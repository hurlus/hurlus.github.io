%%%%%%%%%%%%%%%%%%%%%%%%%%%%%%%%%
% LaTeX model https://hurlus.fr %
%%%%%%%%%%%%%%%%%%%%%%%%%%%%%%%%%

% Needed before document class
\RequirePackage{pdftexcmds} % needed for tests expressions
\RequirePackage{fix-cm} % correct units

% Define mode
\def\mode{a4}

\newif\ifaiv % a4
\newif\ifav % a5
\newif\ifbooklet % booklet
\newif\ifcover % cover for booklet

\ifnum \strcmp{\mode}{cover}=0
  \covertrue
\else\ifnum \strcmp{\mode}{booklet}=0
  \booklettrue
\else\ifnum \strcmp{\mode}{a5}=0
  \avtrue
\else
  \aivtrue
\fi\fi\fi

\ifbooklet % do not enclose with {}
  \documentclass[french,twoside]{book} % ,notitlepage
  \usepackage[%
    papersize={105mm, 297mm},
    inner=12mm,
    outer=12mm,
    top=20mm,
    bottom=15mm,
    marginparsep=3pt,
    marginpar=7mm,
  ]{geometry}
  \usepackage[fontsize=9.5pt]{scrextend} % for Roboto
\else\ifav % A5
  \documentclass[french,twoside]{book} % ,notitlepage
  \usepackage[%
    a5paper
  ]{geometry}
  \usepackage[fontsize=12pt]{scrextend}
\else% A4 2 cols
  \documentclass[twocolumn]{report}
  \usepackage[%
    a4paper,
    inner=15mm,
    outer=10mm,
    top=25mm,
    bottom=18mm,
    marginparsep=0pt,
  ]{geometry}
  \setlength{\columnsep}{20mm}
  \usepackage[fontsize=9.5pt]{scrextend}
\fi\fi

%%%%%%%%%%%%%%
% Alignments %
%%%%%%%%%%%%%%
% before teinte macros

\setlength{\arrayrulewidth}{0.2pt}
\setlength{\columnseprule}{\arrayrulewidth} % twocol
\setlength{\parskip}{0pt} % 1pt allow better vertical justification
\setlength{\parindent}{1.5em}

%%%%%%%%%%
% Colors %
%%%%%%%%%%
% before Teinte macros

\usepackage[dvipsnames]{xcolor}
\definecolor{rubric}{HTML}{0c71c3} % the tonic
\def\columnseprulecolor{\color{rubric}}
\colorlet{borderline}{rubric!30!} % definecolor need exact code
\definecolor{shadecolor}{gray}{0.95}
\definecolor{bghi}{gray}{0.5}

%%%%%%%%%%%%%%%%%
% Teinte macros %
%%%%%%%%%%%%%%%%%
%%%%%%%%%%%%%%%%%%%%%%%%%%%%%%%%%%%%%%%%%%%%%%%%%%%
% <TEI> generic (LaTeX names generated by Teinte) %
%%%%%%%%%%%%%%%%%%%%%%%%%%%%%%%%%%%%%%%%%%%%%%%%%%%
% This template is inserted in a specific design
% It is XeLaTeX and otf fonts

\makeatletter % <@@@

\usepackage{alphalph} % for alph couter z, aa, ab…
\usepackage{blindtext} % generate text for testing
\usepackage{booktabs} % for tables: \toprule, \midrule…
\usepackage[strict]{changepage} % for modulo 4
\usepackage{contour} % rounding words
\usepackage[nodayofweek]{datetime}
\usepackage{enumitem} % <list>
\usepackage{etoolbox} % patch commands
\usepackage{fancyvrb}
\usepackage{fancyhdr}
\usepackage{float}
\usepackage{fontspec} % XeLaTeX mandatory for fonts
\usepackage{footnote} % used to capture notes in minipage (ex: quote)
\usepackage{framed} % bordering correct with footnote hack
\usepackage{graphicx}
\usepackage{lettrine} % drop caps
\usepackage{lipsum} % generate text for testing
\usepackage{manyfoot} % for parallel footnote numerotation
\usepackage[framemethod=tikz,]{mdframed} % maybe used for frame with footnotes inside
\usepackage[defaultlines=2,all]{nowidow} % at least 2 lines by par (works well!)
\usepackage{pdftexcmds} % needed for tests expressions
\usepackage{poetry} % <l>, bad for theater
\usepackage{polyglossia} % bug Warning: "Failed to patch part"
\usepackage[%
  indentfirst=false,
  vskip=1em,
  noorphanfirst=true,
  noorphanafter=true,
  leftmargin=\parindent,
  rightmargin=0pt,
]{quoting}
\usepackage{ragged2e}
\usepackage{setspace} % \setstretch for <quote>
\usepackage{scrextend} % KOMA-common, used for addmargin
\usepackage{tabularx} % <table>
\usepackage[explicit]{titlesec} % wear titles, !NO implicit
\usepackage{tikz} % ornaments
\usepackage{tocloft} % styling tocs
\usepackage[fit]{truncate} % used im runing titles
\usepackage{unicode-math}
\usepackage[normalem]{ulem} % breakable \uline, normalem is absolutely necessary to keep \emph
\usepackage{xcolor} % named colors
\usepackage{xparse} % @ifundefined
\XeTeXdefaultencoding "iso-8859-1" % bad encoding of xstring
\usepackage{xstring} % string tests
\XeTeXdefaultencoding "utf-8"

\defaultfontfeatures{
  % Mapping=tex-text, % no effect seen
  Scale=MatchLowercase,
  Ligatures={TeX,Common},
}
\newfontfamily\zhfont{Noto Sans CJK SC}

% Metadata inserted by a program, from the TEI source, for title page and runing heads
\title{\textbf{ Les luttes de classes en France, 1848-1850 }\par
}
\date{1850}
\author{Karl Marx}
\def\elbibl{Karl Marx. 1850. \emph{Les luttes de classes en France, 1848-1850}}
\def\elsource{\href{https://fr.wikisource.org/wiki/La\_Lutte\_des\_classes\_en\_France\_(1848-1850)}{\dotuline{Wikisource}}\footnote{\href{https://fr.wikisource.org/wiki/La\_Lutte\_des\_classes\_en\_France\_(1848-1850)}{\url{https://fr.wikisource.org/wiki/La\_Lutte\_des\_classes\_en\_France\_(1848-1850)}}}}
\def\eltitlepage{%
{\centering\parindent0pt
  {\LARGE\addfontfeature{LetterSpace=25}\bfseries Karl Marx\par}\bigskip
  {\Large 1850\par}\bigskip
  {\LARGE
\bigskip\textbf{Les luttes de classes en France, 1848-1850}\par

  }
}

}

% Default metas
\newcommand{\colorprovide}[2]{\@ifundefinedcolor{#1}{\colorlet{#1}{#2}}{}}
\colorprovide{rubric}{red}
\colorprovide{silver}{lightgray}
\@ifundefined{syms}{\newfontfamily\syms{DejaVu Sans}}{}
\newif\ifdev
\@ifundefined{elbibl}{% No meta defined, maybe dev mode
  \newcommand{\elbibl}{Titre court ?}
  \newcommand{\elbook}{Titre du livre source ?}
  \newcommand{\elabstract}{Résumé\par}
  \newcommand{\elurl}{http://oeuvres.github.io/elbook/2}
  \author{Éric Lœchien}
  \title{Un titre de test assez long pour vérifier le comportement d’une maquette}
  \date{1566}
  \devtrue
}{}
\let\eltitle\@title
\let\elauthor\@author
\let\eldate\@date




% generic typo commands
\newcommand{\astermono}{\medskip\centerline{\color{rubric}\large\selectfont{\syms ✻}}\medskip\par}%
\newcommand{\astertri}{\medskip\par\centerline{\color{rubric}\large\selectfont{\syms ✻\,✻\,✻}}\medskip\par}%
\newcommand{\asterism}{\bigskip\par\noindent\parbox{\linewidth}{\centering\color{rubric}\large{\syms ✻}\\{\syms ✻}\hskip 0.75em{\syms ✻}}\bigskip\par}%

% lists
\newlength{\listmod}
\setlength{\listmod}{\parindent}
\setlist{
  itemindent=!,
  listparindent=\listmod,
  labelsep=0.2\listmod,
  parsep=0pt,
  % topsep=0.2em, % default topsep is best
}
\setlist[itemize]{
  label=—,
  leftmargin=0pt,
  labelindent=1.2em,
  labelwidth=0pt,
}
\setlist[enumerate]{
  label={\arabic*°},
  labelindent=0.8\listmod,
  leftmargin=\listmod,
  labelwidth=0pt,
}
% list for big items
\newlist{decbig}{enumerate}{1}
\setlist[decbig]{
  label={\bf\color{rubric}\arabic*.},
  labelindent=0.8\listmod,
  leftmargin=\listmod,
  labelwidth=0pt,
}
\newlist{listalpha}{enumerate}{1}
\setlist[listalpha]{
  label={\bf\color{rubric}\alph*.},
  leftmargin=0pt,
  labelindent=0.8\listmod,
  labelwidth=0pt,
}
\newcommand{\listhead}[1]{\hspace{-1\listmod}\emph{#1}}

\renewcommand{\hrulefill}{%
  \leavevmode\leaders\hrule height 0.2pt\hfill\kern\z@}

% General typo
\DeclareTextFontCommand{\textlarge}{\large}
\DeclareTextFontCommand{\textsmall}{\small}

% commands, inlines
\newcommand{\anchor}[1]{\Hy@raisedlink{\hypertarget{#1}{}}} % link to top of an anchor (not baseline)
\newcommand\abbr[1]{#1}
\newcommand{\autour}[1]{\tikz[baseline=(X.base)]\node [draw=rubric,thin,rectangle,inner sep=1.5pt, rounded corners=3pt] (X) {\color{rubric}#1};}
\newcommand\corr[1]{#1}
\newcommand{\ed}[1]{ {\color{silver}\sffamily\footnotesize (#1)} } % <milestone ed="1688"/>
\newcommand\expan[1]{#1}
\newcommand\foreign[1]{\emph{#1}}
\newcommand\gap[1]{#1}
\renewcommand{\LettrineFontHook}{\color{rubric}}
\newcommand{\initial}[2]{\lettrine[lines=2, loversize=0.3, lhang=0.3]{#1}{#2}}
\newcommand{\initialiv}[2]{%
  \let\oldLFH\LettrineFontHook
  % \renewcommand{\LettrineFontHook}{\color{rubric}\ttfamily}
  \IfSubStr{QJ’}{#1}{
    \lettrine[lines=4, lhang=0.2, loversize=-0.1, lraise=0.2]{\smash{#1}}{#2}
  }{\IfSubStr{É}{#1}{
    \lettrine[lines=4, lhang=0.2, loversize=-0, lraise=0]{\smash{#1}}{#2}
  }{\IfSubStr{ÀÂ}{#1}{
    \lettrine[lines=4, lhang=0.2, loversize=-0, lraise=0, slope=0.6em]{\smash{#1}}{#2}
  }{\IfSubStr{A}{#1}{
    \lettrine[lines=4, lhang=0.2, loversize=0.2, slope=0.6em]{\smash{#1}}{#2}
  }{\IfSubStr{V}{#1}{
    \lettrine[lines=4, lhang=0.2, loversize=0.2, slope=-0.5em]{\smash{#1}}{#2}
  }{
    \lettrine[lines=4, lhang=0.2, loversize=0.2]{\smash{#1}}{#2}
  }}}}}
  \let\LettrineFontHook\oldLFH
}
\newcommand{\labelchar}[1]{\textbf{\color{rubric} #1}}
\newcommand{\lnatt}[1]{\reversemarginpar\marginpar[\sffamily\scriptsize #1]{}}
\newcommand{\milestone}[1]{\autour{\footnotesize\color{rubric} #1}} % <milestone n="4"/>
\newcommand\name[1]{#1}
\newcommand\orig[1]{#1}
\newcommand\orgName[1]{#1}
\newcommand\persName[1]{#1}
\newcommand\placeName[1]{#1}
\newcommand{\pn}[1]{\IfSubStr{-—–¶}{#1}% <p n="3"/>
  {\noindent{\bfseries\color{rubric}   ¶  }}
  {{\footnotesize\autour{#1}}}}
\newcommand\reg{}
% \newcommand\ref{} % already defined
\newcommand\sic[1]{#1}
\newcommand\surname[1]{\textsc{#1}}
\newcommand\term[1]{\textbf{#1}}
\newcommand\zh[1]{{\zhfont #1}}


\def\mednobreak{\ifdim\lastskip<\medskipamount
  \removelastskip\nopagebreak\medskip\fi}
\def\bignobreak{\ifdim\lastskip<\bigskipamount
  \removelastskip\nopagebreak\bigskip\fi}

% commands, blocks

\newcommand{\byline}[1]{\bigskip{\RaggedLeft{#1}\par}\bigskip}
% \setlength{\RaggedLeftLeftskip}{2em plus \leftskip}
\newcommand{\bibl}[1]{{\smallskip\RaggedLeft\normalsize\normalfont #1\par\medskip}}
\newcommand{\biblitem}[1]{{\noindent\hangindent=\parindent   #1\par}}
\newcommand{\castItem}[1]{{\noindent\hangindent=\parindent #1\par}}
\newcommand{\dateline}[1]{\medskip{\RaggedLeft{#1}\par}\bigskip}
\newcommand{\docAuthor}[1]{{\large\bigskip #1 \par\bigskip}}
\newcommand{\docDate}[1]{#1 \ifvmode\par\fi }
\newcommand{\docImprint}[1]{\ifvmode\medskip\fi #1 \ifvmode\par\fi }
\newcommand{\labelblock}[1]{\medbreak{\noindent\color{rubric}\bfseries #1}\par\mednobreak}
\newcommand{\salute}[1]{\bigbreak{#1}\par\medbreak}
\newcommand{\signed}[1]{\medskip{\RaggedLeft #1\par}\bigbreak} % supposed bottom
\newcommand{\speaker}[1]{\medskip{\Centering\sffamily #1 \par\nopagebreak}} % supposed bottom
\newcommand{\stagescene}[1]{{\Centering\sffamily\textsf{#1}\par}\bigskip}
\newcommand{\stageblock}[1]{\begingroup\leftskip\parindent\noindent\it\sffamily\footnotesize #1\par\endgroup} % left margin, better than list envs
\newcommand{\spl}[1]{\noindent\hangindent=2\parindent  #1\par} % sp/l
\newcommand{\trailer}[1]{{\Centering\bigskip #1\par}} % sp/l

% environments for blocks (some may become commands)
\newenvironment{borderbox}{}{} % framing content
\newenvironment{citbibl}{\ifvmode\hfill\fi}{\ifvmode\par\fi }
\newenvironment{msHead}{\vskip6pt}{\par}
\newenvironment{msItem}{\vskip6pt}{\par}


% environments for block containers
\newenvironment{argument}{\itshape\parindent0pt}{\bigskip}
\newenvironment{biblfree}{}{\ifvmode\par\fi }
\newenvironment{bibitemlist}[1]{%
  \list{\@biblabel{\@arabic\c@enumiv}}%
  {%
    \settowidth\labelwidth{\@biblabel{#1}}%
    \leftmargin\labelwidth
    \advance\leftmargin\labelsep
    \@openbib@code
    \usecounter{enumiv}%
    \let\p@enumiv\@empty
    \renewcommand\theenumiv{\@arabic\c@enumiv}%
  }
  \sloppy
  \clubpenalty4000
  \@clubpenalty \clubpenalty
  \widowpenalty4000%
  \sfcode`\.\@m
}%
{\def\@noitemerr
  {\@latex@warning{Empty `bibitemlist' environment}}%
\endlist}
\newenvironment{docTitle}{\LARGE\bigskip\bfseries\onehalfspacing}{\bigskip}
% leftskip makes big bugs in Lexmark printing \sffamily
\newenvironment{epigraph}{\begin{addmargin}[2\parindent]{0em}\sffamily\large\setstretch{0.95}}{\end{addmargin}\bigskip}
\newenvironment{quoteblock}% may be used for ornaments
  {\begin{quoting}}
  {\end{quoting}}
\newenvironment{titlePage}
  {\Centering}
  {}






% table () is preceded and finished by custom command
\renewcommand\tabularxcolumn[1]{m{#1}}% for vertical centering text in X column
\newcommand{\tableopen}[1]{%
  \ifnum\strcmp{#1}{wide}=0{%
    \begin{center}
  }
  \else\ifnum\strcmp{#1}{long}=0{%
    \begin{center}
  }
  \else{%
    \begin{center}
  }
  \fi\fi
}
\newcommand{\tableclose}[1]{%
  \ifnum\strcmp{#1}{wide}=0{%
    \end{center}
  }
  \else\ifnum\strcmp{#1}{long}=0{%
    \end{center}
  }
  \else{%
    \end{center}
  }
  \fi\fi
}


% text structure
\newcommand\chapteropen{} % before chapter title
\newcommand\chaptercont{} % after title, argument, epigraph…
\newcommand\chapterclose{} % maybe useful for multicol settings
\setcounter{secnumdepth}{-2} % no counters for hierarchy titles
\setcounter{tocdepth}{5} % deep toc
\renewcommand\tableofcontents{\@starttoc{toc}}
% toclof format
% \renewcommand{\@tocrmarg}{0.1em} % Useless command?
% \renewcommand{\@pnumwidth}{0.5em} % {1.75em}
\renewcommand{\@cftmaketoctitle}{}
\setlength{\cftbeforesecskip}{\z@ \@plus.2\p@}
\renewcommand{\cftchapfont}{}
\renewcommand{\cftchapdotsep}{\cftdotsep}
\renewcommand{\cftchapleader}{\normalfont\cftdotfill{\cftchapdotsep}}
\renewcommand{\cftchappagefont}{\bfseries}
\setlength{\cftbeforechapskip}{0em \@plus\p@}
% \renewcommand{\cftsecfont}{\small\relax}
\renewcommand{\cftsecpagefont}{\normalfont}
% \renewcommand{\cftsubsecfont}{\small\relax}
\renewcommand{\cftsecdotsep}{\cftdotsep}
\renewcommand{\cftsecpagefont}{\normalfont}
\renewcommand{\cftsecleader}{\normalfont\cftdotfill{\cftsecdotsep}}
\setlength{\cftsecindent}{1em}
\setlength{\cftsubsecindent}{2em}
\setlength{\cftsubsubsecindent}{3em}
\setlength{\cftchapnumwidth}{1em}
\setlength{\cftsecnumwidth}{1em}
\setlength{\cftsubsecnumwidth}{1em}
\setlength{\cftsubsubsecnumwidth}{1em}

% footnotes
\newif\ifheading
\newcommand*{\fnmarkscale}{\ifheading 0.70 \else 1 \fi}
\renewcommand\footnoterule{\vspace*{0.3cm}\hrule height \arrayrulewidth width 3cm \vspace*{0.3cm}}
\setlength\footnotesep{1.5\footnotesep} % footnote separator
\renewcommand\@makefntext[1]{\parindent 1.5em \noindent \hb@xt@1.8em{\hss{\normalfont\@thefnmark . }}#1} % no superscipt in foot
\patchcmd{\@footnotetext}{\footnotesize}{\footnotesize\sffamily}{}{} % before scrextend, hyperref
\DeclareNewFootnote{A}[alph] % for editor notes
\renewcommand*{\thefootnoteA}{\alphalph{\value{footnoteA}}} % z, aa, ab…

% poem
\setlength{\poembotskip}{0pt}
\setlength{\poemtopskip}{0pt}
\setlength{\poemindent}{0pt}
\poemlinenumsfalse

%   see https://tex.stackexchange.com/a/34449/5049
\def\truncdiv#1#2{((#1-(#2-1)/2)/#2)}
\def\moduloop#1#2{(#1-\truncdiv{#1}{#2}*#2)}
\def\modulo#1#2{\number\numexpr\moduloop{#1}{#2}\relax}

% orphans and widows, nowidow package in test
% from memoir package
\clubpenalty=9996
\widowpenalty=9999
\brokenpenalty=4991
\predisplaypenalty=10000
\postdisplaypenalty=1549
\displaywidowpenalty=1602
\hyphenpenalty=400
% report h or v overfull ?
\hbadness=4000
\vbadness=4000
% good to avoid lines too wide
\emergencystretch 3em
\pretolerance=750
\tolerance=2000
\def\Gin@extensions{.pdf,.png,.jpg,.mps,.tif}

\PassOptionsToPackage{hyphens}{url} % before hyperref and biblatex, which load url package
\usepackage{hyperref} % supposed to be the last one, :o) except for the ones to follow
\hypersetup{
  % pdftex, % no effect
  pdftitle={\elbibl},
  % pdfauthor={Your name here},
  % pdfsubject={Your subject here},
  % pdfkeywords={keyword1, keyword2},
  bookmarksnumbered=true,
  bookmarksopen=true,
  bookmarksopenlevel=1,
  pdfstartview=Fit,
  breaklinks=true, % avoid long links, overrided by url package
  pdfpagemode=UseOutlines,    % pdf toc
  hyperfootnotes=true,
  colorlinks=false,
  pdfborder=0 0 0,
  % pdfpagelayout=TwoPageRight,
  % linktocpage=true, % NO, toc, link only on page no
}
\urlstyle{same} % after hyperref



\makeatother % /@@@>
%%%%%%%%%%%%%%
% </TEI> end %
%%%%%%%%%%%%%%


%%%%%%%%%%%%%
% footnotes %
%%%%%%%%%%%%%
\renewcommand{\thefootnote}{\bfseries\textcolor{rubric}{\arabic{footnote}}} % color for footnote marks

%%%%%%%%%
% Fonts %
%%%%%%%%%
% \linespread{0.90} % too compact, keep font natural
\ifav % A5
  \usepackage{DejaVuSans} % correct
  \setsansfont{DejaVuSans} % seen, if not set, problem with printer
\else\ifbooklet
  \usepackage[]{roboto} % SmallCaps, Regular is a bit bold
  \setmainfont[
    ItalicFont={Roboto Light Italic},
  ]{Roboto}
  \setsansfont{Roboto Light} % seen, if not set, problem with printer
  \newfontfamily\fontrun[]{Roboto Condensed Light} % condensed runing heads
\else
  \usepackage[]{roboto} % SmallCaps, Regular is a bit bold
  \setmainfont[
    ItalicFont={Roboto Italic},
  ]{Roboto Light}
  \setsansfont{Roboto Light} % seen, if not set, problem with printer
  \newfontfamily\fontrun[]{Roboto Condensed Light} % condensed runing heads
\fi\fi
\renewcommand{\LettrineFontHook}{\bfseries\color{rubric}}
% \renewenvironment{labelblock}{\begin{center}\bfseries\color{rubric}}{\end{center}}

%%%%%%%%
% MISC %
%%%%%%%%

\setdefaultlanguage[frenchpart=false]{french} % bug on part


\newenvironment{quotebar}{%
    \def\FrameCommand{{\color{rubric!10!}\vrule width 0.5em} \hspace{0.9em}}%
    \def\OuterFrameSep{0pt} % séparateur vertical
    \MakeFramed {\advance\hsize-\width \FrameRestore}
  }%
  {%
    \endMakeFramed
  }
\renewenvironment{quoteblock}% may be used for ornaments
  {%
    \savenotes
    \setstretch{0.9}
    \begin{quotebar}
    \smallskip
  }
  {%
    \smallskip
    \end{quotebar}
    \spewnotes
  }


\renewcommand{\headrulewidth}{\arrayrulewidth}
\renewcommand{\headrule}{{\color{rubric}\hrule}}
\renewcommand{\lnatt}[1]{\marginpar{\sffamily\scriptsize #1}}

% delicate tuning, image has produce line-height problems in title on 2 lines
\titleformat{name=\chapter} % command
  [display] % shape
  {\vspace{1.5em}\centering} % format
  {} % label
  {0pt} % separator between n
  {}
[{\color{rubric}\huge\textbf{#1}}\bigskip] % after code
% \titlespacing{command}{left spacing}{before spacing}{after spacing}[right]
\titlespacing*{\chapter}{0pt}{-2em}{0pt}[0pt]

\titleformat{name=\section}
  [display]{}{}{}{}
  [\vbox{\color{rubric}\large\centering\textbf{#1}}]
\titlespacing{\section}{0pt}{0pt plus 4pt minus 2pt}{\baselineskip}

\titleformat{name=\subsection}
  [block]
  {}
  {} % \thesection
  {} % separator \arrayrulewidth
  {}
[\vbox{\large\textbf{#1}}]
% \titlespacing{\subsection}{0pt}{0pt plus 4pt minus 2pt}{\baselineskip}

\ifaiv
  \fancypagestyle{main}{%
    \fancyhf{}
    \setlength{\headheight}{1.5em}
    \fancyhead{} % reset head
    \fancyfoot{} % reset foot
    \fancyhead[L]{\truncate{0.45\headwidth}{\fontrun\elbibl}} % book ref
    \fancyhead[R]{\truncate{0.45\headwidth}{ \fontrun\nouppercase\leftmark}} % Chapter title
    \fancyhead[C]{\thepage}
  }
  \fancypagestyle{plain}{% apply to chapter
    \fancyhf{}% clear all header and footer fields
    \setlength{\headheight}{1.5em}
    \fancyhead[L]{\truncate{0.9\headwidth}{\fontrun\elbibl}}
    \fancyhead[R]{\thepage}
  }
\else
  \fancypagestyle{main}{%
    \fancyhf{}
    \setlength{\headheight}{1.5em}
    \fancyhead{} % reset head
    \fancyfoot{} % reset foot
    \fancyhead[RE]{\truncate{0.9\headwidth}{\fontrun\elbibl}} % book ref
    \fancyhead[LO]{\truncate{0.9\headwidth}{\fontrun\nouppercase\leftmark}} % Chapter title, \nouppercase needed
    \fancyhead[RO,LE]{\thepage}
  }
  \fancypagestyle{plain}{% apply to chapter
    \fancyhf{}% clear all header and footer fields
    \setlength{\headheight}{1.5em}
    \fancyhead[L]{\truncate{0.9\headwidth}{\fontrun\elbibl}}
    \fancyhead[R]{\thepage}
  }
\fi

\ifav % a5 only
  \titleclass{\section}{top}
\fi

\newcommand\chapo{{%
  \vspace*{-3em}
  \centering\parindent0pt % no vskip ()
  \eltitlepage
  \bigskip
  {\color{rubric}\hline}
  \bigskip
  {\Large TEXTE LIBRE À PARTICIPATIONS LIBRES\par}
  \centerline{\small\color{rubric} {\href{https://hurlus.fr}{\dotuline{hurlus.fr}}}, tiré le \today}\par
  \bigskip
}}

\newcommand\cover{{%
  \thispagestyle{empty}
  \centering\parindent0pt
  \eltitlepage
  \vfill\null
  {\color{rubric}\setlength{\arrayrulewidth}{2pt}\hline}
  \vfill\null
  {\Large TEXTE LIBRE À PARTICIPATIONS LIBRES\par}
  \centerline{\href{https://hurlus.fr}{\dotuline{hurlus.fr}}, tiré le \today}\par
}}

\begin{document}
\pagestyle{empty}
\ifbooklet{
  \cover\newpage
  \thispagestyle{empty}\hbox{}\newpage
  \cover\newpage\noindent Les voyages de la brochure\par
  \bigskip
  \begin{tabularx}{\textwidth}{l|X|X}
    \textbf{Date} & \textbf{Lieu}& \textbf{Nom/pseudo} \\ \hline
    \rule{0pt}{25cm} &  &   \\
  \end{tabularx}
  \newpage
  \addtocounter{page}{-4}
}\fi

\thispagestyle{empty}
\ifaiv
  \twocolumn[\chapo]
\else
  \chapo
\fi
{\it\elabstract}
\bigskip
\makeatletter\@starttoc{toc}\makeatother % toc without new page
\bigskip

\pagestyle{main} % after style
\setcounter{footnote}{0}
\setcounter{footnoteA}{0}
  
\chapteropen

\chapter[{I. De février à juin 1848}]{I. De février à juin 1848}
\renewcommand{\leftmark}{I. De février à juin 1848}


\chaptercont
\noindent À l’exception de quelques chapitres, la partie importante des annales révolutionnaires qui va de 1848 à 1849 porte le titre de \emph{défaites de la Révolution}.\par
Ce qui succomba dans ces défaites, ce ne fut pas la Révolution elle-même ; ce furent les accessoires révolutionnaires qui dataient de l’époque précédente. Ils provenaient de rapports sociaux confus, les antagonismes de classe ne s’accusaient pas encore nettement. Avant la révolution de Février, le parti révolutionnaire ne s’était pas encore défait des personnalités, des illusions, de certaines idées, de certains projets. Ce n’était pas la \emph{victoire de Février}, mais une suite de \emph{défaites} qui pouvait l’en délivrer.\par
En un mot : la révolution ne gagna rien à ses succès directs, tragi-comiques ; au contraire. Ce qui la servit, ce fut la constitution d’une contre-révolution puissante, bien limitée, ce fut l’apparition d’un adversaire. En le combattant, le parti insurrectionnel arriva à maturité et devint un véritable parti révolutionnaire.\par
Le but des pages suivantes est d’établir ce point.\par

\section[{1. La défaite de juin 1848}]{1. La défaite de juin 1848}

\noindent Après la révolution de Juillet, quand le banquier libéral Laffite mena en triomphe son compère le duc d’Orléans à l’Hôtel de Ville, il laissa échapper ce mot : \emph{Maintenant le règne des banquiers va commencer}. Laffite trahissait ainsi le secret de la révolution.\par
Sous Louis-Philippe, seule une fraction de la bourgeoisie française régnait. C’étaient les banquiers, les rois de la Bourse et des chemins de fer, les possesseurs de mines de charbon et de fer, les propriétaires de forêts ; une partie de la féodalité foncière s’était en effet ralliée aux premiers. Tous ensemble, ils constituaient ce que l’on a appelé \emph{l’aristocratie financière}. Elle siégeait sur le trône, dictait les lois aux chambres, distribuait les emplois, depuis les ministères jusqu’aux bureaux de tabac.\par
La \emph{bourgeoisie industrielle} proprement dite formait une partie de l’opposition. Elle n’avait qu’une minorité pour la représenter dans les chambres. Son opposition devint d’autant plus vive que l’hégémonie de l’aristocratie financière prenait plus d’extension et qu’elle-même voyait s’asseoir davantage sa domination sur la classe ouvrière après les émeutes de 1832, 1834 et 1839, que l’on étouffa dans le sang. \emph{Grandin}, fabricant de Rouen, l’organe le plus fanatique de la réaction bourgeoise, tant dans la Législative que dans la Constituante, était à la Chambre des députés l’adversaire le plus zélé de Guizot. \emph{Léon Faucher}, que ses efforts impuissants pour se hausser au rôle d’un Guizot de la contre-révolution rendirent célèbre dans la suite, mena, dans les dernières années du règne de Louis-Philippe, une campagne de presse en faveur de l’industrie contre la spéculation et contre son caudataire, le gouvernement. \emph{Bastiat}, au nom de Bordeaux et de toute la France vinicole, faisait de l’agitation contre le système dominant.\par
Toutes les différentes couches de \emph{la petite bourgeoisie}, ainsi que la classe paysanne, étaient complètement exclues du pouvoir politique. Enfin les représentants idéologiques, les interprètes des classes que nous venons de citer, les savants, les avocats, les médecins, etc., bref, ce que l’on appelait les capacités, se rencontraient dans les rangs de l’opposition officielle ou se trouvaient même placés complètement en dehors du « pays légal. »\par
Les difficultés financières soumettaient dès l’abord la monarchie de Juillet à la haute bourgeoisie, et cette dépendance devint précisément une source intarissable de difficultés financières croissantes. Il était impossible de subordonner l’administration de l’État aux intérêts de la production nationale sans établir l’équilibre du budget, l’équilibre entre les dépenses et les recettes de l’État. Et comment établir cet équilibre sans limiter les frais, c’est-à-dire sans léser des intérêts qui consolidaient d’autant le système ? comment y arriver sans une nouvelle réglementation de l’assiette des impôts, sans faire peser une partie importante de leur poids sur la grande bourgeoisie elle-même ?\par
La fraction de la bourgeoisie qui dominait et légiférait dans les Chambres avait un \emph{intérêt direct} à voir l’\emph{État s’endetter}. Le \emph{déficit} était l’objet propre de la spéculation, la source principale d’enrichissement. Chaque année ramenait un nouveau déficit. Au bout de quatre ou cinq ans on faisait un nouvel emprunt. Chaque nouvel emprunt fournissait à l’aristocratie financière une occasion nouvelle de duper l’État, artificiellement maintenu sous la menace d’une banqueroute. Il devenait nécessaire de traiter avec les banquiers dans les conditions les plus défavorables. Chaque nouvel emprunt permettait en outre de piller le public qui place ses capitaux en rentes sur l’État et de le dépouiller par des opérations de bourse dont le secret était abandonné au gouvernement et à la majorité. Les fluctuations du crédit public et la connaissance des secrets d’État permettaient aux banquiers et à leurs affiliés de susciter dans le cours des papiers d’État des variations extraordinaires et soudaines. Le résultat constant des oscillations devait être la ruine d’une masse de petits capitalistes et l’enrichissement fabuleusement rapide des grands spéculateurs. La fraction de la bourgeoisie qui dominait avait donc un intérêt direct à ce que l’État fût en déficit. On s’explique donc que, dans les dernières années du règne de Louis-Philippe, les \emph{crédits extraordinaires} aient dépassé de beaucoup le double de leur montant sous Napoléon. Ils étaient supérieurs à 400 millions de francs, alors que l’exportation annuelle de la France s’éleva rarement à plus de 750 millions. De plus, les sommes énormes qui passaient ainsi entre les mains de l’État laissaient place aux adjudications frauduleuses, aux corruptions, malversations, coquineries de toute espèce. L’État, lésé en grand par les emprunts, l’était en détail dans les travaux publics. Les relations nouées entre la Chambre et le Gouvernement se compliquaient de celles qui s’établissaient entre les administrations et les entrepreneurs isolés.\par
Non contente de tirer profit des dépenses et des emprunts publics, la classe dominante exploitait \emph{les lignes de chemins de fer}. Les Chambres attribuaient à l’État les principales charges et réservaient à l’aristocratie de la spéculation les fruits du trafic. Qu’on se souvienne du scandale qui éclata à la Chambre des députés quand il apparut que beaucoup de membres de la majorité, y compris une partie des ministres, étaient actionnaires des mêmes lignes de chemins de fer que, comme législateurs, ils faisaient construire aux frais de l’État.\par
La plus mince des réformes financières échouait devant l’influence des banquiers, par exemple, la \emph{réforme postale}. Rothschild protesta. L’État pouvait-il amaigrir des sources de revenu dont pouvait profiter la dette sans cesse croissante ?\par
La monarchie de Juillet n’était qu’une compagnie par actions fondée pour l’exploitation de la richesse nationale de la France. Les ministres, les Chambres, deux cent quarante mille électeurs et ceux qui les approchaient s’en partageaient les dividendes. Louis-Philippe était le directeur de cette compagnie. Robert Macaire était sur le trône. Le commerce, l’industrie, l’agriculture, la navigation, les intérêts de la bourgeoisie industrielle étaient condamnés à être constamment exposés, menacés par ce système. Cette bourgeoisie avait inscrit sur ses drapeaux : « Gouvernement à bon marché ».\par
L’aristocratie financière dictait les lois, présidait à l’administration de l’État, disposait d’une grande partie des pouvoirs organisés, régnait sur l’opinion publique grâce aux événements et à la presse. Dans toutes les sphères, de la cour au « café borgne », on retrouvait la même prostitution, la même tromperie éhontée, la même soif de s’enrichir, non en produisant, mais en escamotant la richesse que d’autres possédaient déjà. Les régions supérieures de la société bourgeoise subirent les convoitises malsaines, déréglées, effrénées dont la satisfaction viole incessamment même les lois bourgeoises, par lesquelles la richesse acquise au jeu cherche naturellement à se contenter, où la jouissance devient « crapuleuse\footnote{En français dans le texte} », où la boue et le sang se mêlent à l’argent. \emph{La canaille}\footnote{Lumpenproletariat}\emph{ se trouve transportée dans les sphères supérieures de la société bourgeoise et refleurit} dans l’aristocratie financière, dans ses moyens d’acquérir et dans ses jouissances.\par
Et, cependant, les fractions de la bourgeoisie qui ne dominaient pas criaient à la \emph{corruption} ! Le peuple criait : \emph{À bas les grands voleurs ! À bas les assassins !\footnote{En français dans le texte}} quand, en 1847, sur les scènes les plus distinguées de la société bourgeoise, se jouaient publiquement les actes que la canaille a coutume de commettre dans les bordels, les maisons de fous, les maisons de charité, devant les tribunaux, dans les bagnes et sur l’échafaud. La bourgeoisie industrielle voyait ses intérêts compromis ; la petite bourgeoisie était choquée dans sa morale ; l’imagination populaire s’excitait. Paris était inondé de pamphlets – \emph{la dynastie Rothschild, les Juifs, rois de l’époque}, etc.\footnote{En français dans le texte}, où le règne de l’aristocratie financière était dénoncé et flétri avec plus ou moins d’esprit. « Rien pour la gloire\footnote{En français dans le texte} ! » La gloire ne rapporte rien. « La paix partout et toujours\footnote{En français dans le texte} ! » La guerre fait baisser le cours des 3 et 4 \%. Voilà ce que la France des boursiers juifs avait inscrit sur ses drapeaux. La politique étrangère sombra dans une série d’humiliations du sentiment national. Il s’exalta d’autant plus que l’annexion de Cracovie à l’Autriche consommait le vol commis au préjudice de la Pologne et que Guizot, dans la guerre du Sunderbund, servait activement les intérêts de la Sainte-Alliance. La victoire remportée par les libéraux suisses dans cette guerre fantaisiste releva la conscience de l’opposition bourgeoise en France ; le soulèvement meurtrier du peuple de Palerme fut la décharge électrique qui secoua la masse populaire paralysée, qui réveilla ses souvenirs et ses sentiments révolutionnaires.\par
Enfin, l’explosion du malaise général fut précipitée, le mécontentement mûrit et se changea en révolte, grâce à \emph{deux phénomènes économiques généraux.}\par
La \emph{maladie de la pomme de terre} et les \emph{mauvaises récoltes} de 1845 et de 1846 augmentèrent l’effervescence générale du peuple. La cherté des vivres en 1847 amena des conflits sanglants en France comme sur le reste du continent. Alors que l’aristocratie financière s’abandonnait à une orgie éhontée, le peuple se battait pour se procurer les premiers moyens d’existence ! À Buzançais, on exécuta les émeutiers de la faim ; à Paris, des escrocs repus étaient soustraits aux tribunaux par la famille royale !\par
Le second grand événement économique qui précipita le déchaînement de la Révolution fut une \emph{crise générale de l’industrie et du commerce} qui sévit en Angleterre. Elle s’annonça, dès l’automne de 1845, par la ruine de nombreux spéculateurs en actions de chemins de fer. Elle fut enrayée en 1846 par une foule d’incidents comme la suppression des droits de douane sur les blés. Elle éclata enfin dans l’automne de 1847. Les grands marchands coloniaux de Londres firent banqueroute. Los faillites des banques provinciales et la fermeture des fabriques dans ses districts industriels de l’Angleterre suivirent de près. Le contre-coup de cette crise se faisait encore sentir sur le continent quand éclata la révolution de Février.\par
Le commerce et l’industrie avaient été ruinés par cette épidémie économique, la tyrannie de l’aristocratie financière n’en devint que plus insupportable. Dans toute la France, l’opposition bourgeoise créa, sous prétexte de \emph{réforme électorale, l’agitation des banquets}. Cette réforme devait lui procurer la majorité dans les chambres et renverser le ministère des boursiers. À Paris, la crise industrielle avait eu pour conséquence particulière de lancer sur le marché intérieur une masse de fabricants et de gros commerçants à qui les conditions présentes interdisaient Le marché étranger. Ils fondèrent de grands établissements dont la concurrence ruina quantité d’épiciers et de boutiquiers. Il s’ensuivit un nombre énorme de faillites, frappant cette partie de la bourgeoisie parisienne ; le résultat fut que cette fraction intervint dans la révolution de Février. On sait que Guizot et les Chambres répondirent par une provocation non déguisée aux propositions de réforme. Louis-Philippe se résigna trop tard à un ministère Barrot. L’armée et le peuple en vinrent aux prises. L’attitude passive de la garde nationale désarma l’armée. La monarchie de Juillet dut faire place à un gouvernement provisoire.\par
La composition du \emph{gouvernement provisoire} qui sortit des barricades de Février reflétait nécessairement les différents partis qui se partageaient la victoire. Ce gouvernement ne pouvait être que le résultat d’un \emph{compromis entre les différentes classes} qui avaient renversé de concert le trône de Juillet, mais dont les intérêts étaient opposés. \emph{La grande majorité} était formée de représentants de la bourgeoisie. La petite bourgeoisie républicaine y comptait Ledru-Rollin et Flocon, la bourgeoisie républicaine avait les gens du \emph{National}, l’opposition dynastique Crémieux, Dupont de l’Eure, etc. La classe ouvrière ne possédait que deux représentants : Louis Blanc et Albert. Enfin, dans le gouvernement provisoire, Lamartine ne traduisait aucun intérêt réel, n’était commis par aucune classe déterminée. Lamartine, c’était la révolution de Février elle-même, l’exaltation commune avec ses illusions, sa poésie, son contenu chimérique et ses phrases. D’ailleurs, ce porte-parole de la révolution de Février appartenait à la bourgeoisie par sa situation comme par ses idées.\par
Si la centralisation accorde à Paris la suprématie sur la France, les ouvriers dominent Paris dans les moments de cataclysmes révolutionnaires. Le premier acte du gouvernement provisoire fut une tentative de se soustraire à cette influence victorieuse en en appelant de l’ivresse de Paris au sang-froid de la France. Lamartine contesta aux combattants des barricades le droit de proclamer la République. Seule la majorité des Français avait qualité pour le faire. Il fallait attendre le vote. Le prolétariat parisien ne pouvait souiller sa victoire par une usurpation. La bourgeoisie ne permet au prolétariat qu’\emph{une seule} usurpation : elle lui permet d’usurper sa place au combat.\par
Le 25 février, à midi, si la République n’était pas encore proclamée, du moins les éléments bourgeois du gouvernement provisoire, puis les généraux, les banquiers, les avocats du \emph{National} s’étaient-ils attribué tous les ministères. Mais les ouvriers étaient résolus à ne pas tolérer un escamotage semblable à celui de juillet 1830. Ils étaient prêts à reprendre la lutte et à conquérir la République, les armes à la main. \emph{Raspail} se rendit à l’Hôtel de Ville, porteur d’un message en ce sens. Au nom du prolétariat parisien, il \emph{ordonna} au gouvernement provisoire de proclamer la République. Si cet ordre populaire n’était pas exécuté au bout de deux heures, il devait revenir à la tête de 200 000 hommes. Les cadavres n’avaient pas eu le temps de refroidir ; les barricades étaient toujours dressées ; les ouvriers n’étaient pas désarmés et la garde nationale restait la seule force qu’on pût leur opposer. Dans ces conditions, les considérations politiques et les scrupules juridiques du gouvernement provisoire ne tardèrent pas à s’évanouir. Le délai de deux heures n’était pas écoulé que sur tous les murs de Paris s’étalait la devise géante :

\section[{2. République française ! Liberté ! Égalité ! Fraternité !}]{2. République française ! Liberté ! Égalité ! Fraternité !}

\noindent La proclamation de la République, basée sur le suffrage universel, avait fait oublier le but et les motifs étroits qui avaient entraîné la bourgeoisie dans la révolution de Février. Au lieu de quelques fractions peu nombreuses de cette classe, c’étaient toutes les classes composant la société française qui voyaient s’ouvrir la carrière politique. Elles étaient contraintes de délaisser les loges, le parterre, les galeries, et de venir en personne jouer leur rôle sur la scène révolutionnaire ! La chute de la royauté constitutionnelle dissipa une illusion. On vit que le pouvoir public n’était pas l’ennemi systématique de la bourgeoisie. Toute la série de luttes préliminaires qui avaient leur source dans ce pouvoir apparent devenait par là même inutile !\par
Le prolétariat, en imposant la République au gouvernement provisoire, et par le gouvernement provisoire à toute la France, passa au premier plan et devint un parti indépendant. Mais c’était aussi provoquer toute la France bourgeoise. En conquérant ainsi le terrain indispensable pour s’émanciper par la révolution, le prolétariat ne conquérait nullement cette émancipation elle-même.\par
Une première tâche s’imposait à la République de Février : il lui fallait \emph{parfaire la domination} de la bourgeoisie, puisqu’elle laissait monter sur la scène politique, outre l’aristocratie financière, \emph{toutes les classes possédantes}. La majorité des grands propriétaires français, les légitimistes, étaient délivrés de l’impuissance politique à laquelle les condamnait la monarchie de Juillet. Ce n’était pas en vain que la \emph{Gazette de France} avait participé à la campagne d’agitation menée par les feuilles de l’opposition, que Larochejacquelin avait embrassé le parti de la Révolution à la séance de la Chambre des députés du 24 Février. Grâce au suffrage universel, les propriétaires nominaux qui forment la majorité des Français, les \emph{paysans}, devinrent les arbitres du sort de la France. La République de Février donna un caractère net à la domination de la bourgeoisie en brisant la couronne derrière laquelle se dissimulait le capital.\par
Dans les journées de juillet, les ouvriers avaient conquis la \emph{monarchie bourgeoise}, dans les journées de lévrier, la \emph{république bourgeoise}. La monarchie de Juillet avait été contrainte de se présenter comme \emph{une monarchie entourée d’institutions républicaines} ; la République de Février dut être \emph{une république entourée d’institutions sociales}. Le prolétariat parisien exigea également cette concession.\par
Un ouvrier, Marche, dicta le décret où le gouvernement provisoire, à peine formé, s’engageait à assurer l’existence de l’ouvrier au moyen du travail, à fournir du travail à tous les citoyens, etc. Puis, quand quelques jours plus tard, les engagements pris s’oublièrent et qu’on sembla avoir perdu le prolétariat de vue, une colonne de 20 000 hommes marcha sur l’Hôtel-de-Ville, aux cris de : \emph{Organisation du travail ! Constitution d’un ministère spécial du travail} ! À regret et après de longs débats, le gouvernement provisoire nomma une commission spéciale permanente, chargée de \emph{découvrir} les moyens d’améliorer le sort de la classe ouvrière. Cette commission fut composée des délégués des corporations ouvrières de Paris et présidée par Louis Blanc et par Albert. Le Luxembourg servit de salle de séance. Les représentants de la classe ouvrière étaient donc bannis du gouvernement provisoire. Les bourgeois qui faisaient partie de ce dernier possédaient le pouvoir réel. Ils avaient entre leurs mains les rênes de l’administration et à côté des ministères des Finances, du Commerce et des Travaux publics s’éleva une \emph{synagogue socialiste}, dont les grands prêtres, Louis Blanc et Albert, avaient pour mission de découvrir la terre promise, de publier le nouvel évangile et d’occuper le prolétariat parisien. À la différence du pouvoir profane, cette chapelle n’avait à sa disposition ni budget ni pouvoir exécutif. Le cerveau devait à lui tout seul abattre les fondements de la société bourgeoise. Tandis que le Luxembourg cherchait la pierre philosophale, on frappait, à l’Hôtel-de-Ville, la monnaie ayant cours.\par
Et cependant, comme les prétentions du prolétariat de Paris dépassaient la république bourgeoise, elles ne pouvaient avoir que l’existence nébuleuse que leur donnait le Luxembourg.\par
Les ouvriers avaient fait la révolution de Février de concert avec la bourgeoisie. De même qu’ils avaient installé à côté de la majorité bourgeoise un ouvrier dans le gouvernement provisoire, ils cherchaient à faire prévaloir leurs intérêts \emph{à côté} de la bourgeoisie. \emph{Organisation du travail} ! Mais c’est le salariat qui est l’organisation actuelle, l’organisation bourgeoise du travail. Sans le salariat, pas de capital, pas de bourgeoisie, pas de société bourgeoise. \emph{Ministère spécial du travail} ! Mais les ministères des Finances, du Commerce et des Travaux publics, ne sont-ils pas les \emph{ministères bourgeois} du travail ? \emph{À côté d’eux}, que pouvait être un ministère prolétarien du travail, sinon un organe voué à l’impuissance, un ministère des bonnes intentions, une commission du Luxembourg. Les ouvriers pensaient pouvoir s’émanciper à côté de la bourgeoisie, ils croyaient encore pouvoir accomplir une révolution prolétarienne à côté des autres nations bourgeoises, dans les limites nationales de la France. Mais les conditions de production de la France sont déterminées par le commerce extérieur de ce pays, par sa situation sur le marché international, par les lois de ce dernier. Comment la France aurait-elle pu les violer sans amener une révolution européenne ayant son contre-coup sur l’Angleterre, le despote du marché international ?\par
Dès qu’une classe qui concentre entre ses mains les intérêts révolutionnaires de la société, s’est soulevée, elle trouve dans sa situation même, le contenu, la substance de son activité révolutionnaire. Elle abat ses ennemis, prend les mesures exigées par les nécessités de la lutte ; les conséquences de ses propres actes la font agir. Elle ne se livre pas à des recherches théoriques sur la tâche qui lui est propre. La classe ouvrière en France n’en était pas à ce point. Elle était encore incapable d’accomplir sa propre révolution.\par
Le développement du prolétariat industriel a pour condition générale le progrès de la bourgeoisie industrielle. Dominé par elle, le prolétariat voit son existence s’étendre aux limites de la nation, et la révolution qu’il fait s’élève au rang d’une révolution nationale. Alors seulement il crée ces moyens de production modernes, qui sont autant d’instruments de son émancipation révolutionnaire. Seule cette domination arrache les racines matérielles de la société féodale et aplanit le terrain indispensable à toute révolution prolétarienne. L’industrie française est plus développée, la bourgeoisie française a une éducation révolutionnaire plus élevée que celle de la bourgeoisie du reste du continent. Mais la révolution de Février n’était-elle pas directement dirigée contre l’aristocratie financière ? Ce fait montre que la bourgeoisie industrielle ne régnait pas sur la France. La bourgeoisie industrielle ne peut régner que là où l’industrie moderne a coulé dans son moule tous les rapports de production. La conquête du marché international seule lui confère cette puissance ; les limites nationales, en effet, sont des entraves à son développement. Mais l’industrie française, pour la plus grande part, ne détient le marché national que grâce à un système prohibitif plus ou moins pur. Si, dans le moment d’une révolution parisienne, le prolétariat français jouit d’une puissance et d’une influence réelles qui le poussent à dépasser ses moyens, par contre, dans le reste de la France, il reste concentré en quelques points isolés où l’industrie est centralisée, il disparaît, perdu, pour ainsi dire, dans la foule des paysans et des petits bourgeois. La lutte contre le capital sous sa forme moderne et parfaite, à son degré éminent, la lutte du salarié industriel contre le bourgeois industriel est en France un phénomène partiel. Les journées de Février pouvaient d’autant moins donner à la révolution un caractère national, que la lutte contre les modes d’exploitation inférieurs du capital, la lutte du paysan contre l’usure et l’hypothèque, du petit bourgeois contre le grand commerçant, le banquier, le fabricant, bref la lutte contre la banqueroute, disparaissait dans le soulèvement contre l’aristocratie financière. On s’explique dès lors facilement que le prolétariat dût incliner le drapeau rouge devant le drapeau tricolore, quand il tenta de faire prévaloir son intérêt à côté de celui des bourgeois, au lieu de le présenter comme l’intérêt révolutionnaire de la société elle-même. Pour que les ouvriers français pussent faire un seul pas, pussent toucher à un cheveu de la bourgeoisie, il fallait d’abord que le cours de la révolution eût lancé la masse intermédiaire, placée entre le prolétariat et la bourgeoisie, contre cet ordre même, l’eût soulevée contre la domination du capital, forcée à se joindre à son avant-garde, aux prolétaires. L’épouvantable défaite de Juin devait être le prix de cette victoire ouvrière.\par
La commission du Luxembourg, cette création des ouvriers parisiens, a eu le mérite de trahir, du haut d’une tribune européenne, le secret de la révolution du xixe siècle : \emph{l’émancipation du prolétariat}. Le \emph{Moniteur} s’emportait furieusement quand il se voyait obligé de répandre les « fantaisies sauvages », jusqu’alors ensevelies dans les écrits apocryphes des socialistes. Ces fantaisies ne venaient frapper l’oreille de la bourgeoisie que de loin en loin, semblables à des bruits lointains, moitié effrayants, moitié ridicules. L’Europe se réveilla brusquement, surprise dans son assoupissement bourgeois. Dans l’esprit des prolétaires, qui confondaient l’aristocratie de la finance avec la bourgeoisie ; dans l’imagination des républicains honnêtes qui doutaient de l’existence des classes ou y voyaient tout au plus une conséquence de la monarchie constitutionnelle ; à en croire les discours hypocrites de cette partie de la bourgeoisie qui avait été jusqu’alors exclue du pouvoir, la \emph{domination de la bourgeoisie} avait disparu avec l’avènement de la République. Tous les royalistes se transformaient en républicains ; tous les millionnaires de Paris en travailleurs. Le mot qui traduisait cette suppression imaginaire de la bourgeoisie était la \emph{fraternité\footnote{En français dans le texte}}. Cette abstraction sentimentale des antagonismes de classe, ce doux équilibre des intérêts contradictoires des classes, cette superbe fantaisie s’élevant au-dessus de la lutte des classes, la \emph{fraternité}, en un mot, tel était l’axiome favori de la révolution de Février. Les classes n’étaient séparées que par un \emph{malentendu}, et, le 24 février, Lamartine baptisa le gouvernement provisoire : \emph{un gouvernement qui suspend ce malentendu terrible qui existe entre les différentes classes\footnote{En français dans le texte}}. Le prolétariat parisien se grisa de cette généreuse ivresse.\par
Le gouvernement provisoire, de son côté, une fois placé dans la nécessité de proclamer la République, fit tout pour la rendre acceptable à la bourgeoisie et aux provinces. La Terreur sanglante de la première République fut désavouée par l’abolition de la peine de mort en matière politique. La presse fut ouverte à toutes les opinions. L’armée, les tribunaux, l’administration restèrent, à peu d’exceptions près, entre les mains des anciens dignitaires. On ne demanda de compte à aucun des grands coupables de la monarchie de Juillet. Les républicains bourgeois du \emph{National} s’amusèrent à changer les noms et les costumes de la monarchie contre ceux de l’ancienne République. Pour eux, la République n’était qu’un nouveau déguisement de l’ancienne société bourgeoise. La jeune République trouvait son principal intérêt à n’intimider personne, ou plutôt à n’épouvanter personne. Par sa souplesse, sa condescendance, par sa faiblesse même, elle cherchait à ménager son existence et à désarmer l’opposition. On annonça hautement aux classes privilégiées de l’intérieur, aux puissances despotiques de l’extérieur que la République était de nature pacifique. Vivre et laisser vivre, telle était sa devise. De plus, peu après la révolution de Février, toutes les nations, les Allemands, les Polonais, les Autrichiens, les Hongrois se révoltèrent, chacune poussée par sa situation propre. L’Angleterre et la Russie, la première en proie elle-même à l’agitation, la seconde intimidée, n’étaient pas prêtes. La République ne trouva donc pas devant elle une \emph{nation} ennemie. Il ne se produisit aucune de ces grandes complications étrangères qui auraient pu exciter l’énergie, précipiter le cours de la révolution, aiguillonner le gouvernement provisoire, ou le jeter par dessus bord. Le prolétariat parisien reconnaissait dans la République sa propre créature. Il acclama naturellement tout acte de nature à faciliter l’introduction de ce gouvernement dans la société bourgeoise. Il se laissa transformer par Caussidière en une police chargée de protéger la propriété à Paris, et permit à Louis Blanc de régler les contestations de salaire s’élevant entre les ouvriers et leurs maîtres. Il mettait son « point d’honneur\footnote{En français dans le texte} » à ne pas entamer, sous les yeux de l’Europe, la réputation bourgeoise de la République.\par
La République ne rencontra de résistance ni à l’extérieur ni à l’intérieur. C’est ce qui la désarma. La tâche ne consistait plus à transformer révolutionnairement le monde ; elle était réduite à s’adapter aux conditions de la société bourgeoise. Le gouvernement provisoire s’y employa avec fanatisme. Les \emph{mesures financières} en témoignent de la façon la plus expresse.\par
Le \emph{crédit public} et le \emph{crédit privé} étaient naturellement ébranlés. Le crédit public se fonde sur une certaine confiance. On compte que l’État se laissera exploiter par les Juifs de la finance. Mais l’ancien État avait disparu et la Révolution avait été dirigée surtout contre l’aristocratie financière. Les troubles commerciaux de la dernière crise que nous venons de subir ne s’étaient pas encore déchaînés. Pourtant les banqueroutes succédaient aux banqueroutes.\par
Le \emph{crédit privé} était paralysé, la circulation arrêtée, la production en suspens, avant que n’éclatât la révolution de Février. La crise révolutionnaire exagéra la crise commerciale. Le crédit privé repose sur la conviction que la production bourgeoise et l’ensemble de ses rapports, que l’ordre bourgeois resteront intacts, sont intangibles. Quel ne dût pas être l’effet d’une révolution qui mettait en question le fondement de la production bourgeoise, l’esclavage économique du prolétariat, qui dressait en face de la Bourse le sphinx du Luxembourg ? Le relèvement du prolétariat, c’est l’anéantissement du crédit bourgeois ; c’est en effet la suppression de la production bourgeoise, de l’ordre bourgeois. Le crédit public et le crédit privé sont le thermomètre économique auquel on peut mesurer l’intensité d’une révolution. \emph{Dans la mesure où ils baissent l’un et l’autre, se relèvent la flamme et l’énergie révolutionnaires}.\par
Le gouvernement provisoire voulait dépouiller la République de ce qu’elle avait d’anti-bourgeois. Il devait donc, en premier lieu, tenter d’assurer la \emph{valeur d’échange} de cette nouvelle forme d’État, établir son \emph{cours} à la Bourse. Le crédit privé se releva avec le prix courant auquel la Bourse estima la République.\par
Pour écarter jusqu’au \emph{soupçon} que la République ne voulait ou ne pouvait satisfaire aux obligations contractées par la monarchie, pour faire régner la confiance en la moralité bourgeoise, en la solvabilité de la République, le gouvernement provisoire eut recours à une fanfaronnade aussi enfantine que dépourvue de dignité. Il paya aux créanciers de l’État les intérêts des 5 0/0, 4 1/2 0/0, et 4 0/0 \emph{avant} l’échéance légale. L’aplomb bourgeois, le sentiment de leur valeur se réveillèrent soudain chez les capitalistes, quand ils s’aperçurent de la hâte avec laquelle on achetait leur confiance.\par
L’embarras financier où se trouvait le gouvernement provisoire ne fut naturellement pas diminué par ce coup de théâtre. Cet artifice le privait, au contraire, de l’argent comptant disponible. La crise financière ne pouvait se dissimuler plus longtemps et les \emph{petits bourgeois}, les \emph{employés}, les \emph{ouvriers}, firent les frais de l’agréable surprise ménagée aux créanciers de l’État.\par
On déclara que les \emph{livrets de caisse d’épargne} dont le montant dépassait 100 francs ne seraient plus remboursables en argent. Les sommes déposées dans les caisses furent confisquées et remplacées par une créance remboursable sur l’État. La \emph{petite bourgeoisie}, déjà gênée, s’aigrit contre la République. Elle se vit forcée d’aller à la Bourse vendre les bons d’Etat reçus à la place des livrets. C’était retomber directement entre les mains des Juifs de la Bourse contre lesquels la révolution de Février avait été faite.\par
L’aristocratie financière, toute puissante sous la monarchie de Juillet, avait sa Haute église dans la \emph{Banque}. De même que la Bourse régit le crédit public, la banque gouverne le crédit commercial.\par
Menacée directement par la révolution de Février non seulement dans sa suprématie, mais dans son existence même, la Banque chercha tout d’abord à discréditer la République en généralisant la pénurie du crédit. Elle le refusa brusquement aux banquiers, aux fabricants, aux marchands. Cette manœuvre ne produisit pas une contre-révolution immédiate. Elle se retourna nécessairement contre la Banque. Les capitalistes retirèrent l’argent qu’ils avaient déposé dans ses caves. Les possesseurs de billets se précipitèrent à sa caisse pour se les faire rembourser en or et en argent.\par
Sans intervention violente, en usant de la voie légale, le gouvernement provisoire pouvait forcer la Banque à la \emph{banqueroute}. Il lui suffisait d’adopter une attitude passive et d’abandonner la Banque à son sort. La \emph{banqueroute de la Banque}, c’était le déluge capable de débarrasser en un clin d’œil le sol français de l’aristocratie financière, de délivrer la République de son ennemi le plus puissant et le plus dangereux, de renverser le piédestal d’or sur lequel s’était élevé la monarchie de Juillet. La Banque une fois en faillite, la création d’une banque nationale, la subordination du crédit national au contrôle de la nation, s’imposaient. La bourgeoisie elle-même aurait vu dans cette mesure un dernier moyen de salut, extrême, il est vrai, et désespéré.\par
Le gouvernement provisoire, au contraire, donna aux billets de la Banque le \emph{cours forcé}. Il fit mieux. Il transforma toutes les banques provinciales en succursales de la « Banque de France » et permit à celle-ci de jeter son réseau sur tout le pays. Le gouvernement enfin engagea auprès d’elle les \emph{forêts domaniales}, pour garantir un emprunt qu’il contracta envers elle. La révolution de Février affermit donc la bancocratie qu’elle aurait dû renverser.\par
Cependant le gouvernement provisoire se débattait contre le cauchemar d’un déficit croissant. C’est en vain qu’il mendiait des sacrifices patriotiques. Seuls, les ouvriers lui accordèrent quelques aumônes. Il fallait se résigner à un moyen héroïque, la promulgation d’un \emph{nouvel impôt}. Mais qui donc imposer ? Les loups de la Bourse, les rois de la Banque, les créanciers de l’État, les rentiers, les industriels. Ce n’était pas un moyen de recommander la République aux bourgeois. Cela revenait à compromettre d’un côté le crédit public et le crédit commercial, alors que d’autre part on cherchait à se les concilier au prix des plus grands sacrifices, des plus grandes humiliations. Il fallait cependant que quelqu’un desserrât les cordons de sa bourse. Qui fut sacrifié au crédit bourgeois ? Ce fut « Jacques Bonhomme », le \emph{paysan}.\par
Le gouvernement provisoire établit un impôt additionnel de 45 centimes par franc sur les quatre impôts directs. La presse gouvernementale raconta au prolétariat parisien que cet impôt retomberait heureusement sur la grande propriété, sur les propriétaires du milliard octroyé par la Restauration. En réalité, il atteignit surtout \emph{la classe paysanne}, c’est-à-dire la grande majorité de la nation française. \emph{Les paysans durent payer les frais de la révolution de Février}. La contre-révolution trouva chez eux son principal contingent. L’impôt des 45 centimes devenait une question vitale pour le paysan français. Il en fit une question vitale pour la République. Dès ce moment, le paysan vit dans la \emph{République} l’impôt des 45 centimes et le prolétariat parisien était le dissipateur qui se donnait du bon temps à ses frais.\par
Tandis que la Révolution de 1789 avait commencé par délivrer le paysan du fardeau de la féodalité, la révolution de 1848 se fit connaître par un impôt pesant sur la population campagnarde, et cela pour ne pas inquiéter le capital et pour maintenir en marche le mécanisme de l’Etat capitaliste.\par
Le gouvernement provisoire n’avait qu’\emph{un seul} moyen d’écarter toutes ces difficultés et de faire sortir l’État de l’ancienne ornière. \emph{Il fallait déclarer la banqueroute de l’État}. On se souvint avec quelle indignation vertueuse Ledru-Rollin se prononça au sein de l’Assemblée nationale contre cette proposition du boursier juif Fould, actuellement ministre des Finances. Fould lui présentait cependant une pomme de l’arbre de la science.\par
En reconnaissait les traites que la vieille société bourgeoise avait tirées sur l’État, le gouvernement provisoire s’était rendu à discrétion. Au lieu de rester le créancier menaçant de la bourgeoisie, prêt à encaisser les dettes contractées depuis de nombreuses années envers la révolution, le gouvernement provisoire était devenu un débiteur besogneux. Il dut consolider les rapports bourgeois ébranlés, remplir des engagements dont ces rapports seuls permettent l’exécution. Le crédit devint une condition de son existence. Les concessions, les promesses faites au prolétariat se changèrent en autant d’\emph{entraves} qu’il \emph{fallait} briser. L’émancipation des travailleurs – même à l’état de simple \emph{mot} – était un danger que la République ne pouvait supporter. Le crédit repose sur la reconnaissance certaine et nette des rapports économiques existant entre les classes. L’émancipation des travailleurs protestait en permanence contre cette restauration du crédit. Il fallait donc en \emph{finir avec les ouvriers}.\par
La révolution de Février avait chassé l’armée de Paris. La garde nationale, c’est-à-dire la bourgeoisie à ses différents états, constituait la seule force. Mais elle se sentait inférieure au prolétariat. D’ailleurs, elle était obligée, malgré son extrême répugnance, malgré tous les obstacles qu’elle suscitait, d’ouvrir ses rangs peu à peu, et, partiellement, d’admettre dans son sein des prolétaires armés. Une seule issue restait ouverte : \emph{opposer une partie des prolétaires au reste du prolétariat}.\par
Dans ce but, le gouvernement provisoire forma vingt-quatre bataillons de \emph{gardes mobiles}, de mille hommes chacun et composés de jeunes gens de quinze à vingt ans. Ils appartenaient pour la plus grande partie à canaille\footnote{Lumpenproletariat.} qui, dans toutes les grandes villes, constitue une masse nettement distincte du prolétariat industriel. C’est dans ses rangs que se recrutent les voleurs et les criminels de toute espèce, vivant des déchets de la société, individus sans travail déterminé, rôdeurs, « gens sans feu et sans aveu\footnote{En français dans le texte} », variant avec le degré de développement de la nation à laquelle ils appartiennent et ne démentant jamais le caractère des lazzaroni. L’âge encore jeune auquel le gouvernement les recruta les destinait particulièrement ou bien aux héroïsmes les plus élevés et aux sacrifices les plus exaltés, ou bien au banditisme le plus vulgaire et à la corruption la plus honteuse. Le gouvernement provisoire leur allouait 1 fr. 50 par jour, c’est-à-dire les achetait à ce prix. Il leur donna un uniforme particulier, c’est-à-dire qui se distinguât extérieurement de la blouse. Leurs chefs se composaient pour une partie d’officiers de l’armée permanente, pour l’autre de fils de la bourgeoisie, élus par les gardes. Leurs rodomontades, l’amour de la patrie et le dévouement à la République avaient plu.\par
Ainsi, en face du prolétariat parisien se dressait une armée tirée de son propre milieu, forte de 24 000 hommes doués de la vigueur et de l’exaltation de la jeunesse. Le prolétariat salua de ses \emph{vivats} la garde mobile au cours des marches qu’elle exécuta dans Paris. Il reconnaissait en elle son avant-garde, ceux qui combattraient devant lui sur les barricades. Il la regarda comme la garde prolétarienne en opposition avec la garde nationale bourgeoise. Son erreur était pardonnable.\par
Le gouvernement résolut de rassembler autour de lui, outre la garde mobile, une seconde armée ouvrière, une seconde armée industrielle. Des centaines de mille de travailleurs, jetés sur le pavé par la crise et par la révolution furent enrôlés par le ministre Marie dans ce que l’on a appelé les ateliers nationaux. Ce nom pompeux signifiait seulement que l’on employait les ouvriers à \emph{des travaux de terrassement}, ennuyeux, monotones et improductifs pour un salaire journalier de 23 sous. C’étaient les \emph{work-houses anglais en plein air} et rien de plus. Le gouvernement provisoire pensait avoir formé ainsi \emph{une seconde armée prolétarienne dirigée contre les ouvriers eux-mêmes}. Pour cette fois, la bourgeoisie se trompait sur les ateliers nationaux comme les ouvriers se trompaient sur la garde mobile. La bourgeoisie avait créé \emph{l’armée de l’émeute}.\par
Mais on avait ainsi atteint \emph{un} but.\par
Les \emph{ateliers nationaux}, c’était là le nom des entreprises nationales que Louis Blanc préconisait au Luxembourg. Les ateliers de Marie, inventés pour faire directement échec à la commission du travail, donnèrent lieu, par la similitude du titre, à une intrigue, à un malentendu digne de la comédie espagnole. En sous-main, le gouvernement provisoire répandit lui-même le bruit que ces ateliers nationaux étaient de l’invention de Louis Blanc. La chose parut d’autant plus croyable, que le prophète des ateliers nationaux était lui-même membre du gouvernement provisoire. Grâce à la confusion mi-naïve, mi-préméditée de la bourgeoisie parisienne, grâce à l’opinion où étaient artificiellement entretenues la France et l’Europe, ces \emph{work-houses} passaient pour la première réalisation du socialisme, qui fut ainsi, et avec eux, cloué au pilori.\par
Sinon par leur contenu, du moins par leur titre, les ateliers nationaux donnaient un corps à la protestation du prolétariat contre l’industrie bourgeoise, contre le crédit bourgeois et contre la république bourgeoise. Toute la haine de la bourgeoisie retombait sur eux. Ils présentaient le point faible où elle pourrait diriger ses attaques quand elle se sentirait assez forte pour rompre avec les illusions de février. Tout le malaise, tout le mécontentement des \emph{petits bourgeois} se tourna simultanément contre ces ateliers nationaux, contre cette cible commune. Ils calculèrent avec une véritable fureur les sommes que les prolétaires fainéants engloutissaient alors qu’eux-mêmes voyaient leur sort devenir tous les jours plus insupportable. Une pension de l’État pour un travail illusoire, c’est là le socialisme ! grommelaient-ils en eux-mêmes. Ils cherchaient la cause de leur misère dans les ateliers nationaux, les déclamations du Luxembourg, les promenades des ouvriers dans Paris. Personne ne s’opposa aux prétendues menées des communistes avec autant de fanatisme que le petit bourgeois qui glissait irrémédiablement sur la pente de la faillite.\par
Dans ces premiers engagements de la bourgeoisie aux prises avec le prolétaire, tous les avantages, toutes les positions décisives, toutes les couches moyennes de la société étaient aux mains des bourgeois, alors que les flots de la révolution de Février battaient tout le continent. Chaque courrier apportait un nouveau bulletin révolutionnaire, tantôt d’Italie, tantôt d’Allemagne, tantôt des régions les plus éloignées du Sud-Est de l’Europe, entretenait l’agitation générale du peuple, lui donnait les témoignages continuels d’une victoire qu’il avait remportée.\par
Le 17 \emph{mars} et le 16 \emph{avril} furent les combats d’avant-postes de la grande guerre des classes que la république bourgeoise cherchait à dissimuler.\par
Le 17 \emph{mars} dévoila la situation ambiguë du prolétariat, et montra qu’elle ne laissait place à aucun acte décisif. La démonstration avait à l’origine pour but de remettre le gouvernement provisoire dans la voie de la Révolution, d’obtenir, si les circonstances s’y prêtaient, l’exclusion des membres bourgeois de ce gouvernement, d’exiger la prorogation de la date des élections à l’assemblée et à la garde nationale. Mais, le 16 mars, la bourgeoisie, représentée par cette garde, fit une démonstration hostile au gouvernement provisoire aux cris de : « À bas Ledru-Rollin ! » Elle marcha sur l’Hôtel de Ville. Le peuple se vit forcé de crier, le 17 mars : « Vive Ledru-Rollin ! Vive le gouvernement provisoire ! » Il était obligé de prendre, \emph{contre} la bourgeoisie, le parti de la République bourgeoise dont l’existence lui paraissait remise en question. Il affermit le gouvernement provisoire au lieu de se le soumettre. Le 17 mars aboutit à une scène mélodramatique, et si, ce jour-là encore, le prolétariat parisien fit voir son corps gigantesque, la bourgeoisie, au sein du gouvernement provisoire et en dehors de lui, était d’autant plus décidée à l’abattre.\par
Le 16 avril fut un malentendu machiné par le gouvernement provisoire avec le concours de la bourgeoisie. Les ouvriers s’étaient réunis en nombre au Champs-de-Mars et à l’hippodrome pour préparer l’élection de l’État-Major de la garde nationale. Soudain d’un bout de Paris à l’autre se répandit avec la rapidité de l’éclair le bruit que les ouvriers s’étaient assemblés en armes au Champ de Mars sous la conduite de Louis Blanc, de Blanqui, de Cabet et de Raspail pour marcher de là sur l’Hôtel de Ville, renverser le gouvernement provisoire et proclamer un gouvernement communiste. On battit la générale. – Ledru-Rollin, Marrast, Lamartine se disputèrent plus tard l’honneur de cette initiative ; en une heure 100 000 hommes sont sous les armes ; l’Hôtel de Ville est gardé sur tous les points par les gardes nationaux ; le cri de : « À bas les communistes ! à bas Louis Blanc, à bas Blanqui, à bas Raspail, à bas Cabet ! » gronde dans tout Paris et le gouvernement provisoire reçoit l’hommage d’une foule de délégations, toutes prêtes à sauver la patrie et la société. Quand les ouvriers paraissent devant l’Hôtel de Ville pour remettre au gouvernement provisoire le produit d’une collecte patriotique faite au Champ-de-Mars, ils apprennent à leur grande surprise que la bourgeoisie de Paris a battu leur fantôme en un combat imaginaire très prudemment ménagé. L’effrayant attentat du 16 mars fournit le prétexte du \emph{rappel de l’armée à Paris}, ce qui était le but de cette comédie grossière et fit naître l’occasion de démonstrations réactionnaires et fédéralistes en province.\par
Le 4 mai se réunit l’\emph{Assemblée Nationale}, issue \emph{d’élections directes et générales}. Le suffrage universel ne possédait pas la vertu magique que des républicains d’ancienne marque lui avaient attribuée. Pour eux, toute la France, au moins la majorité des Français étaient des \emph{citoyens} ayant les mêmes intérêts, le même jugement, etc.\par
C’était, chez eux, une conséquence de leur \emph{culte du peuple}. Les élections mirent en lumière, au lieu de leur \emph{peuple imaginaire, le peuple réel} ; elles désignèrent des représentants des classes dont il se compose. Nous avons vu pourquoi les paysans et les petits bourgeois devaient marcher au scrutin sous la conduite des bourgeois prêts à la lutte et des grands propriétaires fonciers enragés de restauration. Mais si le suffrage universel n’était pas la baguette magique que croyaient les braves républicains, il avait au moins l’éminent avantage de déchaîner la lutte des classes, d’éprouver rapidement les illusions et les désillusions des différentes couches moyennes de la société bourgeoise, de placer, d’un seul coup, à la tête de l’État, toutes les fractions de la classe des exploiteurs et de leur arracher ainsi leur masque trompeur. La bourgeoisie, avec son cens, ne laissait se compromettre que certains des siens, tenait les autres à l’écart, dans la coulisse, et entourait ceux-ci de l’auréole commune a l’opposition.\par
Dans l’Assemblée nationale constituante qui se réunit le 4 mai, les \emph{républicains bourgeois}, les républicains du \emph{National} avaient la haute main. Les légitimistes et les orléanistes n’osaient se présenter que sous le masque du républicanisme bourgeois. La lutte contre le prolétariat ne pouvait dès lors s’engager qu’au nom de la République.\par
La \emph{République}, c’est-à-dire la République reconnue par le peuple français, \emph{date du 4 mai} et non du 25 février. Elle n’était pas celle que le prolétariat parisien avait imposée au gouvernement provisoire, la République pourvue d’institutions sociales que rêvaient les combattants des barricades. La République proclamée par l’Assemblée nationale, la seule légitime, ne pouvait devenir une arme révolutionnaire dirigée contre l’ordre bourgeois ; elle était une reconstitution politique, la consolidation politique de la société bourgeoise : en un mot, c’était la \emph{République bourgeoise}. On le proclama à la tribune de l’Assemblée nationale ; toute la presse bourgeoise, républicaine ou non fit écho.\par
Nous avons vu que la République de février n’était et ne pouvait véritablement être qu’une République bourgeoise. Mais nous avons vu aussi que le gouvernement provisoire, sous la pression directe du prolétariat, avait été contraint de proclamer qu’elle était une République \emph{pourvue d’institutions sociales}. Le prolétariat parisien était encore incapable de dépasser la République bourgeoise autrement qu’\emph{en esprit, en imagination}. Chaque fois qu’il accomplissait un acte réel, il agissait au profit de cette République bourgeoise. Les engagements pris à son égard étaient devenus un danger insupportable pour la nouvelle République. Le gouvernement provisoire voyait son existence se passer uniquement en une lutte dirigée contre les revendications du prolétariat.\par
Au sein de l’Assemblée nationale, c’était la France entière qui appelait à sa barre le prolétariat parisien. Elle rompit aussitôt avec les illusions sociales qu’avait fait naître la révolution de Février. Elle proclama nettement la \emph{République bourgeoise}, rien que la République bourgeoise. Elle s’empressa d’exclure de la Commission exécutive qu’elle nomma les représentants du prolétariat : Louis Blanc et Albert. Elle repoussa le projet d’un ministère spécial du travail. Elle accueillit par une tempête approbative la déclaration du ministre Trélat : il s’agissait uniquement de \emph{rendre au travail ses anciennes conditions}.\par
Mais tout cela ne suffisait pas. La République de Février avait été conquise par le prolétariat ; la bourgeoisie l’avait seulement favorisé par son attitude passive. Les prolétaires se considéraient avec justice comme les vainqueurs de Février ; ils avaient les prétentions orgueilleuses des vainqueurs. Il fallait qu’ils fussent vaincus dans la rue, il fallait qu’on leur montrât que leur défaite était inévitable, dès qu’ils combattraient non plus d’\emph{accord} avec la bourgeoisie, mais \emph{contre} elle. Les concessions socialistes de la République de Février supposaient que le prolétariat s’était uni à la bourgeoisie pour livrer bataille à la royauté. Un second combat était nécessaire pour dégager la République des concessions socialistes, pour inaugurer le règne officiel de la \emph{République bourgeoise}. C’est les armes à la main que la bourgeoisie devait repousser les revendications du prolétariat. La naissance véritable de la République bourgeoise date non de la \emph{victoire de Février}, mais de la \emph{défaite de Juin}.\par
Le prolétariat précipita la décision. Le 15 mai, il envahit l’Assemblée nationale, cherchant sans succès à reconquérir son influence révolutionnaire. Il ne réussit qu’à livrer aux cachots de la bourgeoisie ses chefs énergiques : \emph{Il faut en finir}\footnote{En français dans le texte} ! Ce cri trahit la détermination de l’Assemblée nationale à obliger le prolétariat à un combat décisif. La commission exécutive publia une série de décrets provocants, par exemple, le décret interdisant les attroupements. Les ouvriers furent directement défiés, insultés, persiflés du haut de la tribune de l’Assemblée nationale constituante. Mais, comme nous l’avons vu, les ateliers nationaux offraient un but à l’attaque proprement dite. L’Assemblée constituante donna à la commission exécutive, qui n’attendait que cela, l’ordre exprès d’attribuer à ses propres projets la valeur d’un mandat de l’Assemblée nationale.\par
La commission se mit donc à l’ouvrage. Elle rendit plus difficile l’accès des ateliers nationaux. Elle transforma le salaire à la journée en salaire aux pièces, bannit en Sologne les ouvriers nés à Paris, sous prétexte de leur faire exécuter des travaux de terrassement. Ces terrassements n’étaient qu’une formule de rhétorique dont on ornait l’expulsion. De retour dans leurs foyers, les ouvriers désillusionnés l’apprirent à leurs camarades. Enfin, le 21 juin, parut un décret au \emph{Moniteur}, ordonnant l’expulsion brutale des ouvriers non mariés hors des ateliers nationaux ou leur incorporation dans l’armée.\par
Les ouvriers n’avaient plus le choix, il ne leur restait plus qu’à mourir de faim ou à se révolter. Le 22 juin, ils répondirent au décret par la formidable insurrection où se livra la première grande bataille entre les deux classes qui partagent la société moderne. La lutte devait aboutir au maintien ou à l’anéantissement de l’ordre \emph{bourgeois}. Le voile qui cachait la République se déchira.\par
On sait que les ouvriers avec un courage et un génie sans exemple, sans chefs, sans plan commun, sans moyens de défense et manquant d’armes pour la plupart tinrent en échec pendant cinq jours l’armée, la garde mobile, la garde nationale de Paris et la garde nationale des provinces accourue dans la capitale. On sait que la bourgeoisie se dédommagea d’une peur mortelle par une brutalité inouïe et massacra plus de trois mille prisonniers.\par
Les représentants officiels de la démocratie française étaient tellement renfermés dans l’idéologie républicaine qu’ils ne commencèrent à soupçonner le sens des combats de juin que quelques semaines plus tard. La poudre qui assassinait leur république fantastique les avait rendus sourds.\par
Le lecteur nous permettra, pour traduire l’impression première que la nouvelle de la défaite de juin produisit sur nous, de nous servir des termes mêmes de la \emph{Neue rheinische Zeitung}.\par
« Ce qui restait officiellement de la révolution de Février, la commission exécutive, s’est évanoui comme une ombre devant la gravité des circonstances. Les feux d’artifice de Lamartine sont devenus les fusées de Cavaignac. L’expression réelle, sincère, prosaïque de la fraternité entre les classes opposées dont l’une exploite l’autre, de cette fraternité proclamée en février, inscrite en grandes lettres au front de Paris, sur chaque prison, sur chaque caserne, cette fraternité – c’est la guerre civile, la guerre civile sous sa forme la plus épouvantable, la guerre entre le travail et le capital. Cette fraternité brillait à toutes les fenêtres, le soir du 25 juin, quand le Paris de la bourgeoisie illuminait alors que le Paris du prolétariat, incendié et sanglant, gémissait. La fraternité dura juste aussi longtemps que l’accord entre l’intérêt de la bourgeoisie et celui du prolétariat. – Des pédants de la vieille tradition révolutionnaire de 1793 ; des auteurs de systèmes socialistes, mendiant pour le peuple auprès de la bourgeoisie, et auxquels on permit de longs discours, qu’on laissa se compromettre tant qu’il fallut endormir le lion populaire ; des républicains qui désiraient l’ancien ordre bourgeois, mais sans tête couronnée ; l’opposition dynastique à laquelle le sort accorda à la place d’un changement de ministère la chute d’une dynastie ; des légitimistes qui tenaient moins à jeter leur livrée qu’à en modifier la coupe, tels étaient les alliés avec lesquels le prolétariat fit Février. – La Révolution de Février était la \emph{belle} révolution, révolution ayant la sympathie générale parce que les antagonismes qui l’avaient armée contre la royauté n’étaient pas encore développés et sommeillaient en bonne intelligence les uns à côté des autres, parce que la guerre sociale qu’elle menait après elle n’avait encore qu’une réalité nébuleuse, la valeur d’une phrase, d’un mot. La \emph{Révolution de Juin} est la révolution haïssable, la révolution répugnante, parce que la chose prend la place du mot, parce que la République découvre la face du monstre en brisant la couronne qui le couvrait et le cachait. – \emph{Ordre} ! tel était le cri de guerre de Guizot. \emph{Ordre} ! s’écriait le Guizotin Sébastiani quand Varsovie devint russe. \emph{Ordre} ! crie Cavaignac, écho brutal de l’Assemblée nationale et de la bourgeoisie républicaine. \emph{Ordre} ! grondèrent ses cartouches en déchirant les entrailles du prolétariat. Depuis 1789, aucune des nombreuses révolutions de la bourgeoisie française n’avaient attenté à l’ordre, car elles laissaient subsister la domination d’une classe, l’esclavage de l’ouvrier, \emph{l’ordre bourgeois}, en un mot, si souvent qu’ait pu changer la forme politique de cette domination et de cet esclavage. Juin a touché à cet ordre. Malheur à juin ! »\par
{\raggedleft \noindent (\emph{Neue rheinische Zeitung}, 29 juin 1848.)\par}
\noindent Malheur à juin ! répète l’écho de l’Europe.\par
Le prolétariat parisien fut \emph{contraint} à l’insurrection de Juin par la bourgeoisie et sa condamnation était dès lors assurée. Ses besoins présents, immédiats ne l’avaient pas poussé à renverser violemment la bourgeoisie. Il n’était pas non plus assez développé pour entreprendre cette œuvre. Il fallut que le \emph{Moniteur} lui déclarât que le temps était passé où la République était d’humeur à s’incliner devant ses illusions. Seule la défaite put le persuader de la vérité : elle lui apprit que la plus mince amélioration de son sort \emph{dans la société bourgeoise} reste une \emph{utopie}, utopie qui se change en crime dès qu’on s’avise de la réaliser. Au lieu des revendications, excessives de forme, mesquines de contenu, bourgeoises encore, dont il voulait arracher la concession à la République de Février, s’éleva un cri de guerre audacieux, révolutionnaire : \emph{À bas la bourgeoisie ! Dictature de la classe ouvrière} !\par
Le prolétariat, en faisant de son champ funéraire le berceau de sa \emph{République bourgeoise} la força à revêtir sa forme pure. Elle fut l’État dont le but avoué est de perpétuer le règne du capital et l’esclavage du travail. La domination de la bourgeoisie devait se convertir aussitôt en un \emph{terrorisme bourgeois}, frappant l’ennemi couvert de cicatrices, implacable, invincible, invincible parce que l’existence du prolétariat est la condition de l’existence de la bourgeoisie. Le prolétariat restait, pour le moment, à l’écart de la scène ; la dictature de la bourgeoisie était officiellement reconnue. Les couches moyennes de la société allaient se rallier de plus en plus autour du prolétariat à mesure que leur situation deviendrait plus insupportable et que s’aiguiserait leur antagonisme avec la bourgeoisie. Les petits bourgeois voyaient autrefois la cause de leur misère dans les succès des prolétaires. Maintenant, il leur fallait la chercher dans leur défaite.\par
L’insurrection de Juin éleva, sur tout le continent, la bourgeoisie à la conscience de soi-même. Elle la fit nouer alliance avec la royauté féodale contre le peuple. Quelle fut la première victime de cette union ? La bourgeoisie continentale elle-même. La défaite de Juin l’empêcha d’assurer sa suprématie. Elle lui interdit de laisser le peuple moitié satisfait, moitié mécontent, au seuil de la révolution.\par
Enfin la défaite de Juin trahit un secret aux puissances despotiques de l’Europe. Elles surent, dès lors, que la France, en toutes circonstances devait maintenir la paix à l’extérieur pour pouvoir mener la guerre civile à l’intérieur. Aussi les nations qui avaient commencé à lutter pour leur indépendance furent-elles abandonnées à la souveraineté de la Russie, de l’Autriche et de la Prusse. Mais en même temps le destin de ces révolutions nationales fut subordonné au sort de la révolution prolétarienne. L’indépendance, même apparente ne put plus se séparer du grand bouleversement social. Ni le Hongrois, ni le Polonais, ni l’Italien ne pouvaient être libres tant que l’ouvrier restait esclave.\par
Enfin, depuis la victoire de la Sainte-Alliance, l’Europe a pris un aspect tel que tout nouveau soulèvement de prolétariat français devient immédiatement le signal d’une \emph{guerre universelle}. La nouvelle révolution française est obligée de quitter immédiatement le domaine national et de \emph{conquérir le champ de bataille européen}, le seul où la révolution sociale du xixe siècle puisse livrer l’engagement décisif.\par
C’est la défaite de Juin qui la première a créé toutes les conditions nécessaires pour que la France puisse prendre l’\emph{initiative} d’une révolution européenne. C’est parce qu’il a été plongé dans le sang des \emph{insurgés de Juin} que le drapeau tricolore a pu devenir le drapeau de la révolution européenne, – \emph{le drapeau rouge}.\par
Pour nous, nous crions : \emph{La Révolution est morte ! – Vive la Révolution} !
\chapterclose


\chapteropen

\chapter[{II. De juin 1848 au 13 juin 1849}]{II. De juin 1848 au 13 juin 1849}
\renewcommand{\leftmark}{II. De juin 1848 au 13 juin 1849}


\chaptercont
\noindent Le 25 février 1848 avait octroyé la \emph{République} à la France, le 25 juin lui imposa la \emph{Révolution}. Après juin, la révolution signifiait : \emph{bouleversement de la société bourgeoise}, avant février, elle voulait dire : \emph{bouleversement de la forme politique}.\par
La bataille de juin avait été conduite par la fraction \emph{républicaine} de la bourgeoisie. Après la victoire, c’est à cette fraction que revint naturellement le pouvoir public. L’état de siège mettait sans résistance Paris baillonné à ses pieds. Sur les provinces pesait un état de siège moral. L’arrogance, les menaces brutales des bourgeois vainqueurs, un déchaînement de l’amour fanatique des paysans pour la propriété y régnaient. Il n’y avait donc à craindre aucun danger venant \emph{d’en-bas} !\par
L’influence politique des républicains démocrates, des républicains au sens petit bourgeois s’évanouit en même temps que la puissance révolutionnaire. Les démocrates avaient été représentés dans la commission exécutive par Ledru-Rollin, dans l’assemblée nationale constituante par le parti de la Montagne, dans la presse par la « Réforme ». Le 16 avril, ils avaient conspiré de concert avec les bourgeois ; dans les journées de juin, ils avaient combattu côte à côte avec eux. En agir ainsi, c’était détruire la force qui faisait une puissance de leur parti. La petite bourgeoisie ne peut garder une attitude révolutionnaire en face de la bourgeoisie que quand le prolétariat est derrière elle. Elle fut remerciée. L’alliance illusoire, choquante, conclue au moment de la constitution du gouvernement provisoire et de la commission exécutive, fut ouvertement désavouée par les républicains bourgeois. D’alliés que l’on dédaignait, que l’on repoussait même, les petits bourgeois descendirent au rang de gardes du corps des républicains tricolores. Ils ne pouvaient arracher à ces derniers aucune concession, mais ils devaient cependant soutenir leur pouvoir toutes les fois que les républicains tricolores ou que la république même semblait remise en question par les bourgeois anti-républicains. Ceux-ci enfin, orléanistes et légitimistes, se trouvèrent dès l’origine en minorité dans l’assemblée. Avant les journées de juin, il n’osaient même agir que sous le masque du républicanisme bourgeois. Pendant un moment, après la victoire remportée sur les insurgés, toute la France bourgeoise salua en Cavaignac son sauveur. Et quand, après les journées, la presse antirépublicaine recouvra son indépendance, la dictature militaire et l’état de siège proclamé à Paris, ne lui permirent de montrer les cornes qu’avec beaucoup de prudence et de circonspection.\par
Depuis 1830, les \emph{républicains} bourgeois, leurs écrivains et leurs orateurs, leurs gens capables, leurs ambitieux, leurs députés, généraux, banquiers avocats, étaient groupés autour d’un journal de Paris, le \emph{National}. Cette feuille possédait en province ses organes affiliés. La coterie du \emph{National}, c’était toute \emph{la dynastie de la république tricolore}. Elle s’empara aussitôt de toutes les charges publiques, des ministères, de la préfecture de police, de la direction des postes, des préfectures et des grades les plus élevés vacants dans les départements et dans l’armée. À la tête du pouvoir exécutif se tenait leur général Cavaignac ; leur rédacteur en chef Marrast était le président permanent de l’assemblée nationale. En même temps, ce dernier jouait au maître des cérémonies et faisait les honneurs de la république honnête.\par
Dos écrivains français, révolutionnaires cependant, ont, par une sorte de pudeur et pour épargner la tradition républicaine, accrédité l’erreur que les royalistes l’emportaient dans l’Assemblée constituante. C’est le contraire qui est vrai. À partir des journées de juin, cette assemblée \emph{représenta exclusivement le républicanisme bourgeois}. La chose devint d’autant plus apparente que l’influence des républicains tricolores déclinait en dehors de l’assemblée. S’il s’agissait de défendre la \emph{forme} de la république bourgeoise, ils disposaient des voix des républicains démocrates. S’il s’agissait du \emph{contenu}, leur langage même ne se distinguait pas de celui des bourgeois royalistes. Les intérêts de la bourgeoisie, les conditions matérielles de sa suprématie et de son exploitation de classe forment, en effet, le contenu de la république bourgeoise.\par
Ce n’est donc pas le royalisme, mais le républicanisme bourgeois que reflétaient l’existence et les actes de cette Constituante qui ne mourut pas, ne fut pas assassinée, mais tomba en pourriture.\par
Pendant tout le temps que dura la domination de l’assemblée et qu’elle joua sur la scène publique le rôle principal, on sacrifiait sans interruption des victimes dans la coulisse. – Les conseils de guerre condamnaient sans relâche les insurgés de juin faits prisonniers. On déportait sans jugement. La Constituante avait le tact d’avouer que les insurgés de juin n’étaient pas des criminels qu’elle jugeait, mais des ennemis qu’elle écrasait.\par
Le premier acte de l’Assemblée nationale constituante fut la constitution d’une \emph{commission d’enquête}, chargée d’instruire sur les événements de juin, sur le 15 mai et sur la participation à ces journées des chefs des partis socialistes et démocratiques. L’instruction était directement dirigée contre Louis Blanc, Ledru-Rollin et Caussidière. Les républicains bourgeois brûlaient d’impatience de se débarrasser de ces rivaux. Ils ne pouvaient remettre en de meilleures mains la satisfaction de leur rancune qu’en celles d’Odilon Barrot, l’ancien chef de l’opposition dynastique, le libéralisme fait homme, la « nullité grave\footnote{En français dans le texte} », la platitude fondamentale. Il n’avait, en effet, pas seulement une dynastie à venger, mais il avait encore à demander compte aux révolutionnaires de la présidence du ministère qu’ils avaient renversé. Son inflexibilité était assurée. Ce Barrot fut nommé président de la commission d’enquête. Il instruisit contre la révolution de Février un procès complet qui se résume ainsi : 17 mars, \emph{manifestation}, 16 avril, \emph{complot}, 15 mai, \emph{attentat}, 23 juin, \emph{guerre civile} ! Pourquoi ne poussait-il pas ses savantes recherches criminelles jusqu’au 24 février ? Le \emph{Journal des Débats} répondit : le 24 février, c’est la fondation de Rome. L’origine des États s’enveloppe d’un mythe auquel on doit croire, mais qu’on ne doit pas discuter. Louis Blanc et Caussidière furent livrés à la justice. L’Assemblée nationale continuait l’œuvre de sa propre épuration qu’elle avait entreprise le 25 mai.\par
Le projet l’impôt sur le capital, sous forme d’impôt hypothécaire, élaboré par le gouvernement provisoire et repris par Goudchaux, fut rejeté par l’Assemblée constituante. La loi qui limitait à dix heures la journée de travail fut abrogée ; l’emprisonnement pour dettes rétabli ; une grande partie de la population française, celle qui ne sait ni lire ni écrire, fut privée de l’admission au jury. On rétablit le cautionnement des journaux. Le droit d’association fut limité.\par
Mais dans leur hâte à restituer aux anciens rapports bourgeois leur solidité ancienne, à effacer toutes les traces laissées par le flot révolutionnaire, les républicains bourgeois rencontrèrent un obstacle qui les exposa à un danger inattendu.\par
Aux journées de Juin, personne n’avait plus fanatiquement combattu pour la sauvegarde de la propriété et le rétablissement du crédit que les petits bourgeois parisiens, cafetiers, restaurateurs, « marchands de vins », petits commerçants, boutiquiers, artisans, etc. La boutique s’était soulevée et avait fait front contre la barricade pour rétablir la circulation qui mène de la rue à la boutique. Mais derrière la barricade se trouvaient les clients et les débiteurs, devant elle les créanciers de la boutique. Quand les barricades eurent été renversées et les ouvriers écrasés, quand les petits bourgeois se précipitèrent vers leurs boutiques, ils en trouvèrent l’entrée barricadée par un sauveur de la propriété, un agent officiel du crédit qui leur présentait des papiers menaçants : billet échu ! terme de location échu ! créance échue ! boutique déchue ! boutiquier déchu !\par
\emph{Sauvegarde de la propriété} ! Mais la maison que les petits bourgeois habitaient n’était pas leur propriété ; la boutique qu’ils gardaient n’était pas leur propriété ; les marchandises dont ils trafiquaient n’étaient pas leur propriété ; leur commerce, l’assiette où ils mangeaient, le lit où ils dormaient ne leur appartenaient pas. Et il leur fallait \emph{sauver cette propriété} au profit du propriétaire qui loue la maison, du banquier qui escompte le billet, du capitaliste qui fait les avances au comptant, du fabricant qui confie les marchandises pour les vendre, du commerçant en gros qui fait à ces artisans crédit des matières premières.\par
\emph{Rétablissement du crédit} ! Mais le crédit consolidé justifiait sa réputation. C’était un dieu vivant, actif, plein de zèle, jetant hors des murs de son temple les débiteurs insolvables avec leurs femmes et leurs enfants, livrant au capital leur propriété illusoire et les précipitant pour dettes dans la prison qui s’était de nouveau élevée sur les cadavres des insurgés de Juin.\par
Les petits bourgeois reconnurent avec effroi qu’en battant les ouvriers ils s’étaient mis à la merci de leurs créanciers. Leur banqueroute, passée à l’état chronique, languissante, ignorée en apparence, fut publiquement déclarée après les journées de Juin.\par
Leur \emph{propriété nominale} était restée intacte tant qu’il avait été nécessaire de les mener à la bataille \emph{au nom de la propriété}. Maintenant que l’affaire importante avait été réglée avec le prolétariat, on pouvait régler aussi le petit compte de l’épicier. À Paris, le montant des effets en souffrance s’élevait à plus de 21 millions de francs, dans les provinces, à plus de 11 millions. Des commerçants, locataires de plus de 7 090 maisons parisiennes, n’avaient pas payé leur loyer depuis février.\par
Si l’Assemblée nationale avait fait une enquête sur la \emph{dette publique}, remontant jusqu’en février, les petits bourgeois de leur côté réclamaient une enquête sur les \emph{dettes bourgeoises} contractées antérieurement à cette date. Ils se rassemblèrent en masse à la Bourse et demandèrent avec menaces que tout commerçant pouvant prouver qu’il n’avait fait faillite qu’à la suite du trouble commercial apporté au négoce par la révolution, pouvant établir que ses affaires étaient bonnes le 24 février, que ce commerçant se vît accorder une prorogation des échéances par un jugement du Tribunal de commerce et vît obliger le créancier à réduire ses échéances à une répartition proportionnelle modérée. Le projet de loi relatif à cette question fut discutée à la Chambre sous le nom de \emph{concordats à l’amiable}. L’Assemblée hésitait, quand on apprit subitement qu’à la porte Saint-Denis des milliers de femmes et d’enfants, des insurgés de Juin, se préparaient à présenter une pétition d’amnistie.\par
Les petits bourgeois tremblèrent à la résurrection du spectre de Juin et l’assemblée retrouva son inflexibilité. « Les concordats à l’amiable » entre créanciers et débiteurs furent rejetés en leurs points essentiels.\par
Quand, au sein de l’Assemblée nationale, les représentants démocrates des petits bourgeois furent repoussés par les représentants républicains de la bourgeoisie, cette rupture parlementaire prit son sens bourgeois, réel, économique : les petits bourgeois, débiteurs, étaient livrés aux bourgeois, leurs créanciers. Une grande partie des premiers fut complètement ruinée. Il fut permis à ceux qui échappèrent au désastre de continuer leur négoce dans des conditions qui en faisaient des serfs à la discrétion du capital. Le 22 août 1848, l’Assemblée nationale repoussa les \emph{concordats à l’amiable}. Le 19 septembre 1848, en plein état de siège, le prince Louis-Bonaparte et le détenu de Vincennes, le communiste Raspail, furent élus représentants de Paris. La bourgeoisie, de son côté, choisit le changeur juif, l’orléaniste Fould. Ainsi, de tous les côtés à la fois, la guerre était publiquement déclarée à l’Assemblée constituante, au républicanisme bourgeois, à Cavaignac.\par
Il est inutile de s’étendre sur le retentissement qu’eut la banqueroute en masse des petits bourgeois de Paris. Ses effets dépassèrent de beaucoup le cercle de ceux qui en étaient immédiatement frappés. Le commerce bourgeois fut nécessairement ébranlé de nouveau. Le déficit se creusa encore une fois à la suite des dépenses occasionnées par l’insurrection de Juin. Les recettes de l’État baissaient continuellement, la production restait en suspens ; la consommation se restreignait ; l’importation diminuait. Cavaignac et l’Assemblée nationale ne pouvaient recourir qu’à un nouvel emprunt : c’était se mettre davantage encore sous le joug de l’aristocratie financière.\par
Si, pour les petits bourgeois, le fruit de la victoire de Juin avait été la banqueroute et la liquidation judiciaire, la tendre armée des lorettes récompensa les janissaires de Cavaignac, les gardes mobiles. Ces « jeunes sauveurs de la société » reçurent des hommages de toute espèce dans les salons de Marrast, du « gentilhomme » des tricolores, devenu tout à la fois l’amphitryon et le troubadour de la République honnête. Ces préférences de la société et la solde incomparablement plus élevée dont jouissait la garde mobile indisposèrent l’armée. De plus, c’était le moment, où s’évanouissaient toutes les illusions qui, sous Louis-Philippe, avaient rallié autour des républicains bourgeois, grâce à l’attitude de leur journal, le \emph{National}, une partie des militaires et de la classe paysanne. Cavaignac et l’assemblée jouaient dans l’\emph{Italie du Nord} un rôle d’intermédiaire pour la livrer à l’Autriche, d’accord avec l’Angleterre. – Un seul jour de pouvoir anéantit les dix-huit années d’opposition du \emph{National}. Pas de gouvernement moins national que celui du \emph{National}. Pas de gouvernement qui dépendît davantage de l’Angleterre, et, sous Louis-Philippe, ce journal vivait de la répétition constante du mot de Caton : \emph{Carthaginem esse delendam}. Pas de gouvernement plus servile à l’égard de la Sainte-Alliance et un Guizot avait pu demander que l’on déchirât les traités de Vienne.\par
L’ironie de l’histoire fit de Bastide, l’ancien rédacteur de la politique étrangère au \emph{National}, le ministre des Affaires étrangères de la France, pour qu’il pût contredire chacun de ses articles par chacune de ses dépêches.\par
L’armée et les paysans avaient cru un moment que la dictature militaire signifiait la guerre avec l’étranger, la « gloire » mise à l’ordre du jour ; mais Cavaignac n’exerçait pas la dictature du sabre au sein de la société bourgeoise, il exerçait la dictature de la bourgeoisie au moyen du sabre. On avait besoin non de soldats mais de gendarmes. Cavaignac dissimulait sous les traits sévères et résignés d’un républicain antique la fade soumission aux humbles conditions de sa fonction bourgeoise. « L’argent n’a pas de maître\footnote{En français dans le texte} ». Cette ancienne devise du « tiers état » était idéalisée par lui comme elle l’était, en général, par l’Assemblée constituante. En langage politique, elle signifiait : la bourgeoisie n’a pas de roi ; la vraie forme de son pouvoir, c’est la République.\par
Élaborer cette \emph{forme}, élaborer une \emph{Constitution républicaine}, tel était le « grand-œuvre organique » de la Constituante. Débaptiser le calendrier chrétien pour en faire un calendrier républicain, transformer saint Bartholomée en saint Robespierre, ne change pas plus le temps que la constitution ne modifie ou ne peut modifier la société bourgeoise. Quand l’assemblée fît plus que de changer le costume, elle prit acte des faits accomplis. Elle enregistra solennellement l’existence de la République, du suffrage universel, l’existence d’une seule Assemblée nationale souveraine, à la place des deux Chambres constitutionnelles à pouvoirs limités. Elle enregistra et régularisa le fait de la dictature de Cavaignac. Elle changea la royauté héréditaire, irresponsable, stationnaire, en une royauté élective, ambulante et responsable, en une présidence de quatre ans. Elle donna la valeur d’une loi constitutionnelle aux pouvoirs extraordinaires dont l’Assemblée nationale, après la Terreur du 15 mai et du 25 juin, avait soigneusement muni son président dans l’intérêt de sa sécurité. Le reste de la constitution n’était plus qu’une affaire de terminologie. On enleva aux rouages de l’ancienne monarchie leurs étiquettes royalistes pour y mettre des étiquettes républicaines. Marrast, ancien rédacteur en chef du \emph{National}, devenu rédacteur en chef de la Constitution, s’acquitta, non sans talent, de ce travail académique.\par
L’Assemblée nationale ressemblait à ce fonctionnaire chilien qui voulait régulariser les rapports de la propriété foncière par une révision du cadastre, au moment précis où le tonnerre souterrain avait déjà annoncé une éruption volcanique susceptible d’engloutir la terre sous ses pieds. Tandis qu’elle déterminait théoriquement les formules par lesquelles devait s’exprimer républicainement la domination bourgeoise, elle se maintenait en réalité par la négation de toute forme, par la force « sans phrase\footnote{En français dans le texte} », par l’\emph{état de siège}. Deux jours avant de commencer son œuvre constitutionnelle, elle prolongea la durée de cet état d’exception. Autrefois on avait élaboré et accepté des constitutions quand le procès de bouleversement social était arrivé à un point d’arrêt, quand les rapports de classe à classe, nouvellement contractés, s’étaient établis, quand les fractions rivales de la classe au pouvoir avaient recours à un compromis qui leur permettait de continuer à lutter entre elles et d’exclure de ce tournoi la masse populaire domptée. La nouvelle constitution, au contraire, ne sanctionnait pas une Révolution sociale. Elle sanctionnait la victoire momentanée de la vieille société sur la Révolution.\par
Dans le premier projet de constitution, on rencontrait encore le \emph{droit au travail}, première formule confuse où se résumaient les revendications révolutionnaires du prolétariat. Il fut transformé en \emph{droit à l’assistance}. Et quel est donc l’État moderne qui ne nourrit pas ses pauvres sous une forme ou sous une autre ? Le droit au travail est, au sens bourgeois, un contre-sens, un désir pieux, imparfait. Mais ce qui se trouve derrière lui, c’est le pouvoir sur le capital, derrière le pouvoir sur le capital, l’appropriation des moyens de production, leur remise à la classe ouvrière associée, c’est la suppression du salariat, du capital et de ses rapports d’échange. Derrière le \emph{droit au travail} se dressait l’insurrection de Juin. L’Assemblée constituante qui, en fait, mettait le prolétariat révolutionnaire « hors la loi », devait par principe rejeter de la constitution, loi suprême, la formule prolétarienne, et fulminer l’anathème contre le « droit au travail ». Elle n’en demeura pas là. De même que Platon bannissait les poètes de sa République, l’assemblée bannissait pour l’éternité de la sienne – \emph{l’impôt progressif}. Cet impôt n’est pas seulement une mesure bourgeoise, réalisable sur une échelle plus ou moins vaste dans les rapports de production actuels ; c’était encore l’unique moyen d’attacher à la République « honnête » les couches moyennes de la société bourgeoise, de réduire la dette publique, de mettre en échec la majorité anti-républicaine de la bourgeoisie.\par
À l’occasion des concordats à l’amiable, les républicains tricolores avaient sacrifié les petits bourgeois à la grande bourgeoisie. Ce fait isolé fut élevé à la hauteur d’un principe par l’interdiction légale de l’impôt progressif. On mettait sur le même plan une réforme bourgeoise et la révolution prolétarienne. Mais alors quelle était la classe sur laquelle pouvait s’appuyer la République ? La grande bourgeoisie ? Sa masse était anti-républicaine. Si elle exploitait les républicains du \emph{National} pour affermir de nouveau les anciennes conditions économiques, elle cherchait d’autre part à exploiter les rapports sociaux que l’on venait de raffermir, pour rétablir les formes politiques qui leur correspondent. Déjà au commencement d’octobre, Cavaignac se vit forcé de faire de Dufaure et de Vivien, anciens ministres de Louis-Philippe, des ministres de la République, et cela malgré les grondements et le tapage des puritains sans cervelle de son propre parti.\par
Pendant que la constitution tricolore repoussait toute compromission avec la petite bourgeoisie et ne savait attacher aucun élément de la société à la nouvelle forme sociale, elle s’empressait de rendre une intangibilité traditionnelle à un corps où l’ancien État trouvait ses défenseurs les plus acharnés et les plus fanatiques. Elle inscrivit dans la loi constitutionnelle l’\emph{inamovibilité des juges}. Le roi avait été renversé. Il ressuscita par centaines dans ces inquisiteurs inamovibles de la légalité.\par
La presse française a souvent analysé les contradictions de la constitution de M. Marrast, par exemple, l’existence simultanée de deux souverains, l’Assemblée nationale et le président, etc.\par
À la vérité, la contradiction qui enveloppe cette constitution est la suivante : les classes dont elle doit perpétuer l’esclavage social, prolétariat, petite bourgeoisie, classe paysanne, sont mises par elle en possession du pouvoir politique par le suffrage universel. D’autre part, elle soustrait à la classe dont elle sanctionne l’ancienne puissance les garanties politiques de cette puissance. Elle adapte violemment la domination politique de la bourgeoisie à des conditions démocratiques qui procurent la victoire aux classes ennemies et mettent en question les bases mêmes de la société bourgeoise. Elle demande aux unes de ne pas s’avancer de l’émancipation politique à l’émancipation sociale, aux autres de ne pas repasser de la restauration sociale à la restauration politique.\par
Ces contradictions importaient peu aux républicains bourgeois. À mesure qu’ils devenaient \emph{indispensables}, et ils ne l’étaient que s’ils servaient d’avant-garde à la vieille société bourgeoise en combattant contre le prolétariat ; de \emph{parti} qu’ils étaient ils tombaient au rang de \emph{coterie}. La constitution, ils la traitaient comme une grande \emph{intrigue}. Ce qu’il fallait constituer avant tout, c’était la suprématie d’une coterie. Cavaignac devait se prolonger dans le président, l’Assemblée constituante se prolonger dans la Législative. Les républicains espéraient réduire le pouvoir politique des masses populaires à une puissance illusoire. Ils pensaient que cette puissance même serait suffisamment leur jouet et qu’ils pourraient suspendre constamment au-dessus de la majorité de la bourgeoisie le dilemme des journées de Juin, \emph{ou le règne du National ou le règne de l’Anarchie}.\par
L’œuvre constitutionnelle entreprise le 4 septembre fut terminée le 23 octobre. Le 2 septembre, la Constituante avait décidé de ne pas se séparer avant que n’aient été promulguées les lois organiques complétant la constitution. Néanmoins elle se décida à appeler à la vie, le 10 décembre, bien avant que ses propres pouvoirs ne fussent périmés, sa création la plus originale, le président. Tellement elle était sûre de saluer dans l’homunculus constitutionnel le fils dont elle était la mère. Par précaution, on avait décidé que si aucun des candidats ne comptait deux millions de voix, le droit d’élection passerait de la nation à la Constituante.\par
Inutiles mesures ! Le premier jour où la constitution se réalisait était aussi le dernier jour de la Constituante. La condamnation à mort était au fond de l’urne électorale. Elle cherchait le « fils de sa mère », elle trouva le « neveu de son oncle ! » Saül Cavaignac abattit un million de voix, mais David Napoléon en abattit six millions. Saül Cavaignac était six fois battu.\par
Le 10 décembre 1848 fut le jour de l’insurrection des paysans. Ce fut le Février des paysans français. Le symbole qui traduit leur entrée dans le mouvement révolutionnaire, maladroitement astucieux, naïvement gredin, lourdement sublime, superstition calculée, burlesque, pathétique, anachronisme génialement sot, espièglerie historique, hiéroglyphe indéchiffrable pour la raison des civilisés, – ce symbole revêtait indubitablement la physionomie de la classe qui représente la barbarie dans la civilisation. La République s’était fait connaître aux paysans par le percepteur des contributions, les paysans se firent connaître à la République par l’empereur. Napoléon était le seul homme représentant parfaitement les intérêts et l’imagination de la nouvelle classe paysanne, créée par 1789. En écrivant son nom au fronton de l’édifice républicain, cette classe déclarait la guerre à l’étranger ; à l’intérieur elle faisait valoir ses intérêts de classe. Drapeau déployé, musique en tête, elle marcha aux urnes aux cris de : « Plus d’impôts, à bas les riches, à bas la république, vive l’empereur\footnote{En français dans le texte} ! » Derrière l’empereur se cachait la jacquerie. La république contre laquelle les paysans venaient de voter, c’était \emph{la république des riches}.\par
Le 10 décembre était le coup d’État des paysans qui renversaient le gouvernement existant. Du jour où ils avaient ôté, puis donné un gouvernement à la France, leurs regards se dirigèrent fixement sur Paris. Pour un moment héros actifs du drame révolutionnaire, ils ne pouvaient plus se résoudre au rôle inactif et inconscient de choristes.\par
Les autres classes contribuèrent à parfaire la victoire électorale des paysans. L’élection de Napoléon signifiait pour le \emph{prolétariat} la destitution de Cavaignac, le renversement de la Constituante, le renvoi des républicains bourgeois, l’annulation de la victoire de Juin. Pour la \emph{petite bourgeoisie}, Napoléon voulait dire la suprématie du débiteur sur le créancier. Pour la majorité de la \emph{grande bourgeoisie}, l’élection de Napoléon, c’était la rupture ouverte avec ses anciens alliés, auxquels elle avait dû se soumettre un instant pour agir contre la révolution ; mais ce prolétariat lui était devenu insupportable depuis qu’elle essayait de donner à sa suprématie une valeur constitutionnelle. Napoléon remplaçant Cavaignac, c’était la monarchie au lieu de la république, le début de la restauration royaliste, c’étaient les d’Orléans dont on parlait à voix basse, c’était le lys caché sous la violette. L’\emph{armée} enfin, en votant pour Napoléon votait contre la garde mobile, contre l’idylle de la paix, pour la guerre.\par
Il arrivait donc, comme le disait la \emph{Neue rheinische Zeitung} que l’esprit le plus simple de toute la France acquerrait l’importance la plus complexe. Précisément parce qu’il n’était rien, il pouvait signifier tout sans rien signifier par lui-même. Mais quelque varié que fût le sens du mot Napoléon dans la bouche des différentes classes, chacun en inscrivant ce nom sur son bulletin voulait dire : « À bas le parti du \emph{National}. À bas Cavaignac, à bas la Constituante, à bas la république bourgeoise ! » Le ministre Dufaure le déclara publiquement à l’Assemblée Constituante : « Le 10 décembre est un second 24 février. »\par
Bourgeoisie et prolétariat avaient voté « en bloc » pour Napoléon afin de se prononcer \emph{contre} Cavaignac, afin d’arracher à la Constituante, par la comparaison des suffrages, quelque chose de décisif. Cependant la partie la plus avancée de chaque classe avait présenté ses candidats : Napoléon était le \emph{nom collectif} de tous les partis coalisés contre la république bourgeoise ; \emph{Ledru-Rollin} et \emph{Raspail} étaient les \emph{noms propres}, le premier de la petite bourgeoisie démocratique, le second du prolétariat révolutionnaire. Les suffrages exprimés en faveur de Raspail – les prolétaires et leurs interprètes socialistes le disaient bien haut – ne devaient constituer qu’une simple démonstration, être autant de protestations contre la présidence, c’est-à-dire contre la constitution elle-même, autant de votes se prononçant contre Ledru-Rollin. C’était donc le premier acte par lequel le prolétariat se détachait comme parti politique indépendant du parti démocratique. Ce dernier, au contraire, – la petite bourgeoisie démocratique et sa représentation parlementaire, la Montagne, – traitait la candidature de Ledru-Rollin avec tout le sérieux qui accompagne habituellement ses solennelles duperies. D’ailleurs, c’était sa dernière tentative de se poser en parti indépendant en face du prolétariat. Non seulement le parti des bourgeois républicains, mais encore la petite bourgeoisie démocratique et la Montagne étaient battus le 10 décembre.\par
La France possédait alors en face d’une \emph{Montagne} un \emph{Napoléon} : c’était la preuve que l’une et l’autre ne représentaient que les caricatures mortes des grandes réalités dont ils portaient les noms. Louis-Napoléon, avec la couronne impériale et l’aigle, ne parodiait pas plus misérablement l’ancien Napoléon que la Montagne, avec ses phrases empruntées à 1793 et ses poses démagogiques, ne singeait l’ancienne Montagne. La superstition traditionnelle en 1793 fut ainsi détruite en même temps que la superstition traditionnelle en Napoléon. La révolution ne pouvait être chez elle que quand elle aurait acquis son nom \emph{originel et propre} ; elle ne pouvait le faire que si la classe révolutionnaire moderne, le prolétariat industriel était au premier plan. On peut dire que le 10 décembre déconcertait déjà la Montagne et lui faisait perdre le sens en rompant l’analogie classique avec la première révolution par un misérable tour de paysan.\par
Le 20 décembre, Cavaignac résigna son emploi et l’assemblée constituante proclama Louis-Napoléon, président de la République. Le 19 décembre, le dernier jour de sa toute-puissance, elle repoussa la proposition d’amnistie en faveur des insurgés de juin. Rapporter le décret du 27 juin par lequel elle avait condamné sans jugement à la déportation 15 000 insurgés, n’était-ce pas aussi révoquer le combat de Juin ?\par
Odilon Barrot, le dernier ministre de Louis-Philippe, fut le premier ministre de Louis Napoléon. De même que Louis Napoléon ne data pas son pouvoir du 10 décembre, mais d’un sénatus-consulte de 1806, il trouva un président du conseil qui ne datait pas son ministère du 20 décembre, mais d’un décret royal du 24 février. L’héritier légitime de Louis-Philippe, Louis-Napoléon, adoucit le changement de gouvernement en conservant l’ancien ministère qui n’avait pas eu le temps de s’user, puisqu’il n’avait pas trouvé celui de venir au monde.\par
Les chefs des fractions royalistes de la bourgeoisie conseillaient ce choix. La tête de l’ancienne opposition dynastique qui avait ménagé l’alliance avec les républicains du \emph{National} était plus capable encore de ménager avec pleine conscience la transition entre la République bourgeoise et la monarchie.\par
Odilon Barrot était le chef de l’unique parti de l’opposition qui, ayant cherché toujours en vain à saisir un portefeuille de ministre, n’était pas encore usé. La révolution précipitait dans une succession rapide tous les anciens partis d’opposition sur les sommets du pouvoir. Elle les obligeait ainsi, non seulement en fait, mais jusque dans leurs propres phrases à nier et à révoquer les anciennes paroles. Le peuple pouvait alors jeter à la voirie de l’histoire le mélange dégoûtant qu’ils formaient. Aucune apostasie ne fut épargnée à ce Barrot, à cette incorporation du libéralisme bourgeois qui pendant dix-huit ans avait caché le vide misérable de son esprit sous un maintien grave. Si parfois, le contraste trop choquant entre les chardons du présent et les lauriers du passé l’effrayait lui-même, il lui suffisait d’un coup d’œil donné à son miroir pour voir s’y refléter une contenance ministérielle et une suffisance bien humaine. Ce que le miroir lui renvoyait, c’était Guizot qu’il avait constamment envié, constamment censuré, Guizot lui-même, mais paré du front olympien d’Odilon. Ce qu’il ne voyait pas, c’étaient les oreilles de Midas.\par
Le Barrot du 24 février se révéla dans le Barrot du 20 décembre. Orléaniste et voltairien, il s’associa comme ministre des cultes, le légitimiste, le jésuite Falloux.\par
Peu de jours après, le ministère de l’Intérieur était confié à Léon Faucher, le disciple de Malthus. Le droit, la religion, l’économie politique ! Le ministère Barrot contenait tout cela ; de plus, il réunissait les orléanistes et les légitimistes. Le bonapartiste seul faisait défaut. Bonaparte cachait encore l’envie qu’il avait d’être Napoléon. \emph{Soulouque} ne jouait pas encore les Toussaint-Louverture.\par
Le parti du \emph{National} fut aussitôt chassé des postes élevés où il s’était niché. Préfecture de police, direction des postes, parquet général, mairie de Paris, tous ces emplois furent occupés par d’anciennes créatures de la monarchie. Changarnier, le légitimiste, réunit en ses mains le commandement supérieur de la garde nationale du département de la Seine, de la garde mobile et des troupes de ligne de la première division. Bugeaud, l’orléaniste, fut nommé commandant en chef de l’armée des Alpes, le changement des fonctionnaires dura sans interruption pendant toute la durée du gouvernement de Barrot. Le premier acte de son ministère fut la restauration de l’ancienne administration royaliste. En un clin d’œil, la scène officielle se transformait : coulisses, costume, langue, acteurs, figurants, comparses, souffleurs, position des partis, motifs du drame, matière de la catastrophe, situation complète. Seule la Constituante préhistorique se trouvait encore en place ; mais à dater de l’heure où l’Assemblée nationale installa Bonaparte, et Bonaparte Barrot, Barrot Changarnier, la France sortit de la constitution de la république proprement dite pour entrer dans la période de la république constituée. Et qu’avait à faire une Assemblée constituante dans une république constituée ? Quand la terre eut été créée, il ne resta plus à son créateur d’autre ressource que de se réfugier dans le ciel. La Constituante était décidée à ne pas suivre son exemple. L’Assemblée nationale était le dernier asile du parti des républicains bourgeois. Si tout exercice du pouvoir exécutif était interdit à cette assemblée, ne lui restait-il pas la toute-puissance constituante ? Ce qui lui venait d’abord à l’esprit, c’était de revendiquer le poste élevé qui lui était départi, puis de s’en servir pour reconquérir le terrain perdu. Que le ministère Barrot fût remplacé par un gouvernement du \emph{National}, et les créatures royalistes se voyaient obligées de quitter les palais administratifs, le personnel tricolore y rentrait triomphalement. L’Assemblée nationale décida le renversement du ministère. Le ministère lui-même fournit l’occasion. La Constituante ne pouvait en souhaiter de meilleure.\par
On se souvient que, pour les paysans, Louis-Bonaparte signifiait : plus d’impôts nouveaux. Il y avait six jours que ce nouveau président était installé quand son ministère proposa le \emph{maintien de l’impôt sur le sel}. Le gouvernement provisoire en avait décrété la suppression. L’impôt sur le sel partage avec l’impôt sur le vin le privilège d’être le bouc émissaire de l’ancien système financier de la France, surtout aux yeux de la population paysanne. Le ministère Barrot ne pouvait mettre dans la bouche de l’élu des paysans une épigramme plus mordante pour ses électeurs que ces mots : \emph{rétablissement de l’impôt sur le sel}. L’imposition du sel enleva à Bonaparte tout son sel révolutionnaire. Le Napoléon de l’insurrection paysanne s’évanouit comme une ombre. Il ne restait plus qu’à mettre sa confiance dans l’intrigue des bourgeois royalistes et qu’à s’en remettre à un hasard qui pouvait être gros de conséquences. C’est à dessein que le ministère Barrot changea le premier acte gouvernemental du président en une désillusion grossière et brutale.\par
La Constituante, de son côté, saisit avec joie la double occasion qui lui était offerte de renverser le ministère et de représenter les intérêts des paysans contre leur élu. L’Assemblée repoussa le projet du ministre des Finances, réduisit l’impôt sur le sel au tiers de son montant antérieur, augmenta ainsi de 60 millions un déficit public de 560 millions et attendit tranquillement, après son vote de défiance, la retraite du ministère, tellement elle comprenait peu le nouveau monde qui l’entourait, le changement qu’avait subi sa propre position. Derrière le ministère il y avait le président, et derrière celui-ci six millions de citoyens qui avaient déposé dans l’urne un nombre égal de votes de défiance à l’égard de la Constituante. L’Assemblée rendait la pareille à la nation. Ridicule échange de procédés ! La Constituante oubliait que ses votes n’avaient plus cours forcé. Le rejet de l’impôt sur le sel précipita simplement la décision de Bonaparte et de son ministère \emph{d’en finir} avec l’Assemblée. Alors commença ce long duel qui remplit la dernière moitié de l’existence de la Constituante. \emph{Le} 29 \emph{janvier, le} 21 \emph{mars, le} 3 \emph{mai}, sont les « journées » de cette crise, autant de signes précurseurs du 13 juin.\par
Les Français, Louis Blanc, par exemple, ont pensé que le 29 janvier était l’effet d’une contradiction constitutionnelle. Il y avait, certes, contradiction entre l’existence simultanée d’une Assemblée nationale, souveraine, indissoluble, issue du suffrage universel et d’un président, responsable à la lettre, envers elle, mais dont, en réalité, l’élection avait été sanctionnée par le suffrage universel. Le magistrat réunissait, de plus, sur sa personne, tous les suffrages reportés sur les différents membres de l’Assemblée nationale, toutes les voix auparavant dispersées à l’infini. Le président, enfin, était en pleine possession du pouvoir exécutif, tandis que l’Assemblée ne pouvait exercer sur ce pouvoir qu’une influence morale… Si l’on explique ainsi le 29 janvier, c’est qu’on confond les discours échangés au cours de la lutte, prononcés aux tribunes, publiés par la presse, proférés dans les clubs avec leur contenu véritable. L’opposition surgie entre Louis-Bonaparte et l’Assemblée ne représentait pas un conflit isolé entre le pouvoir constitutionnel et un autre pouvoir, entre le pouvoir exécutif et le législatif ; elle correspondait à un choc entre la république bourgeoise constituée et les instruments de sa constitution, entre les intrigues ambitieuses et les exigences idéologiques de la fraction révolutionnaire de la bourgeoisie.\par
Cette fraction avait fondé la République et elle se montrait surprise de la ressemblance de cette république constituée avec une monarchie restaurée. Elle voulait employer la violence à maintenir la période constituante avec ses conditions, ses illusions, son langage et ses personnages. Elle voulait empêcher la république bourgeoise, arrivée à maturité, de revêtir sa forme parfaite, sa forme propre. Si l’Assemblée nationale constituante représentait Cavaignac qui venait de rentrer dans son sein, Napoléon représentait l’Assemblée législative qu’il n’avait pas encore répudiée ; il représentait l’Assemblée nationale de la république bourgeoise constituée.\par
L’élection de Bonaparte ne pouvait s’expliquer qu’à la condition de remplacer le nom par tout ce qu’il signifiait, à la condition de se reproduire par l’élection de la nouvelle Assemblée nationale. Le mandat de la Constituante était échu le 10 décembre. Ce qui entrait en conflit, le 29 janvier, ce n’étaient donc pas le président et l’Assemblée de \emph{la même} République ; c’étaient l’Assemblée de la République en puissance, et le président de la République en acte, deux pouvoirs qui incorporaient deux périodes toutes différentes de l’existence de la République. D’un côté, on rencontrait la petite fraction républicaine de la bourgeoisie, seule capable de proclamer la République, de l’arracher des mains du prolétariat révolutionnaire par la guerre des rues et par la terreur, seule capable de modeler sa constitution d’après un type idéal ; de l’autre, toute la masse royaliste de la bourgeoisie, seule susceptible de régner dans cette République bourgeoise une fois constituée, de débarrasser la constitution des accessoires idéologiques, et de réaliser, par la législation et l’administration, les conditions indispensables à l’asservissement du prolétariat.\par
L’orage qui éclatait le 29 janvier s’était préparé pendant tout le courant du mois. La Constituante voulait, par un vote de défiance, contraindre le ministère Barrot à la démission. Le ministère, de son côté, proposait à la Constituante de se décerner à elle-même un vote de défiance définitif, de décider son suicide, de décréter \emph{sa propre dissolution}. Rateau, un des députés les plus obscurs, le proposa le 6 janvier, sur l’ordre du ministère à la Constituante, à cette Assemblée qui, dès août, avait décidé de ne pas se séparer avant d’avoir promulgué toute une série de lois organiques complétant la constitution. Fould, représentant ministériel, déclara franchement à l’Assemblée que sa dissolution était nécessaire \emph{pour restaurer le crédit ébranlé}. N’ébranlait-elle pas, en effet, le crédit en prolongeant cet état provisoire, en menaçant avec Barrot Bonaparte, et avec Bonaparte la République constituée ? Barrot, L’Olympien, devenu un Roland furieux à la pensée de se voir frustré d’une présidence de cabinet, alors qu’il n’en avait joui que pendant deux semaines à peine, Barrot, dont les républicains avaient une fois déjà prorogé la présidence pour un décennat, c’est-à-dire pour dix mois, Barrot exagéra la tyrannie exercée par le tyran sur cette misérable Assemblée. Le plus doux de ses mots fut que « pour elle, il n’y avait plus d’avenir possible ». Et, en réalité, elle ne représentait que le passé. « Elle était incapable », ajoutait-il ironiquement, « d’entourer la République des institutions indispensables à son affermissement. » Et c’était vrai. Toute son énergie était tombée dès qu’elle avait eu terminé la lutte menée par elle uniquement contre le prolétariat. D’un autre côté, son exaltation républicaine s’était éteinte en même temps que son opposition aux menées royalistes. Elle était donc doublement incapable d’affermir la République bourgeoise qu’elle ne comprenait plus en la dotant des institutions convenables.\par
En même temps que la proposition Rateau, le ministère déchaîna un ouragan de pétition dans tout le pays, et, tous les jours, de tous les coins de la France, des ballots de « billets doux » étaient jetés à la face de la Constituante. On la priait, plus ou moins catégoriquement, de se dissoudre et de faire son testament. La Constituante, de son côté, faisait naître des contre-pétitions où elle se laissait donner l’ordre de rester en vie. La lutte électorale entre Cavaignac et Bonaparte se renouvela sous la forme d’une lutte de pétition pour et contre la dissolution de l’Assemblée nationale. Les pétitions devaient être le commentaire supplémentaire du 10 décembre. Pendant tout le cours de janvier cette agitation persista.\par
Dans le conflit qui s’élevait entre la Constituante et le président, cette assemblée ne pouvait remonter à sa propre origine, à l’élection générale. On en appelait, en effet, au suffrage universel. Elle ne pouvait s’appuyer sur aucun pouvoir régulier. Il s’agissait pour elle d’une lutte contre le pouvoir légal. Elle ne pouvait renverser le ministère par des votes de défiance : elle avait essayé encore de le faire le 6 et le 26 janvier ; mais le gouvernement se souciait peu de sa confiance. Il ne restait qu’une issue : \emph{l’insurrection. La partie républicaine de la garde nationale, la garde mobile} et les centres de réunion du prolétariat révolutionnaire, les \emph{clubs}, formaient les forces de l’insurrection. Les gardes mobiles, ces héros des journées de juin, constituaient, en décembre, les forces organisées des fractions républicaines de la bourgeoisie, comme les \emph{ateliers nationaux} avaient été, avant les journées de juin, les forces organisées du prolétariat révolutionnaire. La commission exécutive de la Constituante avait brutalement attaqué les ateliers nationaux quand elle avait dû mettre lin aux prétentions du prolétariat, devenues insupportables. Le ministère de Bonaparte s’en prit de même à la garde mobile quand il dut mettre fin aux prétentions devenues insupportables des fractions républicaines de la bourgeoisie. Il ordonna le \emph{licenciement de la garde mobile}. Une moitié de celle-ci fut renvoyée et jetée sur le pavé. L’autre moitié reçut, à la place de son organisation démocratique, une organisation monarchique, et sa solde fut réduite à la solde ordinaire des troupes de ligne. La garde mobile était dans la situation où s’étaient trouvés les insurgés de juin. Aussi la presse publiait-elle quotidiennement des \emph{confessions publiques} où la garde avouait son péché, de juin et suppliait le prolétariat de lui pardonner.\par
Et \emph{les clubs} ? La Constituante menaçait dans Barrot le président, dans le président la République constituée ; elle mettait en question, avec la République constituée, la République bourgeoise en général. À partir de ce moment, les éléments qui avaient fondé la République de Février se rangèrent autour de l’Assemblée. Tous les partis qui voulaient renverser la République existante et, par une agression violente, la transformer en une République correspondant à leurs intérêts de classe et à leurs principes, tous ces partis se rallièrent autour de l’Assemblée. Ce qui s’était passé était non avenu. Les cristallisations du mouvement révolutionnaire s’étaient dissoutes. La République pour laquelle on avait combattu redevenait cette République vague des jours de Février que chaque parti se réservait de déterminer. Les partis reprirent un moment leurs anciennes positions, sans partager toutefois les anciennes illusions. Les républicains tricolores du \emph{National} s’appuyèrent de nouveau sur les démocrates de la « Réforme ». Ils les postèrent, en avant-garde, au premier rang de la bataille parlementaire. Les démocrates, de leur côté, s’appuyèrent sur les républicains socialistes ; le 27 janvier, un manifeste public annonça leur réconciliation et leur union ; ils se ménageaient dans les clubs des éléments insurrectionnels. La presse ministérielle accusait, avec raison, les républicains tricolores du \emph{National} de ressusciter les insurgés de juin. Pour pouvoir se mettre à la tête de la République bourgeoise, les tricolores mettaient en question cette République même. Le 26 janvier, le ministre Faucher proposa une loi sur le droit d’association dont le premier paragraphe disait : \emph{les clubs sont interdits}. Il proposa d’accorder à ce projet de loi le bénéfice de l’urgence et de le mettre aussitôt en discussion. La Constituante rejeta la proposition d’urgence, et, le 27 janvier, Ledru-Rollin déposait une motion de mise en accusation du ministère pour violation de la constitution. La proposition était signée de 230 représentants. La mise en accusation du ministère, au moment où un pareil acte dévoilait brutalement l’impuissance du tribunal, la majorité de la Chambre, ou bien se réduisait à une protestation impuissante de l’accusateur contre cette majorité même, voilà le grand atout révolutionnaire que la Montagne jouait quand la crise avait atteint ce caractère d’acuité ! Pauvre Montagne, écrasée sous le poids de son propre nom !\par
Blanqui, Barbès, Raspail, etc, avaient, le 15 mai, tenté de renverser la Constituante en envahissant la salle des séances à la tête du prolétariat parisien. Barrot ménageait à la même Assemblée un 15 mai moral. Il voulait lui dicter sa propre dissolution et fermer la salle des séances. Cette Assemblée avait chargé Barrot de l’enquête sur les événements de mai. Le premier ministre se posait en Blanqui royaliste. L’Assemblée rassemblait contre le ministre des alliés dans les clubs, chez les prolétaires révolutionnaires, dans le parti de Blanqui. À ce moment même, l’inflexible Barrot la tourmentait en lui proposant de soustraire au jury les accusés de mai et de les traduire devant un tribunal suprême, inventé par le parti du \emph{National}, devant la « haute-cour ». Il est remarquable que la crainte anxieuse de perdre un portefeuille pût tirer de la tête d’un Barrot des pointes dignes d’un Beaumarchais ! L’Assemblée nationale, après une longue hésitation, accepta sa proposition. Vis-à-vis des révoltés de mai, elle retrouvait son caractère normal.\par
Si la constituante était contrainte de \emph{s’insurger} contre le président et ses ministres, le ministère et le président étaient obligés au coup d’État : ils n’avaient, en effet, en leur pouvoir aucun moyen légal de dissoudre l’Assemblée ; mais la Constituante était la mère de la constitution, et la constitution, la mère du président. En faisant son coup d’État, le président déchirait la constitution. Il annulait ainsi ses titres républicains. Il lui fallait alors faire reconnaître ses titres impérialistes. C’était tirer l’orléanisme de son sommeil : titres impérialistes et orléanistes pâlissaient à leur tour devant la légitimité. La chute de la République légale ne pouvait élever au pouvoir que son pôle opposé : la monarchie légitimiste. À ce moment, en effet, le parti orléaniste n’était que le vaincu de Février, et Bonaparte le vainqueur du 10 décembre. Ni l’un ni l’autre ne pouvaient opposer à l’usurpation républicaine leurs titres monarchiques également usurpés. Les légitimistes comprenaient combien l’instant était favorable. Ils conspiraient ouvertement. Ils pouvaient espérer trouver leur \emph{Monk} dans le général Changarnier. On annonçait aussi bien dans leurs clubs l’avènement de la \emph{monarchie blanche} que celui de la \emph{république rouge} dans les clubs des prolétaires.\par
Une émeute, heureusement réprimée, aurait délivré le ministère de toutes les difficultés. « La légalité nous tue, » s’écriait Odilon Barrot. Une émeute aurait permis, sous prétexte de « salut public » de dissoudre la Constituante et de violer la constitution dans l’intérêt même de la constitution. La conduite brutale d’Odilon Barrot à l’Assemblée nationale, la proposition d’interdiction des clubs, la révocation bruyante de cinquante préfets tricolores, leur remplacement par des royalistes, le licenciement de la garde mobile, la brutalité de Changarnier envers les chefs de celle-ci, la réintégration de Lerminier, de ce professeur qui s’était déjà rendu impossible sous Guizot, la tolérance envers les fanfaronnades des légitimistes, tout cela constituait autant de provocations à l’émeute. Mais l’émeute ne voulait rien entendre ; elle attendait le signal de la Constituante et non du ministère.\par
Enfin vint le 29 janvier, le jour où il fallait se prononcer sur la proposition de Mathieu (de la Drôme), tendant au rejet sans condition de la proposition Rateau. Légitimistes, orléanistes, bonapartistes, gardes mobiles, montagne, clubs, tout le monde conspirait alors, autant contre l’ennemi prétendu que contre les soi-disant alliés. Bonaparte, du haut de son cheval, passait en revue une partie des troupes sur la place de la Concorde. Changarnier paradait sous prétexte de manœuvres stratégiques. La Constituante trouva la salle de ses séances occupée militairement. Elle, le centre où venaient se joindre toutes les espérances, les craintes, les appréhensions, les ferments, les attentes, les conspirations, cette Assemblée, ce lion n’hésita jamais autant que lorsqu’elle se rapprocha de l’esprit du siècle. Elle valait ce guerrier qui ne craignait pas seulement de se servir de ses propres armes, mais se croyait encore tenu de conserver intactes celles de son adversaire. Méprisant la mort, elle signa sa propre condamnation et repoussa l’ajournement indéterminé de la proposition Rateau. Elle-même en état de siège, elle imposait au pouvoir constituant des limites qui avaient été elles-mêmes déterminées par l’état de siège de Paris. Elle se vengea d’une manière digne d’elle en soumettant le jour suivant à une enquête la terreur dont le ministère l’avait frappée le 29 janvier. La Montagne montra son peu d’énergie révolutionnaire et de sens politique en se laissant imposer par le parti du \emph{National} le rôle de héraut d’armes dans cette grande comédie d’intrigue. Ce parti avait fait sa dernière tentative de ressaisir, sous la République constituée, le monopole du pouvoir qu’il possédait à la période où se constituait la république bourgeoise. Le parti du \emph{National} était terrassé.\par
Si, pendant la crise de janvier, il s’agissait de l’existence de la Constituante, à la crise du 21 mars, il s’agissait de l’existence de la constitution. En janvier, c’était le personnel du parti \emph{National} ; maintenant, c’était l’idéal de ce parti qui était menacé. Nous n’avons pas besoin de dire que les républicains honnêtes prisaient moins le sentiment élevé de l’idéologie républicaine que la jouissance terrestre du pouvoir.\par
Le 21 mars, l’ordre du jour de l’Assemblée nationale comportait le projet de loi de Faucher contre le droit d’association, \emph{la suppression des clubs}. L’article 8 de la constitution garantissait à tous les Français le droit de s’associer. L’interdiction des clubs portait donc une atteinte incontestable à la constitution. L’Assemblée nationale devait bénir elle-même la mutilation de ses saints ; mais les clubs étaient les points de rassemblement, les lieux de conspiration du prolétariat révolutionnaire. L’Assemblée nationale elle-même avait interdit la coalition des ouvriers contre les bourgeois. Et les clubs, qu’était-ce sinon une coalition de la classe ouvrière tout entière contre l’ensemble de la classe bourgeoise ? N’était-ce pas un État ouvrier qui s’élevait en face de l’État bourgeois ? Ne formaient-ils pas autant d’Assemblées constituantes du prolétariat, autant de sections toutes prêtes de l’armée de la révolte ? Ce que la constitution devait constituer avant tout, c’était la suprématie de la bourgeoisie. La constitution ne pouvait donc visiblement entendre par droit d’association que le droit à l’existence des associations cadrant avec la suprématie de la bourgeoisie, s’accordant avec l’ordre bourgeois. Si, par convenance théorique, la formule était générale, le gouvernement et l’Assemblée n’étaient-ils pas là pour l’interpréter et l’appliquer ? Et si, à l’époque primitive de la République, les clubs étaient, en fait, interdits, par l’état de siège, ne devaient-ils pas l’être par la loi dans la République régulière et constituée ? Les républicains tricolores ne pouvaient opposer à cette interprétation prosaïque de la constitution que la phrase redondante de la constitution. Une partie de ceux-ci, Pagnerre, Duclerc, votèrent pour le ministère et lui créèrent ainsi une majorité. L’autre partie, l’archange Cavaignac et le père de l’Église Marrast en tête, quand l’article sur la suppression des clubs eut passé, se retira, avec Ledru-Rollin et la Montagne, en un bureau spécial et « tinrent conseil ». L’Assemblée nationale était paralysée. Elle ne disposait plus du nombre de suffrages nécessaires pour pouvoir prendre une décision valable. M. Crémieux se souvint à temps que le chemin que l’on prenait menait droit à la rue et que l’on n’était plus en février 1848, mais en mars 1849. Le parti du \emph{National}, aussitôt éclairé, rentra dans la salle des séances. Derrière lui la Montagne suivait, la Montagne, qui, tout en étant constamment tourmentée par des envies révolutionnaires, recherchait continuellement des possibilités constitutionnelles et se trouvait toujours mieux à sa place derrière les républicains bourgeois que devant le prolétariat révolutionnaire. La comédie était jouée. La Constituante elle-même avait décrété que la désobéissance à la lettre de la constitution était le seul moyen possible de réaliser son esprit.\par
Un seul point restait à régler : les relations que la République constituée devait entretenir avec la révolution européenne, sa \emph{politique étrangère}. Le 8 mai 1849, une agitation extraordinaire régnait dans la Constituante dont le mandat devait expirer dans quelques jours. L’attaque de Rome par l’armée française, le recul de cette armée devant les Romains, l’infamie politique, la honte militaire, l’assassinat de la République romaine par la République française, la première campagne du second Bonaparte en Italie, tout cela était à l’ordre du jour. La Montagne avait encore une fois joué son grand atout ; Ledru-Rollin avait déposé sur le bureau du président son inévitable acte d’accusation contre le ministère, mais, cette fois, il visait aussi Bonaparte.\par
Le motif du 8 mai fut plus tard celui du 13 juin. Expliquons-nous sur l’expédition romaine.\par
Dès le milieu de novembre 1848, Cavaignac avait expédié une flotte à Civita-Vecchia pour protéger le pape, le prendre à son bord et le transporter en France. Le pape devait bénir la République honnête et assurer l’élection de Cavaignac à la présidence. Cavaignac voulait amorcer les prêtres avec le pape, avec les prêtres les paysans, et, au moyen de ces derniers, pêcher la présidence. Le but prochain de l’expédition de Cavaignac était, d’abord, une réclame électorale ; c’était en même temps une protestation et une menace contre la République romaine. Elle contenait en germe l’intervention de la France en faveur du pape.\par
Cette intervention en faveur du pape, faite de concert avec l’Autriche et Naples et dirigée contre la République romaine, fut décidée le 23 décembre, à la première réunion du Conseil des ministres de Bonaparte. Falloux au ministère, c’était le pape à Rome, et dans la Rome du pape. Bonaparte n’avait plus besoin du pape pour devenir le président des paysans ; mais il avait besoin de conserver le pape pour conserver les paysans au président. Leur crédulité lui avait valu sa dignité ; mais en perdant la foi, ils perdent la crédulité, et en perdant le pape, ils perdent la foi. Et les orléanistes et légitimistes coalisés qui régnaient sous le nom de Bonaparte ! Avant que la royauté ne fût restaurée, il fallait qu’elle fût la puissance qui sacre les rois. Abstraction faite du royalisme, sans l’antique Rome soumise à son pouvoir temporel, pas de pape ; sans pape, pas de catholicisme ; sans catholicisme, pas de religion en France ; et sans religion qu’adviendrait-il de l’ancienne société française ? L’hypothèque que le paysan possède sur les biens célestes garantit l’hypothèque que le bourgeois possède sur les biens du paysan. La révolution romaine était donc un attentat contre la propriété, contre l’ordre bourgeois : elle était autant à craindre que la révolution de Juin. La suprématie bourgeoise restaurée en France exigeait le rétablissement de la suprématie papale à Rome. Enfin on frappait dans les révolutionnaires romains les alliés des révolutionnaires français. L’alliance des classes contre-révolutionnaires dans la République française constituée était nécessairement complétée par l’alliance de cette République avec la Sainte-Alliance, avec Naples et l’Autriche. La décision du Conseil des ministres du 23 décembre n’était pas un secret pour la Constituante. Déjà, le 8 janvier, Ledru-Rollin avait interpellé le cabinet à ce sujet. Le ministère avait nié le fait. L’Assemblée nationale avait passé à l’ordre du jour. Avait-elle confiance dans les paroles du ministère ? Nous savons qu’elle avait employé tout le mois de janvier à lui décerner des votes de défiance. Mais si le ministère était dans son rôle en mentant, l’assemblée était dans le sien en feignant d’avoir foi en ce mensonge et en sauvant ainsi les « dehors » républicains.\par
Cependant le Piémont avait été battu. Charles-Albert avait abdiqué. L’armée autrichienne frappait aux portes de la France. Ledru-Rollin interpella avec vivacité. Le ministère montra que dans l’Italie du Nord il n’avait fait que continuer la politique de Cavaignac, et Cavaignac la politique du gouvernement provisoire, c’est-à-dire de Ledru-Rollin. Pour cette fois, le gouvernement récolta un vote de confiance. Il fut même autorisé à occuper temporairement un point convenable dans l’Italie du Nord et à appuyer ainsi les négociations pacifiques poursuivies avec l’Autriche au sujet de l’intégrité du territoire sarde et touchant la question romaine. Le sort de l’Italie devait se décider certainement sur les champs de bataille du Nord de ce pays. Rome tombait avec la Lombardie et le Piémont, ou bien la France était obligée de déclarer la guerre à l’Autriche et, par là même, à toute la contre-révolution européenne.\par
L’Assemblée nationale prenait-elle subitement le ministère Barrot pour le Comité de Salut public ? Se croyait-elle, elle-même, la Convention ? Pourquoi donc occuper militairement un point du nord de l’Italie ? On cachait sous ce voile transparent l’expédition contre Rome.\par
Le 14 avril, 14 000 hommes s’embarquaient pour Civita-Vecchia sous les ordres d’Oudinot. Le 16 avril, l’Assemblée nationale accorda au ministère un crédit de 1 200 000 francs pour l’entretien, pendant trois mois, d’une flotte d’intervention dans la Méditerranée. Elle donnait ainsi au ministère tous les moyens d’agir contre Rome en feignant de le laisser intervenir contre l’Autriche. Elle ne voyait pas ce que le ministère faisait ; elle se bornait à écouter ce qu’il disait. Israël n’avait pas témoigné une foi pareille. La Constituante en était arrivée à ne pas oser savoir quelle conduite devait tenir la République constituée.\par
Enfin, le 8 mai, se joua la dernière scène de la comédie. La Constituante invita le ministère à prendre des mesures rapides pour ramener l’expédition d’Italie à son véritable but. Bonaparte, le même soir, fait paraître une lettre dans le \emph{Moniteur} où il témoignait à Oudinot la plus grande reconnaissance. Le 11 mai, l’Assemblée nationale repousse la mise en accusation de Bonaparte et de son ministère. Et la Montagne, au lieu de déchirer le voile mensonger, prend au tragique la comédie parlementaire et veut même aller y jouer le rôle des Fouquier-Tinville ! Ne montrait-elle pas sous la peau de lion empruntée à la Convention, sa robe originelle, la peau de veau de la petite bourgeoisie.\par
La seconde moitié de l’existence de la Constituante se résume dans les faits suivants : l’Assemblée avoue, le 29 janvier, que les fractions royalistes de la bourgeoisie sont les chefs naturels de la République constituée ; le 21 mars, que violer la constitution, c’est la réaliser, et, le 11 mai, que l’alliance passive, emphatiquement proclamée entre la République française et les peuples en révolte signifie l’alliance active conclue avec la contre-révolution européenne.\par
Cette misérable assemblée quitta la scène après s’être donnée encore, le 4 mai, deux jours avant l’anniversaire de sa naissance, la satisfaction de rejeter la proposition d’amnistie en faveur des insurgés de juin. Brisée dans sa puissance, haïe à mort par le peuple, repoussée, maltraitée, écartée avec dédain par la bourgeoisie dont elle était l’instrument, contrainte de désavouer dans la deuxième moitié de son existence la première période de sa vie, dépouillée de l’illusion républicaine, n’ayant rien créé de grand dans le passé, n’espérant rien de l’avenir, périssant toute vivante et tombant en morceaux, cette Assemblée savait encore galvaniser son propre cadavre en se rappelant constamment sa victoire de juin, en la revivant avec rancune. Elle s’affirmait en renouvelant constamment sa malédiction contre les maudits. Vampire vivant du sang des insurgés de juin !\par
Elle laissait après elle le déficit augmenté des dépenses occasionnées par l’insurrection de juin, accru par la moins-value de l’impôt sur le sel, par les indemnités qu’elle accorda aux planteurs pour les dédommager de l’abolition de l’esclavage, par les frais de l’expédition romaine, par la moins-value de l’impôt sur le vin, dont, en pleine agonie, elle décida la suppression. Malicieuse vieille, qui riait de joie de charger son héritier d’une dette d’honneur compromettante.\par
Dès le début de mars, avait commencé l’agitation pour les élections à l’\emph{Assemblée nationale législative}. Deux groupes principaux étaient en présence : le \emph{parti de l’ordre} et le \emph{parti démocrate-socialiste} ou \emph{parti rouge}. Entre eux se trouvaient les \emph{amis de la constitution}, nom sous lequel les républicains tricolores du \emph{National} essayaient de présenter un parti. Le \emph{parti de l’Ordre} se constitua immédiatement après les journées de juin. Ce ne fut que quand le 10 décembre lui permit de se débarrasser de la coterie du \emph{National}, des républicains bourgeois, que le secret de son existence se dévoila : c’était la \emph{coalition en un parti des orléanistes et des légitimistes}. La classe bourgeoise se décomposait en deux grandes fractions qui avaient tour à tour prétendu à l’hégémonie : \emph{la grande propriété foncière sous la Restauration, la bourgeoisie industrielle sous la monarchie de juillet}. Bourbon était le nom royal qui couvrait la prépondérance des intérêts d’une fraction ; Orléans désignait la prééminence des intérêts de l’autre. \emph{Le règne anonyme de la République} était le seul sous lequel ces deux fractions pussent faire prévaloir les intérêts communs de leur classe en une domination unique, sans qu’elles dussent pour cela renoncer à leur rivalité réciproque. La République bourgeoise ne pouvait être que la domination parfaite, pure et simple, de la classe bourgeoise tout entière. Pouvait-elle, dès lors, représenter autre chose que le règne des orléanistes complétés par les légitimistes et celui des légitimistes complétés par les orléanistes, autre chose que la \emph{synthèse de la Restauration et de la monarchie de Juillet} ? Les républicains du \emph{National} ne représentaient pas une fraction importante de leur classe au point de vue économique. Ils n’avaient qu’une seule importance, un seul titre historique : c’était d’avoir, sous la monarchie, à l’encontre des deux fractions de la bourgeoisie qui ne concevaient que leur régime particulier, préconisé le régime général de la classe bourgeoise, le règne anonyme de la République, qu’ils idéalisaient, il est vrai, et décoraient d’arabesques antiques, mais en lequel ils saluaient surtout la suprématie de leur coterie. Si le parti du \emph{National} s’était trouvé désorienté en apercevant les royalistes coalisés à la tête de la République qu’il avait fondée, les royalistes, par contre, ne s’illusionnaient pas moins sur le fait de leur suprématie commune. Ils ne comprenaient pas que si chacune de leurs fractions était royaliste, le produit de leur combinaison chimique devait être nécessairement \emph{républicain} : la monarchie blanche et la monarchie bleue devaient se neutraliser dans la République tricolore. S’opposant au prolétariat révolutionnaire et aux classes intermédiaires qui se concentraient autour de ce prolétariat, le parti de l’ordre était obligé d’avoir recours à la coalition de ses forces et de maintenir en état de conservation l’organisation de ses forces coalisées. Chacune des deux fractions de ce parti devait faire prévaloir, à l’encontre des désirs de restauration et d’hégémonie de l’autre, la suprématie commune, la \emph{forme républicaine} de la suprématie bourgeoise. Ces royalistes qui, au début, croyaient à une restauration immédiate, qui plus tard, conservaient la République, l’écume et l’invective aux lèvres, finissaient par accorder qu’ils ne pouvaient vivre en bonne intelligence que sous la seule République et par remettre la Restauration à une date indéterminée. La jouissance commune du pouvoir renforçait même chacune des deux fractions, rendait, par suite, chacune d’elles plus incapable encore et moins disposée à se subordonner à l’autre, c’est-à-dire à restaurer la monarchie.\par
Dans son programme électoral, le \emph{parti de l’ordre} proclama sans détour la suprématie de la classe bourgeoise, le maintien des conditions de cette suprématie, la conservation de la propriété, de la \emph{famille}, de la \emph{religion}, de \emph{l’ordre} ! Sa suprématie de classe et les conditions de cette suprématie devenaient naturellement pour lui le règne de la civilisation et les conditions nécessaires de la production matérielle ainsi que des rapports commerciaux de la société qui en découlent. Le parti de l’ordre disposait de moyens pécuniaires énormes. Il organisa des succursales dans toute la France. Il avait à sa solde tous les idéologues de l’ancienne société. L’influence des pouvoirs existants lui était acquise. Il possédait une armée de vassaux bénévoles dans toute la masse des paysans et des petits bourgeois qui, étrangers encore au mouvement révolutionnaire, voyaient, dans les grands dignitaires de la propriété, les représentants naturels de leur petite propriété et de leurs maigres privilèges. Représenté sur l’étendue du territoire par un nombre énorme de roitelets, ce parti pouvait punir comme une insurrection l’échec de ses candidats, renvoyer les ouvriers rebelles, les salariés agricoles, domestiques, commis, employés de chemins de fer, écrivains, enfin tous les fonctionnaires, ses subordonnés à la mode bourgeoise. Ce parti pouvait enfin entretenir l’illusion que la Constituante républicaine avait entravé, la puissance miraculeuse du Bonaparte du 10 décembre. Nous n’avons pas compté les bonapartistes dans le parti de l’ordre. Ils ne formaient pas une fraction sérieuse de la classe bourgeoise. C’était un assemblage de vieux invalides superstitieux et de jeunes chevaliers d’industrie incrédules. – Le parti de l’ordre triompha aux élections. Il envoya une grande majorité à l’Assemblée législative.\par
En face de la classe bourgeoise, de la contre-révolution coalisée, les fractions de la petite bourgeoisie et de la classe paysanne qui avaient déjà été révolutionnaires, devaient s’unir au défenseur attitré des intérêts révolutionnaires, au prolétariat. Nous avons vu que, dans le Parlement, les porte-paroles démocrates de la petite bourgeoisie, la Montagne, s’étaient, à la suite de leurs défaites parlementaires rapprochés des interprètes socialistes du prolétariat et que, en dehors du Parlement, la véritable petite bourgeoisie s’était rapprochée des vrais prolétaires à la suite de l’échec des « concordats à l’amiable », du triomphe brutal des intérêts bourgeois, à la suite de la banqueroute. Le 27 janvier, la Montagne et les socialistes avaient fêté leur réconciliation. Dans le grand banquet de février, en 1849, on renouvela le pacte d’alliance. Le parti social et le parti démocratique, celui des ouvriers et celui des petits bourgeois, s’unirent en un parti \emph{social-démocratique}, le parti \emph{rouge}.\par
La République française, un instant paralysée par l’agonie qui succéda aux journées de juin, avait, depuis la levée de l’état de siège, depuis le 14 octobre, traversé une série continue d’agitations fiévreuses. D’abord, la lutte au sujet de la présidence ; puis la lutte du président contre la Constituante ; la lutte au sujet des clubs, le procès de Bourges, qui, en face des petites figures du président, des royalistes coalisés, des républicains honnêtes, de la Montagne démocratique, des doctrinaires socialistes, du prolétariat, fit apparaître les révolutionnaires véritables sous l’aspect de ces monstres primitifs qui ne se montrent qu’en deux cas : un déluge peut les laisser remonter à la surface de la société, ou bien ils précédent un déluge ; l’agitation électorale ; l’exécution des meurtriers de Bréa ; les continuels procès de presse ; l’intrusion violente de la police gouvernementale dans les banquets ; les impudentes provocations des royalistes ; la mise au pilori des figures de Louis Blanc et de Caussidière ; la lutte ininterrompue entre la République constituée et la Constituante, lutte ramenant à chaque instant la Révolution à son point de départ, où, à tout moment, le vainqueur devenait le vaincu et réciproquement, où, en un clin d’œil, la position des partis et des classes, leur antagonisme et leur union se modifiaient ; la marche rapide de la contre-révolution européenne ; la lutte glorieuse de la Hongrie ; la levée de boucliers des Allemands ; l’expédition romaine ; la honteuse défaite de l’armée française devant Rome ; – dans ce tourbillon, dans la calamité de ce trouble historique, dans ce dramatique flux et reflux des passions, des espérances et des désillusions révolutionnaires, les différentes classes de la société française ne pouvaient plus compter que par semaines leurs époques d’évolution qu’elles comptaient jadis par demi-siècles. Une partie considérable des paysans et des provinces était révolutionnée. Non seulement on était désillusionné sur le compte de Napoléon, mais le parti rouge offrait à la place du nom le contenu, à la place de la libération illusoire des impôts le remboursement du milliard payé aux légitimistes, la réglementation des hypothèques et la suppression de l’usure.\par
L’armée, elle-même, était atteinte de la fièvre révolutionnaire. Elle avait, en élisant Bonaparte, voté pour la victoire et il lui avait donné la défaite. Elle avait voté pour le petit caporal derrière lequel se cache le grand capitaine révolutionnaire et il lui rendait les grands généraux derrière lesquels se dissimulait le caporal en guêtres blanches. Il n’est pas douteux que le parti rouge, le parti des démocrates coalisés ne dût, sinon remporter la victoire, du moins avoir à fêter de grands succès ; que Paris, l’armée, une grande partie des provinces ne dussent voter pour lui. Ledru-Rollin, le chef de la Montagne, fut élu par cinq départements. Aucun des chefs du parti de l’ordre ne remporta semblable victoire, aucun nom du parti prolétaire proprement dit. Cette élection nous dévoile le secret du parti démocrate-socialiste. La Montagne, avant-garde parlementaire de la petite bourgeoisie démocrate, était, contrainte de s’unir aux doctrinaires socialistes du prolétariat. Le prolétariat était obligé par la formidable défaite de Juin de se relever par des victoires intellectuelles, incapable, vu l’état de développement des autres classes, de s’emparer de la dictature révolutionnaire, et contraint de se jeter dans les bras des théoriciens de son émancipation, des fondateurs de sectes socialistes ; d’autre part, les paysans révolutionnaires, l’armée, les provinces se rangeaient derrière la Montagne qui devenait ainsi le chef du camp révolutionnaire et qui, par son entente avec les socialistes, avait éloigné tout antagonisme du parti de la révolution. Dans la dernière moitié de l’existence de la Constituante, la Montagne y représentait le pathos républicain et avait fait oublier les fautes, commises par elle, sous le gouvernement provisoire, sous la commission exécutive, pendant les journées de Juin. À mesure que le parti du \emph{National}, conformément à l’imperfection de sa nature, se laissait accabler par le ministère royaliste, le parti de la Montagne, tenu à l’écart au temps de l’omnipotence du \emph{National}, s’élevait et devenait le représentant parlementaire de la révolution. En fait le parti du \emph{National} ne pouvait opposer aux fractions royalistes que des personnalités ambitieuses et des bourdes idéalistes. La Montagne, par contre, représentait une masse placée entre la bourgeoisie et le prolétariat, masse dont les intérêts matériels exigeaient des institutions démocratiques. Par rapport aux Cavaignac et aux Marrast, Ledru-Rollin et la Montagne se trouvaient dans la vérité révolutionnaire, et puisaient dans la conscience de cette situation grave un courage d’autant plus grand que la manifestation de l’énergie révolutionnaire se bornait à des effets parlementaires, dépôts d’actes d’accusation, menaces, élévation de la voix, discours tonitruants : on se livrait à des extrémités en parole seulement. Les paysans se trouvaient à peu près dans la même situation que les petits bourgeois et avaient à présenter à peu près les mêmes revendications sociales. Toutes les couches intermédiaires de la société, dans la mesure où elles étaient entraînées dans le mouvement révolutionnaire, devaient voir en Ledru-Rollin leur héros. Ledru-Rollin était le personnage de la petite bourgeoisie. En face du parti de l’ordre, les réformateurs de cet ordre, réformateurs à demi-conservateurs, à demi-révolutionnaires et parfaitement utopistes, devaient prendre le premier rang.\par
Le parti du \emph{National}, les « amis de la Constitution quand même\footnote{En français dans le texte} », « les républicains purs et simples\footnote{En français dans le texte} » furent complètement défaits aux élections. Une minorité ridicule de ses membres fut envoyée à l’Assemblée législative. Ses chefs les plus connus, Marrast lui-même, le « rédacteur en chef », l’Orphée de la République honnête, disparurent de la scène.\par
Le 29 mai, l’Assemblée législative se réunit. Le 11 juin, le conflit du 8 mai se renouvela. Ledru-Rollin déposa au nom de la Montagne une demande de mise en accusation du président et du ministère pour violation de la constitution, pour avoir fait bombarder Rome. Le 12 juin, l’Assemblée législative rejeta la demande de mise en accusation comme la Constituante l’avait fait le 11 mai. Mais cette fois, le prolétariat fit descendre la Montagne dans la rue, non pour s’y battre il est vrai, mais pour y processionner. Il suffit de dire que la Montagne était à la tête de ce mouvement pour qu’on sache qu’il fut vaincu. Juin 1849 fut une caricature aussi ridicule qu’indécente de juin 1848. L’importance de la retraite du 13 juin ne fut éclipsée que par l’importance du bulletin qu’en donna Changarnier, le grand homme improvisé par le parti de l’ordre. Chaque époque a besoin de ses grands hommes et si elle ne les trouve pas, elle les invente, comme dit Helvétius.\par
Le 20 décembre, il n’existait encore qu’une moitié de la République bourgeoise constituée, le \emph{président} ; le 29 mai y ajouta le complément, l’\emph{Assemblée législative}. En juin 1848, la République bourgeoise qui se constituait, avait marqué sa naissance en gravant sur les tables de l’histoire une bataille indicible livrée au prolétariat ; en juin 1849, la République bourgeoise constituée y inscrivait une comédie innommable jouée avec la petite bourgeoisie. Juin 1849 était la Nemesis de juin 1848. En juin 1849, les ouvriers ne furent pas vaincus ; mais les petits bourgeois qui se trouvaient entre les prolétaires et la révolution furent abattus. En juin 1849, ce n’était plus la tragédie sanglante entre le salariat et le capital, mais la comédie lamentable entre le débiteur et le créancier. Le parti de l’ordre avait vaincu. Il était tout puissant. Il lui restait à montrer ce qu’il était.
\chapterclose


\chapteropen

\chapter[{III. Du 13 juin 1849 au 10 mars 1850}]{III. Du 13 juin 1849 au 10 mars 1850}
\renewcommand{\leftmark}{III. Du 13 juin 1849 au 10 mars 1850}


\chaptercont
\noindent Le 20 décembre, la \emph{République constitutionnelle} n’avait encore montré \emph{qu’une} des faces de sa tête de Janus, la face exécutive sous les traits fuyants et plats de Louis Bonaparte. Le 29 mai, elle parut sous sa seconde face, sa face \emph{législative}, toute parsemée des cicatrices qu’y avaient laissé les orgies de la Restauration et de la Monarchie de Juillet. La \emph{République constitutionnelle} était parfaite dès lors. L’Assemblée nationale complétait l’État républicain : ainsi se trouvait parachevée la forme politique qui correspond à la domination bourgeoise, à la suprématie des deux grandes fractions royalistes dont est formée la bourgeoisie française : des légitimistes et les orléanistes coalisés, \emph{du parti de l’ordre}. Tandis que la République devenait ainsi la propriété des partis royalistes, la coalition européenne des puissances contre-révolutionnaires entreprenait simultanément une croisade générale contre les derniers asiles des révolutions de mars. La Russie faisait irruption en Hongrie, la Prusse marchait contre l’armée constitutionnelle de l’Empire, et Oudinot bombardait Rome. La crise européenne approchait incontestablement du moment décisif. Les yeux de toute l’Europe se dirigeaient sur Paris et les yeux de tout Paris étaient fixés sur \emph{l’Assemblée législative}.\par
Le 11 juin, Ledru-Rollin monta à la tribune. Il ne tint pas de discours. Il formula un réquisitoire contre les ministres, nu, simple, réel, concentré, puissant.\par
L’attaque contre Rome est une atteinte portée à la constitution ; l’attaque essuyée par la République romaine est une attaque dirigée contre la République française. L’article V de la constitution dit, en effet : « La République française ne tourne jamais ses forces contre la liberté d’aucune nation, » – et le président dirige l’armée française contre la liberté romaine. L’article IV de la constitution interdit au pouvoir exécutif de déclarer aucune guerre sans l’approbation de l’Assemblée nationale. Le vote, émis par la Constituante le 8 mai, ordonne expressément aux ministres de ramener le plus vite possible l’expédition romaine à sa destination primitive. Par suite, il leur interdit tout aussi expressément d’entrer en guerre avec Rome – et Oudinot bombarde Rome. Ainsi Ledru-Rollin faisait de la constitution même un témoin à charge contre Bonaparte et ses ministres. Lui, le tribun de la constitution lance à la face de l’Assemblée nationale cette déclaration menaçante : « Les républicains sauront faire respecter la constitution par tous les moyens, même par la force des armes ! » \emph{Par la force des armes} ! répéta cent fois l’écho de la Montagne. La majorité répondit par un tumulte effroyable. Le président de l’Assemblée rappela Ledru-Rollin à l’ordre. Ledru-Rollin renouvela sa déclaration provocante et déposa enfin sur le bureau la proposition de mise en accusation de Bonaparte et de ses ministres. L’Assemblée nationale, par 361 voix contre 203, vota l’ordre du jour pur et simple sur le bombardement de Rome.\par
Ledru-Rollin pensait-il que la constitution triompherait de l’Assemblée, et que l’Assemblée l’emporterait sur le président ?\par
La constitution interdisait, il est vrai, toute attaque dirigée contre la liberté des peuples étrangers. Mais ce que l’armée combattait à Rome, ce n’était pas, d’après le ministère, la « liberté », mais bien le « despotisme de l’anarchie ». En dépit de toutes les expériences qu’elle avait pu faire à la Constituante, la Montagne n’avait-elle donc jamais compris que l’interprétation de la constitution n’appartenait pas à ceux qui l’avaient élaborée, mais uniquement à ceux qui l’avaient acceptée ? Sa lettre ne devait-elle pas traduire quelque chose de viable, et la signification bourgeoise n’était-elle pas la seule viable ? Bonaparte et la majorité royaliste de l’Assemblée n’étaient-ils pas les interprètes authentiques de la constitution, comme le prêtre est l’interprète authentique de la Bible et le juge, celui de la loi ? L’Assemblée nationale, fraîchement issue des élections générales devait-elle se croire liée par les dispositions testamentaires de la Constituante défunte, alors qu’un Odilon-Barrot avait pu briser sa volonté quand elle était en vie ? Ledru-Rollin en s’appuyant sur la décision de la Constituante du 8 mai, oubliait que cette même Assemblée avait repoussé le 11 mai sa première proposition de mise en accusation de Bonaparte et des ministres. Elle avait absous le président et le cabinet, sanctionné comme constitutionnelle l’attaque contre Rome. Il ne faisait appel que d’un jugement déjà caduc. Il en appelait de la Constituante républicaine à la Législative royaliste. La constitution elle-même demandait aide à l’insurrection en invitant dans un article spécial chaque citoyen à la défendre. Ledru-Rollin s’appuyait sur cet article. Mais les pouvoirs publics ne sont-ils pas aussi organisés pour protéger la constitution, et la violation ne commence-t-elle pas seulement au moment où l’un des pouvoirs constitutionnels se rebelle contre les autres ? Et le président de la République, les ministres de la République, l’Assemblée nationale de la République étaient dans l’accord le plus parfait.\par
Ce que la Montagne cherchait, le 11 juin, c’était \emph{une insurrection dans le domaine de la raison pure}, une \emph{insurrection purement parlementaire}. La majorité, intimidée à la pensée d’un soulèvement à main armée des masses populaires, devait-elle briser sa propre puissance dans la personne de Bonaparte et de ses ministres et détruire la signification de sa propre élection ? La Constituante n’avait-elle pas cherché de même à casser l’élection de Bonaparte, quand elle avait insisté avec tant d’entêtement pour le renvoi du ministère Barrot-Falloux ?\par
Il ne manquait pas d’exemples d’insurrections parlementaires remontant à l’époque de la Convention où le rapport de majorité à minorité avait été renversé de fond en comble ; et pourquoi la jeune Montagne n’aurait-elle pas réussi aussi bien que l’ancienne ? De plus le moment ne semblait pas défavorable à une pareille entreprise. L’agitation populaire revêtait à Paris un caractère sérieux. L’armée, autant que le faisaient prévoir ses votes, ne semblait pas bien disposée pour le gouvernement. La majorité législative était encore de date trop récente pour s’être consolidée. De plus elle était composée de vieux. Si une insurrection parlementaire réussissait à la Montagne, les rênes de l’État tombaient immédiatement entre ses mains. De son côté la petite bourgeoisie démocrate, comme toujours, ne demandait rien tant que de voir se décider le combat au-dessus de sa tête entre les esprits défunts du Parlement. Enfin la petite bourgeoisie démocrate et ses représentants de la Montagne atteignaient leur but principal par une insurrection parlementaire : on brisait la puissance de la bourgeoisie sans déchaîner le prolétariat ou sans lui laisser entrevoir sa délivrance autrement qu’en perspective. On se servait du prolétariat sans qu’il devînt dangereux.\par
Après le vote de l’Assemblée nationale du 11 juin, quelques membres de la Montagne se rencontrèrent avec des délégués des sociétés secrètes ouvrières. Ces derniers insistaient pour livrer bataille le soir même. La Montagne rejeta décidément ce plan : elle ne voulait à aucun prix abandonner la direction du mouvement. Ses alliés lui étaient aussi suspects que ses adversaires et cela à juste titre. Le souvenir de Juin 1848 était plus vivant que jamais dans les rangs du prolétariat parisien. Il n’en était pas moins contraint de se lier à la Montagne par une alliance. La Montagne, en effet, représentait la plus grande partie des départements. Elle outrait son influence sur l’armée. Elle disposait de la fraction démocrate de la garde nationale. Elle avait derrière elle la puissance morale de la boutique. Commencer à ce moment l’insurrection contre son gré, c’était pour le prolétariat, décimé d’ailleurs par le choléra et chassé en grande partie de Paris par le chômage, renouveler inutilement les journées de Juin sans que la situation imposât ce combat incertain. Les délégués du prolétariat firent la seule chose qui fût rationnelle. Ils engagèrent la Montagne à se \emph{compromettre}, à sortir des limites de la lutte parlementaire au cas où son acte d’accusation serait repoussé. Pendant toute la journée du 13 juin, le prolétariat conserva cette attitude expectante et sceptique. Il attendit que se produisît un engagement sérieux, irrémédiable, entre la garde nationale démocrate et l’armée, pour se jeter alors dans la lutte et faire dépasser à la révolution le but petit bourgeois qu’on lui assignait. Pour le cas où l’on aurait triomphé, la commune prolétarienne qui devait se dresser en face du gouvernement officiel était déjà constituée. Les ouvriers parisiens avaient appris à l’école de Juin 1848.\par
Le 12 juin, le ministre Lacrosse fit lui-même la proposition de passer immédiatement à la discussion de l’acte d’accusation. Le gouvernement avait, pendant la nuit, pris toutes ses dispositions d’attaque et de défense. La majorité de l’Assemblée nationale était décidée à faire descendre dans la rue sa minorité rebelle. La minorité elle-même ne pouvait plus reculer. Le sort en était jeté : 377 voix contre 8 repoussèrent la mise en accusation. La Montagne qui s’était abstenue se précipite en grondant dans les salles de propagande et dans les bureaux de la \emph{Démocratie pacifique}. Un débat se déroula, long, bruyant, interminable. La Montagne était décidée à faire respecter la constitution par tous les moyens, \emph{sauf par la force des armes}. Cette décision reçut l’appui d’un manifeste et d’une députation des « amis de la constitution ». C’est ainsi que s’appelaient les ruines de la coterie du \emph{National}, du parti républicain bourgeois. Alors que six de ses représentants avaient voté \emph{contre}, les autres, tous ensemble avaient voté pour le rejet de la mise en accusation. Tandis que Cavaignac mettait son sabre au service du parti de l’ordre, la plus grande partie de la coterie du \emph{National}, qui n’était pas au Parlement, saisit avec empressement l’occasion de quitter la position de paria politique et d’entrer dans les rangs du parti démocrate. N’étaient-ils pas les hérauts naturels de ce parti qui se couvrait des mêmes armes, et combattait pour le même \emph{principe}, pour la \emph{constitution}.\par
La Montagne resta en travail jusqu’à l’aurore. Elle accoucha d’une \emph{proclamation au peuple} qui parut le matin du 13 juin dans deux journaux socialistes à une place plus ou moins honteuse. Elle mettait le président, les ministres, la majorité de l’Assemblée constituante \emph{hors la constitution}, et invitait la garde nationale, l’armée et finalement le peuple à se soulever. \emph{Vive la constitution} était le mot d’ordre. Il ne signifiait qu’une chose : \emph{À bas la Révolution}.\par
À la proclamation constitutionnelle de la Montagne répondit, le 13 juin, une \emph{démonstration pacifique} des petits bourgeois, une procession partant du « Château-d’Eau » et passant par les boulevards, 30 000 hommes sans armes, gardes nationaux pour la plupart, mêlés aux membres des sections des sociétés secrètes ouvrières, roulant aux cris de \emph{Vive la constitution} ! poussés mécaniquement, froidement par les membres de la démonstration, et que le peuple qui s’agitait sur les boulevards reprenait ironiquement au lieu d’en faire un grondement de tonnerre. Ce concert à plusieurs parties manquait de voix de poitrine. Quand le cortège vint battre le local des « amis de la constitution », quand parut sur le pignon de la maison un héraut constitutionnel soudoyé fendant violemment l’air de son chapeau-claque et faisant sortir d’une poitrine de stentor une grêle de \emph{Vive la constitution} ! sur la tête des pèlerins, ceux-ci semblèrent un moment vaincus eux-mêmes par le comique de la situation. On sait que le cortège, arrivé sur les boulevards à l’entrée de la « rue de la Paix » fut reçu d’une façon très peu parlementaire par les dragons et les chasseurs de Changarnier. Il se dispersa en un clin d’œil dans toutes les directions et lança derrière lui le cri clairsemé de « aux armes ! » uniquement pour que l’appel fait le 11 juin au Parlement reçût exécution.\par
La majorité des Montagnards, rassemblés dans la « rue du Hasard », s’évanouit alors que la dispersion brutale de la procession pacifique, les bruits confus de meurtres commis sur les boulevards sur la personne de citoyens désarmés, l’accroissement du tumulte dans la rue semblaient annoncer l’approche de l’émeute. \emph{Ledru-Rollin}, à la tête d’une poignée de députés, sauva l’honneur de la Montagne sous la protection de l’artillerie de Paris, réunie au « Palais national » ; on se rendit au « Conservatoire des Arts et Métiers » où l’on devait rencontrer la Vᵉ et la VIᵉ légion de la garde nationale. Mais les Montagnards attendirent en vain après elles. Les gardes-nationaux, prudents, abandonnaient les représentants. L’artillerie de Paris empêcha elle-même le peuple d’élever des barricades. Une confusion chaotique rendait toute décision impossible. Les troupes de lignes s’avancèrent, la baïonnette croisée. Une partie des représentants furent faits prisonniers, les autres s’échappèrent. Ainsi se termina le 13 juin.\par
Si le 23 juin 1848 avait été l’insurrection du prolétariat révolutionnaire, le 13 juin 1849 était celle de la petite bourgeoisie démocrate. Chacun de ces mouvements exprimait avec une \emph{pureté classique} la classe qui l’avait créé.\par
À Lyon seulement on en vint à un conflit opiniâtre, sanglant. Dans cette ville où la bourgeoisie et le prolétariat industriels se trouvent directement face à face, où le mouvement ouvrier n’est pas, comme à Paris, enveloppé, déterminé par le mouvement général, le 13 juin perdit, par contrecoup, son caractère originel. Dans les autres endroits de la province où il éclata, il ne mit le feu à rien : \emph{c’était un éclair de chaleur}.\par
Le 13 juin clôt la première \emph{période d’existence de la République constitutionnelle}, dont la vie normale datait de la réunion de l’assemblée législative. Tout ce prologue est rempli par la lutte bruyante du « parti de l’ordre » et de la Montagne, de la bourgeoisie et de la petite bourgeoisie qui s’oppose vainement à l’établissement de la République bourgeoise, en faveur de laquelle cependant elle avait conspiré sans interruption dans le gouvernement provisoire et dans la commission exécutive, au profit de laquelle elle s’était fanatiquement battue contre le prolétariat pendant les journées de juin. Le 13 juin brise sa résistance et fait de la \emph{dictature législative} des royalistes coalisés un « fait accompli ». À partir de cet instant, l’Assemblée nationale n’est plus que le \emph{Comité du Salut publie, du parti de l’ordre}.\par
Paris avait mis \emph{en accusation} le président, les ministres et la majorité de l’Assemblée nationale. Ils mirent Paris en \emph{état de siège}. La Montagne avait mis la majorité de l’Assemblée nationale « hors la constitution » ; la majorité traduisit la Montagne devant la « haute Cour » pour violation de la Constitution et proscrivit tous ceux qui avaient encore quelque vigueur. Elle décima la Montagne au point de la réduire à l’état de tronc sans tête ni cœur. La minorité avait été jusqu’à tenter une \emph{insurrection parlementaire}, la majorité fit une loi de son \emph{despotisme parlementaire}. Elle décréta un nouveau règlement qui anéantissait la liberté de la tribune, et autorisa le président de l’Assemblée à punir pour désordre les représentants par la censure, l’amende, la suspension de l’indemnité, l’expulsion temporaire, la cellule. C’était la férule, non l’épée qu’elle tenait suspendue au-dessus de la Montagne décapitée. Le reste des Montagnards aurait dû tenir à honneur de se retirer en masse. Un acte semblable aurait précipité la dissolution du parti de l’ordre. Il devait, en effet, se résoudre en ses éléments originels dès que l’apparence même d’une opposition ne les réunissait plus.\par
En même temps qu’on dérobait aux petits bourgeois démocrates leur \emph{pouvoir parlementaire}, on les privait aussi de leurs armes. L’artillerie de Paris ainsi que les 8ᵉ, 9ᵉ et 12ᵉ légions de la garde nationale étaient licenciées. Par contre, la légion de la haute finance qui, le 13 juin, avait envahi les imprimeries de Boulé et de Roux, brisé les presses, dévasté les bureaux des journaux républicains, arrêté arbitrairement les rédacteurs, compositeurs, imprimeurs, expéditeurs, commissionnaires, etc, reçut une approbation encourageante du haut de la tribune de l’Assemblée nationale. Sur toute la surface de la France se répétait le licenciement des gardes nationales suspectes de républicanisme.\par
Une nouvelle \emph{loi sur la presse}, une nouvelle \emph{loi sur les associations}, une nouvelle \emph{loi sur l’État de siège}, les prisons de Paris plus que pleines, les réfugiés politiques chassés, tous les journaux plus avancés que le \emph{National} suspendus, Lyon et les cinq départements limitrophes livrés à la chicane brutale du despotisme militaire, les parquets intervenant partout, l’armée des fonctionnaires, si souvent épurée déjà, épurée de nouveau, – c’étaient \emph{des lieux communs} inévitables, accompagnant toujours une victoire de la réaction. Après les massacres et les déportations de juin ils méritaient une mention uniquement parce que cette fois, on s’attaquait non plus seulement à Paris, mais aussi aux départements, non plus seulement au prolétariat, mais surtout aux classes moyennes.\par
Les lois de répression, qui laissaient le gouvernement libre de décréter l’état de siège, enchaînaient la presse encore plus étroitement et supprimaient le droit d’association, ces lois absorbèrent, pour leur confection, toute l’activité législative de l’Assemblée nationale pendant les mois de juin, de juillet et d’août.\par
Go qui caractérise cependant cette époque, c’est qu’on chercha moins à tirer un profit \emph{matériel} de la victoire qu’à la faire servir aux principes ; ce qui importe, c’est moins les décisions de l’Assemblée nationale que les motifs de ces décisions, moins la chose que la phrase, moins la phrase même que l’accent, les gestes qui la vivifient. L’expression franchement impudente du \emph{principe royaliste}, les insultes, d’une distinction méprisante, prodiguées à la République, la divulgation par coquetterie frivole des projets de restauration, en un mot la violation affectée des \emph{convenances républicaines} donnent à cette époque son ton et sa couleur. Le cri des \emph{vaincus} du 13 Juin était : \emph{Vive la constitution}. Les \emph{vainqueurs} n’étaient par suite plus obligés à l’hypocrisie du langage constitutionnel, c’est-à-dire républicain. La contre-révolution soumettait la Hongrie, l’Italie, l’Allemagne et l’on croyait déjà la restauration aux portes de la France : une vraie concurrence s’établit entre les meneurs du parti de l’ordre. C’est à l’envi qu’ils affirmaient leur royalisme dans les documents qui paraissaient au \emph{Moniteur}. À qui mieux mieux, ils confessaient les péchés qu’ils avaient pu commettre par libéralisme sous la monarchie, en faisaient amende honorable, les abjuraient devant Dieu et devant les hommes. Il ne se passait pas de jour que, du haut de la tribune de l’Assemblée nationale, on ne déclarât que la révolution de Février avait été un malheur public. À chaque séance un hobereau quelconque, légitimiste de province, constatait solennellement qu’il n’avait jamais reconnu la République, ou bien un lâche déserteur, traître à la monarchie de Juillet, racontait les exploits supplémentaires que seule la philanthropie du roi Louis-Philippe ou quelque malentendu l’avait empêché d’accomplir. Ce qu’il fallait admirer dans les journées de Février, ce n’était pas la magnanimité du peuple victorieux, mais le désintéressement, la modération des royalistes qui lui avaient permis de vaincre. Un représentant du peuple proposa qu’une partie des secours destinés aux blessés de Février fût consacrée aux \emph{gardes municipaux} qui seuls dans ces journées avaient servi la patrie. Un autre demandait l’érection par décret d’une statue équestre du duc d’Orléans sur la place du Carrousel. Thiers appelait la constitution un morceau de papier sale. Des orléanistes apparaissaient successivement à la tribune et exprimaient leur repentir d’avoir conspiré contre la monarchie légitime ; les légitimistes se reprochaient d’avoir avancé la chute de la royauté en général en se rebellant contre la royauté illégitime. Thiers déplorait d’avoir intrigué contre Molé, Molé contre Guizot, Barrot contre tous les trois. Le cri de : « Vive la République démocratique et sociale ! » fut déclaré inconstitutionnel. Le cri de : « Vive la République ! » fut considéré comme socialiste. À l’anniversaire de la bataille de Waterloo, un représentant proclama : « Je crains moins l’invasion des Prussiens que la rentrée en France des réfugiés révolutionnaires ». Baraguay d’Hilliers répondait aux plaintes qu’on lui exprimait sur le terrorisme organisé à Lyon et dans les départements voisins : « J’aime mieux la terreur blanche que la terreur rouge\footnote{En français dans le texte} ». L’Assemblée applaudissait frénétiquement chaque fois que tombait des lèvres de ses orateurs une épigramme lancée contre la République, la révolution et la constitution, en faveur de la royauté et de la Sainte-Alliance. Contrevenir aux plus petites formalités républicaines, s’adresser aux représentants sans les appeler citoyens, voilà qui excitait l’enthousiasme des chevaliers de l’ordre.\par
Les élections complémentaires du 8 juillet à Paris, survenues en plein état de siège et marquées par l’abstention d’une grande partie du prolétariat, la prise de Rome par l’armée française, la rentrée dans cette ville des éminences rouges et, à leur suite, de l’inquisition et du terrorisme des moines ajoutèrent de nouveaux triomphes à la victoire de Juin et accentuèrent l’ivresse du parti de l’ordre.\par
Enfin à la mi-août, voulant se rendre dans les conseils départementaux qui venaient de se réunir, fatigués aussi par cette orgie tendancieuse qui durait depuis plusieurs mois déjà, les royalistes décrétèrent une prorogation de deux mois de l’Assemblée nationale. Ils laissaient une commission de vingt-cinq représentants, la crème des légitimistes et des orléanistes, un Molé, un Changarnier, chargée, avec une évidente ironie, de représenter l’Assemblée et de \emph{veiller sur la République}. L’ironie était plus profonde qu’ils ne le pensaient. Condamnés par l’histoire à prêter leur concours au renversement de la monarchie qu’ils aimaient, l’histoire les condamnait à conserver la République qu’ils haïssaient.\par
La \emph{prorogation} de l’Assemblée législative termine la \emph{seconde période de l’existence de la République constitutionnelle, sa période de folie}.\par
L’état de siège avait été de nouveau levé à Paris. L’action de la presse se fit de nouveau sentir. Durant la suspension des journaux démocrates-socialistes, alors que régnaient les lois de répression et que le royalisme se déchaînait, le \emph{Siècle}, l’ancien représentant littéraire \emph{des petits bourgeois monarchistes constitutionnels, s’était républicanisé} ; la \emph{Presse}, l’ancien interprète littéraire des \emph{réformateurs bourgeois, s’était démocratisé} ; le \emph{National}, l’ancien organe classique des \emph{bourgeois républicains, s’était socialisé}.\par
Les \emph{sociétés secrètes} croissaient en extension et en activité à mesure que les \emph{clubs publics} devenaient impossibles. Les \emph{associations industrielles} formées par les \emph{travailleurs} et que l’on tolérait à titre de simples compagnies commerciales, nulles économiquement parlant, devenaient politiquement autant de moyens de grouper le prolétariat. Le 13 juin avait enlevé leurs chefs officiels aux divers partis à moitié révolutionnaires. La masse qui restait y gagna de se conduire par elle-même. Les chevaliers de l’ordre avaient intimidé en prophétisant les horreurs de la république rouge. Les excès brutaux, les atrocités hyperboréennes de la contre-révolution victorieuse en Hongrie, à Bade, à Rome lavèrent la république rouge des accusations. Et les classes moyennes de la société française commençaient à préférer les promesses de la république rouge avec sa terreur problématique à la terreur de la monarchie rouge dépourvue de toute espérance réelle. Aucun socialiste ne fit en France plus de propagande révolutionnaire que \emph{Haynau} : « À chaque capacité suivant ses œuvres\footnote{En français dans le texte} »\par
Cependant Louis Bonaparte mettait à profit les vacances de l’Assemblée nationale pour faire des voyages princiers dans les provinces. Les légitimistes les plus ardents allaient en pèlerinage à Ems visiter le descendant de Saint-Louis. La masse des amis de l’ordre représentants du peuple intriguait dans les conseils départementaux qui venaient de se réunir. Il fallait faire déclarer à ces assemblées ce que n’osait encore la majorité de l’Assemblée nationale : \emph{l’urgence de la révision immédiate de la constitution}. En vertu de cette constitution, celle-ci ne pouvait-être révisée qu’en 1852 par une Assemblée nationale, convoquée spécialement à cet effet ; mais si la majorité des assemblées départementales se prononçait dans le sens de la révision immédiate, l’Assemblée nationale ne devrait-elle pas obéir à la voix de la France ? La législative espérait des conseils départementaux ce que les nonnes de la Henriade de Voltaire attendent des Pandours. Mais les Putiphars de l’assemblée n’avaient affaire, en province, à peu d’exception près, qu’à des Josephs. L’énorme majorité ne voulait pas comprendre cette insinuation pressante. La révision de la constitution fut ajournée par le moyen même qui devait la faire naître, par le vote des assemblées départementales. La voix de la France et de la France bourgeoise avait parlé : elle s’était prononcée contre la révision.\par
Au commencement d’octobre se réunit la législative – \emph{quantum mutatus ab illo} —. Sa physionomie était complètement modifiée. Le rejet inattendu de la révision par les conseils départementaux avait replacé l’Assemblée nationale sur le terrain constitutionnel et lui avait rappelé les limites de son existence. Les pèlerinages des légitimistes à Ems avaient rendu les orléanistes défiants. Les légitimistes étaient devenus soupçonneux à la suite des menées des orléanistes à Londres. Les journaux de ces deux fractions avaient attisé le feu et pesé les prétentions réciproques de leurs prétendants. Les orléanistes et les légitimistes réunis gardaient rancune aux bonapartistes de leurs intrigues que dévoilaient les voyages princiers, les tentatives plus ou moins visibles du prétendant de s’émanciper, le langage plein de prétention des journaux bonapartistes. Louis Bonaparte était mécontent d’une assemblée qui n’admettait que la conspiration orléano-légitimiste, mécontent d’un ministère qui continuellement le trahissait au profit de cette assemblée. Enfin le ministère lui-même était divisé sur la politique romaine et sur l’\emph{impôt sur le revenu} proposé par le ministre \emph{Passy} et auquel les conservateurs trouvaient une saveur socialiste.\par
Une des premières propositions que fit le ministère Barrot à l’Assemblée nationale réunie de nouveau, fut une demande de crédit de 300 000 francs, destinés à constituer un douaire à la \emph{duchesse d’Orléans}. L’Assemblée nationale l’accorda et augmenta la dette publique de la nation française d’une somme de 7 millions de francs. Tandis que Louis-Philippe jouait ainsi avec succès le rôle de « pauvre honteux », le ministère n’osait pas proposer une augmentation de traitement en faveur de Bonaparte et l’assemblée de son côté ne paraissait pas disposée à l’accorder. Et Bonaparte comme toujours se trouvait en présence du dilemme : \emph{Aut Cæsar, aut Clichy}.\par
La seconde demande de crédit du ministre qui s’élevait à 9 millions de francs destinés à payer \emph{les frais de l’expédition romaine} rendit plus tendues encore les relations entre Bonaparte d’un côté, les ministres et l’Assemblée de l’autre. Louis Bonaparte avait fait paraître dans le \emph{Moniteur} une lettre adressée à son officier d’ordonnance, Edgard Ney ; où il astreignait le gouvernement papal à des garanties constitutionnelles. Le pape, de son côté, avait prononcé une allocution \emph{motu proprio} où il repoussait toute restriction apportée à son pouvoir restauré. Avec sa lettre, Bonaparte soulevait avec une indiscrétion voulue le voile qui couvrait son cabinet pour se montrer à la galerie sous les traits d’un génie bien intentionné, mais entravé, et méconnu dans sa propre maison. Ce n’était pas la première fois qu’il faisait le coquet et se parait des « coups d’ailes furtifs d’un esprit libre ». \emph{Thiers}, le rapporteur de la commission ignora complètement les « coups d’ailes » de Bonaparte et se contenta de traduire en français l’allocution du pape. Ce ne fut pas le ministère, mais \emph{Victor Hugo} qui essaya de sauver le président par un ordre du jour où l’Assemblée devait approuver la lettre de Bonaparte. \emph{Allons donc ! Allons donc} ! La majorité enterra la proposition de Hugo sous cette interjection d’une légèreté irrévérencieuse. La politique du président ? La lettre du président ? Le président lui-même ? \emph{Allons donc ! Allons donc} ! Qui diable prend donc monsieur Bonaparte « au sérieux ? » Pensez-vous, monsieur Victor Hugo, que nous vous croyions, que vous croyiez au président ? \emph{Allons donc ! Allons donc} !\par
La rupture entre Bonaparte et l’Assemblée fut enfin précipitée par la discussion sur le rappel des d’Orléans et des Bourbons. Faute du ministère, ce fut le cousin du président, le fils de l’ex-roi de Westphalie qui déposa cette proposition. Elle n’avait d’autre but que de mettre au même rang les prétendants, orléaniste et légitimiste, ou plutôt de les placer dans une situation inférieure à celle du prétendant bonapartiste qui lui, du moins, était, en fait, à la tête de l’État.\par
Napoléon Bonaparte avait été assez irrévérencieux pour réunir dans le même projet le rappel des familles royales exilées et l’amnistie en faveur des insurgés de Juin. L’indignation de la majorité l’obligea aussitôt à renoncer à cette liaison criminelle établie entre le sacré et l’infâme, entre les races royales et l’engeance prolétarienne, entre les astres de la société et les feux follets de ses bourbiers. Il fallut accorder à chacune des propositions le rang qui lui était dû. L’Assemblée repoussa énergiquement le rappel des familles royales et \emph{Berryer}, ce Démosthène des légitimistes, dissipa toute équivoque sur le sens de ce vote. La dégradation civile des prétendants, tel est le but que l’on poursuit ! On veut leur dérober leur dernière auréole, la dernière majesté qui leur reste, la \emph{majesté de l’exil} ! Que penserait-on, s’écria Berryer, de celui des prétendants qui, oublieux de son illustre origine, reviendrait vivre en simple particulier ? On ne pouvait dire plus clairement à Bonaparte que sa situation présente ne lui avait rien conféré. Si les royalistes coalisés avaient besoin qu’en France un \emph{homme neutre} siégeât sur le fauteuil présidentiel, les prétendants à la couronne sérieux devaient rester dérobés aux yeux des profanes par le lointain de l’exil.\par
Le 1ᵉʳ novembre, Louis Bonaparte répondit à la Législative par un message où il annonçait, en termes assez brusques, le renvoi du ministère Barrot et la constitution d’un autre cabinet. Le ministère Barrot-Falloux était le ministère de la coalition royaliste, le ministère d’Hautpoul était celui de Bonaparte ; c’était l’organe représentant le président auprès de l’Assemblée ; c’était le \emph{ministère des commis}.\par
Bonaparte n’était plus \emph{l’homme neutre} du 10 décembre 1848. La possession du pouvoir exécutif avait groupé autour de lui une quantité d’intérêts. La lutte contre l’anarchie obligeait le « parti de l’ordre » lui-même à augmenter l’influence présidentielle, et si le président \emph{n’était plus} populaire, le « parti de l’ordre », lui, était \emph{impopulaire}. Bonaparte ne pouvait-il espérer, en mettant à profit leur rivalité et la nécessité d’une restauration monarchique quelconque, contraindre les orléanistes et les légitimistes à reconnaître le \emph{prétendant neutre} ?\par
Du 1ᵉʳ novembre 1849 date la troisième période d’existence de la République constitutionnelle, période qui se termine au 10 mars 1850. Elle n’est pas marquée seulement par le jeu régulier des institutions constitutionnelles tant admiré par Guizot, par la dispute entre le pouvoir exécutif et le pouvoir législatif. Vis-à-vis des velléités de restauration des orléanistes et des légitimistes coalisés, Bonaparte représente le titre juridique du pouvoir réel qu’il exerce : la République. Vis-à-vis des velléités de restauration de Bonaparte, le « parti de l’ordre » représente le titre de la suprématie exercée en commun par les deux fractions : la République. Vis-à-vis des orléanistes, les légitimistes, vis-à-vis des légitimistes, les orléanistes représentent le \emph{statu quo} : la République. Toutes ces diverses fractions du « parti de l’ordre », dont chacun possède \emph{in petto} son roi propre et conserve l’espoir de sa propre restauration, font prévaloir, en présence des velléités d’usurpation et de relèvement de leurs rivales, la forme commune de la domination bourgeoise : la République, où les revendications particulières se neutralisent et se réservent.\par
Kant, considérant que la République est la seule forme rationnelle de l’État, en fait un postulat de la raison pratique dont la réalisation n’est jamais atteinte, mais qu’il faut constamment se poser comme but et avoir à l’esprit. Les royalistes pensaient de même à l’égard de la \emph{royauté}.\par
Ainsi la République constitutionnelle, sortie des mains des républicains bourgeois à l’état de formule idéologique vide, devint entre les mains des royalistes coalisés une forme vivante à contenu plein. Et Thiers disait plus vrai qu’il ne le pensait quand il prétendait : « Nous royalistes, nous sommes les vrais soutiens de la République constitutionnelle. »\par
Le renversement du ministère de la coalition, son remplacement par le ministère des commis avait une seconde signification. Le ministre des Finances s’appelait \emph{Fould}. Fould ministre des Finances, c’est la richesse nationale de la France livrée à la Bourse, les deniers de l’État administrés par la Bourse et au profit de la Bourse. La nomination de Fould, c’était la restauration de l’aristocratie financière paraissant au \emph{Moniteur}. Cette dernière restauration complétait les précédentes, c’était un anneau de plus ajouté à la chaîne.\par
Louis-Philippe n’avait jamais osé faire d’un véritable « loup-cervier\footnote{En français dans le texte} » (Bœrsenwolf) un ministre des finances. Sa royauté était le nom idéal que portait la domination de la haute bourgeoisie. Aussi, dans ses ministères, les intérêts privilégiés devaient-ils recevoir des dominations idéologiquement désintéressées. La République bourgeoise mettait partout en évidence ce que les diverses monarchies, légitimiste ou orléaniste, laissaient à l’arrière-plan. Elle matérialisait ce que l’on idéalisait autrefois. Elle remplaçait les vocables consacrés par les noms propres bourgeois des intérêts de classes dominants.\par
Toute notre exposition a montré que la République, à dater du premier jour de son existence, loin de renverser l’aristocratie financière, ne fit que la consolider ; mais les concessions qu’on lui faisait étaient une nécessité à laquelle on se soumettait sans qu’on fît rien pour la faire naître. Avec Fould, l’initiative gouvernementale revenait à l’aristocratie financière.\par
On se demandera comment la bourgeoisie coalisée pouvait supporter, tolérer la suprématie de la finance, alors que, sous Louis-Philippe, la domination de cette aristocratie impliquait l’exclusion et la subordination des autres fractions bourgeoises ?\par
La réponse est simple.\par
D’abord, l’aristocratie financière constitue une fraction d’une importance décisive de la coalition royaliste dont le gouvernement commun s’appelle la République. Est-ce que les capacités, les interprètes des orléanistes ne sont pas les alliés et les complices de l’aristocratie financière ? N’est-elle pas elle-même la phalange dorée de l’orléanisme ? Pour ce qui est des légitimistes, n’avaient-ils pas pris part à la Bourse, à toutes les orgies de spéculation, sur les mines, les chemins de fer ? D’ailleurs, l’alliance de la grande propriété foncière et de la haute finance est un \emph{fait normal} : nous n’en voulons pour preuve que l’\emph{Angleterre} ou l’\emph{Autriche} même.\par
Dans un pays comme la France, où l’importance de la production nationale n’est pas proportionnée au montant de la dette ; où la rente sur l’État est l’objet essentiel de la spéculation ; où la Bourse constitue le marché principal ; où vient chercher emploi le capital qui veut se mettre en valeur improductivement, dans un pays semblable, il est nécessaire qu’une masse innombrable d’individus venus de toutes les classes bourgeoises ou semi-bourgeoises aient part à la dette publique, au jeu de Bourse, à la finance.\par
Tous ces intéressés subalternes ne trouvent-ils pas leurs appuis et leurs chefs naturels dans la fraction qui représente ces intérêts dans les proportions les plus colossales, en totalité même ?\par
Qu’est-ce qui détermine la main-mise de la haute finance sur les deniers de l’État ? L’endettement toujours croissant de cet État. Et qu’est-ce qui cause l’endettement de l’État lui-même ? L’excès constant des dépenses sur les recettes, disproportion qui forme à la fois la cause et l’effet des emprunts publics.\par
Pour remédier à cet endettement, l’État doit restreindre ses dépenses, simplifier, diminuer le mécanisme gouvernemental, il doit gouverner le moins possible, employer le moins de personnel possible, entrer le moins possible en relation avec la société bourgeoise. Cette voie était impraticable pour le « parti de l’ordre », obligé de s’immiscer officiellement dans tout par raison d’État, d’être partout présent, à tous les instants, par l’entremise des fonctionnaires publics. La nécessité de disposer de ces moyens de répression augmentait à mesure que sa domination, que les conditions d’existence de sa classe étaient menacées davantage. On ne peut réduire la gendarmerie au moment où les attentats contre les personnes et contre les propriétés se multiplient.\par
Ou bien l’État doit chercher à éviter les dettes, essayer d’arriver à un équilibre momentané du budget en faisant peser sur les classes les plus riches des \emph{impôts extraordinaires}. Pour soustraire la richesse nationale à l’exploitation de la Bourse, le « parti de l’ordre » devait-il sacrifier sa propre fortune sur l’autel de la patrie ? « Pas si bête\footnote{En français dans le texte} ».\par
Il était donc impossible, en France, de modifier le déficit sans bouleverser complètement l’État. Ce déficit impliquait l’endettement de l’État ; cet endettement supposait la domination du commerce dont la dette publique est l’objet, la suprématie des créanciers d’État, banquiers, marchands d’argent, « loups-cerviers ». Une seule fraction du « parti de l’ordre » était intéressée au renversement de l’aristocratie financière : c’étaient les fabricants. Nous ne voulons parler ici ni des petits ni des moyens industriels ; nous avons en vue ces régents des intérêts de la fabrique qui avaient formé, sous Louis-Philippe, le fond principal de l’opposition dynastique. Leur intérêt est, incontestablement, de réduire les frais de production, par suite de diminuer les impôts qui l’obèrent, et de restreindre la dette publique dont les intérêts entrent dans les impôts. L’intérêt des fabricants exigeait donc la chute de l’aristocratie financière.\par
En Angleterre – et les plus grands fabricants français ne sont que des petits bourgeois en comparaison de leurs rivaux anglais – nous rencontrons véritablement des fabricants, un Cobden, un Bright à la tête de la croisade contre l’aristocratie financière. Pourquoi n’en est-il pas de même en France ? En Angleterre, c’est l’industrie qui prédomine, en France, c’est l’agriculture. En Angleterre, l’industrie a besoin du \emph{free trade}, en France elle exige la protection, un monopole national s’ajoutant aux autres. L’industrie française ne règne pas en maîtresse sur la production de la France : aussi les industriels français ne dominent-ils pas la bourgeoisie française. Pour faire prévaloir leurs intérêts sur les autres fractions de la bourgeoisie, les fabricants ne peuvent, comme en Angleterre, prendre la tête du mouvement et faire ainsi prédominer leurs intérêts. Il leur faut être à la suite de la révolution et servir des intérêts contraires à ceux de la totalité de leur classe. En Février, ils avaient méconnu la situation, Février leur servit de leçon. Et qui donc est plus directement menacé par les ouvriers que l’employeur, le capitaliste industriel ? Le fabricant devint donc nécessairement un adepte fanatique du « parti de l’ordre ». Qu’est-ce que l’atteinte portée au profil par la finance, comparée à \emph{la suppression du profit par le prolétariat} ?\par
En France, le petit bourgeois fait ce que, normalement, devrait faire le bourgeois industriel. L’ouvrier fait ce qui est, normalement, l’affaire du petit bourgeois. Et le problème qui intéresse l’ouvrier, qui le résout donc ? Personne. En France, on ne résout pas ce problème, on le proclame. Il ne sera jamais résolu dans les limites nationales. La guerre des classes, menée au sein de la société française devient une guerre universelle, où les nations se trouvent en présence. La solution ne peut intervenir qu’au moment où, grâce à une guerre internationale, le prolétariat se trouvera à la tête de la nation qui règne sur le marché du monde, à la tête de l’Angleterre. La révolution alors, trouvant là non son terme, mais son origine et son organisation n’aura plus le souffle court. La génération actuelle ressemble aux Israélites que Moïse conduit à travers le désert. Il ne lui suffit pas de conquérir un nouveau monde. Elle doit disparaître pour faire place à ceux qui sont prédestinés.\par
Revenons à Fould.\par
Le 14 novembre 1849, Fould monta à la tribune de l’Assemblée nationale et exposa son système financier. C’était l’apologie de l’ancien régime des impôts ! Le maintien des droits sur le vin ! Le retrait de l’impôt sur le revenu dû à Passy !\par
Passy pourtant n’était pas un révolutionnaire. Ancien ministre de Louis-Philippe, il comptait au nombre des puritains de la force de Dufaure. Il avait été un des plus intimes confidents de Teste, ce bouc émissaire de la monarchie de Juillet. Passy, lui aussi, avait chanté les louanges de l’ancien régime des impôts, préconisé le maintien des droits sur le vin, mais il avait aussi dévoilé le déficit. Il avait proclamé la nécessité d’une nouvelle taxe, l’impôt sur le revenu, si l’on voulait éviter la banqueroute publique. Fould, qui recommandait à Ledru-Rollin la banqueroute, préconisait le déficit devant la Législative. Il promettait des économies dont le secret se dévoila plus tard. Les dépenses se réduisirent de 60 millions et la dette flottante s’accrut de 200 millions : tours de passe-passe dans le groupement des chiffres, dans l’établissement du bilan, qui aboutissaient tous finalement à de nouveaux emprunts.\par
Avec Fould, l’aristocratie financière, en présence de la jalousie des autres fractions de la bourgeoisie, prit une allure moins impudente que sous Louis-Philippe. Cependant le système restait le même : augmentation constante des dettes, déguisement du déficit. Et, avec le temps, la spéculation d’autrefois se remit de plus en plus ouvertement en évidence. Nous en avons la preuve dans la loi sur les chemins de fer d’Avignon, dans les fluctuations mystérieuses subies par les papiers d’État, dans ces oscillations qui devinrent, un moment, l’objet des conversations de tout Paris, enfin dans les spéculations malheureuses de Fould et de Bonaparte sur les élections du 10 mars.\par
La restauration officielle de l’aristocratie financière ne pouvait manquer de mettre, à brève échéance, le peuple français en présence d’un nouveau 24 Février.\par
La Constituante, dans un accès de misanthropie dirigé contre son héritière, avait supprimé les droits sur les vins à dater de l’an du Seigneur 1850. Ce n’était pas en supprimant d’anciens impôts qu’on pouvait payer de nouvelles dettes. \emph{Creton}, un crétin du « parti de l’ordre », avait proposé le maintien de l’impôt des boissons avant même la prorogation de la Législative. Fould reprit cette proposition au nom du ministère bonapartiste, et le 20 décembre 1849, jour anniversaire de la proclamation de Bonaparte, l’Assemblée décida le \emph{rétablissement de l’impôt des boissons}.\par
Le précurseur de cette restauration n’était pas un financier : c’était le chef des Jésuites, \emph{Montalembert}. La déduction était d’une simplicité frappante. L’impôt est la mamelle qui allaite le gouvernement. Le gouvernement, ce sont les instruments de répression, les organes de l’autorité ; c’est l’armée, c’est la police, ce sont les fonctionnaires, les juges, les ministres, ce sont les \emph{prêtres}. Une attaque contre les impôts, c’est une attaque dirigée par les anarchistes contre les sentinelles de l’ordre qui défendent la production matérielle et intellectuelle de la société bourgeoise contre les assauts des Vandales prolétariens. L’impôt, c’est le cinquième dieu, à côté de la propriété, de la famille, de l’ordre et de la religion. L’impôt des boissons est incontestablement un impôt ; de plus ce n’est pas un impôt commun, il est traditionnel, d’esprit monarchique, respectable : « Vive l’impôt des boissons ! \emph{Three cheers and one cheer more} ! »\par
Quand le paysan français veut voir le diable, il lui donne les traits du percepteur. Du moment que Montalembert fait de l’impôt un dieu, le paysan devient athée et se jette dans les bras du diable, du socialisme. La religion de l’ordre l’avait trompé, les Jésuites l’avaient trompé ; Bonaparte l’avait trompé. Le 20 décembre 1849 avait irrémédiablement compromis le 20 décembre 1848. Le « neveu de son oncle » n’était pas le premier membre de la famille que l’impôt des boissons avait abattu, cet impôt qui, suivant une expression de Montalembert, annonce la tourmente révolutionnaire. Le vrai, le grand Napoléon déclarait à Sainte-Helène que le rétablissement des droits sur le vin avait plus contribué à sa chute que tout le reste, en lui aliénant les paysans du Midi de la France. Objet préféré de la haine populaire déjà sous Louis XIV (cf. les écrits de Boisguillebert et de Vauban), cet impôt avait été rétabli en le modifiant, il est vrai, par Napoléon en 1808. Quand la Restauration fit son apparition en France, les Cosaques n’étaient pas seuls à trotter devant elle, elle était également précédée des assurances de supprimer ces droits. La « gentilhommerie » n’avait naturellement pas besoin de tenir parole à la « gent taillable à merci et miséricorde\footnote{En français dans le texte} ». En 1830, on promit la suppression de l’impôt des boissons ; mais, en 1830, on n’avait l’habitude ni de faire ce qu’on disait, ni de dire ce qu’on faisait. En 1848, on promit la suppression de cet impôt, comme on promit tout. La Constituante, qui ne promit rien, fit, comme nous l’avons dit, une disposition testamentaire en vertu de laquelle l’impôt des boissons devait cesser d’être en vigueur à partir du 1ᵉʳ janvier 1850. Et, précisément dix jours avant cette date, la Législative le rétablit. Le peuple français était condamné à donner continuellement la chasse à cet impôt ; quand il l’avait jeté à la porte, il le voyait rentrer par la fenêtre.\par
La haine dont le peuple poursuit cette taxe s’explique. Elle rassemble, en effet, en elle, tout ce qu’il y avait de haïssable dans l’ancien système des impôts français. La façon dont elle est levée est odieuse, sa répartition est aristocratique : la taxe est la même pour les vins les plus ordinaires comme pour les plus précieux. Cet impôt croît donc en proportion géométrique dans la mesure où la fortune des consommateurs diminue. C’est un impôt progressif à rebours. Il provoque donc directement à l’empoisonnement des classes ouvrières. Il accorde une prime aux vins falsifiés, contrefaits. Il diminue la consommation en plaçant des bureaux d’octroi aux portes de toutes les villes de plus de 4 000 habitants et en transformant chaque cité en un pays étranger protégé par des taxes de douane contre les vins français. Les grands commerçants en vins, les petits à plus forte raison, les « marchands de vins » dont les bénéfices dépendent directement de la vente du vin sont autant d’ennemis déclarés de l’impôt des boissons. Enfin, en diminuant la consommation, cette taxe ferme à la production son débouché. En même temps qu’il empêche les ouvriers des villes de payer le vin, il empêche également les vignerons de le vendre. Et la France compte une population de vignerons s’élevant à 12 millions environ. On comprend, dès lors, la haine du peuple en général et surtout le fanatisme des vignerons contre l’impôt des boissons. De plus on ne voyait pas dans son rétablissement un fait isolé, plus ou moins fortuit. Les paysans possèdent une espèce de tradition historique qui se transmet de père en fils. Dans ces enseignements murmurés à l’oreille, on apprend que tout gouvernement, tant qu’il veut tromper le paysan promet, la suppression de l’impôt sur les vins, mais que, dès qu’il l’a trompé, il le conserve ou le rétablit. C’est à cet impôt que le paysan reconnaît le « bouquet » du gouvernement, sa tendance. Le rétablissement de l’impôt des boissons le 20 décembre, signifiait : \emph{Louis Bonaparte} est comme les autres. Mais il était cependant différent des autres ; il était une \emph{invention des paysans}, et ceux-ci, dans les pétitions contraires à cette taxe et qui comptaient des millions de signatures reprenaient les suffrages qu’ils avaient accordés un an auparavant au « neveu de son oncle ».\par
La population campagnarde, qui forme plus des deux tiers de la population française, consiste principalement en ces \emph{propriétaires fonciers} que l’on qualifie de libres. La première génération, affranchie gratuitement des charges féodales par la révolution de 1789 n’avait pas payé la terre ; mais les générations suivantes payaient, sous la forme de \emph{prix du sol}, ce que leurs devanciers demi-serfs avaient payé sous forme de rente, de dîme, de corvée, etc. À mesure que la population croissait, que d’autre part augmentait la division de la terre, le prix de la parcelle s’élevait, car la demande croissait avec son exiguïté ; mais à mesure que le prix de la parcelle montait, soit que le paysan l’achetât directement, ou qu’il se la fît compter comme capital par ses cohéritiers, l’\emph{endettement du paysan}, c’est-à-dire \emph{l’hypothèque} croissait en proportion. Le titre de la créance dont la terre est chargée se nomme en effet \emph{hypothèque}, c’est la créance dont le sol est le nantissement. De même qu’au Moyen Âge les \emph{privilèges} s’accumulaient sur les biens-fonds, les \emph{hypothèques} s’amoncellent actuellement sur les parcelles. De plus, sous le régime parcellaire, la terre est pour son propriétaire un pur \emph{instrument de production}. Or, à mesure que la division du terrain augmente, sa fertilité diminue. L’application de la machine à la terre, la division du travail, les améliorations principales, canaux d’irrigation, d’asséchement, etc., deviennent de plus en plus impossibles, parce que les \emph{faux-frais} de la culture croissent proportionnellement à la division du moyen de production. Mais, l’état de division augmentant, le bien-fonds et le matériel le plus misérable tendent de plus en plus à devenir l’unique capital du cultivateur parcellaire. Les avances de capital, faites à la terre, diminuent, les petits paysans voient de plus en plus leur faire défaut le sol, l’argent et le savoir nécessaire à l’utilisation des progrès de l’agronomie : l’agriculture rétrograde de plus en plus. Enfin, le \emph{produit net} diminue proportionnellement à l’augmentation de la \emph{consommation brute}. La famille du paysan tout entière se voit interdire par sa propriété même toute autre occupation et cependant la terre ne peut plus la nourrir.\par
Ainsi donc, la population et, avec elle, la division du sol augmentant, le \emph{moyen de production}, la terre, s’élève de prix, sa fertilité diminue en proportion : \emph{l’agriculture périclite et le paysan s’endette} dans la même mesure. Ce qui était effet devient cause à son tour, Chaque génération laisse l’autre plus endettée. Chaque génération débute dans des conditions plus défavorables et plus dures. L’hypothèque donne naissance à l’hypothèque, et quand le paysan ne peut plus offrir sa parcelle en nantissement de \emph{nouvelles dettes}, ne peut plus la charger de nouvelles hypothèques, il devient directement la proie de l’usure et les intérêts usuraires se font de plus en plus énormes.\par
Il arriva donc que le paysan français, sous forme \emph{d’intérêts} pour les \emph{hypothèques} prises sur sa terre, sous forme d’intérêts pour les \emph{avances sans hypothèques} des usuriers, abandonna au capitaliste, non seulement une rente, non seulement le profit industriel, bref, non seulement \emph{tout le bénéfice net}, mais encore une partie du salaire. Il tomba dans la condition du \emph{tenancier irlandais} et tout cela sous le prétexte d’être \emph{propriétaire privé}.\par
Ce procès fut accéléré en France par l’accroissement continu des \emph{charges fiscales} et par les \emph{frais de justice}, provenant soit des formalités dont la législation française entoure la propriété foncière, soit des conflits innombrables qui naissent de la juxtaposition et de l’enchevêtrement des parcelles, soit de la manie chicanière des paysans chez qui la jouissance de la propriété se réduit à faire prévaloir fanatiquement l’illusion de la propriété, le \emph{droit de propriété}.\par
D’après un tableau statistique datant de 1840, le produit brut du sol français s’élevait à 5 237 178 000 francs. Il faut en déduire 3 552 000 000 francs pour les frais de culture, y compris la consommation des travailleurs agricoles. Reste un produit net de 1 685 178 000. francs, dont il faut déduire 550 millions pour les intérêts hypothécaires, 100 millions pour les magistrats, 350 millions d’impôts, et 107 millions pour les droits d’enregistrement, de timbre et d’hypothèque. Reste la troisième partie du produit net, 538 millions. Répartis par tête de la population, cela ne fait pas 25 francs de produit net. Dans ce calcul n’entre naturellement en ligne de compte ni l’usure non hypothécaire, ni les honoraires des avocats, etc.\par
On comprend quelle fut la situation du paysan quand la République eut ajouté de nouvelles charges aux anciennes. On voit que son exploitation ne se distingue que par \emph{la forme} de celle du prolétariat industriel. L’exploiteur est le même, c’est le \emph{capital}. Les capitalistes isolés exploitent les paysans isolés par \emph{l’hypothèque} et \emph{l’usure}. La classe capitaliste exploite la classe paysanne par les \emph{impôts}. Le titre de propriété du paysan est le talisman grâce auquel le capital l’ensorcèle, le prétexte au nom duquel il l’excite contre le prolétariat industriel. Seule la chute du capital peut relever le paysan, seul un gouvernement anti-capitaliste, prolétarien, peut remédier à sa misère économique, à sa dégradation sociale. La \emph{République constitutionnelle} est la dictature des exploiteurs coalisés du campagnard. La République \emph{sociale démocratique}, la République rouge est la dictature de ses alliés. Le plateau de la balance monte ou descend suivant le suffrage que le paysan jette dans l’urne électorale. C’est à lui de décider de son sort. – Ainsi parlaient les socialistes dans des pamphlets, des almanachs, des calendriers, des brochures de toute espèce. Ce langage devint encore plus compréhensible pour le paysan grâce aux écrits contraires du « parti de l’ordre » qui s’adressait à lui, et, par une exagération grossière, une conception, une exposition brutale des desseins et des idées des socialistes, atteignait à la véritable rusticité et irritait la convoitise du fruit défendu. Ce qui parlait le plus clairement à l’esprit des paysans, c’était l’expérience que cette classe avait retirée de l’exercice du droit de suffrage, c’étaient les désillusions qui se succédaient coup sur coup avec une rapidité révolutionnaire. \emph{Les révolutions sont les locomotives de l’histoire}.\par
La transformation graduelle subie par les paysans se montra à différents symptômes. Elle se manifesta dans les élections pour l’Assemblée législative, puis dans la mise en état de siège des départements voisins de Lyon, puis dans l’élection par le département de la Gironde d’un Montagnard à la place de l’ancien président de la « Chambre introuvable » peu de jours après le 13 juin, enfin par l’élection d’un rouge le 20 décembre 1849 à la place d’un légitimiste décédé dans le département du Gard, cette terre promise du légitimisme, théâtre d’atrocités effrayantes subies par les républicains en 1794 et en 1795, centre de la terreur blanche en 1815 où libéraux et protestants furent ouvertement assassinés. Le bouleversement de la classe la plus stationnaire de la population apparut très clairement après le rétablissement de l’impôt des boissons. Les mesures gouvernementales, les lois de janvier et de février 1850 sont presqu’exclusivement destinées aux \emph{départements} et aux \emph{paysans}. C’était la preuve la plus frappante du progrès qu’ils avaient accompli.\par
\emph{La circulaire d’Hautpoul}, qui fait du gendarme un inquisiteur au service du préfet, du sous-préfet et surtout du maire, qui organise l’espionnage au sein des communes rurales les plus éloignées ; \emph{la loi contre les instituteurs}, en vertu de laquelle ces derniers, les gens capables, les porte-paroles, les éducateurs et les interprètes de la classe paysanne étaient soumis à l’arbitraire des préfets ; eux, les prolétaires de la classe instruite, se voyaient chasser d’une commune dans l’autre comme un gibier que l’on veut forcer ; \emph{la proposition de loi contre les maires} qui suspend au-dessus de leur tête l’épée de Damoclès de la révocation et constamment les oppose, eux, les présidents des communes rurales au président de la République et au « parti de l’ordre » ; l’ordonnance qui change les 17 divisions militaires en quatre pachaliks et donne aux Français pour salon national la caserne et le bivouac ; \emph{la loi sur l’instruction} par laquelle le parti de l’ordre proclame que l’inconscience et l’abrutissement de la France sont la condition de son existence sous le régime de suffrage universel. Qu’étaient-ce que toutes ces lois, toutes ces mesures ? C’étaient des tentatives désespérées du « parti de l’ordre » pour reconquérir les départements et les paysans des départements.\par
Considérés comme \emph{mesures} de \emph{répression}, les moyens étaient misérables et allaient contre leur but. Les grandes mesures comme le maintien de l’impôt des boissons, l’impôt des 45 centimes, le rejet dédaigneux des pétitions des paysans demandant le remboursement du milliard, etc., toutes ces foudres législatives ne frappaient la classe paysanne que d’un coup, dans sa totalité, partaient du centre. Les lois et les mesures introduites rendaient l’attaque et la résistance \emph{générales}, devenaient les sujets de conversation de chaque hutte, inoculaient la révolution à chaque village : elles \emph{localisaient cette révolution, en faisaient une révolution paysanne}.\par
D’autre part, ces propositions de Bonaparte, leur adoption par l’Assemblée nationale, tout cela ne démontrait-il pas l’union des deux pouvoirs de la République constitutionnelle dès qu’il s’agit de la répression de l’anarchie, de l’oppression de toutes les classes qui se soulèvent contre la dictature de la bourgeoisie ? Est-ce que \emph{Soulouque} n’avait pas, immédiatement après son message brutal, assuré la Législative de son dévouement à l’ordre par le message de \emph{Carlier} qui suivit immédiatement, de Carlier, cette caricature ignoblement commune de Fouché : Louis Bonaparte lui-même était d’ailleurs la plate caricature de Napoléon.\par
\emph{La loi sur l’instruction} nous montre l’alliance des jeunes catholiques et des vieux voltairiens. La domination des bourgeois coalisés pouvait-elle être autre chose que le despotisme coalisé de la restauration amie des Jésuites et de la monarchie de Juillet libre penseuse ? Est-ce que les armes qu’une fraction de la bourgeoisie avait remises au peuple pour lutter contre l’autre parti bourgeois dans les luttes réciproques dont l’hégémonie était l’enjeu, est-ce que ces armes ne devaient pas être reprises au peuple alors en présence de la dictature de la coalision ? Rien n’a plus irrité le boutiquier de Paris que ce coquet étalage de \emph{Jésuitisme} non pas même le rejet des \emph{concordats à l’amiable}.\par
Pendant ce temps, les conflits continuaient à s’élever aussi bien entre les différentes fractions du parti de l’ordre, qu’entre l’Assemblée nationale et Bonaparte. Beaucoup de choses étaient faites pour déplaire à l’Assemblée : Bonaparte, immédiatement après « son coup d’État », après la constitution d’un ministère bonapartiste proprement dit, mandait devant lui les invalides de la monarchie, nommés préfets, et faisait de leur agitation inconstitutionnelle en faveur de sa réélection à la présidence la condition de leur maintien dans leur fonction. Carlier célébrait son installation par la fermeture d’un club légitimiste. Bonaparte fondait un journal particulier, le \emph{Napoléon}, qui confiait au peuple les intentions secrètes du président que ses ministres étaient obligés de démentir à la tribune de la Législative. Bien des choses semblaient peu plaisantes : le maintien insolent du ministère malgré les votes de défiance répétés ; la tentative de se concilier la faveur des sous-officiers par une haute-paie journalière de quatre sous, et la faveur du prolétariat par un plagiat des \emph{Mystères} d’Eugène Sue, par une banque de prêts sur l’honneur ; l’impudence enfin avec laquelle on faisait proposer par les ministres la déportation à Alger des derniers insurgés de Juin pour frapper « en gros » la Législative d’impopularité, alors que le président conservait sa popularité « en détail » par des grâces isolées. Des paroles menaçantes tombèrent de la bouche de \emph{Thiers} qui parla de « coups d’État » et de « coups de tête ». La Législative se vengea de Bonaparte en rejetant toute proposition de loi qu’il déposait dans son propre intérêt, en cherchant avec une défiance bruyante si chaque projet qu’il déposait dans l’intérêt général n’avait pas pour effet, en augmentant le pouvoir exécutif, de profiter au pouvoir personnel du prince président. En un mot, \emph{l’Assemblée se vengeait par la conspiration du mépris}.\par
Le parti légitimiste, de son côté, voyait avec mécontentement les orléanistes, plus capables, s’emparer de nouveau de presque tous les postes et la \emph{centralisation} croître alors qu’ils voyaient principalement leur salut dans la \emph{décentralisation}. C’était réel. La contre-révolution \emph{centralisait à l’excès}. Elle préparait à l’avance le mécanisme de la révolution. Par le cours forcé accordé aux billets de banque, elle \emph{centralisait} même l’or et l’argent de la France dans la banque de Paris. Elle créait ainsi au profit de la révolution un \emph{trésor de guerre tout fait}.\par
Les orléanistes, enfin, voyaient avec dépit surnager le principe de la légitimité, le voyaient avec déplaisir s’opposer à leur principe bâtard. Ils se trouvaient à chaque instant humiliés et maltraités parce qu’ils représentaient la mésalliance bourgeoise d’un noble époux.\par
Nous avons vu peu à pou les paysans, les petits bourgeois, et, en général, toutes les classes moyennes se ranger aux côtés du prolétariat, poussés à se mettre en opposition officielle avec la République officielle, traités par elle en adversaires. \emph{Révolte contre la dictature de la bourgeoisie ; nécessité d’une modification de la société ; maintien des institutions républicaines et démocratiques considérées comme les moteurs de cette société ; ralliement autour du prolétariat, la seule puissance révolutionnaire décisive} – telles sont les caractéristiques les plus générales de ce qu’on a appelé le \emph{parti de la sociale démocratie}, le \emph{parti de la république rouge}. Ce parti de l’anarchie, comme se sont plus à le baptiser ses adversaires, n’est pas moins que le \emph{parti de l’ordre} une coalition d’intérêts différents. Il part des plus petites réformes apportées au désordre de l’ancienne société et aboutit au bouleversement de l’ancien ordre social ; le libéralisme bourgeois et le terrorisme révolutionnaire sont les lointains extrêmes qui forment le point d’origine et le point terminal du parti de l’ « Anarchie. »\par
La suppression des droits protecteurs, c’est du socialisme ! elle atteint en effet le monopole de la fraction \emph{industrielle} du « parti de l’ordre ». La réglementation du déficit public – c’est du socialisme ! elle atteint le monopole de la fraction \emph{financière} du « parti de l’ordre ». La libre entrée de la viande et des grains étrangers – c’est du socialisme ! elle atteint la troisième fraction du « parti de l’ordre », \emph{la grande propriété foncière}. Les revendications du parti libre-échangiste, des bourgeois anglais les plus avancés paraissent, en France, être autant de revendications socialistes. Le voltairianisme – c’est du socialisme ! Il attaque une quatrième fraction du « parti de l’ordre », la fraction \emph{catholique}. Liberté de la presse, liberté d’association, instruction générale du peuple – Socialisme ! Socialisme ! Ces mesures frappent le monopole du parti de l’ordre dans son ensemble.\par
La marche de la révolution avait si bien mûri la situation que les exigences les plus modestes des classes moyennes, les réformateurs de toutes nuances étaient contraints de se rallier autour du drapeau du parti révolutionnaire le plus avancé, autour du \emph{drapeau rouge}.\par
Quelle que fut, d’ailleurs, la diversité du socialisme, des fractions importantes du parti de l’anarchie – cette diversité correspondait à la variété des conditions économiques et aux différences dans les besoins généraux des classes et des fractions de classes qui découlaient de ces conditions – sur un point du moins il y avait unanimité. On proclamait que le socialisme était le \emph{moyen d’émanciper le prolétariat} et que son but était cette \emph{émancipation}.\par
Dans les phrases socialistes \emph{générales}, ressemblant à peu près à celles du \emph{parti de l’anarchie}, se cache le \emph{socialisme} du \emph{National}, de la \emph{Presse} et du \emph{Siècle}, qui, plus ou moins conséquent, veut renverser la domination de l’aristocratie financière et délivrer l’industrie et le commerce de leurs entraves antérieures. C’est le socialisme de l’industrie, du commerce et de l’agriculture. En effet, les grands capitalistes, membres du parti de l’ordre, renient les intérêts de ces industries qui ne s’accordent plus avec leurs monopoles privés. Ce \emph{socialisme bourgeois}, qui, naturellement, rallie autour de lui une partie des ouvriers et des petits bourgeois comme le font toutes les espèces bâtardes de socialisme, se distingue du \emph{socialisme petit bourgeois proprement dit}, du socialisme « par excellence ». La petite bourgeoisie hait le capital parce qu’elle est débitrice : elle demande des \emph{institutions de crédit}. Le capital l’écrase par la concurrence : elle réclame des associations subventionnées par l’État. Le capital l’accablé par la \emph{concentration} : elle veut des \emph{impôts progressifs}, des restrictions à l’héritage, l’entreprise par l’État des grands travaux, d’autres mesures encore qui \emph{entravent puissamment l’accroissement du capital}. Comme elle rêve à la réalisation pacifique de son socialisme, qu’elle compte même sur une seconde révolution de Février durant pendant quelques jours, elle croit naturellement que le procès historique futur consiste dans l’\emph{application des systèmes} que les penseurs sociaux conçoivent ou ont conçu soit en compagnie, soit en inventeurs isolés. Les petits bourgeois deviennent ainsi les éclectiques ou les adeptes des systèmes socialistes déjà existants, du \emph{socialisme doctrinaire} qui n’est resté l’expression théorique du prolétariat qu’aussi longtemps que celui-ci n’était pas assez développé pour posséder un mouvement historique indépendant.\par
Ainsi donc, pendant que l’\emph{Utopie}, le \emph{socialisme doctrinaire}, qui subordonne le mouvement total à un de ses moments, remplace la production sociale, la production en commun, par la chimère d’un pédant isolé, dont surtout la fantaisie rabaisse la lutte révolutionnaire des classes avec ses nécessités à de petits artifices ou à de grosses sentimentalités ; pendant que ce socialisme doctrinaire, qui au fond se borne à idéaliser la société actuelle, n’en prend qu’une ombre sans âme et veut que son idéal l’emporte sur la réalité sociale ; pendant que ce socialisme est abandonné par le prolétariat à la petite bourgeoisie ; alors que la rivalité des différents chefs socialistes met en évidence chacun de ces soi-disant systèmes, de ces théories, dont chacune relie prétentieusement l’un des moments intermédiaires du bouleversement social au détriment des autres, le \emph{prolétariat}, lui, se groupe de plus en plus autour du \emph{socialisme révolutionnaire}, autour du \emph{communisme} auquel la bourgeoisie elle-même a fourni le nom de \emph{Blanqui}. Ce socialisme, c’est \emph{la révolution à l’état permanent, la dictature de classe du prolétariat}, moment nécessaire qu’il faut franchir pour atteindre à \emph{la suppression générale des différences de classe} ; c’est la suppression de tous les rapports de production sur lesquels elles reposent, la suppression de tous les rapports sociaux qui correspondent à ces rapports de production, le bouleversement enfin de toutes les idées qui découlent de ces rapports sociaux.\par
L’espace qui nous est réservé ne nous permet pas de développer davantage ce sujet.\par
Nous avons vu que si, dans le \emph{parti de l’ordre, l’aristocratie financière} devait nécessairement prendre la tête, dans le parti de l’\emph{Anarchie} ce devait être le \emph{prolétariat}. Les différentes classes, unies en une ligne révolutionnaire, se groupaient donc autour du prolétariat ; les départements devenaient de moins en moins sûrs ; l’Assemblée législative était de plus en plus mécontente des prétentions du Soulouque français. Pendant ce temps, approchaient les élections complémentaires, longtemps ajournées et retardées, qui devaient pourvoir au remplacement des Montagnards proscrits le 13 juin.\par
Le gouvernement, méprisé par ses ennemis, maltraité et journellement humilié par ses soi-disant amis, ne voyait qu’un moyen de sortir de sa situation répugnante, intolérable : l’émeute. Une émeute à Paris eût permis de mettre en état de siège la capitale et les départements, d’être ainsi maître des élections. D’autre part, les amis de l’ordre auraient été contraints à des concessions vis-à-vis d’un gouvernement vainqueur de l’anarchie, s’ils ne voulaient pas, eux-mêmes, passer pour anarchistes.\par
Le gouvernement se mit à l’œuvre. Au commencement de février, on provoqua le peuple en abattant les arbres de la liberté. Ce fut inutile. Si les arbres de la liberté avaient perdu leur place, le gouvernement lui-même perdit la tête, et recula, effrayé, devant sa provocation. L’Assemblée nationale reçut cette maladroite tentative d’émancipation de Bonaparte avec une défiance glacée. L’enlèvement des couronnes d’immortelles de la colonne de juillet n’eut pas plus de succès. Elle fournit à une partie de l’armée l’occasion de démonstrations révolutionnaires et à l’Assemblée nationale le prétexte d’un vote de défiance plus ou moins déguisé contre le ministère. Ce fut en vain que la presse gouvernementale menaça de la suppression du suffrage universel, de l’invasion des Cosaques. En vain d’Hautpoul, en pleine assemblée, somma-t-il la gauche de descendre dans la rue en déclarant que le gouvernement était prêt à la recevoir. D’Hautpoul n’en retira qu’un rappel à l’ordre du président et le « parti de l’ordre » laissa avec une joie maligne et silencieuse un député de gauche persifler les velléités usurpatrices de Bonaparte. En vain prophétisa-t-on une révolution pour le 24 février ; le gouvernement fit en sorte que le 24 février fut ignoré du peuple.\par
Le prolétariat permettait qu’on le provoquât à l’\emph{émeute} parce qu’il avait dessein de faire une révolution.\par
Sans s’arrêter aux excitations du gouvernement qui ne faisaient qu’irriter davantage le mécontentement général contre l’ordre existant, le comité électoral, entièrement sous l’influence des ouvriers, présenta trois candidats pour Paris : \emph{Deflotte, Vidal} et \emph{Carnot}. Deflotte était un déporté de juin, amnistié dans un de ces accès où Bonaparte essayait de se ménager la popularité. C’était un ami de Blanqui, et il avait participé à l’attentat du 15 mars. Vidal, connu comme écrivain communiste par son livre de \emph{La répartition de la richesse} était l’ancien secrétaire de Louis Blanc à la commission du Luxembourg. Carnot, le fils du conventionnel, de l’organisateur de la victoire, le moins compromis des membres du parti du \emph{National} avait été ministre de l’instruction publique. Son démocratique projet de loi sur l’instruction primaire était une protestation vivante contre la loi sur l’instruction due aux Jésuites. Ces trois candidats représentaient les trois classes alliées : en tête, l’insurgé de Juin, le représentant du prolétariat révolutionnaire ; à côté de lui, le socialiste doctrinaire, le représentant de la petite bourgeoisie socialiste ; en troisième lieu, enfin, le représentant du parti républicain bourgeois dont les formules démocratiques, opposées à celles du « parti de l’ordre » avaient pris une couleur socialiste et perdu, depuis longtemps, leur sens propre. C’était \emph{comme en février une coalition générale contre la bourgeoisie et le gouvernement}. Mais, cette fois, le prolétariat était \emph{à la tête de la ligue révolutionnaire}.\par
Malgré tous les efforts, les candidats socialistes l’emportèrent. L’armée elle-même vota pour l’insurgé de Juin, contre son propre ministre de la Guerre, Lahitte. Le parti de l’ordre était comme frappé de la foudre. Les élections départementales le consolaient mal : elles fournissaient une majorité de Montagnards.\par
\emph{L’élection du} 10 \emph{mars} 1850 ! \emph{C’était la revanche de juin} 1848. Les massacreurs et les déporteurs des insurgés de Juin rentraient bien à l’Assemblée nationale, mais l’échine basse, à la suite des déportés, et leurs principes au bout des lèvres. \emph{C’était la revanche du} 13 \emph{Juin} 1849. La Montagne, proscrite par l’Assemblée nationale faisait sa rentrée dans cette Assemblée, mais maintenant trompette avancée de la Révolution, elle n’en avait plus le commandement comme autrefois. \emph{C’était la revanche du} 10 \emph{décembre}. Napoléon avait été battu avec son ministre Lahitte. L’histoire parlementaire de la France ne connaît qu’un cas analogue : l’échec d’Haussy, ministre de Charles X en 1830. L’élection du 10 mars 1850 était enfin la cassation de celle du 13 mai qui avait donné la majorité au « parti de l’ordre. » L’élection du 10 mars protestait contre la majorité du 13 mai. Le 10 mars était une Révolution. Derrière les bulletins de vote, les pavés étaient prêts.\par
« Le vote du 10 mars, c’est la guerre », s’écriait Ségur d’Aguesseau, un des membres les plus avancés du parti de l’ordre.\par
Avec le 10 mars 1850, la République constitutionnelle rentre dans une nouvelle phase, \emph{dans sa phase de dissolution}. Les différentes fractions de la majorité se sont réconciliées entre elles et avec Bonaparte, elles sont de nouveau destinées à sauver l’ordre, et lui redevient leur \emph{homme neutre}. Si elles songent encore à être royalistes, c’est parce qu’elles désespèrent toujours de la possibilité de la République bourgeoise. S’il pense encore à être président, c’est uniquement parce qu’il désespère de le rester.\par
À l’élection de Deflotte, l’insurgé de Juin, Bonaparte répond sur l’indication du « parti de l’ordre », par la nomination de Baroche au ministère de l’Intérieur, de Baroche, l’accusateur de Blanqui et de Barbès, de Ledru-Rollin et de Guinard. À l’élection de Carnot, la Législative répond par le vote de la loi sur l’instruction, à l’élection de Vidal, par l’interdiction de la presse socialiste. Le « parti de l’ordre » cherche à dissimuler sa peur par les coups de trompette que lance sa presse. « Le glaive est saint », s’écrie un de ses organes. « Les défenseurs de l’ordre doivent prendre l’offensive contre le parti rouge », dit un autre. « Entre le socialisme et la société, il y a un duel à mort, une guerre continuelle, sans pitié. Dans ce combat singulier, l’un ou l’autre doit être terrassé. Si la société n’anéantit pas le socialisme, le socialisme anéantira la société », chante un troisième coq de l’ordre. Élevez les barricades de l’ordre, les barricades de la religion, les barricades de la famille ! Il faut en finir avec les 127 000 électeurs de Paris ! Une Saint-Barthélemy pour les socialistes ! Et le parti de l’ordre croit un moment dans sa propre confiance, dans sa victoire.\par
Ce que ses organes prennent le plus fanatiquement à partie, ce sont les \emph{boutiquiers de Paris}. L’insurgé de Juin choisi comme représentant par les boutiquiers de Paris ! Cela Signifie qu’un second Juin 1848 est impossible ; qu’un second 13 Juin 1849 est impossible ; que l’influence morale du capital est battue en brèche ; que l’Assemblée nationale ne représente plus que la bourgeoisie ; que la grande propriété est perdue, puisque son soutien, la petite propriété, cherche son salut dans le camp des sans-avoir.\par
Le « parti de l’ordre » revient naturellement à ses inévitables \emph{lieux communs. Davantage de répression} ! s’écrie-t-il. \emph{Dix fois plus de répression} ! Mais sa force de répression est dix fois plus faible, tandis que la résistance est cent fois plus puissante. L’instrument principal de la répression, l’armée ne doit-elle pas elle-même être réprimée ? Et le parti de l’ordre prononce son dernier mot : « Le cercle de fer d’une légalité qui nous étouffe doit être rompu. \emph{La République constitutionnelle est impossible} ! » Il nous faut lutter avec nos vraies armes. Depuis février 1848, nous avons combattu la Révolution sur \emph{son propre} terrain, avec \emph{ses propres armes} ; nous avons accepté \emph{ses propres} institutions. La Constitution est une citadelle qui ne protège que les assiégeants, non les assiégés ! En nous glissant dans Ilion, la sainte, cachée dans le ventre du cheval de Troie, nous n’avons pas, à la différence de nos modèles, les \emph{Grecs}, conquis la ville ennemie, nous nous sommes laissés faire prisonniers nous-mêmes.\par
Mais le fondement de la Constitution est le \emph{suffrage universel}. La \emph{suppression du suffrage universel} est le dernier mot du « parti de l’ordre », de la dictature bourgeoise.\par
Le suffrage universel avait donné raison à la bourgeoisie le 24 mai 1848, le 20 décembre 1848, le 13 mai 1849, le 8 juillet 1849. Le suffrage universel s’est fait tort à lui-même le 10 mars 1850. La constitution bourgeoise signifie que la domination bourgeoise est l’émanation et le résultat du suffrage universel, l’acte parfait de la volonté souveraine du peuple. Mais du moment que le contenu de ce suffrage, de cette volonté souveraine n’est plus la domination de la bourgeoisie, la Constitution a-t-elle encore quelque sens ? N’est-ce pas le devoir de la bourgeoisie de réglementer le suffrage de telle façon qu’il veuille ce qui est raisonnable, qu’il veuille la domination de la bourgeoisie ? Le suffrage universel en révoquant constamment le pouvoir public, en le créant à nouveau, en le tirant de son propre sein, ne supprime-t-il pas toute stabilité, ne met-il pas tous les pouvoirs existants continuellement en question, n’annihile-t-il pas l’autorité, ne menace-t-il pas même de mettre l’anarchie au même rang que l’autorité ? Après le 10 mars 1850, qui pouvait en douter ?\par
La bourgeoisie, en se dépouillant du suffrage universel dont elle s’était drapée jusqu’alors, d’où elle puisait sa toute-puissance, confesse crûment : \emph{Notre dictature s’est maintenue jusqu’à présent par la volonté du peuple, il faut l’assurer maintenant contre la volonté du peuple}. Et, conséquente, elle cherche son appui non plus en France, mais hors de France, à l’étranger, dans l’\emph{invasion}.\par
Elle suscite ainsi un second Coblence, dont le siège est fait en France même. Elle réveille contre elle toutes les passions nationales. Par son attaque contre le suffrage universel, elle fournit à la nouvelle révolution un \emph{prétexte général}, et la révolution en a besoin. Tout prétexte \emph{particulier} eût divisé les membres de la ligue révolutionnaire et mis leurs divergences en évidence. Le prétexte \emph{général} aveugle les classes à moitié révolutionnaires, leur permet de s’illusionner sur le caractère \emph{déterminé} de la révolution à venir, sur les conséquences de leur propre action. Toute révolution a besoin d’une question de banquet. Le suffrage universel était la question des banquets de la nouvelle révolution.\par
Mais les fractions bourgeoises sont déjà perdues quand elles abandonnent la seule forme où il leur soit possible d’exercer le pouvoir en commun, la forme la plus puissante et la plus parfaite de leur \emph{domination de classe, la République constitutionnelle}, pour avoir recours à la forme inférieure, incomplète et plus faible de la \emph{Monarchie}. Elles ressemblent à ce vieillard qui pour reconquérir sa jeunesse, reprenait sa parure enfantine et cherchait à tourmenter ses membres flétris. Leur République n’avait qu’un avantage : celui d’être \emph{la serre destinée à faire éclore la révolution}.\par
Le 10 mars 1850 porte l’inscription :\par

\begin{center}
\noindent « {\scshape après moi le déluge}. »
\end{center}

\chapterclose


\chapteropen

\chapter[{IV. Abolition du suffrage universel en 1850}]{IV. Abolition du suffrage universel en 1850}
\renewcommand{\leftmark}{IV. Abolition du suffrage universel en 1850}


\chaptercont
\noindent Les mêmes symptômes pouvaient s’observer en France depuis 1849 et surtout depuis le début de 1850. Les industries parisiennes sont en pleine activité. Les fabriques de cotonnade de Rouen et de Mulhouse sont assez prospères bien que les prix élevés de la matière première aient, ici comme en Angleterre, occasionné un certain ralentissement. Le développement de la prospérité fut plus particulièrement accéléré en France par une large modification des tarifs de douanes introduite en Espagne et par l’abaissement des droits opéré par le Mexique sur différents articles de luxe. L’exportation des marchandises françaises avait augmenté d’une façon importante sur ces deux marchés. La multiplication des capitaux fit naître, en France, une série de spéculations dont le prétexte était l’exploitation sur une grande échelle des mines d’or de la Californie. Quantité de sociétés prirent naissance. Le faible montant de leurs actions, leurs prospectus à couleur socialiste visaient directement la bourse des petits bourgeois et des ouvriers. En gros et en détail, ces entreprises se réduisaient à cette escroquerie pure qui est particulière aux Chinois et aux Français. Une de ces sociétés même fut directement protégée par le gouvernement. Les droits sur les importations s’élèvent en France dans les premiers mois de 1848 à 63 millions de francs, en 1849, à 95 millions, et en 1850 à 93 millions. D’ailleurs, au mois de septembre 1850, ils dépassèrent de plus d’un million leur montant pour le même mois de 1847. L’exportation s’est de même accrue en 1849 et plus encore en 1850.\par
La preuve la plus frappante de la renaissance de la prospérité est fournie par le rétablissement des payements en espèces à la Banque et édicté par la loi du 9 septembre 1850. Le 15 mars 1848, la Banque avait été autorisée à suspendre les paiements de cette nature. La circulation en billets, y compris celle des banques provinciales, s’élevait à 373 millions de francs (14 920 000 £) Le 2 novembre 1849, cette circulation montait à 482 millions de francs ou à 19 280 000 £, soit une augmentation de 4 360 000 £ ; et le 2 septembre 1850, à 496 millions de francs, ou à 19 840 000 £, soit une augmentation d’environ 5 millions de livres. Les billets ne s’en trouvèrent pas dépréciés, au contraire. L’augmentation de la circulation des banknotes était accompagnée d’une accumulation sans cesse grandissante de l’or et de l’argent dans les caves de la Banque, si bien qu’en 1850 l’encaisse métallique s’élevait environ à 14 millions de livres, somme inouïe pour la France. La Banque avait donc été mise en position d’augmenter sa circulation, c’est-à-dire son capital actif, de 113 millions de francs, ou de 5 millions de livres. Ce fait prouve d’une façon frappante combien nous avions raison de prétendre, dans une livraison antérieure, que la révolution, loin d’abattre l’aristocratie financière, ne l’avait que consolidée. Cette conclusion est encore plus évidente si nous jetons le coup d’œil suivant sur la législation française de la Banque dans ces dernières années. Le 10 juin 1847, la Banque fut autorisée à émettre des banknotes de 200 francs. Le billet le moins important était jusqu’alors celui de 500 francs. Un décret du 15 mars 1848 donnait aux billets de la Banque la valeur d’une monnaie légale et dispensait cet établissement de l’obligation de les rembourser en espèces. L’émission des billets fut limitée à 350 millions de francs. Elle fut autorisée en même temps à émettre des billets de 100 francs. Un décret du 27 avril ordonna la fusion des banques départementales avec la Banque de France. Un autre décret du 2 mai 1848 élève son émission à 442 millions de francs. Un décret du 22 décembre 1849 élève le maximum de l’émission à 525 millions de francs. Enfin la loi du 6 septembre 1850 rétablit le remboursement des billets en espèces. Tous ces faits, l’accroissement constant de la circulation, la concentration de tout le crédit français entre les mains de la Banque, l’accumulation de l’or français dans ses caves amenèrent M. Proudhon à conclure que la Banque devait dépouiller sa vieille peau et se métamorphoser en une banque populaire à sa mode. Il n’avait pas besoin de connaître l’histoire de la crise de la Banque d’Angleterre de 1799 à 1819 ; il n’avait qu’à porter ses yeux au-delà de la Manche pour voir que le phénomène qui lui paraissait inouï dans l’histoire de la société bourgeoise était des plus normaux dans cette société ; mais seulement il se produisait en France pour la première fois. On voit que les soi-disant théoriciens révolutionnaires qui, à Paris, grondaient après le gouvernement provisoire étaient tout aussi ignorants de la nature et des résultats des mesures prises que ces messieurs du gouvernement eux-mêmes. Malgré la prospérité commerciale et industrielle dont la France se félicite momentanément, la masse de la population, les 25 millions de paysans, souffrent d’une grande dépression. Les bonnes récoltes ont, en France, abaissé le prix du blé plus encore qu’en Angleterre, et la situation des paysans, endettés, saignés à blanc par l’usure, écrasés d’impôts ne peut qu’être rien moins que brillante. L’histoire des trois dernières années a montré à satiété que cette classe de la population est tout à fait incapable d’une initiative révolutionnaire quelconque.\par
La période de crise se fait sentir plus tardivement sur le continent qu’en Angleterre : il en est de même de la période de prospérité. C’est en Angleterre que se produit le procès originel. Ce pays est le Demiurge du cosmos bourgeois. Sur le continent, les différentes phases du cycle que la société bourgeoise parcourt constamment revêtent une forme secondaire et tertiaire. D’abord l’exportation faite par le continent en Angleterre est d’une importance disproportionnée avec celle effectuée dans un pays quelconque. Cette exportation, de son côté, dépend de l’état du marché anglais, surtout vis à-vis du commerce maritime. Puis l’exportation anglaise devient incomparablement plus grande que toute l’exportation continentale. Il s’ensuit que la quantité de marchandises exportées par le continent dans les pays d’outre-mer dépend toujours de l’importance de l’exportation anglaise dans ces pays. Par suite, si les crises amènent des révolutions d’abord sur le continent, la raison de ces mouvements se trouve toujours en Angleterre. Il est naturel que ces convulsions se produisent aux extrémités de l’organisme bourgeois avant d’arriver à son cœur, puisqu’il les chances d’un équilibre sont plus grandes que là. D’autre part, la violence de la réaction dont l’Angleterre a à souffrir du fait de ces crises continentales est le thermomètre où l’on peut lire la gravité de ces crises. Cette réaction indique si ces révolutions mettent réellement en danger les conditions d’existence de la bourgeoisie ou n’atteignent que les formes politiques.\par
À ces époques de prospérité générale où les forces productives de la société bourgeoise se développent autant que les conditions de cette société le permettent, il ne peut être nullement question de véritable révolution. Un semblable bouleversement n’est possible qu’aux périodes où ces \emph{deux facteurs}, les \emph{forces productives modernes} et les \emph{formes de production bourgeoises} entrent en conflit. Les multiples querelles auxquelles participent et dans lesquelles se compromettent réciproquement les fractions isolées du « parti de l’ordre » sur le continent, bien loin de fournir l’occasion de nouvelles révolutions, ne sont, au contraire, possibles que parce que la base qui supporte les rapports est si sûre, et, ce que la réaction ignore, si \emph{bourgeoise}. Les tentatives de réaction destinées à arrêter le développement bourgeois échoueront aussi bien que l’enthousiasme moral et les proclamations enflammées des démocrates. \emph{Une nouvelle révolution n’est possible qu’à la suite d’une nouvelle crise, mais l’une est aussi certaine que l’autre}.\par
Passons maintenant à la France.\par
La victoire que le peuple, uni aux petits bourgeois, avait remportée aux élections du 10 mars, fut annulée par lui-même : il provoqua, en effet, la nouvelle élection du 28 avril. Vidal avait été élu non seulement à Paris, mais encore dans le Bas-Rhin. Le comité parisien où la Montagne et la petite bourgeoisie étaient fortement représentées, l’engagea à opter pour ce dernier département. La victoire du 10 mars cessa d’être décisive. L’échéance fut de nouveau retardée. La vigueur populaire s’énerva. On l’accoutuma aux succès légaux plutôt qu’aux triomphes révolutionnaires. La signification révolutionnaire du 10 mars, la réhabilitation de l’insurrection de juin, fut tout à fait compromise par la candidature d’Eugène Sue, le fantaisiste social, petit bourgeois et sentimental, candidature que le prolétariat ne pouvait accepter tout au plus que comme une plaisanterie destinée à plaire aux grisettes. Le « parti de l’ordre », enhardi par la politique flottante de ses adversaires, opposa à cette candidature bien pensante un candidat qui devait représenter la \emph{victoire} de Juin. Ce candidat comique fut \emph{Leclerc}, le père de famille à la Spartiate, dont l’armure héroïque tomba pièce par pièce sous les coups de la presse et qui subit, d’ailleurs, une brillante défaite le jour de l’élection. La nouvelle victoire électorale du 28 avril remplit de présomption la Montagne et la petite bourgeoisie. Elle se flattait en imagination de remettre le prolétariat au premier plan en usant de la voie purement légale, sans avoir recours à une nouvelle révolution et d’arriver ainsi au comble de ses vœux. Elle comptait fermement, aux nouvelles élections de 1852, installer, grâce au suffrage universel, Ledru-Rollin sur le fauteuil présidentiel et faire entrer dans l’Assemblée une majorité de Montagnards. Le « parti de l’ordre », parfaitement convaincu par le renouvellement de l’élection, par la candidature de Sue, par l’accord de la Montagne et de la petite bourgeoisie que ces deux dernières étaient résolues à rester tranquilles en toutes circonstances, répondit aux deux victoires électorales par la loi qui abolissait le suffrage universel.\par
Le gouvernement se gardait bien de prendre cette proposition sous sa responsabilité. Il fit à la majorité une concession apparente en confiant l’élaboration du projet aux grands dignitaires de la majorité, aux dix-sept burgraves. Ce ne fut pas le gouvernement qui proposa à l’Assemblée, ce fut l’Assemblée qui se proposa à elle-même l’abolition du suffrage universel.\par
Le 8 mai, le projet fut porté devant la Chambre. Toute la presse sociale démocratique se leva comme un seul homme pour recommander au peuple un maintien digne, un « calme majestueux\footnote{En français dans le texte} », la passivité et la confiance en ses représentants. Chaque article de ces journaux confessait qu’une révolution anéantirait d’abord la presse que l’on qualifiait de révolutionnaire : il s’agissait maintenant pour elle d’une question d’existence. La presse soi-disant révolutionnaire dévoilait ainsi son secret. Elle signait son propre arrêt de mort.\par
Le 21 mai, la Montagne mit la question en discussion et proposa le rejet de tout le projet parce qu’il violait la constitution. Le « parti de l’ordre » répondit que l’on violerait la constitution s’il le fallait, mais que, cependant, cela n’était pas nécessaire, parce que la question était susceptible de toute interprétation et que la majorité avait seule compétence pour décider de l’interprétation exacte. Aux attaques effrénées, sauvages, de Thiers et de Montalembert, la Montagne opposa un humanisme convenable et de bon ton. Elle invoquait le terrain juridique. Le « parti de l’ordre » la replaça sur le terrain dans lequel le droit a sa racine, dans le domaine de la propriété bourgeoise. La Montagne demanda en gémissant si l’on voulait à toute force faire naître des révolutions ? Le « parti de l’ordre » répondit qu’on les attendait.\par
Le 22 mai, la question en discussion fut tranchée par 462 voix contre 227. Les hommes qui avaient démontré avec une profondeur si magnifique que l’Assemblée nationale, que chaque député résignait son mandat en quittant le service du peuple son mandant, restèrent sur leurs sièges et essayèrent aussitôt de faire agir le pays à leur place et cela au moyen de pétitions ; ils siégeaient encore, intangibles, quand le 31 mai la loi passa brillamment. Ils essayèrent de se venger par une protestation où ils dressaient procès-verbal de leur innocence dans le viol de la constitution, protestation qu’ils ne déposèrent même pas ouvertement, mais qu’ils glissèrent par-derrière dans la poche du président.\par
Une armée de 150 000 hommes à Paris, le retard apporté à la décision, la tranquillité de la presse, la pusillanimité de la Montagne et des représentants nouvellement élus, le calme majestueux des petits bourgeois, mais surtout la prospérité commerciale et industrielle empêchèrent toute tentative révolutionnaire du côté du prolétariat.\par
Le suffrage universel avait rempli sa mission. La majorité du peuple en avait retiré les leçons que ce suffrage seul peut donner dans une époque révolutionnaire. Il devait être aboli, soit par une révolution, soit par la réaction.\par
La Montagne déploya une énergie plus grande encore dans une occasion née peu de temps après. Le ministre de la Guerre d’Hautpoul avait, du haut de la tribune, qualifié la révolution de Février de catastrophe irrémédiable. Les orateurs de la Montagne qui, comme toujours, se distinguaient par un vacarme causé par une vertueuse indignation, se virent refuser la parole par le président Dupin. Girardin proposa à la Montagne de se retirer en masse. Le résultat fut que la Montagne continua à siéger, mais que Girardin en fut exclu comme indigne.\par
La loi électorale manquait encore d’un complément, d’une \emph{loi sur la presse}. Elle ne se fit pas longtemps attendre. Un projet du gouvernement, notablement agravé par les amendements du « parti de l’ordre » éleva les cautionnements, imposa aux romans-feuilletons un timbre supplémentaire (réponse à l’élection d’Eugène Sue), frappa d’un impôt, jusqu’à concurrence d’un certain nombre de feuilles, toutes les publications paraissant hebdomadairement ou mensuellement et ordonna finalement que chaque article du journal fut muni de la signature de l’auteur. Les prescriptions sur le cautionnement tuèrent la presse que l’on appelait révolutionnaire. Le peuple considéra leur disparition comme une satisfaction donnée à l’abolition du suffrage universel. Cependant ni les tendances, ni les effets de cette loi ne s’étendirent uniquement à cette partie de la presse. Tant que la presse quotidienne fut anonyme, elle sembla être l’organe de l’opinion publique anonyme, innombrable. Elle était le troisième pouvoir dans l’État. La signature de chaque article fit d’un journal la simple collection de contributions littéraires émanant d’individus plus ou moins connus. Chaque article tomba au rang d’annonce. Jusqu’alors les journaux avaient circulé comme papiers-monnaie de l’opinion publique. Ils étaient réduits maintenant à n’être plus qu’une seule lettre de change dont la valeur et la circulation dépendaient du crédit non seulement du tireur, mais encore de l’endosseur. La presse du « parti de l’ordre » avait provoqué non seulement à l’abolition du suffrage universel, mais encore aux mesures les plus extrêmes contre la mauvaise presse. D’ailleurs la bonne presse elle-même, avec son anonymat inquiétant, était incommode à ce parti et plus encore à chacun de ses représentants de province. Ce parti demandait seulement que l’écrivain salarié fit connaître son nom, son domicile et son signalement. C’est en vain que la bonne presse se lamentait sur l’ingratitude dont on payait ses services. La loi passa ; l’obligation de la signature fut prescrite avant tout. Les noms des journalistes républicains étaient assez connus ; mais les raisons sociales respectables, le \emph{Journal des Débats}, l’\emph{Assemblée Nationale}, le \emph{Constitutionnel}, etc., etc., firent piteuse mine ; leur sagesse politique tant prisée prit une figure lamentable, quand leur mystérieuse compagnie se résolut en \emph{penny a liners}, en vieux praticiens à vendre qui avaient défendu contre espèces toutes les causes possibles comme Granier de Cassagnac, ou en vieilles lavettes qui se qualifiaient elles-mêmes d’hommes d’État comme Capefigue, ou en casse-noisettes coquets comme M. Lemoinne des \emph{Débats}.\par
Au moment de la discussion de la loi sur la presse, la Montagne était déjà tombée à un tel degré de corruption morale qu’elle dut se borner à applaudir les brillantes tirades d’une ancienne notabilité philippiste, M. Victor Hugo.\par
La loi électorale et la loi sur la presse font disparaître le parti révolutionnaire et démocratique de la scène officielle. Avant leur retour dans leurs foyers, peu de temps après la clôture de la session, les deux fractions de la Montagne, les démocrates-socialistes et les socialistes-démocrates publièrent deux manifestes, deux \emph{Testimonia paupertatis}. Ils y montraient que si jamais la force et le succès ne s’étaient trouvés de leur côté, ils n’en étaient pas moins restés toujours du côté du droit éternel et de toutes les autres vérités éternelles.\par
Considérons maintenant le « parti de l’ordre ». La \emph{Neue rheinische Zeitung}, disait n° 3, p. 16. « Vis-à-vis des velléités de restauration des orléanistes et des légitimistes coalisés, Bonaparte représente le titre juridique du pouvoir réel qu’il exerce, la république. Vis-à-vis des velléités de restauration de Bonaparte, le « parti de l’ordre » représente le titre de la suprématie exercée en commun par ces deux fractions, la République. Vis-à-vis des orléanistes, les légitimistes ; vis-à-vis des légitimistes, les orléanistes représentent le \emph{statu quo}, la République. Toutes ces diverses fractions du « parti de l’ordre », dont chacune possède \emph{in petto} son roi propre et conserve l’espoir de sa propre restauration, font prévaloir, en présence des velléités d’usurpation et de relèvement de leurs rivales, la forme commune de la domination bourgeoise, la République, où les revendications particulières se neutralisent et se réservent. Et Thiers disait plus vrai qu’il ne le pensait quand il prétendait : « Nous, royalistes, nous sommes les vrais soutiens de la République constitutionnelle. »\par
Cette comédie des « républicains malgré eux » ; la répugnance témoignée au statu quo et sa consolidation constante ; les conflits incessants entre Bonaparte et l’Assemblée nationale ; la menace de la dissolution du « parti de l’ordre » en ses éléments constitutifs, menace sans cesse renouvelée ; les tentatives de chaque fraction de transformer la victoire remportée contre l’ennemi commun en une défaite des alliés momentanés ; la jalousie réciproque ; la rancune, les poursuites, les infatigables levées de boucliers qui se terminent toujours par des baisers Lamourette – toute cette farce peu édifiante ne se poursuivit jamais plus classiquement que pendant ces derniers six mois.\par
Le « parti de l’ordre » regardait tout d’abord la loi électorale comme une victoire remportée sur Bonaparte. Le gouvernement n’avait-il pas abdiqué en abandonnant la rédaction et la responsabilité de sa propre proposition à la commission des dix-sept ? La force principale de Bonaparte vis-à-vis de l’Assemblée ne résidait-elle pas dans sa qualité d’élu de 6 millions ? Bonaparte, de son côté, considérait la loi électorale comme une concession faite à l’Assemblée. Il espérait grâce à elle acheter l’harmonie de l’Exécutif avec le Législatif. Cet aventurier de bas étage demandait pour salaire qu’on augmentât sa liste civile de 3 millions. L’Assemblée nationale pouvait-elle entrer en conflit avec l’Exécutif au moment où elle mettait au ban la grande majorité de Français ? Elle se fâcha, parut vouloir pousser les choses à l’extrême, sa commission rejetta la proposition ; la presse bonapartiste menaça ; une quantité de tentatives de transactions se produisirent, et finalement l’Assemblée capitula ; mais elle se vengea sur le principe. Au lieu d’accorder le principe d’une augmentation annuelle de 3 millions de la liste civile, elle donna un secours de 2 160 000 francs. Non contente de cela, elle ne consentit à cette concession que quand elle eut reçu l’appui de Changarnier, le général du « parti de l’ordre », le protecteur qui s’était imposé à Bonaparte. À vrai dire, elle n’accorda pas les 3 millions à Bonaparte, mais à Changarnier.\par
Ce cadeau fait de « mauvaise grâce » fut reçu parfaitement de même par Bonaparte. La presse bonapartiste s’éleva de nouveau contre l’Assemblée. Quand, au cours des débats soulevés par la loi sur la presse, on proposa un amendement édictant l’obligation de la signature, il était dirigé spécialement contre les journaux inférieurs qui représentaient les intérêts particuliers de Bonaparte. Le principal organe bonapartiste, le \emph{Pouvoir}, se livra à une attaque ouverte et violente contre l’Assemblée nationale. Les ministres durent désavouer la feuille devant l’Assemblée. Le gérant du \emph{Pouvoir} fut cité à la barre de la Législative et condamné au maximum de l’amende, à 5 000 francs. Le jour suivant, le \emph{Pouvoir} publia un article plus insolent encore contre l’Assemblée et, pour venger le gouvernement, le parquet poursuivit aussitôt plusieurs journaux légitimistes pour attaques contre la constitution.\par
Enfin vint la question de la prorogation de la Chambre. Bonaparte désirait cette mesure pour pouvoir opérer sans en être empêché par l’Assemblée. Le « parti de l’ordre » ne la désirait pas moins, pour permettre aux fractions de pousser leurs intrigues, ou aux députés de veiller à leurs intérêts. Tous en avaient besoin pour fortifier en province les succès de la réaction et en récolter les fruits. L’Assemblée s’ajourna du 11 août au 11 novembre ; mais comme Bonaparte n’avait nullement dissimulé que s’il tenait à la prorogation, c’était parce qu’elle le débarrassait du contrôle importun de l’Assemblée, celle-ci donna même à son vote de confiance une signification de défiance. Tous les bonapartistes furent écartés de la commission permanente de vingt-huit membres chargée, pendant les vacances, de veiller sur la vertu de la République. À leur place, on élut même quelques républicains du \emph{Siècle} et du \emph{National} pour marquer au président la fidélité de l’Assemblée envers la République constitutionnelle.\par
Quelque temps avant et surtout immédiatement après la prorogation de l’Assemblée, les deux grandes fractions du « parti de l’ordre », les orléanistes et les légitimistes semblèrent vouloir se reconcilier et cela au moyen d’une fusion des deux maisons royales sous les drapeaux desquelles elles combattaient. Les journaux étaient pleins des projets de réconciliation qui avaient été discutés près du lit de douleur de Louis-Philippe, à Saint-Léonard. La mort du roi détrôné simplifia la situation. Louis-Philippe était l’usurpateur, Henri V le spolié. Le comte de Paris, par contre, en l’absence de descendant bourbonien, était l’héritier légitime de la couronne. Tout obstacle à la fusion des deux intérêts dynastiques disparaissait. Précisément à ce moment, les deux portions de la bourgeoisie découvrirent que ce n’était pas leur enthousiasme pour une maison royale particulière qui les séparait, mais que c’étaient bien plutôt des intérêts de classe différents qui divisaient les deux dynasties. Les légitimistes, en pèlerinage à la cour de Henri V, comme leurs concurrents l’étaient à Saint-Léonard, y apprirent la mort de Louis-Philippe. Ils formèrent aussitôt un ministère \emph{in partibus infidelium}. Il comprenait surtout des membres de la commission chargée de protéger la vertu de la République. Le ministère, prenant prétexte d’une dispute de parti, se présenta en proclamant de la façon la plus franche les principes du droit divin. Les orléanistes se réjouirent du scandale compromettant que ce manifeste suscita dans la presse et ne dissimulèrent à aucun moment leur hostilité ouverte envers les légitimistes.\par
Les conseils généraux se réunirent pendant la prorogation de l’Assemblée nationale. Leur majorité se prononça pour une révision de la constitution, plus ou moins tempérée par telle ou telle disposition. Elle se prononçait pour une restauration monarchique qu’elle ne déterminait pas plus précisément, pour une \emph{solution}. Elle avouait, en même temps, qu’elle était trop lâche et trop incompétente pour résoudre ce problème. Le parti bonapartiste interpréta immédiatement ce vœu en faveur de la révision dans le sens de la prolongation de la présidence de Bonaparte.\par
La solution constitutionnelle : abdication de Bonaparte en mai 1852, élection simultanée d’un président par tous les électeurs du pays, révision de la constitution par une assemblée spéciale pendant les premiers mois de la nouvelle présidence, cette solution est tout à fait inadmissible aux yeux de la classe dominante. Le jour de la nouvelle élection présidentielle serait la date de la rencontre de tous les partis ennemis, légitimistes, orléanistes, républicains bourgeois, révolutionnaires. On en viendrait nécessairement à une action décisive et violente entre les différentes fractions. De même, si le « parti de l’ordre » réussissait à s’unir sur la candidature d’un homme neutre pris en dehors des familles dynastiques, il se trouverait encore en en présence de Bonaparte. Dans sa lutte contre le peuple, le « parti de l’ordre » est constamment obligé d’augmenter le pouvoir de l’Exécutif. Chacune de ces augmentations accroît la puissance de celui qui détient le pouvoir exécutif, de Bonaparte. Dans la mesure donc où le « parti de l’ordre » consolide la domination qu’il exerce en commun, il renforce aussi les moyens d’action que Bonaparte peut mettre au service de ses prétentions dynastiques ; il augmente les chances qu’a le prince de faire avorter violemment la solution constitutionnelle au jour décisif. Par rapport au « parti de l’ordre », Bonaparte ne s’attaquerait pas plus à l’un des piliers de la constitution, que le parti de l’ordre ne l’avait fait, par rapport au peuple, en votant la loi électorale. En un mot, la solution constitutionnelle met en question tout le \emph{statu quo} politique. Et derrière le \emph{statu quo} mis en péril, le bourgeois apercevait le chaos, l’anarchie, la guerre civile. Il voit menacés, le premier dimanche de mai 1852, ses achats et ses ventes, ses effets, ses mariages, ses contrats notariés, ses hypothèques, ses rentes foncières, ses loyers, ses profits, tous ses contrats, toutes ses sources de revenu. Il ne peut courir ce risque. Une fois le \emph{statu quo} en péril, toute la société bourgeoise est en danger de ruine. La seule solution possible, au sens de la bourgeoisie, est l’ajournement de la solution. Cette classe ne peut sauver la République constitutionnelle qu’en violant la constitution, en prolongeant le pouvoir du président. Tel fut aussi le dernier mot de la presse de l’ordre après les interminables et profonds débats auxquels elle se livra sur les « solutions » après la session des conseils généraux. Le puissant « parti de l’ordre » se vit donc, à sa honte, obligé de prendre au sérieux la personnalité ridicule, commune et haïe du pseudo-Bonaparte.\par
Cette sale figure se faisait de son côté illusion sur les causes qui de plus en plus faisaient de lui un homme nécessaire. Tandis que son parti avait assez d’esprit pour attribuer aux circonstances l’importance croissante de Bonaparte, ce dernier croyait pouvoir la rapporter au pouvoir magique de son nom et à sa perpétuelle caricature de Napoléon. Il devenait tous les jours plus entreprenant. Il opposait aux pèlerinages à Wiesbaden et à Saint-Léonard ses tournées en France. Les bonapartistes avaient si peu confiance dans le charme qui se dégageait de sa personne qu’ils le faisaient accompagner, pour lui servir de claqueurs, de gens de la société du 10 décembre, cette organisation de la canaille parisienne ; ils les envoyaient en masse par chemin de fer ou par chaise poste. Ils mettaient dans la, bouche de leur marionnette des discours qui, suivant l’accueil, proclamaient que la résignation républicaine ou la ténacité persévérante étaient la devise électorale de la politique présidentielle. Malgré toutes les manœuvres, ces voyages n’étaient rien moins que des tournées triomphales.\par
Quand Bonaparte crut avoir ainsi enthousiasmé le peuple, il se mit en devoir de se ménager l’armée. Il fait ordonner de grandes revues dans la plaine de Satory, près de Versailles. À cette occasion, il cherche à acheter les soldats en leurs donnant des saucissons, du champagne et des cigares. Si le vrai Napoléon, dans les fatigues de ses marches conquérantes, savait remonter le courage de ses soldats par une familiarité patriarcale momentanée, le pseudo-Napoléon croyait que ses troupes le remerciaient en criant : « Vive Napoléon, vive le saucisson ! »\par
Ces revues firent éclater le différend longtemps dissimulé qui divisait Bonaparte et son ministre de la Guerre, d’Hautpoul, d’un côté, et Changarnier de l’autre. En Changarnier, le « parti de l’ordre » avait trouvé l’homme vraiment neutre qu’il cherchait, l’homme chez lequel il ne pouvait être question de préférences dynastiques particulières. Ce parti l’avait destiné à prendre la succession de Bonaparte. Changarnier, de plus, par son rôle le 29 janvier et le 13 juin 1849, était devenu le grand capitaine du « parti de l’ordre », le moderne Alexandre dont l’intervention brutale avait, aux yeux des bourgeois peureux, tranché le nœud gordien de la révolution. Aussi ridicule, au fond, que Bonaparte, il était devenu à bon marché une puissance et fut chargé par l’Assemblée de surveiller le président. Lui-même faisait parade, par exemple dans la question de la dotation, de la protection qu’il accordait à Bonaparte et devint de plus en plus puissant vis-à-vis de ce dernier et de ses ministres. Quand, à l’occasion du vote de la loi électorale, on s’attendait à une insurrection, il défendit à ses officiers d’exécuter un ordre quelconque venant du ministre de la Guerre ou du président. La presse contribuait à grossir la figure de Changarnier. Comme les grandes personnalités manquaient complètement, le parti de l’ordre se vit forcé d’attribuer à un seul individu la force qui manquait à toute la classe, de l’enfler jusqu’à en faire un monstre. C’est ainsi que naquit le mythe de Changarnier, \emph{boulevard de la société}. Sa charlatanerie présomptueuse, les airs importants et mystérieux avec lesquels il condescendait à porter le monde sur ses épaules, forment le contraste le plus ridicule avec les événements qui se passèrent pendant et après la revue de Satory. Ils montraient incontestablement qu’il suffisait d’un trait de plume de Bonaparte, de l’infiniment petit, pour ramener cette création fantastique de la terreur bourgeoise, le colosse Changarnier, aux dimensions de la médiocrité, et le transformer, lui, le héros sauveur de la société, en un général en retraite.\par
Il y avait longtemps que Bonaparte s’en était pris à Changarnier. Il avait provoqué le ministre de la Guerre à user de tracasseries disciplinaires vis-à-vis de son incommode protecteur. La dernière revue de Satory avait fait éclater cette rancune déjà ancienne. L’indignation constitutionnelle de Changarnier ne connut plus de limites quand il vit défiler les régiments de cavalerie aux cris anticonstitutionnels de : « Vive l’empereur ! » Bonaparte voulut empêcher tout débat désagréable de se produire au sujet de ces acclamations pendant la la session de l’Assemblée. Il éloigna le ministre de la Guerre d’Hautpoul en le nommant gouverneur d’Alger. Il mit à sa place un militaire éprouvé, vieux général de l’empire, dont la brutalité égalait parfaitement celle de Changarnier ; mais pour que le renvoi d’Hautpoul ne parût pas être une concession faite à Changarnier, il renvoya de Paris à Nantes le bras droit du grand sauveur de la société, le général Neumayer. C’était lui qui, à la dernière revue, avait ordonné à l’infanterie de défiler devant le successeur de Napoléon en gardant un silence de fer. Changarnier, atteint en Neumayer, protesta et menaça, mais en vain. Après deux jours de négociations, le décret de déplacement de Neumayer parut au \emph{Moniteur} et il ne resta plus au héros de l’ordre qu’à se soumettre à la discipline ou à se démettre.\par
La lutte de Bonaparte et de Changarnier est la suite de sa lutte contre le « parti de l’ordre ». La reprise des séances de l’Assemblée nationale, à la date du 11 novembre, eut lieu sous des auspices menaçants. Ce sera la tempête dans un verre d’eau. Pour l’essentiel l’ancien jeu continuera. La majorité du parti de l’ordre sera forcée, malgré les cris des hommes de principe de ses différentes fractions, de prolonger les pouvoirs du président. Bonaparte, en dépit de ses protestations antérieures, acculé par le manque de ressources, sera très disposé à recevoir des mains de l’Assemblée nationale cette prolongation de pouvoir sous la forme de simple délégation. Ainsi la solution est retardée, le \emph{statu quo} maintenu, une fraction du parti de l’ordre compromise par l’autre, affaiblie, rendue impossible ; la répression de l’ennemi commun, la masse de la nation, est étendue et poussée à bout, jusqu’à ce que les conditions économiques aient de nouveau atteint un point de développement tel qu’une nouvelle explosion envoie dans les airs tous ces partis querelleurs rejoindre la République constitutionnelle.\par
Pour tranquilliser les bourgeois, ajoutons, d’ailleurs, que le scandale survenu entre Bonaparte et le parti de l’ordre a pour résultat de ruiner à la Bourse une masse de petits capitalistes et de faire empocher leur fortune par les grands « loups-cerviers ».
\chapterclose

 


% at least one empty page at end (for booklet couv)
\ifbooklet
  \pagestyle{empty}
  \clearpage
  % 2 empty pages maybe needed for 4e cover
  \ifnum\modulo{\value{page}}{4}=0 \hbox{}\newpage\hbox{}\newpage\fi
  \ifnum\modulo{\value{page}}{4}=1 \hbox{}\newpage\hbox{}\newpage\fi


  \hbox{}\newpage
  \ifodd\value{page}\hbox{}\newpage\fi
  {\centering\color{rubric}\bfseries\noindent\large
    Hurlus ? Qu’est-ce.\par
    \bigskip
  }
  \noindent Des bouquinistes électroniques, pour du texte libre à participations libres,
  téléchargeable gratuitement sur \href{https://hurlus.fr}{\dotuline{hurlus.fr}}.\par
  \bigskip
  \noindent Cette brochure a été produite par des éditeurs bénévoles.
  Elle n’est pas faite pour être possédée, mais pour être lue, et puis donnée, ou déposée dans une boîte à livres.
  En page de garde, on peut ajouter une date, un lieu, un nom ;
  comme une fiche de bibliothèque en papier qui enregistre le voyage du texte.
  \par

  Ce texte a été choisi parce qu’une personne l’a aimé,
  ou haï, elle a pensé qu’il partipait à la formation de notre présent ;
  sans le souci de plaire, vendre, ou militer pour une cause.
  \par

  L’édition électronique est soigneuse, tant sur la technique
  que sur l’établissement du texte ; mais sans aucune prétention scolaire, au contraire.
  Le but est de s’adresser à tous, sans distinction de science ou de diplôme.
  \par

  Cet exemplaire en papier a été tiré sur une imprimante personnelle
   ou une photocopieuse. Tout le monde peut le faire.
  Il suffit de
  télécharger un fichier sur \href{https://hurlus.fr}{\dotuline{hurlus.fr}},
  d’imprimer, et agrafer (puis lire et donner).\par

  \bigskip

  \noindent PS : Les hurlus furent aussi des rebelles protestants qui cassaient les statues dans les églises catholiques. En 1566 démarra la révolte des gueux dans le pays de Lille. L’insurrection enflamma la région jusqu’à Anvers où les gueux de mer bloquèrent les bateaux espagnols.
  Ce fut une rare guerre de libération dont naquit un pays toujours libre : les Pays-Bas.
  En plat pays francophone, par contre, restèrent des bandes de huguenots, les hurlus, progressivement réprimés par la très catholique Espagne.
  Cette mémoire d’une défaite est éteinte, rallumons-la. Sortons les livres du culte universitaire, débusquons les idoles de l’époque, pour les démonter.
\fi

\end{document}
