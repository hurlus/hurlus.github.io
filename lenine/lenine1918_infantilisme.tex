%%%%%%%%%%%%%%%%%%%%%%%%%%%%%%%%%
% LaTeX model https://hurlus.fr %
%%%%%%%%%%%%%%%%%%%%%%%%%%%%%%%%%

% Needed before document class
\RequirePackage{pdftexcmds} % needed for tests expressions
\RequirePackage{fix-cm} % correct units

% Define mode
\def\mode{a4}

\newif\ifaiv % a4
\newif\ifav % a5
\newif\ifbooklet % booklet
\newif\ifcover % cover for booklet

\ifnum \strcmp{\mode}{cover}=0
  \covertrue
\else\ifnum \strcmp{\mode}{booklet}=0
  \booklettrue
\else\ifnum \strcmp{\mode}{a5}=0
  \avtrue
\else
  \aivtrue
\fi\fi\fi

\ifbooklet % do not enclose with {}
  \documentclass[french,twoside]{book} % ,notitlepage
  \usepackage[%
    papersize={105mm, 297mm},
    inner=12mm,
    outer=12mm,
    top=20mm,
    bottom=15mm,
    marginparsep=0pt,
  ]{geometry}
  \usepackage[fontsize=9.5pt]{scrextend} % for Roboto
\else\ifav
  \documentclass[french,twoside]{book} % ,notitlepage
  \usepackage[%
    a5paper,
    inner=25mm,
    outer=15mm,
    top=15mm,
    bottom=15mm,
    marginparsep=0pt,
  ]{geometry}
  \usepackage[fontsize=12pt]{scrextend}
\else% A4 2 cols
  \documentclass[twocolumn]{book}
  \usepackage[%
    a4paper,
    inner=15mm,
    outer=10mm,
    top=25mm,
    bottom=18mm,
    marginparsep=0pt,
  ]{geometry}
  \setlength{\columnsep}{20mm}
  \usepackage[fontsize=9.5pt]{scrextend}
\fi\fi

%%%%%%%%%%%%%%
% Alignments %
%%%%%%%%%%%%%%
% before teinte macros

\setlength{\arrayrulewidth}{0.2pt}
\setlength{\columnseprule}{\arrayrulewidth} % twocol
\setlength{\parskip}{0pt} % classical para with no margin
\setlength{\parindent}{1.5em}

%%%%%%%%%%
% Colors %
%%%%%%%%%%
% before Teinte macros

\usepackage[dvipsnames]{xcolor}
\definecolor{rubric}{HTML}{902c20} % the tonic
\def\columnseprulecolor{\color{rubric}}
\colorlet{borderline}{rubric!30!} % definecolor need exact code
\definecolor{shadecolor}{gray}{0.95}
\definecolor{bghi}{gray}{0.5}

%%%%%%%%%%%%%%%%%
% Teinte macros %
%%%%%%%%%%%%%%%%%
%%%%%%%%%%%%%%%%%%%%%%%%%%%%%%%%%%%%%%%%%%%%%%%%%%%
% <TEI> generic (LaTeX names generated by Teinte) %
%%%%%%%%%%%%%%%%%%%%%%%%%%%%%%%%%%%%%%%%%%%%%%%%%%%
% This template is inserted in a specific design
% It is XeLaTeX and otf fonts

\makeatletter % <@@@


\usepackage{blindtext} % generate text for testing
\usepackage{contour} % rounding words
\usepackage[nodayofweek]{datetime}
\usepackage{DejaVuSans} % font for symbols
\usepackage{enumitem} % <list>
\usepackage{etoolbox} % patch commands
\usepackage{fancyvrb}
\usepackage{fancyhdr}
\usepackage{fontspec} % XeLaTeX mandatory for fonts
\usepackage{footnote} % used to capture notes in minipage (ex: quote)
\usepackage{framed} % bordering correct with footnote hack
\usepackage{graphicx}
\usepackage{lettrine} % drop caps
\usepackage{lipsum} % generate text for testing
\usepackage[framemethod=tikz,]{mdframed} % maybe used for frame with footnotes inside
\usepackage{pdftexcmds} % needed for tests expressions
\usepackage{polyglossia} % non-break space french punct, bug Warning: "Failed to patch part"
\usepackage[%
  indentfirst=false,
  vskip=1em,
  noorphanfirst=true,
  noorphanafter=true,
  leftmargin=\parindent,
  rightmargin=0pt,
]{quoting}
\usepackage{ragged2e}
\usepackage{setspace}
\usepackage{tabularx} % <table>
\usepackage[explicit]{titlesec} % wear titles, !NO implicit
\usepackage{tikz} % ornaments
\usepackage{tocloft} % styling tocs
\usepackage[fit]{truncate} % used im runing titles
\usepackage{unicode-math}
\usepackage[normalem]{ulem} % breakable \uline, normalem is absolutely necessary to keep \emph
\usepackage{verse} % <l>
\usepackage{xcolor} % named colors
\usepackage{xparse} % @ifundefined
\XeTeXdefaultencoding "iso-8859-1" % bad encoding of xstring
\usepackage{xstring} % string tests
\XeTeXdefaultencoding "utf-8"
\PassOptionsToPackage{hyphens}{url} % before hyperref, which load url package
\usepackage{hyperref} % supposed to be the last one, :o) except for the ones to follow
\urlstyle{same} % after hyperref

% TOTEST
% \usepackage{hypcap} % links in caption ?
% \usepackage{marginnote}
% TESTED
% \usepackage{background} % doesn’t work with xetek
% \usepackage{bookmark} % prefers the hyperref hack \phantomsection
% \usepackage[color, leftbars]{changebar} % 2 cols doc, impossible to keep bar left
% \usepackage[utf8x]{inputenc} % inputenc package ignored with utf8 based engines
% \usepackage[sfdefault,medium]{inter} % no small caps
% \usepackage{firamath} % choose firasans instead, firamath unavailable in Ubuntu 21-04
% \usepackage{flushend} % bad for last notes, supposed flush end of columns
% \usepackage[stable]{footmisc} % BAD for complex notes https://texfaq.org/FAQ-ftnsect
% \usepackage{helvet} % not for XeLaTeX
% \usepackage{multicol} % not compatible with too much packages (longtable, framed, memoir…)
% \usepackage[default,oldstyle,scale=0.95]{opensans} % no small caps
% \usepackage{sectsty} % \chapterfont OBSOLETE
% \usepackage{soul} % \ul for underline, OBSOLETE with XeTeX
% \usepackage[breakable]{tcolorbox} % text styling gone, footnote hack not kept with breakable



% Metadata inserted by a program, from the TEI source, for title page and runing heads
\title{\textbf{ Sur l’infantilisme "de gauche" et les idées petites-bourgeoises }}
\date{1918}
\author{Lénine}
\def\elbibl{Lénine. 1918. \emph{Sur l’infantilisme "de gauche" et les idées petites-bourgeoises}}
\def\elsource{\emph{Pravda}, n° 88, 89, 90 (9, 10,11 mai 1918) - \href{https://www.marxists.org/francais/lenin/works/1918/05/vil19180505.htm}{\dotuline{Œuvres t. 27}}\footnote{\href{https://www.marxists.org/francais/lenin/works/1918/05/vil19180505.htm}{\url{https://www.marxists.org/francais/lenin/works/1918/05/vil19180505.htm}}} }

% Default metas
\newcommand{\colorprovide}[2]{\@ifundefinedcolor{#1}{\colorlet{#1}{#2}}{}}
\colorprovide{rubric}{red}
\colorprovide{silver}{Gray}
\@ifundefined{syms}{\newfontfamily\syms{DejaVu Sans}}{}
\newif\ifdev
\@ifundefined{elbibl}{% No meta defined, maybe dev mode
  \newcommand{\elbibl}{Titre court ?}
  \newcommand{\elbook}{Titre du livre source ?}
  \newcommand{\elabstract}{Résumé\par}
  \newcommand{\elurl}{http://oeuvres.github.io/elbook/2}
  \author{Éric Lœchien}
  \title{Un titre de test assez long pour vérifier le comportement d’une maquette}
  \date{1566}
  \devtrue
}{}
\let\eltitle\@title
\let\elauthor\@author
\let\eldate\@date


\defaultfontfeatures{
  % Mapping=tex-text, % no effect seen
  Scale=MatchLowercase,
  Ligatures={TeX,Common},
}

\@ifundefined{\columnseprulecolor}{%
    \patchcmd\@outputdblcol{% find
      \normalcolor\vrule
    }{% and replace by
      \columnseprulecolor\vrule
    }{% success
    }{% failure
      \@latex@warning{Patching \string\@outputdblcol\space failed}%
    }
}{}

\hypersetup{
  % pdftex, % no effect
  pdftitle={\elbibl},
  % pdfauthor={Your name here},
  % pdfsubject={Your subject here},
  % pdfkeywords={keyword1, keyword2},
  bookmarksnumbered=true,
  bookmarksopen=true,
  bookmarksopenlevel=1,
  pdfstartview=Fit,
  breaklinks=true, % avoid long links
  pdfpagemode=UseOutlines,    % pdf toc
  hyperfootnotes=true,
  colorlinks=false,
  pdfborder=0 0 0,
  % pdfpagelayout=TwoPageRight,
  % linktocpage=true, % NO, toc, link only on page no
}


% generic typo commands
\newcommand{\astermono}{\medskip\centerline{\color{rubric}\large\selectfont{\syms ✻}}\medskip\par}%
\newcommand{\astertri}{\medskip\par\centerline{\color{rubric}\large\selectfont{\syms ✻\,✻\,✻}}\medskip\par}%
\newcommand{\asterism}{\bigskip\par\noindent\parbox{\linewidth}{\centering\color{rubric}\large{\syms ✻}\\{\syms ✻}\hskip 0.75em{\syms ✻}}\bigskip\par}%

% lists
\newlength{\listmod}
\setlength{\listmod}{\parindent}
\setlist{
  itemindent=!,
  listparindent=\listmod,
  labelsep=0.2\listmod,
  parsep=0pt,
  % topsep=0.2em, % default topsep is best
}
\setlist[itemize]{
  label=—,
  leftmargin=0pt,
  labelindent=1.2em,
  labelwidth=0pt,
}
\setlist[enumerate]{
  label={\bf\color{rubric}\arabic*.},
  labelindent=0.8\listmod,
  leftmargin=\listmod,
  labelwidth=0pt,
}
\newlist{listalpha}{enumerate}{1}
\setlist[listalpha]{
  label={\bf\color{rubric}\alph*.},
  leftmargin=0pt,
  labelindent=0.8\listmod,
  labelwidth=0pt,
}
\newcommand{\listhead}[1]{\hspace{-1\listmod}\emph{#1}}

\renewcommand{\hrulefill}{%
  \leavevmode\leaders\hrule height 0.2pt\hfill\kern\z@}

% General typo
\DeclareTextFontCommand{\textlarge}{\large}
\DeclareTextFontCommand{\textsmall}{\small}


% commands, inlines
\newcommand{\anchor}[1]{\Hy@raisedlink{\hypertarget{#1}{}}} % link to top of an anchor (not baseline)
\newcommand\abbr{}
\newcommand{\autour}[1]{\tikz[baseline=(X.base)]\node [draw=rubric,thin,rectangle,inner sep=1.5pt, rounded corners=3pt] (X) {\color{rubric}#1};}
\newcommand\corr{}
\newcommand{\ed}[1]{ {\color{silver}\sffamily\footnotesize (#1)} } % <milestone ed="1688"/>
\newcommand\expan{}
\newcommand\gap{}
\renewcommand{\LettrineFontHook}{\color{rubric}}
\newcommand{\initial}[2]{\lettrine[lines=2, loversize=0.3, lhang=0.3]{#1}{#2}}
\newcommand{\initialiv}[2]{%
  \let\oldLFH\LettrineFontHook
  % \renewcommand{\LettrineFontHook}{\color{rubric}\ttfamily}
  \IfSubStr{QJ’}{#1}{
    \lettrine[lines=4, lhang=0.2, loversize=-0.1, lraise=0.2]{\smash{#1}}{#2}
  }{\IfSubStr{É}{#1}{
    \lettrine[lines=4, lhang=0.2, loversize=-0, lraise=0]{\smash{#1}}{#2}
  }{\IfSubStr{ÀÂ}{#1}{
    \lettrine[lines=4, lhang=0.2, loversize=-0, lraise=0, slope=0.6em]{\smash{#1}}{#2}
  }{\IfSubStr{A}{#1}{
    \lettrine[lines=4, lhang=0.2, loversize=0.2, slope=0.6em]{\smash{#1}}{#2}
  }{\IfSubStr{V}{#1}{
    \lettrine[lines=4, lhang=0.2, loversize=0.2, slope=-0.5em]{\smash{#1}}{#2}
  }{
    \lettrine[lines=4, lhang=0.2, loversize=0.2]{\smash{#1}}{#2}
  }}}}}
  \let\LettrineFontHook\oldLFH
}
\newcommand{\labelchar}[1]{\textbf{\color{rubric} #1}}
\newcommand{\milestone}[1]{\autour{\footnotesize\color{rubric} #1}} % <milestone n="4"/>
\newcommand\name{}
\newcommand\orig{}
\newcommand\orgName{}
\newcommand\persName{}
\newcommand\placeName{}
\newcommand{\pn}[1]{{\sffamily\textbf{#1.}} } % <p n="3"/>
\newcommand\reg{}
% \newcommand\ref{} % already defined
\newcommand\sic{}
\def\mednobreak{\ifdim\lastskip<\medskipamount
  \removelastskip\nopagebreak\medskip\fi}
\def\bignobreak{\ifdim\lastskip<\bigskipamount
  \removelastskip\nopagebreak\bigskip\fi}

% commands, blocks
\newcommand{\byline}[1]{\bigskip{\RaggedLeft{#1}\par}\bigskip}
\newcommand{\bibl}[1]{{\RaggedLeft{#1}\par\bigskip}}
\newcommand{\biblitem}[1]{{\noindent\hangindent=\parindent   #1\par}}
\newcommand{\dateline}[1]{\medskip{\RaggedLeft{#1}\par}\bigskip}
\newcommand{\labelblock}[1]{\medbreak{\noindent\color{rubric}\bfseries #1}\par\mednobreak}
\newcommand{\salute}[1]{\bigbreak{#1}\par\medbreak}
\newcommand{\signed}[1]{\bigbreak\filbreak{\raggedleft #1\par}\medskip}

% environments for blocks (some may become commands)
\newenvironment{borderbox}{}{} % framing content
\newenvironment{citbibl}{\ifvmode\hfill\fi}{\ifvmode\par\fi }
\newenvironment{docAuthor}{\ifvmode\vskip4pt\fontsize{16pt}{18pt}\selectfont\fi\itshape}{\ifvmode\par\fi }
\newenvironment{docDate}{}{\ifvmode\par\fi }
\newenvironment{docImprint}{\vskip6pt}{\ifvmode\par\fi }
\newenvironment{docTitle}{\vskip6pt\bfseries\fontsize{18pt}{22pt}\selectfont}{\par }
\newenvironment{msHead}{\vskip6pt}{\par}
\newenvironment{msItem}{\vskip6pt}{\par}
\newenvironment{titlePart}{}{\par }


% environments for block containers
\newenvironment{argument}{\fontlight\parindent0pt}{\vskip1.5em}
\newenvironment{biblfree}{}{\ifvmode\par\fi }
\newenvironment{bibitemlist}[1]{%
  \list{\@biblabel{\@arabic\c@enumiv}}%
  {%
    \settowidth\labelwidth{\@biblabel{#1}}%
    \leftmargin\labelwidth
    \advance\leftmargin\labelsep
    \@openbib@code
    \usecounter{enumiv}%
    \let\p@enumiv\@empty
    \renewcommand\theenumiv{\@arabic\c@enumiv}%
  }
  \sloppy
  \clubpenalty4000
  \@clubpenalty \clubpenalty
  \widowpenalty4000%
  \sfcode`\.\@m
}%
{\def\@noitemerr
  {\@latex@warning{Empty `bibitemlist' environment}}%
\endlist}
\newenvironment{quoteblock}% may be used for ornaments
  {\begin{quoting}}
  {\end{quoting}}

% table () is preceded and finished by custom command
\newcommand{\tableopen}[1]{%
  \ifnum\strcmp{#1}{wide}=0{%
    \begin{center}
  }
  \else\ifnum\strcmp{#1}{long}=0{%
    \begin{center}
  }
  \else{%
    \begin{center}
  }
  \fi\fi
}
\newcommand{\tableclose}[1]{%
  \ifnum\strcmp{#1}{wide}=0{%
    \end{center}
  }
  \else\ifnum\strcmp{#1}{long}=0{%
    \end{center}
  }
  \else{%
    \end{center}
  }
  \fi\fi
}


% text structure
\newcommand\chapteropen{} % before chapter title
\newcommand\chaptercont{} % after title, argument, epigraph…
\newcommand\chapterclose{} % maybe useful for multicol settings
\setcounter{secnumdepth}{-2} % no counters for hierarchy titles
\setcounter{tocdepth}{5} % deep toc
\markright{\@title} % ???
\markboth{\@title}{\@author} % ???
\renewcommand\tableofcontents{\@starttoc{toc}}
% toclof format
% \renewcommand{\@tocrmarg}{0.1em} % Useless command?
% \renewcommand{\@pnumwidth}{0.5em} % {1.75em}
\renewcommand{\@cftmaketoctitle}{}
\setlength{\cftbeforesecskip}{\z@ \@plus.2\p@}
\renewcommand{\cftchapfont}{}
\renewcommand{\cftchapdotsep}{\cftdotsep}
\renewcommand{\cftchapleader}{\normalfont\cftdotfill{\cftchapdotsep}}
\renewcommand{\cftchappagefont}{\bfseries}
\setlength{\cftbeforechapskip}{0em \@plus\p@}
% \renewcommand{\cftsecfont}{\small\relax}
\renewcommand{\cftsecpagefont}{\normalfont}
% \renewcommand{\cftsubsecfont}{\small\relax}
\renewcommand{\cftsecdotsep}{\cftdotsep}
\renewcommand{\cftsecpagefont}{\normalfont}
\renewcommand{\cftsecleader}{\normalfont\cftdotfill{\cftsecdotsep}}
\setlength{\cftsecindent}{1em}
\setlength{\cftsubsecindent}{2em}
\setlength{\cftsubsubsecindent}{3em}
\setlength{\cftchapnumwidth}{1em}
\setlength{\cftsecnumwidth}{1em}
\setlength{\cftsubsecnumwidth}{1em}
\setlength{\cftsubsubsecnumwidth}{1em}

% footnotes
\newif\ifheading
\newcommand*{\fnmarkscale}{\ifheading 0.70 \else 1 \fi}
\renewcommand\footnoterule{\vspace*{0.3cm}\hrule height \arrayrulewidth width 3cm \vspace*{0.3cm}}
\setlength\footnotesep{1.5\footnotesep} % footnote separator
\renewcommand\@makefntext[1]{\parindent 1.5em \noindent \hb@xt@1.8em{\hss{\normalfont\@thefnmark . }}#1} % no superscipt in foot


% orphans and widows
\clubpenalty=9996
\widowpenalty=9999
\brokenpenalty=4991
\predisplaypenalty=10000
\postdisplaypenalty=1549
\displaywidowpenalty=1602
\hyphenpenalty=400
% Copied from Rahtz but not understood
\def\@pnumwidth{1.55em}
\def\@tocrmarg {2.55em}
\def\@dotsep{4.5}
\emergencystretch 3em
\hbadness=4000
\pretolerance=750
\tolerance=2000
\vbadness=4000
\def\Gin@extensions{.pdf,.png,.jpg,.mps,.tif}
% \renewcommand{\@cite}[1]{#1} % biblio

\makeatother % /@@@>
%%%%%%%%%%%%%%
% </TEI> end %
%%%%%%%%%%%%%%


%%%%%%%%%%%%%
% footnotes %
%%%%%%%%%%%%%
\renewcommand{\thefootnote}{\bfseries\textcolor{rubric}{\arabic{footnote}}} % color for footnote marks

%%%%%%%%%
% Fonts %
%%%%%%%%%
\usepackage[]{roboto} % SmallCaps, Regular is a bit bold
% \linespread{0.90} % too compact, keep font natural
\newfontfamily\fontrun[]{Roboto Condensed Light} % condensed runing heads
\ifav
  \setmainfont[
    ItalicFont={Roboto Light Italic},
  ]{Roboto}
\else\ifbooklet
  \setmainfont[
    ItalicFont={Roboto Light Italic},
  ]{Roboto}
\else
\setmainfont[
  ItalicFont={Roboto Italic},
]{Roboto Light}
\fi\fi
\renewcommand{\LettrineFontHook}{\bfseries\color{rubric}}
% \renewenvironment{labelblock}{\begin{center}\bfseries\color{rubric}}{\end{center}}

%%%%%%%%
% MISC %
%%%%%%%%

\setdefaultlanguage[frenchpart=false]{french} % bug on part


\newenvironment{quotebar}{%
    \def\FrameCommand{{\color{rubric!10!}\vrule width 0.5em} \hspace{0.9em}}%
    \def\OuterFrameSep{\itemsep} % séparateur vertical
    \MakeFramed {\advance\hsize-\width \FrameRestore}
  }%
  {%
    \endMakeFramed
  }
\renewenvironment{quoteblock}% may be used for ornaments
  {%
    \savenotes
    \setstretch{0.9}
    \normalfont
    \begin{quotebar}
  }
  {%
    \end{quotebar}
    \spewnotes
  }


\renewcommand{\pn}[1]{{\footnotesize\autour{ #1}}} % <p n="3"/>
\renewcommand{\headrulewidth}{\arrayrulewidth}
\renewcommand{\headrule}{{\color{rubric}\hrule}}

% delicate tuning, image has produce line-height problems in title on 2 lines
\titleformat{name=\chapter} % command
  [display] % shape
  {\vspace{1.5em}\centering} % format
  {} % label
  {0pt} % separator between n
  {}
[{\color{rubric}\huge\textbf{#1}}\bigskip] % after code
% \titlespacing{command}{left spacing}{before spacing}{after spacing}[right]
\titlespacing*{\chapter}{0pt}{-2em}{0pt}[0pt]

\titleformat{name=\section}
  [block]{}{}{}{}
  [\vbox{\color{rubric}\large\raggedleft\textbf{#1}}]
\titlespacing{\section}{0pt}{0pt plus 4pt minus 2pt}{\baselineskip}

\titleformat{name=\subsection}
  [block]
  {}
  {} % \thesection
  {} % separator \arrayrulewidth
  {}
[\vbox{\large\textbf{#1}}]
% \titlespacing{\subsection}{0pt}{0pt plus 4pt minus 2pt}{\baselineskip}

\ifaiv
  \fancypagestyle{main}{%
    \fancyhf{}
    \setlength{\headheight}{1.5em}
    \fancyhead{} % reset head
    \fancyfoot{} % reset foot
    \fancyhead[L]{\truncate{0.45\headwidth}{\fontrun\elbibl}} % book ref
    \fancyhead[R]{\truncate{0.45\headwidth}{ \fontrun\nouppercase\leftmark}} % Chapter title
    \fancyhead[C]{\thepage}
  }
  \fancypagestyle{plain}{% apply to chapter
    \fancyhf{}% clear all header and footer fields
    \setlength{\headheight}{1.5em}
    \fancyhead[L]{\truncate{0.9\headwidth}{\fontrun\elbibl}}
    \fancyhead[R]{\thepage}
  }
\else
  \fancypagestyle{main}{%
    \fancyhf{}
    \setlength{\headheight}{1.5em}
    \fancyhead{} % reset head
    \fancyfoot{} % reset foot
    \fancyhead[RE]{\truncate{0.9\headwidth}{\fontrun\elbibl}} % book ref
    \fancyhead[LO]{\truncate{0.9\headwidth}{\fontrun\nouppercase\leftmark}} % Chapter title, \nouppercase needed
    \fancyhead[RO,LE]{\thepage}
  }
  \fancypagestyle{plain}{% apply to chapter
    \fancyhf{}% clear all header and footer fields
    \setlength{\headheight}{1.5em}
    \fancyhead[L]{\truncate{0.9\headwidth}{\fontrun\elbibl}}
    \fancyhead[R]{\thepage}
  }
\fi

\ifav % a5 only
  \titleclass{\section}{top}
\fi

\newcommand\chapo{{%
  \vspace*{-3em}
  \centering % no vskip ()
  {\Large\addfontfeature{LetterSpace=25}\bfseries{\elauthor}}\par
  \smallskip
  {\large\eldate}\par
  \bigskip
  {\Large\selectfont{\eltitle}}\par
  \bigskip
  {\color{rubric}\hline\par}
  \bigskip
  {\Large LIVRE LIBRE À PRIX LIBRE, DEMANDEZ AU COMPTOIR\par}
  \centerline{\small\color{rubric} {hurlus.fr, tiré le \today}}\par
  \bigskip
}}


\begin{document}
\pagestyle{empty}
\ifbooklet{
  \thispagestyle{empty}
  \centering
  {\LARGE\bfseries{\elauthor}}\par
  \bigskip
  {\Large\eldate}\par
  \bigskip
  \bigskip
  {\LARGE\selectfont{\eltitle}}\par
  \vfill\null
  {\color{rubric}\setlength{\arrayrulewidth}{2pt}\hline\par}
  \vfill\null
  {\Large LIVRE LIBRE À PRIX LIBRE, DEMANDEZ AU COMPTOIR\par}
  \centerline{\small{hurlus.fr, tiré le \today}}\par
  \newpage\null\thispagestyle{empty}\newpage
  \addtocounter{page}{-2}
}\fi

\thispagestyle{empty}
\ifaiv
  \twocolumn[\chapo]
\else
  \chapo
\fi
{\it\elabstract}
\bigskip
\makeatletter\@starttoc{toc}\makeatother % toc without new page
\bigskip

\pagestyle{main} % after style

  \noindent L’édition par le petit groupe des « communistes de gauche » de leur revue \emph{Kommounist}\footnote{\emph{Kommounist}  [Le communiste] : revue des « communistes de gauche » autour de Boukharine, Radek et \href{https://www.marxists.org/francais/bios/piatakov.htm}{\dotuline{Piatakov}}\footnote{\href{https://www.marxists.org/francais/bios/piatakov.htm}{\url{https://www.marxists.org/francais/bios/piatakov.htm}}}. 4 numéros parurent d’avril à juin 1918.} (n° 1, 20 avril 1918) et de leur « thèses » confirme parfaitement ce que j’avais dit dans ma brochure sur \emph{Les tâches immédiates du pouvoir}. On ne pouvait souhaiter une confirmation plus frappante – dans la littérature politique – de toute la naïveté qu’il y a à défendre le laisser­aller petit‑bourgeois qui se dissimule parfois sous des mots d’ordre « de gauche ». Il est utile et nécessaire de s’arrêter aux raisonnements des « communistes de gauche », car ils sont caractéristiques de la période que nous vivons; ils font ressortir avec une netteté extraordinaire, sous son aspect négatif, ce qu’il y a de plus important dans cette période; ils sont riches d’enseignements, car nous avons affaire ici aux meilleurs représentants de ceux qui n’ont pas compris la situation actuelle, à des hommes qui, par leurs connaissances et leur dévouement, sont de beaucoup, de très loin supérieurs aux représentants \emph{ordinaires} de la même erreur, je veux parler des socialistes‑révolutionnaires de gauche.\par
En sa qualité de facteur politique, ou prétendant à un rôle politique, le groupe des « communistes de gauche » nous a donné ses « thèses sur la situation actuelle ». C’est une excellente habitude marxiste que de présenter un exposé complet et harmonieux des principes de ses conceptions et de sa tactique. Et cette excellente habitude marxiste a contribué à mettre en évidence l’erreur de nos \hspace{1em}« communistes de gauche », car rien que leur tentative de construire une argumentation ‑ au lieu de se cantonner dans des déclamations ‑ suffit à faire éclater l’inconsistance de leurs arguments.\par
Ce qui saute avant tout aux yeux, c’est l’abondance des allusions, des insinuations, des dérobades au sujet de la vieille question de savoir si l’on a bien fait de conclure la paix de Brest‑Litovsk. \hspace{1em}Les « communistes de gauche » n’ont pas osé poser cette question de front; ils ont bonne mine, entassant argument sur argument, ergotant à perte de vue, recherchant tous les « d’une part » et tous les « d’autre part », dissertant de tout et de rien, et s’efforçant d’ignorer combien ils se contredisent eux‑mêmes Ils ont bien soin de relever le chiffre des 12 voix qui se sont affirmées contre la paix au congrès du parti, 28 étant pour, mais ils passent modestement sous silence le fait qu’à la fraction bolchévique du Congrès des Soviets, ils ont réuni moins du dixième des voix sur plusieurs centaines de votants. Ils inventent une « théorie » d’après laquelle ce sont « les éléments fatigués et déclassés » qui étaient pour la paix, tandis que « les ouvriers et les paysans des régions du Sud, économiquement plus d’aplomb et mieux ravitaillées en blé » étaient contre… Comment ne pas rire de ces affirmations ? Pas un mot sur le vote en faveur de la paix émis par le Congrès des Soviets d’Ukraine ; pas un mot sur le caractère social, le caractère de classe du conglomérat politique typiquement petit‑bourgeois et déclassé (le parti socialiste-révolutionnaire de gauche) qui, en Russie, était contre la paix. C'est là une manière typique enfantine de dissimuler sa banqueroute sous d’amusantes explications « scientifiques», en taisant des faits dont la simple énumération eût montré que c’étaient précisément des éléments intellectuels déclassés des couches « supérieures » du parti qui combattaient la paix par des mots d’ordre relevant de la phraséologie petite-bourgeoise révolutionnaire, et que ce sont précisément les \emph{masses} des ouvriers et des paysans exploités qui ont imposé la paix.\par
La simple et claire vérité se fraie néanmoins un chemin à travers toutes ces déclarations et ces arguties des « communistes de gauche » sur la question de la guerre et de la paix. La conclusion de la paix ‑ doivent reconnaître les auteurs des thèses ‑ a momentanément affaibli l’aspiration des impérialistes à s’entendre sur le plan international (les « communistes de gauche » le disent en des termes inexacts, mais ce n’est pas ici le lieu de nous arrêter à leurs inexactitudes). « La conclusion de la paix a déjà entraîné une exacerbation de la lutte entre les puissances impérialistes. »\par
\bigbreak
\noindent Voilà qui est un fait. Voilà qui a une importance \emph{décisive.}\par
Voilà pourquoi les adversaires de la conclusion de la paix faisaient objectivement le jeu des impérialistes, étaient tombés dans leur piège. Car, tant que n’a pas éclaté une révolution socialiste internationale, embrassant plusieurs pays, assez forte pour vaincre \emph{l’impérialisme international}, le premier devoir des socialistes victorieux dans un seul pays (particulièrement arriéré) est de \emph{ne pas} accepter la bataille contre les géants impérialistes, de s’efforcer de l’éviter, d’attendre que la lutte des impérialistes entre eux les affaiblisse \emph{encore plus}, qu’elle rapproche encore la révolution dans les autres pays. Cette simple vérité, nos « communistes de gauche » ne l’ont pas comprise en janvier, en février et en mars ; ils craignent aujourd’hui encore de la reconnaître ouvertement, elle se fraie un chemin à travers tous leurs balbutiements embrouillés : « \emph{On ne saurait, d’une part, ne pas admettre… mais on doit, d’autre part, reconnaître…»}\par
« Dans le courant du printemps et de l’été prochains écrivent les « communistes de gauche » dans leurs thèses, doit commencer l’écroulement du système impérialiste, écroulement que la victoire éventuelle de l’impérialisme allemand dans la phase actuelle de la guerre ne peut que différer, et qui revêtira alors des formes encore plus aiguës. »\par
La formule est ici encore plus enfantine et inexacte, en dépit de tout son appareil pseudo-­scientifique. C'est le propre des enfants de « comprendre » la science comme si elle pouvait prévoir en quelle année, au printemps, en automne ou en hiver, « doit commencer l’écroulement ».\par
Ce sont des tentatives ridicules de connaître l’incon­naissable. Aucun homme politique sérieux ne dira jamais \emph{quand} « doit commencer » l’écroulement d’un « système » (d’autant plus que l’écroulement du \emph{système} a déjà commencé, et qu’il s’agit de dire quand se produira l’explosion dans les \emph{divers} pays). Mais une vérité indéniable se fraie cependant un chemin à travers la puérile impuissance de cette formule : les explosions de la révolution dans les autres pays plus avancés que le nôtre, sont \emph{plus près} de nous maintenant, un mois après la « trêve » ouverte par la signature de la paix, qu’elles ne l’étaient il y a un mois ou un mois et demi.\par

\labelblock{Qu'est‑ce à dire ?}

\noindent C'est dire que les partisans de la paix avaient pleinement raison, et que l’histoire les a déjà justifiés, lors qu’ils s’efforçaient d’expliquer aux amateurs d’attitudes romantiques qu’il fallait savoir calculer le rapport de forces et \emph{ne pas venir en aide} aux impérialistes en leur facilitant la lutte contre le socialisme lorsque celui‑ci est encore faible et que ses chances dans la lutte sont a priori défavorables.\par
Mais nos « communistes de gauche », qui aiment aussi se qualifier de « communistes prolétariens», car ils n’ont pas grand‑chose de prolétarien et sont surtout des petits-bourgeois, ne savent pas réfléchir au rapport des forces ni à la nécessité d’en tenir compte. C'est là l’essentiel du marxisme et de la tactique marxiste, mais ils passent outre à l’ « essentiel », avec des phrases pleines de « superbe » du genre de celle‑ci :\par

\begin{quoteblock}
 \noindent « […] L'enracinement parmi les masses d’une « psychologie de paix » toute de passivité est un fait objectif de « conjoncture politique actuelle[…] »
\end{quoteblock}

\noindent N'est-ce pas là vraiment une perle ? Alors que, après trois années de la plus douloureuse et de la plus réactionnaire des guerres, le peuple a obtenu, grâce au pouvoir des Soviets et à sa juste tactique qui ne s’égare pas dans la phraséologie, une petite, une toute petite trêve, bien précaire et incomplète, nos petits intellectuels « de gauche » déclarent d’un air profond, avec le superbe aplomb d’un Narcisse amoureux de lui-même : «L'enracinement (!!!) parmi les masses (???) d’une psychologie de paix toute de passivité (!!!???). » N'avais‑je pas raison de dire au congrès du parti que le journal ou la revue des « gauches » aurait dû s’appeler le \emph{Gentilhomme} et non le \emph{Kommou­nist} ?\par
Comment un communiste qui comprend tant soit peu les conditions d’existence et la mentalité des masses labo­rieuses et exploitées peut‑il en venir à cette mentalité typi­que d’intellectuel, de petit bourgeois, de déclassé, à l’état d’esprit d’un nobliau ou d’un gentilhomme, qui taxe de « passivité » la « psychologie de paix » et considère comme « actif » le geste de brandir un sabre de carton ? C'est bien brandir un sabre de carton que d’éluder, comme le font nos « communistes de gauche », le fait par­faitement connu et démontré une fois de plus par la guerre en Ukraine que les peuples épuisés par trois années de car­nage ne peuvent poursuivre la guerre sans bénéficier d’une trêve, et que la guerre, si l’on n’a pas la force de l’organiser à l’échelle nationale, engendre très souvent un état de déliquescence inhérent à l’esprit petit-propriétaire, et non la discipline de fer, propre au prolétariat. La revue \emph{Kommounist} nous montre à chaque instant que nos « communistes de gauche » n’ont pas la moindre idée de ce que c’est que la discipline de fer prolétarienne et des moyens de l’assurer, qu’ils sont pénétrés jusqu’à la moelle de la psychologie de l’intellectuel petit‑bourgeois déclassé.\par
Mais peut‑être les phrases des « communistes de gauche » ne sont‑elles que le fruit d’une fougue enfantine, d’ailleurs tournée vers le passé et n’ayant par conséquent aucune portée politique ? C'est ainsi que d’aucuns essaient de défendre nos « communistes de gauche ». Mais c’est faux. Quand on prétend assumer un rôle de direc­tion politique, on doit savoir \emph{méditer} sur les tâches politi­ques, sinon nos « communistes de gauche » en viennent à manquer de caractère et à prêcher des flottements dont le seul sens objectif est \emph{d’aider} les impérialistes à provoquer la République Soviétique de Russie à un combat manifes­tement désavantageux pour cette dernière, \emph{d’aider les} impérialistes à nous faire tomber dans un piège. Écoutez :\par

\begin{quoteblock}
 \noindent « […] La révolution ouvrière de Russie ne peut pas « rester sauve » en abandonnant le chemin de la révolution internationale, en évitant sans cesse le combat et en reculant devant la poussée du capital international, en faisant des concessions au « capital national ».\par
 « De ce point de vue, il est indispensable de poursuivre résolument une politique internationale de classe, joignant à la propagande révolutionnaire internationale par la parole et par l’action, le renforcement de la liaison organique avec le socialisme international (et non avec la bourgeoisie internationale) »[…]
\end{quoteblock}

\noindent Nous aurons l’occasion de reparler des attaques contenues dans ce passage et se rapportant à la politique intérieure. Mais voyez d’abord ce déferlement de grandes phrases ‑ en même temps que cette timidité sur le plan pratique ‑ dans le domaine de la politique extérieure. Quelle est la tactique \emph{obligatoire} pour quiconque ne veut pas être un instrument de la provocation impérialiste et tomber dans le piège qui nous est tendu \emph{actuellement} ? Tout homme politique doit donner à cette question une réponse nette et directe. On connaît la réponse de notre parti : dans la situation \emph{actuelle}, il faut \emph{battre en retraite}, éviter le combat. Nos « communistes de gauche » n’osent pas dire le contraire et tirent en l’air : « poursuivre résolument une politique internationale de classe » !!\par
C'est mystifier les masses. Si vous voulez faire la guerre maintenant, dites‑le franchement. Si vous ne voulez pas \emph{battre en retraite} maintenant, dites‑le franchement. Sinon vous devenez, de par votre rôle objectif, des instruments de la provocation impérialiste. Quant à votre « psychologie » subjective, elle est celle du petit bourgeois exaspéré qui plastronne et fanfaronne, mais sent très bien que le prolétaire \emph{a raison} de battre en retraite et de s’efforcer de le faire d’une façon organisée, que le prolétaire a raison d’estimer que tant qu’on n’a pas encore assez de forces, il faut reculer (devant l’impérialisme occidental et oriental), fût‑ce jusqu’à l’Oural, car c’est la \emph{seule chance} de triompher en cette période de maturation de la révolution en Occident, révolution qui (malgré les bavardages des « communistes de gauche ») ne doit pas commencer « au printemps ou en été » mais devient \emph{de mois en mois} de plus en plus proche et probable.\par
Les « communistes de gauche » n’ont pas de politique « en propre » ; ils \emph{n’osent pas} déclarer que la retraite est inutile \emph{en ce moment.} Ils tergiversent et se dérobent, jouant sur les mots et parlant d’éviter « sans cesse » le combat, alors qu’il est question de l’éluder \emph{en ce moment.} Ils lan­cent des bulles de savon : « propagande révolutionnaire internationale par l’action » !! Qu'est‑ce que cela veut dire ?\par
Cela ne peut vouloir dire que de deux choses l’une : ou bien c’est pure vantardise à la Nozdrev\footnote{Nozdrev : personnage de la pièce Les âmes mortes de Gogol. Type du hobereau grossier et impudent.}, ou bien il s’agit d’une guerre offensive en vue de renverser l’im­périalisme international. Comme on ne saurait avancer ouvertement pareille absurdité, les « communistes de gau­che » se voient obligés, de peur d’être tournés en ridicule par tout prolétaire conscient, de se réfugier derrière des phrases grandiloquentes et creuses: peut‑être le lecteur distrait ne pensera‑t‑il pas ainsi à se demander ce que veut dire exactement cette « propagande révolutionnaire internationale par l’action » ?\par
\bigbreak
\noindent Lancer des phrases ronflantes, c’est le propre des intellectuels, petits‑bourgeois déclassés. Il est certain que les prolétaires communistes organisés condamneront cette « méthode », tout au moins en la ridiculisant et en expulsant les partisans de cette méthode de tout poste responsable.\par
Il faut dire aux masses l’amère vérité simplement, clairement, sans ambages : il est possible et même probable que le parti militaire l’emporte une fois de plus en Allemagne (dans le sens d’une offensive immédiate contre nous) et que l’Allemagne et le Japon essaient de nous étrangler et de partager notre territoire en vertu d’un accord formel ou tacite. Si nous ne voulons pas prêter l’oreille aux braillards, notre tactique doit consister à attendre, à gagner du temps, à éviter le combat, à battre en retraite. Si nous repoussons loin de nous les braillards et si nous nous « reprenons en main », en créant une véritable discipline de fer, véritablement prolétarienne, véritablement communiste, nous avons des chances sérieuses de gagner de nombreux mois. Et alors, même en reculant (en mettant les choses au pire) jusqu’à l’Oural, nous \emph{faciliterons} à notre allié (le prolétariat international) le concours qu’il peut nous apporter, nous lui offrirons (pour employer le vocabulaire sportif) la possibilité de « couvrir » la distance qui sépare le début des explosions révolutionnaires de la révolution elle-même.\par
Cette tactique est la seule qui puisse renforcer effectivement la liaison entre un détachement du socialisme international qui se trouve momentanément isolé et les autres détachements; alors que chez vous, chers « communistes de gauche », on ne voit à vrai dire qu’un « renforcement de la liaison organique » entre une phrase ronflante et un autre phrase ronflante. C'est là une piètre « liaison organique » !\par
Et je vais vous expliquer, chers amis, pourquoi ce malheur vous est arrivé : parce que vous apprenez par cœur et vous retenez les mots d’ordre de la révolution beaucoup plus que vous n’essayez de les comprendre. C'est pourquoi vous mettez entre guillemets les mots « défense de la patrie socialiste », en essayant probablement d’être ironiques, mais en montrant en fait combien vos idées sont brouillées. Vous êtes habitués à considérer la défense nationale comme quelque chose d’infâme et d’ignoble, vous l’avez appris par cœur et vous l’avez retenu, vous l’avez seriné avec tant de zèle que plusieurs d’entre vous en sont arrivés à cette affirmation absurde qu’à l’époque impérialiste la défense de la patrie serait inadmissible (en fait, elle n’est inadmissible que dans une guerre impérialiste, réactionnaire, menée par la bourgeoisie). Mais vous n’avez pas essayé de comprendre pourquoi et quand la « défense de la patrie » est une infamie.\par
Reconnaître la défense de la patrie, c’est reconnaître qu’une guerre est juste et légitime. Juste et légitime à quel point de vue ? Uniquement du point de vue du prolétariat socialiste et de sa lutte pour l’émancipation ; nous n’admettons pas d’autre point de vue. Si c’est la classe des exploiteurs qui fait la guerre pour renforcer sa domination de classe, il s’agit d’une guerre criminelle et la « défense de la patrie » dans \emph{cette} guerre est une infamie et une trahison envers le socialisme. Si c’est le prolétariat qui, après avoir triomphé de la bourgeoisie dans son propre pays, fait la guerre pour consolider et développer le socialisme, il s’agit d’une guerre légitime et « sacrée ».\par
Nous sommes partisans de la défense de la patrie depuis le 25 octobre 1917. Je l’ai dit plus d’une fois avec la plus grande netteté, et vous n’osez pas le contester. C'est précisément pour « renforcer la liaison » avec le socialisme international, qu’il est de notre devoir de défendre la pa­trie \emph{socialiste}. Celui‑là compromettrait la liaison avec le socialisme international qui traiterait avec légèreté la dé­fense du pays où le prolétariat a déjà triomphé. Quand nous étions des représentants de la classe opprimée, nous ne traitions pas avec légèreté la défense de la patrie dans la guerre impérialiste, nous en étions les adversaires de principe. Devenus les représentants de la classe dominante qui a commencé à organiser le socialisme, nous exigeons de tous une attitude \emph{sérieuse} envers la défense du pays. Et cette attitude sérieuse consiste à se préparer activement à la défense du pays et à tenir rigoureusement compte du rapport des forces. S'il est évident que nos forces sont insuffisantes, la \emph{retraite au cœur du pays} est le principal moyen de défense (celui qui voudrait ne voir là qu’une formule de circonstance, forgée pour les besoins de la cause, peut lire chez le vieux \textbf{Clausewitz}, l’un des grands écrivains militaires, le bilan des enseignements de l’histoire qu’il dégage à ce propos). Or, chez les « communistes de gauche », rien ne nous permet de penser qu’ils comprennent l’importance du problème du rapport des forces.\par
Du temps où nous étions les adversaires de principe de la défense de la patrie, nous avions le droit de tourner en ridicule ceux qui voulaient « sauvegarder » leur patrie dans l’intérêt, prétendaient‑ils, du socialisme. Maintenant que nous avons le droit d’être des partisans prolétariens de la défense de la patrie, le problème se pose d’une tout autre façon. Notre devoir devient de mesurer avec la plus grande prudence nos forces, d’examiner minutieusement les possi­bilités de recevoir à temps du renfort de notre allié (le pro­létariat international). L'intérêt du capital est de battre son ennemi (le prolétariat révolutionnaire) par parties, tant que les ouvriers de tous les pays ne se sont pas encore unis (dans l’action, c’est‑à‑dire en commençant la révolu­tion). Notre intérêt à nous est de faire tout notre possible, d’utiliser toutes les chances, aussi minimes soient‑elles, pour différer la bataille décisive jusqu’au moment (ou « \emph{jusqu’après} » le moment) où se produira cette fusion des détachements révolutionnaires au sein de la grande et indivisible armée internationale.\par
Passons maintenant aux déboires de nos « communistes de gauche » en matière de politique intérieure. Il est difficile de ne pas sourire en lisant dans les thèses sur la situation actuelle des phrases comme celle‑ci :\par

\begin{quoteblock}
 \noindent « […] L'utilisation méthodique des moyens de production restés intacts ne se conçoit que dans le cadre de la socialisation la plus résolue » « […] Non pas la capitulation devant la bourgeoisie et ses suppôts intellectuels petits‑bourgeois, mais l’écrasement complet de la bourgeoisie et une action tendant à briser définitivement le sabotage […] »
\end{quoteblock}

\noindent Chers « communistes de gauche », quelle surabondance de résolution […] et quelle insuffisance de réflexion ! Qui, veut dire cette « socialisation la plus résolue » ?\par
On peut être résolu ou irrésolu en matière de nationalisation et de confiscation. Mais aucune \hspace{1em}« résolution », fût-elle la plus grande qui soit, ne suffit pour assurer le passage de la nationalisation et des confiscations à la socialisation. Toute la question est là, précisément. Le malheur de nos \hspace{1em}« communistes de gauche», c’est que par ce naïf et puéril assemblage de mots : « La socialisation […] la plus résolue », ils révèlent leur incompréhension totale du nœud de la question et de la situation \hspace{1em}« actuelle ». Les déboires des « communistes de gauche » viennent précisément de ce qu’ils ne voient pas le trait essentiel de la « situation actuelle », du passage des confiscations (pour lesquelles un homme politique doit surtout faire preuve de résolution) à la socialisation (qui exige des révolutionnaires d’\emph{autres} qualités).\par
Hier, il fallait essentiellement nationaliser, confisquer, battre et achever la bourgeoisie et briser le sabotage avec le maximum de résolution. Aujourd’hui, il n’est que des aveugles pour ne pas voir que nous avons nationalisé, confisqué, brisé et démoli plus \emph{que nous n’avons réussi à compter}. Or, la socialisation diffère de la simple confiscation précisément en ceci qu’on peut confisquer avec la seule « résolution », sans être compétent en matière de recensement et de répartition rationnelle de ce qui a été confisqué, \emph{tandis qu’on ne peut socialiser à défaut de cette compétence.}\par
Notre mérite historique est d’avoir été hier (comme nous le serons demain) résolus dans le domaine des confiscations, de l’écrasement de la bourgeoisie, de la répression du sabotage. En parler aujourd’hui dans des « thèses sur la situation actuelle », c’est se tourner vers le passé et ne pas comprendre le passage vers l’avenir.\par
« […] Briser définitivement le sabotage[…] » La belle tâche que voilà ! Mais les saboteurs sont déjà bien assez « brisés » chez nous. C'est tout autre chose qui nous manque : nous ne savons pas \emph{calculer} où il faut mettre tel ou tel saboteur, nous ne savons pas organiser nos propres forces pour la surveillance, charger un directeur ou un contrôleur bolchévique de surveiller, disons, une centaine de saboteurs qui viennent travailler chez nous. Dans cette situation, lancer des phrases telles que « la socialisation la plus résolue », « l’écrasement », « briser définitivement », c’est se fourrer le doigt dans l’œil. Il est typique, pour un révolutionnaire petit-­bourgeois, de ne pas remarquer qu’il ne suffit pas au socialisme d’achever, de briser, etc. ; cela suffit au petit propriétaire exaspéré contre le grand, mais le révolutionnaire prolétarien ne saurait tomber dans une pareille erreur.\par
Si les paroles que nous venons de citer appellent un sou­rire, c’est un rire homérique que provoque la découverte faite par les « communistes de gauche » et selon laquelle si « la déviation bolchévique de droite » l’emportait, la République des Soviets risquerait d’ « évoluer vers un capita­lisme d’État ».\par
Voilà qui semble bien fait pour nous rem­plir d’effroi ! Et quel zèle nos « communistes de gauche » ne mettent‑ils pas à répéter dans leurs thèses et dans leurs articles cette redoutable découverte…\par
Or, ils n’ont pas songé que le capitalisme d’État serait \emph{un pas en avant} par rapport à l’état actuel des choses dans notre République des Soviets. Si, dans six mois, par exemple nous avions instauré chez nous le capitalisme d’État, ce serait un immense succès et la plus sûre garantie qu’un an plus tard, dans notre pays, le socialisme serait assis et invincible.\par
Je vois d’ici avec quelle noble indignation le « commu­niste de gauche » va repousser cette affirmation et à quelle « critique destructrice » de la « déviation bolchévique de droite » il va se livrer devant les ouvriers. Comment ? Dans la République \emph{socialiste} des Soviets, le passage au \emph{capitalisme d’État} serait un pas en avant ?… N'est‑ce pas trahir le socialisme ?\par
C’est là précisément qu’est l’erreur \emph{économique} des « communistes de gauche ». C'est donc sur ce point qu’il faut nous arrêter plus longuement.\par
\textbf{Premièrement}, les « communistes de gauche » n’ont pas compris quel est exactement le caractère de la \emph{transition} du capitalisme au socialisme qui nous donne le droit et toutes les raisons de nous appeler République socialiste des Soviets.\par
\textbf{Deuxièmement}, ils révèlent leur nature petite‑bourgeoise du fait, justement, qu’ils \emph{ne voient pas} dans l’élément petit-bourgeois l’ennemi \emph{principal} auquel se heurte chez nous le socialisme.\par
\textbf{Troisièmement}, en agitant l’épouvantail du « capitalisme d’État », ils montrent qu’ils ne comprennent pas ce qui, au point de vue économique, distingue l’État soviétique de l’État bourgeois.\par
Examinons ces trois points.\par
Parmi les gens qui se sont intéressés à l’économie de la Russie, personne, semble‑t‑il, n’a nié le caractère transitoire de cette économie. Aucun communiste non plus n’a nié, semble‑t‑il, que l’expression de République socialiste des Soviets traduit la volonté du pouvoir des Soviets d’assurer la transition au socialisme, mais n’entend nullement signifier que le nouvel ordre économique soit socialiste.\par
Mais que veut dire le mot transition ? Ne signifie‑t‑il pas, appliqué à l’économie, qu’il y a dans le régime en question des éléments, des fragments, des parcelles, \emph{à la fois} de capitalisme et de socialisme ? Tout le monde en conviendra. Mais ceux qui en conviennent ne se demandent pas toujours quels sont précisément les éléments qui relèvent, de différents types économiques et sociaux qui coexistent en Russie. Or, là est toute la question.\par
Énumérons ces éléments :\par

\begin{itemize}[itemsep=0pt,]
\item l’économie patriarcale, c’est‑à‑dire, en grande mesure, l’économie naturelle, paysanne ;
\item la petite production marchande (cette rubrique comprend la plupart des paysans qui vendent du blé) ;
\item le capitalisme privé ;
\item le capitalisme d’État ;
\item le socialisme.
\end{itemize}
\noindent La Russie est si grande et d’une telle diversité que toutes ces formes économiques et sociales s’y enchevêtrent étroitement. Et c’est ce qu’il y a de particulier dans no­tre situation.\par
Quels sont donc les types qui prédominent ? Il est évident que, dans un pays de petits paysans, c’est l’élément petit‑bourgeois qui domine et ne peut manquer de dominer ; la majorité, l’immense majorité des agriculteurs sont de petits producteurs. L'enveloppe du capitalisme d’État (monopole du blé, contrôle exercé sur les propriétaires d’usines et des commerçants, coopératives bourgeoises) est déchirée çà et là par les \emph{spéculateurs}, le \emph{blé} étant l’objet principal de la spéculation.\par
C'est dans ce domaine précisément que se déroule la lutte principale. Quels sont les adversaires qui s’affrontent dans cette lutte, si nous parlons par catégories économiques, comme le « capitalisme d’État » ? Sont‑ce le quatrième et le cinquième élément de ceux que je viens d’énumérer ? Non, bien sûr. Ce n’est pas le capitalisme d’État qui est ici aux prises avec le socialisme, mais la petite bourgeoisie et le capitalisme privé qui luttent, au coude à coude, à la fois contre le capitalisme d’État et contre le socialisme. La petite bourgeoisie s’oppose à \emph{toute} intervention de la part de l’État, à tout inventaire, à tout contrôle, qu’il émane d’un capitalisme d’État ou d’un socialisme d’État. C’est là un fait réel, tout à fait indéniable, dont l’incompréhension est à la base de l’erreur économique des « communistes de gauche ». Le spéculateur, le mercanti, le saboteur du monopole, voilà notre pire ennemi « intérieur », l’ennemi des mesures économiques du pouvoir des Soviets. Si, il y a 125 ans, les petits bourgeois français, révolutionnaires des plus ardents et des plus sincères, étaient encore excusables de vouloir vaincre la spéculation en envoyant à l’échafaud un petit nombre d’ « élus » et en usant de foudres déclamatoires, aujourd’hui, les attitudes de phraseurs avec lesquelles tel ou tel socialiste‑révolutionnaire de gauche aborde cette question n’inspirent qu’aversion et dégoût à tous les révolutionnaires conscients. Nous savons parfaitement que la base économique de la spéculation est constituée par la couche des petits propriétaires si largement répandus en Russie et par le capitalisme privé dont \emph{chaque} petit bourgeois est un agent. Nous savons que des millions de tentacules de cette hydre petite‑bourgeoise pénètrent çà et là dans certaines couches de la classe ouvrière et que la spéculation s’introduit dans tous les pores de notre vie économique et sociale, \emph{l’emportant sur le monopole d’État.}\par
Quiconque ne le voit pas montre par son aveuglement à quel point il est prisonnier des préjugés petits‑bourgeois. Tels sont précisément nos « communistes de gauche » qui, en paroles (et dans leur conviction la plus sincère, naturellement), sont des ennemis acharnés de la petite bourgeoisie, mais qui, en réalité, ne font que l’aider, que la servir, que traduire son point de vue à elle, en combattant ‑ en avril 1918 !! ‑ Le… « Capitalisme d’État » ! C'est ce qu’on appelle se fourrer le doigt dans l’œil !\par
Le petit bourgeois a mis de côté une somme coquette, quelques milliers de roubles amassés pendant la guerre par des moyens « licites » (et le plus souvent illicites). Tel est le type économique caractéristique qui forme la base de la spéculation et du capitalisme privé. L'argent est un bon tiré sur la richesse sociale, et des millions de petits propriétaires, cramponnés à ce bon, le cachent à \hspace{1em}« l’État », ne croyant à aucun socialisme, à aucun communisme, et « attendant patiemment » que l’orage prolétarien ait passé. Ou bien nous soumettrons ce petit bourgeois à \emph{notre} contrôle et à \emph{notre} enregistrement (nous pourrons le faire si nous organisons les pauvres, c’est‑à‑dire la majorité de la population ou les semi‑prolétaires, autour de l’avant‑garde prolétarienne consciente), ou bien il culbutera notre pouvoir ouvrier, infailliblement et inéluctablement, comme l’ont fait les Napoléon et les Cavaignac nés précisément sur ce terrain de la petite propriété. Voilà comment se pose la question. Seuls les socialistes‑révolutionnaires de gauche n’aperçoivent, pas, derrière les phrases sur la paysannerie « laborieuse », cette simple et claire vérité. Mais en est‑il qui prennent au sérieux les socialistes‑révolutionnaires de gauche noyés dans la phraséologie ?\par
Le petit bourgeois cramponné à ses billets de mille est l’ennemi du capitalisme d’État ; et ces billets de mille, il entend les réaliser absolument à son propre profit, contre les pauvres, contre tout contrôle général de l’État. Or, le total de ces billets de mille donne une base de plusieurs milliards à la spéculation qui sape notre effort d’édification socialiste. Admettons qu’un certain nombre d’ouvriers produisent un total de biens chiffré à 1 000. Admettons ensuite que 200 unités sur ce total se perdent du fait de la petite spéculation, des vols de tout genre et des procédés des petits‑propriétaires visant à « tourner » les décrets et les réglementations soviétiques. Tout ouvrier conscient dira : si je pouvais donner 300 sur 1000 pour assurer plus d’organisation et plus d’ordre, je les donnerais volontiers au lieu de ces 200, car il nous sera très facile, sous le pouvoir des Soviets, d’abaisser par la suite ce « tribut», par exemple à 100 ou à 50, du fait que l’ordre et l’organisation auront été établis, du fait qu’auront été définitivement brisés les efforts des petits propriétaires pour mettre en échec tout monopole d’État.\par
Cet exemple chiffré tout à fait élémentaire – que j’ai simplifié exprès au maximum dans un but de vulgarisation – explique le \emph{rapport} qui existe à l’heure actuelle entre le capitalisme d’État et le socialisme. Les ouvriers détiennent le pouvoir dans l’État ; ils ont l’entière possibilité juridique de \hspace{1em}« prendre » tout le millier de roubles, c’est‑à‑dire de n’autoriser aucune dépense qui n’ait une destination socialiste. Cette possibilité juridique, s’appuyant sur le passage effectif du pouvoir aux ouvriers, voilà qui va dans le sens du socialisme.\par
Mais la petite propriété et le capitalisme privé sabotent de mille façons cette situation juridique, introduisent par la bande la spéculation, entravent l’application des décrets soviétiques. Le capitalisme d’État serait un immense pas en avant \emph{même si} (et c’est à dessein que j’ai choisi cet exemple chiffré pour renforcer ma démonstration) nous le payions \emph{plus cher} qu’à présent, car cela vaut la peine de payer pour « s’instruire », car c’est utile aux ouvriers, car la victoire sur le désordre, la désorganisation, l’incurie est plus importante que tout, car la continuation de l’anarchie inhérente à la petite propriété est le pire, le plus grave des dangers, celui qui (si nous n’en venons pas à bout) conduira \emph{certainement} à la faillite, tandis que si nous payons un tribut plus élevé au capitalisme D’état, cela ne nous nuira en rien, mais servira au contraire à nous conduire au socialisme par le chemin le plus sûr. Quand la classe ouvrière aura appris à défendre l’ordre d’État contre l’esprit anarchique de la petite propriété, quand elle aura appris à organiser la grande production à l’échelle de l’État, sur les bases du capitalisme d’État, elle aura alors, passez‑moi l’expression, tous les atouts en mains et la consolidation du socialisme sera assurée.\par
Le capitalisme d’État est, au point de vue \emph{économique}, infiniment supérieur à notre économie actuelle. C'est là un premier point.\par
Ensuite, il ne contient rien que le pouvoir des Soviets doive redouter, car l’État soviétique est un État dans lequel le pouvoir des ouvriers et des pauvres est assuré. Les « communistes de gauche » n’ont pas compris ces vérités incontestables, que ne comprendra naturellement jamais aucun \hspace{1em}« socialiste‑révolutionnaire de gauche », incapable en général de lier dans sa tête deux idées quelconques sur l’économie politique, mais que tout marxiste sera \emph{obligé} de reconnaître. Il est inutile de discuter avec un socialiste‑révolutionnaire de gauche, il suffit de le désigner du doigt comme un \hspace{1em}« exemple à ne pas suivre » de bavard futile ; mais il \emph{faut} discuter avec un « communiste de gauche », car dans ce cas l’erreur est commise par des marxistes, et l’analyse de leur erreur aidera la \emph{classe ouvrière} à trouver bon chemin.\par
Pour éclaircir encore plus la question, donnons avant tout un exemple très concret de capitalisme d’État. Tout le monde sait quel est cet exemple : l’Allemagne. Nous trouvons dans ce pays le « dernier mot » de la technique moderne du grand capitalisme et de l’organisation méthodique au \emph{service de l’impérialisme des bourgeois et des junkers. Suppri}mez les mots soulignés, remplacez \emph{l’État} militaire, l’État des junkers, l’État bourgeois et impérialiste, par un \emph{autre État}, mais un État de type social différent, ayant un autre contenu de classe, par l’État \emph{soviétique}, c’est‑à‑dire prolé­tarien, et vous obtiendrez \emph{tout} l’ensemble de conditions qui donne le socialisme.\par
Le socialisme est impossible sans la technique du grand capitalisme, conçue d’après le dernier mot de la science la plus moderne, sans une organisation d’État méthodique qui ordonne des dizaines de millions d’hommes à l’observa­tion la plus rigoureuse d’une norme unique dans la produc­tion et la répartition des produits. Nous, les marxistes, nous l’avons toujours affirmé ; quant aux gens qui ont été incapables de comprendre \emph{au moins} cela (les anarchistes et une bonne moitié des socialistes‑révolutionnaires de gau­che), il est inutile de perdre même deux secondes à discuter avec eux.\par
Le socialisme est également impossible sans que le prolétariat domine dans l’État : cela aussi, c’est de l’a b c. Et l’histoire (dont personne, sauf peut‑être des benêts menchéviques de première grandeur, n’attendait qu’elle produisit sans heurt, dans le calme, facilement et simplement le socialisme \hspace{1em}« intégral ») a suivi des chemins si particuliers qu’elle a \emph{donné naissance}, en 1918, à deux moitiés de socialisme, séparées et voisines comme deux futurs poussins sous la coquille commune de l’impérialisme international. L'Allemagne et la Russie incarnent en 1918, avec une évidence particulière, la réalisation matérielle des conditions du socialisme, des conditions économiques, productives et sociales, d’une part, et des conditions politiques, d’autre part.\par
Une révolution prolétarienne victorieuse en Allemagne briserait d’emblée, avec la plus grande facilité, toutes les coquilles de l’impérialisme (faites, malheureusement, de l’acier le meilleur, et que ne peuvent de ce fait briser les efforts de \emph{n’importe quel…} poussin) et assurerait à coup sûr la victoire du socialisme mondial, sans difficulté ou avec des difficultés insignifiantes, à condition évidemment de considérer les « difficultés » à l’échelle de l’histoire mondiale, et non à celle de quelque groupe de philistins.\par
Tant que la révolution tarde encore à « éclore » en Allemagne, notre devoir est de nous mettre à l’école du capitalisme d’État des Allemands, de nous appliquer de \emph{tous nos forces} à l’assimiler, de ne pas ménager les procédés \emph{dictatoriaux} pour l’implanter en Russie encore plus vite que ne l’a fait Pierre I° pour les mœurs occidentales dans la vieille Russie barbare, sans reculer devant l’emploi de méthodes barbares contre la barbarie. S'il se trouve, parmi les anarchistes et les socialistes‑révolutionnaires de gauche (je me suis, sans le vouloir, souvenu des discours prononcés par Karéline et Gué au Comité exécutif), des gens capables de tenir des raisonnements à la Narcisse comme quoi il ne serait pas digne de nous autres, révolutionnaires, de « nous mettre à l’école » de l’impérialisme allemand, il faut leur dire ceci : une révolution qui prend ces gens au sérieux serait perdue sans rémission (et l’aurait bien mérité).\par
Ce qui prédomine actuellement en Russie, c’est le capitalisme petit‑bourgeois, à partir duquel il n’est qu’\emph{un seul et même chemin} pour parvenir \emph{aussi bien} au grand capitalisme d’État \emph{qu’au} socialisme, et ce chemin passe par \emph{la même} étape intermédiaire qui s’appelle « inventaire et contrôle exercés par le peuple entier sur la production et la répartition des produits ». Quiconque ne le comprend pas commet une erreur économique impardonnable, soit en ignorant les faits réels, en ne voyant pas ce qui est, en ne sachant pas regarder la vérité en face, soit en se contentant d’opposer dans l’abstrait le \hspace{1em}« capitalisme » au « socialisme », sans analyser les formes et les étapes concrètes de cette transition telle qu’elle s’effectue chez nous à l’heure actuelle. Soit dit entre parenthèses, c’est cette même erreur théorique qui a désorienté les meilleurs représentants du camp de la \emph{Novaïa Jizn}\footnote{Novaïa Jizn [Vie Nouvelle] : revue menchévique martovienne. Interdite en juillet 1918.} et du \emph{Vpériod}\footnote{Vpériod [En avant] : Quotidien menchévique interdit en avril 1918.}: les pires d’entre eux et les médiocres terrorisés par la bourgeoisie, se traînent à sa remorque par stupidité ou par manque de caractère; quant aux meilleurs, ils n’ont pas compris que ce n’est pas pour rien que les maîtres du socialisme ont parlé de toute une période de transition du capitalisme au socialisme et que ce n’est pas sans raison qu’ils ont insisté sur les « longues douleur de l’enfantement » de la société nouvelle, celle‑ci n’étant du reste elle-même qu’une abstraction et ne pouvant s’incarner dans la vie qu’à travers toute une série de tentatives concrètes diverses et imparfaites visant à créer tel ou tel État socialiste.\par
C'est justement parce qu’il est impossible, en partant de la situation économique actuelle de la Russie, de progresser sans passer par \emph{ce qu’il y a de commun} au capitalisme d’État et au socialisme (l’inventaire et le contrôle exercés par la nation), qu’il est complètement absurde au point de vue théorique de vouloir terroriser tout le monde et soi‑même en invoquant l’ « évolution \emph{vers} le capitalisme d’État » (\emph{Kommounist}, n° 1, p. 8, I° colonne). C'est, très précisément, laisser sa pensée « s’écarter » du chemin véritable que suit l’« évolution », c’est ne pas comprendre ce chemin. Dans la pratique, cela revient à tirer en arrière vers le capitalisme basé sur la petite propriété.\par
Pour que le lecteur puisse se convaincre que ma « haute » appréciation du capitalisme d’État ne date nullement d’aujourd’hui et que je la professais dès \emph{avant} la prise du pouvoir par les bolchéviks, je me permets de citer ici un passage de ma brochure sur la \emph{Catastrophe imminente et les moyens de la conjurer}, écrite en septembre 1917 :\par

\begin{quoteblock}
 \noindent « […] Eh bien, essayez un peu de substituer à l’État des capitalistes et des hobereaux, à l’État des capitalistes et des grands propriétaires fonciers, l’État démocratique révolutionnaire, c’est‑à‑dire un État qui détruise révolutionnairement tous les privilèges quels qu’ils soient, qui ne craigne pas d’appliquer révolutionnairement le démocratisme le plus complet. Et vous verrez que dans un État véritablement démocratique révolutionnaire, le capitalisme monopoliste d’État signifie inévitablement, infailliblement, un pas, ou des pas en avant vers le socialisme !\par
 […] Car le socialisme n’est pas autre chose que l’étape immédiatement consécutive au monopole capitaliste d’État.\par
 […] Le capitalisme monopoliste d’État est la préparation matérielle la plus complète du socialisme, l’antichambre du socialisme, l’étape de l’Histoire qu’aucune autre étape intermédiaire ne sépare du socialisme » (pp. 27 et 28).
\end{quoteblock}

\noindent Remarquez que ces lignes ont été écrites à l’époque de Kérensky, qu’il ne s’agissait là \emph{ni} de la dictature du prolétariat \emph{ni} d’un État socialiste, mais d’un État « démocratique révolutionnaire ». N'est‑il pas évident que \emph{plus nous nous élevons} au‑dessus de ce degré politique, plus \emph{nous réussissons complètement à} incarner dans les Soviets l’État socialiste et la dictature du prolétariat, et \emph{moins il} nous est permis de redouter le « capitalisme d’État » ? N'est‑il pas évident qu’au point de vue \emph{matériel}, économique, au point de vue de la production, nous n’en sommes pas encore à l’« antichambre » du socialisme ? Et qu’on ne saurait franchir le seuil du socialisme sans passer par cette « antichambre» que nous n’avons pas encore atteinte ?\par
De quelque côté qu’on aborde la question, une seule conclusion s’impose : le raisonnement des \hspace{1em}« communistes de gauche » au sujet de la menace que ferait peser sur nous le « capitalisme d’État » n’est qu’une erreur économique et la preuve manifeste qu’ils sont complètement prisonniers de l’idéologie petite‑bourgeoise.\par
Voici un autre fait extrêmement édifiant.\par
\bigbreak
\noindent Lors de notre discussion au Comité exécutif central avec le camarade Boukharine, celui‑ci a fait notamment la remarque suivante: dans la question des traitements élevés des spécialistes, « nous » (nous, « communistes de gauche », bien entendu) sommes «plus à droite» que Lénine, car nous ne voyons ici aucune atteinte aux principes, nous souvenant de ce qu’avait dit Marx, à savoir que, dans certaines conditions, le plus rationnel serait que la classe ouvrière « se rédime de cette bande » (de la bande des capitalistes, s’entend, c’est‑à‑dire qu’elle \emph{rachète} à la bourgeoisie la terre, les fabriques, les usines et autres moyens de production).\par
Cette remarque extrêmement intéressante montre, tout d’abord, que Boukharine dépasse de deux têtes les socialistes‑révolutionnaires de gauche et les anarchistes, qu’il ne s’est pas irrémédiablement embourbé dans des phrases, mais qu’il s’efforce au contraire de réfléchir aux difficultés \emph{concrètes} de la transition ‑ transition douloureuse et difficile ‑ qui mène du capitalisme au socialisme.\par
D'autre part, cette remarque rend encore plus évidente l’erreur de Boukharine.\par
En effet. Réfléchissez à ce qu’a dit Marx.\par
\bigbreak
\noindent Il s’agissait de l’Angleterre des années 70, à l’apogée capitalisme pré-monopoliste, du pays qui était alors le moins militarisé et le moins bureaucratique, du pays qui offrait à cette époque le plus de possibilités quant à la victoire « pacifique » du socialisme sous la forme d’un « rachat » de la bourgeoisie par les ouvriers. Et Marx disait que, dans certaines conditions, les ouvriers ne s’interdiraient nullement de se racheter de la bourgeoisie. Marx ne se liait pas les mains et n’entravait en rien les futurs artisans de la révolution socialiste quant aux formes, procédés ou méthodes de la révolution, comprenant parfaitement que de nouveaux et nombreux problèmes surgiraient à cette époque, que la situation changerait complètement au cours de la révolution, et qu’elle continuerait à se modifier \emph{souvent} et \emph{considérablement} au fur et à mesure des progrès de la révolution.\par
Or, n’est‑il pas évident que dans la Russie des Soviets, \emph{après} la prise du pouvoir par le prolétariat, \emph{après} la répression de la résistance armée et du sabotage des exploiteurs, nous voyons réalisées \emph{certaines} conditions du genre de celles qui auraient pu se réaliser il y a un demi‑siècle en Angleterre si ce pays s’était mis à évoluer pacifiquement vers le socialisme ? Voici les facteurs qui pouvaient, à cette époque, assurer en Angleterre la soumission des capitalistes aux ouvriers :\par

\begin{itemize}[itemsep=0pt,]
\item une population dont la majorité absolue était composée d’ouvriers, de prolétaires, la paysannerie étant inexistante (certains indices permettaient d’espérer, vers les années 70, que le socialisme ferait en Angleterre des progrès extrêmement rapides parmi les ouvriers agricoles) ;
\item une excellente organisation du prolétariat dans les syndicats (l’Angleterre était alors le premier pays du monde sous ce rapport) ;
\item le niveau culturel relativement élevé du prolétariat, éduqué par un siècle de libertés politiques ;
\item la vieille habitude des capitalistes anglais, admirablement organisés ‑ ils étaient alors les capitalistes les mieux organisés du monde (cette priorité appartient maintenant à l’Allemagne), ‑ de régler les questions politiques et économiques par des compromis.
\end{itemize}
\bigbreak
\noindent Tels étaient les facteurs qui avaient pu faire penser à la possibilité d’une soumission \emph{pacifique} des capitalistes d’Angleterre aux ouvriers de ce pays.\par
Chez nous, cette soumission est actuellement assurée par certaines prémisses fondamentales (la victoire d’Octobre et la répression, entre octobre et février, de la résistance armée et du sabotage des capitalistes). Chez nous, \emph{au lieu} de la majorité absolue des ouvriers, des prolétaires, parmi la population, et de leur haut degré d’organisation, le facteur de victoire a été le soutien accordé aux prolétaires par la paysannerie pauvre qui est en train de se ruiner rapidement. Chez nous, enfin, on ne trouve ni un niveau culturel élevé, ni l’habitude de faire des compromis. Si l’on réfléchit à ces conditions concrètes, il devient évident que nous pouvons et devons arriver maintenant à \emph{conjuguer} les méthodes de répression implacable\footnote{Il s’agit, là aussi, de regarder la vérité bien en face : nous n’avons pas encore suffisamment de cette implacabilité qui est indispensable au succès du socialisme. Non pas que nous ne soyons pas assez résolus ; ce n’est pas la résolution qui nous manque. Mais nous ne savons pas attraper assez vite un nombre assez important de spéculateurs, de maraudeurs, de capitalistes qui enfreignent les prescriptions soviétiques. Car on n’acquiert ce « savoir‑faire » que par l’organisation de l’enregistrement et du contrôle ! En second lieu, nous manquons de fermeté dans nos tribunaux, qui, au lieu de faire fusiller les concussionnaires, leur infligent six mois de prison. Ces deux défauts ont la même racine sociale : l’influence de l’élément petit‑bourgeois, sa débilité. (\emph{Note de l’auteur})} à l’égard des capitalistes sans culture, qui n’acceptent aucun « capitalisme d’État », ne songent à aucun compromis et continuent à contrecarrer les mesures prises par l’État soviétique au moyen de la spéculation, en corrompant les pauvres, etc., avec les \emph{procédés de compromis} ou de rachat à l’égard des capitalistes cultivés qui acceptent le « capitalisme d’État», sont capables de l’appliquer, se montrent utiles au prolétariat en qualité d’organisateurs intelligents et expérimentés des entreprises \emph{les plus importantes}, susceptibles d’assurer l’approvisionnement effectif de dizaines de millions d’hommes.\par
Boukharine est un économiste marxiste d’une excellente culture. Aussi s’est‑il rappelé que Marx avait profondément raison quand il enseignait aux ouvriers combien il est important de conserver l’organisation de la grande production, précisément pour faciliter la transition au socialisme ; quand il enseignait qu’on peut parfaitement envi­sager \emph{de bien payer les capitalistes}, de les « racheter » \emph{si} (par exception, l’Angleterre étant alors cette exception) les circonstances obligent les capitalistes à se soumettre paci­fiquement et à passer au socialisme d’une façon civilisée et organisée moyennant ce \hspace{1em}« rachat ».\par
Si \textbf{Boukharine} s’est trompé, c’est parce qu’il n’a pas réfléchi aux particularités concrètes de la situation actuelle en Russie, situation exceptionnelle, précisément, du fait que nous nous trouvons, nous, prolétariat de Russie, \emph{en avance} sur n’importe quelle Angleterre et n’importe quelle Allemagne par notre régime politique, par la force du pouvoir politique des ouvriers, et en même temps \emph{en retard} par rapport au pays le plus arriéré d’Europe occidentale en ce qui concerne l’organisation d’un capita­lisme d’État digne de ce nom, en ce qui concerne notre niveau culturel et le degré de notre préparation à l’ « instau­ration » du socialisme dans le domaine de la production matérielle. N'est‑il pas évident que de cette situation par­ticulière découle, en ce moment, la nécessité d’une sorte de « rachat » que les ouvriers doivent offrir aux capitalistes les plus cultivés, les plus doués, les plus capables en matière d’organisation, disposés à servir le pouvoir des Soviets en l’aidant honnêtement à organiser la grande et la très grande production « d’État » ? N'est‑il pas évident que, dans cette situation particulière, nous devons nous efforcer d’éviter deux sortes d’erreurs relevant, chacune à sa façon, de l’esprit petit‑bourgeois ? D'une part, nous commettrions une faute irréparable en déclarant que la disproportion entre nos « forces » économiques et notre force politique étant un fait avéré, il en « découle » qu’ il ne fallait pas prendre le pouvoir. C'est là un raisonnement de « maniaques vivant dans du coton », qui oublient qu’il n’y aura jamais de « proportion », qu’il ne saurait y en avoir ni dans le déve­loppement de la nature ni dans celui de la société, que le socialisme achevé ne saurait résulter que de la collaboration révolutionnaire des prolétaires de \emph{tous} les pays et à la suite de nombreuses tentatives dont chacune, considérée isolément, sera unilatérale et souffrira d’une certaine dispropor­tion.\par
D'autre part, il serait profondément erroné de laisser faire les braillards et les phraseurs, qui se laissent séduire par les « belles » attitudes révolutionnaires, mais sont incapables d’un travail révolutionnaire persévérant, réfléchi, mûrement pesé, tenant compte des transitions les plus difficiles.\par
Heureusement, l’histoire du développement des partis révolutionnaires et de la lutte que leur a livrée le bolchévisme nous a légué des types nettement dessinés, parmi lesquels les socialistes‑révolutionnaires de gauche et les anarchistes offrent le type de piètres révolutionnaires. Ils poussent maintenant de grands cris, à perdre haleine, jusqu’à piquer des crises d’hystérie, contre \hspace{1em}l’« esprit conciliateur » des « bolcheviks de droite ». Mais ils sont incapables de comprendre \emph{en quoi} cet « esprit conciliateur » était mauvais et \emph{pourquoi} il a été avec juste raison condamné par l’histoire et par tout le cours de la révolution.\par
L'esprit conciliateur du temps de Kérensky livrait le pouvoir à la bourgeoisie impérialiste ; or, la question du pouvoir est la question capitale de toute révolution. En octobre‑novembre 1917, la tendance conciliatrice d’une partie des bolcheviks les amenait, soit à redouter la prise du pouvoir par\par
le prolétariat, soit à vouloir \emph{partager} le pouvoir sur une base paritaire, non seulement avec des \hspace{1em}« compagnons de route peu sûrs », tels que les socialistes-révolutionnaires de gauche, mais même avec des ennemis, tels que les partisans de Tchernov et les menchéviks, qui nous auraient certainement gênés dans l’essentiel: dans la dissolution de la Constituante, dans l’écrasement impitoyable des Bogaïevski, dans la mise en place généralisée des institutions soviétiques, dans chaque confiscation.\par
Maintenant le pouvoir est conquis, conservé, consolidé entre les mains d’un seul parti, le parti du prolétariat, qui n’a même pas à ses côtés des « compagnons de route peu sûrs ». Parler maintenant d’esprit conciliateur, alors qu’il n’est pas et qu’il ne saurait être question de partager le \emph{pouvoir}, de renoncer à la dictature du prolétariat contre la bourgeoisie, c’est répéter comme un perroquet des mots que l’on a appris par cœur sans les comprendre. Parler d’« esprit conciliateur » à propos du fait que, placés dans une situation où nous pouvons et devons gouverner le pays, nous nous efforçons d’attirer à nous, sans regarder à la dépense, les éléments les plus cultivés parmi ceux que le capitalisme a formés, de les engager à notre service contre la désorganisation inhérente à l’esprit petit‑propriétaire, c’est être totalement incapable de réfléchir aux tâches économiques de l’édification socialiste.\par
Et c’est pourquoi, s’il est tout à l’honneur du camarade Boukharine qu’il se soit tout de suite senti « honteux », au Comité exécutif central, du « service » que lui avaient rendu les Karéline et les Gué, il n’en demeure pas moins que L’existence de tels compagnons de lutte politique garde la valeur d’une sérieuse mise en garde pour la \emph{tendance} des « communistes de gauche ».\par
Prenez le \emph{Znamia Trouda}, organe des socialistes‑révolutionnaires de gauche, qui déclare fièrement dans son numéro du 25 avril 1918 : « En ce qui concerne sa plate‑forme actuelle, notre parti est solidaire d’une autre tendance du bolchévisme (Boukharine, Pokrovski, etc.). » Prenez l’organe menchévique \emph{Vpériod} de la même date ; on y trouve, notamment, la « thèse » suivante du menchévik bien connu Issouv :\par

\begin{quoteblock}
 \noindent « Dépourvue dès le début de tout caractère authentiquement prolétarien, la politique du pouvoir des Soviets s’engage de plus en plus ouvertement, ces derniers temps, dans la voie de la conciliation avec la bourgeoisie et prend un caractère manifestement anti ouvrier. Sous le drapeau de la nationalisation de l’industrie, on poursuit une politique d’implantation des trusts industriels ; sous prétexte de rétablir les forces productives du pays, on tente d’abolir la journée de 8 heures, d’introduire le travail aux pièces et le système Taylor, les listes noires et les passeports d’indésirables. Cette politique menace de ravir au prolétariat ses principales conquêtes dans le domaine économique et d’en faire la victime d’une exploitation illimitée de la part de la bourgeoisie. »
\end{quoteblock}


\labelblock{N'est‑ce pas magnifique ?}

\noindent Les amis de Kerenski qui ont mené avec, lui la guerre impérialiste au nom des traités secrets promettant des annexions aux capitalistes russes, les collègues de \textbf{Tsérételli}\footnote{\emph{Tsérételli} : Dirigeant menchevique.} qui voulait désarmer les ouvriers le 11 juin, les \textbf{Liber}\footnote{\emph{Liber} : Dirigeant du Bund (Parti Socialiste juif).}\textbf{ Dan}\footnote{Dan : Dirigeant menchevique.} qui essayaient de camoufler le pouvoir de la bourgeoisie derrière des phrases ronflantes, ce sont eux qui accusent le pouvoir des Soviets de chercher « une conciliation avec la bourgeoisie », d’« implanter des trusts » (c’est‑à‑dire plus précisément, d’implanter le \hspace{1em}« capitalisme d’État »!), d’introduire le système Taylor.\par
En vérité, les bolchéviks devraient offrir à Issouv une médaille et exposer sa thèse dans chaque club ouvrier et dans chaque syndicat, comme un échantillon des \emph{discours provocateurs de la bourgeoisie.} Aujourd’hui, les ouvriers connaissent bien, ils connaissent partout par expérience les Liber‑Dan, les Tsérételli et les Issouv, et il leur sera archi utile de réfléchir sérieusement aux raisons pour lesquelles \emph{ces valets de la bourgeoisie} les incitent à résister au système Taylor et à l’« implantation des trusts ».\par
Les ouvriers conscients confronteront attentivement, la « thèse » d’Issouv, ami de Messieurs les Liber‑Dan et les Tsérételli, à la thèse suivante des « communistes de gauche » :\par

\begin{quoteblock}
 \noindent «L'introduction de la discipline du travail, liée au rétablissement de la direction des capitalistes dans la production, ne peut augmenter sensiblement la productivité du travail, mais elle diminuera l’initiative de classe, l’activité et le degré d’organisation du prolétariat. Elle menace d’asservir la classe ouvrière, elle suscitera le mécontentement tant des couches arriérées que de l’avant‑garde du prolétariat. Étant donné la haine qui règne dans les milieux prolétariens à l’égard des « capitalistes saboteurs », le Parti communiste devrait, pour appliquer ce système, s’appuyer sur la petite bourgeoisie contre les ouvriers et se perdre ainsi en tant que parti du prolétariat » (Kommounist, n° 1, page 8, 2° colonne).
\end{quoteblock}

\noindent Voilà la preuve éclatante que les « communistes de gauche » sont tombés dans le piège, se sont laissés prendre à la provocation des Issouv et autres Judas capitalistes. Bonne leçon pour les ouvriers qui savent que c’est l’avant-garde du prolétariat qui est pour l’introduction de la discipline du travail, et que c’est la petite bourgeoisie qui se démène le plus pour détruire cette discipline. Des discours tels que la thèse des «communistes de gauche » que nous venons de citer sont une honte insigne et un total abandon du communisme dans la pratique, un ralliement total à la petite bourgeoisie.\par
« Liée au rétablissement de la direction des capitalistes » : c’est avec de tels mots que les \hspace{1em}« communistes de gauche » pensent pouvoir « se défendre ». Leur défense ne vaut rien, parce que la « direction » est accordée aux capitalistes par le pouvoir des Soviets, premièrement, avec des commissaires ouvriers ou des comités ouvriers qui surveillent chaque geste du directeur, qui s’assimilent son expérience de direction et qui ont la possibilité, non seulement de faire appel contre ses décisions, mais de le destituer par le truchement des organes du pouvoir soviétique.\par
\textbf{Deuxièmement}, la « direction » est confiée aux capitalistes afin qu’ils remplissent certaines fonctions exécutives au cours d’un travail dont les conditions sont définies par le pouvoir soviétique, lequel peut également les annuler et les réviser. Troisièmement, le pouvoir soviétique confie la « direction » aux capitalistes non pas en tant que capitalistes, mais en tant que spécialistes‑techniciens ou organisateurs, moyennant des salaires élevés. Et les ouvriers savent parfaitement que 99 \% des organisateurs des grosses et des très grosses entreprises, trusts ou autres établissements, appartiennent à la classe capitaliste, de même que les meilleurs techniciens; mais c’est eux précisément que nous, parti prolétarien, devons embaucher en tant que « dirigeants » du processus de travail et d’organisation de la production, car nous n’avons \emph{personne} d’autre qui connaisse la question pratiquement, par expérience. Car les ouvriers, sortis de cette première enfance où la phrase de « gauche » et le laisser‑aller petit‑bourgeois pouvaient les fourvoyer, s’acheminent vers le socialisme précisément à travers la direction des trusts par les capitalistes, à travers la grande production mécanisée, à travers les entreprises dont le chiffre d’affaires se monte à plusieurs millions par an, et uniquement par la voie de cette production et de ces entreprises. Les ouvriers ne sont pas des petits bourgeois. Ils n’ont pas peur du grand « capitalisme d’État », ils le considèrent comme leur instrument \emph{prolétarien}, dont \emph{leur} pouvoir \emph{soviétique} usera contre le désordre et le gâchis caractérisant la petite propriété.\par
Seuls sont incapables de le comprendre les intellectuels déclassés et par conséquent petits‑bourgeois jusqu’à la moelle, dont le type, au sein du groupe des « communistes de gauche » et dans leur revue, est représenté par Ossinski lorsqu’il écrit :\par

\begin{quoteblock}
 \noindent « […] Toute l’initiative dans l’organisation et la direction de l’entreprise appartiendra aux \hspace{1em}« organisateurs de trusts » : car nous ne voulons pas leur enseigner des choses et en faire de simples collaborateurs, mais nous voulons nous mettre à leur école » (Kommounist, n° 1, p. 14, 2° colonne).
\end{quoteblock}

\noindent Les essais d’ironie contenus dans cette phrase visent mon expression : « Apprendre le socialisme en se mettant à l’école des organisateurs de trusts. »\par
Ossinski trouve cela ridicule. Il veut faire des organisateurs de trusts de « simples collaborateurs». Si cela avait été écrit par un homme de l’âge dont le poète a dit : « Quin­ze printemps ?\footnote{Extrait d’un vers de Pouchkine.} » Il n’y aurait pas lieu de s’éton­ner. Mais il est plutôt étrange d’entendre de pareils propos de la part d’un marxiste qui devrait savoir qu’il est impos­sible de réaliser le socialisme sans utiliser les conquêtes de la technique et de la culture obtenues par le grand capitalis­me. Il ne reste plus là le moindre soupçon de marxisme.\par
Non. Ne sont dignes de s’appeler communistes que ceux qui comprennent qu’on \emph{ne peut pas} créer ou instaurer le socialisme sans \emph{se mettre à l’école} des organisateurs de trusts.\par
Car le socialisme n’est pas une invention ; c’est l’assimilation et l’application, par l’avant‑garde du prolétariat qui a conquis le pouvoir, de ce qui a été créé par les trusts. Nous, parti du prolétariat, nous ne pouvons apprendre \emph{nulle part} l’art d’organiser la grande production à l’instar des trusts, comme les trusts, \emph{‑ nulle part} à moins que nous n’allions le chercher chez les spécialistes les plus qualifiés du capitalisme.\par
\bigbreak
\noindent Nous n’avons rien à leur apprendre, si nous ne nous assignons pas la tâche puérile d’« enseigner » le socialisme à des intellectuels bourgeois : ce qu’il faut, ce n’est pas les instruire, mais les exproprier (ce qui se fait en Russie avec assez de « résolution »), c’est \emph{briser} leur sabotage, c’est les \emph{soumettre} au pouvoir des Soviets en tant que groupe ou couche sociale. Mais, si nous ne sommes pas des communistes d’une mentalité et d’un âge enfantins, nous devons nous instruire à leur école, nous avons des choses à apprendre, car le parti et l’avant‑garde du prolétariat \emph{ne possèdent pas l’expérience} d’un travail indépendant dans l’organisation de vastes entreprises desservant des dizaines de millions d’habitants.\par
Et les meilleurs ouvriers de Russie l’ont compris. Ils se sont mis à l’école des capitalistes‑organisateurs, des ingénieurs‑dirigeants, des techniciens‑spécialistes. Ils ont commencé avec fermeté et prudence, passant graduellement du plus facile au plus difficile. Si, dans la métallurgie et dans les constructions mécaniques, le travail avance lentement, c’est parce que la tâche y est plus difficile. Quant aux ouvriers du textile, du tabac, des cuirs et peaux, ils ne craignent pas le « capitalisme d’État », comme les intellectuels petits‑bourgeois déclassés ; ils ne craignent pas de « se mettre à l’école des organisateurs de trusts ». Dans des administrations dirigeantes centrales telles que la « Direction principale du cuir » ou la « Direction centrale du textile », ces ouvriers siègent à côté des capitalistes \emph{s’instruisent à leur école}, organisent des trusts, organisent le « capitalisme\emph{ d’État} » qui est, sous le pouvoir des Soviets, l’antichambre du socialisme, la condition de la victoire durable du socialisme.\par
\bigbreak
\noindent Ce travail des ouvriers avancés de Russie, parallèle à celui qu’ils accomplissent pour introduire la discipline du travail, a été commencé et se poursuit sans bruit, sans éclat, sans le tapage et le fracas dont certains « communistes de gauche » ne peuvent se passer, avec la plus grande circonspection et par degrés, compte tenu des enseignements de la pratique. Ce travail ardu, qui vise à \emph{apprendre} dans la pratique comment on édifie la très grande production, est le gage que nous sommes dans la bonne voie, que les ouvriers conscients de Russie luttent contre le désordre et le gâchis engendrés par la petite propriété, qu’ils luttent contre l’indiscipline\footnote{Il est extrêmement typique que les auteurs des thèses ne disent pas un mot de la signification de la \emph{dictature} du prolétariat dans le domaine \emph{économique.} Ils ne parlent que d’« organisation », etc. Mais la nécessité de l’organisation est reconnue aussi par le petit bourgeois, qui redoute la \emph{dictature} des ouvriers dans les rapports économiques. Un révolutionnaire prolétarien n’aurait jamais pu, à un pareil moment, « oublier » ce « pivot » de la révolution prolétarienne, qui est dirigée contre les fondements économiques du capitalisme. (\emph{Note de l’auteur})} petite‑bourgeoise ; il est le gage de la victoire du communisme.\par

\labelblock{Deux remarques pour conclure.}

\noindent Quand nous discutions avec les « communistes de gauche ; le 4 avril 1918 (voir \emph{Kommounist}, n° 1), je leur ai posé la question de front : Essayez de nous dire ce qui vous déplaît dans le décret sur les chemins de fer : apportez-y \emph{vos} amendements. C'est votre devoir de dirigeants soviétiques du prolétariat ; sinon, vos paroles ne sont que des phrases.\par
Le 20 avril 1918 paraît le n° 1 du \emph{Kommounist}, qui ne contient \emph{pas un mot} sur les modifications ou les corrections qu’il aurait fallu, de l’avis des « communistes de gauche », apporter au décret sur les chemins de fer.\par
Par ce silence, les « communistes de gauche » se sont eux‑mêmes condamnés. Ils se sont contentés d’insinuations agressives \emph{contre} le décret sur les chemins de fer (pages 8 et 16 du n° 1 de leur revue), mais ils n’ont \emph{rien répondu} de clair à la question : « Dans quel sens corriger le décret s’il est erroné ? »\par
Cela se passe de commentaires. Une \emph{pareille « critique} du décret sur les chemins de fer (modèle de notre ligne politique, politique de fermeté, politique de dictature politique de discipline prolétarienne), les ouvriers conscients la qualifieront soit de « critique à la Issouv », soit de phrase en l’air.\par
L'autre remarque. Le n° 1 du \emph{Kommounist} publie un article critique, très flatteur pour moi, consacré par le camarade Boukharine à ma brochure l’État\emph{ et la Révolution.} Mais quelque prix que j’attache à l’avis d’hommes tels que Boukharine, je dois dire en toute conscience que le \emph{caractère} de cet article révèle un fait affligeant et significatif : c’est tourné vers le \emph{passé}, et non vers l’avenir, que Boukharine envisage les tâches de la dictature du prolétariat. Boukharine a remarqué et souligné ce qu’un révolutionnaire prolétarien et un révolutionnaire petit‑bourgeois peuvent avoir de commun dans la question de l’État. Mais il n’a pas « remarqué » ce qui distingue le premier du second.\par
Boukharine a remarqué et souligné que l’ancien appareil d’État doit être « démoli », qu’il faut « le faire sauter », que la bourgeoisie doit être «étranglée jusqu’au bout », etc… Cela, un petit bourgeois enragé peut le désirer également. Et c’est ce que notre révolution a \emph{déjà} fait dans l’essentiel, entre octobre 1917 et février 1918.\par
Mais ce que ne peut vouloir même le petit bourgeois le plus révolutionnaire, ce que veut le prolétaire conscient, ce que notre révolution n’a \emph{pas encore} fait, ma brochure en parle également. Or, cette tâche, la tâche de demain, Boukharine l’a passée sous silence.\par
J'ai, quant à moi, d’autant moins de raisons de garder le même silence que, tout d’abord, on doit, quand on est communiste, être plus attentif aux tâches de demain qu’à celles d’hier, et qu’ensuite ma brochure a été écrite \emph{avant} la prise du pouvoir par les bolchéviks, à une époque où l’on ne pouvait pas servir aux bolchéviks le vulgaire argument petit‑bourgeois : \par

\begin{quoteblock}
 \noindent « Eh bien, \emph{maintenant} que vous avez pris le pouvoir, vous vous mettrez \emph{naturellement} à faire le grand air de la discipline »[…]\par
 « […] Le socialisme aboutira dans son évolution au communisme[…] car les hommes s’habitueront à observer les conditions élémentaires de la vie en société, sans violence et sans soumission » (\emph{l’État et la Révolution}, pages 77‑78. Il était donc question de « conditions élémentaires » avant la prise du pouvoir).\par
 « […] Alors seulement la démocratie commencera à s’éteindre […] » quand « les hommes s’habitueront graduellement à respecter les règles élémentaires de la vie en société connues depuis des siècles, rebattues durant des millénaires dans toutes les prescriptions morales, à les respecter sans violence, sans contrainte[…] sans cet appareil spécial de coercition qui s’appelle l’État » (ibid., p. 84; il était donc question de « prescriptions » avant la prise du pouvoir).\par
 « […] La phase supérieure du communisme » (à chacun selon ses besoins, de chacun selon ses capacités) « suppose une productivité du travail différente de celle d’aujourd’hui, et la disparition de l’homme moyen d’aujourd’hui, capable, comme les séminaristes de Pomialovski, de gaspiller à plaisir les richesses publiques et d’exiger l’impossible » (ibid., p. 91).\par
 « […] En attendant l’avènement de la phase « supérieure du communisme, les socialistes réclament de et de l’État qu’ils exercent le contrôle le plus rigoureux sur la mesure de travail et la mesure de consommation » (ibid.).\par
 « Enregistrement et contrôle, tel est l’essentiel pour la mise en route, pour le fonctionnement régulier de la société communiste dans sa première phase (ibid., p. 95), Et ce contrôle, il faut l’organiser non seulement « sur l’infime minorité de capitalistes, sur les petits messieurs désireux de conserver leurs pratiques capitalistes », mais aussi sur ceux des ouvriers qui sont \hspace{1em}« profondément corrompus par le capitalisme » (ibid., p. 96) et sur « les parasites, les fils à papa, les filous et autres gardiens des traditions du capitalisme » (ibid.).
\end{quoteblock}

\bigbreak
\noindent Il est significatif que \emph{cela}, \textbf{Boukharine} ne l’ait pas relevé.
 


% at least one empty page at end (for booklet couv)
\ifbooklet
  \newpage\null\thispagestyle{empty}\newpage
\fi

\ifdev % autotext in dev mode
\fontname\font — \textsc{Les règles du jeu}\par
(\hyperref[utopie]{\underline{Lien}})\par
\noindent \initialiv{A}{lors là}\blindtext\par
\noindent \initialiv{À}{ la bonheur des dames}\blindtext\par
\noindent \initialiv{É}{tonnez-le}\blindtext\par
\noindent \initialiv{Q}{ualitativement}\blindtext\par
\noindent \initialiv{V}{aloriser}\blindtext\par
\Blindtext
\phantomsection
\label{utopie}
\Blinddocument
\fi
\end{document}
