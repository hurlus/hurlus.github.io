%%%%%%%%%%%%%%%%%%%%%%%%%%%%%%%%%
% LaTeX model https://hurlus.fr %
%%%%%%%%%%%%%%%%%%%%%%%%%%%%%%%%%

% Needed before document class
\RequirePackage{pdftexcmds} % needed for tests expressions
\RequirePackage{fix-cm} % correct units

% Define mode
\def\mode{a4}

\newif\ifaiv % a4
\newif\ifav % a5
\newif\ifbooklet % booklet
\newif\ifcover % cover for booklet

\ifnum \strcmp{\mode}{cover}=0
  \covertrue
\else\ifnum \strcmp{\mode}{booklet}=0
  \booklettrue
\else\ifnum \strcmp{\mode}{a5}=0
  \avtrue
\else
  \aivtrue
\fi\fi\fi

\ifbooklet % do not enclose with {}
  \documentclass[french,twoside]{book} % ,notitlepage
  \usepackage[%
    papersize={105mm, 297mm},
    inner=12mm,
    outer=12mm,
    top=20mm,
    bottom=15mm,
    marginparsep=0pt,
  ]{geometry}
  \usepackage[fontsize=9.5pt]{scrextend} % for Roboto
\else\ifav
  \documentclass[french,twoside]{book} % ,notitlepage
  \usepackage[%
    a5paper,
    inner=25mm,
    outer=15mm,
    top=15mm,
    bottom=15mm,
    marginparsep=0pt,
  ]{geometry}
  \usepackage[fontsize=12pt]{scrextend}
\else% A4 2 cols
  \documentclass[twocolumn]{report}
  \usepackage[%
    a4paper,
    inner=15mm,
    outer=10mm,
    top=25mm,
    bottom=18mm,
    marginparsep=0pt,
  ]{geometry}
  \setlength{\columnsep}{20mm}
  \usepackage[fontsize=9.5pt]{scrextend}
\fi\fi

%%%%%%%%%%%%%%
% Alignments %
%%%%%%%%%%%%%%
% before teinte macros

\setlength{\arrayrulewidth}{0.2pt}
\setlength{\columnseprule}{\arrayrulewidth} % twocol
\setlength{\parskip}{0pt} % classical para with no margin
\setlength{\parindent}{1.5em}

%%%%%%%%%%
% Colors %
%%%%%%%%%%
% before Teinte macros

\usepackage[dvipsnames]{xcolor}
\definecolor{rubric}{HTML}{0c71c3} % the tonic
\def\columnseprulecolor{\color{rubric}}
\colorlet{borderline}{rubric!30!} % definecolor need exact code
\definecolor{shadecolor}{gray}{0.95}
\definecolor{bghi}{gray}{0.5}

%%%%%%%%%%%%%%%%%
% Teinte macros %
%%%%%%%%%%%%%%%%%
%%%%%%%%%%%%%%%%%%%%%%%%%%%%%%%%%%%%%%%%%%%%%%%%%%%
% <TEI> generic (LaTeX names generated by Teinte) %
%%%%%%%%%%%%%%%%%%%%%%%%%%%%%%%%%%%%%%%%%%%%%%%%%%%
% This template is inserted in a specific design
% It is XeLaTeX and otf fonts

\makeatletter % <@@@


\usepackage{blindtext} % generate text for testing
\usepackage{contour} % rounding words
\usepackage[nodayofweek]{datetime}
\usepackage{DejaVuSans} % font for symbols
\usepackage{enumitem} % <list>
\usepackage{etoolbox} % patch commands
\usepackage{fancyvrb}
\usepackage{fancyhdr}
\usepackage{fontspec} % XeLaTeX mandatory for fonts
\usepackage{footnote} % used to capture notes in minipage (ex: quote)
\usepackage{framed} % bordering correct with footnote hack
\usepackage{graphicx}
\usepackage{lettrine} % drop caps
\usepackage{lipsum} % generate text for testing
\usepackage[framemethod=tikz,]{mdframed} % maybe used for frame with footnotes inside
\usepackage{pdftexcmds} % needed for tests expressions
\usepackage{polyglossia} % non-break space french punct, bug Warning: "Failed to patch part"
\usepackage[%
  indentfirst=false,
  vskip=1em,
  noorphanfirst=true,
  noorphanafter=true,
  leftmargin=\parindent,
  rightmargin=0pt,
]{quoting}
\usepackage{ragged2e}
\usepackage{setspace}
\usepackage{tabularx} % <table>
\usepackage[explicit]{titlesec} % wear titles, !NO implicit
\usepackage{tikz} % ornaments
\usepackage{tocloft} % styling tocs
\usepackage[fit]{truncate} % used im runing titles
\usepackage{unicode-math}
\usepackage[normalem]{ulem} % breakable \uline, normalem is absolutely necessary to keep \emph
\usepackage{verse} % <l>
\usepackage{xcolor} % named colors
\usepackage{xparse} % @ifundefined
\XeTeXdefaultencoding "iso-8859-1" % bad encoding of xstring
\usepackage{xstring} % string tests
\XeTeXdefaultencoding "utf-8"
\PassOptionsToPackage{hyphens}{url} % before hyperref, which load url package
\usepackage{hyperref} % supposed to be the last one, :o) except for the ones to follow
\urlstyle{same} % after hyperref

% TOTEST
% \usepackage{hypcap} % links in caption ?
% \usepackage{marginnote}
% TESTED
% \usepackage{background} % doesn’t work with xetek
% \usepackage{bookmark} % prefers the hyperref hack \phantomsection
% \usepackage[color, leftbars]{changebar} % 2 cols doc, impossible to keep bar left
% \usepackage[utf8x]{inputenc} % inputenc package ignored with utf8 based engines
% \usepackage[sfdefault,medium]{inter} % no small caps
% \usepackage{firamath} % choose firasans instead, firamath unavailable in Ubuntu 21-04
% \usepackage{flushend} % bad for last notes, supposed flush end of columns
% \usepackage[stable]{footmisc} % BAD for complex notes https://texfaq.org/FAQ-ftnsect
% \usepackage{helvet} % not for XeLaTeX
% \usepackage{multicol} % not compatible with too much packages (longtable, framed, memoir…)
% \usepackage[default,oldstyle,scale=0.95]{opensans} % no small caps
% \usepackage{sectsty} % \chapterfont OBSOLETE
% \usepackage{soul} % \ul for underline, OBSOLETE with XeTeX
% \usepackage[breakable]{tcolorbox} % text styling gone, footnote hack not kept with breakable



% Metadata inserted by a program, from the TEI source, for title page and runing heads
\title{\textbf{ Les Bourla-Papey et la Révolution vaudoise }}
\date{1903}
\author{Eugène Mottaz}
\def\elbibl{Eugène Mottaz. 1903. \emph{Les Bourla-Papey et la Révolution vaudoise}}
\def\elabstract{%
 
\labelblock{Préface hurlue}

 \noindent \emph{Bourla-Papey}, mot de patois vaudois que l’on peut traduire par \emph{Brûle-Papiers}, sont des paysans suisses (Pays de Vaud) de 1802, révoltés contre le rétablissement des privilèges féodaux. En 1798, sous la pression des armées révolutionnaires françaises, la Suisse s’institue en République et abolit les charges coutumières, pour établir un impôt constitutionnel. À la suite d’un coup d’état des élites féodales (7 janvier 1800), les seigneurs rétablissent leurs privilèges fiscaux, sans supprimer l’impôt révolutionnaire. Encore dans une agitation révolutionnaire, les paysans vaudois exprimèrent leur mécontentement contre le doublement de l’impôt d’une manière originale. \par
 
\begin{quoteblock}
\noindent Paix aux hommes, guerre aux papiers\end{quoteblock}

 \noindent  Contrairement à la \emph{Grande Peur} en France, qui a précédé l’abolition des privilèges (20 juillet 1789 au 6 août 1789), les paysans vaudois n’ont pas essayé de tuer les seigneurs ou de détruire les châteaux. Ils ont formulé et pratiqué leur rébellion d’une manière plus rationnelle. En effet, tuer un seigneur n’abolit pas les privilèges, puisqu’un autre féodal peut venir reprendre ses droits. Par contre, toutes les taxes féodales se justifiaient par des chartes écrites, remontant souvent au Moyen-Âge. Afin que l’impôt du seigneur ne revienne plus, il suffit de brûler ses papiers. Les insurgés se sont organisés pour détruire les archives féodales, sans atteindre aux personnes et aux biens immobiliers (mais tout de même quelques incursions dans les caves à vins).\par
  Ce livre d’Eugène Mottaz (1903) est encore le meilleur travail historique sur les Bourla-Papey. L’auteur, représentatif d’une société suisse beaucoup plus instruite que par exemple la France de l’époque, a été instituteur puis professeur d’histoire-géographie, et membre fondateur de la Société vaudoise d’histoire et d’archéologie. Son texte est constitué pour un tiers de citations d’archives d’époque, permettant de plonger dans les mots des élites, ainsi que parfois aussi, des paysans révoltés, rapportés par des témoignages. De sa modestie à l’égard de son sujet, il résulte que ce livre est surtout tissé de faits, beaucoup plus passionnants que des considérations politiques. \par
 
\astermono

 \noindent  Cette insurrection devrait entrer dans l’imaginaire actif de tous les révoltés, afin d’élargir les idées d’actions. N’en vouloir qu’aux papiers a sans doute effrayé l’oligarchie d’alors. Elle se savait hors de son droit en voulant rétablir des impôts anciens sans supprimer les nouveaux, et elle ne pouvait pas accuser la révolte d’assassinats pour la réprimer dans le sang. L’\emph{alliance} entre le peuple et les élites étaient rompue, un petit pays entouré de puissants voisins ne peut pas courir le risque de briser son contrat social. Il fallait retrouver les apparences d’une justice pour que la domination soit de nouveau acceptée, et que l’exploitation fiscale continue et s’approfondisse plus discrètement. \par
  Dans un pays actuel mondialisé, par contre, cela ne pose aucun problème d’imposer des taxes sur le diesel aux pauvres, sous prétexte d’écologie, et de réprimer les manifestations avec des gaz et le harcèlement judiciaire. Les gilets jaunes français de 2019 n’ont pas trouvé leurs papiers à brûlés, et de toute façon n’aurait pas su coordonner une action froide, cette foule d’individus enragés ne faisait pas encore peuple. \par
 
\labelblock{{\scshape Lisez} les \emph{Bourla-Papey}}

 \noindent Ce moment historique a aussi été le sujet d’un roman de Ramuz (1878, 1947) \href{https://ebooks-bnr.com/ramuz-charles-ferdinand-guerre-aux-papiers/}{\dotuline{La Guerre aux Papiers}}\footnote{\href{https://ebooks-bnr.com/ramuz-charles-ferdinand-guerre-aux-papiers/}{\url{https://ebooks-bnr.com/ramuz-charles-ferdinand-guerre-aux-papiers/}}} (1942), 
 
\clearpage
}
\def\elsource{ \href{https://archive.org/details/lesbourlapapeye00mottgoog}{\dotuline{archive.org}}\footnote{\href{https://archive.org/details/lesbourlapapeye00mottgoog}{\url{https://archive.org/details/lesbourlapapeye00mottgoog}}} }

% Default metas
\newcommand{\colorprovide}[2]{\@ifundefinedcolor{#1}{\colorlet{#1}{#2}}{}}
\colorprovide{rubric}{red}
\colorprovide{silver}{Gray}
\@ifundefined{syms}{\newfontfamily\syms{DejaVu Sans}}{}
\newif\ifdev
\@ifundefined{elbibl}{% No meta defined, maybe dev mode
  \newcommand{\elbibl}{Titre court ?}
  \newcommand{\elbook}{Titre du livre source ?}
  \newcommand{\elabstract}{Résumé\par}
  \newcommand{\elurl}{http://oeuvres.github.io/elbook/2}
  \author{Éric Lœchien}
  \title{Un titre de test assez long pour vérifier le comportement d’une maquette}
  \date{1566}
  \devtrue
}{}
\let\eltitle\@title
\let\elauthor\@author
\let\eldate\@date


\defaultfontfeatures{
  % Mapping=tex-text, % no effect seen
  Scale=MatchLowercase,
  Ligatures={TeX,Common},
}

\@ifundefined{\columnseprulecolor}{%
    \patchcmd\@outputdblcol{% find
      \normalcolor\vrule
    }{% and replace by
      \columnseprulecolor\vrule
    }{% success
    }{% failure
      \@latex@warning{Patching \string\@outputdblcol\space failed}%
    }
}{}

\hypersetup{
  % pdftex, % no effect
  pdftitle={\elbibl},
  % pdfauthor={Your name here},
  % pdfsubject={Your subject here},
  % pdfkeywords={keyword1, keyword2},
  bookmarksnumbered=true,
  bookmarksopen=true,
  bookmarksopenlevel=1,
  pdfstartview=Fit,
  breaklinks=true, % avoid long links
  pdfpagemode=UseOutlines,    % pdf toc
  hyperfootnotes=true,
  colorlinks=false,
  pdfborder=0 0 0,
  % pdfpagelayout=TwoPageRight,
  % linktocpage=true, % NO, toc, link only on page no
}


% generic typo commands
\newcommand{\astermono}{\medskip\centerline{\color{rubric}\large\selectfont{\syms ✻}}\medskip\par}%
\newcommand{\astertri}{\medskip\par\centerline{\color{rubric}\large\selectfont{\syms ✻\,✻\,✻}}\medskip\par}%
\newcommand{\asterism}{\bigskip\par\noindent\parbox{\linewidth}{\centering\color{rubric}\large{\syms ✻}\\{\syms ✻}\hskip 0.75em{\syms ✻}}\bigskip\par}%

% lists
\newlength{\listmod}
\setlength{\listmod}{\parindent}
\setlist{
  itemindent=!,
  listparindent=\listmod,
  labelsep=0.2\listmod,
  parsep=0pt,
  % topsep=0.2em, % default topsep is best
}
\setlist[itemize]{
  label=—,
  leftmargin=0pt,
  labelindent=1.2em,
  labelwidth=0pt,
}
\setlist[enumerate]{
  label={\bf\color{rubric}\arabic*.},
  labelindent=0.8\listmod,
  leftmargin=\listmod,
  labelwidth=0pt,
}
\newlist{listalpha}{enumerate}{1}
\setlist[listalpha]{
  label={\bf\color{rubric}\alph*.},
  leftmargin=0pt,
  labelindent=0.8\listmod,
  labelwidth=0pt,
}
\newcommand{\listhead}[1]{\hspace{-1\listmod}\emph{#1}}

\renewcommand{\hrulefill}{%
  \leavevmode\leaders\hrule height 0.2pt\hfill\kern\z@}

% General typo
\DeclareTextFontCommand{\textlarge}{\large}
\DeclareTextFontCommand{\textsmall}{\small}


% commands, inlines
\newcommand{\anchor}[1]{\Hy@raisedlink{\hypertarget{#1}{}}} % link to top of an anchor (not baseline)
\newcommand\abbr[1]{#1}
\newcommand{\autour}[1]{\tikz[baseline=(X.base)]\node [draw=rubric,thin,rectangle,inner sep=1.5pt, rounded corners=3pt] (X) {\color{rubric}#1};}
\newcommand\corr[1]{#1}
\newcommand{\ed}[1]{ {\color{silver}\sffamily\footnotesize (#1)} } % <milestone ed="1688"/>
\newcommand\expan[1]{#1}
\newcommand\foreign[1]{\emph{#1}}
\newcommand\gap[1]{#1}
\renewcommand{\LettrineFontHook}{\color{rubric}}
\newcommand{\initial}[2]{\lettrine[lines=2, loversize=0.3, lhang=0.3]{#1}{#2}}
\newcommand{\initialiv}[2]{%
  \let\oldLFH\LettrineFontHook
  % \renewcommand{\LettrineFontHook}{\color{rubric}\ttfamily}
  \IfSubStr{QJ’}{#1}{
    \lettrine[lines=4, lhang=0.2, loversize=-0.1, lraise=0.2]{\smash{#1}}{#2}
  }{\IfSubStr{É}{#1}{
    \lettrine[lines=4, lhang=0.2, loversize=-0, lraise=0]{\smash{#1}}{#2}
  }{\IfSubStr{ÀÂ}{#1}{
    \lettrine[lines=4, lhang=0.2, loversize=-0, lraise=0, slope=0.6em]{\smash{#1}}{#2}
  }{\IfSubStr{A}{#1}{
    \lettrine[lines=4, lhang=0.2, loversize=0.2, slope=0.6em]{\smash{#1}}{#2}
  }{\IfSubStr{V}{#1}{
    \lettrine[lines=4, lhang=0.2, loversize=0.2, slope=-0.5em]{\smash{#1}}{#2}
  }{
    \lettrine[lines=4, lhang=0.2, loversize=0.2]{\smash{#1}}{#2}
  }}}}}
  \let\LettrineFontHook\oldLFH
}
\newcommand{\labelchar}[1]{\textbf{\color{rubric} #1}}
\newcommand{\milestone}[1]{\autour{\footnotesize\color{rubric} #1}} % <milestone n="4"/>
\newcommand\name[1]{#1}
\newcommand\orig[1]{#1}
\newcommand\orgName[1]{#1}
\newcommand\persName[1]{#1}
\newcommand\placeName[1]{#1}
\newcommand{\pn}[1]{\IfSubStr{-—–¶}{#1}% <p n="3"/>
  {\noindent{\bfseries\color{rubric}   ¶  }}
  {{\footnotesize\autour{ #1}  }}}
\newcommand\reg{}
% \newcommand\ref{} % already defined
\newcommand\sic[1]{#1}
\newcommand\surname[1]{\textsc{#1}}
\newcommand\term[1]{\textbf{#1}}

\def\mednobreak{\ifdim\lastskip<\medskipamount
  \removelastskip\nopagebreak\medskip\fi}
\def\bignobreak{\ifdim\lastskip<\bigskipamount
  \removelastskip\nopagebreak\bigskip\fi}

% commands, blocks
\newcommand{\byline}[1]{\bigskip{\RaggedLeft{#1}\par}\bigskip}
\newcommand{\bibl}[1]{{\RaggedLeft{#1}\par\bigskip}}
\newcommand{\biblitem}[1]{{\noindent\hangindent=\parindent   #1\par}}
\newcommand{\dateline}[1]{\medskip{\RaggedLeft{#1}\par}\bigskip}
\newcommand{\labelblock}[1]{\medbreak{\noindent\color{rubric}\bfseries #1}\par\mednobreak}
\newcommand{\salute}[1]{\bigbreak{#1}\par\medbreak}
\newcommand{\signed}[1]{\bigbreak\filbreak{\raggedleft #1\par}\medskip}

% environments for blocks (some may become commands)
\newenvironment{borderbox}{}{} % framing content
\newenvironment{citbibl}{\ifvmode\hfill\fi}{\ifvmode\par\fi }
\newenvironment{docAuthor}{\ifvmode\vskip4pt\fontsize{16pt}{18pt}\selectfont\fi\itshape}{\ifvmode\par\fi }
\newenvironment{docDate}{}{\ifvmode\par\fi }
\newenvironment{docImprint}{\vskip6pt}{\ifvmode\par\fi }
\newenvironment{docTitle}{\vskip6pt\bfseries\fontsize{18pt}{22pt}\selectfont}{\par }
\newenvironment{msHead}{\vskip6pt}{\par}
\newenvironment{msItem}{\vskip6pt}{\par}
\newenvironment{titlePart}{}{\par }


% environments for block containers
\newenvironment{argument}{\itshape\parindent0pt}{\vskip1.5em}
\newenvironment{biblfree}{}{\ifvmode\par\fi }
\newenvironment{bibitemlist}[1]{%
  \list{\@biblabel{\@arabic\c@enumiv}}%
  {%
    \settowidth\labelwidth{\@biblabel{#1}}%
    \leftmargin\labelwidth
    \advance\leftmargin\labelsep
    \@openbib@code
    \usecounter{enumiv}%
    \let\p@enumiv\@empty
    \renewcommand\theenumiv{\@arabic\c@enumiv}%
  }
  \sloppy
  \clubpenalty4000
  \@clubpenalty \clubpenalty
  \widowpenalty4000%
  \sfcode`\.\@m
}%
{\def\@noitemerr
  {\@latex@warning{Empty `bibitemlist' environment}}%
\endlist}
\newenvironment{quoteblock}% may be used for ornaments
  {\begin{quoting}}
  {\end{quoting}}

% table () is preceded and finished by custom command
\newcommand{\tableopen}[1]{%
  \ifnum\strcmp{#1}{wide}=0{%
    \begin{center}
  }
  \else\ifnum\strcmp{#1}{long}=0{%
    \begin{center}
  }
  \else{%
    \begin{center}
  }
  \fi\fi
}
\newcommand{\tableclose}[1]{%
  \ifnum\strcmp{#1}{wide}=0{%
    \end{center}
  }
  \else\ifnum\strcmp{#1}{long}=0{%
    \end{center}
  }
  \else{%
    \end{center}
  }
  \fi\fi
}


% text structure
\newcommand\chapteropen{} % before chapter title
\newcommand\chaptercont{} % after title, argument, epigraph…
\newcommand\chapterclose{} % maybe useful for multicol settings
\setcounter{secnumdepth}{-2} % no counters for hierarchy titles
\setcounter{tocdepth}{5} % deep toc
\markright{\@title} % ???
\markboth{\@title}{\@author} % ???
\renewcommand\tableofcontents{\@starttoc{toc}}
% toclof format
% \renewcommand{\@tocrmarg}{0.1em} % Useless command?
% \renewcommand{\@pnumwidth}{0.5em} % {1.75em}
\renewcommand{\@cftmaketoctitle}{}
\setlength{\cftbeforesecskip}{\z@ \@plus.2\p@}
\renewcommand{\cftchapfont}{}
\renewcommand{\cftchapdotsep}{\cftdotsep}
\renewcommand{\cftchapleader}{\normalfont\cftdotfill{\cftchapdotsep}}
\renewcommand{\cftchappagefont}{\bfseries}
\setlength{\cftbeforechapskip}{0em \@plus\p@}
% \renewcommand{\cftsecfont}{\small\relax}
\renewcommand{\cftsecpagefont}{\normalfont}
% \renewcommand{\cftsubsecfont}{\small\relax}
\renewcommand{\cftsecdotsep}{\cftdotsep}
\renewcommand{\cftsecpagefont}{\normalfont}
\renewcommand{\cftsecleader}{\normalfont\cftdotfill{\cftsecdotsep}}
\setlength{\cftsecindent}{1em}
\setlength{\cftsubsecindent}{2em}
\setlength{\cftsubsubsecindent}{3em}
\setlength{\cftchapnumwidth}{1em}
\setlength{\cftsecnumwidth}{1em}
\setlength{\cftsubsecnumwidth}{1em}
\setlength{\cftsubsubsecnumwidth}{1em}

% footnotes
\newif\ifheading
\newcommand*{\fnmarkscale}{\ifheading 0.70 \else 1 \fi}
\renewcommand\footnoterule{\vspace*{0.3cm}\hrule height \arrayrulewidth width 3cm \vspace*{0.3cm}}
\setlength\footnotesep{1.5\footnotesep} % footnote separator
\renewcommand\@makefntext[1]{\parindent 1.5em \noindent \hb@xt@1.8em{\hss{\normalfont\@thefnmark . }}#1} % no superscipt in foot


% orphans and widows
\clubpenalty=9996
\widowpenalty=9999
\brokenpenalty=4991
\predisplaypenalty=10000
\postdisplaypenalty=1549
\displaywidowpenalty=1602
\hyphenpenalty=400
% Copied from Rahtz but not understood
\def\@pnumwidth{1.55em}
\def\@tocrmarg {2.55em}
\def\@dotsep{4.5}
\emergencystretch 3em
\hbadness=4000
\pretolerance=750
\tolerance=2000
\vbadness=4000
\def\Gin@extensions{.pdf,.png,.jpg,.mps,.tif}
% \renewcommand{\@cite}[1]{#1} % biblio

\makeatother % /@@@>
%%%%%%%%%%%%%%
% </TEI> end %
%%%%%%%%%%%%%%


%%%%%%%%%%%%%
% footnotes %
%%%%%%%%%%%%%
\renewcommand{\thefootnote}{\bfseries\textcolor{rubric}{\arabic{footnote}}} % color for footnote marks

%%%%%%%%%
% Fonts %
%%%%%%%%%
\usepackage[]{roboto} % SmallCaps, Regular is a bit bold
% \linespread{0.90} % too compact, keep font natural
\newfontfamily\fontrun[]{Roboto Condensed Light} % condensed runing heads
\ifav
  \setmainfont[
    ItalicFont={Roboto Light Italic},
  ]{Roboto}
\else\ifbooklet
  \setmainfont[
    ItalicFont={Roboto Light Italic},
  ]{Roboto}
\else
\setmainfont[
  ItalicFont={Roboto Italic},
]{Roboto Light}
\fi\fi
\renewcommand{\LettrineFontHook}{\bfseries\color{rubric}}
% \renewenvironment{labelblock}{\begin{center}\bfseries\color{rubric}}{\end{center}}

%%%%%%%%
% MISC %
%%%%%%%%

\setdefaultlanguage[frenchpart=false]{french} % bug on part


\newenvironment{quotebar}{%
    \def\FrameCommand{{\color{rubric!10!}\vrule width 0.5em} \hspace{0.9em}}%
    \def\OuterFrameSep{\itemsep} % séparateur vertical
    \MakeFramed {\advance\hsize-\width \FrameRestore}
  }%
  {%
    \endMakeFramed
  }
\renewenvironment{quoteblock}% may be used for ornaments
  {%
    \savenotes
    \setstretch{0.9}
    \normalfont
    \begin{quotebar}
  }
  {%
    \end{quotebar}
    \spewnotes
  }


\renewcommand{\headrulewidth}{\arrayrulewidth}
\renewcommand{\headrule}{{\color{rubric}\hrule}}

% delicate tuning, image has produce line-height problems in title on 2 lines
\titleformat{name=\chapter} % command
  [display] % shape
  {\vspace{1.5em}\centering} % format
  {} % label
  {0pt} % separator between n
  {}
[{\color{rubric}\huge\textbf{#1}}\bigskip] % after code
% \titlespacing{command}{left spacing}{before spacing}{after spacing}[right]
\titlespacing*{\chapter}{0pt}{-2em}{0pt}[0pt]

\titleformat{name=\section}
  [block]{}{}{}{}
  [\vbox{\color{rubric}\large\raggedleft\textbf{#1}}]
\titlespacing{\section}{0pt}{0pt plus 4pt minus 2pt}{\baselineskip}

\titleformat{name=\subsection}
  [block]
  {}
  {} % \thesection
  {} % separator \arrayrulewidth
  {}
[\vbox{\large\textbf{#1}}]
% \titlespacing{\subsection}{0pt}{0pt plus 4pt minus 2pt}{\baselineskip}

\ifaiv
  \fancypagestyle{main}{%
    \fancyhf{}
    \setlength{\headheight}{1.5em}
    \fancyhead{} % reset head
    \fancyfoot{} % reset foot
    \fancyhead[L]{\truncate{0.45\headwidth}{\fontrun\elbibl}} % book ref
    \fancyhead[R]{\truncate{0.45\headwidth}{ \fontrun\nouppercase\leftmark}} % Chapter title
    \fancyhead[C]{\thepage}
  }
  \fancypagestyle{plain}{% apply to chapter
    \fancyhf{}% clear all header and footer fields
    \setlength{\headheight}{1.5em}
    \fancyhead[L]{\truncate{0.9\headwidth}{\fontrun\elbibl}}
    \fancyhead[R]{\thepage}
  }
\else
  \fancypagestyle{main}{%
    \fancyhf{}
    \setlength{\headheight}{1.5em}
    \fancyhead{} % reset head
    \fancyfoot{} % reset foot
    \fancyhead[RE]{\truncate{0.9\headwidth}{\fontrun\elbibl}} % book ref
    \fancyhead[LO]{\truncate{0.9\headwidth}{\fontrun\nouppercase\leftmark}} % Chapter title, \nouppercase needed
    \fancyhead[RO,LE]{\thepage}
  }
  \fancypagestyle{plain}{% apply to chapter
    \fancyhf{}% clear all header and footer fields
    \setlength{\headheight}{1.5em}
    \fancyhead[L]{\truncate{0.9\headwidth}{\fontrun\elbibl}}
    \fancyhead[R]{\thepage}
  }
\fi

\ifav % a5 only
  \titleclass{\section}{top}
\fi

\newcommand\chapo{{%
  \vspace*{-3em}
  \centering % no vskip ()
  {\Large\addfontfeature{LetterSpace=25}\bfseries{\elauthor}}\par
  \smallskip
  {\large\eldate}\par
  \bigskip
  {\Large\selectfont{\eltitle}}\par
  \bigskip
  {\color{rubric}\hline\par}
  \bigskip
  {\Large LIVRE LIBRE À PRIX LIBRE, DEMANDEZ AU COMPTOIR\par}
  \centerline{\small\color{rubric} {hurlus.fr, tiré le \today}}\par
  \bigskip
}}


\begin{document}
\pagestyle{empty}
\ifbooklet{
  \thispagestyle{empty}
  \centering
  {\LARGE\bfseries{\elauthor}}\par
  \bigskip
  {\Large\eldate}\par
  \bigskip
  \bigskip
  {\LARGE\selectfont{\eltitle}}\par
  \vfill\null
  {\color{rubric}\setlength{\arrayrulewidth}{2pt}\hline\par}
  \vfill\null
  {\Large LIVRE LIBRE À PRIX LIBRE, DEMANDEZ AU COMPTOIR\par}
  \centerline{\small{hurlus.fr, tiré le \today}}\par
  \newpage\null\thispagestyle{empty}\newpage
  \addtocounter{page}{-2}
}\fi

\thispagestyle{empty}
\ifaiv
  \twocolumn[\chapo]
\else
  \chapo
\fi
{\it\elabstract}
\bigskip
\makeatletter\@starttoc{toc}\makeatother % toc without new page
\bigskip

\pagestyle{main} % after style

  
\chapteropen
\chapter[Préface]{Préface}\renewcommand{\leftmark}{Préface}


\chaptercont
\noindent Le nombre des ouvrages qui parlent de la Révolution vaudoise est déjà considérable : Verdeil, Juste Olivier et Seigneux nous fournissent sur l’émancipation de 1798 et sur la formation laborieuse du Canton de Vaud en 1802 et 1803 des renseignements nombreux et intéressants. Après eux, plusieurs écrivains ont encore Jeté sur divers points de ce sujet une lumière nouvelle et précieuse. En 1898 surtout, les journaux et les revues du canton et de l’étranger ont tenu le public au courant des événements qui s’étaient accomplis un siècle auparavant. Dernièrement, enfin, M. Charles Burnier nous a donné dans son remarquable ouvrage sur \emph{La Vie vaudoise et la Révolution}, un tableau philosophique, raisonné et intéressant, de cette époque importante pour nous.\par
Les événements de 1798 peuvent donc être considérés comme connus par la génération actuelle. Il n’en est pas de même, me semble-t-il, pour ceux des années suivantes, pour ceux, surtout, de 1802, qui ont exercé une influence si considérable sur les destinées de notre petit pays.\par
La Révolution de 1798 fut surtout l’œuvre des villes ; elle eut pour conséquence l’émancipation politique du Pays de Vaud. Celle de 1802 fut accomplie essentiellement par les campagnards et les patriotes aux idées les plus avancées ; elle amena l’émancipation économique par le moyen de la suppression totale et définitive des droits féodaux. C’est de cette dernière, surtout, qu’il sera question ici. Moins connue que la première, elle n’est pas moins digne d’intérêt et elle nous fait pénétrer davantage dans la vie du peuple. Différents auteurs nous en ont parlé d’une manière plus ou moins exacte ; mais, jusqu’à maintenant, elle est restée, malgré tout, enveloppée d’un certain mystère ; des légendes se sont formées à son sujet et le roman lui-même s’en est emparé\footnote{Les \emph{Bourla-Papey}, par A. de Bougy.} \par
Après un siècle d’indépendance, le moment semble venu de raconter les événements de cette époque. Les historiens vaudois ont insisté jusqu’à maintenant sur le dévouement du Canton du Léman au reste de l’Helvétie, sur ses sacrifices militaires, sur son ardeur à défendre le nouvel ordre de choses. Ils ont volontiers laissé dans l’ombre certains côtés de notre Révolution afin de ne pas être obligés de montrer qu’un certain nombre de patriotes du Léman furent amenés à hésiter un moment entre l’alliance helvétique et celle de la France. Le moment est venu aussi de montrer quelle était la vraie situation des esprits en — 1802 et comment les unitaires de 1798 devinrent les cantonalistes de 1803.\par
Ce volume est le résultat de longues recherches. Les archives de plusieurs villes vaudoises, celles de Morges et d’Yverdon entre autres, ont fourni quelques renseignements intéressants. La plupart des pièces, correspondances officielles et documents de tout genre sur cette époque se trouvent cependant à Berne, aux archives fédérales. Beaucoup, des plus importants, ont été publiés, ou le seront, dans la grande collection des \emph{Actes de l’Helvétique} que le gouvernement fédéral fait paraître sous la direction du très savant et consciencieux Dr Strickler. Je remercie vivement le service des archives fédérales, qui a bien voulu faciliter mes recherches et même m’adresser, sur la question des \emph{Bourla-Papey}, un nombre considérable de volumes de pièces originales et authentiques qui ne figurent pas toutes dans les \emph{Actes} mais présentent parfois cependant un très vif intérêt.\par
Sachant que les notes très nombreuses sont un ennui pour beaucoup de lecteurs, je n’ai cité mes sources que lorsqu’elles sont tirées d’archives vaudoises, de publications antérieures ou de renseignements reçus de divers côtés. Tous les autres faits et renseignements reposent sur des documents authentiques des archives fédérales, et il m’a paru aussi inutile que fastidieux de le rappeler à chaque page.
\chapterclose


\chapteropen
\chapter[Introduction]{Introduction}\renewcommand{\leftmark}{Introduction}


\chaptercont
\noindent Les impôts proprement dits étaient peu nombreux sous l’ancien régime. Presque tous étaient indirects ; ils se composaient surtout des péages et de la régie du sel qui produisaient, à la veille de la Révolution, une somme de 160 000 francs dans le territoire actuel du Canton de Vaud. Quelques autres existaient encore, mais beaucoup moins importants. Le chiffre ci-dessus est fourni par Henri Monod, fonctionnaire lui-même de la régie des sels et, à ce titre-là, parfaitement qualifié pour connaître un peu la situation. La population des villes, celle des contrées montagneuses où l’élevage du bétail constituait l’occupation principale, ne versaient donc que fort peu de chose dans la caisse de l’État.\par
Si toutes les redevances payées par les Vaudois se fussent bornées à cela, la situation de ces derniers eût été certainement enviable. Malheureusement, la ville de Berne avait conquis en 1536 une province dans laquelle une grande partie du territoire dépendait des seigneurs qui possédaient sur les terres et leurs habitants des droits nombreux et 4 imprescriptibles. Ces seigneurs virent sans doute leur puissance et leurs privilèges diminuer très sensiblement à partir de cette date, mais ils n’en conservèrent pas moins certains avantages dont le plus important consistait à prélever diverses redevances sur les produits du sol, qui constituait leur fief ou leur seigneurie. Suivant que le seigneur était un simple particulier ou l’État, le cultivateur payait la redevance à celui-ci ou à celui-là. Il résulte de ces considérations que les droits féodaux perçus sous l’ancien régime dans le Pays de Vaud ne doivent pas être assimilés, comme on a l’habitude de le faire souvent, à des impôts proprement dits. Ils résultaient de contrats conclus, pour la plupart, fort anciennement, entre celui qui cultivait le sol et celui qui avait des droits sur ce dernier.\par
Ces redevances étaient entre autres la dîme, le cens et le lod.\par
La dîme était prélevée sur les récoltes en vin, en blé, etc. Le cens – le cens était devenu chez nous un mot féminin – était une redevance payée quelquefois en nature, quelquefois aussi en argent, mais beaucoup moins importante et moins générale que la dîme. Le lod – ou laud – était notre droit de mutation et de succession.\par
Quand la Révolution éclata, les campagnards apprirent que dorénavant ils n’auraient plus à payer les redevances seigneuriales ; c’est du moins ce qui leur fut promis de divers côtés. Très défiants à l’égard du nouveau régime, ils ne s’y rallièrent que dans la ferme espérance de jouir de l’émancipation économique, comme leurs voisins de France.\par
Quelle était la valeur annuelle de ces redevances ? Il est assez difficile de le dire d’une manière exacte. Henri Monod en donne dans ses \emph{Mémoires} une estimation passablement détaillée ; mais deux pages plus loin il reconnaît que la somme indiquée par lui est trop forte. Des tableaux dressés par les soins du Ministère helvétique des finances renferment à ce sujet des indications sensiblement différentes. De tous ces témoignages, il résulte cependant que les droits seigneuriaux payés à la caisse de l’État formaient une somme deux fois plus forte, au moins, que ceux qui étaient dûs aux communes, à des institutions et à des particuliers. Le Bureau de liquidation qui fut institué à Lausanne, au mois d’octobre 1802, évalua ces derniers à la somme de 284 800 francs. Ajoutons, enfin, que les droits perçus par l’État servaient tout d’abord à rétribuer les professeurs de l’académie, les pasteurs, les instituteurs – pensionnés aussi par les communes – et à subventionner divers établissements.\par
Ensuite de la Révolution, le Pays de Vaud se trouva affranchi de ce qu’il avait dû livrer jusqu’alors au trésor de LL. EE. Les propriétés particulières ne furent pas abolies de la même manière et la question se posa bientôt de savoir si les droits seigneuriaux seraient supprimés et, dans le cas de l’affirmative, si on accorderait une indemnité aux possesseurs de fiefs.\par
L’immense majorité de la nation vaudoise voulait la suppression des dîmes ; citadins et campagnards étaient d’accord à ce sujet. Cette unanimité était moins parfaite lorsqu’il s’agissait de déterminer la méthode à suivre pour effectuer cette suppression. Les patriotes exaltés, ceux qui voulaient en tout imiter la France, estimaient qu’il fallait abolir d’un trait de plume l’ancien régime féodal et que les droits seigneuriaux étant injustes, il n’y avait pas lieu d’accorder une indemnité à leurs propriétaires. Cette solution très simple paraissait trop révolutionnaire à la majorité des Vaudois et surtout aux chefs principaux du parti patriote. Le nouveau régime devant forcément exiger lui aussi des impositions, il semblait d’autre part impraticable de faire racheter Complètement les anciennes par ceux seulement qui les avaient payées auparavant. La première autorité constitutionnelle que le nouveau Canton du Léman posséda en 1798, la Chambre administrative, proposa donc un projet en vertu duquel une partie des biens nationaux devait servir à indemniser les propriétaires de fiefs. La modeste somme qui aurait manqué peut-être pour liquider cette question devait être répartie sur le pays par le moyen d’une légère contribution foncière.\par
La Chambre administrative n’eut malheureusement pas le temps d’exécuter ce projet équitable pour tous. La question de la suppression des droits féodaux avec ou sans rachat resta donc à la base des préoccupations du Canton du Léman. Après avoir attendu avec patience l’effet des promesses qui leur avaient été faites en 1798, les campagnards montrèrent un mécontentement de plus en plus grand lorqu’ils s’aperçurent que le gouvernement central ne faisait rien de sérieux pour les satisfaire. Lorsque, enfin, le gouvernement usa des anciens droits de LL. EE. pour exiger de nouveau les redevances seigneuriales, en 1801, ce mécontentement se changea en colère et il en résulta l’insurrection de 1802 à la suite de laquelle on donna satisfaction aux vœux légitimes des paysans en revenant à la solution préconisée en 1798 par la Chambre administrative et, dès lors, par tous les hommes d’État du parti patriote.\par
Les considérations qui précédent seront suffisantes sans doute pour servir d’introduction au récit des événements. Quelques éclaircissements sur l’organisation de la République helvétique et celle du Canton du Léman pendant la période révolutionnaire pourront être, en revanche, de quelque utilité pour faire comprendre certains termes ou titres qui reviendront souvent dans ce volume.\par
Le Pays de Vaud se constitua dès le mois de janvier 1798 en République Lémanique. Le 9 février suivant, l’Assemblée provisoire adopta la nouvelle Constitution de la Suisse, rédigée à Paris par Pierre Ochs, de Bâle, et recommandée – sinon imposée – par la France.\par
Cette loi fondamentale n’admettait plus de pays sujets ou alliés et groupait toute la Suisse en un seul État, la République helvétique, une et indivisible, composée de vingt-deux cantons \footnote{Ce nombre ne fut pas maintenu. La division du territoire fut modifiée très souvent. Plusieurs cantons furent parfois réunis en un seul.}. La partie romande du pays en constitua trois : le Valais, que la France chercha à se rattacher dès 1801 en le faisant terroriser par le général Turreau ; le Canton du Léman, formé du territoire vaudois actuel, moins les districts de Payerne et d’Avenches réunis au troisième canton, celui de Sarine et Broie.\par
La Constitution donnait à la République helvétique un seul gouvernement qui se constitua au milieu du mois d’avril à Aarau. Le Pouvoir exécutif était confié à un \emph{Directoire}, secondé par quelques Ministres qui se partageaient le travail administratif. Le pouvoir législatif était composé de deux Conseils : le \emph{Sénat} et le \emph{Grand Conseil.} Les cantons étaient représentés dans le premier par quatre députés et dans le second par huit. Le \emph{Tribunal suprême} était composé d’un juge par canton.\par
Les cantons n’étaient plus que des circonscriptions administratives dans lesquelles le gouvernement central était représenté par un \emph{Préfet national} dont les pouvoirs étaient très étendus. Il nommait la plupart des fonctionnaires subalternes du canton, veillait à la sûreté intérieure, pouvait disposer de la force armée et ordonner des arrestations. Il servait d’intermédiaire obligatoire entre le gouvernement central, les citoyens et les autorités cantonales, aux séances desquelles il pouvait assister. Un \emph{Tribunal de Canton}, composé de treize juges prononçait \emph{« en première instance dans les causes criminelles majeures, et en dernière instance dans les autres causes criminelles, dans les causes civiles et dans celles de police »}. Une \emph{Chambre administrative}, composée d’un président et de quatre membres, était chargée d’assurer dans le Canton l’exécution des lois et la rentrée des impositions.\par
Il y avait dans chaque district un \emph{Sous-préfet} qui veillait au maintien de l’ordre, à la police et à l’exécution des ordres des autorités cantonales. Un \emph{Tribunal de district}, de neuf membres, rendait des jugements en matière correctionnelle et de police.\par
Dans chaque section d’une ville et dans chaque village, le sous-préfet avait sous ses ordres un \emph{Agent national.}\par
Les autorités cantonales prévues par la Constitution de 1798 et indiquées ci-dessus, continuèrent à subsister jusqu’à la mise en vigueur de l’Acte de médiation, le 14 avril 1803. Il n’en fut pas de même du gouvernement central. Après avoir habité Aarau pendant quelques mois, il alla se fixer à Lucerne. Lorsque la Suisse devint, l’année suivante, le théâtre de la guerre étrangère et que les Autrichiens et les Russes occupèrent Zurich, le gouvernement helvétique ne se crut plus en sûreté à Lucerne et se retira à Berne où il résida jusqu’en 1803, excepté pendant l’automne de 1802 où on le verra venir se réfugier à Lausanne.\par
Dès que le général français Masséna eut délivré la Suisse de la présence des armées russes et autrichiennes par la bataille de Zurich, au mois de septembre 1799, la division se mit dans le parti unitaire. Les uns voulaient, sous l’impulsion de La Harpe, attacher les populations au nouveau régime, sous la protection de la France, en mettant en œuvre la plus grande fermeté et même la force. D’autres, plus modérés, comme le Vaudois Maurice Glayre, cherchaient à dégager le pays de la suzeraineté exclusive de la « Grande Nation », et à amener peu à peu le peuple suisse au nouveau régime par le moyen de la persuasion et de la douceur. Ces derniers réussirent, par le Coup d’État du 7 janvier 1800, à abattre la puissance de La Harpe qui fut expulsé du Directoire et, bientôt, se retira en France. Un second Coup d’État, le 7 août de la même année, transforma les Conseils comme le premier avait transformé le pouvoir exécutif. \emph{« Justice vous est rendue, disait alors le sénateur vaudois J.-J. Cart à ses collègues ; le 7 janvier, vous avez destitué le Directoire ; le 7 août, on vous destitue. »}\par
Dès lors, la période helvétique est une succession de Coups d’États et de transformations plus ou moins violentes du gouvernement central. Pendant l’hiver de 1801 à 1802, le parti aristocratique ou de l’ancien régime crut enfin être à la veille de voir ses vœux se réaliser lorsque son illustre chef, Aloïs Reding, devint Landammann de la Suisse. Plus homme de guerre et d’action que diplomate, ce patriote n’eut cependant pas suffisamment de souplesse pour éviter les écueils, et, le 17 avril 1802, un nouveau Coup d’État, appuyé par le représentant de la France, vint rendre le pouvoir aux unitaires qui le conservèrent, mais sans gloire, jusqu’à la mise en vigueur de l’Acte de Médiation.\par
La Constitution de 1798 ne déterminait pas l’organisation intérieure des communes. Elles conservèrent provisoirement leurs anciennes autorités en attendant que les Conseils helvétiques leur eussent donné une administration définitive en rapport avec les idées nouvelles. Les lois du 13 novembre 1798 et des 13 et 15 février 1799 statuèrent à ce sujet pour toute la Suisse, mais renfermèrent des principes qui ne correspondaient pas à nos idées et coutumes romandes. Elles introduisaient dans chaque localité un double organisme : la commune d’habitants et la commune de bourgeoisie. Dans chaque commune, tous les citoyens actifs habitant le territoire nommaient une \emph{Municipalité} de trois à onze membres qui avait dans ses attributions l’administration générale et la police locale. Les bourgeois – co-propriétaires des biens communaux – nommaient de leur côté une \emph{Chambre de Régie} qui s’occupait essentiellement de l’administration de ces biens. Il y avait ainsi deux pouvoirs existant côte à côte : l’un basé sur un principe démocratique, jouant un rôle politique, mais sans ressources pécuniaires ; l’autre purement économique et possédant des revenus. La loi vaudoise du 18 juin 1803 supprima ce dualisme.
\chapterclose


\chapteropen
\chapter[I. Le Canton du Léman, de 1798 à 1802]{I. Le Canton du Léman, de 1798 à 1802}\renewcommand{\leftmark}{I. Le Canton du Léman, de 1798 à 1802}


\chaptercont
\section[Premiers projets de rachat]{Premiers projets de rachat}
\noindent La question des redevances seigneuriales fut une des plus importantes parmi celles dont les autorités vaudoises s’occupèrent après l’émancipation politique de 1798. Lorsque, à la fin du mois de mars, le Canton du Léman fut organisé, la Chambre administrative, qui possédait provisoirement le pouvoir législatif, chercha à profiter du faible temps de sa puissance pour obtenir l’abolition du régime féodal.\par

\begin{quoteblock}
 \noindent « Les fiefs dans le Pays de Vaud étaient une charge très onéreuse, dit Henri Monod\footnote{H. Monod, \emph{Mémoires I} 143-144.}  ; ils étaient d’une nature toute différente de ceux de la Suisse allemande. Prêts à passer sous la même administration, si rien n’était réglé auparavant à cet égard, nos terres allaient payer infiniment plus à la masse collective que celles du reste de la Suisse, et comme nous apportions infiniment plus de biens que la plupart des autres cantons, nous risquions d’être infiniment lésés. »
 \end{quoteblock}

\noindent D’autres raisons importantes devaient encore engager les autorités vaudoises à résoudre si possible cette question.\par
Notre révolution était fille de celle de la France ; il paraissait difficile, sinon impossible, de conserver chez nous les fiefs et leurs conséquences, alors qu’ils avaient été abolis de l’autre côté du Jura. Les campagnards – surtout ceux qui habitaient près de la frontière – étaient certains que ce qui restait du régime féodal disparaîtrait aussitôt après l’émancipation. Leur attente fut déçue ; leur impatience devint bien vite très grande et le fameux Desportes, Résident de France à Genève, n’eut pas de peine à leur persuader que s’ils devenaient citoyens français, ils s’en trouveraient mieux et ne paieraient plus les redevances féodales.\par
Les paysans ne comprirent la révolution que dans la suppression des fiefs. Que le peuple fût administré par MM. de Berne ou par MM. de Lausanne, c’était pour eux une affaire secondaire. Si l’on voulait changer quelque chose dans la situation constitutionnelle du pays, il fallait donc commencer par la question économique. L’émancipation, pour le cultivateur, c’était la suppression de la dîme.\par
La révolution avait été faite par les villes, et cela seul avait excité la défiance des campagnards. Si l’administration nouvelle voulait pouvoir compter sur l’appui de ces derniers et les attacher tout à fait au nouveau régime, elle devait chercher une solution rapide de la question des fiefs.\par
La plus simple eût été de supprimer d’un seul trait de plume l’ancien régime. C’est ce que l’on avait fait en France ; c’est ce que les plus exaltés voulaient faire chez nous. Une mesure de ce genre répugnait cependant à la plupart, puisque c’était une spoliation pure et simple. Les propriétaires de droits seigneuriaux avaient, en effet, acheté ces derniers de bonne foi ou les avaient reçus de leurs ancêtres. Ils constituaient une propriété aussi sacrée que toute autre et on ne pouvait les supprimer sans indemnité qu’en commettant un de ces actes révolutionnaires et injustes qui répugnent à notre peuple.\par
Le plan de la Chambre administrative était simple. Il consistait à donner aux propriétaires de fiefs des reconnaissances pour le montant de leurs droits évalués à un taux équitable. Les biens nationaux auraient été mis en vente et les reconnaissances ci-dessus acceptées en paiement. Si le produit de cette vente n’avait pas suffi, on eût levé un léger impôt foncier.\par
L’administration vaudoise n’eut malheureusement pas le temps de terminer ce grand travail avant que ses pouvoirs extraordinaires eussent passé au gouvernement helvétique qui fut organisé à Aarau au milieu d’avril. Cette circonstance fut fâcheuse pour le canton du Léman qui resta cinq ans sans pouvoir obtenir le règlement de la question des fiefs, règlement qu’il désirait avant toute chose et qu’il avait failli obtenir par ses seules forces. La révolution eût été terminée chez nous. Au lieu de cela, le mécontentement et le malaise allèrent en augmentant d’année en année. L’esprit de parti s’empara de cette question qui devint la source de réclamations de plus en plus violentes et enfin de troubles graves.\par
Cette situation fournit au Résident de France à Genève, Félix Desportes, un excellent prétexte pour exciter au plus haut point les passions politiques dans la région de la Côte et surtout celle de Nyon. Les députés des communes eurent des entrevues ; Desportes alla même à Céligny – alors français – encourager les populations vaudoises à se jeter dans les bras de la France, à laquelle Genève venait d’être annexée. La plupart des localités envoyèrent alors à Lausanne des représentants avec des pétitions demandant l’abolition immédiate des droits féodaux. La Chambre administrative ne put que faire parvenir ces adresses au Directoire qui venait d’être organisé à Aarau en lui faisant part de la grande agitation qui existait dans une partie du Léman, agitation qui deviendrait sans doute dangereuse pour la sécurité du pays si on n’y mettait fin aussitôt par des mesures vraiment efficaces. Le gouvernement nouveau répondit qu’il ne désirait que le bonheur du peuple et que les perturbateurs seraient punis « suivant la rigueur des lois ».\par
Il n’y avait pas là de quoi mettre fin au mécontentement, d’autant plus que le « perturbateur » principal était le représentant officiel de la France à Genève. Il aurait fallu que les Conseils helvétiques eussent leur nuit du 4 août ; que poussés par le désir de mettre fin à une situation incompatible avec le nouveau régime, ils anéantissent dans un juvénile enthousiasme ce brandon qui menaçait de mettre le feu à l’édifice tout entier. Ils discutèrent au contraire pendant plus de six mois et donnèrent le spectacle d’une assemblée profondément divisée et dans laquelle les idées des différentes régions de la République ne pouvaient guère se concilier.\par
L’agitation ne fît donc que s’étendre. Le 13 mai, le Préfet du Léman, Henri Polier, annonça au Directoire qu’elle se montrait déjà dans le district de Morges, parcouru par de nombreux agents. Il fallait un prompt remède. Les seigneurs étaient disposés à faire des sacrifices. Necker, baron de Coppet, se déclarait prêt à réduire ses rentes de 40 000 à 20 000 francs ; d’autres suivaient cet exemple, parmi lesquels Henri Polier lui-même. De son côté, la Chambre administrative insista de nouveau un mois plus tard, recommandant le projet qu’elle n’avait pas eu le temps de mettre à exécution, espérant aussi que Ton accorderait une indemnité équitable aux propriétaires.\par

\begin{quoteblock}
\noindent « La bonne foi helvétique est notre premier bien, écrivait-elle. Les autorités que le peuple vaudois a choisies, conserveront intact ce dépôt sacré, plus glorieux pour le nom suisse chez les autres nations que notre réputation de bravoure. »\end{quoteblock}

\noindent Ce langage ne fut pas écouté par tous et, malgré les discours savants ou enthousiastes des députés vaudois, de Secretan surtout, les conseils continuèrent chaque jour leur discussion fastidieuse et décidèrent seulement que les récoltes de l’année pourraient être rentrées \emph{« en attendant le décret qui sera rendu sur les dîmes »\footnote{Décret du 8 juin 1798.} }\par
Ces atermoiements amenèrent de nouvelles manifestations de l’esprit public dans les diverses contrées du Léman, à Lausanne surtout, où la Société des amis de la liberté s’éleva avec force sous la direction de Louis Reymond contre « les restes honteux de la servitude féodale », et cela autant par son journal le \emph{Régénérateur} que par ses adresses enflammées aux Conseils législatifs. Le \emph{Régénérateur} fut supprimé par le Directoire, la Société des Amis de la Liberté fut dissoute et Louis Reymond, principal boute-en-train des mesures extrêmes, fut condamné à la prison.\par
On commençait à reconnaître la faute que l’on avait commise lorsque, au premier jour de la Révolution, on avait promis au peuple des campagnes des avantages considérables qui, pour lui, ne pouvaient être à ce moment là que d’ordre financier.\par

\begin{quoteblock}
 \noindent « Les plus légers soulagements qu’on lui eût faits sur les droits féodaux auraient (sans cela) excité sa reconnaissance, écrivait Polier ; aujourd’hui, tout ce qu’on lui en laissera lui paraîtra une injustice criante \footnote{Lettre au Directoire, 13 novembre 1798.}. »
 \end{quoteblock}

\noindent Le 10 novembre enfin, les Conseils votèrent une loi qui était un compromis entre la manière de voir des différentes parties de la République et des principaux partis et qui, par conséquent, n’était pas de nature à rallier tous les suffrages. Elle abolissait les redevances seigneuriales, mais distinguait entre les petites dîmes qui étaient supprimées sans indemnité et les grandes dîmes qui devaient être rachetées.\par
Cette loi, très compliquée dans ses détails, ne fut pas exécutée. Le gouvernement s’aperçut sans peine des difficultés considérables que présenterait sa mise en vigueur et en ajourna l’application. Le dernier mot n’était donc pas dit ; la question restait ouverte et devait être à la base de la politique vaudoise pendant plusieurs années.\par
Les circonstances au milieu desquelles le pays se trouva en 1799 contribuèrent encore à aigrir les esprits. Malgré les intrigues du parti réactionnaire, le canton du Léman supporta allègrement sa grande part des charges écrasantes résultant de la guerre de la seconde coalition, dont la Suisse allemande fut le théâtre principal. Ses bataillons allèrent généralement avec entrain combattre aux côtés des troupes de Masséna et méritèrent hautement les éloges de cet illustre homme de guerre « l’enfant chéri de la victoire. » Il fournit le quart des impositions générales de la République et ses citoyens et citoyennes se dévouèrent encore largement en faveur des orphelins du Valais et de la Suisse centrale. Ses hommes d’État patriotes déployèrent le plus grand zèle pour maintenir l’esprit public, mais ils étaient battus en brèche par les réactionnaires qui regrettaient l’ancien régime, et les exaltés – les anarchistes, disait-on alors – qui auraient voulu diriger le mouvement, et s’acharnaient tous à profiter des circonstances fâcheuses du moment pour rechercher l’amitié de la population.\par
Partagée entre ces deux courants contraires, cette dernière en arrivait ou bien à regretter LL. EE. ou bien à demander les mesures les plus extrêmes. Dans les régions du nord du Canton qui avoisinent le Jura, ou buvait à la santé de MM. de Berne, « ces bons et braves seigneurs ». « Oh ! la pauvre bête ! » disait-on en parlant de l’ours \footnote{Monnard XVI, 297-298.}. Les amis de la liberté de Lausanne, soutenus par les campagnards de la partie occidentale du Canton, continuaient en revanche à demander \emph{« l’abolition des droits féodaux sans rachat et se plaignaient, dans une adresse célèbre au Grand Conseil Helvétique du modérantisme qui pénétrerait jusque dans les enfers s’il pouvait en arracher des tisons propres à incendier sa patrie et à détruire ses défenseurs\footnote{Adresse du 10 juin 1799.}  »}.
\section[L’ « Adresse anarchique »]{L’ « Adresse anarchique »}
\noindent L’année 1800 s’ouvrit par le Coup d’État du 7 janvier, ensuite duquel le pouvoir fut enlevé au parti de La Harpe et donné à celui qui était accusé de « modérantisme ». Elle ne fut pas plus tranquille que la précédente. Les modérés voulurent compléter l’œuvre du 7 janvier, en faisant prononcer l’ajournement indéfini des Conseils législatifs. La République serait dans ce cas sous la direction d’un nouveau gouvernement provisoire dont le premier devoir consisterait à donner au pays une constitution mieux appropriée à ses mœurs et à ses habitudes.\par
La destitution des deux Directeurs du Léman, La Harpe et Ph. Secretan, et d’autres modifications dans le personnel administratif du canton vinrent encore s’ajouter à ces nouvelles causes de discussions, de divisions et de troubles. Beaucoup de communes se prononcèrent énergiquement en faveur du parti vaincu ; d’autres en grand nombre signèrent des adresses en sens inverse. Ces dernières furent mieux entendues à Berne, où siégeait alors le gouvernement central, et le Coup d’État du 7 août vint compléter l’œuvre du 7 janvier.\par
Le nouveau Conseil exécutif modéré annonça qu’il allait s’occuper aussitôt que possible de la question des dîmes, mais le peuple avait été déjà si souvent trompé dans son attente à l’époque où les patriotes étaient au gouvernail, qu’il ne crut pas aux promesses d’un gouvernement auquel il lui était impossible d’accorder une confiance complète.\par

\begin{quoteblock}
 \noindent « Si la loi continue à exiger le rachat, écrivait Polier au Pouvoir exécutif, elle ne l’obtiendra que par une force armée considérable et étrangère au canton \footnote{Lettre du 3 septembre 1800.}  »
 \end{quoteblock}

\noindent C’est en effet ce qui allait arriver peu de temps après. Malgré divers avis semblables à celui du Préfet Polier, le Corps législatif rapporta, le 15 septembre 1800, la loi du 10 novembre 1798 sur les dîmes, la considérant comme contraire aux principes de justice et tarissant une des principales sources du revenu public. Un second projet beaucoup moins favorable aux campagnards fut mis en discussion.\par
Le ministre de la Justice ne tarda pas, en conséquence, à recevoir de mauvaises nouvelles du Léman.\par

\begin{quoteblock}
 \noindent « Les délibérations du Corps législatif sur les dîmes et les censes occasionnent dans plusieurs contrées une grande fermentation, lui écrivait Polier, et les paysans paraissent déterminés à s’exposer à tout plutôt que de se soumettre derechef à ces charges odieuses. Il y a eu à Morges, le 24 [septembre], une assemblée fort nombreuse de cultivateurs et députés des communes qui se sont engagés réciproquement à ne point obéir à une telle loi, si elle était rendue \footnote{« L’exaspération contre le gouvernement fut telle dans cette assemblée que les délégués de la Côte déclarèrent que leurs commettants préféraient une réunion à La France plutôt que d’être aussi malmenés par des Allemands et des aristocrates ». Verdeil III, 388.}. Ces symptômes insurrectionnels inspirent de vives inquiétudes pour le maintien de la tranquillité publique et il paraît nécessaire, ou que le Corps législatif ne suive pas au plan qu’il paraît s’être proposé, ou que le gouvernement attire à lui des moyens assez puissants pour vaincre les résistances » \footnote{Lettre du 26 septembre.}.
 \end{quoteblock}

\noindent Le Corps législatif n’en continua pas moins son œuvre et, dans le canton du Léman, on procéda à la levée des censes arriérées et à celle d’une contribution du trois pour cent imposée par la Chambre administrative pour l’entretien des troupes françaises. Le Préfet Polier publia une proclamation très sévère à l’occasion d’une récente loi sur les associations politiques et des assemblées répréhensibles qui avaient eu lieu. Le calme sembla renaître un peu et se maintint pendant quelques semaines.\par
Cependant, le feu ne faisait que couver sous la cendre. Plusieurs des hommes auxquels les Coups d’État du 7 janvier et du 7 août avaient enlevé leurs emplois et leur influence dans les affaires générales du pays, d’autres qui étaient encore fonctionnaires cantonaux ou communaux, mais qui désapprouvaient la conduite des nouvelles autorités centrales, cherchaient à faire partager leur manière de voir par les populations et à tenir en échec les modérés que l’on accusait de vouloir ramener l’ancien ordre de choses. Ces derniers avaient déjà fondé en divers lieux des clubs ou cercles des « amis de l’ordre » ; les patriotes se groupèrent autant que possible de la même manière, mais réussirent à mettre un peu d’ensemble dans leurs mouvements.\par
La ville de Morges qui, dès le commencement de la Révolution, avait joué un rôle essentiel dans les progrès de celle-ci, fut encore, en 1800, le centre de ralliement du parti patriote qui y comptait plusieurs de ses chefs les plus importants. De là partirent les agents qui allaient transmettre dans la plus grande partie du canton et même plus loin encore, les conseils et les directions des hommes de confiance. Là aussi, arrivaient de divers côtés les représentants des communes, constituant parfois un vrai parlement régional dont les séances avaient lieu dans la maison de Henri Monod, qui avait rempli pendant plus de deux ans les fonctions de président de la Chambre administrative.\par
Le général français Turreau, connu par sa dureté et les mauvais traitements qu’il fit subir aux populations de diverses contrées, occupait alors le Valais avec des troupes de son pays. Il avait pour mission occulte de préparer l’annexion de ce canton à la France. Le bruit se répandit dans nos districts que le Léman était destiné à subir le même sort. L’alarme fut grande ; quelques autorités constituées, beaucoup de communes et un grand nombre de citoyens manifestèrent par des adresses au gouvernement helvétique leur volonté formelle de rester Suisses. Le parti patriote fut accusé par ses adversaires de rechercher au contraire l’annexion. Cette allégation était sans doute exagérée, mais il n’en est pas moins certain que quelques adresses furent rédigées dans ce but et signées clandestinement. Il est certain aussi que le même parti ayant réussi à obtenir l’émancipation politique du pays avec l’aide de la France, en 1798, crut pouvoir encore cette fois compter sur ce même pays pour arriver à son but, qui consistait à obtenir, par tous les moyens, le maintien intégral de son œuvre, de ses principes, de son influence et l’abolition des redevances seigneuriales. Placés entre « les baïonnettes et le désespoir », les campagnards se décidèrent enfin à manifester leur volonté et le parti avancé résolut de frapper un grand coup pour les attacher définitivement à sa cause.\par
\emph{« Il nous paraît une injustice criante de faire racheter des impositions dont la majeure partie a été établie par le fanatisme et la terreur »}, lit-on dans une adresse d’Ecublens au sous-préfet de Morges. Et les mêmes citoyens, après avoir montré leur détresse financière, insistaient en faveur de l’interdiction des droits féodaux \emph{« promise au peuple par des imprimés au commencement de la Révolution au su et au vu de l’Assemblée des représentants provisoires du Léman, sans empêchement de cette autorité alors souveraine \footnote{Adresse du 29 novembre 1800.}  »}\par
Les esprits étant ainsi bien préparés, le parti patriote fit signer dans quelques districts \emph{à l’Adresse des sous signés aux Autorités du Canton Léman}, qui devait avoir une influence considérable sur les événements et même les destinées du pays. La voici :\par

\begin{quoteblock}
 \noindent « La crainte de notre réunion à la République Française est aujourd’hui le mot d’ordre des ennemis de notre révolution ; cette crainte, vraie ou feinte, leur a fait naître l’idée d’en tirer parti pour consolider le gouvernement et lui faire connaître à quel point il peut hasarder ses entreprises contre la liberté. Tout est donc en rumeur dans cet instant ; les adresses fourmillent, les émissaires de vœux pour conserver le nom Suisse sont répandus à profusion ; mais ce qu’il y a d’étrange dans tout cela, c’est que les sollicitations des agents subalternes du Gouvernement se dirigent de manière à noter une partie des Citoyens, comme partisans de cette réunion qu’on paraît tant redouter, partie à laquelle on ne propose aucune souscription, et dont le silence sera interprété défavorablement.\par
 « Nous, tous membres des Communes du Canton du Léman, en Helvétie, voulons aussi émettre notre vœu ; nous voulons aussi en consigner l’acte authentique entre les mains des trois premières autorités de notre Canton que seules nous pouvons envisager comme constitutionnelles \footnote{Les nouvelles autorités centrales n’avaient pas été organisées et nommées d’après les formes prévues par la constitution de 1798, qui n’avait pas encore été remplacée.}.\par
 « Oui, nous le jurons à la face de l’Être Suprême ; oui, nous en attestons l’Univers ; oui, nous le déclarons sincèrement, et avec vérité à tous nos concitoyens :\par
 « Le nom de Suisses fut toujours celui que nous chérîmes ; perdre cette qualité nous serait infiniment douloureux ; nous signons le vœu de le conserver, et nous le scellerons de notre sang. Si le nom de Suisse doit être celui que doit porter un peuple libre et indépendant ; si ce peuple doit être régi par une constitution basée sur les principes de l’égalité et de la liberté ; si ce peuple ne doit jamais avoir sous les yeux l’odieux spectacle d’un régime arbitraire et contraire à la constitution qu’il a jurée ; si ce peuple est assuré que les magistratures quelconques ne deviendront point l’apanage d’un certain nombre de familles, contradictoirement à ses droits qui lui en donnent l’éligibilité indirecte ; si ce peuple, ballotté par des factions, ne voit pas des loix fondées sur les grands principes de son état politique tout à coup bouleversé, pour faire face à des arrêtés basés sur des principes absolument differens, et qui sembleraient provoquer cette réunion ; si, enfin, et sur toutes choses, ce peuple auquel on a promis si solennellement l’abolition des censes, des dîmes, et de toutes autres droitures féodales, qui tiennent de la barbarie et de l’esclavage, vient à jouir avec certitude de ces avantages, et qu’à cet effet tous les titres qui les constituent soyent lacérés et anéantis, sauf à indemniser les propriétaires par la vente des domaines nationaux, alors, nous le jurons ; nous sommes Suisses, et nous ne cesserons de l’être qu’avec l’existence ».
 \end{quoteblock}

\noindent Remarquons tout d’abord la demande des pétitionnaires tendant à ce que les titres et parchemins attestant le droit des seigneurs à percevoir les dîmes et censes fussent lacérés et anéantis. Ce désir, exprimé un an et demi avant le soulèvement des campagnards, montre qu’il était répandu déjà dans une grande partie du pays et même chez des personnes d’opinion modérée. On l’avait vu exprimé quelques semaines auparavant dans une lettre du citoyen Burnier, vice-président du Tribunal du Canton, adressée au Corps législatif. \emph{« Les documents, sans en conserver vestiges, seraient mis au néant »}, disait-il.\par
On peut aussi remarquer en parcourant la fin de ce document, une fixité remarquable dans la méthode préconisée par les chefs du parti patriote pour liquider la question des dîmes. On a vu plus haut, en effet, qu’au printemps de l’année 1798, la Chambre administrative du Léman n’avait manqué que du temps nécessaire pour supprimer par le même moyen le régime féodal. On retrouve du reste cette manière de voir dans un grand nombre de lettres du temps, particulières ou officielles, dans les journaux et plusieurs brochures. L’opinion publique, dans sa généralité, insistait pour qu’elle fût adoptée, et ce mouvement devint assez puissant pour prévaloir plus tard.\par
L’adresse reproduite plus haut était un acte politique d’une gravité toute particulière. C’était l’explosion d’un mécontentement contenu à grand’peine depuis quelques mois et dont la manifestation explique en partie comment les patriotes vaudois, fervents unitaires en 1798, devinrent plus tard – la plupart du moins – des cantonalistes convaincus.\par
Le parti patriote fut accusé par ses adversaires de vouloir préparer une réunion à la France. Il n’en était rien. Sans doute, cette idée était entrée dans l’esprit d’un nombre assez considérable de citoyens généralement peu instruits sur les affaires, générales ; sans doute aussi les intrigues de Desportes avaient laissé de profondes traces dans la partie occidentale du canton, et celles moins connues du général Turreau produisaient déjà quelque effet ; sans doute enfin, des agents français poussaient les citoyens dans cette direction, mais la grande masse du peuple et des chefs patriotes ne voulaient pas rejeter le nom Suisse. Ce qu’ils réprouvaient, c’était la réaction et la conduite politique du gouvernement central. Le parti patriote fut accusé par ce dernier d’avoir \emph{« adroitement supposé la réunion à la France dans le but anarchique de ne reconnaître pour constitutionnelles que les trois autorités du Canton Léman »}, soit le Préfet national, la Chambre administrative et le Tribunal du Canton. Ce que les patriotes voulaient, remarque Juste Olivier, c’était \emph{« défendre la révolution et l’achever, surtout par l’abolition des titres féodaux, afin de détruire du même coup l’influence de l’aristocratie en diminuant sa fortune et toute possibilité de retour sur un point essentiel, à l’ancien ordre de choses \footnote{J. Olivier : \emph{Hist. de la Révol. helv.} 221.}  »}.\par
Malgré les menaces du gouvernement central et les proclamations du Préfet national et du général Montchoisy, commandant des troupes françaises en Suisse, le mouvement devint de plus en plus grave et quatre-vingts communes donnèrent en peu de temps plus de 4000 signatures à l’adresse. Pensant être protégés encore par la Grande Nation, les meneurs ne tinrent pas compte du fait qu’ils étaient déclarés rebelles, et une centaine d’entre eux se présentèrent à l’audience de Henri Polier, le premier décembre, pour s’annoncer comme tels, malgré « l’arrêté de proscription qui pesait sur eux ».\par
C’était payer d’audace, mais ils se savaient soutenus par leurs concitoyens qui, laissant un peu de côté, cette fois, leur crainte instinctive de se compromettre, osaient enfin manifester leurs vœux.\par
Le six décembre, le Ministre de la Justice communiqua au gouvernement des nouvelles importantes du Léman.\par

\begin{quoteblock}
 \noindent « Parmi les meneurs, disait-il, d’après le rapport de Polier, il remarque quelques membres du ci-devant Conseil législatif, quelques fonctionnaires et surtout des membres des tribunaux de district. Il se plaint de quelques sous-préfets… entre autres ceux d’Aubonne et de Cossonay. Les anarchistes ont eu à Morges une assemblée tumultueuse, ensuite de laquelle ils ont planté l’arbre de la liberté devant le Cercle des amis du gouvernement qu’ils appellent le Cercle des \emph{Chouans} ; ils ont enterré le quarteron, la raclette et d’autres instruments de la recouvre des cens et n’ont rien négligé pour exciter le peuple contre les lois et l’autorité du gouvernement. Ce qui leur a donné le courage de se porter à ces mouvements est une lettre du président de l’administration, Monod, actuellement à Paris, qui leur promettait l’appui de la France pourvu qu’ils allassent en avant \footnote{Communication du 6 décembre 1800.}  ».
 \end{quoteblock}

\noindent Le Préfet National prononça peu de temps après la dissolution du Club de Morges, « centre et point de ralliement des anarchistes, auteurs, fauteurs, prédicateurs et colporteurs de l’adresse ». Ils avaient déployé en dehors de la croisée de leur salle de réunion donnant sur la rue, un grand drapeau helvétique, à l’imitation de l’Assemblée provisoire du Pays de Vaud et des Conseils suprêmes. A Morges, à Cossonay et ailleurs, des rassemblements avaient eu lieu aussi « dans certains cabarets d’où l’on sortait plus acharné que jamais \footnote{Lettre du 27 décembre 1800, écrite par Polier.}  »,\par
Des agents continuaient à parcourir le canton, de même que celui de Fribourg et quelques régions plus éloignées de Berne, Soleure et Argovie. Le Préfet de Fribourg apprit que les communes de son ressort devaient envoyer des délégués à Payerne le 19 décembre, pour y délibérer sur les « moyens d’appuyer les réclamations du Canton du Léman ». L’Assemblée ne put pas avoir lieu, paraît-il.\par
Sous la haute et mystérieuse direction de quelques hommes influents, les principaux agents et meneurs signalés étaient Jaquerod, agent de Villars-sous-Yens ; le juge Rouge à Lausanne ; Bonnard de Lausanne ; le fougueux Claude Mandrot, membre du tribunal de Morges ; Samuel Clerc ; le juge Epars de Cossonay ; les sous-préfets Vionnet, à Aubonne, et Duchat, à Cossonay, etc. ; la plupart se retrouveront au premier rang en 1802. On craignit un mouvement contre l’arsenal de Morges qui renfermait 56 canons et dont le gardien Guibert, était, selon le Préfet Polier, « un anarchiste prononcé ».\par
Le gouvernement déploya quelque énergie pour réprimer ce mouvement, et les représentants de la France se montrèrent très corrects en mettant à sa disposition leur autorité militaire et morale. Le général Montchoisy rappela le commandant de la place de Lausanne, le citoyen Lecorps, « homme très dangereux » par ses relations avec les « anarchistes » et y envoya le général Guétard accompagné de nombreuses forces. Ce dernier publia le 18 décembre une proclamation par laquelle il détruisait les illusions de ceux qui avaient cru que les Français les soutiendraient. La fermeté de cet officier et sa volonté fermement arrêtée et proclamée de rétablir l’ordre, contribuèrent grandement à calmer les esprits.\par
L’accusateur public, Auguste Pidou, et le Tribunal du Canton furent invités à poursuivre les auteurs et les colporteurs de l’adresse. Ils agirent avec tant de lenteur, de mollesse ou de mauvais vouloir ; ils montrèrent tellement, par leur conduite, qu’ils ne voulaient pas sévir contre des amis, que le gouvernement destitua l’accusateur public, plusieurs membres du Tribunal du Canton, accepta la démission de Henri Monod, président de la Chambre administrative, et fit remplacer un grand nombre de fonctionnaires subalternes, entre autres les sous-préfets d’Aubonne et de Cossonay. Les chefs ostensibles du mouvement furent sérieusement recherchés ; un certain nombre – Rouge, Bonnard, Clerc, Epars, Durand, etc. – se réfugièrent à Genève ; d’autres, plus nombreux, furent emprisonnés. Une enquête judiciaire fut instruite ; elle ne tarda pas à prendre des proportions considérables à cause de la tactique des inculpés qui consista à compromettre le plus grand nombre de personnes, afin de rendre la procédure d’autant plus difficile, embrouillée et illusoire.\par
Des contingents de troupes françaises furent envoyés dans les communes qui avaient participé au mouvement. Elles en imposèrent aux populations pour lesquelles elles constituaient une charge nouvelle et assurèrent la rentrée de toutes les impositions arriérées. On obtint enfin de plusieurs communes des adresses de rétractation, et toute cette affaire, qui avait troublé profondément le pays, se termina bientôt par un oubli général et officiel de ce qui s’était passé.\par
Cette solution ne pouvait pas être définitive ; la situation économique du cultivateur restait la même ou plutôt ne faisait que s’aggraver ; les causes de mécontentement subsistaient toujours et ne devaient pas tarder à se manifester de nouveau.
\section[Causes et préparatifs]{Causes et préparatifs}
\noindent Le calme qui succéda pendant quelques mois à cette agitation extrême fut tout de surface et dû à la seule contrainte. On put bientôt distinguer deux courants d’opinion au milieu de la masse des mécontents.\par
L’un de ces courants renfermait toutes les personnes qui désespéraient de pouvoir jamais obtenir un régime politique convenable tant que l’on resterait uni à la République helvétique. Le Canton du Léman ne pouvant pas à lui seul constituer un État suffisamment fort pour faire reconnaître et respecter son existence nationale, ils estimaient que sa réunion à la France était probablement la meilleure solution à apporter à une situation de jour en jour plus mauvaise. Les hommes au courant des affaires générales cherchaient inutilement à leur faire comprendre que si, de cette manière, on obtenait l’abolition complète des dîmes, cet avantage serait compensé et bien au-delà par les charges immenses, civiles et militaires, qui pesaient sur le peuple français et que notre pays était incapable de supporter. Les partisans de l’union avec la France étaient trop aigris pour écouter les conseils de la raison. Ils préféraient s’en rapporter aux affirmations d’agents de la Grande Nation qui espéraient, en préparant cette annexion, mériter la reconnaissance du gouvernement consulaire. Voici, à ce sujet, ce que l’on mandait de Nyon au mois de mars 1801 au Préfet national.\par

\begin{quoteblock}
\noindent « J’ai l’honneur de vous prévenir que je viens d’apprendre par une voie sûre qu’il y a eu un rassemblement de nos fugitifs à Carouge, où s’est trouvé Muret, d’Aubonne, ci-devant inspecteur général, de retour de sa mission à Paris ; Rolaz-Croisier, de Gilly, etc., de même que nombre de députés de toutes les villes de Lausanne en ça, ainsi que de plusieurs villages ; toutes leurs démarches tendent à la réunion de leur patrie à la France. Les courriers révolutionnaires vont en mission tous les jours et la trame qui doit nous perdre s’ourdit tous les instants. »\end{quoteblock}

\noindent Le représentant de la Suisse à Paris, Ph.-A. Stapfer, confirmait le 21 mars à Bégoz, ministre helvétique des relations extérieures, les intrigues des partisans de la France et s’élevait avec force contre leur manque de patriotisme. Ces tentatives n’étaient pas destinées, heureusement, à obtenir de succès, d’autant plus que le gouvernement consulaire ne leur était pas favorable et qu’elles n’étaient encouragées que par une faible minorité des citoyens vaudois.\par
L’autre courant politique renfermait tous ceux qui croyaient le nouveau régime incapable de donner à l’Helvétie le repos, la tranquillité et le bonheur dont elle avait joui auparavant pendant plusieurs siècles. Ils avaient tressailli de joie en 1798 lorsqu’on avait fait miroiter devant leurs yeux les avantages nombreux qu’ils devaient retirer de la Révolution ; ils avaient espéré encore, lorsque les premières difficultés et les premières désillusions étaient survenues. Trois années s’étaient écoulées depuis lors et la situation, déjà mauvaise auparavant, était devenue presque insupportable. Au milieu de la lutte violente des partis, de l’épuisement de plus en plus complet du pays, dans un moment où le gouvernement demandait de nouveaux impôts considérables venant s’ajouter aux anciens, levés avec l’aide de troupes étrangères, le présent et surtout les perspectives affreuses de l’avenir, faisaient songer au passé et poussaient les populations à désirer son retour. \emph{« Tombons à genoux et invoquons l’ours I »} disait J.-J. Cart, de Morges, en protestant contre les actes du gouvernement. Les citoyens en arrivèrent en effet à « invoquer l’ours » et à demander la « réunion au canton de Berne ». Des pamphlets de toute nature circulèrent dans les campagnes et, malgré les mesures prises par l’administration cantonale, ils eurent suffisamment de succès pour qu’une pétition couverte d’environ 18 000 signatures – la moitié du nombre total des citoyens – vînt soutenir cette manière de voir pendant l’hiver de 1801 à 1802.\par
La question des dîmes avait cependant attiré dans l’intervalle l’attention des autorités centrales et cantonales. La Diète vaudoise de l’été 1801, composée en grande partie de patriotes modérés, n’avait malheureusement pas su prendre une décision franche et définitive capable de mettre fin à l’agitation. « Toute redevance est rachetable, » disait-elle ; ce n’était pas suffisant.\par

\begin{quoteblock}
 \noindent « Pressera-t-on le peuple chaque année entre les baïonnettes et le désespoir ? demandait à ce sujet J.-J. Cart. Renouvellera-t-on chaque année les scènes déchirantes dont nous avons été les témoins et ne craindra-t-on pas que le développement répété d’une force étrangère contre le vœu général ou ne nous plonge dans les horreurs de la guerre civile ou ne détruise pour toujours le peu d’esprit public qui nous reste \footnote{J.-J. Cart : \emph{Quelques mots sur le projet de la Commission diétale du Canton de Vaud.}}  ? »
 \end{quoteblock}

\noindent La Diète helvétique, de son côté, avait montré en automne 1801 le spectacle de la plus profonde division. Les députés vaudois protestèrent surtout contre la prétention de l’Assemblée de vouloir faire retomber les charges du rachat sur les seuls débiteurs, et Louis Secretan s’abstint dès lors d’assister aux séances de la Diète dont les délibérations ne devaient aboutir qu’à une tentative de réaction, marquée par l’arrivée au pouvoir des fédéralistes avec Aloïs Reding comme Landammann de la Suisse.\par
Ce dernier événement fut un coup de fouet pour le parti patriote dans le Léman. La situation politique devenant très grave pour ce canton, sérieusement menacé d’un retour à l’ancien régime, les partisans du nouveau cherchèrent encore une fois à rallier les campagnards – dont un grand nombre étaient indécis – en se servant du levier puissant de l’intérêt et de la suppression des droits féodaux.\par
Il ne lui fut pas très difficile d’arriver à ce résultat.\par

\begin{quoteblock}
 \noindent « Il semblait qu’on eût cherché à les pousser à bout, dit Henri Monod en parlant des paysans. Après avoir aboli et rétabli les droits féodaux ; après avoir modifié, changé et rechangé plusieurs fois les lois à ce sujet, on fait payer coup sur coup les nouvelles impositions générales, puis en différents endroits les impositions locales et l’on ordonne d’acquitter dans un terme donné les redevances féodales arriérées, dont la perception avait été comme abandonnée pendant deux ou trois ans. Il résulta de ces mesures précipitées et vraiment inconcevables, que le campagnard se trouva chargé d’une telle masse de dettes à payer tout à coup qu’il devint impossible à la plupart de se suffire. On vit, dans un seul tribunal, une cinquantaine de saisies de fonds demandées le même jour pour défaut de payement des droits féodaux ; une commune peu considérable fut attaquée en droit pour le payement de ses censes retardées dont une seule année montait au moins à cinq ou six cents louis. Ajoutez à ces motifs d’exaspération le mécontentement occasionné par l’expulsion hors du gouvernement des représentants du Pays de Vaud qui avaient la confiance, et par là même le manque absolu de confiance dans les chefs de l’État \footnote{Monod : \emph{Mémoires} I, 204-205.}. »
 \end{quoteblock}


\begin{quoteblock}
 \noindent « En effet, ajoute Verdeil, Muret, Secretan, J.-J. Cart, Bourgeois, Lafléchère étaient exclus des Conseils helvétiques ; les tribunaux où siégeaient Pidou, Potterat, Agassiz, Duchat, Claude Mandrot, Soulier, Dentan, Jan de Châtillens et d’autres, étaient cassés ; les sous-préfets patriotes comme Bergier de Jouxtens, Vionnet d’Aubonne, Correvon de Martines, d’Yverdon, étaient destitués ; l’inspecteur des milices, Muret-Grivel, était révoqué » \footnote{Verdeil III, 398.}.
 \end{quoteblock}

\noindent Les Vaudois étaient mécontents de voir les charges de la République retomber le plus souvent sur eux.\par

\begin{quoteblock}
 \noindent « Leur territoire, qui avait été le premier envahi, dit le réactionnaire Seigneux, fut aussi le premier obligé d’entretenir l’armée française dont les demandes de tout genre étaient exorbitantes. Ensuite, l’opposition continuelle de la part des autres cantons contre le nouvel ordre de choses, fit encore retomber la plus grande partie des impôts et des réquisitions sur le Pays de Vaud. Enfin lorsque la moitié de la Suisse se fut rangée du côté des armées autrichiennes et russes, cette même province eut à faire face, presque à elle seule, aux frais de la guerre parce que les hommes placés à la tête de son administration étaient forcés de tout sacrifier à la cause qu’ils voulaient défendre. C’est pourquoi, tandis que les autres Confédérés ne fournissaient que lentement une partie de leurs contingents en hommes et en argent, les chefs du Canton de Vaud, poussés par la crainte d’une contre-révolution, allaient au devant de tous les besoins du nouveau gouvernement et faisaient les plus grands efforts pour le soutenir » \footnote{Seigneux : \emph{Précis historique} II, 63.} 
 \end{quoteblock}

\noindent J.-J. Cart se trouve d’accord sur ce point avec le réactionnaire Seigneux :\par

\begin{quoteblock}
 \noindent « Si l’on considère, dit-il, que toutes nos communes ont été obligées depuis quatre ans à des dépenses décuples de leurs dépenses ordinaires ; que l’on a exigé de nos cultivateurs, les censes et les dîmes, arriérées depuis la révolution, outre les impôts qu’ils n’avaient jamais payés ; que la plupart sont abîmés de dettes et harcelés par leurs créanciers, l’on trouvera un peuple au désespoir et dans ce désespoir les causes de l’insurrection ».\par
 « Et malgré la révolution dont on se promettait tant, disait-il encore, l’on exigera des Vaudois, dîmes, censes et impôts. On l’exigera tandis que les habitants des petits Cantons n’en payent point ; nous serons ainsi leurs tributaires. Jamais, jamais, et quoi qu’il arrive, jamais ! \footnote{J.-J. Cart : \emph{De la Suisse avant la Révolution et pendant la Révolution.}}  »
 \end{quoteblock}

\noindent Beaucoup pensaient comme J.-J. Cart, et l’on trouve dans les extraits précédents, reflétant l’opinion de divers partis, les causes de l’adresse aux autorités du Canton du Léman et, plus tard, de l’insurrection des \emph{Bourla-Papey.}\par
Les nombreux fonctionnaires patriotes destitués cherchaient à opérer une réaction contre le vent d’aristocratie qui semblait souffler sur la Suisse depuis quelque temps. L’ambition d’un certain nombre, le ressentiment de quelques autres, le mépris que l’on affichait pour un gouvernement central sans force, sans énergie, sans ligne fixe de conduite, toutes ces choses facilitèrent l’explosion. A l’époque de l’Adresse aux autorités du Canton du Léman, on avait appris à correspondre entre communes, on avait eu des conciliabules secrets ; des agents avaient parcouru le pays. Cette organisation savante et occulte continua à subsister malgré tout et facilita l’organisation du soulèvement de 1802.\par
Henri Monod dit dans ses \emph{Mémoires} que le pays était persuadé que \emph{« toute résistance au gouvernement aurait l’appui de la France \footnote{Monod : \emph{Mémoires} I, 205.}  »} Cette opinion fut soutenue, encouragée et fortifiée par des agents français. Le général Turreau, entre autres, chargé de préparer l’annexion du Valais, se distingua par ses intrigues dans ce sens et eut avec les mécontents du Pays de Vaud, des relations secrètes et fréquentes qui poussèrent les plus exaltés à demander la réunion à la France si on ne faisait pas droit à toutes leurs réclamations. Ces intrigues furent d’autant plus importantes et fructueuses qu’elles restèrent pendant trop longtemps imparfaitement connues et que le gouvernement helvétique se trouvait dans l’impossibilité de les empêcher complètement. Un des commissaires envoyés dans le Canton du Léman, Frédéric May, dévoila tout à fait les actes de cet homme néfaste. a Un général français, disait-il, connu par les cruautés qu’il a exercées dans la Vendée et, depuis deux ans, le fléau du Valais, espérant se rendre agréable à son gouvernement s’il pouvait parvenir à réunir à la France une des plus belles contrées de l’Helvétie, attisait le feu de la discorde, entrait dans les plans des premiers moteurs et faisait espérer l’impunité et même l’appui de nos puissants voisins » \footnote{Rapport du 14 juillet 1802 au Conseil Exécutif.}.\par
A l’époque du gouvernement de Reding, les patriotes de divers cantons eurent des conciliabules pour rechercher les moyens de le renverser. Une réunion importante eut lieu à Payerne et décida qu’une insurrection aurait lieu au printemps dans le but de placer à la tête du gouvernement central des hommes favorables aux nouvelles idées. Les paysans vaudois furent encouragés à soutenir ce mouvement en s’attaquant aux propriétés des seigneurs et en détruisant les titres féodaux.\par
L’entreprise était aussi importante que grave. Qui prendrait la direction du mouvement ? Qui voudrait bien se compromettre publiquement et au grand jour, dans un pays où l’on avait appris depuis des siècles à cacher ses idées et à agir dans l’ombre ? On sentait ce qu’il y avait de répréhensible dans l’envahissement des propriétés particulières, on avait conscience de la gravité de ses actes et des chances contraires que l’on avait à courir. On était suffisamment aigri et mécontent pour vouloir faire quelque chose ; on voulait arriver au résultat par les moyens préconisés, mais qui commencerait, d’où partirait le signal ?\par
C’est dans la région qui s’étend de Morges à La Sarraz que la « fermentation » était la plus grande, sans cependant que les agents du gouvernement eussent la possibilité de citer des faits. Des personnes dont on ne se défiait pas avaient cependant aperçu divers mouvements suspects dans les villages. On avait vu souvent Claude Mandrot, ex-juge, au tribunal de Morges, venir rejoindre les frères Bourgeois à St-Saphorin et se rendre avec ces derniers auprès de Marc Cart, dans le même village. Là, avaient lieu de longs conciliabules. Cart s’en allait ensuite à cheval du côté de Cossonay et de La Sarraz et ne revenait qu’au bout de quelques jours. Sa femme s’exprimait d’une manière très véhémente au sujet des aristocrates et excitait les voisins :\par

\begin{quoteblock}
\noindent « Laissez-les faire, disait-elle, quand on aura fait le tour de Lausanne, nous voulons engraisser ici les roues de nos chars avec le sang des aristocrates. »\end{quoteblock}

\noindent On savait que l’ancien juge du Canton, Potterat d’Orny, personnage influent dans la région de La Sarraz, suivait de près les événements du temps ; il avait assisté à l’assemblée de Payerne et il était au courant de toutes les intrigues. Là encore quelquefois, les femmes se laissaient entraîner par la passion politique. Ursule Monnier d’Eclépens était continuellement « par voies et par chemins », elle « courait les villages » et transmettait des avis et des ordres.
\section[La Sarraz et Bière]{La Sarraz et Bière}
\noindent Ces intrigues étaient peu connues dans le public, celui des villes notamment, et le \emph{Nouvelliste Vaudois} insérait dans son numéro du 5 février 1802, les lignes suivantes :\par

\begin{quoteblock}
 \noindent « Le Pays de Vaud jouit de la plus grande tranquillité ; s’il y existe encore des partis divers, il n’en est aucun qui ne sente manifestement qu’une insurrection quelconque ne profiterait qu’aux ennemis du nom Suisse, et que, pour en être dupe, il faut ne pas voir au delà de son nez ou de l’heure présente ; on peut même dire que jamais les partis n’ont été plus près de se rallier pour la cause commune que dans ce moment ».
 \end{quoteblock}

\noindent Il semble sans doute résulter de la lecture de ces lignes la conviction que le rédacteur connaissait un peu les intentions des meneurs, mais son étonnement fut cependant assez grand sans doute lorsque, le 23 du même mois, il dut annoncer à ses lecteurs que dans la nuit du 19 au 20, le \emph{« ci-devant château de La Sarraz et la chambre forte renfermant les archives de la baronnie »} avaient été forcés et les titres des droits féodaux enlevés et anéantis pour la plupart.\par
Cette nouvelle fut un coup de foudre pour beaucoup de personnes, pour les autorités, pour le gouvernement. Le sous-préfet de Charrière, à Cossonay, averti aussitôt, fit procéder sans retard aux constatations légales.\par
Ces dernières donnèrent la certitude qu’un grand nombre de personnes avaient dû coopérer à l’événement. La grande porte, dite la Pothélaz, était forcée et la serrure enlevée. Les envahisseurs avaient ensuite traversé un fossé et escaladé un mur. Ils étaient enfin parvenus par une fenêtre dans la petite cour ; de là, ils étaient descendus aux archives. Ces dernières étaient renfermées dans une chambre protégée par une porte en fer ayant une serrure fermant à double tour et un cadenas. Le tout était forcé ; les titres avaient disparu à l’exception des papiers de famille. Des débris de parchemins et de papiers se trouvaient encore dans la « maison du tirage », d’autres peu nombreux dans la Venoge. Les envahisseurs avaient donc pris le temps de faire un choix dans les archives et une partie de leur larcin avait fourni les éléments d’un auto-da-fé, allumé sur la colline du Mauremont.\par
Le Préfet national du Léman publia le même jour, 20 février, une proclamation dans laquelle il faisait voir la gravité du crime qui venait d’être commis.\par

\begin{quoteblock}
\noindent « Un tel événement, disait-il, par la violation de la propriété…, par la nature du vol, par le grand nombre des intéressés, étant le premier de ce genre qui ait souillé la révolution, excitera sans doute l’indignation de tous les bons Vaudois. »\end{quoteblock}

\noindent Il dévoilait ensuite les agissements de ceux qui persuadaient aux paysans que\par

\begin{quoteblock}
 \noindent « la France n’était point disposée à soutenir notre gouvernement. »\par
 « Je vous déclare, ajoutait-il, que c’est là une insigne calomnie, démentie à ce moment même par la présence des brigades françaises dans les communes du canton de Zurich qui ont opposé de la résistance. »
 \end{quoteblock}

\noindent Il priait enfin les citoyens de la baronnie de La Sarraz de montrer par une adresse qu’ils réprouvaient l’événement de la nuit précédente.\par
Le secret fut bien gardé et, d’autre part, c’est à peine si cinquante-deux citoyens signèrent l’adresse sollicitée par le premier magistrat du canton. Une enquête fut ouverte et quarante-six personnes interrogées inutilement.\par
Les signataires de l’adresse de La Sarraz promirent 50 louis de récompense à ceux qui feraient connaître les auteurs de la spoliation ; la famille de Gingjns s’engagea pour la même somme et le gouvernement helvétique pour 600 francs, en assurant le secret aux dénonciateurs. Ces offres importantes ne furent pas capables de décider quelqu’un à rompre le silence inquiétant qui continua à planer sur cette affaire.\par
Les personnes compromises étaient nombreuses et l’événement tenait à un plan dont les autorités ne pouvaient que présumer l’existence. Le gouvernement central, pressentant l’importance du fait, chargea encore Polier, au commencement de mars, de faire procéder à une enquête complémentaire par deux hommes de confiance. Cette dernière mesure resta aussi sans résultat. Quelques jours auparavant, ce gouvernement ayant appris que l’on signait \emph{« des adresses de réunion à la France »} avait déjà ordonné au même fonctionnaire de mettre la plus grande diligence \emph{« à découvrir les auteurs de ces criminelles entreprises. »}\par
L’événement de La Sarraz faisait encore l’objet des conversations du public lorsque Polier reçut du sous-préfet d’Aubonne l’avis suivant :\par

\begin{quoteblock}
 \noindent « Le citoyen Monod, chargé des affaires du citoyen Necker (seigneur de Bière) me remet un verbal qui constate qu’on est entré dans la maison du citoyen Necker à Bière ; la chambre des archives où déposaient les grosses, plans et cottets, relatifs au territoire de Bière et Bérolles a été forcée et vidée de tous titres. Il y a eu effraction de la serrure ».
 \end{quoteblock}

\noindent Le sous-préfet se rendit aussitôt à Bière où les titres déposés dans la maison communale avaient aussi été enlevés. Le château était inhabité depuis longtemps et dans un état de grand délabrement ; on ne put en conséquence spécifier d’une manière certaine la date du vol qui pouvait avoir eu lieu dans l’intervalle entre le 27 février et le 17 mars.\par
Le ruisseau du Flon entraîna des débris de papiers. Une enquête fut instruite à Bière et à Bérolles, par les soins du sous-préfet ; une récompense de six cents francs et le secret furent de nouveau promis par le gouvernement à tout dénonciateur. Le silence le plus complet continua, malgré tout, à planer sur les spoliations de Bière et La Sarraz.\par
Cette conspiration du silence, qui peignait bien la situation et qui constitue une partie de l’originalité du mouvement insurrectionnel de 1802, ne pouvait cependant tromper l’ensemble du public. La plupart des campagnards des districts de Morges, Cossonay, Aubonne, Orbe, Rolle, et même Nyon et Yverdon, étaient au courant des intrigues et si un certain nombre de personnages importants agissaient de manière à ce que l’on ne pût pas les accuser de participer aux événements, les chefs secondaires, les agents les plus actifs étaient connus. C’est ainsi qu’un jeune homme des environs de Cossonay écrivait les lignes suivantes dans son journal, à la date du 20 février :\par

\begin{quoteblock}
 \noindent « Le fournier qui est revenu de La Sarraz a rapporté que, la nuit dernière, on a forcé les archives du château et jeté à la Venoge une partie des papiers du baron de Gingins. Le reste a été porté sur le Mauremont où ils en ont fait un feu de joie. Cardinaux, de Mont-la-Ville, qui est à la tête du mouvement, a fait signifier au syndic d’avoir à transporter à Lussery les titres des archives, que sans cela il viendrait lui-même avec ses \emph{Bourla-Papay}. Pendant qu’on délibérait sur ce qu’il y avait à faire, quelques hommes, conduits par l’envoyé de Cardinaux, ont forcé la maison de commune et vidé la caisse des archives. Le lieutenant Devenoge a réuni tous les titres seigneuriaux dans une serviette et l’on s’est rendu entre Lussery et Villars où étaient déjà amoncelées les archives de quelques villages avoisinants. On en a fait un feu de joie et cependant bien des cœurs, à ce qu’il m’a paru, étouffaient des regrets.\par
 « Au retour, j’ai trouvé tout le village sur pied, et Pied-fin sur le toit de la tour, où il était monté pour abattre la girouette. Quand il l’a eu détachée à grand’peine…, il l’a jetée dans la rue. « Voilà où je te voudrais, Dégenève\footnote{Receveur de M. de St-Saphorin.}, avec toutes mes nielles » a-t-il dit. J’en ai eu du regret et d’autres avec moi, parce que ces bravades ne servent de rien et que M. de St-Saphorin a toujours été bon pour le paysan. On sera cependant content si on peut s’entendre avec le gouvernement pour n’avoir plus de dîmes à payer \footnote{\emph{In memoriam}, Jean-Daniel, 3\textsuperscript{e} édition, p. 26-28.}  ».
 \end{quoteblock}

\noindent Ce récit montre que les paysans, tout en gardant souvent des sentiments de respect pour leurs anciens seigneurs, voulaient absolument la fin du régime féodal. Leur résolution ferme et inébranlable était capable d’inspirer de la frayeur au reste de la population, qui regarda en spectatrice ce mouvement étrange de 1802 sans vouloir l’entraver par une dénonciation quelconque auprès de l’autorité, ignorante du détail des faits.
\section[Attente et intrigues]{Attente et intrigues}
\noindent Le moment vint cependant où le mystère ne pouvait plus être complet.\par
A la fin de mars, le bruit se répandit dans le public qu’un complot allait éclater sous peu avec le secours d’un général français dont le nom était caché soigneusement. Le Préfet national apprit dès le 28 qu’il devait lui-même être arrêté en même temps qu’un certain nombre d’autres propriétaires de dîmes. Il fallait même s’attendre à ce qu’il y eût quelques victimes du ressentiment populaire contre « l’oligarchie cantonale \footnote{Le sous-préfet d’Orbe, Thomasset, à Polier, 31 mars.}  » Les personnes arrêtées devaient être conduites au château de Chillon. Les archives déposées dans la tour de la cathédrale allaient être enlevées et l’on voulait probablement profiter de cette occasion pour demander l’annexion à la France. Les meneurs promettaient à ceux qui voudraient bien les suivre, le pillage des châteaux, des maisons des aristocrates et de celles appartenant aux membres du parti modéré que l’on appelait alors le parti Polier.\par
Les sous-préfets d’Orbe et de Cossonay signalaient les courses nombreuses de quelques meneurs. Un seigneur – Pillichody de Bavois – annonçait que « 600 hommes étaient armés pour cette entreprise, que les nommés David et Duchat la dirigeaient dans le district de Cossonay, Mandrot à Morges, Potterat à Orny, Rouge à Lausanne \footnote{Les \emph{Bourla-Papay. Journal de Ami Rigot de Begnins. Gaz. de Lausanne.} 1857.}  ».\par
L’insurrection devait probablement éclater le samedi suivant, 3 avril.\par
Ces indications n’étaient basées que sur des rapports un peu vagues et ne permettaient pas au Préfet national de prendre des mesures précises contre une personne quelconque. Polier put néanmoins dès ce moment signaler au gouvernement central l’activité du capitaine Louis Reymond qui se trouvait alors à Lausanne, mais dont les actes n’étaient pas encore de nature à légitimer une arrestation.\par
En présence d’indications aussi peu précises, le gouvernement ne crut pas devoir accorder une importance considérable aux rapports de Polier. Il envoya cependant dans le Pays de Vaud quelques détachements de troupes françaises et helvétiques qui furent disséminées dans les districts où l’agitation semblait être la plus grande. Il recommanda enfin au Préfet de montrer de la vigilance, de la vigueur et de la fermeté, et il attendit des rapports plus précis.\par
Le soulèvement annoncé pour le trois avril n’eut pas lieu. Soit que les intrigues politiques qui s’organisaient à ce moment-là à Berne contre le gouvernement de Reding fussent connues des meneurs ; soit, surtout, que la présence des troupes françaises inspirât quelque crainte, le mot d’ordre fut de rester tranquille provisoirement. \emph{« Nous ne ferons rien tant que ces gens-là sont ici »}, disaient les citoyens de Cossonay en parlant des hussards français.\par
Les campagnards s’impatientaient de leur côté et, à chaque instant, l’annonce d’un soulèvement circulait dans le public. C’est ainsi que le 6 avril, le citoyen Duplan, de Cossonay, fit la déclaration suivante : « Dimanche dernier, environ les six heures de l’après-midi, étant au bouchon \emph{(sic)} de Cossonay, le fils de Jaques Epars me dit que je ne devais pas avoir peur si j’entendais sonner les cloches. Puis il a ajouté qu’on devait se rendre à St-Sulpice mercredi dans la nuit, qu’il y aurait un grand feu et qu’il devait s’y trouver deux généraux français pour commander les troupes, qui ne seront pas des traîtres comme Montchoisy… Que les individus qui devaient se rendre à St-Sulpice venaient de Goumoens et Oulens et des villages de la baronnie de La Sarraz, qui prendraient ceux de Daillens et de Penthaz en passant et qu’eux mêmes venaient de Grancy et Cottens pour donner l’alerte \footnote{Le 8 au soir, il y eut en effet un commencement d’alerte à Cossonay où la cloche du feu fut sonnée avec le secours des citoyens Duchat, ex-sous-préfet et Solliard, ex-greffier. Les jeunes gens de Penthalaz survinrent bientôt avec des gourdins qui remplaçaient, pour les initiés, les pompes à feu. L’affaire fut manquée.}. »\par
Cette déposition, et le fait de l’inaction des mécontents au commencement d’avril, montre que malgré les nombreux agents dont disposaient les meneurs et la bonne volonté des campagnards, leur parti manquait un peu de cohésion et qu’aucune personne ne possédait une autorité reconnue par tous. Les plus zélés dans chaque contrée semblaient agir souverainement sans trop se préoccuper de la coordination à établir entre tous ces efforts locaux ou personnels. Le complot était connu de tous, mais l’exécution était encore subordonnée dans chaque région à diverses circonstances particulières.\par
Pendant ce temps, les sous-préfets faisaient leur possible pour se renseigner.\par

\begin{quoteblock}
\noindent « Les meneurs vont et viennent, \emph{écrivait de Charrière, à Cossonay}. Il est difficile de savoir ce qu’ils font, malgré une surveillance active. Outre les patrouilles de la cavalerie, trois hommes sûrs sont envoyés pendant la nuit sur un lieu élevé pour voir et écouter. »\end{quoteblock}

\noindent Le sous-préfet d’Orbe, Thomasset, insistait en faveur de la suppression des dîmes, \emph{« seul et unique moyen d’obtenir la tranquillité. »}\par
S’il en faut croire le \emph{Journal} de Ami Rigot de Begnins, J.-J. Cart aurait remis le trois avril à M. de St-Saphorin – un des principaux seigneurs du pays – un Mémoire \emph{« sur le rachat des dîmes et censes que l’on opérerait au moyen des biens nationaux et d’un impôt additionnel. »} Il ajoute que \emph{« les citoyens Rouge, Mandrot et d’autres avaient représenté au même seigneur a que l’adoption du projet Cart était le seul moyen de sauver notre pays »}. Il eût été prudent, sans doute, de la part des propriétaires de fiefs de négocier eux-mêmes la liquidation de leurs droits seigneuriaux. Et cependant, plus approchait le moment où ils allaient en être dépouillés violemment, et plus aussi ils s’y montraient attachés. Le revirement politique qui s’était produit en Suisse depuis plus de deux ans en faveur des idées modérées leur avait persuadé que les droits seigneuriaux n’étaient plus sérieusement menacés. L’arrivée de Reding au pouvoir avait inspiré une confiance plus grande encore et, malgré le mécontentement qui grandissait chaque jour contre eux et la sympathie bien connue de la France et de Bonaparte pour les patriotes, les aristocrates continuaient à se bercer de funestes illusions.\par
Un événement important et imprévu vint tout à coup jeter l’effroi dans ce parti.\par
Le gouvernement de Reding, qui avait toujours soutenu les fédéralistes et les aristocrates, fut culbuté par un nouveau coup d’État organisé par quelques unitaires habiles et influents que le Ministre de France, Verninac, soutenait et encourageait. Ils profitèrent de l’absence du Landammann, qui s’en était allé célébrer les fêtes de Pâques au milieu des siens et, le 17 avril, décidèrent que toutes les mesures prises pour mettre à exécution la constitution élaborée pendant l’hiver et acceptée à contre-cœur par la généralité des cantons seraient suspendues. Une assemblée des Notables de tout le pays devait se réunir au plus tôt pour donner à ce dernier une organisation définitive ; Pidou, Carrard et Dan.-Alex. Chavannes devaient y représenter le canton du Léman.\par
Cet événement important semblait de nature à donner autant de satisfaction aux patriotes que de consternation à leurs adversaires. La confiance allait évidemment renaître dans les campagnes et les villes du Léman. Les brûleurs de papier allaient sans doute attendre calmement les décisions des notables qui ne pourraient que donner à la Suisse un régime capable de l’acheminer vers des jours meilleurs.\par
Il n’en fut rien cependant ; les mécontents vaudois ne suivirent pas sous ce rapport l’exemple du plus grand nombre de leurs Confédérés et persistèrent dans leurs intentions subversives. Un certain nombre d’adresses de félicitations et d’attachement furent, il est vrai, envoyées par les communes au nouveau gouvernement unitaire, mais elles exprimaient le ferme espoir que la question des dîmes recevrait une solution aussi prochaine que favorable. Le reste du pays – la campagne surtout – resta silencieux et se prépara secrètement à exécuter le projet si bien préparé pendant l’hiver et dont l’explosion n’avait été que retardée au commencement d’avril.\par
Les populations avaient été déçues trop souvent dans leurs espérances depuis 1798 pour posséder encore quelque confiance dans un gouvernement helvétique nouveau, même composé d’unitaires. Chaque régime avait promis un ordre de choses définitif ; chaque Coup d’État avait été accueilli avec espérance ; tous les hommes politiques qui s’étaient succédé au pouvoir avaient rallié autour d’eux une grande partie du peuple, et successivement, tous avaient failli à leurs promesses. La confiance dans un gouvernement quelconque issu des luttes et des intrigues des factions était complètement détruite. Les populations cherchaient vainement dans le pays l’homme qui parviendrait à concilier les opinions divergentes, à donner satisfaction aux justes demandes et aux revendications basées sur les principes mêmes qui avaient légitimé la Révolution. Ayant été obligées pendant plusieurs siècles de ne s’occuper que de leurs intérêts matériels, on ne pouvait pas s’attendre à les voir tout à coup oublier totalement ces derniers pour se confiner dans l’idéalisme politique et les spéculations philosophiques d’une organisation sociale renfermant plus ou moins d’unité ou de fédéralisme. Après quatre ans d’attente inutile, d’espoirs déçus, de promesses abandonnées, de contributions aussi variées qu’écrasantes, de dévouement infructueux à une cause que l’on avait embrassée avec ferveur comme une planche de salut alors qu’on l’avait d’abord considérée avec défiance, ces populations étaient plongées dans un désespoir mélangé de beaucoup de colère et même de haine contre ceux que les meneurs faisaient regarder comme les obstacles principaux à l’avènement définitif de la liberté et de l’égalité. Tout en restant attachées généralement au nom Suisse, elles se demandaient avec anxiété si le Canton du Léman, le seul qui soit tout entier de race latine, pourrait vivre heureux et satisfait dans l’alliance helvétique avec le système de l’unité. On voyait dans tous les événements passés et présents la main vigilante de la France ; on savait que Bonaparte avait soutenu de la manière la plus éclatante, en face de Reding, les droits du Pays de Vaud à l’existence autonome ; les hommes politiques du Léman avaient confiance dans l’appui du gouvernement consulaire et ce sentiment avait passé dans l’esprit des populations.\par
La « Grande nation » avait facilité, sinon rendu possible l’émancipation politique de 1798 ; elle agirait sans doute de même en 1802, en faveur de l’émancipation économique et sociale. Et comment ne le ferait-elle pas ? N’avait-elle pas supprimé complètement le régime féodal chez elle dès 1789 ? N’avait-elle pas agi de même dans les contrées conquises par ses troupes ? Son gouvernement ne favoriserait-il pas cette politique dans un pays ami ? Et si l’on se voyait continuellement refuser le bénéfice légitime et nécessaire de la Révolution, disaient ceux qui mettaient cette question au dessus de l’attachement à la patrie suisse, ne fallait-il pas s’unir tout à fait à la France, plutôt que de rester fidèle à l’alliance helvétique qui n’avait fourni, selon eux, que des maîtres absolus pendant » deux siècles et demi et dans laquelle, après quatre années d’attente pénible, on désespérait de pouvoir trouver des égaux et des frères ? On voulait arriver à la liberté, à l’égalité et à l’émancipation économique ; on voulait cette fois encore compter, dans ce but sur le secours de la France et, au besoin, s’unir à elle pour obtenir ces avantages.\par
Le Résident de France à Genève, Félix Desportes, avait intrigué en 1798 dans le même but. Pour décider les Genevois à émettre leur vœu de réunion, il ne cessait de leur répéter : – \emph{« Montez sur le clocher de St-Pierre et tout ce que vous découvrirez du haut de cette tour, aussi loin que l’horizon peut s’étendre, fera partie du Département dont Genève sera le chef-lieu. – Dans le même temps, il agitait le Pays de Vaud et surtout le district de Nyon en leur persuadant d’exiger l’abolition de toutes les redevances territoriales sans rachat, de la même manière que cela s’était opéré en France \footnote{Lettre de L. Frossard de Saugy, sénateur, au Préfet Polier. 27 décembre 1799.}  »}. Ces intrigues étaient encore présentes à l’esprit d’un très grand nombre de citoyens. Maintenant, le général Turreau reprenait l’œuvre de Desportes, correspondait avec les mécontents, leur promettait l’appui de son pays, recevait leurs émissaires et les poussait à manifester leur vœu de réunion à la Grande République.\par
Ces considérations nécessaires montrent quel était l’état d’esprit des populations d’une partie du Léman au mois d’avril 1802 et peuvent faire comprendre pourquoi le coup d’État du 17 avril ne fut pas capable de mettre fin à une entreprise préparée au milieu de circonstances politiques qui n’existaient plus.
\chapterclose


\chapteropen
\chapter[II. Les Bourla-Papey]{II. Les Bourla-Papey}\renewcommand{\leftmark}{II. Les Bourla-Papey}


\chaptercont
\section[Le premier mai]{Le premier mai}
\noindent L’événement du 17 avril ne fit que faciliter et précipiter l’insurrection.\par
\emph{« Nous ne ferons rien tant que ces gens-là seront ici »}, avaient dit les gens de Cossonay, en parlant des troupes françaises cantonnées dans cette localité. Le nouveau gouvernement unitaire, craignant sans doute une réaction, fit rentrer à Berne tous les contingents qui avaient été envoyés dans le Pays de Vaud quelques semaines auparavant. Les mécontents purent donc aussitôt s’occuper de l’exécution de leur projet.\par
L’agitation secrète se montra de nouveau et, à la fin d’avril seulement, les agents du gouvernement eurent une connaissance approximative de ce qui se tramait dans l’ombre.\par
Le 29 avril, le sous-préfet d’Orbe put annoncer à Henri Polier :\par

\begin{quoteblock}
\noindent « qu’il était plus que jamais question de brûler les archives des châteaux et qu’on en voulait surtout au Préfet national. »\end{quoteblock}

\noindent Bourgeois, Châtelain des Clées, écrivait de son côté que le projet avait été repris à la dernière foire de La Sarraz et que l’exécution serait prochaine.\par
Les populations de l’ancienne baronnie de La Sarraz paraissaient tout particulièrement actives et agitées. Le Commis d’exercice d’Orny colportait les ordres des chefs prescrivant à chacun de se tenir prêt à marcher sur Lausanne dans la nuit du 1\textsuperscript{ᵉʳ} mai. A Eclépens, Louis Monnier, sa femme et Pierre Besson parcouraient les villages de la contrée et envoyaient des agents jusqu’à Yverdon pour s’assurer, si possible, la coopération des localités du nord du Canton. Le Receveur de La Sarraz voyagea aussi pendant toute la journée du 30 avril et on remarqua qu’il y avait encore de la lumière chez lui, de même que chez le ministre Ribet, à deux heures du matin, le 1\textsuperscript{ᵉʳ} mai. A Penthalaz, l’agent national apprenait le 30 avril au soir, par des enfants, que l’on allait battre la caisse pendant la nuit ; il la fit mettre en lieu sûr.\par
Il n’en fut pas de même partout et, à partir de dix heures du soir, un mouvement inusité put se remarquer dans un grand nombre de villages. Des groupes de citoyens se formaient et se dirigeaient ensuite par différentes routes du côté de Lausanne. Le tambour battait dans la baronnie de La Sarraz, dans le district de Cossonay ; on entendait des appels nombreux dans les villages et des coups frappés contre les portes et les fenêtres des maisons pour réveiller les citoyens et leur rappeler que le moment était venu de s’armer et de marcher. Les plus résolus partaient équipés comme pour se rendre à une revue ; d’autres, plus prudents, se bornaient à ajouter à leurs vêtements habituels, leurs gamaches ou grandes guêtres blanches. Tous étaient plus ou moins complètement armés.\par
La partie de la population qui n’était pas dans le secret ou qui ne voulait pas participer au mouvement, se réveillait effrayée et les bruits les plus contradictoires circulaient. Quelques-uns rapportaient que l’on avait vu passer des contingents de dix à quinze hommes ; d’autres soutenaient au contraire que plusieurs milliers d’individus étaient en marche.\par

\begin{quoteblock}
 \noindent « Cette nuit, disait une personne de Penthaz, ils ont réveillé de maison en maison tous les citoyens… en les sommant de se lever, de s’armer et de marcher avec eux contre Lausanne pour y détruire les archives nationales, les dits individus proférant de grandes menaces contre ceux qui ne marcheraient pas… » \par
 « On dit qu’il en a passé sept ou huit cents, écrivait l’agent national du même village au sous-préfet de Cossonay. Vous comprenez… dans quelle combustion les braves ont été réduits ; je vous mande ce peu de mots pour vous prier, Monsieur, s’il n’y aurait pas moyen d’avoir quelque sûreté ; plusieurs personnes peuvent courir des dangers, entre autres M. de Penthaz et Louis Quelen, à qui Ton a juré la mort, n’ayant pas voulu partir, quoiqu’il ait été obligé de donner un valet… »
 \end{quoteblock}

\noindent Dès le milieu de la nuit du 30 avril au 1\textsuperscript{ᵉʳ} mai, des groupes de campagnards commencèrent à se montrer à l’entrée de Lausanne, à Crissier, à Prilly surtout. Le lieu de rassemblement fixé était la hauteur de Montétan. Les contingents de la contrée d’Oron et surtout ceux de La Côte devaient y prendre contact avec les gens du district de Cossonay. Les chefs militaires du mouvement devaient s’y trouver aussi ; il devait y avoir parmi eux, disait-on, un général français.\par
Dans la direction du nord, on entendait au loin, dans la nuit, le bruit du tambour. Les unes après les autres, les bandes arrivaient et cherchaient à se reconnaître. La confiance et la joie étaient assez générales. N’allait-on pas enfin apprendre aux aristocrates qu’ils devaient dorénavant tenir compte de la volonté du peuple ? N’allait-on pas, enfin, secouer le joug féodal ? L’animation la plus grande se montra donc de Prilly aux portes de Lausanne pendant un temps assez long, mais bientôt des gestes de mécontentement, des paroles d’impatience lui succédèrent. On avait compté sur plusieurs colonnes qui ne venaient pas. Les contingents de la région de Morges à Rolle devaient se trouver au rendez-vous ; on n’en avait aucune nouvelle. Nulle rumeur n’était entendue du côté de l’ouest ; on ne percevait pas le moindre bruit de tambour. Les chefs auraient dû être là ; on n’en voyait arriver aucun.\par
Bientôt, une vague inquiétude se montra dans cette foule de campagnards qui n’avaient pas craint de se compromettre. Les circonstances du moment, l’acte qu’il s’agissait de commettre, le manque d’ordre et de discipline dans cette cohue, et, au-dessus de tout, l’absence de ceux sur lesquels on comptait le plus et qui avaient été les principaux moteurs de cette expédition, tout cela abattit les courages, enleva quelques illusions et fit maugréer.\par
L’aube commençait à poindre et personne encore ne venait. \emph{« Craïou pardieu que nos an bailli on mai d’auri} \footnote{Je crois, pardieu, qu’ils nous ont donné un mois d’avril.}, » cria une grosse voix. Cette parole, bien vaudoise, donna un nouveau cours aux idées. Tout à coup, on chuchota mystérieusement de groupe en groupe une nouvelle importante. Les agents envoyés de divers côtés avaient annoncé qu’il fallait se réunir à Montétan dans la nuit du 1\textsuperscript{ᵉʳ} mai, qui était à Lausanne le jour du marché principal. Les hommes du district de Cossonay et de la baronnie de La Sarraz étaient donc venus dans la nuit du 30 avril au 1\textsuperscript{ᵉʳ} mai. Les gens de La Côte, comprenant d’une autre manière les instructions reçues, crurent au contraire qu’il fallait se mettre en route dans la nuit qui suivrait le 1\textsuperscript{ᵉʳ} mai.\par
Cette équivoque est une des nombreuses choses qui, dans la guerre des \emph{Bourla-Papey}, dépeignent bien le caractère vaudois. Cette indécision et ce manque de clarté eurent rarement, toutefois, des conséquences aussi importantes que ce jour-là. Les campagnards comprirent que, le jour étant venu et les autres contingents n’arrivant pas. L’entreprise était complètement manquée.\par

\begin{quoteblock}
\noindent « Retournons-nous-en, dirent les campagnards, nous n’avons plus rien à faire ici. »\end{quoteblock}

\noindent Ils manifestèrent en même temps la plus grande mauvaise humeur contre les chefs et surtout contre Claude Mandrot, de Morges, qu’ils ne furent pas loin, pendant quelques jours, de considérer comme un traître.\par
Pendant ce temps, le public des districts de Cossonay, Orbe, etc., se demandait avec anxiété quel avait été le résultat de l’expédition. Les personnes les plus impatientes s’en allèrent, au matin, sur les routes, dans la direction de Lausanne, pour avoir le plus tôt possible des nouvelles. Elles rencontrèrent bientôt de petits groupes d’hommes équipés et armés qui rentraient paisiblement et un peu attristés dans leurs villages respectifs. – Que disiez-vous ? demandait-on à l’un d’entre eux quelques années plus tard. – Ma foi, répondait-il, on était « capot ».\par
Les prudents, ceux qui n’avaient pas endossé l’uniforme, déposèrent leurs armes dans diverses maisons plus ou moins amies des environs de Lausanne, au Bois-de-Vaud, chez le citoyen Joseph ; à la Vallombreuse, chez le citoyen Pache ; à Valency, au château de Prilly, etc. Ils entrèrent ensuite à Lausanne isolément et se confondirent bientôt avec la foule des autres campagnards venus à la foire.\par
Le Préfet national fut averti le 30 avril au soir par les sous-préfets d’Orbe et d’Yverdon de la réorganisation du complot. Comme celui-ci ne devait éclater, disait-on, que dans le cours de la semaine suivante, Polier ne crut pas nécessaire de prendre le soir même des mesures spéciales de sécurité.\par
Le premier mai, à 4 ½ heures du matin, il fut réveillé, et le citoyen Carrard d’Orbe introduit auprès de lui tout couvert de transpiration et fort agité. Il raconta :\par

\begin{quoteblock}
\noindent « qu’étant sorti de la ville une demi-heure auparavant pour se rendre à Romainmôtier, il avait rencontré des groupes de gens de la campagne… tous fort animés et annonçant divers projets sinistres qu’ils voulaient exécuter immédiatement ; qu’arrivé à Montétan, il en avait trouvé un corps considérable. Pensant à tous les maux qui pouvaient suivre, il était immédiatement revenu en arrière. »\end{quoteblock}

\noindent Le Préfet fit réveiller son fils et son gendre, de Constant, et les envoya à la découverte, accompagnés de quelques hommes de bonne volonté.\par
Le capitaine de dragons, Auguste de Constant d’Hermenches, partit dès les cinq heures du matin, parcourut le pays avoisinant et vit des groupes de paysans rentrant chez eux.\par
Il fit une seconde patrouille après onze heures du matin avec quatre dragons et une vingtaine de fantassins pour saisir les armes déposées dans diverses maisons. La plupart avaient déjà été reprises et emportées ; quelques-unes seulement se trouvèrent encore dans la propriété du citoyen Pache à la Vallombreuse, cachées dans une grange au milieu d’un tas de foin.\par
Le Préfet avertit, dès le premier moment, le Petit Conseil de ce qui venait de se passer et des projets plus ou moins avérés des campagnards. Le commandant français de la place et du Pays de Vaud montra, de son côté, le plus grand dévouement pour concourir au maintien de la paix et ordonna aux compagnies françaises qui se trouvaient à Vevey et à Nyon de venir à Lausanne. Lorsqu’il apprit, en outre, que les insurgés comptaient sur l’appui de la République, et même de ses troupes, il en fut fort fâché et publia aussitôt une proclamation par laquelle il annonçait qu’il allait mettre tout en œuvre pour empêcher le désordre et qu’il pouvait assurer que les factieux n’avaient à attendre que l’animadversion des autorités françaises.\par
Le Préfet appela un détachement de chasseurs qui se trouvait à Aigle et fit mettre sur pied quelques contingents de Lausanne et des environs sous la direction de l’inspecteur des milices. Le dépôt des archives fut placé sous la garde d’une centaine d’hommes disposant d’une pièce de canon.\par
Dès les deux heures après-midi, les troupes françaises cantonnées à Vevey firent leur entrée à Lausanne. A ce moment-là, les paysans, un peu surpris par le contenu de la proclamation du commandant Veilande, avaient tous quitté la ville dans laquelle le calme régna bientôt.\par
Le Préfet Polier annonça le soir au Petit Conseil, par courrier spécial, les événements de la journée et les mesures prises. \emph{« Malheureusement, disait-il, les anarchistes ont si fort corrompu la masse des campagnards et des citadins au moyen de l’abolition des dîmes et censes que l’on ne sait sur qui compter, en sorte que le secours de la troupe de ligne et celui des Français est indispensable »}.
\section[Le plan des insurgés et leur chef]{Le plan des insurgés et leur chef}
\noindent Le calme continua pendant la nuit et les deux jours suivants. On savait, sans doute, qu’une grande agitation régnait de divers côtés, mais il était bien difficile d’être renseigné exactement sur les intentions des campagnards.\par

\begin{quoteblock}
\noindent « Personne ne veut rien dire, annonçait le sous-préfet de Cossonay. – Il règne un bel accord entre les coupables et les non coupables » \emph{ajoutait-il le 3 mai}.\end{quoteblock}

\noindent Dans la nuit du 3 au 4, des patrouilles furent envoyées dans les environs de Lausanne. L’une d’entre elles a donna à un quart d’heure de la ville dans un corps de paysans. Elle fut entourée, mais relâchée peu après ». Une balle atteignit le cheval du citoyen Constant d’Hermenches, gendre du Préfet.\par
Le jour suivant, ce dernier fut averti par le sous-préfet de Cossonay que :\par

\begin{quoteblock}
\noindent « l’opération devait se renouveler sous peu. L’insurrection sera générale, disait-il. Quand les insurgés auront réussi, ils s’adresseront au Premier Consul pour demander la réunion à la France. On double partout la garde de sûreté. »\end{quoteblock}

\noindent Que faisait le gouvernement central en face d’une situation aussi inquiétante ? Son attitude fut d’abord singulièrement passive. Le Coup d’État du 17 avril, l’arrivée au pouvoir du parti unitaire ou démocratique, le grand nombre d’adresses de félicitations et de dévouement qui lui avaient été envoyées de différentes parties du Canton du Léman, toutes ces choses lui firent supposer que le Préfet National croyait la situation plus grave qu’elle ne l’était en réalité. Il lui sembla que les forces dont son représentant pouvait disposer étaient suffisantes pour parer au danger éventuel.\par

\begin{quoteblock}
\noindent « Le Léman est-il donc réduit à un tel état de désorganisation que l’homme qui y exerce depuis quatre ans la première magistrature à là satisfaction de ses supérieurs et sûrement à celle de la grande majorité de ses concitoyens, mandait-il à Polier \footnote{Lettre du 2 mai.} ne puisse pas compter sur la masse des amis de l’ordre, surtout les hommes intéressés à la conservation des propriétés, pour s’opposer aux projets de quelques anarchistes ? Est-ce que les habitants de Lausanne les laisseraient tranquillement venir dans leurs murs, enlever leur Préfet et piller la maison nationale ? Ne doutez pas, disait cependant le Petit Conseil, en terminant, que si la tranquillité publique dans votre canton est réellement en danger, il ne mette en œuvre tous les moyens pour la conserver. »\end{quoteblock}

\noindent Ces lignes montrent que le gouvernement n’avait pas une connaissance exacte de l’état des esprits dans le Canton du Léman. La situation était telle, en effet, qu’il était extrêmement difficile à une personne étrangère de s’en faire une idée juste. C’est ce que le Préfet chercha à faire comprendre le 4 mai à ses supérieurs.\par

\begin{quoteblock}
 \noindent « Pour toute autre question que celle des dîmes et des censes, disait-il, le gouvernement trouverait dans le Léman autant de soldats fidèles qu’il compte de citoyens en état de porter les armes ; mais, dans cette malheureuse question où chacun se trouve intéressé et sur laquelle tant de circonstances ont altéré ou corrompu l’opinion, ce serait s’aveugler volontairement que de compter sur les milices pour la défense de ces propriétés ou des titres qui les constatent ; le secret gardé malgré le secret promis et les récompenses, sur les spoliations des archives de Bière ou de La Sarraz, malgré la foule qui les connaît nécessairement, porte déjà la preuve de cette assertion et elle trouve son complément dans l’événement du premier mai, dont les préparatifs n’ont été trahis par qui ce soit. Dans un tel état de choses et sans moyens pécuniaires pour la police secrète, le Petit Conseil sentira que les données sur l’esprit public, sur les complots qui se préparent, sur leurs auteurs et fauteurs ne peuvent qu’être vagues et réduites à des aperçus plus ou moins plausibles, d’où il résulte souvent que je ne puis, malgré ma meilleure volonté et mes efforts, mettre le gouvernement à même d’apprécier le degré et l’étendue des dangers… Le capitaine de Constant ayant sept ou huit dragons sous ses ordres et quelques citoyens à cheval, battent les routes dès la tombée de la nuit pour éviter toute surprise ; environ soixante volontaires de mes amis ont promis de me joindre à la première alarme ».
 \end{quoteblock}

\noindent Pour parer aux événements prévus, le Petit Conseil avait invité le Ministre de la guerre à faire envoyer à Lausanne une compagnie d’infanterie « aussi complète que possible, » et prié le général Montrichard, commandant des troupes françaises en Suisse, d’ordonner aux officiers sous ses ordres dans le Canton du Léman de faire droit aux demandes du Préfet national\par
Ces moyens étaient dérisoires et le gouvernement s’en aperçut bientôt.\par
Le 4 mai, en effet, Henri Polier reçut par le sous-préfet de Cossonay un avis de l’agent national de Penthaz, lui disant :\par

\begin{quoteblock}
 \noindent « La bagarre de vendredi doit se répéter demain avec beaucoup d’acharnement et de grandes précautions. Les troupes ne les détourneront pas de leurs projets. Ils ont un grand nombre de courriers pour leurs communications. »
 \end{quoteblock}

\noindent Cet avertissement fut le plus exact de tous ceux, très nombreux, qui avaient été transmis à l’autorité jusqu’à ce moment-là. Ce que le Préfet national ignora cependant pendant un certain nombre d’heures encore, ce fut le but que se proposaient cette fois les paysans. En réalité, ceux-ci, après l’échec piteux de l’expédition du premier mai contre Lausanne et l’annonce des mesures de sûreté qui avaient été prises dans cette ville, pensèrent que pour y pénétrer, ils devaient non seulement être nombreux, mais encore constituer une troupe déjà organisée. Il fallait rendre aux paysans la confiance qui avait abandonné plusieurs d’entre eux et, pour cela, leur procurer un premier succès plus facile quoique moins important.\par
La ville de Morges avait été, par le moyen d’un bon nombre de ses citoyens les plus influents, le centre de l’agitation politique depuis bien des mois. Cette ville fut appelée par conséquent à servir de base d’opération à la troupe des \emph{Bourla-Papey.} Ceux-ci pouvaient y compter sur la sympathie de la grande majorité de la population et même sur celle de l’autorité communale. Le château renfermait de nombreuses pièces de canon et n’était occupé que par un très faible détachement de troupes. Tout encourageait donc les \emph{Bourla-Papey} à s’emparer d’abord de Morges, à brûler ses archives féodales importantes, à appeler à eux tous les hommes de bonne volonté et à marcher aussi nombreux que possible sur Lausanne.\par
Ce plan de campagne pouvait être excellent à condition qu’il fût rapidement exécuté par une troupe nombreuse et disciplinée, ne laissant pas au gouvernement et à ses représentants dans le Léman le temps de concentrer des forces considérables.\par
Il fallait aussi que les paysans eussent à leur tête un chef reconnu et responsable. Il ne manquait pas sans doute d’officiers capables dans le grand nombre des hommes qui avaient poussé les campagnards dans leur entreprise. La plupart ne voulurent en aucune mesure se compromettre ostensiblement dans une aventure de ce genre et ils préférèrent en laisser la direction à un personnage secondaire à ce moment-là, mais qui avait montré le zèle le plus ardent et le plus audacieux en 1798 et dont l’exaltation politique était connue.\par
Ils se servirent de Louis Reymond.\par
Originaire des Grands-Bayards au canton de Neuchâtel, Louis Reymond était né à Lausanne en 1770. Il y terminait son apprentissage d’imprimeur lorsque la Révolution éclata. Poussé par son caractère violent et son exaltation politique, il joua un rôle en évidence à l’extrême gauche du parti patriote. Il fut un membre influent du Comité de Réunion et de la Société des Amis de la liberté, et sa parole éloquente et enflammée retentit souvent dans le temple de St-Laurent où avaient lieu les séances de cette association. Le journal \emph{l’Ami de la Liberté}, dont il fut un des principaux rédacteurs, parlait souvent de l’activité de ce citoyen et de l’influence énorme qu’il exerçait au milieu de son parti.\par

\begin{quoteblock}
\noindent « Le brave Reymond monte de nouveau à la tribune, dit par exemple le compte-rendu de la séance du 13 février, il lance la foudre sur la tête des aristocrates ; il exhorte, il réveille les patriotes, il fait passer dans leurs cœurs le feu de la liberté qui brûle le sien. Il est brusque et rapide dans sa déclamation, son geste est animé, son regard expressif ; il s’enflamme, il enflamme tous les cœurs et ravit tous les suffrages. Le citoyen Will, transporté, s’élance à la tribune et l’embrasse avec la franchise d’un homme qui aime la liberté. »\end{quoteblock}

\noindent Le lendemain, Reymond n’était pas là.\par

\begin{quoteblock}
\noindent « La séance languit, dit le journal ; on chante, on rit, on cause et l’on s’aperçoit de l’absence de Reymond ; il est l’àme des assemblées, le bras et la tête du peuple. »\end{quoteblock}

\noindent Lorsque \emph{l’Ami de la Liberté} cessa de paraître, au mois de mai, Reymond publia sous son nom le \emph{Régénérateur}, dont le titre était un programme.\par

\begin{quoteblock}
\noindent « Le crédit du journaliste et l’importance relative de sa feuille se révélèrent bientôt d’une part par l’élection du citoyen Reymond à la place de juge au tribunal du district de Lausanne, et de l’autre par les plaintes dirigées contre le journal qui, tiré à cinq cents exemplaires, entretenait et fomentait dans le pays les opinions démagogiques de son rédacteur \footnote{Voir \emph{La Presse périodique vaudoise}, par J. Chavannes dans la Bibl. universelle et Revue suisse de nov. 1863.}. »\end{quoteblock}

\noindent Le gouvernement lui-même se montra disposé à sévir contre la Société des Amis de la Liberté et contre le \emph{Régénérateur}, et il était à prévoir que Louis Reymond ne tarderait pas à sentir les dangers de son activité politique.\par
Le 31 août 1798, ce journal publia une Adresse au Corps législatif, protestant contre la distinction que l’on voulait perpétuer dans les communes entre les bourgeois et les simples habitants et déclarant que les signataires se refuseraient à accorder force de loi à une décision basée sur ce principe.\par
Le Directoire décréta l’arrestation et la mise en jugement de Louis Reymond, qui fut condamné par le tribunal de Lausanne à trois mois d’arrêt. L’Accusateur public, Auguste Pidou, en appela au Tribunal suprême qui infligea à Reymond trois ans de détention et dix ans de privation de ses droits civiques. Il fut emprisonné à Lucerne, siège du gouvernement central. Ce dernier l’occupa à des travaux d’impression, et enfin, au bout de trois mois, demanda sa libération aux Conseils. Il rentra triomphalement à Lausanne après avoir été nommé capitaine dans la deuxième demi-brigade helvétique. C’est en cette qualité et comme officier de recrutement, ayant à sa disposition des sommes de quelque importance, qu’il se trouvait dans le Pays de Vaud en 1802.\par
Reymond avait surtout cherché en 1798 à obtenir la suppression sans rachat des redevances féodales.\par

\begin{quoteblock}
 \noindent « Non, bon cultivateur, disait-il alors dans son journal, toi qui depuis plusieurs siècles arroses la terre de tes sueurs, pour en laisser recueillir les fruits à la mollesse et à la corruption, tu ne te verras plus arracher le produit de tes travaux par ceux qui ne te récompensent que par le mépris et l’ingratitude. Laboure ton champ en paix, sans crainte de voir les descendants de tes anciens oppresseurs fondre sur ta récolte comme l’oiseau de proie sur le paisible ramier. C’est maintenant que tu es rendu à ta dignité que tu peux élever un front courbé longtemps sous le joug et l’ignominie. Tu ne te dois plus qu’à ta patrie ; celle qui te protège a seule des droits à ta reconnaissance. »
 \end{quoteblock}

\noindent Ce langage, on le voit, préparait admirablement Louis Reymond, à prendre la direction des \emph{Bourla-Papey.}
\section[Les premiers Auto-da-fé]{Les premiers Auto-da-fé}
\noindent Dans la soirée du 4 mai et pendant la nuit suivante, le tambour fut entendu dans toute la partie du canton s’étendant d’Orbe et La Sarraz jusqu’à Aubonne et même Rolle. Les contingents des \emph{Bourla-Papey} s’organisèrent dans les communes, se placèrent sous la direction de la personne qui avait su le mieux obtenir leur confiance et s’en allèrent assaillir les maisons seigneuriales en demandant avec menaces les titres féodaux.\par
Ces expéditions régionales, auxquelles on voyait participer les contingents des différents villages ou hameaux placés sous la suzeraineté du même seigneur, s’effectuaient presque toujours de la même manière. Profondément excités par les discours enflammés de divers démagogues, les campagnards profitaient des heures les plus sombres de la nuit pour exécuter leur entreprise. La spoliation était alors accomplie avec un certain désordre et au milieu des cris les plus variés et quelquefois les plus discordants.\par
Il n’y a que le premier pas qui coûte. Lorsque les insurgés eurent commencé leur œuvre, lorsqu’ils virent que le mouvement s’étendait chaque jour davantage, ils devinrent plus hardis ; ils osèrent – surtout lorsque leurs chefs étaient eux-mêmes courageux – s’attaquer en plein jour aux propriétaires de titres féodaux.\par
La nuit du 4 au 5 mai fut marquée déjà par quelques destructions d’archives dans la région s’étendant du Jura à Morges. Avant de s’avancer sur cette ville, et de grossir l’armée qui devait s’y former, quelques contingents voulurent faire main basse sur un certain nombre de dépôts qui les intéressaient spécialement. Ce fut le cas à Moliens, à Pampigny, à Vullierens, à Denens, à l’Isle.\par
Voyons d’abord ce qui se passa dans cette dernière localité ; la spoliation des archives du château montrera du reste assez bien de quelle manière procédaient les insurgés.\par
Le citoyen Wagnon, homme d’affaires de la famille de Chandieu et inspecteur des forêts, fut réveillé par des coups violents et redoublés à la porte de sa maison. Il courut à la fenêtre demander ce que l’on voulait. Il aperçut une demi-douzaine d’hommes armés, la baïonnette au bout de leurs fusils. \emph{Ouvrez la porte, ou nous l’enfonçons}, dirent-ils, et ils frappèrent de plus en plus, exigeant les titres et que l’on vînt avec eux au château où se trouvaient quelques hommes de garde possédant la clef des archives.\par

\begin{quoteblock}
 \noindent « Sur ce, en continuant toujours à dire : \emph{Ouvrez la porte, ou nous l’enfonçons}, et en tirant un assez grand nombre de coups de fusil, ils ont exigé que moi-même je leur donnasse cette clef, raconte Wagnon. – Donnez-moi donc le temps de m’habiller, leur ai-je dit, mais leur furieuse impatience ne m’en a pas laissé le temps ; je suis descendu à peu près nu et sans armes et ai ouvert [la porte] au moment où ils la frappaient encore en redoublant. Aussitôt ces gens armés se sont précipités en foule dans le corridor avec tous les gestes et le ton de la fureur ; l’un d’eux ayant tiré son sabre m’a demandé les droits féodaux. J’ai répondu : Ils sont aux archives du château, mais tranquillisez-vous et épargnez deux femmes enceintes qui sont dans la maison. »
 \end{quoteblock}

\noindent Ayant enfin trouvé le temps de s’habiller, Wagnon sortit avec sa femme qui voulut l’accompagner au milieu de ces « furieux », au nombre de quarante au moins.\par

\begin{quoteblock}
 \noindent « Arrivé au château, j’ai dit à la garde qu’elle était insuffisante et qu’elle devait céder la place pour ne pas répandre du sang. Aussitôt, j’ai voulu délivrer la clef des archives qui était dans la chambre des gardes, mais entouré dans ce moment de plus de soixante hommes armés qui s’étaient précipités dans la chambre, et d’autres qui obstruaient la cour, j’ai dû encore ressortir de là et, forcé d’ouvrir moi-même les archives où se sont précipités encore autant d’hommes qu’il y en a pu entrer ; puis le citoyen Baudat, agent à l’Isle, y est parvenu accompagné d’un factionnaire de la garde bourgeoise, qui m’a rapporté que sa maison avait aussi été entourée de gens armés pour l’empêcher de sortir et il a vu, comme ma femme et moi, que tous les titres en volumes ou papiers que cette troupe arrachait de toutes parts et qu’ils ont dit concerner les droits féodaux, ont été par eux enlevés, notamment ceux des fiefs de Cuarnens, de Chavannes et de l’Isle… Comme pendant cette scène, des imprécations accompagnées de menaces et d’accusations que tous les titres n’étaient pas là et qu’on me les ferait bien délivrer, ont été faites à réitérées fois, j’ai invité cette troupe à venir fouiller ma maison ; mais arrivé de retour à ma porte, j’ai déclaré que je ne souffrirais pas qu’elle fût violée de nouveau, mais j’ai invité le chef et quelques hommes à entrer pour faire la visite, tandis que les autres resteraient dehors. M’ayant répondu qu’ils n’avaient point de chef là, je leur ai adressé un petit discours dont le sens était que, pénétré de l’acte de violence qu’ils venaient de commettre, ce n’est pas pour ceux qu’ils venaient de dépouiller de leurs titres que je gémissais, puisque l’on pouvait se consoler de la perte de ses biens, mais pour ceux qui avaient perdu pour toujours la tranquillité de leur patrie, leur honneur et le repos de leur conscience. Là dessus ils se sont retirés sans vouloir entrer ni proférer un seul mot ; un seulement, des rangs les plus éloignés a dit en partant : \emph{Arrivera ce qui pourra}, et tous en silence ont tourné le dos et s’en sont allés. »
 \end{quoteblock}

\noindent Les \emph{Bourla-Papey} ne furent pas toujours aussi modérés qu’à l’Isle. Pendant la même nuit, à Moliens, le citoyen de Watteville vit la porte de sa maison enfoncée \emph{« à coups de batterans \footnote{Marteau en fonte dont on se sert pour casser des pierres.}  »}. Il fut insulté, après quoi on lui prit tous ses titres et même des ciseaux, des pistolets et une lettre de rente de 450 francs. Tout fut brûlé immédiatement. D’autres intéressés vinrent encore pendant la nuit suivante, fouillèrent soigneusement le local des archives et se retirèrent en cassant une glace.\par
Un des hommes les plus actifs pendant ces premières heures de l’insurrection, fut le citoyen David, de Chavannes. Il se rendit tout d’abord avec sa troupe dans le village de Pampigny. La porte de la maison du citoyen Bolens fut enfoncée, ses archives furent chargées sur un véhicule et, quelques instants plus tard, brûlées hors du village. Dans la matinée du 5 mai, David alla assiéger le château de Denens. Les gens de la localité, qui n’avaient sans doute pas de raisons sérieuses de mécontentement à l’égard du citoyen de Büren, refusèrent, malgré les plus grandes menaces, de coopérer à la spoliation. Le tocsin fut sonné et quelques personnes, entre autres Étienne Pernet, vinrent prendre les titres qui concernaient la localité. Les archives ne furent pas enlevées sans quelques violences. Dans le désordre et au milieu des menaces et des cris, le citoyen de Büren reçut quelques « bourrades » et M\textsuperscript{elle} de Tavel, « un coup de pied au derrière ». Les papiers furent ensuite brûlés sur la route de Morges.\par
Après avoir détruit les archives du citoyen Bolens, David de Chavannes s’était d’abord rendu pendant la nuit précédente à Vullierens pour y demander les titres du citoyen de Mestral d’Aruffens. Il était accompagné par des gens de Pampigny, sous la direction de Berthi et Pierre Pittet ; il fut rejoint bientôt par la troupe de Vullierens, aux ordres de Jean Martigny. Le propriétaire opposa quelque résistance et ne céda qu’au moment où il se trouva complètement débordé par une foule toujours plus grande. Les insurgés prirent tous les papiers concernant Vullierens, Pampigny, Aruffens, Clarmont et Réverolles. De Mestral et sa famille étaient déjà depuis quatre heures en proie aux plus vives angoisses, lorsque arriva Wagnon, de l’Isle, avec une nouvelle troupe ; il réclama encore les armes d’une collection et il força les portes. La lettre suivante, écrite le 5 mai par Madame d’Aruffens, née Golowkin, à une de ses amies, peut donner une idée de l’état psychologique dans lequel se trouvait une famille seigneuriale après la visite des \emph{Bourla-Papey} :\par

\begin{quoteblock}
 \noindent « Tout est dit, ma bonne amie ; cent-cinquante hommes armés sont venus ici à sept heures, ont ouvert les archives et, dans ce moment, tout brûle au milieu de la cour, au son du tambour et des cris de cette troupe. C’est la quatrième expédition depuis hier au soir. Nous avons craint tout au monde. Jusqu’à présent, le bon Dieu nous a préservés. Il faut le prier avec ferveur qu’il daigne, dans sa bonté, nous garantir de tout malheur. Notre pauvre petit arsenal a été tout vuidé. Tous ces gens étaient dans la maison. Cette troupe va faire une tournée. Montrez cette lettre à M. Pr [Polier], il saura ce qu’il doit faire. Je suis à moitié morte et n’ai pas la force de vous en dire davantage… »
 \end{quoteblock}

\noindent Toutes les troupes de paysans dont il vient d’être question, beaucoup d’autres venues de diverses régions des districts de Lausanne, Cossonay et Morges, se rapprochèrent de cette dernière ville où elles allaient se concentrer sous le commandement de Louis Reymond et de son adjudant Marcel. Elles devaient y être rejointes encore par un contingent venu de la contrée d’Oron où les \emph{Bourla-Papey} eurent de chauds partisans.\par
Ce furent quelques agitateurs lausannois qui entraînèrent Oron dans l’entreprise. Le premier mai, jour de l’abbaye de Palézieux, Marcel, qui fut l’adjudant de Reymond, était venu à Oron où il avait cherché à voir les citoyens Jan, ex-juge de canton, et Georges, commandant d’arrondissement. Il se rendit à Palézieux et, le soir, eut une longue entrevue à l’auberge de la Croix-d’Or, à Oron, avec les citoyens Rouge, juge au tribunal de Lausanne, Demiéville, président du tribunal d’Oron ; Pernet, instituteur, Jan, Georges, etc. \emph{« Ils causaient ensemble quand ils étaient seuls et se taisaient quand il entrait quelqu’un »}, déclara plus tard l’aubergiste. Le nommé Avocat, de Penthéréaz, était venu aussi à Oron et Palézieux faire un exposé de la situation et encourager les campagnards à prendre les armes. Un de ces derniers, le citoyen Dufey, ayant combattu cette manière de voir, fut apostrophé et s’en alla. – \emph{« Va, tu n’es qu’un vieux fou »}, lui dit un des assistants.\par
Quelques villages se décidèrent à prendre les armes : ce furent surtout Oron, Palézieux, Vuibroye, Châtillens, Essertes et Chésalles qui fournirent du monde. \emph{« De Servion, disait le sous-préfet Gilliéron, il n’est parti qu’un homme qui fait la croix de sa femme et de ses enfants »}.\par
Le 4 mai au soir, les meneurs survinrent dans les différents villages, firent battre la caisse et « au nom de la nation » entraînèrent autant d’hommes que possible. Sous la direction du citoyen Georges, commandant d’arrondissement, ils se dirigèrent du côté de Servion et Mézières où ils espéraient rallier un certain nombre de personnes avant de marcher sur Lausanne par Montpreveyres. A Servion, ils réveillèrent le commis d’exercice Devaud, à 11 ½ heures du soir et lui ordonnèrent de réunir sa troupe. Il refusa, mais fut obligé, à plusieurs reprises, de menacer les insurgés de ses armes pour les obliger à se retirer.\par

\begin{quoteblock}
\noindent « Mais sa femme, qui est enceinte, \emph{dit le sous-préfet dans son rapport, }se trouva malade de l’émotion qu’elle avait eue, et il fut obligé de la saigner le lendemain matin. »\end{quoteblock}

\noindent A l’aube, la troupe d’Oron, précédée du citoyen Georges, à cheval, rejoignit la route de Lausanne, près de Mézières.\par
Le contenu d’un billet non signé et envoyé de ce district au Préfet national, résume assez bien la manière dont tout s’était passé et aussi la crainte que les révoltés inspiraient au public.\par

\begin{quoteblock}
\noindent « Ces jeunes gens, dit-il, le dit Mellet, Pache, Miéville et les Jorge et Jean ; c’est eusse qui ont fait le plus. Le président leur a payé à boire et les a fait danser à ses frets… Je vous prie de ne pas m’esposé {{\sic (sic)}}. »\end{quoteblock}

\noindent Le 5, à onze heures du matin, le citoyen Georges était déjà à la brasserie du Bois de Vaud, près de Lausanne, et parlait confidentiellement à un citoyen de Penthalaz. Au bout d’un instant, ce dernier se leva.\par
— Je m’en vais rejoindre mes gens, dit-il.\par
— Moi aussi, répondit Georges, et ils se séparèrent.\par
A ce moment, les troupes de paysans se voyaient déjà sur le Crêt de la Bourdonnette et dans le bois de Dorigny. Elles ne tardèrent pas à se diriger du côté de Morges.
\section[A Morges et à Berne]{A Morges et à Berne}
\noindent Dès le premier mai, la municipalité de Morges s’était décidée, sur l’invitation du sous-préfet Mandrot, à adjoindre deux hommes au guet afin d’assurer la tranquillité publique. Tout fut cependant calme jusqu’au 4 mai et un spectateur superficiel ne se fût pas douté que la ville était le foyer d’une révolution prête à éclater.\par
Le 4 mai, le sous-préfet, secondé par le commandant d’arrondissement Dellient, convoqua une garde de trente hommes au château sous la direction de Jean Cart-Muller. Entre huit et neuf heures du soir, il apprit qu’un rassemblement se formait au Signal, à une petite distance de la ville. Il fit aussitôt battre le tambour pour réunir les citoyens de bonne volonté.\par

\begin{quoteblock}
 \noindent « J’entends battre la générale, raconte l’ex-sénateur J.-J. Cart. Je sors ; un huissier proclame l’ordre à la réserve et à l’élite de se rencontrer incessamment à la place d’armes. Sur une population d’environ trois mille âmes, vingt hommes se rendent à l’ordre. Qu’est-ce donc ? L’on se chuchote d’abord ; bientôt les vieilles femmes crient : C’est une insurrection ! Les gens de la campagne sont armés ; ils accourent de toutes parts ; ils menacent de brûler les titres féodaux partout où il y a des titres féodaux…\par
 « Qui le croirait ? le nombre des insurgés accroît à chaque instant ; on en compte bientôt trois mille, équipés, organisés, commandés. Point de licence, ils respectent les personnes, les propriétés, mais ils veulent brûler les titres féodaux, où qu’ils soient et ne veulent pas en démordre \footnote{J.-J. Cart : De \emph{la Suisse avant la Révolution et pendant la Révolution.}}  ».
 \end{quoteblock}

\noindent Comme on peut le voir, les sympathies de J.-J. Cart allaient toutes aux paysans ; mais, quoi qu’il fût certainement dans le secret de l’événement, il voulait se considérer comme pris au dépourvu et complètement ignorant.\par
Les vingt hommes qui avaient bien voulu répondre à l’appel du citoyen Mandrot furent envoyés au château. L’élite et la réserve de sept villages des environs furent convoquées d’urgence. Il ne vint qu’une centaine d’hommes qui renforcèrent les détachements précédents. Les citoyens de Denges et d’Echandens avaient été arrêtés en route. D’autrès n’osèrent pas partir. A Morges même, plusieurs personnes excitèrent à la désobéissance.\par
Une patrouille de trois hommes alla reconnaître les environs. Ils aperçurent un attroupement au Signal. A leur \emph{qui vive} ! il fut répondu : \emph{Contingent de Vullierens} !\par
Une seconde patrouille de douze hommes fut envoyée dans la même direction. Au \emph{qui vive} ! il fut cette fois répondu par une décharge de mousqueterie qui blessa le fils du sous-préfet. Pendant que quatre hommes se retiraient, les huit autres s’élancèrent contre les paysans qui décampèrent et furent poursuivis jusqu’au moment où ils disparurent tout à fait dans la nuit. Marc Cart, de St-Saphorin, était à cheval au milieu d’eux, de même que Louis Reymond. Quatorze hommes furent faits prisonniers et emmenés au château. Une troisième patrouille trouva près d’une haie le nommé Rayroud, de Cottens, blessé grièvement ; il fut emporté aussi au château.\par
Dans le même temps, la Municipalité se décidait à siéger en permanence et le 5, à une heure et demie du matin, le sous-préfet apprit enfin l’arrivée très prochaine d’un détachement de troupes françaises. A trois heures, il envoya un courrier au Préfet national pour lui faire part de la situation.\par

\begin{quoteblock}
 \noindent « Nous allons être attaqués, disait-il… On en veut à nos canons pour nous attaquer avec plus d’avantage. Si vous pouvez nous envoyer des Français, qu’ils viennent en toute hâte, autrement nous sommes forcés. »
 \end{quoteblock}

\noindent Une heure plus tard, vingt-cinq soldats français arrivaient au château. C’était bien peu pour défendre une ville dont une grande partie de la population et même la Municipalité étaient plus ou moins d’accord avec les insurgés.\par
A neuf heures, le sous-préfet expédia un nouveau message au Préfet national.\par

\begin{quoteblock}
 \noindent « Les vingt-cinq Français sont arrivés et monteront bien peu de chose à une troupe un peu considérable. Je ne puis la soutenir en rien, mes moyens sont usés complètement. Je m’en remets à la Providence et à ce que vous pourrez m’envoyer de surplus… Je suis harassé… »
 \end{quoteblock}

\noindent A onze heures, un courrier fut encore envoyé à Lausanne porteur des nouvelles les plus alarmantes…\par

\begin{quoteblock}
 \noindent « Je vous réitère que nombre d’archives ont été brûlées dans la matinée, qu’on viendra à nous au plus tôt, car la colonne n’est plus qu’à une lieue et demie. Si vous voulez donc sauver l’arsenal et nos archives, renforcez-moi considérablement. Une compagnie que vous m’annoncez n’est pas suffisante et n’est pas là, et j’estime qu’il me faut en tout cent-cinquante hommes. Je n’ai rien à attendre de la ville et des environs. Ils sont lassés, quelques-uns épouvantés, le plus grand nombre mal disposés. Je me décharge donc de toute responsabilité, ne pouvant rien faire avec rien. J’ai fait mon devoir aussi bien que j’ai pu, mais ma position est des plus pénibles ».
 \end{quoteblock}

\noindent Une heure plus tard, enfin, il arriva un détachement de cinquante hommes de troupes françaises sous la direction du capitaine Demney qui fut commandant du château.\par
Pour faire face à tant de danger, le Préfet national ne pouvait disposer que de très faibles détachements et le gouvernement central n’avait pas encore accordé une grande importance aux rapports qui lui étaient arrivés de Lausanne. Ayant enfin des renseignements précis sur les actes des \emph{Bourla-Papey}, Polier se hâta, le 5 mai, d’envoyer à Berne le capitaine de Constant avec un message circonstancié.\par

\begin{quoteblock}
 \noindent « Dans mes rapports du premier mai et jours suivants, disait-il, j’ai exposé la situation critique et éminemment dangereuse où se trouvent dans ce Canton les amis de l’ordre, des lois et du gouvernement par l’éclat du complot des chefs de l’anarchie qui… ont préludé par des actes de violence et des spoliations sans nombre de maisons des ci-devant propriétaires de droits féodaux et ont enfin depuis hier donné ouvertement le signal de la guerre civile… La présente vous est portée par mon gendre… qui ce matin, en faisant une patrouille, a essuyé une fusillade qui a blessé son cheval et percé son manteau ; le fils du sous-préfet de Morges, en se défendant contre ces scélérats a reçu un coup de feu au visage. Mon gendre vous exposera la situation où nous nous trouvons par l’approche de la colonne des insurgés qui se trouve à Chavannes, à trois quarts de lieue d’ici, tandis que l’avant-garde d’une colonne du district d’Oron est actuellement à deux lieues d’ici. Nous avons en tout six cents hommes, dont quatre cents Français, à leur opposer. Je ne cacherai point au Conseil exécutif que le brave chef de bataillon Veilande, de la quatre-vingt-septième, ne cesse de me témoigner son extrême surprise de la nullité des moyens que vous opposez à un si grand mal puisqu’une seule compagnie d’infanterie est annoncée ».
 \end{quoteblock}

\noindent Le Petit Conseil dut se rendre à l’évidence des faits. Il se décida à envoyer dans le Léman un de ses membres, le citoyen Kuhn, en qualité de « Commissaire général et extraordinaire » en lui donnant les pouvoirs les plus étendus.\par

\begin{quoteblock}
 \noindent « Toutes les autorités civiles et militaires, disait le décret du gouvernement, les commandants de la force armée helvétique sont subordonnés au citoyen Kuhn et tenus de respecter ses ordres. Tous les bons citoyens sont sommés de lui prêter main forte à sa première réquisition. Les commandants des troupes françaises stationnés dans le Léman sont invités d’appuyer le dit Commissaire de tous leurs moyens ».
 \end{quoteblock}

\noindent Le Petit Conseil s’adressa d’autre part au commandant des troupes françaises en Suisse pour le prier de mettre toutes les forces nécessaires à la disposition du citoyen Kuhn. Le général Montrichard répondit aussitôt de la manière la plus rassurante. \par

\begin{quoteblock}
 \noindent « Je crois devoir renouveler au Petit Conseil l’assurance de mes efforts constants et de ma volonté bien prononcée pour le maintien de l’ordre, disait-il. Le concours des troupes françaises et helvétiques parmi lesquelles il règne la meilleure harmonie, malgré que l’on ait cherché à la troubler, en est un sûr garant… »
 \end{quoteblock}

\noindent En arrivant à Lausanne le 7 mai, le Commissaire du gouvernement s’empressa d’annoncer sa mission et son but « aux citoyens du Canton du Léman et en particulier des districts de Cossonay, Morges, Aubonne, Oron, Orbe, Lausanne et Rolle : C’est avec la plus vive douleur et une indignation profonde que j’apprends jusques à quel point vous vous êtes rendus coupables, disait-il. Le pillage, l’incendie doivent-ils donc déshonorer la Révolution dans le Canton du Léman, et le moment où une Constitution définitive va assurer les destinées de notre patrie, guérir les maux du régime provisoire, assurer le triomphe de la liberté et d’une sage égalité, devait-il être celui où s’allume le feu de la guerre civile ? Rentrez dans vos foyers, soumettez-vous à vos autorités, obéissez à la loi. Je puis écouter les citoyens repentants et paisibles ; mais contre des rebelles armés, je ne connais que la force des armes et la punition serait terrible.\par
L’arrivée du citoyen Kuhn, le contenu de sa proclamation, la présence de nouvelles troupes venues de divers côtés, tout contribua à rassurer un peu les personnes qui avaient le plus à craindre les insurgés ou qui ne demandaient que le maintien de l’ordre.\par
Malheureusement, le mouvement insurrectionnel était trop puissant, les campagnards étaient trop exaspérés et la plupart des citadins trop favorables à leur cause pour que l’on pût espérer une pacification immédiate. Le 5 au soir, le Préfet national avait fait enjoindre à tous les citoyens du chef-lieu, de 18 à 60 ans, d’avoir à se réunir sur la place de Montbenon à sept heures « pour défendre les personnes et les propriétés ». Un nombre bien minime de personnes avait répondu à cet appel, montrant ainsi que l’indifférence, la peur ou le mépris pour les autorités constituées avaient pénétré partout.\par
Avant l’arrivée du citoyen Kuhn, il était du reste survenu à Morges et dans les environs, des événements graves qui avaient contribué à donner une nouvelle vigueur à l’insurrection.
\section[Le Traité de Rion-Bosson]{Le Traité de Rion-Bosson}
\noindent Le sous-préfet Mandrot passa la nuit du 5 au 6 mai au château de Morges. Il put entendre sur les collines environnantes le bruit du tambour et même les cris poussés par les troupes de campagnards qui venaient encore de divers côtés, grossir la troupe réunie sous la direction de Louis Reymond. Pendant cette même nuit, deux membres de la Municipalité vinrent lui offrir leur médiation, vu les grands dangers que courait la commune. Il refusa, mais il apprit ainsi que les insurgés se rapprochaient et qu’ils pouvaient compter sur l’appui de l’autorité locale.\par
A six heures du matin, cette dernière fut invitée par le commandant français du château à venir prendre connaissance de la demande que lui avait présentée Louis Reymond de livrer les prisonniers de l’avant-veille et les archives communales. Elle chargea deux de ses membres – son président, Muret-Martin, et le citoyen Bourgeois – \emph{« de se transporter où il sera nécessaire pour travailler selon leur prudence à la sûreté et à la tranquillité de la commune »}. Les autres membres de la Municipalité restèrent en permanence à la maison de ville.\par
A la même heure, le sous-préfet avertit Polier de ce qui se passait, ajoutant que si on ne lui envoyait pas sur le champ du renfort, il en résulterait le plus grand dommage, les insurgés devenant rapidement plus nombreux et le château n’ayant que 143 hommes pour le défendre. Ce fonctionnaire fidèle s’aperçut bientôt qu’il ne pouvait même plus compter sur le commandant de la garnison.\par
Ce dernier lui annonça en effet, dit-il dans son rapport, que :\par

\begin{quoteblock}
\noindent « Reymond était à nos portes avec sa troupe et qu’il demandait une entrevue ; je répondis que je n’avais rien à faire avec lui, sur quoi le commandant me répliqua qu’honnêtement il ne pouvait la refuser, qu’il connaissait beaucoup Reymond, qu’il avait mangé avec lui à Lausanne et que cet homme ne méritait pas d’être renvoyé ainsi. »\end{quoteblock}

\noindent Le commandant français vit en effet Louis Reymond et revint bientôt auprès du sous-préfet avec les deux délégués de la Municipalité pour l’exhorter encore à ne pas repousser toute négociation. a J’aurais refusé toute réponse à cette demande sans les sollicitations du commandant français et des deux municipaux. On ne cessait de me dire que les gens qui pensent comme moi étaient fort exposés et qu’on ne savait ce qui pouvait arriver, et cela fut même poussé si loin que j’envoyai ma famille à Lausanne où elle n’arriva pas sans avoir essuyé des menaces sur la route. Le commandant français me parlait de se retirer, qu’il était bien ridicule d’avoir tant de bruit pour des paperasses et me refusa tout moyen de défense pour la ville. Tout cela me détermina à traiter ; je vis clairement que j’étais isolé. »\par
Avant de négocier avec Reymond, le sous-préfet avertit encore le Préfet national. Polier s’empressa d’envoyer à Morges son lieutenant Clavel (de Brenles) avec un détachement d’infanterie. Il le chargea de montrer aux insurgés la honte qui rejaillirait sur le Canton s’ils continuaient leur œuvre, de leur annoncer que le gouvernement s’occupait à détruire le plus tôt possible « le levain de division », et de leur rappeler le sort du Valais.\par

\begin{quoteblock}
 \noindent « Engagez-les, lui avait-il encore dit, engagez-les à rentrer dans leurs foyers et à s’en remettre sur l’objet de leurs vœux à la sagesse du gouvernement, sous le Conseil de l’Assemblée des Notables et sous la haute protection du Premier Consul de la République française. »
 \end{quoteblock}

\noindent Pendant ce temps, Louis Reymond avait insisté encore, annonçant que sa troupe s’impatientait et qu’il n’en était presque plus le maître. Mandrot l’invita à venir négocier, mais le chef des \emph{Bourla-Papey} préféra que l’on se rendît auprès de lui.\par
Clavel arriva à Morges à ce moment-là. Le citoyen Muret-Fasnacht alla au devant de lui, lui annonça que la ville était cernée par trois rassemblements et que la compagnie qui l’escortait ne pouvait plus être utile. Clavel entra donc seul dans la localité où il trouva la situation plus grave qu’on ne l’avait cru à Lausanne. Il fit part aussitôt de ce sentiment au Préfet Polier :\par

\begin{quoteblock}
 \noindent « Reymond qui a, dans ce moment, plus de 2000 hommes à ses ordres, disait-il, et avec lequel on se voit dans la nécessité de traiter, va m’envoyer les conditions qu’il veut proposer pour faire rentrer ses troupes dans leurs foyers. L’abandon total du gouvernement nous force malheureusement à nous conduire différemment de ce que l’on aurait dû faire dans d’autres circonstances ».
 \end{quoteblock}

\noindent Mandrot exposa à Clavel que les paysans formaient deux camps principaux, l’un à Tolochenaz, et l’autre au-dessus de ce village. Les insurgés étaient bien armés, extrêmement excités, « échauffés » un peu par le vin et parfois par la faim, ils étaient résolus à aller jusqu’au bout et il était inutile, dans ces conditions, de vouloir leur faire entendre raison. Clavel insista encore, disant que l’on pouvait disposer de troupes suisses et françaises, que ces dernières ne seraient pas attaquées par les paysans, que leur chef était bien disposé à maintenir l’ordre et que l’on ne pouvait pas livrer les archives placées sous la responsabilité des fonctionnaires. Tous les assistants lui firent comprendre qu’il était absolument inutile d’insister.\par
Clavel s’en alla en conséquence avec le sous-préfet, la délégation municipale et le commandant français à dix minutes de la ville, à Rion-Bosson, maison de campagne appartenant au citoyen Warnery-Blanchenay. Il y trouva Louis Reymond, Claude Mandrot, Guibert, et d’autres chefs des \emph{Bourla-Papey.}\par

\begin{quoteblock}
 \noindent « En m’adressant au citoyen Reymond comme chef du rassemblement, raconta Clavel dans une lettre adressée le soir à Polier, je lui représentai les malheurs qu’il était prêt à faire fondre sur sa tête et sur celle des malheureux paysans qu’il commandait. Je l’assurai qu’ils étaient trompés s’ils croyaient que les troupes françaises ne contribueraient pas de tout leur pouvoir à maintenir l’ordre et que les intentions du chef qui les commandait à Lausanne étaient parfaitement annoncées de répondre à la force par la force si cela devenait nécessaire. – Le citoyen Reymond me dit qu’il en avait trop fait à présent pour pouvoir reculer, que la troupe qu’il commandait insistait pour qu’on lui remît les prisonniers de Morges et les archives, et que lui-même essaierait vainement de leur faire changer d’avis sur ce sujet ; qu’ils espéraient être pas dans le cas d’en venir aux mains avec les Français et que ses gens y étaient si peu disposés qu’ils avaient tous une cocarde française dans leur poche qu’ils arboreraient à la première occasion pour fraterniser avec eux. »
 \end{quoteblock}

\noindent Le sous-préfet et le président de la Municipalité passèrent alors dans une autre chambre avec Clavel et insistèrent auprès de lui pour qu’il ne s’opposât pas davantage à une transaction. Le lieutenant de Polier déclara alors qu’il les laissait libres d’agir tout en leur laissant la responsabilité de leurs actes.\par
La conclusion du \emph{Traité de Rion-Bosson} fut la conséquence des discussions qui venaient d’avoir lieu. En voici les clauses :\par
I. Le sous-préfet fera relâcher les prisonniers détenus au château de Morges comme prisonniers de guerre et les fera conduire en Boujean pour être remis au détachement que le dit Reymond enverra pour les recevoir…\par
II. Le malheureux blessé qui est maintenant à Morges y restera en sûreté et sera soigné par qui lui conviendra et il ne lui arrivera rien, mais il sera relâché et libre dès ce moment pour se retirer quand il le voudra ; le même article regarde tous ceux qui peuvent avoir été blessés.\par
III. Les titres des droits féodaux contenus dans la Maison nationale de Morges seront remis à quatre personnes envoyées par le citoyen Reymond ; on fera aussi inviter les particuliers qui ont de pareils titres à les remettre au même instant aux mêmes dites quatre personnes.\par
IV. La troupe du citoyen Reymond ne pourra entrer en armes à Morges, mais pourra y envoyer un certain nombre d’individus faisant partie du même rassemblement pour y acheter ou s’y pourvoir de ce qui leur conviendra, moyennant qu’ils y arrivent sans armes et que leur chef promette qu’ils y observeront le meilleur ordre.\par
V. La troupe du citoyen Reymond ne pourra aussi entrer en armes dans les communes de Lully, Tolochenaz, Echichens, Lonay et Préverenges, et les personnes et propriétés de toutes les communes ci-dessus, ainsi que de celle de Morges, soit publiques, soit particulières, seront complètement et parfaitement respectées sans qu’on puisse les arrêter et maltraiter sous aucun prétexte, tout comme les individus du rassemblement qui s’y présenteront sans armes n’y éprouveront aucun dommage en leurs personnes et en leurs biens.\par
La présente convention, faite et exécutable sur le champ et par considération pour le bien public et la tranquillité générale et par égard, de la part du sous-préfet, aux représentations qui lui ont paru fondées, du Président municipal de Morges, des municipaux Bourgeois, Mercier, Guibert et de nombres d’autres citoyens du dit lieu. »\par
Le traité de Rion-Bosson fut exécuté dès le même jour. Les prisonniers reprirent leurs armes et allèrent rejoindre les contingents de leurs villages respectifs. Louis Reymond envoya à Morges quatre commissaires pour prendre livraison des archives de la nation et des particuliers ; c’étaient les citoyens Jaïn, ancien membre de la Chambre administrative ; Claude Mandrot, ex-juge de district ; Sterky, greffier du tribunal, et Dautun. Un certain nombre de voitures furent chargées de ces liasses nombreuses de papiers, registres, grosses, « cottets » et parchemins de tout genre, et conduites dans la direction de Tolochenaz.\par
Le traité de Rion-Bosson fut assez mal exécuté à d’autres égards. Beaucoup d’insurgés entrèrent, paraît-il, dans la ville sans avoir préalablement déposé leurs armes ; quelques-uns menacèrent même le sous-préfet. Le commandant français ne fit rien pour porter remède à cette situation. Quant à la Municipalité, qui n’avait pas voulu jusqu’alors fournir les choses les plus nécessaires à la troupe, elle se décida à envoyer du vin au château et une provision de pain et de fromage aux insurgés, qui avaient du reste le plus grand besoin d’être ravitaillés.\par
La capitulation de Morges était un grand succès pour les \emph{Bourla-Papey. –} Leur joie fut grande ; elle se montra de la manière la plus bruyante pendant la soirée et une partie de la nuit. Toute la contrée environnante retentit de leurs chants et de leurs cris pendant que, dans le camp de Tolochenaz, s’élevaient de grandes flammes et une immense lueur, annonçant à tout le pays le sort des archives de Morges.
\section[Le Manifeste de Reymond]{Le Manifeste de Reymond}
\noindent Pendant les négociations qui avaient eu lieu à Rion-Bosson, Clavel avait prié Louis Reymond de lui indiquer quelles étaient ses intentions, celles des troupes qu’il commandait et ce qu’ils avaient à demander au gouvernement ou à ses représentants. Le chef des \emph{Bourla-Papey} lui remit, en réponse, et par l’intermédiaire du citoyen Warnery, une note qui renferme à ce sujet les indications les plus précises. En voici l’essentiel :\par

\begin{quoteblock}
 \noindent « Il [Reymond] demande, au nom de sa troupe, que l’on mette à sa disposition tous les titres quelconques, quel que puisse en être le nom, relatifs aux droitures féodales, quel que puisse être le dépôt de ces titres, qu’ils existent dans les archives de Lausanne ou dans d’autres archives publiques ou particulières, ou enfin dans tous les autres lieux où il pourrait en être découvert ; les représentants du gouvernement s’engagent de mettre de même à sa disposition tous les titres qui pourraient avoir été transportés hors du canton ou hors de l’Helvétie… ; que ceux déposés dans les archives quelconques de Lausanne lui soient remis dès demain matin et que tout le reste suive avant le licenciement de ses troupes, et qu’il espère obtenir une fois ces conditions remplies. – Il demande, en outre, qu’amnistie générale soit prononcée pour tous les citoyens quelconques qui ont pris part à cet armement. Il demande, enfin, que réponse leur soit donnée à minuit sur les objets pour lesquels il accorde un délai… »
 \end{quoteblock}

\noindent Louis Reymond demandait enfin une suspension des hostilités jusqu’au moment où ses demandes recevraient une réponse.\par
Le chef des \emph{Bourla-Papey} fit encore plus et mieux. Lorsqu’il apprit, le 7 mai, que le gouvernement central avait envoyé, dans le Canton du Léman, un Commissaire, il se hâta de s’adresser à ce dernier pour lui soumettre les causes du soulèvement, les griefs des paysans, et les résolutions qu’ils avaient prises. Cette lettre de Louis Reymond au citoyen Kuhn fait penser aussitôt à la fameuse adresse aux autorités du Canton du Léman. Elle constitue, en quelque sorte, le manifeste de l’insurrection de 1802, mais un manifeste aussi habile que net et énergique. C’est le document politique principal de cette Jacquerie vaudoise. Il est daté de St-Saphorin-sur-Morges, le 7 mai. Le voici :\par

\begin{quoteblock}
 \noindent « Le Commandant des troupes vaudoises des divers districts du Canton de Vaud, au citoyen ministre de la Justice et de la Police, actuellement Commissaire du gouvernement de la République helvétique dans le Canton de Vaud,\par
 « Citoyen Ministre :\par
 « Organe des braves troupes sous mes ordres, je n’ai pu qu’être satisfait, ainsi que tous les braves citoyens que j’ai l’honneur de commander, du choix qu’a fait le Gouvernement de votre personne, pour vous rendre dans ce canton ; la pleine confiance que l’on a en vos lumières et en votre intégrité les invite à mettre sous vos yeux tous les griefs qu’ils ont à former.\par
 » Vous le savez, Citoyen Ministre, au moment de la révolution, il fut promis aux agriculteurs la libération de toutes redevances féodales, et comment n’y auraient-ils pas cru, lorsque des proclamations du premier magistrat du canton, émanées pour ce jour solennel où ils furent appelés à prêter serment en face de l’Être suprême, leur assuraient qu’il n’existerait plus pour eux de seigneurs et, par conséquent, de redevances seigneuriales. Pendant deux ans ils jouissent des effets de ces promesses ; mais la troisième année arrivée, ils voient des arrêtés rigoureux et injustes dans leurs formes sortir d’une autorité provisoire, et qui empiraient leur sort. Fondés sur des arrêtés pareils, les propriétaires de ces redevances, tous ennemis du nouvel ordre de choses, s’en sont prévalus pour tyranniser leurs vassaux, auxquels tous moyens de se faire entendre étaient enlevés à raison des formalités que prescrivaient ces arrêtés. Il y a plus, c’est qu’outre le payement de ces redevances féodales arréagées, on exige d’eux des impôts continuels partant d’autorités auxquelles un tel pouvoir, ne peut et ne doit jamais être attribué, de manière que tous ces impôts, tant directs qu’indirects, réunis au payement des censes et des dîmes anéantissent toutes leurs ressources et les ont jetés dans le désespoir.\par
 » D’où il résulte : 1° que le mouvement auquel ils se sont portés n’a d’autre cause que la manière dure et tyrannique dont on les traite depuis quelque temps, l’exaction rigoureuse des dîmes et des impôts répétés, la menace de nouveaux impôts plus insupportables encore ; par la mise en exécution d’un système d’imposition aussi oppressif qu’absurde, la ruine, enfin, de plusieurs agriculteurs, qui ne peuvent plus suffire au payement de leurs intérêts. – 2° Que loin de mériter les qualifications odieuses dont ils savent qu’on les décore, loin d’être des rebelles, ils ne sont que des infortunés réduits au désespoir et qu’on cherche à flétrir, mais tout malheureux qu’ils sont, ils donnent au gouvernement actuel l’assurance positive de leur attachement inviolable, de leur disposition absolue de défendre le gouvernement et de l’appuyer si cela était nécessaire, contre les manœuvres de ceux qui voudraient le renverser pour ramener un ordre de choses qu’ils ne verraient rétablir qu’avec horreur. – 3° Qu’ils sont assez convaincus de la sagesse du gouvernement pour croire qu’il n’écoutera, dans le parti, qu’il a à prendre en ce moment, aucune des suggestions perfides des ennemis de toute liberté, des partisans des privilèges, de ces caméléons politiques toujours au service du pouvoir qui domine, toujours se pliant à tous les événements, toujours portés aux mesures violentes, tout en se répandant en paroles mielleuses \footnote{On verra facilement dans cette phrase une attaque violente contre la politique du Préfet Polier qui occupait cet emploi depuis 1796.}.\par
 » Enfin, Citoyen Ministre, si séduit par les conseils perfides de ces hommes qu’on vient de vous dépeindre, vous ordonniez des moyens de rigueur et des voies de fait pour opposer au projet invariablement pris par mes troupes, elles vous déclarent qu’outre la résistance que vous trouverez en elles, elles émettront incontinent un vœu de réunion à la République française, sous la protection de laquelle elles se mettent dès cet instant et dont elles arborent déjà les couleurs.\par
 Agréez, etc. {\scshape Reymond.}
 \end{quoteblock}

\section[De Morges à Lausanne]{De Morges à Lausanne}
\noindent Au moment où Louis Reymond lançait son manifeste, les contingents placés sous ses ordres se signalaient par de nouvelles spoliations dans différentes localités.\par
Le 5 mai, à sept heures du soir, trente hommes conduits par Pittet, de Pampigny, envahirent la maison du citoyen La Harpe à Yens \footnote{Le fils aîné du général Amédée de la Harpe.}. Les titres furent jetés par les fenêtres et leur propriétaire obligé, « la baïonnette dans les reins », de signer une renonciation de ses droits.\par

\begin{quoteblock}
 \noindent « Je ne pus empêcher, malgré toute ma vigilance, raconte La Harpe, qu’il me fût volé une belle pipe garnie, en argent, à laquelle j’étais fort attaché, vu qu’elle était un cadeau d’un camarade de régiment à qui elle avait coûté un louis d’or. Il me fut de même volé une carabine qu’un paysan de Lully prit en échange de son fusil dont il avait cassé la crosse ».
 \end{quoteblock}

\noindent Le lendemain, une colonne de paysans, passant à Echichens, village où le sous-préfet de Morges possédait une maison, y commit quelques excès. Un coup de fusil fut tiré contre la propriété de ce fonctionnaire. Pittet, de Pampigny, insinua que si l’on trouvait les deux fils du citoyen Mandrot, il fallait leur casser la tête. Le domestique de ce dernier fut « mis en joue » par un insurgé. Un autre – à instincts plus pratiques – tua une oie et un dernier cassa la crosse de son fusil en cherchant inutilement à en abattre une autre.\par
Le 7, à sept heures du matin, cinquante hommes, sous la direction de Wagnon, de l’Isle, se présentèrent chez Mme de Mestral, à St-Saphorin.\par

\begin{quoteblock}
\noindent « Les archives qui furent brûlées auraient chargé un chariot à deux chevaux. »\end{quoteblock}

\noindent Une personne de ce village dont il a déjà été parlé, la femme de Marc Cart, se distingua par la violence de son langage et en déchirant fiévreusement les parchemins sur lesquels elle sautait et dansait ensuite.\par
Au même moment, Louis Reymond, accompagné de 1500 hommes, vint au château de Vufïlens demander les titres du citoyen de Senarclens. Le désordre dura pendant quatre heures. Les titres, une fois livrés et détruits par le feu, quelques hommes prétendirent que tout n’avait pas été donné et les locaux furent fouillés une seconde fois. Le propriétaire livra aux campagnards quatre chars de vin et Reymond ferma ensuite la porte de la cave. Une demi-heure plus tard, elle fut forcée de nouveau et des coups de fusil tirés contre le plus grand tonneau. Le jardinier reçut un coup de crosse en voulant s’interposer. Le propriétaire descendit encore une fois et livra de nouveau du vin. Les paysans se décidèrent alors à partir après avoir une dernière fois frappé à coups de crosse contre la porte du château.\par
Pendant la même matinée du 7 mai, une troupe aux ordres de Marc Cart, de St-Saphorin et de Wagnon, de l’Isle, arriva tambour battant à Monnaz devant la maison d’Aruffens. Ils y trouvèrent le régisseur, Jean Massin, qui nous a laissé un récit succinct de ce qui se passa.\par

\begin{quoteblock}
 \noindent « Cette troupe, qui avait l’air en partie ivre, dit-il, est entrée en poussant des cris, des hurlements, des jurements, des insultes, des menaces et tirant des coups de fusil ».
 \end{quoteblock}

\noindent Wagnon, avec une quinzaine d’hommes, commença à jeter les titres par les fenêtres.\par

\begin{quoteblock}
 \noindent « Comme on triait les papiers dans les tiroirs, dit Massin, il s’est élevé des murmures, des cris, des menaces de me jeter par la fenêtre ; j’ai été saisi par les bras et éloigné, par la force, des archives, par un nommé Gruaz, maréchal de l’Isle. Dès ce moment, ils ont arraché tous les tiroirs et la boiserie des dites archives et jeté indistinctement en bas les fenêtres tous les papiers, même les papiers de famille. Le tout a été chargé sur un char et conduit sur le chemin tendant à Joulens et brûlé par cette troupe au milieu de leurs danses, de leurs hurlements, malédictions, injures, au bruit du tambour et des coups de fusil… Le chef a ensuite demandé du vin, je leur en ai donné dans une seille et dans des bouteilles qui ont été cassées. »
 \end{quoteblock}

\noindent Comme on peut le supposer d’après ce qui vient d’être dit, la situation était inquiétante pour les possesseurs de fiefs et même pour toutes les personnes qui réprouvaient la révolte des paysans.\par

\begin{quoteblock}
 \noindent « La campagne est tenue de tous les côtés, mandait le sous-préfet de Morges à Polier, et aucun homme seul ne pourrait et ne voudrait s’y hasarder ; chacun se tient chez soi, et plusieurs cachés. »
 \end{quoteblock}

\noindent A Lausanne, la nouvelle des événements de Morges et des environs, avait aussi augmenté l’inquiétude des représentants de l’autorité. Le Préfet national et le Commandant français Veilande se persuadèrent de plus en plus qu’ils ne pouvaient compter sur aucun secours de la ville et des milices du canton. Ne croyant pas à la possibilité de défendre le chef-lieu et ne voulant pas négocier avec Reymond, ils allèrent passer la nuit du 6 au 7 mai à la Cité pour y défendre, avec quelques détachements, le dépôt des titres de droitures seigneuriales. Le Préfet laissa vingt-cinq soldats français dans sa maison de la rue de Bourg pour y protéger sa famille. Il répondait en outre aux demandes que Reymond lui avait fait parvenir par l’intermédiaire de Clavel, que le gouvernement pouvait seul décider d’une question de ce genre. Il y avait donc lieu de s’attendre à une attaque de la ville ; elle n’eut pas lieu.\par
Lorsque Kuhn arriva le lendemain, il était déjà trop tard pour arrêter le développement de l’insurrection. Il fut frappé de la gravité de la situation. Il resta convaincu cependant à son arrivée que, sans négliger les voies de la douceur, il fallait se préparer immédiatement à repousser la force par la force. Il s’adressa tout d’abord à l’ex-sénateur Jules Muret, dont l’influence et la popularité pouvaient être fort utiles dans ce moment et le chargea de se rendre aussitôt que possible auprès des insurgés pour les inviter à rentrer dans leurs foyers. Il ajoutait que la force ferait exécuter cet ordre s’il n’était pas suivi volontairement.\par
» Si cela n’est pas exécuté ce soir, écrivait Kuhn au Petit Conseil, je prierai le commandant français : 1. de laisser un détachement pour la garde du château ; 2. de concentrer le reste de ses troupes disponibles et de s’avancer avec elles contre les rebelles, les sommer de rendre les armes et de rentrer dans leurs foyers. S’ils ne donnent pas suite à cette sommation, on les convaincra par une attaque très vive que les menaces sont sérieuses ».\par
Kuhn retarda cette attaque jusqu’au lendemain afin d’attendre l’arrivée de plusieurs troupes et de laisser aux documents envoyés aux insurgés le temps de produire une impression « bienfaisante ».\par
Le Commissaire ne doutait pas, le jour de son arrivée, du succès de l’un ou l’autre des moyens qu’il venait de choisir pour mettre fin aux troubles. Il changea bientôt d’opinion. Il apprit, en effet, que l’insurrection s’étendait de plus en plus dans le pays, et que, d’autre part, la pacification à laquelle il avait songé ne serait pas obtenue facilement. Jules Muret arriva bientôt auprès de lui et déclara qu’il n’accomplirait la mission dont il venait d’être chargé que moyennant deux conditions. II exigeait tout d’abord que le gouvernement décrétât la suppression totale des droits féodaux ; il demandait ensuite qu’une amnistie complète fût accordée aux \emph{Bourla-Papey.} Ces demandes étaient trop opposées aux instructions données par le Petit Conseil à son Commissaire pour que celui-ci pût les prendre pour base d’une négociation. Après une discussion dont il ne nous a pas laissé le résumé, il modifia cependant ses intentions primitives et résolut d’entrer en relations avec les chefs mêmes des insurgés et de se rendre auprès d’eux le lendemain pour obtenir par la persuasion leur retour dans leurs foyers.\par

\begin{quoteblock}
 \noindent « Si l’on veut pacifier le Canton du Léman, écrivit-il le même soir au Petit Conseil, la Constitution doit contenir cet article : Les décisions au sujet des droits féodaux sont laissées aux cantons sous la condition d’indemniser les propriétaires ».
 \end{quoteblock}

\noindent Pendant la matinée du 7 mai, les contingents insurgés avaient réduit en cendres les archives de diverses localités ; pendant l’après-midi, ils se rapprochèrent de Lausanne et passèrent la nuit suivante dans la région des Plaines du Loup, au-dessus de la Pontaise. Ils montrèrent ainsi leur intention de pénétrer dans le chef-lieu que le citoyen Kuhn mit aussitôt en état de siège. Voici les dispositions qu’il publia à cet effet :\par

\begin{quoteblock}
 \noindent « Le citoyen Chastellain sera chargé de l’inspection de la police dans la localité ; conjointement avec le citoyen Deverre, commandant de la place… Tous les citoyens domiciliés dans cette commune viendront au bureau du commandant de la place, à la maison commune, se munir de cartes de sûreté. Tout citoyen domicilié dans cette commune devra, soit de jour, soit de nuit, être muni de sa carte de sûreté, à défaut de quoi, il sera arrêté et conduit au commandant de place pour être examiné. Il sera placé des postes et des plantons à toutes les portes et avenues de la ville et tout individu qui se présentera pour entrer devra être conduit au commandant de place qui examinera son passe-port et lui délivrera, s’il le juge convenable, une carte de sûreté pour le temps qu’il devra résider en ville… Le commandant de place fera visiter fréquemment les maisons publiques, pintes, cabarets, par des patrouilles, lesquelles arrêteront tous les individus qui ne seront pas munis de cartes de sûreté. Il pourra ordonner, suivant l’exigence du cas, des visites domiciliaires… »
 \end{quoteblock}

\noindent Le lendemain, 8 mai, étant un samedi, il fut ordonné, en outre, que le marché eût lieu hors de la ville, sur la place de Montbenon…\par
Le 8 mai, au matin, J.-J. Cart, ex-sénateur, se présenta au citoyen Kuhn, comme parlementaire des insurgés. Informé, d’autre part, des intentions de ces derniers, le Commissaire venait de faire battre la générale dans la ville. Quinze hommes seulement se présentèrent, parmi lesquels six domestiques du Préfet national.\par
Les paysans s’impatientèrent bientôt et, sans attendre davantage le résultat de la démarche du citoyen Cart, ils entrèrent à Lausanne quelques instants avant neuf heures, au nombre d’environ quinze cents, par les portes de St-Laurent et de Chaucrau.\par
Le défilé des campagnards, avec leurs uniformes variés, eut un immense succès de curiosité pour la plus grande partie de la population et jeta aussi la terreur dans l’esprit de quelques personnes. Beaucoup de paysans portaient à la pointe de leurs fusils ou de leurs baïonnettes des fragments de parchemins. Presque tous avaient des fleurs ou des rubans à leurs chapeaux. Ils paraissaient d’excellente humeur, poussaient leur cri habituel de ralliement : \emph{Paix aux hommes, guerre aux papiers}, et, tambour battant, suivaient allègrement leurs chefs etledrapeau vert qui flottait au milieu d’eux. Les postes de troupes françaises qui se trouvaient aux portes et dans la ville se bornèrent à sortir de leurs corps de garde et à se ranger pour voir passer les campagnards. « Nous sommes vos amis », leur disaient ces derniers.\par
La troupe des \emph{Bourla-Papey} traversa une partie de la ville et alla se ranger sur la place de la Palud. Le Commissaire s’y rendit en toute hâte, accompagné du Commandant français Veilande. Des troupes françaises et helvétiques arrivèrent de leur côté et se formèrent en ligne en face des insurgés. Reymond demanda les archives, des vivres, la suppression des droits féodaux et une amnistie générale. Le Commissaire le somma de sortir de la ville. Louis Reymond demanda alors de pouvoir parler au citoyen Veilande, qui lui réitéra la même sommation et lui parla, en même temps qu’à sa troupe, avec la plus grande énergie. Cette conduite du Commissaire et de la troupe jeta quelque incertitude et quelque indécision dans l’esprit de Reymond et de ses gens. Il réfléchit, en outre, à sa situation stratégique fâcheuse au centre d’une ville occupée par une force armée nombreuse qui pouvait facilement lui fermer toutes les issues. Il se décida, en conséquence, à obéir à la sommation du Commissaire et du Commandant Veilande. Il ordonna la retraite, sortit de la ville et alla prendre position sur la place de Montbenon \footnote{Ce fait est raconté d’après le rapport présenté le lendemain matin par Kuhn au Petit Conseil. La plupart des récits de cette journée représentent Reymond et sa troupe pénétrant dans la rue de Bourg où habitait le Préfet national et au-dessus de laquelle la colonne des \emph{Bourla-Papey} fut arrêtée par un bataillon helvétique descendant Martheray.}.\par
Le commandant Veilande suivit de près les \emph{Bourla-Papey} avec des troupes françaises et les disposa de manière à exercer sur eux une impression aussi grande que possible. Le Commissaire arriva bientôt aussi et eut une entrevue avec Reymond en présence des deux armées. Il lui communiqua les lettres du général Montrichard, commandant en chef des troupes françaises en Suisse, lettres adressées au Petit Conseil et au citoyen Veilande, et attestant sa volonté de soutenir le gouvernement et de contribuer à dissoudre l’insurrection. Sans paraître attacher une importance considérable à ces documents, Reymond continua à demander les archives de Lausanne qui lui furent de nouveau refusées. Il demanda ensuite, comme Muret et Cart l’avaient fait de leur côté, la suppression complète des droits féodaux et une amnistie totale pour tout ce qui s’était passé jusqu’à ce moment. Il ajouta que si l’on ne faisait pas droit à ces demandes, sa troupe voterait l’annexion à la France. « Le Commissaire refuse l’un et l’autre. Il s’élève en ce moment, parmi les campagnards, des cris montrant que leur chef exprimait bien leur manière de voir. Du reste, presque tous portaient déjà la cocarde tricolore et avaient signé une adresse au Premier Consul \footnote{Rapport de Kuhn au Petit Conseil.}.\par
Ne pouvant obtenir par la persuasion le licenciement des \emph{Bourla-Papey} et ne voulant pas dans ce moment user de la force des armes pour l’obtenir, le Commissaire du gouvernement n’avait pas des pouvoirs suffisants pour accorder à Louis Reymond ce qu’il exigeait. Il fut convenu en conséquence que le citoyen Kuhn se rendrait aussitôt à Berne pour conférer avec le Petit Conseil et qu’en attendant son retour – pour lequel il demandait quarante-huit heures – les campagnards iraient camper près de St-Sulpice.
\section[De Lausanne à Berne]{De Lausanne à Berne}
\noindent Il était onze heures du matin quand le Commissaire conclut avec le chef des \emph{Bourla-Papey} l’espèce de capitulation dont il vient d’être parlé. Il partit pour Berne à deux heures après midi.\par
On a vu plus haut quelles étaient ses intentions un jour auparavant, au moment de son arrivée à Lausanne. On peut facilement se rendre compte maintenant de la rapidité avec laquelle sa manière de voir s’était modifiée pendant vingt-quatre heures. Il lui avait suffi de se trouver au milieu de ce canton profondément troublé et de cette population surexcitée pour comprendre que la situation était bien différente qu’on ne la croyait à Berne. Il put aussi s’apercevoir que les insurgés étaient en communauté d’idées sur le fond de la question avec la grande majorité de la population restée dans ses foyers et enfin avec les hommes politiques les plus influents du pays. Il put enfin probablement se convaincre déjà un peu que les chefs des troupes françaises n’étaient pas également sûrs, que si Veilande montrait la fermeté la plus remarquable, il n’était pas secondé par tous ses frères d’armes et que même plusieurs – Turreau surtout – soutenaient plus ou moins ouvertement les \emph{Bourla-Papey.} Dans une situation semblable et en face de la fermeté des paysans, il s’était déjà convaincu de l’impossibilité de rétablir le calme sans donner à la population des campagnes des garanties au sujet de ce qu’elle réclamait. La capitulation, conclue dans un moment où il eût été facile de forcer les insurgés à mettre bas les armes, persuada d’autre part ces derniers qu’ils pouvaient compter sur une impunité à peu près complète. L’insurrection se répandit en conséquence de plus en plus et les spoliations de titres continuèrent au milieu de désordres importants.\par
Quelques heures après le départ du Commissaire, Potier expédia en effet à Berne un courrier porteur des nouvelles les plus fâcheuses.\par

\begin{quoteblock}
 \noindent « Les avis alarmants se succèdent, disait-il. Les sous-préfets de Rolle, de Nyon, d’Aubonne, d’Echallens, d’Orbe et d’Yverdon « décrivent que les mêmes violations de domicile et de propriété, les mêmes prises d’armes, soit de gré, soit de force s’exécutent dans leurs districts. Rien ne peut exprimer la terreur qu’inspire un tel état de choses aux amis du gouvernement, de l’ordre et de la paix. Le chef avoué de cette grande insurrection, ceux non avoués, mais connus, le changement déplorable dans la moralité de ce peuple, l’état d’exaspération où ses meneurs l’ont porté, tout justifie l’effroi qui glace tous les cœurs. Depuis le départ de Kuhn, les avis se multiplient que les campagnards armés ne tiendront pas l’armistice promis ce matin pour trois jours et que cette nuit même ils se proposent de rentrer à Lausanne. Le chef Veilande est plein d’énergie, de talent et de dévouement, mais il n’est pas suffisamment secondé par le citoyen Deverre, commandant la place, qui est un simple lieutenant… Tous les bons citoyens du Léman tendent les bras vers le Petit Conseil et lui exposent que des mesures grandes et énergiques peuvent seules nous tirer du précipice dans lequel nous sommes entraînés. »
 \end{quoteblock}

\noindent Ce qui alarmait, en outre, le Préfet national, c’était la conduite du général Turreau, toujours en relations avec les plus exaltés des meneurs vaudois, toujours disposé à leur donner des conseils et des encouragements. Une lettre écrite à ce sujet le 8 mai par le citoyen Deloës, sous-préfet d’Aigle, renfermait des renseignements précis et intéressants.\par

\begin{quoteblock}
 \noindent « Les citoyens Guibert sont allés auprès du général Turreau en qualité de députés, disait-il, pour demander qu’il n’envoyàt pas de troupes ; ils ont été très bien reçus, invités, etc. On attendait hier sa réponse à Bex ; le général est à Martigny… Les dits députés étant ivres à leur retour ont été entrepris et ont dit qu’ils rapportaient la certitude qu’il n’y avait point de troupes ; on leur a dit qu’ils étaient dans l’erreur, ce qui les a fort capotisés \emph{(sic].} On leur a demandé des détails sur leurs projets ; ils ont parlé des archives, droits féodaux, etc., et ont ajouté que quant au Préfet, il fallait qu’il passât le goût du pain \emph{(sic].} Celui qui a passé, Mandrot, ne s’est arrêté à Bex que pour se rafraîchir ; il a filé en Valais, disant aller chez le général… Guibert a dit qu’on respecterait les Français, mais qu’on hâcherait les Suisses qui viendraient. »
 \end{quoteblock}

\noindent On peut voir, par la lettre pittoresque du sous-préfet d’Aigle, combien Polier était détesté des \emph{Bourla-Papey.} Ce sentiment était sensiblement le même chez les hommes qui inspiraient les campa – gnards ; ils l’exprimaient toutefois d’une manière plus parlementaire. Polier était un homme d’une moralité et d’une austérité religieuse reconnues de tous, mais qui portait ombrage à beaucoup de personnages influents dont les principes étaient empreints de la philosophie voltairienne de la fin du XVIII\textsuperscript{me} siècle. Ses idées politiques très modérées le séparaient aussi d’un très grand nombre de ses concitoyens qui voyaient en lui un partisan secret des aristocrates et par conséquent un obstacle à la réalisation de leurs vœux. Ces derniers sentiments sont exprimés dans une lettre écrite le 11 mai 1802 par Aug. Pidou, sénateur, et adressée de Berne à son ami Jaïn, à Morges, ancien membre de la Chambre administrative.\par

\begin{quoteblock}
 \noindent « Je vous conjure, disait-il, qu’on aille de tous les districts auprès de Kuhn ; qu’on lui parle, qu’on lui écrive ; qu’on empêche qu’il ne soit circonvenu par la clique nobiliaire. J’ai de fortes raisons de croire qu’il l’est déjà sur quelques points : ces gens-là sont comme des serpents : ils se replient et sont là quand vous les croyez bien loin. Kuhn a fait son premier voyage avec Monsieur Constant d’Hermenche, beau-fils du Préfet, et son second, avec Monsieur de Crousaz, neveu de ce même Préfet ; bon moyen pour arriver bien instruit ! Sans compter ceux qui l’auront attendu à sa descente de voiture, complimenté, choyé, flatté, etc. Encore une fois, soyez alerte ! Parlez-lui surtout du grand arbre qui nuit à votre courtil, et que nous voudrions qu’on déplaçât pour en substituer un autre dont l’influence fût moins maligne. Qu’on ne fasse pas une égratignure à cet arbre, qu’on ne lui enlève pas une feuille, mais qu’il cesse de nous ombrager, de nous dominer \footnote{Choix de lettres et documents tirés de papiers de famille, II. 52. On aura reconnu le « grand arbre » dans la personne du Préfet Polier.} . »
 \end{quoteblock}

\noindent Kuhn avait déjà montré pendant son petit séjour dans le Canton du Léman, qu’il savait agir avec beaucoup d’indépendance et s’affranchir des idées et des désirs de ceux qui l’entouraient le plus. Polier ne fut évidemment pas satisfait de la conduite du Commissaire sur la place de la Palud et surtout sur celle de Montbenon. La lettre alarmante qu’il expédia à Berne quelques heures après le départ de ce magistrat était destinée sans doute, dans une certaine mesure, à faire prévaloir dans les délibérations du Petit Conseil, ses propres vœux et ceux de son parti. Il y réussit du reste dans une grande mesure.\par
Kuhn arriva à Berne le 9 mai à l’aube du jour, et dès les premières heures de la matinée, le Petit Conseil s’assembla. Le Commissaire lui montra la gravité du mouvement insurrectionnel, le manque complet de dévouement de la population civile, l’insuffisance notoire des troupes dont il pouvait disposer. Les députés du Léman, à l’Assemblée des notables, ayant été appelés ensuite, il fut pris en leur présence les décisions qui suivent :\par

\begin{quoteblock}
 \noindent « Le Petit Conseil, ensuite des rapports qui lui ont été présentés par le citoyen Kuhn sur la situation du Canton du Léman et en particulier sur les propositions faites par les chefs des insurgés comme conditions auxquelles ils veulent se retirer dans leurs foyers, arrête ce qui suit : 1° Ces propositions sont rejetées et le Commissaire du gouvernement reçoit l’instruction positive de n’entrer en aucune capitulation avec les rebelles ; 2° La force sera employée s’il est nécessaire pour dissiper les rassemblements armés des insurgés ; 3° Le général de division Montrichard, commandant en Helvétie, est invité à mettre en œuvre tous les moyens à sa disposition pour rétablir l’ordre et la tranquillité dans le Canton du Léman. »
 \end{quoteblock}

\noindent Le Petit Conseil s’adressa ensuite au général Montrichard et au ministre de France, Verninac. Il pria le premier de bien vouloir se rendre luimême à Lausanne :\par

\begin{quoteblock}
\noindent « Votre présence, lui disait-il, dissipera l’erreur dans laquelle sont un très grand nombre de paysans… Nous vous prions de vouloir bien saisir ce moyen, le seul qui reste peut-être pour sauver le Léman des maux de la guerre ou, si elle doit éclater, pour la terminer promptement. »\end{quoteblock}

\noindent Au second, il exposa la situation et le laissa juge de la conduite à tenir.\par
Le général Montrichard répondit le même jour qu’il donnait l’ordre au général Amey de se rendre à Lausanne, d’y prendre toutes les dispositions nécessaires, d’employer les moyens de conciliation les plus propres à faire rentrer les rebelles dans le devoir, de les sommer de se dissoudre et de rejoindre leurs foyers, enfin de repousser la force par la force s’il était obligé d’en venir à ces extrémités. »\par
Quant au ministre Verninac, il adréssa au Petit Conseil une lettre qui, par sa netteté, sa franchise et sa bienveillance, contrastait avec la conduite un peu équivoque de la France à ce moment-là et qui rendit un peu de confiance au gouvernement.\par

\begin{quoteblock}
 \noindent « Parmi les excès dont les chefs des insurgés se sont rendus coupables, disait Verninac, le plus criminel sans doute est celui de s’être servi du nom français pour porter à la révolte des agriculteurs simples et paisibles. Le gouvernement français n’entendra pas sans la plus vive indignation qu’ils aient osé se couvrir des couleurs françaises ; qu’ils se soient flattés de n’être point désapprouvés par lui et qu’ils aient menacé les autorités helvétiques d’une injurieuse émission de vœu de réunion à la France. Les dispositions que le général Montrichard vient de faire pour coopérer au rétablissement de l’ordre montreront aux instigateurs de l’insurrection de quel œil la France voit dans un pays allié ses couleurs sur le front des rebelles ; quel jugement elle porte de leurs vues et de quel poids est pour elle l’autorité de leurs suffrages. »
 \end{quoteblock}

\noindent Cette déclaration diplomatique spontanée étonna quelques personnes, donna des craintes à beaucoup et laissa supposer que le Ministre ne devait pas avoir sur ce point des idées conformes à celles de son gouvernement. Auguste Pidou écrivait à son ami Jaïn qu’après cela il ne restait plus aux \emph{Bourla-Papey} qu’à poser les armes.\par

\begin{quoteblock}
 \noindent « Nous n’aurions jamais osé croire à une déclaration aussi catégorique, ajoutait-il ; c’est véritablement comme un sceau d’eau froide qu’on leur a jeté sur la tête. Qu’il est fâcheux qu’on ait fait cette levée de boucliers dans cette circonstance pour nous ôter ici (à Berne) tout poids et tout crédit. »
 \end{quoteblock}

\noindent A Paris, on feignit de se fâcher de que ce Verninac s’était interposé. Il aurait dû savoir que la France \emph{« ne se mêlait en rien de ce qui se passait dans l’intérieur des autres États »}. Je vous garantis que cette comédie a été jouée, écrivait Henri Monod, qui habitait alors Paris, comme s’il n’y avait de jugement que dans une tête et que tout le reste ne fût qu’un tas d’oisons. »
\section[Les Gamaches]{Les Gamaches}
\noindent A la suite de l’armistice conclu sur la place de Montbenon, Louis Reymond se mit en marche avec toute sa troupe par la route de Morges et alla camper dans la région de St-Sulpice et d’Ecublens, sur cette plaine où, deux ans auparavant, Bonaparte avait passé son armée en revue avant de passer le St-Bemard. De tous côtés, le public et les autorités d’un certain nombre de communes leur envoyèrent des vivres. Des réquisitions furent faites aussi dans beaucoup de villages.\par
Fatigués par quelques jours de campagne et profitant de l’armistice qui était intervenu, un grand nombre de \emph{Bourla-Papey} quittèrent l’armée provisoirement et se rendirent dans leurs communes respectives, afin de voir leur famille, leurs amis, et aussi de renseigner le pays et l’encourager à soutenir l’insurrection. Ils furent, du reste, remplacés par des individus isolés et des contingents qui, après avoir brûlé les archives de leur contrée, voulurent se joindre à l’armée principale.\par
Le camp de St-Sulpice, appelé communément le camp des Gamaches – non à cause des banquets plantureux qui y auraient été faits, mais bien plutôt du nom des grandes guêtres blanches que portaient les paysans – fut, pendant deux jours, la grande préoccupation du pays. On y vit accourir, le dimanche 9 mai, des flots considérables de promeneurs de tout âge et de tout sexe.\par
Les citadins voulurent aller se rendre compte de l’importance du mouvement, de l’organisation de cette armée d’un nouveau genre et fraterniser aussi avec les insurgés. Des villageois arrivèrent de toutes parts, quelquefois de fort loin, apportant des vivres aux parents et amis. Les femmes, les enfants, les jeunes gens, les Lausannois et les Morgiens se mélangèrent aux Gamaches qui passèrent des heures agréables au milieu d’une abondance – provisoire du reste – dont quelques-uns, sans doute, abusèrent au milieu du grand enthousiasme qui animait cette foule.\par
Pendant que les « Gamaches » festoyaient un peu, les chefs de la troupe et les meneurs politiques qui les suivaient ou étaient venus les rejoindre, se concertaient sur les moyens de fortifier et d’augmenter encore l’insurrection. Le très fougueux Claude Mandrot rédigeait, sur une caisse de tambour, des adresses de réunion à la France, en compagnie du juge Warnéry chez lequel avait été signé le traité de Rion-Bosson. Il rassurait aussi ceux qui craignaient que les troupes françaises ne voulussent combattre l’insurrection.\par

\begin{quoteblock}
\noindent – « N’ayez pas plus peur des Français que de moi, leur disait-il. Ils ne se mêleront pas de cette affaire. »\end{quoteblock}

\noindent Le dragon Chamot, de Penthaz, qui voulait se hausser à la hauteur de l’éloquence de Louis Reymond, mais en remplaçant les arguments par des invectives, menaçait de la prison, de la fusillade, de la guillottine, ceux qui \emph{« ne voulaient pas de la réunion. »} Il était cependant applaudi par ses auditeurs.\par
Reymond, de son côté, ne perdait pas un moment et sentait le poids de sa responsabilité. 11 avait peut-être hésité à prendre la direction de l’armée des \emph{Bourla-Papey.} Maintenant que le mouvement avait gagné un certain nombre de districts, il s’agissait d’en assurer le succès par tous les moyens possibles. C’est lui qui recevait les rapports des différentes colonnes qui parcouraient le pays ; c’est lui qui expédiait les courriers dans toutes les directions, faisant appel à toutes les bonnes volontés, secouant la torpeur des craintifs, des attardés, de ceux qui hésitaient encore. Il demandait de nouveaux contingents et des vivres. Voici, entre beaucoup, un exemple de ces adresses :\par

\begin{quoteblock}
 \noindent {\scshape liberté} \emph{9 mai.} {\scshape égalité}\par
 « Le Commandant des troupes vaudoises armées pour l’abolition des droits féodaux,\par
 « Aux braves des communes de Prilly, Jouxtens et Mézery,\par
 « Vous êtes invités à vous joindre le plus tôt possible à l’armée sous mes ordres pour pouvoir agir de concert.\par
 « Au Camp près Ecublens, sous Lausanne.\par
 \emph{« P-S.} Des vivres autant que possible devront être fournis par les communes ou les particuliers qui marcheront.\par
 « Vos frères vous attendent pour venir partager leur gloire et sauver la patrie, écrivait-il le 10 mai aux Lausannois. Hâtez-vous de les joindre avec des vivres en suffisance. »
 \end{quoteblock}

\noindent Des adresses de ce genre furent envoyées jusqu’à Concise, à la frontière neuchâteloise. \emph{« Le rendez-vous est à Mathoud \emph{(sic)} demain matin à trois heures »}, y lisait-on. Presque toujours, au bas de ces adresses se trouvait cette recommandation : \emph{« Vous êtes prié de faire passer des copies à vos villages voisins afin qu’ils n’ignorent pas l’invitation. »}\par
Pendant cette journée du dimanche 9 mai, les autorités officielles se montrèrent plus craintives que le public à l’égard des \emph{Bourla-Papey.} Le \emph{Nouvelliste Vaudois} dit qu’au chef-lieu \emph{« on craignit que les insurgés, sans attendre la réponse de Berne, et voyant que les troupes qui sont à Lausanne se renforcent toujours, ne tentassent quelque coup de main. »} On craignit aussi un peu qu’une colonne de paysans venue d’une autre direction, n’entreprît une attaque ou que même un parti en ville ne formât quelque projet. La Municipalité prit en conséquence, de concert avec le commandant de place, les mesures suivantes :\par

\begin{quoteblock}
 \noindent « Dans le cas d’alarme, chacun éclairera ses fenêtres sur la rue et tous les citoyens fermeront leurs portes, boutiques et magasins et resteront chez eux. Toutes les pintes, traiteries, cafés et autres endroits publics seront fermés à huit heures et demie ; les citoyens qui s’y trouveront après cette heure là seront conduits à la maison de commune… Après huit heures du soir, tout citoyen qui sera rencontré n’étant pas muni d’une carte de sûreté, sera arrêté et conduit à la maison de commune… Tout rassemblement même non armé est défendu ; sera considéré comme rassemblement tout groupe où il y aura plus de six personnes. Tout individu annonçant publiquement des nouvelles alarmantes, appelant aux armes ou provoquant à la désobéissance et au mépris des autorités sera immédiatement arrêté et interrogé. Passé neuf heures, toute personne rencontrée sans lumière dans les rues sera conduite à la maison de commune· »
 \end{quoteblock}

\noindent Le dépôt important d’archives qui se trouvait à Lausanne était le point de mire des insurgés et donnait de perpétuelles inquiétudes aux représentants du Petit Conseil. Celui-ci proposa que ce dépôt fût transporté à Berne. La Chambre administrative décida le 8 mai que :\par

\begin{quoteblock}
\noindent « sous la protection des troupes françaises, elles [les archives] étaient autant en sûreté ici qu’à Berne. L’emballage prendrait beaucoup de temps et le convoi ne pourrait opposer une résistance suffisante à une levée en masse qui aurait un motif et un appui puissant dans l’espérance de réussir. »\end{quoteblock}

\section[Reculade du Commissaire]{Reculade du Commissaire}
\noindent Kuhn rentra à Lausanne le 10 mai à six heures du matin.\par

\begin{quoteblock}
 \noindent « J’ai trouvé la situation plus mauvaise, écrivit-il au Petit Conseil. Les hordes armées qui parcourent le pays ne se bornent plus au pillage des archives, mais commettent d’autres excès. Chaque particulier s’adresse à moi, mais il m’est impossible de prendre soin de la sûreté générale en même temps que de celle des individus. Il nous faudrait six à huit colonnes de cinq à six cents hommes chacune pour dissiper cet orage. Nous n’avons pas plus de 800 hommes dont la moitié seulement est disponible. »
 \end{quoteblock}

\noindent Kuhn apprit dès le matin que les \emph{Bourla-Papey} campés à St-Sulpice devaient avoir l’intention de venir attaquer Lausanne. Les autorités étaient si craintives en face de l’insurrection qu’à chaque instant et au moindre rapport, elles croyaient voir déjà l’armée de Louis Reymond aux portes de la ville.\par
Toutes les troupes françaises et helvétiques furent mises sur pied. On ordonna aux citoyens de fermer leurs portes et de rentrer chez eux. Des sentinelles furent postées dans toutes les rues. Des détachements de troupes furent chargés de surveiller les routes principales aboutissant à Lausanne.\par

\begin{quoteblock}
 \noindent « Il s’agissait de leur montrer [aux insurgés] qu’on ne les craignait pas, raconte Kuhn. L’intention était d’aller les attaquer si on pouvait le faire avantageusement, sinon de se contenter d’une simple démonstration, de leur communiquer les lettres de Montrichard et de Verninac, et de les sommer de se rendre. »
 \end{quoteblock}

\noindent Une troupe que le Commissaire dit avoir été composée de 400 hommes, et les journaux du temps de 6 à 700 hommes, fut rangée en bataille sur la place de Montbenon sous le commandement du citoyen Veilande. Elle se composait d’infanterie française et suisse, de cavalerie et d’artillerie suisses. Cette force armée partit par la route de Morges, suivie de près par une voiture contenant le Commissaire Kuhn et le Sénateur Pellis.\par
Le représentant du gouvernement helvétique disait dans sa lettre adressée à Berne, le soir, qu’il avait voulu montrer aux insurgés qu’il ne les craignait pas. Ce qui se passa à St-Sulpice montre que cette assurance ne fut pas de longue durée et que les \emph{Bourla-Papey} firent sur lui une impression plus grande qu’il ne l’avait supposé.\par
Il s’aperçut dès le premier moment que les campagnards étaient résolus, opiniâtres, exaltés même et excités par une campagne de plusieurs jours. Leur nombre avait augmenté et, quoiqu’ils ne constituassent pas une armée organisée comme les troupes françaises ou helvétiques, ils n’en savaient pas moins, dans un moment essentiel, se soumettre à une certaine discipline et adopter une formation tactique capable d’inspirer du respect.\par
La rencontre du Commissaire et des \emph{Bourla-Papey} est racontée de diverses manières.\par
La colonne du commandant Veilande ne s’arrêta que lorsqu’elle fut en présence des insurgés et à bout portant, dit le \emph{Nouvelliste vaudois.} De longues conférences commencèrent aussitôt. Reymond et les autres chefs des campagnards tinrent les mêmes discours qu’auparavant, continuèrent à protester qu’ils ne voulaient point se battre avec les Français, qu’ils ne voulaient qu’affranchir leurs terres, qu’ils étaient amis de la Grande Nation et qu’il voulaient se donner à elle· Kuhn leur représenta que leur sort serait certainement allégé, que depuis la dernière révolution du 17 avril, on y travaillait et qu’au moment où une bonne constitution allait fixer le sort de la Suisse, il ne fallait pas que les Lémans se déshonorassent par des insurrections.\par
Ce récit de personnes qui devaient être bien au courant des faits, mais qui voulaient aussi peutêtre arranger ces derniers à leur façon, ne nous montre pas comment Veilande et Kuhn, allant « attaquer » les \emph{Bourla-Papey}, ne firent que discuter avec eux. La vérité doit être bien plutôt dans la lettre écrite le même soir par le Commissaire au Petit Conseil.\par

\begin{quoteblock}
 \noindent « Nous n’avons pas réussi, dit-elle. Ils insistaient sur l’amnistie. Ils étaient plus obstinés que deux jours auparavant, mais aussi plus forts. Ils étaient plus de trois mille. A l’arrivée des troupes, ils se rangèrent très rapidement sur deux colonnes protégées sur les deux flancs. Ils nous débordaient sur nos deux ailes. Le commandant français m’expliqua qu’il ne pouvait pas prendre sur lui la responsabilité de les attaquer. Les vieux militaires étaient de la même opinion. Le commandant se retira sous la condition que les insurgés iraient camper derrière la Venoge. C’est ce qui fut fait. On le réprimandera sans doute, mais le succès de notre entreprise dépendait de l’issue de cette première rencontre. »
 \end{quoteblock}

\noindent Pour la seconde fois, le représentant officiel du gouvernement helvétique se trouvait en face des \emph{Bourla-Papey} ; pour. la seconde fois aussi, il il se voyait dans la pénible obligation de négocier avec eux, de conclure une sorte de traité ou de capitulation. Le Commandant Veilande aurait pu, peut-être, forcer les campagnards à reculer ; d’autre part, et sans tenir compte du danger qu’il courait en agissant ainsi, n’était-il pas difficile pour lui, à d’autres égards, de commander l’attaque de gens résolus, très nombreux, et qui se déclaraient être les meilleurs amis de lui-même, de ses soldats, de son pays ? Malgré sa volonté de contribuer à ramener l’ordre dans le Léman, ne devait-il pas désirer que le conflit s’apaisât par d’autres moyens que ceux de la force militaire ? On voit dans ce fait en quoi consista surtout l’habileté des \emph{Bourla-Papey}, ce qui amena les officiers français à s’interposer plutôt qu’à combattre et comment le Commissaire du gouvernement se vit peu à peu amené à changer de ligne de conduite.\par
Les troupes rentrèrent à Lausanne à quatre heures après midi. Le résultat de cette équipée donna toute confiance aux campagnards et jeta le désespoir ou la colère dans l’esprit des « amis de l’ordre », c’est-à-dire des modérés et aristocrates.\par
Quant au Commissaire, il vit dès ce moment l’impossibilité presque complète d’arriver au résultat que le Petit Conseil lui avait prescrit, et il l’avoua sans détour, le soir même, à ce dernier :\par

\begin{quoteblock}
 \noindent « Le Canton du Léman sera dans quelques jours une Vendée, disait-il. Les insurgés sont au dernier point du désespoir. Il faut de deux choses l’une : Ou bien des troupes assez nombreuses pour pouvoir attaquer sur tous les points, ou bien l’amnistie. Le résultat vous prouvera que je vois mieux ici que vous à la distance où vous vous trouvez. Cela résulte d’une réflexion raisonnée et non de la peur. »
 \end{quoteblock}

\section[La nuit du 10 au 11 mai à Morges]{La nuit du 10 au 11 mai à Morges}
\noindent Les dispositions favorables du Commissaire au sujet de l’amnistie allaient s’accentuer à la suite des événements qui se passèrent à Morges pendant la nuit du 10 au 11 mai et dans lesquels les campagnards montrèrent une audace de plus en plus grande.\par
Pendant l’après-midi, alors que leurs chefs parlementaient avec Kuhn et Veilande, ils avaient brûlé dans leur camp les archives de Daillens que que quelques-uns d’entre eux avaient découvertes au Timonet, près de Crissier. Après avoir réussi à en imposer pour la seconde fois au représentant du gouvernement, ils se rendirent au delà de la Venoge, comme cela avait été convenu, et ils organisèrent un nouveau campement près du village de Préverenges.\par
Ils pouvaient, certes, être satisfaits des résultats déjà obtenus. Leur nombre augmentait toujours ; de nouveaux contingents s’organisaient encore dans diverses contrées et, de Nyon à Yverdon, bien peu d’archives féodales restaient à détruire.\par
Cependant, le dépôt de Lausanne hantait toujours l’esprit des \emph{Bourla-Papey} ; et, soit pour parvenir mieux à mettre la main dessus, soit pour faire encore davantage impression sur le Commissaire et le gouvernement, ils résolurent de chercher à s’emparer de tout ou partie des canons déposés à l’arsenal de Morges. Ils pouvaient compter sur la population et l’autorité municipale de cette ville ; plusieurs chefs connaissaient le commandant de la place, Demney, et se persuadaient qu’il n’opposerait aucune résistance sérieuse.\par
L’un des chefs les plus résolus des \emph{Bourla-Papey}, le capitaine Cart, tenancier de l’Hôtel de l’Ange, à Nyon, fut chargé, avec un détachement de quatre cents hommes, d’aller présenter une sommation régulière à la garnison et à la ville de Morges, pendant que d’autres colonnes cerneraient cette localité du côté du Signal.\par
A quatre heures après-midi, au moment où le Commissaire Kuhn rentrait à Lausanne, Cart se présenta aux portes de Morges. Le commandant Demney se rendit auprès de lui et l’amena au sous-préfet. Cart demanda de pouvoir traverser la ville. Mandrot lui rappela le traité de Rion-Bosson et, refusa. Demney accepta, au contraire, et alla conduire cette troupe sur la place, près du château. Cart remit alors la sommation suivante :\par

\begin{quoteblock}
 \noindent « Le citoyen capitaine Cart, commandant l’avantgarde des troupes vaudoises aux ordres du commandant en chef Reymond, somme le citoyen Demney, commandant la place de Morges de lui remettre le poste de la place dans le terme de quatre heures, à partir de 5 7« heures.\par
 (Signé) {\scshape Cart. »}
 \end{quoteblock}

\noindent En présence d’une situation aussi dangereuse, le sous-préfet se hâta d’avertir Polier et le commandant Veilande.\par

\begin{quoteblock}
 \noindent « Nous ne pouvons nous défendre avec le peu de monde que nous avons, leur dit-il, Au nom de Dieu, du secours et prompt, car à 11 7« heures du soir, nous sommes attaqués et pris et Dieu sait ce qui arrivera à notre pauvre ville. Si le citoyen Veilande peut venir lui-même, ce sera le mieux. Il en imposera. Ils sont plus de mille et nous avons tout à craindre. »
 \end{quoteblock}

\noindent Le sous-préfet se rendit ensuite au château pour voir de quelle manière le commandant Demney exécutait l’ordre précis qu’il avait reçu d’y rester, de n’en laisser sortir personne, et de mettre des sentinelles aux endroits faibles. Il apprit que l’officier français était au cabaret, en compagnie de Cart.\par
« Ce désordre m’affligea, dit Mandrot, et j’en tirai les plus mauvais augures. La ville était pleine de paysans, qui se mêlaient avec les soldats et qui pouvaient, par le moyen du vin, les mettre hors d’état de se défendre. A 8 ½ heures du soir, mon courrier revint de Lausanne avec une lettre du Préfet et une du citoyen Veilande. Toutes deux portaient au citoyen Demney l’ordre de défendre le château à toute extrémité. Demney fit chercher Cart ; on lui lut les ordres positifs de défense… Je l’exhortai à retirer sa demande et à s’en aller· Je lui fis des observations très fortes ; il persista et dit qu’il attaquerait. Demney déclara alors qu’il se défendrait au péril de sa vie, mais qu’il était bien triste pour un vieux militaire à la veille de se retirer, d’être obligé de défendre une place qui n’était pas soutenable, qu’il serait forcé, mais qu’il y périrait. » Pendant ce temps, la Municipalité siégeait au milieu d’un certain nombre de personnes de la localité. Toutes désiraient que l’on évitât de plus grands maux et que le sous-préfet fût invité à user de son pouvoir pour y parvenir. Ce dernier, averti de ce qui se passait, se rendit à la maison de ville, où il ne fut pas étonné de trouver le capitaine Cart. On lui remit aussitôt la réquisition suivante de l’autorité locale :\par

\begin{quoteblock}
 \noindent « Environnés de détachements armés, sans assurances de leur part non plus que de la vôtre et de celle du commandant français que les propriétés de cette commune, et la sûreté des individus tranquilles qui la composent seront respectées…, notre devoir est de vous inviter à conjurer l’orage qui nous menace ; vous en avez les moyens, mettez-les à exécution ; pesez le parti que vous avez à prendre dans les balances des circonstances et de la prudence, et si votre parti est extrême, nous mettons tout ce qui peut arriver de fâcheux sous votre responsabilité personnelle, puisque vous donnez \emph{seul} des ordres à la troupe française, représentez-vous les suites des ordres que vous donnerez et les reproches que vous aurez à vous faire si l’obstination est égale d’un côté et de l’autre et que les détachements armésqui environnent une commune qui, quoique tranquille, peut être mise dans un instant à feu et à sang (reçoivent l’ordre d’attaquer)… Prévenez les malheurs, il en est encore temps. »
 \end{quoteblock}

\noindent Comme on le voit, la Municipalité de Morges et les personnes réunies autour d’elle invitaient le sous-préfet à'assurer l’exécution de la demande présentée par le capitaine Cart. Ce fonctionnaire était entouré, en outre, par quelques insurgés et le commandant français se bornait à dire qu’il était à ses ordres et les exécuterait… Mandrot rappela les instructions précises de Polier et de Veilande, mais on lui répondit avec la même insistance et les mêmes arguments.\par
Cart allait se retirer pour donner le signal de l’attaque lorsqu’une personne s’écria :\par
— Les paysans se contenteront de quatre canons.\par
Cart consentit à cela et le commandant Demney déclara qu’il les accorderait s’il y était autorisé par le sous-préfet \footnote{Lettre de J. Muret au Commissaire Kuhn, dans \emph{Papiers…}, etc.} . Ce dernier refusa encore.\par

\begin{quoteblock}
 \noindent « Alors, dit-il, les propos s’échauffèrent ; municipaux et bourgeois s’exaltèrent en propos violents ; j’entendis des menaces ; les militaires se rapprochèrent de moi… et je ne voyais que trop que je n’étais plus libre. Je vis qu’il fallait céder ; je voulus gagner du temps et avoir encore une nouvelle sommation de la Municipalité. »
 \end{quoteblock}

\noindent Cette demande fut aussitôt rédigée, invitant le sous-préfet \emph{« à remettre au citoyen Cart… quatre pièces de canon de campagne avec les caissons au complet et garnis de leurs munitions »}. La Municipalité déclarait d’ailleurs que malgré les ordres formels du Préfet national et du Commandant Veilande, \emph{« elle mettait sous la responsabilité personnelle du sous-préfet les suites affreuses qu’aurait un refus »}. Cart, de son côté, s’engageait dans ce cas à respecter le traité conclu le 6 mai à Rion-Bosson \footnote{Archives de Morges. Parmi les signataires de cette réquisition, on lit les noms de Muret-Fasnacht, président du tribunal de district ; Muret-Baron, ex-sénateur ; Henri Warnery, juge de district ; Guex, greffier de district ; M.-J. Devenoge, « ministre du culte » ; P. Devenoge, membre de la Diète cantonale ; Claude Mandrot, Jaïn, etc.} .\par

\begin{quoteblock}
 \noindent « Il était onze heures du soir, dit Mandrot, j’étais comme prisonnier. Je ne pouvais plus espérer de secours ni de Lausanne, ni du commandant français qui était là, qui m’abandonnait et même se cachait derrière moi et je cédai. »
 \end{quoteblock}

\noindent Une capitulation fut rédigée et signée. La voici :\par

\begin{quoteblock}
 \noindent « Le sous-préfet de Morges autorise le Commandant Demney à délivrer les quatre pièces ci-devant désignées, ainsi que leurs caissons et munitions y attachés, cela devant être fait sans que qui que ce soit entre au dit Château, que les troupes françaises qui y sont.\par
 11 heures du soir.\par
 (Signé) {\scshape Mandrot}, sous-préfet. »
 \end{quoteblock}

\noindent La capitulation allait être exécutée lorsqu’un grand bruit se fit entendre dans la maison. Plusieurs individus entrèrent effarés en criant :\par
— Il vient d’arriver un renfort de Français au Château.\par

\begin{quoteblock}
 \noindent « Cart s’approcha alors de moi, raconte Mandrot, et me dit d’un ton furieux :\par
 — « Vous nous trahissez ; vous avez traîné en longueur, vous avez refusé, disputé, chicané sur tous les mots, espérant qu’il vous viendrait du secours ; il vient d’arriver, mais je n’en serai pas la dupe. Je vous arrête et je vous remets en otage entre les mains de la Municipalité jusqu’à ce qu’on ait exécuté ce que vous avez promis.\par
 « Cet acte de violence ne fut contredit ni par le Commandant français, ni par la Municipalité… Des cris s’élevèrent en faveur de cette mesure. Je me récriai. Je déclarai que cette arrestation était injuste et sentait la trahison. Je ne fus pas écouté… Ce procédé prouve que dès que je me fus rendu à la Municipalité, je ne fus plus libre. »
 \end{quoteblock}

\noindent Pendant ce temps, une troupe d’une centaine d’hommes, français et suisses, arrivait d’Ouchy par le lac, sous la direction du capitaine helvétique Gilly. Le Commandant Demney annonça à ce dernier qu’on l’attendait à la Maison-de-Ville, mais sans lui faire mention de ce qui s’y passait, lui assurant seulement qu’il n’y avait aucun danger pour lui de s’y rendre.\par

\begin{quoteblock}
 \noindent « Vers la porte de la Maison de commune, raconte Gilly, le municipal Mercier m’invita à monter et me dit que le sous-préfet voulait me parler. J’entrai donc ; les portes se fermèrent sur moi et des gardes d’insurgés y furent placées pour que je ne pusse sortir. En montant l’escalier, le dit Mercier me demanda si j’étais le commandant des \emph{Rablais} ou Helvétiques. Je lui dis oui et il me répondit : \emph{C’est justement à ceux-là que nous en voulons.} En entrant dans la salle de la Municipalité, je vois des gardes d’insurgés aux portes, le sabre à la main. La chambre était remplie, outre les Municipaux, de militaires insurgés et même de citoyens de Morges, dont quelques-uns avaient des armes. »
 \end{quoteblock}

\noindent Gilly fut sommé d’exécuter la convention conclue quelques instants auparavant. Il refusa nettement et la colère des assistants fut grande contre lui.\par
Il fut maintenu à la Maison-de-Ville comme prisonnier et, pendant ce temps, Mandrot et Demney furent conduits au Château pour faire exécuter la capitulation. Ils en donnèrent l’ordre ; mais un officier venu avec Gilly et tous les soldats déclarèrent qu’ils ne livreraient aucune pièce de canon et que leur commandant l’ordonnerait inutilement.\par
Mandrot fut ramené encore une fois à la Maisonde-Ville où le tumulte ne fit que grandir et où Gilly demanda vainement d’avoir un instant d’entretien avec le sous-préfet.\par

\begin{quoteblock}
 \noindent « Je voulus alors sortir, raconte l’officier suisse, je déclarai qu’on ne pouvait m’arrêter ; mais sur le champ je fus invectivé et menacé, tant par des municipaux, que par d’autres individus et les portes étaient toujours garnies d’hommes, le sabre à la main. Je me fâchai, je parlai avec force et les injures augmentèrent ; je vis même le moment où j’aurais à me défendre. »
 \end{quoteblock}

\noindent Le sous-préfet demanda enfin et obtint la permission d’aller à Lausanne rendre compte de ce qui se passait. Une voiture fut préparée, et après maintes difficultés nouvelles, il parvint à s’échapper. Il était arrivé à dix minutes de la ville, lorsqu’un courrier du Préfet national le rejoignit et lui remit la déclaration du Ministre de France, Verninac, au Petit-Conseil. Cette pièce apparut au sous-préfet de Morges comme une planche de salut et il revint immédiatement en arrière pour en donner lecture aux personnes rassemblées à la Maison-de-Ville.\par
Un événement important s’était passé pendant son absence. Un détachement de troupes françaises et helvétiques était accouru à la Maison-deVille depuis le Château pour réclamer la libération de son chef. Cette demande ayant été formellement repoussée, elles envahirent la maison, et bousculèrent tout ce qui se présenta devant elles. Un certain nombre de bourgeois furent légèrement blessés dans la bagarre ; la Municipalité fit enfin déployer un grand drapeau helvétique. Les soldats saisirent alors leur chef et l’emmenèrent.\par
L’instant suivant, le sous-préfet se présenta et lut la lettre du Ministre de France au Petit Conseil. Elle fit une certaine impression. Quant au capitaine Cart, il parut un peu découragé et abattu.\par

\begin{quoteblock}
 \noindent « Je l’engageai, dit Mandrot, à renvoyer sur le champ les troupes qu’il avait autour de Morges et il s’en vint à Lausanne avec moi pour se présenter au citoyen Kuhn. »
 \end{quoteblock}

\noindent Ainsi se termina un des incidents les plus curieux de la « guerre » des \emph{Bourla-Papey.}
\section[Nouvelle reculade du Commissaire]{Nouvelle reculade du Commissaire}
\noindent L’événement dont il vient d’être question et la reculade du jour précédent à St-Sulpice montrèrent au Commissaire Kuhn qu’il était bien difficile, sinon impossible dans ce moment, de se servir de la force pour dissiper le rassemblement principal des \emph{Bourla-Papey}, aussi bien que ceux moins considérables qui existaient au même moment dans plusieurs autres parties du canton. Des compagnies helvétiques et d’autres, moins nombreuses, de troupes françaises continuaient, il est vrai, à arriver à Lausanne. Le Commissaire du gouvernement s’était adressé aussi au général Séras, commandant de la place de Genève, pour le prier de faire entrer quelques contingents dans le Pays de Vaud. Séras s’était empressé de lui répondre qu’il serait heureux de lui rendre ce service, mais il devait pour cela demander la permission de son supérieur, le général Molitor, à Grenoble, et le courrier qui lui était envoyé ne pouvait être de retour avant plusieurs jours. Il était, d’autre part, urgent que l’insurrection prit fin aussitôt, car elle ne cessait de s’étendre à de nouveaux districts. Quelques jours encore, et il faudrait de six à huit mille hommes pour ramener le calme par la force.\par
Le Petit Conseil continuait sans doute à préconiser, à exiger même la fermeté et l’emploi de la force. Dans une lettre du 11 mai, à quatre heures du soir, il écrivait :\par

\begin{quoteblock}
\noindent « Après tant d’excès commis par les insurgés, tant de mesures prises de notre part, il ne peut nullement être question de proclamer une amnistie générale. »\end{quoteblock}

\noindent C’était trop tard ; la résolution du Commissaire était prise et, après tout ce qui s’était passé, cette résolution allait être irrévocable, d’autant plus qu’elle lui était recommandée par le général français Amey, envoyé dans le Pays de Vaud par son supérieur Montrichard. Kuhn expliqua le 11 mai à son gouvernement combien il serait dangereux d’attaquer les \emph{Bourla-Papey}.\par

\begin{quoteblock}
\noindent « Le plan sur lequel je me suis mis d’accord est le suivant, disait-il. Nous chercherons avant tout à obtenir que les insurgés se retirent sous la promesse de l’amnistie. Si cela arrive, les troupes avanceront néanmoins et se disperseront de manière qu’aucune nouvelle révolte ne puisse s’organiser et qu’on puisse commencer le désarmement. Dans le cas contraire, nous les amuserons jusqu’à ce que les troupes qui sont en route soient arrivées et que nous puissions prendre l’offensive. »\end{quoteblock}

\noindent Kuhn savait qu’en agissant de cette manière, il contrevenait aux ordres du Petit Conseil. Cependant sa résolution était si formelle qu’il s’engagea \emph{« en toute justice à prendre la responsabilité de tout »} et se déclara \emph{« prêt à justifier cette mesure publiquement »}.\par
Dans la soirée du même 11 mai, les chefs de la troupe campée à Préverenges vinrent à Lausanne, où ils avaient été convoqués par le général Amey.\par

\begin{quoteblock}
 \noindent « Il leur présenta amicalement les suites de leur conduite passée et les maux terribles qui seraient la conséquence de la guerre civile. Il les somma ensuite de rentrer dans leurs foyers. Il leur montra combien ils avaient agi contre leurs propres intérêts en abandonnant le gouvernement patriote qui leur était cependant favorable, pour prononcer l’annexion à la France. Ils se montrèrent très disdisposés à obéir à condition que le gouvernement oubliât le passé et les amnistiât. Le général et moi nous promîmes d’intercéder dans ce but auprès du gouvernement, raconte le commissaire \footnote{Lettre du 12 mai au Petit Conseil.} . »
 \end{quoteblock}

\noindent Ce dernier justifia immédiatement sa conduite en faisant valoir les raisons suivantes :\par
1° La trop petite quantité de troupes françaises et suisses dans le Léman.\par
2° Ce n’est qu’au bout de quatre ou cinq jours que l’on aurait pu prendre l’offensive et cela sans assurance de succès.\par
3° On parvenait de cette manière à empêcher le mouvement qui commençait à se manifester dans la vallée de la Broie et le Canton de Fribourg, de s’étendre davantage.\par
4° Une des raisons les plus importantes était le fait que les chefs ostensibles de l’insurrection n’en étaient pas les vrais auteurs, mais bien plutôt de \emph{« misérables instruments de quelques ambitieux. C’est ceux-là qu’il faudrait punir et non ceux qu’ils ont séduits, »} disait Kuhn. Du reste, ajoutait-il, Reymond et Marcel \emph{« ont fait tout leur possible pour empêcher des excès et ont observé la discipline autant que cela est possible dans une milice »}.\par
5° En promettant une amnistie, Kuhn croyait enfin avoir évité une guerre civile non seulement au Léman, mais encore à une grande partie de la Suisse, et pour des raisons sur lesquelles il y aura lieu de revenir plus loin.\par
Le Commissaire du gouvernement avait ainsi, en trois jours, composé trois fois avec les insurgés du Léman. Il reste à voir maintenant quels sont les événements qui, accomplis dans un certain nombre de districts, avaient contribué pour une part importante à lui faire prendre cette détermination.
\section[Yverdon]{Yverdon}
\noindent La région d’Yverdon et de Grandson vit se former une des troupes insurgées les plus considérables, en dehors de celle commandée par Louis Reymond. C’est cependant assez tard que l’agitation se répandit dans le nord du Canton.\par
\emph{« Les têtes se montent »}, écrivait le sous-préfet d’Yverdon, Doxat, à la date du 2 mai. La tranquillité resta néanmoins complète jusqu’au 6, mais les nouvelles qui arrivaient d’autres districts inquiétèrent beaucoup de personnes.\par

\begin{quoteblock}
 \noindent « Mon ami, nous sommes perdus, écrivait Louis Lambert à son ami Auguste Pidou, alors à Berne… L’impunité organise le brigandage. Ce torrent va nous inonder. Notre malheureux canton est perdu ; cherchez, je vous en conjure, d’exciter l’intérêt du gouvernement. »
 \end{quoteblock}

\noindent Le 6 mai, des rumeurs inquiétantes commençaient à courir dans le public, le sous-préfet fit rétablir les ponts-levis et placer des gardes aux trois portes de la ville. Des éclaireurs furent envoyés de divers côtés. Tout fut calme dans la contrée et cette petite troupe put être licenciée le 7 au matin. \emph{« Je ne doute pas que je sois obligé de la remettre sur pied ce soir »}, écrivit Doxat au Préfet national.\par
Le sous-préfet avait raison. C’est dans la journée du 7 mai qu’eut lieu la prise d’armes des campagnards des districts d’Yverdori et de Grandson. Malgré la tranquillité apparente de ces contrées durant les jours précédents, un mouvement général y avait été organisé. Trois troupes devaient se former. L’une d’entre elles, composée des gens d’Yvonand et des villages voisins, devait se rendre à Font, près d’Estavayer, puis revenir pendant la nuit suivante à Yverdon où elle trouverait les deux contingents de la contrée de Mathod et de celle de Fiez.\par
Le 7 mai au matin, les citoyens d’Yvonand furent assemblés ; ils décidèrent d’infliger une amende à tous ceux qui ne marcheraient pas \footnote{Cette amende fut d’abord fixée à deux écus neufs, mais fut bientôt trouvée un peu trop forte.} et de prendre pour chef le capitaine Besson de Niédens. Le gouverneur (syndic), Emmanuel Vonnez se rendit auprès de ce dernier et le somma de prendre le commandement. Des citoyens de quelques communes du voisinage ayant rejoint la colonne, celleci se rendit, par Cheyres, à Font, où beaucoup de campagnards de la contrée payaient des droits féodaux. Le receveur étant absent, on s’adressa au curé Bielmann.\par
Quatre hommes se présentèrent chez lui à midi pour payer leurs censes et demander divers renseignements au sujet des impôts. Pendant ce temps, d’autres payeurs se présentèrent et finirent par encombrer l’appartement.\par

\begin{quoteblock}
 \noindent « Quand leur nombre est assez grand, raconte Bielmann, l’un des citoyens arrache les livres et papiers relatifs aux censes diverses d’Yvonand en disant qu’il les emporterait ; le curé, aussitôt, saute à sa carabine pour décocher un coup de fusil sur le voleur des papiers. A peine est-il armé que six hommes lui sautent au bras et au collet et l’empêchent de commettre une imprudence que sa vivacité et son zèle pour le bien public lui inspiraient. Saisi, il crie ; son domestique arrive, saute à son sabre, et deux des citoyens qui tenaient le curé arrêtent le domestique. Le curé, voyant son domestique captif comme lui, envoie sa servante chercher la Municipalité. A peine fut-elle sortie que les insurgés redevinrent honnêtes, et, tout en le tenant, le prièrent de s’apaiser, qu’ils sont au nombre de deux mille pour lui rendre raison, qu’il ne s’attirerait que la mort par une plus longue résistance. Alors le curé dépose les armes et les quatre citoyens le lâchent et prennent la fuite en chantant. »
 \end{quoteblock}

\noindent Le curé se rendit à Châbles avec son domestique, mais on trouva qu’il n’était pas possible d’arrêter la troupe d’Yvonand. Bielmann alla cependant au-devant d’elle avec une délégation de la Municipalité pour demander un acte attestant que les papiers avaient été pris. \emph{« Point de décharge ! s’écrièrent les insurgés. Nous avons les livres, qu’ils s’arrangent ! Avancez ! point de déclaration ! »}\par
La cloche sonna de nouveau le soir à Yvonand et les villages voisins furent avertis, surtout Cuarny, où l’enthousiasme insurrectionnel était grand. Une assemblée générale eut lieu dans laquelle il fut décidé qu’à deux heures du matin les gens de tous les villages devaient se trouver au-dessus de Clindy, un hameau d’Yverdon, pour s’emparer des titres déposés dans cette ville.\par
Pendant cette même soirée, les gens de Mathod, de Suscévaz, de Villars-sous-Champvent, etc. se réunirent sous la direction de Rodolphe Décoppet, président de la Municipalité de Suscévaz, pour se rendre au château de Champvent, propriété de la famille Doxat. Tout y fut pris, même de nombreux papiers qui avaient un grand intérêt pour le seigneur, mais aucun pour les insurgés. Ces archives furent brûlées immédiatement. Le citoyen Doxat croyait que tout était terminé, lorsque quelques envahisseurs recommencèrent à parcourir le château, fouillant tout et commettant quelques violences. Un domestique eut le visage « chauffé » d’un coup de fusil. L’instant d’après, il voulut encore empêcher que l’on ne reprît des actes d’acquit.\par

\begin{quoteblock}
\noindent « La réponse fut qu’on pointa une baïonnette contre sa poitrine avec menace de l’enfoncer s’il continuait à raisonner. »\end{quoteblock}

\noindent Un tonneau fut encore rempli de papier et emporté sur la place publique de Mathod où le contenu fut brûlé au milieu de la joie générale. La troupe se mit en route bientôt après pour Yverdon.\par
L’agitation avait fini par gagner cette dernière ville, où un certain nombre de personnes faisaient des vœux pour le succès de l’insurrection. Malgré l’activité du commis d’exercice Develey, on n’y put réunir que vingt-cinq hommes de la garde bourgeoise et les courriers envoyés de divers côtés rapportèrent les nouvelles les plus fâcheuses.\par
La Municipalité fut convoquée au milieu de la nuit et discuta, en compagnie du sons-préfet et de quelques autres personnes, ce qu’il fallait faire. Elle renonça à l’idée de faire battre la générale. Elle résolut de faire son possible pour modérer le mouvement révolutionnaire sans avoir l’espérance de l’arrêter et elle pria les particuliers d’éclairer leurs maisons.\par
La troupe d’Yvonand arriva la première, annoncée longtemps à l’avance par le bruit du tambour. Une délégation de quatre citoyens fut conduite auprès de la Municipalité pour annoncer les intentions des paysans. Le sous-préfet leur représenta inutilement l’« inconvenance » de l’acte qu’ils voulaient commettre. Les campagnards conseillèrent de leur côté de ne pas chercher à opposer de résistance puisqu’ils exécuteraient leur projet de vive force et que, par contre, \emph{« si on les laissait faire, ils promettaient sûreté aux citoyens et à toutes autres propriétés »}.\par
Le bruit du tambour annonça l’arrivée de la \textsubscript{\#} troupe de Mathod alors que la ville était déjà envahie par celle d’Yvonand.\par
Les campagnards se rendirent tout d’abord chez le receveur national Vulliemin, père de l’historien, qui demeurait en face du château. Il annonça qu’avant de rien livrer, il voulait un ordre du sous-préfet. Cela lui donna le temps de mettre en sûreté quelques papiers importants avant que sa maison fût complètement envahie.\par
Entouré de tous côtés, porté, en quelque sorte, par les insurgés qui le pressaient de toutes parts, il ne put qu’assister au pillage de ses livres de droitures féodales. Les \emph{Bourla-Papey} s’emparèrent des titres de vingt-trois localités et de ceux de quelques familles déposés au même endroit. Ils se rendirent ensuite au château où toutes les archives déposées dans la tour des Juifs furent enlevées. Les registres des anciens notaires du bailliage disparurent aussi en grande partie. Un peu plus tard enfin, un grand nombre de campagnards pénétrèrent dans la maison de ville après avoir laissé deux hommes en sentinelle à la porte. Ils exigèrent les titres appartenant à l’hôpital :\par

\begin{quoteblock}
\noindent « menaçant, dit la Régie, d’employer la baïonnette si on ne les leur remettait pas sur-le-champ… en sorte qu’il ne fut pas possible de prendre un inventaire \footnote{Archives d’Yverdon.}. »\end{quoteblock}

\noindent Les troupes de Mathod et du district de Grandson étaient entrées dans la ville qui se trouva, dès ce moment, occupée par environ un millier de \emph{Bourla-Papey}. Pendant que les uns fouillaient le château, la Maison de ville et celle du Receveur, d’autres allaient enlever les titres que possédaient un certain nombre de particuliers : Doxat, de Démoret ; Mme Jenner, née Manuel ; M\textsuperscript{elle} de Coppet, de Souville (\emph{« la troupe a été très polie »}, dit-elle), Casimir Décoppet ; Henri de Treytorrens, des Bains ; Christin, avocat ; Mme Cordey, etc.\par
Des quantités considérables de papiers furent amoncelées dès les premiers moments sur la place où s’élève aujourd’hui le monument Pestalozzi et les flammes qui les consumaient s’élevèrent bientôt suffisamment pour inspirer la crainte d’un plus vaste incendie.\par
On laissa donc éteindre ce foyer et, les débris encore fumants, joints aux masses nouvelles de titres qui affluaient toujours, remplirent bientôt trois grands chars qui furent emmenés sur la place d’armes où l’incendie des archives féodales d’Yverdon eut lieu au milieu de l’enthousiasme bruyant d’un foule considérable.\par
La troupe, maintenant très grande des \emph{Bourla-Papey}, se mit en route pour Grandson où la Municipalité se réunit à sept heures du matin avec le sous-préfet Delachaux. Une délégation de quatre paysans vint annoncer l’arrivée de la troupe et ses intentions. Cette dernière entra ensuite dans la ville et pendant qu’une partie pénétrait dans la maison du receveur national de Ribeaupierre, l’autre se rangeait en bataille dans la rue et fermait toutes les issues. Ce fonctionnaire livra tout ce qu’il avait, même les clefs des greniers nationaux, qui furent, du reste, rapportées plus tard. Les propriétaires de dîmes livrèrent aussi leurs titres qui furent brûlés avec les précédents sur la place du château.\par
Les incendies considérables d’Yverdon et de Grandson avaient détruit la presque totalité des papiers féodaux des deux districts. Quelques dépôts existaient cependant encore en divers lieux. Une colonne de plus de cent hommes alla s’emparer, au milieu de la journée, à Montavaux, près Orges, des papiers que possédait la famille Masset. Quelques villages vinrent réclamer encore lés leurs, soit pour les détruire, soit pour les mettre en sûreté. C’est ainsi que les jeunes gens d’Arissoules se rendirent à Yverdon pour demander ceux du citoyen Casimir Décoppet. \emph{« Cette demande fut faite très honnêtement et même en tremblant »}, dit ce dernier. A Ependes, les habitants du village avaient entendu, à deux heures du matin, les gens de Essert-Pittet et de Chavornay qui se rendaient à Yverdon en poussant des cris et en invitant tout le monde à se joindre à eux. Ils envoyèrent aussitôt quelques personnes auprès du citoyen Duplessis, seigneur d’Ependes, pour lui proposer de mettre ses titres en sûreté dans les archives communales.\par
Cette demande fut acceptée avec empressement.\par
La plupart des paysans qui s’étaient rendus à Grandson revinrent ensuite à Yverdon où, enflammés un peu par l’heureux résultat de leur expédition, ils firent beaucoup de bruit et commirent quelques excès. Un certain nombre poussèrent l’audace jusqu’à demander l’argent qu’ils avaient déjà donné pour les dîmes et censes de 1801. \emph{« Dans sept jours, ils exigeront les créances »}, écrivait à Henri Polier l’avocat de Félice, au nom du sous-préfet Doxat.\par
Les paysans furent enfin rejoints par des messagers apportant un appel de Louis Reymond, les invitant à le rejoindre près d’Ecublens. Un certain nombre partirent dans cette direction pendant que les autres allaient raconter leurs exploits dans leurs villages respectifs.
\section[Orbe]{Orbe}
\noindent i\par
Le district d’Orbe fut un de ceux où l’agitation fut la plus grande et où se commirent quelquesuns des excès les plus graves. Le chef-lieu avait été depuis quelques mois le rendez-vous d’un grand nombre de chefs et d’émissaires ; le district contenait plusieurs hommes qui, destitués de leurs fonctions en 1800 et 1801, furent des boute-en-train de l’insurrection. L’ex-juge de canton Potterat, à Orny ; le juge Agassiz, àBavois, et surtout Abram Gleyre, président de la Municipalité de Chevilly, furent les principaux meneurs dans l’ancienne baronie de La Sarraz et dans quelques contrées voisines.\par
Le sous-préfet d’Orbe, Thomasset, homme déjà âgé, mais encore plein de vigueur, était lui-même propriétaire de droits féodaux importants et se trouvait, en conséquence, au nombre des citoyens détestés du district. Plusieurs autres seigneurs, les de Gingins d’Eclépens, les Duplessis, les Pillichody, les de Goumoêns, vinrent au fort de la tourmente se réfugier à Orbe ; cette ville, renfermant d’autre part une Municipalité et beaucoup de citoyens favorables aux insurgés, se vit en butte, de bonne heure, aux attaques des \emph{Bourla-Papey.}\par
Le 5 mai déjà, Thomasset fut un peu alarmé par le fait que les citoyens de la baronnie de La Sarraz et quelques-uns de Bavois et Chavornay prenaient les armes. Une garde bourgeoise fut formée ; la nuit suivante fut cependant tranquille et le lendemain on apprit qu’un contingent considérable avait été rejoindre l’armée de Louis Reymond.\par
La garde bourgeoise fut augmentée, mais les précautions ne purent empêcher bien des gens de se déclarer pour les \emph{Bourla-Papey.} Les nouvelles de Lausanne leur étaient favorables et le public se persuadait que la France soutenait l’insurrection.\par
Les propriétaires de fiefs étant venus demander un conseil au sous-préfet, celui-ci leur recommanda la modération. Le greffier du tribunal, le citoyen Maubert, entra à ce moment et conseilla de son côté au citoyen Thomasset de ne pas chercher à défendre les propriétaires de fiefs à main armée, attendu que dans ce cas \emph{« sa vie serait en danger et que l’on pourrait mettre le feu à la ville. »} Ce fonctionnaire répondit avec fermeté et se rendit auprès de la Municipalité. Malgré ses sollicitations, cette autorité se borna à décider que :\par

\begin{quoteblock}
\noindent « s’il arrivait une colonne, deux Municipaux iraient à sa rencontre, la prieraient de s’arrêter et qu’on satisferait à ses réquisitions. »\end{quoteblock}

\noindent Thomasset se rendit encore à un Cercle \emph{« que l’on disait être un peu dans le mouvement anarchique. »} L’accueil fut glacial… Il voulut ensuite lever un détachement de la réserve qu’il avait fait inspecter quelques jours auparavant, mais on l’avertit qu’il ne devait pas compter sur ces hommes \emph{« que l’on avait fort travaillés depuis lors. »} Il se décida, en conséquence, à convoquer des volontaires connus et fidèles, sous les ordres de l’un de ses fils.\par
Deux de ces derniers se rendirent à Vallorbes où une soixantaine de citoyens avaient offert leurs services au sous-préfet.\par

\begin{quoteblock}
 \noindent « Ils y arrivèrent au moment où ce village recevait une sommation d’envoyer un contingent au camp de Lausanne et où le village était rempli d’émissaires qui cherchaient à le séduire. La crainte chez les uns, l’effet de la suggestion chez les autres, empêchèrent mes fils, \emph{dit Thomasset, }de réunir ce corps. »
 \end{quoteblock}

\noindent Au moment où le sous-préfet d’Orbe apprit qu’il ne pouvait plus compter sur personne, le 8 mai, cette ville fut envahie par une colonne de soixante hommes armés, venant des villages de la contrée de Chavomay, demander les titres de la seigneurie de Corcelles, déposés dans une maison particulière.\par

\begin{quoteblock}
 \noindent « Je leur envoyai la proclamation du citoyen Kuhn, dit Thomasset. Ils répondirent qu’ils s’en moquaient comme d’une saucisse. Ils entrèrent en ville tambour battant. Ils en repartirent en poussant des cris de joie et brûlèrent les titres près du canal en tirant des salves et battant du tambour. »
 \end{quoteblock}

\noindent Cette colonne rentra à Chavornay puis se rendit à Bavois où le tocsin sonna et où l’on battit la générale. Il s’agissait de se rendre au château pour y demander les titres du citoyen Pillichody, bien connu par ses sentiments réactionnaires.\par
Le juge Agassiz, qui était à la tête du rassemblement, présenta la demande des insurgés. Le domestique était seul au château ; il lui fut accordé un délai expirant le 9 mai au soir, pour livrer les archives. Il arriva aussitôt après une centaine d’hommes suivis de femmes et d’enfants qui ne voulurent pas se contenter de celte promesse et pénétrèrent dans le château où beaucoup de choses furent détériorées. La porte de la cave fut enfoncée ; toute cette population y pénétra et vécut à discrétion jusqu’au lendemain où le juge Agassiz fit présenter au citoyen Pillichody une sommation d’avoir à livrer ses titres le dix, à huit heures du soir. On le menaçait, en cas de refus, de démolir son château.\par

\begin{quoteblock}
\noindent « On a demandé du secours\emph{, dit Thomasset}. Mais que puis-je faire contre deux cents hommes ivres-morts ? »\end{quoteblock}

\noindent Le dix, à huit heures du soir, les titres furent livrés à la population qui s’était de nouveau rendue au château. Voici, d’après le rapport du propriétaire, ce qui avait été consommé dans les caves ou emporté, du 8 au 10 mai :\par

\begin{quoteblock}
\noindent « 400 bouteilles de La Côte 1774 ; 250 bouteilles de vin rouge de Cortaillod de 1795 – le citoyen Pillichody avait un bon bouteiller — ; 60 bouteilles de bière ; 300 pots de vin du pays, de 1800. Les principaux dégâts commis étaient les suivants : deux portes enfoncées, trois chaises cassées ; deux douzaines de verres de Bohême cassés, six dites ordinaires, trois serrures brisées, une table en acajou brisée, quatre estampes anglaises détruites, un lit échiré et cassé. »\end{quoteblock}

\noindent Si le citoyen Pillichody n’a pas trop exagéré les dégâts commis dans sa cave, il est certain que le sous-préfet était autorisé à écrire à Polier :\par

\begin{quoteblock}
\noindent « Que puis-je faire contre deux cents hommes ivres-morts ? »\end{quoteblock}

\noindent Pendant que les gens de Bavois et Chavornay buvaient le vin de leur seigneur, une autre troupe s’organisait au pied du Jura. Les gens de Bofflens, Agiez, Brethonnières, Croy, Premier, etc., se présentèrent dans la nuit du 8 au 9 mai, à trois heures du matin, chez le receveur de Romainmôtier. Us enlevèrent les titres qui furent emmenés sur un char pour être brûlés « à la croisée de Croy ». Une autre troupe, se rendant de Vallorbes à l’armée de Louis Reymond, s’assura encore pendant la matinée que tout était bien détruit.\par
L’ex-membre du Directoire helvétique, Maurice Glayre, vit aussi ses propriétés visitées ce même jour par la troupe de Romaininôtier. Les gens d’Arnex, où il avait une grande partie de ses propriétés et qui ne possédaient pas de mauvais sentiments à son égard, se joignirent au mouvement pour le modérer. C’est à eux que Glayre remit ses titres et non à Cochet, commandant de la troupe. Ce dernier exigea du vin pour son contingent. Glayre lui répondit avec fermeté :\par

\begin{quoteblock}
\noindent « Je ne donne pas à boire à des gens qui le demandent le sabre à la main. »\end{quoteblock}

\noindent Cette troupe, très surexcitée, se rendit ensuite à Agiez, où elle investit la maison du sous-préfet d’Orbe, occupée à ce moment par un de ses fils. A défaut de papiers, il donna une renonciation formelle de ses droits. Après avoir fait beaucoup de bruit, cette troupe se rendit à Orbe pour y prendre livraison des titres.\par
Thomasset eut, paraît-il, à supporter les menaces et les injures les plus graves.\par

\begin{quoteblock}
\noindent « Je reçus l’injonction de restituer ce que j’avais tiré pour la dîme de cette année. Crainte d’être assassiné, j’ai obéi à ce qu’ils m’ont demandé, en leur offrant, de plus, ma bourse. »\end{quoteblock}

\noindent Le sous-préfet fut encore visité le même jour par les gens de Champ vent et de Villars, qui reçurent leurs titres mais n’envahirent pas la maison. Quant à Duplessis, d’Ependes, il fut insulté chez lui et reçut des lettres menaçantes parce qu’il avait recueilli dans sa maison les citoyens de Goumoêns, Pillichody de Bavois, et de Gingins.\par

\begin{quoteblock}
 \noindent « Il n’existe plus d’autorité, écrivit Thomasset le même soir à Polier. L’anarchie est à son comble et la barbarie dont ils usent à l’égard de ceux qui ne se soumettent pas sur-le-champ ne me laisse aucun doute que si l’on n’y met ordre, il n’y aura plus de sûreté pour nos vies et nos personnes ».
 \end{quoteblock}

\noindent Cette opinion du sous-préfet dOrbe se serait encore fortifiée s’il avait dû assister aux scènes de désordre qui eurent lieu à Orny, près de La Sarraz. Le citoyen de Gingins de Chevilly, ancien trésorier du Pays de Vaud, y possédait un « château » ; âgé et infirme, il se réfugia à Orbe, laissant sa propriété sous la surveillance de son « chargé d’affaires », le citoyen Magnenat et sa famille.\par
La troupe venue d’Agiez, après avoir visité Thomasset à Orbe, se rendit à Orny où, à onze heures du soir, le château fut investi par une foule, évaluée par Magnenat à six cents hommes, deux cents femmes et une centaine d’enfants, et par une autre personne, favorable aux \emph{Bourla-Papey}, à trois cents hommes seulement. L’ex-juge de canton, Potterat, et Abram Glayre, président de la Municipalité de Chevilly, commandaient. Les titres n’étant pas au château, il fallut prendre l’engagement écrit suivant :\par

\begin{quoteblock}
 \noindent « Moi, sous-signé, Louis Magnenat, connaissant les intentions de M. le trésorier de Gingins, m’engage, en son nom, de faire rendre ici, pour mercredi matin (12), tous les titres relatifs aux droits féodaux de ses trois terres d’Orny, Chevilly et Moiry. »
 \end{quoteblock}

\noindent 9 mai 1802, à minuit.\par
Tout cela ne s’était pas fait sans bruit et sans quelques injures qui ne furent, du reste, que le prélude de désordres plus graves racontés par le citoyen Magnenat de la manière suivante :\par

\begin{quoteblock}
 \noindent « Ils me forcèrent de donner à boire et à manger et, forçant la porte d’entrée, se jetèrent dans la chambre à manger, le colidor (sic), la cuisine, le vestibule et même jusque dans la cave ; me poursuivirent à différentes reprises avec ma famille et les autres domestiques, les baïonnettes dans les reins et le sabre nu à la main. Je ne dus ma liberté qu’à force de courir et à l’ombre de la nuit. Ce train a duré jusqu’à lundi à midi. »
 \end{quoteblock}

\noindent Une grande partie de l’attroupement se dissipa à ce moment. Le château resta cependant occupé jusqu’au mercredi où il revint un nombre considérable de personnes.\par

\begin{quoteblock}
 \noindent « L’inquiétude était grande, dit Thomasset, à cause de l’attroupement d’Orny qui grossissait toujours et que l’ivresse rendait redoutable au point que l’ex-juge de canton Potterat, qui était parmi eux, ne pouvait plus en être le maître. Ils annonçaient hautement vouloir venir à Orbe, où le citoyen de Gingins avait réfugié ses effets les plus précieux chez le citoyen Duplessis, pour chercher ses titres qu’ils supposaient y être. »
 \end{quoteblock}

\noindent Thomasset obtint enfin de pouvoir disposer à Orbe des détachements de troupe qui se trouvaient sur la frontière, à Ste-Croix, à Ballaigues et à Vallorbes ; ces vingt-quatre hommes joints à quarante-quatre citoyens de confiance, sous les ordres du commandant Darbonnier, furent pourvus ostensiblement de cartouches. Le sous-préfet donna les ordres les plus stricts et assura enfin la sécurité et la tranquillité.\par
Dès le mercredi matin, 12 mai, le château d’Orny fut entouré et envahi par une foule plus nombreuse et plus menaçante que précédemment. Potterat n’était plus là, malheureusement, pour chercher à la calmer.\par

\begin{quoteblock}
\noindent « On n’entendait parler que de piller, démolir, brûler, raconte Magnenat. Moi et ma femme, dit-il, furent contraints de nous sauver. »\end{quoteblock}

\noindent Les têtes se calmèrent tout à coup quand, à quatre heures après-midi, arriva à Orny le citoyen Binet, capitaine dans la soixante-treizième demibrigage, avec une troupe française. Abram Glayre alla au devant de lui pour lui demander par quel ordre il arrivait.\par
— Par l’ordre du commandant français de Lausanne, répondit Binet.\par
— Dans ce cas, dit Glayre, je vais faire retirer ma troupe.\par
C’est ce qui eut lieu ; le citoyen Binet prit la place des \emph{Bourla-Papey.}\par
La surexcitation était cependant encore si grande que :\par

\begin{quoteblock}
\noindent « le lendemain matin, raconte le \emph{Nouvelliste vaudois}, les femmes d’Orny reprochèrent à leurs maris qu’ils n’étaient que des lâches : elles se rassemblèrent au nombre de vingt-cinq, commandées par la femme de l’agent national, s’armèrent de fourches et de bâtons et arrivèrent devant le château où elles vomirent pendant trois quarts d’heure toutes sortes d’injures, puis allèrent chez un homme du village où elles croyaient que les titres de M. d’Orny étaient déposés, menacèrent sa vie et sa maison et ne furent calmées que par un détachement de Français qui survint. Le commandant Binet fit ensuite rassembler la commune, parla avec bonté et fermeté et rétablit enfin l’ordre dans cette commune \footnote{Voir \emph{Nouvelliste vaudois} du 28 mai et du 8 juin. Voir aussi J. Olivier : \emph{Hist. de la Révolution helv. dans le canton de Vaud ou du Léman, 232.}}.  »\end{quoteblock}

\noindent Le chargé d’affaires de l’ex-trésorier de Gingins nous a laissé la liste suivante de ce qui fut bu, mangé, détruit, au château d’Orny, ou emporté du 9 au 12 mai :\par

\begin{itemize}[itemsep=0pt,]
\item 800 pots de vin nouveau ;
\item 80 pots de vin bouché ;
\item 500 livres de pain (!) ;
\item 200 livres de fromage ;
\item 150 bouteilles et verres brisés ;
\item 70 tasses de fine porcelaine brisées ; couteaux, etc., disparus ;
\item 2 tableaux détruits ;
\item 100 livres de viande de porc.
\end{itemize}
\noindent Deuxième jour : 300 pots de vin et 200 livres de pain.\par
Aucune description ne peut mieux, que cette énumération, montrer la psychologie de certains incidents de la guerre des \emph{Bourla-Papey.}\par
Le district d’Orbe renferma, au reste, lui aussi, des communes qui restèrent tranquilles. Ce fut le cas de Lignerolles et de Valleyres-sous-Rances, où le citoyen Bourgeois, ex-châtelain des Clées, se plut à reconnaître la fidélité des censitaires. En revanche, il n’eut pas à se féliciter de la conduite du village de Rances qui lui devait aussi des dîmes. Cette localité renfermait un des meneurs de l’insurrection, le juge Recordon, et un jeune homme très fougueux, le fils de l’instituteur, Henri Caillachon. Ce dernier disait un jour au fils du pasteur Détraz :\par

\begin{quoteblock}
 \noindent « Ne me parlez pas de lois, ni de justice ; elles ne sont plus de saison. Vous avez vu la révolution de France ; eh bien ! nous la commençons. »
 \end{quoteblock}

\noindent Quarante hommes de Rances arrachèrent à l’exchâtelain des Clées les copies de ses titres et enfin les originaux.
\section[Echallens et Goumoëns]{Echallens et Goumoëns}
\noindent Le district d’Echallens fournit un grand nombre d’hommes à l’armée de Louis Reymond ; il renferma quelques meneurs écoutés et plusieurs chefs de troupes dont le plus connu est Jean Isaac, d’Etagnières.\par
A partir du 7 mai, beaucoup de contingents traversèrent la localité, se dirigeant du côté de Lausanne.\par
Le huit, à deux heures du matin, une colonne d’environ 150 hommes sous la direction des citoyens Cevey de Cheseaux et Jean Isaac d’Etagnières, arriva tambour battant au château d’Echallens. Les coffres furent enfoncés et les titres enlevés. Un certain nombre de campagnards se rendirent ensuite chez le Receveur national et chez les particuliers dépositaires de titres féodaux. Le souspréfet livra aussi à deux heures et demie du matin ceux qu’il possédait pour les seigneuries de St-Barthélemy et de Brétigny.\par
Tout ce qui avait été pris fut chargé sur un grand véhicule et la troupe se dirigea du côté de. Lausanne. Une personne d’Etagnières, qui aperçut un peu plus tard ce convoi, en fit la description suivante dans une lettre adressée au Préfet national.\par

\begin{quoteblock}
 \noindent « Je les vis passer tambour battant, huant, chantant, portant au bout de leurs baïonnettes des papiers d’archives pillés à Echallens, avec le convoi d’un char qui en était chargé et sur lequel était assis le citoyen Jean Isaac. »
 \end{quoteblock}

\noindent C’est le village de Goumoêns qui, dans le district d’Echallens, vit se passer les scènes les plus caractéristiques.\par
Le 7 mai, à deux heures après midi, deux Municipaux vinrent au château demander les titres de M. de Goumoêns. Il y eut quelques menaces lorsque ce dernier annonça que ses papiers se trouvaient à Berne et à Neuchâtel.\par
On entendit ensuite le roulement du tambour dans les environs. Le seigneur demanda du secours à Lausanne d’où on lui envoya une sauvegarde de deux sous-officiers et neuf soldats français. Les hommes de la localité avouèrent à ces derniers que leur arrivée avait empêché que l’on attaquât le château.\par
Le 8, M. de Goumoêns, apprenant qu’on en voulait à sa personne, alla se réfugier à Orbe, laissant le château sous le commandement de son beau-frère, le citoyen de Mellet, qui avait l’autorisation de livrer quelques vieux papiers. Ces derniers furent emportés en effet par les gens du village conduits par le pasteur Louis Dufour, dit « le Grand Ministre \footnote{Louis Dufour fut pasteur à Goumoêns de 1800 à 1831.}  », et le municipal Bezencenet, lequel trouvait que \emph{« c’était bien dommage que l’on n’eût pas tous ces titres pour en faire un beau feu qui montât bien haut. »}\par
Le dimanche 9 mai, le culte fut interrompu par le bruit du tambour, et le pasteur lui-même ne tarda pas à se montrer sur la place publique. Une cinquantaine d’hommes environ, la baïonnette au fusil et portant une « seille » pleine de cartouches, se présentèrent devant le château, sous le commandement de Samuel Dufour. Ils reçurent la même réponse que deux jours auparavant.\par
— Eh bien ! faites-les venir, dit un des insurgés. Il nous les faut dans une heure.\par
Il y eut un éclat de rire général, puisque les titres se trouvaient à Berne. On voulut exiger qu’ils fussent livrés le jeudi. Le citoyen de Mellet refusa, mais, en revanche, donna quelques papiers « pour calmer les plus enragés. » Il finit cependant par céder, mais en exigeant une capitulation écrite. Les paysans demandèrent tout d’abord du vin ; il leur fut répondu que l’on ne donnait pas à boire à des voleurs. Pour s’en débarrasser, on leur accorda cependant à chacun une bouteille au cabaret.\par

\begin{quoteblock}
 \noindent « On dressa la capitulation qu’on leur envoya par le sergent français à qui ils refusèrent de la signer après avoir bu le vin accordé pour cela, menaçant de venir de suite pour forcer le château. Il leur assura qu’ils n’avaient qu’à essayer, qu’ils seraient reçus chaudement. Il dit dans son rapport que comme les chefs étaient disposés à signer, un grand jeune homme en habit bleu, en pantalon de tricot, s’était élevé fortement contre cette capitulation qu’il ne fallait jamais accorder, que M. de Mellet les avait leurrés, qn’il ne voulait que gagner du temps, qu’il fallait avoir les titres coûte que coûte, qu’il ne voyait pas pourquoi on ménageait M. de Goumoêns, que ce b…-là n’avait jamais cherché qu’à les mordre, etc. Peu après\emph{, raconte le citoyen de Goumoêns}, mon beau-frère reçut une députation de la commune dont l’orateur était cet homme véhément qui avait fait rompre la capitulation ; on le reconnut à son signalement ; c’était le grand ministre, Louis Dufour. »\par
 Le citoyen de Mellet continua à résister.\par
  Madame ayant voulu parler :\par
 « — Non, Madame, dit le Ministre, nous respectons la propriété, mais nous voulons vos titres. Nous savons que nous vous faisons du mal, mais il nous les faut, et cela aujourd’hui. Si nous manquons cette occasion, nous ne la retrouverons plus. »\par
 « L’homme divin », comme l’appelle le seigneur de Goumoëns, se retira enfin sans avoir rien obtenu et après avoir accordé un délai finissant le 12 mai, à trois heures. »
 \end{quoteblock}

\noindent Ce jour-là, la guerre était terminée et :\par

\begin{quoteblock}
\noindent « les insurgés, revenant du camp, couronnés de feuillage et d’hellébore, passèrent devant le château, tambour battant, et avec deux violons, mais avec le plus grand ordre et toute la décence d’une troupe réglée. Ils ne firent pas la moindre insulte. Le grand ministre Dufour alla à leur rencontre et les conduisit chez la citoyenne Jordan, sa sœur, où une table bien garnie les attendait et où l’on porta les santés réitérées du dit Grand Ministre, de la citoyenne Jordan et du général Reymond \footnote{Rapport du citoyen de Goumoëns.}. »\end{quoteblock}

\section[Cossonay]{Cossonay}
\noindent Le district de Cossonay fut un de ceux qui contribuèrent le plus à grossir l’armée de Louis Reymond. Pendant que des détachements allaient la rejoindre, des citoyens en revenaient pour augmenter le zèle des différents villages et y retourner avec de nouvelles colonnes.\par
Le chef-lieu, résidence du sous-préfet de Charrière, fut relativement calme et ne fournit que très peu de renforts à l’insurrection.\par
Un certain nombre de dépôts d’archives furent visités dès le premier jour. Ce fut le cas pour Grancy où le citoyen Monnet, dit Baron, montra beaucoup de zèle.\par
Le 7 mai, à dix heures du soir, Jean Isaac d’Etagnières, qui devait quelques heures plus tard s’emparer des archives d’Echallens, arriva à Boussens, devant la maison du citoyen de Saussure.\par
— Que voulez-vous ? demanda-t-on à cette troupe.\par
— Nous venons demander les droits féodaux.\par
— Par quel ordre ?\par
— Par la force armée.\par
— Ils ne sont pas ici.\par
— Nous ferons « la cherche ».\par
Ils fouillèrent tout ; on promit les titres pour le lendemain et la troupe s’en alla.\par
Il en revint une autre le 12 mai, composée de citoyens qui criaient : – A bas le seigneur de Boussens ! A bas tous les seigneurs ! A bas les droits féodaux !\par
Le chef envoya la sommation suivante :\par

\begin{quoteblock}
 \noindent « Le chef de la colonne vous invite pour la dernière fois à donner les papiers concernant les droits féodaux ; à ce défaut, j’ai toute autorité pour faire entrer chez vous la colonne de ce lieu à vos frais ».\par
 (signé) : \emph{Le chef de la colonne de ce lieu.}
 \end{quoteblock}

\noindent Ils eurent les papiers.\par

\begin{quoteblock}
\noindent « Ils les enlevèrent, les frappèrent sur le pavé de la cour, les foulant aux pieds. »\end{quoteblock}

\noindent Le même jour, à sept heures du soir, le citoyen Mercier, seigneur de Bettens, signait une renonciation de ses droits. Le jour précédent, le citoyen d’Albenas, à Sullens, avait pu promettre tranquillement ses titres qui ne furent emportés que le 12 mai.\par
Le 9 mai, ce fut le tour de La Sarraz. A trois heures du soir, il y arriva une troupe de Pompaples et de Chevilly aux ordres de Abram Glayre. Elle se rangea en bataille devant l’auberge de la Couronne et obligea Potterat, d’Orny, qui s’y trouvait, de prendre le commandement. II se soumit de mauvaise grâce ; on l’affubla a d’un baudrier et d’un chapeau à trois cornes » et il marcha avec les insurgés.\par
Cette troupe voulut tout d’abord pénétrer dans le château.\par

\begin{quoteblock}
 \noindent « Ils s’y portèrent en foule et en enragés, bien que leur chef, Potterat, cherchât à les apaiser. Ils prirent des brantes qui étaient dans la cour, les remplirent de papiers et de titres qui n’avaient aucun rapport avec les droits féodaux, les emportèrent et détruisirent toutes les archives à l’exception de deux layettes ».
 \end{quoteblock}

\noindent Ils allèrent ensuite demander à l’agent national ce qu’il possédait des archives du château.\par

\begin{quoteblock}
 \noindent « Il leur remit tous les chiffons et débris qu’il avait pêchés dans la rivière le lendemain et jours suivants du vol du 19 février. Ils en remplirent un tonneau, conduisirent le tout sur le Mauremont et le tout a été brûlé de suite ».
 \end{quoteblock}

\noindent Potterat, Gleyre et leur troupe se rendirent enfin à Eclépens où leurs hommes envahirent le château et allèrent brûler les titres au bas du village.\par

\begin{quoteblock}
 \noindent « Il fallut leur donner six seilles de vin… Peu de personnes de l’endroit prirent part à cet événement, à l’exception des femmes et des enfants, parce que tout ce qui était en état de porter les armes se trouvait sons les ordres du citoyen Reymond ».
 \end{quoteblock}

\noindent On a déjà vu que Potterat et Gleyre participèrent ensuite aux événements d’Orny.\par
Le jour précédent, à Cuarnens, les citoyens avaient absolument voulu détruire les archives de la commune où ne se trouvaient cependant pas de titres féodaux.\par

\begin{quoteblock}
 \noindent « Ils prirent les papiers, des comptes de la Municipalité, tous les décrets, arrêtés, lois et ordres que la Municipalité gardait, disant qu’il n’y avait plus de gouvernement ni d’autorités quelconques, qu’ils se gouverneraient bien eux-mêmes… Tous les papiers furent mis dans le fourneau et brûlés ».
 \end{quoteblock}

\noindent Le 10 mai, une colonne, conduite par Cardinaux et venant d’Orny, demanda les archives de Cossonay qui lui furent livrées. La Régie dut donner aussi une quittance générale en faveur de toutes les communes qui devaient des droits à celle de Cossonay. Le citoyen de Charrière, de Penthaz, fut spolié à la fin de la même journée. A Mex, à Daillens, à Bournens, les archives furent détruites le 12 mai, lorsque les insurgés rentrèrent dans leurs foyers.\par
De tous les propriétaires du district, le citoyen Crinsoz de Cottens fut celui qui eut le plus à souffrir des menaces et des invectives des insurgés. Voici comment il raconte lui même l’événement, qui eut lieu dans la nuit du 4 au 5 mai :\par

\begin{quoteblock}
 \noindent « Environ les cinq heures du matin, arrive au bas du village et près du cabaret, une troupe armée de septante à huitante individus, tambour battant, avec des cris affreux : après s’être arrêtés là près d’une heure, je les entendis s’acheminer en haut du village ; je rentre et ferme ma porte ; la troupe se range en bataille devant ma maison ; je n’en connus d’abord aucun individu ; l’un d’eux bourre et heurte à force de bras ; je lui demande par la fenêtre ce qu’il veut ; il m’intime l’ordre d’ouvrir sous menace d’enfoncer ma porte, comme il dit avoir fait ailleurs. Il était armé d’une hache ; je lui ouvre et veux le faire entrer seul ; alors toute la troupe force le passage et se jette en tumulte dans les appartements ; celui qui avait porté la parole me demande \emph{au nom des paysans armés pour la destruction de la féodalité}, de lui livrer mes titres ; sur la réponse que je les ai mis hors de chez moi, en lieu de sûreté, s’élève un cri de fureur dans toute la troupe : \emph{tes titres ou ta tête} ; je répète que je ne les ai pas, même cri répété : \emph{ta tête ou tes titres} ; je leur répondis que vu force majeure, la première était entre leurs mains, mais que les titres ne leur reviendraient pas pour cela, puisqu’on ignorait où ils étaient. A ces mots, la rage devient à son comble, tous crient à la fois : \emph{il faut le massacrer}, et au même moment, on lève la hache sur ma tête, ainsi que vingt crosses de fusil ; d’autres m’appuyaient sur le corps des baïonnettes ; l’on propose de me fusiller pour l’exemple, sur ma terrasse ou devant ma porte ; deux opinent pour me pendre et détachent leurs cravates en guise de corde. On me prodigue des épithètes les plus injurieuses ; on me traite de voleur, de brigand, de sangsue du peuple. Sur ces entrefaites, un de la troupe s’écrie qu’il faut avant tout avoir les titres et qu’on doit persister et ouvrir les appartements et les armoires. On me traîne, on me pousse de bas en haut ; la troupe se partage ; chacun visite de son côté ; la maison retentit de cris et de menaces ; ma femme accourt et veut prendre mon parti ; un furieux ouvre la fenêtre du corridor, veut la saisir et la jeter dehors ; je parviens à la pousser contre sa chambre ; cet enragé voyant que sa proie lui échappe, la poursuit et veut lui donner un coup de crosse qui tombe sur la double porte ; il retourne son fusil et lui lance sa baïonnette au moment où elle fermait la porte intérieure, contre laquelle la baïonnette glissa. Les feuilles, livres et tous les papiers qu’on avait trouvés furent déchirés ; cette troupe, furieuse de ne rien trouver que de vieux titres qu’on mit en lambeaux, recommença à m’accabler d’injures, menaces et bourrades, délibère en tumulte sur le parti à prendre et décide d’environner la maison de seringues \footnote{Pompes à feu.} pour préserver les bâtiments voisins, et de m’y brûler moi, ma famille et mes richesses, disent-ils, ainsi que mes titres qu’ils croyaient y être renfermés. Du milieu de ces débats, quelques voix proposent de me donner vingt-quatre heures pour les produire, et on me dicte de suite un engagement sur papier timbré de les livrer aux communes sous vingt-quatre heures, et d’en faire une renonciation formelle pour moi et les miens : je le fis pour éviter de plus grands maux. On alla la présenter au chef de l’attroupement et le porteur revint dire qu’on l’acceptait \footnote{\emph{Nouvelliste Vaudois} du 1\textsuperscript{er} juin 1802.}  ».
 \end{quoteblock}

\noindent Les titres furent livrés et emportés le lendemain.
\section[Aubonne]{Aubonne}
\noindent Le district d’Aubonne, de même que ceux de Rolle et de Nyon, fut très agité et fournit à l’armée de Reymond des contingents importants. Le chef-lieu du district participa lui-même dans une certaine mesure à l’insurrection et la Municipalité lui fut en partie favorable.\par
Le citoyen Grivel, sous-préfet, se joignit dès le 5 mai à l’autorité communale pour prendre quelques mesures de précaution et de prudence. Il organisa des patrouilles et les papiers féodaux, qui se trouvaient chez le receveur, furent transportés aux archives nationales et placés sous la garde des huissiers qui logeaient au-dessus.\par
Pendant la nuit suivante, à une heure du matin, Grivel et la Municipalité furent avertis qu’il y avait un grand feu dans le vallon derrière le château. Ils trouvèrent le local des archives forcé et vide. Une garde de quelques hommes avait empêché l’huissier de sortir de chez lui.\par
Le 6 mai, à onze heures du soir, plusieurs troupes, dont l’une venait du camp de Reymond et une autre de Lavigny, se rencontrèrent à Aubonne pour aller s’emparer des titres du citoyen de Mestral de St-Saphorin dans sa propriété de l’Aspre. Ils s’y rendirent sous la direction de Benjamin Fusay, de Lavigny. Ils s’emparèrent de tout ce qui concernait ce village. Ils restèrent encore pendant environ deux heures dans le vestibule à insulter le citoyen de Mestral et ne se retirèrent qu’à l’arrivée du sous-préfet.\par
Reymond pria à ce moment la ville d’Aubonne de lui envoyer un contingent et le municipal Galliker se donna beaucoup de peine pour que l’on répondit favorablement. Le 7 mai, au matin, il fit battre la générale par l’huissier du sous-préfet et distribuer 20 batz aux partants. La Régie leur promit de son côté un écu neuf et plus tard envoya au camp un char de pain et de fromage. Le président de la Municipalité donna une gratification au tambour et la colonne se mit en route aussitôt après midi. Un drapeau vert flottait au vent et Galliker avait le commandement de la troupe.\par
Les gens de Saubraz vinrent de leur côté, brûler les titres de la Régie d’Aubonne, et le sous-préfet remit ceux qu’il possédait sur ce village et celui de Gimel. Les cadastres de Yens et Bussy furent enlevés aussi et détruits.\par
Le 9 mai, à onze heures du soir, par l’ordre de Reymond, une expédition eut lieu à Trévelin, près d’Aubonne, chez le citoyen Crinsoz. L’aubergiste Comte, membre de la Municipalité et de la Régie, en fut le chef.\par

\begin{quoteblock}
 \noindent « Ne me trouvant pas à la maison, dit Crinsoz, il dit à mon cocher qu’il venait chercher ce qui leur appartenait. Le domestique voulut aller chercher son maître ; mais le citoyen Comte lui dit qu’il ne sortirait pas, le consigna et le somma de lui faire voir où étaient renfermés les titres. Le cocher, pressé par les menaces et les baïonnettes, indiqua une armoire. Comte enfonce l’armoire et, n’y trouvant que des papiers insignifiants pour son but, passe à l’armoire contenant le linge de ma femme et la force aussi. Le cocher voulut, pour m’en aviser à Aubonne, sauter par la fenêtre du premier étage. La cuisinière le retint en lui assurant qu’il se tuerait ou se casserait les jambes ; elle l’engagea à descendre au rez-de-chaussée pour en sortir par la fenêtre du côté des jardins, toutes les avenues, portes extérieures et intérieures étant garnies de sentinelles. »
 \end{quoteblock}

\noindent Le citoyen Crinsoz, enfin prévenu, arriva bientôt à Trévelin avec le sous-préfet. Comte se modéra un peu tout en continuant cependant à exiger les titres.\par

\begin{quoteblock}
\noindent « Mon cocher, à son retour, ne put plus se modérer, dit Crinsoz, et le chargea d’insultes, qui étaient autant de vérités. Pour éviter un malheur, j’enfermai mon domestique sous clé. »\end{quoteblock}

\noindent Le propriétaire finit par remettre les titres qu’il avait à Trévelin, mais Comte l’accusa de ne pas avoir tout donné, le menaça de brûler sa maison et ne se retira avec sa troupe qu’à deux heures du matin.
\section[Rolle]{Rolle}
\noindent Dans le district de Rolle, l’agitation eut pour centre le village de Perroy, où existaient plusieurs propriétaires de titres féodaux : les citoyennes May et de Chandieu et le citoyen Demartines-Couvreu, agent national. Les hommes qui contribuèrent le plus à préparer l’incendie des archives furent les frères Bron, de Bougy ; l’un, François, membre de la Municipalité, et Henri, le plus actif des deux. Dès le commencement des troubles, ce dernier s’était rendu au camp de Louis Reymond ; il en était revenu le 7 ou le 8 au matin pour chercher à donner une nouvelle impulsion au soulèvement. Les deux frères descendirent à Perroy et eurent une entrevue avec la Municipalité, au cabaret tenu par le Savoisien Jacob Blanchard.\par
Par ordre de l’autorité locale, les hommes portant les armes furent convoqués pour le 8, à neuf heures du soir, sur le chemin d’Etraz, où ils devaient rencontrer le contingent de Bougy.\par
Une avant-garde descendit au pas de charge, suivie bientôt du gros de la troupe. Le « château » May fut entouré et les titres qu’il renfermait livrés aux campagnards.\par
Ces derniers allèrent aussitôt se ranger en bataille devant la maison du citoyen Demartines qui donna à Bron quatre grands coffres renfermant les titres de Pailly, Essertines, Mont et St-Georges. Les insurgés réquisitionnèrent un grand tombereau, le chargèrent des titres et partirent.\par
La troupe se rendit alors chez « la citoyenne de Chandieu », personne nonagénaire qui vivait entourée d’employés et domestiques formant une famille, dont le chef était lui-même très âgé. Les titres se trouvant à Lausanne, Blanchard et Bron furent très irrités.\par

\begin{quoteblock}
 \noindent « Ils se firent ouvrir l’armoire où avaient été les papiers précédemment, dit Mme de Chandieu. Ils ne virent que de vieux parchemins du temps des papes ; ils parcoururent toutes les armoires de la maison ; ils ne trouvèrent rien, non plus qu’au premier. Pendant ces altercations, d’autres se firent ouvrir la cave par le maréchal du lieu. Ils sortirent tous ivres avec des imprécations. Ces scènes durèrent de neuf heures et demie à minuit. »
 \end{quoteblock}

\noindent Après avoir donné à M\textsuperscript{elle} de Chandieu un délai de vingt-quatre heures pour livrer ses titres, les insurgés se rendirent à Rolle, afin d’y demander ceux du baron, le citoyen Kirchberger.\par
Le sous-préfet, Preud’homme, et la Municipalité s’étaient assemblés et siégeaient en permanence pendant que quatre citoyens sûrs faisaient le guet. La colonne entra à Rolle à deux heures du matin et rencontra bientôt le sous-préfet qui chercha inutilement à lui faire abandonner son projet. Le citoyen Humbert, homme d’affaires du baron de Rolle, dut par conséquent ouvrir la chambre des archives, lesquelles furent enlevées aussitôt et détruites l’instant d’après.\par
Une colonne s’en alla ensuite auprès du citoyen de Watteville, à Malessert.\par

\begin{quoteblock}
\noindent « La troupe avait une cinquantaine d’hommes presque tous ivres-morts, dit ce dernier. Deux chefs (les frères Bron) entrèrent, prirent les titres et s’en allèrent. Ils revinrent après et me dirent : « \emph{Notre troupe se recommande pour un coup à boire} ; ils me firent observer que je perdais très peu de chose et ne pouvais refuser. »\end{quoteblock}

\noindent Sachant déjà ce qui venait de se passer à Perroy, de Watteville descendit à la cave et donna une « seille » pleine de vin.\par

\begin{quoteblock}
\noindent « Ils burent à ma santé, crièrent \emph{qu’il vive} ! et s’en allèrent tambour battant à Perroy, où ils brûlèrent mes titres. »\end{quoteblock}

\noindent Les insurgés n’oublièrent pas, le 9 mai, le rendez-vous qu’ils avaient donné à Mme de Chandieu. Une soixantaine d’entre eux se rendirent au Prieuré, suivis d’une foule de gens armés d’échalas.\par

\begin{quoteblock}
 \noindent « Mme de Chandieu ayant encore ses titres à Lausanne chez M. de Grancy, raconte l’agent Demartines, ils se livrèrent à tous les excès, cassèrent toutes les fenêtres qui donnaient sur la cour, mirent en joue Mme de Chandieu. Ils maltraitèrent cruellement le valet, parcoururent les appartements, fracturèrent un buffet, volèrent des chandelles, forcèrent la cave où Louis Martin leur distribua, de concert avec Jacob Blanchard, plus de cent pots de vin, pillèrent des fromages… Cette nuit fut affreuse…\par
 « Jugez de mon anxiété, raconte l’agent national, lorsque, un moment après avoir entendu commander : \emph{En avant} ! A \emph{bas les tyrans} ! J’entendis des voix étouffées crier : \emph{En aide} ! à réitérées fois ; mes cheveux s’en hérissaient sur ma tête. »\par
 Mme de Chandieu resta, dit-elle, « froidement inflexible. Un des chefs menaça de brûler ma maison.\par
 — « Vous le pouvez, dis-je, vous verrez ce qui vous arrivera.\par
 « Dans ce moment, je vis les jeunes filles pleurer et crier :\par
 — « On assassine mon père !\par
 « Je voulus aller à lui, sa femme n’étant point à la maison, l’ayant fait absenter sur l’avis d’un complot. Je ne pus jamais parvenir jusqu’à lui, ce vieux domestique étant couché par terre, tenu par dix hommes, dont un pesait sur sa gorge jusqu’à l’étouffer. Il fut délivré. Je revins donc à ma place, où je fus persécutée de nouveau par un de ces scélérats qui me coucha en joue. Je me jetai de côté pour l’éviter… Pendant ce temps, une partie de ces infâmes occupèrent la galerie, tirant des coups de fusils et cassant des vitres, cherchant à voler… »
 \end{quoteblock}

\noindent Le lendemain, les titres furent transportés de Lausanne à Perroy, où une vingtaine d’hommes vinrent d’Allaman pour les prendre.\par
Pendant que, dans la nuit du 8 ou 9 mai, les archives de Perroy et de Rolle étaient détruites, une autre troupe alla demander, au Rosay, celles du citoyen Rolaz. Ce dernier arriva à sa porte au moment où elle était déjà assiégée par un grand nombre d’hommes. Il refusa ses titres, disant que leur sort serait décidé par le gouvernement.\par

\begin{quoteblock}
 \noindent « En conséquence, dit-il, une partie de la troupe se débanda ; une autre s’amusa à tirer contre les tours et les toits aussi longtemps qu’ils eurent des cartouches ; sept ou huit entrèrent dans ma chambre et persévérèrent pendant longtemps à vouloir que je leur donnasse quelque chose à brûler. »
 \end{quoteblock}

\noindent Tous s’en allèrent enfin ; le citoyen Rolaz garda ses titres.\par
Pendant la même nuit, une troupe, aux ordres de Jean-François Bourguignon, de Gilly, s’empara des archives du citoyen Jean-lsaac Thélusson, à Dullit.\par
Le citoyen Vasserot, seigneur de Viney, reçut, pendant la nuit suivante, la visite d’une colonne commandée par Baptiste Caillat, de Tartegnins. N’obtenant pas de titres, elle dut se contenter d’un acte de renonciation.\par
Le 10 mai, à sept heures, deux hommes apportèrent encore au citoyen Vasserot la sommation suivante, qui est trop originale pour ne pas être placée sous les yeux du lecteur.\par

\begin{quoteblock}
 \noindent « Du 9 mai 1802.\par
 « Citoyen Vasserot, je viens vous inviter à me donner les deux livres de plans que vous avez encore chez vous et que nous ne pouvons nous en passer, et que d’ailleurs c’est une chose précieuse pour la commune, et que par ainsi je vous prie de me les envoyer par la commission que je vous envoie, sans quoi je suis obligé de monter avec ma colonne qui ne montera qu’à double pas.\par
 Signé : Je suis le Chephe de Colone. »
 \end{quoteblock}

\noindent Vasserot envoya les livres demandés.\par
Ce dernier raconte, au sujet de ces événements, un incident qui montre que la vie des seigneurs fut quelquefois menacée en dehors des heures où leur maison était envahie. Il se promenait à onze heures, avec sa femme, le 10 mai, sur la terrasse de sa maison, lorsqu’une balle siffla à ses oreilles. Il recommandait à sa compagne de rentrer lorsqu’un second coup de fusil se fit entendre.\par

\begin{quoteblock}
 \noindent « La balle, après avoir passé à peu près à une toise de nous, alla frapper contre un arbre voisin. »\par
 « Ma femme, enfin rentrée, on me fit observer un homme à trente pas de nous sur le sentier de vignes tendant de Gilly à Viney. J’eus le temps encore de voir l’homme faire un mouvement avec son fusil. »
 \end{quoteblock}

\noindent Vasserot observa cet individu pendant un instant et rentra chez lui. Le lendemain, ajoute-t-il :\par

\begin{quoteblock}
\noindent « ma femme me dit qu’après les deux coups de fusil, la domestique avait entendu, de sa fenêtre, une voix qui disait : \emph{Est-il bas ?} »\end{quoteblock}

\section[Nyon]{Nyon}
\noindent A Nyon, le sous-préfet Nicole profita du rassemblement considérable causé par la foire du 6 mai, pour engager les agents nationaux des différents villages à montrer toute la fermeté possible. Il dut se convaincre, du reste, qu’il ne devait compter que sur un bien petit nombre de personnes. Deux propriétaires de fiefs, Arpeau et Desvignes, lui assurèrent que :\par

\begin{quoteblock}
\noindent « nul de la campagne n’obéirait pour marcher d’après ses ordres –, que les plus obéissants seraient calmes spectateurs. »\end{quoteblock}

\noindent La journée du 7 mai fut encore fort tranquille en apparence. Pendant la nuit du 7 au 8, le citoyen Nicole fut averti que, « dans la campagne du côté de Genève, on entendait la caisse et, peu d’instants après, qu’une armée considérable marchait sur Nyon. 11 se rendit avec le président de la Municipalité au-devant de cette colonne qui, après un instant d’arrêt, se remit en route, tambour battant, entra à Nyon et se rangea en bataille devant la maison du citoyen Bonnard, Receveur national. Le sous-préfet, voulant encore s’interposer, fut écarté avec violence et tous les papiers furent enlevés, de même que ceux des particuliers qui en possédaient aussi.\par
A quatre heures du matin, cette troupe se présenta au château de Coppet. Les papiers de famille furent respectés et les autres brûlés sur la place d’armes. Dans son rapport, le citoyen Necker, ancien ministre des finances de Louis XVI, avoue avec une pointe de malice que :\par

\begin{quoteblock}
\noindent « la scène s’est passée avec convenance, le genre une fois admis. »\end{quoteblock}

\noindent C’est pendant la nuit suivante, celle du 8 au 9, que le plus grand nombre des propriétaires reçurent la visite des \emph{Bourla-Papey.}\par
Dès les quatre heures après midi, une troupe, conduite par le citoyen Saugy, de Founex, et Deblue, du même village, se rendit chez le citoyen Saladin, à Crans, où tout fut indistinctement enlevé.\par
Vers le soir, les archives du citoyen de Saint-Georges furent brûlées devant chez lui, au château de Duillier. A sept heures, la même opération eut lieu au château de Prangins, propriété du citoyen Guiguer.\par
Le citoyen Gottraux, meunier à Promenthoux, s’en alla pendant la nuit, avec une petite troupe, brûler les titres de Coinsins, sans commettre de violences inutiles. Il se joignit ensuite à la colonne, beaucoup plus nombreuse, des gens de Begnins, conduits par le président de la Municipalité, le citoyen Morsier, et Charles Francfort. Ils s’en allèrent tous demander, à Begnins, les archives de la seigneurie de Cottens, appartenant à la citoyenne veuve Garcin, née Stûrler. Le plus fougueux de la troupe fut le nommé Moïse Bugnion qui, au moment où les titres étaient déjà devant la maison, rentra encore une fois seul, le sabre au clair, pour prendre les plans territoriaux.\par

\begin{quoteblock}
 \noindent « Ils sont enfin partis, dit Mme Garcin, emmenant avec eux, à bras, un petit char chargé de plus de vieux papiers que de titres, car leur inexpérience s’est montrée égale à leur brigandage. Au bout d’une heure et demie, quatre d’entre eux sont revenus et parmi eux ce même Bugnion qui, outre beaucoup de mauvais propos tenus, a dit devant quelqu’un de la maison digne de foi, que ce n’était pas la dernière fois qu’ils comptent faire visite et qu’il espère bien avoir sa part du partage du domaine. »
 \end{quoteblock}

\noindent Avant de se rendre chez la citoyenne Garcin, la troupe de Nicolas Francfort et Samuel Morsier avait été demander les titres de Ami Rigot dans le même village. Laissons celui-ci raconter l’événement.\par

\begin{quoteblock}
 \noindent « Le neuf mai, à deux heures et demie du matin, on a eu infiniment de peine à me réveiller. J’étais harassé par une veille consécutive de plusieurs jours. Quelques coups de fusil se font entendre autour de ma demeure. La caisse annonce aussi la cohorte des pillards qui se rend premièrement à la Municipalité, et chez moi, seulement vers les trois heures du matin. J’entends rudement frapper à ma porte. Je m’y rends avec mon frère, qui crie :\par
 — Qui est là ?\par
 — Ouvre, amis, répond-on.\par
 « Ayant ouvert, un homme, le sabre à la main, se présente suivi d’une troupe armée.\par
 — Que voulez-vous ? lui dis-je.\par
 — Les titres féodaux de Begnins.\par
 — De quel droit et par quel ordre ?\par
 — De la part du peuple.\par
 — Qui est le chef, ici ?\par
 — C’est moi, me dit le même homme.\par
 « Je lui déclarai que je ne résisterais pas à une force armée ; que je donnerais les clés ; mais que je n’entendais pas que personne d’autre que lui et ceux qu’il nommerait pénétrât dans ma maison, à la sûreté de laquelle il pourvoirait.\par
 « Il donna, en effet, des ordres en conséquence à deux factionnaires qu’il plaça en dedans, et, avec quatre hommes, il se porta aux archives. – L’un d’eux faisait le commissaire, demandait les cottets, les quernets, les grosses. Je les lui montrai dans une cachette et il se saisit de tout, et même de mes autres livres qui étaient sur des layettes. Il m’emportait même le \emph{Traité d’agriculture} de l’abbé Rozier, Bayle, Moreri et d’autres gros dictionnaires, divers papiers étrangers aux droits féodaux. Je l’obligeai à les remettre. – Je lui ouvris le bureau des parchemins où étaient les titres originaux des dixmes et plusieurs autres que mon frère m’avait enjoint de livrer par prudence. Une caisse renfermait une foule de titres inutiles ici, concernant Trélex, Nyon, Prangins, etc. J’avais une copie vidimée pour les dixmes. Tout cela fut pris et enlevé avec une multitude d’abergements, de chartes de fondation et autres titres authentiques, scellés et non scellés.\par
 « Tout cela avec nos grosses forma bientôt un gros tas dans la cour du château de Martheray, près de la porte de la maison. Après cela, le président municipal, Samuel Morsier pénétra à son tour dans les archives avec la force armée, en disant que \emph{la Municipalité devait y être aussi.} Il a exigé les plans du domaine, que j’ai refusés. 11 a insisté auprès du commandant qui ne l’a pas appuyé comme il le croyait.\par
 « Le tas de titres et de parchemins ayant été fait, une foule égarée qui paraissait haletante de soif et de pillage s’est portée autour, d’un air mécontent et menaçant. J’en ai reconnu quinze de Begnins armés et trois non armés.\par
 « Le chef, que Ton appelait commandant, était Goton, de Benex. Il y en avait aussi de Gland, de Vich, de Duillier, et un jeune orphelin de Genollier, nommé des Vignes, enfin mon bovéron, Jean Grobet, de Gland, qui me servait depuis cinq semaines, et auquel j’avais avancé deux écus pour acheter un quarteron de chenevis que sa pauvre mère voulait semer, disait-il. Il avait employé cet argent à acheter de la poudre à tirer pour venir contre ses maîtres.\par
 « Je lui avais confié pour cette nuit la garde des écuries, le bouvier-chef étant auprès de moi. Mais il les avait ouvertes et était allé joindre la bande des brigands.\par
 « Il y avait en tout cinquante à cinquante-cinq hommes armés, mais je n’ai pu tous les reconnaître. Ils criaient (ceux de Begnins particulièrement) :\par
 {\itshape — Tout n’y est pas, qu’on fouille la maison !}\par
 Après plusieurs récidives de propos pareils, ils ont fait mine de me saisir. On a crié au tambour de battre \emph{un rouf fie} pour faire avancer l’arrière-garde.\par
 « Cela a été exécuté et j’ai vu en effet briller dans l’obscurité quelques baïonnettes qui s’avançaient. Le tumulte devenait de plus en plus sérieux. Les cris de : \emph{A bas le tyran} ! s’en mêlaient. Je me contentai de répondre avec fermeté :\par
 « Vous êtes les plus forts et vous pouvez me massacrer, saccager ma maison, mais non me faire livrer d’autres titres que ceux-là.\par
 « Je répétai cela lorsque le commandant, faisant avec d’autres, mine de me saisir et de pénétrer chez moi me dit : – Croyez-moi, exécutez-vous et évitez des malheurs. – Enfin, après mille injures, menaces et vociférations, les brigands me sommèrent de livrer du pain, du vin, la clef des caves, les titres des eaux pour lesquels j’avais été en procès avec la commune. Je refusai toujours nettement, et nos gens de Begnins criaient de plus belle que dans un quart d’heure ils les trouveraient bien.\par
 « Le commandant me somma une dernière fois de produire le reste. Quatre baïonnettes, dirigées sur ma poitrine, me firent reculer contre le mur de la terrasse qui est dans ma cour. Le commandant me donna deux heures de réflexion, \emph{au bout desquelles il ne répondait plus de rien.} Il plaça deux factionnaires en dehors de la porte de la maison, pour la garde des titres, je suppose, et il emmena sa troupe.\par
 « Ces deux heures ont été bien longues et bien angoissantes. En attendant l’issue, les bandits arrivaient les uns après les autres, fouillaient le monceau de titres, lisaient, commentaient et m’insultaient. Les factionnaires dormaient, (car ils avaient déjà été à Changin, Duillier, Prangins, etc.). J.-J. Francfort a beaucoup fouillé partout. Les titres pouvaient ainsi être saisis, ce qui aurait fourni un nouveau prétexte contre nous. Il était environ cinq heures et demie lorsque la bande reparut. Le tumulte recommence, on me somme de nouveau de produire ce qui manque. Même réponse. Les injures, la cruelle et basse ironie se « donnent carrière. \emph{A bas le tyran} ! A \emph{Viney ! à Viney} !\par
 \emph{Ah ! Monsieur de Begnins, nos vos baillions on bé mai de may}, disaient surtout les gens de Begnins. Le commandant m’avertissait de nouveau de prendre garde à moi s’il manquait des titres. Mais, au plus fort du vacarme, il ordonne qu’on emmène le char qui porte nos dépouilles. Le char se meut et après lui cette tourbe impatiente de voir l’incendie des titres représentant près de 80 000 livres, sans compter ceux de Serraux et de Cottens \footnote{\emph{Les Bourla-Papey. Gazette de Lausanne 1857.}}. »
 \end{quoteblock}

\section[Faits épars]{Faits épars}
\noindent On a pu voir dans les pages précédentes les principaux événements qui se passèrent du 4 au 12 mai dans les divers districts où l’insurrection se répandit. Avant de montrer quelle fut la conduite postérieure des paysans et des autorités légales de l’époque, il ne sera pas inutile d’indiquer les quelques faits épars qui se passèrent dans d’autres parties du Canton où le mouvement fut beaucoup moins général.\par
Dans la vallée de la Broie, les populations restèrent en général tranquilles. Les gens de Marnand, cependant, sans vouloir détruire les archives du citoyen Thormann, propriétaire du château, cherchèrent toutefois à s’en assurer la possession. Pendant la nuit du 10 au 11 mai, une troupe d’une cinquantaine d’hommes ayant au milieu d’eux la Municipalité de Marnand, alla surprendre le seigneur dans son lit ; ils enlevèrent ses titres et les emportèrent. A la fin du mois de juin, ils envoyèrent auprès du citoyen Thormann un délégué pour lui exprimer leur repentir et lui rendre ses archives.\par
Le 12 mai, à deux heures et demie du matin, le Receveur de Lucens fut réveillé par huit hommes qui demandèrent ses titres. N’ayant rien obtenu, ils revinrent l’instant d’après avec beaucoup d’autres personnes et pénétrèrent jusqu’à la porte du bureau. Le citoyen Briod avait eu le temps de se barricader et les assaillants, malgré beaucoup de coups de crosse et de baïonnettes, ne parvinrent pas à rejoindre ce fonctionnaire. Un membre de la Municipalité arriva au bout de cinq minutes.\par

\begin{quoteblock}
\noindent « Ils partirent sur le champ, dit le Receveur, et allèrent former un groupe à cinquante pas de la maison, puis se retirèrent tout à fait en marquant leur colère. »\end{quoteblock}

\noindent Les habitants du village de Chavannes sur Moudon devaient quelques droits seigneuriaux au curé du village voisin, Morlens, au Canton de Fribourg. Ils se rendirent auprès de lui pendant la nuit, sous la direction de Pierre-Daniel Veyre, de Jean Dutoit et de Samuel Crausaz. A l’aube, la cure de Morlens fut entourée. \emph{« On donnait des bourrades à la porte comme des coups de tonnerre »}, raconte le prêtre de l’endroit, le citoyen Gremaud. Il sortit pour protéger ses gens : \emph{« Voilà six baïonnettes à mon estomach. »} Il donna ses titres, mais les assaillants l’accusèrent de ne pas avoir tout donné et le menacèrent. \emph{« La colère me prit\emph{, dit-il}, et je leur ai dit qu’ils étaient de la canaille d’exiger de moi ce que je n’avais pas »}. Ils se calmèrent et s’en allèrent \emph{« laissant le curé tout malade »}. \emph{« Je me croyais près de ma fin »}, écrivait quelques jours plus tard ce brave ecclésiastique.\par
Le district d’Oron, dans lequel il y eut quelques agitateurs ; qui envoya un contingent à l’armée de Reymond et en plaça un autre en observation sur la route du Jorat, ne brûla cependant pas les archives du bailliage pendant l’insurrection. Les documents se trouvaient dans le grenier du château, « la Grenatterie » ; ils furent remis plus tard aux communes intéressées et brûlés solennellement en 1803, après mûre délibération, au lieu dit « Le Bosson de la Croix », près d’Oron-la-ville \footnote{Pasche. \emph{La contrée d’Oron}, 579.}.\par
A Aigle, quelques personnes auraient désiré participer à l’insurrection et détruire les titres du bailliage. Elles en furent empêchées par le sous-préfet de Loës qui se servit pour cela des troupes françaises cantonnées dans son district. Comme à Oron, cependant, on brûla en 1803, sur les Glariers, et après délibération, une grande charrette de livres et de papiers seigneuriaux.\par

\begin{quoteblock}
 \noindent « Un jeune garçon qui cherchait à retirer un parchemin pour en faire une couverture de livre, fut reçu par un coup de crosse et se garda bien de renouveler sa tentative \footnote{Suppl, au \emph{Dlct. hist. du Canton de Vaud} par M. Favey, Aigle.}. »
 \end{quoteblock}

\noindent La tradition veut que les \emph{Bourla-Papey} aient détruit en partie les archives de Bex, et M. Cherix, ancien Syndic, en avait entendu parler par un des assistants. Quoique cette localité fût alors très agitée par plusieurs personnes et surtout par le voisinage du général Turreau, il est probable que l’incendie des titres eut lieu comme à Aigle, en 1803.\par
Eugène Rambert raconte dans sa notice sur Montreux, qu’aux Planches, les \emph{Bourla-Papey} :\par

\begin{quoteblock}
\noindent « chauffèrent deux fois le four communal et peu s’en fallut que dans leur ardeur ils n’incendiassent le village \footnote{\emph{Montreux}, p. 113.}. »\end{quoteblock}

\noindent Lutry eut un fervent admirateur des \emph{Bourla-Papey} dans la personne du capitaine Mûller. Pendant la nuit du 8 mai, le tambour Besançon battit la caisse par son ordre. Le lendemain, il se rendit au camp et revint ensuite à Cully où il engagea beaucoup de personnes à partir. Il y battit aussi la caisse, mais sans beaucoup de succès. Lorsque Mûller eut envoyé ses grenadiers rejoindre Reymond, il fut mis aux arrêts. Il recouvra du reste bientôt sa liberté en promettant au sous-préfet, le citoyen Gay, de rester tranquille.\par
Si l’insurrection se répandit si peu dans la vallée de la Broie et dans les régions de Lavaux, Vevey et Aigle, il faut en voir la cause dans les efforts que firent les \emph{Bourla-Papey} pour obtenir l’appui de la France et même la réunion à ce pays. Tous les Vaudois voulaient la fin du régime féodal, mais les régions seules qui se soulevèrent envisageaient sans crainte l’union avec la grande république voisine. Si les chefs du mouvement n’avaient pas regardé du côté de la frontière de l’ouest, ils auraient été suivis ou soutenus par une masse beaucoup plus considérable de citoyens. Le succès relatif des intrigues de Turreau et des autres agents français fit dévier l’agitation politique de son principe et jeta la défiance dans une partie du pays.
\chapterclose


\chapteropen
\chapter[III. Le Canton de Vaud]{III. Le Canton de Vaud}\renewcommand{\leftmark}{III. Le Canton de Vaud}


\chaptercont
\section[Vainqueurs ou vaincus ?]{Vainqueurs ou vaincus ?}
\noindent Les pages qui précèdent auront fait comprendre pourquoi le citoyen Kuhn ne crut pas pouvoir, le 11 mai, se servir uniquement de la force pour arrêter le développement de l’insurrection. Il avait pu s’apercevoir bien vite, en effet, que le mouvement, loin d’être accidentel, tenait à des causes lointaines et générales, que les populations vaudoises, au sortir d’une très longue période de nullité politique et d’infériorité sociale, étaient, en immense majorité, résolues d’arriver par tous les moyens à un régime plus égalitaire. Il s’agissait d’une crise politique et surtout économique qu’il fallait absolument résoudre au plus tôt, et non d’une insurrection partielle comme on en avait vu plusieurs déjà et qu’on pouvait réprimer avec le secours de quelques troupes.\par
Au milieu de circonstances semblables, environné de personnes qui étaient ou paralysées par la peur ou animées des plus vifs ressentiments, n’ayant pour le seconder que des troupes helvétiques qui n’inspiraient aucune crainte et aucun respect, et des contingents français dont les chefs \emph{« étaient tenus à de grands ménagements »}, le citoyen Kuhn fut dans la délicate obligation de désobéir aux ordres de son gouvernement et de prendre sur lui une détermination de nature à mécontenter beaucoup de personnes et à faire planer un doute dans l’esprit de la population.\par
Lorsque, le 11 mai au soir, les représentants des \emph{Bourla-Papey} eurent avec le général Amey et le Commissaire Kuhn une entrevue à la suite de laquelle ils licencièrent leurs contingents, il ne fut malheureusement pas conclu de traité formel et écrit, et il y eut toujours un peu d’incertitude, même dans l’esprit du Petit Conseil, sur les décisions qui avaient été prises. La version des paysans ne fut pas, en effet, exactement la même que celle qui eut cours dans les sphères gouvernementales.\par
Que dirent les \emph{Bourla-Papey} ? Quelques indications intéressantes peuvent être fournies à ce sujet.\par

\begin{quoteblock}
 \noindent « Le mercredi 12, de bon matin, raconte J.-J. Cart\footnote{J.-J. Cart : \emph{De la Suisse avant la Révolution et pendant la Révolution}}, j’entends des détentes de boêtes, des cris de joie, je vois des insurgés. – Qu’est-ce donc ? – La Paix ! la Paix ! – A quelle condition ? – Pardon réciproque, les censes et les dîmes abolies sans indemnités. – Je m’enquiers plus particulièrement ; c’est une amnistie ; les droits féodaux doivent être remis aux autorités cantonales et celles-ci feront droit aux réclamants ».\par
 « Le 12, revenant du camp, ils dirent que les droits féodaux étaient abolis, dit-on dans un rapport relatif au village d’Echandens. On se réunit et, après quelques verres, les exaltés se levèrent, insultèrent l’agent (national), lui donnèrent des coups de poing ».
 \end{quoteblock}

\noindent Ces lignes nous montrent quel respect avaient les campagnards pour les représentants du gouvernement. Ce sentiment se retrouvait du reste partout dans les régions insurgées. Les gens de Penthaz prétendaient \emph{« qu’il n’y avait plus de magistrats »} et lorsque l’agent national de la localité eut affiché la lettre du Ministre Verninac au Petit Conseil, \emph{« ils conclurent que c’était un écrit fabriqué »} et ils l’arrachèrent.\par

\begin{quoteblock}
 \noindent « Le jour où le camp a été levé, – raconte cet agent, le citoyen Descombes, – ils sont rentrés en triomphe et ont rapporté qu’ils avaient eu une victoire complète, que tout leur avait été accordé, que les communes, par députations, iraient à Lausanne pour retirer dans les archives nationales les droits qui les concernaient. Ils revinrent tout couverts de verdure en place de laurier. Le soir, ils s’assemblèrent à la maison de commune, avertis par le son de la cloche, pour faire une orgie de boisson et ils tirèrent de petites pièces pour boire des santés. »
 \end{quoteblock}

\noindent Deux jours plus tard, le citoyen Grivel, sous-préfet d’Aubonne, écrivait de son côté à Polier les lignes suivantes qui résument très bien la situation à ce moment-là :\par

\begin{quoteblock}
 \noindent « Le peuple est rentré dans ses foyers avec la persuasion que dans peu de jours l’abolition des droits féodaux sera décrétée ; il est dans cette attente. Si le gouvernement fait droit à ce vœu, je crois que la tranquillité publique est assurée ».
 \end{quoteblock}

\noindent Les \emph{Bourla-Papey} et leurs amis croyaient donc être arrivés au résultat qu’ils avaient tant désiré. Quelques jours encore et on leur confirmerait l’exécution des promesses faites par le citoyen Kuhn.\par
Celui-ci était-il entièrement d’accord avec cette manière de voir ? Il ne le semble pas, d’après ses rapports au gouvernement, dans lesquels il n’est pas du tout question de l’abolition immédiate des droits féodaux. Et, du reste, aurait-il voulu admettre cette revendication populaire, qu’il n’aurait pas eu les pouvoirs nécessaires pour le faire. Cette question relevait de la législation centrale et l’Assemblée des Notables était précisément occupée à la discuter à ce moment-là en préparant une nouvelle constitution pour la République helvétique.\par

\begin{quoteblock}
 \noindent « Il n’a pas été question de proclamer une amnistie générale, écrivait d’autre part le citoyen Kuhn au Petit Conseil \footnote{Lettre du 13 mai.}, mais je vous demande de me laisser Reymond et Marcel et de ne pas vous inquiéter des bruits de capitulation. \emph{Il n’en a été conclu aucune, ni oralement ni par écrit}. »
 \end{quoteblock}

\noindent Kuhn avait tout fait pour obtenir le licenciement des paysans. Il avait pris sur lui de pardonner à Reymond et Marcel, mais il espérait, en revanche, que les vrais auteurs du mouvement seraient atteints :\par

\begin{quoteblock}
 \noindent « Je suis entrain, disait-il, de recueillir, avec l’aide du Préfet, les charges (indices) nécessaires contre ces hommes qui ont conduit le peuple d’un crime à l’autre. J’espère qu’ils n’échapperont pas à une punition méritée. Ils doivent être arrêtés dès que les circonstances le permettront. »
 \end{quoteblock}

\noindent L’amnistie accordée par le Commissaire du gouvernement était-elle du moins complète pour ceux qui avaient été entraînés dans le mouvement ? pour les populations qui avaient écouté les conseils des agitateurs politiques ? Non, car il préparait les moyens de maintenir la tranquillité par la force.\par

\begin{quoteblock}
 \noindent « Les troupes qui vont arriver, dit-il dans sa lettre du 12 mai au Petit Conseil, ne viendront pas à Lausanne, mais seront dispersées dans le canton de manière à dominer le pays. Aussitôt que cela sera fait, on devra ordonner le désarmement et l’exécuter immédiatement. »
 \end{quoteblock}

\noindent Un peuple que l’on désarme et que l’on fait surveiller par une force armée étrangère n’est pas un peuple auquel on pardonne sa conduite antérieure.\par
Quelque profitable qu’ait pu être la conduite du citoyen Kuhn pour la tranquillité du pays, elle n’était donc pas de nature à mettre fin au conflit. Le peuple crut avoir tout obtenu. En réalité, comme on vient de le voir, sa victoire n’était pas si complète et bientôt, des deux côtés, on allait ressentir les conséquences fâcheuses de cette équivoque.
\section[L’amnistie et la Suisse]{L’amnistie et la Suisse}
\noindent Il faut dire maintenant quelques mots d’une autre cause importante de la conduite du Commissaire.\par
On a vu que les « patriotes » du Pays de Vaud avaient eu, dans le courant de l’hiver, des entrevues avec des amis politiques d’autres cantons. Ils s’étaient rencontrés, parait-il, à Payerne, dans une conférence secrète, et ils avaient pris des mesures communes pour qu’une insurrection éclatât au printemps dans toutes les contrées où il existait encore des redevances féodales. Les circonstances avaient sans doute changé dès lors par suite de la chute du gouvernement d’Aloïs Reding et, dans plusieurs cantons, on avait renoncé plus ou moins complètement au projet formé à Payerne. Il n’en subsistait pas moins une disposition soit à soutenir l’insurrection du Léman, soit à l’imiter, et lorsque celle-ci eut remporté quelques succès par l’incendie d’un grand nombre d’archives, on put voir le mouvement, localisé d’abord dans quelques districts, se répandre de proche en proche avec une grande rapidité.\par
Kuhn fut à même de connaître cela et il n’est pas étonnant qu’il ait cherché, par des moyens qui ne rentraient pas dans le cadre de ses instructions, à arrêter le développement de l’incendie qui menaçait de tout envahir.\par

\begin{quoteblock}
 \noindent « Je puis vous assurer, écrivait-il au Petit Conseil, qu’on vient de renverser un plan combiné ne visant à rien moins qu’à mettre en rébellion tous les cantons qui paient encore des droits féodaux. »
 \end{quoteblock}

\noindent Kuhn était donc persuadé qu’il avait obtenu un grand succès en traitant avec les insurgés.\par

\begin{quoteblock}
 \noindent « Je vous prie de retenir votre sentiment d’honneur, disait-il à son gouvernement, et de vous abstenir de toute appréciation jusqu’à ce que je vous aie fait connaître l’affaire sous son vrai jour. J’ai le sentiment d’avoir rendu à ma patrie un grand service, ce qui me permet de ne pas craindre pour le moment l’impression que pourraient produire des jugements erronés ; dans ce moment de tels jugements ne troublent nullement ma conscience, et j’espère, malgré l’impression fâcheuse qu’auraient pu provoquer des appréciations inexactes, me tenir bientôt aussi assuré vis-à-vis de vous. »
 \end{quoteblock}

\noindent Quant au général Amey, qui avait contribué pour une grande part à provoquer un compromis, il s’aperçut bientôt, sans doute, de l’équivoque qui en résultait et, se retranchant derrière le Commissaire du gouvernement pour tout ce qui tenait au côté politique de la question, il fit publier la note suivante :\par

\begin{quoteblock}
 \noindent « Le général Amey a sommé les chefs des communes à la tête des rassemblements armés, campés au-dessus du pont de la Venoge, près Morges, à se dissoudre, et dans le cas contraire qu’il emploierait la force armée pour y parvenir, tel que le lui ordonne le général Montrichard ; les chefs ont promis de s’y conformer et, ce matin, (12 mai) de très bonne heure, tout le monde s’est retiré paisiblement dans sa commune ».
 \end{quoteblock}

\noindent Le général Montrichard suivit l’exemple de son subordonné et, le 13.mai, fit publier à son tour une note très nette :\par

\begin{quoteblock}
 \noindent « Je n’entreprendrai pas de réfuter les absurdités que les ennemis de toute espèce d’ordre débitent à cette occasion, disait-il ; ils savent bien que le gouvernement helvétique n’écoute pas des propositions extravagantes et qu’un général français ne capitule pas avec des rebelles ».
 \end{quoteblock}

\noindent Comme on le voit, il y avait loin des appréciations de Kuhn et des généraux français à celles des campagnards. Le Petit Conseil, de son côté, fut loin d’être rassuré sur l’issue définitive de cette affaire et il craignit que la mansuétude de son Commissaire ne provoquât précisément cette insurrection générale que l’on avait voulu conjurer. Divers incidents vinrent d’ailleurs légitimer ces craintes pendant quelques jours.\par

\begin{quoteblock}
 \noindent « Par une fatalité que nous ne pouvons nous expliquer, écrivait le Petit Conseil au citoyen Kuhn \footnote{Lettre du 15 mai qui arriva à Lausanne après le départ du Commissaire.}, le bruit d’une amnistie générale promise par vous aux insurgés s’est répandu comme l’éclair dans le canton de Fribourg, dans celui de Berne et sans doute aussi dans tous les autres cantons de la République. Dans divers lieu du canton de Berne, nous apprenons qu’il s’est tenu hier des conférences et qu’il y a eu des rassemblements ensuite de cette même nouvelle. Le Préfet de Zurich nous écrit qu’il n’y a qu’un exemple de sévérité donné dans le Pays de Vaud qui puisse éviter à son canton les mêmes scènes. En un mot nous avons devant les yeux la perspective presque certaine d’un embrasement général si des mesures sévères à l’égard des chefs de l’insurrection dans le Canton de Vaud ne viennent pas faire voir à ceux qui seraient tentés de suivre leur exemple, que la punition est inévitable ».
 \end{quoteblock}

\noindent Les faits semblèrent vouloir un instant donner raison, dans le canton de Fribourg, aux craintes du Petit Conseil. Les populations du Vully possédaient au milieu d’elles des agitateurs nombreux en relations avec ceux du Léman et suivaient les événements avec attention. Le district de la Gruyère apprit dès le 14 mai que les insurgés vaudois venaient d’imposer leurs conditions au représentant du gouvernement et cette nouvelle ne tarda pas à échauffer les esprits. A Estavayer, le sous-préfet, le citoyen Endrion, fut averti le même jour au matin que les habitants du bailliage de Vuissens, armés de fusils, de hallebardes et de bâtons se présentaient à l’entrée de la ville. Son intervention resta inutile.\par

\begin{quoteblock}
 \noindent « Étant obligé de céder à la force, dit-il, je les ai invités à m’envoyer des députés avec lesquels j’ai eu des pourparlers, mais la troupe a désavoué ce que les commis avaient arrangé avec moi. Je me suis transporté chez le receveur qui a remis les titres du ci-devant bailliage de Vuissens et Cheyres. Les communes insurgées ont promis de ne point brûler les titres \footnote{Lettre du 14 mai au Préfet national de Fribourg. Les insurgés se repentirent un peu plus tard et le 30 mai, vinrent « honteux de leur conduite rapporter les titres intacts ». \emph{Journal helvétique} du 1\textsuperscript{er} juin.}. »
 \end{quoteblock}

\noindent Quant au Préfet national, le citoyen d’Église, il ne cacha pas ses craintes pour la tranquillité de son canton. Sa situation était du reste rendue très pénible par le fait que la clef de la ville de Fribourg se trouvait entre les mains du commandant français de la place qui s’en servait selon son bon plaisir.\par

\begin{quoteblock}
 \noindent « Le représentant du pouvoir exécutif sous la responsabilité duquel est mise la tranquillité de tout le canton, disait-il, ne peut pas disposer de l’entrée ou de la sortie de la ville dans laquelle il réside, mais il faut qu’il s’adresse à un étranger qui ne dépend pas de lui. Cet étranger peut lui en refuser l’entrée, ce qui m’est arrivé personnellement \footnote{Lettre du Préfet d’Église au Petit Conseil, 12 mai.}. »
 \end{quoteblock}

\noindent Le citoyen d’Église fut encore plus mécontent lorsque quelques troupes françaises furent envoyées dans le canton et placées dans différentes localités sans qu’on l’ait averti ou consulté. C’est ainsi que deux compagnies se rendirent à Gruyères qui était fort tranquille tandis qu’il ne pouvait disposer d’un contingent semblable pour occuper le district d’Estavayer.\par
Une proclamation annonça enfin aux populations que les nouvelles répandues au milieu d’elles au sujet de l’issue de l’insurrection des \emph{Bourla-Papey} étaient complètement fausses. L’effet en fut aussi heureux que rapide et le citoyen d’Église put bientôt espérer que la tranquillité publique ne serait pas troublée davantage dans son canton.\par
En face des dangers que courait la paix intérieure, le Petit Conseil ne pouvait donc que demander la sévérité la plus grande. Le 15 mai, il ordonnait encore au citoyen Kuhn d’arrêter sans retard Reymond, Marcel et les autres chefs de bandes, de faire désarmer les insurgés et de mettre les frais d’entretien des troupes à la charge des districts qui avaient participé au mouvement.\par

\begin{quoteblock}
 \noindent « Qui pourrait croire à la justice du gouvernement à l’égard des chefs cachés, écrivait-il au Commissaire, si on le voit, dans un canton où tout a été dans le désordre, laisser aller les chefs de 5000 hommes armés et dire qu’il faut rechercher d’au· très coupables ? Une telle conduite exciterait la risée générale. »
 \end{quoteblock}

\noindent C’était là certainement une manière de voir bien naturelle, mais quand la lettre du Petit Conseil arriva à Lausanne, le citoyen Kuhn ne s’y trouvait déjà plus et elle resta par conséquent sans effet.
\section[De Kuhn à Lanther]{De Kuhn à Lanther}
\noindent Kuhn quitta Lausanne le 13 mai, laissant le pays dans un malaise politique et social des plus graves. Le Préfet national confirma en effet le même jour à son gouvernement que les campagnards continuaient à croire que leurs chefs avaient fait la loi aux généraux et possédaient l’amnistie complète avec l’abolition des dîmes et des censes. Il était en conséquence aisé de prévoir que s’ils n’entraient pas bientôt en possession de ces deux avantages, des désordres se produiraient certainement de nouveau. D’autre part, avant de rentrer à Berne, le citoyen Kuhn avait placé :\par

\begin{quoteblock}
\noindent « les propriétés nationales ou publiques, celles des fonctionnaires publics restés fidèles à leur devoir, celles de tous les citoyens paisibles… sous la responsabilité des communes dans lesquelles ils demeuraient. »\end{quoteblock}

\noindent S’il se commettait des désordres dans une localité, celle-ci serait aussitôt mise en état de siège et les meneurs seraient arrêtés et conduits à Lausanne pour y être emprisonnés. Cette décision du Commissaire provoqua sans doute quelque crainte dans le pays, d’autant plus que ce dernier se voyait occupé par de nouvelles troupes françaises, mais elle augmenta encore le mécontentement et le mépris des campagnards pour le gouvernement et ses représentants.\par
Ces derniers, aussi bien que les citoyens paisibles, n’eurent bientôt d’autre souci que celui de se faire oublier autant que possible, de vivre à l’écart, de ne se mêler en rien des affaires publiques. Des hommes qui avaient, au premier moment, montré le plus de bonne volonté pour le maintien de l’ordre, refusaient maintenant leurs services ; d’autres, qui étaient restés tranquilles spectateurs des événements, en étaient réduits à se cacher ou à quitter provisoirement leur domicile pendant que les \emph{Bourla-Papey} célébraient leur victoire. A Lausanne, le citoyen Chastellain, que Polier voulait placer à la tête de la garde bourgeoise, déclarait à ce magistrat qu’il pouvait consentir à être un simple soldat, mais non un chef. Dans les communes de la région occidentale du Canton, les autorités locales donnaient une rétribution à ceux qui avaient marché ; ailleurs, on infligeait des amendes aux citoyens qui n’avaient pas voulu partir.\par

\begin{quoteblock}
 \noindent « Dans le district de Cossonay, nombre de citoyens, particulièrement des Municipaux… ont été obligés de fuir leurs villages ou de se cacher, écrivait Polier. Dans quelques communes, on a sonné l’appel des citoyens pour voter la réunion à la France \footnote{Polier au Petit Conseil, 15 mai.}. »
 \end{quoteblock}

\noindent Le général Séras, commandant français de la place de Genève, avait enfin obtenu de son supérieur, le général Molitor, l’ordre d’entrer dans le Pays de Vaud avec des forces nombreuses. Mille cinq cents hommes vinrent l’occuper à partir du 14 mai et leur chef envoya des détachements importants dans les districts et les principales localités ayant participé à l’insurrection. Le général Séras lui-même vint habiter Lausanne. Homme énergique et ferme, plus disposé que son collègue Amey à employer la force pour maintenir l’ordre public, il vécut en bons termes avec les représentants du gouvernement helvétique, réprouva la conduite de son collègue Turreau et prit une part essentielle à la pacification du pays. Ses troupes étant sur pied de guerre, les populations devaient leur fournir un supplément de solde de deux sous par homme et les officiers devaient être nourris. Cette charge financière nouvelle, venant s’ajouter à beaucoup d’autres, donna lieu à bien des difficultés et des froissements, et il fallut tout le tact et toute la patience du général Séras pour que l’on pût éviter des conflits locaux d’une certaine gravité.\par
Pour comble de malheur, le canton du Léman, déjà si éprouvé par les événements et la détresse financière, fut atteint par un nouveau fléau. La période des « saints de glace » fut tout particulièrement critique en 1802. Dès le 14 mai, la température baissa avec une rapidité inquiétante. La neige apparut jusque près des rivages du Léman et le gel détruisit dès le 15 une grande partie des espérances des cultivateurs. Ce qui échappa au mal à ce moment fut pour la plus grande partie anéanti quelques jours plus tard.\par
La Chambre administrative se vit dans l’obligation d’exposer sa détresse financière au gouvernement qui faisait appel à sa bonne volonté.\par

\begin{quoteblock}
 \noindent « Nous ne pouvons pas même obtenir la rentrée du solde de la contribution du deux pour mille ordonnée par l’arrêté du 6 janvier, disait-elle. La meilleure partie du canton vient d’être abymée par une gelée qui réduit une foule de familles à la mendicité. Le canton se trouve dans la plus triste position si on n’allège pas ses charges. Nous sommes surpris et profondément affligés que les généraux français exigent dans un pays allié une haute paye que nous ne pouvons considérer que comme une contribution militaire. »
 \end{quoteblock}

\noindent Les troupes ne demandaient cependant que ce qu’elles estimaient devoir leur revenir et il fallut bien que la Chambre administrative, malgré sa détresse, s’exécutât dans la mesure du possible. Les frais de l’occupation militaire devaient, du reste, être supportés entièrement par les districts insurgés.\par
Tous ces faits n’étaient pas de nature à mettre fin à l’agitation qui continuait à régner dans le pays et l’exécution des promesses vraies ou supposées qui avaient été faites pouvait seule ramener le calme.\par

\begin{quoteblock}
 \noindent « Tous les amis du gouvernement, de la paix intérieure et du bon ordre désirent avec ardeur qu’il soit pris les mesures les plus promptes pour effectuer la liquidation des droits féodaux de manière à écarter à jamais cette pomme de discorde \footnote{Préfet Polier au Petit Conseil, 15 mai.}. »
 \end{quoteblock}

\noindent Deux jours après avoir adressé au Petit Conseil les lignes qui précèdent, Polier lui mandait que les chefs de l’insurrection, à Morges et ailleurs, étaient grandement agités et qu’une nouvelle insurrection était à craindre si on ne satisfaisait pas aux exigences de la population. Le général Séras, de son côté, exprimait les mêmes idées, attendu que les campagnards commençaient à croire qu’on les avait trompés. Il demandait que le gouvernement publiât une proclamation capable de ramener le calme, mais ce dernier préféra attendre le moment où il pourrait adopter « le langage de la clémence », malgré les rapports précis de ceux qui avaient la responsabilité du maintien de l’ordre.\par

\begin{quoteblock}
 \noindent « Le peuple,… ballotté entre les déclarations de ses chefs sur son prétendu triomphe et celles du Ministre plénipotentiaire, du général Montrichard et de l’arrêté du citoyen Kuhn du 15 courant, travaillé d’ailleurs par les meneurs secrets qui courent les communes… paraît prêt à se porter aux extrémités plus violentes que les dernières et il n’y a pas de temps à perdre pour l’éclairer, d’autant plus que la retraite d’une grande partie des troupes françaises et celle des troupes helvétiques accrédite l’opinion répandue que le peuple n’a rien de sérieux à craindre de la part des Français \footnote{Préfet Polier au Petit Conseil, 19 mai.}  ».
 \end{quoteblock}

\noindent Le 20 mai, jour de foire à Cossonay, l’animation fut très grande dans cette localité ; le sous-préfet d’Orbe signala à la même époque des conciliabules qui avaient lieu entre Claude Mandrot, Dautun, Duchat, etc.\par
La présence de Louis Reymond et de son adjudant Marcel dans le Canton était un nouveau motif de crainte à cause de l’influence que le premier exerçait sur les populations. Vêtu de son uniforme d’officier de recrutement, il continuait à aller de lieu en lieu et à encourager les \emph{Bourla-Papey}, au grand scandale des représentants du gouvernement et des généraux français. L’ordre d’arrestation lancé contre lui le 15 mai par le Petit Conseil lui était cependant connu et le général Séras lui conseillait de se rendre à Genève, mais il trouvait encore le moyen de rester et de se montrer sur différents points du pays et même à Lausanne. Le 20 mai, Polier écrivit encore à son sujet les lignes suivantes :\par

\begin{quoteblock}
 \noindent « Le général Séras le croyait depuis deux jours de l’autre côté du lac où il lui avait dit qu’il se rendait. Sur quoi il est de mon devoir d’exposer au Petit Conseil que cet homme est aussi dangereux de l’autre côté du lac que de celui-ci, puisqu’il peut le passer dans deux heures pour se mettre à la tête d’une insurrection. »
 \end{quoteblock}

\noindent Les représentants du gouvernement ne pouvaient donc que demander l’arrestation du capitaine recruteur. Le Petit Conseil maintenait que cela n’était plus possible.\par

\begin{quoteblock}
 \noindent « Vous savez ce qui s’est passé, disait-il au Préfet, entre le Commissaire, le général Amey et les nommés Reymond et Marcel. La promesse donnée à ces deux individus qu’on solliciterait en leur faveur la clémence du gouvernement les engagea à licencier leur bande. Le Petit Conseil attache un trop grand prix à tout ce qui peut tenir au respect de la foi publique ; il a trop de motifs de croire à la sagesse du citoyen Kuhn pour ne pas désirer que ces individus, par un prompt éloignement du territoire de la République, aient évité au gouvernement la nécessité de sévir contre eux. Cependant, il le ferait s’ils continuaient à insulter le gouvernement en paraissant encore dans le chef-lieu ou dans d’autres communes du Canton. S’ils sont encore au pays, vous leur ferez savoir sous main qu’ils aient à disparaître sur le champ et, à défaut de quoi vous devrez vous saisir de leurs personnes \footnote{Petit Conseil à Polier, lettre du 20 mai.}  ».
 \end{quoteblock}

\noindent Par une contradiction que l’on retrouve dans un grand nombre d’actes de cette époque-là, le même Petit Conseil déclarait ne se faire aucun scrupule d’ordonner l’arrestation des chefs des autres troupes \emph{« et en particulier de celles d’Yverdon. – Ce sont des brigands »}, disait-il.\par
Au milieu des dangers du moment et de l’excitation qui ne faisait qu’augmenter dans les campagnes, il était nécessaire qu’il y eût un Commissaire dans le Canton. Le général Séras, qui devait se concerter avec ce fonctionnaire pour toutes les mesures à prendre, était étonné de n’en avoir aucune nouvelle. Le citoyen Kuhn, qui connaissait bien les circonstances du canton, avait d’autre part des raisons sérieuses de ne pas vouloir retourner dans le Pays de Vaud. Le Petit Conseil se décida enfin, sur les instances de Polier et du général Séras, à lui donner un successeur dans la personne du citoyen Lanther, de Fribourg, ancien ministre de la guerre, en lui adjoignant comme secrétaire Frédéric May, qui remplissait déjà ces fonctions depuis quelques jours.
\section[Anarchie]{Anarchie}
\noindent Un des premiers actes du nouveau Commissaire du gouvernement fut l’interdiction de \emph{« tous les tirages, exercices militaires et port d’armes »} sous peine de 40 francs d’amende ; les districts de l’est étaient seuls exceptés. Il publia aussi une proclamation destinée, si possible, à ramener les citoyens à des dispositions plus pacifiques et à leur recommander la Constitution que l’Assemblée des Notables venait de terminer et de publier. Ce nouvel acte constitutionnel renfermait, sur la question des dîmes et censes, deux principes qui excitaient la défiance des campagnards. Le premier proclamait ces redevances rachetables et le second prévoyait que le mode de rachat serait déterminé avant le mois de janvier 1803. La liquidation de cette question capitale pour la tranquillité publique était laissée aux autorités que chaque canton allait se donner librement.\par
Ces décisions laissèrent quelque doute dans l’esprit d’un très grand nombre de citoyens et les explications données par les trois députés vaudois à l’Assemblée des Notables, Auguste Pidou, Dan. Alex. Chavannes et Henri Carrard, ne parvinrent pas à le dissiper entièrement.\par

\begin{quoteblock}
 \noindent  « La certitude que les autorités cantonales s’occuperont du rachat ne tranquillise pas le peuple exaspéré, écrivait Polier \footnote{Polier au Petit Conseil, 2 juin.}. Les démagogues lui disent que ces autorités seront composées de seigneurs qui fixeront un taux exorbitant. J’ai cru avoir persuadé quelques-uns, mais d’autres ont des vues cachées… Je me suis vu forcé de m’ériger pour ainsi dire en professeur. Mes disciples sont souvent nombreux, leurs objections embarrassantes et mes peines infinies… »
 \end{quoteblock}

\noindent Le citoyen Kuhn et plus encore le Petit Conseil avaient demandé que le désarmement des districts insurgés fût effectué aussitôt que possible. Cette opération ne pouvait avoir lieu qu’avec le secours des troupes françaises et le général Montrichard la déconseilla bientôt. Il estimait que les insurgés n’ayant pas été invités à mettre bas les armes avant la dissolution de la troupe de Louis Reymond, il était trop tard maintenant pour les leur demander militairement. » Cette mesure exaspère les esprits, disait-il \footnote{Montrichard au Petit Conseil 3 prairial au 10 (23 mai).}, réveille les passions et les haines et fait remporter un avantage à un parti sur un autre, et, dans ce moment, le gouvernement helvétique veut les réunir tous ». Le Petit Conseil dut en conséquence ordonner à son Commissaire d’abandonner toute disposition à ce sujet.\par
Les perturbateurs surent profiter admirablement de ces hésitations dans lesquelles ils voyaient avec plaisir la main de la France, pour tenir sur le qui-vive le gouvernement et ses agents et les amener à exécuter au plus tôt les promesses faites dans la nuit du 11 au 12 mai par Amey et Kuhn. Ils auraient même sans doute passé à l’exécution de leurs menaces s’ils n’avaient pas connu la ferme intention du général Séras de réprimer avec rigueur toute nouvelle tentative de soulèvement. La hardiesse de leurs projets augmentait tous les jours, et si quelques détachements de troupes d’occupation étaient rappelés ailleurs, on apprenait aussitôt que les paysans étaient prêts à prendre les armes. Les districts de Morges, d’Aubonne, de Cossonay, d’Orbe et d’Yverdon, la région de La Sarraz, surtout, étaient en perpétuelle agitation. Le 28 mai, on signalait des gens armés en divers lieux et on envoyait un nouveau détachement de troupes françaises à Orny. Le dimanche 30 mai, on avait entendu à Rances\par

\begin{quoteblock}
 \noindent « les propos les plus atroces ». « Leur but est Lausanne, écrivait Thomasset d’Orbe, mais ils se proposent d’égorger avant que d’y arriver ». « Pendant tout le jour les cabarets étaient pleins de monde, annonçait Lanther, et l’on voyait souvent passer des courriers à cheval qui allaient d’une commune à l’autre. On entend les injures les plus grossières et les calomnies les plus infâmes contre le gouvernement et tous les fonctionnaires. On menace de reprendre les armes, de piller et de brûler toutes les maisons des seigneurs et des partisans du gouvernement dès que les troupes françaises seront parties. Par contre, les officiers et les soldats français sont traités avec tous les égards possibles, on loue leur nation et leur gouvernement et l’on parle ouvertement de se réunir à la France. »
 \end{quoteblock}

\noindent Le sous-préfet de Cossonay annonçait le 6 juin, jour de la Pentecôte, que des rassemblements avaient lieu dans diverses localités:\par

\begin{quoteblock}
 \noindent « On fait force emplettes de poudre et de plomb et on sait que tous les montagnards et surtout du côté de Vallorbes se préparent à descendre… »\par
 « On doit forcer tout ce qui peut marcher à partir, mandait le sous-préfet de Morges, et un individu a dit il y a cinq ou six jours : Tuons, pillons, massacrons et n’épargnons personne, vaincre ou mourir ! »
 \end{quoteblock}

\noindent Il annonçait aussi qu’une grande assemblée devait avoir lieu à Montricher et que Reymond s’y rendrait probablement pour maintenir le zèle des campagnards.\par

\begin{quoteblock}
\noindent « Il circule des munitions et des armes, écrivait Thomasset d’Orbe, le 12 juin. Des individus influents se rencontrent aux foires et aux marchés d’Yverdon ; d’autres du côté de Lausanne. »\end{quoteblock}

\noindent Le 6 juin, une grande assemblée de députés des communes eut lieu à Grancy.\par

\begin{quoteblock}
\noindent « Tous les rapports de hier et d’aujourd’hui, écrivait Lanthei le lendemain, annoncent pour ces jours prochains une nouvelle prise d’armes dans le district de Cossonay. »\end{quoteblock}

\noindent D’autres disaient que l’Assemblée de Grancy avait décidé l’envoi d’une députation à Berne. Le huit juin, enfin, le Commissaire demandait au Petit Conseil l’établissement d’un petit état de siège dans les districts insurgés.\par

\begin{quoteblock}
\noindent « L’audace du peuple n’a plus de bornes, dit Séras, toutes les autorités et le gouvernement même sont vilipendés. »\end{quoteblock}

\noindent Le petit Conseil donna dès le lendemain l’autorisation nécessaire.\par
Les agitateurs, craignant le ressentiment du Commissaire et du Petit Conseil, cherchaient de plus en plus à se mettre sous la protection de la France en faisant signer des adresses pour la réunion du canton à ce pays. Henri Monod, qui habitait alors à Paris, mettait sans doute ses concitoyens en garde contre les dangers de cette décision, mais il le faisait avec des réticences qui étaient seules retenues par beaucoup.\par

\begin{quoteblock}
 \noindent « Dites bien à ceux… qui voudraient se réunir qu’ils ne connaissent guère ce grand théâtre, écrivait-il le 29 mai, à son ami Jaïn à Morges. Veulent-ils pour la plus petite misère être obligés de courir ici, y languir plusieurs mois, y dépenser un argent effroyable, et s’ils n’en ont pas à donner pour obtenir ce qu’ils désirent, s’en retourner comme ils sont venus ? Prêchez donc à chacun patience et espérance… Mais enfin si après avoir épuisé cette provision de patience et d’espérance, nous n’obtenons pas la liberté après laquelle nous courons et qui mérite bien quelques années de sacrifices, le pis aller sera d’en venir à cette réunion… »
 \end{quoteblock}

\noindent Chez les campagnards, la patience était à bout et l’espérance avait peu à peu disparu. Dès le premier juin, le général Séras annonçait qu’une députation de cinq personnes allait partir pour Paris avec des adresses de réunion et que plusieurs de ces députés étaient de Morges. Les colporteurs de ces adresses parcouraient le pays, convoquaient les citoyens dans les cabarets qui, ces jours-là, ne « désemplissaient pas ». Le Commissaire reçut l’ordre de chercher à connaître les députés qui allaient se rendre à Paris et de mettre la main sur eux. Si cela était impossible, il devait demander au Ministre suisse en France, Stapfer, de les faire arrêter dans ce pays.\par
Le gouvernement insistait pour que ses agents montrassent la plus grande fermeté. « Le moment des ménagements est passé, écrivait-il le 8 juin à\par
Lanther. Les palliatifs sont sans vertu. » Cet ordre du Petit Conseil était sans doute excellent si Ton admettait son point de vue politique, mais les moyens étaient-ils suffisants dans le Pays de Vaud pour l’appliquer ? Dans l’état d’anarchie où se trouvaient les esprits et avec le malentendu qui grandissait toujours entre le peuple et ses représentants légaux, il fallait ou un pardon général, avec le rachat, ou une répression systématique avec le jugement des coupables par le moyen d’un tribunal spécial. Le Petit Conseil ne sut adopter complètement aucune de ces deux politiques et fut acculé en conséquence à une situation de plus en plus inextricable.\par
Dès la fin de mai, le Préfet national et le Commissaire demandèrent un tribunal dont les membres devraient être choisis hors du canton du Léman, puisque personne dans ce dernier ne voulait en faire partie ou même accepter un emploi quelconque sous la direction du gouvernement helvétique. Le Petit Conseil hésita pendant plusieurs semaines et laissa ainsi le temps aux citoyens de se persuader toujours davantage qu’ils pouvaient compter sur l’impunité.\par
La force armée qui se trouvait dans le canton avait encore quelque importance sans doute, quoique se composant seulement de 1000 à 1500 hommes de troupes françaises, mais ces dernières y resteraient-elles encore longtemps, et si elles devaient partir, que deviendrait l’ordre public ?\par
Dès le 6 juin, le général Séras fut, en effet, rappelé à Genève avec ses troupes par le général Molitor, et il fallut l’intervention du Ministre de France, Verninac, pour obtenir qu’il pût rester plus longtemps dans le canton.\par

\begin{quoteblock}
 \noindent « Le général Séras n’a pas 1200 combattants, écrivait Polier le 6 juin, et quoi qu’il soit rempli de bonne volonté, et de courage…, je ne puis cacher au Petit Conseil qu’il est navré et très mécontent qu’on (le gouvernement) lui ait refusé les cinquante hussards qu’il a demandés avec tant d’insistance (pour le service des courriers) ; aussi il ne faut pas moins que mes vives sollicitations et l’intérêt qu’il prend à ce malheureux canton pour suspendre l’exécution de l’ordre qu’il vient de recevoir du général Molitor \footnote{Lettre du 6 juin.}. »
 \end{quoteblock}

\noindent Si le commandant français des troupes d’occupation était si peu soutenu, la situation du Commissaire et des fonctionnaires publics était encore plus précaire.\par

\begin{quoteblock}
 \noindent « Quelle que soit la bonne volonté d’un sous préfet d’agir avec vigueur au moyen de la troupe dont il dispose aujourd’hui, il n’ose s’y livrer, écrivait Polier, dans la crainte de se trouver isolé dans quelques jours et exposé à la vengeance des anarchistes \footnote{Polier au gouvernement ; lettre du 4 juin.}. »
 \end{quoteblock}

\noindent Quant au citoyen Lanther, il fut complètement découragé lorsque, au commencement de juin, le Conseil Exécutif blâma sa patience. Il exposa sa triste situation et la nullité de ses moyens réels :\par

\begin{quoteblock}
\noindent « Ses renseignements sont toujours vagues et ils ne sont connus qu’après qu’ils sont consommés… La troupe n’est pas assez nombreuse pour occuper toutes les communes en insurrection ; les communications entre elles sont si bien prises qu’il est impossible de les arrêter, et les rapports sur les événements, lents, indirects, presque toujours tardifs et insuffisants, parce que sans argent la police secrète ne peut se faire. »\end{quoteblock}

\noindent Lanther était au milieu d’un brasier\par

\begin{quoteblock}
\noindent « entouré d’hommes passionnés qui l’attisent et n’ayant d’appui que dans une force étrangère \emph{qui a de grands ménagements à garder}. Le Conseil exécutif demandait l’arrestation des meneurs, colporteurs d’adresses, etc. « Mais qui jugera ces délits ? demandait le citoyen Lanther. »\end{quoteblock}

\noindent Il ne cessait de demander un tribunal et il ne pouvait l’obtenir.\par
Laissé ainsi sans pouvoir réel, sans force, sans appui, le Commissaire en était réduit à demander sa démission. On ne voulut pas la lui accorder.
\section[Nouvelle insurrection à Yverdon]{Nouvelle insurrection à Yverdon}
\noindent Les événements qui se passèrent à Yverdon les 12 et 13 juin montrèrent que le Préfet national et le Commissaire avaient raison de faire part de leurs craintes au Conseil Exécutif.\par
Le district d’Yverdon avait été un des derniers à prendre part à l’insurrection. Les esprits semblèrent y être plus tard d’autant plus excités, et le sous-préfet demanda avec instance à ses supérieurs de bien vouloir au plus tôt mettre fin à une situation aussi décourageante par une prompte solution de la question des dîmes.\par

\begin{quoteblock}
 \noindent  « Sous-tensions de nous ôter le poignard que nous avons dans le cœur, pour parvenir enfin à un état stable, écrivait-il le cinq juin. De tous côtés l’on voit naître la défiance ; les passions haineuses ont succédé au sourire ; le gouvernement du prince le plus barbare vaut mieux que l’anarchie où nous sommes. »
 \end{quoteblock}

\noindent Le 7 juin, Séras demandait du renfort au général Montrichard et, le lendemain, le capitaine Ferrier, commandant de place à Yverdon, annonçait qu’il fallait :\par

\begin{quoteblock}
 \noindent « s’attendre à une prise d’armes et aux malheurs qui en seraient la suite. »\par
 \emph{Il demandait un renfort,}\par
 « n’ayant que 65 hommes, et le départ… de 350 hommes de la 45\textsuperscript{me} demi-brigade ayant persuadé aux paysans qu’il n’y aurait plus de Français dans quelques jours. »\par
 \emph{Les craintes étaient, du reste, générales ; le 10, le bruit ayant couru, dans le district de Cossonay, d’une prise d’armes imminente, quelques propriétaires se hâtèrent}\par
 « de quitter leur domicile en emportant leurs effets les plus précieux \footnote{Henri Polier au Conseil Exécutif, lettre du 11 juin.}. »
 \end{quoteblock}

\noindent Le 9 juin, le citoyen Lanther, obéissant aux ordres de son gouvernement, invita le sous-préfet Doxat, à Yverdon, à faire arrêter les citoyens Jean-François Rebaud, de Rovray, Favre et Correvon, de Cuarny, qui étaient signalés comme perturbateurs influents. Ces trois hommes furent saisis le 12 au matin et emprisonnés au château d’Yverdon.\par
Cette nouvelle se répandit avec la plus grande rapidité dans la contrée et y causa une colère et une consternation d’autant plus vives que ces trois hommes allaient, disait-on, être fusillés ou transportés à Lausanne.\par
Les citoyens furent convoqués dans chaque village ; les cloches sonnèrent à Yvonand, à Chavannes, à Chêne et Pâquier, à Rovray, à la Mauguettaz, etc. Il fut décidé que l’on se rendrait aussi nombreux que possible au chef-lieu pour remettre les prisonniers en liberté. Chaque maison fournirait un homme au moins et l’on se réunirait pendant la nuit sur la grande route, à une petite distance d’Yverdon. Des signaux furent préparés pendant qu’au chef-lieu le commandant de place envoyait une patrouille en reconnaissance dans la campagne. Ces quelques hommes se virent entourés par des campagnards armés et eurent beaucoup de peine à rentrer le soir à Yverdon, ayant dû se cacher et faire de grands détours dans les forêts.\par
A Cuarny, les citoyennes Favre et Correvon, suivies de la population, s’en allèrent auprès du gouverneur du village qui conseilla de se rendre nombreux à Yverdon, afin de répondre des détenus et les faire remettre en liberté. Elles s’adressèrent aussi au citoyen Emmanuel Besson, agent national. Celui-ci se rendit aussitôt auprès du sous-préfet. Ce fonctionnaire vit arriver un peu plus tard un certain nombre d’autres citoyens du même village. Il leur répondit avec fermeté, tout en les rassurant au sujet du sort des prisonniers et en les exhortant au calme et à la paix. L’insistance des « réclamants » devint bientôt si grande que le citoyen Doxat avertit le commandant de place ; celui-ci prit aussitôt des mesures de prudence.\par
Le sous-préfet venait à peine de s’endormir qu’on vint le réveiller et lui apprendre que les paysans armés se montraient aux environs.\par
Les campagnards de la région d’Yvonand s’étaient groupés en deux troupes. La première, formée surtout des gens de la Mauguettaz et forte seulement d’une douzaine d’hommes, arriva en vue d’Yverdon à onze heures du soir. La seconde, composée d’environ deux cents citoyens, se trouva réunie à Clindy (faubourg d’Yverdon) à deux heures du matin.\par
Peu après dix heures du soir, le commandant de place, Ferrier, envoya une forte patrouille sur la route d’Yvonand, sous la direction du capitaine Linaz. Ce dernier rencontra au-dessus de Clindy, à la lisière de la forêt qui y existait alors, une douzaine de paysans qui avaient leurs armes à feu, sur lesquelles ils se couchèrent pendant que d’autres se retiraient quelques pas plus loin.\par
— Que faites-vous là ? demanda le capitaine Linaz.\par
— Nous prenons le frais, lui fut-il répondu.\par
Linaz leur fit observer que l’on ne « prend pas le frais avec ses armes » et à plusieurs kilomètres de son village. Ils annoncèrent alors leur intention réelle. Ils furent désarmés ; un seul protesta en donnant un coup de poing dans la figure du soldat français qui lui enlevait son fusil.\par
En ramenant à Yverdon les paysans qui venaient d’être pris, Linaz rencontra à l’endroit dit « les quatre marronniers », un groupe d’une dizaine de paysans qui cherchèrent à s’opposer à son passage, tout en demandant à la troupe de relâcher leurs amis. Un coup de feu fut tiré à ce moment-là sur les Français ; il ne fit de mal à personne, mais le capitaine Linaz, voyant que ses prisonniers étaient sur le point de lui échapper, tira un coup de fusil sur l’un d’entre eux qui fuyait déjà. Ce citoyen, Abram-Louis Marrel, de la Mauguettaz, eut la cuisse fracassée. Cela fit une grande impression sur tout le monde et une quinzaine d’hommes furent emprisonnés au château quelques instants plus tard.\par
Les citoyens que le capitaine Linaz venait de rencontrer étaient descendus de Cuarny à la rencontre des députés qui étaient venus auprès du sous-préfet. Ces derniers rentrèrent bientôt chez eux où ils racontèrent le sort de leurs combourgeois et les dispositions militaires du commandant de place. Les gens de Cuarny jugèrent alors inutile de se joindre à la troupe principale des campagnards de la région d’Yvonand. Quant au citoyen blessé, il fut transporté à Yverdon pour y être soigné.\par
A deux heures de matin, le bruit du tambour se fit entendre du côté de Clendy où les contingents des paysans se groupaient.\par
Le commandant Ferrier envoya dans cette direction une forte patrouille sous la direction du capitaine Bulisson, avec l’ordre de désarmer tous les paysans qu’elle rencontrerait. Elle aperçut la troupe des campagnards au bas de la forêt de Clindy.\par
— Qui vive ! cria-t-on,\par
— Nous sommes des amis, fut-il répondu.\par
— Que votre chef s’avance.\par
— Nous n’avons pas de chefs.\par
David Rebaud et Jean-Pierre Michoud s’avancent au-devant du capitaine français et annoncent leurs intentions.\par
Voyant qu’il avait trop peu de monde, le capitaine Bulisson se retire et les paysans s’engagent à ne pas dépasser Clindy.\par
Le commandant Ferrier fait battre la générale en ville à trois reprises, mais c’est à peine si vint-cinq personnes se présentent. Un courrier est expédié en hâte au général Séras, à Lausanne ; la Municipalité se rassemble pour prendre quelques mesures de précaution et le capitaine Bulisson, à la tête d’une seconde patrouille plus forte que la première, se met de nouveau en marche.\par
Pendant ce temps, les campagnards avaient occupé Clindy, mais quand ils entendirent battre la générale en ville, ils se retirèrent prudemment dans leur première position où la patrouille française les trouva rangés en bataille pour s’opposer à son passage.\par
Rebaud et Michoud s’avancent de nouveau.\par
— Que voulez-vous ? leur demande-t-on.\par
— La paix ! nous ne voulons point faire de mal ; nous voulons aller délivrer les prisonniers.\par
Le capitaine Bulisson somme les campagnards de déposer leurs armes, sinon il ordonnera de faire feu. Les paysans du premier rang déposent aussitôt leurs fusils ; les autres prennent la fuite. L’un d’entre eux fait feu, mais sans succès.\par
La patrouille rentra dans la ville avec vingt-neuf prisonniers.\par
Au matin du 13 mai, le sous-préfet Doxat fit publier une proclamation qui décida un certain nombre de citoyens à prendre les armes. A ce moment-là, les rassemblements de campagnards ne faisaient qu’augmenter autour de la ville et l’on affirmait que s’ils pouvaient y pénétrer, ils avaient l’intention de mettre à mort :\par

\begin{quoteblock}
\noindent « une douzaine de personnes et de brûler leurs maisons \footnote{Séras à Lanther ; lettre du 24 prairial (13 juin).}. »\end{quoteblock}

\noindent La situation restait donc inquiétante et les « amis de l’ordre » ne furent rassurés qu’à l’approche du général et de nombreuses troupes.\par

\begin{quoteblock}
 \noindent « Dès son arrivée, il se mit en marche pour aller attaquer les rebelles qui étaient postés sur les hauteurs aux environs de la ville \footnote{Lettre de Lanther au Petit Conseil, 14 juin.} Ce mouvement fit dissoudre cette horde de brigands qui ne put être atteinte. Le général a cependant suivi les traces des fugitifs ; tous les villages qu’il a parcourus étaient dans un état de stupeur qui représentait la tranquillité. Enfin il est entré à Yverdon où plusieurs députés vinrent l’implorer en abjurant leurs erreurs, offrirent inutilement de répondre des prisonniers si on voulait bien les remettre en liberté ou, à défaut, ne pas les conduire à Lausanne. »
 \end{quoteblock}

\noindent Cette marche militaire et la fermeté que le général Séras avait désiré quelques jours auparavant avoir une occasion de déployer, fit une impression très grande sur les populations et leur montra enfin de quelle manière serait accueillie toute nouvelle tentative de soulèvement. Cette impression fut sans doute encore augmentée lorsque l’on sut que, le 14 juin au matin, les cinquante-sept prisonniers détenus au château d’Yverdon avaient eu les mains liées derrière le dos pour être conduits à Lausanne sous l’escorte des soldats français.\par
En envoyant, le 15 juin, au Petit Conseil, son rapport sur les événements, le citoyen Lanther annonça que les communes devaient être désarmées ce même jour et qu’il était vraiment embarrassant de posséder un si grand nombre de prisonniers sans un tribunal pour les juger et même un fonctionnaire pour les interroger et remettre en liberté ceux qui n’étaient pas responsables.\par

\begin{quoteblock}
 \noindent « Tous les citoyens de ce canton, excepté le Préfet national, quelques sous-préfets et quelques autres citoyens, refusent une fonction quelconque, disait-il. »\par
 {\itshape La Chambre administrative elle-même montrait, suivant lui,}\par
 « une mauvaise volonté caractérisée. »\par
 « Ne soyez pas surpris d’apprendre dans quelques jours que les districts insurgés se sont levés en masse pour délivrer les prisonniers et que, par le retard des secours promis, ils sont arrivés à exécuter leur projet ; dans ce cas, je vous prie de me lever d’avance de toute responsabilité \footnote{Lanther au Petit Conseil, 14 juin.}. »
 \end{quoteblock}

\section[Agitation]{Agitation}
\noindent Le soulèvement qui venait d’être réprimé décida le Petit Conseil et son Commissaire à user enfin d’une sévérité plus grande à l’égard des agitateurs, à organiser un tribunal et à faire prélever dans les districts insurgés une contribution extraordinaire pour subvenir aux frais de l’occupation militaire.\par
Dès le 2 juin, le Petit Conseil enleva à Louis Reymond sa charge de capitaine dans la deuxième demi-brigade auxiliaire. Le 14 du même mois, le Commissaire ordonna le désarmement de toutes les communes qui avaient participé à la dernière insurrection. Celle d’Yverdon fut comprise dans le nombre afin de punir ses citoyens de n’avoir pas répondu à l’appel qui leur avait été adressé dans la nuit du 12 au 13. Des détachements de troupes devaient faire subir la même peine aux habitants des localités dans lesquelles il éclaterait des événements semblables. Trois jours plus tard une contribution de guerre de 36 000 francs environ fut imposée aux communes qui avaient pris part à l’insurrection du mois de mai. Si elles n’avaient pas livré la somme qu’elles devaient, au bout d’un nombre déterminé de jours, elles devraient être occupées aussitôt par des troupes nourries, rétribuées et logées par les particuliers les plus coupables. Peu après, une contribution supplémentaire d’environ 25000 francs fut encore imposée à un certain nombre de communes des mêmes districts insurgés. Le 8 juin, enfin, le Petit Conseil décréta la formation d’un tribunal spécial chargé de juger les meneurs politiques et les chefs des \emph{Bourla-Papey.}\par
Ces mesures, jointes à la conduite du général Séras à Yverdon, commencèrent enfin à ouvrir les yeux de beaucoup de personnes et de communes qui avaient cru être à l’abri de toute recherche judiciaire et de toute peine quelconque. La levée de la contribution de guerre fut surtout sensible pour les campagnards, déjà pressurés de toutes manières depuis quelque temps et dont les récoltes étaient détruites ou compromises par le gel et la grêle. Cette dernière venait en effet de ravager quelques-unes des meilleures régions du canton. Le 8 juin, entre deux et trois heures du matin – ce qui était tout à fait exceptionnel – une forte colonne de grêle s’était abattue dans la contrée s’étendant d’Aubonne par Morges, Cossonay et Echallens, jusqu’à Payerne, détruisant presque tout ce que le gel avait pu laisser subsister quelques jours auparavant.\par
Beaucoup de citoyens qui avaient suivi avec plaisir les bandes armées au commencement du mois de mai, virent alors la situation sous des couleurs plus sombres et, bientôt, la désunion se montra un peu dans les rangs des \emph{Bourla-Papey.} Tandis que les uns continuaient à parcourir le pays et à exciter les passions politiques, d’autres, de plus en plus nombreux, crurent devoir montrer une déférence subite pour le gouvernement et manifestèrent leur repentir dans l’espérance de voir diminuer les charges qui pesaient sur eux.\par
Louis Reymond et le général Turreau étaient encore les deux personnes qui donnaient le plus de craintes au gouvernement et à son Commissaire.\par
Le premier avait sans doute, au commencement de juin, déconseillé à ses amis de prendre les armes, mais on savait d’autre part qu’il dirigeait toujours l’agitation politique dans le pays et que les campagnards écoutaient ses avis avec beaucoup d’empressement. Il avait passé le lac Léman et s’était fixé à Thonon, où ses amis venaient le voir et où, sous les yeux des autorités françaises, avaient lieu de nombreux conciliabules. Il logeait chez son conseiller, le Dr Devaud, à l’entrée de la ville. C’est là que le juge Rouge de Lausanne et quelques personnes connues, de Morges et d’Aubonne, allaient le voir, rapportant ensuite sur terre vaudoise un conseil ou un mot d’ordre.\par
Le Commissaire, le Petit Conseil et surtout le général Séras se plaignirent de cela auprès du sous-préfet de Thonon et l’invitèrent à éloigner cet agitateur. Cela fut complètement inutile. Non seulement ce magistrat ne fît rien pour entraver les faits et gestes de Louis Reymond, mais il sembla encore le protéger.\par
Ce dernier portait encore généralement son uniforme de capitaine. Parfois cependant il le remplaçait par le costume civil pour aller faire un petit voyage à Genève où beaucoup d’hommes compromis se retirèrent lorsque le tribunal spécial fut organisé à Lausanne et que les arrestations commencèrent. Le citoyen Vauthey, de Yens, possédait une campagne au Petit Saconnex. Fugitif luimême, il donnait l’hospitalité à une colonie vaudoise de quelque importance dans laquelle on remarquait entre autres Guibert, de Morges ; Guibert, de Lussy ; Muret-Grivel, ex-inspecteur général des milices ; Cart de l’Ange, à Nyon, qui avait présenté la sommation du 11 mai à la ville de Morges, et Marc Cart, de St-Saphorin. Le nombre des réfugiés en Savoie et à Genève ou aux environs, monta jusqu’à deux cents.\par
Le Commissaire pria le Conseil Exécutif de négocier avec la France l’extradition du chef des \emph{Bourla-Papey.} Le gouvernement ne crut pas devoir suivre ce conseil ; il répondit qu’il ne fallait pas aller au devant d’un refus presque certain de Bonaparte et que, du reste, il était nécessaire de tenir compte des promesses faites le 11 mai à Louis Reymond par le citoyen Kuhn. Quelles avaient été ces promesses ? Le gouvernement interrogea à plusieurs reprises, à ce sujet, son premier Commissaire, mais ne put jamais, parait-il, en recevoir une réponse précise et complète.\par
Quant au général Turreau, il fut toujours considéré par le citoyen Kuhn comme le principal moteur de l’insurrection et c’est même là, au dire de ce magistrat, une des raisons essentielles pour lesquelles il ne voulut pas se servir de la force et, plus tard encore, réclama l’amnistie en faveur des hommes compromis dans l’événement. L’influence prêtée à Turreau par le premier Commissaire fut, sans doute, un peu exagérée. Elle n’en fut pas moins considérable, par les émissaires qui venaient auprès de lui, longtemps encore après la dissolution de la troupe des \emph{Bourla-Papey} et qui maintenaient dans le Pays de Vaud la persuasion que la France était complètement dévouée aux insurgés. Lorsque le général Séras commença à montrer une fermeté à laquelle ses collègues et prédécesseurs n’avaient pas habitué les populations, on vit, parait-il, circuler parmi ces dernières des pétitions par lesquelles elles demandaient l’éloignement de cet homme et son remplacement par Turreau ou Amey. Si ces adresses n’aboutirent pas au résultat désiré, elles n’en furent pas moins portées à Paris par des députés.\par
Le gouvernement partageait, au reste, les sentiments et les craintes du Commissaire à l’égard du « bourreau du Valais ».\par

\begin{quoteblock}
 \noindent « Ce que vous nous marquez au sujet du général Turreau est analogue à notre manière de voir, lui disait-il \footnote{Conseil exécutif au Commissaire F. May : lettre du 6 juillet.}. La proximité de cet homme est une vraie calamité publique. Déjà plus d’une fois, soit par des démarches indirectes, soit par des notes officielles, nous avons demandé son rappel. Mais c’est si peu de chose qu’une demande adressée par le gouvernement de la pauvre Helvétie au gouvernement tout-puissant de la France. Nous devons cependant à la vérité de dire que le Ministre Verninac et le général Montrichard appuient, de bien bonne foi, nos démarches ».
 \end{quoteblock}

\noindent Sous l’influence de Turreau et de Louis Reymond, secondés par d’autres personnages importants, l’agitation se perpétuait dans le pays et les adresses pour la réunion à la France continuaient à circuler et à se couvrir de signatures. Au milieu du mois de juin, une personne rentrant de Paris au Pays de Vaud annonça avoir rencontré, en route, deux députés qui s’y rendaient. Cette nouvelle fut confirmée au commencement de juillet et le bruit se répandit que les représentants des annexionistes avaient été écoutés en France, qu’on leur avait fait des promesses et que toutes les recherches judiciaires commencées allaient être abandonnées. Claude Mandrot et Jaïn, ex-membre de la Chambre administrative, accompagnés d’autres personnes, avaient soutenu avec force, disait-on, les arguments de leurs amis contre le gouvernement helvétique et la Constitution nouvelle. Le Conseil exécutif chercha à détruire cette opinion en annonçant que son Ministre à Paris, Ph.-A. Stapfer, avait pris, avec le Préfet de police, toutes les mesures nécessaires pour contrecarrer ces intrigues, mais il ne réussit pas à convaincre complètement le public.\par
De divers côtés, on annonçait, du reste, que l’agitation et le mécontentement étaient encore très grands dans les campagnes, toujours excitées par des démagogues. C’est dans cette dernière classe que devait rentrer, sans doute, le régent d’Oron, le citoyen Pernet, qui avait été un des principaux boute-en-train de l’insurrection dans la contrée. On peut juger de la haine que lui portaient les modérés par les lignes suivantes, extraites d’une lettre du sous-préfet du district :\par

\begin{quoteblock}
 \noindent  « On le charge d’avoir dit qu’il faudrait que Bonaparte fût en cendres et que de ses cendres naquit des Robespierre ; qu’il fallait la guillotine, que sans cela, jamais la révolution ne marcherait. Si Pernet est un Robespierre,… si aux défauts qu’on lui reproche, il joint une âme atroce, la hyène de Gévaudan n’est rien en comparaison de cet homme et Dieu nous préserve qu’il ait jamais rien à dire dans le gouvernement helvétique \footnote{Sous-préfet Gilliéron au Préfet national, lettre du 3 juillet.}  ».
 \end{quoteblock}

\noindent Un autre indice de la surexcitation des esprits fut le contenu d’un placard ou proclamation qui fut affiché à plusieurs endroits à Yverdon, pendant la nuit du 7 au 8 juillet. Voici ce libelle dont le contenu était à la hauteur du style :\par

\begin{quoteblock}
 \noindent « Doxat, de Turin, fils, sous-préfet téméraire, exécrable brigand, tu as cordelé les frères de ton pays, tonnerre de Chouan.\par
 « Les familles éperdues que tu as désolées dans tout notre district, vengeance doi [vent] crier. C’en est fait et fini ; tous les brigands de la clique doivent périr, ta tête sautera ou la terre s’écroulera. C’est juré, il faut, pour le repos public, que ta tête on fasse sauter. C’en est fait, la mort t’est jurée ».\par
 A la suite du nom du sous-préfet se trouvait encore la liste des suivants :\par
 « Bezencenet-Rablet – Paillard, orfèvre – Pétregnet, huissier – Le russe Christin \footnote{Ferdinand Christin, fils du Banneret.} – Pillichody, major \footnote{Fr. Pillichody, ex-seigneur de Bavois.} – Roctis 1… – Son maître \footnote{Bourgeois, ex-châtelain des Clées, à Valleyres.}. – Faciot de Blonay — Demartines. »
 \end{quoteblock}

\section[Repentir]{Repentir}
\noindent Ainsi que cela a déjà été dit plus haut, beaucoup de personnes croyaient qu’il fallait chercher par une soumission plus ou moins apparente à arrêter les mesures de rigueur que prenaient le gouvernement et son Commissaire.\par
Au commencement de juillet, on apprit qu’un grand nombre de communes de la région de Cossonay envoyaient des députés à Berne pour y présenter une adresse marquant leur repentir. Ces mandataires étaient les citoyens Bourgeois, de Morges, ex-membre du Corps législatif ; Solliard, de Cossonay, ex-greffier du tribunal du canton ; Marc-Louis Vionnet, ex-juge de district, et Samuel Clerc, hommes modérés, mais, surtout, amis des campagnards. Ils remirent leur supplique au Conseil exécutif le 9 juillet.\par
Après avoir reconnu leurs torts, les communes s’exprimaient de la manière suivante :\par

\begin{quoteblock}
 \noindent « Par l’organe de leurs députés, elles viennent déposer en vos mains leurs promesses d’obéissance aux lois et leurs respects pour le gouvernement constitutionnel. Mieux éclairées, elles remettent leur sort entre vos mains ; elles sont convaincues que le bonheur du peuple et son soulagement est votre premier vœu, tout comme il est celui du peuple qui vous a remis le pouvoir de nous gouverner.\par
 « Veuillez donc, Citoyens Magistrats, porter vos regards sur les tristes habitants du Canton de Vaud ; en voyant les maux qui les accablent, vos cœurs en seront émus. Le gel, la grêle ont dévasté nos vignes et nos champs ; le pays est sans ressources, chargé de militaires ; des contributions Font frappé ; plusieurs individus gémissent dans les fers ; ils succombent sous le fardeau.\par
 « Ils viennent solliciter votre commisération ; confiants en votre bonté, ils espèrent en vous et vous demandent avec instance un prompt soulagement à leurs maux et l’oubli général du passé \footnote{Cette lettre représentait les vœux de trente-deux communes des districts de Cossonay, Morges, Aubonne, Oron et Eehallens.} … »
 \end{quoteblock}

\noindent Les députés des communes restèrent à Berne pendant dix-sept jours et intervinrent auprès de diverses personnes influentes pour chercher à obtenir du gouvernement une amnistie en faveur des insurgés. S’ils n’arrivèrent pas au but, ils obtinrent du moins quelques importantes marques d’intérêt et des promesses qu’un avenir peu éloigné devait, du reste, réaliser. Avant de rentrer dans leurs foyers, ils adressèrent au Conseil exécutif un message pour le remercier de sa bienveillance et plaider encore une fois la cause de leurs concitoyens. Ce message est une pièce importante. Il indique bien l’opinion moyenne qui devait régner dans les régions agitées du Canton du Léman à ce moment-là. Il rappelle tout d’abord l’adresse des communes repentantes et continue de la manière suivante :\par

\begin{quoteblock}
 \noindent « D’après ces assurances (celles données par le gouvernement aux députés), les communes exposantes osaient espérer et vous suppliaient de mettre fin aux maux qui pèsent tant sur elles que sur plusieurs individus qui gémissent dans les cachots, accablés de fers ou de privations de toute espèce, même de nourriture salubre, ainsi que sur d’autres qui ont fui loin de leur patrie pour éviter ces incarcérations et les peines extraordinaires qui en sont la suite.\par
 « Déjà le magistrat respectable que le gouvernement provisoire avait envoyé au Canton de Vaud, fit espérer aux insurgés un oubli complet du passé s’ils rentraient dans leurs foyers ; tel fut au moins le sens qu’ils attachèrent à ses discours. Et même quand il serait vrai que ces hommes égarés n’en avaient pas bien saisi le sens, au moins serait-il bien constant qu’alors, sur sa simple sommation, ils se seraient retirés à l’instant dans leurs maisons, sans commettre d’excès, en s’abandonnant avec confiance à la clémence du gouvernement.\par
 « Que ne doit donc pas attendre d’un pareil peuple qui, pendant quatre ans de révolution, ne s’est oublié que ce seul instant, un gouvernement maintenant constitutionnel qui prendra à tâche de raviver l’esprit public et de rassembler sous ses ailes tous les partis pour ne plus faire qu’un peuple de frères ?\par
 « D’après ces promesses déjà présentées au gouvernement et que les députés des communes répètent encore, ils osent croire, citoyens Magistrats du Peuple, qu’écoutant plutôt vos cœurs et les sentiments paternels qui portent toujours au pardon, vous vous hâterez de prononcer un oubli général sur tout ce qui s’est passé et que vous montrerez ainsi à la nation entière que si vous savez punir, vous aimez encore mieux pardonner à ceux qui reconnaissent leurs torts.\par
 « Les soussignés que, dans leurs informations particulières, vous avez daigné recevoir avec la plus grande bienveillance, eussent bien désiré, en se rendant chez eux, pouvoir porter à leurs commettants des paroles de paix et de consolation ; mais s’ils n’en sont pas porteurs par un décret d’amnistie solennel, au moins ils porteront les espérances que vous avez bien voulu leur donner. Ils consoleront de leur mieux les communautés qui les ont envoyés, en attendant qu’incessamment vous réalisiez les promesses que le Peuple du Léman, dont les souffrances et les maux sont si grands, attend de votre bonté, de votre justice et de votre humanité \footnote{Cette adresse est dans l’original, de la main de M. L. Vionnet.} … »
 \end{quoteblock}

\noindent La a Constitution des Notables » qui avait quelque analogie avec celle de la Malmaison, avait été, pendant ce temps, soumise à la sanction du peuple. Les citoyens devant aller inscrire leur vote dans un registre déposé dans les différentes sous-préfectures, il en résulta qu’un grand nombre d’entre eux préférèrent s’abstenir.\par
Le Canton du Léman comptait, en 1802, 35,308 citoyens actifs ; 14,304 refusèrent le projet et 5711 l’acceptèrent. Par une procédure que nous ne pouvons considérer aujourd’hui que comme absolument abusive, il avait été décidé que les citoyens qui n’inscriraient pas leur vote dans le registre civique seraient considérés comme acceptant tacitement le pacte nouveau ; 15293 s’abstinrent dans le Léman et il en résulta que la Constitution des Notables y fut officiellement acceptée par une majorité considérable. Ce n’était pas là, on le voit, un moyen certain de donner à l’administration nouvelle le prestige et la force morale dont elle allait avoir le plus pressant besoin.
\section[Pardon]{Pardon}
\noindent Le citoyen Lanther, qui avait succédé en qualité de Commissaire au citoyen Kuhn, était destiné à occuper une place dans le nouveau Sénat de la République helvétique. Son départ pour Berne fut annoncé dès la fin de juin et, le 6 juillet, le nouveau Conseil Exécutif le remplaça par Frédéric May, qui, en qualité de secrétaire, connaissait déjà dans tous ses détails la situation politique du\par
Pays de Vaud. Ce citoyen, modéré d’opinion, se montra d’une grande fermeté dans l’exercice de ses fonctions et fit son possible pour obtenir la punition des coupables.\par
Le Tribunal spécial, décrété par le Petit Conseil, fut formé des citoyens Ringier, de Zofingue, membre du Tribunal suprême, président ; Steck, de Berne ; Badoud, de Romont ; Herrenschwand, de Morat ; Barras, de Fribourg ; Fasnacht, de Morat, Reinhard, de Soleure ; Kuhni, de Glaris, et Burnand, de Moudon, le seul Vaudois qui consentit à en faire partie. Herrenschwand fut chargé de remplir les fonctions d’accusateur public.\par
Le tribunal spécial siégea dès la fin du mois de juin et lança des ordres d’arrestation contre un grand nombre d’individus compromis dans l’insurrection.\par
Beaucoup de ces derniers quittèrent, comme On l’a déjà vu, le territoire du Canton ; d’autres, avertis à temps, purent s’échapper. Ce fut le cas pour Besson, de Niédens, qui avait commandé la colonne des gens de la région d’Yvonand. Lorsque l’agent de ce village arriva chez lui, il apprit que l’inculpé avait quitté son domicile depuis très peu de temps. Le citoyen Potterat, d’Orny, fut plus courageux. Ne s’étant pas trouvé chez lui lorsqu’on vint l’arrêter, il se rendit le lendemain à Lausanne et fit annoncer au président du tribunal qu’il venait se constituer prisonnier. Le nombre des accusés détenus dans les prisons devint bientôt très considérable et la procédure fut beaucoup plus longue et compliquée que le gouvernement l’avait supposé.\par
On a pu voir dans l’adresse remise au Conseil Exécutif par les représentants des communes vaudoises, que ces derniers considéraient la nourriture donnée aux prisonniers comme insuffisante et il semble bien, en effet, que la sévérité à leur égard fut poussée trop loin à ce point de vue. Le tribunal avait décidé qu’ils recevraient une livre de pain et « trois bonnes soupes nourrissantes » par jour. Lorsque l’un d’entre eux devrait être soumis à un régime correctionnel, cette pitance se composerait de deux livres de pain et d’eau chaque jour et d’une soupe chaque deuxième jour.\par
La Chambre administrative se plaignit au citoyen Kuhn, ministre de la justice, delà sévérité de ce régime, appliqué à des citoyens détenus en prison préventive pour délits politiques. Il résulta de cette plainte une correspondance peu aimable entre l’ex-Commissaire et son dernier successeur, Frédéric May, qui pria le Conseil Exécutif de se persuader que « les prisonniers ne mourraient pas de faim ».\par
Cette divergence de vues et d’opinion entre le Commissaire et le Gouvernement qu’il représentait ne fit qu’augmenter.\par
Une circonstance nouvelle était venue en effet modifier tout à coup l’attitude du Conseil Exécutif et le décider à se montrer plus aimable à l’égard des Vaudois. La Constitution des Notables, adoptée sans enthousiasme par l’ensemble de la Suisse, allait être mise à exécution. La République jouirait enfin, semblait-il, d’institutions définitives et d’une paix intérieure que tout le monde demandait depuis longtemps. Toujours désireux de faire voir à l’Europe qu’il ne voulait influer en rien sur la politique intérieure de la République helvétique, Bonaparte crut devoir, à ce moment, donner aux troupes françaises et au général Montrichard, leur chef, l’ordre de quitter le territoire suisse.\par
Cet événement était désiré depuis longtemps sans doute, mais le Conseil Exécutif se demanda avec une vive anxiété si son prestige et son pouvoir seraient suffisants pour assurer le maintien de l’ordre sans le secours de cette force étrangère. Ses craintes ne tardèrent pas à augmenter encore.\par
Les cantons fédéralistes de la Suisse centrale s’empressèrent, en effet, de protester contre la Constitution des Notables. Des comités insurrectionnels se formèrent rapidement de divers côtés et même jusque dans la résidence du Conseil Exécutif. Le parti fédéraliste se prépara ainsi à renverser un gouvernement sans influence dans le pays, faible, incapable aux yeux de beaucoup et n’ayant que deux mille hommes à opposer à ses adversaires résolus et enthousiastes.\par
Dans le Léman, la retraite des troupes françaises remplit de joie les \emph{Bourla-Papey} qui comptaient déjà sur une amnistie prochaine.\par

\begin{quoteblock}
\noindent « Il résulte de cet état de choses, écrivait Polier, l’effroi des amis de l’ordre, le découragement absolu des fonctionnaires. Les conséquences nécessaires seront la déconsidération du gouvernement, le rejet de toute constitution cantonale qui ne donnerait pas aux insurgés les moyens de placer leurs chefs à la tête de toutes les autorités \footnote{Polier au Conseil Exécutif, 27 juillet.} … »\end{quoteblock}

\noindent Polier et May, dont les opinions étaient semblables, demandaient en conséquence au gouvernement une fermeté inébranlable et une conduite prudente.\par
Menacé par les fédéralistes, le Petit Conseil serait-il obligé d’employer les faibles forces dont il pouvait disposer, à maintenir l’ordre dans le Canton du Léman ? Ce dernier avait été l’appui le plus important de la République helvétique et du parti unitaire et il avait fourni environ le quart des impositions générales. Une révolte très grave y avait eu lieu sans doute, mais n’était-elle pas l’œuvre même du parti patriote et unitaire qui voulait enfin jouir, coûte que coûte, des avantages économiques que la révolution lui avait fait entrevoir depuis cinq ans ? Insurrection grave et malheureuse à certains égards, peut-être, mais qui montrait d’autant plus, chez un peuple généralement impassible et bon, l’état d’anarchie morale dans laquelle des désillusions renouvelées avaient fini par le jeter. Les plus exaltés avaient, sans doute, commis aux yeux du gouvernement le crime de vouloir abandonner l’Helvétie ; mais après plus de deux siècles de sujétion, les espérances qu’ils avaient placées dans le nouveau régime ne se transformaient-elles pas en un mirage décevant, ne laissant en réalité que la perspective d’une situation toujours inférieure ? Ils avaient appris depuis longtemps à voir dans les Confédérés des dominateurs que cinq années de troubles ne semblaient pas avoir transformés. Poussés par des flatteurs puissants et peut-être autorisés, qui leur montraient la liberté économique dans l’alliance avec la « Grande Nation », ils tendaient les bras vers cette puissance qui les avait soutenus énergiquement et chez laquelle beaucoup persistaient à ne voir que de la générosité. Ils voulaient rester Suisses, sans doute, et le Conseil Exécutif le savait, mais il fallait pour cela qu’on voulût bien les considérer comme des égaux. C’était l’opinion de tous les patriotes vaudois. a Je m’étais convaincu, dit Monod, qu’une nouvelle réunion du Canton de Vaud à celui de Berne serait la ruine de celui-là ; j’en étais tellement pénétré que si j’avais à opter, je préférerais la réunion à la France. »\par
Le Conseil Exécutif arriva donc à se convaincre que pour faire face au parti fédéraliste, il devait rattacher tout à fait les Vaudois à la Suisse et, pour cela, adopter à l’égard des \emph{Bourla-Papey} une politique de conciliation. L’ex-Commissaire Kuhn fut encore, dans ce moment-là, le citoyen qui plaida avec le plus de force en faveur d’un oubli aussi complet que possible des événements du mois de mai. Il présenta dans ce but au gouvernement un rapport étendu dans lequel il chercha surtout à montrer que le plus grand coupable ne pouvant être atteint, puisqu’il était un général français, il ne fallait pas frapper ceux qui avaient été entraînés par lui. Parmi ces derniers, il y avait sans doute des hommes qui, par leur attitude, s’étaient suffisamment compromis pour qu’un tribunal pût les faire paraître à sa barre ; mais à côté d’eux, et surtout au-dessus d’eux, combien d’autres étaient peut-être plus coupables encore, mais avaient agi dans l’ombre et ne pouvaient être poursuivis ! Plusieurs des chefs ostensibles du mouvement n’étaient – à l’origine du moins – que les instruments de quelques hommes politiques en face desquels la justice restait impuissante.\par
Ces arguments essentiels entraînèrent le gouvernement, et le Commissaire May chercha inutilement, par ses volumineux rapports du 28 juillet, à les combattre par les raisonnements les plus pressants. Tout ce qu’il parvint à obtenir, c’est que le tribunal spécial pût continuer ses travaux pendant quelques jours encore et prononcer le jugement des hommes les plus compromis.\par
Ces derniers, pour la plupart absents, furent jugés sévèrement. Reymond, Marcel, Cart de l’Ange, Dautun et Claude Mandrot furent condamnés à la peine de mort ; Besson de Niédens, Abram Gleyre de Chevilly, Jean-Louis Deblue de Founex, Jean-Isaac d’Etagnières et Gottraux de Promenthoux, respectivement à 10, 16, 24, 14, 15 et 20 ans de fers. Potterat d’Orny, le seul qui comparut devant ses juges, fut frappé de six ans de réclusion.\par
Les dernières sentences furent rendues pendant la première moitié du mois d’août. Les membres du tribunal spécial s’empressèrent alors de se disperser et de rentrer chez eux avec la conviction qu’ils avaient accompli une besogne aussi ingrate que vaine.
\section[Le Préfet Monod]{Le Préfet Monod}
\noindent Un événement important pour le Canton venait de survenir, qui devait transformer complètement les circonstances politiques. C’est la nomination de Henri Monod comme Préfet national.\par
Depuis plus d’un an et demi, Monod vivait à Paris dans une retraite plus ou moins complète. Il avait continué sans doute à s’intéresser aux événements de son pays, mais découragé à la suite des événements du 7 janvier et du 7 août 1800, il avait résolu de n’y rentrer que le jour :\par

\begin{quoteblock}
\noindent « où ses destinées seraient fixées d’une manière qui parût promettre la liberté et le bonheur. »\end{quoteblock}

\noindent Dans le courant du mois de juillet 1802, il apprit que son beau-père était tombé gravement malade et il ne crut pas pouvoir se dispenser d’accourir avec sa femme auprès de lui.\par

\begin{quoteblock}
 \noindent « Tel fut, dit-il, l’unique, mais impérieux motif d’un retour auquel, par toutes sortes de raisons, j’avais le plus grand regret. J’étais si éloigné de l’idée de rester en Suisse que j’avais gardé mes meubles à Paris. »
 \end{quoteblock}

\noindent A peine Monod eut-il foulé le sol du Canton de Vaud que le bruit se répandit de sa nomination imminente à la charge de Préfet National. Dès le 5 août, en effet, il reçut du Conseil Exécutif la lettre suivante :\par

\begin{quoteblock}
 \noindent « Informé de votre retour dans votre patrie, le Conseil d’Exécution s’empresse de vous y attacher. 11 vous a nommé à la Préfecture du Canton de Vaud, en remplacement du citoyen Polier. Votre attachement connu aux principes libéraux, votre réputation de justice et de fermeté, la confiance et l’estime que vous ont vouées vos concitoyens, ont déterminé ce choix ; le gouvernement espère que vous ne tromperez pas son attente. En acceptant sans délai l’emploi qui vous est offert, vous pouvez assurer le maintien de l’ordre et la tranquillité publique dans le Canton de Vaud sans que le Gouvernement se voie dans la nécessité d’y employer des troupes qui sont nécessaires ailleurs. – Il est inutile que le Conseil vous invite à témoigner au citoyen Polier, en recevant de lui les affaires de la préfecture, les égards auxquels ses vertus et les services qu’il a rendus lui donnent droit. Peut-être n’a-t-il manqué à cet homme estimable, pour être heureux dans son administration, que des circonstances différentes… »
 \end{quoteblock}

\noindent Croyant peut-être obliger Henri Monod à accepter, en le laissant seul en face des fonctions qui lui étaient offertes, le Conseil Exécutif annonça le même jour à Polier qu’il était remplacé.\par

\begin{quoteblock}
 \noindent « Cette décision a coûté au Gouvernement, lui disait-il, car, pendant quatre ans, vous avez travaillé avec un zèle infatigable et un dévouement entier au bien de votre pays ; vos vues ont été pures ; les moyens que vous avez mis en œuvre, dignes d’un homme vertueux ; votre conduite publique n’a jamais démenti votre caractère privé. Mais les circonstances critiques où se trouve la République obligeant le gouvernement à maintenir la tranquillité et l’ordre dans le Canton de Vaud avec le moins de forces militaires possible, votre remplacement par le citoyen Monod a paru promettre quelque avantage sous ce rapport. – Vous emportez l’estime, les vœux et les regrets du gouvernement ; vous emportez surtout le témoignage d’une conscience sans reproche, et vous trouverez en vous-même bien des sujets de vous féliciter d’être rendu aux douceurs, aux habitudes et au bonheur de la vie privée… »
 \end{quoteblock}

\noindent Quoique Monod voulût bien, moyennant une amnistie en faveur de la plupart des coupables, accepter les pénibles et délicates fonctions de Préfet national, ce fut cependant sans enthousiasme qu’il rentra dans la carrière politique.\par

\begin{quoteblock}
 \noindent « Sans parler, dit-il, des embarras, des peines et des soucis de toute espèce où je savais que je me serais replongé en acceptant, j’éprouvais alors une répugnance presqu’invincible à me charger d’emploi. Parmi les membres du gouvernement helvétique, j’en connaissais personnellement quelques-uns qui avaient toute mon estime, d’autres que je ne connaissais que de réputation l’avaient de même, quoique souvent leur conduite politique n’eùt pas été conforme à ma manière de voir. Mais il y en avait sous les ordres desquels j’aurais eu un grand éloignement à me trouver. Je n’hésite pas même à l’avouer, et pourquoi ne le dirais-je pas puisque je ne parle que de ce qui était public ? Tous les changements survenus dans le gouvernement helvétique, sa faiblesse, ses vacillations, l’avaient plongé dans un tel état de discrédit et de déconsidération, que je ne me sentais à ma place dans aucune de celles qu’il eût pu me confier \footnote{Monod : \emph{Mémoires} I 209.}. »
 \end{quoteblock}

\noindent Henri Monod montra donc certainement beaucoup de bonne volonté et un grand dévouement lorsqu’il se décida enfin à remplacer Polier, dont le tempérament, les idées, les habitudes et les opinions n’étaient plus en harmonie, en ce moment-là, avec ceux de la majorité de ses concitoyens.\par
Le gouvernement ne laissa pas même à Monod le temps qu’il désirait pour mettre en ordre ses affaires personnelles.\par

\begin{quoteblock}
 \noindent  « Tout l’état des choses appelle à la tête de votre Canton un homme sage, ferme et jouissant de la confiance générale, lui écrivait-il le 11 août. Le Conseil a compté beaucoup sur votre entrée en fonctions coïncidant avec le décret d’amnistie qui va être incessamment rendu et que vous recevrez à la fin de la semaine. Il ne peut dès lors que vous réitérer l’invitation de prendre incessamment en mains les affaires de la préfecture ; il compte sur vous et, sans doute, vous remplirez son attente… »
 \end{quoteblock}

\noindent Monod annonça donc à ses concitoyens, son entrée en fonctions dès le 16 août, par une proclamation dont quelques passages au moins doivent être placés sous les yeux du lecteur. Après avoir rappelé qu’il succédait à \emph{« un homme justement estimé par ses talents, sa bienfaisance et sa piété »}, il montrait les dangers politiques du moment et continuait de la manière suivante :\par

\begin{quoteblock}
 \noindent « L’égarement dans lequel on vous a conduits vous a coûté cher ; le gouvernement a dû punir ; il a montré qu’il le pouvait, mais la clémence est dans son cœur, il fait espérer qu’il va la laisser parler. Votre reconnaissance vous fera donc fortement tenir à l’ordre ; votre intérêt vous le commande, de nouveaux troubles seraient la ruine de la patrie. N’en doutez pas, citoyens, des agitateurs spéculent sur ces troubles ; les uns y comptent parce qu’ils croient voir leur intérêt à vous faire renoncer au nom \emph{d’Helvétiens}, de Suisses ; les autres y comptent parce qu’ils voient le leur à vous ramener à un ordre de choses détruit sans retour, parce que toutes ses bases le sont. Si les premiers réussissaient, leur patrie reprendrait bientôt le nom de désert qu’elle avait dans le moyen-âge, et les seconds aimeraient-ils donc mieux dominer sur des ruines, que de vivre tranquilles et heureux sous l’empire de la vraie liberté ? La vraie liberté 1 non, non, citoyens, elle ne périra pas, elle vivra chez les Vaudois, j’en atteste la Constitution qui se prépare, et les sacrifices que vous avez faits. L’indépendance marchera de front avec la liberté ; la retraite des troupes étrangères la proclame mieux que tous les discours. Pour conserver les deux premiers biens des hommes en société, vous prouverez que rien ne vous coûte. Au courage que vous avez toujours montré dans les combats, que vous saurez montrer toutes les fois que vos chefs vous y appelleront, vous allierez des vertus moins brillantes, et non moins utiles, la constance dans l’adversité, la patience dans les maux ; c’est ainsi que les peuples maîtrisent la fortune… L’union, tel doit être maintenant notre cri de ralliement. Pour la ramener, votre premier magistrat vous promet la plus exacte impartialité ; il ne connaît et ne veut connaître d’autre parti que celui du bien public ; il le voit et ne peut le voir que dans la consolidation de l’ordre établi ; quiconque y travaillera aura sa confiance, et quiconque s’y soumettra sera aussi sûr de sa protection que de celle de la loi. Union, confiance chez le peuple ; impartialité, fermeté dans les autorités ; amour ardent de la Patrie, constance chez tous et le Canton de Vaud, l’Helvétie est sauvée… »
 \end{quoteblock}

\noindent L’amnistie partielle, demandée par Henri Monod et recommandée depuis longtemps par Kuhn, fut enfin votée par le Sénat helvétique le 19 août. Elle fut le résultat d’une discussion importante dans laquelle le rapport de minorité présenté par Auguste Pidou finit par rallier les voix du plus grand nombre. Ce rapport remarquable est un tableau précieux de la situation du pays et de celle des insurgés. Ancien accusateur public, Pidou y résuma avec autant de prudence que d’habileté les circonstances atténuantes qui devaient décider le Sénat à voter le décret d’amnistie proposé par le gouvernement. Voici les points sur lesquels Pidou insista le plus fortement.\par

\begin{quoteblock}
 \noindent \emph{a)}« Premièrement provocation, provocation violente, longue, continue, parvenue enfin à sa dernière période… Au moment de l’insurrection, cet infortuné peuple était graduellement arrivé à ce point fatal de misère et d’exaspération où l’homme, machine frêle, ne sait plus ce qu’il fait et se jette dans le crime par désespoir.\par
 \emph{b)} « Ils étaient trois mille, quatre mille, cinq mille ; ils étaient organisés, armés. Ils ont fait du mal, ils ont fait de grands maux. Mais, nous le demandons au plus rigide des juges, ce qu’ils ont fait, dans quelle proportion est-il avec les délits atroces dont, dans des circonstances pareilles, un peuple moins bon, une fois échauffé, n’eût pas manqué de se souiller ? Tournez les yeux vers l’Ouest, et répondez.\par
 \emph{c)} « J’ajoute que les insurgés du Léman ont déjà tous éprouvé, plus ou moins, les suites cuisantes de leur criminelle entreprise… »
 \end{quoteblock}

\noindent Pidou énumérait ensuite toutes les charges qui pesaient sur eux ensuite de l’insurrection, les incarcérations, l’exil, la perte d’un temps précieux et enfin les fléaux du ciel qui réduisaient \emph{« un grand nombre d’entre eux, pour plusieurs années, à la dernière misère »}. Il montrait enfin que les insurgés avaient été « humiliés » puisqu’un grand nombre de communes s’étaient décidées à « témoigner leur repentir et à demander grâce ». La multitude des coupables lui paraissait aussi un obstacle presque insurmontable à l’application d’une justice impartiale, d’autant plus, ajoutait-il, que les plus dangereux des chefs ne s’étaient pas compromis suffisamment pour craindre des recherches juridiques.\par

\begin{quoteblock}
 \noindent « Si à ces considérations internes, disait-il enfin, on ajoute la position relative où l’État se trouve, et la nécessité de rallier promptement par la douceur et par l’amour les cœurs de ceux qui veulent encore être Helvétiens, et plus encore l’impossibilité de jamais faire adopter dans le Léman… aucune organisation quelconque, tant que le peuple, aigri, mécontent,… sans espoir, ne pourra prendre d’intérêt à ce qui se fait que pour le contrarier. – Qui peut peser toutes ces considérations et ne pas conclure avec la minorité de la Commission à ce que les dispositions contenues dans le décret… soient adoptées ? »
 \end{quoteblock}

\noindent Le décret d’amnistie, promulgué le 19 août, renfermait les dispositions principales suivantes :\par

\begin{quoteblock}
 \noindent « La peine de mort prononcée contre Louis-Gabriel Reymond et Henri Marcel… est commuée en un bannissement perpétuel hors du territoire de la République helvétique.\par
 » Les autres sentences de mort sont commuées en un bannissement de dix ans hors du même territoire.\par
 » Les condamnés à la peine des fers seront suspendus de leur droit de citoyens actifs et garderont les arrêts dans leur commune et son territoire pour un espace de temps égal au quart de celui fixé pour leur peine…\par
 » La peine de ceux qui ont été condamnés à une simple réclusion est remise au moyen d’une caution de cinq mille francs qu’ils fourniront pour la moitié du temps que devait durer leur réclusion.\par
 » Quant aux individus… qui sont ou décrétés de prise de corps ou actuellement incarcérés, ils seront de suite libérés, à charge toutefois de fournir chacun une caution… dont le maximum sera de mille francs. »
 \end{quoteblock}

\noindent L’amnistie en faveur de la plupart des accusés, la dissolution du tribunal spécial qui en fut la conséquence immédiate, l’arrivée à la préfecture du citoyen Monod et enfin le départ de toutes les troupes françaises ou helvétiques, furent autant d’événements qui ne tardèrent pas à ramener un peu de calme et de confiance chez les habitants du Canton de Vaud.
\section[Liquidation des dîmes]{Liquidation des dîmes}
\noindent Dans le courant du mois de juillet, le Sénat helvétique avait désigné les citoyens qui, dans chaque canton, allaient être chargés de préparer une Constitution. La Commission vaudoise fut formée des citoyens Monod, Bergier, membre de la Chambre administrative ; Secrétan, ex-Directeur ; Fayod, ex-président du tribunal du Canton ; Bégoz, ex-ministre des affaires étrangères ; Glayre, ex-Directeur ; Soulié, de Nyon, avocat ; Carrard, d’Orbe, avocat ; Alex-César Chavannes, ex-membre de l’Assemblée des Notables ; Correvon, d’Yverdon, ex-sous préfet, et Jomini, de Payerne. Elle se mit aussitôt à l’ouvrage et termina ses travaux le 29 août par l’adoption d’une Constitution cantonale basée sur quelques principes essentiels dont les suivants se rapportent plus spécialement à la question des impôts et redevances :\par
1. Les propriétés du peuple vaudois ne peuvent être grevées à l’avenir d’autres impositions que celles décrétées par ses représentants légitimes.\par
2. Tous les droits féodaux sont à jamais abolis.\par
3. Les dîmes et les censes sont rachetables.\par
Avant de se dissoudre, la Commission voulut attirer l’attention du gouvernement central sur l’urgence qu’il y avait à ce que la Constitution cantonale nouvelle fût mise à exécution aussitôt que possible, afin d’éviter les chances de désorganisations qui subsistaient encore. Elle demanda aussi que la décision prise au sujet des dîmes et censes fût appliquée au plus tôt \footnote{Ce projet ne fut du reste jamais exécuté.} que l’amnistie partielle décrétée le 19 août fût rendue plus générale et plus complète et enfin que les sommes encore dues par les districts insurgés leur fussent remises.\par
Ces demandes importantes, présentées au milieu d’une période de calme et de sécurité pour le gouvernement, eussent sans doute couru le risque de n’être prises en considération sérieuse qu’au bout d’un temps très long. On sait, au contraire, dans quelle situation critique se trouvaient à ce moment la République helvétique et ses chefs. L’insurrection fédéraliste s’était développée avec une grande rapidité. Une Diète nationale se réunit à Schwytz et renferma bientôt des représentants de douze cantons ; Zurich, puis Berne tombèrent au pouvoir des troupes du général Bachmann et le gouvernement helvétique, profondément divisé lui aussi, se voyait dans l’obligation de venir demander l’hospitalité à ce même peuple vaudois qui constituait maintenant son plus ferme et même son unique appui.\par
Au milieu de circonstances aussi désastreuses, le Conseil Exécutif, réduit au désespoir et ne comptant plus que sur une intervention de la France et sur l’appui des patriotes vaudois, se décida facilement à accorder à ces derniers ce qu’ils demandaient.\par
Deux jours après leur arrivée à Lausanne, le Gouvernement et le Sénat décrétèrent que tout ce qui avait été décidé auparavant au sujet des droits féodaux était rapporté, pour ce qui concernait le Canton de Vaud. Les dîmes et les censes devaient être rachetées par les autorités vaudoises qui y consacreraient \emph{« les biens cantonaux et à défaut de ces biens, une répartition modique sur les fonds »} où elles étaient dues auparavant \footnote{Décret du 22 septembre 1802.}. Quant aux contributions encore dues par les districts qui avaient participé à l’insurrection, elles furent mises à la charge du « trésor » national.\par
La Chambre administrative du Léman se mit aussitôt à l’œuvre pour préparer l’exécution de ce décret. Elle montra cette fois tellement d’activité que sept jours plus tard, elle put faire connaître déjà le résultat de ses travaux. Un tableau avait été dressé et imprimé, renfermant des indications précises sur plus de neuf cents propriétés nationales qui allaient être mises aux enchères publiques à partir du 11 octobre suivant. En vertu de l’arrêté de la Chambre administrative, les propriétaires devaient déposer entre les mains du Bureau de liquidation, avant le 15 octobre, un état sommaire du produit annuel de leurs censes tant en argent qu’en denrées et celui des dîmes qu’ils avaient perçues chaque année de 1776 à 1790. Ils devaient recevoir en échange des \emph{« bons, à compte de leurs prétentions sur le canton. »} Les membres du clergé et les instituteurs recevraient aussi des « bons » du même genre \emph{« pour le solde de leurs pensions, calculées jusqu’au premier janvier 1803 \footnote{Arrêté de la Chambre administrative, du 29 septembre 1802.}  »}.\par
La liquidation des dîmes et des censes par le moyen de la vente des biens nationaux, dont un grand nombre composés de vignobles, fut une opération très compliquée et qui se fit dans un moment très peu favorable, étant donnée la situation économique désastreuse du pays et la confiance bien limitée d’une partie de la population à l’égard des institutions que Bonaparte préparait à ce moment-là pour les cantons suisses.\par
On sait, en effets que la révolte du parti fédéraliste avait failli aboutir à la chute du gouvernement unitaire réfugié à Lausanne. Les troupes du général Bachmann gravissaient les pentes du Jorat, le Conseil exécutif et le Sénat faisaient leurs préparatifs de départ pour la côte savoisienne, lorsque se produisit l’intervention et la médiation du Premier Consul, attendues à Lausanne avec une impatience qui croissait à mesure que le danger se rapprochait lui-même.\par
Le Préfet Monod montra, au milieu des circonstances pénibles que traversa alors le Pays de Vaud, une fermeté remarquable. Grâce à la confiance dont il jouissait auprès de ses concitoyens, il put lever plusieurs bataillons de nouvelles troupes, entretenir le zèle et le bon vouloir d’une grande partie du public et arrêter en quelques heures le développement de la contre-révolution que le major Pillichody, ex-seigneur de Bavois, avait fomentée dans les districts du nord du Canton et qui s’était manifestée par l’échauffourée d’Orbe.\par
Le Capitaine Louis Reymond, le chef des \emph{Bourla-Papey} était rentré dans le pays et avait participé avec un zèle remarquable au siège d’Orbe où il fut blessé grièvement au genou. Avant de rentrer à Berne, le gouvernement central voulut témoigner sa sympathie aux patriotes vaudois qui l’avaient soutenu dans sa détresse ; il accorda aux condamnés du tribunal spécial une amnistie plus complète que la première. Reymond fut récompensé ainsi du dévouement qu’il avait voulu montrer encore pour la cause qui lui était chère, au moment où le gouvernement bernois croyait déjà rentrer en possession du Pays de Vaud.\par
La Diète vaudoise se réunit les 1 et 2 novembre pour nommer ses députés à la Consulta de Paris et leur donner des instructions. Henri Monod, Jules Muret et Louis Secrétan furent choisis. Ils devaient offrir au Premier Consul l’hommage de la reconnaissance du peuple vaudois et lui exprimer leur espoir que dans :\par

\begin{quoteblock}
\noindent « aucun temps, il ne puisse être donné la moindre atteinte à l’indépendance du Canton de Vaud dans ses rapports avec les autres, avec lesquels il doit être dans la plus entière égalité quant aux droits politiques. »\end{quoteblock}

\noindent La défiance que les patriotes vaudois avaient montrée pour le gouvernement central depuis les coups d’État du 7 janvier et du 7 août 1800 et qui avait été exprimée ouvertement dans la fameuse « Adresse aux autorités du Canton du Léman », existait encore un peu, le 2 novembre 1802. Elle se manifesta par un article des instructions données aux députés et en vertu duquel l’assemblée formait le vœu :\par

\begin{quoteblock}
\noindent « que la Constitution helvétique fût basée sur les principes d’une fédération sous le lien d’un pouvoir central auquel seraient confiés les relations extérieures et le militaire. »\end{quoteblock}

\noindent C’était là une profession de foi singulièrement fédéraliste de la part des anciens unitaires vaudois, mais les événements leur avaient montré que le canton devait posséder son existence autonome, s’il voulait pouvoir jouir d’institutions conformes au génie de ses habitants. Cette dernière volonté était si forte, elle avait pénétré si profondément dans les esprits qu’elle l’emportait même sur celle de l’union avec le reste de la Suisse en général et avec Berne en particulier. Cela est attesté par un Décret de cette même Diète vaudoise qui constitue une des instructions données à ses députés. « Si les autres cantons, dit ce Décret, venaient à émettre, dans la prochaine Consulta, une opinion tendante à altérer, soit dans son principe, soit dans ses résultats, la pleine indépendance du Canton de Vaud à l’égard d’un autre Canton, ou à le placer dans la ligne des cantons à un degré d’infériorité quelconque, et si cette opinion prévalait, les députés du Canton de Vaud sont spécialement chargés de demander au Premier Consul la séparation du Canton de Vaud du reste de l’Helvétie et son aveu pour qu’il puisse s’ériger en république souveraine, sous la garantie de la France \footnote{On peut comparer ce décret et les événements du temps avec la devise qui figure sur l’écusson vaudois et où la liberté prend place avant la patrie.}  ».\par
Après plusieurs mois de séjour à Paris, les députés vaudois purent rentrer dans leur pays avec l’Acte de Médiation et surtout une Constitution cantonale. Une commission composée des citoyens Monod, Glayre, Bergier, Pidou, Carrard, de Mellet et Muret était chargée d’en assurer la mise en vigueur. Enfin, le 14 avril 1803, le Grand Conseil se réunit et put donner au Canton de Vaud, maintenant souverain, son premier gouvernement national.\par
Le produit de la vente des biens nationaux avait suffi pour payer les trois quarts des indemnités dûes aux propriétaires de droits féodaux. Restant fidèle au Décret rendu par le Sénat helvétique le 22 septembre 1802, le Grand Conseil adopta le 31 mai 1804 une loi par laquelle le reste de ces indemnités était réparti entre ceux qui, avant la Révolution, avaient payé ces droits. L’État garantissait ces sommes aux anciens propriétaires de fiefs et les débiteurs les remboursaient par le moyen de sept annuités. Deux propriétaires, Ami Rigot, de Begnins, et Charles-Albert de Mestral, de St-Saphorin, protestèrent contre certaines dispositions de cette loi et déclarèrent ne vouloir s’y soumettre que par la force. Ils furent arrêtés, et, à la suite d’un long procès qui passionna les esprits, ils furent frappés d’un mois d’arrestation dans leur propriété \footnote{Voir sur ce curieux procès : Exposé de la Procédure \emph{instruite contre M. Ami Rigot, ci-devant seigneur de Begnins} et \emph{Recueil des faits relatifs d l’arrestation, la détention, le procès et le jugement du Colonel Charles-Albert de Mestral-Saint-Saphorin.}}.\par
Ce fut là l’épilogue des discussions véhémentes et des troubles qui avaient agité le pays depuis le commencement de la Révolution. \emph{« Le régime féodal est proscrit à jamais du territoire du Canton de Vaud »}, disait le dernier article de la loi du 31 mai 1804. Les campagnards purent dès lors envisager l’avenir avec plus de confiance. Cet article de loi mettait fin, en effet, à la Révolution au point de vue économique, comme l’Acte de Médiation y avait mis fin au point de vue politique.
\chapterclose

 


% at least one empty page at end (for booklet couv)
\ifbooklet
  \newpage\null\thispagestyle{empty}\newpage
\fi

\ifdev % autotext in dev mode
\fontname\font — \textsc{Les règles du jeu}\par
(\hyperref[utopie]{\underline{Lien}})\par
\noindent \initialiv{A}{lors là}\blindtext\par
\noindent \initialiv{À}{ la bonheur des dames}\blindtext\par
\noindent \initialiv{É}{tonnez-le}\blindtext\par
\noindent \initialiv{Q}{ualitativement}\blindtext\par
\noindent \initialiv{V}{aloriser}\blindtext\par
\Blindtext
\phantomsection
\label{utopie}
\Blinddocument
\fi
\end{document}
