%%%%%%%%%%%%%%%%%%%%%%%%%%%%%%%%%
% LaTeX model https://hurlus.fr %
%%%%%%%%%%%%%%%%%%%%%%%%%%%%%%%%%

% Needed before document class
\RequirePackage{pdftexcmds} % needed for tests expressions
\RequirePackage{fix-cm} % correct units

% Define mode
\def\mode{a4}

\newif\ifaiv % a4
\newif\ifav % a5
\newif\ifbooklet % booklet
\newif\ifcover % cover for booklet

\ifnum \strcmp{\mode}{cover}=0
  \covertrue
\else\ifnum \strcmp{\mode}{booklet}=0
  \booklettrue
\else\ifnum \strcmp{\mode}{a5}=0
  \avtrue
\else
  \aivtrue
\fi\fi\fi

\ifbooklet % do not enclose with {}
  \documentclass[french,twoside]{book} % ,notitlepage
  \usepackage[%
    papersize={105mm, 297mm},
    inner=12mm,
    outer=12mm,
    top=20mm,
    bottom=15mm,
    marginparsep=0pt,
  ]{geometry}
  \usepackage[fontsize=9.5pt]{scrextend} % for Roboto
\else\ifav
  \documentclass[french,twoside]{book} % ,notitlepage
  \usepackage[%
    a5paper,
    inner=25mm,
    outer=15mm,
    top=15mm,
    bottom=15mm,
    marginparsep=0pt,
  ]{geometry}
  \usepackage[fontsize=12pt]{scrextend}
\else% A4 2 cols
  \documentclass[twocolumn]{report}
  \usepackage[%
    a4paper,
    inner=15mm,
    outer=10mm,
    top=25mm,
    bottom=18mm,
    marginparsep=0pt,
  ]{geometry}
  \setlength{\columnsep}{20mm}
  \usepackage[fontsize=9.5pt]{scrextend}
\fi\fi

%%%%%%%%%%%%%%
% Alignments %
%%%%%%%%%%%%%%
% before teinte macros

\setlength{\arrayrulewidth}{0.2pt}
\setlength{\columnseprule}{\arrayrulewidth} % twocol
\setlength{\parskip}{0pt} % classical para with no margin
\setlength{\parindent}{1.5em}

%%%%%%%%%%
% Colors %
%%%%%%%%%%
% before Teinte macros

\usepackage[dvipsnames]{xcolor}
\definecolor{rubric}{HTML}{800000} % the tonic 0c71c3
\def\columnseprulecolor{\color{rubric}}
\colorlet{borderline}{rubric!30!} % definecolor need exact code
\definecolor{shadecolor}{gray}{0.95}
\definecolor{bghi}{gray}{0.5}

%%%%%%%%%%%%%%%%%
% Teinte macros %
%%%%%%%%%%%%%%%%%
%%%%%%%%%%%%%%%%%%%%%%%%%%%%%%%%%%%%%%%%%%%%%%%%%%%
% <TEI> generic (LaTeX names generated by Teinte) %
%%%%%%%%%%%%%%%%%%%%%%%%%%%%%%%%%%%%%%%%%%%%%%%%%%%
% This template is inserted in a specific design
% It is XeLaTeX and otf fonts

\makeatletter % <@@@


\usepackage{blindtext} % generate text for testing
\usepackage[strict]{changepage} % for modulo 4
\usepackage{contour} % rounding words
\usepackage[nodayofweek]{datetime}
% \usepackage{DejaVuSans} % seems buggy for sffont font for symbols
\usepackage{enumitem} % <list>
\usepackage{etoolbox} % patch commands
\usepackage{fancyvrb}
\usepackage{fancyhdr}
\usepackage{float}
\usepackage{fontspec} % XeLaTeX mandatory for fonts
\usepackage{footnote} % used to capture notes in minipage (ex: quote)
\usepackage{framed} % bordering correct with footnote hack
\usepackage{graphicx}
\usepackage{lettrine} % drop caps
\usepackage{lipsum} % generate text for testing
\usepackage[framemethod=tikz,]{mdframed} % maybe used for frame with footnotes inside
\usepackage{pdftexcmds} % needed for tests expressions
\usepackage{polyglossia} % non-break space french punct, bug Warning: "Failed to patch part"
\usepackage[%
  indentfirst=false,
  vskip=1em,
  noorphanfirst=true,
  noorphanafter=true,
  leftmargin=\parindent,
  rightmargin=0pt,
]{quoting}
\usepackage{ragged2e}
\usepackage{setspace} % \setstretch for <quote>
\usepackage{tabularx} % <table>
\usepackage[explicit]{titlesec} % wear titles, !NO implicit
\usepackage{tikz} % ornaments
\usepackage{tocloft} % styling tocs
\usepackage[fit]{truncate} % used im runing titles
\usepackage{unicode-math}
\usepackage[normalem]{ulem} % breakable \uline, normalem is absolutely necessary to keep \emph
\usepackage{verse} % <l>
\usepackage{xcolor} % named colors
\usepackage{xparse} % @ifundefined
\XeTeXdefaultencoding "iso-8859-1" % bad encoding of xstring
\usepackage{xstring} % string tests
\XeTeXdefaultencoding "utf-8"
\PassOptionsToPackage{hyphens}{url} % before hyperref, which load url package

% TOTEST
% \usepackage{hypcap} % links in caption ?
% \usepackage{marginnote}
% TESTED
% \usepackage{background} % doesn’t work with xetek
% \usepackage{bookmark} % prefers the hyperref hack \phantomsection
% \usepackage[color, leftbars]{changebar} % 2 cols doc, impossible to keep bar left
% \usepackage[utf8x]{inputenc} % inputenc package ignored with utf8 based engines
% \usepackage[sfdefault,medium]{inter} % no small caps
% \usepackage{firamath} % choose firasans instead, firamath unavailable in Ubuntu 21-04
% \usepackage{flushend} % bad for last notes, supposed flush end of columns
% \usepackage[stable]{footmisc} % BAD for complex notes https://texfaq.org/FAQ-ftnsect
% \usepackage{helvet} % not for XeLaTeX
% \usepackage{multicol} % not compatible with too much packages (longtable, framed, memoir…)
% \usepackage[default,oldstyle,scale=0.95]{opensans} % no small caps
% \usepackage{sectsty} % \chapterfont OBSOLETE
% \usepackage{soul} % \ul for underline, OBSOLETE with XeTeX
% \usepackage[breakable]{tcolorbox} % text styling gone, footnote hack not kept with breakable


% Metadata inserted by a program, from the TEI source, for title page and runing heads
\title{\textbf{ Éthique à Nicomaque }}
\date{-300}
\author{Aristote (-384, -322)}
\def\elbibl{Aristote (-384, -322). -300. \emph{Éthique à Nicomaque}}
\def\elsource{}

% Default metas
\newcommand{\colorprovide}[2]{\@ifundefinedcolor{#1}{\colorlet{#1}{#2}}{}}
\colorprovide{rubric}{red}
\colorprovide{silver}{lightgray}
\@ifundefined{syms}{\newfontfamily\syms{DejaVu Sans}}{}
\newif\ifdev
\@ifundefined{elbibl}{% No meta defined, maybe dev mode
  \newcommand{\elbibl}{Titre court ?}
  \newcommand{\elbook}{Titre du livre source ?}
  \newcommand{\elabstract}{Résumé\par}
  \newcommand{\elurl}{http://oeuvres.github.io/elbook/2}
  \author{Éric Lœchien}
  \title{Un titre de test assez long pour vérifier le comportement d’une maquette}
  \date{1566}
  \devtrue
}{}
\let\eltitle\@title
\let\elauthor\@author
\let\eldate\@date


\defaultfontfeatures{
  % Mapping=tex-text, % no effect seen
  Scale=MatchLowercase,
  Ligatures={TeX,Common},
}


% generic typo commands
\newcommand{\astermono}{\medskip\centerline{\color{rubric}\large\selectfont{\syms ✻}}\medskip\par}%
\newcommand{\astertri}{\medskip\par\centerline{\color{rubric}\large\selectfont{\syms ✻\,✻\,✻}}\medskip\par}%
\newcommand{\asterism}{\bigskip\par\noindent\parbox{\linewidth}{\centering\color{rubric}\large{\syms ✻}\\{\syms ✻}\hskip 0.75em{\syms ✻}}\bigskip\par}%

% lists
\newlength{\listmod}
\setlength{\listmod}{\parindent}
\setlist{
  itemindent=!,
  listparindent=\listmod,
  labelsep=0.2\listmod,
  parsep=0pt,
  % topsep=0.2em, % default topsep is best
}
\setlist[itemize]{
  label=—,
  leftmargin=0pt,
  labelindent=1.2em,
  labelwidth=0pt,
}
\setlist[enumerate]{
  label={\bf\color{rubric}\arabic*.},
  labelindent=0.8\listmod,
  leftmargin=\listmod,
  labelwidth=0pt,
}
\newlist{listalpha}{enumerate}{1}
\setlist[listalpha]{
  label={\bf\color{rubric}\alph*.},
  leftmargin=0pt,
  labelindent=0.8\listmod,
  labelwidth=0pt,
}
\newcommand{\listhead}[1]{\hspace{-1\listmod}\emph{#1}}

\renewcommand{\hrulefill}{%
  \leavevmode\leaders\hrule height 0.2pt\hfill\kern\z@}

% General typo
\DeclareTextFontCommand{\textlarge}{\large}
\DeclareTextFontCommand{\textsmall}{\small}

% commands, inlines
\newcommand{\anchor}[1]{\Hy@raisedlink{\hypertarget{#1}{}}} % link to top of an anchor (not baseline)
\newcommand\abbr[1]{#1}
\newcommand{\autour}[1]{\tikz[baseline=(X.base)]\node [draw=rubric,thin,rectangle,inner sep=1.5pt, rounded corners=3pt] (X) {\color{rubric}#1};}
\newcommand\corr[1]{#1}
\newcommand{\ed}[1]{ {\color{silver}\sffamily\footnotesize (#1)} } % <milestone ed="1688"/>
\newcommand\expan[1]{#1}
\newcommand\foreign[1]{\emph{#1}}
\newcommand\gap[1]{#1}
\renewcommand{\LettrineFontHook}{\color{rubric}}
\newcommand{\initial}[2]{\lettrine[lines=2, loversize=0.3, lhang=0.3]{#1}{#2}}
\newcommand{\initialiv}[2]{%
  \let\oldLFH\LettrineFontHook
  % \renewcommand{\LettrineFontHook}{\color{rubric}\ttfamily}
  \IfSubStr{QJ’}{#1}{
    \lettrine[lines=4, lhang=0.2, loversize=-0.1, lraise=0.2]{\smash{#1}}{#2}
  }{\IfSubStr{É}{#1}{
    \lettrine[lines=4, lhang=0.2, loversize=-0, lraise=0]{\smash{#1}}{#2}
  }{\IfSubStr{ÀÂ}{#1}{
    \lettrine[lines=4, lhang=0.2, loversize=-0, lraise=0, slope=0.6em]{\smash{#1}}{#2}
  }{\IfSubStr{A}{#1}{
    \lettrine[lines=4, lhang=0.2, loversize=0.2, slope=0.6em]{\smash{#1}}{#2}
  }{\IfSubStr{V}{#1}{
    \lettrine[lines=4, lhang=0.2, loversize=0.2, slope=-0.5em]{\smash{#1}}{#2}
  }{
    \lettrine[lines=4, lhang=0.2, loversize=0.2]{\smash{#1}}{#2}
  }}}}}
  \let\LettrineFontHook\oldLFH
}
\newcommand{\labelchar}[1]{\textbf{\color{rubric} #1}}
\newcommand{\milestone}[1]{\autour{\footnotesize\color{rubric} #1}} % <milestone n="4"/>
\newcommand\name[1]{#1}
\newcommand\orig[1]{#1}
\newcommand\orgName[1]{#1}
\newcommand\persName[1]{#1}
\newcommand\placeName[1]{#1}
\newcommand{\pn}[1]{\IfSubStr{-—–¶}{#1}% <p n="3"/>
  {\noindent{\bfseries\color{rubric}   ¶  }}
  {{\footnotesize\autour{ #1}  }}}
\newcommand\reg{}
% \newcommand\ref{} % already defined
\newcommand\sic[1]{#1}
\newcommand\surname[1]{\textsc{#1}}
\newcommand\term[1]{\textbf{#1}}

\def\mednobreak{\ifdim\lastskip<\medskipamount
  \removelastskip\nopagebreak\medskip\fi}
\def\bignobreak{\ifdim\lastskip<\bigskipamount
  \removelastskip\nopagebreak\bigskip\fi}

% commands, blocks
\newcommand{\byline}[1]{\bigskip{\RaggedLeft{#1}\par}\bigskip}
\newcommand{\bibl}[1]{{\RaggedLeft{#1}\par\bigskip}}
\newcommand{\biblitem}[1]{{\noindent\hangindent=\parindent   #1\par}}
\newcommand{\dateline}[1]{\medskip{\RaggedLeft{#1}\par}\bigskip}
\newcommand{\labelblock}[1]{\medbreak{\noindent\color{rubric}\bfseries #1}\par\mednobreak}
\newcommand{\salute}[1]{\bigbreak{#1}\par\medbreak}
\newcommand{\signed}[1]{\bigbreak\filbreak{\raggedleft #1\par}\medskip}

% environments for blocks (some may become commands)
\newenvironment{borderbox}{}{} % framing content
\newenvironment{citbibl}{\ifvmode\hfill\fi}{\ifvmode\par\fi }
\newenvironment{docAuthor}{\ifvmode\vskip4pt\fontsize{16pt}{18pt}\selectfont\fi\itshape}{\ifvmode\par\fi }
\newenvironment{docDate}{}{\ifvmode\par\fi }
\newenvironment{docImprint}{\vskip6pt}{\ifvmode\par\fi }
\newenvironment{docTitle}{\vskip6pt\bfseries\fontsize{18pt}{22pt}\selectfont}{\par }
\newenvironment{msHead}{\vskip6pt}{\par}
\newenvironment{msItem}{\vskip6pt}{\par}
\newenvironment{titlePart}{}{\par }


% environments for block containers
\newenvironment{argument}{\itshape\parindent0pt}{\vskip1.5em}
\newenvironment{biblfree}{}{\ifvmode\par\fi }
\newenvironment{bibitemlist}[1]{%
  \list{\@biblabel{\@arabic\c@enumiv}}%
  {%
    \settowidth\labelwidth{\@biblabel{#1}}%
    \leftmargin\labelwidth
    \advance\leftmargin\labelsep
    \@openbib@code
    \usecounter{enumiv}%
    \let\p@enumiv\@empty
    \renewcommand\theenumiv{\@arabic\c@enumiv}%
  }
  \sloppy
  \clubpenalty4000
  \@clubpenalty \clubpenalty
  \widowpenalty4000%
  \sfcode`\.\@m
}%
{\def\@noitemerr
  {\@latex@warning{Empty `bibitemlist' environment}}%
\endlist}
\newenvironment{quoteblock}% may be used for ornaments
  {\begin{quoting}}
  {\end{quoting}}

% table () is preceded and finished by custom command
\newcommand{\tableopen}[1]{%
  \ifnum\strcmp{#1}{wide}=0{%
    \begin{center}
  }
  \else\ifnum\strcmp{#1}{long}=0{%
    \begin{center}
  }
  \else{%
    \begin{center}
  }
  \fi\fi
}
\newcommand{\tableclose}[1]{%
  \ifnum\strcmp{#1}{wide}=0{%
    \end{center}
  }
  \else\ifnum\strcmp{#1}{long}=0{%
    \end{center}
  }
  \else{%
    \end{center}
  }
  \fi\fi
}


% text structure
\newcommand\chapteropen{} % before chapter title
\newcommand\chaptercont{} % after title, argument, epigraph…
\newcommand\chapterclose{} % maybe useful for multicol settings
\setcounter{secnumdepth}{-2} % no counters for hierarchy titles
\setcounter{tocdepth}{5} % deep toc
\markright{\@title} % ???
\markboth{\@title}{\@author} % ???
\renewcommand\tableofcontents{\@starttoc{toc}}
% toclof format
% \renewcommand{\@tocrmarg}{0.1em} % Useless command?
% \renewcommand{\@pnumwidth}{0.5em} % {1.75em}
\renewcommand{\@cftmaketoctitle}{}
\setlength{\cftbeforesecskip}{\z@ \@plus.2\p@}
\renewcommand{\cftchapfont}{}
\renewcommand{\cftchapdotsep}{\cftdotsep}
\renewcommand{\cftchapleader}{\normalfont\cftdotfill{\cftchapdotsep}}
\renewcommand{\cftchappagefont}{\bfseries}
\setlength{\cftbeforechapskip}{0em \@plus\p@}
% \renewcommand{\cftsecfont}{\small\relax}
\renewcommand{\cftsecpagefont}{\normalfont}
% \renewcommand{\cftsubsecfont}{\small\relax}
\renewcommand{\cftsecdotsep}{\cftdotsep}
\renewcommand{\cftsecpagefont}{\normalfont}
\renewcommand{\cftsecleader}{\normalfont\cftdotfill{\cftsecdotsep}}
\setlength{\cftsecindent}{1em}
\setlength{\cftsubsecindent}{2em}
\setlength{\cftsubsubsecindent}{3em}
\setlength{\cftchapnumwidth}{1em}
\setlength{\cftsecnumwidth}{1em}
\setlength{\cftsubsecnumwidth}{1em}
\setlength{\cftsubsubsecnumwidth}{1em}

% footnotes
\newif\ifheading
\newcommand*{\fnmarkscale}{\ifheading 0.70 \else 1 \fi}
\renewcommand\footnoterule{\vspace*{0.3cm}\hrule height \arrayrulewidth width 3cm \vspace*{0.3cm}}
\setlength\footnotesep{1.5\footnotesep} % footnote separator
\renewcommand\@makefntext[1]{\parindent 1.5em \noindent \hb@xt@1.8em{\hss{\normalfont\@thefnmark . }}#1} % no superscipt in foot
\patchcmd{\@footnotetext}{\footnotesize}{\footnotesize\sffamily}{}{} % before scrextend, hyperref


%   see https://tex.stackexchange.com/a/34449/5049
\def\truncdiv#1#2{((#1-(#2-1)/2)/#2)}
\def\moduloop#1#2{(#1-\truncdiv{#1}{#2}*#2)}
\def\modulo#1#2{\number\numexpr\moduloop{#1}{#2}\relax}

% orphans and widows
\clubpenalty=9996
\widowpenalty=9999
\brokenpenalty=4991
\predisplaypenalty=10000
\postdisplaypenalty=1549
\displaywidowpenalty=1602
\hyphenpenalty=400
% Copied from Rahtz but not understood
\def\@pnumwidth{1.55em}
\def\@tocrmarg {2.55em}
\def\@dotsep{4.5}
\emergencystretch 3em
\hbadness=4000
\pretolerance=750
\tolerance=2000
\vbadness=4000
\def\Gin@extensions{.pdf,.png,.jpg,.mps,.tif}
% \renewcommand{\@cite}[1]{#1} % biblio

\usepackage{hyperref} % supposed to be the last one, :o) except for the ones to follow
\urlstyle{same} % after hyperref
\hypersetup{
  % pdftex, % no effect
  pdftitle={\elbibl},
  % pdfauthor={Your name here},
  % pdfsubject={Your subject here},
  % pdfkeywords={keyword1, keyword2},
  bookmarksnumbered=true,
  bookmarksopen=true,
  bookmarksopenlevel=1,
  pdfstartview=Fit,
  breaklinks=true, % avoid long links
  pdfpagemode=UseOutlines,    % pdf toc
  hyperfootnotes=true,
  colorlinks=false,
  pdfborder=0 0 0,
  % pdfpagelayout=TwoPageRight,
  % linktocpage=true, % NO, toc, link only on page no
}

\makeatother % /@@@>
%%%%%%%%%%%%%%
% </TEI> end %
%%%%%%%%%%%%%%


%%%%%%%%%%%%%
% footnotes %
%%%%%%%%%%%%%
\renewcommand{\thefootnote}{\bfseries\textcolor{rubric}{\arabic{footnote}}} % color for footnote marks

%%%%%%%%%
% Fonts %
%%%%%%%%%
\usepackage[]{roboto} % SmallCaps, Regular is a bit bold
% \linespread{0.90} % too compact, keep font natural
\newfontfamily\fontrun[]{Roboto Condensed Light} % condensed runing heads
\ifav
  \setmainfont[
    ItalicFont={Roboto Light Italic},
  ]{Roboto}
\else\ifbooklet
  \setmainfont[
    ItalicFont={Roboto Light Italic},
  ]{Roboto}
\else
\setmainfont[
  ItalicFont={Roboto Italic},
]{Roboto Light}
\fi\fi
\renewcommand{\LettrineFontHook}{\bfseries\color{rubric}}
% \renewenvironment{labelblock}{\begin{center}\bfseries\color{rubric}}{\end{center}}

%%%%%%%%
% MISC %
%%%%%%%%

\setdefaultlanguage[frenchpart=false]{french} % bug on part


\newenvironment{quotebar}{%
    \def\FrameCommand{{\color{rubric!10!}\vrule width 0.5em} \hspace{0.9em}}%
    \def\OuterFrameSep{\itemsep} % séparateur vertical
    \MakeFramed {\advance\hsize-\width \FrameRestore}
  }%
  {%
    \endMakeFramed
  }
\renewenvironment{quoteblock}% may be used for ornaments
  {%
    \savenotes
    \setstretch{0.9}
    \normalfont
    \begin{quotebar}
  }
  {%
    \end{quotebar}
    \spewnotes
  }


\renewcommand{\headrulewidth}{\arrayrulewidth}
\renewcommand{\headrule}{{\color{rubric}\hrule}}

% delicate tuning, image has produce line-height problems in title on 2 lines
\titleformat{name=\chapter} % command
  [display] % shape
  {\vspace{1.5em}\centering} % format
  {} % label
  {0pt} % separator between n
  {}
[{\color{rubric}\huge\textbf{#1}}\bigskip] % after code
% \titlespacing{command}{left spacing}{before spacing}{after spacing}[right]
\titlespacing*{\chapter}{0pt}{-2em}{0pt}[0pt]

\titleformat{name=\section}
  [block]{}{}{}{}
  [\vbox{\color{rubric}\large\raggedleft\textbf{#1}}]
\titlespacing{\section}{0pt}{0pt plus 4pt minus 2pt}{\baselineskip}

\titleformat{name=\subsection}
  [block]
  {}
  {} % \thesection
  {} % separator \arrayrulewidth
  {}
[\vbox{\large\textbf{#1}}]
% \titlespacing{\subsection}{0pt}{0pt plus 4pt minus 2pt}{\baselineskip}

\ifaiv
  \fancypagestyle{main}{%
    \fancyhf{}
    \setlength{\headheight}{1.5em}
    \fancyhead{} % reset head
    \fancyfoot{} % reset foot
    \fancyhead[L]{\truncate{0.45\headwidth}{\fontrun\elbibl}} % book ref
    \fancyhead[R]{\truncate{0.45\headwidth}{ \fontrun\nouppercase\leftmark}} % Chapter title
    \fancyhead[C]{\thepage}
  }
  \fancypagestyle{plain}{% apply to chapter
    \fancyhf{}% clear all header and footer fields
    \setlength{\headheight}{1.5em}
    \fancyhead[L]{\truncate{0.9\headwidth}{\fontrun\elbibl}}
    \fancyhead[R]{\thepage}
  }
\else
  \fancypagestyle{main}{%
    \fancyhf{}
    \setlength{\headheight}{1.5em}
    \fancyhead{} % reset head
    \fancyfoot{} % reset foot
    \fancyhead[RE]{\truncate{0.9\headwidth}{\fontrun\elbibl}} % book ref
    \fancyhead[LO]{\truncate{0.9\headwidth}{\fontrun\nouppercase\leftmark}} % Chapter title, \nouppercase needed
    \fancyhead[RO,LE]{\thepage}
  }
  \fancypagestyle{plain}{% apply to chapter
    \fancyhf{}% clear all header and footer fields
    \setlength{\headheight}{1.5em}
    \fancyhead[L]{\truncate{0.9\headwidth}{\fontrun\elbibl}}
    \fancyhead[R]{\thepage}
  }
\fi

\ifav % a5 only
  \titleclass{\section}{top}
\fi

\newcommand\chapo{{%
  \vspace*{-3em}
  \centering % no vskip ()
  {\Large\addfontfeature{LetterSpace=25}\bfseries{\elauthor}}\par
  \smallskip
  {\large\eldate}\par
  \bigskip
  {\Large\selectfont{\eltitle}}\par
  \bigskip
  {\color{rubric}\hline\par}
  \bigskip
  {\Large TEXTE LIBRE À PARTICPATION LIBRE\par}
  \centerline{\small\color{rubric} {hurlus.fr, tiré le \today}}\par
  \bigskip
}}

\newcommand\cover{{%
  \thispagestyle{empty}
  \centering
  {\LARGE\bfseries{\elauthor}}\par
  \bigskip
  {\Large\eldate}\par
  \bigskip
  \bigskip
  {\LARGE\selectfont{\eltitle}}\par
  \vfill\null
  {\color{rubric}\setlength{\arrayrulewidth}{2pt}\hline\par}
  \vfill\null
  {\Large TEXTE LIBRE À PARTICPATION LIBRE\par}
  \centerline{{\href{https://hurlus.fr}{\dotuline{hurlus.fr}}, tiré le \today}}\par
}}

\begin{document}
\pagestyle{empty}
\ifbooklet{
  \cover\newpage
  \thispagestyle{empty}\hbox{}\newpage
  \cover\newpage\noindent Les voyages de la brochure\par
  \bigskip
  \begin{tabularx}{\textwidth}{l|X|X}
    \textbf{Date} & \textbf{Lieu}& \textbf{Nom/pseudo} \\ \hline
    \rule{0pt}{25cm} &  &   \\
  \end{tabularx}
  \newpage
  \addtocounter{page}{-4}
}\fi

\thispagestyle{empty}
\ifaiv
  \twocolumn[\chapo]
\else
  \chapo
\fi
{\it\elabstract}
\bigskip
\makeatletter\@starttoc{toc}\makeatother % toc without new page
\bigskip

\pagestyle{main} % after style

  \section[{Livre I}]{Livre I}\renewcommand{\leftmark}{Livre I}

\subsection[{1 (1094a — 1095a) < Le bien et l’activité humaine. La hiérarchie des biens >}]{1 (1094a — 1095a) < Le bien et l’activité humaine. La hiérarchie des biens >}
\noindent  Tout art [τεχνη] et toute investigation, et pareillement toute action [πραξις] et tout choix tendent vers quelque bien, à ce qu’il semble [δοκει]. Aussi a-t-on déclaré avec raison que le Bien est ce à quoi toutes choses tendent.\par
Mais on observe, en fait, une certaine différence entre les fins : les unes consistent dans des activités, et les autres dans \\
certaines œuvres, distinctes des activités elles-mêmes. Et là où existent certaines fins distinctes des actions, dans ces cas-là les œuvres sont par nature supérieures aux activités qui les produisent.\par
Or, comme il y a multiplicité d’actions, d’arts et de sciences, leurs fins aussi sont multiples : ainsi l’art médical a pour fin la santé, l’art de construire des vaisseaux le navire, l’art stratégique la victoire, et l’art économique la richesse. \\
Mais dans tous les arts de ce genre qui relèvent d’une unique potentialité (de même, en effet, que sous l’art hippique tombent l’art de fabriquer des freins et tous les autres métiers concernant le harnachement des chevaux, et que l’art hippique lui-même et toute action se rapportant à la guerre tombent à leur tour sous l’art stratégique, c’est de la même façon que d’autres arts sont subordonnés à d’autres), dans tous ces cas, disons-nous, les fins des arts architectoniques doivent être \\
préférées à toutes celles des arts subordonnés, puisque c’est en vue des premières fins qu’on poursuit les autres. Peu importe, au surplus, que les activités elles-mêmes soient les fins des actions, ou que, à part de ces activités, il y ait quelque autre chose, comme dans le cas des sciences dont nous avons parlé.\par
Si donc il y a, de nos activités, quelque fin que nous souhaitons par elle-même, et les autres seulement à cause \\
d’elle, et si nous ne choisissons pas indéfiniment une chose en vue d’une autre (car on procéderait ainsi à l’infini, de sorte que le désir serait futile et vain), il est clair que cette fin-là ne saurait être que le bien, le Souverain Bien. N’est-il pas vrai dès lors que, pour la conduite de la vie, la connaissance de ce bien est d’un grand poids, et que, semblables à des archers qui ont une cible sous les yeux, nous pourrons plus aisément atteindre le but qui convient ? S’il en est ainsi, nous devons \\
essayer d’embrasser, tout au moins dans ses grandes lignes, la nature du Souverain Bien, et de dire de quelle science particulière ou de quelle potentialité il relève. On sera d’avis qu’il dépend de la science suprême et architectonique par excellence. Or une telle science est manifestement la Politique, car c’est elle qui dispose quelles sont parmi les sciences celles qui  sont nécessaires dans les cités, et quelles sortes de sciences chaque classe de citoyens doit apprendre, et jusqu’à quel point l’étude en sera poussée ; et nous voyons encore que même les potentialités les plus appréciées sont subordonnées à la Politique : par exemple la stratégie, l’économique, la rhétorique. Et puisque la Politique se sert des autres sciences pratiques, et \\
qu’en outre elle légifère sur ce qu’il faut faire et sur ce dont il faut s’abstenir, la fin de cette science englobera les fins des autres sciences ; d’où il résulte que la fin de la Politique sera le bien proprement humain. Même si, en effet, il y a identité entre le bien de l’individu et celui de la cité, de toute façon c’est une tâche manifestement plus importante et plus parfaite d’appréhender et de sauvegarder le bien de la cité : car le bien est assurément aimable même pour un individu isolé, mais il \\
est plus beau et plus divin appliqué à une nation ou à des cités.\par
Voilà donc les buts de notre enquête, qui constitue une forme de politique.\par
Nous aurons suffisamment rempli notre tâche si nous donnons les éclaircissements que comporte la nature du sujet que nous traitons. C’est qu’en effet on ne doit pas chercher la même rigueur dans toutes les discussions indifféremment, pas plus qu’on ne l’exige dans les productions de l’art. Les choses belles et les choses justes qui sont l’objet de la Politique, \\
donnent lieu à de telles divergences et à de telles incertitudes qu’on a pu croire qu’elles existaient seulement par convention et non par nature. Une pareille incertitude se présente aussi dans le cas des biens de la vie, en raison des dommages qui en découlent souvent : on a vu, en effet, des gens périr par leur richesse, et d’autres périr par leur courage. On doit donc se contenter, en traitant de pareils sujets et partant de pareils principes, \\
de montrer la vérité d’une façon grossière et approchée ; et quand on parle de choses simplement constantes et qu’on part de principes également constants, on ne peut aboutir qu’à des conclusions de même caractère. C’est dans le même esprit, dès lors, que devront être accueillies les diverses vues que nous émettons : car il est d’un homme cultivé de ne chercher la rigueur pour chaque genre de choses que dans la \\
mesure où la nature du sujet l’admet : il est évidemment à peu près aussi déraisonnable d’accepter d’un mathématicien des raisonnements probables que d’exiger d’un rhéteur des démonstrations proprement dites. D’autre part, chacun juge correctement de ce qu’il connaît, et en ce domaine il est bon juge. Ainsi donc, dans un domaine  déterminé, juge bien celui qui a reçu une éducation appropriée, tandis que, dans une matière excluant toute spécialisation, le bon juge est celui qui a reçu une culture générale. Aussi le jeune homme n’est-il pas un auditeur bien propre à des leçons de Politique, car il n’a aucune expérience des choses de la vie, qui sont pourtant le point de départ et l’objet des raisonnements de cette science. De plus, étant enclin à suivre ses \\
passions, il ne retirera de cette étude rien d’utile ni de profitable, puisque la Politique a pour fin, non pas la connaissance, mais l’action. Peu importe, du reste, qu’on soit jeune par l’âge ou jeune par le caractère : l’insuffisance à cet égard n’est pas une question de temps, mais elle est due au fait qu’on vit au gré de ses passions et qu’on s’élance à la poursuite de tout ce qu’on voit. Pour des étourdis de cette sorte, la connaissance ne sert à rien, pas plus que pour les intempérants ; pour ceux, au \\
contraire, dont les désirs et les actes sont conformes à la raison, le savoir en ces matières sera pour eux d’un grand profit.
\subsection[{2 (1095a — 1095b) < Le bonheur ; diverses opinions sur sa nature. Méthode à employer >}]{2 (1095a — 1095b) < Le bonheur ; diverses opinions sur sa nature. Méthode à employer >}
\noindent En ce qui regarde l’auditeur ainsi que la manière dont notre enseignement doit être reçu et l’objet que nous nous proposons de traiter, toutes ces choses-là doivent constituer une introduction suffisante.\par
Revenons maintenant en arrière. Puisque toute connaissance, \\
tout choix délibéré aspire à quelque bien, voyons quel est selon nous le bien où tend la Politique, autrement dit quel est de tous les biens réalisables celui qui est le Bien suprême. Sur son nom, en tout cas, la plupart des hommes sont pratiquement d’accord : c’est le bonheur, au dire de la foule aussi bien que des gens cultivés ; tous assimilent le fait de bien vivre \\
et de réussir au fait d’être heureux. Par contre, en ce qui concerne la nature du bonheur, on ne s’entend plus, et les réponses de la foule ne ressemblent pas à celles des sages. Les uns, en effet, identifient le bonheur à quelque chose d’apparent et de visible, comme le plaisir, la richesse ou l’honneur : pour les uns c’est une chose et pour les autres une autre chose ; souvent le même homme change d’avis à son sujet : malade, il place le bonheur dans la santé, et pauvre, dans la richesse ; à \\
d’autres moments, quand on a conscience de sa propre ignorance, on admire ceux qui tiennent des discours élevés et dépassant notre portée. Certains, enfin, pensent qu’en dehors de tous ces biens multiples il y a un autre bien qui existe par soi et qui est pour tous ces biens-là cause de leur bonté. Passer en revue la totalité de ces opinions est sans doute assez vain ; il suffit de s’arrêter à celles qui sont le plus répandues ou qui \\
paraissent avoir quelque fondement rationnel.\par
N’oublions pas la différence qui existe entre les raisonnements qui partent des principes et ceux qui remontent aux principes. C’est en effet à juste titre que Platon se posait la question, et qu’il recherchait si la marche à suivre est de partir des principes ou de remonter aux principes, tout comme dans  le stade les coureurs vont des athlothètes à la borne, ou inversement. Il faut, en effet, partir des choses connues, et une chose est dite connue en deux sens, soit pour nous, soit d’une manière absolue. Sans doute devons-nous partir des choses qui sont connues pour nous. C’est la raison pour laquelle il faut avoir été élevé dans des mœurs honnêtes, quand on se dispose à \\
écouter avec profit un enseignement portant sur l’honnête, le juste, et d’une façon générale sur tout ce qui a trait à la Politique (car ici le point de départ est le fait, et si le fait était suffisamment clair, nous serions dispensés de connaître en sus le pourquoi). Or l’auditeur tel que nous le caractérisons, ou bien est déjà en possession des principes, ou bien est capable de les recevoir facilement. Quant à celui qui ne les possède d’aucune de ces deux façons, qu’on le renvoie aux paroles d’Hésiode :\par
 \\
 {\itshape Celui-là est absolument parfait qui de lui-même réfléchit sur toutes choses.} \par
 {\itshape Est sensé encore celui qui se rend aux bons conseils qu’on lui donne.} \par
{\itshape Quant à celui qui ne sait ni réfléchir par lui-même, ni, en écoutant les leçons d’autrui}, \par
 {\itshape Les accueillir dans son cœur, celui-là en revanche est un homme bon à rien.} 
\subsection[{3 (1095b — 1096b) < Les théories courantes sur la nature du bonheur : le plaisir, l’honneur, la richesse >}]{3 (1095b — 1096b) < Les théories courantes sur la nature du bonheur : le plaisir, l’honneur, la richesse >}
\noindent \\
Nous revenons au point d’où nous nous sommes écartés. Les hommes, et il ne faut pas s’en étonner, paraissent concevoir le bien et le bonheur d’après la vie qu’ils mènent. La foule et les gens les plus grossiers disent que c’est le plaisir : c’est la raison pour laquelle ils ont une préférence pour la vie de jouissance. C’est qu’en effet les principaux types de vie sont au nombre de trois : celle dont nous venons de parler, la vie politique, \\
et en troisième lieu la vie contemplative. — La foule se montre véritablement d’une bassesse d’esclave en optant pour une vie bestiale, mais elle trouve son excuse dans le fait que beaucoup de ceux qui appartiennent à la classe dirigeante ont les mêmes goûts qu’un Sardanapale. — Les gens cultivés, et qui aiment la vie active, préfèrent l’honneur, car c’est là, à tout prendre, la fin de la vie politique. Mais l’honneur apparaît comme une chose trop superficielle pour être l’objet cherché, car, de l’avis général. il dépend plutôt de ceux qui honorent que \\
de celui qui est honoré ; or nous savons d’instinct que le bien est quelque chose de personnel à chacun et qu’on peut difficilement nous ravir. En outre, il semble bien que l’on poursuit l’honneur en vue seulement de se persuader de son propre mérite ; en tout cas, on cherche à être honoré par les hommes sensés et auprès de ceux dont on est connu, et on veut l’être pour son excellence. Il est clair, dans ces conditions, que, tout \\
au moins aux yeux de ceux qui agissent ainsi, la vertu l’emporte sur l’honneur. Peut-être pourrait-on aussi supposer que c’est la vertu plutôt que l’honneur qui est la fin de la vie politique. Mais la vertu [αρετη] apparaît bien, elle aussi, insuffisante, car il peut se faire, semble-t-il. que, possédant la vertu, on  passe sa vie entière à dormir ou à ne rien faire, ou même, bien plus, à supporter les plus grands maux et les pires infortunes. Or nul ne saurait déclarer heureux l’homme vivant ainsi, à moins de vouloir maintenir à tout prix une thèse. Mais sur ce sujet en voilà assez (il a été suffisamment traité, même dans les discussions courantes).\par
\\
Le troisième genre de vie, c’est la vie contemplative, dont nous entreprendrons l’examen par la suite.\par
Quant à la vie de l’homme d’affaires, c’est une vie de contrainte, et la richesse n’est évidemment pas le bien que nous cherchons : c’est seulement une chose utile, un moyen en vue d’une autre chose. Aussi vaudrait-il encore mieux prendre pour fins celles dont nous avons parlé précédemment, puisqu’elles sont aimées pour elles-mêmes. Mais il est manifeste que ce ne \\
sont pas non plus ces fins-là, en dépit de nombreux arguments qu’on a répandus en leur faveur.
\subsection[{4 (1096b — 1097a) < Critique de la théorie platonicienne de l’Idée du Bien >}]{4 (1096b — 1097a) < Critique de la théorie platonicienne de l’Idée du Bien >}
\noindent Laissons tout cela. Il vaut mieux sans doute faire porter notre examen sur le Bien pris en général, et instituer une discussion sur ce qu’on entend par là, bien qu’une recherche de ce genre soit rendue difficile du fait que ce sont des amis qui ont introduit la doctrine des Idées. Mais on admettra peut-être qu’il est préférable, et c’est aussi pour nous une obligation, si \\
nous voulons du moins sauvegarder la vérité, de sacrifier même nos sentiments personnels, surtout quand on est philosophe : vérité et amitié nous sont chères l’une et l’autre, mais c’est pour nous un devoir sacré d’accorder la préférence à la vérité.\par
Ceux qui ont apporté l’opinion dont nous parlons ne constituaient pas d’Idées des choses dans lesquelles ils admettaient de l’antérieur et du postérieur (et c’est la raison pour laquelle ils n’établissaient pas non plus d’Idée des nombres). \\
Or le Bien s’affirme et dans l’essence et dans la qualité et dans la relation ; mais ce qui est en soi, la substance, possède une antériorité naturelle à la relation (laquelle est semblable à un rejeton et à un accident de l’Être). Il en résulte qu’il ne saurait y avoir quelque Idée commune pour ces choses-là.\par
En outre, puisque le Bien s’affirme d’autant de façons que l’Être (car il se dit dans la substance, par exemple Dieu et \\
l’intellect, dans la qualité, comme les vertus, dans la quantité, comme la juste mesure, dans la relation, comme l’utile, dans le temps, comme l’occasion, dans le lieu, comme l’habitat, et ainsi de suite), il est clair qu’il ne saurait être quelque chose de commun, de général et d’un : car s’il l’était, il ne s’affirmerait pas de toutes les catégories, mais d’une seule.\par
\\
De plus, puisque des choses tombant sous une seule Idée il n’y a aussi qu’une seule science, de tous les biens sans exception il ne devrait y avoir également qu’une science unique : or, en fait, les biens sont l’objet d’une multiplicité de sciences, même ceux qui tombent sous une seule catégorie : ainsi pour l’occasion, dans la guerre il y a la stratégie, et dans la maladie, la médecine ; pour la juste mesure, dans l’alimentation c’est la médecine, et dans les exercices fatigants c’est la gymnastique.\par
On pourrait se demander encore ce qu’en fin de compte \\
les Platoniciens veulent dire par la Chose en soi, s’il est vrai  que l’Homme en soi et l’homme répondent à une seule et même définition, à savoir celle de l’homme, car en tant qu’il s’agit de la notion d’homme il n’y aura aucune différence entre les deux cas. Mais s’il en est ainsi, il faudra en dire autant du Bien. Et ce n’est pas non plus parce qu’on l’aura rendu éternel que le Bien en soi sera davantage un bien, puisque \\
une blancheur de longue durée n’est pas plus blanche qu’une blancheur éphémère. À cet égard les Pythagoriciens donnent l’impression de parler du Bien d’une façon plus plausible en posant l’Un dans la colonne des biens, et c’est d’ailleurs eux que Speusippe semble avoir suivis. Mais tous ces points doivent faire l’objet d’une autre discussion.\par
Quant à ce que nous avons dit ci-dessus, une incertitude se laisse entrevoir, du fait que les Platoniciens n’ont pas visé dans \\
leurs paroles tous les biens, mais que seuls dépendent d’une Idée unique les biens qui sont poursuivis et aimés pour eux-mêmes, tandis que les biens qui assurent la production des premiers, ou leur conservation d’une façon ou d’une autre, ou encore qui empêchent l’action de leurs contraires, ne sont appelés des biens qu’à cause des premiers, et dans un sens secondaire. Évidemment alors, les biens seraient \\
entendus en un double sens : d’une part, les choses qui sont des biens par elles-mêmes, et, d’autre part, celles qui ne sont des biens qu’en \\
raison des précédentes. Ayant donc séparé les biens par eux-mêmes des biens simplement utiles, examinons si ces biens par soi sont appelés biens par référence à une Idée unique. Quelles sont les sortes de choses que nous devrons poser comme des biens en soi ? Est-ce celles qu’on poursuit même isolées de tout le reste, comme la prudence, la vision, certains plaisirs et certains honneurs ? Ces biens-là, en effet, même si nous les poursuivons en vue de quelque autre chose, on n’en doit pas moins les poser dans la classe des biens en soi. Ou bien est-ce qu’il n’y a aucun autre bien en soi que l’Idée du \\
Bien ? Il en résultera dans ce cas que la forme du Bien sera quelque chose de vide. Si on veut, au contraire, que les choses désignées plus haut fassent aussi partie des biens en soi, il faudra que la notion du Bien en soi se montre comme quelque chose d’identique en elles toutes, comme dans la neige et la céruse se retrouve la notion de la blancheur. Mais l’honneur, la prudence et le plaisir ont des définitions distinctes, et qui \\
diffèrent précisément sous le rapport de la bonté elle-même. Le bien n’est donc pas quelque élément commun dépendant d’une Idée unique.\par
Mais alors en quel sens les biens sont-ils appelés du nom de {\itshape bien} ? Il ne semble pas, en tout cas, qu’on ait affaire à des homonymes accidentels. L’homonymie provient-elle alors de ce que tous les biens dérivent d’un seul bien ou de ce qu’ils concourent tous à un seul bien ? Ne s’agirait-il pas plutôt d’une unité d’analogie : ainsi, ce que la vue est au corps, l’intellect l’est à l’âme, et de même pour d’autres analogies ? \\
Mais sans doute sont-ce là des questions à laisser de côté pour le moment, car leur examen détaillé serait plus approprié à une autre branche de la philosophie. Même raison d’écarter ce qui a rapport à l’Idée. En admettant même, en effet, qu’il y ait un seul Bien comme prédicat commun à tous les biens, ou possédant l’existence séparée et par soi, il est évident qu’il ne serait ni praticable, ni accessible à l’homme, alors que le bien que \\
nous cherchons présentement c’est quelque chose qui soit à notre portée. Peut-être pourrait-on croire qu’il est tout de  même préférable de connaître le Bien en soi, en vue de ces biens qui sont pour nous accessibles et réalisables : ayant ainsi comme un modèle sous les yeux, nous connaîtrons plus aisément, dira-t-on, les biens qui sont à notre portée, et si nous les connaissons, nous les atteindrons. Cet argument n’est pas sans quelque apparence de raison, mais il semble en désaccord avec la façon dont procèdent les sciences : si toutes les sciences en \\
effet, tendent à quelque bien et cherchent à combler ce qui les en sépare encore, elles laissent de côté la connaissance du Bien en soi. Et pourtant ! Que tous les gens de métier ignorent un secours d’une telle importance et ne cherchent même pas à l’acquérir, voilà qui n’est guère vraisemblable ! On se demande aussi quel avantage un tisserand ou un charpentier retirera pour son art de la connaissance de ce Bien en soi, ou comment sera \\
meilleur médecin ou meilleur général celui qui aura contemplé l’Idée en elle-même : il est manifeste que ce n’est pas de cette façon-là que le médecin observe la santé, mais c’est la santé de l’être humain qu’il observe, ou même plutôt sans doute la santé de tel homme déterminé, car c’est l’individu qui fait l’objet de ses soins.
\subsection[{5 (1097a — 1097b) < Nature du bien : fin parfaite, qui se suffit à elle-même >}]{5 (1097a — 1097b) < Nature du bien : fin parfaite, qui se suffit à elle-même >}
\noindent \\
Tous ces points ont été suffisamment traités. — Revenons encore une fois sur le bien qui fait l’objet de nos recherches, et demandons-nous ce qu’enfin il peut être. Le bien, en effet, nous apparaît comme une chose dans telle action ou tel art, et comme une autre chose dans telle autre action ou tel autre art : il est autre en médecine qu’il n’est en stratégie, et ainsi de suite pour le reste des arts. Quel est donc le bien dans chacun de ces cas ? N’est-ce pas la fin en vue de quoi tout le reste est effectué ? C’est en médecine la santé, en stratégie la victoire, dans l’art de \\
bâtir, une maison, dans un autre art c’est une autre chose, mais dans toute action, dans tout choix, le bien c’est la fin, car c’est en vue de cette fin qu’on accomplit toujours le reste. Par conséquent, s’il y a quelque chose qui soit fin de tous nos actes, c’est cette chose-là qui sera le bien réalisable, et s’il y a plusieurs choses, ce seront ces choses-là.\par
Voilà donc que par un cours différent, l’argument aboutit au même résultat qu’auparavant. — Mais ce que nous disons là, nous devons tenter de le rendre encore plus clair.\par
\\
Puisque les fins sont manifestement multiples, et nous choisissons certaines d’entre elles (par exemple la richesse, les flûtes et en général les instruments) en vue d’autres choses, il est clair que ce ne sont pas là des fins parfaites, alors que le Souverain Bien est, de toute évidence, quelque chose de parfait. Il en résulte que s’il y a une seule chose qui soit une fin \\
parfaite, elle sera le bien que nous cherchons, et s’il y en a plusieurs, ce sera la plus parfaite d’entre elles. Or, ce qui est digne d’être poursuivi par soi, nous le nommons plus parfait que ce qui est poursuivi pour une autre chose, et ce qui n’est jamais désirable en vue d’une autre chose, nous le déclarons plus parfait que les choses qui sont désirables à la fois par elles-mêmes et pour cette autre chose, et nous appelons parfait au sens absolu ce qui est toujours désirable en soi-même et ne l’est jamais en vue d’une autre chose. Or le bonheur semble être au  suprême degré une fin de ce genre, car nous le choisissons toujours pour lui-même et jamais en vue d’une autre chose : au contraire, l’honneur, le plaisir, l’intelligence ou toute vertu quelconque, sont des biens que nous choisissons assurément pour eux-mêmes (puisque, même si aucun avantage n’en découlait pour nous, nous les choisirions encore), mais nous \\
les choisissons aussi en vue du bonheur, car c’est par leur intermédiaire que nous pensons devenir heureux. Par contre, le bonheur n’est jamais choisi en vue de ces biens, ni d’une manière générale en vue d’autre chose que lui-même.\par
On peut se rendre compte encore qu’en partant de la notion de suffisance [αυταρκεια] on arrive à la même conclusion. Le bien parfait semble en effet se suffire à lui-même. Et par ce qui se suffit à soi-même, nous entendons non pas ce qui suffit à un seul homme menant une vie solitaire, mais aussi à ses parents, ses \\
enfants, sa femme, ses amis et ses concitoyens en général, puisque l’homme est par nature un être politique. Mais à cette énumération il faut apporter quelque limite, car si on l’étend aux grands-parents, aux descendants et aux amis de nos amis, on ira à l’infini. Mais nous devons réserver cet examen pour une autre occasion. En ce qui concerne le fait de se suffire \\
à soi-même, voici quelle est notre position : c’est ce qui, pris à part de tout le reste, rend la vie désirable et n’ayant besoin de rien d’autre. Or tel est, à notre sentiment, le caractère du bonheur. Nous ajouterons que le bonheur est aussi la chose la plus désirable de toutes, tout en ne figurant pas cependant au nombre des biens, puisque s’il en faisait partie il est clair qu’il serait encore plus désirable par l’addition fût-ce du plus infime des biens : en effet, cette addition produit une somme de biens plus élevée, et de deux biens le plus grand est toujours le plus \\
désirable. On voit donc que le bonheur est quelque chose de parfait et qui se suffit à soi-même, et il est la fin de nos actions.
\subsection[{6 (1097b — 1098a) < Le bonheur défini par la fonction propre de l’homme >}]{6 (1097b — 1098a) < Le bonheur défini par la fonction propre de l’homme >}
\noindent Mais sans doute l’identification du bonheur et du Souverain Bien apparaît-elle comme une chose sur laquelle tout le monde est d’accord ; ce qu’on désire encore, c’est que nous disions plus clairement quelle est la nature du bonheur. Peut-être pourrait-on y arriver si on déterminait la {\itshape fonction} [εργον] de \\
l’homme. De même, en effet, que dans le cas d’un joueur de flûte, d’un statuaire, ou d’un artiste quelconque, et en général pour tous ceux qui ont une fonction ou une activité déterminée, c’est dans la fonction que réside, selon l’opinion courante, le bien, le « réussi », on peut penser qu’il en est ainsi pour l’homme, s’il est vrai qu’il y ait une certaine fonction spéciale à l’homme. Serait-il possible qu’un charpentier ou un cordonnier aient une fonction et une activité à exercer, mais que \\
l’homme n’en ait aucune et que la nature l’ait dispensé de toute œuvre à accomplir ? Ou bien encore, de même qu’un œil, une main, un pied et, d’une manière générale, chaque partie d’un corps, a manifestement une certaine fonction à remplir, ne doit-on pas admettre que l’homme a, lui aussi, en dehors de toutes ces activités particulières, une fonction déterminée ? Mais alors en quoi peut-elle consister ? Le simple fait de vivre est, de toute évidence, une chose que l’homme partage en commun même avec les végétaux ; or ce que nous recherchons, c’est ce qui est propre à l’homme. Nous devons  donc laisser de côté la vie de nutrition et la vie de croissance. Viendrait ensuite la vie sensitive, mais celle-là encore apparaît commune avec le cheval, le bœuf et tous les animaux. Reste donc une certaine vie pratique de la partie rationnelle de l’âme, partie qui peut être envisagée, d’une part, au sens où elle est soumise à la raison, et, d’autre part. au sens où elle possède \\
la raison et l’exercice de la pensée. L’expression < vie rationnelle > étant ainsi prise en un double sens, nous devons établir qu’il s’agit ici de la vie selon le point de vue de l’exercice, car c’est cette vie-là qui paraît bien donner au terme son sens le plus plein. Or s’il y a une fonction de l’homme consistant dans une activité de l’âme conforme à la raison, ou qui n’existe pas sans la raison, et si nous disons que cette fonction est génériquement la même dans un individu quelconque et dans un individu de mérite (ainsi, dans un cithariste et dans un bon \\
cithariste, et ceci est vrai, d’une manière absolue, dans tous les cas), l’excellence due au mérite s’ajoutant à la fonction (car la fonction du cithariste est de jouer de la cithare, et celle du bon cithariste d’en bien jouer) : s’il en est ainsi ; si nous posons que la fonction de l’homme consiste dans un certain genre de vie, c’est-à-dire dans une activité de l’âme et dans des actions accompagnées de raison ; si la fonction d’un homme vertueux est d’accomplir cette tâche, et de l’accomplir bien et avec \\
succès, chaque chose au surplus étant bien accomplie quand elle l’est selon l’excellence qui lui est propre : — dans ces conditions, c’est donc que le bien pour l’homme consiste dans une activité de l’âme en accord avec la vertu, et, au cas de pluralité de vertus, en accord avec la plus excellente et la plus parfaite d’entre elles. Mais il faut ajouter : « et cela dans une vie accomplie jusqu’à son terme », car une hirondelle ne fait pas le printemps, ni non plus un seul jour : et ainsi la félicité et le bonheur ne sont pas davantage l’œuvre d’une seule journée, ni \\
d’un bref espace de temps.
\subsection[{7 (1098a — 1098b) < Questions de méthode — La connaissance des principes >}]{7 (1098a — 1098b) < Questions de méthode — La connaissance des principes >}
\noindent Voilà donc le bien décrit dans ses grandes lignes (car nous devons sans doute commencer par une simple ébauche, et ce n’est qu’ultérieurement que nous appuierons sur les traits). On peut penser que n’importe qui est capable de poursuivre et d’achever dans le détail ce qui a déjà été esquissé avec soin ; et le temps, en ce genre de travail, est un facteur de découverte ou du moins un auxiliaire précieux : cela même \\
est devenu pour les arts une source de progrès, puisque tout homme peut ajouter à ce qui a été laissé incomplet. Mais nous devons aussi nous souvenir de ce que nous avons dit précédemment et ne pas chercher une égale précision en toutes choses, mais au contraire, en chaque cas particulier tendre à l’exactitude que comporte la matière traitée, et seulement dans une mesure appropriée à notre investigation. Et, en effet, un charpentier et un géomètre font bien porter leur recherche l’un et l’autre sur l’angle droit, mais c’est de façon différente : le \\
premier veut seulement un angle qui lui serve pour son travail, tandis que le second cherche l’essence de l’angle droit ou ses propriétés, car le géomètre est un contemplateur de la vérité. C’est de la même façon dès lors qu’il nous faut procéder pour tout le reste, de manière à éviter que dans nos travaux les à-côtés ne l’emportent sur le principal. On ne doit pas non plus exiger la cause en toutes choses indifféremment : il suffit,  dans certains cas, que le fait soit clairement dégagé, comme par exemple en ce qui concerne les principes : le fait vient en premier, c’est un point de départ. Et parmi les principes, les uns sont appréhendés par l’induction [επαγωγη], d’autres par la sensation, d’autres enfin par une sorte d’habitude, les différents principes étant ainsi connus de différentes façons ; et nous devons \\
essayer d’aller à la recherche de chacun d’eux d’une manière appropriée à sa nature, et avoir soin de les déterminer exactement, car ils sont d’un grand poids pour ce qui vient à leur suite. On admet couramment, en effet, que {\itshape le commencement est plus que la moitié du tout} et qu’il permet d’apporter la lumière à nombre de questions parmi celles que nous nous posons.
\subsection[{8 (1098b) < La définition aristotélicienne du bonheur confirmée par les opinions courantes >}]{8 (1098b) < La définition aristotélicienne du bonheur confirmée par les opinions courantes >}
\noindent Mais nous devons porter notre examen sur le principe, \\
non seulement à la lumière de la conclusion et des prémisses de notre raisonnement, mais encore en tenant compte de ce qu’on en dit communément, car avec un principe vrai toutes les données de fait s’harmonisent, tandis qu’avec un principe faux la réalité est vite en désaccord.\par
On a divisé les biens en trois classes : les uns sont dits biens {\itshape extérieurs}, les autres sont ceux qui se rapportent à l’âme ou au corps, et les biens ayant rapport à l’âme, nous les appelons biens au sens strict et par excellence. Or comme nous \\
plaçons les actions et les activités spirituelles parmi les biens qui ont rapport à l’âme, il en résulte que notre définition doit être exacte, dans la perspective du moins de cette opinion qui est ancienne et qui a rallié tous ceux qui s’adonnent à la philosophie. C’est encore à bon droit que nous identifions certaines actions et certaines activités avec la fin, car de cette façon la fin est mise au rang des biens de l’âme et non des biens \\
extérieurs. Enfin s’adapte également bien à notre définition l’idée que l’homme heureux est celui qui vit bien et réussit, car pratiquement nous avons défini le bonheur une forme de vie heureuse et de succès.
\subsection[{9 (1098b — 1099b) < Suite du chapitre précédent : accord de la définition du bonheur avec les doctrines qui identifient le bonheur à la vertu, ou au plaisir, ou aux biens extérieurs >}]{9 (1098b — 1099b) < Suite du chapitre précédent : accord de la définition du bonheur avec les doctrines qui identifient le bonheur à la vertu, ou au plaisir, ou aux biens extérieurs >}
\noindent Il est manifeste aussi que les caractères qu’on requiert d’ordinaire pour le bonheur appartiennent absolument tous à notre définition.\par
Certains auteurs, en effet, sont d’avis que le bonheur c’est la vertu ; pour d’autres, c’est la prudence ; pour d’autres, une forme de sagesse ; d’autres encore le font consister dans ces différents biens à la fois, ou seulement dans l’un d’entre eux, \\
avec accompagnement de plaisir ou n’existant pas sans plaisir ; d’autres enfin ajoutent à l’ensemble de ces caractères la prospérité extérieure. Parmi ces opinions, les unes ont été soutenues par une foule de gens et depuis fort longtemps, les autres l’ont été par un petit nombre d’hommes illustres : il est peu vraisemblable que les uns et les autres se soient trompés du tout au tout, mais, tout au moins sur un point déterminé, ou même sur la plupart, il y a des chances que ces opinions soient conformes à la droite raison.\par
Pour ceux qui prétendent que le bonheur consiste dans la \\
vertu en général ou dans quelque vertu particulière, notre définition est en plein accord avec eux, car l’{\itshape activité conforme à la vertu} appartient bien à la vertu. Mais il y a sans doute une différence qui n’est pas négligeable, suivant que l’on place le Souverain Bien dans la possession ou dans l’usage, dans une disposition ou dans une activité. En effet, la disposition peut  très bien exister sans produire aucun bien, comme dans le cas de l’homme en train de dormir ou inactif de quelque autre façon ; au contraire, pour la vertu en activité, c’est là une chose impossible, car celui dont l’activité est conforme à la vertu agira nécessairement et agira bien. Et de même qu’aux Jeux Olympiques, ce ne sont pas les plus beaux et les plus forts qui sont couronnés, mais ceux qui combattent (car c’est parmi eux \\
que sont pris les vainqueurs), de même aussi les nobles et bonnes choses de la vie deviennent à juste titre la récompense de ceux qui agissent. Et leur vie est encore en elle-même un plaisir, car le sentiment du plaisir rentre dans la classe des états de l’âme, et chacun ressent du plaisir par rapport à l’objet, quel qu’il soit, qu’il est dit aimer : par exemple, un \\
cheval donne du plaisir à l’amateur de chevaux, un spectacle à l’amateur de spectacles ; de la même façon, les actions justes sont agréables à celui qui aime la justice, et, d’une manière générale, les actions conformes à la vertu plaisent à l’homme qui aime la vertu. Mais tandis que chez la plupart des hommes les plaisirs se combattent parce qu’ils ne sont pas des plaisirs par leur nature même, ceux qui aiment les nobles actions trouvent au contraire leur agrément dans les choses qui sont des plaisirs par leur propre nature. Or tel est précisément ce qui caractérise les actions conformes à la vertu, de sorte qu’elles sont des plaisirs à la fois pour ceux qui les accomplissent et en \\
elles-mêmes. Dès lors la vie des gens de bien n’a nullement besoin que le plaisir vienne s’y ajouter comme un surcroît postiche, mais elle a son plaisir en elle-même. Ajoutons, en effet, à ce que nous avons dit, qu’on n’est pas un véritable homme de bien quand on n’éprouve aucun plaisir dans la pratique des bonnes actions, pas plus que ne saurait être jamais appelé juste celui qui accomplit sans plaisir des actions justes, ou libéral celui qui n’éprouve aucun plaisir à faire des actes de \\
libéralité, et ainsi de suite. S’il en est ainsi, c’est en elles-mêmes que les actions conformes à la vertu doivent être des plaisirs. Mais elles sont encore en même temps bonnes et belles, et cela au plus haut degré, s’il est vrai que l’homme vertueux est bon juge en ces matières ; or son jugement est fondé, ainsi que nous l’avons dit. Ainsi donc le bonheur est en même temps ce qu’il y a de meilleur, de plus beau et de \\
plus agréable, et ces attributs ne sont pas séparés comme dans l’inscription de Délos :\par
{\itshape Ce qu’il y a de plus beau, c’est ce qu’il y a de plus juste, et ce qu’il y a de meilleur, c’est de se bien porter} ; \par
 {\itshape Mais ce qu’il y a par nature de plus agréable, c’est d’obtenir l’objet de son amour.} \par
En effet, tous ces attributs appartiennent à la fois aux activités \\
qui sont les meilleures, et ces activités, ou l’une d’entre elles, celle qui est la meilleure, nous disons qu’elles constituent le bonheur même. Cependant il apparaît nettement qu’on doit faire aussi entrer en ligne de compte les biens extérieurs, ainsi que nous l’avons dit, car il est impossible, ou du moins malaisé, d’accomplir les bonnes actions quand on est dépourvu de ressources pour y faire face. En effet, dans un grand nombre  de nos actions, nous faisons intervenir à titre d’instruments les amis ou la richesse, ou l’influence politique ; et, d’autre part, l’absence de certains avantages gâte la félicité : c’est le cas, par exemple, pour la noblesse de race, une heureuse progéniture, la beauté physique. On n’est pas, en effet, complètement heureux si on a un aspect disgracieux, si on est d’une basse extraction, ou si on vit seul et sans enfants ; et, pis encore sans \\
doute, si on a des enfants ou des amis perdus de vices, ou si enfin, alors qu’ils étaient vertueux, la mort nous les a enlevés. Ainsi donc que nous l’avons dit, il semble que le bonheur ait besoin, comme condition supplémentaire, d’une prospérité de ce genre ; de là vient que certains mettent au même rang que le bonheur, la fortune favorable, alors que d’autres l’identifient à la vertu.
\subsection[{10 (1099b — 1100a) < Mode d’acquisition du bonheur : il n’est pas l’œuvre de la fortune, mais le résultat d’une perfection >}]{10 (1099b — 1100a) < Mode d’acquisition du bonheur : il n’est pas l’œuvre de la fortune, mais le résultat d’une perfection >}
\noindent Cette divergence de vues a donné naissance à la difficulté de savoir si le bonheur est une chose qui peut s’apprendre, ou \\
s’il s’acquiert par l’habitude ou quelque autre exercice, ou si enfin il nous échoit en partage par une certaine faveur divine ou même par le hasard. Et, de fait, si jamais les dieux ont fait quelque don aux hommes, il est raisonnable de supposer que le bonheur est bien un présent divin, et cela au plus haut degré parmi les choses humaines, d’autant plus qu’il est la meilleure de toutes. Mais cette question serait sans doute mieux appropriée à un autre ordre de recherches. Il semble bien, en tout cas, que même en admettant que le bonheur ne soit pas envoyé par les dieux, mais survient en nous par l’effet de la vertu ou de \\
quelque étude ou exercice, il fait partie des plus excellentes réalités divines : car ce qui constitue la récompense et la fin même de la vertu est de toute évidence un bien suprême, une chose divine et pleine de félicité. Mais en même temps, ce doit être une chose accessible au grand nombre, car il peut appartenir à tous ceux qui ne sont pas anormalement inaptes à la vertu, s’ils y mettent quelque étude et quelque soin. Et s’il est \\
meilleur d’être heureux de cette façon-là que par l’effet d’une chance imméritée, on peut raisonnablement penser que c’est bien ainsi que les choses se passent en réalité, puisque les œuvres de la nature sont naturellement aussi bonnes qu’elles peuvent l’être, ce qui est le cas également pour tout ce qui relève de l’art ou de toute autre cause, et notamment de la cause par excellence. Au contraire, abandonner au jeu du hasard ce qu’il y a de plus grand et de plus noble serait une solution par trop discordante.\par
\\
La réponse à la question que nous nous posons ressort clairement aussi de notre définition du bonheur. Nous avons dit, en effet, qu’il était {\itshape une activité de l’âme conforme à la vertu}, c’est-à-dire une activité d’une certaine espèce, alors que pour les autres biens, les uns font nécessairement partie intégrante du bonheur, les autres sont seulement des adjuvants et sont utiles à titre d’instruments naturels. — Ces considérations, au surplus, ne sauraient qu’être en accord avec ce que nous avons dit tout au début : car nous avons établi que la fin de \\
la Politique est la fin suprême ; or cette science met son principal soin à faire que les citoyens soient des êtres d’une certaine qualité, autrement dit des gens honnêtes et capables de nobles actions. C’est donc à juste titre que nous n’appelons heureux ni un bœuf, ni un cheval, ni aucun autre animal, car  aucun d’eux n’est capable de participer à une activité de cet ordre. Pour ce motif encore, l’enfant non plus ne peut pas être heureux, car il n’est pas encore capable de telles actions, en raison de son âge, et les enfants qu’on appelle heureux ne le sont qu’en espérance, car le bonheur requiert, nous l’avons dit, à la fois une vertu parfaite et une vie venant à son terme. \\
De nombreuses vicissitudes et des fortunes de toutes sortes surviennent, en effet, au cours de la vie, et il peut arriver à l’homme le plus prospère de tomber dans les plus grands malheurs au temps de sa vieillesse, comme la légende héroïque le raconte de Priam : quand on a éprouvé des infortunes pareilles aux siennes et qu’on a fini misérablement, personne ne vous qualifie d’heureux.
\subsection[{11 (1100a — 1101b) < Le bonheur et la vie présente. Le bonheur après la mort >}]{11 (1100a — 1101b) < Le bonheur et la vie présente. Le bonheur après la mort >}
\noindent \\
Est-ce donc que pas même aucun autre homme ne doive être appelé heureux tant qu’il vit, et, suivant la parole de Solon, devons-nous pour cela {\itshape voir la fin} ? Même si nous devons admettre une pareille chose, irons-nous jusqu’à dire qu’on n’est heureux qu’une fois qu’on est mort ? Ou plutôt n’est-ce pas là une chose complètement absurde, surtout de notre part à nous qui prétendons que le bonheur consiste dans une certaine activité ? Mais si, d’un autre côté, nous refusons d’appeler \\
heureux celui qui est mort (le mot de Solon n’a d’ailleurs pas cette signification), mais si nous voulons dire que c’est seulement au moment de la mort qu’on peut, d’une manière assurée, qualifier un homme d’heureux, comme étant désormais hors de portée des maux et des revers de fortune, même ce sens-là ne va pas sans soulever quelque contestation : on croit, en effet, d’ordinaire que pour l’homme une fois mort il existe encore quelque bien et quelque mal, tout comme chez l’homme vivant \\
qui n’en aurait pas conscience, dans le cas par exemple des honneurs ou des disgrâces qui affectent les enfants ou en général les descendants, leurs succès ou leurs revers. Mais sur ce point encore une difficulté se présente. En effet, l’homme qui a vécu dans la félicité jusqu’à un âge avancé, et dont la fin a été en harmonie avec le restant de sa vie, peut fort bien subir de nombreuses vicissitudes dans ses descendants, certains \\
d’entre eux étant des gens vertueux et obtenant le genre de vie qu’ils méritent, d’autres au contraire se trouvant dans une situation tout opposée ; et évidemment aussi, suivant les degrés de parenté, les relations des descendants avec leurs ancêtres sont susceptibles de toute espèce de variations. Il serait dès lors absurde de supposer que le mort participe à toutes ces vicissitudes, et pût devenir à tel moment heureux pour redevenir ensuite malheureux ; mais il serait tout aussi absurde de penser \\
qu’en rien ni en aucun temps le sort des descendants ne pût affecter leurs ancêtres.\par
Mais nous devons revenir à la précédente difficulté, car peut-être son examen facilitera-t-il la solution de la présente question. Admettons donc que l’on doive voir la fin et attendre ce moment pour déclarer un homme heureux, non pas comme étant actuellement heureux, mais parce qu’il l’était dans un temps antérieur : comment n’y aurait-il pas une absurdité dans \\
le fait que, au moment même où cet homme est heureux, on refusera de lui attribuer avec vérité ce qui lui appartient, sous  prétexte que nous ne voulons pas appeler heureux les hommes qui sont encore vivants, en raison des caprices de la fortune et de ce que nous avons conçu le bonheur comme quelque chose de stable et ne pouvant être facilement ébranlé d’aucune façon, alors que la roue de la fortune tourne souvent pour le même individu ? Il est évident, en effet, que si nous le suivons pas à \\
pas dans ses diverses vicissitudes, nous appellerons souvent le même homme tour à tour heureux et malheureux, faisant ainsi de l’homme heureux une sorte de {\itshape caméléon} ou une {\itshape maison menaçant ruine}. Ne doit-on pas plutôt penser que suivre la fortune dans tous ses détours est un procédé absolument incorrect ? Ce n’est pas en cela, en effet, que consistent la prospérité ou l’adversité : ce ne sont là, nous l’avons dit, que de simples adjuvants dont la vie de tout homme a besoin. La \\
cause véritablement déterminante du bonheur réside dans l’activité conforme à la vertu, l’activité en sens contraire étant la cause de l’état opposé.\par
Et la difficulté que nous discutons présentement témoigne en faveur de notre argument. Dans aucune action humaine, en effet, on ne relève une fixité comparable à celle des activités conformes à la vertu, lesquelles apparaissent plus stables encore que les connaissances scientifiques. Parmi ces activités \\
vertueuses elles-mêmes, les plus hautes sont aussi les plus stables, parce que c’est dans leur exercice que l’homme heureux passe la plus grande partie de sa vie et avec le plus de continuité, et c’est là, semble-t-il bien, la cause pour laquelle l’oubli ne vient pas les atteindre.\par
Ainsi donc, la stabilité que nous recherchons appartiendra à l’homme heureux, qui le demeurera durant toute sa vie : car toujours, ou du moins préférablement à toute autre chose, il s’engagera dans des actions et des contemplations conformes à \\
la vertu, et il supportera les coups du sort avec la plus grande dignité et un sens en tout point parfait de la mesure, si du moins il est véritablement homme de bien et {\itshape d’une carrure sans reproche}.\par
Mais nombreux sont les accidents de la fortune, ainsi que leur diversité en grandeur et en petitesse. S’agit-il de succès minimes aussi bien que de revers légers, il est clair qu’ils ne pèsent pas d’un grand poids dans la vie. Au contraire si \\
on a affaire à des événements dont la gravité et le nombre sont considérables, alors, dans le cas où ils sont favorables ils rendront la vie plus heureuse (car en eux-même ils contribuent naturellement à embellir l’existence, et, de plus, leur utilisation peut être noble et généreuse), et dans le cas où ils produisent des résultats inverses, ils rétrécissent et corrompent le bonheur, car, en même temps qu’ils apportent des chagrins avec eux, ils mettent obstacle à de multiples activités. \\
Néanmoins, même au sein de ces contrariétés transparaît la noblesse de l’âme, quand on supporte avec résignation de nombreuses et sévères infortunes, non certes par insensibilité, mais par noblesse et grandeur d’âme. Et si ce sont nos activités qui constituent le facteur déterminant de notre vie, ainsi que nous l’avons dit, nul homme heureux ne saurait devenir misérable, \\
puisque jamais il n’accomplira des actions odieuses et viles. En effet, selon notre doctrine, l’homme véritablement  bon et sensé supporte toutes les vicissitudes du sort avec sérénité et tire parti des circonstances pour agir toujours avec le plus de noblesse possible, pareil en cela à un bon général qui utilise à la guerre les forces dont il dispose de la façon la plus efficace, ou à un bon cordonnier qui du cuir qu’on lui a confié \\
fait les meilleures chaussures possibles, et ainsi de suite pour tous les autres corps de métier. Et s’il en est bien ainsi, l’homme heureux ne saurait jamais devenir misérable, tout en n’atteignant pas cependant la pleine félicité s’il vient à tomber dans des malheurs comme ceux de Priam. Mais il n’est pas non plus sujet à la variation et au changement, car, d’une part, \\
il ne sera pas ébranlé aisément dans son bonheur, ni par les premières infortunes venues : il y faudra pour cela des échecs multipliés et graves ; et, d’autre part, à la suite de désastres d’une pareille ampleur, il ne saurait recouvrer son bonheur en un jour, mais s’il y arrive, ce ne pourra être qu’à l’achèvement d’une longue période de temps, au cours de laquelle il aura obtenu de grandes et belles satisfactions.\par
Dès lors, qui nous empêche d’appeler heureux l’homme \\
dont l’activité est conforme à une parfaite vertu et suffisamment pourvu des biens extérieurs, et cela non pas pendant une durée quelconque mais pendant une vie complète ? Ne devons-nous pas ajouter encore : {\itshape dont la vie se poursuivra dans les mêmes conditions et dont la fin sera en rapport avec le reste de l’existence}, puisque l’avenir nous est caché et que nous posons le bonheur comme une fin, comme quelque chose d’absolument parfait ? S’il en est ainsi, nous qualifierons de bienheureux \\
ceux qui, parmi les hommes vivants, possèdent et posséderont les biens que nous avons énoncés, — mais bienheureux toutefois comme des hommes peuvent l’être.\par
Sur toutes ces questions, nos explications doivent suffire. Quant à soutenir que les vicissitudes de nos descendants et de l’ensemble de nos amis n’influencent en rien notre bonheur, c’est là de toute évidence une doctrine par trop étrangère à l’amitié et contraire aux opinions reçues. Mais étant donné que les événements qui nous atteignent sont nombreux et \\
d’une extrême variété, que les uns nous touchent de plus près et d’autres moins, vouloir les distinguer un par un serait manifestement une besogne de longue haleine et autant dire illimitée ; des indications générales et sommaires seront sans doute suffisantes. Si donc parmi les malheurs qui nous frappent personnellement, les uns pèsent d’un certain poids et exercent une certaine influence sur notre vie, tandis que les autres nous \\
paraissent plus supportables, il doit en être de même aussi pour les infortunes qui atteignent l’ensemble de nos amis ; et s’il y a une différence suivant que chacun de ces malheurs affecte des vivants ou des morts (différence bien plus grande au surplus que celle que nous constatons, dans les tragédies, entre les crimes et les horreurs survenus antérieurement et ce qui s’accomplit sur la scène), il faut tenir compte également de \\
cette différence-là ; ou plutôt sans doute faut-il se poser la question préalable de savoir si les défunts ont encore part à un  bien ou à un mal quelconque, car il semble résulter de ces considérations que si une impression quelconque, en bien ou en mal, pénètre jusqu’à eux, elle doit être quelque chose de faible et de négligeable, soit en elle-même soit par rapport à eux, ou, s’il n’en est rien, être du moins d’une intensité 36 et d’une nature telles qu’elle soit insuffisante pour rendre heureux ceux qui ne le sont pas, ou pour ôter la félicité à ceux qui la possèdent.\par
\\
Les succès aussi bien que les insuccès de leurs amis semblent donc bien affecter dans une certaine mesure le sort des défunts, tout en n’étant pas cependant d’une nature et d’une importance telles qu’ils puissent rendre malheureux ceux qui sont heureux, ni produire quelque autre effet de cet ordre.
\subsection[{12 (1101b — 1102a) < Le bonheur est-il un bien digne d’éloge ou digne d’honneur ? >}]{12 (1101b — 1102a) < Le bonheur est-il un bien digne d’éloge ou digne d’honneur ? >}
\noindent \\
Ces explications une fois données, examinons si le bonheur appartient à la classe des biens dignes d’éloge ou plutôt à celle des biens dignes d’honneur, car il est évident que, de toute façon, il ne rentre pas dans les potentialités. Il apparaît bien que ce qui est digne d’éloge est toujours loué par le fait de posséder quelque qualité et d’être dans une certaine relation à quelque chose, car l’homme juste, l’homme courageux, \\
et en général l’homme de bien et la vertu elle-même sont objet de louanges de notre part en raison des actions et des œuvres qui en procèdent, et nous louons aussi l’homme vigoureux, le bon coureur, et ainsi de suite, parce qu’ils possèdent une certaine qualité naturelle et se trouvent dans une certaine relation avec quelque objet bon ou excellent. Cela résulte encore clairement des louanges que nous donnons aux dieux : il nous paraît, en effet, ridicule de rapporter les dieux à nous, \\
et cela tient à ce que les louanges se font par référence à autre chose, ainsi que nous l’avons dit. Mais si la louange s’applique à des choses de ce genre, il est évident que les réalités les plus nobles sont objet, non pas de louange, mais de quelque chose de plus grand et de meilleur, comme on peut d’ailleurs s’en rendre compte : ce que nous faisons, en effet, aussi bien pour les dieux que pour ceux des hommes qui sont le plus semblables aux dieux, c’est de proclamer leur béatitude et leur \\
félicité. Nous agissons de même en ce qui concerne les biens proprement dits, car nul ne fait l’éloge du bonheur comme il le fait de la justice, mais on proclame sa félicité comme étant quelque chose de plus divin et de meilleur encore. Et Eudoxe semble avoir eu raison de prendre la défense du plaisir pour lui décerner le prix de la plus haute excellence : il pensait, en effet, que si le plaisir, tout en étant au nombre des biens, n’est jamais pris pour sujet d’éloge, c’était là un signe de sa supériorité sur les choses dont on fait l’éloge, caractère qui \\
appartient aussi à Dieu et au bien, en ce qu’ils servent de référence pour tout le reste. L’éloge [επαινος] s’adresse, en effet, à la vertu (puisque c’est elle qui nous rend aptes à accomplir les bonnes actions), tandis que la glorification [εγκωµιον] porte sur les actes soit du corps soit de l’âme, indifféremment. Mais l’examen détaillé de \\
ces questions relève sans doute plutôt de ceux qui ont fait une étude approfondie des glorifications : pour nous, il résulte  clairement de ce que nous avons dit, que le bonheur rentre dans la classe des choses dignes d’honneur et parfaites. Et si telle est sa nature, cela tient aussi, semble-t-il, à ce qu’il est un principe, car c’est en vue de lui que tous nous accomplissons toutes les autres choses que nous faisons ; et nous posons le principe et la cause des biens comme quelque chose digne d’être honoré et de divin.
\subsection[{13 (1102a — 1103a) < Les facultés de l’âme. Vertus intellectuelles et vertus morales >}]{13 (1102a — 1103a) < Les facultés de l’âme. Vertus intellectuelles et vertus morales >}
\noindent \\
Puisque le bonheur est une certaine activité de l’âme en accord avec une vertu parfaite, c’est la nature de la vertu qu’il nous faut examiner : car peut-être ainsi pourrons-nous mieux considérer la nature du bonheur lui-même. Or il semble bien que le véritable politique [πολιτικος] soit aussi celui qui s’est adonné spécialement à l’étude de la vertu, puisqu’il veut faire de ses \\
concitoyens des gens honnêtes et soumis aux lois (comme exemple de ces politiques nous pouvons citer les législateurs de la Crète et de Lacédémone, et tous autres du même genre dont l’histoire peut faire mention). Et si cet examen relève de la Politique, il est clair que nos recherches actuelles rentreront dans notre dessin primitif.\par
Mais la vertu qui doit faire l’objet de notre examen est \\
évidemment une vertu humaine, puisque le bien que nous cherchons est un bien humain, et le bonheur, un bonheur humain. Et par vertu humaine nous entendons non pas l’excellence du corps. mais bien celle de l’âme, et le bonheur est aussi pour nous une activité de l’âme. Mais s’il en est ainsi, il est évident que le politique doit posséder une certaine connaissance de ce qui a rapport à l’âme, tout comme le médecin \\
appelé à soigner les yeux doit connaître aussi d’une certaine manière le corps dans son ensemble ; et la connaissance de l’âme s’impose d’autant plus dans l’espèce que la Politique dépasse en noblesse et en élévation la médecine, et d’ailleurs chez les médecins eux-mêmes, les plus distingués d’entre eux s’appliquent avec grand soin à acquérir la connaissance du corps. Il faut donc aussi que le politique considère ce qui a rapport à l’âme et que son étude soit faite dans le but que nous avons indiqué, et seulement dans la mesure requise pour ses \\
recherches, car pousser plus loin le souci du détail est sans doute une tâche trop lourde eu égard à ce qu’il se propose.\par
On traite aussi de l’âme dans les discussions exotériques : certains points y ont été étudiés d’une manière satisfaisante et nous devons en faire notre profit : c’est ainsi que nous admettons qu’il y a dans l’âme la partie irrationnelle et la partie \\
rationnelle. Quant à savoir si ces deux parties sont réellement distinctes comme le sont les parties du corps ou de tout autre grandeur divisible, ou bien si elles sont logiquement distinctes mais inséparables par nature, comme le sont dans la circonférence le convexe et le concave, cela n’a aucune importance pour la présente discussion.\par
Dans la partie irrationnelle elle-même, on distingue la partie qui semble être commune à tous les êtres vivants y compris les végétaux, je veux dire cette partie qui est cause de la nutrition et de l’accroissement. C’est, en effet, une potentialité psychique de ce genre que l’on peut assigner à tous les  êtres qui se nourrissent et même aux embryons ; cette même faculté est au surplus également présente dans les êtres pleinement développés, car il est plus raisonnable de la leur attribuer que de leur en donner quelque autre. — Quoi qu’il en soit, cette faculté possède une certaine excellence, laquelle se révèle comme étant commune à toutes les espèces et non comme étant proprement humaine. En effet, c’est dans le sommeil que cette partie de l’âme, autrement dit cette potentialité, semble \\
avoir son maximum d’activité, alors qu’au contraire l’homme bon et l’homme vicieux ne se distinguent en rien pendant leur sommeil, et c’est même de là que vient le dicton qu’il n’y a aucune différence durant la moitié de leur vie entre les gens heureux et les misérables. Cela résulte tout naturellement de ce fait que le sommeil est pour l’âme une suspension de cette activité par où se caractérise l’âme vertueuse ou perverse, sauf à admettre toutefois que, dans une faible mesure, certaines \\
impressions [κινησις] parviennent à la conscience, et qu’ainsi les rêves des gens de bien sont meilleurs que ceux du premier venu. Mais sur ce sujet nous en avons assez dit, et nous devons laisser de côté la partie nutritive, puisque par sa nature même elle n’a rien à voir avec l’excellence spécifiquement humaine.\par
Mais il semble bien qu’il existe encore dans l’âme une autre nature irrationnelle, laquelle toutefois participe en quelque manière à la raison. En effet, dans l’homme tempérant comme dans l’homme intempérant, nous faisons l’éloge \\
de leur principe raisonnable, ou de la partie de leur âme qui possède la raison, parce qu’elle les exhorte avec rectitude à accomplir les plus nobles actions. Mais il se manifeste aussi en eux un autre principe, qui se trouve par sa nature même en dehors du principe raisonnable, principe avec lequel il est en conflit et auquel il oppose de la résistance. Car il en est exactement comme dans les cas de paralysie où les parties du corps, \\
quand nous nous proposons de les mouvoir à droite, se portent au contraire à gauche. Eh bien, pour l’âme il en est de même : c’est dans des directions contraires à la raison que se tournent les impulsions des intempérants. Il y a pourtant cette différence que, dans le cas du corps, nous voyons de nos yeux la déviation du membre, tandis que dans le cas de l’âme nous ne voyons rien : il n’en faut pas moins admettre sans doute qu’il existe aussi dans l’âme un facteur en dehors du principe raisonnable, qui lui est opposé et contre lequel il lutte. Quant à savoir \\
en quel sens ces deux parties de l’âme sont distinctes, cela n’a aucune importance.\par
Mais il apparaît bien aussi que ce second facteur participe au principe raisonnable, ainsi que nous l’avons dit : dans le cas de l’homme tempérant tout au moins, ce facteur obéit au principe raisonnable, et il est peut-être encore plus docile chez l’homme modéré et courageux, puisque en lui tout est en accord avec le principe raisonnable.\par
On voit ainsi que la partie irrationnelle de l’âme est elle-même double : il y a, d’une part, la partie végétative, qui n’a rien de commun avec le principe raisonnable, et, d’autre part, \\
la partie appétitive ou, d’une façon générale, désirante, qui participe en quelque manière au principe raisonnable en tant qu’elle l’écoute et lui obéit, et cela au sens où nous disons « tenir compte » de son père ou de ses amis, et non au sens où les mathématiciens parlent de « raison ». Et que la partie irrationnelle subisse une certaine influence de la part du principe raisonnable, on en a la preuve dans la pratique des admonestations, et, d’une façon générale, des reproches et exhortations.  Mais si cet élément irrationnel doit être dit aussi posséder la raison, c’est alors la partie raisonnable qui sera double : il y aura, d’une part, ce qui, proprement et en soi-même, possède la raison, et, d’autre part, ce qui ne fait que lui obéir, à la façon dont on obéit à son père.\par
La vertu se divise à son tour conformément à cette différences. \\
Nous distinguons, en effet, les vertus intellectuelles et les vertus morales : la sagesse, l’intelligence, la prudence [φρονησις] sont des vertus intellectuelles ; la libéralité et la modération sont des vertus morales. En parlant, en effet, du caractère moral de quelqu’un, nous ne disons pas qu’il est sage ou intelligent, mais qu’il est doux ou modéré. Cependant nous louons aussi le sage en raison de la disposition où il se trouve, et, parmi les \\
dispositions, celles qui méritent la louange, nous les appelons des vertus.
\section[{Livre II}]{Livre II}\renewcommand{\leftmark}{Livre II}

\subsection[{1 (1103a — 1103b) < La vertu, résultat de l’habitude s’ajoutant à la nature >}]{1 (1103a — 1103b) < La vertu, résultat de l’habitude s’ajoutant à la nature >}
\noindent La vertu est de deux sortes, la vertu intellectuelle et la vertu \\
morale. La vertu intellectuelle dépend dans une large mesure de l’enseignement reçu, aussi bien pour sa production que pour son accroissement ; aussi a-t-elle besoin d’expérience et de temps. La vertu morale, au contraire, est le produit de l’habitude, d’où lui est venu aussi son nom, par une légère modification de εθος. — Et par suite il est également évident qu’aucune des vertus morales n’est engendrée en nous naturellement, car \\
rien de ce qui existe par nature ne peut être rendu autre par l’habitude : ainsi la pierre, qui se porte naturellement vers le bas, ne saurait être habituée à se porter vers le haut, pas même si des milliers de fois on tentait de l’y accoutumer en la lançant en l’air ; pas davantage ne pourrait-on habituer le feu à se porter vers le bas, et, d’une manière générale, rien de ce qui a une nature donnée ne saurait être accoutumé à se comporter autrement. Ainsi donc, ce n’est ni par nature, ni contrairement à la nature que naissent en nous les vertus, mais la nature nous a \\
donné la capacité de les recevoir, et cette capacité est amenée à maturité par l’habitude.\par
En outre, pour tout ce qui survient en nous par nature, nous le recevons d’abord à l’état de puissance, et c’est plus tard que nous le faisons passer à l’acte, comme cela est manifeste dans le cas des facultés sensibles (car ce n’est pas à la suite d’une multitude d’actes de vision ou d’une multitude d’actes d’audition \\
que nous avons acquis les sens correspondants, mais c’est l’inverse : nous avions déjà les sens quand nous en avons fait usage, et ce n’est pas après en avoir fait usage que nous les avons eus). Pour les vertus, au contraire, leur possession suppose un exercice antérieur, comme c’est aussi le cas pour les autres arts. En effet, les choses qu’il faut avoir apprises pour les faire, c’est en les faisant que nous les apprenons : par exemple, c’est en construisant qu’on devient constructeur, et en jouant de la cithare qu’on devient cithariste ; ainsi encore,  c’est en pratiquant les actions justes que nous devenons justes, les actions modérées que nous devenons modérés, et les actions courageuses que nous devenons courageux. Cette vérité est encore attestée par ce qui se passe dans les cités, où les législateurs rendent bons les citoyens en leur faisant contracter certaines \\
habitudes : c’est même là le souhait de tout législateur, et s’il s’en acquitte mal, son œuvre est manquée, et c’est en quoi une bonne constitution se distingue d’une mauvaise.\par
De plus, les actions qui, comme causes ou comme moyens, sont à l’origine de la production d’une vertu quelconque, sont les mêmes que celles qui amènent sa destruction, tout comme dans le cas d’un art : en effet, jouer de la cithare forme indifféremment les bons et les mauvais citharistes. On \\
peut faire une remarque analogue pour les constructeurs de maisons et tous les autres corps de métiers : le fait de bien construire donnera de bons constructeurs, et le fait de mal construire, de mauvais. En effet, s’il n’en était pas ainsi, on n’aurait aucun besoin du maître, mais on serait toujours de naissance bon ou mauvais dans son art. Il en est dès lors de même pour les vertus : c’est en accomplissant tels ou tels actes dans notre commerce avec les autres hommes que nous \\
devenons, les uns justes, les autres injustes ; c’est en accomplissant de même telles ou telles actions dans les dangers, et en prenant des habitudes de crainte ou de hardiesse que nous devenons, les uns courageux, les autres poltrons. Les choses se passent de la même façon en ce qui concerne les appétits et les impulsions : certains hommes deviennent modérés et doux, \\
d’autres déréglés et emportés, pour s’être conduits, dans des circonstances identiques, soit d’une manière soit de l’autre. En un mot, les dispositions morales proviennent d’actes qui leur sont semblables. C’est pourquoi nous devons orienter nos activités dans un certain sens, car la diversité qui les caractérise entraîne les différences correspondantes dans nos dispositions. Ce n’est donc pas une œuvre négligeable de contracter dès la plus tendre enfance telle ou telle habitude, c’est au contraire \\
d’une importance majeure, disons mieux totale.
\subsection[{2 (1103b — 1105a) < Théorie et pratique dans la morale. Rapports du plaisir et de la peine avec la vertu >}]{2 (1103b — 1105a) < Théorie et pratique dans la morale. Rapports du plaisir et de la peine avec la vertu >}
\noindent Puisque le présent travail n’a pas pour but la spéculation pure comme nos autres ouvrages (car ce n’est pas pour savoir ce qu’est la vertu en son essence que nous effectuons notre enquête, mais c’est afin de devenir vertueux, puisque autrement cette étude ne servirait à rien), il est nécessaire de porter \\
notre examen sur ce qui a rapport à nos actions, pour savoir de quelle façon nous devons les accomplir, car ce sont elles qui déterminent aussi le caractère de nos dispositions morales, ainsi que nous l’avons dit.\par
Or le fait d’agir {\itshape conformément à la droite règle} est une chose communément admise et qui doit être pris pour base : nous y reviendrons plus tard, nous dirons ce qu’est la droite règle et son rôle à l’égard des autres vertus.\par
 Mais mettons-nous préalablement d’accord sur le point suivant : notre exposé tout entier, qui roule sur les actions qu’il faut faire, doit s’en tenir aux généralités et ne pas entrer dans le détail. Ainsi que nous l’avons dit en commençant, les exigences de toute discussion dépendent de la matière que l’on traite. Or sur le terrain de l’action et de l’utile, il n’y a rien de \\
fixe, pas plus que dans le domaine de la santé. Et si tel est le caractère de la discussion portant sur les règles générales de la conduite, à plus forte raison encore la discussion qui a pour objet les différents groupes de cas particuliers manque-telle également de rigueur, car elle ne tombe ni sous aucun art, ni sous aucune prescription, et il appartient toujours à l’agent lui-même d’examiner ce qu’il est opportun de faire, comme dans le cas de l’art médical, ou de l’art de la navigation.\par
\\
Mais, en dépit de ce caractère du présent exposé, nous devons cependant nous efforcer de venir au secours du moraliste. Ce que tout d’abord il faut considérer, c’est que les vertus en question sont naturellement sujettes à périr à la fois par excès et par défaut, comme nous le voyons dans le cas de la vigueur corporelle et de la santé (car on est obligé pour éclaircir les choses obscures, de s’appuyer sur des preuves manifestes) : \\
en effet, l’excès comme l’insuffisance d’exercice font perdre également la vigueur ; pareillement, dans le boire et le manger, une trop forte ou une trop faible quantité détruit la santé, tandis que la juste mesure la produit, l’accroît et la conserve. Eh bien, il en est ainsi pour la modération, le courage \\
et les autres vertus : car celui qui fuit devant tous les périls, qui a peur de tout et qui ne sait rien supporter devient un lâche, tout comme celui qui n’a peur de rien et va au-devant de n’importe quel danger, devient téméraire ; pareillement encore, celui qui se livre à tous les plaisirs et ne se refuse à aucun devient un homme dissolu, tout comme celui qui se prive de tous les plaisirs comme un rustre, devient une sorte d’être insensible. \\
Ainsi donc, la modération et le courage se perdent également par l’excès et par le défaut, alors qu’ils se conservent par la juste mesure.\par
Mais non seulement les vertus ont pour origine et pour source de leur production et de leur croissance les mêmes actions qui président d’autre part à leur disparition, mais encore leur activité se déploiera dans l’accomplissement de ces mêmes actions. Il en est effectivement ainsi pour les autres \\
qualités plus apparentes que les vertus, Prenons, par exemple, la vigueur du corps : elle a sa source dans la nourriture abondante qu’on absorbe et dans les nombreuses fatigues qu’on endure ; mais ce sont là aussi des actions que l’homme vigoureux se montre particulièrement capable d’accomplir. Or c’est ce qui se passe pour les vertus : c’est en nous abstenant des plaisirs que nous devenons modérés, et une fois que nous le \\
sommes devenus, c’est alors que nous sommes le plus capables  de pratiquer cette abstention, Il en est de même au sujet du courage : en nous habituant à mépriser le danger et à lui tenir tête, nous devenons courageux, et une fois que nous le sommes devenus, c’est alors que nous serons le plus capables d’affronter le danger.\par
D’autre part, nous devons prendre pour signe distinctif de nos dispositions le plaisir ou la peine qui vient s’ajouter \\
à nos actions. En effet, l’homme qui s’abstient des plaisirs du corps et qui se réjouit de cette abstention même, est un homme modéré, tandis que s’il s’en afflige, il est un homme intempérant ; et l’homme qui fait face au danger et qui y trouve son plaisir, ou tout au moins n’en éprouve pas de peine, est un homme courageux, alors que s’il en ressent de la peine, c’est un lâche. — Plaisirs et peines sont ainsi, en fait, ce sur quoi roule la vertu morale.\par
En effet, c’est à cause du plaisir que nous en ressentons \\
que nous commettons le mal, et à cause de la douleur que nous nous abstenons du bien, Aussi devons-nous être amenés d’une façon ou d’une autre, dès la plus tendre enfance, suivant la remarque de Platon, à trouver nos plaisirs et nos peines là où il convient, car la saine éducation consiste en cela. — En second lieu, si les vertus concernent les actions et les passions, et si toute passion et toute action s’accompagnent logiquement \\
de plaisir ou de peine, pour cette raison encore la vertu aura rapport aux plaisirs et aux peines. — Une autre indication résulte de ce fait que les sanctions se font par ces moyens : car le châtiment est une sorte de cure, et il est de la nature de la cure d’obéir à la loi des opposés. — De plus, comme nous l’avons noté aussi plus haut, toute disposition de l’âme est par sa nature même en rapports et en conformité avec le genre de \\
choses qui peuvent la rendre naturellement meilleure ou pire. Or c’est à cause des plaisirs et des peines que les hommes deviennent méchants, du fait qu’ils les poursuivent ou les évitent, alors qu’il s’agit de plaisirs et de peines qu’on ne doit pas rechercher ou fuir, ou qu’on le fait à un moment où il ne le faut pas, ou de la façon qu’il ne faut pas, ou selon tout autre modalité rationnellement déterminée. Et c’est pourquoi certains définissent les vertus comme étant des états d’impassibilité \\
et de repos ; mais c’est là une erreur, due à ce qu’ils s’expriment en termes absolus, sans ajouter {\itshape de la façon qu’il faut} et {\itshape de la façon qu’il ne faut pas} ou {\itshape au moment où il faut}, et toutes autres additions. Qu’il soit donc bien établi que la vertu dont il est question est celle qui tend à agir de la meilleure façon au regard des plaisirs et des peines, et que le vice fait tout le contraire.\par
Nous pouvons, à l’aide des considérations suivantes, \\
apporter encore quelque lumière aux points que nous venons de traiter. Il existe trois facteurs qui entraînent nos choix, et trois facteurs nos répulsions : le beau, l’utile, le plaisant, et leurs contraires, le laid, le dommageable et le pénible. En face de tous ces facteurs l’homme vertueux peut tenir une conduite ferme, alors que le méchant est exposé à faillir et tout spécialement en ce qui concerne le plaisir, car le plaisir est commun à \\
l’homme et aux animaux, et de plus il accompagne tout ce qui dépend de notre choix, puisque même le beau et l’utile nous  apparaissent comme une chose agréable.\par
En outre, dès l’enfance, l’aptitude au plaisir a grandi avec chacun de nous : c’est pourquoi il est difficile de se débarrasser de ce sentiment, tout imprégné qu’il est dans notre vie. — De plus, nous mesurons nos actions, tous plus ou moins, au plaisir et à la peine qu’elles nous donnent. Pour cette raison encore, nous devons nécessairement centrer toute notre étude sur ces notions, car il n’est pas indifférent pour la conduite de la vie \\
que notre réaction au plaisir et à la peine soit saine ou viciée. Ajoutons enfin qu’il est plus difficile de combattre le plaisir que les désirs de son cœur, suivant le mot d’Héraclite ; or la vertu, comme l’art également, a toujours pour objet ce qui est plus difficile, car le bien est de plus haute qualité quand il est \\
contrarié. Par conséquent, voilà encore une raison pour que plaisirs et peines fassent le principal objet de l’œuvre entière de la vertu comme de la Politique, car si on en use bien on sera bon, et si on en use mal, mauvais.
\subsection[{3 (1105a — 1105b) < Vertus et arts — Conditions de l’acte moral >}]{3 (1105a — 1105b) < Vertus et arts — Conditions de l’acte moral >}
\noindent Qu’ainsi donc la vertu ait rapport à des plaisirs et à des peines, et que les actions qui la produisent soient aussi celles qui la font croître ou, quand elles ont lieu d’une autre façon, la \\
font disparaître ; qu’enfin les actions dont elle est la résultante soient celles mêmes où son activité s’exerce ensuite, — tout cela, considérons-le comme dit.\par
Mais on pourrait se demander ce que nous entendons signifier quand nous disons qu’on ne devient juste qu’en faisant des actions justes, et modéré qu’en faisant des actions modérées : car enfin, si on fait des actions justes et des actions \\
modérées, c’est qu’on est déjà juste et modéré, de même qu’en faisant des actes ressortissant à la grammaire et à la musique on est grammairien et musicien. Mais ne peut-on pas dire plutôt que cela n’est pas exact, même dans le cas des arts ? C’est qu’il est possible, en effet, qu’on fasse une chose ressortissant à la grammaire soit par chance, soit sous l’indication d’autrui : on ne sera donc grammairien que si, à la fois, on a fait quelque chose de grammatical, et si on l’a fait d’une façon grammaticale, \\
à savoir conformément à la science grammaticale qu’on possède en soi-même.\par
De plus, il n’y a pas ressemblance entre le cas des arts et celui des vertus. Les productions de l’art ont leur valeur en elles-mêmes ; il suffit donc que la production leur confère certains caractères, Au contraire, pour les actions faites selon la vertu, ce n’est pas par la présence en elles de certains caractères intrinsèques qu’elles sont faites d’une façon juste ou \\
modérée ; il faut encore que l’agent lui-même soit dans une certaine disposition quand il les accomplit : en premier lieu, il doit savoir ce qu’il fait ; ensuite, choisir librement l’acte en question et le choisir en vue de cet acte lui-même ; et en troisième lieu, l’accomplir dans une disposition d’esprit ferme  et inébranlable. Or ces conditions n’entrent pas en ligne de compte pour la possession d’un art quel qu’il soit, à l’exception du savoir lui-même, alors que, pour la possession des vertus, le savoirs ne joue qu’un rôle minime ou même nul, à la différence des autres conditions, lesquelles ont une influence non pas médiocre, mais totale, en tant précisément que la possession de la vertu naît de l’accomplissement répété des actes justes et modérés.\par
\\
Ainsi donc, les actions sont dites justes et modérées quand elles sont telles que les accomplirait l’homme juste ou l’homme modéré ; mais est juste et modéré non pas celui qui les accomplit simplement, mais celui qui, de plus, les accomplit de la façon dont les hommes justes et modérés les accomplissent. \\
On a donc raison de dire que c’est par l’accomplissement des actions justes qu’on devient juste, et par l’accomplissement des actions modérées qu’on devient modéré, tandis qu’à ne pas les accomplir nul ne saurait jamais être en passe de devenir bon, Mais la plupart des hommes, au lieu d’accomplir des actions vertueuses, se retranchent dans le domaine de la discussion, et pensent qu’ils agissent ainsi en philosophes et que cela suffira à les rendre vertueux : ils ressemblent en cela \\
aux malades qui écoutent leur médecin attentivement, mais n’exécutent aucune de ses prescriptions. Et de même que ces malades n’assureront pas la santé de leur corps en se soignant de cette façon, les autres non plus n’obtiendront pas celle de l’âme en professant une philosophie de ce genre.
\subsection[{4 (1105b — 1106a) < Définition générique de la vertu : la vertu est un « habitus » >}]{4 (1105b — 1106a) < Définition générique de la vertu : la vertu est un « habitus » >}
\noindent Qu’est-ce donc que la vertu, voilà ce qu’il faut examiner.\par
\\
Puisque les phénomènes de l’âme sont de trois sortes, les états affectifs [παθος], les facultés et les dispositions, c’est l’une de ces choses qui doit être la vertu. J’entends par {\itshape états affectifs}, l’appétit, la colère, la crainte, l’audace, l’envie, la joie, l’amitié, la haine, le regret de ce qui a plu, la jalousie, la pitié, bref toutes les inclinations accompagnées de plaisir ou de peine ; par facultés, les aptitudes qui font dire de nous que nous sommes capables d’éprouver ces affections, par exemple la \\
capacité d’éprouver colère, peine ou pitié ; par {\itshape dispositions}, enfin, notre comportement bon ou mauvais relativement aux affections : par exemple, pour la colère, si nous l’éprouvons ou violemment ou nonchalamment, notre comportement est mauvais, tandis qu’il est bon si nous l’éprouvons avec mesure, et ainsi pour les autres affections.\par
Or ni les vertus, ni les vices ne sont des affections, parce \\
que nous ne sommes pas appelés vertueux ou pervers d’après les affections que nous éprouvons, mais bien d’après nos vertus et nos vices, et parce que ce n’est pas non plus pour nos affections que nous encourons l’éloge ou le blâme (car on ne loue pas l’homme qui ressent de la crainte ou éprouve de la colère, pas plus qu’on ne blâme celui qui se met simplement en  colère, mais bien celui qui s’y met d’une certaine façon), mais ce sont nos vertus et nos vices qui nous font louer ou blâmer. En outre, nous ressentons la colère ou la crainte indépendamment de tout choix délibéré, alors que les vertus sont certaines façons de choisir, ou tout au moins ne vont pas sans un choix réfléchi. Ajoutons à cela que c’est en raison de nos affections \\
que nous sommes dits être mus, tandis qu’en raison de nos vertus et de nos vices nous sommes non pas mus, mais disposés d’une certaine façon.\par
Pour les raisons qui suivent, les vertus et les vices ne sont pas non plus des facultés. Nous ne sommes pas appelés bons ou mauvais d’après notre capacité à éprouver simplement ces états, pas plus que nous ne sommes loués ou blâmés. De plus, nos facultés sont en nous par notre nature, alors que nous ne \\
naissons pas naturellement bons ou méchants. Mais nous avons traité ce point précédemment.\par
Si donc les vertus ne sont ni des affections, ni des facultés, il reste que ce sont des dispositions.
\subsection[{5 (1106a — 1106b) < Définition spécifique de la vertu : la vertu est une médiété >}]{5 (1106a — 1106b) < Définition spécifique de la vertu : la vertu est une médiété >}
\noindent Ainsi, nous avons établi génériquement la nature de la vertu, Mais nous ne devons pas seulement dire de la vertu qu’elle est une disposition, mais dire encore quelle espèce de \\
disposition elle est. Nous devons alors remarquer que toute « vertu », pour la chose dont elle est « vertu », a pour effet à la fois de mettre cette chose en bon état et de lui permettre de bien accomplir son œuvre propre : par exemple, la « vertu » de l’œil rend l’œil et sa fonction également parfaits, car c’est par la vertu de l’œil que la vision s’effectue en nous comme il faut. De même la « vertu » du cheval rend un cheval à la fois parfait \\
en lui-même et bon pour la course, pour porter son cavalier et faire face à l’ennemi. Si donc il en est ainsi dans tous les cas, l’excellence, la vertu de l’homme ne saurait être qu’une disposition par laquelle un homme devient bon et par laquelle aussi son œuvre propre sera rendue bonne.\par
Comment cela se fera-t-il, nous l’avons déjà indiqué, \\
mais nous apporterons un complément de clarté si nous considérons ce qui constitue la nature spécifique de la vertu.\par
En tout ce qui est continu et divisible, il est possible de distinguer le plus, le moins et l’égal, et cela soit dans la chose même, soit par rapport à nous, l’égal étant quelque moyen entre l’excès et le défaut. J’entends par {\itshape moyen dans la chose} ce \\
qui s’écarte à égale distance de chacun des deux extrêmes, point qui est unique et identique pour tous les hommes, et par {\itshape moyen par rapport à nous} ce qui n’est ni trop, ni trop peu, et c’est là une chose qui n’est ni une, ni identique pour tout le monde. Par exemple, si 10 est beaucoup, et 2 peu, 6 est le moyen pris dans la chose, car il dépasse et est dépassé par une \\
quantité égale ; et c’est là un moyen établi d’après la proportion arithmétique. Au contraire, le moyen par rapport à nous ne doit pas être pris de cette façon : si, pour la nourriture de tel  individu déterminé, un poids de 10 mines est beaucoup et un poids de 2 mines peu, il ne s’ensuit pas que le maître de gymnase prescrira un poids de 6 mines, car cette quantité est peut-être aussi beaucoup pour la personne qui l’absorbera, ou peu : pour Milon ce sera peu, et pour un débutant dans les exercices du gymnase, beaucoup. Il en est de même pour la course et la \\
lutte. C’est dès lors ainsi que l’homme versé dans une discipline quelconque évite l’excès et le défaut ; c’est le moyen qu’il recherche et qu’il choisit, mais ce moyen n’est pas celui de la chose, c’est celui qui est relatif à nous.\par
Si donc toute science aboutit ainsi à la perfection de son œuvre, en fixant le regard sur le moyen et y ramenant ses \\
œuvres (de là vient notre habitude de dire en parlant des œuvres bien réussies, qu’il est impossible d’y rien retrancher ni d’y rien ajouter, voulant signifier par là que l’excès et le défaut détruisent la perfection, tandis que la médiété la préserve), si donc les bons artistes, comme nous les appelons, ont les yeux fixés sur cette médiété quand ils travaillent, et si en outre, la vertu, comme la nature, dépasse en exactitude et en valeur \\
tout autre art, alors c’est le moyen vers lequel elle devra tendre. J’entends ici la vertu morale, car c’est elle qui a rapport à des affections et des actions, matières en lesquelles il y a excès, défaut et moyen. Ainsi, dans la crainte, l’audace, l’appétit, la colère, la pitié, et en général dans tout sentiment de plaisir et \\
de peine, on rencontre du trop et du trop peu, lesquels ne sont bons ni l’un ni l’autre ; au contraire, ressentir ces émotions au moment opportun, dans les cas et à l’égard des personnes qui conviennent, pour les raisons et de la façon qu’il faut, c’est à la fois moyen et excellence, caractère qui appartient précisément à la vertu. Pareillement encore, en ce qui concerne les actions, il peut y avoir excès, défaut et moyen. Or la vertu a rapport à \\
des affections et à des actions dans lesquelles l’excès est erreur et le défaut objet de blâme, tandis que le moyen est objet de louange et de réussite, double avantage propre à la vertu. La vertu est donc une sorte de médiété, en ce sens qu’elle vise le moyen.\par
De plus l’erreur est multiforme (car le mal relève de \\
l’Illimité, comme les Pythagoriciens l’ont conjecturé, et le bien, du Limité), tandis qu’on ne peut observer la droite règle que d’une seule façon : pour ces raisons aussi, la première est facile, et l’autre difficile ; il est facile de manquer le but, et difficile 49 de l’atteindre. Et c’est ce qui fait que le vice a pour caractéristiques l’excès et le défaut, et la vertu la médiété :\par
 {\itshape L’honnêteté n’a qu’une seule forme, mais le vice en a de} \\
 {\itshape nombreuses.} 
\subsection[{6 (1106b — 1107a) < Définition complète de la vertu morale, et précisions nouvelles >}]{6 (1106b — 1107a) < Définition complète de la vertu morale, et précisions nouvelles >}
\noindent Ainsi donc, la vertu est une disposition à agir d’une façon délibérée, consistant en une médiété relative à nous, laquelle  est rationnellement déterminée et comme le déterminerait l’homme prudent. Mais c’est une médiété entre deux vices, l’un par excès et l’autre par défaut ; et < c’est encore une médiété > en ce que certains vices sont au-dessous, et d’autres au-dessus du « ce qu’il faut » dans le domaine des affections aussi bien que des actions, tandis que la vertu, elle, découvre et \\
choisit la position moyenne, C’est pourquoi dans l’ordre de la substance [ουσια] et de la définition exprimant la quiddité [το τι ην ειναι], la vertu est une médiété, tandis que dans l’ordre de l’excellence et du parfait, c’est un sommet.\par
Mais toute action n’admet pas la médiété, ni non plus toute affection, car pour certaines d’entre elles leur seule dénomination \\
implique immédiatement la perversité, par exemple la malveillance, l’impudence, l’envie, et, dans le domaine des actions, l’adultère, le vol, l’homicide : ces affections et ces actions, et les autres de même genre, sont toutes, en effet, objets de blâme parce qu’elles sont perverses en elles-mêmes, et ce n’est pas seulement leur excès ou leur défaut que l’on condamne. Il n’est donc jamais possible de se tenir à leur sujet \\
dans la voie droite, mais elles constituent toujours des fautes, On ne peut pas non plus, à l’égard de telles choses, dire que le bien ou le mal dépend des circonstances, du fait, par exemple, que l’adultère est commis avec la femme qu’il faut, à l’époque et de la manière qui conviennent, mais le simple fait d’en commettre un, quel qu’il soit, est une faute. Il est également absurde de supposer que commettre une action injuste ou lâche ou déréglée, comporte une médiété, un excès et un défaut, car il \\
y aurait à ce compte-là une médiété d’excès et de défaut, un excès d’excès et un défaut de défaut. Mais de même que pour la modération et le courage il n’existe pas d’excès et de défaut du fait que le moyen est en un sens un extrême, ainsi pour les actions dont nous parlons il n’y a non plus ni médiété, ni excès, ni défaut, mais, quelle que soit la façon dont on les accomplit, \\
elles constituent des fautes : car, d’une manière générale, il n’existe ni médiété d’excès et de défaut, ni excès et défaut de médiété.
\subsection[{7 (1107a — 1108b) < Étude des vertus particulières >}]{7 (1107a — 1108b) < Étude des vertus particulières >}
\noindent Nous ne devons pas seulement nous en tenir à des généralités, mais encore en faire l’application aux vertus particulières. \\
En effet, parmi les exposés traitant de nos actions, ceux qui sont d’ordre général sont plus vides, et ceux qui s’attachent aux particularités plus vrais, car les actions ont rapport aux faits individuels, et nos théories doivent être en accord avec eux. Empruntons donc les exemples de vertus particulières à notre tableau.\par

\tableopen{}
\begin{tabularx}{\linewidth}
{|l|X|X|}
\hlineIrascibilité & Impassibilité & Douceur \\
\hline
Témérité & Lâcheté & Courage \\
\hline
Impudence & Embarras & Pudeur \\
\hline
Intempérance & Insensibilité & Tempérance \\
\hline
Haine (envie) & Anonyme & Indignation vertueuse \\
\hline
Gain & Perte & Le juste \\
\hline
Prodigalité & Illibéralité & Libéralité \\
\hline
Fanfaronnade & Dissimulation & Vérité \\
\hline
Flatterie & Hostilité & Amitié \\
\hline
Complaisance & Égoïsme & Dignité \\
\hline
Mollesse & Grossièreté & Endurance \\
\hline
Vanité & Pusillanimité & Magnanimité \\
\hline
Ostentation & Mesquinerie & Magnificence \\
\hline
Fourberie & Niaiserie & Sagesse \\
\hline
\end{tabularx}
\tableclose{}

\noindent  En ce qui concerne la peur et la témérité, le courage est une médiété, et parmi ceux qui pèchent par excès, celui qui le fait par manque de peur n’a pas reçu de nom (beaucoup d’états n’ont d’ailleurs pas de nom), tandis que celui qui le fait par audace est un téméraire, et celui qui tombe dans l’excès de crainte et manque d’audace est un lâche.\par
\\
Pour ce qui est des plaisirs et des peines (non pas de tous, et à un moindre degré en ce qui regarde les peines), la médiété est la modération, et l’excès le dérèglement. Les gens qui pèchent par défaut en ce qui regarde les plaisirs se rencontrent rarement, ce qui explique que de telles personnes n’ont pas non plus reçu de nom ; appelons-les des {\itshape insensibles}.\par
Pour ce qui est de l’action de donner et celle d’acquérir des richesses, la médiété est la libéralité ; l’excès et le défaut sont \\
respectivement la prodigalité et la parcimonie, C’est de façon opposée que dans ces actions on tombe dans l’excès ou le défaut : en effet, le prodigue pèche par excès dans la dépense et par défaut dans l’acquisition, tandis que le parcimonieux pèche par excès dans l’acquisition et par défaut dans la dépense. — Pour le moment, nous traçons là une simple esquisse, très \\
sommaire, qui doit nous suffire pour notre dessein ; plus tard, ces états seront définis avec plus de précision. — Au regard des richesses, il existe aussi d’autres dispositions : la médiété est la magnificence (car l’homme magnifique diffère d’un homme libéral : le premier vit dans une ambiance de grandeur, et l’autre dans une sphère plus modeste), l’excès, le manque de goût ou \\
vulgarité, le défaut la mesquinerie. Ces vices diffèrent des états opposés à la libéralité, et la façon dont ils diffèrent sera indiquée plus loin.\par
En ce qui concerne l’honneur et le mépris, la médiété est la grandeur d’âme, l’excès ce qu’on nomme une sorte de boursouflure, le défaut la bassesse d’âme, Et de même que nous avons montré la libéralité en face de la magnificence, différant \\
de cette dernière par la modicité de la sphère où elle se meut, ainsi existe-t-il pareillement, en face de la grandeur d’âme, laquelle a rapport à un honneur de grande classe, un certain état ayant rapport à un honneur plus modeste. On peut, en effet, désirer un honneur de la façon qu’on le doit, ou plus qu’on ne le doit, ou moins qu’on ne le doit ; et l’homme aux désirs excessifs s’appelle un ambitieux, l’homme aux désirs insuffisants, un homme sans ambition, tandis que celui qui tient la position \\
moyenne n’a pas reçu de dénomination. Sans désignations spéciales sont aussi les dispositions correspondantes, sauf celle de l’ambitieux, qui est l’ambition. De là vient que les extrêmes se disputent le terrain intermédiaire, et il arrive que nous-mêmes appelions celui qui occupe la position moyenne tantôt ambitieux  et tantôt dépourvu d’ambition, et que nous réservions nos éloges tantôt à l’ambitieux et tantôt à celui qui n’a pas d’ambition. Pour quelle raison agissons-nous ainsi, nous le dirons dans la suite ; pour le moment, parlons des états qui nous restent à voir, en suivant la marche que nous avons indiquée.\par
En ce qui concerne la colère, il y a aussi excès, défaut et \\
médiété. Ces états sont pratiquement dépourvus de toute dénomination. Cependant, puisque nous appelons débonnaire celui qui occupe la position moyenne, nous pouvons appeler débonnaireté la médiété elle-même. Pour ceux qui sont aux points extrêmes, irascible sera celui qui tombe dans l’excès, et le vice correspondant l’irascibilité ; et celui qui pèche par défaut sera une sorte d’être indifférent, et son vice sera l’indifférence.\par
Il y a encore trois autres médiétés ayant une certaine \\
ressemblance entre elles, tout en étant différentes les unes des autres : toutes, en effet, concernent les relations sociales entre les hommes dans les paroles et dans les actions, mais diffèrent en ce que l’une a rapport au vrai que ces paroles et ces actions renferment, les deux autres étant relatives à l’agrément soit dans le badinage, soit dans les circonstances générales de la vie. Nous devons donc parler aussi de ces divers états, de façon \\
à mieux discerner qu’en toutes choses la médiété est digne d’éloge, tandis que les extrêmes ne sont ni corrects, ni louables, mais au contraire répréhensibles. Ici encore, la plupart de ces états ne portent aucun nom ; nous devons cependant essayer, comme dans les autres cas, de forger nous-mêmes des noms, en vue de la clarté de l’exposé et pour qu’on puisse nous suivre facilement. — En ce qui regarde le vrai, la position moyenne \\
peut être appelée véridique, et la médiété véracité, tandis que la feinte par exagération est vantardise et celui qui la pratique un vantard, et la feinte par atténuation, réticence, et celui qui la pratique, un réticent. — Passons à l’agrément, et voyons d’abord celui qu’on rencontre dans le badinage : l’homme qui occupe la position moyenne est un homme enjoué, et sa disposition \\
une gaieté de bon aloi ; l’excès est bouffonnerie, et celui qui la pratique, un bouffon ; l’homme qui pèche au contraire par défaut est un rustre, et son état est la rusticité. Pour l’autre genre d’agrément, à savoir les relations agréables de la vie, l’homme agréable comme il faut est un homme aimable, et la médiété l’amabilité ; celui qui tombe dans l’excès, s’il n’a aucune fin intéressée en vue est un complaisant, et si c’est pour son avantage propre, un flatteur ; celui qui pèche par défaut et \\
qui est désagréable dans toutes les circonstances est un chicanier et un esprit hargneux.\par
Il existe aussi dans les affections et dans tout ce qui se rapporte aux affections, des médiétés. En effet, la réserve n’est pas une vertu, et pourtant on loue aussi l’homme réservé, car même en ce domaine tel homme est dit garder la position moyenne, un autre tomber dans l’excès, < un autre enfin pécher par défaut. Et celui qui tombe dans l’excès > est par exemple le \\
timide qui rougit de tout ; celui qui pèche par défaut ou qui n’a pas du tout de pudeur est un impudent ; et celui qui garde la position moyenne, un homme réservé.\par
 D’autre part, la juste indignation est une médiété entre l’envie et la malveillance, et ces états se rapportent à la peine et au plaisir qui surgissent en nous pour tout ce qui arrive au prochain : l’homme qui s’indigne s’afflige des succès immérités, l’envieux va au-delà et s’afflige de tous les succès d’autrui, < et tandis que l’homme qui s’indigne s’afflige des malheurs immérités >, \\
le malveillant, bien loin de s’en affliger, va jusqu’à s’en réjouir. Mais nous aurons l’occasion de décrire ailleurs ces divers états34. — En ce qui concerne la justice, étant donné que le sens où on la prend n’est pas simple, après avoir décrit les autres états, nous la diviserons en deux espèces et nous indiquerons pour chacune d’elles comment elle constitue \\
une médiété. — Et nous traiterons pareillement des vertus intellectuelles.
\subsection[{8 (1108b — 1109a) < Les oppositions entre les vices et la vertu >}]{8 (1108b — 1109a) < Les oppositions entre les vices et la vertu >}
\noindent Il existe ainsi trois dispositions : deux vices, l’un par excès et l’autre par défaut, et une seule vertu consistant dans la médiété ; et toutes ces dispositions sont d’une certaine façon opposées à toutes. En effet, les états extrêmes sont contraires à \\
la fois à l’état intermédiaire et l’un à l’autre, et l’état intermédiaire aux états extrêmes : de même que l’égal est plus grand par rapport au plus petit et plus petit par rapport au plus grand, ainsi les états moyens sont en excès par rapport aux états déficients, et en défaut par rapport aux états excessifs, aussi bien dans les affections que dans les actions. En effet, l’homme \\
courageux, par rapport au lâche apparaît téméraire, et par rapport au téméraire, lâche ; pareillement l’homme modéré, par rapport à l’insensible est déréglé, et par rapport au déréglé, insensible ; et l’homme libéral, par rapport au parcimonieux est un prodigue, et par rapport au prodigue, parcimonieux. De là vient que ceux qui sont aux extrêmes poussent respectivement \\
celui qui occupe le milieu vers l’autre extrême : le lâche appelle le brave un téméraire, et le téméraire l’appelle un lâche ; et dans les autres cas, le rapport est le même.\par
Ces diverses dispositions étant ainsi opposées les unes aux autres, la contrariété maxima est celle des extrêmes l’un par rapport à l’autre plutôt que par rapport au moyen, puisque ces extrêmes sont plus éloignés l’un de l’autre que du moyen, comme le grand est plus éloigné du petit, et le petit du grand, \\
qu’ils ne le sont l’un et l’autre de l’égal. En outre, il y a des extrêmes qui manifestent une certaine ressemblance avec le moyen, par exemple dans le cas de la témérité par rapport au courage, et de la prodigalité par rapport à la libéralité. Par contre, c’est entre les extrêmes que la dissemblance est à son plus haut degré ; or les choses qui sont {\itshape le plus} éloignées l’une de l’autre sont définies comme des contraires, et par conséquent les choses qui sont {\itshape plus} éloignées l’une de l’autre sont \\
aussi {\itshape plus} contraires.\par
 À l’égard du moyen, dans certains cas c’est le défaut qui lui est le plus opposé, et dans certains autres, l’excès : ainsi, au courage ce n’est pas la témérité (laquelle est un excès) qui est le plus opposé, mais la lâcheté (laquelle est un manque) ; inversement, à la modération ce n’est pas l’insensibilité, laquelle est \\
une déficience, mais bien le dérèglement, lequel est un excès. Cela a lieu pour deux raisons. La première vient de la chose elle-même : une plus grande proximité et une ressemblance plus étroite entre l’un des extrêmes et le moyen fait que nous n’opposons pas cet extrême au moyen, mais plutôt l’extrême contraire. Par exemple, du fait que la témérité paraît ressembler davantage au courage et s’en rapprocher plus étroitement, \\
et que la lâcheté y ressemble moins, c’est plutôt cette dernière que nous lui opposons : car les choses qui sont plus éloignées du moyen lui sont aussi, semble-t-il, plus contraires. — Voilà donc une première cause, qui vient de la chose elle-même. Il y en a une autre, qui vient de nous : les choses, en effet, pour lesquelles notre nature éprouve un certain penchant paraissent plus contraires au moyen. Par exemple, de nous-mêmes nous \\
ressentons un attrait naturel plus fort vers le plaisir, c’est pourquoi nous sommes davantage enclins au dérèglement qu’à une vie rangée. Nous qualifions alors plutôt de contraires au moyen les fautes dans lesquelles nous sommes plus exposés à tomber, et c’est pour cette raison que le dérèglement, qui est un excès, est plus spécialement contraire à la modération.
\subsection[{9 (1109a — 1109b) < Règles pratiques pour atteindre la vertu >}]{9 (1109a — 1109b) < Règles pratiques pour atteindre la vertu >}
\noindent \\
Qu’ainsi donc la vertu, la vertu morale, soit une médiété, et en quel sens elle l’est, à savoir qu’elle est une médiété entre deux vices, l’un par excès et l’autre par défaut, et qu’elle soit une médiété de cette sorte parce qu’elle vise la position intermédiaire dans les affections et dans les actes, — tout cela nous l’avons suffisamment établi.\par
Voilà pourquoi aussi c’est tout un travail que d’être \\
vertueux. En toute chose, en effet, on a peine à trouver le moyen : par exemple trouver le centre d’un cercle n’est pas à la portée de tout le monde, mais seulement de celui qui sait. Ainsi également, se livrer à la colère est une chose à la portée de n’importe qui, et bien facile, de même donner de l’argent et le dépenser ; mais le faire avec la personne qu’il faut, dans la mesure et au moment convenables, pour un motif et d’une façon légitimes, c’est là une œuvre qui n’est plus le fait de tous, ni d’exécution facile, et c’est ce qui explique que le bien soit à la fois une chose rare, digne d’éloge et belle.\par
\\
Aussi celui qui cherche à atteindre la position moyenne doit-il tout d’abord s’éloigner de ce qui y est le plus contraire, et suivre le conseil de Calypso :\par
 {\itshape Hors de cette vapeur et de cette houle, écarte} \par
 {\itshape Ton vaisseau.} \par
En effet, des deux extrêmes l’un nous induit plus en faute que l’autre ; par suite, étant donné qu’il est extrêmement difficile \\
d’atteindre le moyen, nous devons, comme on dit, {\itshape changer de navigation}, et choisir le moindre mal, et la meilleure façon  d’y arriver sera celle que nous indiquons.\par
Mais nous devons, en second lieu, considérer quelles sont les fautes pour lesquelles nous-mêmes avons le plus fort penchant, les uns étant naturellement attirés vers telles fautes et les autres vers telles autres, Nous reconnaîtrons cela au plaisir et à la peine que nous en ressentons. Nous devons nous \\
en arracher nous-mêmes vers la direction opposée, car ce n’est qu’en nous écartant loin des fautes que nous commettons, que nous parviendrons à la position moyenne, comme font ceux qui redressent le bois tordu.\par
En toute chose, enfin, il faut surtout se tenir en garde contre ce qui est agréable et contre le plaisir, car en cette matière nous ne jugeons pas avec impartialité. Ce que les \\
Anciens du peuple ressentaient pour Hélène, nous devons nous aussi le ressentir à l’égard du plaisir, et en toutes circonstances appliquer leurs paroles : en répudiant ainsi le plaisir, nous serons moins sujets à faillir. Et si nous agissons ainsi, pour le dire d’un mot, nous nous trouverons dans les conditions les plus favorables pour atteindre le moyen.\par
Mais sans doute est-ce là une tâche difficile, surtout quand \\
on passe aux cas particuliers. Il n’est pas aisé, en effet, de déterminer par exemple de quelle façon, contre quelles personnes, pour quelles sortes de raisons et pendant combien de temps on doit se mettre en colère, puisque nous-mêmes accordons nos éloges tantôt à ceux qui pèchent par défaut en cette matière, et que nous qualifions de {\itshape doux}, tantôt à ceux qui sont d’un caractère irritable et que nous nommons des gens {\itshape virils}. Cependant celui qui dévie légèrement de la droite ligne, que ce soit du côté de l’excès ou du côté du défaut, n’est pas répréhensible ; l’est \\
seulement celui dont les écarts sont par trop considérables, car celui-là ne passe pas inaperçu. Quant à dire jusqu’à quel point et dans quelle mesure la déviation est répréhensible, c’est là une chose qu’il est malaisé de déterminer rationnellement, comme c’est d’ailleurs le cas pour tous les objets perçus par les sens : de telles précisions sont du domaine de l’individuel, et la discrimination est du ressort de la sensation. Mais nous en avons dit assez pour montrer que l’état qui occupe la position \\
moyenne est en toutes choses digne de notre approbation, mais que nous devons pencher tantôt vers l’excès, tantôt vers le défaut, puisque c’est de cette façon que nous atteindrons avec le plus de facilité le juste milieu et le bien.
\section[{Livre III}]{Livre III}\renewcommand{\leftmark}{Livre III}

\subsection[{1 (1109b — 1110b) < Actes volontaires et actes involontaires. De la contrainte >}]{1 (1109b — 1110b) < Actes volontaires et actes involontaires. De la contrainte >}
\noindent \\
Puisque la vertu a rapport à la fois à des affections et à des actions, et que ces états peuvent être soit volontaires, et encourir l’éloge ou le blâme, soit involontaires, et provoquer l’indulgence et parfois même la pitié, il est sans doute indispensable, pour ceux qui font porter leur examen sur la vertu, de distinguer entre le volontaire et l’involontaire ; et cela est également utile au législateur pour établir des récompenses et des châtiments.\par
\\
On admet d’ordinaire qu’un acte est involontaire quand  il est fait sous la contrainte, ou par ignorance. Est fait par contrainte tout ce qui a son principe hors de nous, c’est-à-dire un principe dans lequel on ne relève aucun concours de l’agent ou du patient : si, par exemple, on est emporté quelque part, soit par le vent, soit par des gens qui vous tiennent en leur pouvoir.\par
Mais pour les actes accomplis par crainte de plus grands \\
maux ou pour quelque noble motif (par exemple, si un tyran nous ordonne d’accomplir une action honteuse, alors qu’il tient en son pouvoir nos parents et nos enfants, et qu’en accomplissant cette action nous assurerions leur salut, et en refusant de la faire, leur mort), pour de telles actions la question est débattue de savoir si elles sont volontaires ou involontaires. C’est là encore ce qui se produit dans le cas d’une cargaison que l’on jette par-dessus bord au cours d’une tempête : dans l’absolu, personne ne se débarrasse ainsi de son bien volontairement, \\
mais quand il s’agit de son propre salut et de celui de ses compagnons un homme de sens agit toujours ainsi. De telles actions sont donc mixtes, tout en ressemblant plutôt à des actions volontaires, car elles sont librement choisies au moment où on les accomplit, et la fin de l’action varie avec les circonstances de temps. On doit donc, pour qualifier une action de \\
volontaire ou d’involontaire, se référer au moment où elle s’accomplit. Or ici l’homme agit volontairement, car le principe qui, en de telles actions, meut les parties instrumentales de son corps, réside en lui, et les choses dont le principe est en l’homme même, il dépend de lui de les faire ou de ne pas les faire. Volontaires sont donc les actions de ce genre, quoique dans l’absolu elles soient peut-être involontaires, puisque personne ne choisirait jamais une pareille action en elle-même.\par
\\
Les actions de cette nature sont aussi parfois objet d’éloge quand on souffre avec constance quelque chose de honteux ou d’affligeant en contrepartie de grands et beaux avantages ; dans le cas opposé, au contraire, elles sont objet de blâme, car endurer les plus grandes indignités pour n’en retirer qu’un avantage nul ou médiocre est le fait d’une âme basse. Dans le cas de certaines actions, ce n’est pas l’éloge qu’on provoque, mais l’indulgence : c’est lorsqu’on accomplit une action qu’on \\
ne doit pas faire, pour éviter des maux qui surpassent les forces humaines et que personne ne pourrait supporter. Cependant il existe sans doute des actes qu’on ne peut jamais être contraint d’accomplir, et auxquels nous devons préférer subir la mort la plus épouvantable : car les motifs qui ont contraint par exemple l’Alcméon d’Euripide à tuer sa mère apparaissent bien ridicules. Et s’il est difficile parfois de discerner, dans une action \\
donnée, quel parti nous devons adopter et à quel prix, ou quel mal nous devons endurer en échange de quel avantage, il est encore plus difficile de persister dans ce que nous avons décidé, car la plupart du temps ce à quoi l’on s’attend est pénible et ce qu’on est contraint de faire, honteux ; et c’est pourquoi louange et blâme nous sont dispensés suivant que nous cédons ou que nous résistons à cette contrainte.\par
 Quelles sortes d’actions faut-il dès lors appeler forcées ? Ne devons-nous pas dire qu’au sens absolu, c’est lorsque leur cause réside dans les choses hors de nous, et que l’agent n’y a en rien contribué ? Les actions qui, en elles-mêmes, sont involontaires, mais qui, à tel moment et en retour d’avantages déterminés, ont été librement choisies et dont le principe réside dans l’agent, sont assurément en elles-mêmes involontaires, \\
mais, à tel moment et en retour de tels avantages, deviennent volontaires et ressemblent plutôt à des actions volontaires : car les actions font partie des choses particulières, et ces actions particulières sont ici volontaires. Mais quelles sortes de choses doit-on choisir à la place de quelles autres, cela n’est pas aisé à établir, car il existe de multiples diversités dans les actes particuliers.\par
\\
Et si on prétendait que les choses agréables et les choses nobles ont une force contraignante (puisqu’elles agissent sur nous de l’extérieur), toutes les actions seraient à ce compte-là des actions forcées, car c’est en vue de ces satisfactions qu’on accomplit toujours toutes ses actions. De plus, les actes faits par contrainte et involontairement sont accompagnés d’un sentiment de tristesse, tandis que les actes ayant pour fin une chose agréable ou noble sont faits avec plaisir. Il est dès lors \\
ridicule d’accuser les choses extérieures et non pas soi-même, sous prétexte qu’on est facilement capté par leurs séductions, et de ne se considérer soi-même comme cause que des bonnes actions, rejetant la responsabilité des actions honteuses sur la force contraignante du plaisir.\par
Ainsi donc, il apparait bien que l’acte forcé soit celui qui a son principe hors de nous, sans aucun concours de l’agent qui subit la contrainte.
\subsection[{2 (1110b — 1111a) < Actes involontaires résultant de l’ignorance >}]{2 (1110b — 1111a) < Actes involontaires résultant de l’ignorance >}
\noindent L’acte fait par ignorance est toujours non volontaire ; il n’est involontaire que si l’agent en éprouve affliction et repentir. En effet, l’homme qui, après avoir accompli par ignorance \\
une action quelconque, ne ressent aucun déplaisir de son acte, n’a pas agi volontairement, puisqu’il ne savait pas ce qu’il faisait, mais il n’a pas non plus agi involontairement, puisqu’il n’en éprouve aucun chagrin. Les actes faits par ignorance sont dès lors de deux sortes : si l’agent en ressent du repentir, on estime qu’il a agi involontairement ; et s’il ne se repent pas, on pourra dire, pour marquer la distinction avec le cas précédent, qu’il a agi {\itshape non} volontairement : puisque ce second cas est différent du premier, il est préférable, en effet, de lui donner un nom qui lui soit propre.\par
\\
Il y a aussi, semble-t-il bien, une différence entre agir par ignorance et accomplir un acte {\itshape dans} l’ignorance : ainsi, l’homme ivre ou l’homme en colère, pense-t-on, agit non par ignorance mais par l’une des causes que nous venons de mentionner, bien qu’il ne sache pas ce qu’il fait mais se trouve en état d’ignorance. Ainsi donc, tout homme pervers ignore les choses qu’il doit faire et celles qu’il doit éviter, et c’est cette sorte d’erreur qui engendre chez l’homme l’injustice et le vice \\
en général. Mais on a tort de vouloir appliquer l’expression {\itshape involontaire} à une action dont l’auteur est dans l’ignorance de ce qui lui est avantageux. En effet, ce n’est pas l’ignorance dans le choix délibéré qui est cause du caractère involontaire de l’acte (elle est seulement cause de sa perversité), et ce n’est pas non plus l’ignorance des règles générales de conduite (puisque une ignorance de ce genre attire le blâme) : < ce qui rend l’action involontaire >, c’est l’ignorance des particularités  de l’acte, c’est-à-dire de ses circonstances et de son objet, car c’est dans ces cas-là que s’exercent la pitié et l’indulgence, parce que celui qui est dans l’ignorance de quelqu’un de ces facteurs agit involontairement.\par
Dans ces conditions, il n’est peut-être pas sans intérêt de déterminer quelle est la nature et le nombre de ces particularités. Elles concernent : l’agent lui-même ; l’acte ; la personne ou la chose objet de l’acte ; quelquefois encore ce par quoi l’acte est fait (c’est-à-dire l’instrument) ; le résultat qu’on en \\
attend (par exemple, sauver la vie d’un homme) ; la façon enfin dont il est accompli (doucement, par exemple, ou avec force). Ces différentes circonstances, personne, à moins d’être fou, ne saurait les ignorer toutes à la fois ; il est évident aussi que l’ignorance ne peut pas non plus porter sur l’agent, car comment s’ignorer soi-même ? Par contre, l’ignorance peut porter sur l’acte, comme, par exemple, quand on dit : {\itshape cela leur a échappé en parlant, ou ils ne savaient pas qu’il s’agissait} \\
{\itshape de choses secrètes}, comme Eschyle le dit des Mystères, ou {\itshape voulant seulement faire une démonstration}, il a lâché le trait, comme le disait l’homme au catapulte. On peut aussi prendre son propre fils pour un ennemi, comme Mérope, ou une lance acérée pour une lance mouchetée, ou une pierre ordinaire pour une pierre ponce ; ou encore, avec l’intention de lui sauver la vie, tuer quelqu’un en lui donnant une potion ; ou en voulant le \\
toucher légèrement, comme dans la lutte à main plate, le frapper pour de bon. L’ignorance pouvant dès lors porter sur toutes ces circonstances au sein desquelles l’action se produit, l’homme qui a ignoré l’une d’entre elles est regardé comme ayant agi involontairement, surtout si son ignorance porte sur les plus importantes, et parmi les plus importantes sont, semble-t-il, celles qui tiennent à l’acte lui-même et au résultat qu’on espérait.\par
Telle est donc la sorte d’ignorance qui permet d’appeler \\
un acte, involontaire, mais encore faut-il que cet acte soit accompagné, chez son auteur, d’affliction et de repentir.
\subsection[{3 (1111a — 1111b) < Acte volontaire >}]{3 (1111a — 1111b) < Acte volontaire >}
\noindent Étant donné que ce qui est fait sous la contrainte ou par ignorance est involontaire, l’acte volontaire semblerait être ce dont le principe réside dans l’agent lui-même connaissant les circonstances particulières au sein desquelles son action se produit. Sans doute, en effet, est-ce à tort qu’on appelle involontaires \\
les actes faits par impulsivité ou par concupiscence. D’abord, à ce compte-là on ne pourrait plus dire qu’un animal agit de son plein gré, ni non plus un enfant. Ensuite, est-ce que nous n’accomplissons jamais volontairement les actes qui sont dus à la concupiscence ou à l’impulsivité, ou bien serait-ce que les bonnes actions sont faites volontairement, et les actions honteuses involontairement ? Une telle assertion n’est-elle pas ridicule, alors qu’une seule et même personne est la cause des unes comme des autres ? Mais sans doute est-il absurde de \\
décrire comme involontaires ce que nous avons le devoir de désirer : or nous avons le devoir, à la fois de nous emporter dans certains cas, et de ressentir de l’appétit pour certaines choses, par exemple pour la santé et l’étude. D’autre part, on admet que les actes involontaires s’accompagnent d’affliction, et les actes faits par concupiscence, de plaisir. En outre, quelle différence y a-t-il, sous le rapport de leur nature involontaire, entre les erreurs commises par calcul, et celles commises par impulsivité ? On doit éviter les unes comme les autres, et il  nous semble aussi que les passions irrationnelles ne relèvent pas moins de l’humaine nature, de sorte que les actions qui procèdent de l’impulsivité ou de la concupiscence appartiennent aussi à l’homme qui les accomplit. Il est dès lors absurde de poser ces actions comme involontaires.
\subsection[{4 (1111b — 1112a) < Analyse du choix préférentiel >}]{4 (1111b — 1112a) < Analyse du choix préférentiel >}
\noindent Après avoir défini à la fois l’acte volontaire et l’acte \\
involontaire, nous devons ensuite traiter en détail du choix préférentiel [προαιρεσις] : car cette notion semble bien être étroitement apparentée à la vertu, et permettre, mieux que les actes, de porter un jugement sur le caractère de quelqu’un.\par
Ainsi donc, le choix est manifestement quelque chose de volontaire, tout en n’étant pas cependant identique à l’acte volontaire, lequel a une plus grande extension. En effet, tandis qu’à l’action volontaire enfants et animaux ont part, il n’en est pas de même pour le choix ; et les actes accomplis spontanément, nous pouvons bien les appeler volontaires, \\
mais non pas dire qu’ils sont faits par choix.\par
Ceux qui prétendent que le choix est un appétit, ou une impulsivité, ou un souhait, ou une forme de l’opinion, soutiennent là, semble-t-il, une vue qui n’est pas correcte.\par
En effet, le choix n’est pas une chose commune à l’homme et aux êtres dépourvus de raison, à la différence de ce qui a lieu pour la concupiscence et l’impulsivité. De plus, l’homme intempérant agit par concupiscence, mais non par choix, tandis \\
que l’homme maître de lui, à l’inverse, agit par choix et non par concupiscence. En outre, un appétit peut être contraire à un choix, mais non un appétit à un appétit. Enfin, l’appétit relève du plaisir et de la peine, tandis que le choix ne relève ni de la peine, ni du plaisir.\par
Encore moins peut-on dire que le choix est une impulsion, car les actes dus à l’impulsivité semblent être tout ce qu’il y a de plus étranger à ce qu’on fait par choix.\par
\\
Mais le choix n’est certainement pas non plus un souhait, bien qu’il en soit visiblement fort voisin. Il n’y a pas de choix, en effet, des choses impossibles, et si on prétendait faire porter son choix sur elles on passerait pour insensé ; au contraire, il peut y avoir souhait des choses impossibles, par exemple de l’immortalité. D’autre part, le souhait peut porter sur des choses qu’on ne saurait d’aucune manière mener à bonne fin par soi-même, par exemple faire que tel acteur ou tel athlète \\
remporte la victoire ; au contraire, le choix ne s’exerce jamais sur de pareilles choses, mais seulement sur celles qu’on pense pouvoir produire par ses propres moyens. En outre, le souhait porte plutôt sur la fin, et le choix, sur les moyens pour parvenir à la fin : par exemple, nous souhaitons être en bonne santé, mais nous choisissons les moyens qui nous feront être en bonne santé ; nous pouvons dire encore que nous souhaitons d’être heureux, mais il est inexact de dire que nous choisissons \\
de l’être : car, d’une façon générale, le choix porte, selon toute apparence, sur les choses qui dépendent de nous. On ne peut pas non plus dès lors identifier le choix à l’opinion. L’opinion, en effet, semble-t-il bien, a rapport à toute espèce d’objets, et non moins aux choses éternelles ou impossibles qu’aux choses qui sont dans notre dépendance ; elle se divise selon le vrai et le faux, et non selon le bien et le mal, tandis que le choix, c’est plutôt selon le bien et le mal qu’il se partage.\par
 À l’opinion prise en général, personne sans doute ne prétend identifier le choix ; mais le choix ne peut davantage s’identifier avec une certaine sorte d’opinion. En effet, c’est le choix que nous faisons de ce qui est bien ou de ce qui est mal qui détermine la qualité de notre personne morale, et nullement nos opinions. Et tandis que nous choisissons de saisir ou de fuir quelque bien ou quelque mal, nous opinons sur la nature d’une chose, ou sur la personne à qui cette chose est utile, ou enfin sur \\
la façon de s’en servir ; mais on peut difficilement dire que nous avons l’opinion de saisir ou de fuir quelque chose. En outre, le choix est loué plutôt parce qu’il s’exerce sur un objet conforme au devoir qu’en raison de sa propre rectitude à l’égard de cet objet ; pour l’opinion, au contraire, c’est parce qu’elle est dans un rapport véridique avec l’objet. Et nous choisissons les choses que nous savons, de la science la plus certaine, être bonnes, tandis que nous avons des opinions sur ce que nous ne savons qu’imparfaitement. Enfin, il apparaît que ce ne sont pas les mêmes personnes qui à la fois pratiquent \\
les meilleurs choix et professent les meilleures opinions : certaines gens ont d’excellentes opinions, mais par perversité choisissent de faire ce qui est illicite. — Que l’opinion précède le choix ou l’accompagne, peu importe ici : ce n’est pas ce point que nous examinons, mais s’il y a identité du choix avec quelque genre d’opinion.\par
Qu’est-ce donc alors que le choix, ou quelle sorte de chose est-ce, puisqu’il n’est rien de tout ce que nous venons de dire ? Il est manifestement une chose volontaire, mais tout ce qui est volontaire n’est pas objet de choix. Ne serait-ce pas, en réalité, \\
le prédélibéré ? Le choix, en effet, s’accompagne de raison et de pensée discursive. Et même son appellation semble donner à entendre que c’est ce qui a été {\itshape choisi avant} d’autres choses.
\subsection[{5 (1112a — 1113a) < Analyse de la délibération. Son objet >}]{5 (1112a — 1113a) < Analyse de la délibération. Son objet >}
\noindent Est-ce qu’on délibère sur toutes choses, autrement dit est-ce que toute chose est objet de délibération, ou bien y a-t-il certaines choses dont il n’y a pas délibération ? Nous devons \\
sans doute appeler un objet de délibération non pas ce sur quoi délibérerait un imbécile, ou un fou, mais ce sur quoi peut délibérer un homme sain d’esprit. Or, sur les entités éternelles il n’y a jamais de délibération : par exemple, l’ordre du Monde ou l’incommensurabilité de la diagonale avec le côté du carré. Il n’y a pas davantage de délibération sur les choses qui sont en mouvement mais se produisent toujours de la même façon, \\
soit par nécessité, soit par nature, soit par quelque autre cause : tels sont par exemple, les solstices et le lever des astres. Il n’existe pas non plus de délibération sur les choses qui arrivent tantôt d’une façon, tantôt d’une autre, par exemple les sécheresses et les pluies, ni sur les choses qui arrivent par fortune, par exemple la découverte d’un trésor. Bien plus : la délibération ne porte même pas sur toutes les affaires humaines sans exception : ainsi, aucun Lacédémonien ne délibère sur la meilleure forme de gouvernement pour les Scythes. C’est \\
qu’en effet, rien de tout ce que nous venons d’énumérer ne pourrait être produit par nous.\par
Mais nous délibérons sur les choses qui dépendent de nous et que nous pouvons réaliser : et ces choses-là sont, en fait, tout ce qui reste, car on met communément au rang des causes, nature, nécessité et fortune, et on y ajoute l’intellect et toute action dépendant de l’homme. Et chaque classe d’hommes délibère sur les choses qu’ils peuvent réaliser par eux-mêmes.\par
 Dans le domaine des sciences, celles qui sont précises et pleinement constituées ne laissent pas place à la délibération : par exemple, en ce qui concerne les lettres de l’alphabet (car nous n’avons aucune incertitude sur la façon de les écrire). Par contre, tout ce qui arrive par nous et dont le résultat n’est pas toujours le même, voilà ce qui fait l’objet de nos délibérations : par exemple, les questions de médecine ou d’affaires d’argent. \\
Et nous délibérons davantage sur la navigation que sur la gymnastique, vu que la navigation a été étudiée d’une façon moins approfondie, et ainsi de suite pour le reste. De même nous délibérons davantage sur les arts que sur les sciences, car nous sommes à leur sujet dans une plus grande incertitude. La délibération a lieu dans les choses qui, tout en se produisant avec fréquence, demeurent incertaines dans leur aboutissement, ainsi que là où l’issue est indéterminée. Et nous nous \\
faisons assister d’autres personnes pour délibérer sur les questions importantes, nous défiant de notre propre insuffisance à discerner ce qu’il faut faire.\par
Nous délibérons non pas sur les fins elles-mêmes, mais sur les moyens d’atteindre les fins. Un médecin ne se demande pas s’il doit guérir son malade, ni un orateur s’il entraînera la persuasion, ni un politique s’il établira de bonnes lois, et dans les autres domaines on ne délibère jamais non plus sur la fin à \\
atteindre. Mais, une fois qu’on a posé la fin, on examine comment et par quels moyens elle se réalisera ; et s’il apparaît qu’elle peut être produite par plusieurs moyens, on cherche lequel entraînera la réalisation la plus facile et la meilleure. Si au contraire la fin ne s’accomplit que par un seul moyen, on considère comment par ce moyen elle sera réalisée, et ce moyen à son tour par quel moyen il peut l’être lui-même, jusqu’à ce qu’on arrive à la cause immédiate, laquelle, dans l’ordre de la \\
découverte, est dernière. En effet, quand on délibère on semble procéder, dans la recherche et l’analyse dont nous venons de décrire la marche, comme dans la construction d’une figure (s’il est manifeste que toute recherche n’est pas une délibération, par exemple l’investigation en mathématiques, par contre toute délibération est une recherche), et ce qui vient dernier dans l’analyse est premier dans l’ordre de la génération. Si on \\
se heurte à une impossibilité, on abandonne la recherche, par exemple s’il nous faut de l’argent et qu’on ne puisse pas s’en procurer ; si au contraire une chose apparaît possible, on essaie d’agir.\par
Sont possibles les choses qui peuvent être réalisées par nous, < et cela au sens large >, car celles qui se réalisent par nos amis sont en un sens réalisées par nous, puisque le principe de leur action est en nous. L’objet de nos recherches, c’est tantôt l’instrument lui-même, \\
tantôt son utilisation. Il en est de même dans les autres domaines : c’est tantôt l’instrument, tantôt la façon de s’en servir, autrement dit par quel moyen. Il apparaît ainsi, comme nous l’avons dit, que l’homme est principe de ses actions et que la délibération porte sur les choses qui sont réalisables par l’agent lui-même ; et nos actions tendent à d’autres fins qu’elles-mêmes. En effet, la fin ne saurait être un objet de délibération, mais seulement les moyens  en vue de la fin. Mais il faut exclure aussi les choses particulières, par exemple si ceci est du pain, ou si ce pain a été cuit comme il faut, car ce sont là matières à sensation. — Et si on devait toujours délibérer, on irait à l’infini. L’objet de la délibération et l’objet du choix sont identiques, sous cette réserve que lorsqu’une chose est choisie elle a déjà été déterminée, puisque c’est la chose jugée préférable \\
à la suite de la délibération qui est choisie. En effet, chacun cesse de rechercher comment il agira quand il a ramené à lui-même le principe de son acte, et à la partie directrice de lui-même, car c’est cette partie qui choisit. Ce que nous disons là s’éclaire encore à la lumière des antiques constitutions qu’Homère nous a dépeintes : les rois annonçaient à leur peuple le parti qu’ils avaient adopté. \\
L’objet du choix étant, parmi les choses en notre pouvoir, un objet de désir sur lequel on a délibéré, le choix sera un désir délibératif des choses qui dépendent de nous ; car une fois que nous avons décidé à la suite d’une délibération, nous désirons alors conformément à notre délibération. 6 ( — 1113b) < Analyse du souhait raisonné > Ainsi donc, nous pouvons considérer avoir décrit le choix dans ses grandes lignes, déterminé la nature de ses objets et établi qu’il s’applique aux moyens conduisant à la fin. \\
Passons au souhait. Qu’il ait pour objet la fin elle-même, nous l’avons déjà indiqué ; mais tandis qu’aux yeux de certains40 son objet est le bien véritable, pour d’autres, au contraire, c’est le bien apparent. Mais ceux pour qui le bien véritable est l’objet du souhait, en arrivent logiquement à ne pas reconnaître pour objet de souhait ce que souhaite l’homme qui choisit une fin injuste (car si on admettait que c’est là un objet de souhait, on admettrait aussi que c’est une chose bonne ; or, dans le cas supposé, on souhaitait une chose mauvaise). \\
En revanche, ceux pour qui c’est le bien apparent qui est objet de 6\\
40 Platon (Gorgias, 466e et ss.). souhait, sont amenés à dire qu’il n’y a pas d’objet de souhait par nature, mais que c’est seulement ce qui semble bon à chaque individu : or telle chose paraît bonne à l’un, et telle autre chose à l’autre, sans compter qu’elles peuvent même, le cas échéant, être en opposition.\par
Si ces conséquences ne sont guère satisfaisantes, ne doit-on pas dire que, dans l’absolu et selon la vérité, c’est le bien réel qui est l’objet du souhait, mais que pour chacun de nous c’est ce qui lui apparaît comme tel ? Que, par conséquent, pour \\
l’honnête homme, c’est ce qui est véritablement un bien, tandis que pour le méchant c’est tout ce qu’on voudra ? N’en serait-il pas comme dans le cas de notre corps : un organisme en bon état trouve salutaire ce qui est véritablement tel, alors que pour un organisme débilité ce sera autre chose qui sera salutaire ; et il en serait de même pour les choses amères, douces, chaudes, pesantes, et ainsi de suite dans chaque cas ? En effet, l’homme de bien juge toutes choses avec rectitude, et toutes lui apparaissent \\
comme elles sont véritablement. C’est que, à chacune des dispositions de notre nature il y a des choses bonnes et agréables qui lui sont appropriées ; et sans doute, ce qui distingue principalement l’homme de bien, c’est qu’il perçoit en toutes choses la vérité qu’elles renferment, étant pour elles en quelque sorte une règle et une mesure. Chez la plupart des hommes, au contraire, l’erreur semble bien avoir le plaisir pour cause, car, tout en n’étant pas un bien, il en a l’apparence ;  aussi choisissent-ils ce qui est agréable comme étant un bien, et évitent-ils ce qui est pénible comme étant un mal.
\subsection[{7 (1113b — 1114b) < La vertu et le vice sont volontaires >}]{7 (1113b — 1114b) < La vertu et le vice sont volontaires >}
\noindent La fin étant ainsi objet de souhait, et les moyens pour atteindre à la fin, objets de délibération et de choix, les actions concernant ces moyens seront faites par choix et seront volontaires ; \\
or l’activité vertueuse a rapport aux moyens ; par conséquent, la vertu dépend aussi de nous. Mais il en est également ainsi pour le vice. En effet, là où il dépend de nous d’agir, il dépend de nous aussi de ne pas agir, et là où il dépend de nous de dire non, il dépend aussi de nous de dire oui ; par conséquent, si agir, quand l’action est bonne, dépend de nous, ne pas agir, quand l’action est honteuse, dépendra aussi de nous, et si \\
ne pas agir, quand l’abstention est bonne, dépend de nous, agir, quand l’action est honteuse, dépendra aussi de nous. Mais s’il dépend de nous d’accomplir les actions bonnes et les actions honteuses, et pareillement encore de ne pas les accomplir, et si c’est là essentiellement, disions-nous, être bons ou mauvais, il en résulte qu’il est également en notre pouvoir d’être intrinsèquement vertueux ou vicieux. La maxime suivant laquelle :\par
 \\
 {\itshape Nul n’est volontairement pervers, ni malgré soi bienheureux.} \par
est, semble-t-il, partiellement vraie et partiellement fausse. Si personne, en effet, n’est bienheureux à contrecœur, par contre la perversité est bien volontaire. Ou alors, il faut remettre en question ce que nous avons déjà soutenu, et refuser à l’homme d’être principe et générateur de ses actions, comme il l’est de ses enfants. Mais s’il est manifeste que l’homme est bien l’auteur de ses propres actions, et si nous ne pouvons pas \\
ramener nos actions à d’autres principes que ceux qui sont en nous, alors les actions dont les principes sont en nous dépendent elles-mêmes de nous et sont volontaires.\par
En faveur de ces considérations, on peut, semble-t-il, appeler en témoignage à la fois le comportement des individus dans leur vie privée et la pratique des législateurs eux-mêmes : on châtie, en effet, et on oblige à réparation ceux qui commettent des actions perverses, à moins qu’ils n’aient agi sous la contrainte ou par une ignorance dont ils ne sont pas \\
eux-mêmes causes, et, d’autre part, on honore ceux qui accomplissent de bonnes actions, et on pense ainsi encourager ces derniers et réprimer les autres. Mais les choses qui ne dépendent pas de nous et ne sont pas volontaires, personne n’engage à les faire, attendu qu’on perdrait son temps à nous persuader de ne pas avoir chaud, de ne pas souffrir, de ne pas avoir faim, et ainsi de suite, puisque nous n’en serons pas moins sujets \\
à éprouver ces impressions. Et, en effet, nous punissons quelqu’un pour son ignorance même, si nous le tenons pour responsable de son ignorance, comme par exemple dans le cas d’ébriété où les pénalités des délinquants sont doublées41, parce que le principe de l’acte réside dans l’agent lui-même, qui était maître de ne pas s’enivrer et qui est ainsi responsable de son ignorance. On punit également ceux qui sont dans l’ignorance de quelqu’une de ces dispositions légales dont la connaissance est obligatoire et ne présente aucune difficulté. Et nous  agissons de même toutes les autres fois où l’ignorance nous paraît résulter de la négligence, dans l’idée qu’il dépend des intéressés de ne pas demeurer dans l’ignorance, étant maîtres de s’appliquer à s’instruire.\par
Mais sans doute, < dira-t-on >, un pareil homme est fait de telle sorte qu’il est incapable de toute application ? Nous répondons qu’en menant une existence relâchée les hommes sont personnellement responsables d’être devenus eux-mêmes \\
relâchés, ou d’être devenus injustes ou intempérants, dans le premier cas en agissant avec perfidie et dans le second en passant leur vie à boire ou à commettre des excès analogues : en effet, c’est par l’exercice des actions particulières qu’ils acquièrent un caractère du même genre qu’elles. On peut s’en rendre compte en observant ceux qui s’entraînent en vue d’une compétition ou d’une activité quelconque : tout leur temps se passe en exercices. Aussi, se refuser à reconnaître que c’est à \\
l’exercice de telles actions particulières que sont dues les dispositions de notre caractère est le fait d’un esprit singulièrement étroit. En outre, il est absurde de supposer que l’homme qui commet des actes d’injustice ou d’intempérance ne souhaite pas être injuste ou intempérant ; et si, sans avoir l’ignorance pour excuse, on accomplit des actions qui auront pour conséquence de nous rendre injuste, c’est volontairement qu’on sera injuste. Il ne s’ensuit pas cependant qu’un simple souhait suffira pour cesser d’être injuste et pour être juste, pas plus que \\
ce n’est ainsi que le malade peut recouvrer la santé, quoiqu’il puisse arriver qu’il soit malade volontairement en menant une vie intempérante et en désobéissant à ses médecins : c’est au début qu’il lui était alors possible de ne pas être malade, mais une fois qu’il s’est laissé aller, cela ne lui est plus possible, de même que si vous avez lâché une pierre vous n’êtes plus capable de la rattraper, mais pourtant il dépendait de vous de la jeter et de la lancer, car le principe de votre acte était en vous. Ainsi en est-il pour l’homme injuste ou intempérant : au début \\
il leur était possible de ne pas devenir tels, et c’est ce qui fait qu’ils le sont volontairement ; et maintenant qu’ils le sont devenus, il ne leur est plus possible de ne pas l’être.\par
Et non seulement les vices de l’âme sont volontaires, mais ceux du corps le sont aussi chez certains hommes, lesquels encourent pour cela le blâme de notre part. Aux hommes qui sont laids par nature, en effet, nous n’adressons aucun reproche, tandis que nous blâmons ceux qui le sont par défaut \\
d’exercice et de soin. Même observation en ce qui concerne la faiblesse ou l’infirmité corporelle : on ne fera jamais grief à quelqu’un d’être aveugle de naissance ou à la suite d’une maladie ou d’une blessure, c’est plutôt de la pitié qu’on ressentira ; par contre, chacun blâmera celui qui devient aveugle par l’abus du vin ou par une autre forme d’intempérance. Ainsi donc, parmi les vices du corps, ce sont ceux qui sont sous notre dépendance qui encourent le blâme, à l’exclusion de ceux qui \\
ne dépendent pas de nous. Mais s’il en est ainsi, dans les autres cas également les vices qui nous sont reprochés doivent aussi être des vices qui dépendent de nous.\par
Objectera-t-on que tous les hommes ont en vue le bien qui leur apparaît comme tel, mais qu’on n’est pas maître de ce que telle chose nous apparaît comme bonne, et que le tempérament  de chacun détermine la façon dont la fin lui apparaît. < À cela nous répliquons > que si chacun est en un sens cause de ses propres dispositions, il sera aussi en un sens cause de l’apparence ; sinon personne n’est responsable de sa mauvaise conduite, mais c’est par ignorance de la fin qu’il \\
accomplit ses actions, pensant qu’elles lui procureront le bien le plus excellent ; et la poursuite de la fin n’est pas ainsi l’objet d’un choix personnel, mais exige qu’on soit né, pour ainsi dire, avec un œil qui nous permettra de juger sainement et de choisir le bien véritable ; et on est bien doué quand la nature s’est montrée libérale pour nous à cet égard (c’est là, en effet, le plus grand et le plus beau des dons, et qu’il n’est pas possible de \\
recevoir ou d’apprendre d’autrui, mais qu’on possédera tel qu’on l’a reçu en naissant, et le fait d’être heureusement et noblement doué par la nature sur ce point constituera, au sens complet et véritable, un bon naturel). Si dès lors ces considérations sont vraies, en quoi la vertu sera-t-elle plus volontaire que le vice ? Dans les deux cas la situation est la même : pour l’homme bon comme pour le méchant, la fin apparaît et se trouve posée par nature ou de la façon que l’on voudra, et c’est \\
en se référant pour tout le reste à cette fin qu’ils agissent en chaque cas. Qu’on admette donc que pour tout homme, la vue qu’il a de sa fin, quelle que soit cette fin, ne lui est pas donnée par la nature mais qu’elle est due en partie à lui-même, ou qu’on admette que la fin est bien donnée par la nature, mais que l’homme de bien accomplissant tout le reste volontairement la vertu demeure volontaire, < dans un cas comme dans l’autre > il \\
n’en est pas moins vrai que le vice sera volontaire comme la vertu, puisque le méchant, tout comme l’homme de bien, est cause par lui-même de ses actions, même s’il n’est pas cause de la fin. Si donc, comme il est dit, nos vertus sont volontaires (et, en fait, nous sommes bien nous-mêmes, dans une certaine mesure, partiellement causes de nos propres dispositions, et, d’autre part, c’est la nature même de notre caractère qui nous fait poser telle ou telle fin), nos vices aussi seront volontaires, \\
 car le cas est le même. Mais nos actions ne sont pas volontaires de la même façon que nos dispositions : en ce qui concerne nos actions, elles sont sous notre dépendance absolue du commencement à la fin, quand nous en savons les circonstances singulières ; par contre, en ce qui concerne nos dispositions, elles dépendent bien de nous au début, mais les actes singuliers qui s’y ajoutent par la suite échappent à notre conscience, comme dans le cas des maladies ; cependant, parce qu’il dépendait de nous d’en faire tel ou tel usage, pour cette raison-là nos dispositions sont volontaires.
\subsection[{8 (1114b) < Résumé des chapitres précédents >}]{8 (1114b) < Résumé des chapitres précédents >}
\noindent \\
En ce qui regarde les vertus en général, nous avons marqué, dans les grandes lignes, quel était leur genre, à savoir que ce sont des médiétés et que ce sont des dispositions ; nous avons établi aussi que, par leur essence, elles nous rendent aptes à accomplir les mêmes actions que celles dont elles procèdent ; qu’elles sont dans notre dépendance, et volontaires ; qu’enfin elles agissent selon les prescriptions de la \\
droite règle.
\subsection[{9 (1115a — 1115b) < Examen des vertus spéciales. Le courage >}]{9 (1115a — 1115b) < Examen des vertus spéciales. Le courage >}
\noindent  Reprenant chacune des différentes vertus, indiquons \\
quelle est leur nature, sur quelles sortes d’objets elles portent et de quelle façon ; ce faisant, nous montrerons aussi quel est leur nombre. Tout d’abord, parlons du courage.\par
Que le courage soit une médiété par rapport à la crainte et à la témérité, c’est là une chose que nous avons déjà rendue manifeste. Or il est clair que les choses que nous craignons sont les choses redoutables, et ces choses-là, pour le dire tout uniment, sont des maux ; et c’est pourquoi on définit la crainte \\
{\itshape une attente d’un mal}. Quoi qu’il en soit, nous ressentons la crainte à l’égard de tous les maux, comme par exemple le mépris, la pauvreté, la maladie, le manque d’amis, la mort ; par contre, on ne considère pas d’ordinaire que le courage ait rapport à tous ces maux : il y a, en effet, certains maux qu’il est de notre devoir, qu’il est même noble, de redouter et honteux de ne pas craindre, par exemple le mépris [αδοξια]. Celui qui craint le mépris est un homme de bien, un homme réservé, et celui qui \\
ne le craint pas un impudent, quoique on appelle parfois ce dernier, par extension, homme courageux, parce qu’il offre quelque ressemblance avec l’homme courageux, l’homme courageux étant lui aussi quelqu’un qui n’a pas peur. Quant à la pauvreté, sans doute ne devons-nous pas la redouter, ni non plus la maladie, ni en général aucun des maux qui ne proviennent pas d’un vice ou qui ne sont pas dus à l’agent lui-même. Mais celui qui n’éprouve aucune crainte à leur sujet n’est pas non plus pour autant un homme courageux (quoique nous lui \\
appliquions à lui aussi cette qualification par similitude) : car certains hommes, qui sont lâches dans les dangers de la guerre, n’en sont pas moins d’une nature libérale dans les questions d’argent, et supportent avec constance la perte de leur fortune. On n’est pas non plus un lâche si on redoute l’insulte faite à ses enfants et à sa femme, ou l’envie, ou quelque mal de ce genre ; ni brave, si on montre du cœur au moment de recevoir le fouet.\par
Dans ces conditions, pour quelles sortes de choses redoutables se montre-ton courageux ? Ne serait-ce pas quand il \\
s’agit de choses de première importance ? Personne, en effet, n’endure plus intrépidement les dangers que l’homme courageux. Or le plus redoutable de tous est la mort, car elle est un point final, et pour celui qui est mort, rien, selon l’opinion courante, ne peut plus lui arriver de bon ou de mauvais. Cependant, même pour affronter la mort, ce n’est pas, semblerait-il, en toutes circonstances qu’on peut être qualifié d’homme courageux, par exemple dans les dangers courus en mer ou dans la maladie. À quelles occasions donc est-on courageux ? \\
Ne serait-ce pas dans les occasions les plus nobles ? Or la plus noble forme de la mort est celle qu’on rencontre à la guerre, au sein du plus grand et du plus beau des dangers. Cette façon de voir est confirmée par l’exemple des honneurs qui sont décernés dans les cités et à la cour des monarques.\par
Au sens principal du terme, on appellera dès lors courageux celui qui demeure sans crainte en présence d’une noble mort, ou de quelque péril imminent pouvant entraîner la \\
mort : or tels sont particulièrement les dangers de la guerre.  Non pas toutefois que, même sur mer et dans la maladie, l’homme courageux ne soit pas aussi un homme sans peur, quoique ce ne soit pas de la même façon que le sont les marins eux-mêmes : il a abandonné tout espoir de salut et se révolte à la pensée de mourir de cette façon-là, alors que les marins, eux, gardent bon espoir en raison de leur expérience. En même temps aussi, on montre du courage dans des circonstances \\
où on peut faire preuve de valeur ou mourir d’une belle mort ; mais dans ces différentes sortes de mort, aucune des deux conditions que nous avons posées n’est réalisée.
\subsection[{10 (1115b — 1116a) < Le courage, suite >}]{10 (1115b — 1116a) < Le courage, suite >}
\noindent Bien que les mêmes choses ne soient pas redoutables pour tout le monde, il y a cependant des choses que nous affirmons dépasser les forces humaines, et qui sont par suite redoutables pour tout homme, du moins pour tout homme sain d’esprit. Mais les choses que l’homme peut endurer diffèrent en \\
grandeur et par le plus et le moins, et il en est de même pour celles qui inspirent confiance. Or l’homme courageux est à l’épreuve de la crainte autant qu’homme peut l’être. Aussi tout en éprouvant même de la crainte dans les choses qui ne sont pas au-delà des forces humaines, il leur fera face comme il convient et comme la raison le demande, en vue d’un noble but, car c’est là la fin à laquelle tend la vertu. D’autre part, il est possible de redouter ces choses-là plus ou moins, et il est possible en outre de redouter des choses non redoutables \\
comme si elles étaient redoutables. Des erreurs qui se produisent à cet égard, l’une consiste à redouter ce qui ne doit pas l’être, l’autre à le redouter d’une façon qui ne convient pas, ou en un temps inopportun, et ainsi de suite ; et il en est de même pour les choses qui inspirent confiance. Celui donc qui attend de pied ferme et redoute les choses qu’il faut, pour une fin droite, de la façon qui convient et au moment opportun, ou qui se montre confiant sous les mêmes conditions, celui-là est un homme courageux (car l’homme courageux pâtit et agit pour \\
un objet qui en vaut la peine et de la façon qu’exige la raison. Et la fin de toute activité est celle qui est conforme aux dispositions du caractère dont elle procède, et c’est là une vérité pour l’homme courageux également : son courage est une noble chose ; par suite sa fin aussi est noble, puisqu’une chose se définit toujours par sa fin ; et par conséquent c’est en vue d’une fin noble que l’homme courageux fait face aux dangers et accomplit les actions que lui dicte son courage).\par
De tous ceux qui, en ce domaine, pèchent par excès, l’un \\
pèche par manque de peur et n’a pas reçu de désignation (nous avons dit plus haut que beaucoup de qualités n’ont pas de nom) : ce pourrait être une sorte de maniaque ou d’être insensible s’il n’avait peur de rien, ni d’un tremblement de terre, ni des vagues, comme on le raconte des Celtes ; — l’autre, qui pèche par excès de confiance en soi dans les choses redoutables, est un téméraire (le téméraire est encore considéré \\
comme un vantard, et qui se donne des airs de courage : ce que l’homme courageux {\itshape est} à l’égard des choses redoutables, le téméraire veut seulement le {\itshape paraître}, et dans les situations où il lui est possible de se trouver il imite le premier. C’est pourquoi aussi la plupart de ces sortes de gens sont des poltrons qui font les braves : car dans ces situations, tout en faisant bonne contenance, ils ne tiennent pas ferme longtemps contre les choses qu’ils craignent) ; — l’autre, enfin, qui pèche par excès de crainte, est un lâche. Il ressent à la fois ce qu’on ne doit pas \\
ressentir et d’une façon qui ne convient pas, et toutes les autres caractéristiques de cette sorte s’attachent à lui. La confiance  aussi lui fait défaut, mais c’est dans les situations alarmantes que sa peur exagérée éclate surtout aux yeux. Le lâche est, dès lors, une sorte d’homme sans espoir, car il s’effraie de tout. Pour l’homme courageux, c’est tout le contraire, et sa bravoure est la marque d’une disposition tournée vers l’espérance.\par
\\
Ainsi, le lâche, le téméraire et le courageux ont rapport aux mêmes objets ; la différence qui les sépare porte uniquement sur la façon dont ils se comportent envers les dits objets. Les deux premiers, en effet, pèchent par excès ou par défaut, et le troisième se tient dans un juste milieu et comme il doit être. Les téméraires, en outre, sont emportés et appellent de leurs vœux les dangers, mais au moment critique s’en détournent, tandis que les hommes courageux sont vifs dans l’action et calmes au temps qui la précède.
\subsection[{11 (1116a — 1117a) < Le courage, suite >}]{11 (1116a — 1117a) < Le courage, suite >}
\noindent Ainsi donc que nous l’avons dit, le courage est une médiété par rapport aux choses qui inspirent confiance et à celles qui \\
inspirent de la crainte, dans les circonstances que nous avons indiquées ; et il [le courageux N.d.É] choisit ou endure ces choses parce qu’il est noble de le faire, ou parce qu’il est honteux de ne pas le faire. Or mourir pour échapper à la pauvreté où à des chagrins d’amour, ou à quelque autre souffrance, c’est le fait non d’un homme courageux, mais bien plutôt d’un lâche : c’est, en effet, un manque d’énergie que de fuir les tâches pénibles, et on endure la mort non pas parce qu’il est noble d’agir ainsi, mais pour échapper à un mal.\par
\\
Telle est donc la nature du courage, mais ce terme s’emploie encore pour désigner cinq types différents.\par
<1> En premier lieu, vient le courage civique, car c’est lui qui ressemble le plus au courage proprement dit. Le citoyen, en effet, paraît supporter les dangers à cause des pénalités provenant de la loi, des récriminations ou des honneurs. Et pour \\
cette raison les peuples les plus courageux sont apparemment ceux chez lesquels les lâches sont voués au mépris, et les braves à l’estime publique. Ce sont des hommes courageux de ce type que dépeint Homère sous les traits de Diomède et d’Hector :\par
 {\itshape Polydamas sera le premier à me charger d’un blâme.} \par
Et Diomède :\par
\\
{\itshape Car Hector, un jour, dira en parlant devant les Troyens} : \par
 {\itshape Le fils de Tydée, par moi…} \par
Ce genre de courage est celui qui ressemble le plus à celui que nous avons décrit plus haut, parce qu’il est produit par une vertu (à savoir, par un sentiment de pudeur) et par un désir de quelque chose de noble (à savoir, de l’honneur) et aussi par le désir d’éviter le blâme, qui est une chose honteuse. On pourrait \\
aussi ranger dans cette même classe les soldats qui sont forcés par leurs chefs de se montrer courageux ; mais c’est là un courage d’ordre inférieur, en tant que leur conduite est dictée non pas par le sentiment de l’honneur, mais par la crainte et le désir d’éviter non la honte mais la souffrance ; car leurs maîtres les y forcent à la façon d’Hector disant :\par
{\itshape Mais celui que j’apercevrai en train de se blottir à l’écart du combat}, \par
 \\
 {\itshape Sera bien assuré de ne pas échapper aux chiens.} \par
Et les officiers qui assignent leurs postes aux soldats et les frappent quand ils lâchent pied n’agissent pas autrement, non  plus que ceux qui alignent leurs hommes en avant des fossés et autres retranchements de ce genre : tous emploient la contrainte. Or on ne doit pas être courageux parce qu’on est forcé de l’être, mais parce que c’est une chose noble.\par
<2> L’expérience de certains dangers particuliers est aussi regardée comme étant une forme de courage : c’est ce qui \\
explique que, dans la pensée de Socrate, le courage est une science. Les uns font preuve de ce genre de courage dans telles circonstances, les autres dans telles autres, et notamment, dans les dangers de la guerre, les soldats de métier. Il semble, en effet, y avoir, dans la guerre beaucoup de vaines alarmes, que ces hommes embrassent d’un coup d’œil des plus sûrs : ils ont ainsi toute l’apparence de la bravoure, parce que les autres ne savent pas le véritable état des choses. Ensuite l’expérience les rend capables au plus haut point de prendre \\
l’offensive et de parer les coups, vu leur habileté à se servir de leurs armes et à s’équiper avec tout ce qu’il peut y avoir de plus parfait à la fois pour l’attaque et pour la défense. Leur situation est ainsi celle d’hommes armés combattant une foule désarmée, ou d’athlètes entraînés luttant avec de simples amateurs ; et, en effet, même dans ces dernières sortes de compétitions, ce ne sont pas les plus courageux qui sont les meilleurs combattants, \\
mais ceux qui sont les plus vigoureux et dont le corps est le mieux entraîné. Mais les soldats de métier deviennent lâches quand le danger se montre par trop pressant et qu’ils ont l’infériorité du nombre et de l’équipement : ils sont alors les premiers à fuir, alors que les troupes composées de citoyens meurent à leur poste, comme cela est arrivé à la bataille du temple d’Hermès. Pour les soldats-citoyens, en effet, il est honteux \\
de fuir, et la mort est préférable à un salut acquis à ce prix ; les autres, au contraire, commencent par affronter le danger en pensant qu’ils sont les plus forts, mais la vérité une fois connue ils prennent la fuite, craignant la mort plus que la honte. Mais l’homme courageux est d’une autre trempe.\par
<3> L’impulsivité est encore rapportée au courage. On regarde aussi en effet comme des gens courageux ceux qui par \\
impulsivité se comportent à la façon des bêtes sauvages se jetant sur le chasseur qui les a blessées, parce que les gens courageux sont aussi des gens pleins de passion. Car rien de tel que la passion pour se lancer impétueusement dans les dangers ; et de là les expressions d’Homère :\par
{\itshape Il a placé sa force dans son ardeur}, \par
et :\par
{\itshape Il excitait leur animosité et leur colère}, \par
et encore \par
{\itshape Un âpre picotement irritait ses narines}, \par
et enfin :\par
 {\itshape Son sang bouillonnait.} \par
Car tous les symptômes de ce genre semblent indiquer \\
l’excitation et l’élan de la passion. Ainsi, les hommes courageux agissent pour l’amour du bien, quoique la passion opère en même temps en eux ; les bêtes sauvages, au contraire, sont poussées par la souffrance, à cause par exemple d’une blessure reçue, ou par peur, puisque à l’abri dans une forêt ou dans un marécage elles n’approchent pas. Ce n’est donc pas du courage quand, chassées par la souffrance et l’impulsivité, elles se \\
ruent au danger, sans rien prévoir des périls qui les attendent : car à ce compte-là, même les ânes seraient courageux quand ils  ont faim, puisque les coups ne parviennent pas à leur faire quitter le pâturage. Et les libertins poussés par la concupiscence accomplissent aussi beaucoup d’actions audacieuses. Mais la forme de courage inspirée par la passion semble être la plus naturelle de toutes et, quand s’y ajoute le choix et le motif, \\
constituer le courage au sens propre. — Les hommes donc aussi, quand ils sont en colère ressentent de la souffrance, et quand ils se vengent éprouvent du plaisir. Mais ceux qui se battent pour ces raisons-là, tout en combattant vaillamment ne sont pas courageux au sens propre, car ils n’agissent ni poussés par le bien ni comme la raison le veut, mais sous l’effet de la passion ; ils ont cependant quelque chose qui rappelle le vrai courage.\par
\\
<4> Pas davantage les gens confiants en eux-mêmes ne sont des hommes courageux : c’est, en effet, parce qu’ils ont de nombreuses victoires à leur actif et sur beaucoup d’adversaires qu’ils gardent leur assurance au milieu des dangers. Ils ont une certaine ressemblance avec les hommes courageux, en ce que les uns comme les autres sont pleins d’assurance. Mais les hommes courageux tirent leur confiance des raisons que nous avons précédemment exposées, tandis que les autres, c’est parce qu’ils pensent être les plus forts et n’avoir rien à subir en retour. (Tel est aussi le comportement des gens en état d’ivresse \\
et qui deviennent pleins d’assurance). Mais quand les choses ne tournent pas comme ils l’espèrent, ils prennent la fuite. Or, nous l’avons vu, la marque d’un homme courageux est de supporter ce qui est réellement redoutable à l’homme ou ce qui lui apparaît tel, avec ce motif qu’il est beau d’agir ainsi et honteux de ne pas le faire. C’est pourquoi encore on considère qu’un homme montre un plus grand courage en demeurant sans crainte et sans trouble dans les dangers qui s’abattent brusquement que dans les dangers qu’on peut prévoir à l’avance, car le courage provient alors davantage d’une disposition du \\
caractère, et demande moins de préparation : en effet, les dangers prévisibles peuvent faire l’objet d’un choix calculé et raisonnable, tandis que les périls soudains exigent une disposition stable du caractère.\par
<5> Les gens ignorant le danger apparaissent eux aussi courageux, et ils ne sont pas fort éloignés des hommes confiants en eux-mêmes ; ils leur sont cependant inférieurs par leur manque total d’assurance, alors que les autres en possèdent. Aussi les hommes qui se fient à eux-mêmes tiennent-ils fermement pendant un certain temps, tandis que ceux qui ne se \\
rendent pas compte du danger et ont éprouvé des déceptions à cet égard, dès qu’ils s’aperçoivent, ou même soupçonnent, que la réalité est toute différente, prennent la fuite, comme cela est arrivé pour les Argiens quand ils tombèrent inopinément sur les Spartiates qu’ils prenaient pour des Sicyoniens.
\subsection[{12 (1117a — 1117b) < Le courage, fin >}]{12 (1117a — 1117b) < Le courage, fin >}
\noindent Nous venons ainsi d’indiquer les caractères à la fois de l’homme courageux et de ceux qui passent d’ordinaire pour courageux.\par
Bien que le courage ait rapport à la confiance et à la crainte, ce n’est pas de la même façon qu’il a rapport à l’une et à l’autre, \\
mais il se montre surtout dans les choses qui inspirent la crainte. En effet, celui qui demeure imperturbable au milieu des dangers et qui se comporte à leur égard comme il se doit, est plus véritablement courageux que celui qui se comporte ainsi dans les situations rassurantes. Dès lors, c’est par sa fermeté envers les choses qui apportent de la souffrance, ainsi que nous l’avons dit, qu’un homme est appelé courageux. C’est pourquoi le courage est en lui-même une chose pénible, \\
et il est à bon droit objet de nos éloges, parce qu’il est plus difficile d’endurer les peines que de s’abstenir des plaisirs.  Non pas qu’il faille penser que la fin que se propose le courage ne soit pas une chose agréable ; seulement, elle est obscurcie par les circonstances qui l’accompagnent, comme cela se produit également dans les compétitions du gymnase : car chez les pugilistes, la fin pour laquelle ils combattent est agréable, c’est la couronne et les honneurs, alors que les coups qu’ils reçoivent \\
sont pour eux, qui sont des êtres de chair, une chose douloureuse et pénible, comme d’ailleurs l’ensemble de leur travail d’entraînement. Et tous ces efforts font par leur nombre apparaître l’objet final comme insignifiant et sans agrément. Si dès lors la fin concernant le courage est de même ordre, la mort et les blessures seront pénibles à l’homme courageux, qui les souffrira à contre-cœur ; il les endurera néanmoins, parce qu’il \\
est noble d’agir ainsi, ou qu’il est honteux de s’y dérober. Et plus la vertu qu’il possède est complète et grand son bonheur, plus aussi la pensée de la mort lui sera pénible : car c’est pour un pareil homme que la vie est surtout digne d’être vécue, c’est lui que la mort privera des plus grands biens, et il en a pleinement conscience : tout cela ne va pas sans l’affliger. Mais il n’en est pas moins courageux, peut-être même l’est-il davantage, parce qu’il préfère les nobles travaux de la guerre à ces grands biens dont nous parlons.\par
\\
Il n’appartient donc pas à toutes les vertus de s’exercer d’une façon agréable, sinon dans la mesure où leur fin se trouve atteinte. Mais rien sans doute ne nous empêche de penser que ce ne sont pas ceux qui possèdent le genre de bravoure que nous avons décrit, qui font les meilleurs soldats : ce sont plutôt ceux qui, tout en étant moins braves, ne disposent d’aucun autre bien que leur vie même, car c’est avec empressement \\
qu’ils s’exposent aux dangers, et ils donnent leur vie en échange de maigres profits.
\subsection[{13 (1117b — 1118b) < La modération >}]{13 (1117b — 1118b) < La modération >}
\noindent Le courage a été suffisamment étudié (quant à sa nature, il n’est pas difficile, tout au moins dans les grandes lignes, de la comprendre, à l’aide des explications qui précèdent).\par
Après le courage, parlons de la modération [σωφροσυνη], car il semble bien que ces deux vertus soient celles des parties irrationnelles \\
de l’âme. Nous avons dit56 que la modération est une médiété par rapport aux plaisirs (elle l’est à un moindre degré et d’une façon différente par rapport aux peines) ; c’est dans la même sphère aussi que se manifeste le dérèglement. À quelles sortes de plaisirs ces deux états se rapportent-ils donc ? C’est ce que nous allons maintenant déterminer.\par
On peut admettre que les plaisirs se divisent en plaisirs du corps et en plaisirs de l’âme. Comme exemples de plaisirs de l’âme, nous avons l’ambition et l’amour du savoir : en effet, \\
pour chacun de ces cas on trouve son plaisir dans l’objet qu’on est porté à aimer sans que le corps on soit affecté en rien, mais c’est plutôt l’esprit qui l’est. Mais ceux qui recherchent les plaisirs de l’ambition ou du savoir ne sont appelés ni modérés, ni déréglés, et il en est de même pour tous ceux qui se livrent aux autres plaisirs non corporels : ceux qui se plaisent à écouter \\
ou à raconter des fables et qui passent leurs journées à musarder çà et là sont des bavards, mais nous ne les appelons pas des  gens déréglés, pas plus d’ailleurs que ceux qui ont des ennuis d’argent ou des peines de cœur.\par
La modération ne saurait donc s’appliquer qu’aux plaisirs corporels, et encore n’est-ce pas à tous indistinctement : par exemple, les hommes qui trouvent leur plaisir dans les spectacles de la vue, comme les couleurs, les formes, le dessin, ne sont appelés ni modérés ni déréglés, et pourtant on \\
pourrait penser que, même dans ce domaine, il peut y avoir un plaisir ou légitime, ou excessif, ou déficient. Même remarque pour ceux qui recherchent les plaisirs de l’ouïe : les personnes qui ont un goût immodéré pour la musique ou le théâtre, on ne les appelle jamais déréglées, pas plus qu’on n’appelle modérées celles qui ne dépassent pas la juste mesure. Pas davantage on ne donne ces noms à ceux qui aiment les plaisirs de l’odorat, \\
sinon par accident : ceux qui se plaisent à l’odeur des pommes ou des roses ou des parfums, nous ne les appelons pas des hommes déréglés, mais nous appelons plutôt ainsi ceux qui se délectent à l’odeur d’onguents ou de mets, car les gens déréglés y trouvent leur plaisir du fait que ces odeurs leur rappellent les objets de leur concupiscence.\par
On peut constater assurément que même les autres personnes, quand elles ont faim, ont \\
plaisir à sentir la nourriture ; mais prendre plaisir à ce genre d’odeurs est le fait d’un homme déréglé, car ce sont là pour lui des objets de concupiscence.\par
Il n’existe pas non plus chez les animaux de plaisirs par ces sens, sinon accidentellement. Les chiens, en effet, ne prennent pas plaisir à l’odeur des lièvres, ils prennent plaisir à les manger : l’odeur leur a donné seulement la perception du \\
lièvre. De même le lion ne s’intéresse pas au mugissement du bœuf, ce qu’il veut c’est le dévorer : le mugissement lui a seulement fait percevoir que le bœuf est à sa portée, et il paraît ainsi trouver plaisir au mugissement. De même il ne se réjouit pas de voir\par
{\itshape un cerf ou une chèvre sauvage}, \par
mais il se réjouit de pouvoir en faire son régal.\par
Ainsi donc, la modération et le dérèglement n’ont rapport qu’à ces sortes de plaisirs que l’homme possède en commun \\
avec les animaux, et qui par suite apparaissent d’un caractère vil et bestial, je veux dire les plaisirs du toucher et du goût. Bien plus, les plaisirs ne paraissent tirer du goût qu’un usage médiocre ou même nul. En effet, c’est du goût que relève la discrimination des saveurs, telle qu’elle est pratiquée par les dégustateurs et les bons cuisiniers ; or ces discriminations ne procurent pas beaucoup de plaisir, et en tout cas n’en \\
donnent pas aux gens déréglés : ceux-ci ne recherchent que la jouissance, qui leur vient tout entière par le toucher, à la fois dans le boire et dans le manger, ainsi que dans ce qu’on nomme les plaisirs de l’amour59. C’est pourquoi encore certain gourmand priait que son gosier devînt plus long que celui d’une grue, ce qui montre bien que son plaisir venait du toucher.  Ainsi donc, le sens auquel le dérèglement est lié est celui de tous qui nous est le plus commun avec les animaux, et le dérèglement ne semblerait être à si juste titre répréhensible que parce qu’il existe en nous non pas en tant qu’hommes, mais en tant qu’animaux : se plaire à de pareilles sensations et les aimer par-dessus tout a quelque chose de bestial. En effet, on exclut \\
même les plaisirs tactiles les plus épurés, tels que les plaisirs que procurent au gymnase frictions et bains chauds, car ce n’est pas le contact portant sur le corps entier qui intéresse le débauché, mais seulement celui qui porte sur certaines de ses parties.\par
Des appétits concupiscibles, les uns semblent être communs à tous les hommes < et naturels >, les autres propres et adventices : par exemple l’appétit de la nourriture est \\
naturel, puisque tout homme appète la nourriture solide ou liquide dont il a besoin, et parfois même les deux à la fois ; et c’est le cas aussi du plaisir sexuel, comme le dit Homère, quand on est jeune et en pleine force. Mais le fait de désirer telle ou telle sorte de nourriture ou de plaisir amoureux est variable selon les individus, et leur désir ne porte pas non plus sur les mêmes objets. C’est pourquoi de tels appétits nous paraissent véritablement nôtres. Ces préférences individuelles n’en ont pas moins cependant elles aussi quelque chose de naturel : telles choses sont agréables aux uns, et telles autres le sont aux autres, et certaines choses sont, pour tous les hommes, plus agréables que les premières venues.\par
\\
Quoi qu’il en soit, dans les appétits naturels on se trompe rarement et seulement dans une seule direction, à savoir dans le sens de l’exagération (car manger ou boire ce qui se présente jusqu’à en être gavé, c’est dépasser la quantité fixée par la nature, puisque l’appétit naturel est seulement satisfaction d’un besoin. Aussi appelle-t-on ceux qui commettent ces excès des \\
{\itshape goinfres}, du fait qu’ils remplissent leur ventre au-delà de la mesure convenable ; et ce sont les gens d’un caractère particulièrement vil qui tombent dans un pareil excès). Par contre, dans les appétits propres à chacun, les erreurs sont nombreuses et de formes variées. En effet, alors qu’on dit généralement de quelqu’un qu’il {\itshape aime à la folie telle ou telle chose}, soit parce qu’il prend plaisir à des choses qu’on ne doit pas désirer, ou parce qu’il dépasse la mesure courante, ou enfin parce qu’il prend son plaisir d’une mauvaise manière, c’est au contraire de toutes ces façons à la fois que les gens déréglés tombent dans \\
l’exagération : en effet, ils mettent leur plaisir dans certaines choses illicites (et effectivement détestables), et même s’il arrive que certaines d’entre elles soient permises, ils se livrent à leur goût plus que de raison ou plus qu’on ne le fait généralement.\par
Ainsi, il est évident que l’excès dans les plaisirs est un dérèglement et une chose blâmable. En ce qui regarde d’autre part les peines, on n’est pas, comme pour le courage, appelé \\
modéré parce qu’on les endure, ni déréglé parce qu’on ne les supporte pas, mais on est appelé déréglé parce qu’on s’afflige outre mesure de ne pas trouver les plaisirs qu’on recherche (et même c’est le plaisir qui nous cause de la peine), et modéré quand on ne s’afflige pas de l’absence du plaisir.
\subsection[{14 (1119a) < La modération, suite >}]{14 (1119a) < La modération, suite >}
\noindent  L’homme déréglé a ainsi l’appétit de toutes les choses agréables ou de celles qui le sont le plus, et il est conduit par la concupiscence à accorder sa préférence à ces choses-là sur toutes les autres, et c’est pourquoi il s’afflige non seulement de les manquer mais encore de les désirer (car l’appétit s’accompagne de souffrance, quoiqu’il paraisse absurde d’éprouver de la peine à cause du plaisir).\par
\\
Des personnes péchant par défaut en ce qui regarde les plaisirs et s’en délectant moins qu’il ne convient, se rencontrent rarement, car une pareille insensibilité n’a rien d’humain. En effet, même les animaux font des discriminations dans la nourriture, et se plaisent à certains aliments à l’exclusion d’autres ; et s’il existe un être à ne trouver rien d’agréable et à n’établir aucune différence entre une chose et une autre, cet \\
être-là sera très loin de l’humaine nature. Au surplus, un pareil homme n’a pas reçu de nom, parce qu’il se rencontre peu fréquemment. Quant à l’homme modéré, il se tient dans un juste milieu à cet égard. Car il ne prend pas plaisir aux choses qui séduisent le plus l’homme déréglé (elles lui répugnent plutôt), ni généralement à toutes les choses qu’on ne doit pas rechercher, ni à rien de ce genre d’une manière excessive, pas plus qu’il ne ressent de peine ou de plaisir à leur absence (sinon \\
d’une façon mesurée), ni plus qu’on ne doit, ni au moment où il ne faut pas, ni en général rien de tel. Par contre, toutes les choses qui, étant agréables, favorisent la santé ou le bon état du corps, ces choses-là il y aspirera d’une façon modérée et convenable, ainsi que tous les autres plaisirs qui ne sont pas un obstacle aux fins que nous venons de dire, ou contraires à ce qui est noble, ou enfin au-dessus de ses moyens. L’homme qui dépasse ces limites aime les plaisirs de ce genre plus qu’ils \\
ne le méritent ; mais l’homme modéré n’est rien de tel, il se comporte envers les plaisirs comme la droite règle le demande.
\subsection[{15 (1119a — 1119b) < Dérèglement et lâcheté. Comparaison avec l’enfance >}]{15 (1119a — 1119b) < Dérèglement et lâcheté. Comparaison avec l’enfance >}
\noindent Le dérèglement est plus semblable à un état volontaire que la lâcheté, car il a pour cause le plaisir, et la lâcheté la souffrance, deux sentiments dont le premier est objet de choix, et l’autre, objet de répulsion seulement. Or la souffrance met hors de soi l’être qui l’éprouve et détruit sa nature, tandis que le plaisir n’opère rien de pareil. Aussi le dérèglement est-il plus \\
volontaire, et par suite encore, plus répréhensible. En effet, on s’accoutume assez facilement à garder la modération dans les plaisirs, parce que les occasions de ce genre sont nombreuses dans le cours de la vie et que l’exercice de cette habitude n’entraîne aucun danger, à l’inverse de ce qui se passe dans les situations qui inspirent de la crainte. D’autre part, la lâcheté semblerait bien n’être un état volontaire qu’en la distinguant de ses manifestations particulières : en elle-même elle n’est pas une souffrance, mais dans les manifestations particulières dont nous parlons, la souffrance nous met hors de notre assiette au \\
point de nous faire jeter nos armes ou adopter d’autres attitudes honteuses, ce qui donne à nos actes l’apparence d’être accomplis sous la contrainte. Pour l’homme déréglé, c’est l’inverse : ses actions particulières sont volontaires (puisqu’il en a l’appétit et le désir), mais son caractère en général l’est moins, puisque personne ne désire être un homme déréglé.\par
Nous étendons encore le terme dérèglement aux fautes commises par les enfants, fautes qui présentent une certaine  similitude avec ce que nous avons vu. Quant à dire lequel des deux sens tire son nom de l’autre, cela importe peu pour notre présent dessein, mais il est clair que c’est le plus récent qui emprunte son nom au plus ancien. En tout cas, cette extension de sens semble assez judicieuse, car c’est ce qui aspire aux choses honteuses, et dont les appétits prennent un grand développement, qui a besoin d’émondage, et pareille description \\
s’applique principalement aussi bien à l’appétit qu’à l’enfant : les enfants, en effet, vivent aussi sous l’empire de la concupiscence, et c’est surtout chez eux que l’on rencontre le désir de l’agréable. Si donc on ne rend pas l’enfant docile et soumis à l’autorité, il ira fort loin dans cette voie : car dans un être sans raison, le désir de l’agréable est insatiable et s’alimente de tout, et l’exercice même de l’appétit renforce la tendance innée ; et \\
si ces appétits sont grands et forts, ils vont jusqu’à chasser le raisonnement. Aussi doivent-ils être modérés et en petit nombre et n’être jamais en conflit avec la raison. Et c’est là ce que nous appelons un caractère docile et contenu. Et de même que l’enfant doit vivre en se conformant aux prescriptions de son gouverneur, ainsi la partie concupiscible de l’âme doit-elle \\
se conformer à la raison. C’est pourquoi il faut que la partie concupiscible de l’homme modéré soit en harmonie avec la raison, car pour ces deux facultés le bien est le but visé, et l’homme modéré a l’appétit des choses qu’on doit désirer, de la manière dont elles doivent l’être et au moment convenable, ce qui est également la façon dont la raison l’ordonne.
\section[{Livre IV}]{Livre IV}\renewcommand{\leftmark}{Livre IV}

\subsection[{1 (1119b — 1120a) < La libéralité >}]{1 (1119b — 1120a) < La libéralité >}
\noindent \\
Nous avons assez parlé de la modération. Passons maintenant à l’étude de la libéralité.\par
Cette vertu semble être la médiété dans les affaires d’argent, car l’homme libéral est l’objet de nos éloges non pas dans les travaux de la guerre, ni dans le domaine où se distingue l’homme modéré, ni non plus dans les décisions de \\
justice, mais dans le fait de donner et d’acquérir de l’argent, et plus spécialement dans le fait de donner. Nous entendons par {\itshape argent} toutes les choses dont la valeur est mesurée en monnaie.\par
D’autre part, la prodigalité et la parcimonie constituent l’une et l’autre des modes de l’excès et du défaut dans les affaires d’argent. Si nous attribuons toujours le terme parcimonie à ceux qui montrent pour l’argent une avidité plus \\
grande qu’il ne convient, par contre nous appliquons parfois le mot prodigalité en un sens complexe, puisque nous appelons également du nom de prodigues les gens intempérants et qui dépensent beaucoup pour leurs dérèglements. C’est aussi la raison pour laquelle cette dernière sorte de prodigues nous semble atteindre le comble de la perversité, car il y a en eux cumul de plusieurs vices en même temps, Aussi le nom qu’on leur assigne n’est-il pas pris dans son sens propre : le terme prodigue signifie plutôt un homme atteint d’un vice bien  particulier, qui consiste à dilapider sa fortune, car tout espoir de salut est interdit à qui se ruine par sa propre faute, et la dilapidation du patrimoine semble être une sorte de ruine de la personne elle-même, en ce sens que ce sont nos biens qui nous permettent de vivre.\par
Tel est donc le sens où nous prenons ici le terme prodigue. — Les choses dont nous avons l’usage peuvent être bien ou mal \\
employées, et la richesse est au nombre des choses dont on fait usage ; or, pour une chose déterminée, l’homme qui en fait le meilleur usage est celui qui possède la vertu relative à cette chose ; par suite, pour la richesse également, l’homme qui en fera le meilleur usage est celui qui possède la vertu ayant rapport à l’argent, c’est-à-dire l’homme libéral. Mais l’usage de l’argent apparaît consister dans la dépense et dans le don, tandis que l’acquisition et la conservation intéressent de préférence la possession. C’est pourquoi, ce qui caractérise \\
l’homme libéral, c’est plutôt de disposer en faveur de ceux qu’il convient d’obliger, que de recevoir d’une source licite et de ne pas recevoir d’une source illicite. La marque de la vertu en effet, c’est plutôt de faire le bien que de le recevoir, et d’accomplir des bonnes actions plutôt que de s’abstenir des honteuses ; et il est de toute évidence que faire le bien et accomplir de bonnes actions va de pair avec le fait de donner, et qu’au contraire recevoir un bienfait ou s’abstenir d’actions \\
honteuses va de pair avec le fait de prendre. Ajoutons que la gratitude s’adresse à celui qui donne et non à celui qui se borne à ne pas recevoir, et l’éloge s’adresse aussi davantage au premier. Du reste, il est plus facile de ne pas prendre que de donner, car on se défait moins facilement de son propre bien qu’on ne s’abstient de prendre ce qui appartient à un autre. Et ceux qui sont appelés libéraux sont ceux qui donnent ; ceux qui \\
se contentent de ne pas prendre ne sont pas loués pour leur libéralité, mais plutôt pour leur sens de la justice ; et ceux qui reçoivent sont privés de tout éloge. Enfin les hommes libéraux sont peut-être de tous les gens vertueux ceux qu’on aime le plus, en raison des services qu’ils rendent, c’est-à-dire en ce qu’ils donnent.
\subsection[{2 (1120a — 1121a) < La libéralité, suite >}]{2 (1120a — 1121a) < La libéralité, suite >}
\noindent Les actions conformes à la vertu sont nobles et accomplies en vue du bien ; l’homme libéral donnera donc en vue du bien ; et il donnera d’une façon correcte, c’est-à-dire à ceux à qui il \\
faut, dans la mesure et au moment convenables, et il obéira aux autres conditions d’une générosité droite. Et cela, il le fera avec plaisir, ou du moins sans peine, car l’acte vertueux est agréable ou tout au moins sans souffrance, mais n’est sûrement pas une chose pénible. Au contraire, celui qui donne à ceux à qui il ne faut pas, ou qui n’agit pas en vue d’un noble but mais pour quelque autre motif, ne sera pas appelé libéral mais recevra un autre nom. Pas davantage n’est libéral celui qui \\
donne avec peine, car il semble ainsi faire passer l’argent avant la bonne action, ce qui n’est pas la marque d’une nature libérale. L’homme libéral n’acquerra pas non plus un bien d’une source illicite, une pareille acquisition n’étant pas davantage le fait de quelqu’un qui ne fait aucun cas de l’argent. Ne saurait être non plus un homme libéral celui qui est prompt à solliciter pour lui-même, car recevoir à la légère un bienfait n’est pas la marque d’un homme bienfaisant pour autrui. Mais, d’autre part, l’homme libéral ne prendra qu’à des sources non suspectes,  provenant par exemple de ses propriétés personnelles, non pas parce qu’il est noble d’agir ainsi, mais par nécessité, de façon à être en état de donner. Il ne négligera pas non plus son propre patrimoine, lui qui souhaite l’employer à secourir autrui. Il ne donnera pas au premier venu, de façon à pouvoir se montrer généreux envers ceux à qui il faut donner, au moment et au lieu où il est bon de donner. Mais il est hautement caractéristique \\
d’un homme libéral de ne pas mesurer ses largesses, et par suite de ne laisser à lui-même qu’une moindre part, car ne pas regarder à ses propres intérêts est le fait d’une nature libérale.\par
D’autre part, c’est d’après les ressources que la libéralité doit s’entendre : le caractère libéral d’un don ne dépend pas, en effet, de son montant, mais de la façon de donner du donateur, et celle-ci est fonction de ses ressources. Rien n’empêche dès lors \\
que celui qui donne moins ne soit cependant plus libéral, si c’est à des moyens plus modestes qu’il a recours. Et on considère ordinairement comme étant plus libéraux ceux qui n’ont pas acquis par eux-mêmes leur fortune, mais l’ont reçue par héritage : car, d’abord, l’expérience ne leur a pas appris ce que c’est que le besoin, et, en outre, tous les hommes ont une préférence marquée pour les ouvrages dont ils sont les auteurs, comme on le voit par l’exemple des parents et des poètes.\par
\\
Mais il n’est pas facile à l’homme libéral d’être riche, puisqu’il n’est apte ni à prendre ni à conserver, et qu’au contraire il se montre large dans ses dépenses, et n’apprécie pas l’argent en lui-même mais comme moyen de donner. Et c’est pourquoi le reproche que l’on adresse d’ordinaire au sort, c’est que ce sont les plus dignes de l’être qui sont le moins riches, Mais c’est là un fait qui n’a rien de surprenant, car il n’est pas possible d’avoir de l’argent si on ne se donne pas de peine pour l’acquérir, et c’est d’ailleurs ainsi pour tout le reste.\par
\\
Mais l’homme libéral ne sera pas du moins généreux envers ceux qu’il ne faut pas, ni en temps inopportun, et ainsi de suite : car agir ainsi ne serait plus être dans la ligne de la libéralité, et après avoir dépensé son argent à cela, il ne pourrait plus le dépenser à bon escient. En fait, comme nous l’avons dit, est libéral celui qui dépense selon ses facultés et pour les \\
choses qu’il faut, tandis que celui qui transgresse ces règles est un prodigue. Cela explique que nous n’appelons pas les tyrans des prodigues, car il semble difficile que leurs largesses et leurs dépenses puissent jamais dépasser le montant de ce qu’ils possèdent.\par
Si donc la libéralité est une médiété en ce qui touche l’action de donner et d’acquérir de l’argent, l’homme libéral, à la fois donnera et dépensera pour les choses qui conviennent et dans la mesure qu’il faut, pareillement dans les petites choses \\
et dans les grandes, et tout cela avec plaisir ; d’autre part, il ne prendra qu’à des sources licites et dans une mesure convenable. En effet, la vertu étant une médiété ayant rapport à la fois à ces deux sortes d’opérations, pour chacune d’elles l’activité de l’homme libéral sera comme elle doit être : car le fait de prendre de la façon indiquée va toujours de pair avec le fait de donner équitablement, alors que le fait de prendre d’une autre façon lui est au contraire opposé ; par conséquent, la bonne façon de donner et la bonne façon de prendre, qui ne vont pas l’une sans l’autre, sont présentes à la fois dans la même personne, tandis que pour les façons opposées, ce n’est évidemment pas possible.\par
 S’il arrive à l’homme libéral de dépenser au-delà de ce qui est convenable et de ce qui est bon, il en ressentira de la peine, mais ce sera d’une façon mesurée et comme il convient, la vertu ayant pour caractère de ne ressentir du plaisir ou de la peine que dans les circonstances où l’on doit en éprouver, et comme il le faut. Enfin, l’homme libéral se montre le plus \\
accommodant du monde dans les questions d’argent : il est capable de souffrir dans ce domaine l’injustice, puisqu’il ne fait aucun cas de l’argent, et il ressent plus d’affliction à ne pas dépenser ce qu’il faut qu’il n’éprouve de chagrin à dépenser ce qu’il ne faut pas, et il n’est pas sur ce point d’accord avec Simonide.
\subsection[{3 (1121a — 1122a) < La libéralité. La prodigalité et la parcimonie >}]{3 (1121a — 1122a) < La libéralité. La prodigalité et la parcimonie >}
\noindent Le prodigue, dans ce domaine aussi, commet des erreurs : il ne se réjouit ni dans les occasions convenables, ni de la bonne façon, et il en est de même pour l’affliction qu’il ressent. Mais cela deviendra plus clair par la suite.\par
\\
Nous avons dit que la prodigalité et la parcimonie sont des modes de l’excès et du défaut, et cela dans les deux genres d’activité, c’est-à-dire dans le fait de donner comme dans celui de recevoir, la dépense étant rattachée au fait de donner. La prodigalité est ainsi un excès dans le fait de donner et dans celui de ne pas prendre, et une déficience dans le fait de prendre, tandis que la parcimonie est au contraire une déficience dans le \\
fait de donner et un excès dans le fait de prendre, sauf quand il s’agit de petites choses.\par
Ces deux caractères de la prodigalité se rencontrent rarement associés : il n’est pas facile quand on ne reçoit rien de personne, de donner à tout le monde, car les ressources ne tardent pas à faire défaut lorsque ce sont de simples particuliers qui donnent et qui sont précisément les seuls à être regardés comme des prodigues. Cependant l’homme de cette sorte semblerait être de beaucoup supérieur à l’homme parcimonieux : \\
il est, en effet facile à guérir sous la double influence de l’âge et de la pénurie, et il est capable d’atteindre la juste mesure. Il possède les qualités de l’homme libéral puisqu’il donne et ne prend pas : la seule réserve à faire, c’est que, dans un cas comme dans l’autre, il n’agit ni comme il faudrait, ni d’une façon satisfaisante. Si donc il contractait l’habitude de donner ou de prendre à bon escient, ou de changer de conduite à cet égard d’une façon ou d’une autre, ce serait un homme libéral, puisqu’il donnerait à ceux à qui il faut et prendrait là où il faut. Et c’est la raison pour laquelle on ne reconnaît \\
d’ordinaire à son caractère aucune perversité : dépasser la mesure dans la générosité et dans le refus de prendre n’est, en effet, la marque ni d’un méchant, ni d’un être vil, mais seulement d’un homme dépourvu de jugement. Le prodigue de ce type semble être nettement préférable à l’homme parcimonieux, d’abord pour les raisons que nous venons d’indiquer, et aussi parce que le prodigue rend service à beaucoup de gens, tandis que l’autre n’est utile à personne, pas même à soi. \\
Mais la majorité des prodigues, comme nous l’avons noté, prennent aussi de sources suspectes, et ils sont sous ce rapport des hommes parcimonieux. Ce qui les rend fortement enclins à prendre, c’est qu’ils veulent dépenser mais ne peuvent pas facilement le faire, parce que les ressources leur font rapidement défaut et qu’ainsi ils sont obligés, pour s’en procurer de  nouvelles, de s’adresser à d’autres sources. En même temps aussi, dans leur indifférence pour le bien, ils s’inquiètent peu de la façon dont ils prennent l’argent et en acceptent de toutes mains : ce qu’ils désirent, c’est donner, et peu leur importe comment ni d’où ils prennent l’argent. Aussi leurs générosités ne sont-elles pas des libéralités véritables : elles n’ont rien de noble, ne poursuivent aucune fin honnête et ne sont pas faites \\
de la façon requise : au contraire, ils enrichissent parfois ceux qui devraient être pauvres, ne donneront rien aux gens de mœurs irréprochables mais réservent leurs largesses aux flatteurs ou aux ministres de leurs plaisirs. Et c’est pourquoi la plupart des prodigues sont aussi des hommes déréglés, car ils sont facilement dépensiers et gaspilleurs pour leurs débauches, et, faute \\
de mener une vie conforme au bien, s’abandonnent à tous les plaisirs.\par
Voilà donc vers quoi se tourne le prodigue quand il est laissé sans conducteur, alors que trouvant quelqu’un pour s’intéresser à lui, il pourrait atteindre le juste milieu et le point convenable. La parcimonie, au contraire, est un vice incurable, car c’est la vieillesse ou une autre impuissance quelconque, qui semble bien rendre les hommes parcimonieux. Elle est d’ailleurs enracinée dans l’humaine nature plus \\
profondément que la prodigalité, car la plupart des gens sont cupides plutôt que généreux. Ce vice prend une grande extension et revêt de multiples aspects, car c’est de nombreuses façons que la parcimonie se fait jour.\par
Consistant, en effet, en deux éléments, le défaut dans le fait de donner et l’excès dans le fait de prendre, elle ne se rencontre pas toujours à l’état complet, et ses deux éléments existent \\
parfois séparément, certains hommes dépassant la mesure dans l’acquisition de la richesse, et d’autres péchant par défaut dans ce qu’ils donnent. Les uns, en effet, gratifiés de surnoms tels que {\itshape avares, fesse-mathieux}, ladres, manquent tous de facilité pour donner, mais ne convoitent pas le bien des autres et ne désirent pas s’en emparer, soit par une sorte d’honnêteté et de timidité à commettre des actions honteuses (puisque certains \\
semblent conserver jalousement leur argent, c’est du moins ce qu’ils disent, pour la seule raison de ne se trouver ainsi jamais dans la nécessité d’accomplir une mauvaise action : à ce groupe appartient le {\itshape scieur de cumin} ou autre maniaque de ce genre, qui tire son nom d’une excessive répugnance à ne jamais rien donner), soit encore que la crainte les détourne de s’approprier le bien d’autrui, dans la pensée qu’il n’est pas \\
facile de s’emparer soi-même du bien des autres sans que ceux-ci à leur tour s’emparent du vôtre, se déclarant ainsi satisfaits de ne rien prendre comme de ne rien donner.\par
D’autres, au contraire, dépassent la mesure quand il s’agit d’acquérir, en prenant de tous côtés et tout ce qu’ils peuvent : c’est le cas de ceux qui exercent des métiers dégradants, tenanciers de mauvais lieu et toutes autres gens de cette espèce, usuriers prêtant de petites sommes à gros intérêts, qui tous  recueillent l’argent de sources inavouables et dépassent toute mesure. Leur vice commun, c’est manifestement une cupidité sordide, puisque tous, pour l’amour du gain, gain au surplus médiocre, endurent les pires avanies. Ceux, en effet, qui réalisent des gains sur une grande échelle, sans se soucier de leur \\
provenance ni de leur nature, par exemple les tyrans qui saccagent les villes et dépouillent les temples, nous ne les nommons pas des hommes parcimonieux, mais plutôt des hommes pervers, ou impies, ou injustes. Cependant le joueur, le pillard et le brigand rentrent dans la classe des parcimonieux par leur sordide amour du gain, car c’est en vue du gain que les uns comme les autres déploient leur habileté et endurent les pires \\
hontes, les voleurs s’exposant aux plus grands dangers dans l’espoir du butin, les joueurs réalisant des gains au détriment de leurs amis, pour lesquels ils devraient plutôt se montrer généreux, Ainsi les uns et les autres, en voulant réaliser des gains d’origine inavouable, sont poussés par un sordide amour du profit. Et dès lors toutes ces différentes façons de prendre sont de la parcimonie. C’est donc à bon droit que la parcimonie est appelée le contraire de la libéralité, car, en même temps \\
qu’elle constitue un plus grand mal que la prodigalité, on est sujet à commettre plus d’erreurs en ce sens-là que dans le sens de la prodigalité telle que nous l’avons décrite.
\subsection[{4 (1122a — 1122b) < La magnificence >}]{4 (1122a — 1122b) < La magnificence >}
\noindent Nous avons suffisamment parlé de la libéralité et des vices qui lui sont opposés, On pensera qu’après cela doit venir la discussion sur la magnificence [µεγαλοπρεπεια], laquelle est, semble-t-il bien, elle aussi, une vertu ayant rapport à l’argent. Mais, à la différence \\
de la libéralité, elle ne s’étend pas à toutes les actions ayant l’argent pour objet, mais seulement à celles qui concernent la dépense, et, dans ce domaine, elle surpasse la libéralité en grandeur.\par
Comme son nom même le suggère, elle consiste dans une dépense convenant à la grandeur de son objet. Or la grandeur est quelque chose de relatif, car les dépenses à engager pour un \\
triérarque ne sont pas les mêmes que pour un chef de théorie. Le convenable en matière de dépenses est donc relatif à l’agent, aux circonstances et à l’objet. Mais l’homme qui, dans les petites choses ou dans les moyennes, dépense selon qu’elles le méritent n’est pas ce qu’on nomme un homme magnifique (tel celui qui dit : {\itshape Souvent j’ai donné au vagabond}), mais c’est seulement celui qui agit ainsi dans les grandes choses : car, bien que l’homme magnifique soit un homme libéral, l’homme libéral n’est pas pour autant un homme magnifique.\par
\\
Dans une disposition de ce genre, la déficience s’appelle mesquinerie, et l’excès, vulgarité, manque de goût, et autres dénominations analogues. Ce dernier vice constitue un excès, non pas en ce qu’on dépense largement pour des objets qui en valent la peine, mais en ce qu’on engage des dépenses de pure ostentation dans des occasions et d’une façon également inopportunes. Nous parlerons plus loin de ces vices.\par
Le magnifique est une sorte de connaisseur, car il a la \\
capacité de discerner ce qu’il sied de faire et de dépenser sur  une grande échelle avec goût. Nous l’avons dit, en effet, au début, la disposition du caractère se définit par ses activités et par ses objets. Or les dépenses du magnifique sont à la fois considérables et répondent à ce qu’il est séant d’accomplir ; tels sont par suite également les caractères des œuvres réalisées, car ainsi il y aura dépense considérable et en pleine convenance avec l’œuvre accomplie. Par conséquent, comme le \\
résultat doit répondre dignement à la dépense, ainsi aussi la dépense doit être proportionnée au résultat, ou même lui être supérieure. — En outre, l’homme magnifique, en dépensant de pareilles sommes aura le bien pour fin, ce qui est un caractère commun à toutes les vertus. Et il le fera aussi avec joie et avec profusion, car se montrer pointilleux dans les comptes est le fait d’une nature mesquine. Et il examinera la façon d’obtenir le plus beau résultat et le plus hautement convenable, plutôt \\
que s’inquiéter du prix et du moyen de payer le moins possible. Le magnifique sera donc aussi nécessairement un homme libéral, car l’homme libéral également dépensera ce qu’il faut et comme il faut ; et c’est dans l’observation de cette double règle que ce qu’il y a de grand dans l’homme magnifique, en d’autres termes sa grandeur, se révèle, puisque c’est là ce qu’il y a de commun avec l’exercice de la libéralité. Et d’une égale dépense il tirera un résultat plus magnifique. En effet, la même excellence n’est pas attachée à une chose qu’on \\
possède et à une œuvre qu’on réalise : en matière de possession, c’est ce qui a la plus grande valeur marchande qu’on prise le plus, l’or par exemple ; tandis que s’il s’agit d’une œuvre, la plus estimée est celle qui est grande et belle, car la contemplation d’une œuvre de ce genre soulève l’admiration du spectateur, et le fait de causer l’admiration appartient précisément à l’œuvre magnifique réalisée ; et l’œuvre a son excellence, c’est-à- dire sa magnificence, dans sa grandeur.
\subsection[{5 (1122b — 1123a) < La magnificence, suite >}]{5 (1122b — 1123a) < La magnificence, suite >}
\noindent La magnificence résulte des dépenses dont la qualité est \\
pour nous du plus haut prix : ce seront, par exemple, celles qui concernent les dieux, comme les offrandes votives, les édifices, les sacrifices ; pareillement celles qui touchent à tout ce qui présente un caractère religieux ; ou encore celles qu’on ambitionne de faire pour l’intérêt public, comme l’obligation dans certains endroits d’organiser un chœur avec faste, ou d’équiper une trirème, ou même d’offrir un repas civique. Mais dans tous ces cas, comme nous l’avons dit, on doit apprécier la dépense par référence à l’agent lui-même, c’est-à-dire se demander à quelle personnalité on a affaire et de quelles \\
ressources il dispose : car la dépense doit répondre dignement aux moyens, et être en convenance non seulement avec l’œuvre projetée, mais encore avec son exécutant. C’est pourquoi un homme pauvre ne saurait être magnifique, parce qu’il ne possède pas les moyens de faire de grandes dépenses d’une manière appropriée, et toute tentative en ce sens est un manque de jugement, car il dépense au-delà de ce qu’on attend de lui et de ce à quoi il est tenu, alors que l’acte conforme à la vertu est celui qui est fait comme il doit l’être. Mais les dépenses de \\
magnificence conviennent à ceux qui sont en possession des moyens appropriés, provenant soit de leur propre travail, soit de leurs ancêtres, soit de leurs relations, ou encore aux personnes de haute naissance, ou aux personnages illustres, et ainsi de suite, car toutes ces distinctions emportent grandeur et prestige. Tel est donc avant tout l’homme magnifique, et la magnificence se montre dans les dépenses de ce genre, ainsi \\
que nous l’avons dit, car ce sont les plus considérables et les plus honorables. Parmi les grandes dépenses d’ordre  privé, citons celles qui n’ont lieu qu’une fois, par exemple un mariage ou un événement analogue, et ce qui intéresse la cité tout entière, ou les personnes de rang élevé ; ou encore pour la réception ou le départ d’hôtes étrangers, ainsi que dons et rémunérations. Le magnifique, en effet, ne dépense pas pour \\
lui-même, mais dans l’intérêt commun, et ses dons présentent quelque ressemblance avec les offrandes votives. C’est aussi le fait d’un homme magnifique que de se ménager une demeure en rapport avec sa fortune (car même une belle maison est une sorte de distinction), et ses dépenses devront même porter de préférence sur ces travaux, qui sont destinés à durer (car ce sont les plus nobles), et en chaque occasion il dépensera ce qu’il est séant de dépenser. Ce ne sont pas, en effet, les mêmes dons qui \\
conviennent à des dieux et à des hommes, pour un temple et pour un tombeau. Et puisque chaque forme de dépense peut être grande dans le genre considéré, et, bien que la plus magnifique de toutes soit une grande dépense pour une grande chose, que dans tel cas particulier la plus magnifique est celle qui est grande dans le cas en question ; puisque, de plus, la grandeur existant dans l’œuvre réalisée est différente de celle existant dans la dépense (car la plus jolie balle à jouer ou la plus belle \\
fiole est une chose magnifique pour un cadeau à un enfant, quoique son prix soit modeste et mesquin), — il s’ensuit de tout cela que ce qui caractérise l’homme magnifique, c’est, quel que soit le genre de résultat auquel il aboutit, de le réaliser avec magnificence (un pareil résultat n’étant pas facile à dépasser), et d’une façon qui réponde dignement à la dépense.
\subsection[{6 (1123a) < La magnificence et ses contraires, suite >}]{6 (1123a) < La magnificence et ses contraires, suite >}
\noindent Tel est donc l’homme magnifique ; et l’homme qui, au \\
contraire, tombe dans l’excès, l’homme vulgaire, exagère en dépensant au-delà de ce qui convient, ainsi que nous l’avons dit. En effet, dans les petites occasions de dépenses, il gaspille des sommes considérables et déploie un faste démesuré : par exemple, à un repas par écot il donne l’éclat d’un repas de noces, et s’il équipe un chœur de comédie il le fait s’avancer à sa première entrée sur de la pourpre, comme à Mégare. Et \\
toutes ces sottises, il les accomplira non pas pour un noble motif, mais pour étaler sa richesse, pensant exciter ainsi l’admiration. Dans les circonstances où il faut dépenser largement il se montre parcimonieux, et là où une faible dépense suffirait, prodigue.\par
À l’opposé, l’homme mesquin pèche en toutes choses par défaut : même après avoir dépensé l’argent à pleines mains, il gâtera pour une bagatelle la beauté du résultat, hésitant en \\
tout ce qu’il fait, étudiant de quelle façon dépenser le moins possible, ce qui ne l’empêche pas de pousser des lamentations et de s’imaginer toujours faire les choses plus grandement qu’il ne faut.\par
Ces dispositions du caractère sont assurément vicieuses, mais n’apportent du moins avec elles aucun déshonneur, parce qu’elles ne sont ni dommageables pour le prochain, ni d’un aspect par trop repoussant.
\subsection[{7 (1123a — 1124a) < La magnanimité >}]{7 (1123a — 1124a) < La magnanimité >}
\noindent La magnanimité [µεγαλοψυχια] a rapport à de grandes choses, comme semble encore l’indiquer son nom. Mais de quelles grandes \\
choses s’agit-il ? C’est là ce que nous devons tout d’abord  saisir. Peu importe d’ailleurs que nous examinions la disposition en elle-même ou l’homme qui répond à cette disposition.\par
On pense d’ordinaire qu’est magnanime celui qui se juge lui-même digne de grandes choses, et qui en est réellement digne ; car celui qui, sans en être digne, agit de même, est un homme sans jugement, et au nombre des gens vertueux ne figurent ni l’homme sans jugement, ni le sot.\par
Magnanime, donc, est l’homme que nous venons de \\
décrire (celui qui n’est digne que de petites choses et qui s’estime lui-même digne d’elles est un homme modeste, mais non un homme magnanime, puisque c’est dans la grandeur que se situe la magnanimité, tout comme la beauté dans un corps majestueux : les gens de petite taille peuvent être élégants et bien proportionnés, mais ne peuvent pas être beaux). D’autre part, celui qui s’estime lui-même digne de grandes choses, tout en étant réellement indigne d’elles, est un vaniteux (quoique celui qui s’estime au-dessus de son mérite ne soit pas toujours un vaniteux) ; celui qui se juge moins qu’il ne vaut est un \\
pusillanime, qu’il soit digne de grandes choses ou de choses moyennes, ou même, quoique n’étant digne que de petites choses, s’il s’estime encore audessous d’elles. Le plus haut degré de la pusillanimité semblera se rencontrer dans celui qui est digne de grandes choses : car que ferait-il, si son mérite n’était pas aussi grand ? Ainsi, l’homme magnanime, d’une part est un extrême par la grandeur < de ce à quoi il peut prétendre >, et d’autre part un moyen par la juste mesure où il se tient (puisqu’il ne se juge digne que de ce dont il est effectivement digne), alors que l’homme vain et l’homme \\
pusillanime tombent dans l’excès ou le défaut.\par
Si donc l’homme magnanime est celui qui se juge lui-même digne de grandes choses et en est effectivement digne, et si l’homme le plus magnanime est celui qui se juge digne, et qui l’est, des choses les plus grandes, son principal objet ne saurait être qu’une seule et unique chose. Or le mérite se dit par relation avec les biens extérieurs ; et le plus grand de tous ces biens, nous pouvons l’assurer, est celui que nous offrons en hommage aux dieux, que les personnes élevées en dignité convoitent avec le plus d’ardeur, et qui est une récompense \\
accordée aux actions les plus nobles : à cette description nous reconnaissons l’honneur (qui est effectivement le plus grand des biens extérieurs). Par suite, le magnanime est celui qui, en ce qui regarde l’honneur et le déshonneur, adopte l’attitude qui convient. En dehors même de tout raisonnement, il est manifeste que la magnanimité a rapport à l’honneur, puisque c’est surtout de l’honneur que les grands s’estiment eux-mêmes dignes, et cela en conformité avec leur mérite.\par
Quant à l’homme pusillanime, il est dans un état d’insuffisance à la fois par rapport à ses propres mérites et par comparaison \\
avec ce dont se juge capable l’homme magnanime, tandis que le vaniteux dépasse la mesure par rapport à ses propres mérites, mais non du moins par rapport à ce dont le magnanime se juge capable.\par
L’homme magnanime, puisqu’il est digne des plus grandes choses, ne saurait qu’être un homme parfait : en effet, meilleur est l’homme et toujours plus grands sont les biens dont il est digne, et celui-là est digne des plus grands biens qui est parfait. Par conséquent, l’homme véritablement magnanime doit être un homme de bien. Et on pensera qu’à la grandeur \\
d’âme appartient ce qu’il y a de grand en chaque vertu. Il serait absolument contraire au caractère d’un homme magnanime, à la fois de s’enfuir à toutes jambes et de commettre une injustice : dans quel but ferait-il des actes honteux, lui pour qui rien n’a grande importance ? Et, à examiner chacune des vertus, il paraîtrait complètement ridicule que l’homme magnanime ne fût pas homme de bien, pas plus qu’il ne serait digne \\
d’être honoré s’il était pervers, puisque l’honneur est une récompense de la vertu et que c’est aux gens de bien qu’il est  rendu. La magnanimité semble donc être ainsi une sorte d’ornement des vertus, car elle les fait croître et ne se rencontre pas sans elles. C’est pourquoi il est difficile d’être véritablement un homme magnanime, car cela n’est pas possible sans une vertu parfaite.\par
Ainsi donc, c’est surtout en ce qui touche l’honneur \\
et le déshonneur que l’homme magnanime se révèle, et les honneurs éclatants, quand ils sont décernés par les gens de bien, lui feront ressentir une joie mesurée, dans la conviction qu’il n’obtient là que ce qui lui appartient en propre, ou même moins (puisqu’il ne saurait y avoir d’honneur digne d’une parfaite vertu) ; il ne les en acceptera pas moins de toute façon, \\
parce que les hommes n’ont rien de mieux à lui offrir. Quant à l’honneur rendu par des gens quelconques et pour des raisons futiles, il n’en fera absolument aucun cas (car ce n’est pas cela dont il est digne), et il agira de même pour le déshonneur (puisque aucun déshonneur ne peut qu’injustement s’attacher à lui). — C’est donc principalement de ce qui touche l’honneur, comme nous l’avons dit, que l’homme magnanime se préoccupe. Cependant, en ce qui concerne la richesse, le pouvoir, et la bonne ou mauvaise fortune en général, il se comportera avec \\
modération, de quelque façon que ces avantages se présentent à lui : il ne se réjouira pas avec excès dans la prospérité, ni ne s’affligera outre mesure dans l’adversité. En effet, même à l’égard de l’honneur il n’agit pas ainsi, et pourtant c’est le plus grand des biens (la puissance et la richesse n’étant des choses désirables que pour l’honneur qu’elles procurent : du moins ceux qui les possèdent souhaitent être honorés à cause d’elles) ; celui dès lors pour qui même l’honneur est peu de chose, à celui-là aussi tout le reste demeure indifférent. C’est pourquoi de tels hommes passent d’ordinaire pour dédaigneux.
\subsection[{8 (1124a — 1125a) < La magnanimité, suite >}]{8 (1124a — 1125a) < La magnanimité, suite >}
\noindent \\
On admet d’ordinaire que les dons de la fortune contribuent aussi à la magnanimité. En effet, les gens bien nés sont jugés dignes d’être honorés, ainsi que les personnes des classes dirigeantes ou les gens riches, parce qu’ils occupent une position supérieure aux autres, et que ce qui possède une supériorité en quelque bien jouit toujours d’une plus grande considération. C’est pourquoi même des avantages de cette nature ont pour effet de rendre les hommes plus magnanimes, car leurs possesseurs en retirent de la considération auprès de certains. \\
En toute vérité, l’homme de bien seul devrait être honoré ; cependant celui en qui résident à la fois la vertu et les avantages dont nous parlons est regardé comme plus digne d’honneur encore. Mais ceux qui, dépourvus de vertu, possèdent les biens de ce genre, ne sont ni justifiés à se croire eux-mêmes dignes de grandes choses, ni en droit de prétendre au nom de magnanime, tous avantages qui ne se rencontrent pas indépendamment d’une parfaite vertu. Mais ceux qui possèdent uniquement \\
les dons de la fortune deviennent eux aussi dédaigneux et insolents, car sans vertu il n’est pas facile de supporter avec aisance la prospérité, et de tels hommes, dans leur incapacité  d’y parvenir et se croyant supérieurs à tout le monde, méprisent les autres et font eux-mêmes tout ce qui leur passe par la tête. Ils imitent, en effet, l’homme magnanime sans être réellement pareils à lui et le copient en tout ce qu’ils peuvent ; ainsi, tout en n’agissant pas selon la vertu, ils méprisent les \\
autres. L’homme magnanime, en effet, méprise les autres parce qu’il en a le droit (puisqu’il juge avec vérité), tandis que la plupart des hommes le font au petit bonheur. L’homme magnanime ne se jette pas dans des dangers qui n’en valent pas la peine, pas plus qu’il n’aime les dangers en eux-mêmes, car il y a peu de choses qu’il apprécie. Mais il affronte le danger pour des motifs importants, et quand il s’expose ainsi il n’épargne pas sa propre vie, dans l’idée qu’on ne doit pas vouloir conserver la vie à tout prix. Par nature, il aime à répandre des bienfaits, mais il rougit d’en recevoir, \\
parce que, dans le premier cas c’est une marque de supériorité, et dans le second d’infériorité. Il est enclin à rendre plus qu’il ne reçoit, car de cette façon le bienfaiteur originaire contractera une nouvelle dette envers lui et sera l’obligé. En outre, les hommes magnanimes semblent ne garder mémoire que de ceux à qui ils ont fait du bien, à l’exclusion de ceux qui les ont eux-mêmes obligés : car celui qui reçoit un service est l’inférieur de celui qui le lui rend, alors que l’homme magnanime souhaite garder la supériorité. Et si son oreille est flattée des bienfaits qu’il a accordés, c’est sans plaisir qu’il entend parler \\
de ceux qu’il a reçus. Telle est la raison pour laquelle Thétis ne rappelle pas à Zeus les services qu’elle lui a rendus, et pour laquelle aussi les Lacédémoniens n’ont pas rappelé aux Athéniens les bons offices dont ils les avaient gratifiés, mais seulement les bienfaits qu’ils en avaient eux-mêmes reçus. Et c’est encore le fait d’un homme magnanime, de ne rien demander à personne, ou de ne le faire qu’avec répugnance, mais par contre de rendre service avec empressement. De même, s’il se montre plein de hauteur avec les puissants ou les \\
heureux de ce monde, il sait garder la modération avec les gens de condition moyenne : en effet, c’est une chose malaisée et qui impose le respect, de l’emporter sur les grands en excellence, tandis qu’avec les autres, c’est facile ; d’autre part, se montrer hautain envers les premiers n’a rien d’incivil, alors que c’est une grossièreté à l’égard du menu peuple, tout comme de déployer sa force contre les faibles. En outre, l’homme magnanime ne va pas chercher les honneurs ni les places où d’autres occupent le premier rang. Il est lent, il temporise, sauf là où une grave question d’honneur ou une affaire sérieuse sont en jeu ; il \\
ne s’engage que dans un petit nombre d’entreprises, mais qui sont d’importance et de renom. Son devoir impérieux est de se montrer à découvert dans ses haines comme dans ses amitiés, la dissimulation étant la marque d’une âme craintive. Il se soucie davantage de la vérité que de l’opinion publique, il parle et agit au grand jour, car le peu de cas qu’il fait des autres lui permet de s’exprimer avec franchise. C’est pourquoi aussi il \\
aime à dire la vérité, sauf dans les occasions où il emploie l’ironie, quand il s’adresse à la masse. Il est incapable de vivre  selon la loi d’autrui, sinon celle d’un ami, car c’est là un esclavage, et c’est ce qui fait que les flatteurs sont toujours serviles, et les gens de peu, des flatteurs. Il n’est pas non plus enclin à l’admiration, car rien n’est grand pour lui. Il est sans rancune : ce n’est pas une marque de magnanimité que de conserver du ressentiment, surtout pour les torts qu’on a subis, il vaut mieux \\
les dédaigner. Il n’aime pas non plus les commérages : il ne parlera ni de lui-même ni d’un autre, car il n’a souci ni d’éloge pour lui-même ni de blâme pour les autres, et il n’est pas davantage prodigue de louanges : de là vient qu’il n’est pas mauvaise langue, même quand il s’agit de ses ennemis, sinon par insolence délibérée. Dans les nécessités de la vie ou dans les circonstances insignifiantes, il est l’homme le moins \\
geignard et le moins quémandeur, car c’est prendre les choses trop à cœur que d’agir ainsi dans ces occasions. Sa nature le pousse à posséder les choses belles et inutiles plutôt que les choses profitables et avantageuses : cela est plus conforme à un esprit qui se suffit à soi-même.\par
En outre, une démarche lente est généralement regardée comme la marque d’un homme magnanime, ainsi qu’une voix grave et un langage posé : l’agitation ne convient pas à qui ne prend à cœur que peu de choses, ni l’excitation à qui pense \\
que rien n’a d’importance ; au contraire une voix aiguë et une démarche précipitée sont l’effet d’un tempérament agité et excitable.
\subsection[{9 (1125a) < La magnanimité et ses contraires, suite >}]{9 (1125a) < La magnanimité et ses contraires, suite >}
\noindent Tel est donc le caractère de l’homme magnanime. Celui qui, dans ce domaine, pèche par défaut est un homme pusillanime, et celui qui tombe dans l’excès un vaniteux. Ces deux derniers ne sont pas non plus généralement regardés comme des gens vicieux (ils ne font aucun mal), mais seulement comme des gens qui font fausse route. En effet, le pusillanime, \\
tout en étant digne de grands biens, se prive lui-même des avantages qu’il mérite, et il donne l’impression de recéler en lui quelque chose de mauvais, du fait qu’il se juge lui-même indigne de tous biens. Et il semble aussi se méconnaître lui-même, car autrement il convoiterait les choses dont il est digne, puisque ce sont là des biens. Non pas toutefois que les hommes de cette sorte soient tenus pour des sots : ce sont plutôt des timides. Mais cette opinion qu’ils ont d’eux-mêmes ne fait, semble-t-il bien, que renforcer leur infériorité : chaque classe \\
d’hommes, en effet, tend aux biens correspondant à son mérite ; or les pusillanimes s’abstiennent de toute action et de toute occupation vertueuses dans la pensée qu’ils en sont indignes, et ils se comportent de même à l’égard des biens extérieurs.\par
Les vaniteux, au contraire, sont des sots qui s’ignorent eux-mêmes, et on s’en aperçoit (car, tout comme s’ils en étaient dignes, ils entreprennent des tâches honorables, et \\
l’événement ne tarde pas à les confondre). Ils veulent briller par la parure, le maintien et autres avantages de ce genre ; ils souhaitent que les dons que la fortune leur a départis apparaissent au grand jour et ils en font état dans leurs paroles, croyant en tirer de la considération.\par
La pusillanimité s’oppose davantage à la magnanimité que la vanité, car elle est à la fois plus répandue et plus mauvaise.
\subsection[{10 (1125a — 1125b) < L’ambition, le manque d’ambition et la vertu intermédiaire >}]{10 (1125a — 1125b) < L’ambition, le manque d’ambition et la vertu intermédiaire >}
\noindent \\
La magnanimité a donc rapport à un honneur d’ordre élevé, comme il a été dit déjà.\par
 Il semble bien aussi y avoir, dans le domaine de l’honneur, ainsi que nous l’avons indiqué dans notre première partie, une vertu qui apparaîtrait voisine de la magnanimité, comme la libéralité l’est de la magnificence. Ces deux vertus, en effet, se tiennent en dehors de la grandeur, mais nous mettent dans \\
la position qui convient, en ce qui concerne les objets de moyenne et de petite importance. De même que dans l’acquisition et le don des richesses il existe une médiété aussi bien qu’un excès et un défaut, de même encore l’honneur peut être désiré plus qu’il ne faut ou moins qu’il ne faut, ou cherché à sa véritable source et d’une façon convenable. En effet, nous blâmons à la fois, d’une part l’ambitieux, en ce qu’il convoite \\
l’honneur plus qu’il ne convient et le cherche là où il ne faut pas, et, d’autre part, l’homme sans ambition, en ce qu’il se montre indifférent à l’honneur qu’on lui rend, même quand c’est pour de belles actions. Mais, à d’autres moments, nous louons, au contraire, l’ambitieux d’agir en homme et d’être plein d’une noble ardeur, et l’homme sans ambition pour son sens de la mesure et de la modération, ainsi que nous l’avons noté dans nos premières études73. On voit que l’expression {\itshape passionné pour telle ou telle chose} se prend en plusieurs sens, \\
et que nous n’appliquons pas toujours à la même chose le terme ambitieux < passionné pour l’honneur > : c’est une expression élogieuse quand nous avons en vue celui qui aime l’honneur plus que ne le fait la majorité des hommes, et elle revêt un sens péjoratif au contraire quand nous pensons à celui qui aime l’honneur plus qu’il ne convient. Et comme la moyenne à observer n’a pas de nom spécial, les deux extrêmes paraissent se disputer sa place comme si elle était vacante. Mais là où il y a excès et défaut existe aussi le moyen ; or on peut convoiter l’honneur à la fois plus et moins qu’on ne le devrait ; il est donc \\
aussi possible de le désirer comme il est convenable, et c’est cette dernière disposition du caractère qui est l’objet de nos éloges, disposition qui constitue dans le domaine de l’honneur une médiété dépourvue de désignation spéciale. Comparée à l’ambition, elle apparaît manque d’ambition, et comparée au manque d’ambition, ambition ; comparée enfin à l’un et à l’autre, elle est, en un sens, les deux en même temps. Cela semble bien être également le cas pour les autres vertus, mais, dans l’espèce présente, les extrêmes paraissent seulement \\
opposés l’un à l’autre, du fait que la vertu moyenne n’a pas reçu de nom.
\subsection[{11 (1125b — 1126b) < La douceur >}]{11 (1125b — 1126b) < La douceur >}
\noindent La douceur [πραοτης] est une médiété dans le domaine des sentiments de colère, mais l’état intermédiaire n’ayant pas de nom, et les extrêmes se trouvant presque dans le même cas, nous appliquons le terme douceur au moyen, quoique la douceur incline plutôt du côté de la déficience. Celle-ci est dépourvue de nom, mais l’excès pourrait s’appeler une sorte \\
d’irascibilité, car la passion en question est une colère, bien que les causes qui la produisent soit multiples et diverses.\par
L’homme donc qui est en colère pour les choses qu’il faut et contre les personnes qui le méritent, et qui en outre l’est de la façon qui convient, au moment et aussi longtemps qu’il faut, un tel homme est l’objet de notre éloge. Cet homme sera dès lors un homme doux, s’il est vrai que le terme de douceur est pour nous un éloge (car le terme doux signifie celui qui reste \\
imperturbable et n’est pas conduit par la passion, mais ne s’irrite que de la façon, pour les motifs et pendant le temps que  la raison peut dicter ; il semble toutefois errer plutôt dans le sens du manque, l’homme doux n’étant pas porté à la vengeance, mais plutôt à l’indulgence).\par
La déficience, d’autre part, qu’elle soit une sorte d’indifférence à la colère ou tout ce qu’on voudra, est une disposition que nous blâmons (car ceux qui ne s’irritent pas pour les choses \\
où il se doit sont regardés comme des niais, ainsi que ceux qui ne s’irritent pas de la façon qu’il faut, ni quand il faut, ni avec les personnes qu’il faut : de pareilles gens donnent l’impression de n’avoir de la position où ils se trouvent ni sentiment, ni peine, et, faute de se mettre en colère, d’être incapables de se défendre : or endurer d’être bafoué ou laisser avec indifférence insulter ses amis, est le fait d’une âme vile).\par
L’excès, de son côté, a lieu de toutes les façons dont nous \\
avons parlé (on peut être en colère, en effet, avec des personnes qui ne le méritent pas, pour des choses où la colère n’est pas de mise, plus violemment, ou plus rapidement, ou plus longtemps qu’il ne faut), bien que tous ces traits ne se rencontrent pas dans la même personne, ce qui serait d’ailleurs une impossibilité, car le mal va jusqu’à se détruire lui-même, et quand il est complet devient intolérable.\par
Quoi qu’il en soit, il y a d’abord les irascibles qui se mettent en colère sans crier gare, contre des gens qui n’en peuvent mais, pour des choses qui n’en valent pas la peine et \\
plus violemment qu’il ne convient. Mais leur colère tombe vite, et c’est même là le plus beau côté de leur caractère : cela tient chez eux à ce qu’ils ne compriment pas leur colère, mais réagissent ouvertement à cause de leur vivacité, et ensuite leur colère tombe à plat. — Les caractères très colériques sont vifs à l’excès et portés à la colère envers tout le monde et en toute occasion ; d’où leur nom. — Les caractères amers, \\
d’autre part, sont difficiles à apaiser et restent longtemps sur leur colère, car ils contiennent leur emportement, mais le calme renaît une fois qu’ils ont rendu coup pour coup : la vengeance fait cesser leur colère, en faisant succéder en eux le plaisir à la peine. Mais si ces représailles n’ont pas lieu, ils continuent à porter le fardeau de leur ressentiment, car leur rancune n’apparaissant pas au dehors, personne ne tente de les apaiser, et digérer en soi-même sa propre colère est une \\
chose qui demande du temps. De pareilles gens sont les plus insupportables à la fois à eux-mêmes et à leurs plus chers amis. — Enfin, nous qualifions de caractères difficiles ceux qui s’irritent dans les choses qui n’en valent pas la peine, plus qu’il ne faut et trop longtemps, et qui ne changent de sentiments qu’ils n’aient obtenu vengeance ou châtiment.\par
À la douceur nous donnons comme opposé plutôt l’excès \\
que le défaut, parce que l’excès est plus répandu (le désir de se venger est un sentiment plus naturel à l’homme < que l’oubli des injures >, et aussi parce que les caractères difficiles s’adaptent avec plus de peine à la vie en société.\par
Ce que nous avons indiqué dans nos précédentes analyses reçoit un surcroît d’évidence de ce que nous disons présentement, à savoir qu’il n’est pas aisé de déterminer comment, à l’égard de qui, pour quels motifs et pendant combien de temps on doit être en colère, et à quel point précis, en agissant ainsi, \\
on cesse d’avoir raison et on commence à avoir tort. En effet, une légère transgression de la limite permise n’est pas pour autant blâmée, qu’elle se produise du côté du plus ou du côté du moins : ainsi parfois nous louons ceux qui pèchent par insuffisance  et les qualifions de doux, et, d’autre part, nous louons les caractères difficiles, pour leur virilité qui, dans notre pensée, les rend aptes au commandement. Dès lors il n’est pas aisé de définir dans l’abstrait de combien et de quelle façon il faut franchir la juste limite pour encourir le blâme : cela rentre dans le domaine de l’individuel, et la discrimination est du ressort \\
de la sensation. Mais ce qui du moins est clair, c’est l’appréciation favorable que mérite la disposition moyenne, selon laquelle nous nous mettons en colère avec les personnes qu’il faut, pour des choses qui en valent la peine, de la façon qui convient, et ainsi de suite, et que, d’autre part, l’excès et le défaut sont également blâmables, blâme léger pour un faible écart, plus accentué si l’écart est plus grand, et d’une grande sévérité enfin quand l’écart est considérable. On voit donc clairement que c’est à la disposition moyenne que nous devons nous attacher.
\subsection[{12 (1126b — 1127a) < L’affabilité et ses vices opposés >}]{12 (1126b — 1127a) < L’affabilité et ses vices opposés >}
\noindent \\
Les états ayant rapport à la colère ont été suffisamment étudiés.\par
Dans les relations journalières, la vie en société, le commerce de la conversation et des affaires, certains sont considérés comme des gens complaisants [αρεσκεια], qui se font un plaisir de tout approuver et de n’opposer jamais de résistance, estimant que c’est pour eux un devoir d’éviter toute contrariété à leur entourage. Et ceux qui, à l’inverse des précédents, \\
soulèvent des difficultés sur toutes choses, sans se soucier le moins du monde de causer de la peine à autrui, sont qualifiés d’esprits hargneux et chicaniers. Il est bien clair que les dispositions dont nous venons de parler sont blâmables, et que la position moyenne entre ces états est au contraire digne d’éloge : c’est celle qui nous fera accueillir, et pareillement repousser, les choses qu’il faut et de la façon qu’il faut. Mais aucun nom \\
n’a été assigné à cette disposition, quoiqu’elle ait la plus grande ressemblance avec l’amitié : car celui qui répond à cette disposition moyenne est cette sorte d’hommes que nous entendons désigner par l’expression de « bon ami », s’il s’y ajoute l’affection. Toutefois cet état diffère de l’amitié en ce qu’il est exempt de tout facteur sentimental et d’affection pour ceux avec lesquels on a commerce, car ce n’est pas par amour ou par haine qu’on accueille tout ce qui vient des autres comme il se doit, mais parce qu’on est constitué de cette façon-là. En effet, \\
qu’il s’agisse d’inconnus ou de gens de connaissance, de familiers ou d’indifférents, on agira de même, sauf à s’adapter à la diversité des cas, car on ne saurait avoir la même sollicitude envers des familiers ou des étrangers, ni non plus les traiter sur un pied d’égalité pour les peines qu’on peut leur causer.\par
Nous avons dit en termes généraux que l’homme de cette sorte se comportera dans ses rapports avec autrui comme il doit se comporter ; mais c’est en se référant à des considérations d’honnêteté et d’utilité qu’il cherchera à ne pas contrister les \\
autres ou à contribuer à leur agrément, puisqu’il est entendu qu’il s’agit ici de plaisirs et de peines se produisant dans la vie de société ; et dans les cas où il est déshonorant ou dommageable pour l’homme dont nous parlons de contribuer à l’agrément des autres, il s’y refusera avec indignation et préférera leur causer de la peine. D’autre part, si son approbation apporte à l’auteur de l’acte, à son tour, un discrédit qui soit d’une importance considérable, ou un tort quelconque, alors que son \\
opposition ne peut lui causer qu’une peine légère, il n’accordera pas son assentiment mais ne craindra pas de déplaire. Et, dans ses relations sociales, il traitera différemment les  personnes de rang élevé et les gens du commun, ainsi que les personnes qui sont plus ou moins connues de lui ; il aura pareillement égard aux autres distinctions, rendant à chaque classe d’individus ce qui lui est dû. Et s’il estime préférable en soi de contribuer à l’agrément des autres et d’éviter de les contrister, en fait il aura égard aux conséquences, si elles sont \\
plus fortes, je veux dire à l’honnêteté et à l’utilité. Et pour procurer un grand plaisir à venir, il causera une peine légère dans le présent.\par
Tel est donc l’homme qui occupe la position moyenne, sans toutefois porter de désignation spéciale. De ceux qui causent du plaisir aux autres, celui qui vise uniquement à faire plaisir sans poursuivre aucune autre fin, est un complaisant, et celui qui agit ainsi pour l’avantage qu’il en retire personnellement, soit en argent soit en valeur appréciable en argent, celui-là \\
est un flatteur [κολαξ]. Celui qui, au contraire, fait des difficultés en toute occasion est, comme nous l’avons dit, un homme hargneux et chicanier. Et les extrêmes paraissent être opposés l’un à l’autre, du fait que le moyen terme n’a pas de nom.
\subsection[{13 (1127a — 1127b) < L’homme véridique et ses opposés >}]{13 (1127a — 1127b) < L’homme véridique et ses opposés >}
\noindent Dans une sphère sensiblement la même se rencontre encore la médiété opposée à la fois à la vantardise < et à la réticence >, et qui elle non plus n’a pas reçu de nom. Mais il n’est pas mauvais d’approfondir aussi les dispositions de ce \\
genre : nous connaîtrons mieux ce qui a trait à la moralité après avoir passé en revue chacune de ses manifestations, et nous acquerrons en outre la conviction que les vertus sont bien des médiétés, si d’un seul regard nous voyons qu’il en est ainsi dans tous les cas.\par
Dans la vie en société, les hommes qui n’ont en vue que de causer du plaisir ou de la peine à ceux qu’ils fréquentent ont déjà été étudiés. Parlons maintenant de ceux qui recherchent la vérité ou le mensonge pareillement dans leurs discours et \\
dans leurs actes, ainsi que dans leurs prétentions. De l’avis général, alors, le vantard est un homme qui s’attribue des qualités susceptibles de lui attirer de la réputation tout en ne les possédant pas, ou encore des qualités plus grandes qu’elles ne sont en réalité ; inversement, le réticent dénie les qualités qu’il possède ou les atténue ; enfin, celui qui se tient dans un juste milieu est un homme sans détours [αυθεκαστος], sincère à la fois dans sa vie \\
et dans ses paroles, et qui reconnaît l’existence de ses qualités propres, sans y rien ajouter ni retrancher. La sincérité et la fausseté peuvent l’une et l’autre être pratiquées soit en vue d’une fin déterminée, soit sans aucun but. Mais en tout homme le véritable caractère se révèle dans le langage, les actes et la façon de vivre, toutes les fois qu’il n’agit pas en vue d’une fin. Et en elle-même, la fausseté est une chose basse et répréhensible, \\
et la sincérité une chose noble et digne d’éloge. Ainsi également l’homme sincère qui se tient au milieu des deux opposés mérite la louange, tandis que l’homme faux, aussi bien dans un sens que dans l’autre, est un être méprisable, mais plus particulièrement le vantard. Traitons à la fois de l’homme véridique et de l’homme faux, en commençant par le premier.\par
Nous ne parlons pas ici de la bonne foi dans les contrats, ni dans les matières qui se rapportent à la justice ou à l’injustice  (c’est d’une autre vertu que ces choses-là doivent relever)7\\
: nous parlons des cas où, aucune considération de ce genre n’offrant d’intérêt, un homme est véridique dans ses paroles et dans sa vie parce que telle est la disposition habituelle de son caractère. On peut penser qu’un pareil homme est un homme de bien. En effet, celui qui aime la vérité et se montre sincère même dans des choses où cela n’importe en rien, sera, à plus \\
forte raison encore, sincère dans les cas où cela présente de l’intérêt : il se gardera alors de la fausseté comme d’une action honteuse, lui qui s’en détournait déjà par simple répulsion de ce qu’elle est en elle-même ; et un tel homme mérite nos éloges. Il aura même plutôt tendance à rester au-dessous de la vérité, et c’est là une attitude qui, de toute évidence, est de meilleur ton, toute exagération étant à charge aux autres.\par
Quant à l’homme qui a des prétentions dépassant la réalité de ses propres mérites, tout en n’ayant aucune fin en vue, il \\
apparaît assurément comme un être méprisable (autrement il ne prendrait pas plaisir à mentir), mais il donne pourtant l’impression d’être plus vain que méchant. Supposons maintenant qu’il agisse dans un but déterminé : si c’est en vue de la gloire ou de l’honneur, il n’y a trop rien à reprendre (c’est précisément le cas du vantard), mais si c’est en vue de l’argent ou dans un intérêt pécuniaire, sa conduite est alors plus honteuse. D’autre part, ce n’est pas la simple potentialité qui \\
fait le vantard, mais le choix délibéré : c’est en raison de la disposition de son caractère et parce qu’il est un homme de telle nature qu’il est un vantard. Mêmes distinctions en ce qui concerne le menteur : tel est menteur parce qu’il aime le mensonge pour le mensonge même, tel autre par désir de la gloire ou du gain. Or ceux qui font les vantards en vue d’acquérir la réputation se donnent les qualités de nature à susciter les éloges ou les félicitations ; mais ceux qui recherchent un avantage matériel prétendent à des qualités dont leur voisinage tire profit, et dont au surplus l’absence passe facilement inaperçue : \\
par exemple le talent d’un devin, d’un savant ou d’un médecin. C’est pour cela que les qualités de cette sorte sont celles que la plupart des vantards se donnent et qui font l’objet de leurs forfanteries, car on trouve en eux celles que nous venons de décrire.\par
Les réticents qui ne parlent d’eux-mêmes qu’en atténuant la vérité, apparaissent comme étant de mœurs plus aimables, (car on admet qu’ils ne parlent pas en vue du gain, mais qu’ils fuient l’ostentation). Pour ces derniers aussi, il s’agit principalement des qualités donnant une bonne réputation, qualités \\
qu’ils déclarent ne pas posséder, suivant la manière de faire de Socrate. Ceux qui nient posséder des qualités sans importance ou des qualités qu’ils possèdent manifestement, sont appelés finassiers et sont à juste titre plus méprisables (parfois même cette affectation a toute l’apparence de la vantardise, comme le vêtement des Lacédémoniens77, car l’excès aussi bien que la déficience poussée trop loin ont quelque chose de fanfaron). \\
Mais ceux qui usent avec modération de la réticence, et pour des qualités dont l’évidence ne soit pas par trop apparente, apparaissent comme des gens de distinction.\par
Enfin, le vantard paraît bien être l’opposé de l’homme sincère, car il est pire que le réticent.
\subsection[{14 (1127b — 1128b) < Le bon goût dans l’activité de jeu >}]{14 (1127b — 1128b) < Le bon goût dans l’activité de jeu >}
\noindent Comme il y a aussi des moments de repos dans l’existence, et qu’une forme de ce repos consiste dans le loisir accompagné d’amusement, dans ce domaine également il semble bien y avoir un certain bon ton des relations sociales,  qui détermine quelles sortes de propos il est de notre devoir de tenir et comment les exprimer, et pareillement aussi quels sont ceux que nous pouvons nous permettre d’entendre. Il y aura à cet égard une différence suivant la qualité des interlocuteurs auxquels nous nous adresserons ou que nous écouterons. On voit que dans ces matières aussi il peut y avoir à la fois excès et défaut par rapport au juste milieu.\par
Ceux qui pèchent par exagération dans la plaisanterie sont \\
considérés comme de vulgaires bouffons, dévorés du désir d’être facétieux à tout prix, et visant plutôt à provoquer le rire qu’à observer la bienséance dans leurs discours et à ne pas contrister la victime de leurs railleries. Ceux, au contraire, qui ne peuvent ni proférer eux-mêmes la moindre plaisanterie ni entendre sans irritation les personnes qui en disent, sont tenus pour des rustres et des grincheux. Quant à ceux qui plaisantent \\
avec bon goût, ils sont ce qu’on appelle des gens d’esprit, ou, si l’on veut, des gens à l’esprit alerte, car de telles saillies semblent être des mouvements du caractère, et nous jugeons le caractère des hommes comme nous jugeons leur corps, par leurs mouvements. Mais comme le goût de la plaisanterie est très répandu et que la plupart des gens se délectent aux facéties et aux railleries plus qu’il ne faudrait, même les bouffons se \\
voient gratifiés du nom d’hommes d’esprit et passent pour des gens de bon ton ; mais qu’en fait ils diffèrent d’une façon nullement négligeable du véritable homme d’esprit, c’est là une chose qui résulte manifestement de ce que nous venons de dire.\par
La disposition du caractère qui occupe le juste milieu est encore marquée par le tact : c’est le fait d’un homme de tact de dire et d’écouter seulement les choses qui s’accordent avec la nature de l’homme vertueux et libre, car il y a certaines choses qu’il sied à un homme de cette sorte de dire ou d’entendre par \\
manière de plaisanterie, et la plaisanterie de l’homme libre diffère de celle de l’homme d’une nature servile, comme, de son côté, la plaisanterie d’un homme bien élevé n’est pas celle d’un homme sans éducation. On peut se rendre compte de cette différence en comparant les comédies anciennes et les nouvelles : pour les anciens auteurs comiques, c’était l’obscénité qui faisait rire, tandis que pour les nouveaux auteurs, ce sont plutôt les sous-entendus, ce qui constitue un progrès, qui n’est \\
pas négligeable, vers la bonne tenue. Dans ces conditions, devons-nous définir le railleur bien élevé en disant que ses plaisanteries ne sont jamais malséantes au jugement d’un homme libre, ou devons-nous dire que c’est parce qu’il évite de contrister celui qui l’écoute ou même qu’il s’efforce de le réjouir ? Mais cette dernière définition ne porte-t-elle pas sur quelque chose de bien vague ? Car ce qu’on aime et ce qu’on déteste varie avec les différents individus. Telle sera aussi la nature des plaisanteries que le railleur de bon ton écoutera, car les plaisanteries qu’il supporte d’entendre sont aussi celles qu’il trouve bon de faire lui-même. Il ne lancera donc pas \\
n’importe quelle plaisanterie, car la raillerie constitue une sorte d’outrage, et certaines formes d’outrages sont prohibées par le législateur ; peut-être aussi devrait-on interdire certaines formes de raillerie. — Ainsi donc, l’homme libre et de bon ton se comportera comme nous l’avons indiqué, étant en quelque sorte sa loi à lui-même.\par
Tel est donc le caractère de celui qui se tient dans le juste milieu, qu’on l’appelle homme de tact ou homme d’esprit. Le bouffon, lui, est l’esclave de son goût de la plaisanterie, ne \\
ménageant ni lui ni les autres dès qu’il s’agit de faire rire, et  tenant des propos que ne tiendrait jamais l’homme de bon ton, qui ne voudrait même pas écouter certains d’entre eux. Quant au rustre, il est absolument impropre aux conversations de ce genre, car il n’y apporte aucune contribution, et critique tout, et pourtant la détente et l’amusement sont, de l’avis général, un élément essentiel de l’existence.\par
\\
Nous avons ainsi mentionné trois façons d’observer un juste milieu dans notre vie, et toutes ont rapport à un commerce réciproque de paroles et d’actions. Elles diffèrent cependant en ce que l’une de ces médiétés a rapport à la vérité, et les deux autres à l’agrément : de ces deux dernières, la première se manifeste dans les distractions, et la seconde dans les rapports sociaux intéressant une vie toute différente.
\subsection[{15 (1128b) < La modestie >}]{15 (1128b) < La modestie >}
\noindent En ce qui concerne la modestie [αιδως], il ne convient pas d’en \\
parler comme d’une vertu, car elle ressemble plutôt à une affection qu’à une disposition. Quoiqu’il en soit, on la définit comme une sorte de crainte de donner une mauvaise opinion de soi, et elle produit des effets analogues à ceux que provoque la crainte du danger : on rougit, en effet, quand on a honte, et on pâlit quand on craint pour sa vie. Dans un cas comme dans l’autre, il semble donc bien qu’il s’agisse là de quelque chose \\
de corporel en un sens, ce qui, on l’admet communément, est plutôt le fait d’une affection que d’une disposition.\par
L’affection en question ne convient pas à tout âge, mais seulement à la jeunesse. Nous pensons que les jeunes gens ont le devoir d’être modestes, parce que, vivant sous l’empire de la passion, ils commettent beaucoup d’erreurs, dont la modestie peut les préserver ; et nous louons les jeunes gens quand ils sont modestes, alors qu’on ne s’aviserait jamais de louer une \\
personne plus âgée de ce qu’elle est sensible à la honte, car nous pensons qu’elle a le devoir de ne rien faire de ce qui peut causer de la honte. Un homme vertueux, en effet, ne ressent jamais la honte, s’il est vrai qu’elle naisse à l’occasion des actions perverses (puisqu’on ne doit pas accomplir les actions de ce genre ; et même en admettant que certaines d’entre elles sont réellement honteuses et que les autres ne le sont qu’aux yeux de l’opinion, cette distinction n’importe ici en rien, car les unes comme les autres devant être évitées, nous n’avons pas \\
par suite à éprouver de honte à leur sujet) ; et la honte est le propre d’un homme pervers, et elle est due au fait qu’il est d’une nature capable d’accomplir quelque action honteuse. Et avoir le caractère constitué de telle sorte qu’on ressente de la honte si on a commis une action de ce genre, et penser qu’à cause de cela on est un homme de bien, c’est une absurdité : c’est, en effet, à l’occasion des actes volontaires que la modestie est ressentie, mais l’homme de bien ne commettra jamais volontairement les mauvaises actions. La modestie peut cependant \\
être un acte de vertu, dans l’hypothèse où un homme de bien ayant commis un acte vil, en éprouverait ensuite de la honte : mais cela ne peut pas se produire dans le domaine de la vertu. Et si l’impudence, autrement dit le fait de n’avoir pas honte d’accomplir les actions honteuses, est une chose vile, il n’en résulte pas pour autant que ressentir de la honte quand on accomplit de mauvaises actions soit un acte vertueux : pas davantage la tempérance n’est non plus une vertu, mais c’est \\
un mélange < de vertu et de vice >. Nous montrerons cela par la suite. Pour le moment, parlons de la justice.
\section[{Livre V}]{Livre V}\renewcommand{\leftmark}{Livre V}

\subsection[{1 (1129a) < Nature de la justice et de l’injustice >}]{1 (1129a) < Nature de la justice et de l’injustice >}
\noindent  Au sujet de la justice et de l’injustice, nous devons examiner sur quelles sortes d’actions elles portent en fait, \\
quelle sorte de médiété est la justice, et de quels extrêmes le juste est un moyen. Notre examen suivra la même marche que nos précédentes recherches.\par
Nous observons que tout le monde entend signifier par {\itshape justice} cette sorte de disposition qui rend les hommes aptes à accomplir les actions justes, et qui les fait agir justement et vouloir les choses justes ; de la même manière, l’{\itshape injustice} est \\
cette disposition qui fait les hommes agir injustement et vouloir les choses injustes. Posons donc, nous aussi, cette définition comme point de départ, à titre de simple esquisse. Il n’en est pas, en effet, pour les dispositions du caractère comme il en est pour les sciences et les potentialités : car il n’y a, semble-t-il, qu’une seule et même puissance, une seule et même science, pour les contraires, tandis qu’une disposition qui produit un certain effet ne peut pas produire aussi les effets contraires : \\
par exemple, en partant de la santé on ne produit pas les choses contraires à la santé, mais seulement les choses saines, car nous disons qu’un homme marche sainement quand il marche comme le ferait l’homme en bonne santé.\par
Souvent la disposition contraire est connue par son contraire, et souvent les dispositions sont connues au moyen \\
des sujets qui les possèdent : si, en effet, le bon état du corps nous apparaît clairement, le mauvais état nous devient également clair ; et nous connaissons le bon état aussi, au moyen des choses qui sont en bon état, et les choses qui sont en bon état, par le bon état. Supposons par exemple que le bon état en question soit une fermeté de chair : il faut nécessairement, d’une part, que le mauvais état soit une flaccidité de chair, et, d’autre part, que le facteur productif du bon état soit ce qui produit la fermeté dans la chair. Et il s’ensuit la plupart du \\
temps que si une paire de termes est prise en plusieurs sens, l’autre paire aussi sera prise en plusieurs sens : par exemple, si le terme juste est pris en plusieurs sens, injuste et injustice le seront aussi.
\subsection[{2 (1129a — 1129b) < Justice universelle et justice particulière >}]{2 (1129a — 1129b) < Justice universelle et justice particulière >}
\noindent Or, semble-t-il bien, la justice est prise en plusieurs sens, et l’injustice aussi, mais du fait que ces différentes significations sont voisines, leur homonymie échappe, et il n’en est pas comme pour les notions éloignées l’une de l’autre où l’homonymie est plus visible : par exemple (car la différence est considérable quand elle porte sur la forme extérieure), on appelle \\
{\itshape κλεις} [« clé »], en un sens homonyme, à la fois la clavicule des animaux et l’instrument qui sert à fermer les portes. — Comprenons donc en combien de sens se dit l’homme injuste. On considère généralement comme étant injuste à la fois celui qui viole la loi, celui qui prend plus que son dû, et enfin celui qui manque à l’égalité, de sorte que de toute évidence l’homme juste sera à la fois celui qui observe la loi et celui qui respecte l’égalité. Le juste, donc, est ce qui est conforme à la loi et ce qui respecte  l’égalité, et l’injuste ce qui est contraire à la loi et ce qui manque à l’égalité.\par
Et puisque l’homme injuste est celui qui prend au-delà de son dû, il sera injuste en ce qui a rapport aux biens, non pas tous les biens mais seulement ceux qui intéressent prospérité ou adversité, et qui, tout en étant toujours des biens au sens absolu, ne le sont pas toujours pour une personne déterminée. Ce sont cependant ces biens-là que les hommes demandent \\
dans leurs prières et poursuivent, quoi qu’ils ne dussent pas le faire, mais au contraire prier que les biens au sens absolu soient aussi des biens pour eux, et choisir les biens qui sont des biens pour eux. Mais l’homme injuste ne choisit pas toujours {\itshape plus}, il choisit aussi {\itshape moins} dans le cas des choses qui sont mauvaises au sens absolu ; néanmoins, du fait que le mal moins mauvais semble être en un certain sens un bien, et que l’avidité [πλεονεξια] a le bien pour objet, pour cette raison l’homme injuste semble être \\
un homme qui prend plus que son dû. Il manque aussi à l’égalité, car l’inégalité est une notion qui enveloppe les deux choses à la fois et leur est commune.
\subsection[{3 (1129b — 1130a) < La justice universelle ou légale >}]{3 (1129b — 1130a) < La justice universelle ou légale >}
\noindent Puisque, disions-nous, celui qui viole la loi est un homme injuste, et celui qui l’observe un homme juste, il est évident que toutes les actions prescrites par la loi sont, en un sens, justes : en effet, les actions définies par la loi positive sont légales, et chacune d’elles est juste, disons-nous. Or les lois prononcent \\
sur toutes sortes de choses, et elles ont en vue l’utilité commune, soit de tous les citoyens, soit des meilleurs, soit seulement des chefs désignés en raison de leur valeur ou de quelque autre critère analogue ; par conséquent, d’une certaine manière, nous appelons actions justes toutes celles qui tendent à produire ou à conserver le bonheur avec les éléments qui le composent, pour la communauté politique. — Mais la loi nous commande aussi d’accomplir les actes de l’homme \\
courageux (par exemple, ne pas abandonner son poste, ne pas prendre la fuite, ne pas jeter ses armes), ceux de l’homme tempérant (par exemple, ne pas commettre d’adultère, ne pas être insolent), et ceux de l’homme de caractère agréable (comme de ne pas porter des coups et de ne pas médire des autres), et ainsi de suite pour les autres formes de vertus ou de vices, prescrivant les unes et interdisant les autres, tout cela correctement si \\
la loi a été elle-même correctement établie, ou d’une façon critiquable, si elle a été faite à la hâte. Cette forme de justice, alors, est une vertu complète, non pas cependant au sens absolu, mais dans nos rapports avec autrui. Et c’est pourquoi souvent on considère la justice comme la plus parfaite des vertus, et ni {\itshape l’étoile du soir, ni l’étoile du matin} ne sont ainsi admirables. Nous avons encore l’expression proverbiale :\par
 {\itshape Dans la justice est en somme toute vertu.} \par
\\
Et elle est une vertu complète au plus haut point, parce qu’elle est usage de la vertu complète, et elle est complète parce que l’homme en possession de cette vertu est capable d’en user aussi à l’égard des autres et non seulement pour lui-même : si, en effet, beaucoup de gens sont capables de pratiquer la vertu dans leurs affaires personnelles, dans celles qui,  au contraire, intéressent les autres ils en demeurent incapables. Aussi doit-on approuver la parole de Bias, que {\itshape le commandement révélera l’homme}, car celui qui commande est en rapports avec d’autres hommes, et dès lors est membre d’une communauté. C’est encore pour cette même raison que la justice, seule de toutes les vertus, est considérée comme étant un {\itshape bien étranger}, parce qu’elle a rapport à autrui : elle accomplit \\
ce qui est avantageux à un autre, soit à un chef, soit à un membre de la communauté. Et ainsi l’homme le pire de tous est l’homme qui fait usage de sa méchanceté à la fois envers lui-même et envers ses amis ; et l’homme le plus parfait n’est pas l’homme qui exerce sa vertu seulement envers lui-même, mais celui qui la pratique aussi à l’égard d’autrui, car c’est là une œuvre difficile.\par
Cette forme de justice, alors, n’est pas une partie de la \\
vertu, mais la vertu tout entière, et son contraire, l’injustice, n’est pas non plus une partie du vice, mais le vice tout entier. (Quant à la différence existant entre la vertu et la justice ainsi comprise, elle résulte clairement de ce que nous avons dit : la justice est identique à la vertu, mais sa quiddité n’est pas la même : en tant que concernant nos rapports avec autrui, elle est justice, et en tant que telle sorte de disposition pure et simple, elle est vertu.)
\subsection[{4 (1130a — 1130b) < La justice spéciale ou particulière >}]{4 (1130a — 1130b) < La justice spéciale ou particulière >}
\noindent Mais ce que nous recherchons, de toute façon, c’est la justice qui est une partie de la vertu, puisqu’il existe une \\
justice de cette sorte, comme nous le disons ; et pareillement pour l’injustice, prise au sens d’injustice particulière. L’existence de cette forme d’injustice est prouvée comme suit. Quand un homme exerce son activité dans la sphère des autres vices, il commet certes une injustice tout en ne prenant en rien plus que sa part (par exemple, l’homme qui jette son bouclier par lâcheté, ou qui, poussé par son caractère difficile, prononce des paroles blessantes, ou qui encore refuse un secours en argent par lésinerie) ; quand, au contraire, il prend plus que sa \\
part, souvent son action ne s’inspire d’aucun de ces sortes de vices, encore moins de tous à la fois, et cependant il agit par une certaine perversité (puisque nous le blâmons) et par injustice. Il existe donc une autre sorte d’injustice comme une partie de l’injustice totale, et un injuste qui est une partie de l’injuste total, de cet injuste contraire à la loi. Autre preuve : si un homme commet un adultère en vue du gain, et en en retirant \\
un bénéfice, tandis qu’un autre agit ainsi par concupiscence, déboursant même de l’argent et y laissant des plumes, ce dernier semblerait être un homme déréglé plutôt qu’un homme prenant plus que son dû, tandis que le premier est injuste, mais non déréglé ; il est donc évident que ce qui rend ici l’action injuste, c’est qu’elle est faite en vue du gain. Autre preuve encore : tous les autres actes injustes sont invariablement rapportés à quelque forme de vice particulière, par exemple \\
l’adultère au dérèglement, l’abandon d’un camarade de combat à la lâcheté, la violence physique à la colère ; mais si, au contraire, l’action est dictée par l’amour du gain, on ne la rapporte à aucune forme particulière de perversité, mais seulement à l’injustice. — On voit ainsi que, en dehors de l’injustice au sens universel, il existe une autre forme d’injustice, qui est une partie de la première et qui porte le même nom, du fait  que sa définition tombe dans le même genre, l’une et l’autre étant caractérisées par ce fait qu’elles intéressent nos rapports avec autrui. Mais tandis que l’injustice au sens partiel a rapport à l’honneur ou à l’argent ou à la sécurité (ou quel que soit le nom dans lequel nous pourrions englober tous ces avantages), et qu’elle a pour motif le plaisir provenant du gain, l’injustice \\
prise dans sa totalité a rapport à toutes les choses sans exception qui rentrent dans la sphère d’action de l’homme vertueux.
\subsection[{5 (1130b — 1131a) < La justice totale et la justice particulière >}]{5 (1130b — 1131a) < La justice totale et la justice particulière >}
\noindent Qu’ainsi donc il existe plusieurs formes de justice, et qu’il y en ait une qui soit distincte et en dehors de la vertu totale, c’est là une chose évidente. Quelle est-elle et quelle est sa nature, c’est ce que nous devons comprendre.\par
Nous avons divisé l’injuste en le contraire à la loi et \\
l’inégal, et le juste en le conforme à la loi et l’égal. Au {\itshape contraire à la loi} correspond l’injustice au sens indiqué précédemment. Mais puisque l’inégal et le contraire à la loi ne sont pas identiques mais sont autres, comme une partie est autre que le tout (car tout inégal est contraire à la loi, tandis que tout contraire à la loi n’est pas inégal), l’injuste et l’injustice < au sens particulier > ne sont pas identiques < à l’injuste et à l’injustice au sens total >, mais sont autres qu’eux, et sont à leur égard comme les parties aux touts (car l’injustice sous cette \\
forme est une partie de l’injustice totale, et pareillement la justice, de la justice totale) : il en résulte que nous devons traiter à la fois de la justice particulière et de l’injustice particulière, ainsi que du juste et de l’injuste pris en ce même sens.\par
La justice au sens où elle est coextensive à la vertu totale, et l’injustice correspondante, qui sont respectivement l’usage de \\
la vertu totale ou du vice total à l’égard d’autrui, peuvent être laissées de côté. Quant à la façon dont le juste et l’injuste répondant à ces précédentes notions doivent être distingués à leur tour, c’est là une chose manifeste. (On peut dire, en effet, que la plupart des actes légaux sont ceux qui relèvent de la vertu prise dans sa totalité, puisque la loi nous prescrit une manière de vivre conforme aux diverses vertus particulières et nous interdit de nous livrer aux différents vices particuliers. Et \\
les facteurs susceptibles de produire la vertu totale sont ceux des actes que la loi a prescrits pour l’éducation de l’homme en société. Quant à l’éducation de l’individu comme tel, qui fait devenir simplement homme de bien, la question se pose de savoir si elle relève de la science politique ou d’une autre science, et c’est là un point que nous aurons à déterminer ultérieurement : car, sans doute, n’est-ce pas la même chose d’être un homme de bien et d’être un bon citoyen de quelque État.)\par
\\
De la justice particulière et du juste qui y correspond, une première espèce est celle qui intervient dans la distribution des honneurs, ou des richesses, ou des autres avantages qui se répartissent entre les membres de la communauté politique (car dans ces avantages il est possible que l’un des membres ait  une part ou inégale ou égale à celle d’un autre), et une seconde espèce est celle qui réalise la rectitude dans les transactions privées. Cette justice corrective comprend elle-même deux parties : les transactions privées, en effet, sont les unes volontaires et les autres involontaires : sont volontaires les actes tels qu’une vente, un achat, un prêt de consommation, une caution, un prêt à usage, un dépôt, une location (ces actes sont dits \\
volontaires parce que le fait qui est à l’origine de ces transactions est volontaire) ; des actes involontaires, à leur tour, les uns sont clandestins, tels que vol, adultère, empoisonnement, prostitution, corruption d’esclave, assassinat par ruse, faux témoignage ; les autres sont violents, tels que voies de fait, séquestration, meurtre, vol à main armée, mutilation, diffamation, outrage.
\subsection[{6 (1131a — 1131b) < La justice distributive, médiété proportionnelle >}]{6 (1131a — 1131b) < La justice distributive, médiété proportionnelle >}
\noindent Et puisque, à la fois, l’homme injuste est celui qui manque \\
à l’égalité et que l’injuste est inégal, il est clair qu’il existe aussi quelque moyen entre ces deux sortes d’inégal. Or ce moyen est l’égal, car en toute espèce d’action admettant le plus et le moins il y a aussi l’égal. Si donc l’injuste est inégal, le juste est égal, et c’est là, sans autre raisonnement, une opinion unanime. Et puisque l’égal est moyen, le juste sera un certain \\
moyen. Or l’égal suppose au moins deux termes. Il s’ensuit nécessairement, non seulement que le juste est à la fois moyen, égal, et aussi relatif, c’est-à-dire juste pour certaines personnes, mais aussi qu’en tant que moyen, il est entre certains extrêmes < qui sont le plus et le moins >, qu’en tant qu’égal, il suppose deux choses < qui sont égales >, et qu’en tant que juste, il suppose certaines personnes < pour lesquelles il est juste >. Le juste implique donc nécessairement au moins quatre termes : les personnes pour lesquelles il se trouve en fait juste, et qui sont deux, et les choses dans lesquelles il se manifeste, \\
au nombre de deux également. Et ce sera la même égalité pour les personnes et pour les choses : car le rapport qui existe entre ces dernières, à savoir les choses à partager, est aussi celui qui existe entre les personnes. Si, en effet, les personnes ne sont pas égales, elles n’auront pas des parts égales ; mais les contestations et les plaintes naissent quand, étant égales, les personnes possèdent ou se voient attribuer des parts non égales, ou quand, les personnes n’étant pas égales, leurs parts sont égales. On peut encore montrer cela en s’appuyant sur le fait qu’on tient compte de la valeur propre des personnes. \\
Tous les hommes reconnaissent, en effet, que la justice dans la distribution doit se baser sur un mérite de quelque sorte, bien que tous ne désignent pas le même mérite, les démocrates le faisant consister dans une condition libre, les partisans de l’oligarchie, soit dans la richesse, soit dans la noblesse de race, et les défenseurs de l’aristocratie dans la vertu.\par
Le juste est, par suite, une sorte de proportion (car la \\
proportion n’est pas seulement une propriété d’un nombre formé d’unités abstraites, mais de tout nombre en général), la proportion étant une égalité de rapports et supposant quatre termes au moins. — Que la proportion discontinue implique quatre termes, cela est évident, mais il en est de même aussi pour la proportion continue, puisqu’elle emploie un seul  terme comme s’il y en avait deux et qu’elle le mentionne deux fois : par exemple, ce que la ligne A est à la ligne B, la ligne B l’est à la ligne Γ ; la ligne B est donc mentionnée deux fois, de sorte que si l’on pose B deux fois, il y aura quatre termes proportionnels. — Et le juste, donc, implique quatre termes au moins, et le rapport < entre la première paire de termes > est le même < que celui qui existe entre la seconde paire >, car la division \\
s’effectue d’une manière semblable entre les personnes et les choses. Ce que le terme A, alors, est à B, le terme Γ le sera à Δ ; et de là, par interversion, ce que A est à Γ, B l’est à Δ ; et par suite aussi le rapport est le même pour le total à l’égard du total. Or c’est là précisément l’assemblage effectué par la distribution des parts, et si les termes sont joints de cette façon, l’assemblage est effectué conformément à la justice.
\subsection[{7 (1131b — 1132b) < La justice distributive, suite. La justice corrective >}]{7 (1131b — 1132b) < La justice distributive, suite. La justice corrective >}
\noindent Ainsi donc, l’assemblage du terme A avec le terme Γ, et \\
de B avec Δ, constitue le juste dans la distribution, et ce juste est un moyen entre deux extrêmes qui sont en dehors de la proportion, puisque la proportion est un moyen, et le juste une proportion. — Les mathématiciens désignent la proportion de ce genre du nom de {\itshape géométrique}, car la proportion géométrique est celle dans laquelle le total est au total dans le même rapport que chacun des deux termes au terme correspondant. \\
Mais la proportion de la justice distributive n’est pas une proportion continue, car il ne peut pas y avoir un terme numériquement un pour une personne et pour une chose. — Le juste en question est ainsi la proportion, et l’injuste ce qui est en dehors de la proportion. L’injuste peut donc être soit le trop, soit le trop peu, et c’est bien là ce qui se produit effectivement, puisque celui qui commet une injustice a plus que sa part du bien \\
distribué, et celui qui la subit moins que sa part. S’il s’agit du mal, c’est l’inverse : car le mal moindre comparé au mal plus grand fait figure de bien, puisque le mal moindre est préférable au mal plus grand ; or ce qui est préférable est un bien, et ce qui est préféré davantage, un plus grand bien.\par
\\
Voilà donc une première espèce du juste. Une autre, la seule restante, est le juste {\itshape correctif}, qui intervient dans les transactions privées, soit volontaires, soit involontaires. Cette forme du juste a un caractère spécifique différent de la précédente. En effet, le juste distributif des biens possédés en commun s’exerce toujours selon la proportion dont nous avons \\
parlé (puisque si la distribution s’effectue à partir de richesses communes, elle se fera suivant la même proportion qui a présidé aux apports respectifs des membres de la communauté ; et l’injuste opposé à cette forme du juste est ce qui est en dehors de ladite proportion). Au contraire, le juste dans les transactions privées, tout en étant une sorte d’égal, et l’injuste  une sorte d’inégal, n’est cependant pas l’égal selon la proportion de tout à l’heure, mais selon la proportion arithmétique. Peu importe, en effet, que ce soit un homme de bien qui ait dépouillé un malhonnête homme, ou un malhonnête homme un homme de bien, ou encore qu’un adultère ait été commis par un homme de bien ou par un malhonnête homme : la loi n’a \\
égard qu’au caractère distinctif du tort causé, et traite les parties à égalité, se demandant seulement si l’une a commis, et l’autre subi, une injustice, ou si l’une a été l’auteur et l’autre la victime d’un dommage. Par conséquent, cet injuste dont nous parlons, qui consiste dans une inégalité, le juge s’efforce de l’égaliser : en effet, quand l’un a reçu une blessure et que l’autre est l’auteur de la blessure, ou quand l’un a commis un meurtre et que l’autre a été tué, la passion et l’action ont été divisées en parties inégales ; mais le juge s’efforce, au moyen du châtiment, d’établir l’égalité, en enlevant le gain obtenu. \\
— On applique en effet indistinctement le terme {\itshape gain} aux cas de ce genre, même s’il n’est pas approprié à certaines situations, par exemple pour une personne qui a causé une blessure, et le terme {\itshape perte} n’est pas non plus dans ce cas bien approprié à la victime ; mais, de toute façon, quand le dommage souffert a été évalué, on peut parler de perte et de gain. — Par conséquent, \\
l’égal est moyen entre le plus et le moins, mais le gain et la perte sont respectivement plus et moins en des sens opposés, plus de bien et moins de mal étant du gain, et le contraire étant une perte ; et comme il y a entre ces extrêmes un moyen, lequel, avons-nous dit, est l’égal — égal que nous identifions au juste —, il s’ensuit que le juste rectificatif sera le moyen entre une perte \\
et un gain. C’est pourquoi aussi, en cas de contestation, on a recours au juge. Aller devant le juge c’est aller devant la justice, car le juge tend à être comme une justice vivante ; et on cherche dans un juge un moyen terme (dans certains pays on appelle les juges des {\itshape médiateurs}), dans la pensée qu’en obtenant ce qui est moyen on obtiendra ce qui est juste. Ainsi le juste est une sorte de moyen, s’il est vrai que le juge l’est aussi.\par
\\
Le juge restaure l’égalité. Il en est à cet égard comme d’une ligne divisée en deux segments inégaux : au segment le plus long le juge enlève cette partie qui excède la moitié de la ligne entière et l’ajoute au segment le plus court ; et quand le total a été divisé en deux moitiés, c’est alors que les plaideurs déclarent qu’ils ont ce qui est proprement leur bien, c’est-à-dire quand ils ont reçu l’égal. Et l’égal est moyen entre ce qui est \\
plus grand et ce qui est plus petit, selon la proportion arithmétique. C’est pour cette raison aussi que le moyen reçoit le nom de {\itshape juste} (δικαιον), parce qu’il est une division en {\itshape deux parts égales} (διχα), c’est comme si on disait διχαιον, et le {\itshape juge} (δικαστης) est un homme qui {\itshape partage en deux} (διχαστης). Quand, en effet, de deux choses égales on enlève une partie de l’une pour l’ajouter à l’autre, cette autre chose excède la première de deux fois ladite partie, puisque si ce qui a été  enlevé à l’une n’avait pas été ajouté à l’autre, cette seconde chose excéderait la première d’une fois seulement la partie en question ; cette seconde chose, donc, excède le moyen d’une fois ladite partie, et le moyen excède la première, qui a fait l’objet du prélèvement, d’une fois la partie. Ce processus nous permettra ainsi de connaître à la fois quelle portion il faut enlever de ce qui a plus, et quelle portion il faut ajouter à ce qui a moins : nous apporterons à ce qui a moins la quantité dont le \\
moyen le dépasse, et enlèverons à ce qui a le plus la quantité dont le moyen est dépassé. Soit les lignes AA’, BB’, ΓΓ’, égales entre elles ; de la ligne AA’ admettons qu’on enlève le segment AE, et qu’on ajoute à la ligne ΓΓ’ le segment ΓΔ, de telle sorte que la ligne entière ΔΓΓ’ dépasse la ligne EA’ des segments ΓΔ et ΓΖ ; c’est donc qu’elle dépasse BB’ de la longueur ΓΔ — \{Et cela s’applique aussi aux autres arts, car \\
ils seraient voués à la disparition si ce que l’élément actif produisait et en quantité et en qualité n’entraînait pas de la part de l’élément passif une prestation équivalente en quantité et qualité.\}\par
Les dénominations en question, à savoir la perte et le gain, sont venues de la notion d’échange volontaire. Dans ce domaine, en effet, avoir plus que la part qui vous revient en propre s’appelle {\itshape gagner}, et avoir moins que ce qu’on avait en \\
commençant, {\itshape perdre} : c’est ce qui se passe dans l’achat, la vente et toutes autres transactions laissées par la loi à la liberté des contractants. Quand, au contraire, la transaction n’entraîne pour eux ni enrichissement ni appauvrissement, mais qu’ils reçoivent exactement ce qu’ils ont donné, ils disent qu’ils ont ce qui leur revient en propre et qu’il n’y a ni perte, ni gain. Ainsi donc, le juste est moyen entre une sorte de gain et une sorte de perte dans les transactions non volontaires : il \\
consiste à posséder après, une quantité égale à ce qu’elle était auparavant.
\subsection[{8 (1132b — 1133b) < La justice et la réciprocité. Rôle économique de la monnaie >}]{8 (1132b — 1133b) < La justice et la réciprocité. Rôle économique de la monnaie >}
\noindent Dans l’opinion de certains, c’est la réciprocité qui constitue purement et simplement la justice : telle était la doctrine des Pythagoriciens, qui définissaient le juste simplement comme la réciprocité. Mais la réciprocité ne coïncide ni avec la justice distributive, ni même avec la justice corrective (bien \\
qu’on veuille d’ordinaire donner ce sens à la justice de Rhadamante :\par
{\itshape Subir ce qu’on a fait aux autres sera une justice équitable}, \par
car souvent réciprocité et justice corrective sont en désaccord : par exemple, si un homme investi d’une magistrature a frappé un particulier, il ne doit pas être frappé à son tour, et si un particulier a frappé un magistrat, il ne doit pas seulement être \\
frappé mais recevoir une punition supplémentaire. En outre, entre l’acte volontaire et l’acte involontaire, il y a une grande différence. Mais dans les relations d’échanges, le juste sous sa forme de réciprocité est ce qui assure la cohésion des hommes entre eux, réciprocité toutefois basée sur une proportion et non sur une stricte égalité. C’est cette réciprocité-là qui fait subsister la cité : car les hommes cherchent soit à répondre au mal  par le mal, faute de quoi ils se considèrent en état d’esclavage, soit à répondre au bien par le bien, sans quoi aucun échange n’a lieu, alors que c’est pourtant l’échange qui fait la cohésion des citoyens. Et c’est pourquoi un temple des Charites se dresse sur la place publique : on veut rappeler l’idée de reconnaissance, qui est effectivement un caractère propre de la grâce, puisque c’est un devoir non seulement de rendre service pour service à celui qui s’est montré aimable envers nous, mais \\
encore à notre tour de prendre l’initiative d’être aimable.\par
Or la réciprocité, j’entends celle qui est proportionnelle, est réalisée par l’assemblage en diagonale. Soit par exemple A un architecte, B un cordonnier, C une maison et D une chaussure : il faut faire en sorte que l’architecte reçoive du cordonnier le produit du travail de ce dernier, et lui donne en \\
contrepartie son propre travail. Si donc tout d’abord on a établi l’égalité proportionnelle des produits et qu’ensuite seulement l’échange réciproque ait lieu, la solution sera obtenue ; et faute d’agir ainsi, le marché n’est pas égal et ne tient pas, puisque rien n’empêche que le travail de l’un n’ait une valeur supérieure à celui de l’autre, et c’est là ce qui rend une péréquation préalable indispensable. — Il en est de même aussi dans \\
le cas des autres arts, car ils disparaîtraient si ce que l’élément actif produisait à la fois en quantité et qualité n’entraînait pas de la part de l’élément passif une prestation équivalente en quantité et en qualité. — En effet, ce n’est pas entre deux médecins que naît une communauté d’intérêts, mais entre un médecin par exemple et un cultivateur, et d’une manière générale entre des contractants différents et inégaux qu’il faut pourtant égaliser. C’est pourquoi toutes les choses faisant objet de transaction doivent être d’une façon quelconque commensurables entre elles. C’est à cette fin que la monnaie a été introduite, \\
devenant une sorte de moyen terme, car elle mesure toutes choses et par suite l’excès et le défaut, par exemple combien de chaussures équivalent à une maison ou à telle quantité de nourriture. Il doit donc y avoir entre un architecte et un cordonnier le même rapport qu’entre un nombre déterminé de chaussures et une maison (ou telle quantité de nourriture), faute de quoi il n’y aura ni échange ni communauté d’intérêts ; et ce rapport ne pourra être établi que si entre les biens à \\
échanger il existe une certaine égalité. Il est donc indispensable que tous les biens soient mesurés au moyen d’un unique étalon, comme nous l’avons dit plus haut. Et cet étalon n’est autre, en réalité, que le besoin, qui est le lien universel (car si les hommes n’avaient besoin de rien, ou si leurs besoins n’étaient pas pareils, il n’y aurait plus d’échange du tout, ou les échanges seraient différents) ; mais la monnaie est devenue une sorte de substitut du besoin et cela par convention, et c’est \\
d’ailleurs pour cette raison que la monnaie reçoit le nom de νοµισµα, parce qu’elle existe non pas par nature, mais en vertu de la loi (νοµος), et qu’il est en notre pouvoir de la changer et de la rendre inutilisable.\par
Il y aura dès lors réciprocité, quand les marchandises ont été égalisées de telle sorte que le rapport entre cultivateur et cordonnier soit le même qu’entre l’œuvre du cordonnier et celle du cultivateur. Mais on ne doit pas les faire entrer dans la  forme d’une proportion une fois qu’ils ont effectué l’échange (autrement, l’un des deux extrêmes aurait les deux excédents à la fois), mais quand ils sont encore en possession de leur propre marchandise. C’est seulement de cette dernière façon qu’ils sont en état d’égalité et en communauté d’intérêts, car alors l’égalité en question peut se réaliser pour eux — (Appelons un \\
cultivateur A, une certaine quantité de nourriture Γ, un cordonnier B, et le travail de ce dernier égalisé, Δ) — ; si au contraire il n’avait pas été possible pour la réciprocité d’être établie de la façon que nous venons de dire, il n’y aurait pas communauté d’intérêts.\par
Que ce soit le besoin qui, jouant le rôle d’étalon unique, constitue le lien de cette communauté d’intérêts, c’est là une chose qui résulte clairement de ce fait que, en l’absence de tout besoin réciproque, soit de la part des deux contractants, soit seulement de l’un d’eux, aucun échange n’a lieu, comme c’est le cas si quelqu’un a besoin d’une marchandise qu’on possède soi-même, du vin par exemple, alors que les facilités d’exportation \\
n’existent que pour le blé. — Ainsi donc il convient de réaliser la péréquation.\par
Mais pour les échanges éventuels, dans l’hypothèse où nous n’avons besoin de rien pour le moment, la monnaie est pour nous une sorte de gage, donnant l’assurance que l’échange sera possible si jamais le besoin s’en fait sentir, car on doit pouvoir, en remettant l’argent, obtenir ce dont on manque. La monnaie, il est vrai, est soumise aux mêmes fluctuations que les autres marchandises (car elle n’a pas toujours un égal pouvoir d’achat) ; elle tend toutefois à une plus grande stabilité. De là vient que toutes les marchandises doivent \\
être préalablement estimées en argent, car de cette façon il y aura toujours possibilité d’échange, et par suite communauté d’intérêts entre les hommes. La monnaie, dès lors, jouant le rôle de mesure, rend les choses commensurables entre elles et les amène ainsi à l’égalité : car il ne saurait y avoir ni communauté d’intérêts sans échange, ni échange sans égalité, ni enfin égalité sans commensurabilité. Si donc, en toute rigueur, il n’est pas possible de rendre les choses par trop différentes commensurables entre elles, du moins, pour nos besoins courants, peut-on y parvenir d’une façon suffisante. Il doit \\
donc y avoir quelque unité de mesure, fixée par convention, et qu’on appelle pour cette raison νοµισµα, car c’est cet étalon qui rend toutes choses commensurables, puisque tout se mesure en monnaie. Appelons par exemple une maison A, dix mines B, un lit Γ. Alors A est moitié de B si la maison vaut cinq \\
mines, autrement dit est égale à cinq mines ; et le lit Γ est la dixième partie de B : on voit tout de suite combien de lits équivalent à une maison, à savoir cinq. Qu’ainsi l’échange ait existé avant la création de la monnaie, cela est une chose manifeste, puisqu’il n’y a aucune différence entre échanger cinq lits contre une maison ou payer la valeur en monnaie des cinq lits.
\subsection[{9 (1133b — 1134a) < La justice-médiété >}]{9 (1133b — 1134a) < La justice-médiété >}
\noindent Nous avons ainsi déterminé la nature du juste et celle de \\
l’injuste. Des distinctions que nous avons établies il résulte clairement que l’action juste est un moyen entre l’injustice commise et l’injustice subie, l’une consistant à avoir trop, 1\\
et l’autre trop peu. La justice est à son tour une sorte de médiété, non pas de la même façon que les autres vertus, mais en ce sens  qu’elle relève du juste milieu, tandis que l’injustice relève des extrêmes. Et la justice est une disposition d’après laquelle l’homme juste se définit celui qui est apte à accomplir, par choix délibéré, ce qui est juste, celui qui, dans une répartition à effectuer soit entre lui-même et un autre, soit entre deux autres personnes, n’est pas homme à s’attribuer à lui-même, dans le bien désiré, une part trop forte, et à son voisin une part \\
trop faible (ou l’inverse, s’il s’agit d’un dommage à partager), mais donne à chacun la part proportionnellement égale qui lui revient, et qui agit de la même façon quand la répartition se fait entre des tiers. L’injustice, en sens opposé, a pareillement rapport à ce qui est injuste, et qui consiste dans un excès ou un défaut disproportionné de ce qui est avantageux ou dommageable. C’est pourquoi l’injustice est un excès et un défaut en ce sens qu’elle est génératrice d’excès et de défaut : quand on \\
est soi-même partie à la distribution, elle aboutit à un excès de ce qui est avantageux en soi et à un défaut de ce qui est dommageable ; s’agit-il d’une distribution entre des tiers, le résultat dans son ensemble est bien le même que dans le cas précédent, mais la proportion peut être dépassée indifféremment dans un sens ou dans l’autre. Et l’acte injuste a deux faces : du côté du trop peu, il y a injustice subie, et du côté du trop, injustice commise.
\subsection[{10 (1134a — 1136a) < Justice sociale — Justice naturelle et justice positive >}]{10 (1134a — 1136a) < Justice sociale — Justice naturelle et justice positive >}
\noindent Sur la justice et l’injustice, et sur la nature de chacune \\
d’elles, voilà tout ce que nous avions à dire, aussi bien d’ailleurs que sur le juste et l’injuste en général.\par
Mais étant donné qu’on peut commettre une injustice sans être pour autant injuste, quelles sortes d’actes d’injustice doit-on dès lors accomplir pour être injuste dans chaque forme d’injustice, par exemple pour être un voleur, un adultère ou un brigand ? Ne dirons-nous pas que la différence ne tient en rien à la nature de l’acte ? Un homme, en effet, pourrait avoir \\
commerce avec une femme, sachant qui elle était, mais le principe de son acte peut être, non pas un choix délibéré, mais la passion. Il commet bien une injustice, mais il n’est pas un homme injuste ; de même on n’est pas non plus un voleur, même si on a volé, ni un adultère, même si on a commis l’adultère ; et ainsi de suite.\par
La relation de la réciprocité et de la justice a été étudiée précédemment.\par
\\
Mais nous ne devons pas oublier que l’objet de notre investigation est non seulement le juste au sens absolu, mais encore le juste politique. Cette forme du juste est celle qui doit régner entre des gens associés en vue d’une existence qui se suffise à elle-même, associés supposés libres et égaux en droits, d’une égalité soit proportionnelle, soit arithmétique, de telle sorte que, pour ceux ne remplissant pas cette condition, il n’y a pas dans leurs relations réciproques, justice politique proprement dite, mais seulement une sorte de justice prise en \\
un sens métaphorique. Le juste, en effet, n’existe qu’entre ceux dont les relations mutuelles sont sanctionnées par la loi, et il n’y a de loi que pour des hommes chez lesquels l’injustice peut se rencontrer, puisque la justice légale est une discrimination du juste et de l’injuste. Chez les hommes, donc, où l’injustice peut exister, des actions injustes peuvent aussi se commettre chez eux (bien que là où il y a action injuste il n’y ait pas toujours injustice), actions qui consistent à s’attribuer à soi-même une part trop forte des choses en elles-mêmes bonnes, et une part trop faible des choses en elles-mêmes mauvaises. C’est la raison pour laquelle nous ne laissons pas un homme nous gouverner, nous voulons que ce soit la loi, parce qu’un  homme ne le fait que dans son intérêt propre et devient un tyran ; mais le rôle de celui qui exerce l’autorité est de garder la justice, et gardant la justice, de garder aussi l’égalité. Et puisqu’il est entendu qu’il n’a rien de plus que sa part s’il est juste (car il ne s’attribue pas à lui-même une part trop forte des choses en elles-mêmes bonnes, à moins qu’une telle part ne \\
soit proportionnée à son mérite ; aussi est-ce pour autrui qu’il travaille, et c’est ce qui explique la maxime {\itshape la justice est un bien étranger}, comme nous l’avons dit précédemment), on doit donc lui allouer un salaire sous forme d’honneurs et de prérogatives. Quant à ceux pour qui de tels avantages sont insuffisants, ceux-là deviennent des tyrans.\par
La justice du maître ou celle du père n’est pas la même que la justice entre citoyens, elle lui ressemble seulement. En effet, il n’existe pas d’injustice au sens absolu du mot, à l’égard de ce \\
qui nous appartient en propre ; mais ce qu’on possède en pleine propriété, aussi bien que l’enfant, jusqu’à ce qu’il ait atteint un certain âge et soit devenu indépendant, sont pour ainsi dire une partie de nous-mêmes, et nul ne choisit délibérément de se causer à soi-même du tort, ni par suite de se montrer injuste envers soi-même : il n’est donc pas non plus question ici de justice ou d’injustice au sens politique, lesquelles, avons-nous dit, dépendent de la loi et n’existent que pour ceux qui vivent naturellement sous l’empire de la loi, à savoir, comme nous l’avons dit encore, ceux à qui appartient une part égale dans le \\
droit de gouverner et d’être gouverné. De là vient que la justice qui concerne l’épouse se rapproche davantage de la justice proprement dite que celle qui a rapport à l’enfant et aux propriétés, car il s’agit là de la justice domestique, mais même celle-là est différente de la forme politique de la justice.\par
La justice politique elle-même est de deux espèces, l’une naturelle et l’autre légale. Est naturelle celle qui a partout la \\
même force et ne dépend pas de telle ou telle opinion ; légale, celle qui à l’origine peut être indifféremment ceci ou cela, mais qui une fois établie, s’impose : par exemple, que la rançon d’un prisonnier est d’une mine, ou qu’on sacrifie une chèvre et non deux moutons, et en outre toutes les dispositions législatives portant sur des cas particuliers, comme par exemple le sacrifice en l’honneur de Brasidas et les prescriptions prises sous forme de décrets.\par
Certains sont d’avis que toutes les prescriptions juridiques \\
appartiennent à cette dernière catégorie, parce que, disent-ils, ce qui est naturel est immuable et a partout la même force (comme c’est le cas pour le feu, qui brûle également ici et en Perse), tandis que le droit est visiblement sujet à variations. Mais dire que le droit est essentiellement variable n’est pas exact d’une façon absolue, mais seulement en un sens déterminé. Certes, chez les dieux, pareille assertion n’est peut-être pas vraie du tout ; dans notre monde, du moins, bien qu’il existe aussi une certaine justice naturelle, tout dans ce domaine \\
est cependant passible de changement ; néanmoins on peut distinguer ce qui est naturel et ce qui n’est pas naturel. Et parmi les choses qui ont la possibilité d’être autrement qu’elles ne sont, il est facile de voir quelles sortes de choses sont naturelles et quelles sont celles qui ne le sont pas mais reposent sur la loi et la convention, tout en étant les unes et les autres pareillement sujettes au changement. Et dans les autres domaines, la même distinction s’appliquera : par exemple, bien que par nature la main droite soit supérieure à la gauche, il est cependant \\
toujours possible de se rendre ambidextre. Et parmi les règles de droit, celles qui dépendent de la convention et de l’utilité  sont semblables aux unités de mesure : en effet, les mesures de capacité pour le vin et le blé ne sont pas partout égales, mais sont plus grandes là où on achète, et plus petites là où l’on vend. Pareillement les règles de droit qui ne sont pas fondées sur la nature, mais sur la volonté de l’homme, ne sont pas partout les mêmes, puisque la forme du gouvernement elle-même ne \\
l’est pas, alors que cependant il n’y a qu’une seule forme de gouvernement qui soit partout naturellement la meilleure.\par
Les différentes prescriptions juridiques et légales sont, à l’égard des actions qu’elles déterminent, dans le même rapport que l’universel aux cas particuliers : en effet, les actions accomplies sont multiples, et chacune de ces prescriptions est une, étant universelle.\par
Il existe une différence entre l’action injuste et ce qui est injuste, et entre l’action juste et ce qui est juste : car une chose \\
est injuste par nature ou par une prescription de la loi, et cette même chose, une fois faite, est une action injuste, tandis qu’avant d’être faite, elle n’est pas encore une action injuste, elle est seulement quelque chose d’injuste. Il en est de même aussi d’une action juste (bien que le terme général soit plutôt δικαιοπραγηµα, le terme δικαιωµα étant réservé au redressement de l’action injuste). Quant aux différentes prescriptions juridiques et légales, ainsi que la nature et le nombre de leurs espèces et les sortes de choses sur lesquelles elles portent en \\
fait, tout cela devra être examiné ultérieurement.\par
Les actions justes et injustes ayant été ainsi décrites, on agit justement ou injustement quand on les commet volontairement. Mais quand c’est involontairement, l’action n’est ni juste ni injuste sinon par accident, car on accomplit alors des actes dont la qualité de justes ou d’injustes est purement \\
accidentelle. La justice (ou l’injustice) d’une action est donc déterminée par son caractère volontaire ou involontaire : est-elle volontaire, elle est objet de blâme, et elle est alors aussi en même temps un acte injuste ; par conséquent, il est possible pour une chose d’être injuste, tout en n’étant pas encore un acte injuste si la qualification de volontaire ne vient pas s’y ajouter. J’entends par volontaire, comme il a été dit précédemment, tout ce qui, parmi les choses qui sont au pouvoir de l’agent, est accompli en connaissance de cause, c’est-à-dire sans ignorer ni \\
la personne subissant l’action, ni l’instrument employé, ni le but à atteindre (par exemple, l’agent doit connaître qui il frappe, avec quelle arme et en vue de quelle fin), chacune de ces déterminations excluant au surplus toute idée d’accident ou de contrainte (si, par exemple, prenant la main d’une personne on s’en sert pour en frapper une autre, la personne à qui la main appartient n’agit pas volontairement, puisque l’action ne dépendait pas d’elle). Il peut se faire encore que la personne frappée soit par exemple le père de l’agent et que celui-ci, tout en sachant qu’il a affaire à un homme ou à l’une des personnes \\
présentes, ignore que c’est son père ; et une distinction de ce genre peut également être faite en ce qui concerne la fin à atteindre, et pour toutes les modalités de l’action en général. Dès lors, l’acte fait dans l’ignorance, ou même fait en connaissance de cause mais ne dépendant pas de nous ou résultant d’une contrainte, un tel acte est involontaire (il y a, en effet, beaucoup de processus naturels que nous accomplissons ou  subissons sciemment, dont aucun n’est ni volontaire, ni involontaire, comme par exemple vieillir ou mourir). Mais dans les actes justes ou injustes, la justice ou l’injustice peuvent pareillement être quelque chose d’accidentel : si un homme restitue un dépôt malgré lui et par crainte, on ne doit pas dire \\
qu’il fait une action juste, ni qu’il agit justement, sinon par accident. De même encore celui qui, sous la contrainte et contre sa volonté, ne restitue pas le dépôt confié, on doit dire de lui que c’est par accident qu’il agit injustement et accomplit une action injuste.\par
Les actes volontaires se divisent en actes qui sont faits par choix réfléchi et en actes qui ne sont pas faits par choix : sont \\
faits par choix ceux qui sont accomplis après délibération préalable, et ne sont pas faits par choix ceux qui sont accomplis sans être précédés d’une délibération. Il y a dès lors trois sortes d’actes dommageables dans nos rapports avec autrui : les torts qui s’accompagnent d’ignorance sont des {\itshape fautes}, quand la victime, ou l’acte, ou l’instrument, ou la fin à atteindre sont autres que ce que l’agent supposait (il ne pensait pas frapper, ou pas avec telle arme, ou pas telle personne, ou pas en vue de telle fin, mais l’événement a tourné dans un sens auquel il ne \\
s’attendait pas : par exemple, ce n’était pas dans l’intention de blesser, mais seulement de piquer, ou encore ce n’était pas la personne ou ce n’était pas l’instrument qu’il croyait). Quand alors le dommage a eu lieu contrairement à toute attente raisonnable, c’est une {\itshape méprise}, et quand on devait le prévoir raisonnablement, mais qu’on a agi sans méchanceté, c’est une simple {\itshape faute} (on commet une simple faute quand le principe de l’ignorance réside en nous-mêmes, et une méprise quand la cause \\
vient du dehors). Quand l’acte est fait en pleine connaissance, mais sans délibération préalable, c’est un {\itshape acte injuste}, par exemple tout ce qu’on fait par colère, ou par quelque autre de ces passions qui sont irrésistibles ou qui sont la conséquence de l’humaine nature (car en commettant ces torts et ces fautes les hommes agissent injustement, et leurs actions sont des actions injustes, bien qu’ils ne soient pas pour autant des êtres injustes ni pervers, le tort n’étant pas causé par méchanceté). Mais \\
quand l’acte procède d’un choix délibéré, c’est alors que l’agent est un homme injuste et méchant. — De là vient que les actes accomplis par colère sont jugés à bon droit comme faits sans préméditation, car ce n’est pas celui qui agit par colère qui est le véritable auteur du dommage, mais bien celui qui a provoqué sa colère. En outre, le débat ne porte pas sur la question de savoir s’il s’est produit ou non un fait dommageable, mais s’il a été causé justement (puisque c’est l’image d’une injustice qui déclenche la colère) : le fait lui-même n’est pas mis en \\
discussion (comme c’est le cas quand il s’agit des contrats, où l’une des deux parties est forcément malhonnête, à moins que son acte ne soit dû à un oubli), mais tout en étant d’accord sur la chose, les intéressés discutent le point de savoir lequel des deux a la justice de son côté (tandis que celui qui a fait délibérément du tort n’ignore pas ce point), de telle sorte que l’un croit qu’il est victime d’une injustice et que l’autre le conteste.\par
 Si, au contraire, c’est par mûre délibération qu’un homme a causé un tort, il agit injustement, et dès lors les actes injustes qu’il commet impliquent que celui qui agit ainsi est un homme injuste quand son acte viole la proportion ou l’égalité. Pareillement, un homme est juste quand, par choix réfléchi, il accomplit un acte juste, mais il accomplit un acte juste si seulement il le fait volontairement.\par
\\
Des actions involontaires, enfin, les unes sont pardonnables, et les autres ne sont pas pardonnables. En effet, les fautes non seulement faites dans l’ignorance, mais qui encore sont dues à l’ignorance, sont pardonnables, tandis que celles qui ne sont pas dues à l’ignorance, mais qui, tout en étant faites dans l’ignorance, ont pour cause une passion qui n’est ni naturelle ni humaine, ne sont pas pardonnables.
\subsection[{11 (1136a — 1136b) < Examen de diverses apories relatives à la justice >}]{11 (1136a — 1136b) < Examen de diverses apories relatives à la justice >}
\noindent \\
On pourrait se poser la question de savoir si nos déterminations de l’injustice subie et de l’injustice commise sont suffisantes, et, en premier lieu, si les choses se comportent comme le dit Euripide dans cette étrange parole :\par
 {\itshape — J’ai tué ma mère : tel est mon bref propos.} \par
{\itshape — Est-ce de votre consentement et du sien ? Ou bien n’y avez-vous consenti ni l’un ni l’autre} ? \par
\\
Est-ce qu’il est, en effet, véritablement possible de subir volontairement l’injustice, ou au contraire n’est-ce pas là quelque chose de toujours involontaire, de même que commettre l’injustice est toujours volontaire ? En outre, est-ce que subir l’injustice est toujours volontaire ou toujours involontaire, ou bien dans certains cas volontaire et dans certains autres, involontaire ? Même question en ce qui concerne le fait d’être traité avec justice : agir justement est toujours volontaire, de sorte qu’il est raisonnable de supposer semblable \\
opposition dans les deux cas, entre être traité injustement et être traité justement, d’une part, et être volontaire ou involontaire, d’autre part. Pourtant il pourrait sembler étrange de soutenir que même le fait d’être traité justement est toujours volontaire, car on est parfois traité justement contre sa volonté.\par
Ensuite, on pourrait aussi se poser la question suivante : l’homme qui a subi ce qui est injuste est-il toujours traité injustement, ou bien en est-il du fait de supporter l’injustice comme il en est du fait de la commettre ? En effet, comme agent \\
aussi bien que comme patient, on peut participer par accident à une action juste, et il en est évidemment de même pour les actions injustes : accomplir ce qui est injuste n’est pas la même chose qu’agir injustement, et subir ce qui est injuste n’est pas non plus la même chose qu’être traité injustement, et il en est de même du fait d’agir justement et d’être traité justement, car il est impossible d’être traité injustement si un autre \\
n’agit pas injustement, ou d’être traité justement si un autre n’agit pas justement.\par
Mais si agir injustement consiste purement et simplement à causer du tort volontairement à quelqu’un, et si {\itshape volontairement} a le sens de {\itshape avoir pleine connaissance et de la personne lésée, et de l’instrument, et de la manière}, et si l’homme intempérant se fait volontairement du tort à lui-même, il s’ensuivra à la fois que volontairement il sera injustement traité et qu’il lui sera possible d’agir envers lui-même injustement (c’est là d’ailleurs aussi une des questions que nous avons à nous poser : 1l36b peut-on agir injustement envers soi-même ?). De plus, on pourrait volontairement, par son intempérance, subir un dommage de la part d’une autre personne agissant volontairement, de sorte qu’on pourrait être volontairement traité injustement. Mais notre définition n’est-elle pas incorrecte, et ne doit-on pas ajouter à {\itshape causer du tort en ayant pleine connaissance et de la personne lésée, et de l’instrument, et de la manière}, la \\
précision suivante : {\itshape contrairement au souhait réfléchi de ladite personne} ? Ceci une fois admis, un homme peut assurément subir volontairement un dommage et supporter ce qui est injuste, mais il ne peut jamais consentir à être traité injustement, car personne ne souhaite cela, pas même l’homme intempérant, mais il agit contrairement à son propre souhait, puisque personne ne veut ce qu’il ne croit pas bon pour lui, et l’homme intempérant fait des choses qu’il pense lui-même n’être pas celles qu’il doit faire. D’ailleurs, celui qui donne ce qui lui appartient en propre, comme, selon Homère, Glaucus donnait à Diomède :\par
\\
{\itshape Des armes d’or pour des armes de bronze, la valeur de cent bœufs pour neuf bœufs}, \par
celui-là n’est pas injustement traité : car, bien que donner dépende de lui, être injustement traité n’est pas en son pouvoir, mais il faut qu’il y ait une autre personne qui le traite injustement.
\subsection[{12 (1136b — 1137a) < Autres apories relatives à la justice >}]{12 (1136b — 1137a) < Autres apories relatives à la justice >}
\noindent On voit donc qu’il n’est pas possible de subir volontairement l’injustice.\par
\\
Des questions que nous nous étions proposé de discuter, il en reste encore deux à examiner : est-ce, en fin de compte, celui qui a assigné à une personne la part excédant son mérite qui commet une injustice, ou bien est-ce celui qui reçoit ladite part ? Et peut-il se faire qu’on agisse envers soi-même injustement ?\par
Si on reconnaît la possibilité de la première solution, c’est-à-dire si c’est le distributeur de parts qui commet l’injustice, et non celui qui reçoit la part trop forte, alors, quand un homme, sciemment et volontairement, assigne à un autre une part plus grande qu’à lui-même, cet homme commet personnellement \\
un acte injuste envers lui-même, ce que font précisément, semble-t-il, les gens honnêtes, puisque l’homme équitable est enclin à prendre moins que son dû. Mais cette explication n’est-elle pas non plus, dans sa simplicité, inexacte ? Il peut arriver en effet, que l’homme en question possédait plus que sa part d’un autre bien, plus que sa part d’honneur, par exemple, ou de vertu proprement dite. Il y a encore une solution : c’est d’appliquer notre définition de l’action injuste. L’homme dont nous parlons, en effet, ne subit rien de contraire à sa propre volonté : par conséquent, il ne subit pas d’injustice, du fait tout au moins qu’il s’est attribué la plus petite part ; mais, le cas échéant, il supporte seulement un dommage.\par
\\
Cependant il n’est pas douteux que c’est bien le distributeur de parts qui commet l’injustice, tandis que celui qui reçoit la part excessive ne commet pas l’injustice. En effet, ce n’est pas celui dans lequel réside ce qui est injuste qui agit injustement, mais celui qui commet volontairement l’acte injuste, c’est-à-dire celui d’où l’action tire son origine, origine qui se trouve dans celui qui distribue et non dans celui qui reçoit. De plus, étant donné que le terme faire comporte de \\
nombreuses acceptions et qu’en un sens on peut qualifier de meurtriers les objets inanimés, ou la main, ou le serviteur agissant par ordre, celui qui reçoit une part excessive n’agit pas injustement, quoiqu’il fasse là ce qui est injuste.\par
En outre, si le distributeur de parts a décidé dans l’ignorance, il n’agit pas injustement au sens où on parle de justice légale, et sa décision n’est pas non plus injuste en ce sens-là, mais elle est cependant en un certain sens injuste (puisque la justice légale est autre que la justice première) ; mais si tout en le sachant il a jugé d’une manière injuste,  il prend lui-même une part excessive soit de gratitude, soit de vengeance. Ainsi, tout comme s’il recevait une part du produit de l’injustice, le juge qui, pour les raisons ci-dessus, rend une décision injuste, obtient plus que son dû ; car, même dans l’hypothèse d’une participation au butin, si par exemple il attribue dans son jugement un fonds de terre, ce n’est pas de la terre mais de l’argent qu’il reçoit.
\subsection[{13 (1137a) < La justice est une disposition >}]{13 (1137a) < La justice est une disposition >}
\noindent \\
Les hommes s’imaginent qu’il est en leur pouvoir d’agir injustement, et que par suite il est facile d’être juste. Mais cela n’est pas exact. Avoir commerce avec la femme de son voisin, frapper son prochain, glisser de l’argent dans la main, c’est là assurément chose facile et en notre pouvoir, mais faire tout cela en vertu de telle disposition déterminée du caractère, n’est ni facile, ni en notre dépendance.\par
\\
Pareillement, on croit que la connaissance du juste et de l’injuste ne requiert pas une profonde sagesse, sous prétexte qu’il n’est pas difficile de saisir le sens des diverses prescriptions de la loi (quoique, en réalité, les actions prescrites par la loi ne soient justes que par accident). Mais savoir de quelle façon doit être accomplie une action, de quelle façon doit être effectuée une distribution pour être l’une et l’autre justes, c’est là une étude qui demande plus de travail que de connaître les remèdes qui procurent la santé. Et même dans ce dernier domaine, s’il est facile de savoir ce que c’est que du miel, du \\
vin, de l’ellébore, un cautère, un coup de lancette, par contre savoir comment, à qui et à quel moment on doit les administrer pour produire la santé, c’est une affaire aussi importante que d’être médecin.\par
Et pour la même raison, les hommes pensent aussi que l’homme juste est non moins apte que l’homme injuste à commettre l’injustice, parce que l’homme juste n’est en rien moins capable, s’il ne l’est davantage, d’accomplir, le cas échéant, quelqu’une des actions injustes dont nous avons parlé : n’est-il \\
pas capable, en effet, d’avoir commerce avec une femme, ou de frapper quelqu’un ? Et l’homme courageux est capable aussi de jeter son bouclier, de faire demi-tour et de s’enfuir dans n’importe quelle direction. Mais, en réalité, se montrer lâche ou injuste ne consiste pas à accomplir lesdites actions, sinon par accident, mais à les accomplir en raison d’une certaine disposition, tout comme exercer la médecine et l’art de guérir \\
ne consiste pas à faire emploi ou à ne pas faire emploi du scalpel ou de drogues, mais à le faire d’une certaine façon.\par
Les actions justes n’existent qu’entre les êtres qui ont part aux choses bonnes en elles-mêmes et admettant en elles excès et défaut. Il y a, en effet, des êtres pour lesquels un excès de bien ne se conçoit pas (c’est le cas sans doute des dieux) ; d’autres, au contraire, sont incapables de tirer profit d’aucune portion de ces biens, ce sont ceux qui sont irrémédiablement vicieux et à qui tout est nuisible ; d’autres, enfin, n’en tirent avantage que jusqu’à un certain point. Et c’est la raison pour \\
laquelle la justice est quelque chose de purement humain.
\subsection[{14 (1137a — 1138a) < L’équité et l’équitable >}]{14 (1137a — 1138a) < L’équité et l’équitable >}
\noindent Nous avons ensuite à traiter de l’équité et de l’équitable, et montrer leurs relations respectives avec la justice et avec le juste108. En effet, à y regarder avec attention, il apparaît que la justice et l’équité ne sont ni absolument identiques, ni génériquement \\
différentes : tantôt nous louons ce qui est équitable et l’homme équitable lui-même, au point que, par manière  d’approbation, nous transférons le terme {\itshape équitable} aux actions autres que les actions justes, et en faisons un équivalent de {\itshape bon}, en signifiant par {\itshape plus équitable} qu’une chose est simplement {\itshape meilleure} ; tantôt, par contre, en poursuivant le raisonnement, il nous paraît étrange que l’équitable, s’il est une chose qui s’écarte du juste, reçoive notre approbation. S’ils sont différents, en effet, ou bien le juste, ou bien l’équitable n’est pas \\
bon ; ou si tous deux sont bons, c’est qu’ils sont identiques.\par
Le problème que soulève la notion d’équitable est plus ou moins le résultat de ces diverses affirmations, lesquelles sont cependant toutes correctes d’une certaine façon, et ne s’opposent pas les unes aux autres. En effet, l’équitable, tout en étant supérieur à une certaine justice, est lui-même juste, et ce n’est pas comme appartenant à un genre différent qu’il est supérieur \\
au juste. Il y a donc bien identité du juste et de l’équitable, et tous deux sont bons, bien que l’équitable soit le meilleur des deux. Ce qui fait la difficulté, c’est que l’équitable, tout en étant juste, n’est pas le juste selon la loi, mais un correctif de la justice légale. La raison en est que la loi est toujours quelque chose de général, et qu’il y a des cas d’espèce pour lesquels il n’est pas possible de poser un énoncé général qui s’y applique avec rectitude. Dans les matières, donc, où on doit nécessairement \\
se borner à des généralités et où il est impossible de le faire correctement, la loi ne prend en considération que les cas les plus fréquents, sans ignorer d’ailleurs les erreurs que cela peut entraîner. La loi n’en est pas moins sans reproche, car la faute n’est pas à la loi, ni au législateur, mais tient à la nature des choses, puisque par leur essence même la matière des choses de l’ordre pratique revêt ce caractère d’irrégularité. Quand, par \\
suite, la loi pose une règle générale, et que là-dessus survient un cas en dehors de la règle générale, on est alors en droit, là où le législateur a omis de prévoir le cas et a péché par excès de simplification, de corriger l’omission et de se faire l’interprète de ce qu’eût dit le législateur lui-même s’il avait été présent à ce moment, et de ce qu’il aurait porté dans sa loi s’il avait connu le cas en question. De là vient que l’équitable est juste, et qu’il est supérieur à une certaine espèce de juste, non pas supérieur au juste absolu, mais seulement au juste où peut se \\
rencontrer l’erreur due au caractère absolu de la règle. Telle est la nature de l’équitable : c’est d’être un correctif de la loi, là où la loi a manqué de statuer à cause de sa généralité. En fait, la raison pour laquelle tout n’est pas défini par la loi, c’est qu’il y a des cas d’espèce pour lesquels il est impossible de poser une loi, de telle sorte qu’un décret est indispensable. En effet, de ce qui \\
est indéterminé la règle aussi est indéterminée, à la façon de la règle de plomb utilisée dans les constructions de Lesbos : de même que la règle épouse les contours de la pierre et n’est pas rigide, ainsi le décret est adapté aux faits.\par
On voit ainsi clairement ce qu’est l’équitable, que l’équitable est juste et qu’il est supérieur à une certaine sorte de juste. De là résulte nettement aussi la nature de l’homme équitable : \\
celui qui a tendance à choisir et à accomplir les actions équitables  et ne s’en tient pas rigoureusement à ses droits dans le sens du pire, mais qui a tendance à prendre moins que son dû, bien qu’il ait la loi de son côté, celui-là est un homme équitable, et cette disposition est l’équité, qui est une forme spéciale de la justice et non pas une disposition entièrement distincte.
\subsection[{15 (1138a — 1138b) < Dernière aporie : de l’injustice envers soi-même >}]{15 (1138a — 1138b) < Dernière aporie : de l’injustice envers soi-même >}
\noindent Mais est-il possible ou non de commettre l’injustice envers \\
soi-même ? La réponse à cette question résulte clairement de ce que nous avons dit. En effet, parmi les actions justes figurent les actions conformes à quelque vertu, quelle qu’elle soit, qui sont prescrites par la loi : par exemple, la loi ne permet pas expressément le suicide, et ce qu’elle ne permet pas expressément, elle le défend. En outre, quand, contrairement à la loi, un homme cause du tort (autrement qu’à titre de représailles) et cela volontairement, il agit injustement, — et agir volontairement c’est connaître à la fois et la personne qu’on lèse et l’instrument dont on se sert ; or celui qui, dans un accès de \\
colère, se tranche à lui-même la gorge, accomplit cet acte contrairement à la droite règle, et cela la loi ne le permet pas ; aussi commet-il une injustice. Mais contre qui ? N’est-ce pas contre la cité, et non contre lui-même ? Car le rôle passif qu’il joue est volontaire, alors que personne ne subit volontairement une injustice. Telle est aussi la raison pour laquelle la cité inflige une peine ; et une certaine dégradation civique s’attache à celui qui s’est détruit lui-même, comme ayant agi injustement envers la cité.\par
En outre, au sens où celui qui agit injustement est injuste \\
seulement et n’est pas d’une perversité totale, il n’est pas possible de commettre une injustice envers soi-même (c’est là un cas distinct du précédent, parce que, en ce sens du terme, l’homme injuste est pervers de la même façon que le lâche, et non pas comme possédant la perversité totale, de sorte que son action injuste ne manifeste pas non plus une perversité totale). En effet, si cela était possible, la même chose pourrait en même temps être enlevée et être ajoutée à la même chose, ce qui est impossible, le juste et l’injuste se réalisant nécessairement \\
toujours en plus d’une personne. En outre, une action injuste est non seulement à la fois volontaire et le résultat d’un libre choix, mais elle est encore quelque chose d’antérieur (car l’homme qui, parce qu’il a été éprouvé lui-même, rend mal pour mal, n’est pas regardé comme agissant injustement) ; or quand on commet une injustice envers soi-même, on est pour les mêmes choses passif et actif, et cela en même temps. De plus, ce serait admettre qu’on peut subir volontairement l’injustice. Ajoutons à cela qu’on n’agit jamais injustement sans \\
accomplir des actes particuliers d’injustice ; or on ne peut jamais commettre d’adultère avec sa propre femme, ni pénétrer par effraction dans sa propre maison, ni voler ce qui est à soi.\par
D’une manière générale, la question de savoir si on peut agir injustement envers soi-même se résout à la lumière de la distinction que nous avons posée au sujet de la possibilité de subir volontairement l’injustice.\par
Il est manifeste aussi que les deux choses sont également mauvaises, à savoir subir une injustice et commettre une injustice, puisque, dans le premier cas, on a moins, et, dans le \\
second, plus que la juste moyenne, laquelle joue ici le rôle du {\itshape sain} en médecine et du {\itshape bon état corporel} en gymnastique. Mais cependant le pire des deux, c’est commettre l’injustice, car commettre l’injustice s’accompagne de vice et provoque notre désapprobation, — vice qui, au surplus, est d’une espèce achevée et atteint l’absolu ou presque (presque, car une action injuste commise volontairement ne s’accompagne pas toujours de vice), tandis que subir l’injustice est indépendant de vice et \\
d’injustice < chez la victime >. Ainsi, en soi, subir l’injustice est  un mal moins grand, quoique par accident rien n’empêche que ce ne soit un plus grand mal. Mais l’art se désintéresse de l’accident : il déclare qu’une pleurésie est une maladie plus grave qu’une foulure ; cependant dans certains cas une foulure peut devenir accidentellement plus grave qu’une pleurésie, si par exemple la foulure provoque une chute qui vous fait \\
tomber aux mains de l’ennemi ou cause votre mort.\par
Par extension de sens et simple similitude, il y a justice, non pas entre un homme et lui-même, mais entre certaines parties de lui-même : ce n’est pas d’ailleurs n’importe quelle justice, mais cette justice qui existe entre maître et esclave, ou entre mari et femme. En effet, dans les discussions sur ces questions, on a établi une distinction entre la partie rationnelle de l’âme et sa partie irrationnelle ; et dès lors c’est en fixant son \\
attention sur ces diverses parties qu’on pense d’ordinaire qu’il existe une injustice envers soi-même, parce que ces parties peuvent être affectées dans un sens contraire à leurs tendances respectives. Ainsi, il peut y avoir aussi entre elles une certaine forme de justice, analogue à celle qui existe entre gouvernant et gouverné.
\section[{Livre VI}]{Livre VI}\renewcommand{\leftmark}{Livre VI}

\subsection[{1 (1138b) < Passage aux vertus intellectuelles. La « droite règle » >}]{1 (1138b) < Passage aux vertus intellectuelles. La « droite règle » >}
\noindent \\
Au sujet de la justice et des autres vertus morales, prenons-les comme définies de la façon que nous avons indiquée.\par
Et puisque, en fait, nous avons dit plus haut que nous \\
devons choisir le moyen terme, et non l’excès ou le défaut, et que le moyen terme est conforme à ce qu’énonce la droite règle [ορθος λογος] analysons maintenant ce dernier point.\par
Dans toutes les dispositions morales dont nous avons parlé aussi bien que dans les autres domaines il y a un certain but sur lequel, fixant son regard, l’homme qui est en possession de la droite règle intensifie ou relâche son effort et il existe un certain principe de détermination des médiétés, lesquelles constituent, disons-nous, un état intermédiaire entre \\
l’excès et le défaut, du fait qu’elles sont en conformité avec la droite règle. Mais une telle façon de s’exprimer, toute vraie qu’elle soit, manque de clarté. En effet, même en tout ce qui rentre par ailleurs dans les préoccupations de la science, on peut dire avec vérité assurément que nous ne devons déployer notre effort, ou le relâcher, ni trop ni trop peu, mais observer le juste milieu, et cela comme le demande la droite règle ; seulement, \\
la simple possession de cette vérité ne peut accroître en rien notre connaissance : nous ignorerions, par exemple, quelles sortes de remèdes il convient d’appliquer à notre corps si quelqu’un se contentait de nous dire : « Ce sont tous ceux que prescrit l’art médical et de la façon indiquée par l’homme de l’art ». Aussi faut-il également, en ce qui concerne les dispositions de l’âme, non seulement établir la vérité de ce que nous avons dit ci-dessus, mais encore déterminer quelle est la nature de la droite règle, et son principe de détermination.
\subsection[{2 (1138b — 1139b) < Objet de la vertu intellectuelle ; combinaison du désir et de l’intellect >}]{2 (1138b — 1139b) < Objet de la vertu intellectuelle ; combinaison du désir et de l’intellect >}
\noindent \\
Nous avons divisé les vertus de l’âme, et distingué, d’une  part les vertus du caractère, et d’autre part celles de l’intellect. Nous avons traité en détail des vertus morales ; pour les autres restantes, après quelques remarques préalables au sujet de l’âme, voici ce que nous avons à dire.\par
Antérieurement nous avons indiqué qu’il y avait deux parties de l’âme, à savoir la partie rationnelle et la partie \\
irrationnelle. Il nous faut maintenant établir, pour la partie rationnelle elle-même, une division de même nature. Prenons pour base de discussion que les parties rationnelles sont au nombre de deux, l’une par laquelle nous contemplons ces sortes d’êtres dont les principes ne peuvent être autrement qu’ils ne sont, et l’autre par laquelle nous connaissons les choses contingentes : quand, en effet, les objets diffèrent par le genre, les parties de l’âme adaptées naturellement à la connaissance des uns et des autres doivent aussi différer par le \\
genre, s’il est vrai que c’est sur une certaine ressemblance et affinité entre le sujet et l’objet que la connaissance repose. Appelons l’une de ces parties la partie {\itshape scientifique}, et l’autre la {\itshape calculative}, délibérer et calculer étant une seule et même chose, et on ne délibère jamais sur les choses qui ne peuvent être autrement qu’elles ne sont. Par conséquent, la partie \\
calculative est seulement une partie de la partie rationnelle de l’âme. Il faut par suite bien saisir quelle est pour chacune de ces deux parties sa meilleure disposition : on aura là la vertu de chacune d’elles, et la vertu d’une chose est relative à son œuvre propre [εργον].\par
Or il y a dans l’âme trois facteurs prédominants qui déterminent l’action et la vérité : sensation, intellect et désir. \\
De ces facteurs, la sensation n’est principe d’aucune action [πραξις]118, comme on peut le voir par l’exemple des bêtes, qui possèdent bien la sensation mais n’ont pas l’action en partage. Et ce que l’affirmation et la négation sont dans la pensée, la recherche et l’aversion le sont dans l’ordre du désir ; par conséquent, puisque la vertu morale est une disposition capable de choix, et que le choix est un désir délibératif, il faut par là même qu’à la fois la règle soit vraie et le désir droit, si le choix est bon, et \\
qu’il y ait identité entre ce que la règle affirme et ce que le désir poursuit. Cette pensée et cette vérité dont nous parlons ici sont de l’ordre pratique ; quant à la pensée contemplative, qui n’est ni pratique, ni poétique, son bon et son mauvais état consiste dans le vrai et le faux auxquels son activité aboutit, puisque c’est là l’œuvre de toute partie intellective, tandis que pour la \\
partie de l’intellect pratique, son bon état consiste dans la vérité correspondant au désir, au désir correct.\par
Le principe de l’action morale est ainsi le libre choix ({\itshape principe} étant ici le point d’origine du mouvement et non la fin où il tend), et celui du choix est le désir et la règle dirigée vers quelque fin, C’est pourquoi le choix ne peut exister ni sans intellect et pensée, ni sans une disposition morale, la bonne \\
conduite et son contraire dans le domaine de l’action n’existant pas sans pensée et sans caractère. La pensée par elle-même cependant n’imprime aucun mouvement, mais seulement la pensée dirigée vers une fin et d’ordre pratique. Cette dernière  sorte de pensée commande également l’intellect poétique [ποιητικη] puisque dans la production l’artiste agit toujours en vue d’une fin ; la production n’est pas une fin au sens absolu, mais est quelque chose de relatif et production d’une chose déterminée. Au contraire, dans l’action, ce qu’on fait < est une fin au sens absolu >, car la vie vertueuse est une fin, et le désir a cette fin pour objet.\par
Aussi peut-on dire indifféremment que le choix \\
préférentiel est un intellect désirant ou un désir raisonnant, et le principe qui est de cette sorte est un homme.\par
Le passé ne peut jamais être objet de choix : personne ne choisit d’avoir saccagé Troie ; la délibération, en effet, porte, non sur le passé, mais sur le futur et le contingent, alors que le passé ne peut pas ne pas avoir été. Aussi Agathon a-t-il raison de dire :\par
\\
{\itshape Car il y a une seule chose dont Dieu même est privé}, \par
 {\itshape C’est de faire que ce qui a été fait ne l’ait pas été.} \par
Ainsi les deux parties intellectuelles de l’âme ont pour tâche la vérité. C’est pourquoi les dispositions qui permettent à chacune d’elles d’atteindre le mieux la vérité constituent les vertus respectives de l’une et de l’autre.
\subsection[{3 (1139b) < Énumération des vertus intellectuelles. Étude de la science >}]{3 (1139b) < Énumération des vertus intellectuelles. Étude de la science >}
\noindent Reprenons donc depuis le début, et traitons à nouveau de ces dispositions.\par
\\
Admettons que les états par lesquels l’âme énonce ce qui est vrai sous une forme affirmative ou négative121 sont au nombre de cinq : ce sont l’art, la science, la prudence [φρονησις], la sagesse et la raison intuitive, car par le jugement et l’opinion il peut arriver que nous soyons induits en erreur.\par
La nature de la science (si nous employons ce terme dans son sens rigoureux, et en négligeant les sens de pure similitude), résulte clairement des considérations suivantes. Nous \\
concevons tous que les choses dont nous avons la science ne peuvent être autrement qu’elles ne sont : pour les choses qui peuvent être autrement, dès qu’elles sont sorties du champ de notre connaissance, nous ne voyons plus si elles existent ou non. L’objet de la science existe donc nécessairement ; il est par suite éternel, car les êtres qui existent d’une nécessité absolue sont tous éternels ; et les êtres éternels sont inengendrés \\
et incorruptibles. De plus, on pense d’ordinaire que toute science est susceptible d’être enseignée, et que l’objet de science peut s’apprendre. Mais tout enseignement donné vient de connaissances préexistantes, comme nous l’établissons aussi dans les {\itshape Analytiques}, puisqu’il procède soit par induction, soit par syllogisme. L’induction dès lors est principe aussi de l’universel, tandis que le syllogisme procède à \\
partir des universels. Il y a par conséquent des principes qui servent de point de départ au syllogisme, principes dont il n’y a pas de syllogisme possible, et qui par suite sont obtenus par induction. Ainsi la science est une disposition capable de démontrer, en ajoutant à cette définition toutes les autres caractéristiques mentionnées dans nos {\itshape Analytiques}, car lorsque un homme a sa conviction établie d’une certaine façon et que les principes lui sont familiers, c’est alors qu’il a la science, car si les principes ne lui sont pas plus connus que la conclusion il aura seulement la science par accident.
\subsection[{4 (1139b — 1140a) < Étude de l’art >}]{4 (1139b — 1140a) < Étude de l’art >}
\noindent \\
Telle est donc la façon dont nous pouvons définir la science.\par
 Les choses qui peuvent être autres qu’elles ne sont comprennent à la fois les choses qu’on fabrique et les actions qu’on accomplit. Production et action sont distinctes (sur leur nature nous pouvons faire confiance aux discours exotériques) ; il s’ensuit que la disposition à agir accompagnée de règle est différente de la disposition à produire accompagnée \\
de règle. De là vient encore qu’elles ne sont pas une partie l’une de l’autre, car ni l’action n’est une production, ni la production une action. Et puisque l’architecture est un art, et est essentiellement une certaine disposition à produire, accompagnée de règle, et qu’il n’existe aucun art qui ne soit une disposition à produire accompagnée de règle, ni aucune disposition de ce \\
genre qui ne soit un art, il y aura identité entre art et disposition à produire accompagnée de règle exacte. L’art concerne toujours un devenir, et s’appliquer à un art, c’est considérer la façon d’amener à l’existence une de ces choses qui sont susceptibles d’être ou de n’être pas, mais dont le principe d’existence réside dans l’artiste et non dans la chose produite : l’art, en effet, ne concerne ni les choses qui existent ou deviennent \\
nécessairement, ni non plus les êtres naturels, qui ont en eux-mêmes leur principe. Mais puisque production et action sont quelque chose de différent, il faut nécessairement que l’art relève de la production et non de l’action. Et en un sens la fortune et l’art ont rapport aux mêmes objets, ainsi qu’Agathon le dit :\par
 {\itshape L’art affectionne la fortune, et la fortune l’art.} \par
\\
Ainsi donc, l’art, comme nous l’avons dit, est une certaine disposition, accompagnée de règle vraie, capable de produire ; le défaut d’art, au contraire, est une disposition à produire accompagnée de règle fausse ; dans un cas comme dans l’autre, on se meut dans le domaine du contingent.
\subsection[{5 (1140a — 1140b) < Étude de la prudence [φρονησις] >}]{5 (1140a — 1140b) < Étude de la prudence [φρονησις] >}
\noindent 132 Une façon dont nous pourrions appréhender la nature de la \\
prudence, c’est de considérer quelles sont les personnes que nous appelons prudentes. De l’avis général, le propre d’un homme prudent c’est d’être capable de délibérer correctement sur ce qui est bon et avantageux pour lui-même, non pas sur un point partiel (comme par exemple quelles sortes de choses sont favorables à la santé ou à la vigueur du corps), mais d’une façon générale, quelles sortes de choses par exemple conduisent à la vie heureuse. Une preuve, c’est que nous appelons aussi prudents ceux qui le sont en un domaine déterminé, quand ils calculent avec justesse en vue d’atteindre une fin \\
particulière digne de prix, dans des espèces où il n’est pas question d’art ; il en résulte que, en un sens général aussi, sera un homme prudent celui qui est capable de délibération.\par
Mais on ne délibère jamais sur les choses qui ne peuvent pas être autrement qu’elles ne sont, ni sur celles qu’il nous est impossible d’accomplir ; par conséquent s’il est vrai qu’une science s’accompagne de démonstration, mais que les choses dont les principes peuvent être autres qu’ils ne sont n’admettent \\
pas de démonstration (car toutes sont également susceptibles d’être autrement qu’elles ne sont), et s’il n’est pas  possible de délibérer sur les choses qui existent nécessairement, la prudence ne saurait être ni une science, ni un art : une science, parce que l’objet de l’action peut être autrement qu’il n’est ; un art, parce que le genre de l’action est autre que celui de la production. Reste donc que la prudence est une disposition, \\
accompagnée de règle vraie, capable d’agir dans la sphère de ce qui est bon ou mauvais pour un être humain. Tandis que la production, en effet, a une fin autre qu’elle-même, il n’en saurait être ainsi pour l’action, la bonne pratique étant elle-même sa propre fin. C’est pourquoi nous estimons que Périclès et les gens comme lui sont des hommes prudents en ce qu’ils possèdent la faculté d’apercevoir ce qui est bon pour eux-mêmes et ce qui est bon pour l’homme en général, et \\
tels sont aussi, pensons-nous, les personnes qui s’entendent à l’administration d’une maison ou d’une cité. — De là vient aussi le nom par lequel nous désignons la {\itshape tempérance} (σωφροσυνη), pour signifier qu’elle {\itshape conserve la prudence} (σωζουσα την φρονησιν), et ce qu’elle conserve, c’est le jugement dont nous indiquons la nature : car le plaisir et la douleur ne détruisent pas et ne faussent pas tout jugement quel qu’il soit, \\
par exemple le jugement que le triangle a ou n’a pas ses angles égaux à deux droits, mais seulement les jugements ayant trait à l’action. En effet, les principes de nos actions consistent dans la fin à laquelle tendent nos actes ; mais à l’homme corrompu par l’attrait du plaisir ou la crainte de la douleur, le principe n’apparaît pas immédiatement, et il est incapable de voir en vue de quelle fin et pour quel motif il doit choisir et accomplir tout ce qu’il fait, car le vice est destructif du principe. Par \\
conséquent, la prudence est nécessairement une disposition, accompagnée d’une règle exacte, capable d’agir, dans la sphère des biens humains.\par
En outre, dans l’art on peut parler d’excellence, mais non dans la prudence. Et, dans le domaine de l’art, l’homme qui se trompe volontairement est préférable à celui qui se trompe involontairement, tandis que dans le domaine de la prudence c’est l’inverse qui a lieu, tout comme dans le domaine des vertus également. On voit donc que la prudence est une excellence et non un art.\par
\\
Des deux parties de l’âme, douées de raison, l’une des deux, la faculté d’opiner, aura pour vertu la prudence : car l’opinion a rapport à ce qui peut être autrement qu’il n’est, et la prudence aussi. Mais cependant la prudence n’est pas simplement une disposition accompagnée de règle : une preuve, c’est \\
que l’oubli peut atteindre la disposition de ce genre, tandis que pour la prudence il n’en est rien.
\subsection[{6 (1140b — 1141a) < Étude de la raison intuitive >}]{6 (1140b — 1141a) < Étude de la raison intuitive >}
\noindent Puisque la science consiste en un jugement portant sur les universels et les êtres nécessaires, et qu’il existe des principes d’où découlent les vérités démontrées et toute science en général (puisque la science s’accompagne de raisonnement), il en résulte que le principe de ce que la science connaît ne saurait être lui-même objet ni de science, ni d’art, ni de prudence : en effet, l’objet de la science est démontrable, et d’autre part l’art et la prudence se trouvent avoir rapport aux \\
choses qui peuvent être autrement qu’elles ne sont. Mais la  sagesse n’a pas non plus dès lors les principes pour objet, puisque le propre du sage c’est d’avoir une démonstration pour certaines choses. Si donc les dispositions qui nous permettent d’atteindre la vérité et d’éviter toute erreur dans les choses qui ne peuvent être autrement qu’elles ne sont ou dans celles qui peuvent être autrement, si ces dispositions-là sont la science, la prudence, la sagesse et l’intellect, et si trois d’entre elles ne \\
peuvent jouer aucun rôle dans l’appréhension des principes (j’entends la prudence, la science et la sagesse), il reste que c’est la raison intuitive qui les saisit.
\subsection[{7 (1141a — 1141b) < La sagesse théorétique >}]{7 (1141a — 1141b) < La sagesse théorétique >}
\noindent Le terme {\itshape sagesse} [σοφια] dans les arts est par nous appliqué à ceux \\
qui atteignent la plus exacte maîtrise dans l’art en question, par exemple à Phidias comme sculpteur habile et à Polyclète comme statuaire ; et en ce premier sens, donc, nous ne signifions par sagesse rien d’autre qu’excellence dans un art. Mais nous pensons aussi que certaines personnes sont sages d’une manière générale, et non sages dans un domaine particulier, ni sages en quelque autre chose, pour parler comme Homère dans {\itshape Margitès} :\par
{\itshape Celui-là les dieux ne l’avaient fait ni vigneron, ni laboureur}, \par
 \\
 {\itshape Ni sage en quelque autre façon.} \par
Il est clair, par conséquent, que la sagesse sera la plus achevée des formes du savoir. Le sage doit donc non seulement connaître les conclusions découlant des principes, mais encore posséder la vérité sur les principes eux-mêmes. La sagesse sera ainsi à la fois raison intuitive et science, science munie en quelque sorte d’une tête et portant sur les réalités les plus \\
hautes. Il est absurde, en effet, de penser que l’art politique ou la prudence soit la forme la plus élevée du savoir, s’il est vrai que l’homme n’est pas ce qu’il y a de plus excellent dans le Monde. Si dès lors {\itshape sain} et {\itshape bon} est une chose différente pour des hommes et pour des poissons, tandis que {\itshape blanc} et {\itshape rectiligne} est toujours invariable, on reconnaîtra chez tous les hommes que ce qui est sage est la même chose, mais que ce qui est \\
prudent est variable : car c’est l’être qui a une vue nette des diverses choses qui l’intéressent personnellement, qu’on désigne du nom de {\itshape prudent}, et c’est à lui qu’on remettra la conduite de ces choses-là. De là vient encore que certaines bêtes sont qualifiées de prudentes : ce sont celles qui, en tout ce qui touche à leur propre vie, possèdent manifestement une capacité de prévoir.\par
Il est de toute évidence aussi que la sagesse ne saurait être \\
identifiée à l’art politique : car si on doit appeler la connaissance de ses propres intérêts une sagesse, il y aura multiplicité de sagesses : il n’existe pas, en effet, une seule sagesse s’appliquant au bien de tous les êtres animés, mais il y a une sagesse différente pour chaque espèce, de même qu’il n’y a pas non plus un seul art médical pour tous les êtres. Et si on objecte qu’un homme l’emporte en perfection sur les autres animaux, cela n’importe ici en rien : il existe, en effet, d’autres êtres  d’une nature beaucoup plus divine que l’homme, par exemple, pour s’en tenir aux réalités les plus visibles, les Corps dont le Monde est constitué.\par
Ces considérations montrent bien que la sagesse est à la fois science et raison intuitive des choses qui ont par nature la dignité la plus haute. C’est pourquoi nous disons qu’Anaxagore, Thalès et ceux qui leur ressemblent, possèdent \\
la sagesse, mais non la prudence, quand nous les voyons ignorer les choses qui leur sont profitables à eux-mêmes, et nous reconnaissons qu’ils ont un savoir hors de pair, admirable, difficile et divin, mais sans utilité, du fait que ce ne sont pas les biens proprement humains qu’ils recherchent.
\subsection[{8 (1141b) < La prudence et l’art politique >}]{8 (1141b) < La prudence et l’art politique >}
\noindent Or la prudence a rapport aux choses humaines et aux choses qui admettent la délibération : car le prudent, disons-nous, \\
a pour œuvre principale de bien délibérer ; mais on ne délibère jamais sur les choses qui ne peuvent être autrement qu’elles ne sont, ni sur celles qui ne comportent pas quelque fin à atteindre, fin qui consiste en un bien réalisable. Le bon délibérateur au sens absolu est l’homme qui s’efforce d’atteindre le meilleur des biens réalisables pour l’homme, et qui le fait par raisonnement.\par
\\
La prudence n’a pas non plus seulement pour objet les universels, mais elle doit aussi avoir la connaissance des faits particuliers, car elle est de l’ordre de l’action, et l’action a rapport aux choses singulières. C’est pourquoi aussi certaines personnes ignorantes sont plus qualifiées pour l’action que d’autres qui savent : c’est le cas notamment des gens d’expérience : si, tout en sachant que les viandes légères sont faciles à digérer et bonnes pour la santé, on ignore quelles sortes de viandes sont légères, on ne produira pas la santé, tandis que si \\
on sait que la chair de volaille est légère, on sera plus capable de produire la santé.\par
La prudence étant de l’ordre de l’action, il en résulte qu’on doit posséder les deux sortes de connaissances, et de préférence celle qui porte sur le singulier. Mais ici encore elle dépendra d’un art architectonique.\par
La sagesse politique et la prudence sont une seule et même disposition, bien que leur essence ne soit cependant pas \\
la même. De la prudence appliquée à la cité, une première espèce, en tant qu’elle a sous sa dépendance toutes les autres, est législative ; l’autre espèce, en tant que portant sur les choses particulières, reçoit le nom, qui lui est d’ailleurs commun avec la précédente, de politique. Cette dernière espèce a rapport à l’action et à la délibération, puisque tout décret doit être rendu dans une forme strictement individuelle. C’est pourquoi {\itshape administrer la cité} est une expression réservée pour ceux qui entrent dans la particularité des affaires, car ce sont les seuls qui accomplissent la besogne, semblables en cela aux artisans.\par
\\
Dans l’opinion commune, la prudence aussi est prise surtout sous la forme où elle ne concerne que la personne privée, c’est-à-dire un individu ; et cette forme particulière reçoit le nom général de prudence, Des autres espèces, l’une est appelée {\itshape économie domestique}, une autre {\itshape législation}, une autre enfin, {\itshape politique}, celle-ci se subdivisant en délibérative et judiciaire.
\subsection[{9 (1141b — 1142a) < La prudence et l’art politique, suite. L’intuition des singuliers >}]{9 (1141b — 1142a) < La prudence et l’art politique, suite. L’intuition des singuliers >}
\noindent Une des formes de la connaissance sera assurément de savoir le bien qui est propre à soi-même, mais cette connaissance-là est très différente des autres espèces. Et on pense  d’ordinaire que celui qui connaît ses propres intérêts et qui y consacre sa vie, est un homme prudent, tandis que les politiques s’occupent d’une foule d’affaires. D’où les vers d’Euripide :\par
 {\itshape Mais comment serais-je sage, moi à qui il était possible de vivre à l’abri des affaires.} \par
{\itshape Simple numéro perdu dans la foule des soldats}, \par
 \\
 {\itshape Participant au sort commun ?…} \par
 {\itshape Car les gens hors de pair et qui en font plus que les autres…} \par
Ceux qui pensent ainsi ne recherchent que leur propre bien, et ils croient que c’est un devoir d’agir ainsi. Cette opinion a fait naître l’idée que de pareils gens sont des hommes prudents ; peut-être cependant la poursuite par chacun de son bien propre \\
ne va-t-elle pas sans économie domestique ni politique. En outre, la façon dont on doit administrer ses propres affaires n’apparaît pas nettement et demande examen.\par
Ce que nous avons dit est d’ailleurs confirmé par ce fait que les jeunes gens peuvent devenir géomètres ou mathématiciens ou savants dans les disciplines de ce genre, alors qu’on n’admet pas communément qu’il puisse exister de jeune homme prudent. La cause en est que la prudence a rapport aussi aux faits particuliers, qui ne nous deviennent familiers \\
que par l’expérience, dont un jeune homme est toujours dépourvu (car c’est à force de temps que l’expérience s’acquiert). On pourrait même se demander pourquoi un enfant, qui peut faire un mathématicien, est incapable d’être philosophe ou même physicien. Ne serait-ce pas que, parmi ces sciences, les premières s’acquièrent par abstraction, tandis que les autres ont leurs principes dérivés de l’expérience, et que, dans ce dernier cas, les jeunes gens ne se sont formés aucune conviction et se contentent de paroles, tandis \\
que les notions mathématiques, au contraire, ont une essence dégagée de toute obscurité ? — Ajoutons que l’erreur dans la délibération peut porter soit sur l’universel, soit sur le singulier, si on soutient par exemple que toutes les eaux pesantes sont pernicieuses, ou bien que telle eau déterminée est pesante.\par
Et que la prudence ne soit pas science, c’est là une chose manifeste : elle porte, en effet, sur ce qu’il y a de plus particulier, comme nous l’avons dit, car l’action à accomplir est \\
elle-même particulière. La prudence dès lors s’oppose à la raison intuitive : la raison intuitive, en effet, appréhende les définitions, pour lesquelles on ne peut donner aucune raison, tandis que la prudence est la connaissance de ce qu’il y a de plus individuel, lequel n’est pas objet de science, mais de perception : non pas la perception des sensibles propres, mais une perception de la nature de celle par laquelle nous percevons que telle figure mathématique particulière est un triangle ; car dans cette direction aussi on devra s’arrêter. Mais cette \\
intuition mathématique est plutôt perception que prudence, et de la prudence l’intuition est spécifiquement différente.
\subsection[{10 (1142a — 1142b) < Les vertus intellectuelles mineures. La bonne délibération >}]{10 (1142a — 1142b) < Les vertus intellectuelles mineures. La bonne délibération >}
\noindent La recherche et la délibération diffèrent, car la délibération est une recherche s’appliquant à une certaine chose. — Nous devons aussi appréhender quelle est la nature de la bonne délibération, si elle est une forme de science, ou opinion, ou justesse de coup d’ œil, ou quelque autre genre différent.\par
 Or elle n’est pas science (on ne cherche pas les choses qu’on sait, alors que la bonne délibération est une forme de délibération, et que celui qui délibère cherche et calcule). — Mais elle n’est pas davantage justesse de coup d’œil, car la justesse de coup d’œil a lieu indépendamment de tout calcul conscient, et d’une manière rapide, tandis que la délibération exige beaucoup de temps, et on dit que s’il faut exécuter avec \\
rapidité ce qu’on a délibéré de faire, la délibération elle-même doit être lente. Autre raison : la vivacité d’esprit est une chose différente de la bonne délibération ; or la vivacité d’esprit est une sorte de justesse de coup d’œil. — La bonne délibération n’est pas non plus une forme quelconque d’opinion. Mais puisque celui qui délibère mal se trompe et que celui qui délibère bien délibère correctement, il est clair que la bonne délibération est une certaine rectitude. Mais on ne peut affirmer la rectitude ni de la science, ni de l’opinion : pour la science, en \\
effet. on ne peut pas parler de rectitude (pas plus que d’erreur), et pour l’opinion sa rectitude est vérité ; et en même temps, tout ce qui est objet d’opinion est déjà déterminé. Mais la bonne délibération ne va pas non plus sans calcul conscient. Il reste donc qu’elle est rectitude de pensée, car ce n’est pas encore une assertion, puisque l’opinion n’est pas une recherche mais est \\
déjà une certaine assertion, tandis que l’homme qui délibère bien ou mal, recherche quelque chose et calcule.\par
Mais la bonne délibération étant une certaine rectitude de délibération, nous devons donc d’abord rechercher ce qu’est la délibération en général et sur quel objet elle porte. Et {\itshape rectitude} étant un terme à sens multiples, il est clair qu’il ne s’agit pas ici de toute rectitude quelle qu’elle soit. En effet, l’homme intempérant ou pervers, s’il est habile, atteindra ce qu’il se propose à l’aide du calcul, de sorte qu’il aura délibéré correctement, \\
alors que c’est un mal considérable qu’il s’est procuré : or on admet couramment qu’avoir bien délibéré est en soi-même un bien, car c’est cette sorte de rectitude de délibération qui est bonne délibération, à savoir celle qui tend à atteindre un bien. — Mais il est possible d’atteindre même le bien par un faux syllogisme, et d’atteindre ce qu’il est de notre devoir de faire, mais en se servant non pas du moyen qui convient, mais à l’aide d’un moyen terme erroné. Par conséquent, cet état, en \\
vertu duquel on atteint ce que le devoir prescrit mais non cependant par la voie requise, n’est toujours pas bonne délibération. — On peut aussi arriver au but par une délibération de longue durée, alors qu’un autre l’atteindra rapidement : dans le premier cas, ce n’est donc pas encore une bonne délibération, laquelle est rectitude eu égard à ce qui est utile, portant à la fois sur la fin à atteindre, la manière et le temps. — En outre, on peut avoir bien délibéré soit au sens absolu, soit par rapport à une \\
fin déterminée. La bonne délibération au sens absolu est dès lors celle qui mène à un résultat correct par rapport à la fin prise absolument, alors que la bonne délibération en un sens déterminé est celle qui n’aboutit à un résultat correct que par rapport à une fin elle-même déterminée. Si donc les hommes prudents ont pour caractère propre le fait d’avoir bien délibéré, la bonne délibération sera une rectitude en ce qui concerne ce qui est utile à la réalisation d’une fin, utilité dont la véritable conception est la prudence elle-même.
\subsection[{11 (1142b — 1143a) < Les vertus intellectuelles mineures, suite. L’intelligence et le jugement >}]{11 (1142b — 1143a) < Les vertus intellectuelles mineures, suite. L’intelligence et le jugement >}
\noindent L’intelligence aussi et la perspicacité, qui nous font dire  des gens qu’ils sont intelligents et perspicaces, ne sont pas absolument la même chose que la science ou l’opinion (car, dans ce dernier cas, tout le monde serait intelligent), et ne sont pas davantage quelqu’une des sciences particulières, comme la médecine, science des choses relatives à la santé, ou la géométrie, \\
science des grandeurs. Car l’intelligence ne roule ni sur les êtres éternels et immobiles, ni sur rien de ce qui devient, mais seulement sur les choses pouvant être objet de doute et de délibération. Aussi porte-t-elle sur les mêmes objets que la prudence, bien qu’intelligence et prudence ne soient pas identiques. La prudence est, en effet, directive (car elle a pour fin de déterminer ce qu’il est de notre devoir de faire ou de ne \\
pas faire), tandis que l’intelligence est seulement judicative (car il y a identité entre intelligence et perspicacité, entre un homme intelligent et un homme perspicace).\par
L’intelligence ne consiste ni à posséder la prudence, ni à l’acquérir. Mais de même que {\itshape apprendre} s’appelle {\itshape comprendre} quand on exerce la faculté de connaître scientifiquement, ainsi {\itshape comprendre} s’applique à l’exercice de la faculté d’opinion, quand il s’agit de porter un jugement sur ce qu’une autre personne énonce dans des matières relevant de la prudence, \\
et par jugement j’entends un jugement fondé, car {\itshape bien} est la même chose que {\itshape fondé}. Et l’emploi du terme {\itshape intelligence} pour désigner la qualité des gens perspicaces est venu de l’{\itshape intelligence} au sens d’{\itshape apprendre}, car nous prenons souvent {\itshape apprendre} au sens de {\itshape comprendre}.\par
Ce qu’on appelle enfin {\itshape jugement}, qualité d’après laquelle \\
nous disons des gens qu’ils ont un {\itshape bon jugement} ou qu’ils ont {\itshape du jugement}, est la correcte discrimination de ce qui est équitable. Ce qui le montre bien, c’est le fait que nous disons que l’homme équitable est surtout favorablement disposé pour autrui, et que montrer dans certains cas de la largeur d’esprit est équitable. Et dans la largeur d’esprit on fait preuve de jugement en appréciant correctement ce qui est équitable ; et juger correctement c’est juger ce qui est vraiment équitable.
\subsection[{12 (1143a — 1143b) < Relations des vertus dianoétiques entre elles et avec la prudence >}]{12 (1143a — 1143b) < Relations des vertus dianoétiques entre elles et avec la prudence >}
\noindent \\
Toutes les dispositions dont il a été question convergent, comme cela est normal, vers la même chose. En effet, nous attribuons jugement, intelligence, prudence et raison intuitive indifféremment aux mêmes individus quand nous disons qu’ils ont atteint l’âge du jugement et de la raison, et qu’ils sont prudents et intelligents. Car toutes ces facultés portent sur les choses ultimes et particulières ; et c’est en étant capable de juger des choses rentrant dans le domaine de l’homme prudent \\
qu’on est intelligent, bienveillant et favorablement disposé pour les autres, les actions équitables étant communes à tous les gens de bien dans leurs rapports avec autrui. Or toutes les actions que nous devons accomplir rentrent dans les choses particulières et ultimes, car l’homme prudent doit connaître les faits particuliers, et de leur côté l’intelligence et le jugement roulent sur les actions à accomplir, lesquelles sont des choses \\
ultimes. La raison intuitive s’applique aussi aux choses particulières, dans les deux sens à la fois, puisque les termes premiers aussi bien que les derniers sont du domaine de la  raison intuitive et non de la discursion : dans les démonstrations, la raison intuitive appréhende les termes immobiles et premiers, et dans les raisonnements d’ordre pratique, elle appréhende le fait dernier et contingent, c’est-à-dire la prémisse mineure, puisque ces faits-là sont principes de la fin à atteindre, les cas particuliers servant de point de départ pour \\
atteindre les universels. Nous devons donc avoir une perception des cas particuliers, et cette perception est raison intuitive.\par
C’est pourquoi on pense d’ordinaire que ces états sont des qualités naturelles, et, bien que personne ne soit philosophe naturellement, qu’on possède naturellement jugement, intelligence et raison intuitive. Une preuve, c’est que nous croyons que ces dispositions accompagnent les différents âges de la vie, et que tel âge déterminé apporte avec lui raison intuitive et jugement, convaincus que nous sommes que la nature en est la \\
cause. — Voilà pourquoi encore la raison intuitive est à la fois principe et fin, choses qui sont en même temps l’origine et l’objet des démonstrations. — Par conséquent, les paroles et les opinions indémontrées des gens d’expérience, des vieillards et des personnes douées de sagesse pratique sont tout aussi dignes d’attention que celles qui s’appuient sur des démonstrations, car l’expérience leur a donné une vue exercée qui leur permet de voir correctement les choses.
\subsection[{13 (1143b — 1145a) < Utilité de la sagesse théorique et de la sagesse pratique (ou prudence) — Rapports des deux sagesses >}]{13 (1143b — 1145a) < Utilité de la sagesse théorique et de la sagesse pratique (ou prudence) — Rapports des deux sagesses >}
\noindent \\
Nous avons donc établi quelle est la nature de la prudence et de la sagesse théorique, et quelles sont en fait leurs sphères respectives ; et nous avons montré que chacune d’elles est vertu d’une partie différente de l’âme.\par
Mais on peut se poser la question de savoir quelle est l’utilité de ces vertus. La sagesse théorique, en effet, n’étudie \\
aucun des moyens qui peuvent rendre un homme heureux (puisqu’elle ne porte en aucun cas sur un devenir) ; la prudence, par contre, remplit bien ce rôle, mais en vue de quoi avons-nous besoin d’elle ? La prudence a sans doute pour objet les choses justes, belles et bonnes pour l’homme, mais ce sont là des choses qu’un homme de bien accomplit naturellement. Notre action n’est en rien facilitée par la connaissance que nous avons de ces choses, s’il est vrai que les vertus sont des \\
dispositions du caractère, pas plus que ne nous sert la connaissance des {\itshape choses saines} ou des {\itshape choses en bon état}, en prenant ces expressions non pas au sens de {\itshape productrices de santé}, mais comme un résultat de l’état de santé, car nous ne sommes rendus en rien plus aptes à nous bien porter ou à être en bon état, par le fait de posséder l’art médical ou celui de la gymnastique.\par
Mais si, d’un autre côté, on doit poser qu’un homme est prudent non pas afin de connaître les vérités morales, mais afin de devenir vertueux, alors, pour ceux qui le sont déjà, la \\
prudence ne saurait servir à rien. Bien plus, elle ne servira pas davantage à ceux qui ne le sont pas, car peu importera qu’on possède soi-même la prudence ou qu’on suive seulement les conseils d’autres qui la possèdent : il nous suffirait de faire ce que nous faisons en ce qui concerne notre santé, car tout en souhaitant de nous bien porter, nous n’apprenons pas pour autant l’art médical.\par
Ajoutons à cela qu’on peut trouver étrange que la prudence, bien qu’inférieure à la sagesse théorique ait une autorité supérieure à celle de cette dernière, puisque l’art qui \\
produit une chose quelconque gouverne et régit tout ce qui concerne cette chose. Telles sont donc les questions que nous devons discuter, car jusqu’ici nous n’avons fait que poser des problèmes.\par
 D’abord nous soutenons que la sagesse et la prudence sont nécessairement désirables en elles-mêmes, en tant du moins qu’elles sont vertus respectives de chacune des deux parties de l’âme, et cela même si ni l’une ni l’autre ne produisent rien. — Secondement, ces vertus produisent en réalité quelque chose, non pas au sens où la médecine produit la santé, mais au \\
sens où l’état de santé est cause de la santé : c’est de cette façon que la sagesse produit le bonheur, car étant une partie de la vertu totale, par sa simple possession et par son exercice elle rend un homme heureux.\par
En outre, l’œuvre propre de l’homme n’est complètement achevée qu’en conformité avec la prudence aussi bien qu’avec la vertu morale : la vertu morale, en effet, assure la rectitude du but que nous poursuivons, et la prudence celle des moyens pour parvenir à ce but. — Quant à la quatrième partie de l’âme, \\
la nutritive, elle n’a aucune vertu de cette sorte, car son action ou son inaction n’est nullement en son pouvoir.\par
En ce qui regarde enfin le fait que la prudence ne nous rend en rien plus aptes à accomplir les actions nobles et justes, il nous faut reprendre d’un peu plus haut en partant d’un principe qui est le suivant. De même que nous disons de certains qui accomplissent des actions justes, qu’ils ne sont pas encore des hommes justes, ceux qui font, par exemple, ce qui est \\
prescrit par les lois, soit malgré eux, soit par ignorance, soit pour tout autre motif, mais non pas simplement en vue d’accomplir l’action (bien qu’ils fassent assurément ce qu’il faut faire, et tout ce que l’homme vertueux est tenu de faire), ainsi, semble-t-il bien, il existe un certain état d’esprit dans lequel on accomplit ces différentes actions de façon à être homme de bien, je veux dire qu’on les fait par choix délibéré et \\
en vue des actions mêmes qu’on accomplit. Or la vertu morale assure bien la rectitude du choix, mais accomplir les actes tendant naturellement à la réalisation de la fin que nous avons choisie, c’est là une chose qui ne relève plus de la vertu, mais d’une autre potentialité. — Mais il nous faut insister sur ce point et parler plus clairement. Il existe une certaine puissance, \\
appelée {\itshape habileté}, et celle-ci est telle qu’elle est capable de faire les choses tendant au but que nous nous proposons et de les atteindre. Si le but est noble, c’est une puissance digne d’éloges, mais s’il est pervers, elle n’est que rouerie, et c’est pourquoi nous appelons {\itshape habiles} les hommes prudents aussi bien que les roués. La prudence n’est pas la puissance dont nous parlons, mais elle n’existe pas sans cette puissance, Mais \\
ladite disposition ne se réalise pas pour cet « œil de l’âme » sans l’aide de la vertu : nous l’avons dit, et cela est d’ailleurs évident. En effet, les syllogismes de l’action ont comme principe : « Puisque la fin, c’est-à-dire le Souverain Bien, est de telle nature », (quoi que ce puisse être d’ailleurs, et nous pouvons prendre à titre d’exemple la première chose venue) ; mais ce Souverain Bien ne se manifeste qu’aux yeux de \\
l’homme de bien : car la méchanceté fausse l’esprit et nous induit en erreur sur les principes de la conduite. La conséquence évidente, c’est l’impossibilité d’être prudent sans être vertueux.\par
 Examinons donc de nouveau encore la nature de la vertu.\par
Le cas de la vertu est, en effet, voisin de celui de la prudence dans ses rapports avec l’habileté. Sans qu’il y ait à cet égard identité, il y a du moins ressemblance, et la vertu naturelle entretient un rapport de même sorte avec la vertu au sens strict. Tout le monde admet, en effet que chaque type de caractère appartient à son possesseur en quelque sorte par \\
nature (car nous sommes justes, ou enclins à la tempérance, ou braves, et ainsi de suite, dès le moment de notre naissance). Mais pourtant nous cherchons quelque chose d’autre, à savoir le bien au sens strict, et voulons que de telles qualités nous appartiennent d’une autre façon. En effet, même les enfants et les bêtes possèdent les dispositions naturelles, mais, faute d’être accompagnées de raison, ces dispositions nous apparaissent \\
comme nocives. De toute façon, il y a une chose qui tombe semble-t-il sous le sens, c’est que, tout comme il arrive à un organisme vigoureux mais privé de la vue, de tomber lourdement quand il se meut, parce qu’il n’y voit pas, ainsi en est-il dans le cas des dispositions dont nous parlons ; une fois au contraire que la raison est venue, alors dans le domaine de l’action morale c’est un changement radical, et la disposition qui n’avait jusqu’ici qu’une ressemblance avec la vertu sera alors vertu au sens strict. Par conséquent, de même que pour la \\
partie opinante on distingue deux sortes de qualités, l’habileté et la prudence, ainsi aussi pour la partie morale de l’âme il existe deux types de vertus, la vertu naturelle et la vertu proprement dite, et de ces deux vertus la vertu proprement dite ne se produit pas sans être accompagnée de prudence. C’est pourquoi certains prétendent que toutes les vertus sont des formes de prudence, et Socrate, dans sa méthode d’investigation, avait raison en un sens et tort en un autre : en pensant que toutes les \\
vertus sont des formes de la prudence, il commettait une erreur, mais en disant qu’elles ne pouvaient exister sans la prudence, il avait entièrement raison. Et la preuve, c’est que tout le monde aujourd’hui144, en définissant la vertu, après avoir indiqué la disposition qu’elle est et précisé les choses qu’elle a pour objet, ajoute qu’elle est une disposition {\itshape conforme à la droite règle}, et la droite règle est celle qui est selon la prudence. Il apparaît dès lors que tous les hommes pressentent en quelque sorte obscurément \\
que la disposition présentant ce caractère est vertu, je veux dire la disposition selon la prudence.\par
Mais il nous faut aller un peu plus loin : ce n’est pas seulement la disposition {\itshape conforme} à la droite règle qui est vertu, il faut encore que la disposition soit intimement {\itshape unie} à la droite règle : or dans ce domaine la prudence est une droite règle. — Ainsi donc, Socrate pensait que les vertus sont des règles (puisqu’elles sont toutes selon lui des formes de science), tandis que, à notre avis à nous, les vertus sont intimement unies à une règle.\par
\\
On voit ainsi clairement, d’après ce que nous venons de dire, qu’il n’est pas possible d’être homme de bien au sens strict, sans prudence, ni prudent sans la vertu morale. Mais en outre on pourrait de cette façon réfuter l’argument dialectique qui tendrait à établir que les vertus existent séparément les unes des autres, sous prétexte que le même homme n’est pas naturellement \\
le plus apte à les pratiquer toutes, de sorte qu’il aura déjà acquis l’une et n’aura pas encore acquis l’autre. Cela est assurément possible pour ce qui concerne les vertus naturelles ; par contre, en ce qui regarde celles auxquelles nous devons le  nom d’homme de bien proprement dit, c’est une chose impossible, car en même temps que la prudence, qui est une seule vertu, toutes les autres seront données. — Et il est clair que, même si la prudence n’avait pas de portée pratique, on aurait tout de même besoin d’elle, parce qu’elle est la vertu de cette partie de l’intellect à laquelle elle appartient ; et aussi, que le choix délibéré ne sera pas correct sans prudence, pas plus que \\
sans vertu morale, car la vertu morale est ordonnée à la fin, et la prudence nous fait accomplir les actions conduisant à la fin.\par
Il n’en est pas moins vrai que la prudence ne détient pas la suprématie sur la sagesse théorique, c’est-à-dire sur la partie meilleure de l’intellect, pas plus que l’art médical n’a la suprématie sur la santé : l’art médical ne dispose pas de la santé, mais veille à la faire naître ; il formule donc des prescriptions {\itshape en vue} de \\
la santé, mais non {\itshape à} elle. En outre, on pourrait aussi bien dire que la politique gouverne les dieux, sous prétexte que ses prescriptions s’appliquent à toutes les affaires de la cité.
\section[{Livre VII}]{Livre VII}\renewcommand{\leftmark}{Livre VII}

\subsection[{1 (1145a — 1145b) < Vice, intempérance, bestialité, et leurs contraires >}]{1 (1145a — 1145b) < Vice, intempérance, bestialité, et leurs contraires >}
\noindent \\
Après cela, il nous faut établir, en prenant un autre point de départ, qu’en matière de moralité les attitudes à éviter sont de trois espèces : vice, intempérance [ακρασια]145, bestialité. Les états contraires aux deux premiers sautent aux yeux (nous appelons l’un vertu, et l’autre tempérance) ; mais à la bestialité on pourrait le plus justement faire correspondre la vertu surhumaine, \\
sorte de vertu héroïque et divine, comme Homère a représenté Priam qualifiant Hector de parfaitement vertueux,\par
 {\itshape Et il ne semblait pas} \par
 {\itshape Être enfant d’un homme mortel, mais d’un dieu.} \par
Par conséquent, si, comme on le dit, les hommes deviennent des dieux par excès de vertu, c’est ce caractère que \\
revêtira évidemment la disposition opposée à la bestialité : de même, en effet, qu’une bête brute n’a ni vice ni vertu, ainsi en est-il d’un dieu : son état est quelque chose de plus haut que la vertu, et celui de la brute est d’un genre tout différent du vice. Et puisqu’il est rare d’être un homme {\itshape divin}, au sens habituel donné à ce terme par les Lacédémoniens quand ils admirent profondément quelqu’un (un {\itshape homme divin} disent-ils), ainsi \\
également la bestialité est rare dans l’espèce humaine : c’est principalement chez les barbares qu’on la rencontre, mais elle se montre aussi parfois comme le résultat de maladies ou de difformités ; et nous appelons encore de ce terme outrageant les hommes qui surpassent les autres en vice. Mais la disposition dont nous parlons doit faire ultérieurement l’objet d’une mention de notre part, et le vice, de son côté, a été étudié plus \\
haut ; nous devons pour le moment parler de l’intempérance et de la mollesse ou sensualité, ainsi que de la tempérance et de  l’endurance : aucune de ces deux classes de dispositions ne doit en effet être conçue comme identique à la vertu ou au vice, ni pourtant comme étant d’un genre différent. Et nous devons, comme dans les autres matières, poser devant nous les faits tels qu’ils apparaissent, et après avoir d’abord exploré les problèmes, arriver ainsi à prouver le mieux possible la vérité de \\
toutes les opinions communes concernant ces affections de l’âme, ou tout au moins des opinions qui sont les plus répandues et les plus importantes, car si les objections soulevées sont résolues pour ne laisser subsister que les opinions communes, notre preuve aura suffisamment rempli son objet.
\subsection[{2 (1145b) < Énumération des opinions communes à vérifier >}]{2 (1145b) < Énumération des opinions communes à vérifier >}
\noindent On est généralement d’accord sur les points suivants : 1- la tempérance comme l’endurance font partie des états vertueux et louables, et, d’autre part, l’intempérance aussi bien que la mollesse rentrent dans les états à la fois pervers et \\
blâmables. 2- L’homme tempérant se confond avec celui qui s’en tient fermement à son raisonnement, et l’homme intempérant est celui qui est enclin à s’en écarter. 3- L’intempérant sachant que ce qu’il fait est mal, le fait par passion, tandis que le tempérant, sachant que ses appétits sont pervers, refuse de les suivre, par la règle qu’il s’est donnée. \\
4- L’homme modéré est toujours un homme tempérant et endurant, tandis que l’homme tempérant et endurant n’est toujours modéré qu’au sentiment de certains à l’exclusion des autres : les uns identifient l’homme déréglé avec l’intempérant, et l’intempérant avec l’homme déréglé, en les confondant ensemble, tandis que les autres les distinguent. 5- Quant à l’homme prudent, tantôt on prétend qu’il ne lui est pas possible d’être intempérant, tantôt au contraire que certains hommes, tout en étant prudents et habiles, sont intempérants. 6- De \\
plus, on dit qu’il y a des hommes intempérants même en ce qui concerne colère, honneur et gain.
\subsection[{3 (1145b — 1146b) < Examen des apories >}]{3 (1145b — 1146b) < Examen des apories >}
\noindent Voilà donc les propositions que l’on pose d’ordinaire.\par
Mais on peut se demander comment un homme jugeant avec rectitude verse dans l’intempérance. Quand on a la science, cela n’est pas possible, au dire de certains, car il serait étrange, ainsi que Socrate le pensait, qu’une science résidant en quelqu’un pût se trouver sous le pouvoir d’une autre force et \\
tirée en tous sens à sa suite comme une esclave. Socrate, en effet, combattait à fond cette façon de penser, dans l’idée qu’il n’existe pas d’intempérance, puisque personne, selon lui, exerçant son jugement, n’agit contrairement à ce qu’il croit être le meilleur parti ; ce serait seulement par ignorance qu’on agit ainsi. — Or la théorie socratique est visiblement en désaccord avec les faits, et nous devons nous livrer à des recherches sur l’attitude en question. Si on agit ainsi par ignorance, il faut \\
voir quelle sorte d’ignorance est en jeu (que l’homme, en effet, qui tombe dans l’intempérance ne croie pas, avant de se livrer à sa passion, qu’il devrait agir ainsi, c’est là une chose évidente). — Mais il y a des auteurs qui n’acceptent la doctrine socratique que sur certains points, et rejettent les autres. Que rien ne soit plus fort que la science, ils raccordent volontiers, mais qu’un homme n’agisse jamais à l’encontre de ce que l’opinion lui présente comme meilleur, ils refusent de l’admettre, et pour cette raison prétendent que l’intempérant \\
n’est pas en possession d’un véritable savoir quand il est asservi à ses plaisirs, mais seulement d’une opinion. Nous répondons que si c’est bien une opinion et non une science, si ce n’est pas une forte conviction qui oppose de la résistance,  mais seulement une conviction débile, semblable à celle de l’homme qui hésite entre deux partis, nous ne pouvons que nous montrer indulgent envers celui qui sent fléchir ses opinions en face de puissants appétits ; et pourtant, en fait, la méchanceté ne rencontre chez nous aucune indulgence, pas plus qu’aucun autre état digne de blâme. — Est-ce alors quand c’est la prudence qui oppose de la résistance ? car c’est elle \\
le plus fort de tous les états dont nous parlons. Mais cela est absurde : le même homme serait en même temps prudent et intempérant, alors que personne ne saurait prétendre qu’un homme prudent est propre à commettre volontairement les actions les plus viles. En outre, nous avons montré plus haut que l’homme prudent est celui qui est apte à agir (puisque c’est un homme engagé dans les faits particuliers) et qui possède les autres vertus.\par
\\
De plus, si la tempérance implique la possession d’appétits puissants et pervers, l’homme modéré ne sera pas tempérant, ni l’homme tempérant modéré, car le propre d’un homme modéré c’est de n’avoir ni appétits excessifs, ni appétits pervers. Mais l’homme tempérant, lui, doit en avoir, car si ses appétits sont bons, la disposition qui le détourne de les suivre \\
sera mauvaise, et ainsi la tempérance ne sera pas toujours elle-même bonne ; si, au contraire, les appétits sont débiles sans être pervers, il n’y aura rien de glorieux à les vaincre, ni s’ils sont pervers et débiles, rien de remarquable.\par
De plus, si la tempérance rend capable de demeurer ferme dans toute opinion quelle qu’elle soit, elle est mauvaise dans le cas par exemple où elle fait persister même dans une opinion erronée ; et si l’intempérance, par contre, rend apte à se dégager de toute opinion quelle qu’elle soit, il y aura une intempérance vertueuse, dont le Néoptolème de Sophocle, dans le \\
{\itshape Philoctète}, est un exemple : on doit l’approuver, en effet, de ne pas persister dans une résolution inspirée par Ulysse, à cause de sa répugnance pour le mensonge.\par
En outre, il y a l’aporie provenant de l’argument sophistique que voici. — Du fait que les Sophistes veulent enfermer leur adversaire dans des propositions contraires aux opinions communes, de façon à montrer leur habileté en cas de succès, le syllogisme qui en résulte aboutit à une aporie : la pensée, \\
en effet, est enchaînée quand, d’une part, elle ne veut pas demeurer où elle est parce que la conclusion ne la satisfait pas, et que, d’autre part, elle est incapable d’aller de l’avant parce qu’elle ne peut résoudre l’argument qui lui est opposé. — Or de l’un de ces arguments il suit que la folie combinée avec de l’intempérance est une vertu : on accomplit le contraire de ce qu’on juge devoir faire, grâce à l’intempérance, et, d’un autre côté, on juge que ce qui est bon est mauvais et qu’on ne doit pas \\
le faire ; et le résultat sera ainsi qu’on accomplira ce qui est bon et non ce qui est mauvais.\par
En outre, l’homme qui, par conviction, accomplit et poursuit ce qui est agréable et le choisit librement, peut être considéré comme meilleur que celui qui agit de même, non pas par calcul mais par intempérance. Il est plus facile, en effet, de guérir le premier, du fait qu’on peut le persuader de changer de conviction ; au contraire, l’intempérant se verra appliquer le \\
proverbe qui dit : {\itshape Quand l’eau vous étouffe, que faut-il boire par-dessus} ? Car si l’intempérant avait la conviction qu’il doit  faire ce qu’il fait, assurément sa conviction une fois modifiée, il cesserait de le faire, mais, en réalité, tout en étant convaincu, il n’en fait pas moins des choses toutes différentes.\par
Enfin, si l’intempérance a rapport à toutes sortes d’objets, et la tempérance également, quel homme est intempérant purement et simplement ? Personne, en effet, n’a toutes les formes d’intempérance, et pourtant nous disons que certains \\
sont intempérants d’une façon absolue.
\subsection[{4 (1146b) < Solution des apories — Tempérance et connaissance >}]{4 (1146b) < Solution des apories — Tempérance et connaissance >}
\noindent Les apories se présentent donc sous les différentes formes que nous venons d’indiquer : certains de ces points doivent être résolus, et les autres laissés debout, car résoudre l’aporie c’est découvrir le vrai.\par
En premier lieu il faut examiner si l’homme intempérant agit sciemment ou non, et, si c’est sciemment, en quel sens \\
il sait : ensuite, quelles sortes d’objets devons-nous poser comme rentrant dans la sphère de l’homme intempérant et de l’homme tempérant, je veux dire s’il s’agit de toute espèce de plaisirs et de peines, ou seulement de certaines espèces déterminées ; et si l’homme tempérant est identique à l’homme endurant, ou s’il est autre ; et pareillement en ce qui concerne les autres questions de même espèce que la présente étude.\par
Un point de départ pour notre examen est de savoir si \\
l’homme tempérant, ainsi que l’homme intempérant, sont différenciés par leurs objets ou par leur façon de se comporter, autrement dit si l’homme intempérant est intempérant simplement par rapport à tels ou tels objets, ou si ce n’est pas plutôt parce qu’il se comporte de telle manière, ou si ce n’est pas plutôt encore pour ces deux raisons à la fois. Ensuite, nous nous demanderons si l’intempérance et la tempérance s’étendent à la conduite tout entière, ou seulement à certaine partie de celle-ci. L’homme, en effet, qui est intempérant au sens absolu n’est pas tel par rapport à tout objet quel qu’il soit, mais \\
seulement par rapport aux choses 149 où se révèle l’homme déréglé. Il n’est pas non plus caractérisé par le fait d’avoir simplement rapport à ces choses (car alors son état se confondrait avec le dérèglement) mais par le fait d’être avec elles dans un rapport d’une certaine espèce : l’homme déréglé, en effet, est conduit à satisfaire ses appétits par un choix délibéré, pensant que son devoir est de toujours poursuivre le plaisir présent ; l’homme intempérant, au contraire, n’a aucune pensée de ce genre, mais poursuit néanmoins le plaisir.
\subsection[{5 (1146b — 1147b) < Solution de l’aporie sur les rapports de la science et de la tempérance >}]{5 (1146b — 1147b) < Solution de l’aporie sur les rapports de la science et de la tempérance >}
\noindent La doctrine d’après laquelle c’est en réalité à l’encontre \\
d’une opinion vraie et non d’un savoir véritablement scientifique que nous agissons dans l’intempérance, cette doctrine ne présente aucun intérêt pour notre raisonnement. (Certains, en effet, de ceux qui professent une opinion n’ont aucune hésitation et croient posséder une connaissance exacte ; si donc on prétend que c’est grâce à la faiblesse de leur conviction que ceux qui ont une simple opinion sont plus portés à agir à l’encontre de leur conception du bien que ceux qui possèdent la science, il n’y aura aucune différence à cet égard entre science et opinion, puisque certains hommes ne sont pas moins \\
convaincus des choses dont ils ont l’opinion que d’autres des choses dont ils ont la science, et on peut le voir par l’exemple d’Héraclite).\par
Mais, puisque le terme {\itshape avoir la science} se prend en un double sens (car celui qui possède la science, mais ne l’utilise pas, et celui qui l’utilise actuellement, sont dits l’un et l’autre, avoir la science), il y aura une différence entre un homme qui, possédant la science mais ne l’exerçant pas, fait ce qu’il ne faut pas faire, et un autre qui fait de même en possédant la science et \\
en l’exerçant : ce dernier cas paraît inexplicable, mais il n’en est plus de même s’il s’agit d’une science ne s’exerçant pas.\par
 En outre, puisqu’il y a deux sortes de prémisses, rien n’empêche qu’un homme en possession des deux prémisses ensemble, n’agisse contrairement à la science qu’il a, pourvu toutefois qu’il utilise la prémisse universelle et non la prémisse particulière : car ce qui est l’objet de l’action, ce sont les actes singuliers. — Il y a aussi une distinction à établir pour le terme \\
universel : un universel est prédicable de l’agent lui-même, et l’autre de l’objet. Par exemple : {\itshape les aliments secs sont bons pour tout homme}, et : {\itshape je suis un homme}, ou : {\itshape telle espèce d’aliments est sèche}. Mais si c’est : {\itshape cette nourriture que voici est de telle sorte}, l’homme intempérant n’en possède pas la science, ou n’en a pas la science en exercice. Dès lors, entre ces deux modes de savoir, il y aura une différence tellement considérable qu’on ne verra rien de surprenant à ce que l’homme intempérant connaisse d’une certaine façon, tandis que connaître d’une autre façon serait extraordinaire.\par
\\
De plus, la possession de la science en un autre sens encore que ceux dont nous avons parlé, peut se rencontrer chez l’homme : car même dans la possession de la science indépendamment de son utilisation, nous observons une différence de disposition, de sorte qu’on peut avoir la science en un sens et ne pas l’avoir, comme dans le cas de l’homme en sommeil, ou fou, ou pris de vin. Or c’est là précisément la condition \\
de ceux qui sont sous l’influence de la passion, puisque les accès de colère, les appétits sexuels et quelques autres passions de ce genre, de toute évidence altèrent également l’état corporel, et même dans certains cas produisent la folie. Il est clair, par conséquent, que la possession de la science chez l’homme intempérant doit être déclarée de même nature que pour ces gens-là. Le fait pour les intempérants de parler le langage découlant de la science n’est nullement un signe qu’ils la possèdent : car même ceux qui se trouvent dans les états affectifs que nous avons indiqués répètent machinalement des \\
démonstrations de géométrie ou des vers d’Empédocle, et ceux qui ont commencé à apprendre une science débitent tout d’une haleine ses formules, quoiqu’ils n’en connaissent pas encore la signification : la science, en effet, doit s’intégrer à leur nature, mais cela demande du temps. Par suite c’est par comparaison avec le langage des histrions que nous devons apprécier celui qu’emploient les hommes qui versent dans l’intempérance.\par
De plus, voici encore de quelle façon, en nous plaçant \\
sur le terrain des faits, nous pouvons considérer la cause de l’intempérance. La prémisse universelle est une opinion, et l’autre a rapport aux faits particuliers, où la perception dès lors est maîtresse. Or quand les deux prémisses engendrent une seule proposition, il faut nécessairement que, dans certains cas, l’âme affirme la conclusion, et que dans le cas de prémisses relatives à la production, l’action suive immédiatement. Soit, par exemple les prémisses : {\itshape il faut goûter à tout ce qui est doux}, et : {\itshape ceci est doux} (au sens d’être une chose douce particulière) : \\
il faut nécessairement que l’homme capable d’agir et qui ne rencontre aucun empêchement, dans le même temps accomplisse aussi l’acte. Quand donc, d’un côté, réside dans l’esprit l’opinion universelle nous défendant de goûter, et que, d’autre part, est présente aussi l’opinion que {\itshape tout ce qui est doux est agréable} et que {\itshape ceci est doux} (cette dernière opinion déterminant l’acte), et que l’appétit se trouve également présent en nous, alors, si la première opinion universelle nous invite bien à fuir l’objet, par contre l’appétit nous y conduit (puisqu’il est \\
capable de mettre en mouvement chaque partie du corps) : il en résulte, par conséquent, que c’est sous l’influence d’une règle  en quelque sorte ou d’une opinion qu’on devient intempérant, opinion qui est contraire, non pas en elle-même mais seulement par accident (car c’est l’appétit qui est réellement contraire, et non l’opinion), à la droite règle. Une autre conséquence en découle encore : la raison pour laquelle on ne peut parler d’intempérance pour les bêtes, c’est qu’elles ne possèdent pas de jugement portant sur les universels, mais qu’elles ont \\
seulement image et souvenir des choses particulières.\par
Quant à dire comment l’ignorance de l’homme intempérant se résout pour faire place de nouveau à l’état de savoir, l’explication est la même que pour un homme pris de vin ou en sommeil, et n’est pas spéciale à l’état dont nous traitons : nous devons nous renseigner à cet effet auprès de ceux qui sont versés dans la science de la nature.\par
Mais la dernière prémisse étant une opinion qui à la fois \\
porte sur un objet sensible et détermine souverainement nos actes, cette opinion-là, un homme sous l’empire de la passion, ou bien ne la possède pas du tout, ou bien ne la possède qu’au sens où, comme nous l’avons dit, posséder la science veut dire seulement parler machinalement, à la façon dont l’homme pris de vin récite les vers d’Empédocle. Et du fait que le dernier terme n’est pas un universel, ni considéré comme étant un objet de science semblablement à un universel, on est, semble-t-il, amené logiquement à la conclusion que Socrate cherchait \\
à établir : en effet, ce n’est pas en la présence de ce qui est considéré comme la science au sens propre que se produit la passion dont il s’agit, pas plus que ce n’est la vraie science qui est tiraillée par la passion, mais c’est lorsque est présente la connaissance sensible.
\subsection[{6 (1147b — 1149a) < Domaine de l’intempérance. Les diverses formes : l’intempérance simpliciter et l’intempérance secundum quid >}]{6 (1147b — 1149a) < Domaine de l’intempérance. Les diverses formes : l’intempérance {\itshape simpliciter} et l’intempérance {\itshape secundum quid} >}
\noindent Si l’intempérance peut ou non s’accompagner de savoir, et, le cas échéant, de quel genre de savoir, c’est là une question qui a été suffisamment traitée.\par
\\
Mais peut-on être intempérant purement et simplement, ou doit-on toujours l’être par rapport à certaines choses particulières, et, dans l’affirmative, à quelles sortes de choses ? C’est une question à discuter maintenant.\par
Que les plaisirs et les peines rentrent dans la sphère d’action à la fois des hommes tempérants et des hommes endurants ainsi que des hommes intempérants et des hommes adonnés à la mollesse, c’est là une chose évidente. Or, parmi les choses qui donnent du plaisir, les unes sont nécessaires, et \\
les autres sont souhaitables en elles-mêmes mais susceptibles d’excès. Sont nécessaires les causes corporelles de plaisir (j’entends par là, à la fois celles qui intéressent la nutrition et les besoins sexuels, en d’autres termes ces fonctions corporelles que nous avons posées comme étant celles qui constituent la sphère du dérèglement et de la modération) ; les autres causes de plaisirs ne sont pas nécessaires, mais sont souhaitables \\
en elles-mêmes (par exemple, la victoire, l’honneur, la richesse, et autres biens et plaisirs de même sorte). Ceci posé, les hommes qui tombent dans l’excès en ce qui concerne ce dernier groupe de plaisirs, contrairement à la droite règle qui est en eux, nous ne les appelons pas des intempérants au sens strict, mais nous ajoutons une spécification et disons qu’ils sont intempérants en matière d’argent, de gain, d’honneur ou de colère, et non simplement intempérants, attendu qu’ils sont différents des gens intempérants proprement dits et qu’ils \\
ne reçoivent ce nom que par similitude (comme dans le cas d’Anthropos, vainqueur aux Jeux olympiques : la définition  générale de l’homme différait peu de la notion qui lui était propre, mais elle était néanmoins autre). — En voici une preuve : nous blâmons l’intempérance non comme une erreur seulement, mais comme une sorte de vice, qu’il s’agisse de l’intempérance pure et simple ou de l’intempérance portant sur quelque plaisir < corporel > particulier, tandis que nous ne blâmons aucun intempérant de l’autre classe. — Mais parmi les hommes dont l’intempérance a rapport aux jouissances corporelles, \\
jouissances qui, disons-nous, rentrent dans la sphère de l’homme modéré et de l’homme déréglé, celui qui, à la fois, poursuit les plaisirs excessifs et évite les peines du corps comme la pauvreté, la soif, la chaleur, le froid et toutes les sensations pénibles du toucher et du goût, et cela non pas par choix réfléchi mais contrairement à son choix et à sa raison, \\
celui-là est appelé intempérant, non pas avec la spécification qu’il est intempérant en telle chose, la colère par exemple, mais intempérant au sens strict seulement. Et une preuve, c’est qu’on parle de mollesse, seulement en ce qui regarde ces plaisirs et jamais en ce qui regarde les autres. Et c’est pour cette raison que nous plaçons dans le même groupe l’intempérant et le déréglé, le tempérant et le modéré, à l’exclusion de tous les autres, parce qu’ils ont pour sphère d’activité, en quelque sorte \\
les mêmes plaisirs et les mêmes peines. Mais, tout en s’intéressant aux mêmes objets, leur comportement à l’égard de ces objets n’est pas le même, les uns agissant par choix délibéré, et les autres en dehors de tout choix. Aussi donnerions-nous le nom de déréglé à l’homme qui, sans concupiscence ou n’en éprouvant qu’une légère, poursuit les plaisirs excessifs et évite les peines modérées plutôt qu’à celui qui en fait autant sous \\
l’empire de violents appétits : car que ne ferait pas le premier si un appétit juvénile ou un chagrin violent venait s’ajouter en lui quand il se verrait privé des plaisirs nécessaires ? Parmi les appétits et les plaisirs, les uns appartiennent à la classe des choses génériquement nobles et bonnes (car certaines choses agréables sont naturellement dignes de choix, tandis que d’autres leur sont contraires, et les autres, enfin, sont intermédiaires, conformément à nos distinctions antérieures) : \\
tels sont l’argent, le gain, la victoire, l’honneur. Et en ce qui concerne toutes ces choses-là et celles de même sorte, ainsi que celles qui sont intermédiaires, ce n’est pas le fait d’être affecté par elles, ou de les désirer, ou de les aimer, qui rend l’homme blâmable, mais c’est le fait de les aimer d’une certaine façon, autrement dit avec excès. C’est pourquoi tous ceux qui, en violation de la règle, ou bien se laissent dominer par l’une des choses naturellement nobles et bonnes, ou bien les recherchent \\
trop, par exemple ceux qui montrent plus d’ardeur qu’il ne faudrait pour l’honneur, ou pour leurs enfants ou leurs parents, < ne sont pas des hommes pervers > (car ces objets font partie des biens, et on approuve ceux qui s’y attachent avec zèle ; mais cependant il y a un excès même dans ce domaine, si par exemple comme Niobé on luttait contre les dieux eux-mêmes, ou si on avait pour son père une affection semblable à celle  de Satyros, surnommé Philopator, dont l’exagération sur ce point passait pour de la folie). — Il n’y a donc aucune perversité en ce qui regarde ces objets de notre attachement, pour la raison que nous avons indiquée, à savoir que chacun d’eux est naturellement digne de choix en lui-même, bien que l’excès soit ici condamnable et doive être évité. Et pareillement, il ne \\
saurait y avoir non plus d’intempérance à leur sujet (car l’intempérance n’est pas seulement une chose qu’on doit éviter, c’est aussi une chose qui fait partie des actions blâmables) ; seulement, par similitude, on emploie le terme {\itshape intempérance} en y ajoutant une spécification dans chaque cas, tout comme on qualifie de mauvais médecin ou de mauvais acteur celui qu’on ne pourrait pas appeler mauvais au sens propre. De même donc que dans ces exemples nous n’appliquons pas le terme {\itshape mauvais} sans spécification, parce que l’insuffisance \\
du médecin ou de l’acteur n’est pas un vice mais lui ressemble seulement par analogie, ainsi il est clair que, dans l’autre cas également, seule doit être considérée comme étant véritablement intempérance ou tempérance celle qui a rapport aux mêmes objets que la modération et le dérèglement, et que nous n’appliquons à la colère que par similitude ; et c’est pourquoi, ajoutant une spécification, nous disons {\itshape intempérant dans la colère}, comme nous disons {\itshape intempérant dans l’honneur} ou {\itshape le gain}.\par
\\
Certaines choses sont agréables par leur nature, les unes d’une façon absolue, et les autres pour telle classe d’animaux ou d’hommes ; d’autres choses, par contre, ne sont pas agréables par nature, mais le deviennent soit comme conséquence d’une difformité, soit par habitude ; d’autres enfin le sont par dépravation naturelle. Ceci posé, il est possible, pour chacune de ces dernières espèces de plaisirs, d’observer des dispositions du caractère correspondantes. J’entends par là les dispositions \\
bestiales, comme dans l’exemple de la femme qui, dit-on, éventre de haut en bas les femmes enceintes et dévore leur fruit, ou encore ces horreurs où se complaisent, à ce qu’on raconte, certaines tribus sauvages des côtes du Pont, qui mangent des viandes crues ou de la chair humaine, ou échangent mutuellement leurs enfants pour s’en repaître dans leurs festins, ou enfin ce qu’on rapporte de Phalaris.\par
\\
Ce sont là des états de bestialité, mais d’autres ont pour origine la maladie (ou parfois la folie, comme dans le cas de l’homme qui offrit sa mère en sacrifice aux dieux et la mangea, ou celui de l’esclave qui dévora le foie de son compagnon) ; d’autres encore sont des propensions morbides résultant de l’habitude, comme par exemple s’arracher les cheveux, ronger ses ongles ou mêmes du charbon et de la terre, sans oublier l’homosexualité. Ces pratiques sont le résultat, dans certains \\
cas de dispositions naturelles, et dans d’autres de l’habitude, comme chez ceux dont on a abusé dès leur enfance.\par
Ceux chez qui la nature est la cause de ces dépravations, on ne saurait les appeler intempérants, pas plus qu’on ne qualifierait ainsi les femmes, sous le prétexte que dans la copulation leur rôle est passif et non actif ; il en va de même pour ceux qui sont dans un état morbide sous l’effet de l’habitude.\par
 La possession de ces diverses dispositions se situe hors des limites du vice, comme c’est aussi le cas pour la bestialité ; et quand on les a, s’en rendre maître ou s’y laisser asservir ne constitue pas < la tempérance ou > l’intempérance proprement dites, mais seulement ce qu’on appelle de ce nom par similitude, tout comme celui qui se comporte de cette façon dans ses colères doit être appelé {\itshape intempérant dans} ladite passion, et non intempérant au sens strict.\par
\\
En effet, tous excès d’insanité ou de lâcheté ou d’intempérance ou d’humeur difficile, sont soit des traits de bestialité, soit des états morbides. L’homme constitué naturellement de façon à avoir peur de tout, même du bruit d’une souris, est lâche d’une lâcheté tout animale, et celui qui avait la phobie des belettes était sous l’influence d’une maladie ; et parmi les \\
insensés, ceux qui sont naturellement privés de raison et vivent seulement par les sens, comme certaines tribus barbares éloignées, sont assimilables aux brutes, tandis que ceux qui ont perdu la raison à la suite de maladies, de l’épilepsie par exemple, ou par un accès de folie, sont des êtres morbides. Avec des penchants de ce genre, il peut se faire que l’on n’ait parfois qu’une simple disposition à les suivre, sans s’y laisser asservir, si, par exemple, Phalaris avait réprimé son désir de manger un jeune enfant ou de se livrer à des plaisirs sexuels contre nature ; mais il est possible également de s’abandonner à \\
ces penchants et ne pas se contenter de les avoir. De même donc que, dans le cas de la perversité, celle qui est sur le plan humain est appelée perversité au sens strict, tandis que l’autre espèce se voit ajouter la spécification de {\itshape bestiale} ou de {\itshape morbide}, mais n’est pas appelée perversité proprement dite, de la même façon il est évident que, dans le cas de l’intempérance, il y a celle qui est bestiale et celle qui est morbide, et que \\
l’intempérance au sens strict est seulement celle qui correspond au dérèglement proprement humain.
\subsection[{7 (1149a — 1150a) < Intempérance dans la colère et intempérance des appétits — La bestialité >}]{7 (1149a — 1150a) < Intempérance dans la colère et intempérance des appétits — La bestialité >}
\noindent Ainsi donc l’intempérance et la tempérance portent exclusivement sur les mêmes objets que le dérèglement et la modération, et, d’autre part, l’intempérance qui porte sur les autres objets est d’une espèce différente, appelée seulement ainsi par extension de sens et non au sens strict : tout cela est maintenant clair.\par
Que l’intempérance dans la colère soit moins déshonorante \\
que l’intempérance des appétits, c’est cette vérité que nous allons à présent considérer. — La colère, en effet, semble bien prêter jusqu’à un certain point l’oreille à la raison, mais elle entend de travers, à la façon de ces serviteurs pressés qui sortent en courant avant d’avoir écouté jusqu’au bout ce qu’on leur dit, et puis se trompent dans l’exécution de l’ordre, ou encore à la façon des chiens, qui avant même d’observer si c’est un ami, au moindre bruit qui se produit se mettent à \\
aboyer. Pareillement la colère, par sa chaleur et sa précipitation naturelles, tout en entendant n’entend pas un ordre, et s’élance pour assouvir sa vengeance. La raison ou l’imagination, en effet, présente à nos regards une insulte ou une marque de dédain ressenties, et la colère, après avoir conclu par une sorte de raisonnement que notre devoir est d’engager les hostilités contre un pareil insulteur, éclate alors brusquement ; l’appétit, \\
au contraire, dès que la raison ou la sensation a seulement dit  qu’une chose est agréable, s’élance pour en jouir. Par conséquent, la colère obéit à la raison en un sens, alors que l’appétit n’y obéit pas. La honte est donc plus grande dans ce dernier cas, puisque l’homme intempérant dans la colère est en un sens vaincu par la raison, tandis que l’autre l’est par l’appétit et non par la raison.\par
\\
En outre, on pardonne plus aisément de suivre les désirs naturels, puisque, même dans le cas des appétits on pardonne plus facilement de les suivre quand ils sont communs à tous les hommes, et cela dans la mesure même où ils sont communs. Or la colère et l’humeur difficile sont une chose plus naturelle que les appétits portant sur des plaisirs excessifs et qui n’ont rien de nécessaire. Donnons comme exemple l’homme qui, en réponse à l’accusation de frapper son père, disait : « Mais lui aussi a frappé le sien, et le père de mon père aussi ! », et désignant \\
son petit garçon : « Et celui-ci, dit-il, en fera autant quand il sera devenu un homme ! car c’est inné dans notre famille ». C’est encore l’histoire de l’homme qui, traîné par son fils hors de sa maison, lui demandait de s’arrêter à la porte, car lui-même n’avait traîné son père que jusque-là.\par
De plus, on est d’autant plus injuste qu’on use davantage de manœuvres perfides. Or l’homme violent n’a aucune perfidie, \\
ni non plus la colère, qui agit à visage découvert ; l’appétit, au contraire, est comme l’Aphrodite dont on dit :\par
{\itshape Cyprogeneia qui ourdit des embûches}, \par
et Homère, décrivant la ceinture brodée de la déesse :\par
 {\itshape Un conseil perfide, qui s’emparait de l’esprit du sage, si sensé fût-il.} \par
Par conséquent, si cette forme d’intempérance est plus injuste, elle est aussi plus honteuse que celle qui est relative à la colère, et elle est intempérance proprement dite, et vice en un sens.\par
\\
De plus, si nul ne fait subir un outrage avec un sentiment d’affliction, par contre tout homme agissant par colère agit en ressentant de la peine, alors que celui qui commet un outrage le fait avec accompagnement de plaisir. Si donc les actes contre lesquels une victime se met le plus justement en colère sont plus injustes que d’autres, l’intempérance causée par l’appétit est aussi plus injuste que l’intempérance de la colère, car il n’y a dans la colère aucun outrage.\par
Qu’ainsi donc l’intempérance relative à l’appétit soit plus \\
déshonorante que celle qui a rapport à la colère, et que la tempérance et l’intempérance aient rapport aux appétits et plaisirs du corps, c’est clair.\par
Mais parmi ces appétits et ces plaisirs eux-mêmes nous devons établir des distinctions. Ainsi, en effet, que nous l’avons dit en commençant, certains d’entre eux sont sur le plan humain et sont naturels à la fois en genre et en grandeur, d’autres ont un caractère bestial, et d’autres sont dus à des \\
difformités ou des maladies. Or c’est seulement aux plaisirs que nous avons nommés en premier lieu que la modération et le dérèglement ont rapport ; et c’est pourquoi nous ne disons pas des bêtes qu’elles sont modérées ou déréglées, sinon par extension de sens et seulement dans le cas où en totalité quelque espèce d’animaux l’emporte sur une autre en lascivité, en instincts destructeurs ou en voracité (les animaux. en effet, n’ont ni faculté de choisir, ni raisonnement) : ce sont là \\
des aberrations de la nature, tout comme les déments chez  les hommes. La bestialité est un moindre mal que le vice, quoique plus redoutable : non pas que la partie supérieure ait été dépravée, comme dans l’homme, mais elle est totalement absente. Par suite, c’est comme si, comparant une chose inanimée avec un être animé, on demandait lequel des deux l’emporte en méchanceté : car la perversité d’une chose qui n’a \\
pas en elle de principe d’action est toujours moins pernicieuse, et l’intellect est un principe de ce genre (c’est donc à peu près comme si on comparait l’injustice avec un homme injuste : chacun de ces deux termes est en un sens pire que l’autre) : car un homme mauvais peut causer infiniment plus de maux qu’une bête brute.
\subsection[{8 (1150a — 1150b) < Intempérance et mollesse, tempérance et endurance. L’impétuosité et la faiblesse >}]{8 (1150a — 1150b) < Intempérance et mollesse, tempérance et endurance. L’impétuosité et la faiblesse >}
\noindent À l’égard des plaisirs et des peines dues au toucher et au goût, ainsi que des appétits et des aversions correspondants, \\
toutes choses que nous avons définies plus haut comme rentrant dans la sphère à la fois du dérèglement et de la modération, il est possible de se comporter des deux façons suivantes : ou bien nous succombons même à des tentations que la majorité des hommes peut vaincre, ou bien, au contraire, nous triomphons même de celles où la plupart des hommes succombent. De ces deux dispositions, celle qui a rapport aux plaisirs est tempérance et intempérance, et celle qui a rapport \\
aux peines, mollesse et endurance. La disposition de la plupart des hommes tient le milieu entre les deux, même si, en fait, ils penchent davantage vers les états moralement plus mauvais.\par
Puisque parmi les plaisirs les uns sont nécessaires et les autres ne le sont pas, les premiers étant nécessaires seulement jusqu’à un certain point, alors que ni l’excès en ce qui les concerne, ni le défaut ne sont soumis à cette nécessité (et on peut en dire autant des appétits et des peines) : dans ces conditions, l’homme qui poursuit ceux des plaisirs qui dépassent la mesure, ou qui poursuit à l’excès des plaisirs nécessaires, et \\
cela par choix délibéré, et qui les poursuit pour eux-mêmes et nullement en vue d’un résultat distinct du plaisir, celui-là est un homme déréglé : car cet homme est nécessairement incapable de se repentir, et par suite il est incurable, puisque pour qui est impuissant à se repentir il n’y a pas de remède. — L’homme déficient dans la recherche du plaisir est l’opposé du précédent, et celui qui occupe la position moyenne un homme modéré. Pareillement encore, est un homme déréglé celui qui évite les peines du corps, non pas parce qu’il est sous \\
l’empire de la passion mais par choix délibéré. (De ceux qui agissent sans choix délibéré, les uns sont menés par le plaisir, les autres parce qu’ils veulent fuir la peine provenant de l’appétit insatisfait, ce qui entraîne une différence entre eux. Mais, au jugement de tout homme, si quelqu’un, sans aucune concupiscence ou n’en ressentant qu’une légère, commet quelque action honteuse, il est pire que s’il est poussé par de violents appétits, et s’il frappe sans colère, il est pire que s’il \\
frappe avec colère : que ne ferait pas, en effet, le premier, s’il était sous l’empire de la passion ? C’est pourquoi le déréglé est pire que l’intempérant). Donc, des états décrits ci-dessus, le dernier est plutôt une espèce de la mollesse, et l’autre est l’intempérance. — À l’homme intempérant est opposé l’homme tempérant, et à l’homme mou l’homme endurant : car l’endurance consiste dans le fait de résister, et la tempérance dans le \\
fait de maîtriser ses passions, et résister et maîtriser sont des notions différentes, tout comme éviter la défaite est différent de remporter la victoire ; et c’est pourquoi la tempérance est  une chose préférable à l’endurance. — L’homme qui manque de résistance à l’égard des tentations où la plupart des hommes à la fois tiennent bon et le peuvent, celui-là est un homme mou et voluptueux (et, en effet, la volupté est une sorte de mollesse), lequel laisse traîner son manteau pour éviter la peine de le relever, ou feint d’être malade, ne s’imaginant pas qu’étant \\
semblable à un malheureux il est lui-même malheureux.\par
Même observation pour la tempérance et l’intempérance. Qu’un homme, en effet, succombe sous le poids de plaisirs ou de peines violents et excessifs, il n’y a là rien de surprenant et il est même excusable s’il a succombé en résistant, à l’exemple de Philoctète dans Théodecte, quand il est mordu par la \\
vipère, ou de Cercyon dans l’{\itshape Alope} de Carcinos, ou encore de ceux qui, essayant de réprimer leur rire, éclatent d’un seul coup, mésaventure qui arriva à Xénophantos ; mais ce qui est surprenant, c’est qu’à l’égard de plaisirs ou de peines auxquels la plupart des gens sont capables de résister, un homme ait le dessous et ne puisse pas tenir bon quand cette faiblesse n’est pas due à l’hérédité ou à une maladie, comme c’est le cas chez \\
les rois scythes, où la mollesse tient à la race, ou encore pour l’infériorité physique qui distingue le sexe féminin du sexe masculin.\par
L’homme passionné pour l’amusement est considéré également comme un homme déréglé, mais c’est en réalité chez lui de la mollesse : car le jeu est une détente puisque c’est un repos, et c’est dans la classe de ceux qui pèchent par excès à cet égard que rentre l’amateur de jeu.\par
La première forme de l’intempérance est l’impétuosité, et \\
l’autre la faiblesse. Certains hommes, en effet, après qu’ils ont délibéré, ne persistent pas dans le résultat de leur délibération, et cela sous l’effet de la passion ; pour d’autres, au contraire, c’est grâce à leur manque de délibération qu’ils sont menés par la passion : certains, en effet (pareils en cela à ceux qui ayant pris les devants pour chatouiller ne sont pas eux-mêmes chatouillés), s’ils ont préalablement senti et vu ce qui va leur arriver, et s’ils ont auparavant pu donner l’éveil à eux-mêmes et à leur faculté de raisonner, ne succombent pas alors \\
sous l’effet de la passion, que ce soit un plaisir ou une peine. Ce sont surtout les hommes d’humeur vive et les hommes de tempérament excitable qui sont sujets à l’intempérance sous sa forme d’impétuosité : les premiers par leur précipitation, et les seconds par leur violence n’ont pas la patience d’attendre la raison, enclins qu’ils sont à suivre leur imagination.
\subsection[{9 (1150b — 1151a) < Intempérance et dérèglement >}]{9 (1150b — 1151a) < Intempérance et dérèglement >}
\noindent L’homme déréglé, comme nous l’avons dit, n’est pas \\
sujet au repentir (car il persiste dans son état par son libre choix), alors que l’homme intempérant est toujours susceptible de regretter ce qu’il fait. C’est pourquoi la position adoptée dans l’énoncé que nous avons donné du problème n’est pas exacte : au contraire, c’est l’homme déréglé qui est incurable, et l’homme intempérant qui est guérissable : car la perversité est semblable à ces maladies comme l’hydropisie ou la consomption, tandis que l’intempérance ressemble à l’épilepsie, la perversité étant un mal continu et l’intempérance un mal \\
intermittent. Effectivement l’intempérance et le vice sont d’un genre totalement différent : le vice est inconscient, alors que l’intempérance ne l’est pas.\par
 Parmi les intempérants eux-mêmes, les impulsifs valent mieux que ceux qui possèdent la règle mais n’y persistent pas : car ces derniers succombent sous une passion moins pressante, et en outre ne s’y abandonnent pas sans délibération préalable, comme le font les impulsifs : l’intempérant, en effet, est semblable à ceux qui s’enivrent rapidement et avec une faible quantité de vin, moindre qu’il n’en faut à la plupart des hommes.\par
\\
Qu’ainsi l’intempérance ne soit pas un vice, voilà qui est clair (quoiqu’elle le soit peut-être en un sens) ; car l’intempérance agit contrairement à son choix, et le vice conformément au sien. Mais cependant il y a ressemblance du moins dans leurs actions respectives, et comme disait Démodocos aux Milésiens :\par
 {\itshape Les Milésiens ne sont pas dénués d’intelligence, mais ils agissent tout à fait comme les imbéciles.} \par
\\
pareillement, les hommes intempérants ne sont pas des hommes injustes, mais ils commettent des actions injustes.\par
Puisque l’homme intempérant est constitué de telle sorte qu’il poursuit, sans croire pour autant qu’il a raison de le faire, les plaisirs corporels excessifs et contraires à la droite règle, tandis que l’homme déréglé est convaincu qu’il doit agir ainsi, et cela parce qu’il est constitué de façon à poursuivre ces plaisirs : il en résulte que c’est au contraire le premier qu’on peut aisément persuader de changer de conduite, alors que \\
pour le second ce n’est pas possible. En effet, la vertu et le vice, respectivement conservent et détruisent le principe, et dans le domaine de la pratique c’est la cause finale qui est principe, comme les hypothèses en mathématiques ; dès lors, pas plus dans les matières que nous traitons ici que dans les mathématiques, le raisonnement n’est apte à nous instruire des principes, mais c’est une vertu soit naturelle, soit acquise par l’habitude, qui nous fait avoir une opinion correcte au sujet du principe. L’homme répondant à cette description est par suite un homme modéré, et son contraire un homme déréglé.\par
\\
Mais il y a un genre d’hommes qui, sous l’influence de la passion, abandonne les voies de la droite règle ; c’est un homme que la passion domine au point de l’empêcher d’agir conformément à la droite règle, mais cette domination ne va cependant pas jusqu’à le rendre naturellement capable de croire que son devoir est de poursuivre en toute liberté les plaisirs dont nous parlons : c’est là l’homme intempérant, qui est meilleur que l’homme déréglé et qui n’est même pas \\
vicieux à proprement parler, puisque en lui est sauvegardé ce qu’il y a de plus excellent, je veux dire le principe. Opposé enfin à l’intempérant est un autre genre d’hommes : c’est celui qui demeure ferme et ne s’écarte pas du principe, sous l’effet du moins de la passion. Ces considérations montrent donc clairement que cette dernière disposition du caractère est bonne et que l’autre ne vaut rien.
\subsection[{10 (1151a — 1151b) < Tempérance et obstination >}]{10 (1151a — 1151b) < Tempérance et obstination >}
\noindent Est-ce donc qu’est tempérant celui qui demeure ferme \\
dans n’importe quelle règle et n’importe quel choix, ou seulement celui qui demeure ferme dans la droite règle, et, d’autre part, est-ce qu’est intempérant celui qui ne persiste pas dans n’importe quel choix et n’importe quelle règle, ou celui qui fait seulement abandon de la règle exempte de fausseté et du choix correct ? Telle était la façon dont le problème a été posé précédemment. — Ne serait-ce pas que, par accident, ce peut être une règle ou un choix quelconque, mais que, en soi, c’est seulement la règle vraie et le choix correct où le tempérant \\
persiste et où l’intempérant ne persiste pas ? Si, en effet, on  choisit ou poursuit telle chose en vue de telle autre chose, c’est cette dernière que par soi on poursuit et choisit, mais par accident c’est la première. Or {\itshape d’une façon absolue} a pour nous le sens de par soi. Par conséquent, en un sens c’est n’importe quelle opinion à laquelle le tempérant s’attache fermement et que l’intempérant abandonne, mais absolument parlant c’est seulement à celle qui est vraie.\par
\\
Mais il y a des personnes capables de persister dans leur opinion, qu’on appelle des {\itshape opiniâtres}, c’est-à-dire qui sont difficiles à convaincre et qu’on ne fait pas facilement changer de conviction. Ces gens-là présentent une certaine ressemblance avec l’homme tempérant, comme le prodigue ressemble à l’homme libéral, et le téméraire à l’homme sûr de lui : mais en réalité ils diffèrent de lui sous bien des aspects. L’homme tempérant, en effet, sous l’assaut de la passion et de la concupiscence \\
demeure inébranlable, mais il sera prêt, le cas échéant, à céder à la persuasion ; l’homme opiniâtre, au contraire, refuse de céder à la raison, car de telles gens ressentent des appétits et beaucoup d’entre eux sont menés par leurs plaisirs. Or parmi les opiniâtres on distingue les entêtés, les ignorants et les rustres : l’entêtement des premiers tient au plaisir ou à la peine \\
que leur propre attitude leur cause : ils se plaisent à chanter victoire quand on ne réussit pas à les faire changer d’opinion, et ils s’affligent quand leurs propres décisions deviennent nulles et non avenues, comme cela arrive pour des décrets ; aussi ressemblent-ils davantage à l’homme intempérant qu’à l’homme tempérant.\par
D’autre part, il y a des gens qui ne persistent pas dans leurs opinions pour une cause étrangère à l’intempérance, par exemple Néoptolème dans le {\itshape Philoctète} de Sophocle. Il est vrai que c’est au plaisir que fut dû son changement de résolution, mais c’était un noble plaisir : car dire la vérité était pour \\
lui quelque chose de noble, et il n’avait consenti à mentir qu’à l’instigation d’Ulysse. En effet, celui qui accomplit une action par plaisir n’est pas toujours un homme déréglé ou pervers ou intempérant ; mais c’est celui qui l’accomplit par un plaisir honteux.
\subsection[{11 (1151b — 1152a) < Insensibilité — Intempérance et prudence >}]{11 (1151b — 1152a) < Insensibilité — Intempérance et prudence >}
\noindent Puisqu’il existe aussi un genre d’homme constitué de telle façon qu’il ressent moins de joie qu’il ne devrait des plaisirs corporels, et qu’il ne demeure pas fermement attaché à la \\
règle, celui qui occupe la position intermédiaire entre lui et l’homme intempérant est l’homme tempérant. En effet, l’homme intempérant ne demeure pas dans la règle parce qu’il aime trop les plaisirs du corps, et cet autre dont nous parlons, parce qu’il ne les aime pas assez ; l’homme tempérant, au contraire, persiste dans la règle et ne change sous l’effet d’aucune de ces deux causes. Mais il faut bien, si la tempérance est une chose bonne, que les deux dispositions qui y sont contraires soient l’une et l’autre mauvaises, et c’est d’ailleurs \\
bien ainsi qu’elles apparaissent. Mais du fait que l’une d’elles ne se manifeste que dans un petit nombre d’individus et rarement, on croit d’ordinaire que la tempérance est le seul contraire de l’intempérance, tout comme on admet que la modération est le seul contraire du dérèglement.\par
Étant donné, d’autre part, qu’un grand nombre d’expressions sont employées par similitude, c’est par similitude que nous venons naturellement à parler de la {\itshape tempérance} de l’homme modéré, parce que l’homme tempérant et l’homme modéré sont l’un et l’autre constitués de façon à ne rien faire à \\
l’encontre de la règle sous l’impulsion des plaisirs corporels. Mais tandis que le premier a des appétits pervers, le second  n’en a pas, et sa nature est telle qu’il ne ressent aucun plaisir dans les choses qui sont contraires à la règle, alors que l’homme tempérant est naturellement apte à goûter le plaisir dans ces choses-là mais à ne pas s’y abandonner. — Il y a également une ressemblance entre l’homme intempérant et l’homme \\
déréglé, bien qu’ils soient en réalité différents : tous deux poursuivent les plaisirs du corps, mais l’homme déréglé pense qu’il doit le faire, et l’homme intempérant ne le pense pas.\par
Il n’est pas possible non plus que la même personne soit en même temps prudente et intempérante, car nous avons montré que la prudence et le caractère vertueux vont toujours ensemble. Ajoutons que la prudence ne consiste pas seulement dans la connaissance purement théorique du bien, mais encore dans la capacité de le faire, capacité d’agir que l’homme intempérant \\
ne possède pas. — Rien n’empêche au surplus que l’homme habile soit intempérant (et c’est la raison pour laquelle on pense parfois qu’il y a des gens qui tout en étant prudents sont cependant intempérants), parce que si l’habileté et la prudence diffèrent, c’est de la façon indiquée dans nos premières discussions : en tant que se rapportant à la raison ce sont des notions voisines, mais elles diffèrent pour ce qui est du choix. — On ne doit dès lors pas comparer non plus l’homme intempérant à celui qui sait et contemple, mais seulement à celui qui est en état de sommeil ou d’ivresse. Et il agit certes \\
volontairement (puisqu’il sait, d’une certaine manière, à la fois ce qu’il fait et en vue de quoi il le fait), mais il n’est pas pervers, parce que son choix est équitable de telle sorte qu’il n’est qu’à demi pervers. Et il n’est pas injuste, car il n’a aucune malice, puisque des deux types d’hommes intempérants, l’un ne persiste pas dans le résultat de ses délibérations, et que l’autre, l’homme d’humeur excitable, ne délibère pas du tout. \\
Dès lors, l’homme intempérant est semblable à une cité qui rend toujours les décrets qu’il faut et possède des lois sages, mais qui n’en fait aucun usage, comme le remarque en raillant Anaxandride :\par
 {\itshape La cité le souhaitait, elle qui n’a aucun souci des lois.} \par
L’homme vicieux, au contraire, ressemble à une cité qui se sert de ses lois, mais ces lois ne valent rien à l’usage.\par
\\
Tempérance et intempérance ont rapport à ce qui dépasse l’état habituel de la majorité des hommes : l’homme tempérant, en effet, montre une fermeté plus grande, et l’homme intempérant une fermeté moindre que ne sont capables d’en montrer la plupart des hommes.\par
De toutes les formes d’intempérance, celle dont les hommes à humeur excitable sont atteints est plus facile à guérir que celle des hommes qui délibèrent sans persister ensuite dans leur décision, et ceux qui sont intempérants par habitude se guérissent plus aisément que ceux qui le sont par nature, car on change d’habitude plus facilement que de \\
nature ; même l’habitude est difficile à changer, précisément pour cette raison qu’elle ressemble à la nature, suivant la parole d’Evenus :\par
 {\itshape Je dis que l’habitude n’est qu’un exercice de longue haleine, mon ami, et dès lors} \par
 {\itshape Elle finit par devenir chez les hommes une nature.} 
\subsection[{12 (1152a — 1152b) < Théories sur le plaisir : leurs arguments >}]{12 (1152a — 1152b) < Théories sur le plaisir : leurs arguments >}
\noindent Nous avons traité de la nature de la tempérance et de \\
l’intempérance, de celle de l’endurance et de la mollesse, et montré comment ces états se comportent les uns envers les autres.\par
 L’étude du plaisir et de la peine est l’affaire du philosophe politique : c’est lui, en effet, dont l’art architectonique détermine la fin sur laquelle nous fixons les yeux pour appeler chaque chose bonne ou mauvaise au sens absolu. En outre, cette investigation est l’une de nos tâches indispensables : car \\
non seulement nous avons posé que la vertu morale et le vice ont rapport à des plaisirs et à des peines, mais encore, au dire de la plupart des hommes, le bonheur ne va pas sans le plaisir, et c’est la raison pour laquelle l’homme {\itshape bienheureux} est désigné par un nom dérivé de {\itshape se réjouir}.\par
Certains sont d’avis qu’aucun plaisir n’est un bien, ni en \\
lui-même ni par accident (car il n’y a pas identité, disent-ils, entre bien et plaisir). Pour d’autres, certains plaisirs seulement sont bons, mais la plupart sont mauvais. Selon une troisième opinion, enfin, même en supposant que tous les plaisirs soient un bien, il n’est cependant pas possible que le plaisir soit le Souverain Bien.\par
<I> Le plaisir n’est pas du tout un bien, dit-on, parce que : <1> tout plaisir est un devenir senti, vers un état naturel, et qu’un devenir n’est jamais du même genre que sa fin : par exemple un processus de construction n’est jamais du même genre qu’une maison. <2> De plus, l’homme modéré évite \\
les plaisirs. <3> De plus, l’homme prudent poursuit ce qui est exempt de peine, non l’agréable. <4> De plus, les plaisirs sont un obstacle à la prudence, et cela d’autant plus que la jouissance ressentie est plus intense, comme dans le cas du plaisir sexuel, où nul n’est capable de penser quoi que ce soit en l’éprouvant. <5> De plus, il n’existe aucun art productif du plaisir ; cependant toute chose bonne est l’œuvre d’un art. <6> De plus, enfants et bêtes poursuivent les plaisirs.\par
\\
<II> Tous les plaisirs ne sont pas bons, dit-on d’autre part, parce que : <1> il y en a de honteux et de répréhensibles et qu’en outre <2> il y en a de nuisibles, puisque certaines choses qui plaisent sont funestes à la santé.\par
<III> Enfin, que le plaisir ne soit pas le Souverain Bien est prouvé parce fait qu’il n’est pas une fin mais un devenir.
\subsection[{13 (1152b — 1153a) < Discussion de la théorie que le plaisir n’est pas un bien >}]{13 (1152b — 1153a) < Discussion de la théorie que le plaisir n’est pas un bien >}
\noindent Telles sont donc, à peu près, les opinions qui ont cours.\par
\\
Qu’il ne résulte pas de ces arguments que le plaisir ne soit pas un bien, ni même le Souverain Bien, les considérations suivantes le font voir.\par
En premier lieu, puisque le bien est pris en un double sens (il y a le bien au sens absolu et le bien pour telle personne), il s’ensuivra que les états naturels et les dispositions seront aussi appelés bons en un double sens, et par suite également les mouvements et les devenirs correspondants. Et de ces mouvements et devenirs considérés comme mauvais, les uns seront mauvais au sens absolu, < les autres, mauvais pour une personne déterminée > et non pour une autre, mais au contraire désirables pour tel individu ; certains autres ne seront même pas désirables \\
en général pour tel individu, mais seulement à un moment donné et pour peu de temps et non < toujours > ; les autres devenirs, enfin, ne sont pas même des plaisirs, mais le paraissent seulement, ce sont tous ceux qui s’accompagnent de peine et ont pour fin une guérison, par exemple les processus des maladies.\par
En outre, puisque une sorte de bien est activité, et une autre sorte, disposition, les processus qui nous restaurent dans notre état naturel sont agréables seulement par accident, l’activité \\
en travail dans nos appétits étant celle de cette partie de nous-mêmes demeurée dans son état naturel : c’est qu’il existe  aussi des plaisirs sans accompagnement de peine ou d’appétit, par exemple l’activité contemplative, où la nature ne souffre d’aucun manque. Et ce qui indique < que les plaisirs liés à un processus sont seulement accidentels > c’est qu’on ne se réjouit pas du même objet agréable au moment où la nature remplit ses vides et après qu’elle est restaurée : dans la nature restaurée, on se plaît aux choses qui sont agréables au sens absolu ; dans la nature en train de se refaire, on se plaît même à \\
leurs contraires : car on aimera même les substances piquantes et amères dont aucune n’est naturellement agréable ni absolument agréable, de sorte que les plaisirs que nous en ressentons ne sont non plus ni naturellement ni absolument agréables, la distinction qui sépare les différents objets plaisants l’un de l’autre s’étendant aux plaisirs qui en découlent.\par
En outre, il ne s’ensuit pas qu’on doive nécessairement poser quelque chose de meilleur que le plaisir, pour la raison qu’au dire de certains la fin est meilleure que le devenir. Les plaisirs, en effet, ne sont pas réellement des devenirs, ni ne sont \\
pas tous liés à un devenir : ils sont activités et fin ; ils ne se produisent pas non plus au cours de nos devenirs mais quand nous faisons usage de nos puissances ; tous enfin n’ont pas une fin différente d’eux-mêmes, cela n’est vrai que des plaisirs de ceux qui reviennent à la perfection de leur nature. Et c’est pourquoi il n’est pas exact de dire que le plaisir est un devenir {\itshape senti}, il faut plutôt le définir comme une activité de la manière \\
d’être qui est selon la nature, et, au lieu de {\itshape senti}, mettre {\itshape non empêché}. — Il y a aussi les gens qui regardent le plaisir comme un devenir, parce que c’est pour eux un bien au sens absolu, car à leurs yeux l’activité est un devenir, alors qu’en fait elle est tout autre chose.\par
L’opinion suivant laquelle les plaisirs sont mauvais parce que certaines choses agréables sont nuisibles à la santé, revient à dire que la santé est mauvaise parce que certaines choses utiles à la santé ne valent rien pour gagner de l’argent. À cet égard assurément les choses agréables comme les choses utiles à la santé sont mauvaises, mais elles ne sont pas mauvaises du moins pour cette raison-là, puisque même la contemplation \\
peut parfois être nuisible à la santé.\par
D’autre part, ni la prudence, ni aucune disposition en général n’est entravée par le plaisir découlant d’elle-même, mais seulement par les plaisirs étrangers, puisque les plaisirs nés du fait de contempler et d’apprendre nous feront contempler et apprendre davantage.\par
Qu’aucun plaisir ne soit l’œuvre d’un art, c’est là un fait \\
assez naturel : aucune autre activité non plus n’est le produit d’un art, mais l’art se contente de donner la capacité, bien qu’en fait l’art du parfumeur et celui du cuisinier soient généralement considérés comme des arts productifs de plaisir.\par
Les arguments qui s’appuient sur le fait que l’homme modéré évite le plaisir et que l’homme prudent poursuit la vie exempte de peine seulement, et que d’autre part les enfants et les bêtes poursuivent le plaisir, ces arguments-là sont réfutés \\
tous à la fois par la même considération : nous avons indiqué, en effet, comment les plaisirs sont bons au sens absolu, et comment certains plaisirs ne sont pas bons ; or ce sont ces derniers plaisirs que les bêtes et les enfants poursuivent (et c’est l’absence de la peine causée par la privation des plaisirs de ce genre que recherche l’homme prudent), c’est-à-dire les plaisirs qui impliquent appétit et peine, en d’autres termes les plaisirs corporels (qui sont bien de cette sorte-là) et leurs formes excessives, plaisirs qui rendent précisément déréglé l’homme déréglé. Telle est la raison pour laquelle l’homme modéré fuit \\
ces plaisirs, car même l’homme modéré a des plaisirs.
\subsection[{14 (1153b — 1154a) < Le plaisir et le Souverain Bien. Plaisirs bons et plaisirs mauvais >}]{14 (1153b — 1154a) < Le plaisir et le Souverain Bien. Plaisirs bons et plaisirs mauvais >}
\noindent  En outre, que la peine aussi soit un mal et doive être évitée, c’est ce que tout le monde reconnaît : car la peine est tantôt un mal au sens absolu, tantôt un mal en ce qu’elle est propre à entraver de quelque façon notre activité. Or le contraire d’une chose qu’on doit éviter, en tant qu’elle est à éviter et est un mal, ce contraire est un bien. Le plaisir est donc \\
nécessairement un bien. Speusippe tentait de réfuter cet argument en s’appuyant sur cette comparaison que {\itshape plus grand} est contraire à la fois à {\itshape plus petit} et à {\itshape égal} ; mais sa réfutation est inopérante, car on ne saurait prétendre que le plaisir est dans son essence quelque espèce de mal.\par
D’autre part, rien ne s’oppose à ce que le Souverain Bien ne soit lui-même un plaisir déterminé, même si on accorde que certains plaisirs sont mauvais : tout comme le Souverain Bien pourrait consister en une science déterminée, même si certaines sciences sont mauvaises. Peut-être mêmes est-ce une nécessité, si chacune de nos dispositions a son activité correspondante \\
s’exerçant sans entraves (qu’on fasse consister le bonheur soit dans l’activité de l’ensemble de nos dispositions, soit dans l’activité de l’une d’entre elles, cette activité < sous l’une ou l’autre forme > étant supposée sans entraves), que l’activité en question soit la plus digne de notre choix : or cette activité est plaisir. Ainsi le Souverain Bien serait un certain plaisir, bien que la plupart des plaisirs soient mauvais, et même, le cas échéant, mauvais absolument. — Et c’est pourquoi tous les hommes pensent que la vie heureuse est une vie \\
agréable, et qu’ils entrelacent étroitement le plaisir au bonheur. En cela ils ont raison, aucune activité n’étant parfaite quand elle est empêchée, alors que le bonheur rentre dans la classe des activités parfaites. Aussi l’homme heureux a-t-il besoin, en sus du reste, des biens du corps, des biens extérieurs et des dons de la fortune, de façon que son activité ne soit pas entravée de ce côté-là. Et ceux qui prétendent que l’homme attaché à la \\
roue ou tombant dans les plus grandes infortunes est un homme heureux à la condition qu’il soit bon, profèrent, volontairement ou non, un non-sens. À l’opposé, sous prétexte que l’on a besoin, en sus du reste, du secours de la fortune, on identifie parfois la fortune favorable au bonheur ; or ce sont des choses toutes différentes, car la fortune favorable elle-même, quand elle excède la mesure, constitue un empêchement à l’activité, et peut-être n’est-il plus juste de l’appeler alors fortune favorable, sa limite étant déterminée par sa relation au bonheur.\par
\\
Et le fait que tous les êtres, bêtes et hommes, poursuivent le plaisir est un signe que le plaisir est en quelque façon le Souverain Bien.\par
 {\itshape Nulle rumeur ne meurt tout entière, que tant de gens…} \par
Mais, comme ce n’est ni la même nature, ni la même disposition qui est la meilleure pour tout le monde, ou qui du moins apparaît telle à chacun, tous les hommes ne poursuivent \\
pas non plus le même plaisir, bien que tous poursuivent le plaisir. Peut-être aussi poursuivent-ils non pas le plaisir qu’ils s’imaginent ou qu’ils voudraient dire qu’ils recherchent, mais un plaisir le même pour tous, car tous les êtres ont naturellement en eux quelque chose de divin. Mais les plaisirs corporels ont accaparé l’héritage du nom de plaisir, parce que c’est vers eux que nous dirigeons le plus fréquemment notre course \\
et qu’ils sont le partage de tout le monde ; et ainsi, du fait qu’ils sont les seuls qui nous soient familiers, nous croyons que ce sont les seuls qui existent.\par
 Il est manifeste aussi que si le plaisir, autrement dit l’activité, n’est pas un bien, la vie de l’homme heureux ne sera pas une vie agréable : pour quelle fin aurait-il besoin du plaisir si le plaisir n’est pas un bien ? Au contraire sa vie peut même être chargée de peine : car la peine n’est ni un bien ni un mal, si \\
le plaisir n’est non plus ni l’un ni l’autre : dans ces conditions pourquoi fuirait-on la peine ? Dès lors aussi la vie de l’homme vertueux ne sera pas plus agréable qu’une autre, si ses activités ne le sont pas non plus davantage.\par
Au sujet des plaisirs du corps, il faut examiner la doctrine de ceux qui disent qu’assurément certains plaisirs sont hautement désirables, par exemple les plaisirs nobles, mais qu’il n’en est pas ainsi des plaisirs corporels et de ceux qui sont le \\
domaine de l’homme déréglé. S’il en est ainsi, pourquoi les peines contraires sont-elles mauvaises ? car le contraire d’un mal est un bien. Ne serait-ce pas que les plaisirs qui sont nécessaires sont bons au sens où ce qui n’est pas mauvais est bon ? Ou encore que ces plaisirs sont bons jusqu’à un certain point ? En effet, si dans les dispositions et les mouvements qui n’admettent pas d’excès du mieux il n’y a pas non plus d’excès possible du plaisir correspondant, dans les états admettant au \\
contraire cette sorte d’excès il y aura aussi excès du plaisir. Or les biens du corps admettent l’excès, et c’est la poursuite de cet excès qui rend l’homme pervers, et non pas celle des plaisirs nécessaires : car si tous les hommes jouissent d’une façon quelconque des mets, des vins et des plaisirs sexuels, tous n’en jouissent pas dans la mesure qu’il faut. C’est tout le contraire pour la peine : on n’en évite pas seulement l’excès, mais on la \\
fuit complètement ; c’est que ce n’est pas au plaisir excessif qu’une peine est contraire, excepté pour l’homme qui poursuit l’excès de plaisir.
\subsection[{15 (1154a — 1154b) < Le plaisir — Les plaisirs corporels >}]{15 (1154a — 1154b) < Le plaisir — Les plaisirs corporels >}
\noindent Puisqu’il faut non seulement énoncer le vrai, mais encore montrer la cause de l’erreur contraire (car c’est renforcer la croyance au vrai : quand, en effet, on a fourni une explication plausible de la raison pour laquelle ce qui apparaît comme vrai \\
ne l’est pas en réalité, on rend plus forte la croyance au vrai), il en résulte que nous devons indiquer pourquoi les plaisirs du corps apparaissent comme plus désirables que d’autres.\par
La première raison, donc, est que le plaisir détourne la peine ; l’excès de la peine pousse les hommes à rechercher, en guise de remède, le plaisir lui-même excessif, et, d’une manière générale, le plaisir corporel. Et ces plaisirs curatifs \\
revêtent eux-mêmes une grande intensité (et c’est pourquoi on les poursuit) parce qu’ils apparaissent en contraste avec la peine opposée. L’opinion, dès lors, suivant laquelle le plaisir n’est pas un bien tient aussi à ces deux faits que nous avons signalés, à savoir : que certains plaisirs sont des actes relevant d’une nature perverse (qu’elle le soit dès la naissance, comme chez la brute, ou par l’effet de l’habitude, comme les plaisirs des hommes vicieux) ; que les autres qui agissent comme remèdes supposent un manque, et qu’il est préférable d’être en  bon état que d’être en voie de guérison ; mais ces plaisirs curatifs accompagnent en fait des processus de restauration d’un état parfait, et sont donc bons par accident.\par
En outre, les plaisirs corporels sont poursuivis, en raison même de leur intensité, par les gens qui ne sont pas capables d’en goûter d’autres : ainsi il y en a qui vont jusqu’à provoquer en eux la soif. Quand ces plaisirs n’entraînent aucun dommage, \\
il n’y a rien à redire, mais s’ils sont pernicieux, c’est un mal. Le fait est que ces gens-là n’ont pas d’autres sources de jouissance, et l’état qui n’est ni agréable ni pénible est pour beaucoup d’entre eux une chose difficile à supporter, en raison de leur constitution naturelle. L’être animé vit, en effet, dans un état perpétuel d’effort, au témoignage même des physiologues, d’après lesquels la vision et l’audition sont quelque chose de pénible ; il est vrai que depuis longtemps, disent-ils, nous y sommes accoutumés. Dans le même ordre d’idées, les \\
jeunes gens, à cause de la croissance, sont dans un état semblable à celui de l’homme pris de vin, et c’est même là le charme de la jeunesse ; d’autre part, les gens d’humeur naturellement excitable ont un perpétuel besoin de remède, car même leur corps vit dans un continuel état d’irritation dû à leur tempérament, et ils sont toujours en proie à un désir violent ; mais le plaisir chasse la peine, aussi bien le plaisir qui y est contraire que n’importe quel autre, à la condition qu’il soit fort, et c’est ce qui fait que l’homme d’humeur excitable devient déréglé et pervers.\par
\\
D’autre part, les plaisirs non accompagnés de peine n’admettent pas l’excès, et ces plaisirs sont ceux qui découlent des choses agréables par nature et non par accident. Par {\itshape choses agréables par accident}, j’entends celles qui agissent comme remèdes (il se trouve, en effet, que leur vertu curative vient d’une certaine activité de la partie de nous-mêmes demeurée saine, ce qui fait que le remède lui-même semble agréable), et \\
par {\itshape choses agréables par nature}, celles qui stimulent l’activité d’une nature donnée.\par
Il n’y a aucune chose cependant qui soit pour nous toujours agréable : cela tient à ce que notre nature n’est pas simple, mais qu’elle renferme aussi un second élément, en vertu de quoi nous sommes des êtres corruptibles, de sorte que si le premier élément fait une chose, cette chose est pour l’autre élément naturel quelque chose de contraire à sa nature, et quand les deux éléments sont en état d’équilibre, l’action accomplie n’est ressentie ni comme pénible ni comme agréable ; car \\
supposé qu’il existe un être quelconque possesseur d’une nature simple, la même activité serait pour lui toujours le plus haut degré de plaisir. C’est pourquoi Dieu jouit perpétuellement d’un plaisir un et simple ; car il y a non seulement une activité de mouvement, mais encore une activité d’immobilité, et le plaisir consiste plutôt dans le repos que dans le mouvement. Mais :\par
{\itshape Le changement en toutes choses est bien doux}, \par
suivant le poète, en raison d’une certaine imperfection de \\
notre nature : car de même que l’homme pervers est un homme versatile, ainsi est perverse la nature qui a besoin de changement, car elle n’est ni simple, ni bonne.\par
La tempérance et l’intempérance, le plaisir et la peine ont fait jusqu’ici l’objet de nos discussions, et nous avons établi la nature de chacune de ces notions et en quel sens les unes sont bonnes et les autres mauvaises. Il nous reste à parler de l’amitié.
\section[{Livre VIII}]{Livre VIII}\renewcommand{\leftmark}{Livre VIII}

\subsection[{1 (1155a) < L’amitié — Sa nécessité >}]{1 (1155a) < L’amitié — Sa nécessité >}
\noindent  Après ces considérations, nous pouvons passer à la discussion sur l’amitié. L’amitié est en effet une certaine vertu, ou ne va pas sans vertu ; de plus, elle est ce qu’il y a de plus nécessaire pour vivre. Car sans amis personne ne choisirait \\
de vivre, eût-il tous les autres biens (et de fait les gens riches, et ceux qui possèdent autorité et pouvoir semblent bien avoir plus que quiconque besoin d’amis : à quoi servirait une pareille prospérité, une fois ôtée la possibilité de répandre des bienfaits, laquelle se manifeste principalement et de la façon la plus digne d’éloge, à l’égard des amis ? Ou encore, comment \\
cette prospérité serait-elle gardée et préservée sans amis ? car plus elle est grande, plus elle est exposée au risque). Et dans la pauvreté comme dans tout autre infortune, les hommes pensent que les amis sont l’unique refuge. L’amitié d’ailleurs est un secours aux jeunes gens, pour les préserver de l’erreur ; aux vieillards, pour leur assurer des soins et suppléer à leur manque d’activité dû à la faiblesse ; à ceux enfin qui sont dans la fleur de l’âge, pour les inciter aux nobles actions :\par
\\
{\itshape Quand deux vont de compagnie}, \par
car on est alors plus capable à la fois de penser et d’agir. De plus, l’affection est, semble-t-il, un sentiment naturel du père pour sa progéniture et de celle-ci pour le père, non seulement chez l’homme mais encore chez les oiseaux et la plupart des animaux ; les individus de même race ressentent aussi une \\
amitié mutuelle, principalement dans l’espèce humaine, et c’est pourquoi nous louons les hommes qui sont bons pour les autres. Même au cours de nos voyages au loin, nous pouvons constater à quel point l’homme ressent toujours de l’affinité et de l’amitié pour l’homme. L’amitié semble aussi constituer le lien des cités, et les législateurs paraissent y attacher un plus grand prix qu’à la justice même : en effet, la concorde, qui \\
paraît bien être un sentiment voisin de l’amitié, est ce que recherchent avant tout les législateurs, alors que l’esprit de faction, qui est son ennemie, est ce qu’ils pourchassent avec le plus d’énergie. Et quand les hommes sont amis il n’y a plus besoin de justice, tandis que s’ils se contentent d’être justes ils ont en outre besoin d’amitié, et la plus haute expression de la justice est, dans l’opinion générale, de la nature de l’amitié.\par
Non seulement l’amitié est une chose nécessaire, mais elle est aussi une chose noble : nous louons ceux qui aiment leurs \\
amis, et la possession d’un grand nombre d’amis est regardée comme un bel avantage ; certains pensent même qu’il n’y a aucune différence entre un homme bon et un véritable ami.
\subsection[{2 (1155a -1156a) < Les diverses théories sur la nature de l’amitié >}]{2 (1155a -1156a) < Les diverses théories sur la nature de l’amitié >}
\noindent Les divergences d’opinion au sujet de l’amitié sont nombreuses. Les uns la définissent comme une sorte de ressemblance, et disent que ceux qui sont semblables sont amis, d’où les dictons : {\itshape le semblable va à son semblable, le choucas va au} \\
{\itshape choucas}, et ainsi de suite. D’autres au contraire, prétendent que les hommes qui se ressemblent ainsi sont toujours comme  {\itshape des potiers l’un envers l’autre}. Sur ces mêmes sujets, certains recherchent une explication plus relevée et s’appuyant davantage sur des considérations d’ordre physique : pour Euripide, {\itshape la terre, quand elle est desséchée, est éprise de pluie, et le ciel majestueux, saturé de pluie, aime à tomber sur la terre} ; pour \\
Héraclite, {\itshape c’est ce qui est opposé qui est utile, et des dissonances résulte la plus belle harmonie, et toutes choses sont engendrées par discorde}. Mais l’opinion contraire est soutenue par d’autres auteurs et notamment par Empédocle, suivant lequel {\itshape le semblable tend vers le semblable}.\par
Laissons de côté les problèmes d’ordre physique (qui n’ont rien à voir avec la présente enquête) ; examinons \\
seulement les problèmes proprement humains et qui concernent les mœurs et les passions : par exemple, si l’amitié se rencontre chez tous les hommes, ou si au contraire il est impossible que des méchants soient amis ; et s’il n’y a qu’une seule espèce d’amitié ou s’il y en a plusieurs. Ceux qui pensent que l’amitié est d’une seule espèce pour la raison qu’elle admet le plus et le moins, ajoutent foi à une indication insuffisante, puisque même les choses qui \\
diffèrent en espèce sont susceptibles de plus et de moins. Mais nous avons discuté ce point antérieurement.\par
Peut-être ces matières gagneraient-elles en clarté si nous connaissions préalablement ce qui est objet de l’amitié. Il semble, en effet, que tout ne provoque pas l’amitié, mais seulement ce qui est aimable, c’est-à-dire ce qui est bon, agréable ou utile. On peut d’ailleurs admettre qu’est utile ce par quoi est \\
obtenu un certain bien ou un certain plaisir, de sorte que c’est seulement le bien et l’agréable qui seraient aimables, comme des fins. Dans ces conditions, est-ce que les hommes aiment le bien réel, ou ce qui est bien pour eux ? car il y a parfois désaccord entre ces deux choses. Même question en ce qui concerne aussi l’agréable. Or on admet d’ordinaire que chacun aime ce qui est bon pour soi-même, et que ce qui est réellement un bien est aimable d’une façon absolue, tandis que ce qui est \\
bon pour un homme déterminé est aimable seulement pour lui. Et chaque homme aime non pas ce qui est réellement un bien pour lui, mais ce qui lui apparaît tel ; cette remarque n’a du reste ici aucune importance : nous dirons que l’aimable est l’aimable apparent.\par
Il y a donc trois objets qui font naître l’amitié. L’attachement pour les choses inanimées ne se nomme pas amitié, puisqu’il n’y a pas attachement en retour, ni possibilité pour nous de leur désirer du bien (il serait ridicule sans doute \\
de vouloir du bien au vin par exemple ; tout au plus souhaite-t-on sa conservation, de façon à l’avoir en notre possession) ; s’agit-il au contraire d’un ami, nous disons qu’il est de notre devoir de lui souhaiter ce qui est bon pour lui. Mais ceux qui veulent ainsi du bien à un autre, on les appelle bienveillants quand le même souhait ne se produit pas de la part de ce dernier, car ce n’est que si la bienveillance est réciproque qu’elle est amitié. Ne faut-il pas ajouter encore que cette \\
bienveillance mutuelle ne doit pas demeurer inaperçue ? Beaucoup de gens ont de la bienveillance pour des personnes  qu’ils n’ont jamais vues mais qu’ils jugent honnêtes ou utiles, et l’une de ces personnes peut éprouver ce même sentiment à l’égard de l’autre partie. Quoiqu’il y ait manifestement alors bienveillance mutuelle, comment pourrait-on les qualifier d’amis, alors que chacun d’eux n’a pas connaissance des sentiments personnels de l’autre ? Il faut donc qu’il y ait bienveillance mutuelle, chacun souhaitant le bien de l’autre : que cette bienveillance ne reste pas ignorée des intéressés ; et \\
qu’elle ait pour cause l’un des objets dont nous avons parlé.
\subsection[{3 (1156a — 1156b) < Les espèces de l’amitié : l’amitié fondée sur l’utilité et l’amitié fondée sur le plaisir >}]{3 (1156a — 1156b) < Les espèces de l’amitié : l’amitié fondée sur l’utilité et l’amitié fondée sur le plaisir >}
\noindent Or ces objets aimables diffèrent l’un de l’autre en espèce, et par suite aussi les attachements et les amitiés correspondantes. On aura dès lors trois espèces d’amitiés, en nombre égal à leurs objets, car répondant à chaque espèce il y a un attachement réciproque ne demeurant pas inaperçu des intéressés. Or quand les hommes ont l’un pour l’autre une amitié partagée, ils se souhaitent réciproquement du bien d’après \\
l’objet qui est à l’origine de leur amitié. Ainsi donc, ceux dont l’amitié réciproque a pour source l’utilité ne s’aiment pas l’un l’autre pour eux-mêmes, mais en tant qu’il y a quelque bien qu’ils retirent l’un de l’autre. De même encore ceux dont l’amitié repose sur le plaisir : ce n’est pas en raison de ce que les gens d’esprit sont ce qu’ils sont en eux-mêmes qu’ils les chérissent, mais parce qu’ils les trouvent agréables personnellement. Par suite ceux dont l’amitié est fondée sur l’utilité \\
aiment pour leur propre bien, et ceux qui aiment en raison du plaisir, pour leur propre agrément, et non pas dans l’un et l’autre cas en tant ce qu’est en elle-même la personne aimée, mais en tant qu’elle est utile ou agréable. Dès lors ces amitiés ont un caractère accidentel, puisque ce n’est pas en tant ce qu’elle est essentiellement que la personne aimée est aimée, mais en tant qu’elle procure quelque bien ou quelque plaisir, suivant le cas. Les amitiés de ce genre sont par suite fragiles, \\
dès que les deux amis ne demeurent pas pareils à ce qu’ils étaient : s’ils ne sont plus agréables ou utiles l’un à l’autre, ils cessent d’être amis. Or l’utilité n’est pas une chose durable, mais elle varie suivant les époques. Aussi, quand la cause qui faisait l’amitié a disparu, l’amitié elle-même est-elle rompue, attendu que l’amitié n’existe qu’en vue de la fin en question.\par
\\
C’est surtout chez les vieillards que cette sorte d’amitié se rencontre (car les personnes de cet âge ne poursuivent pas l’agrément mais le profit), et aussi chez ceux des hommes faits et des jeunes gens qui recherchent leur intérêt. Les amis de cette sorte ne se plaisent guère à vivre ensemble, car parfois ils ne sont pas même agréables l’un à l’autre ; ils n’ont dès lors nullement besoin d’une telle fréquentation, à moins qu’ils n’y trouvent leur intérêt, puisqu’ils ne se plaisent l’un avec l’autre \\
que dans la mesure où ils ont l’espérance de quelque bien. — À ces amitiés on rattache aussi celle envers les hôtes.\par
D’autre part, l’amitié chez les jeunes gens semble avoir pour fondement le plaisir ; car les jeunes gens vivent sous l’empire de la passion, et ils poursuivent surtout ce qui leur plaît personnellement et le plaisir du moment ; mais en avançant en âge, les choses qui leur plaisent ne demeurent pas les mêmes. \\
C’est pourquoi ils forment rapidement des amitiés et les abandonnent avec la même facilité, car leur amitié change avec  l’objet qui leur donne du plaisir, et les plaisirs de cet âge sont sujets à de brusques variations. Les jeunes gens ont aussi un penchant à l’amour, car une grande part de l’émotion amoureuse relève de la passion et a pour source le plaisir. De là vient qu’ils aiment et cessent d’aimer avec la même rapidité, changeant plusieurs fois dans la même journée. Ils souhaitent aussi \\
passer leur temps et leur vie en compagnie de leurs amis, car c’est de cette façon que se présente pour eux ce qui a trait à l’amitié.
\subsection[{4 (1156b) < L’amitié fondée sur la vertu >}]{4 (1156b) < L’amitié fondée sur la vertu >}
\noindent Mais la parfaite amitié est celle des hommes vertueux et qui sont semblables en vertu : car ces amis-là se souhaitent pareillement du bien les uns aux autres en tant qu’ils sont bons, et ils sont bons par eux-mêmes. Mais ceux qui souhaitent du \\
bien à leurs amis pour l’amour de ces derniers sont des amis par excellence (puisqu’ils se comportent ainsi l’un envers l’autre en raison de la propre nature de chacun d’eux, et non par accident) : aussi leur amitié persiste-telle aussi longtemps qu’ils sont eux-mêmes bons, et la vertu est une disposition stable. Et chacun d’eux est bon à la fois absolument et pour son ami, puisque les hommes bons sont en même temps bons absolument et utiles les uns aux autres. Et de la même façon qu’ils \\
sont bons, ils sont agréables aussi l’un pour l’autre : les hommes bons sont à la fois agréables absolument et agréables les uns pour les autres, puisque chacun fait résider son plaisir dans les actions qui expriment son caractère propre, et par suite dans celles qui sont de même nature, et que, d’autre part, les actions des gens de bien sont identiques ou semblables à celles des autres gens de bien. Il est normal qu’une amitié de ce genre soit stable, car en elle sont réunies toutes les qualités qui doivent appartenir aux amis. Toute amitié, en effet, a pour \\
source le bien ou le plaisir, bien ou plaisir envisagés soit au sens absolu, soit seulement pour celui qui aime, c’est-à-dire en raison d’une certaine ressemblance ; mais dans le cas de cette amitié, toutes les qualités que nous avons indiquées appartiennent aux amis par eux-mêmes (car en cette amitié les amis sont semblables aussi pour les autres qualités), et ce qui est bon absolument est aussi agréable absolument. Or ce sont là les principaux objets de l’amitié, et dès lors l’affection et l’amitié existent chez ces amis au plus haut degré et en la forme la plus excellente.\par
\\
Il est naturel que les amitiés de cette espèce soient rares, car de tels hommes sont en petit nombre. En outre elles exigent comme condition supplémentaire, du temps et des habitudes communes, car, selon le proverbe, il n’est pas possible de se connaître l’un l’autre avant d’avoir {\itshape consommé ensemble la mesure de sel} dont parle le dicton, ni d’admettre quelqu’un dans son amitié, ou d’être réellement amis, avant que chacun des intéressés se soit montré à l’autre comme un digne objet d’amitié et lui ait inspiré de la confiance. Et ceux qui s’engagent rapidement dans les liens d’une amitié réciproque ont \\
assurément la volonté d’être amis, mais ils ne le sont pas en réalité, à moins qu’ils ne soient aussi dignes d’être aimés l’un et l’autre, et qu’ils aient connaissance de leurs sentiments : car si la volonté de contracter une amitié est prompte, l’amitié ne l’est pas.
\subsection[{5 (1156b — 1157a) < Comparaison entre l’amitié parfaite et les autres amitiés >}]{5 (1156b — 1157a) < Comparaison entre l’amitié parfaite et les autres amitiés >}
\noindent Cette amitié, donc, est parfaite aussi bien en raison de sa durée que pour le reste ; et à tous ces points de vue, chaque partie reçoit de l’autre les mêmes avantages ou des avantages \\
semblables, ce qui est précisément la règle entre amis.\par
 L’amitié fondée sur le plaisir a de la ressemblance avec la précédente (puisque les hommes bons sont aussi des gens agréables les uns aux autres) ; et il en est encore de même pour celle qui est basée sur l’utilité (puisque les hommes de bien sont utiles aussi les uns aux autres). Dans ces deux derniers cas l’amitié atteint son maximum de durée quand l’avantage que retirent réciproquement les deux parties est le même, par \\
exemple le plaisir, et non seulement cela, mais encore quand sa source est la même, comme c’est le cas d’une amitié entre personnes d’esprit, alors qu’il en est tout différemment dans le commerce de l’amant et de l’aimé. Ces derniers, en effet, ne trouvent pas leur plaisir dans les mêmes choses : pour l’un, le plaisir consiste dans la vue de l’aimé, et pour l’autre, dans le fait de recevoir les petits soins de l’amant ; et la fleur de la jeunesse venant à se faner, l’amour se fane aussi (à celui qui aime, la vue de l’aimé ne cause pas de plaisir, et à l’être aimé on \\
ne rend plus de soins) ; dans beaucoup de cas, en revanche, l’amour persiste quand l’intimité a rendu cher à chacun d’eux le caractère de l’autre, étant tous les deux d’un caractère semblable. Mais ceux dont les relations amoureuses reposent sur une réciprocité non pas même de plaisir mais seulement d’utilité, ressentent aussi une amitié moins vive et moins durable. Et l’amitié basée sur l’utilité disparaît en même temps que \\
le profit : car ces amis-là ne s’aimaient pas l’un l’autre, mais n’aimaient que leur intérêt.\par
Ainsi donc, l’amitié fondée sur le plaisir ou sur l’utilité peut exister entre deux hommes vicieux, ou entre un homme vicieux et un homme de bien, ou enfin entre un homme ni bon ni mauvais et n’importe quel autre ; mais il est clair que seuls les hommes vertueux peuvent être amis pour ce qu’ils sont en eux-mêmes : les méchants, en effet, ne ressentent aucune joie l’un de l’autre s’il n’y a pas quelque intérêt en jeu.\par
\\
Seule encore l’amitié entre gens de bien est à l’abri des traverses : on ajoute difficilement foi à un propos concernant une personne qu’on a soi-même pendant longtemps mise à l’épreuve ; et c’est parmi les gens vertueux qu’on rencontre la confiance, l’incapacité de se faire jamais du tort, et toutes autres qualités qu’exige la véritable amitié. Dans les autres formes d’amitié, rien n’empêche les maux opposés de se produire.\par
\\
Mais étant donné que les hommes appellent aussi amis à la fois ceux qui ne recherchent que leur utilité, comme cela arrive pour les cités (car on admet généralement que les alliances entre cités se forment en vue de l’intérêt), et ceux dont la tendresse réciproque repose sur le plaisir, comme c’est le cas chez les enfants : dans ces conditions, peut-être nous aussi devrions-nous désigner du nom d’amis ceux qui entretiennent \\
des relations de ce genre, et dire qu’il existe plusieurs espèces d’amitié, dont l’une, prise au sens premier et fondamental, est l’amitié des gens vertueux en tant que vertueux, tandis que les 176 189 Banquet, 183e. deux autres ne sont des amitiés que par ressemblance : en effet, dans ces derniers cas, on n’est amis que sous l’angle de quelque bien ou de quelque chose de semblable, puisque même le plaisir est un bien pour ceux qui aiment le plaisir. Mais ces deux formes inférieures de l’amitié sont loin de coïncider entre elles, et les hommes ne deviennent pas amis à la \\
fois par intérêt et par plaisir, car on ne trouve pas souvent unies ensemble les choses liées d’une façon accidentelle.
\subsection[{6 (1157b) < L’habitus et l’activité dans l’amitié >}]{6 (1157b) < L’habitus et l’activité dans l’amitié >}
\noindent  Telles étant les différentes espèces entre lesquelles se distribue l’amitié, les hommes pervers seront amis par plaisir ou par intérêt, étant sous cet aspect semblables entre eux, tandis que les hommes vertueux seront amis par ce qu’ils sont en eux-mêmes, c’est-à-dire en tant qu’ils sont bons. Ces derniers sont ainsi des amis au sens propre, alors que les précédents ne le sont que par accident et par ressemblance avec les véritables amis.\par
\\
De même que, dans la sphère des vertus, les hommes sont appelés bons soit d’après une disposition, soit d’après une activité, ainsi en est-il pour l’amitié : les uns mettent leur plaisir à partager leur existence et à se procurer l’un à l’autre du bien, tandis que ceux qui sont endormis ou habitent des lieux séparés ne sont pas des amis en acte, mais sont cependant dans une \\
disposition de nature à exercer leur activité d’amis. Car les distances ne détruisent pas l’amitié absolument, mais empêchent son exercice. Si cependant l’absence se prolonge, elle semble bien entraîner l’oubli de l’amitié elle-même. D’où le proverbe :\par
 {\itshape Un long silence a mis fin à de nombreuses amitiés.} \par
On ne voit d’ailleurs ni les vieillards ni les gens moroses \\
êtres enclins à l’amitié : médiocre est en eux le côté plaisant, et personne n’est capable de passer son temps en compagnie d’un être chagrin et sans agrément, la nature paraissant par-dessus tout fuir ce qui est pénible et tendre à ce qui est agréable. — Quant à ceux qui se reçoivent dans leur amitié tout en ne vivant pas ensemble, ils sont plutôt semblables à des gens bienveillants qu’à des amis. Rien, en effet, ne caractérise mieux l’amitié que la vie en commun : ceux qui sont dans le \\
besoin aspirent à l’aide de leurs amis, et même les gens comblés souhaitent passer leur temps ensemble, car la solitude leur convient moins qu’à tous autres. Mais il n’est pas possible de vivre les uns avec les autres si on n’en retire aucun agrément et s’il n’y a pas communauté de goûts, ce qui, semble-t-il, est le lien de l’amitié entre camarades.
\subsection[{7 (1157b — 1158a) < Étude de rapports particuliers entre les diverses amitiés >}]{7 (1157b — 1158a) < Étude de rapports particuliers entre les diverses amitiés >}
\noindent \\
L’amitié est donc surtout celle des gens vertueux, comme nous l’avons dit à plusieurs reprises. On admet, en effet, que ce qui est bon, ou plaisant, au sens absolu, est digne d’amitié et de choix, tandis que ce qui est bon ou plaisant pour telle personne déterminée n’est digne d’amitié et de choix que pour elle. Et l’homme vertueux l’est pour l’homme vertueux pour ces deux raisons à la fois. (L’attachement semble être une émotion, et l’amitié une disposition, car l’attachement \\
s’adresse même aux êtres inanimés, mais l’amour réciproque s’accompagne de choix délibéré, et le choix provient d’une disposition). Et quand les hommes souhaitent du bien à ceux qu’ils aiment pour l’amour même de ceux-ci, ce sentiment relève non pas d’une émotion, mais d’une disposition. Et en aimant leur ami ils aiment ce qui est bon pour eux-mêmes, puisque l’homme bon, en devenant un ami, devient un bien pour celui qui est son ami. Ainsi chacun des deux amis, à la fois \\
aime son propre bien et rend exactement à l’autre ce qu’il en reçoit, en souhait et en plaisir : on dit, en effet, que {\itshape l’amitié est}  {\itshape une égalité}, et c’est principalement dans l’amitié entre gens de bien que ces caractères se rencontrent.\par
Chez les personnes moroses ou âgées l’amitié naît moins fréquemment en tant qu’elles ont l’humeur trop chagrine et se plaisent médiocrement aux fréquentations, alors que les qualités opposées sont considérées comme les marques les plus caractéristiques de l’amitié et les plus favorables à sa production. \\
Aussi, tandis que les jeunes gens deviennent rapidement amis, pour les vieillards il en est tout différemment : car on ne devient pas amis de gens avec lesquels on n’éprouve aucun sentiment de joie. Même observation pour les personnes de caractère morose. Il est vrai que ces deux sortes de gens peuvent ressentir de la bienveillance les uns pour les autres (ils se souhaitent du bien, et vont au secours l’un de l’autre dans leurs besoins) ; mais on peut difficilement les appeler des amis, pour la raison qu’ils ne vivent pas ensemble, ni ne se plaisent \\
les uns avec les autres : or ce sont là les deux principaux caractères qu’on reconnaît à l’amitié.\par
On ne peut pas être un ami pour plusieurs personnes, dans l’amitié parfaite, pas plus qu’on ne peut être amoureux de plusieurs personnes en même temps (car l’amour est une sorte d’excès, et un état de ce genre n’est naturellement ressenti qu’envers un seul) ; et peut-être même n’est-il pas aisé de trouver un grand nombre de gens de bien. On doit aussi acquérir quelque expérience de son ami et entrer dans son \\
intimité, ce qui est d’une extrême difficulté. Par contre, si on recherche l’utilité ou le plaisir, il est possible de plaire à beaucoup de personnes, car nombreux sont les gens de cette sorte, et les services qu’on en reçoit ne se font pas attendre longtemps.\par
De ces deux dernières formes d’amitié, celle qui repose sur le plaisir ressemble davantage à la véritable amitié, quand les deux parties retirent à la fois les mêmes satisfactions l’une de l’autre et qu’elles ressentent une joie mutuelle ou se plaisent \\
aux mêmes choses : telles sont les amitiés entre jeunes gens, car il y a en elles plus de générosité : au contraire, l’amitié basée sur l’utilité est celle d’âmes mercantiles. Quand à ceux qui sont comblés par la vie, ils ont besoin non pas d’amis utiles, mais d’amis agréables, parce qu’ils souhaitent vivre en compagnie de quelques personnes ; et bien qu’ils puissent supporter un court temps ce qui leur est pénible, ils ne pourraient jamais l’endurer d’une façon continue, pas plus qu’ils ne le pourraient \\
même pour le Bien en soi, s’il leur était à charge. C’est pourquoi les gens heureux recherchent les amis agréables. Sans doute devraient-ils aussi rechercher des amis qui, tout en ayant cette dernière qualité, soient aussi gens de bien, et en outre bons et plaisants pour eux, possédant ainsi tous les caractères exigés de l’amitié.\par
Les hommes appartenant aux classes dirigeantes ont, c’est un fait, leurs amis séparés en groupes distincts : les uns leur sont utiles, et d’autres agréables, mais ce sont rarement les mêmes à \\
la fois. Ils ne recherchent pour amis ni ceux dont l’agrément s’accompagne de vertu, ni ceux dont l’utilité servirait de nobles desseins, mais ils veulent des gens d’esprit quand ils ont envie de s’amuser, et quant aux autres ils les veulent habiles à exécuter leurs ordres, toutes exigences qui se rencontrent rarement dans la même personne. Nous avons dit que l’homme de bien est en même temps utile et agréable, mais un tel homme ne devient pas ami d’un autre occupant une position sociale plus \\
élevée, à moins que cet autre ne le surpasse aussi en vertu : sinon, l’homme de bien, surpassé par le supérieur, ne peut réaliser une égalité proportionnelle. Mais on n’est pas habitué à rencontrer fréquemment des hommes puissants de cette espèce.
\subsection[{8 (1158b) < L’égalité et l’inégalité dans l’amitié >}]{8 (1158b) < L’égalité et l’inégalité dans l’amitié >}
\noindent  Quoi qu’il en soit, les amitiés dont nous avons parlé impliquent égalité : les deux parties retirent les mêmes avantages l’une de l’autre et se souhaitent réciproquement les mêmes biens, ou encore échangent une chose contre une autre, par exemple plaisir contre profit. Nous avons dit que ces dernières formes de l’amitié sont d’un ordre inférieur et durent \\
moins longtemps. Mais du fait qu’à la fois elles ressemblent et ne ressemblent pas à la même chose, on peut aussi bien penser qu’elles sont des amitiés et qu’elles n’en sont pas : par leur ressemblance, en effet, avec l’amitié fondée sur la vertu, elles paraissent bien être des amitiés (car l’une comporte le plaisir et l’autre l’utilité, et ces caractères appartiennent aussi à l’amitié fondée sur la vertu) ; par contre, du fait que l’amitié basée sur la vertu est à l’abri des traverses et demeure stable, tandis que les autres amitiés changent rapidement et diffèrent en outre de \\
la 179 première sur beaucoup d’autres points, ces amitiés-là ne semblent pas être des amitiés, à cause de leur dissemblance avec l’amitié véritable.\par
Mais il existe une autre espèce d’amitié, c’est celle qui comporte une supériorité d’une partie sur l’autre, par exemple l’affection d’un père à l’égard de son fils, et, d’une manière générale, d’une personne plus âgée à l’égard d’une autre plus jeune, ou encore celle du mari envers sa femme, ou d’une personne exerçant une autorité quelconque envers un inférieur. Ces diverses amitiés diffèrent aussi entre elles : l’affection \\
des parents pour leurs enfants n’est pas la même que celle des chefs pour leurs inférieurs ; bien plus, celle du père pour son fils n’est pas la même que celle du fils pour son père, ni celle du mari pour sa femme la même que celle de la femme pour son mari. En effet, chacune de ces personnes a une vertu et une fonction différentes, et différentes sont aussi les raisons qui les font s’aimer : il en résulte une différence dans les attachements \\
et les amitiés. Dès lors il n’y a pas identité dans les avantages que chacune des parties retire de l’autre, et elles ne doivent pas non plus y prétendre ; mais quand les enfants rendent à leurs parents ce qu’ils doivent aux auteurs de leurs jours et que les parents rendent à leurs enfants ce qu’ils doivent à leur progéniture, l’amitié entre de telles personnes sera stable et équitable. Et dans toutes les amitiés comportant supériorité, il faut \\
aussi que l’attachement soit proportionnel : ainsi, celui qui est meilleur que l’autre doit être aimé plus qu’il n’aime ; il en sera de même pour celui qui est plus utile, et pareillement dans chacun des autres cas. Quand, en effet, l’affection est fonction du mérite des parties, alors il se produit une sorte d’égalité, égalité qui est considérée comme un caractère propre de l’amitié.
\subsection[{9 (1158b — 1159a) < L’égalité dans la justice et dans l’amitié. Amitié donnée et amitié rendue >}]{9 (1158b — 1159a) < L’égalité dans la justice et dans l’amitié. Amitié donnée et amitié rendue >}
\noindent Mais l’égalité ne semble pas revêtir la même forme dans le \\
domaine des actions justes et dans l’amitié. Dans le cas des actions justes, l’égal au sens premier est ce qui est proportionné au mérite, tandis que l’égal en quantité n’est qu’un sens dérivé ; au contraire, dans l’amitié l’égal en quantité est le sens premier, et l’égal proportionné au mérite, le sens secondaire. Ce que nous disons là saute aux yeux, quand une disparité considérable se produit sous le rapport de la vertu, ou du vice, ou des ressources matérielles, ou de quelque autre chose : les amis ne sont plus longtemps amis, et ils ne prétendent même \\
pas à le rester. Mais le cas le plus frappant est celui des dieux, chez qui la supériorité en toute espèce de biens est la plus  indiscutable. Mais on le voit aussi quand il s’agit des rois : en ce qui les concerne, les hommes d’une situation par trop inférieure ne peuvent non plus prétendre à leur amitié, pas plus d’ailleurs que les gens dépourvus de tout mérite ne songent à se lier avec les hommes les plus distingués par leur excellence ou leur sagesse. Il est vrai qu’en pareil cas on ne peut déterminer avec précision jusqu’à quel point des amis sont encore des amis : les motifs sur lesquels elle repose disparaissant en grande partie, l’amitié persiste encore. Toutefois si l’un des amis est séparé par un intervalle considérable, comme par \\
exemple Dieu est éloigné de l’homme, il n’y a plus d’amitié possible. C’est même ce qui a donné lieu à la question de savoir si, en fin de compte, les amis souhaitent vraiment pour leurs amis les biens les plus grands, comme par exemple d’être des dieux, car alors ce ne seront plus des amis pour eux, ni par suite des biens, puisque les amis sont des biens. Si donc nous avons eu raison de dire que l’ami désire du bien à son ami en vue de cet ami même, celui-ci devrait demeurer ce qu’il est, quel qu’il puisse être, tandis que l’autre souhaitera à son ami seulement \\
les plus grands biens compatibles avec la persistance de sa nature d’homme. Peut-être même ne lui souhaitera-t-il pas tous les plus grands biens, car c’est surtout pour soi-même que tout homme souhaite les choses qui sont bonnes.\par
La plupart des hommes, poussés par le désir de l’honneur, paraissent souhaiter être aimés plutôt qu’aimer (de là vient qu’on aime généralement les flatteurs, car le flatteur est un ami \\
en état d’infériorité ou qui fait du moins semblant d’être tel et d’aimer plus qu’il n’est aimé) ; or être aimé et être honoré sont, semble-t-il, des notions très rapprochées, et c’est à être honorés que la majorité des hommes aspirent. Mais il apparaît qu’on ne choisit pas l’honneur pour lui-même, mais seulement par accident. En effet, on se plaît la plupart du temps à recevoir des marques de considération de la part des hommes en place, \\
en raison des espérances qu’ils font naître (car on pense obtenir d’eux ce dont on peut avoir besoin, quoi que ce soit : dès lors, c’est comme signe d’un bienfait à recevoir qu’on se réjouit de l’honneur qu’ils vous rendent). Ceux qui, d’autre part, désirent être honorés par les gens de bien et de savoir, aspirent, ce faisant, à renforcer leur propre opinion sur eux-mêmes. Ils se réjouissent dès lors de l’honneur qu’ils reçoivent, parce qu’ils sont assurés de leur propre valeur morale sur la foi du jugement \\
porté par ceux qui la répandent. D’un autre côté, on se réjouit d’être aimé par cela même. Il résulte de tout cela qu’être aimé peut sembler préférable à être honoré, et que l’amitié est désirable par elle-même.\par
Mais il paraît bien que l’amitié consiste plutôt à aimer qu’à être aimé. Ce qui le montre bien, c’est la joie que les mères ressentent à aimer leurs enfants. Certaines les mettent en nourrice, et elles les aiment en sachant qu’ils sont leurs enfants, \\
mais ne cherchent pas à être aimées en retour, si les deux choses à la fois ne sont pas possibles, mais il leur paraît suffisant de les voir prospérer ; et elles-mêmes aiment leurs enfants même si ces derniers ne leur rendent rien de qui est dû à une mère, à cause de l’ignorance où ils se trouvent.
\subsection[{10 (1159a — 1159b) < Amitié active et amitié passive, suite. Amitiés entre inégaux >}]{10 (1159a — 1159b) < Amitié active et amitié passive, suite. Amitiés entre inégaux >}
\noindent Étant donné que l’amitié consiste plutôt dans le fait d’aimer, et qu’on loue ceux qui aiment leurs amis, il semble bien qu’aimer soit la vertu des amis, de sorte que ceux dans lesquels ce sentiment se rencontre proportionné au mérite de \\
leur ami, sont des amis constants, et leur amitié l’est aussi. —  C’est de cette façon surtout que même les hommes de condition inégale peuvent être amis, car ils seront ainsi rendus égaux. Or l’égalité et la ressemblance constituent l’affection, particulièrement la ressemblance de ceux qui sont semblables en vertu : car étant stables en eux-mêmes, ils le demeurent aussi dans leurs rapports mutuels, et ils ne demandent ni ne rendent des services dégradants, mais on peut même dire qu’ils y \\
mettent obstacle : car le propre des gens vertueux c’est à la fois d’éviter l’erreur pour eux-mêmes et de ne pas la tolérer chez leurs amis. Les méchants, au contraire, n’ont pas la stabilité, car ils ne demeurent même pas semblables à eux-mêmes ; mais ils ne deviennent amis que pour un temps fort court, se délectant à leur méchanceté réciproque. Ceux dont l’amitié repose sur l’utilité ou le plaisir demeurent amis plus longtemps que \\
les précédents, à savoir aussi longtemps qu’ils se procurent réciproquement des plaisirs ou des profits.\par
C’est l’amitié basée sur l’utilité qui, semble-t-il, se forme le plus fréquemment à partir de personnes de conditions opposées : par exemple l’amitié d’un pauvre pour un riche, d’un ignorant pour un savant : car quand on se trouve dépourvu d’une chose dont on a envie, on donne une autre chose en \\
retour pour l’obtenir. On peut encore ranger sous ce chef le lien qui unit un amant et son aimé, un homme beau et un homme laid. C’est pourquoi l’amant apparaît parfois ridicule, quand il a la prétention d’être aimé comme il aime : s’il était pareillement aimable, sans doute sa prétention serait-elle justifiée, mais s’il n’a rien de tel à offrir, elle est ridicule.\par
\\
Mais peut-être le contraire ne tend-il pas au contraire par sa propre nature, mais seulement par accident, le désir ayant en réalité pour objet le moyen, car le moyen est ce qui est bon : ainsi il est bon pour le sec non pas de devenir humide, mais d’atteindre à l’état intermédiaire, et pour le chaud et les autres qualités il en est de même.
\subsection[{11 (1159b — 1160a) < Amitié et justice. Les types d’amitié. Associations particulières et cité >}]{11 (1159b — 1160a) < Amitié et justice. Les types d’amitié. Associations particulières et cité >}
\noindent Mais laissons de côté ces dernières considérations (et de fait elles sont par trop étrangères à notre sujet).\par
\\
Il semble bien, comme nous l’avons dit au début, que l’amitié et la justice ont rapport aux mêmes objets et interviennent entre les mêmes personnes. En effet, en toute communauté, on trouve, semble-t-il, quelque forme de justice et aussi d’amitié coextensive : aussi les hommes appellent-ils du nom d’amis leurs compagnons de navigation et leurs compagnons d’armes, ainsi que ceux qui leur sont associés dans les autres \\
genres de communauté. Et l’étendue de leur association est la mesure de l’étendue de leurs droits. En outre, le proverbe {\itshape ce que possèdent des amis est commun}, est bien exact, car c’est dans une mise en commun que consiste l’amitié. Il y a entre frères ainsi qu’entre camarades communauté totale, mais pour les autres amis la mise en commun ne porte que sur des choses déterminées, plus ou moins nombreuses suivant les cas : car les amitiés aussi suivent les mêmes variations en plus ou en moins. \\
Les rapports de droit admettent aussi des différences : les  droits des parents et des enfants ne sont pas les mêmes que ceux des frères entre eux, ni ceux des camarades les mêmes que ceux des citoyens ; et il en est de même pour les autres formes d’amitié. Il y a par suite aussi des différences en ce qui concerne les injustices commises dans chacune de ces différentes classes d’associés, et l’injustice acquiert un surcroît de gravité quand elle s’adresse davantage à des amis : par exemple, il est plus \\
choquant de dépouiller de son argent un camarade qu’un concitoyen, plus choquant de refuser son assistance à un frère qu’à un étranger, plus choquant enfin de frapper son père qu’une autre personne quelconque. Et il est naturel aussi que la justice croisse en même temps que l’amitié, attendu que l’une et l’autre existent entre les mêmes personnes et possèdent une égale extension.\par
Mais toutes les communautés ne sont, pour ainsi dire, que des fractions de la communauté politique. On se réunit, par \\
exemple, pour voyager ensemble en vue de s’assurer quelque avantage déterminé, et de se procurer quelqu’une des choses nécessaires à la vie ; et c’est aussi en vue de l’avantage de ses membres, pense-t-on généralement, que la communauté politique s’est constituée à l’origine et continue à se maintenir. Et cette utilité commune est le but visé par les législateurs, qui appellent juste ce qui est à l’avantage de tous. Ainsi les autres \\
communautés recherchent leur avantage particulier : par exemple les navigateurs, en naviguant ensemble, ont en vue l’avantage d’acquérir de l’argent ou quelque chose d’analogue ; pour les compagnons d’armes, c’est le butin, que ce soit richesses, ou victoire, ou prise d’une ville qu’ils désirent ; et c’est le cas également des membres d’une tribu ou d’un dème. [Certaines communautés semblent avoir pour origine l’agrément, par exemple celles qui unissent les membres d’un thiase ou d’un cercle dans lequel chacun paye son écot, associations \\
constituées respectivement en vue d’offrir un sacrifice ou d’entretenir des relations de société. Mais toutes ces communautés semblent bien être subordonnées à la communauté politique, car la communauté politique n’a pas pour but l’avantage présent, mais ce qui est utile à la vie tout entière], qui offrent des sacrifices et tiennent des réunions à cet effet, rendant ainsi des honneurs aux dieux et se procurant en même \\
temps pour eux-mêmes des distractions agréables. En effet, les sacrifices et les réunions d’ancienne origine ont lieu, c’est un fait, après la récolte des fruits et présentent le caractère d’une offrande des prémices : car c’est la saison de l’année où le peuple avait le plus de loisir. Toutes ces communautés sont donc manifestement des fractions de la communauté politique, et les espèces particulières d’amitiés correspondent \\
aux espèces particulières de communautés.
\subsection[{12 (1160a — 1161a) < Constitutions politiques et amitiés correspondantes >}]{12 (1160a — 1161a) < Constitutions politiques et amitiés correspondantes >}
\noindent Il y a trois espèces de constitutions et aussi un nombre égal de déviations, c’est-à-dire de corruptions auxquelles elles sont sujettes. Les constitutions sont la royauté, l’aristocratie et en troisième lieu celle qui est fondée sur le cens et qui, semble-t-il, peut recevoir le qualificatif approprié de {\itshape timocratie}, quoique en fait on a coutume de l’appeler la \\
plupart du temps {\itshape république}. La meilleure de ces constitutions est la royauté, et la plus mauvaise la timocratie. La déviation  de la royauté est la tyrannie. Toutes deux sont des monarchies, mais elles diffèrent du tout au tout : le tyran n’a en vue que son avantage personnel, tandis que le roi a en vue celui de ses sujets. En effet, n’est pas réellement roi celui qui ne se suffit pas à lui-même, c’est-à-dire ne possède pas la supériorité en \\
toutes sortes de biens ; mais le roi tel que nous le supposons, n’ayant besoin de rien de plus qu’il n’a, n’aura pas en vue ses propres intérêts mais ceux de ses sujets, car le roi ne possédant pas ces caractères ne serait qu’un roi désigné par le sort. La tyrannie est tout le contraire de la royauté, car le tyran poursuit son bien propre. Et on aperçoit plus clairement dans le cas de la tyrannie qu’elle est la pire des déviations, le contraire de ce qu’il y a de mieux étant ce qu’il y a de plus mauvais.\par
\\
De la royauté on passe à la tyrannie, car la tyrannie est une perversion de monarchie, et dès lors le mauvais roi devient tyran. De l’aristocratie on passe à l’oligarchie par le vice des gouvernants qui distribuent ce qui appartient à la cité sans tenir compte du mérite, et s’attribuent à eux-mêmes tous les biens ou la plupart d’entre eux, et réservent les magistratures \\
toujours aux mêmes personnes, ne faisant cas que de la richesse : dès lors le gouvernement est aux mains d’un petit nombre d’hommes pervers au lieu d’appartenir aux plus capables. De la timocratie on passe à la démocratie : elles sont en effet limitrophes, puisque la timocratie a aussi pour idéal le règne de la majorité, et que sont égaux tous ceux qui répondent aux conditions du cens. La démocratie est la moins mauvaise \\
< des gouvernements corrompus >, car elle n’est qu’une légère déviation de la forme du gouvernement républicain. — Telles sont donc les transformations auxquelles les constitutions sont surtout exposées (car ce sont là des changements minimes et qui se produisent le plus facilement).\par
On peut trouver des ressemblances à ces constitutions, des modèles en quelque sorte, jusque dans l’organisation domestique. En effet, la communauté existant entre un père et ses \\
enfants est de type royal (puisque le père prend soin de ses enfants ; de là vient qu’Homère désigne Zeus du nom de {\itshape père}, car la royauté a pour idéal d’être un gouvernement paternel). Chez les Perses, l’autorité paternelle est tyrannique (car ils se servent de leurs enfants comme d’esclaves). Tyrannique aussi est l’autorité du maître sur ses esclaves (l’avantage du maître \\
s’y trouvant seul engagé ; or si cette dernière sorte d’autorité apparaît comme légitime, l’autorité paternelle de type perse est au contraire fautive, car des relations différentes appellent des formes de commandement différentes). La communauté du mari et de sa femme semblent être de type aristocratique (le mari exerçant l’autorité en raison de la dignité de son sexe, et dans des matières où la main d’un homme doit se faire sentir ; \\
mais les travaux qui conviennent à une femme, il les lui abandonne). Quand le mari étend sa domination sur toutes choses, il transforme la communauté conjugale en oligarchie (puisqu’il  agit ainsi en violation de ce qui sied à chaque époux, et non en vertu de sa supériorité). Parfois cependant ce sont les femmes qui gouvernent quand elles sont héritières, mais alors leur autorité ne s’exerce pas en raison de l’excellence de la personne, mais elle est due à la richesse et au pouvoir, tout comme dans les oligarchies. La communauté entre frères est semblable à une timocratie (il y a égalité entre eux, sauf dans la mesure où \\
ils diffèrent par l’âge ; et c’est ce qui fait précisément que si la différence d’âge est considérable, l’affection qui les unit n’a plus rien de fraternel). La démocratie se rencontre principalement dans les demeures sans maîtres (car là tous les individus sont sur un pied d’égalité), et dans celles où le chef est faible et où chacun a licence de faire ce qui lui plaît.
\subsection[{13 (1161a — 1161b) < Formes de l’amitié correspondant aux constitutions politiques >}]{13 (1161a — 1161b) < Formes de l’amitié correspondant aux constitutions politiques >}
\noindent \\
Pour chaque forme de constitution on voit apparaître une amitié, laquelle est coextensive aussi aux rapports de justice. L’affection d’un roi pour ses sujets réside dans une supériorité de bienfaisance, car un roi fait du bien à ses sujets si, étant lui-même bon, il prend soin d’eux en vue d’assurer leur prospérité, comme un berger le fait pour son troupeau. De là \\
vient qu’Homère a appelé Agamemnon {\itshape pasteur des peuples}. De même nature est aussi l’amour paternel, lequel cependant l’emporte ici par la grandeur des services rendus, puisque le père est l’auteur de l’existence de son enfant (ce qui de l’avis général est le plus grand des dons), ainsi que de son entretien et de son éducation ; et ces bienfaits sont attribués également aux ancêtres. Et, de fait, c’est une chose naturelle qu’un père gouverne ses enfants, les ancêtres leurs descendants, et un roi ses sujets. Ces diverses amitiés impliquent supériorité \\
< de bienfaits de la part d’une des parties >, et c’est pourquoi encore les parents sont honorés par leurs enfants. Dès lors, les rapports de justice entre les personnes dont nous parlons ne sont pas identiques des deux côtés, mais sont proportionnés au mérite de chacun, comme c’est le cas aussi de l’affection qui les unit.\par
L’affection entre mari et femme est la même que celle qu’on trouve dans le régime aristocratique, puisqu’elle est proportionnée à l’excellence personnelle, et qu’au meilleur revient une plus large part de biens, chaque époux recevant ce qui lui est exactement approprié ; et il en est ainsi encore pour les rapports de justice.\par
\\
L’affection entre frères ressemble à celle des camarades : ils sont, en effet, égaux et de même âge, et tous ceux qui remplissent cette double condition ont la plupart du temps mêmes sentiments et même caractère. Pareille à l’affection fraternelle est celle qui existe dans le régime timocratique, car ce gouvernement a pour idéal l’égalité et la vertu des citoyens, de sorte que le commandement appartient à ces derniers à tour de rôle et que tous y participent sur un pied d’égalité. Cette égalité caractérise aussi l’amitié correspondante.\par
\\
Dans les formes déviées de constitutions, de même que la justice n’y tient qu’une place restreinte, ainsi en est-il de l’amitié, et elle est réduite à un rôle insignifiant dans la forme la plus pervertie, je veux dire dans la tyrannie, où l’amitié est nulle ou faible. En effet, là où il n’y a rien de commun entre gouvernant et gouverné, il n’y a non plus aucune amitié, puisqu’il n’y a pas même de justice : il en est comme dans la \\
relation d’un artisan avec son outil, de l’âme avec le corps,  d’un maître avec son esclave : tous ces instruments sans doute peuvent être l’objet de soins de la part de ceux qui les emploient, mais il n’y a pas d’amitié ni de justice envers les choses inanimées. Mais il n’y en a pas non plus envers un cheval ou un bœuf, ni envers un esclave en tant qu’esclave. Dans ce dernier cas, les deux parties n’ont en effet rien de commun : l’esclave est un outil animé, et l’outil un esclave \\
inanimé. En tant donc qu’il est esclave on ne peut pas avoir d’amitié pour lui, mais seulement en tant qu’il est homme, car de l’avis général il existe certains rapports de justice entre un homme, quel qu’il soit, et tout autre homme susceptible d’avoir participation à la loi ou d’être partie à un contrat ; dès lors il peut y avoir aussi amitié avec lui, dans la mesure où il est homme. Par suite encore, tandis que dans les tyrannies l’amitié et la justice ne jouent qu’un faible rôle, dans les démocraties au \\
contraire leur importance est extrême : car il y a beaucoup de choses communes là où les citoyens sont égaux.
\subsection[{14 (1161b — 1162a) < L’affection entre parents et entre époux >}]{14 (1161b — 1162a) < L’affection entre parents et entre époux >}
\noindent C’est donc au sein d’une communauté que toute amitié se réalise, ainsi que nous l’avons dit. On peut cependant mettre à part du reste, à la fois l’affection entre parents et celle entre camarades. L’amitié qui unit les membres d’une cité ou d’une tribu ou celle contractée au cours d’une traversée commune, et tous autres liens de ce genre, se rapprochent davantage \\
des amitiés caractérisant les membres d’une communauté, car elles semblent reposer pour ainsi dire sur une convention déterminée. Dans ce dernier groupe on peut ranger l’amitié à l’égard des étrangers.\par
L’affection entre parents apparaît revêtir plusieurs formes, mais toutes semblent se rattacher à l’amour paternel. Les parents, en effet, chérissent leurs enfants comme étant quelque chose d’eux-mêmes, et les enfants leurs parents comme étant quelque chose d’où ils procèdent. Or, d’une part, les parents \\
savent mieux que leur progéniture vient d’eux-mêmes que les enfants ne savent qu’ils viennent de leurs parents, et, d’autre part, il y a communauté plus étroite du principe d’existence à l’égard de l’être engendré que de l’être engendré à l’égard de la cause fabricatrice : car ce qui procède d’une chose appartient proprement à la chose dont il sort (une dent, par exemple, un cheveu, n’importe quoi, à son possesseur), tandis que le principe d’existence n’appartient nullement à ce qu’il a produit, ou du moins lui appartient à un plus faible degré. Et l’affection des parents l’emporte encore en longueur de temps : les parents \\
chérissent leurs enfants aussitôt nés, alors que ceux-ci n’aiment leurs parents qu’au bout d’un certain temps, quand ils ont acquis intelligence ou du moins perception. Ces considérations montrent clairement aussi pour quelles raisons l’amour de la mère est plus fort que celui du père. Ainsi les parents aiment leurs enfants comme eux-mêmes (les êtres qui procèdent d’eux sont comme d’autres eux-mêmes, « autres » du fait qu’ils sont séparés du père), et les enfants aiment leurs parents \\
comme étant nés d’eux ; les frères s’aiment entre eux comme étant nés des mêmes parents, car leur identité avec ces derniers les rend identiques entre eux, et de là viennent les expressions {\itshape être du même sang, de la même souche}, et autres semblables. Les frères sont par suite la même chose en un sens, mais dans des individus distincts. Ce qui contribue grandement aussi à l’affection entre eux, c’est l’éducation commune et la similitude d’âge : {\itshape les jeunes se plaisent avec ceux de leur âge} ; et \\
{\itshape des habitudes communes engendrent la camaraderie}, et c’est pourquoi l’amitié entre frères est semblable à celle entre camarades.  La communauté de sentiments entre cousins ou entre les autres parents dérive de celle des frères entre eux, parce qu’ils descendent des mêmes ancêtres. Mais ils se sentent plus étroitement unis ou plus étrangers l’un à l’autre suivant la proximité ou l’éloignement de l’ancêtre originel.\par
L’amour des enfants pour leurs parents, comme l’amour \\
des hommes pour les dieux, est celui qu’on ressent pour un être bon et qui nous est supérieur : car les parents ont concédé à leurs enfants les plus grands des bienfaits en leur donnant la vie, en les élevant, et en assurant une fois nés leur éducation. Et cet amour entre parents et enfants possède encore en agrément et en utilité une supériorité par rapport à l’affection qui unit des personnes étrangères, supériorité qui est d’autant plus grande que leur communauté de vie est plus étroite. On trouve \\
aussi dans l’amitié entre frères tout ce qui caractérise l’amitié soit entre camarades (et à un plus haut degré entre camarades vertueux), soit, d’une façon générale, entre personnes semblables l’une à l’autre ; cette amitié est d’autant plus forte que les frères sont plus intimement unis et que leur affection réciproque remonte à la naissance ; d’autant plus forte encore, qu’une plus grande conformité de caractère existe entre les individus nés des mêmes parents, élevés ensemble et ayant reçu la même éducation ; et c’est dans leur cas que l’épreuve du temps se montre la plus décisive et la plus sûre. Entre les autres \\
parents les degrés de l’amitié varient proportionnellement.\par
L’amour entre mari et femme semble bien être conforme à la nature, car l’homme est un être naturellement enclin à former un couple, plus même qu’à former une société politique, dans la mesure où la famille est quelque chose d’antérieur à la cité et de plus nécessaire qu’elle, et la procréation des enfants une chose plus commune aux êtres vivants. Quoi qu’il \\
en soit, chez les animaux la communauté ne va pas au-delà de la procréation, tandis que dans l’espèce humaine la cohabitation de l’homme et de la femme n’a pas seulement pour objet la reproduction, mais s’étend à tous les besoins de la vie : car la division des tâches entre l’homme et la femme a lieu dès l’origine, et leurs fonctions ne sont pas les mêmes ; ainsi, ils se portent une aide mutuelle, mettant leurs capacités propres au service de l’œuvre commune. C’est pour ces raisons que l’utilité et l’agrément semblent se rencontrer à la fois dans l’amour \\
conjugal. Mais cet amour peut aussi être fondé sur la vertu, quand les époux sont gens de bien : car chacun d’eux a sa vertu propre, et tous deux mettront leur joie en la vertu de l’autre. Les enfants aussi, semble-t-il, constituent un trait d’union, et c’est pourquoi les époux sans enfants se détachent plus rapidement l’un de l’autre : les enfants, en effet, sont un bien commun aux deux, et ce qui est commun maintient l’union.\par
La question de savoir quelles sont les règles qui président \\
aux relations mutuelles du mari et de la femme, et, d’une manière générale, des amis entre eux, apparaît comme n’étant rien d’autre que de rechercher les règles concernant les rapports de justice entre ces mêmes personnes : car la justice ne se manifeste pas de la même manière à l’égard d’un ami, d’un étranger, d’un camarade ou d’un condisciple.
\subsection[{15 (1162a — 1163a) < Règles pratiques relatives à l’amitié entre égaux. — L’amitié utilitaire >}]{15 (1162a — 1163a) < Règles pratiques relatives à l’amitié entre égaux. — L’amitié utilitaire >}
\noindent Il existe donc trois espèces d’amitié, ainsi que nous l’avons \\
dit au début, et pour chaque espèce il y a à la fois les amis qui vivent sur un pied d’égalité, et ceux où l’une des parties l’emporte sur l’autre (car non seulement deux hommes également vertueux peuvent devenir amis, mais encore un homme plus vertueux peut se lier avec un moins vertueux ; pareillement,  pour l’amitié basée sur le plaisir ou l’utilité, il peut y avoir égalité ou disparité dans les avantages qui en découlent) : dans ces conditions, les amis qui sont égaux doivent réaliser l’égalité dans une égalité d’affection et du reste, et chez ceux qui sont inégaux, < la partie défavorisée réalisera cette égalité > en fournissant en retour un avantage proportionné à la supériorité, quelle qu’elle soit, de l’autre partie.\par
\\
Les griefs et les récriminations se produisent uniquement, ou du moins principalement, dans l’amitié fondée sur l’utilité, et il n’y a rien là que de naturel. En effet, ceux dont l’amitié repose sur la vertu s’efforcent de se faire réciproquement du bien (car c’est le propre de la vertu et de l’amitié), et entre gens qui rivalisent ainsi pour le bien, il ne peut y avoir ni plaintes ni querelles (nul, en effet, n’éprouve d’indignation envers la \\
personne qui l’aime et qui lui fait du bien, mais au contraire, si on a soi-même quelque délicatesse, on lui rend la pareille en bons offices. Et celui qui l’emporte décidément sur l’autre en bienfaits, atteignant ainsi le but qu’il se propose, ne saurait se plaindre de son ami, puisque chacun des deux aspire à ce qui est bien). Les récriminations ne sont pas non plus fréquentes entre amis dont l’affection repose sur le plaisir (tous deux, en effet, atteignent en même temps l’objet de leur désir, puisqu’ils \\
se plaisent à vivre ensemble ; et même on paraîtrait ridicule de reprocher à son ami de ne pas vous causer de plaisir, étant donné qu’il vous est loisible de ne pas passer vos journées avec lui).\par
Au contraire l’amitié basée sur l’utilité a toujours tendance à se plaindre : les amis de cette sorte se fréquentant par intérêt, ils demandent toujours davantage, s’imaginent avoir moins que leur dû et en veulent à leur ami parce qu’ils n’obtiennent \\
pas autant qu’ils demandent, eux qui en sont dignes ! De son côté, le bienfaiteur est dans l’incapacité de satisfaire à toutes les demandes de son obligé.\par
De même que la justice est de deux espèces, la justice non-écrite et la justice selon la loi, de même il apparaît que l’amitié utilitaire peut être soit morale soit légale. Et ainsi les griefs ont cours principalement quand les intéressés ont passé une convention et s’en acquittent en se réclamant d’un type \\
d’amitié qui n’est pas le même. Or l’amitié utilitaire de type légal est celle qui se réfère à des clauses déterminées ; l’une de ses variétés est purement mercantile, avec paiement de la main à la main ; l’autre variété est plus libérale pour l’époque du paiement, tout en conservant son caractère de contrat, obligeant à remettre une chose déterminée contre une autre chose (dans cette dernière variété, l’obligation est claire et sans ambiguïté, mais renferme cependant un élément affectif, à savoir le délai octroyé ; c’est pourquoi chez certains peuples il n’existe pas d’actions en justice pour sanctionner ces obligations, mais \\
on estime que ceux qui ont traité sous le signe de la confiance doivent en supporter les conséquences). Le type moral, d’autre part, ne se réfère pas à des conditions déterminées, mais le don ou tout autre avantage quelconque est consenti à titre amical, bien que celui qui en est l’auteur s’attende à recevoir en retour une valeur égale ou même supérieure, comme s’il n’avait pas fait un don mais un prêt ; et du fait qu’à l’expiration du contrat il n’est pas dans une situation aussi favorable qu’au moment où il a traité, il fera entendre des récriminations. La raison de cet \\
état de choses vient de ce que tous les hommes, ou la plupart d’entre eux, souhaitent assurément ce qui est noble, mais choisissent ce qui est profitable ; et s’il est beau de faire du bien sans espoir d’être payé de retour, il est profitable d’être  soi-même l’objet de la faveur d’autrui.\par
Dès lors, quand on le peut, il faut rendre l’équivalent de ce qu’on a reçu, et cela sans se faire prier : car on ne doit pas faire de quelqu’un son ami contre son gré. Reconnaissant par suite que nous avons commis une erreur au début en recevant un bienfait d’une personne qui n’avait pas à nous l’octroyer, puisqu’elle n’était pas notre ami et qu’elle n’agissait pas pour \\
le plaisir de donner, nous devons nous libérer comme si la prestation dont nous avons bénéficié résultait de clauses strictement déterminées. Effectivement, nous aurions à ce moment consenti à rendre, dans la mesure de nos moyens, une prestation équivalente, et, en cas d’impossibilité, celui qui nous a avantagé n’aurait pas compté sur cette réciprocité. Ainsi donc, si nous le pouvons, nous devons rendre l’équivalent. Mais dès le début nous ferons bien de considérer de quelle personne nous recevons les bons offices, et en quels termes l’accord est passé, de façon qu’on puisse en accepter le bénéfice sur les bases fixées, ou à défaut le décliner.\par
\\
Il y a discussion sur le point suivant : doit-on mesurer un service par l’utilité qu’en retire celui qui le reçoit et calculer sur cette base la rémunération à fournir en retour, ou bien faut-il considérer le prix qu’il coûte au bienfaiteur ? L’obligé dira que ce qu’il a reçu de son bienfaiteur était peu de chose pour ce dernier et qu’il aurait pu le recevoir d’autres personnes, minimisant ainsi l’importance du service qui lui est rendu. Le bienfaiteur, \\
en revanche, prétendra que ce qu’il a donné était la chose la plus importante de toutes celles dont il disposait, que personne d’autre n’était capable de la fournir, et qu’au surplus elle était concédée à un moment critique ou pour parer à un besoin urgent. Ne devons-nous pas dire que, dans l’amitié de type utilitaire, c’est l’avantage de l’obligé qui est la mesure ? C’est, en effet, l’obligé qui demande, tandis que l’autre vient à son aide dans l’idée qu’il recevra l’équivalent en retour ; ainsi l’assistance consentie a été à la mesure de l’avantage reçu par \\
l’obligé, et dès lors ce dernier doit rendre à l’autre autant qu’il en a reçu, ou même, ce qui 191 est mieux, davantage. — Dans les amitiés fondées sur la vertu, les griefs sont inexistants, et c’est le choix délibéré du bienfaiteur qui joue le rôle de mesure, car le choix est le facteur déterminant de la vertu et du caractère.
\subsection[{16 (1163a — 1163b) < Règles de conduite pour l’amitié entre personnes inégales >}]{16 (1163a — 1163b) < Règles de conduite pour l’amitié entre personnes inégales >}
\noindent Des différends se produisent aussi au sein des amitiés où existe une supériorité : car chacun des deux amis a la prétention \\
de recevoir une part plus grande que l’autre, mais cette prétention, quand elle se fait jour, entraîne la ruine de l’amitié. Le plus vertueux estime que c’est à lui que doit revenir la plus large part (puisque à l’homme vertueux on assigne d’ordinaire une part plus considérable) ; même état d’esprit chez celui qui rend plus de services, car un homme bon à rien n’a pas droit, disent ces gens-là, à une part égale : c’est une charge gratuite que l’on supporte et ce n’est plus de l’amitié, dès lors que les \\
avantages qu’on retire de l’amitié ne sont pas en rapport avec l’importance du travail qu’on accomplit. Ils pensent, en effet, qu’il doit en être de l’amitié comme d’une société de capitaux où les associés dont l’apport est plus considérable reçoivent une plus grosse part de bénéfices. Mais, d’un autre côté, l’ami dénué de ressources ou en état d’infériorité quelconque, tient un raisonnement tout opposé : à son avis, c’est le rôle d’un véritable ami que d’aider ceux qui ont besoin de lui. À quoi sert, \\
dira-t-il, d’être l’ami d’un homme de bien ou d’un homme puissant, si on n’a rien d’avantageux à en attendre ?\par
 Il semble bien que les deux parties aient des prétentions également justifiées, et que chacun des amis soit en droit de se faire attribuer, en vertu de l’amitié, une part plus forte que l’autre ; seulement ce ne sera pas une part de la même chose : à celui qui l’emporte en mérite on donnera plus d’honneur, et à celui qui a besoin d’assistance plus de profit matériel : car la vertu et la bienfaisance ont l’honneur pour récompense, et l’indigence, pour lui venir en aide, a le profit.\par
\\
Qu’il en soit encore ainsi dans les diverses organisations politiques, c’est là un fait notoire. On n’honore pas le citoyen qui ne procure aucun bien à la communauté : car ce qui appartient au patrimoine de la communauté est donné à celui qui sert les intérêts communs, et l’honneur est une de ces choses qui font partie du patrimoine commun. On ne peut pas, en effet, tirer à la fois de la communauté argent et honneur. De fait, personne ne supporte d’être dans une situation défavorisée en \\
toutes choses en même temps : par suite, celui qui amoindrit son patrimoine est payé en honneur, et celui qui accepte volontiers des présents, en argent, puisque la proportionnalité au mérite rétablit l’égalité et conserve l’amitié, ainsi que nous l’avons dit.\par
Telle est donc aussi la façon dont les amis de condition inégale doivent régler leurs relations : celui qui retire un avantage en argent ou en vertu doit s’acquitter envers l’autre en \\
honneur, payant avec ce qu’il peut. L’amitié, en effet, ne réclame que ce qui rentre dans les possibilités de chacun, et non ce que le mérite exigerait, chose qui, au surplus, n’est même pas toujours possible, comme par exemple dans le cas des honneurs que nous rendons aux dieux ou à nos parents : personne ne saurait avoir pour eux la reconnaissance qu’ils méritent, mais quand on les sert dans la mesure de son pouvoir on est regardé comme un homme de bien. Aussi ne saurait-on admettre qu’il fût permis à un fils de renier son père, bien qu’un père puisse \\
renier son fils : quand on doit, il faut s’acquitter, mais il n’est rien de tout ce qu’un fils ait pu faire qui soit à la hauteur des bienfaits qu’il a reçus de son père, de sorte qu’il reste toujours son débiteur. Cependant ceux envers qui on a des obligations ont la faculté de vous en décharger, et par suite un père peut le faire. En même temps, aucun père sans doute, de l’avis général, ne voudrait jamais faire abandon d’un enfant qui ne serait pas un monstre de perversité (car l’affection naturelle mise à part, il n’est pas dans l’humaine nature de repousser \\
l’assistance éventuelle d’un fils). Un fils au contraire, quand il est vicieux, évitera de venir en aide à son père ou du moins n’y mettra pas d’empressement : c’est que la plupart des gens souhaitent qu’on leur fasse du bien, mais se gardent d’en faire eux-mêmes aux autres, comme une chose qui ne rapporte rien.
\section[{Livre IX}]{Livre IX}\renewcommand{\leftmark}{Livre IX}

\subsection[{1 (1163b — 1164b) < Les amitiés d’espèces différentes. Fixation de la rémunération >}]{1 (1163b — 1164b) < Les amitiés d’espèces différentes. Fixation de la rémunération >}
\noindent \\
Les matières qui précèdent ont été suffisamment étudiées. Dans toutes les amitiés d’espèce différente, c’est la proportionnalité qui établit l’égalité entre les parties et qui préserve l’amitié, ainsi que nous l’avons indiqué : ainsi, dans la communauté politique, le cordonnier reçoit pour ses chaussures \\
une rémunération proportionnée à la valeur fournie, et de  même le tisserand et les autres artisans. Dans ce secteur on a institué une commune mesure, la monnaie, et c’est l’étalon auquel dès lors on rapporte toutes choses et avec lequel on les mesure. — Dans les relations amoureuses, l’amant se plaint parfois que son amour passionné ne soit pas payé de retour, quoique, le cas échéant, il n’y ait en lui rien d’aimable ; de son côté, \\
l’aimé se plaint fréquemment que l’autre, qui lui avait précédemment fait toutes sortes de promesses, n’en remplisse à présent aucune. Pareils dissentiments se produisent lorsque l’amant aime l’aimé pour le plaisir, tandis que l’aimé aime l’amant pour l’utilité, et que les avantages attendus ne se rencontrent ni dans l’un ni dans l’autre. Dans l’amitié basée sur ces motifs, une rupture a lieu quand les deux amis n’obtiennent pas les satisfactions en vue desquelles leur amitié s’était \\
formée : ce n’est pas, en effet, la personne en elle-même qu’ils chérissaient, mais bien les avantages qu’ils en attendaient, et qui n’ont rien de stable ; et c’est ce qui fait que de telles amitiés ne sont pas non plus durables, Au contraire, celle qui repose sur la similitude des caractères, n’ayant pas d’autre objet qu’elle-même, est durable, ainsi que nous l’avons dit.\par
Des dissentiments éclatent encore quand les amis obtiennent des choses autres que celles qu’ils désirent : car c’est en somme ne rien obtenir du tout que de ne pas obtenir ce qu’on a \\
en vue, On connaît l’histoire de cet amateur qui avait promis à un joueur de cithare de le payer d’autant plus cher que son jeu serait meilleur : au matin, quand le cithariste réclama l’exécution de la promesse, l’autre répondit qu’il avait déjà rendu plaisir pour plaisir. Certes, si tous deux avaient souhaité du plaisir, pareille solution eût été satisfaisante ; mais quand l’un veut de l’amusement et l’autre un gain matériel, si le premier obtient ce qu’il veut, et l’autre non, les conditions de leur \\
accord mutuel ne sauraient être remplies comme il faut : car la chose dont en fait on a besoin, c’est elle aussi qui intéresse, et c’est pour l’obtenir, elle, qu’on est prêt à donner soi-même ce qu’on a.\par
Mais auquel des deux appartient-il de fixer le prix ? Est-ce à celui dont le service émane ? Ne serait-ce pas plutôt à celui qui a bénéficié le premier de l’opération ? car, enfin, celui qui rend d’abord service paraît bien s’en remettre sur ce point à l’autre partie. Telle était, dit-on, la façon de faire de Protagoras212 : \\
quand il donnait des leçons sur un sujet quelconque, il invitait son élève à évaluer lui-même le prix des connaissances qu’il avait acquises, et il recevait le salaire ainsi fixé. Cependant, dans des circonstances de ce genre, certains préfèrent s’en tenir à l’adage {\itshape que le salaire convenu avec un ami < lui soit assuré >}. Mais ceux qui commencent par prendre l’argent, et qui ensuite ne font rien de ce qu’ils disaient, à cause de l’exagération de leurs promesses, sont l’objet de plaintes \\
bien naturelles, puisqu’ils n’accomplissent pas ce qu’ils ont accepté de faire. Cette façon de procéder est peut-être pour les Sophistes une nécessité, parce que personne ne voudrait donner de l’argent en échange de leurs connaissances. Ainsi donc, ces gens qu’on paie d’avance, s’ils ne remplissent pas les services pour lesquels ils ont reçu leur salaire, soulèvent à juste titre des récriminations.\par
Dans les cas où il n’existe pas de convention fixant la rémunération du service rendu, et où on agit par pure bienveillance pour son ami, aucune récrimination, avons-nous dit, \\
n’est à redouter (et, de fait, cette absence de tout dissentiment caractérise l’amitié fondée sur la vertu), et le montant de la  rémunération donnée en retour doit être fixé conformément au choix délibéré du bienfaiteur (puisque le choix délibéré est le fait d’un ami et en général de la vertu). Telle est encore, semble-t-il, la façon de nous acquitter envers ceux qui nous ont dispensé leur enseignement philosophique ; car sa valeur n’est pas mesurable en argent, et aucune marque de considération ne saurait non plus entrer en balance avec le service rendu, mais \\
sans doute suffit-il, comme dans nos rapports avec les dieux et avec nos parent, de nous acquitter dans la mesure où nous le pouvons. Quand, au contraire, le service accordé ne présente pas ce caractère de gratuité mais qu’il est fait pour quelque avantage corrélatif, la meilleure solution sera sans doute que la rémunération payée en retour soit celle qui semble aux deux parties conforme à la valeur du service ; et si l’accord des parties ne peut se réaliser, il semblera non seulement nécessaire, \\
mais juste, que ce soit la partie ayant bénéficié d’abord du service qui fixe le montant de la rémunération, puisque l’autre partie, en recevant en compensation l’équivalent de l’avantage conféré au bénéficiaire ou le prix librement consenti par ce dernier en échange du plaisir, recouvrera ainsi du bénéficiaire le prix justement dû. Pour les marchandises mises en vente, en effet, c’est manifestement encore de cette façon-là qu’on procède ; et dans certains pays il existe même des lois refusant route action en justice pour les transactions de gré à gré, en vertu de cette idée qu’il convient, quand on fait confiance à quelqu’un, de s’acquitter envers lui dans le même esprit qui a \\
présidé à la formation du contrat. Dans la pensée du législateur, en effet, il est plus juste d’abandonner la fixation du prix à la personne en qui on a mis sa confiance qu’à celle qui s’est confiée. C’est que, la plupart du temps, le possesseur d’une chose ne lui attribue pas la même valeur que celui qui souhaite l’acquérir : chacun, c’est là un fait notoire, estime à haut prix les choses qui lui appartiennent en propre et celles qu’il donne. Il n’en est pas moins vrai que la rémunération fournie en \\
retour est évaluée au taux fixé par celui qui reçoit la chose. Mais sans doute faut-il que ce dernier apprécie la chose non pas à la valeur qu’elle présente pour lui quand il l’a en sa possession, mais bien à la valeur qu’il lui attribuait avant de la posséder.
\subsection[{2 (1164b — 1165a) < Conflits entre les diverses formes de l’amitié >}]{2 (1164b — 1165a) < Conflits entre les diverses formes de l’amitié >}
\noindent Une difficulté est également soulevée par des questions telles que celle-ci : doit-on tout concéder à son père et lui obéir en toutes choses, ou bien quand on est malade doit-on plutôt faire confiance à son médecin, et, dans le choix d’un stratège, faut-il plutôt voter pour l’homme apte à la guerre ? Pareillement, \\
doit-on rendre service à un ami plutôt qu’à un homme de bien, doit-on montrer sa reconnaissance à un bienfaiteur plutôt que faire un don à un camarade, si on ne peut pas accomplir les deux choses à la fois ?\par
N’est-il pas vrai que, pour toutes les questions de ce genre, il n’est pas facile de déterminer une règle précise ? (Elles comportent, en effet, une foule de distinctions de toutes sortes, d’après l’importance plus ou moins grande du service rendu, et \\
la noblesse ou la nécessité d’agir). Mais que nous ne soyons pas tenus de tout concéder à la même personne, c’est un point qui n’est pas douteux. D’autre part, nous devons, la plupart du temps, rendre les bienfaits que nous avons reçus plutôt que de faire plaisir à nos camarades, tout comme nous avons l’obligation de rembourser un prêt à notre créancier avant de donner de l’argent à un camarade. Et même ces règles ne sont-elles pas sans doute applicables dans tous les cas. Supposons, par exemple, un homme délivré, moyennant rançon, des mains des brigands : doit-il à son tour payer rançon pour délivrer son \\
propre libérateur, quel qu’il soit (ou même dans l’hypothèse où ce dernier n’a pas été enlevé par les brigands, mais demande  seulement à être rémunéré du service rendu, doit-il payer ?), ou ne doit-il pas plutôt racheter contre rançon son propre père ? Car on pensera qu’il doit faire passer l’intérêt de son père avant même le sien propre. Ainsi donc que nous venons de le dire, en règle générale on doit rembourser la dette contractée ; mais si cependant un don pur et simple l’emporte en noblesse morale ou en nécessité, c’est en faveur de ce don qu’il faut faire pencher \\
la balance. Il existe, en effet, des circonstances où il n’est même pas équitable de rendre l’équivalent de ce qu’on a d’abord reçu : quand, par exemple, un homme a fait du bien à un autre homme qu’il sait vertueux, et qu’à son tour ce dernier est appelé à rendre son bienfait au premier, qu’il estime être un malhonnête homme. Car même si une personne vous a prêté de l’argent, vous n’êtes pas toujours tenu de lui en prêter à votre tour : cette personne peut, en effet, vous avoir prêté à vous, qui êtes honnête, pensant qu’elle rentrera dans son argent, alors \\
que vous-même n’avez aucun espoir de vous faire rembourser par un coquin de son espèce. Si donc on se trouve réellement dans cette situation, la prétention de l’autre partie n’est pas équitable ; et même si on n’a pas affaire à un coquin, mais qu’il en ait la réputation, personne ne saurait trouver étrange que vous agissiez de la sorte.\par
La conclusion est celle que nous avons indiquée à plusieurs reprises : nos raisonnements concernant les passions et les actions humaines ne sont pas autrement définis que les objets dont ils traitent.\par
Que nous ne soyons pas tenus d’acquitter à tous indistinctement les mêmes rémunérations en retour de leurs services, ni \\
de déférer en toutes choses aux désirs d’un père, pas plus qu’on n’offre à Zeus tous les sacrifices, c’est ce qui ne fait pas de doute. Mais puisque ce sont des satisfactions différentes que réclament parents, frères, camarades ou bienfaiteurs, il faut attribuer à chacun de ces groupes les avantages qui lui sont appropriés et qui sont à sa mesure. C’est d’ailleurs ainsi qu’en fait on procède : aux noces, par exemple, on invite les personnes de sa parenté (car elles font partie de la famille, et par suite \\
participent aux actes qui la concernent) ; pour les funérailles aussi on estime qu’avant tout le monde les gens de la famille doivent s’y présenter et cela pour la même raison. On pensera encore que l’assistance due à nos parents pour assurer leur subsistance passe avant tout autre devoir, puisque c’est une dette que nous acquittons, et que l’aide que nous apportons à cet égard aux auteurs de nos jours est quelque chose de plus honorable encore que le souci de notre propre conservation. L’honneur aussi est dû à nos parents, comme il l’est aux dieux, mais ce n’est pas n’importe quel honneur dans tous les cas : \\
l’honneur n’est pas le même pour un père ou pour une mère, ni non plus pour le philosophe ou pour le stratège, mais on doit rendre au père l’honneur dû à un père, et pareillement à la mère l’honneur dû à une mère. À tout vieillard aussi nous devons rendre l’honneur dû à son âge, en nous levant à son approche, en le faisant asseoir, et ainsi de suite. En revanche, à l’égard de camarades ou de frères on usera d’un langage plus libre, et \\
on mettra tout en commun avec eux. Aux membres de notre famille, de notre tribu, de notre cité, ou d’autres groupements, à tous nous devons toujours nous efforcer d’attribuer ce qui leur revient en propre, et de comparer ce que chacune de ces catégories d’individus est en droit de prétendre, eu égard à leur degré de parenté, à leur vertu ou à leur utilité. Entre personnes appartenant à une même classe, la discrimination est relativement aisée, mais entre personnes de groupements différents, elle est plus laborieuse. Ce n’est pourtant pas une raison pour \\
y renoncer, mais, dans la mesure du possible, il convient d’observer toutes ces distinctions.
\subsection[{3 (1165a — 1165b) < De la rupture de l’amitié >}]{3 (1165a — 1165b) < De la rupture de l’amitié >}
\noindent On se pose encore la question de savoir si l’amitié sera  rompue ou non à l’égard des amis ne demeurant pas ce qu’ils étaient. Ne devons-nous pas répondre que dans le cas des amitiés reposant sur l’utilité ou le plaisir, dès que les intéressés ne possèdent plus ces avantages, il n’y a rien d’étonnant à ce qu’elles se rompent ? (Car c’était de ces avantages qu’on était épris ; une fois qu’ils ont disparu. il est normal que l’amitié cesse). Mais on se plaindrait à juste titre de celui qui ne recherchant \\
en réalité dans l’amitié que l’utilité ou le plaisir qu’elle procure, ferait semblant d’y être poussé par des raisons morales. Comme nous l’avons dit au début, des conflits entre amis se produisent le plus souvent lorsqu’ils ne sont pas amis de la façon qu’ils croient l’être. Quand donc on a commis une erreur sur ce point et qu’on a supposé être aimé pour des raisons morales, si l’autre ne fait rien pour accréditer cette \\
supposition on ne saurait s’en prendre qu’à soi-même ; si, au contraire, ce sont ses feintes qui nous ont induits en erreur, il est juste d’adresser des reproches à celui qui nous a dupés, et qui les mérite davantage que s’il avait falsifié la monnaie, d’autant que sa perfidie porte sur un objet plus précieux encore.\par
Mais si on reçoit dans son amitié quelqu’un comme étant un homme de bien et qu’il devienne ensuite un homme pervers et nous apparaisse tel, est-ce que nous devons encore l’aimer ? N’est-ce pas plutôt là une chose impossible, s’il est vrai que \\
rien n’est aimable que ce qui est bon, et que, d’autre part, nous ne pouvons, ni ne devons aimer ce qui est pervers ? Car notre devoir est de ne pas être un amateur de vice, et de ne pas ressembler à ce qui est vil ; et nous avons dit218 que le semblable est ami du semblable. Est-ce donc que nous devions rompre sur-le-champ ? N’est-ce pas là plutôt une solution qui n’est pas applicable à tous les cas, mais seulement quand il s’agit d’amis dont la perversité est incurable ? Nos amis sont-ils, au contraire, susceptibles de s’amender, nous avons alors le devoir de leur venir moralement en aide, bien plus même que s’il s’agissait de les aider pécuniairement, et cela dans la \\
mesure où les choses d’ordre moral l’emportent sur l’argent et se rapprochent davantage de l’amitié. On admettra cependant que celui qui rompt une amitié de ce genre ne fait rien là que de naturel : car ce n’était pas à un homme de cette sorte que s’adressait notre amitié ; si donc son caractère s’est altéré et qu’on soit impuissant à le remettre dans la bonne voie, on n’a plus qu’à se séparer de lui.\par
Si, d’autre part, l’un des deux amis demeurait ce qu’il était et que l’autre eût progressé dans le bien et l’emportât grandement en vertu, celui-ci doit-il garder le premier pour ami ? N’y \\
a-t-il pas plutôt là une impossibilité ? Quand l’intervalle qui sépare les deux amis est considérable, cette impossibilité apparaît au grand jour, comme dans le cas des amitiés entre enfants : si, en effet, l’un restait enfant par l’esprit, tandis que l’autre serait devenu un homme de haute valeur, comment pourraient-ils être amis, n’ayant ni les mêmes goûts, ni les mêmes plaisirs, ni les mêmes peines ? Même dans leurs rapports mutuels, cette communauté de sentiments leur fera défaut ; or c’est là une \\
condition sans laquelle, nous le savons, ils ne peuvent être amis, puisqu’il ne leur est pas possible de vivre l’un avec l’autre. Mais nous avons déjà traité cette question.\par
Devons-nous donc nous comporter envers un ancien ami exactement de la même façon que s’il n’avait jamais été notre ami ? Ne doit-on pas plutôt conserver le souvenir de l’intimité passée, et de même que nous pensons qu’il est de notre devoir de nous montrer plus aimable pour des amis que pour des étrangers, ainsi également à ceux qui ont été nos amis ne \\
devons-nous pas garder encore quelque sentiment d’affection en faveur de notre amitié d’antan, du moment que la rupture n’a pas eu pour cause un excès de perversité de leur part ?
\subsection[{4 (1166a — 1166b) < Analyse de l’amitié. Altruisme et égoïsme >}]{4 (1166a — 1166b) < Analyse de l’amitié. Altruisme et égoïsme >}
\noindent  Les sentiments affectifs que nous ressentons à l’égard de nos amis, et les caractères qui servent à définir les diverses amitiés semblent bien dériver des relations de l’individu avec lui-même. En effet, on définit un ami : celui qui souhaite et fait ce qui est bon en réalité ou lui semble tel, en vue de son ami même ; ou encore, celui qui souhaite que son ami ait l’existence \\
et la vie, pour l’amour de son ami même (c’est précisément ce sentiment que ressentent les mères à l’égard de leurs enfants, ainsi que les amis qui se sont querellés). D’autres définissent un ami : celui qui passe sa vie avec un autre et qui a les mêmes goûts que lui ; ou celui qui partage les joies et les tristesses de son ami (sentiment que l’on rencontre aussi tout particulièrement chez les mères). L’amitié se définit enfin par l’un ou l’autre de ces caractères.\par
\\
Or chacune de ces caractéristiques se rencontre aussi dans la relation de l’homme de bien avec lui-même (comme aussi chez les autres hommes, en tant qu’ils se croient eux-mêmes des hommes de bien ; or, de l’avis général, ainsi que nous l’avons dit, la vertu et l’homme vertueux sont mesure de toutes choses). En effet, les opinions sont chez lui en complet accord entre elles, et il aspire aux mêmes choses avec son âme tout entière. Il se souhaite aussi à lui-même ce qui est bon \\
en réalité et lui semble tel, et il le fait (car c’est le propre de l’homme bon de travailler activement pour le bien), et tout cela en vue de lui-même (car il agit en vue de la partie intellective qui est en lui et qui paraît constituer l’intime réalité de chacun de nous). Il souhaite encore que lui-même vive et soit conservé, et spécialement cette partie par laquelle il pense. L’existence est, en effet, un bien pour l’homme vertueux, et chaque homme souhaite à soi-même ce qui est bon : et nul ne \\
choisirait de posséder le monde entier en devenant d’abord quelqu’un d’autre que ce qu’il est devenu (car Dieu possède déjà tout le bien existant), mais seulement en restant ce qu’il est, quel qu’il soit. Or il apparaîtra que l’intellect constitue l’être même de chaque homme, ou du moins sa partie principale. En outre, l’homme vertueux souhaite de passer sa vie \\
avec lui-même : il est tout aise de le faire, car les souvenirs que lui laissent ses actions passées ont pour lui du charme, et en ce qui concerne les actes à venir, ses espérances sont celles d’un homme de bien et en cette qualité lui sont également agréables. Sa pensée enfin abonde en sujets de contemplation. Et avec cela, il sympathise par-dessus tout avec ses propres joies et ses propres peines, car toujours les mêmes choses sont pour lui pénibles ou agréables, et non telle chose à tel moment et telle autre à tel autre, car on peut dire qu’il ne regrette jamais rien.\par
\\
Dès lors, du fait que chacun de ces caractères appartient à l’homme de bien dans sa relation avec lui-même, et qu’il est avec son ami dans une relation semblable à celle qu’il entretient avec lui-même (car l’ami est un autre soi-même) il en résulte que l’amitié semble consister elle aussi en l’un ou l’autre de ces caractères, et que ceux qui les possèdent sont liés d’amitié. — Quant à la question de savoir s’il peut ou non y avoir amitié entre un homme et lui-même, nous pouvons la laisser de côté pour le moment ; on admettra cependant qu’il peut y \\
avoir amitié en tant que chacun de nous est un être composé de deux parties ou davantage, à en juger d’après les caractères  mentionnés plus haut, et aussi parce que l’excès dans l’amitié ressemble à celle qu’on se porte à soi-même.\par
C’est un fait d’expérience que les caractères que nous avons décrits appartiennent aussi à la plupart des hommes, si pervers qu’ils puissent être. Ne pouvons-nous alors dire que, en tant qu’ils se complaisent en eux-mêmes et se croient des hommes de bien, ils participent réellement à ces caractères ? Car enfin aucun homme d’une perversité ou d’une scélératesse \\
achevée n’est en possession de ces qualités, et il ne donne même pas l’impression de les avoir. On peut même à peu près assurer qu’elles ne se rencontrent pas chez les individus d’une perversité courante : ces gens-là sont en désaccord avec eux-mêmes, leur concupiscence les poussant à telles choses, et leurs désirs rationnels à telles autres : c’est par exemple le cas des intempérants, qui, au lieu de ce qui, à leurs propres yeux, \\
est bon, choisissent ce qui est agréable mais nuisible. D’autres, à leur tour, par lâcheté et par fainéantise, renoncent à faire ce qu’ils estiment eux-mêmes le plus favorable à leurs propres intérêts. Et ceux qui ont commis de nombreux et effrayants forfaits et sont détestés pour leur perversité en arrivent à dire adieu à l’existence et à se détruire eux-mêmes. De même encore, les méchants recherchent la société d’autres personnes avec lesquelles ils passeront leurs journées, mais ils se fuient eux-mêmes, car seuls avec eux-mêmes ils se ressouviennent \\
d’une foule d’actions qui les accablent et prévoient qu’ils en commettront à l’avenir d’autres semblables, tandis qu’au contraire la présence de compagnons leur permet d’oublier. De plus, n’ayant en eux rien d’aimable, ils n’éprouvent aucun sentiment d’affection pour eux-mêmes. Par suite, de tels hommes demeurent étrangers à leurs propres joies et à leurs propres peines, car leur âme est déchirée par les factions : l’une de ses parties, en raison de sa dépravation, souffre quand \\
l’individu s’abstient de certains actes, tandis que l’autre partie s’en réjouit ; l’une tire dans un sens et l’autre dans un autre, mettant ces malheureux pour ainsi dire en pièces. Et s’il n’est pas strictement possible qu’ils ressentent dans un même moment du plaisir et de la peine, du moins leur faut-il peu de temps pour s’affliger d’avoir cédé au plaisir et pour souhaiter que ces jouissances ne leur eussent jamais été agréables : car les hommes vicieux sont chargés de regrets.\par
\\
Ainsi donc, il est manifeste que l’homme pervers n’a même pas envers lui-même de dispositions affectueuses, parce qu’il n’a en lui rien qui soit aimable. Si dès lors un pareil état d’esprit est le comble de la misère morale, nous devons fuir la perversité de toutes nos forces et essayer d’être d’honnêtes gens : ainsi pourrons-nous à la fois nous comporter en ami avec nous-mêmes et devenir un ami pour un autre.
\subsection[{5 (1166b — 1167a) < Analyse de la bienveillance >}]{5 (1166b — 1167a) < Analyse de la bienveillance >}
\noindent \\
La bienveillance est une sorte de sentiment affectif, tout en n’étant pas cependant amitié. La bienveillance, en effet, est ressentie même à l’égard de gens qu’on ne connaît pas, et elle peut demeurer inaperçue, ce qui n’est pas le cas de l’amitié. Nous avons précédemment discuté ce point.\par
Mais la bienveillance n’est pas non plus amour proprement dit. Elle n’enveloppe, en effet, ni distension, ni désir, caractères qui au contraire accompagnent toujours l’amour ; et l’amour ne va pas sans fréquentation habituelle, tandis que la \\
bienveillance prend naissance même d’une façon soudaine, comme celle qu’il nous arrive d’éprouver en faveur de ceux qui  prennent part à une compétition sportive : nous ressentons de la bienveillance pour eux, notre volonté s’associe à la leur, mais nous ne les seconderions en rien : ainsi que nous venons de le dire, notre bienveillance pour eux s’éveille d’une façon soudaine et notre affection est superficielle.\par
La bienveillance semble dès lors un commencement d’amitié, tout comme le plaisir causé par la vue de l’être aimé est le commencement de l’amour : nul, en effet, n’est amoureux sans avoir été auparavant charmé par l’extérieur de la \\
personne aimée, mais celui qui éprouve du plaisir à l’aspect d’un autre n’en est pas pour autant amoureux, mais c’est seulement quand on regrette son absence et qu’on désire passionnément sa présence. Ainsi également, il n’est pas possible d’être amis sans avoir d’abord éprouvé de la bienveillance l’un pour l’autre, tandis que les gens bienveillants ne sont pas pour autant liés d’amitié : car ils se contentent de souhaiter du bien à ceux qui sont l’objet de leur bienveillance, et ne voudraient \\
les seconder en rien ni se donner du tracas à leur sujet. Aussi pourrait-on dire, en étendant le sens du terme amitié, que la bienveillance est une amitié paresseuse, mais avec le temps et une fois parvenue à une certaine intimité elle devient amitié, < amitié véritable >, et non pas cette sorte d’amitié basée sur l’utilité ou le plaisir, car la bienveillance non plus ne prend pas naissance sur ces bases. L’homme qui en effet, a reçu un bienfait, et qui, en échange des faveurs dont il a été gratifié, \\
répond par de la bienveillance, ne fait là que ce qui est juste, et, d’autre part, celui qui souhaite la prospérité d’autrui dans l’espoir d’en tirer amplement profit, paraît bien avoir de la bienveillance, non pas pour cet autre, mais plutôt pour lui-même, pas plus qu’on n’est ami de quelqu’un si les soins dont on l’entoure s’expliquent par quelque motif intéressé. En somme, la bienveillance est suscitée par une certaine excellence et une certaine valeur morale : quand, par exemple, une personne se montre à une autre, noble, ou brave, ou douée de \\
quelque qualité analogue, comme nous l’avons indiqué pour le cas des compétiteurs sportifs.
\subsection[{6 (1167a — 1167b) < Analyse de la concorde >}]{6 (1167a — 1167b) < Analyse de la concorde >}
\noindent La concorde est, elle aussi, l’expérience le montre, un sentiment affectif. Pour cette raison elle n’est pas simple conformité d’opinion, qui pourrait exister même entre personnes inconnues les unes aux autres. Pas davantage, on ne dit des gens qui ont la même manière de voir sur une question quelconque que la concorde règne entre eux : par exemple, \\
ceux qui sont du même avis sur les phénomènes célestes (car la façon de penser commune sur ces matières n’a rien d’affectif). Au contraire, nous disons que la concorde prévaut dans les cités, quand les citoyens sont unanimes sur leurs intérêts, choisissent la même ligne de conduite et exécutent les décisions prises en commun. C’est donc aux fins d’ordre pratique que la concorde se rapporte, mais à des fins pratiques d’importance \\
et susceptibles d’intéresser les deux parties à la fois ou même toutes les parties en cause : c’est le cas pour les cités, quand tous les citoyens décident que les magistratures seront électives, ou qu’une alliance sera conclue avec les Lacédémoniens, ou que Pittacos exercera le pouvoir, à l’époque où lui-même y consentait de son côté. Quand au contraire chacun des deux partis rivaux souhaite pour lui-même la chose débattue, comme les chefs dans {\itshape les Phéniciennes}, c’est le règne des factions : car la concorde ne consiste pas pour chacun des deux compétiteurs à penser la même chose, quelle que soit au surplus la \\
chose, mais à penser la même chose réalisée dans les mêmes  mains, quand, par exemple, le peuple et les classes dirigeantes sont d’accord pour remettre le pouvoir au parti aristocratique, car c’est seulement ainsi que tous les intéressés voient se réaliser ce qu’ils avaient en vue. Il apparaît dès lors manifeste que la concorde est une amitié politique, conformément d’ailleurs au sens ordinaire du terme : car elle roule sur les intérêts et les choses se rapportant à la vie.\par
\\
La concorde prise en ce sens n’existe qu’entre les gens de bien, puisqu’ils sont en accord à la fois avec eux-mêmes et les uns à l’égard des autres, se tenant pour ainsi dire sur le même terrain. Chez les gens de cette sorte, en effet, les volontés demeurent stables et ne sont pas le jouet du reflux comme les eaux d’un détroit ; et ils souhaitent à la fois ce qui est juste et ce qui est avantageux, toutes choses pour lesquelles leurs aspirations aussi sont communes. Les hommes pervers, au contraire, sont impuissants à faire régner entre eux la concorde, \\
sinon dans une faible mesure, tout comme ils sont incapables d’amitié, du fait qu’ils visent à obtenir plus que leur part dans les profits, et moins que leur part dans les travaux et dans les charges publiques. Et comme chacun souhaite ces avantages pour lui personnellement, il surveille jalousement son voisin et l’empêche d’en bénéficier : faute d’y veiller, l’intérêt général court à sa ruine. Le résultat est que des dissensions éclatent \\
entre les citoyens, chacun contraignant l’autre à faire ce qui est juste, mais ne voulant pas s’y plier lui-même.
\subsection[{7 (1167b — 1168a) < Analyse de la bienfaisance >}]{7 (1167b — 1168a) < Analyse de la bienfaisance >}
\noindent Les bienfaiteurs aiment ceux auxquels ils ont fait du bien, semble-t-il, plus que ceux auxquels on a fait du bien n’aiment ceux qui leur en ont fait ; et comme c’est là une constatation contraire à toute raison, on en recherche l’explication.\par
\\
Aux yeux de la plupart, la cause est que les obligés sont dans la position de débiteurs, et les bienfaiteurs dans celle de créanciers : il en est donc comme dans le cas du prêt d’argent, où l’emprunteur verrait d’un bon œil son prêteur disparaître, tandis que le prêteur veille au contraire avec soin à la conservation de son débiteur : ainsi également, pense-t-on, le bienfaiteur souhaite que son obligé demeure bien vivant afin d’en \\
recueillir de la reconnaissance, alors que l’obligé se soucie peu de s’acquitter de sa dette. Épicharme dirait peut-être de ceux qui donnent cette explication qu’{\itshape ils voient les choses par leur mauvais côté} ; elle paraît bien cependant conforme à l’humaine nature, tant la plupart des hommes ont la mémoire courte, et aspirent plutôt à recevoir qu’à donner.\par
Mais on peut penser que la cause tient davantage à la nature même des choses, et qu’il n’y a aucune ressemblance avec ce \\
qui se passe dans le cas du prêt. Le prêteur n’a, en effet, en lui aucune affection pour son emprunteur, il désire seulement sa conservation afin de recouvrer ce qu’il lui a prêté ; au contraire, le bienfaiteur ressent de l’amitié et de l’attachement pour la personne de son obligé, même si ce dernier ne lui est d’aucune utilité et ne peut lui rendre dans l’avenir aucun service.\par
En fait, le cas est exactement le même chez les artistes : ils \\
ont tous plus d’amour pour l’œuvre de leurs mains qu’ils n’en recevraient de celle-ci si elle devenait animée. Peut-être ce  sentiment se rencontre-t-il surtout chez les poètes, qui ont une affection excessive pour leurs propres productions et les chérissent comme leurs enfants. La position du bienfaiteur ressemble ainsi à celle de l’artiste : l’être qui a reçu du bien de lui est son ouvrage, et par suite il l’aime plus que l’ouvrage \\
n’aime celui qui l’a fait. La raison en est que l’existence est pour tout être objet de préférence et d’amour, et que nous existons par notre acte (puisque nous existons par le fait de vivre et d’agir), et que l’œuvre est en un sens son producteur en acte ; et dès lors, le producteur chérit son œuvre parce qu’il chérit aussi l’existence. Et c’est là un fait qui prend son origine dans la nature même des choses, car ce que l’agent est en puissance, son œuvre l’exprime en acte.\par
\\
En même temps aussi, pour le bienfaiteur il y a quelque chose de noble dans son action, de sorte qu’il se réjouit dans ce en quoi son action réside ; par contre, pour le patient il n’y a rien de noble dans l’agent, mais tout au plus quelque chose de profitable, et cela est moins agréable et moins digne d’amour que ce qui est noble.\par
Trois choses donnent du plaisir : l’activité du présent, l’espoir du futur et le souvenir du passé, mais le plus agréable des trois est ce qui est attaché à l’activité, et c’est pareillement \\
ce qui est aimable. Or pour l’agent qui a concédé le bienfait, son œuvre demeure (car ce qui est noble a une longue durée), alors que pour celui qui l’a reçu, l’utilité passe vite. Et le souvenir des choses nobles est agréable, tandis que celui des choses utiles ne l’est pas du tout ou l’est moins. Quant à l’attente, c’est au contraire l’inverse qui semble avoir lieu. \\
\par
En outre, aimer est semblable à un processus de production, et être aimé à une passivité ; et par suite ce sont ceux qui ont la supériorité dans l’action que l’amour et les sentiments affectifs accompagnent naturellement.\par
De plus, tout homme chérit davantage les choses qu’il a obtenues à force de travail : ainsi ceux qui ont acquis leur argent y tiennent plus que ceux qui l’ont reçu par héritage. Or recevoir un bienfait semble n’impliquer aucun travail pénible, tandis que faire du bien à autrui demande un effort. — C’est également pour ces raisons que les mères ont pour leurs enfants \\
un amour plus grand que celui du père, car elles ont peiné davantage pour les mettre au monde et savent mieux que lui que l’enfant est leur propre enfant. Ce dernier point paraît bien être aussi un caractère propre aux bienfaiteurs.
\subsection[{8 (1168a — 1169b) < L’égoïsme, son rôle et ses formes >}]{8 (1168a — 1169b) < L’égoïsme, son rôle et ses formes >}
\noindent On se pose aussi la question de savoir si on doit faire passer avant tout l’amour de soi-même ou l’amour de quelqu’un d’autre. On critique, en effet, ceux qui s’aiment eux-mêmes \\
par-dessus tout, et on leur donne le nom d’égoïstes, en un sens péjoratif. Et on pense à la fois que l’homme pervers a pour caractère de faire tout ce qu’il fait en vue de son propre intérêt, et qu’il est d’autant plus enfoncé dans sa perversité qu’il agit davantage en égoïste (ainsi, on l’accuse de ne rien faire de lui-même), et qu’au contraire l’homme de bien a pour caractère de faire une chose parce qu’elle est noble, et que sa valeur morale est d’autant plus grande qu’il agit davantage pour de nobles motifs et dans l’intérêt même de son ami, laissant de côté tout avantage personnel.\par
\\
Mais à ces arguments les faits opposent un démenti, et ce  n’est pas sans raison. On admet, en effet, qu’on doit aimer le mieux son meilleur ami, le meilleur ami étant celui qui, quand il souhaite du bien à une personne, le souhaite pour l’amour de cette personne, même si nul ne doit jamais le savoir. Or ces caractères se rencontrent à leur plus haut degré, dans la relation \\
du sujet avec lui-même, ainsi que tous les autres attributs par lesquels on définit un ami : nous l’avons dit en effet, c’est en partant de cette relation de soi-même à soi-même que tous les sentiments qui constituent l’amitié se sont par la suite étendus aux autres hommes. Ajoutons que les proverbes confirment tous cette manière de voir : par exemple, {\itshape une seule âme, ce que possèdent des amis est commun, amitié est égalité, le genou est plus près que la jambe}, — toutes réflexions qui ne sauraient s’appliquer avec plus d’à-propos à la relation de l’homme avec lui-même, car un homme est à lui-même son \\
meilleur ami, et par suite il doit s’aimer lui-même par-dessus tout. Et il est raisonnable de se demander laquelle des deux opinions nous devons suivre, attendu que l’une comme l’autre ont quelque chose de plausible.\par
Peut-être, en présence d’opinions ainsi en conflit, devons-nous les distinguer nettement l’une de l’autre, et déterminer dans quelle mesure et sous quel aspect chacune des deux thèses est vraie. Si dès lors nous parvenions à saisir quel sens chacune d’elles attache au terme égoïste, nous pourrions probablement y voir clair.\par
\\
Ceux qui en font un terme de réprobation appellent {\itshape égoïstes} ceux qui s’attribuent à eux-mêmes une part trop large dans les richesses, les honneurs ou les plaisirs du corps, tous avantages que la plupart des hommes désirent et au sujet desquels ils déploient tout leur zèle, dans l’idée que ce sont là les plus grands biens et par là-même les plus disputés. Ainsi, ceux qui prennent une part excessive de ces divers avantages \\
s’abandonnent à leurs appétits sensuels, et en général à leurs passions et à la partie irrationnelle de leur âme. Tel est d’ailleurs l’état d’esprit de la majorité des hommes, et c’est la raison pour laquelle l’épithète {\itshape égoïste} a été prise au sens où elle l’est : elle tire sa signification du type le plus répandu, et qui n’a rien que de vil. C’est donc à juste titre qu’on réprouve les hommes qui sont égoïstes de cette façon. Que, d’autre part, ce soit seulement ceux qui s’attribuent à eux-mêmes les biens de ce genre qui sont habituellement et généralement désignés du nom \\
d’égoïstes, c’est là un fait qui n’est pas douteux : car si un homme mettait toujours son zèle à n’accomplir lui-même et avant toutes choses que les actions conformes à la justice, à la tempérance, ou à n’importe quelle autre vertu, et, en général, s’appliquait toujours à revendiquer pour lui-même ce qui est honnête, nul assurément ne qualifierait cet homme d’égoïste, ni ne songerait à le blâmer. Et pourtant un tel homme peut sembler, plus que le précédent, être un égoïste : du moins s’attribue-t-il à lui-même les avantages qui sont les plus nobles et le plus véritablement des biens ; et il met ses complaisances \\
dans la partie de lui-même qui a l’autorité suprême et à laquelle tout le reste obéit. Et de même que dans une cité la partie qui a le plus d’autorité est considérée comme étant, au sens le plus plein, la cité elle-même (et on doit en dire autant de n’importe quelle autre organisation), ainsi en est-il pour un homme ; et par suite est égoïste par excellence celui qui aime cette partie supérieure et s’y complaît. En outre, un homme est dit tempérant ou intempérant suivant que son intellect possède ou non \\
la domination, ce qui implique que chacun de nous est son propre intellect. Et les actions qui nous semblent le plus proprement  nôtres, nos actions vraiment volontaires, sont celles qui s’accompagnent de raison. Qu’ainsi donc chaque homme soit cette partie dominante même, ou qu’il soit tout au moins principalement cette partie, c’est là une chose qui ne souffre aucune obscurité, comme il est évident aussi que l’homme de bien aime plus que tout cette partie qui est en lui. D’où il suit que l’homme de bien sera suprêmement égoïste, quoique d’un autre type que celui auquel nous réservons notre réprobation, \\
et dont il diffère dans toute la mesure où vivre conformément à un principe diffère de vivre sous l’empire de la passion, ou encore dans toute la mesure où désirer le bien est autre que désirer ce qui semble seulement avantageux. Ceux donc qui s’appliquent avec une ardeur exceptionnelle à mener une conduite conforme au bien sont l’objet d’une approbation et d’une louange unanimes ; et si tous les hommes rivalisaient en noblesse morale et tendaient leurs efforts pour accomplir les actions les plus parfaites, en même temps que la communauté \\
trouverait tous ses besoins satisfaits, dans sa vie privée chacun s’assurerait les plus grands des biens, puisque la vertu est précisément un bien de ce genre.\par
Nous concluons que l’homme vertueux a le devoir de s’aimer lui-même (car il trouvera lui-même profit en pratiquant le bien, et en fera en même temps bénéficier les autres), alors que l’homme vicieux ne le doit pas (car il causera du tort à la fois à lui-même et à ses proches, en suivant comme il fait \\
ses mauvaises passions). Chez l’homme vicieux, donc, il y a désaccord entre ce qu’il doit faire et ce qu’il fait, alors que l’homme de bien, ce qu’il doit faire il le fait aussi, puisque toujours l’intellect choisit ce qu’il y a de plus excellent pour lui-même, et que l’homme de bien obéit au commandement de son intellect.\par
Mais il est vrai également de l’homme vertueux qu’il agit souvent dans l’intérêt de ses amis et de son pays, et même, s’il \\
en est besoin, donne sa vie pour eux : car il sacrifiera argent, honneurs et généralement tous les biens que les hommes se disputent, conservant pour lui la beauté morale de l’action : il ne saurait, en effet, que préférer un bref moment d’intense joie à une longue période de satisfaction tranquille, une année de vie exaltante à de nombreuses années d’existence terre à terre, une seule action, mais grande et belle, à une multitude d’actions \\
mesquines. Ceux qui font le sacrifice de leur vie atteignent probablement ce résultat ; et par là ils choisissent pour leur part un bien de grand prix. Ils prodigueront aussi leur argent si leurs amis doivent en retirer un accroissement de profit : aux amis l’argent, mais à eux la noblesse morale, et ils s’attribuent ainsi à eux-mêmes la meilleure part. Et en ce qui concerne honneurs \\
et charges publiques, l’homme de bien agira de la même façon : tous ces avantages il les abandonnera à son ami, car pareil abandon est pour lui-même quelque chose de noble et qui attire la louange. C’est dès lors à bon droit qu’on le considère comme un homme vertueux, puisque à toutes choses il préfère le bien. Il peut même arriver qu’il laisse à son ami l’occasion d’agir en son lieu et place ; il peut être plus beau pour lui de devenir la cause de l’action accomplie par son ami que de l’accomplir lui-même.\par
\\
Par suite, dans toute la sphère d’une activité digne d’éloges, l’homme vertueux, on le voit, s’attribue à lui-même  la plus forte part de noblesse morale. En ce sens, donc, on a le devoir de s’aimer soi-même, ainsi que nous l’avons dit : mais au sens où la plupart des hommes sont égoïstes, nous ne devons pas l’être.
\subsection[{9 (1169b — 1170b) < Si l’homme heureux a besoin d’amis >}]{9 (1169b — 1170b) < Si l’homme heureux a besoin d’amis >}
\noindent On discute également, au sujet de l’homme heureux, s’il aura ou non besoin d’amis.\par
On prétend que ceux qui sont parfaitement heureux et se suffisent à eux-mêmes n’ont aucun besoin d’amis : ils sont déjà en possession des biens de la vie, et par suite se suffisant à eux-mêmes \\
n’ont besoin de rien de plus ; or l’ami, qui est un autre soi-même, a pour rôle de fournir ce qu’on est incapable de se procurer par soi-même. D’où l’adage :\par
{\itshape Quand la fortune est favorable, à quoi bon des amis} ? \par
Pourtant il semble étrange qu’en attribuant tous les biens à l’homme heureux on ne lui assigne pas des amis, dont la \\
possession est considérée d’ordinaire comme le plus grand des biens extérieurs. De plus, si le propre d’un ami est plutôt de faire du bien que d’en recevoir, et le propre de l’homme de bien et de la vertu de répandre des bienfaits, et si enfin il vaut mieux faire du bien à des amis qu’à des étrangers, l’homme vertueux aura besoin d’amis qui recevront de lui des témoignages de sa bienfaisance. Et c’est pour cette raison qu’on se pose encore la question de savoir si le besoin d’amis se fait sentir davantage dans la prospérité ou dans l’adversité, attendu \\
que si le malheureux a besoin de gens qui lui rendront des services, les hommes dont le sort est heureux ont besoin eux-mêmes de gens auxquels s’adresseront leurs bienfaits. — Et sans doute est-il étrange aussi de faire de l’homme parfaitement heureux un solitaire : personne, en effet, ne choisirait de posséder tous les biens de ce monde pour en jouir seul, car l’homme est un être politique et naturellement fait pour vivre en société. Par suite, même à l’homme heureux cette caractéristique \\
appartient, puisqu’il est en possession des avantages qui sont bons par nature. Et il est évidemment préférable de passer son temps avec des amis et des hommes de bien qu’avec des étrangers ou des compagnons de hasard. Il faut donc à l’homme heureux des amis.\par
Que veulent donc dire les partisans de la première opinion et sous quel angle sont-ils dans la vérité ? Ne serait-ce pas que la plupart des hommes considèrent comme des amis les gens qui sont seulement utiles ? Certes l’homme parfaitement heureux n’aura nullement besoin d’amis de cette dernière sorte, \\
puisqu’il possède déjà tous les biens ; par suite, il n’aura pas besoin non plus, ou très peu, des amis qu’on recherche pour le plaisir (sa vie étant en soi agréable, il n’a besoin en rien d’un plaisir apporté du dehors) : et comme il n’a besoin d’aucune de ces deux sortes d’amis, on pense d’ordinaire qu’il n’a pas besoin d’amis du tout.\par
Mais c’est là une vue qui n’est sans doute pas exacte. Au début, en effet, nous avons dit que le bonheur est une certaine activité ; et l’activité est évidemment un devenir et non une chose qui existe une fois pour toutes comme quelque chose \\
qu’on a en sa possession. Or, si le bonheur consiste dans la vie et dans l’activité, et si l’activité de l’homme de bien est vertueuse et agréable en elle-même, ainsi que nous l’avons dit en commençant ; si, d’autre part, le fait qu’une chose est proprement nôtre est au nombre des attributs qui nous la rendent agréable ; si enfin nous pouvons contempler ceux qui nous \\
entourent mieux que nous-mêmes, et leurs actions mieux que les nôtres, et si les actions des hommes vertueux qui sont leurs  amis, sont agréables aux gens de bien (puisque ces actions possèdent ces deux attributs qui sont agréables par leur nature), dans ces conditions l’homme parfaitement heureux aura besoin d’amis de ce genre, puisque ses préférences vont à contempler des actions vertueuses et qui lui sont propres, deux qualités que revêtent précisément les actions de l’homme de bien qui est son ami.\par
En outre, on pense que l’homme heureux doit mener une \\
vie agréable. Or pour un homme solitaire la vie est lourde à porter, car il n’est pas facile, laissé à soi-même, d’exercer continuellement une activité, tandis que, en compagnie d’autrui et en rapports avec d’autres, c’est une chose plus aisée. Ainsi donc l’activité de l’homme heureux sera plus continue < exercée avec d’autres >, activité qui est au surplus agréable par soi, et ce sont là les caractères qu’elle doit revêtir chez l’homme parfaitement heureux. (Car l’homme vertueux, en tant que vertueux, se réjouit des actions conformes à la vertu et s’afflige \\
de celles dont le vice est la source, pareil en cela au musicien qui ressent du plaisir aux airs agréables, et qui souffre à écouter de la mauvaise musique). — Ajoutons qu’un certain entraînement à la vertu peut résulter de la vie en commun avec les honnêtes gens, suivant la remarque de Théognis.\par
En outre, à examiner de plus près la nature même des choses, il apparaît que l’ami vertueux est naturellement \\
désirable pour l’homme vertueux. Car ce qui est bon par nature, nous l’avons dit, est pour l’homme vertueux bon et agréable en soi. Or la vie se définit, dans le cas des animaux par une capacité de sensation, et chez l’homme par une capacité de sensation ou de pensée ; mais la capacité se conçoit par référence à l’acte, et l’élément principal réside dans l’acte. Il apparaît par suite que la vie humaine consiste principalement dans l’acte de sentir ou de penser. Mais la vie fait partie des \\
choses bonnes et agréables en elles-mêmes, puisqu’elle est quelque chose de déterminé, et que le déterminé relève de la nature du bien ; et ce qui est bon par nature l’est aussi pour l’homme de bien (et c’est pourquoi la vie apparaît agréable à tous les hommes). Mais nous ne devons pas entendre par là une vie dépravée et corrompue, ni une vie qui s’écoule dans la peine, car une telle vie est indéterminée, comme le sont ses \\
attributs. — Dans la suite de ce travail, cette question de la peine deviendra plus claire. — Mais si la vie elle-même est une chose bonne et agréable (comme elle semble bien l’être, à en juger par l’attrait qu’elle inspire à tout homme et particulièrement aux hommes vertueux et parfaitement heureux, car à ceux-ci la vie est désirable au suprême degré, et leur existence est la plus parfaitement heureuse), et si celui qui voit a \\
conscience qu’il voit, celui qui entend, conscience qu’il entend, celui qui marche, qu’il marche, et si pareillement pour les autres formes d’activité il y a quelque chose qui a conscience que nous sommes actifs, de sorte que nous aurions conscience que nous percevons, et que nous penserions que nous pensons, et si avoir conscience que nous percevons ou pensons est avoir conscience que nous existons (puisque  exister, avons-nous dit, est percevoir ou penser), et si avoir conscience qu’on vit est au nombre des plaisirs agréables par soi (car la vie est quelque chose de bon par nature, et avoir conscience qu’on possède en soi-même ce qui est bon est une chose agréable) ; et si la vie est désirable, et désirable surtout pour les bons, parce que l’existence est une chose bonne pour \\
eux et une chose agréable (car la conscience qu’ils ont de posséder en eux ce qui est bon par soi est pour eux un sujet de joie) ; et si l’homme vertueux est envers son ami comme il est envers lui-même (son ami étant un autre lui-même), — dans ces conditions, de même que pour chacun de nous sa propre existence est une chose désirable, de même est désirable pour lui au même degré, ou à peu de chose près, l’existence de son ami. Mais nous avons dit que ce qui rend son existence désirable c’est la conscience qu’il a de sa propre bonté, et une telle \\
conscience est agréable par elle-même. Il a besoin, par conséquent, de participer aussi à la conscience qu’a son ami de sa propre existence, ce qui ne saurait se réaliser qu’en vivant avec lui et en mettant en commun discussions et pensées : car c’est en ce sens-là, semblera-t-il, qu’on doit parler de vie en société quand il s’agit des hommes, et il n’en est pas pour eux comme pour les bestiaux où elle consiste seulement à paître dans le même lieu.\par
Si donc pour l’homme parfaitement heureux l’existence \\
est une chose désirable en soi, puisqu’elle est par nature bonne et agréable, et si l’existence de son ami est aussi presque autant désirable pour lui, il s’ensuit que l’ami sera au nombre des choses désirables. Mais ce qui est désirable pour lui, il faut bien qu’il l’ait en sa possession, sinon sur ce point particulier il souffrira d’un manque. Nous concluons que l’homme heureux aura besoin d’amis vertueux.
\subsection[{10 (1170b — 1171a) < Sur le nombre des amis >}]{10 (1170b — 1171a) < Sur le nombre des amis >}
\noindent \\
Est-ce que nous devons nous faire le plus grand nombre d’amis possible, ou bien (de même que, dans le cas de l’hospitalité, on estime qu’il est judicieux de dire :\par
{\itshape Ni un homme de beaucoup d’hôtes, ni un homme sans hôtes}, \par
appliquerons-nous à l’amitié la formule : n’être ni sans amis, ni non plus avec des amis en nombre excessif ? S’agit-il d’amis qu’on recherche pour leur utilité, ce propos paraîtra certainement applicable (car s’acquitter de services rendus envers un grand nombre de gens est une \\
lourde charge, et la vie n’est pas suffisante pour l’accomplir. Par suite, les amis dont le nombre excède les besoins normaux de notre propre existence sont superflus et constituent un obstacle à la vie heureuse ; on n’a donc nullement besoin d’eux). Quant aux amis qu’on recherche pour le plaisir, un petit nombre doit suffire, comme dans la nourriture il faut peu d’assaisonnement.\par
Mais en ce qui regarde les amis vertueux, doit-on en avoir le plus grand nombre possible, ou bien existe-t-il aussi une \\
limite au nombre des amis, comme il y en a une pour la population d’une cité ? Si dix hommes, en effet, ne sauraient constituer une cité, cent mille hommes ne sauraient non plus en former encore une. Mais la quantité à observer n’est sans doute pas un nombre nettement déterminé, mais un nombre quelconque compris entre certaines limites. Ainsi, le nombre des  amis est-il également déterminé, et sans doute doit-il tout au plus atteindre le nombre de personnes avec lesquelles une vie en commun soit encore possible (car, nous l’avons dit, la vie en commun est d’ordinaire regardée comme ce qui caractérise le mieux l’amitié) : or qu’il ne soit pas possible de mener une vie commune avec un grand nombre de personnes et de se partager soi-même entre toutes, c’est là une chose qui n’est pas douteuse. De plus, il faut encore que nos amis soit amis les uns \\
des autres, s’ils doivent tous passer leurs jours en compagnie les uns des autres : or c’est là une condition laborieuse à remplir pour des amis nombreux. On arrive difficilement aussi à compatir intimement aux joies et aux douleurs d’un grand nombre, car on sera vraisemblablement amené dans un même moment à se réjouir avec l’un et à s’affliger avec un autre.\par
Peut-être, par conséquent, est-il bon de ne pas chercher à avoir le plus grand nombre d’amis possible, mais seulement \\
une quantité suffisante pour la vie en commun ; car il apparaîtra qu’il n’est pas possible d’entretenir une amitié solide avec beaucoup de gens. Telle est précisément la raison pour laquelle l’amour sensuel ne peut pas non plus avoir plusieurs personnes pour objet : l’amour, en effet, n’est pas loin d’être une sorte d’exagération d’amitié, sentiment qui ne s’adresse qu’à un seul : par suite, l’amitié solide ne s’adresse aussi qu’à un petit nombre.\par
Ce que nous disons semble également confirmé par les faits. Ainsi, l’amitié entre camarades ne rassemble qu’un petit nombre d’amis, et les amitiés célébrées par les poètes ne se \\
produisent qu’entre deux amis. Ceux qui ont beaucoup d’amis et se lient intimement avec tout le monde passent pour n’être réellement amis de personne (excepté quand il s’agit du lien qui unit entre eux des concitoyens), et on leur donne aussi l’épithète de complaisants. Pour l’amitié entre concitoyens, il est assurément possible d’être lié avec un grand nombre d’entre eux sans être pour autant complaisant et en restant un véritable homme de bien. Toujours est-il qu’on ne peut pas avoir pour une multitude de gens cette sorte d’amitié basée sur la vertu et sur la considération de la personne elle-même, et il \\
faut même se montrer satisfait quand on a découvert un petit nombre d’amis de ce genre.
\subsection[{11 (1171a — 1171b) < Le besoin d’amis dans la prospérité et dans l’adversité >}]{11 (1171a — 1171b) < Le besoin d’amis dans la prospérité et dans l’adversité >}
\noindent Est-ce dans la prospérité que nous avons davantage besoin d’amis, ou dans l’adversité ? Dans un cas comme dans l’autre, en effet, on est à leur recherche : d’une part, les hommes défavorisés par le sort ont besoin d’assistance, et, d’autre part, ceux à qui la fortune sourit ont besoin de compagnons et de gens auxquels ils feront du bien, puisqu’ils souhaitent pratiquer la bienfaisance. L’amitié, par suite, est une chose plus nécessaire \\
dans la mauvaise fortune, et c’est pourquoi on a besoin d’amis utiles dans cette circonstance, mais l’amitié est une chose plus belle dans la prospérité, et c’est pourquoi alors on recherche aussi les gens de bien, puisqu’il est préférable de pratiquer la bienfaisance envers eux et de vivre en leur compagnie. En effet, la présence même des amis est agréable à la fois dans la bonne et la mauvaise fortune. Car les personnes affligées éprouvent du soulagement quand leurs amis compatissent \\
à leurs souffrances. Et de là vient qu’on peut se demander si ces amis ne reçoivent pas en quelque sorte une part de notre fardeau, ou si, sans qu’il y ait rien de tel, leur seule présence, par le plaisir qu’elle nous cause, et la pensée qu’ils compatissent à nos souffrances, n’ont pas pour effet de rendre notre peine moins vive. Que ce soit pour ces raisons ou pour quelque autre qu’on éprouve du soulagement, laissons cela : de toute façon, l’expérience montre que ce que nous venons de dire a réellement lieu.\par
\\
Mais la présence d’amis semble bien procurer un plaisir qui n’est pas sans mélange. La simple vue de nos amis est, il  est vrai, une chose agréable, surtout quand on se trouve dans l’infortune, et devient une sorte de secours contre l’affliction (car un ami est propre à nous consoler à la fois par sa vue et ses paroles, si c’est un homme de tact, car il connaît notre caractère et les choses qui nous causent du plaisir ou de la peine). Mais, d’un autre côté, s’apercevoir que l’ami ressent lui-même de \\
l’affliction de notre propre infortune est quelque chose de pénible, car tout le monde évite d’être une cause de peine pour ses amis. C’est pourquoi les natures viriles se gardent bien d’associer leurs amis à leurs propres peines, et, à moins d’être d’une insensibilité portée à l’excès, un homme de cette trempe ne supporte pas la peine que sa propre peine fait naître chez ses amis, et en général il n’admet pas que d’autres se lamentent \\
avec lui, pour la raison qu’il n’est pas lui-même enclin aux lamentations. Des femmelettes, au contraire, et les hommes qui leur ressemblent, se plaisent avec ceux qui s’associent à leurs gémissements, et les aiment comme des amis et des compagnons de souffrance. Mais en tout cela nous devons évidemment prendre pour modèle l’homme de nature plus virile.\par
D’un autre côté, la présence des amis dans la prospérité non seulement est une agréable façon de passer le temps, mais encore nous donne la pensée qu’ils se réjouissent de ce qui nous arrive personnellement de bon.\par
\\
C’est pourquoi il peut sembler que notre devoir est de convier nos amis à partager notre heureux sort (puisqu’il est noble de vouloir faire du bien), et dans la mauvaise fortune, au contraire, d’hésiter à faire appel à eux (puisqu’on doit associer les autres le moins possible à nos maux, d’où l’expression : {\itshape C’est assez de ma propre infortune}). Mais là où il nous faut principalement appeler à l’aide nos amis, c’est lorsque, au prix d’un léger désagrément pour eux-mêmes, ils sont en situation \\
de nous rendre de grands services. — Inversement, il convient sans doute que nous allions au secours de nos amis malheureux sans attendre d’y être appelés, et de tout cœur (car c’est le propre d’un ami de faire du bien, et surtout à ceux qui sont dans le besoin et sans qu’ils l’aient demandé : pour les deux parties l’assistance ainsi rendue est plus conforme au bien et plus agréable) ; mais quand ils sont dans la prospérité, tout en leur apportant notre coopération avec empressement (car même pour cela ils ont besoin d’amis), nous ne mettrons aucune hâte \\
à recevoir leurs bons offices (car il est peu honorable de montrer trop d’ardeur à se faire assister). Mais sans doute faut-il éviter une apparence même de grossièreté en repoussant leurs avances, chose qui arrive parfois.\par
La présence d’amis apparaît donc désirable en toutes circonstances.
\subsection[{12 (1171b — 1172a) < La vie commune dans l’amitié >}]{12 (1171b — 1172a) < La vie commune dans l’amitié >}
\noindent Ne doit-on pas le dire ? De même que pour les amoureux la \\
vue de l’aimé est ce qui les réjouit par-dessus tout, et qu’ils préfèrent le sens de la vue à tous les autres, dans la pensée que c’est de lui que dépendent principalement l’existence et la naissance de leur amour, pareillement aussi pour les amis la vie en commun n’est-elle pas ce qu’il y a de plus désirable ?\par
L’amitié, en effet, est une communauté. Et ce qu’un homme est à soi-même, ainsi l’est-il pour son ami ; or en ce qui le concerne personnellement, la conscience de son existence est désirable, et dès lors l’est aussi la conscience de \\
l’existence de son ami ; mais cette conscience s’actualise dans  la vie en commun, de sorte que c’est avec raison que les amis aspirent à cette vie commune. En outre, tout ce que l’existence peut représenter pour une classe déterminée d’individus, tout ce qui rend la vie désirable pour eux, c’est à cela qu’ils souhaitent passer leur vie avec leurs amis. De là vient que les uns se réunissent pour boire, d’autres pour jouer aux dés, d’autres encore pour s’exercer à la gymnastique, chasser, \\
étudier la philosophie, tous, dans chaque groupement, se livrant ensemble à longueur de journée au genre d’activité qui leur plaît au-dessus de toutes les autres occupations de la vie : souhaitant, en effet, vivre avec leurs amis, ils s’adonnent et participent de concert à ces activités, qui leur procurent le sentiment d’une vie en commun.\par
Quoi qu’il en soit, l’amitié qui unit les gens pervers est mauvaise (car en raison de leur instabilité ils se livrent en commun à des activités coupables, et en outre deviennent \\
méchants en se rendant semblables les uns aux autres), tandis que l’amitié entre les gens de bien est bonne et s’accroît par leur liaison même. Et ils semblent aussi devenir meilleurs en agissant et en se corrigeant mutuellement, car ils s’impriment réciproquement les qualités où ils se complaisent, d’où le proverbe :\par
 {\itshape Des gens de bien viennent les bonnes leçons.} 
\section[{Livre X}]{Livre X}\renewcommand{\leftmark}{Livre X}

\subsection[{1 (1172a — 1172b) < Introduction à la théorie du plaisir : les thèses en présence >}]{1 (1172a — 1172b) < Introduction à la théorie du plaisir : les thèses en présence >}
\noindent En ce qui concerne l’amitié, restons-en là. Nous pourrons \\
ensuite traiter du plaisir. Après les considérations qui précèdent suit sans doute naturellement une discussion sur le plaisir. On admet, en effet, d’ordinaire que le plaisir est ce qui touche \\
le plus près à notre humaine nature ; et c’est pourquoi dans l’éducation des jeunes gens, c’est par le plaisir et la peine qu’on les gouverne. On est également d’avis que pour former l’excellence du caractère, le facteur le plus important est de se plaire aux choses qu’il faut et de détester celles qui doivent l’être. En effet, plaisir et peine s’étendent tout au long de la \\
vie, et sont d’un grand poids et d’une grande force pour la vertu comme pour la vie heureuse, puisqu’on élit ce qui est agréable et qu’on évite ce qui est pénible. Et les facteurs de cette importance ne doivent d’aucune façon, semblera-t-il, être passés sous silence, étant donné surtout le grand débat qui s’élève à leur sujet. Les uns, en effet, prétendent que le plaisir est le bien ; d’autres, au contraire, qu’il est entièrement mauvais ; parmi ces derniers, certains sont sans doute persuadés qu’il en est réellement ainsi, tandis que d’autres pensent qu’il est préférable \\
dans l’intérêt de notre vie morale de placer ouvertement le plaisir au nombre des choses mauvaises, même s’il n’en est rien : car la plupart des hommes ayant pour lui une forte inclination et étant esclaves de leurs plaisirs, il convient, disent-ils, de les mener dans la direction contraire, car ils atteindront ainsi le juste milieu.\par
Mais il est à craindre que cette manière de voir ne soit pas exacte. En effet, quand il s’agit des sentiments et des actions, \\
les arguments sont d’une crédibilité moindre que les faits, et ainsi lorsqu’ils sont en désaccord avec les données de la perception ils sont rejetés avec mépris et entraînent la vérité dans  leur ruine. Car, une fois qu’on s’est aperçu que le contempteur du plaisir y a lui-même tendance, son inclination au plaisir semble bien indiquer que tout plaisir est digne d’être poursuivi, les distinctions à faire n’étant pas à la portée du grand public. Il apparaît ainsi que ce sont les arguments conformes à la vérité qui sont les plus utiles, et cela non seulement pour la \\
connaissance pure, mais encore pour la vie pratique : car, étant en harmonie avec les faits, ils emportent la conviction, et de cette façon incitent ceux qui les comprennent à y conformer leur vie. — Mais en voilà assez sur ces questions ; passons maintenant en revue les opinions qu’on a avancées sur le plaisir.
\subsection[{2 (1172b — 1174a) < Critique des théories d’Eudoxe et de Speusippe >}]{2 (1172b — 1174a) < Critique des théories d’Eudoxe et de Speusippe >}
\noindent \\
Eudoxe, donc, pensait que le plaisir est le bien, du fait qu’il voyait tous les êtres, raisonnables ou irraisonnables, tendre au plaisir ; or chez tous les êtres, ce qui est désiré est ce qui leur convient équitablement, et ce qui est désiré au plus haut degré est le Bien par excellence ; et le fait que tous les êtres sont portés vers le même objet est le signe que cet objet est pour tous ce qu’il y a de mieux (puisque chaque être découvre ce qui est bon pour lui, comme il trouve aussi la nourriture qui lui est appropriée) ; dès lors ce qui est bon pour tous les êtres et vers quoi ils tendent tous est le Souverain Bien.\par
\\
Si cependant ces arguments entraînaient la conviction, c’était plutôt à cause de la gravité du caractère de leur auteur qu’en raison de leur valeur intrinsèque. Eudoxe avait, en effet, la réputation d’un homme exceptionnellement tempérant, et par suite on admettait que s’il soutenait cette théorie, ce n’était pas par amour du plaisir, mais parce qu’il en est ainsi dans la réalité.\par
Il croyait encore que sa doctrine résultait non moins manifestement de cet argument {\itshape a contrario} : la peine étant en soi un objet d’aversion pour tous les êtres, il suit que son contraire doit pareillement être en soi un objet de désir pour \\
tous. — En outre, selon lui, est désirable au plus haut point ce que nous ne choisissons pas à cause d’une autre chose, ni en vue d’une autre chose : tel est précisément, de l’aveu unanime, le caractère du plaisir, car on ne demande jamais à quelqu’un en vue de quelle fin il se livre au plaisir, ce qui implique bien que le plaisir est désirable par lui-même. — De plus, le plaisir, ajouté à un bien quelconque, par exemple à une activité juste \\
ou tempérante, rend ce bien plus désirable : or le bien ne peut être augmenté que par le bien lui-même.\par
Ce dernier argument, en tout cas, montre seulement, semble-t-il, que le plaisir est l’un des biens, et nullement qu’il est meilleur qu’un autre bien, car tout bien, uni à un autre bien, est plus désirable que s’il est seul. Aussi, est-ce par un argument de ce genre que Platon ruine l’identification du bien au \\
plaisir : la vie de plaisir, selon lui, est plus désirable unie à la prudence que séparée d’elle, et si la vie mixte est meilleure, c’est que le plaisir n’est pas le bien, car aucun complément ajouté au bien ne peut rendre celui-ci plus désirable. Il est clair aussi qu’aucune autre chose non plus ne saurait être le bien, si, par l’adjonction de quelqu’une des choses qui sont bonnes en elles-mêmes, elle devient plus désirable. Quelle est donc la chose qui répond à la condition posée et à laquelle nous \\
puissions avoir part ? Car c’est un bien de ce genre que nous recherchons.\par
Ceux, d’autre part, qui objectent que ce à quoi tous les êtres tendent n’est pas forcément un bien, il est à craindre qu’ils ne parlent pour ne rien dire. Les choses, en effet, que tous les  hommes reconnaissent comme bonnes, nous disons qu’elles sont telles en réalité : et celui qui s’attaque à cette conviction trouvera lui-même difficilement des vérités plus croyables. Si encore les êtres dépourvus de raison étaient seuls à aspirer aux plaisirs, ce que disent ces contradicteurs pourrait présenter un certain sens ; mais si les êtres doués d’intelligence manifestent aussi la même tendance, quel sens pourront bien présenter leurs allégations ? Et peut-être même, chez les êtres inférieurs existe-t-il quelque principe naturel et bon, supérieur à ce que \\
ces êtres sont par eux-mêmes, et qui tend à réaliser leur bien propre.\par
Il ne semble pas non plus que leur critique de l’argument {\itshape a contrario} soit exacte. Ils prétendent, en effet, que si la peine est un mal, il ne s’ensuit pas que le plaisir soit un bien : car un mal peut être opposé aussi à un mal, et ce qui est à la fois bien et mal peut être opposé à ce qui n’est ni bien ni mal. Ce raisonnement n’est pas sans valeur, mais il n’est pas conforme à la vérité, du moins dans le présent cas. Si, en effet, plaisir et peine \\
sont tous deux des maux, ils devraient aussi tous deux être objet d’aversion, et s’ils ne sont tous deux ni bien ni mal ils ne devraient être ni l’un ni l’autre objet d’aversion ou devraient l’être tous deux pareillement. Mais ce qu’en réalité on constate, c’est que l’on fuit l’une comme un mal, et que l’on préfère l’autre comme un bien : c’est donc comme bien et mal que le plaisir et la peine sont opposés l’un à l’autre.\par
Mais il ne s’ensuit pas non plus, dans l’hypothèse où le plaisir n’est pas au nombre des qualités, qu’il ne soit pas pour autant au nombre des biens, car les activités vertueuses ne sont \\
pas davantage des qualités, ni le bonheur non plus. — Ils disent encore que le bien est déterminé, tandis que le plaisir est indéterminé, parce qu’il est susceptible de plus et de moins. Si c’est sur l’expérience même du plaisir qu’ils appuient ce jugement, quand il s’agira de la justice et des autres vertus (à propos desquelles on dit ouvertement que leurs possesseurs sont plus ou moins dans cet état, et leurs actions plus ou moins conformes \\
à ces vertus) on pourra en dire autant (car il est possible d’être plus juste ou plus brave que d’autres, et il est possible de pratiquer aussi la justice ou la tempérance mieux que d’autres). Mais si leur jugement se fonde sur la nature même des plaisirs, je crains qu’ils n’indiquent pas la véritable cause, s’il est vrai qu’il existe d’une part les plaisirs sans mélange, et d’autre part les plaisirs mixtes. Qui empêche, au surplus, qu’il n’en soit du plaisir comme de la santé, laquelle, tout en étant déterminée, \\
admet cependant le plus et le moins ? La même proportion, en effet, ne se rencontre pas en tous les individus, et dans le même individu elle n’est pas non plus toujours identique, mais elle peut se relâcher et cependant persister jusqu’à un certain point, et différer ainsi selon le plus et le moins. Tel peut être aussi, par conséquent, le cas du plaisir.\par
De plus, ils posent en principe à la fois que le bien est parfait, et les mouvements et les devenirs imparfaits, puis ils \\
s’efforcent de montrer que le plaisir est un mouvement et un devenir. Mais ils ne semblent pas s’exprimer exactement, même quand ils soutiennent que le plaisir est un mouvement : tout mouvement, admet-on couramment, a pour propriétés vitesse ou lenteur, et si un mouvement, celui du Ciel par exemple, n’a pas ces propriétés par lui-même, il les possède du moins relativement à un autre mouvement. Or au plaisir n’appartiennent ni l’une ni l’autre de ces sortes de mouvements. Il est assurément possible d’être {\itshape amené} vers le plaisir plus ou moins rapidement, comme aussi de se mettre en colère,  mais on ne peut pas être dans l’état de plaisir rapidement pas même par rapport à une autre personne, alors que nous pouvons marcher, croître, et ainsi de suite, plus ou moins rapidement. Ainsi donc, il est possible de passer à l’état de plaisir rapidement ou lentement, mais il n’est pas possible d’être en acte dans cet état (je veux dire être dans l’état de plaisir) plus ou moins rapidement. De plus, en quel sens le \\
plaisir serait-il un devenir ? Car on n’admet pas d’ordinaire que n’importe quoi naisse de n’importe quoi, mais bien qu’une chose se résout en ce dont elle provient ; et la peine est la destruction de ce dont le plaisir est la génération.\par
Ils disent encore que la peine est un processus de déficience de notre état naturel, et le plaisir un processus de réplétion. Mais ce sont là des affections intéressant le corps. Si dès lors le plaisir est une réplétion de l’état naturel, c’est le \\
sujet en lequel s’accomplit la réplétion qui ressentira le plaisir ; ce sera donc le corps. Mais c’est là une opinion qu’on n’accepte pas d’ordinaire ; le plaisir n’est donc pas non plus un processus de réplétion ; tout ce qu’on peut dire, c’est qu’au cours d’un processus de réplétion on ressentira du plaisir, comme au cours d’une opération chirurgicale on ressentira de la souffrance. En fait, cette opinion semble avoir pour origine les souffrances et les plaisirs ayant rapport à la nutrition : quand, en effet, le manque de nourriture nous a d’abord fait ressentir de la souffrance, \\
nous éprouvons ensuite du plaisir en assouvissant notre appétit. Mais cela ne se produit pas pour tous les plaisirs : par exemple les plaisirs apportés par l’étude ne supposent pas de peine antécédente, ni parmi les plaisirs des sens ceux qui ont l’odorat pour cause, et aussi un grand nombre de sons et d’images, ainsi que des souvenirs ou des attentes. De quoi donc ces plaisirs-là seront-ils des processus de génération ? \\
Aucun manque de quoi que ce soit ne s’est produit dont ils seraient une réplétion.\par
À ceux qui mettent en avant les plaisirs répréhensibles, on pourrait répliquer que ces plaisirs ne sont pas agréables en soi : car en supposant même qu’ils soient agréables aux gens de constitution vicieuse, il ne faut pas croire qu’ils soient agréables aussi à d’autres qu’à eux, pas plus qu’on ne doit penser que les choses qui sont salutaires, ou douces, ou amères aux malades soient réellement telles, ou que les choses qui \\
paraissent blanches à ceux qui souffrent des yeux soient réellement blanches. On pourrait encore répondre ainsi : les plaisirs sont assurément désirables, mais non pas du moins quand ils proviennent de ces sources-là, de même que la richesse est désirable, mais non comme salaire d’une trahison, ou la santé, mais non au prix de n’importe quelle nourriture. Ou peut-être encore les plaisirs sont-ils spécifiquement différents : ceux, en effet, qui proviennent de sources nobles sont autres que ceux qui proviennent de sources honteuses, et il n’est pas possible de ressentir le plaisir de l’homme juste sans être soi-même \\
juste, ni le plaisir du musicien sans être musicien, et ainsi pour tous les autres plaisirs. Et de plus, le fait que l’ami est autre que le flatteur semble montrer clairement que le plaisir n’est pas un bien, ou qu’il y a des plaisirs spécifiquement différents. L’ami, en effet, paraît rechercher notre compagnie pour notre bien, et le flatteur pour notre plaisir, et à ce dernier on adresse des reproches et à l’autre des éloges, en  raison des fins différentes pour lesquelles ils nous fréquentent. En outre, nul homme ne choisirait de vivre en conservant durant toute son existence l’intelligence d’un petit enfant, même s’il continuait à jouir le plus possible des plaisirs de l’enfance ; nul ne choisirait non plus de ressentir du plaisir en accomplissant un acte particulièrement déshonorant, même s’il n’en devait jamais en résulter pour lui de conséquence pénible. Et il y a aussi bien des avantages que nous mettrions \\
tout notre empressement à obtenir, même s’ils ne nous apportaient aucun plaisir, comme voir, se souvenir, savoir, posséder les vertus. Qu’en fait des plaisirs accompagnent nécessairement ces avantages ne fait pour nous aucune différence, puisque nous les choisirions quand bien même ils ne seraient pour nous la source d’aucun plaisir.\par
Qu’ainsi donc le plaisir ne soit pas le bien, ni que tout \\
plaisir soit désirable, c’est là une chose, semble-t-il, bien évidente, et il est non moins évident que certains plaisirs sont désirables par eux-mêmes, parce qu’ils sont différents des autres par leur espèce ou par les sources d’où ils proviennent.
\subsection[{3 (1174a — 1174b) < La nature du plaisir >}]{3 (1174a — 1174b) < La nature du plaisir >}
\noindent Les opinions relatives au plaisir et à la peine ont été suffisamment étudiées.\par
Qu’est-ce que le plaisir et quelle sorte de chose est-il ? Cela deviendra plus clair si nous reprenons le sujet à son début. On admet d’ordinaire que l’acte de vision est parfait à \\
n’importe quel moment de sa durée (car il n’a besoin d’aucun complément qui surviendrait plus tard et achèverait sa forme). Or telle semble bien être aussi la nature du plaisir : il est, en effet, un tout, et on ne saurait à aucun moment appréhender un plaisir dont la prolongation dans le temps conduirait la forme à sa perfection. C’est la raison pour laquelle il n’est pas non plus un mouvement. Tout mouvement, en effet, se déroule dans \\
le temps, et en vue d’une certaine fin, comme par exemple le processus de construction d’une maison, et il est parfait quand il a accompli ce vers quoi il tend ; dès lors il est parfait soit quand il est pris dans la totalité du temps qu’il occupe, soit à son moment final. Et dans les parties du temps qu’ils occupent tous les mouvements sont imparfaits, et diffèrent spécifiquement du mouvement total comme ils diffèrent aussi l’un de l’autre. Ainsi, l’assemblage des pierres est autre que le travail de cannelure de la colonne, et ces deux opérations sont elles-mêmes autres que la construction du temple comme un tout. Et \\
tandis que la construction du temple est un processus parfait (car elle n’a besoin de rien d’autre pour atteindre la fin proposée), le travail du soubassement et celui du triglyphe sont des processus imparfaits (chacune de ces opérations ne produisant qu’une partie du tout). Elles diffèrent donc spécifiquement, et il n’est pas possible, à un moment quelconque de sa durée, de saisir un mouvement qui soit parfait selon sa forme, mais s’il apparaît tel, c’est seulement dans la totalité de sa durée. On peut en dire autant de la marche et des autres formes de locomotion. \\
Si, en effet, la translation est un mouvement d’un point à un autre, et si on relève en elle des différences spécifiques, le vol, la marche, le saut, et ainsi de suite, ce ne sont cependant pas les seules, mais il en existe aussi dans la marche elle-même par exemple, car le mouvement qui consiste à aller d’un point à un autre n’est pas le même dans le parcours total du stade et dans le parcours partiel, ni dans le parcours de telle partie ou de telle autre, ni dans le franchissement de cette ligne-ci  et de celle-là, puisqu’on ne traverse pas seulement une ligne quelconque, mais une ligne tirée dans un lieu déterminé, et que l’une est dans un lieu différent de l’autre. Nous avons traité avec précision du mouvement dans un autre ouvrage, mais on peut dire ici, semble-t-il, qu’il n’est parfait à aucun moment de sa durée, mais que les multiples mouvements partiels dont il est composé sont imparfaits et différents en espèce, puisque \\
c’est le point de départ et le point d’arrivée qui les spécifient. Du plaisir, au contraire, la forme est parfaite à n’importe quel moment. — On voit ainsi que plaisir et mouvement ne sauraient être que différents l’un de l’autre, et que le plaisir est au nombre de ces choses qui sont des touts parfaits. Cette conclusion pourrait ainsi résulter du fait qu’il est impossible de se mouvoir autrement que dans le temps, alors qu’il est possible de ressentir le plaisir indépendamment du temps, car ce qui a lieu dans l’instant est un tout complet.\par
\\
De ces considérations il résulte clairement encore qu’on a tort de parler du plaisir comme étant le résultat d’un mouvement ou d’une génération, car mouvement et devenir ne peuvent être affirmés de toutes les choses, mais seulement de celles qui sont divisibles en parties et ne sont pas des touts : il n’y a devenir, en effet, ni d’un acte de vision, ni d’un point, ni d’une unité, et aucune de ces choses n’est mouvement ou devenir. Il n’y a dès lors rien de tel pour le plaisir, puisqu’il est un tout.
\subsection[{4 (1174b — 1175a) < Plaisir et acte >}]{4 (1174b — 1175a) < Plaisir et acte >}
\noindent Chaque sens passant à l’acte par rapport à l’objet sensible \\
correspondant, et cette opération se révélant parfaite quand le sens est dans une disposition saine par rapport au meilleur des objets qui tombent sous lui (car telle est, semble-t-il, la meilleure description qu’on puisse donner de l’activité parfaite : que ce soit au reste le sens lui-même qu’on dise passer à l’acte ou l’organe dans lequel il réside, peu importe), il s’ensuit que pour chaque sens l’acte le meilleur est celui du sens le mieux disposé par rapport au plus excellent de ses objets ; et l’acte répondant à ces conditions ne saurait être que \\
le plus parfait comme aussi le plus agréable. Car pour chacun des sens il y a un plaisir qui lui correspond, et il en est de même pour la pensée discursive et la contemplation, et leur activité la plus parfaite est la plus agréable, l’activité la plus parfaite étant celle de l’organe qui se trouve en bonne disposition par rapport au plus excellent des objets tombant sous le sens en question ; et le plaisir est l’achèvement de l’acte. Le plaisir cependant n’achève pas l’acte de la même façon que le font à la fois le \\
sensible et le sens quand ils sont l’un et l’autre en bon état, tout comme la santé et le médecin ne sont pas au même titre cause du rétablissement de la santé. — Que pour chaque sens naisse un plaisir correspondant, c’est là une chose évidente, puisque nous disons que des images et des sons peuvent être agréables. Il est évident encore que le plaisir atteint son plus haut point, quand à la fois le sens est dans la meilleure condition et s’actualise par rapport à l’objet également le meilleur. \\
Et le sensible comme le sentant étant tels que nous venons de les décrire, toujours il y aura plaisir dès que seront mis en présence le principe efficient et le principe passif.\par
Le plaisir achève l’acte, non pas comme le ferait une disposition immanente au sujet, mais comme une sorte de fin survenue par surcroît, de même qu’aux hommes dans la force de l’âge vient s’ajouter la fleur de la jeunesse. Aussi longtemps donc que l’objet intelligible ou sensible est tel qu’il doit être,  ainsi que le sujet discernant ou contemplant, le plaisir résidera dans l’acte : car l’élément passif et l’élément actif restant tous deux ce qu’ils sont et leurs relations mutuelles demeurant dans le même état, le même résultat se produit naturellement.\par
Comment se fait-il alors que personne ne ressente le plaisir d’une façon continue ? La cause n’en est-elle pas la fatigue ? En effet, toutes les choses humaines sont incapables d’être \\
dans une continuelle activité, et par suite le plaisir non plus ne l’est pas, puisqu’il est un accompagnement de l’acte. C’est pour la même raison que certaines choses nous réjouissent quand elles sont nouvelles, et que plus tard elles ne nous plaisent plus autant : au début, en effet, la pensée se trouve dans un état d’excitation et d’intense activité à l’égard de ces objets, comme pour la vue quand on regarde avec attention ; mais par la suite l’activité n’est plus ce qu’elle était, mais elle se relâche, \\
ce qui fait que le plaisir aussi s’émousse.\par
On peut croire que si tous les hommes sans exception aspirent au plaisir, c’est qu’ils ont tous tendance à vivre. La vie est une certaine activité, et chaque homme exerce son activité dans le domaine et avec les facultés qui ont pour lui le plus d’attrait : par exemple, le musicien exerce son activité, au moyen de l’ouïe, sur les mélodies, l’homme d’étude, au moyen \\
de la pensée, sur les spéculations de la science, et ainsi de suite dans chaque cas. Et le plaisir vient parachever les activités, et par suite la vie à laquelle on aspire. Il est donc normal que les hommes tendent aussi au plaisir, puisque pour chacun d’eux le plaisir achève la vie, qui est une chose désirable.
\subsection[{5 (1175a — 1176a) < La diversité spécifique des plaisirs >}]{5 (1175a — 1176a) < La diversité spécifique des plaisirs >}
\noindent Quant à savoir si nous choisissons la vie à cause du plaisir, ou le plaisir à cause de la vie, c’est une question que nous pouvons laisser de côté pour le moment. En fait, ces deux tendances sont, de toute évidence, intimement associées et \\
n’admettent aucune séparation : sans activité, en effet, il ne naît pas de plaisir, et toute activité reçoit son achèvement du plaisir.\par
De là vient aussi qu’on reconnaît une différence spécifique entre les plaisirs. En effet, nous pensons que les choses différentes en espèce reçoivent leur achèvement de causes elles-mêmes différentes (tel est manifestement ce qui se passe pour les êtres naturels et les produits de l’art, comme par exemple les animaux et les arbres, d’une part, et, d’autre part, un tableau, \\
une statue, une maison, un ustensile) ; de même nous pensons aussi que les activités qui diffèrent spécifiquement sont achevées par des causes spécifiquement différentes. Or les activités de la pensée diffèrent spécifiquement des activités 222 sensibles, et toutes ces activités diffèrent à leur tour spécifiquement entre elles : et par suite les plaisirs qui complètent ces activités diffèrent de la même façon.\par
Cette différence entre les plaisirs peut encore être rendue manifeste au moyen de l’indissoluble union existant entre chacun des plaisirs et l’activité qu’il complète. Une activité est, \\
en effet, accrue par le plaisir qui lui est approprié, car dans tous les domaines on agit avec plus de discernement et de précision quand on exerce son activité avec plaisir : ainsi ceux qui aiment la géométrie deviennent meilleurs géomètres et comprennent mieux les diverses propositions qui s’y rapportent ; et de même ce sont les passionnés de musique, d’architecture et autres arts \\
qui font des progrès dans leur tâche propre, parce qu’ils y trouvent leur plaisir. Les plaisirs accroissent les activités qu’ils accompagnent, et ce qui accroît une chose doit être approprié à  cette chose. Mais à des choses différentes en espèce les choses qui leur sont propres doivent elles-mêmes différer en espèce.\par
Une autre confirmation plus claire encore peut être tirée du fait que les plaisirs provenant d’autres activités constituent une gêne pour les activités en jeu : par exemple, les amateurs de flûte sont incapables d’appliquer leur esprit à une argumentation dès qu’ils écoutent un joueur de flûte, car ils se plaisent \\
davantage à l’art de la flûte qu’à l’activité où ils sont présentement engagés : ainsi, le plaisir causé par le son de la flûte détruit l’activité se rapportant à la discussion en cours. Le même phénomène s’observe aussi dans tous les autres cas où on exerce son activité sur deux objets en même temps : l’activité plus agréable chasse l’autre, et cela d’autant plus qu’elle l’emporte davantage sous le rapport du plaisir, au point d’amener \\
la cessation complète de l’autre activité. C’est pourquoi, lorsque nous éprouvons un plaisir intense à une occupation quelconque, nous pouvons difficilement nous livrer à une autre ; et, d’autre part, nous nous tournons vers une autre occupation quand l’occupation présente ne nous plaît que médiocrement : par exemple, ceux qui au théâtre mangent des sucreries le font surtout quand les acteurs sont mauvais. Et puisque le plaisir approprié aux activités aiguise celles-ci, prolonge leur durée et les rend plus efficaces, et qu’au contraire les \\
plaisirs étrangers les gâtent, il est clair qu’il existe entre ces deux espèces de plaisirs un écart considérable. Les plaisirs résultant d’activités étrangères produisent sur les activités en cours à peu près le même effet que les peines propres à ces dernières, puisque les activités sont détruites par leurs propres peines : par exemple, si écrire ou calculer est pour quelqu’un une chose désagréable et fastidieuse, il cesse alors d’écrire ou \\
de calculer, l’activité en question lui étant pénible. Ainsi donc, les activités sont affectées en sens opposé par les plaisirs et les peines qui leur sont propres, et sont propres les plaisirs et les peines qui surviennent à l’activité en raison de sa nature même. Quant aux plaisirs qui relèvent d’activités étrangères, ils produisent, nous l’avons dit, sensiblement le même effet que la peine, car ils détruisent l’activité, bien que ce ne soit pas de la même manière.\par
Et puisque les activités diffèrent par leur caractère moralement \\
honnête ou pervers, et que les unes sont désirables, d’autres à éviter, d’autres enfin ni l’un ni l’autre, il en est de même aussi pour les plaisirs, puisque à chaque activité correspond un plaisir propre. Ainsi donc, le plaisir propre à l’activité vertueuse est honnête, et celui qui est propre à l’activité perverse, mauvais : car même les appétits qui se proposent une fin noble provoquent la louange, et ceux qui se proposent une fin \\
honteuse, le blâme. Or les plaisirs inhérents à nos activités sont plus étroitement liés à ces dernières que les désirs : car les désirs sont distincts des activités à la fois chronologiquement et par leur nature, tandis que les plaisirs sont tout proches des activités et en sont à ce point inséparables que la question est débattue de savoir si l’acte n’est pas identique au plaisir. Cependant, à ce qu’il semble du moins, le plaisir n’est ni pensée, ni sensation (ce qui serait absurde), mais l’impossibilité \\
de les séparer les rend, aux yeux de certains, une seule et même chose. Ainsi, de même que les activités sont différentes, ainsi en est-il des plaisirs.\par
 En outre, la vue l’emporte sur le toucher en pureté, et l’ouïe et l’odorat sur le goût ; il y a dès lors une différence de même nature entre les plaisirs correspondants ; et les plaisirs de la pensée sont supérieurs aux plaisirs sensibles, et dans chacun de ces deux groupes il y a des plaisirs qui l’emportent sur d’autres.\par
De plus, on admet d’ordinaire que chaque espèce animale a son plaisir propre, tout comme elle a une fonction propre, à savoir le plaisir qui correspond à son activité. Et à considérer \\
chacune des espèces animales, on ne saurait manquer d’en être frappé : cheval, chien et homme ont des plaisirs différents : comme le dit Héraclite, {\itshape un âne préférera la paille à l’or}, car la nourriture est pour des ânes une chose plus agréable que l’or. Ainsi donc, les êtres spécifiquement différents ont aussi des plaisirs spécifiquement distincts. D’un autre côté on s’attendrait à ce que les plaisirs des êtres spécifiquement identiques \\
fussent eux-mêmes identiques. En fait, les plaisirs accusent une extrême diversité, tout au moins chez l’homme : les mêmes choses charment certaines personnes et affligent les autres, et ce qui pour les uns est pénible et haïssable est pour les autres agréable et attrayant. Pour les saveurs douces il en va aussi de même : la même chose ne semble pas douce au fiévreux et à l’homme bien portant, ni pareillement chaude à l’homme \\
chétif et à l’homme robuste. Et ce phénomène arrive encore dans d’autres cas. Mais dans tous les faits de ce genre on regarde comme existant réellement ce qui apparaît à l’homme vertueux. Et si cette règle est exacte, comme elle semble bien l’être, et si la vertu et l’homme de bien, en tant que tel, sont mesure de chaque chose, alors seront des plaisirs les plaisirs qui à cet homme apparaissent tels, et seront plaisantes en réalité les choses auxquelles il se plaît. Et si les objets qui sont \\
pour lui ennuyeux paraissent plaisants à quelque autre, cela n’a rien de surprenant, car il y a beaucoup de corruptions et de perversions dans l’homme ; et de tels objets ne sont pas réellement plaisants, mais le sont seulement pour les gens dont nous parlons et pour ceux qui sont dans leur état. Les plaisirs qu’on s’accorde à reconnaître pour honteux, on voit donc qu’ils ne doivent pas être appelés des plaisirs, sinon pour les gens corrompus. Mais parmi les plaisirs considérés comme honnêtes, de quelle classe de plaisirs ou de quel plaisir \\
déterminé doit-on dire qu’il est proprement celui de l’homme ? La réponse ne résulte-t-elle pas avec évidence des activités humaines ?\par
Les plaisirs, en effet, en sont l’accompagnement obligé. Qu’ainsi donc l’activité de l’homme parfait et jouissant de la béatitude soit une ou multiple, les plaisirs qui complètent ces activités seront appelés au sens absolu plaisirs propres de l’homme, et les autres ne seront des plaisirs qu’à titre secondaire et à un moindre degré, comme le sont les activités correspondantes.
\subsection[{6 (1176a — 1177a) < Bonheur, activité et jeu >}]{6 (1176a — 1177a) < Bonheur, activité et jeu >}
\noindent \\
Après avoir parlé des différentes sortes de vertus, d’amitiés et de plaisirs, il reste à tracer une esquisse du bonheur, puisque c’est ce dernier que nous posons comme fin des affaires humaines. Mais si nous reprenons nos précédentes analyses, notre discussion y gagnera en concision.\par
Nous avons dit que le bonheur n’est pas une disposition, car alors il pourrait appartenir même à l’homme qui passe sa \\
vie à dormir, menant une vie de végétal, ou à celui qui subit les plus grandes infortunes. Si ces conséquences ne donnent pas  satisfaction, mais si nous devons plutôt placer le bonheur dans une certaine activité, ainsi que nous l’avons antérieurement indiqué, et si les activités sont les unes nécessaires et désirables en vue d’autres choses, et les autres désirables en elles-mêmes, il est clair qu’on doit mettre le bonheur au nombre des activités désirables en elles-mêmes et non de celles qui ne sont \\
désirables qu’en vue d’autre chose : car le bonheur n’a besoin de rien, mais se suffit pleinement à lui-même.\par
Or sont désirables en elles-mêmes les activités qui ne recherchent rien en dehors de leur pur exercice. Telles apparaissent être les actions conformes à la vertu, car accomplir de nobles et honnêtes actions est l’une de ces choses désirables en elles-mêmes.\par
Mais parmi les jeux ceux qui sont agréables font aussi \\
partie des choses désirables en soi : nous ne les choisissons pas en vue d’autres choses, car ils sont pour nous plus nuisibles qu’utiles, nous faisant négliger le soin de notre corps et de nos biens. Pourtant la plupart des hommes qui sont réputés heureux ont recours à des distractions de cette sorte, ce qui fait qu’à la cour des tyrans on estime fort les gens d’esprit qui s’adonnent à de tels passe-temps, car en satisfaisant les désirs \\
de leurs maîtres ils se montrent eux-mêmes agréables à leurs yeux, et c’est de ce genre de complaisants dont les tyrans ont besoin. Quoi qu’il en soit, on pense d’ordinaire que les amusements procurent le bonheur pour la raison que les puissants de ce monde y consacrent leurs loisirs, — quoique sans doute la conduite de tels personnages n’ait en l’espèce aucune signification. Ce n’est pas, en effet, dans le pouvoir absolu que résident la vertu et l’intelligence, d’où découlent les activités vertueuses, et si les gens dont nous parlons, qui ne ressentent \\
aucun goût pour un plaisir pur et digne d’un homme libre, s’évadent vers les plaisirs corporels, nous ne devons pas croire pour cela que ces plaisirs sont plus souhaitables : car les enfants, aussi, s’imaginent que les choses qui ont pour eux-mêmes du prix sont d’une valeur incomparable. Il en découle logiquement que les appréciations des gens pervers et des gens de bien sont tout aussi différentes les unes des autres que sont visiblement différentes celles des enfants et des adultes. Par \\
conséquent, ainsi que nous l’avons dit à maintes reprises, sont à la fois dignes de prix et agréables les choses qui sont telles pour l’homme de bien ; et pour tout homme l’activité la plus désirable étant celle qui est en accord avec sa disposition propre, il en résulte que pour l’homme de bien c’est l’activité conforme à la vertu. Ce n’est donc pas dans le jeu que consiste le bonheur. Il serait en effet étrange que la fin de l’homme fût le jeu, et qu’on dût se donner du tracas et du mal pendant toute \\
sa vie afin de pouvoir s’amuser ! Car pour le dire en un mot, tout ce que nous choisissons est choisi en vue d’une autre chose, à l’exception du bonheur, qui est une fin en soi. Mais se dépenser avec tant d’ardeur et de peine en vue de s’amuser ensuite est, de toute évidence, quelque chose d’insensé et de puéril à l’excès ; au contraire, s’amuser en vue d’exercer une activité sérieuse, suivant le mot d’Anacharsis, voilà, semble-t-il, la règle à suivre. Le jeu est, en effet, une sorte de délassement, \\
du fait que nous sommes incapables de travailler d’une façon ininterrompue et que nous avons besoin de relâche. Le  délassement n’est donc pas une fin, car il n’a lieu qu’en vue de l’activité. Et la vie heureuse semble être celle qui est conforme à la vertu ; or une vie vertueuse ne va pas sans un effort sérieux et ne consiste pas dans un simple jeu. Et nous affirmons, à la fois, que les choses sérieuses sont moralement supérieures à celles qui font rire ou s’accompagnent d’amusement, et que l’activité la plus sérieuse est toujours celle de la partie \\
meilleure de nous-mêmes ou celle de l’homme d’une moralité plus élevée. Par suite, l’activité de ce qui est meilleur est elle-même supérieure et plus apte à procurer le bonheur. De plus, le premier venu, fût-ce un esclave, peut jouir des plaisirs du corps, tout autant que l’homme de la plus haute classe, alors que personne n’admet la participation d’un esclave au bonheur, à moins de lui attribuer aussi une existence humaine. Ce n’est pas, en effet, dans de telles distractions que réside le bonheur, \\
mais dans les activités en accord avec la vertu, comme nous l’avons dit plus haut.
\subsection[{7 (1177a — 1178a) < La vie contemplative ou théorétique >}]{7 (1177a — 1178a) < La vie contemplative ou théorétique >}
\noindent Mais si le bonheur est une activité conforme à la vertu, il est rationnel qu’il soit activité conforme à la plus haute vertu, et celle-ci sera la vertu de la partie la plus noble de nous-mêmes. Que ce soit donc l’intellect ou quelque autre faculté qui soit regardé comme possédant par nature le commandement et la \\
direction et comme ayant la connaissance des réalités belles et divines, qu’au surplus cet élément soit lui-même divin ou seulement la partie la plus divine de nous-mêmes, c’est l’acte de cette partie selon la vertu qui lui est propre qui sera le bonheur parfait. Or que cette activité soit théorétique, c’est ce que nous avons dit.\par
Cette dernière affirmation paraîtra s’accorder tant avec nos précédentes conclusions qu’avec la vérité. En effet, \\
en premier lieu, cette activité est la plus haute, puisque l’intellect est la meilleure partie de nous-mêmes et qu’aussi les objets sur lesquels porte l’intellect sont les plus hauts de tous les objets connaissables. Ensuite elle est la plus continue, car nous sommes capables de nous livrer à la contemplation d’une manière plus continue qu’en accomplissant n’importe quelle action. Nous pensons encore que du plaisir doit être mélangé au bonheur ; or l’activité selon la sagesse est, tout le monde le reconnaît, la plus plaisante des \\
activités conformes à la vertu ; de toute façon, on admet que la philosophie renferme de merveilleux plaisirs sous le rapport de la pureté et de la stabilité, et il est normal que la joie de connaître soit une occupation plus agréable que la poursuite du savoir. De plus, ce qu’on appelle la pleine suffisance [αυταρκεια] appartiendra au plus haut point à l’activité de contemplation : car s’il est vrai qu’un homme sage, un homme juste, ou tout autre \\
possédant une autre vertu, ont besoin des choses nécessaires à la vie, cependant, une fois suffisamment pourvu des biens de ce genre, tandis que l’homme juste a encore besoin de ses semblables, envers lesquels ou avec l’aide desquels il agira avec justice (et il en est encore de même pour l’homme tempéré, l’homme courageux et chacun des autres), l’homme sage, au contraire, fût-il laissé à lui-même, garde la capacité de contempler, et il est même d’autant plus sage qu’il contemple dans cet état davantage. Sans doute est-il préférable pour lui d’avoir des collaborateurs, mais il n’en est pas moins l’homme qui se  suffit le plus pleinement à lui-même. Et cette activité paraîtra la seule à être aimée pour elle-même : elle ne produit, en effet, rien en dehors de l’acte même de contempler, alors que des activités pratiques nous retirons un avantage plus ou moins considérable à part de l’action elle-même. De plus, le bonheur semble consister dans le loisir : car nous ne nous adonnons \\
à une vie active qu’en vue d’atteindre le loisir, et ne faisons la guerre qu’afin de vivre en paix. Or l’activité des vertus pratiques s’exerce dans la sphère de la politique ou de la guerre ; mais les actions qui s’y rapportent paraissent bien être étrangères à toute idée de loisir, et, dans le domaine de la guerre elles revêtent même entièrement ce caractère, puisque personne ne \\
choisit de faire la guerre pour la guerre, ni ne prépare délibérément une guerre : on passerait pour un buveur de sang accompli, si de ses propres amis on se faisait des ennemis en vue de susciter des batailles et des tueries. Et l’activité de l’homme d’État est, elle aussi, étrangère au loisir, et, en dehors de l’administration proprement dite des intérêts de la cité, elle s’assure la possession du pouvoir et des honneurs, ou du moins le bonheur pour l’homme d’État lui-même et pour ses concitoyens, bonheur qui est différent de l’activité politique, et \\
qu’en fait nous recherchons ouvertement comme constituant un avantage distinct. Si dès lors, parmi les actions conformes à la vertu, les actions relevant de l’art politique ou de la guerre viennent en tête par leur noblesse et leur grandeur, et sont cependant étrangères au loisir et dirigées vers une fin distincte et ne sont pas désirables par elles-mêmes ; si, d’autre part, l’activité de l’intellect, activité contemplative, paraît bien \\
à la fois l’emporter sous le rapport du sérieux et n’aspirer à aucune autre fin qu’elle-même, et posséder un plaisir achevé qui lui est propre (et qui accroît au surplus son activité) ; si enfin la pleine suffisance, la vie de loisir, l’absence de fatigue (dans les limites de l’humaine nature), et tous les autres caractères qu’on attribue à l’homme jouissant de la félicité, sont les manifestations rattachées à cette activité : il en résulte que c’est cette dernière qui sera le parfait bonheur de l’homme, — quand \\
elle est prolongée pendant une vie complète, puisque aucun des éléments du bonheur ne doit être inachevé.\par
Mais une vie de ce genre sera trop élevée pour la condition humaine : car ce n’est pas en tant qu’homme qu’on vivra de cette façon, mais en tant que quelque élément divin est présent en nous. Et autant cet élément est supérieur au composé humain, autant son activité est elle-même supérieure à celle de l’autre sorte de vertu. Si donc l’intellect est quelque chose \\
de divin par comparaison avec l’homme, la vie selon l’intellect est également divine comparée à la vie humaine. Il ne faut donc pas écouter ceux qui conseillent à l’homme, parce qu’il est homme, de borner sa pensée aux choses humaines, et, mortel, aux choses mortelles, mais l’homme doit, dans la mesure du possible, s’immortaliser, et tout faire pour vivre selon la  partie la plus noble qui est en lui ; car même si cette partie est petite par sa masse, par sa puissance et sa valeur elle dépasse de beaucoup tout le reste. On peut même penser que chaque homme s’identifie avec cette partie même, puisqu’elle est la partie fondamentale de son être, et la meilleure. Il serait alors étrange que l’homme accordât la préférence non pas à la vie qui lui est propre, mais à la vie de quelque chose autre que lui. \\
Et ce que nous avons dit plus haut s’appliquera également ici : ce qui est propre à chaque chose est par nature ce qu’il y a de plus excellent et de plus agréable pour cette chose. Et pour l’homme, par suite, ce sera la vie selon l’intellect, s’il est vrai que l’intellect est au plus haut degré l’homme même. Cette vie-là est donc aussi la plus heureuse.
\subsection[{8 (1178a — 1178b) < Prééminence de la vie contemplative >}]{8 (1178a — 1178b) < Prééminence de la vie contemplative >}
\noindent C’est d’une façon secondaire qu’est heureuse la vie selon \\
l’autre sorte de vertu : car les activités qui y sont conformes sont purement humaines : les actes justes, en effet, ou courageux, et tous les autres actes de vertu, nous les pratiquons dans nos relations les uns avec les autres, quand, dans les contrats, les services rendus et les actions les plus variées ainsi que dans nos passions, nous observons fidèlement ce qui doit revenir à chacun, et toutes ces manifestations sont choses simplement humaines. Certaines mêmes d’entre elles sont regardées comme résultant de la constitution physique, et la vertu éthique \\
comme ayant, à beaucoup d’égards, des rapports étroits avec les passions. Bien plus, la prudence elle-même est intimement liée à la vertu morale, et cette dernière à la prudence, puisque les principes de la prudence dépendent des vertus morales, et la rectitude des vertus morales de la prudence. Mais \\
les vertus morales étant aussi rattachées aux passions, auront rapport au composé ; or les vertus du composé sont des vertus simplement humaines ; et par suite le sont aussi, à la fois la vie selon ces vertus et le bonheur qui en résulte. Le bonheur de l’intellect est, au contraire, séparé : qu’il nous suffise de cette brève indication à son sujet, car une discussion détaillée dépasse le but que nous nous proposons.\par
Le bonheur de l’intellect semblerait aussi avoir besoin du cortège des biens extérieurs, mais seulement à un faible \\
degré ou à un degré moindre que la vertu éthique. On peut admettre, en effet, que les deux sortes de vertus aient l’une et l’autre besoin, et cela à titre égal, des biens nécessaires à la vie (quoique, en fait, l’homme en société se donne plus de tracas pour les nécessités corporelles et autres de même nature) : car il ne saurait y avoir à cet égard qu’une légère différence entre elles. Par contre, en ce qui concerne leurs activités propres, la différence sera considérable. L’homme libéral, en effet, aura besoin d’argent pour répandre ses libéralités, et par suite \\
l’homme juste pour rétribuer les services qu’on lui rend (car les volontés demeurent cachées, et même les gens injustes prétendent avoir la volonté d’agir avec justice) ; de son côté l’homme courageux aura besoin de force, s’il accomplit quelqu’une des actions conformes à sa vertu, et l’homme tempérant a besoin d’une possibilité de se livrer à l’intempérance. Autrement, comment ce dernier, ou l’un des autres dont nous parlons, pourra-t-il manifester sa vertu ? On discute aussi le point de savoir quel est l’élément le plus important de la vertu, si c’est le \\
choix délibéré ou la réalisation de l’acte, attendu que la vertu  consiste dans ces deux éléments. La perfection de la vertu résidera évidemment dans la réunion de l’un et de l’autre, mais l’exécution de l’acte requiert le secours de multiples facteurs, et plus les actions sont grandes et nobles, plus ces conditions sont nombreuses. Au contraire, l’homme livré à la contemplation n’a besoin d’aucun concours de cette sorte, en vue du moins d’exercer son activité : ce sont même là plutôt, pour ainsi dire, \\
des obstacles, tout au moins à la contemplation ; mais en tant qu’il est homme et qu’il vit en société, il s’engage délibérément dans des actions conformes à la vertu : il aura donc besoin des moyens extérieurs en question pour mener sa vie d’homme.\par
Que le parfait bonheur soit une certaine activité théorétique, les considérations suivantes le montreront encore avec clarté. Nous concevons les dieux comme jouissant de la suprême félicité et du souverain bonheur. Mais quelles sortes \\
d’actions devons-nous leur attribuer ? Est-ce les actions justes ? Mais ne leur donnerons-nous pas un aspect ridicule en les faisant contracter des engagements, restituer des dépôts et autres opérations analogues ? Sera-ce les actions courageuses, et les dieux affronteront-ils les dangers et courront-ils des risques pour la beauté de la chose ? Ou bien alors ce sera des actes de libéralité ? Mais à qui donneront-ils ? Il serait étrange aussi qu’ils eussent à leur disposition de la monnaie ou quelque \\
autre moyen de paiement analogue ! Et les actes de tempérance, qu’est-ce que cela peut signifier dans leur cas ? N’est-ce pas une grossièreté de les louer de n’avoir pas d’appétits dépravés ? Si nous passons en revue toutes ces actions, les circonstances dont elles sont entourées nous apparaîtront mesquines et indignes de dieux.\par
Et pourtant on se représente toujours les dieux comme possédant la vie et par suite l’activité, car nous ne pouvons pas \\
supposer qu’ils dorment, comme Endymion264. Or, pour l’être vivant, une fois qu’on lui a ôté l’action et à plus forte raison la production, que lui laisse-t-on d’autre que la contemplation ? Par conséquent, l’activité de Dieu, qui en félicité surpasse toutes les autres, ne saurait être que théorétique. Et par suite, de toutes les activités humaines celle qui est la plus apparentée à l’activité divine sera aussi la plus grande source de bonheur.\par
Un signe encore, c’est que les animaux autres que \\
l’homme n’ont pas de participation au bonheur, du fait qu’ils sont totalement démunis d’une activité de cette sorte. Tandis qu’en effet chez les dieux la vie est tout entière bienheureuse, comme elle l’est aussi chez les hommes dans la mesure où une certaine ressemblance avec l’activité divine est présente en eux, dans le cas des animaux, au contraire, il n’y a pas trace de bonheur, parce que, en aucune manière, l’animal n’a part à la contemplation. Le bonheur est donc coextensif à la contemplation, \\
et plus on possède la faculté de contempler, plus aussi on est heureux, heureux non pas par accident, mais en vertu de la contemplation même, car cette dernière est par elle-même d’un grand prix. Il en résulte que le bonheur ne saurait être qu’une forme de contemplation.
\subsection[{9 (1178b — 1179a) < La vie contemplative et ses conditions matérielles >}]{9 (1178b — 1179a) < La vie contemplative et ses conditions matérielles >}
\noindent Mais le sage aura aussi besoin de la prospérité extérieure, puisqu’il est un homme : car la nature humaine ne se suffit pas pleinement à elle-même pour l’exercice de la contemplation, \\
mais il faut aussi que le corps soit en bonne santé, qu’il reçoive de la nourriture et tous autres soins. Cependant, s’il n’est pas  possible sans l’aide des biens extérieurs d’être parfaitement heureux, on ne doit pas s’imaginer pour autant que l’homme aura besoin de choses nombreuses et importantes pour être heureux : ce n’est pas, en effet, dans un excès d’abondance que résident la pleine suffisance et l’action, et on peut, sans \\
posséder l’empire de la terre et de la mer, accomplir de nobles actions, car même avec des moyens médiocres on sera capable d’agir selon la vertu. L’observation au surplus le montre clairement : les simples particuliers semblent en état d’accomplir des actions méritoires, tout autant que les puissants, et même mieux. Il suffit d’avoir la quantité de moyens strictement exigés par l’action vertueuse : alors sera heureuse la vie de l’homme agissant selon la vertu. Solon aussi donnait sans \\
doute un aperçu exact de l’homme heureux, quand il le montrait modérément entouré des biens extérieurs et ayant accompli (dans la pensée de Solon tout au moins) les plus beaux exploits, et ayant vécu dans la tempérance : car on peut, en possédant des biens médiocres, accomplir ce que l’on doit. De son côté, Anaxagore semble avoir pensé que l’homme heureux n’est ni \\
riche ni puissant, puisqu’il dit qu’il ne serait pas étonné qu’un tel homme apparût à la foule sous un aspect déconcertant : car la foule juge par les caractères extérieurs, qui sont les seuls qu’elle perçoit. Les opinions des sages semblent donc en plein accord avec nos propres arguments.\par
De pareilles considérations entraînent ainsi la conviction dans une certaine mesure, mais, dans le domaine de la conduite, la vérité se discerne aussi d’après les faits et la manière de vivre, car c’est sur l’expérience que repose la décision finale. \\
Nous devons dès lors examiner les conclusions qui précèdent en les confrontant avec les faits et la vie : si elles sont en harmonie avec les faits, il faut les accepter, mais si elles sont en désaccord avec eux, les considérer comme de simples vues de l’esprit.\par
L’homme qui exerce son intellect et le cultive semble être à la fois dans la plus parfaite disposition et le plus cher aux \\
dieux. Si, en effet, les dieux prennent quelque souci des affaires humaines, ainsi qu’on l’admet d’ordinaire, il sera également raisonnable de penser, d’une part qu’ils mettent leur complaisance dans la partie de l’homme qui est la plus parfaite et qui présente le plus d’affinité avec eux (ce ne saurait être que l’intellect), et, d’autre part, qu’ils récompensent généreusement les hommes qui chérissent et honorent le mieux cette partie, voyant que ces hommes ont le souci des choses qui leur sont chères à eux-mêmes, et se conduisent avec droiture et \\
noblesse. Or que tous ces caractères soient au plus haut degré l’apanage du sage, cela n’est pas douteux. Il est donc l’homme le plus chéri des dieux. Et ce même homme est vraisemblablement aussi le plus heureux de tous. Par conséquent, de cette façon encore, le sage sera heureux au plus haut point.
\subsection[{10 (1179a — 1181b) < Éthique et Politique >}]{10 (1179a — 1181b) < Éthique et Politique >}
\noindent Une question se pose : si ces matières et les vertus, en y ajoutant l’amitié et le plaisir, ont été suffisamment traitées dans leurs grandes lignes, devons-nous croire que notre dessein \\
a été totalement rempli ? Ou plutôt, comme nous l’assurons, ne doit-on pas dire que dans le domaine de la pratique, la fin ne  consiste pas dans l’étude et la connaissance purement théoriques des différentes actions, mais plutôt dans leur exécution ? Dès lors, en ce qui concerne également la vertu, il n’est pas non plus suffisant de savoir ce qu’elle est, mais on doit s’efforcer aussi de la posséder et de la mettre en pratique, ou alors tenter par quelque autre moyen, s’il en existe, de devenir des hommes de bien.\par
Quoi qu’il en soit, si les raisonnements étaient en eux-mêmes \\
suffisants pour rendre les gens honnêtes, {\itshape ils recevraient de nombreux et importants honoraires}, pour employer l’expression de Théognis, et cela à bon droit, et nous devrions en faire une ample provision. Mais en réalité, et c’est là un fait d’expérience, si les arguments ont assurément la force de stimuler et d’encourager les jeunes gens doués d’un esprit généreux, comme de rendre un caractère bien né et véritablement épris de noblesse morale perméable à la vertu, ils sont cependant \\
impuissants à inciter la grande majorité des hommes à une vie noble et honnête : la foule, en effet, n’obéit pas naturellement au sentiment de l’honneur, mais seulement à la crainte, ni ne s’abstient des actes honteux à cause de leur bassesse, mais par peur des châtiments ; car, vivant sous l’empire de la passion, les hommes poursuivent leurs propres satisfactions et les moyens de les réaliser, et évitent les peines qui y sont opposées, \\
et ils n’ont même aucune idée de ce qui est noble et véritablement agréable, pour ne l’avoir jamais goûté. Des gens de cette espèce, quel argument pourrait transformer leur nature ? Il est sinon impossible, du moins fort difficile d’extirper par un raisonnement les habitudes invétérées de longue date dans le caractère. Nous devons sans doute nous estimer heureux si, en possession de tous les moyens qui peuvent, à notre sentiment, nous rendre honnêtes, nous arrivons à participer en quelque mesure à la vertu.\par
\\
Certains pensent qu’on devient bon par nature, d’autres disent que c’est par habitude, d’autres enfin par enseignement. Les dons de la nature ne dépendent évidemment pas de nous, mais c’est par l’effet de certaines causes divines qu’ils sont l’apanage de ceux qui, au véritable sens du mot, sont des hommes fortunés. Le raisonnement et l’enseignement, de leur côté, ne sont pas, je le crains, également puissants chez tous les \\
hommes, mais il faut cultiver auparavant, au moyen d’habitudes, l’âme de l’auditeur, en vue de lui faire chérir ou détester ce qui doit l’être, comme pour une terre appelée à faire fructifier la semence. Car l’homme qui vit sous l’empire de la passion ne saurait écouter un raisonnement qui cherche à le détourner de son vice, et ne le comprendrait même pas. Mais l’homme qui est en cet état, comment est-il possible de le faire changer de sentiment ? Et, en général, ce n’est pas, semble-t-il, au raisonnement que cède la passion, c’est à la contrainte. Il \\
faut donc que le caractère ait déjà une certaine disposition propre à la vertu, chérissant ce qui est noble et ne supportant pas ce qui est honteux.\par
Mais recevoir en partage, dès la jeunesse, une éducation tournée avec rectitude vers la vertu est une chose difficile à imaginer quand on n’a pas été élevé sous de justes lois : car vivre dans la tempérance et la constance n’a rien d’agréable pour la plupart des hommes, surtout quand ils sont jeunes. Aussi convient-il de régler au moyen de lois la façon de les \\
élever, ainsi que leur genre de vie, qui cessera d’être pénible en  devenant habituel. Mais sans doute n’est-ce pas assez que pendant leur jeunesse des hommes reçoivent une éducation et des soins également éclairés ; puisqu’ils doivent, même parvenus à l’âge d’homme, mettre en pratique les choses qu’ils ont apprises et les tourner en habitudes, nous aurons besoin de lois pour cet âge aussi, et, d’une manière générale, pour toute la durée de la vie : la plupart des gens, en effet, obéissent à la nécessité plutôt qu’au raisonnement, et aux châtiments plutôt \\
qu’au sens du bien.\par
Telle est la raison pour laquelle certains pensent que le législateur a le devoir, d’une part, d’inviter les hommes à la vertu et de les exhorter en vue du bien, dans l’espoir d’être entendu de ceux qui, grâce aux habitudes acquises, ont déjà été amenés à la vertu : et, d’autre part, d’imposer à ceux qui sont désobéissants et d’une nature par trop ingrate, des punitions et des châtiments, et de rejeter totalement les incorrigibles \\
hors de la cité. L’homme de bien, ajoutent-ils, et qui vit pour la vertu, se soumettra au raisonnement, tandis que l’homme pervers, qui n’aspire qu’au plaisir, sera châtié par une peine, comme une bête de somme. C’est pourquoi ils disent encore que les peines infligées aux coupables doivent être de telle nature qu’elles soient diamétralement opposées aux plaisirs qu’ils ont goûtés.\par
\\
Si donc, comme nous l’avons dit, l’homme appelé à être bon doit recevoir une éducation et des habitudes d’homme de bien, et ensuite passer son temps dans des occupations honnêtes et ne rien faire de vil, soit volontairement, soit même involontairement, et si ces effets ne peuvent se réaliser que dans une vie soumise à une règle intelligente et à un ordre parfait, disposant de la force : dans ces conditions, l’autorité paternelle ne possède ni la force, ni la puissance coercitive (et il \\
en est de même, dès lors, de tout particulier pris individuellement, s’il n’est roi ou quelqu’un d’approchant), alors que la loi, elle, dispose d’un pouvoir contraignant, étant une règle qui émane d’une certaine prudence et d’une certaine intelligence. Et tandis que nous détestons les individus qui s’opposent à nos impulsions, même s’ils agissent ainsi à bon droit, la loi n’est à charge à personne en prescrivant ce qui est honnête. Mais ce \\
n’est qu’à Lacédémone et dans un petit nombre de cités qu’on voit le législateur accorder son attention à la fois à l’éducation et au genre de vie des citoyens ; dans la plupart des cités, on a complètement négligé les problèmes de ce genre, et chacun vit comme il l’entend, {\itshape dictant}, à la manière des Cyclopes, {\itshape la loi aux enfants et à l’épouse}. La meilleure solution est donc de \\
s’en remettre à la juste sollicitude de l’autorité publique et d’être capable de le faire. Mais si l’autorité publique s’en désintéresse, on estimera que c’est à chaque individu qu’il appartient d’aider ses propres enfants et ses amis à mener une vie vertueuse, ou du moins d’avoir la volonté de le faire. Mais il résultera, semble-t-il, de notre exposé qu’on sera particulièrement apte à s’acquitter de cette tâche, si on s’est pénétré de la science du législateur. Car l’éducation publique s’exerce \\
évidemment au moyen de lois, et seulement de bonnes lois produisent une bonne éducation : que ces lois soient écrites ou  non écrites, on jugera ce point sans importance ; peu importe encore qu’elles pourvoient à l’éducation d’un seul ou de tout un groupe, et à cet égard il en est comme pour la musique, la gymnastique et autres disciplines. De même, en effet, que dans les cités, les dispositions légales et les coutumes ont la force pour les sanctionner, ainsi en est-il dans les familles pour les \\
injonctions du père et les usages privés, et même dans ce cas la puissance coercitive est-elle plus forte en raison du lien qui unit le père aux enfants et des bienfaits qui en découlent : car chez les enfants préexistent une affection et une docilité naturelles. En outre, l’éducation individuelle est supérieure à l’éducation publique : il en est comme en médecine, où le repos et la diète sont en général indiqués pour le fiévreux, mais ne le sont peut-être pas pour tel fiévreux déterminé ; et sans doute encore \\
le maître de pugilat ne propose pas à tous ses élèves la même façon de combattre. On jugera alors qu’il est tenu un compte plus exact des particularités individuelles quand on a affaire à l’éducation privée, chaque sujet trouvant alors plus facilement ce qui répond à ses besoins.\par
Toutefois, les soins les plus éclairés seront ceux donnés à un homme pris individuellement, par un médecin ou un maître de gymnastique ou tout autre ayant la connaissance de l’universel, et sachant ce qui convient à tous ou à ceux qui \\
rentrent dans telle catégorie : car la science a pour objet le général, comme on le dit et comme cela est en réalité, non pas qu’il ne soit possible sans doute qu’un individu déterminé ne soit traité avec succès par une personne qui ne possède pas la connaissance scientifique, mais a observé avec soin, à l’aide de la seule expérience, les phénomènes survenant en chaque cas particulier, tout comme certains semblent être pour eux-mêmes \\
d’excellents médecins, mais seraient absolument incapables de soulager autrui. Néanmoins on admettra peut-être que celui qui souhaite devenir un homme d’art ou de science doit s’élever jusqu’à l’universel et en acquérir une connaissance aussi exacte que possible : car, nous l’avons dit, c’est l’universel qui est l’objet de la science. Il est vraisemblable dès lors que celui qui souhaite, au moyen d’une discipline éducative, rendre les hommes meilleurs, qu’ils soient en grand nombre ou en petit nombre, doit s’efforcer de devenir lui-même capable de légiférer, \\
si c’est bien par les lois que nous pouvons devenir bons : mettre, en effet, un individu quel qu’il soit, celui qu’on propose à vos soins, dans la disposition morale convenable, n’est pas à la portée du premier venu, mais si cette tâche revient à quelqu’un, c’est assurément à l’homme possédant la connaissance scientifique, comme cela a lieu pour la médecine, et les autres arts qui font appel à quelque sollicitude d’autrui et à la prudence.\par
Ne doit-on pas alors après cela examiner à quelle source et de quelle façon nous pouvons acquérir la science de la législation ? Ne serait-ce pas, comme dans le cas des autres arts, en \\
s’adressant aux hommes adonnés à la politique 2\\
active ? Notre opinion était, en effet, que la science législative est une partie de la politique. Mais n’est-il pas manifeste qu’il n’existe pas de ressemblance entre la politique et les autres sciences et potentialités ? En effet, dans les autres sciences on constate que les mêmes personnes, à la fois transmettent à leurs élèves leurs potentialités et exercent leur propre activité en s’appuyant \\
sur celles-ci, par exemple les médecins et les peintres ; au contraire, les réalités de la politique, que les Sophistes font  profession d’enseigner, ne sont pratiquées par aucun d’eux, mais bien par ceux qui gouvernent la cité, et dont l’action, croirait-on, repose sur une sorte d’habileté tout empirique plutôt que sur la pensée abstraite : car on ne les voit jamais écrire ou discourir sur de telles matières (ce qui serait pourtant une tâche peut-être plus honorable encore que de prononcer \\
des discours devant les tribunaux ou devant l’assemblée du peuple), pas plus que, d’autre part, nous ne les voyons avoir jamais fait des hommes d’État de leurs propres enfants ou de certains de leurs amis. Ce serait pourtant bien naturel, s’ils en avaient le pouvoir, car ils n’auraient pu laisser à leurs cités un héritage préférable à celui-là, ni souhaiter posséder pour eux-mêmes, et par suite pour les êtres qui leur sont le plus chers, rien qui soit supérieur à cette habileté politique.\par
Il n’en est pas moins vrai que l’expérience semble en \\
pareille matière apporter une contribution qui n’est pas négligeable : sans elle, en effet, jamais personne ne pourrait devenir homme d’État en se familiarisant simplement avec les réalités de la politique. C’est pourquoi, ceux qui désirent acquérir la science de la politique sont dans l’obligation, semble-t-il, d’y ajouter la pratique des affaires.\par
Quant à ceux des Sophistes qui se vantent d’enseigner la Politique, ils sont manifestement fort loin du compte. D’une façon générale, en effet, ils ne savent ni quelle est sa nature, ni quel est son objet : sans cela, ils ne l’auraient pas confondue \\
avec la Rhétorique, ou même placée à un rang inférieur à cette dernière ; ils n’auraient pas non plus pensé que légiférer est une chose facile, consistant seulement à collectionner celles des lois qui reçoivent l’approbation de l’opinion publique. Car ils disent qu’il est possible de sélectionner les meilleures lois comme si cette sélection n’était pas elle-même œuvre d’intelligence, et comme si ce discernement fait correctement n’était pas ce qu’il y a de plus important ! C’est tout à fait comme ce qui se passe dans l’art musical. Ceux qui, en effet, ont acquis l’expérience dans un art quel qu’il soit, jugent correctement \\
les productions de cet art, comprenant par quels moyens et de quelle façon la perfection de l’œuvre est atteinte, et savent quels sont les éléments de l’œuvre qui par leur nature s’harmonisent entre eux ; au contraire, les gens à qui l’expérience fait défaut doivent s’estimer satisfaits de pouvoir tout juste distinguer si l’œuvre produite est bonne ou mauvaise, comme cela a lieu pour la peinture. Or les lois ne sont que des produits  en quelque sorte de l’art politique : comment, dans ces conditions, pourrait-on apprendre d’elles à devenir législateur, ou à discerner les meilleures d’entre elles ? Car on ne voit jamais personne devenir médecin par la simple étude des recueils d’ordonnances. Pourtant les écrivains médicaux essayent bien d’indiquer non seulement les traitements, mais encore les méthodes de cure et la façon dont on doit soigner chaque catégorie \\
de malades, distinguant à cet effet les différentes dispositions du corps. Mais ces indications ne paraissent utiles qu’à ceux qui possèdent l’expérience, et perdent toute valeur entre les mains de ceux qui en sont dépourvus. Il peut donc se faire également que les recueils de lois ou de constitutions rendent des services à ceux qui sont capables de les méditer et de discerner ce qu’il y a de bon ou de mauvais, et quelles sortes de dispositions légales doivent répondre à une situation donnée. \\
Quant à ceux qui se plongent dans des collections de ce genre sans avoir la disposition requise, ils ne sauraient porter un jugement qualifié, à moins que ce ne soit instinctivement, quoique leur perspicacité en ces matières soit peut-être susceptible d’en recevoir un surcroît de développement.\par
Nos devanciers ayant laissé inexploré ce qui concerne la science de la législation, il est sans doute préférable que nous procédions à cet examen, et en étudiant le problème de la constitution en général, de façon à parachever dans la mesure \\
du possible notre philosophie des choses humaines.\par
Ainsi donc, en premier lieu, si quelque indication partielle intéressante a été fournie par les penseurs qui nous ont précédé, nous nous efforcerons de la reprendre à notre tour ; ensuite, à la lumière des constitutions que nous avons rassemblées, nous considérerons à quelles sortes de causes sont dues la conservation ou la ruine des cités ainsi que la conservation ou la ruine des formes particulières de constitutions, et pour quelles raisons certaines cités sont bien gouvernées et d’autres tout le contraire.\par
\\
Après avoir étudié ces différents points, nous pourrons peut-être, dans une vue d’ensemble, mieux discerner quelle est la meilleure des constitutions, quel rang réserver à chaque type, et de quelles lois et de quelles coutumes chacun doit faire usage. Commençons donc notre exposé.
 


% at least one empty page at end (for booklet couv)
\ifbooklet
  \pagestyle{empty}
  \clearpage
  % 2 empty pages maybe needed for 4e cover
  \ifnum\modulo{\value{page}}{4}=0 \hbox{}\newpage\hbox{}\newpage\fi
  \ifnum\modulo{\value{page}}{4}=1 \hbox{}\newpage\hbox{}\newpage\fi


  \hbox{}\newpage
  \ifodd\value{page}\hbox{}\newpage\fi
  {\centering\color{rubric}\bfseries\noindent\large
    Hurlus ? Qu’est-ce.\par
    \bigskip
  }
  \noindent Des bouquinistes électroniques, pour du texte libre à participation libre,
  téléchargeable gratuitement sur \href{https://hurlus.fr}{\dotuline{hurlus.fr}}.\par
  \bigskip
  \noindent Cette brochure a été produite par des éditeurs bénévoles.
  Elle n’est pas faîte pour être possédée, mais pour être lue, et puis donnée.
  Que circule le texte !
  En page de garde, on peut ajouter une date, un lieu, un nom ; pour suivre le voyage des idées.
  \par

  Ce texte a été choisi parce qu’une personne l’a aimé,
  ou haï, elle a en tous cas pensé qu’il partipait à la formation de notre présent ;
  sans le souci de plaire, vendre, ou militer pour une cause.
  \par

  L’édition électronique est soigneuse, tant sur la technique
  que sur l’établissement du texte ; mais sans aucune prétention scolaire, au contraire.
  Le but est de s’adresser à tous, sans distinction de science ou de diplôme.
  Au plus direct ! (possible)
  \par

  Cet exemplaire en papier a été tiré sur une imprimante personnelle
   ou une photocopieuse. Tout le monde peut le faire.
  Il suffit de
  télécharger un fichier sur \href{https://hurlus.fr}{\dotuline{hurlus.fr}},
  d’imprimer, et agrafer ; puis de lire et donner.\par

  \bigskip

  \noindent PS : Les hurlus furent aussi des rebelles protestants qui cassaient les statues dans les églises catholiques. En 1566 démarra la révolte des gueux dans le pays de Lille. L’insurrection enflamma la région jusqu’à Anvers où les gueux de mer bloquèrent les bateaux espagnols.
  Ce fut une rare guerre de libération dont naquit un pays toujours libre : les Pays-Bas.
  En plat pays francophone, par contre, restèrent des bandes de huguenots, les hurlus, progressivement réprimés par la très catholique Espagne.
  Cette mémoire d’une défaite est éteinte, rallumons-la. Sortons les livres du culte universitaire, cherchons les idoles de l’époque, pour les briser.
\fi

\ifdev % autotext in dev mode
\fontname\font — \textsc{Les règles du jeu}\par
(\hyperref[utopie]{\underline{Lien}})\par
\noindent \initialiv{A}{lors là}\blindtext\par
\noindent \initialiv{À}{ la bonheur des dames}\blindtext\par
\noindent \initialiv{É}{tonnez-le}\blindtext\par
\noindent \initialiv{Q}{ualitativement}\blindtext\par
\noindent \initialiv{V}{aloriser}\blindtext\par
\Blindtext
\phantomsection
\label{utopie}
\Blinddocument
\fi
\end{document}
