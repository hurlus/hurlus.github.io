%%%%%%%%%%%%%%%%%%%%%%%%%%%%%%%%%
% LaTeX model https://hurlus.fr %
%%%%%%%%%%%%%%%%%%%%%%%%%%%%%%%%%

% Needed before document class
\RequirePackage{pdftexcmds} % needed for tests expressions
\RequirePackage{fix-cm} % correct units

% Define mode
\def\mode{a4}

\newif\ifaiv % a4
\newif\ifav % a5
\newif\ifbooklet % booklet
\newif\ifcover % cover for booklet

\ifnum \strcmp{\mode}{cover}=0
  \covertrue
\else\ifnum \strcmp{\mode}{booklet}=0
  \booklettrue
\else\ifnum \strcmp{\mode}{a5}=0
  \avtrue
\else
  \aivtrue
\fi\fi\fi

\ifbooklet % do not enclose with {}
  \documentclass[french,twoside]{book} % ,notitlepage
  \usepackage[%
    papersize={105mm, 297mm},
    inner=12mm,
    outer=12mm,
    top=20mm,
    bottom=15mm,
    marginparsep=0pt,
  ]{geometry}
  \usepackage[fontsize=9.5pt]{scrextend} % for Roboto
\else\ifav
  \documentclass[french,twoside]{book} % ,notitlepage
  \usepackage[%
    a5paper,
    inner=25mm,
    outer=15mm,
    top=15mm,
    bottom=15mm,
    marginparsep=0pt,
  ]{geometry}
  \usepackage[fontsize=12pt]{scrextend}
\else% A4 2 cols
  \documentclass[twocolumn]{report}
  \usepackage[%
    a4paper,
    inner=15mm,
    outer=10mm,
    top=25mm,
    bottom=18mm,
    marginparsep=0pt,
  ]{geometry}
  \setlength{\columnsep}{20mm}
  \usepackage[fontsize=9.5pt]{scrextend}
\fi\fi

%%%%%%%%%%%%%%
% Alignments %
%%%%%%%%%%%%%%
% before teinte macros

\setlength{\arrayrulewidth}{0.2pt}
\setlength{\columnseprule}{\arrayrulewidth} % twocol
\setlength{\parskip}{0pt} % classical para with no margin
\setlength{\parindent}{1.5em}

%%%%%%%%%%
% Colors %
%%%%%%%%%%
% before Teinte macros

\usepackage[dvipsnames]{xcolor}
\definecolor{rubric}{HTML}{800000} % the tonic 0c71c3
\def\columnseprulecolor{\color{rubric}}
\colorlet{borderline}{rubric!30!} % definecolor need exact code
\definecolor{shadecolor}{gray}{0.95}
\definecolor{bghi}{gray}{0.5}

%%%%%%%%%%%%%%%%%
% Teinte macros %
%%%%%%%%%%%%%%%%%
%%%%%%%%%%%%%%%%%%%%%%%%%%%%%%%%%%%%%%%%%%%%%%%%%%%
% <TEI> generic (LaTeX names generated by Teinte) %
%%%%%%%%%%%%%%%%%%%%%%%%%%%%%%%%%%%%%%%%%%%%%%%%%%%
% This template is inserted in a specific design
% It is XeLaTeX and otf fonts

\makeatletter % <@@@


\usepackage{blindtext} % generate text for testing
\usepackage[strict]{changepage} % for modulo 4
\usepackage{contour} % rounding words
\usepackage[nodayofweek]{datetime}
% \usepackage{DejaVuSans} % seems buggy for sffont font for symbols
\usepackage{enumitem} % <list>
\usepackage{etoolbox} % patch commands
\usepackage{fancyvrb}
\usepackage{fancyhdr}
\usepackage{float}
\usepackage{fontspec} % XeLaTeX mandatory for fonts
\usepackage{footnote} % used to capture notes in minipage (ex: quote)
\usepackage{framed} % bordering correct with footnote hack
\usepackage{graphicx}
\usepackage{lettrine} % drop caps
\usepackage{lipsum} % generate text for testing
\usepackage[framemethod=tikz,]{mdframed} % maybe used for frame with footnotes inside
\usepackage{pdftexcmds} % needed for tests expressions
\usepackage{polyglossia} % non-break space french punct, bug Warning: "Failed to patch part"
\usepackage[%
  indentfirst=false,
  vskip=1em,
  noorphanfirst=true,
  noorphanafter=true,
  leftmargin=\parindent,
  rightmargin=0pt,
]{quoting}
\usepackage{ragged2e}
\usepackage{setspace} % \setstretch for <quote>
\usepackage{tabularx} % <table>
\usepackage[explicit]{titlesec} % wear titles, !NO implicit
\usepackage{tikz} % ornaments
\usepackage{tocloft} % styling tocs
\usepackage[fit]{truncate} % used im runing titles
\usepackage{unicode-math}
\usepackage[normalem]{ulem} % breakable \uline, normalem is absolutely necessary to keep \emph
\usepackage{verse} % <l>
\usepackage{xcolor} % named colors
\usepackage{xparse} % @ifundefined
\XeTeXdefaultencoding "iso-8859-1" % bad encoding of xstring
\usepackage{xstring} % string tests
\XeTeXdefaultencoding "utf-8"
\PassOptionsToPackage{hyphens}{url} % before hyperref, which load url package

% TOTEST
% \usepackage{hypcap} % links in caption ?
% \usepackage{marginnote}
% TESTED
% \usepackage{background} % doesn’t work with xetek
% \usepackage{bookmark} % prefers the hyperref hack \phantomsection
% \usepackage[color, leftbars]{changebar} % 2 cols doc, impossible to keep bar left
% \usepackage[utf8x]{inputenc} % inputenc package ignored with utf8 based engines
% \usepackage[sfdefault,medium]{inter} % no small caps
% \usepackage{firamath} % choose firasans instead, firamath unavailable in Ubuntu 21-04
% \usepackage{flushend} % bad for last notes, supposed flush end of columns
% \usepackage[stable]{footmisc} % BAD for complex notes https://texfaq.org/FAQ-ftnsect
% \usepackage{helvet} % not for XeLaTeX
% \usepackage{multicol} % not compatible with too much packages (longtable, framed, memoir…)
% \usepackage[default,oldstyle,scale=0.95]{opensans} % no small caps
% \usepackage{sectsty} % \chapterfont OBSOLETE
% \usepackage{soul} % \ul for underline, OBSOLETE with XeTeX
% \usepackage[breakable]{tcolorbox} % text styling gone, footnote hack not kept with breakable


% Metadata inserted by a program, from the TEI source, for title page and runing heads
\title{\textbf{ Les Annales (traduction Burnouf, 1859) }}
\date{110}
\author{Tacite}
\def\elbibl{Tacite. 110. \emph{Les Annales (traduction Burnouf, 1859)}}
\def\elsource{\href{http://remacle.org/bloodwolf/historiens/tacite/table.htm}{\dotuline{Remacle}}\footnote{\href{http://remacle.org/bloodwolf/historiens/tacite/table.htm}{\url{http://remacle.org/bloodwolf/historiens/tacite/table.htm}}}}

% Default metas
\newcommand{\colorprovide}[2]{\@ifundefinedcolor{#1}{\colorlet{#1}{#2}}{}}
\colorprovide{rubric}{red}
\colorprovide{silver}{lightgray}
\@ifundefined{syms}{\newfontfamily\syms{DejaVu Sans}}{}
\newif\ifdev
\@ifundefined{elbibl}{% No meta defined, maybe dev mode
  \newcommand{\elbibl}{Titre court ?}
  \newcommand{\elbook}{Titre du livre source ?}
  \newcommand{\elabstract}{Résumé\par}
  \newcommand{\elurl}{http://oeuvres.github.io/elbook/2}
  \author{Éric Lœchien}
  \title{Un titre de test assez long pour vérifier le comportement d’une maquette}
  \date{1566}
  \devtrue
}{}
\let\eltitle\@title
\let\elauthor\@author
\let\eldate\@date


\defaultfontfeatures{
  % Mapping=tex-text, % no effect seen
  Scale=MatchLowercase,
  Ligatures={TeX,Common},
}


% generic typo commands
\newcommand{\astermono}{\medskip\centerline{\color{rubric}\large\selectfont{\syms ✻}}\medskip\par}%
\newcommand{\astertri}{\medskip\par\centerline{\color{rubric}\large\selectfont{\syms ✻\,✻\,✻}}\medskip\par}%
\newcommand{\asterism}{\bigskip\par\noindent\parbox{\linewidth}{\centering\color{rubric}\large{\syms ✻}\\{\syms ✻}\hskip 0.75em{\syms ✻}}\bigskip\par}%

% lists
\newlength{\listmod}
\setlength{\listmod}{\parindent}
\setlist{
  itemindent=!,
  listparindent=\listmod,
  labelsep=0.2\listmod,
  parsep=0pt,
  % topsep=0.2em, % default topsep is best
}
\setlist[itemize]{
  label=—,
  leftmargin=0pt,
  labelindent=1.2em,
  labelwidth=0pt,
}
\setlist[enumerate]{
  label={\bf\color{rubric}\arabic*.},
  labelindent=0.8\listmod,
  leftmargin=\listmod,
  labelwidth=0pt,
}
\newlist{listalpha}{enumerate}{1}
\setlist[listalpha]{
  label={\bf\color{rubric}\alph*.},
  leftmargin=0pt,
  labelindent=0.8\listmod,
  labelwidth=0pt,
}
\newcommand{\listhead}[1]{\hspace{-1\listmod}\emph{#1}}

\renewcommand{\hrulefill}{%
  \leavevmode\leaders\hrule height 0.2pt\hfill\kern\z@}

% General typo
\DeclareTextFontCommand{\textlarge}{\large}
\DeclareTextFontCommand{\textsmall}{\small}

% commands, inlines
\newcommand{\anchor}[1]{\Hy@raisedlink{\hypertarget{#1}{}}} % link to top of an anchor (not baseline)
\newcommand\abbr[1]{#1}
\newcommand{\autour}[1]{\tikz[baseline=(X.base)]\node [draw=rubric,thin,rectangle,inner sep=1.5pt, rounded corners=3pt] (X) {\color{rubric}#1};}
\newcommand\corr[1]{#1}
\newcommand{\ed}[1]{ {\color{silver}\sffamily\footnotesize (#1)} } % <milestone ed="1688"/>
\newcommand\expan[1]{#1}
\newcommand\foreign[1]{\emph{#1}}
\newcommand\gap[1]{#1}
\renewcommand{\LettrineFontHook}{\color{rubric}}
\newcommand{\initial}[2]{\lettrine[lines=2, loversize=0.3, lhang=0.3]{#1}{#2}}
\newcommand{\initialiv}[2]{%
  \let\oldLFH\LettrineFontHook
  % \renewcommand{\LettrineFontHook}{\color{rubric}\ttfamily}
  \IfSubStr{QJ’}{#1}{
    \lettrine[lines=4, lhang=0.2, loversize=-0.1, lraise=0.2]{\smash{#1}}{#2}
  }{\IfSubStr{É}{#1}{
    \lettrine[lines=4, lhang=0.2, loversize=-0, lraise=0]{\smash{#1}}{#2}
  }{\IfSubStr{ÀÂ}{#1}{
    \lettrine[lines=4, lhang=0.2, loversize=-0, lraise=0, slope=0.6em]{\smash{#1}}{#2}
  }{\IfSubStr{A}{#1}{
    \lettrine[lines=4, lhang=0.2, loversize=0.2, slope=0.6em]{\smash{#1}}{#2}
  }{\IfSubStr{V}{#1}{
    \lettrine[lines=4, lhang=0.2, loversize=0.2, slope=-0.5em]{\smash{#1}}{#2}
  }{
    \lettrine[lines=4, lhang=0.2, loversize=0.2]{\smash{#1}}{#2}
  }}}}}
  \let\LettrineFontHook\oldLFH
}
\newcommand{\labelchar}[1]{\textbf{\color{rubric} #1}}
\newcommand{\milestone}[1]{\autour{\footnotesize\color{rubric} #1}} % <milestone n="4"/>
\newcommand\name[1]{#1}
\newcommand\orig[1]{#1}
\newcommand\orgName[1]{#1}
\newcommand\persName[1]{#1}
\newcommand\placeName[1]{#1}
\newcommand{\pn}[1]{\IfSubStr{-—–¶}{#1}% <p n="3"/>
  {\noindent{\bfseries\color{rubric}   ¶  }}
  {{\footnotesize\autour{ #1}  }}}
\newcommand\reg{}
% \newcommand\ref{} % already defined
\newcommand\sic[1]{#1}
\newcommand\surname[1]{\textsc{#1}}
\newcommand\term[1]{\textbf{#1}}

\def\mednobreak{\ifdim\lastskip<\medskipamount
  \removelastskip\nopagebreak\medskip\fi}
\def\bignobreak{\ifdim\lastskip<\bigskipamount
  \removelastskip\nopagebreak\bigskip\fi}

% commands, blocks
\newcommand{\byline}[1]{\bigskip{\RaggedLeft{#1}\par}\bigskip}
\newcommand{\bibl}[1]{{\RaggedLeft{#1}\par\bigskip}}
\newcommand{\biblitem}[1]{{\noindent\hangindent=\parindent   #1\par}}
\newcommand{\dateline}[1]{\medskip{\RaggedLeft{#1}\par}\bigskip}
\newcommand{\labelblock}[1]{\medbreak{\noindent\color{rubric}\bfseries #1}\par\mednobreak}
\newcommand{\salute}[1]{\bigbreak{#1}\par\medbreak}
\newcommand{\signed}[1]{\bigbreak\filbreak{\raggedleft #1\par}\medskip}

% environments for blocks (some may become commands)
\newenvironment{borderbox}{}{} % framing content
\newenvironment{citbibl}{\ifvmode\hfill\fi}{\ifvmode\par\fi }
\newenvironment{docAuthor}{\ifvmode\vskip4pt\fontsize{16pt}{18pt}\selectfont\fi\itshape}{\ifvmode\par\fi }
\newenvironment{docDate}{}{\ifvmode\par\fi }
\newenvironment{docImprint}{\vskip6pt}{\ifvmode\par\fi }
\newenvironment{docTitle}{\vskip6pt\bfseries\fontsize{18pt}{22pt}\selectfont}{\par }
\newenvironment{msHead}{\vskip6pt}{\par}
\newenvironment{msItem}{\vskip6pt}{\par}
\newenvironment{titlePart}{}{\par }


% environments for block containers
\newenvironment{argument}{\itshape\parindent0pt}{\vskip1.5em}
\newenvironment{biblfree}{}{\ifvmode\par\fi }
\newenvironment{bibitemlist}[1]{%
  \list{\@biblabel{\@arabic\c@enumiv}}%
  {%
    \settowidth\labelwidth{\@biblabel{#1}}%
    \leftmargin\labelwidth
    \advance\leftmargin\labelsep
    \@openbib@code
    \usecounter{enumiv}%
    \let\p@enumiv\@empty
    \renewcommand\theenumiv{\@arabic\c@enumiv}%
  }
  \sloppy
  \clubpenalty4000
  \@clubpenalty \clubpenalty
  \widowpenalty4000%
  \sfcode`\.\@m
}%
{\def\@noitemerr
  {\@latex@warning{Empty `bibitemlist' environment}}%
\endlist}
\newenvironment{quoteblock}% may be used for ornaments
  {\begin{quoting}}
  {\end{quoting}}

% table () is preceded and finished by custom command
\newcommand{\tableopen}[1]{%
  \ifnum\strcmp{#1}{wide}=0{%
    \begin{center}
  }
  \else\ifnum\strcmp{#1}{long}=0{%
    \begin{center}
  }
  \else{%
    \begin{center}
  }
  \fi\fi
}
\newcommand{\tableclose}[1]{%
  \ifnum\strcmp{#1}{wide}=0{%
    \end{center}
  }
  \else\ifnum\strcmp{#1}{long}=0{%
    \end{center}
  }
  \else{%
    \end{center}
  }
  \fi\fi
}


% text structure
\newcommand\chapteropen{} % before chapter title
\newcommand\chaptercont{} % after title, argument, epigraph…
\newcommand\chapterclose{} % maybe useful for multicol settings
\setcounter{secnumdepth}{-2} % no counters for hierarchy titles
\setcounter{tocdepth}{5} % deep toc
\markright{\@title} % ???
\markboth{\@title}{\@author} % ???
\renewcommand\tableofcontents{\@starttoc{toc}}
% toclof format
% \renewcommand{\@tocrmarg}{0.1em} % Useless command?
% \renewcommand{\@pnumwidth}{0.5em} % {1.75em}
\renewcommand{\@cftmaketoctitle}{}
\setlength{\cftbeforesecskip}{\z@ \@plus.2\p@}
\renewcommand{\cftchapfont}{}
\renewcommand{\cftchapdotsep}{\cftdotsep}
\renewcommand{\cftchapleader}{\normalfont\cftdotfill{\cftchapdotsep}}
\renewcommand{\cftchappagefont}{\bfseries}
\setlength{\cftbeforechapskip}{0em \@plus\p@}
% \renewcommand{\cftsecfont}{\small\relax}
\renewcommand{\cftsecpagefont}{\normalfont}
% \renewcommand{\cftsubsecfont}{\small\relax}
\renewcommand{\cftsecdotsep}{\cftdotsep}
\renewcommand{\cftsecpagefont}{\normalfont}
\renewcommand{\cftsecleader}{\normalfont\cftdotfill{\cftsecdotsep}}
\setlength{\cftsecindent}{1em}
\setlength{\cftsubsecindent}{2em}
\setlength{\cftsubsubsecindent}{3em}
\setlength{\cftchapnumwidth}{1em}
\setlength{\cftsecnumwidth}{1em}
\setlength{\cftsubsecnumwidth}{1em}
\setlength{\cftsubsubsecnumwidth}{1em}

% footnotes
\newif\ifheading
\newcommand*{\fnmarkscale}{\ifheading 0.70 \else 1 \fi}
\renewcommand\footnoterule{\vspace*{0.3cm}\hrule height \arrayrulewidth width 3cm \vspace*{0.3cm}}
\setlength\footnotesep{1.5\footnotesep} % footnote separator
\renewcommand\@makefntext[1]{\parindent 1.5em \noindent \hb@xt@1.8em{\hss{\normalfont\@thefnmark . }}#1} % no superscipt in foot
\patchcmd{\@footnotetext}{\footnotesize}{\footnotesize\sffamily}{}{} % before scrextend, hyperref


%   see https://tex.stackexchange.com/a/34449/5049
\def\truncdiv#1#2{((#1-(#2-1)/2)/#2)}
\def\moduloop#1#2{(#1-\truncdiv{#1}{#2}*#2)}
\def\modulo#1#2{\number\numexpr\moduloop{#1}{#2}\relax}

% orphans and widows
\clubpenalty=9996
\widowpenalty=9999
\brokenpenalty=4991
\predisplaypenalty=10000
\postdisplaypenalty=1549
\displaywidowpenalty=1602
\hyphenpenalty=400
% Copied from Rahtz but not understood
\def\@pnumwidth{1.55em}
\def\@tocrmarg {2.55em}
\def\@dotsep{4.5}
\emergencystretch 3em
\hbadness=4000
\pretolerance=750
\tolerance=2000
\vbadness=4000
\def\Gin@extensions{.pdf,.png,.jpg,.mps,.tif}
% \renewcommand{\@cite}[1]{#1} % biblio

\usepackage{hyperref} % supposed to be the last one, :o) except for the ones to follow
\urlstyle{same} % after hyperref
\hypersetup{
  % pdftex, % no effect
  pdftitle={\elbibl},
  % pdfauthor={Your name here},
  % pdfsubject={Your subject here},
  % pdfkeywords={keyword1, keyword2},
  bookmarksnumbered=true,
  bookmarksopen=true,
  bookmarksopenlevel=1,
  pdfstartview=Fit,
  breaklinks=true, % avoid long links
  pdfpagemode=UseOutlines,    % pdf toc
  hyperfootnotes=true,
  colorlinks=false,
  pdfborder=0 0 0,
  % pdfpagelayout=TwoPageRight,
  % linktocpage=true, % NO, toc, link only on page no
}

\makeatother % /@@@>
%%%%%%%%%%%%%%
% </TEI> end %
%%%%%%%%%%%%%%


%%%%%%%%%%%%%
% footnotes %
%%%%%%%%%%%%%
\renewcommand{\thefootnote}{\bfseries\textcolor{rubric}{\arabic{footnote}}} % color for footnote marks

%%%%%%%%%
% Fonts %
%%%%%%%%%
\usepackage[]{roboto} % SmallCaps, Regular is a bit bold
% \linespread{0.90} % too compact, keep font natural
\newfontfamily\fontrun[]{Roboto Condensed Light} % condensed runing heads
\ifav
  \setmainfont[
    ItalicFont={Roboto Light Italic},
  ]{Roboto}
\else\ifbooklet
  \setmainfont[
    ItalicFont={Roboto Light Italic},
  ]{Roboto}
\else
\setmainfont[
  ItalicFont={Roboto Italic},
]{Roboto Light}
\fi\fi
\renewcommand{\LettrineFontHook}{\bfseries\color{rubric}}
% \renewenvironment{labelblock}{\begin{center}\bfseries\color{rubric}}{\end{center}}

%%%%%%%%
% MISC %
%%%%%%%%

\setdefaultlanguage[frenchpart=false]{french} % bug on part


\newenvironment{quotebar}{%
    \def\FrameCommand{{\color{rubric!10!}\vrule width 0.5em} \hspace{0.9em}}%
    \def\OuterFrameSep{\itemsep} % séparateur vertical
    \MakeFramed {\advance\hsize-\width \FrameRestore}
  }%
  {%
    \endMakeFramed
  }
\renewenvironment{quoteblock}% may be used for ornaments
  {%
    \savenotes
    \setstretch{0.9}
    \normalfont
    \begin{quotebar}
  }
  {%
    \end{quotebar}
    \spewnotes
  }


\renewcommand{\headrulewidth}{\arrayrulewidth}
\renewcommand{\headrule}{{\color{rubric}\hrule}}

% delicate tuning, image has produce line-height problems in title on 2 lines
\titleformat{name=\chapter} % command
  [display] % shape
  {\vspace{1.5em}\centering} % format
  {} % label
  {0pt} % separator between n
  {}
[{\color{rubric}\huge\textbf{#1}}\bigskip] % after code
% \titlespacing{command}{left spacing}{before spacing}{after spacing}[right]
\titlespacing*{\chapter}{0pt}{-2em}{0pt}[0pt]

\titleformat{name=\section}
  [block]{}{}{}{}
  [\vbox{\color{rubric}\large\raggedleft\textbf{#1}}]
\titlespacing{\section}{0pt}{0pt plus 4pt minus 2pt}{\baselineskip}

\titleformat{name=\subsection}
  [block]
  {}
  {} % \thesection
  {} % separator \arrayrulewidth
  {}
[\vbox{\large\textbf{#1}}]
% \titlespacing{\subsection}{0pt}{0pt plus 4pt minus 2pt}{\baselineskip}

\ifaiv
  \fancypagestyle{main}{%
    \fancyhf{}
    \setlength{\headheight}{1.5em}
    \fancyhead{} % reset head
    \fancyfoot{} % reset foot
    \fancyhead[L]{\truncate{0.45\headwidth}{\fontrun\elbibl}} % book ref
    \fancyhead[R]{\truncate{0.45\headwidth}{ \fontrun\nouppercase\leftmark}} % Chapter title
    \fancyhead[C]{\thepage}
  }
  \fancypagestyle{plain}{% apply to chapter
    \fancyhf{}% clear all header and footer fields
    \setlength{\headheight}{1.5em}
    \fancyhead[L]{\truncate{0.9\headwidth}{\fontrun\elbibl}}
    \fancyhead[R]{\thepage}
  }
\else
  \fancypagestyle{main}{%
    \fancyhf{}
    \setlength{\headheight}{1.5em}
    \fancyhead{} % reset head
    \fancyfoot{} % reset foot
    \fancyhead[RE]{\truncate{0.9\headwidth}{\fontrun\elbibl}} % book ref
    \fancyhead[LO]{\truncate{0.9\headwidth}{\fontrun\nouppercase\leftmark}} % Chapter title, \nouppercase needed
    \fancyhead[RO,LE]{\thepage}
  }
  \fancypagestyle{plain}{% apply to chapter
    \fancyhf{}% clear all header and footer fields
    \setlength{\headheight}{1.5em}
    \fancyhead[L]{\truncate{0.9\headwidth}{\fontrun\elbibl}}
    \fancyhead[R]{\thepage}
  }
\fi

\ifav % a5 only
  \titleclass{\section}{top}
\fi

\newcommand\chapo{{%
  \vspace*{-3em}
  \centering % no vskip ()
  {\Large\addfontfeature{LetterSpace=25}\bfseries{\elauthor}}\par
  \smallskip
  {\large\eldate}\par
  \bigskip
  {\Large\selectfont{\eltitle}}\par
  \bigskip
  {\color{rubric}\hline\par}
  \bigskip
  {\Large TEXTE LIBRE À PARTICPATION LIBRE\par}
  \centerline{\small\color{rubric} {hurlus.fr, tiré le \today}}\par
  \bigskip
}}

\newcommand\cover{{%
  \thispagestyle{empty}
  \centering
  {\LARGE\bfseries{\elauthor}}\par
  \bigskip
  {\Large\eldate}\par
  \bigskip
  \bigskip
  {\LARGE\selectfont{\eltitle}}\par
  \vfill\null
  {\color{rubric}\setlength{\arrayrulewidth}{2pt}\hline\par}
  \vfill\null
  {\Large TEXTE LIBRE À PARTICPATION LIBRE\par}
  \centerline{{\href{https://hurlus.fr}{\dotuline{hurlus.fr}}, tiré le \today}}\par
}}

\begin{document}
\pagestyle{empty}
\ifbooklet{
  \cover\newpage
  \thispagestyle{empty}\hbox{}\newpage
  \cover\newpage\noindent Les voyages de la brochure\par
  \bigskip
  \begin{tabularx}{\textwidth}{l|X|X}
    \textbf{Date} & \textbf{Lieu}& \textbf{Nom/pseudo} \\ \hline
    \rule{0pt}{25cm} &  &   \\
  \end{tabularx}
  \newpage
  \addtocounter{page}{-4}
}\fi

\thispagestyle{empty}
\ifaiv
  \twocolumn[\chapo]
\else
  \chapo
\fi
{\it\elabstract}
\bigskip
\makeatletter\@starttoc{toc}\makeatother % toc without new page
\bigskip

\pagestyle{main} % after style

  \section[{Livre premier (14, 15)}]{Livre premier (14, 15)}\renewcommand{\leftmark}{Livre premier (14, 15)}

\subsection[{Introduction – Rappel historique – Le sujet : Tibère, Caligula, Claude, Néron.}]{Introduction – Rappel historique – Le sujet : Tibère, Caligula, Claude, Néron.}
\noindent \labelchar{I.} Rome fut d’abord soumise à des rois. L. Brutus fonda la liberté et le consulat. Les dictatures étaient passagères ; le pouvoir décemviral ne dura pas au-delà de deux années, et les tribuns militaires se maintinrent peu de temps à la place des consuls. La domination de Cinna, celle de Sylla, ne furent pas longues, et la puissance de Pompée et de Crassus passa bientôt dans les mains de César, les armes de Lépide et d’Antoine dans celles d’Auguste, qui reçut sous son obéissance le monde fatigué de discordes, et resta maître sons le nom de prince \footnote{Le titre de \emph{prince} ne conférait aucune autorité ni civile ni militaire. Du temps de la République, il se donnait au citoyen que les censeurs avaient inscrit le premier sur le tableau des sénateurs, et qui pour cela était appelé \emph{princeps senatus}. Quand Auguste eut réuni dans ses mains les pouvoirs de toutes les magistratures, il préféra ce nom de prince à tout autre, comme moins propre à exciter l’envie.}. Les prospérités et les revers de l’ancienne république ont eu d’illustres historiens ; et les temps même d’Auguste n’en ont pas manqué, jusqu’au moment où les progrès de l’adulation gâtèrent les plus beaux génies. L’histoire de Tibère, de Caius, de Claude et de Néron, falsifiée par la crainte aux jours de leur grandeur, fut écrite, après leur mort, sous l’influence de haines trop récentes. Je dirai donc peu de mots d’Auguste, et de sa fin seulement. Ensuite je raconterai le règne de Tibère et les trois suivants, sans colère comme sans faveur, sentiments dont les motifs sont loin de moi.
\subsection[{Auguste – Son arrivée au pouvoir}]{Auguste – Son arrivée au pouvoir}
\noindent \labelchar{II.} Lorsque, après la défaite de Brutus et de Cassius, la cause publique fut désarmée, que Pompée \footnote{Sextus Pompée} eut succombé en Sicile, que l’abaissement de Lépide et la mort violente d’Antoine n’eurent laissé au parti même de César d’autre chef qu’Auguste, celui-ci abdiqua le nom de \emph{triumvir}, s’annonçant comme simple consul, et content, disait-il, pour protéger le peuple, de la puissance tribunitienne. Quand il eut gagné les soldats par des largesses, la multitude par l’abondance des vivres, tous par les douceurs du repos, on le vit s’élever insensiblement et attirer à lui l’autorité du sénat, des magistrats, des lois. Nul ne lui résistait : les plus fiers républicains avaient péri par la guerre ou la proscription ; ce qui restait de nobles trouvaient, dans leur empressement à servir, honneurs et opulence, et, comme ils avaient gagné au changement des affaires, ils aimaient mieux le présent et sa sécurité que le passé avec ses périls. Le nouvel ordre des choses ne déplaisait pas non plus aux provinces, qui avaient en défiance le gouvernement du Sénat et du peuple, à cause des querelles des grands et de l’avarice des magistrats, et qui attendaient peu de secours des lois, impuissantes contre la force, la brigue et l’argent.
\subsection[{Problèmes avec ses héritiers}]{Problèmes avec ses héritiers}
\noindent \labelchar{III.} Auguste, pour donner des appuis à sa domination, éleva aux dignités d’édile curule et de pontife Claudius Marcellus \footnote{C’est ce jeune Marcellus, tant célébré dans les beaux vers de Virgile, Énéide, VI, 860 et suiv. Il était fils d’Octavie, et il avait épousé Julie, fille d’Auguste. Il mourut à vingt ans, l’an de Rome 731.}, fils de sa sœur, à peine entré dans l’adolescence, et honora de deux consulats consécutifs M. Agrippa, d’une naissance obscure, mais grand homme de guerre et compagnon de sa victoire ; il le prit même pour gendre \footnote{Agrippa eut de la fille d’Auguste Agrippine, femme de Germanicus, la seconde Julie, les Césars Gaius et Lucius, et enfin Postumus, qui naquit après la mort de son père. D’une première femme, Attica, fille de Pomponius Atticus, il avait déjà eu Vipsania Agrippina, épouse de Tibère et mère du jeune Drusus, qui fut depuis empoisonné par Séjan.}, après la mort de Marcellus, et il décora du titre d’Imperator les deux fils de sa femme, Tibérius Néro et Claudius Drusus \footnote{Tibérius Néro (l’empereur Tibère) et Claudius Drusus étaient fils de Tibérius Claudius et de Livia Drusilla, que Tibérius céda pour femme à Auguste, pendant qu’elle était enceinte de Drusus.}, quoique sa propre maison fût encore florissante : car il avait fait entrer dans la famille des Césars Caius et Lucius \footnote{Par adoption.}, fils d’Agrippa, qui, même avant d’avoir quitté la robe de l’enfance, furent nommés princes de la jeunesse \footnote{Le chevalier romain que les censeurs avaient inscrit le premier sur le tableau de son ordre s’appelait \emph{princeps equestris ordinis}. Le titre de \emph{princeps juventutis} paraît analogue à celui-là.} et désignés consuls ; ce qu’Auguste, tout en feignant de le refuser, avait ardemment désiré. Mais Agrippa cessa de vivre ; les deux Césars, Lucius en allant aux armées d’Espagne, Caius en revenant blessé d’Arménie, furent enlevés par une mort que hâtèrent les destins ou le crime de leur marâtre Livie ; depuis longtemps Drusus n’était plus, il ne restait à Auguste d’autre beau-fils que Tibère. Alors celui-ci fut le centre où tout vint aboutir : il est adopté, associé à l’autorité suprême et à la puissance tribunitienne, montré avec affectation à toutes les armées. Ce n’était plus par d’obscures intrigues, mais par de publiques sollicitations, que sa mère allait à son but. Elle avait tellement subjugué la vieillesse d’Auguste, qu’il jeta sans pitié dans l’île de Planasie \footnote{Voisine de l’île d’Elbe ; on la nomme aujourd’hui \emph{Planosa}.} son unique petit-fils, Agrippa Postumus, jeune homme, il est vrai, d’une ignorance grossière et stupidement orgueilleux de la force de son corps, mais qui n’était convaincu d’aucune action condamnable. Toutefois il mit Germanicus, fils de Drusus, à la tête de huit légions sur le Rhin, et obligea Tibère de l’adopter, quoique celui-ci eût un fils déjà sorti de l’adolescence ; mais Auguste voulait multiplier les soutiens de sa maison. Il ne restait alors aucune guerre, si ce n’est celle contre les Germains ; et l’on combattait plutôt pour effacer la honte du désastre de Varus que pour l’agrandissement de l’empire ou les fruits de la victoire. Au-dedans tout était calme ; rien de changé dans le nom des magistratures ; tout ce qu’il y avait de jeune était né depuis la bataille d’Actium, la plupart des vieillards au milieu des guerres civiles : combien restait-il de Romains qui eussent vu la République ?
\subsection[{Quel successeur ?}]{Quel successeur ?}
\noindent \labelchar{IV.} La révolution était donc achevée ; un nouvel esprit avait partout remplacé l’ancien ; et chacun, renonçant à l’égalité, les yeux fixés sur le prince, attendait ses ordres. Le présent n’inspira pas de craintes, tant que la force de l’âge permit à Auguste de maintenir son autorité, sa maison, et la paix. Quand sa vieillesse, outre le poids des ans, fut encore affaissée par les maladies, et que sa fin prochaine éveilla de nouvelles espérances, quelques-uns formèrent pour la liberté des vœux impuissants ; beaucoup redoutant la guerre, d’autre la désiraient, le plus grand nombre épuisaient, sur les maîtres dont Rome était menacée, tous les traits de la censure. « Agrippa, d’une humeur farouche, irrité par l’ignominie, n’était ni d’un âge ni d’une expérience à porter le fardeau de l’empire. Tibère, mûri par les années, habile capitaine, avait en revanche puisé dans le sang des Clodius l’orgueil héréditaire de cette famille impérieuse ; et, quoi qu’il fît pour cacher sa cruauté, plus d’un indice le trahissait. Élevé, dès le berceau, parmi les maîtres du monde, chargé, tout jeune encore, de triomphes et de consulats, les années même de sa retraite ou plutôt de son exil à Rhodes n’avaient été qu’un perpétuel exercice de vengeance, tous les caprices d’un sexe dominateur. Il faudra donc ramper sous une femme et sous deux enfants \footnote{Drusus, fils de Tibère, et Germanicus, son neveu.}, qui pèseront sur la République, en attendant qu’ils la déchirent. »
\subsection[{Mort d’Auguste}]{Mort d’Auguste}
\noindent \labelchar{V.} Pendant que ces pensées occupaient les esprits, Auguste s’affaiblissait de jour en jour. Quelques soupçons tombèrent sur son épouse. Un bruit avait couru que, peu de mois auparavant, le prince, après s’être ouvert à des confidents choisis, s’était rendu, accompagné du seul Fabius Maximus, dans l’île de Planasie, pour voir Agrippa. Beaucoup de larmes coulèrent de part et d’autre, et des signes d’une mutuelle tendresse firent espérer que le jeune homme reverrait le palais de son aïeul. Maximus révéla ce secret à sa femme Marcia, celle-ci à Livie. Auguste le sut ; et, bientôt après, Maximus ayant fini ses jours par une mort qui peut-être ne fut pas naturelle, on entendit à ses funérailles, Marcia s’accuser en gémissant d’avoir causé la perte de son époux. Quoi qu’il en soit, à peine entré dans l’Illyricum, Tibère est rappelé par une lettre pressante de sa mère. On ne saurait dire si Auguste respirait encore ou n’était déjà plus, lorsqu’il arriva à Nole ; car Livie avait entouré la maison de gardes qui en fermaient soigneusement les avenues. De temps en temps elle faisait publier des nouvelles rassurantes, et, lorsqu’elle eut bien concerté ses mesures, on apprit qu’Auguste était mort et Tibère empereur.
\subsection[{Meurtre de Postumus Agrippa}]{Meurtre de Postumus Agrippa}
\noindent \labelchar{VI.} Le coup d’essai du nouveau règne fut le meurtre de Postumus Agrippa : un centurion déterminé le surprit sans armes et cependant ne le tua qu’avec peine. Tibère ne parla point au sénat de cet événement. Il feignait qu’un ordre de son père avait enjoint au tribun qui veillait sur le jeune homme de lui donner la mort, aussitôt que lui-même aurait fini sa destinée ? Il est vrai qu’Auguste, après s’être plaint avec aigreur du caractère de Postumus, avait fait confirmer son exil par un sénatus-consulte. Mais sa rigueur n’alla jamais jusqu’à tuer aucun des siens ; et il n’est pas croyable qu’il ait immolé son petit-fils à la sécurité du fils de sa femme. Il est plus vraisemblable que Tibère et Livie, l’un par crainte, l’autre par haine de marâtre, se hâtèrent d’abattre une tête suspecte et odieuse. Quand le centurion, suivant l’usage militaire, vint annoncer que les ordres de César étaient exécutés, celui-ci répondit qu’il n’avait point donné d’ordres, et qu’on aurait à rendre compte au sénat de ce qui s’était fait. À cette nouvelle, Sallustius Crispus \footnote{Neveu et fils adoptif de l’historien Salluste.}, confident du prince, et qui avait envoyé le billet au tribun, craignant de voir retomber sur lui-même une accusation également dangereuse, soit qu’il soutînt le mensonge ou déclarât la vérité, fit sentir à Livie « qu’il importait de ne point divulguer les mystères du palais, les conseils des amis de César, les services des gens de guerre ; que Tibère énerverait l’autorité, en renvoyant tout au sénat ; que la première condition du pouvoir, c’est qu’il n’y ait de comptes reconnus que ceux qui se rendent à un seul. »
\subsection[{Entrée en fonction de Tibère}]{Entrée en fonction de Tibère}
\noindent \labelchar{VII.} Cependant, à Rome, tout se précipite dans la servitude, consuls, sénateurs, chevaliers, plus faux et plus empressés à proportion de la splendeur des rangs. On se compose le visage pour ne paraître ni joyeux à la mort du prince, ni triste à l’avènement d’un autre, et chacun s’étudie à mêler les pleurs, l’allégresse, les plaintes, l’adulation. Les consuls Sext. Pompeius et Sext. Apuleius jurèrent les premiers obéissance à Tibère César ; et entre leurs mains firent serment Seius Strabo et C. Turranius, préfets, celui-ci des vivres et l’autre du prétoire, puis le sénat, les soldats et le peuple. Car Tibère laissait aux consuls l’initiative de tous les actes, à l’imitation de l’ancienne République, et comme s’il n’était pas sûr que l’empire fût à lui. L’édit même par lequel il appela les sénateurs au conseil, il ne le rendit qu’en vertu de la puissance tribunitienne qu’il avait reçue sous Auguste. Le texte en était court et le sens très modeste : « il voulait consulter le sénat sur les honneurs dus à son père, dont il ne quittait pas le corps ; ce serait son seul acte d’autorité publique. » Et cependant, Auguste à peine mort, il avait donné l’ordre comme empereur aux cohortes prétoriennes ; des veilles se faisaient à sa porte ; il avait des gardes, une cour ; des soldats l’escortaient au Forum, l’accompagnaient au sénat ; il écrivit aux armées comme un prince déjà reconnu ; il ne manquait d’hésitation que devant les sénateurs. La principale cause de ce contraste était la crainte que Germanicus, maître de tant de légions et d’un nombre immense d’auxiliaires, jouissant d’ailleurs d’une merveilleuse popularité, n’aimât mieux posséder l’empire que de l’attendre. Il tenait aussi, dans l’intérêt de sa renommée, à paraître avoir été appelé et choisi par la République, plutôt qu’imposé furtivement par les intrigues d’une femme et l’adoption d’un vieillard. On reconnut dans la suite que sa feinte irrésolution avait encore pour objet de lire dans la pensée des grands. Il tournait les paroles, les regards, en autant de crimes que sa haine mettait en réserve.
\subsection[{Lecture du testament – funérailles}]{Lecture du testament – funérailles}
\noindent \labelchar{VIII.} Tibère voulut que la première séance fût consacrée tout entière à Auguste. Le testament de ce prince, apporté par les Vestales \footnote{C’était l’usage de déposer les testaments et les traités dans les temples, et particulièrement dans celui de Vesta.}, nommait Tibère et Livie ses héritiers ; Livie était adoptée dans la famille des Jules, et recevait le nom d’Augusta. Après eux il appelait ses petits-fils et arrière-petits-fils, et à leur défaut les premiers personnages de l’état, la plupart objets de sa haine ; mais il affectait la générosité au profit de sa mémoire. Ses legs n’excédaient pas ceux d’un particulier : seulement il donnait au peuple romain et aux tribus de la ville quarante-trois millions cinq cent mille sesterces \footnote{Ou 7 951 910 F. Le sesterce, à l’époque d’Auguste, valait 20 cent.}, mille à chaque soldat prétorien et trois cents par tête aux légions et aux cohortes de citoyens romains. On délibéra ensuite sur les honneurs funèbres, dont les plus remarquables furent, « que le convoi passât par la porte triomphale »; cet avis fut ouvert par Asinius Gallus : « que les titres des lois dont Auguste était l’auteur, et les noms de peuples qu’il avait vaincus, fussent portés en tête du cortège »; ainsi opina L. Arruntius. Messala Valerius ajoutait à son vote celui de renouveler chaque année le serment de Tibère. Interrogé par le prince s’il l’avait chargé de faire cette proposition, il répondit « qu’il l’avait faite de son propre mouvement, et que, dans tout ce qui intéresserait le bien public, il ne prendrait conseil que de lui-même, dût-il déplaire. » C’était le seul raffinement qui manquât à la flatterie. Les sénateurs proposèrent par acclamation de porter le corps au bûcher sur leurs épaules. Tibère se fit, avec une arrogante modestie, arracher son consentement. Il publia un édit pour avertir le peuple « de ne point troubler les funérailles d’Auguste, comme autrefois celles de César, par un excès de zèle, et de ne pas exiger que son corps fût brûlé dans le Forum plutôt que dans le Champ de Mars, où l’attendait son mausolée ». Le jour de la cérémonie funèbre, les soldats furent sous les armes comme pour prêter main-forte : grand sujet de risée pour ceux qui avaient vu par eux-mêmes ou connu par les récits de leurs pères, cette journée d’une servitude encore toute récente et d’une délivrance vraiment essayée, où le meurtre de César paraissait à ceux-ci un crime détestable, à ceux-là une action héroïque. « Fallait-il donc maintenant tout l’appareil de la force militaire, pour protéger les obsèques d’un prince vieilli dans le pouvoir, et mort après avoir assuré contre la République la fortune de ses héritiers? »
\subsection[{Louanges et critiques post mortem}]{Louanges et critiques \emph{post mortem}}
\noindent \labelchar{IX.} Auguste lui-même devint le sujet de mille entretiens. Le peuple, frappé des plus futiles circonstances, remarquait « que le prince avait cessé de vivre le jour même où jadis il avait reçu l’empire ; qu’il était mort à Nole dans la même maison, dans la même chambre que son père Octavius. » On comptait ses consulats, « égaux en nombre à ceux de Marius et de Valerius Corvus réunis \footnote{Valérius Corvus fut consul six fois, Marius sept.}, ses trente-sept années consécutives de puissance tribunitienne, le nom d’\emph{Imperator} reçu vingt et une fois, et tant d’autres honneurs ou souvent réitérés ou entièrement nouveaux. » Les gens éclairés s’entretenaient de sa vie, dont ils faisaient l’éloge ou la censure. Suivant les uns, « la piété filiale et les malheurs de la République livrée à l’anarchie l’avaient seuls entraîné dans les guerres civiles, qu’on ne peut ni entreprendre, ni soutenir par des voies légitimes. Il avait, pour venger son père, accordé beaucoup à Antoine, beaucoup à Lépide. Quand celui-ci se fut perdu par sa lâche indolence, l’autre par ses folles amours, il ne restait de remède aux divisions de la patrie que le gouvernement d’un seul. Toutefois le pacificateur de l’état, content du nom de prince, ne s’était fait ni roi ni dictateur. Il avait donné pour barrières à l’empire l’Océan ou des fleuves lointains, réuni par un lien commun les légions, les flottes, les provinces, respecté les droits des citoyens, ménagé les alliés, embelli Rome elle-même d’une magnificence inconnue. Quelques rigueurs en petit nombre n’avaient fait qu’assurer le repos général. »\par
\labelchar{X.} On disait, d’un autre côté, « que sa tendresse pour son père et les désordres de la République ne lui avaient servi que de prétextes ; que c’était par ambition qu’il avait rassemblé les vétérans à force de largesses, levé une armée au sortir de l’enfance et sans titre public, corrompu les légions d’un consul, affecté pour le parti de Pompée un zèle hypocrite ; c’était par ambition qu’ayant usurpé, à la faveur d’un sénatus-consulte, les faisceaux et l’autorité de préteur, il s’était emparé des troupes d’Hirtius et de Pansa, tués par l’ennemi peut-être, mais peut-être aussi par les artifices de César, s’il est vrai que du poison fut versé dans la blessure de Pansa \footnote{Il se livra, près de Modène, deux batailles sanglantes, dont la première eut lieu le 15 avril 711, et qui coûtèrent la vie aux deux consuls.}, et qu’Hirtius périt de la main de ses propres soldats. Que dire du consulat envahi malgré les sénateurs ? Des armes reçues contre Antoine et tournées contre la République ? De cette proscription de citoyens, de ces distributions de terres, qui n’avaient même pas l’approbation de leurs auteurs ? Que la mort de Cassius et des deux Brutus eût été vraiment offerte aux mânes paternels, on pouvait le croire ; et encore eût-il pu, sans impiété, immoler à l’intérêt public ses ressentiments domestiques. Mais Sextus, mais Lépide, il les avait trompés, l’un par un simulacre de paix, l’autre par une feinte amitié ; mais Antoine, il l’avait entraîné dans le piège par les traités de Tarente et de Brindes et l’hymen de sa sœur, alliance perfide que le malheureux Antoine avait payée de sa vie. La paix sans doute était venue ensuite, mais une paix sanglante : au dehors, les désordres de Lollius et de Varus ; à Rome, le meurtre des Varron, des Egnatius, des Iule. » On n’épargnait pas même sa vie privée : on lui reprochait « la femme de Tibérius enlevée au lit conjugal ; les pontifes interrogés par dérision si, enceinte d’un premier époux, il lui était permis de se marier à un autre ; et le luxe effréné de Q. Tedius et de Vedius Pollio ; et Livie, fatale, comme mère, à la République, plus fatale, comme marâtre, à la maison des Césars. Et les honneurs des dieux ravis par un homme qui avait voulu comme eux des temples, des images sacrées, des flamines, des prêtres. Même en appelant Tibère à lui succéder, il avait consulté ni son cœur ni le bien public ; mais il avait deviné cette âme hautaine et cruelle, et cherché de la gloire dans un odieux contraste. » En effet, peu d’années avant sa mort, Auguste, demandant une seconde fois pour Tibère la puissance tribunitienne, avait, dans un discours, d’ailleurs à sa louange, jeté sur son maintien, son extérieur et ses mœurs, quelques traits d’une censure déguisée en apologie. La solennité des funérailles terminée, on décerne au prince mort un temple et les honneurs divins.
\subsection[{Tibère répond aux prières}]{Tibère répond aux prières}
\noindent \labelchar{XI.} Puis toutes les prières s’adressent à Tibère. Celui-ci répond par des discours vagues sur la grandeur de l’empire et sa propre insuffisance. Selon lui, « le génie d’Auguste pouvait seul embrasser toutes les parties d’un aussi vaste corps ; appelé par ce prince à partager le fardeau des affaires, lui-même avait appris par expérience combien il est difficile et hasardeux de le porter tout entier ; dans un empire qui comptait tant d’illustres appuis, il ne fallait pas que tout reposât sur une seule tête. La tâche de gouverner l’État serait plus facile, si plusieurs y travaillaient de concert. » Il y avait dans ce langage plus de dignité que de franchise. Tibère, lors même qu’il ne dissimulait pas, s’exprimait toujours, soit par caractère soit par habitude, en termes obscurs et ambigus. Mais il cherchait ici à se rendre impénétrable, et des ténèbres plus épaisses que jamais enveloppaient sa pensée. Les sénateurs, qui n’avaient qu’une crainte, celle de paraître le deviner, se répandent en plaintes, en larmes, en vœux. Ils lèvent les mains vers les statues des dieux, vers l’image d’Auguste ; ils embrassent les genoux de Tibère. Alors il fait apporter un registre dont il ordonne la lecture ; c’était le tableau de la puissance publique : on y voyait combien de citoyens et d’alliés étaient en armes, le nombre des flottes, des royaumes, des provinces, l’état des tributs et des péages, l’aperçu des dépenses nécessaires et des gratifications. Auguste avait tout écrit de sa main, et il ajoutait le conseil de ne plus reculer les bornes de l’empire : on ignore si c’était prudence ou jalousie.
\subsection[{Impair d’Asinius Gallus}]{Impair d’Asinius Gallus}
\noindent \labelchar{XII.} Le sénat s’abaissant alors aux plus humiliantes supplications, il échappa à Tibère de dire que, s’il ne peut supporter tout entier le poids du gouvernement, il se chargera cependant de la partie qu’on voudra lui confier. « Apprends-nous donc, César, fit alors Asinius Gallus, quelle partie de la chose publique tu veux qu’on te confie. » Déconcerté par cette question inattendue, Tibère garde un instant le silence. Puis, remis de son trouble, il répond « que sa délicatesse ne lui permet ni choix ni exclusion parmi les devoirs dont il désirerait être tout à fait dispensé. » Gallus avait démêlé par son visage les signes du dépit : il répliqua « qu’il n’avait pas fait cette question pour que César divisât ce qui était indivisible, mais pour qu’il fût convaincu, par son propre aveu, que la République, formant un seul corps, devait être régie par une seule âme. » Ensuite il fit l’éloge d’Auguste, et pria Tibère de se rappeler ses propres victoires et tant d’années d’une glorieuse expérience dans les fonctions de la paix. Toutefois il ne put adoucir sa colère : Tibère le haïssait de longue main, prévenu de l’idée que son mariage avec Vispania, fille d’Agrippa, que lui-même avait eue pour femme, cachait des projets au-dessus de la condition privée, et qu’il avait hérité tout l’orgueil de son père Asinius Pollio.
\subsection[{Discours de L. Arruntius et de Q Hatérius}]{Discours de L. Arruntius et de Q Hatérius}
\noindent \labelchar{XIII.} Bientôt L. Arruntius, par un discours à peu près semblable à celui de Gallus, s’attira la même disgrâce. Ce n’est pas que Tibère eût contre lui d’anciens ressentiments ; mais Arruntius, riche, homme d’action, doué de qualités éminentes, honorées de l’estime publique, excitait sa défiance. Auguste en effet, parlant dans ses derniers entretiens de ceux qu’il croyait dignes du rang suprême, mais peu jaloux d’y monter ou ambitieux de l’obtenir sans en être dignes ou enfin ambitieux et capables tout à la fois avait dit « que M. Lépidus serait digne de l’empire, mais le dédaignait ; que Gallus le désirait sans le mériter ; que L. Arruntius ne manquait pas de capacité, et, dans l’occasion, ne manquerait pas d’audace. » On est d’accord sur les deux premiers ; quelques-uns nomment Cn. Pison au lieu d’Arruntius. Tous, excepté Lépidus, périrent depuis, victimes de différentes accusations que Tibère leur suscita. Q. Hatérius et Mamercus Scaurus blessèrent encore cet esprit soupçonneux ; le premier pour lui avoir dit : « Jusques à quand, César, laisseras-tu la République sans chef ? » L’autre pour avoir fait espérer « que César ne serait pas inexorable aux prières du sénat, puisqu’il n’avait point opposé sa puissance tribunitienne à la délibération que venaient d’ouvrir les consuls. » Tibère éclata sur-le-champ contre Hatérius ; quant à Scaurus, objet d’une haine plus implacable, il n’eut point de réponse. Las enfin des clameurs de l’assemblée et des instances de chaque membre, Tibère céda peu à peu, sans avouer pourtant qu’il acceptait l’empire : mais au moins il cessa de refuser et de se faire prier. Hatérius se rendit au palais pour implorer son pardon. C’est un fait certain que, s’étant prosterné sur le passage de Tibère afin d’embrasser ses genoux, il pensa être tué par les gardes, parce que le hasard ou peut-être les mains du suppliant, firent tomber le prince. Toutefois le péril d’un homme si distingué n’adoucit pas Tibère : il fallut qu’Hatérius eût recours à Augusta, dont les instantes prières purent seules le sauver.
\subsection[{Flagornerie des sénateurs}]{Flagornerie des sénateurs}
\noindent \labelchar{XIV.} Les sénateurs prodiguèrent aussi les adulations à Augusta. Les uns voulaient qu’on lui donnât le titre de Mère, d’autres qu’on l’appelât Mère de la patrie, la plupart qu’au nom de César on ajoutât « Fils de Julie. » Tibère répondit « que les honneurs de ce sexe devaient avoir des bornes ; que lui-même n’accepterait qu’avec discrétion ceux qui lui seraient offerts. » La vérité est que son inquiète jalousie voyait dans l’élévation d’une femme son propre abaissement ; aussi ne souffrit-il pas même qu’on donnât un licteur à sa mère : on allait voter un autel de l’adoption et d’autres choses semblables ; il s’y opposa. Cependant il demanda pour Germanicus la puissance proconsulaire, et une députation fut envoyée à ce général pour lui porter le décret, et lui adresser des consolations au sujet de la mort d’Auguste. S’il ne fit point la même demande pour Drusus, c’est que Drusus était présent et désigné consul. Tibère nomma douze candidats pour la préture : c’était le nombre fixé par Auguste ; et, comme le sénat le pressait d’y ajouter, il fit serment au contraire de ne l’excéder jamais.
\subsection[{Les comices passent du Champ de Mars au sénat}]{Les comices passent du Champ de Mars au sénat}
\noindent \labelchar{XV.} Alors, pour la première fois, les comices passèrent du Champ de Mars au sénat : car, si jusqu’à ce jour le prince avait disposé des plus importantes élections, quelques-unes cependant étaient encore abandonnées aux suffrages des tribus. Le peuple, dépouillé de son droit, ne fit entendre que de vains murmures ; et le sénat se saisit volontiers d’une prérogative qui lui épargnait des largesses ruineuses et des prières humiliantes. Tibère d’ailleurs se bornait à recommander quatre candidats, dispensés il est vrai, des soins de la brigue et des chances d’un refus. Dans le même temps, les tribuns du peuple demandèrent à donner à leurs frais des jeux qui seraient ajoutés aux fastes, et, du nom d’Auguste, appelés Augustaux. Mais on assigna des dons sur le trésor, et l’on permit aux tribuns de paraître au cirque en robe triomphale : le char ne leur fut pas accordé. Bientôt la célébration annuelle de ces jeux fut transportée à celui des préteurs qui juge les contestations entre les citoyens et les étrangers.
\subsection[{Révolte des légions de Pannonie — Causes}]{Révolte des légions de Pannonie — Causes}
\noindent \labelchar{XVI.} Telle était à Rome la situation des affaires, quand l’esprit de révolte s’empara des légions de Pannonie ; révolte sans motif, si ce n’est le changement de prince, qui leur montrait la carrière ouverte au désordre et des récompenses à gagner dans une guerre civile. Trois légions étaient réunies dans les quartiers d’été, sous le commandement de Junius Blésus. En apprenant la fin d’Auguste et l’avènement de Tibère, ce général avait, en signe de deuil ou de réjouissance, interrompu les exercices accoutumés. De là naquirent, parmi les soldats, la licence, la discorde, l’empressement à écouter les mauvais conseils, enfin l’amour excessif des plaisirs et du repos, le dégoût du travail et de la discipline. Il y avait dans le camp un certain Percennius, autrefois chef d’entreprises théâtrales, depuis simple soldat, parleur audacieux, et instruit, parmi les cabales des histrions, à former des intrigues. Comme il voyait ces esprits simples en peine de ce que serait après Auguste la condition des gens de guerre, il les ébranlait peu à peu dans des entretiens nocturnes ; ou bien, sur le soir, lorsque les hommes tranquilles étaient retirés, il assemblait autour de lui tous les pervers
\subsection[{Un mutin : Percennius}]{Un mutin : Percennius}
\noindent \labelchar{XVII.} Enfin lorsqu’il se fut associé de nouveaux artisans de sédition, prenant le ton d’un général qui harangue, il demandait aux soldats « pourquoi ils obéissaient en esclaves à un petit nombre de centurions, à un petit nombre de tribuns. Quand donc oseraient-ils réclamer du soulagement, s’ils n’essayaient, avec un prince nouveau et chancelant encore, les prières ou les armes ? C’était une assez longue et assez honteuse lâcheté, de courber, trente ou quarante ans, sous le poids du service, des corps usés par l’âge ou mutilés par les blessures. Encore si le congé finissait leurs misères ! Mais après le congé il fallait rester au drapeau \footnote{Quand les années de service légionnaire étaient finies, les soldats n’étaient pas encore renvoyés chez eux. Il leur était dû une récompense en argent ou en fonds de terres ; et, en attendant qu’ils la reçussent, on les retenait sous un drapeau nommé \emph{vexillum}, où ils servaient en qualité de vétérans.}, et, sous un autre nom, subir les mêmes fatigues. Quelqu’un échappait-il vivant à de si rudes épreuves ? On l’entraînait en des régions lointaines, où il recevait comme fonds de terre, la fange des marais et des roches incultes. Le service en lui-même était pénible, infructueux : dix as par jour, voilà le prix qu’on estimait l’âme et le corps du soldat ; là-dessus, il devait se fournir d’armes, d’habits, de tentes, se racheter de la cruauté des centurions, payer les moindres dispenses. Mais les verges, mais les blessures, de rigoureux hivers, des étés laborieux, des guerres sanglantes, des paix stériles, à cela jamais de fin. Le seul remède était qu’on ne devînt soldat qu’à des conditions fixes : un denier \footnote{Le denier valait 16 as, et l’as environ 5 centimes.} par jour ; le congé au bout de la seizième année ; passé ce terme, plus d’obligation de rester sous le drapeau, et, dans le camp même, la récompense argent comptant. Les cohortes prétoriennes, qui recevaient deux deniers par tête, qui après seize ans étaient rendues à leurs foyers, couraient-elles donc plus de hasards ? Il n’ôtait rien de leur mérite aux veilles qui se faisaient dans Rome ; mais lui, campé chez des peuples sauvages, de sa tente il voyait l’ennemi.\par
\labelchar{XVIII.} Les soldats répondaient par des cris confus, et, s’animant à l’envi, l’un montrait les coups dont il fut déchiré, l’autre ses cheveux blancs, la plupart leurs vêtements en lambeaux et leurs corps demi-nus. Enfin, leur fureur s’allumant par degrés, ils parlèrent de réunir les trois légions en une seule. L’esprit de corps fit échouer ce dessein, parce que chacun voulait la préférence pour sa légion : ils prennent un autre parti, et placent ensemble les trois aigles et les enseignes des cohortes. En même temps ils amassent du gazon et dressent un tribunal, afin que le point de ralliement s’aperçoive de plus loin. Pendant qu’ils se hâtaient, Blésus accourt, menace, arrête tantôt l’un tantôt l’autre. « Soldats, s’écrie-t-il, trempez plutôt vos mains dans mon sang : ce sera un crime moins horrible de tuer votre général que de trahir votre empereur. Ou vivant, je maintiendrai mes légions dans le devoir ou, massacré par elles, ma mort avancera leur repentir. »
\subsection[{Discours du général Blésus}]{Discours du général Blésus}
\noindent \labelchar{XIX.} Le tertre de gazon ne s’en élevait pas moins ; déjà il avait atteint la hauteur de la poitrine, lorsque, vaincus par l’inébranlable fermeté du général, ils l’abandonnèrent. Blésus, avec une adroite éloquence, leur représente « que ce n’est point par la sédition et le désordre que les vœux des soldats doivent être portés à César ; que jamais armées sous les anciens généraux, jamais eux-mêmes sous Auguste, n’avaient formé des demandes si imprévues ; qu’il était peu convenable d’ajouter ce surcroît aux soucis d’un nouveau règne. S’ils voulaient cependant essayer, en pleine paix, des prétentions que n’élevèrent jamais dans les guerres civiles les vainqueurs les plus exigeants, pourquoi, au mépris de la subordination et des lois sacrées de la discipline, recourir à la force ? Ils pouvaient nommer une députation et lui donner des instructions en sa présence. » Un cri universel désigna pour député le fils de Blésus, tribun des soldats, et lui enjoignit de demander congé au bout de seize ans ; « on s’expliquerait sur le reste, quand ce premier point serait accordé. » Le départ du jeune homme ramena un peu de calme. Mais le soldat, fier de voir le fils de son général devenu l’orateur de la cause commune, sentit que la contrainte avait arraché ce que la soumission n’aurait pas obtenu.
\subsection[{Pillage}]{Pillage}
\noindent \labelchar{XX.} Cependant quelques manipules, envoyés à Nauport \footnote{Cellarius croit que c’est Oberlaybach, dans la Carniole, à quelques lieues de Laybach.}, avant la sédition, pour l’entretien des chemins et des ponts et les autres besoins de service, en apprenant que la révolte a éclaté dans le camp, partent avec les enseignes et pillent les villages voisins, sans excepter Nauport, qui était une espèce de ville. Les centurions qui les retiennent sont poursuivis de huées, d’outrages, à la fin même de coups. Le principal objet de leur colère était le préfet de camp \footnote{Le préfet de camp était, dans les armées romaines, tout à la fois l’officier de génie et l’administrateur militaire. Il s’occupait de tout ce qui concernait les campements, les transports, les machines de guerre, les malades et les médecins, etc.} Aufidénius Rufus. Arraché de son chariot et chargé de bagages, ils le faisaient marcher devant eux, lui demandant par dérision « s’il aimait à porter de si lourds fardeaux, à faire de si longues routes. » C’est que Rufus, longtemps simple soldat, puis centurion, ensuite préfet de camp, remettait en vigueur l’ancienne et austère discipline ; homme vieilli dans la peine et le travail, et dur à proportion de ce qu’il avait souffert.\par
\labelchar{XXI.} À l’arrivée de ces mutins la sédition recommence, et une multitude de pillards se répand dans la campagne. Blésus en arrête quelques-uns, principalement ceux qui étaient chargés de butin ; et, pour effrayer les autres, il ordonne qu’on les batte de verges et qu’on les jette en prison : alors le général était encore obéi des centurions et de ce qu’il y avait de bon parmi les soldats. Les coupables entraînés résistent, embrassent les genoux de leurs camarades, les appellent par leurs noms ; ou bien, invoquant chacun sa centurie, sa cohorte, sa légion, ils s’écrient que tous sont menacés d’un sort pareil. En même temps ils chargeaient le lieutenant d’imprécations, attestaient le ciel et les dieux, n’omettaient rien pour exciter l’indignation, la pitié, la crainte, la fureur. Tout le monde accourt en foule ; la prison est forcée, les prisonniers dégagés de leurs fers ; et cette fois on s’associe les déserteurs et les criminels condamnés à mort.
\subsection[{Un mutin : Vibulénus}]{Un mutin : Vibulénus}
\noindent \labelchar{XXII.} Alors l’embrasement redouble de violence, et la sédition trouve de nouveaux chefs. Un certain Vibulénus, simple soldat, se fait élever sur les épaules de ses camarades, devant le tribunal de Blésus ; et, au milieu de cette multitude émue et attentive à ce qu’il allait faire : « Amis, s’écrie-t-il, vous venez de rendre la jouissance de la lumière et de l’air à ces innocentes et malheureuses victimes ; mais mon frère, qui lui rendra la vie ? Il était envoyé vers vous par l’armée de Germanie, pour traiter de nos intérêts communs ; et, la nuit dernière, ce tyran l’a fait égorger par les gladiateurs qu’il entretient et qu’il arme pour être les bourreaux des soldats. Réponds-moi, Blésus : où as-tu jeté le cadavre de mon frère ? À la guerre même on n’envie pas la sépulture à un ennemi. Laisse-moi rassasier ma douleur de baisers et de larmes, ensuite commande qu’on m’égorge à mon tour ; pourvu que ces braves amis rendent les derniers devoirs à deux infortunés, dont tout le crime est d’avoir défendu la cause des légions. »\par
\labelchar{XXIII.} À ces paroles incendiaires, il ajouta des pleurs, et se frappait la poitrine et le visage. Bientôt il écarte ceux qui le soutenaient, se jette à terre, et, se roulant aux pieds de ses camarades, il excite un transport si universel de pitié et de vengeance, qu’une partie des soldats met aux fers les gladiateurs de Blésus, tandis que les autres enchaînent ses esclaves ou se répandent de tous côtés pour chercher le cadavre. Si l’on n’eût promptement acquis la certitude que nulle part on ne trouvait de corps, que les esclaves mis à la torture, niaient l’assassinat, enfin que Vibulénus n’avait jamais eu de frère, la vie du général courait de grands dangers. Cependant ils chassent les tribuns et le préfet de camp, pillent leurs bagages, et tuent le centurion Lucillius, que, dans leurs plaisanteries militaires, ils avaient surnommé \emph{Encore une}, parce qu’après avoir rompu sur le dos d’un soldat sa verge de sarment \footnote{Le cep de vigne était la marque distinctive des centurions. C’est avec cette verge qu’ils châtiaient les soldats coupables ou indociles.}, il criait d’une voix retentissante qu’on lui en donnât encore une, et après celle-là une troisième. Les autres centurions échappèrent en se cachant ; un seul fut retenu, Julius Clémens, qui, par facilité de son esprit, sembla propre à porter la parole au nom des soldats. Enfin les légions elles-mêmes se divisèrent, et la huitième allait en venir aux mains avec la quinzième pour un centurion nommé Sirpicus \footnote{\emph{Sirpicus} paraît venir de \emph{sirpus} ou \emph{scirpus}, jonc. Peut-être le centurion dont il s’agit se servait-il de jonc, au lieu de vigne, pour frapper le soldat.}, que celle-ci défendait tandis que l’autre demandait sa mort, si la neuvième n’eût interposé ses prières, appuyées de menaces contre ceux qui les repousseraient.
\subsection[{Tibère envoie son fils Drusus}]{Tibère envoie son fils Drusus}
\noindent \labelchar{XXIV.} Instruit de ces mouvements, Tibère, quoique impénétrable et soigneux de cacher surtout les mauvaises nouvelles, se décide à faire partir son fils Drusus avec les premiers de Rome et deux cohortes prétoriennes. Drusus ne reçut pas d’instructions précises : il devait se régler sur les circonstances. Les cohortes furent renforcées de surnuméraires choisis. On y ajouta une grande partie de la cavalerie prétorienne, et l’élite des Germains que l’empereur avait alors dans sa garde. Le préfet de prétoire Elius Séjanus, donné pour collègue à son père Strabon, et tout-puissant auprès de Tibère, partit aussi, pour être le conseil du jeune homme et montrer de loin à chacun les faveurs et les disgrâces. À l’approche de Drusus, les légions, par une apparence de respect, allèrent au-devant de lui, non toutefois avec les signes ordinaires d’allégresse, ni parées de leurs décorations, mais dans la tenue la plus négligée, et avec des visages qui, en affectant la tristesse, laissaient percer la révolte.\par
\labelchar{XXV.} Lorsqu’il fut entré dans le camp, elles s’assurèrent des portes et distribuèrent à l’intérieur des pelotons armés : le reste environna le tribunal d’un immense concours. Drusus était debout, et de la main demandait le silence. Les soldats, enhardis par la vue de leur nombre, poussaient des cris menaçants ; puis tout à coup, en regardant César, ils s’intimidaient : c’était tour à tour un murmure confus, d’horribles clameurs, un calme soudain ; agités de passions contraires, ils tremblaient et faisaient trembler. Enfin, le tumulte cessant un moment, Drusus lit une lettre de son père. Elle portait « que ses premiers soins étaient pour ces vaillantes légions avec lesquelles il avait enduré les fatigues de tant de guerres ; que dès l’instant où le deuil lui laisserait quelque repos, il entretiendrait le sénat de leurs demandes ; qu’en attendant il leur avait envoyé son fils, qui accorderait sans retard ce qu’il était permis d’accorder sur-le-champ ; que le reste devait être réservé au sénat, auquel il était juste de laisser sa part dans la distribution ou le refus des grâces. »
\subsection[{Le centurion Clémens parle pour tous}]{Le centurion Clémens parle pour tous}
\noindent \labelchar{XXVI.} L’armée répondit que le centurion Clémens était chargé de s’expliquer pour tous. Celui-ci, prenant la parole, demande le congé après seize ans, les récompenses à la fin du service, un denier de paye par jour, enfin que les vétérans ne soient plus retenus sous le drapeau. » Drusus parlait d’attendre une décision suprême du sénat et de son père ; des cris l’interrompent : « Qu’est-il venu faire, s’il ne peut augmenter la paye du soldat, ni soulager ses maux ? Il est donc sans pouvoir pour le bien ? Ah ! Les pouvoirs ne manquent à personne, quand il s’agit de frapper ou de tuer. Tibère jadis empruntait le nom d’Auguste pour refuser justice aux légions ; Drusus renouvelle les mêmes artifices : ne leur viendra-t-il donc jamais que des enfants en tutelle ? Chose étrange ! L’empereur ne renvoie au sénat que ce qui est en faveur des gens de guerre : il faut donc aussi consulter le sénat toutes les fois qu’on les mène au combat ou au supplice. Récompenser est-il le privilège de quelques-uns ; punir, le droit de tous ? »\par
\labelchar{XXVII.} Ils quittent enfin le tribunal, et, à mesure qu’ils rencontrent des prétoriens ou des amis de Drusus, ils le menacent du geste, dans l’intention d’engager une querelle et de tirer l’épée. Ils en voulaient principalement à Cn. Lentulus, le plus distingué de tous par son âge et sa gloire militaire, et, à ce titre, soupçonné d’affermir l’esprit du jeune César, et de s’indigner plus qu’un autre de ces attentats contre la discipline. Peu de moments après, il se retirait avec Drusus, et retournait par prudence au camp d’hiver, lorsqu’on l’entoure en lui demandant « où il va" ; si c’est vers le sénat ou vers l’empereur, afin d’y combattre aussi la cause des légions. » En même temps on fond sur lui avec une grêle de pierres ; et, déjà tout sanglant d’un coup qui l’atteignit, sa mort était certaine, si la troupe qui accompagnait Drusus ne fût accourue pour le sauver.
\subsection[{Une éclipse calme les mutins}]{Une éclipse calme les mutins}
\noindent \labelchar{XXVIII.} La nuit était menaçante et aurait enfanté des crimes, si le hasard n’eût tout calmé. On vit, dans un ciel serein, la lune pâlir tout à coup. Frappé de ce phénomène, dont il ignorait la cause, le soldat crut y lire l’annonce de sa destinée. Cet astre qui s’éteignait lui parut l’image de sa propre misère ; il conçut l’espoir que ses vœux seraient accomplis, si la déesse reprenait son majestueux éclat. Ils font donc retentir l’air du bruit de l’airain, du son des clairons et des trompettes \footnote{Les éclipses de la lune étaient imputées à des maléfices, et les peuples s’efforçaient de la secourir par des bruits confus et tumultueux. Ils s’imaginaient que les cris des hommes, le son retentissant de l’airain et des trompettes, empêcheraient la déesse d’entendre les enchantements de la magicienne qui essayait de la faire descendre sur la terre.} ; tour à tout joyeux ou affligés, suivant qu’elle apparaît plus brillante ou plus obscure. Enfin des nuées qui s’élèvent la dérobent à leurs regards, et ils la croient ensevelie pour jamais dans les ténèbres. C’est alors que, passant, par une pente naturelle, de la frayeur à la superstition, ils s’écrient en gémissant que le ciel leur annonce d’éternelles infortunes, et que les dieux ont horreur de leurs excès. Attentif à ce mouvement des esprits, et persuadé que la sagesse devait profiter de ce qu’offrait le hasard, Drusus ordonna qu’on parcourût les tentes. Il fait appeler le centurion Clemens, et avec lui tous ceux qui jouissaient d’une popularité honnêtement acquise. Ceux-ci se mêlent parmi les soldats chargés de veiller sur le camp ou de garder les portes ; ils invitent à l’espérance, ils font agir les craintes :"Jusques à quand assiégerons-nous le fils de notre empereur ? Quel sera le terme de nos dissensions ? Prêterons-nous serment à Percennius et à Vibulénus ? Sans doute Percennius et Vibulénus donneront au soldat sa paye, des terres aux vétérans ! Ils iront, à la place des Nérons et des Drusus, dicter des lois au peuple romain ! Ah ! Plutôt, si nous avons été les derniers à faillir, soyons les premiers à détester notre faute. Ce qu’on demande en commun se fait attendre ; une faveur personnelle est obtenue aussitôt que méritée. » Après avoir ainsi ébranlé les esprits et semé de mutuelles défiances, ils détachent les jeunes soldats des vieux, une légion d’une autre. Alors l’amour du devoir rentre peu à peu dans les cœurs ; les veilles cessent aux portes ; les enseignes, réunies au commencement de la sédition, sont reportées chacune à sa place.
\subsection[{Fin de la révolte}]{Fin de la révolte}
\noindent \labelchar{XXIX.} Drusus, au lever du jour, convoque les soldats, et, avec une dignité naturelle qui lui tenait lieu d’éloquence, il condamne le passé, loue le présent ; déclare « qu’il est inaccessible à la terreur et aux menaces ; que, s’il les voit soumis, s’il entend de leur bouche des paroles suppliantes, il écrira à son père d’accueillir avec bonté les prières des légions. » Sur leur demande, le fils de Blésus est envoyé une seconde fois vers Tibère avec L. Apronius, chevalier romain de la suite de Drusus, et Justus Catonius, centurion primipilaire \footnote{Le centurion primipilaire (le premier de tous) avait rang immédiatement après les tribuns.}. Les avis furent ensuite partagés : les uns voulaient qu’on attendît le retour de ces députés, et que dans l’intervalle on achevât de ramener le soldat par la douceur. D’autres penchaient pour les remèdes violents, soutenant « que la multitude était toujours extrême ; terrible, si elle ne tremble, et une fois qu’elle a peur, se laissant impunément braver ; qu’il fallait ajouter aux terreurs de la superstition la crainte du pouvoir, en faisant justice des chefs de la révolte. » Drusus était naturellement enclin à la rigueur : il mande Vibulénus et Percennius, et ordonne qu’on les tue. La plupart disent que leurs corps furent enfouis dans la tente du général, plusieurs qu’on les jeta hors du camp, en spectacle aux autres.\par
\labelchar{XXX.} Ensuite on rechercha les principaux séditieux. Plusieurs, épars dans la campagne, furent tués par les centurions ou les prétoriens. Les manipules eux-mêmes, pour gage de leur fidélité, en livrèrent quelques-uns. Un hiver prématuré causait aux soldats de nouvelles alarmes : des pluies affreuses et continuelles les empêchaient de sortir des tentes et de se rassembler ; à peine pouvaient-ils préserver leurs enseignes des coups de vent et des torrents d’eau qui les emportaient. Ajoutons la colère céleste, dont la crainte durait encore : « Ce n’était pas en vain qu’ils voyaient les astres pâlir, et les tempêtes se déchaîner sur leurs têtes impies. Le seul remède à tant de maux était d’abandonner un camp dévoué au malheur et souillé par le crime, et de se soustraire à la vengeance des dieux en regagnant leurs quartiers d’hiver. » La neuvième demandait à grands cris qu’on attendît la réponse de Tibère. Enfin, restée seule par le départ des autres, elle prévint d’elle-même une nécessité désormais inévitable ; et Drusus, voyant le calme entièrement rétabli, reprit le chemin de Rome sans attendre le retour de la députation.
\subsection[{Révolte des légions de Germanie – Causes}]{Révolte des légions de Germanie – Causes}
\noindent \labelchar{XXXI.} Presque dans le même temps et pour les mêmes raisons, les légions de Germanie s’agitèrent plus violemment encore, étant en plus grand nombre. Elles espéraient d’ailleurs que Germanicus ne pourrait souffrir un maître, et qu’il se donnerait à des légions assez fortes pour entraîner tout l’empire. Deux armées étaient sur le Rhin : l’une, appelée supérieure, avait pour chef C. Silius ; l’autre, inférieure, obéissait à A. Cécina. La direction suprême de toutes les deux appartenait à Germanicus, occupé alors à régler le cens des Gaules \footnote{Jules César avait imposé à la Gaule un tribut annuel ; mais il ne paraît pas qu’il eût soumis les habitants à une assiette régulière d’impôts : il laissait probablement aux cités le soin d’acquitter collectivement cette dette des vaincus. Ce fut seulement en 727 que le cens fut institué : c’était un dénombrement des personnes et des biens, d’après lequel on réglait la contribution de chacun.}. Les légions de Silius, encore irrésolues, observaient quel serait pour autrui le succès de la révolte. Celles de l’armée inférieure s’y jetèrent avec rage. Le mal commença par la vingt et unième et la cinquième, qui entraînèrent la vingtième et la première. Toutes quatre étaient réunies dans un camp d’été, sur les frontières des Ubiens, oisives ou faisant peu de service. Quand on apprit la fin d’Auguste, une foule de gens du peuple, enrôlés depuis peu dans Rome, et qui en avaient apporté l’habitude de la licence et de la haine du travail, remplirent ces esprits grossiers de l’idée « que le temps était venu, pour les vieux soldats, d’obtenir un congé moins tardif, pour les jeunes d’exiger une plus forte paye, pour tous de demander du soulagement à leurs maux et de punir la cruauté des centurions. » Et ces discours, ce n’est point un seul homme qui les débite, comme Percennius en Pannonie, à des oreilles craintives, au milieu d’une armée qui en voit derrière elle de plus puissantes. Ici la sédition a mille bouches, mille voix qui répètent « que les légions germaniques font le destin de l’empire ; que leurs victoires en reculent les bornes ; que les généraux empruntent d’elles leur surnom. »\par
\labelchar{XXXII.} Le lieutenant n’essayait point de les contenir : ce délire universel lui avait ôté le courage. Soudain la fureur les emporte, et ils fondent l’épée à la main sur les centurions, éternels objets de la haine du soldat, et premières victimes de ses vengeances. Ils les terrassent et les chargent de coups, s’acharnant soixante sur un seul, comme les centurions étaient soixante par légion. Enfin ils les jettent déchirés, mutilés, la plupart morts, dans le Rhin ou devant les retranchements. Septimius s’était réfugié sur le tribunal et se tenait prosterné aux pieds de Cécina : ils le réclamèrent avec tant d’obstination qu’il fallut l’abandonner à leur rage. Cassius Chéréa, qui depuis s’est assuré un nom dans la postérité par le meurtre de Caius, et qui était alors jeune et intrépide, s’ouvrit un passage avec son épée à travers les armes de ces furieux. Dès lors ni tribun, ni préfet de camp, ne trouva d’obéissance : les soldats se partageaient entre eux les veilles, les gardes, les autres soins du moment. Ce qui parut, à quiconque avait étudié l’esprit des camps, le principal symptôme d’une grande et implacable rébellion, c’est qu’au lieu de s’agiter en désordre et à la voix de quelques factieux, tous éclataient, tous se taisaient à la fois, avec tant d’ensemble et de concert, qu’on aurait cru leurs mouvements commandés.
\subsection[{Germanicus}]{Germanicus}
\noindent \labelchar{XXXIII.} Cependant Germanicus, occupé, comme nous l’avons dit, à régler le cens des Gaules, reçut la nouvelle qu’Auguste n’était plus. Il avait épousé sa petite-fille Agrippine, dont il avait plusieurs enfants. Lui-même était fils de Drusus, neveu de Tibère, et petit-fils d’Augusta. Mais ces titres ne le rassuraient pas contre la haine secrète de son oncle et de son aïeule, haine dont les causes étaient d’autant plus actives, qu’elles étaient injustes. La mémoire de Drusus était grande auprès des Romains, et l’on croyait que, s’il fût parvenu à l’empire, il eût rétabli la liberté. De là leur affection pour Germanicus, à qui s’attachaient les mêmes espérances. En effet, l’esprit populaire et les manières affables du jeune César contrastaient merveilleusement avec l’air et le langage de Tibère, si hautain et si mystérieux. À cela se joignaient des animosités de femmes : Livie montrait pour Agrippine toute l’aigreur d’une marâtre ; Agrippine elle-même ne savait pas assez se contenir. Toutefois sa chasteté et sa tendresse conjugale faisaient tourner au profit de la vertu cette hauteur de caractère.\par
\labelchar{XXXIV.} Mais plus Germanicus était près du rang suprême, plus il s’efforçait d’y affermir Tibère. Il le fit reconnaître par les cités les plus voisines, celles des Séquanes et des Belges. Bientôt instruit de la révolte de ses légions, il part à la hâte et les trouve hors du camp. Elles venaient à sa rencontre, les yeux baissés vers la terre, comme par repentir. Quand il fut entré dans l’enceinte, des murmures confus commencèrent à s’élever. Quelques soldats, prenant sa main sous prétexte de le baiser, glissèrent ses doigts dans leur bouche, afin qu’il touchât leurs gencives sans dents ; d’autres lui montraient leurs corps courbés par la vieillesse. Tout le monde était assemblé pêle-mêle : il leur ordonne de se ranger par manipules, afin de mieux entendre sa réponse ; de prendre leurs enseignes, afin qu’il pût au moins distinguer les cohortes. On obéit, mais lentement. Alors, commençant par rendre un pieux hommage à Auguste, il passe aux victoires et aux triomphes de Tibère, et célèbre avant tout ses glorieuses campagnes en Germanie, à la tête de ces mêmes légions. Il leur montre l’accord unanime de l’Italie, la fidélité des Gaules, enfin la paix et l’union régnant dans tout l’empire. Ces paroles furent écoutées en silence ou n’excitèrent que de légers murmures.
\subsection[{Plaintes des soldats}]{Plaintes des soldats}
\noindent \labelchar{XXXV.} Mais lorsque, arrivé à la sédition, il leur demanda ce qu’était devenue la subordination militaire, où était l’antique honneur de la discipline, ce qu’ils avaient fait des centurions, des tribuns, alors se dépouillant tous à la fois de leurs vêtements, ils lui demandent à leur tour s’il voit les cicatrices de leurs blessures, les traces de coups de verges. Bientôt des milliers de voix accusent en même temps le trafic des exemptions, l’insuffisance de la solde, la dureté des travaux, qu’ils énumèrent en détail : retranchements, fossés, transport de fourrage et de bois, enfin tout ce qu’on exige du soldat pour les besoins du service ou pour bannir l’oisiveté des camps. Les vétérans se distinguaient par la violence de leurs cris, nombrant les trente années et plus qu’ils portaient les armes, et implorant sa pitié pour des fatigues sans mesure. « Passeraient-ils donc immédiatement du travail à la mort ? Quand trouveraient-ils la fin d’une si laborieuse milice, et un repos qui ne fût pas la misère ? » Il y en eut aussi qui réclamèrent le legs d’Auguste, en ajoutant des vœux pour la grandeur de Germanicus, et l’offre de leurs bras s’il voulait l’empire. À ce mot, comme si un crime eût souillé son honneur, il s’élance de son tribunal et veut s’éloigner. Les soldats lui présentent la pointe de leurs armes et l’en menacent s’il ne remonte. Il s’écrie alors qu’il mourra plutôt que de trahir sa foi ; et, tirant son épée, il la levait déjà pour la plonger dans son sein, lorsque ceux qui l’entouraient lui saisirent le bras et le retinrent de force. Des séditieux qui se pressaient à l’extrémité de l’assemblée, et dont plusieurs, chose à peine croyable, s’avancèrent exprès hors de la foule, l’exhortaient à frapper ; et un soldat, nommé Calusidius, lui offrit son épée nue, en disant qu’elle était plus tranchante. Ce trait parut cruel et révoltant, même aux plus furieux ; et il y eut un moment de relâche dont les amis de César profitèrent pour l’entraîner dans la tente.\par
\labelchar{XXXVI.} Là il fut délibéré sur le choix des remèdes : on annonçait que les mutins préparaient une députation pour attirer à leur parti l’armée du haut Rhin ; qu’ils avaient résolu de saccager la ville des Ubiens, et, que, les mains une fois souillées de cette proie, ils s’élanceraient sur les Gaules et y porteraient le ravage. Pour surcroît d’alarmes, l’ennemi connaissait nos discordes, et, si on abandonnait la rive, il ne manquerait pas de s’y jeter. Armer les auxiliaires et les alliés contre les légions rebelles, c’était allumer la guerre civile : la sécurité était dangereuse, la faiblesse humiliante ; tout refuser, tout accorder, mettait également la République en péril. Toutes les raisons mûrement examinées, on prit le parti de supposer des lettres de l’empereur ; elles promettaient « le congé après vingt ans, la vétérance après seize, à condition de rester sous le drapeau, sans autre devoir que de repousser l’ennemi ; quant au legs d’Auguste, il serait payé et porté au double. »\par
\labelchar{XXXVI.} Le soldat comprit que c’était une ruse pour gagner du temps et voulut qu’on tînt parole sans délai. Les tribuns donnent aussitôt les congés ; pour les largesses, chaque légion devait les recevoir dans ses quartiers d’hiver. Mais la cinquième et la vingt et unième ne relâchèrent rien de leur obstination qu’on eût payé dans le camp même, avec l’argent que César et ses amis avaient apporté pour leurs besoins personnels. Cécina ramena dans la ville des Ubiens \footnote{Qui depuis fut Cologne, \emph{Colonia Agrippensis}.} la première et la vingtième ; marche honteuse, où l’on voyait traîner entre les aigles et les enseignes un trésor conquis sur le général. Germanicus se rendit à l’armée supérieure pour recevoir son serment. La seconde, la treizième et la seizième légion le prêtèrent sans balancer. La quatorzième avait montré quelque hésitation : on y distribua, sans que personne l’eût demandé, les congés et l’argent.\par
\labelchar{XXXVIII.} Il y eut chez les Chauques un essai de révolte, tenté par les vexillaires \footnote{Corps détachés d’un corps principal auquel ils appartiennent. L’enseigne de la cohorte s’appelait \emph{vexillum}, celle de la légion était l’aigle.} des légions rebelles, qui gardaient ce pays, et réprimé un moment par un prompt supplice de deux soldats. Cet exemple que fit, avec moins de droit que d’utilité, le préfet de camp Memmius. Bientôt l’orage devient plus terrible et Memmius fugitif est découvert : la sûreté que ne lui offrait point sa retraite, il la trouve dans son audace. « Ce n’est pas à un préfet, s’écrie-t-il, que vous faites la guerre ; c’est à Germanicus, votre général ; c’est à Tibère votre empereur. » Il intimide tout ce qui résiste, saisit le drapeau, tourne droit vers le fleuve, et, menaçant de traiter comme déserteur quiconque s’écartera des rangs, il les ramène au camp d’hiver, agités mais contenus.
\subsection[{Germanicus en danger}]{Germanicus en danger}
\noindent \labelchar{XXXIX.} Cependant les envoyés du sénat arrivent auprès de Germanicus, déjà revenu à l’Autel des Ubiens \footnote{Quelques-uns pensent que c’est Bonn, d’autres Cologne ou un lieu voisin.}. Deux légions, la première et la vingtième, y étaient en quartier d’hiver, avec les corps des vétérans nouvellement formés. Ces esprits, égarés par le délire de la peur et du remords, se persuadent qu’on vient, au nom du sénat, révoquer les faveurs que la sédition avait extorquées, et, comme il faut à la multitude un coupable, n’y eût-il pas de crime, ils accusent le consulaire Munatius Plancus, chef de la députation, d’être l’auteur du sénatus-consulte. Au milieu de la nuit, ils commencent à demander l’étendard placé dans la maison de Germanicus, courent en foule à sa demeure et en brisent les portes. Le général est arraché de son lit, et contraint, pour échapper à la mort, de livrer l’étendard. Les mutins, errant ensuite par la ville, rencontrent des députés qui, au premier bruit de ce tumulte, se rendaient chez Germanicus. Ils les chargent d’injures et s’apprêtent à les massacrer. Plancus surtout, qui avait cru la fuite indigne de son rang. Il n’eut, en ce péril, d’autre refuge que le camp de la première légion. Là, tenant embrassés l’aigle et les enseignes, il se couvrait en vain de leur protection sacrée, et, si l’aquilifère Calpurnius n’avait empêché les dernières violences, on aurait vu, dans un camp romain, un envoyé du peuple romain, victime d’un attentat rare même chez les ennemis, souiller de son sang les autels des dieux. Lorsque enfin le jour éclaira de sa lumière général et soldats et permit de distinguer les hommes et leurs actions, Germanicus entra dans le camp, se fit amener Plancus, et le plaça auprès de lui sur son tribunal. Alors, condamnant ces nouveaux transports, dont il accuse moins les soldats que la fatalité et la colère des dieux, il explique le sujet de la députation, déplore éloquemment l’outrage fait au caractère d’ambassadeur, le malheur si cruel et si peu mérité de Plancus, l’opprobre dont la légion vient de se couvrir, et, après avoir étonné plutôt que calmé les esprits, il renvoie les députés avec une escorte de cavalerie auxiliaire ?\par
\labelchar{XL.} En ces moments critiques, tout le monde blâmait Germanicus de ne pas se rendre à l’armée supérieure, où il trouverait obéissance et secours contre les rebelles. « Les congés, les dons, la faiblesse, n’avaient, disait-on, que trop aggravé le mal. Si la vie n’était rien pour lui, pourquoi laisser un fils en bas âge, une épouse enceinte à la merci de forcenés, violateurs des droits les plus saints ? Qu’il les rendît au moins à un aïeul, à la République ! » Germanicus balança longtemps ; Agrippine repoussait l’idée de fuir, protestant qu’elle était fille d’Auguste et qu’elle ne dérogerait pas en face du danger. À la fin son époux, embrassant avec larmes leur jeune enfant et ce sein dépositaire d’un autre gage, la détermine à partir. On vit alors un départ déplorable, l’épouse d’un général fugitive et emportant son enfant dans ses bras, autour d’elle les femmes éplorées de leurs amis, qu’elle entraînait dans sa fuite, et, avec la douleur de ce triste cortège, la douleur non moins grande de ceux qui restaient.\par
\labelchar{XLI.} Ce tableau, qui annonçait plutôt une ville prise par l’ennemi que le camp et la fortune d’un César, ces pleurs, ces gémissements, attirèrent l’attention des soldats eux-mêmes. Ils sortirent de leurs tentes : « Quels sont ces cris lamentables ? Qu’est-il donc arrivé de sinistre ? Des femmes d’un si haut rang, et pas un centurion, pas un soldat pour les protéger ! L’épouse de César, sans suite, sans aucune des marques de sa grandeur ! Et c’est aux Trévires, c’est à une foi étrangère, qu’elle va confier sa tête ! » Alors la honte et la pitié, le souvenir d’Agrippa son père, d’Auguste son aïeul, de son beau-père Drusus, l’heureuse fécondité d’Agrippine elle-même et sa vertu irréprochable, cet enfant né sous la tente, élevé au milieu des légions, qui lui donnaient le surnom militaire de Caligula, parce que, afin de le rendre agréable aux soldats, on lui faisait souvent porter leu chaussure \footnote{La chaussure des soldats s’appelait \emph{caliga}.}, tout concourt à les émouvoir. Mais rien n’y contribua comme le dépit de se voir préférer les Trévires. Ils se jettent au-devant d’Agrippine, la supplient de revenir, de rester ; et, tandis qu’une partie essaye d’arrêter ses pas, le plus grand nombre retourne vers Germanicus. Lui, encore ému de douleur et de colère, s’adressant à la foule qui l’environne :
\subsection[{Discours de Germanicus aux mutins}]{Discours de Germanicus aux mutins}
\noindent \labelchar{XLII.} « Ne croyez pas, dit-il, que mon épouse et mon fils me soient plus chers que mon père et la République. Mais mon père a pour sauvegarde sa propre majesté ; l’empire a ses autres armées. Ma femme et mes enfants, que j’immolerais volontiers à votre gloire, je les dérobe maintenant à votre fureur, afin que, si le crime ensanglante ces lieux, je sois la seule victime, et que le meurtre de l’arrière-petit-fils d’Auguste et de la belle-fille de Tibère n’en comble pas la mesure. En effet, qu’y a-t-il eu pendant ces derniers jours que n’ait violé votre audace ? Quel nom donnerai-je à cette foule qui m’entoure ? Vous appellerai-je soldats ? Vous avez assiégé comme un ennemi le fils de votre empereur ; citoyens ? Vous foulez aux pieds l’autorité du sénat : les lois même de la guerre, le caractère sacré d’ambassadeur, le droit des gens, vous avez tout méconnu. Jules César apaisa d’un mot une sédition de son armée, en appelant \emph{Quirites} des hommes qui trahissaient leurs serments \footnote{Ces soldats mutinés, qui ne respectaient plus la discipline, respectaient encore leur nom de soldats. L’appellation de \emph{Quirites} leur parut la même injure que si l’on apostrophait un de nos bataillons du nom de \emph{bourgeois}.}. Auguste, d’un seul de ses regards, fit trembler les légions d’Actium. Si nous n’égalons pas encore ces héros, nous sommes leurs rejetons ; et l’on verrait avec surprise et indignation le soldat d’Espagne ou de Syrie nous manquer de respect. Et c’est la première légion, tenant les enseignes de Tibère ; c’est vous, soldats de la vingtième, compagnons de ses victoires, riches de ses bienfaits, qui payez votre général d’une telle reconnaissance ! Voilà donc ce que j’annoncerai à mon père, qui de toutes les autres provinces ne reçoit que des nouvelles heureuses ! Je lui dirai que ses jeunes soldats, que ses vétérans, ne se rassasient ni de congés ni d’argent ; qu’ici seulement les centurions sont tués, les tribuns chassés, les députés prisonniers, qu’ici le sang inonde les camps, rougit les fleuves, qu’ici enfin ma vie est à la merci d’une multitude furieuse.\par
\labelchar{XLIII.} « Pourquoi, le premier jour où j’élevai la voix, m’arrachiez-vous le fer que j’allais me plonger dans le cœur, trop aveugles amis ? Il me rendait un bien plus généreux office, celui qui m’offrait son glaive : j’aurais péri du moins avant d’avoir vu la honte de mon armée. Vous auriez choisi un autre chef, qui sans doute eût laissé ma mort impunie, mais qui eût vengé le massacre de Varus et des trois légions. Car nous préservent les dieux de voir passer aux Belges, malgré l’empressement de leur zèle, l’éclatant honneur d’avoir soutenu la puissance romaine et abaissé l’orgueil de la Germanie ! Âme du divin Auguste, reçue au séjour des Immortels, image de mon père Drusus \footnote{L’image de Drusus était parmi les étendards.}, mémoire sacrée d’un grand homme, venez, avec ces mêmes soldats, sur qui la gloire et la vertu reprennent leurs droits, venez effacer une tache humiliante, et tournez à la ruine de l’ennemi ces fureurs domestiques. Et vous, dont je vois les visages, dont je vois les cœurs heureusement changés, si vous rendez au sénat ses députés, à l’empereur votre obéissance, à moi ma femme et mon fils, rompez avec la sédition, séparez de vous les artisans de trouble. Ce sera la marque d’un repentir durable, et le gage de votre fidélité. »
\subsection[{Fin de la révolte}]{Fin de la révolte}
\noindent \labelchar{XLIV.} Touchés par ce discours, ils lui demandent grâce, et, reconnaissant la vérité de ses reproches, ils le conjurent de punir le crime, de pardonner à l’erreur, et de les mener à l’ennemi : « Que César rappelle son épouse ; que le nourrisson des légions revienne, et ne soit pas livré en otage aux Gaulois. » Germanicus répondit que l’hiver et une grossesse trop avancée s’opposaient au retour d’Agrippine ; que son fils reviendrait ; que c’était aux soldats de faire le reste. À ces mots, devenus d’autres hommes, ils courent arrêter les plus séditieux, et les traînent enchaînés devant C. Cétronius, lieutenant de la première légion, qui en fit justice de cette manière. Les légions se tenaient, l’épée nue, autour du tribunal. On y plaçait le prévenu, et un tribun le montrait à l’assemblée. Si le cri général le déclarait coupable, il était jeté en bas et mis à mort. Le soldat versait ce sang avec plaisir, croyant par là s’absoudre lui-même. Germanicus laissait faire : comme il n’avait donné aucun ordre, l’excès de ces cruautés retombait sur leurs auteurs. Les vétérans suivirent cet exemple, et furent bientôt envoyés en Rhétie, sous prétexte de défendre cette province, menacée par les Suèves ; on voulait, au fond, les arracher d’un camp où la violence du remède, autant que le souvenir du crime, entretenait de sinistres pensées. On fit ensuite la revue des centurions : chacun d’eux, appelé par le général, déclarait son nom, sa centurie, son pays, ses années de service, ses faits d’armes et les récompenses militaires qu’il pouvait avoir reçues. Ceux dont les tribuns et la légion attestaient le mérite et la probité conservaient leur grade. Tout centurion qu’une voix unanime accusait de cruauté ou d’avarice était renvoyé de l’armée.
\subsection[{Révolte de la 5e et 21e légions}]{Révolte de la 5e et 21e légions}
\noindent \labelchar{XLV.} Le calme rétabli de ce côté, restait un autre péril, aussi grand que le premier, dans l’obstination de la cinquième et de la vingt et unième légions, en quartier d’hiver à soixante milles de distance, au lieu nommé Vétéra. C’était par elles qu’avait commencé la révolte, par leurs mains qu’avaient été commis les plus coupables excès. Ni l’effrayante punition ni le mémorable repentir de leurs compagnons ne désarmaient leur colère. Germanicus se prépare donc à descendre le Rhin avec une flotte chargée d’armes et de troupes alliées, résolu, si l’on bravait son autorité, de recourir à la force.
\subsection[{Réactions à Rome : hésitations de Tibère}]{Réactions à Rome : hésitations de Tibère}
\noindent \labelchar{XLVI.} À Rome, on ne savait pas encore l’issue des troubles d’Illyrie, quand on en apprit le soulèvement des légions germaniques. La ville alarmée se plaint hautement de ce que « Tibère s’amuse à jouer par ses feintes irrésolutions un peuple sans armes et un sénat sans pouvoir, tandis que le soldat se révolte, et certes ne sera pas réduit à l’obéissance par la jeune autorité de deux enfants. Ne devait-il pas se montrer lui-même, et opposer la majesté impériale à des rebelles dont la fureur tomberait devant un prince fort de sa longue expérience et arbitre souverain des châtiments et des grâces ? Auguste, chargé d’années, avait tant de fois visité la Germanie, et Tibère, dans la vigueur de l’âge, ne savait que rester au sénat pour y tourner en crime les paroles des sénateurs ! On n’avait que trop pourvu à l’esclavage de Rome ; c’était l’esprit des soldats qu’il s’agissait de calmer, afin de leur apprendre à supporter la paix. »\par
\labelchar{XLVII.} Peu touché de ces murmures, Tibère fut inébranlable dans la résolution de ne point quitter la capitale de l’empire, et de ne pas mettre au hasard le sort de la République et le sien. Il était combattu de mille pensées diverses. « L’armée de Germanie était plus puissante, celle de Pannonie plus voisine ; la première s’appuyait sur toutes les forces de la Gaule, la seconde menaçait l’Italie. Laquelle visiter de préférence, sans faire à l’autre un affront dont elle s’indignerait ? Mais il pouvait par ses fils les visiter toutes les deux à la fois, sans commettre la majesté suprême, qui de loin impose plus de respect. On excuserait d’ailleurs les jeunes Césars de renvoyer quelque chose à la décision de leur père ; et, si les rebelles résistaient à Germanicus ou à Drusus, lui-même pourrait encore les apaiser ou les réduire ; mais quelle ressource, s’ils avaient une fois bravé l’empereur ? » Au reste, comme s’il eût dû partir à chaque instant, il nomma sa suite, fit rassembler des bagages, équiper des vaisseaux ; puis, prétextant un jour la saison, un autre les affaires, il tint dans l’erreur d’abord jusqu’aux plus clairvoyants, ensuite la multitude, et très longtemps les provinces.
\subsection[{Germanicus décide d’attaquer les rebelles}]{Germanicus décide d’attaquer les rebelles}
\noindent \labelchar{XLVIII.} Cependant Germanicus avait déjà réuni son armée, et tout prêt pour le châtiment des rebelles. Voulant toutefois leur donner le temps d’imiter un exemple récent et de prendre eux-mêmes leur parti, il écrit à Cécina qu’il arrive en force, et que, si l’on ne prévient pas sa justice par la punition des coupables, le fer n’épargnera personne. Cécina lit secrètement cette lettre aux porte-enseigne des légions et des cohortes, et à la plus saine partie des soldats. Il les exhorte à sauver l’armée de l’infamie, à se sauver eux-mêmes de la mort : « car, en paix, chacun est traité selon son mérite et ses œuvres ; une foi la guerre allumée, l’innocent périt avec le criminel. » Ceux-ci sondent adroitement les esprits, et, s’étant assurés de la fidélité du plus grand nombre, ils fixent un jour avec le lieutenant, pour tomber l’épée à la main, sur ce qu’il y avait de plus pervers et de plus séditieux. Au signal convenu, ils se jettent dans les tentes, égorgent sans qu’on ait le temps de se reconnaître, et sans que personne, excepté ceux qui étaient dans le secret, sache comment le massacre a commencé, ni quand il finira.
\subsection[{Carnage entre soldats romains}]{Carnage entre soldats romains}
\noindent \labelchar{XLIX.} Ce fut un spectacle tel que nulle autre guerre civile n’en offrit de pareil. Les combattants ne s’avancent point, de deux camps opposés, sur un champ de bataille : c’est au sortir des mêmes lits, après avoir mangé la veille aux mêmes tables, goûté ensemble le repos de la nuit, qu’ils se divisent et s’attaquent. Les traits volent, on entend les cris, on voit le sang et les blessures ; la cause, on l’ignore. Le hasard conduisit le reste ; et quelques soldats fidèles périront comme les autres, quand les coupables, comprenant à qui l’on faisait la guerre, eurent pris aussi les armes. Ni lieutenant, ni tribuns n’intervinrent pour modérer le carnage : la vengeance fut laissée à la discrétion du soldat, et n’eut de mesure que la satiété. Peu de temps après, Germanicus entre dans le camp, et, les larmes aux yeux, comparant un si cruel remède à une bataille perdue, il ordonne qu’on brûle les morts. Bientôt ces courages encore frémissants sont saisis du désir de marcher à l’ennemi pour expier de si tristes fureurs, et ne voient d’autre moyen d’apaiser les mânes de leurs compagnons que d’offrir à de glorieuses blessures des cœurs sacrilèges. Germanicus répondit à leur ardeur : il jette un pont sur le Rhin, passe le fleuve avec douze milles légionnaires, vingt-six cohortes alliées, et huit ailes \footnote{Les \emph{ailes} de cavalerie étaient généralement composées de provinciaux et d’étrangers. Le nombre d’hommes variait de 500 à 1000. Elles étaient divisées en \emph{turmes} de trente hommes, et chaque turme en trois \emph{décuries}.} de cavalerie, qui, pendant la sédition, étaient restées soumises et irréprochables.
\subsection[{Lutte contre les Germains – Passage du Rhin}]{Lutte contre les Germains – Passage du Rhin}
\noindent \labelchar{L.} Joyeux et rapprochés de nos frontières, les Germains triomphaient de l’inaction où nous avait retenus d’abord le deuil d’Auguste, ensuite la discorde. L’armée romaine, après une marche rapide, perce la forêt de Caesia \footnote{Celle qu’on appelle aujourd’hui Heserwald, dans le duché de Clèves.}, ouvre le rempart construit par Tibère \footnote{Dans les pays où l’empire n’était point défendu par des fleuves ou des montagnes, les Romains élevaient entre eux et les barbares une barrière factice : c’était un rempart immense, garni de palissades, qui s’étendait d’un poste militaire à l’autre et régnait tout le long de la frontière.}, et campe sur ce rempart même, couverte en avant et en arrière par des retranchements, sur les deux flancs par des abatis d’arbres. Ensuite elle traverse des bois épais. On délibéra si, de deux chemins, on prendrait le plus court et le plus fréquenté ou l’autre plus difficile, non frayé, et que pour cette raison l’ennemi ne surveillait point. On choisit la route la plus longue, mais on redoubla de vitesse ; car nos éclaireurs avaient rapporté que la nuit suivante était une fête chez les Germains, et qu’ils la célébraient par des festins solennels. Cécina eut l’ordre de s’avancer le premier avec les cohortes sans bagages, et d’écarter les obstacles qu’il trouverait dans la forêt ; les légions suivaient à quelque distance. Une nuit éclairée par les astres favorisa la marche. On arrive au village des Marses, et on les investit. Les barbares étaient encore étendus sur leurs lits ou près des tables, sans la moindre inquiétude, sans gardes qui veillassent pour eux : tant leur négligence laissait tout à l’abandon. Ils ne songeaient point à la guerre, et leur sécurité même était moins celle de la paix que le désordre et l’affaissement de l’ivresse.
\subsection[{Massacres}]{Massacres}
\noindent \labelchar{LI.} César, pour donner à ses légions impatientes plus de pays à ravager, les partage en quatre colonnes. Il porte le fer et la flamme sur un espace de cinquante milles. Ni l’âge ni le sexe ne trouvent de pitié ; le sacré n’est pas plus épargné que le profane, et le temple le plus célèbre de ces contrées, celui de Tanfana, est entièrement détruit. Nos soldats revinrent sans blessures ; ils n’avaient qu’à égorger des hommes à moitié endormis, désarmés ou épars. Ce massacre réveilla les Bructères, les Tubantes, les Usipiens ; ils se postèrent dans les bois par où l’armée devait faire sa retraite. Le général en fut instruit, et disposa tout pour la marche et le combat. Une partie de la cavalerie et les cohortes auxiliaires formaient l’avant-garde ; venait ensuite la première légion ; au centre étaient les bagages ; la vingt et unième légion occupait le flanc gauche, la cinquième le flanc droit ; la vingtième protégeait les derrières, suivie du reste des alliés. Les ennemis attendirent tranquillement que toute la longueur de la colonne fût engagée dans les défilés. Alors, faisant sur le front et les ailes de légères attaques, ils tombent de toute leur force sur l’arrière-garde. Les bataillons serrés des Germains commençaient à porter le désordre dans nos cohortes légèrement armées, lorsque César accourut vers la vingtième légion et lui cria d’une voix forte « que le temps était venu d’effacer la mémoire de la sédition ; qu’elle marchât donc, et qu’elle se hâtât de changer en gloire le blâme qu’elle avait mérité. » Les courages s’enflamment : l’ennemi, enfoncé d’un choc, est rejeté dans la plaine et taillé en pièces. Au même moment la tête de l’armée, sortie du bois, se retranchait déjà. Le retour s’acheva paisiblement. Fier du présent, oubliant le passé, le soldat rentre dans ses quartiers d’hiver.
\subsection[{À Rome – Joie et inquiétude}]{À Rome – Joie et inquiétude}
\noindent \labelchar{LII.} Ces nouvelles causèrent à l’empereur de la joie et de l’inquiétude. Il voyait avec plaisir la révolte étouffée ; mais la faveur des soldats, que Germanicus avait acquise en avançant les congés et en distribuant les gratifications, et aussi la gloire militaire de ce général, lui donnaient de l’ombrage. Cependant il rendit compte au sénat de ses services, et fit de son courage un grand éloge, mais en termes trop pompeux pour qu’on le crût sincère. Il loua Drusus et l’ordre rétabli dans l’Illyrie en moins de mots, mais avec plus de chaleur et de franchise. Il ratifia toutes les concessions de Germanicus, et les étendit aux armées de Pannonie.
\subsection[{Mort de Julie et de son amant Sempronius Gracchus}]{Mort de Julie et de son amant Sempronius Gracchus}
\noindent \labelchar{LIII.} Cette même année mourut Julie, fille d’Auguste que son père avait confinée jadis, à cause de ses désordres, dans l’île de Pandatère \footnote{Voisine de la Campanie.}, ensuite à Rhégium, sur le détroit de Sicile. Mariée à Tibère dans le temps où florissaient les Césars Caius et Lucius, elle avait trouvé cette alliance inégale ; et, au fond, nulle cause n’influa autant que ses mépris sur la retraite de Tibère à Rhodes. Bannie, déshonorée, privée, par le meurtre d’Agrippa Postumus, de sa dernière espérance, elle survécut peu à l’avènement de ce prince : il la fit périr lentement de faim et de misère, persuadé qu’à la suite d’un si long exil \footnote{Il y avait quinze ans que Julie était reléguée, et le peuple, qui d’abord s’était fort intéressé à elle, avait eu le temps de l’oublier.} sa mort passerait inaperçue. De semblables motifs armèrent sa cruauté contre Sempronius Gracchus. Cet homme, d’une haute naissance, d’un esprit délié, doué d’une éloquence dont il usait pour le mal, avait séduit cette même Julie, quand elle était femme de M. Agrippa. Et l’adultère ne cessa pas avec cette union. Son amour obstiné la suivit dans la maison de Tibère, et il aigrissait contre ce nouvel époux son orgueil et sa haine. Une lettre injurieuse pour Tibère, qu’elle écrivit à Auguste, fut même regardée comme l’ouvrage de Gracchus. Relégué en conséquence dans l’île de Cercine, sur les côtes d’Afrique, il y endurait depuis quatorze ans les rigueurs de l’exil. Les soldats envoyés pour le tuer le trouvèrent sur une pointe du rivage, n’attendant rien moins qu’une bonne nouvelle. À leur arrivée, il demanda quelques instants pour écrire ses dernières volontés à sa femme Alliaria. Ensuite il présenta sa tête aux meurtriers et reçut la mort avec un courage digne du nom de Sempronius, qu’il avait démenti par sa vie. Quelques-uns rapportent que ces soldats ne vinrent point de Rome, mais que le proconsul Asprénas les envoya d’Afrique, par ordre de Tibère, qui s’était flatté vainement de faire retomber sur Asprénas l’odieux de ce meurtre.
\subsection[{Création du collège des prêtres d’Auguste}]{Création du collège des prêtres d’Auguste}
\noindent \labelchar{LIV.} On créa, la même année, une nouvelle institution religieuse, le collège des prêtres d’Auguste, comme jadis Titus Tatius, pour conserver le culte des Sabins, avait créé les prêtres Titiens. À vingt et un membres tirés au sort parmi les principaux de Rome, on ajouta Tibère, Drusus, Claude et Germanicus. Les jeux Augustaux furent troublés par le premier désordre auquel aient donné lieu les rivalités des histrions. Auguste avait toléré cette espèce d’acteurs afin de complaire Mécène ; qui était passionné pour Bathylle. Lui-même ne haïssait pas les amusements de ce genre, et il croyait se montrer ami du peuple en partageant ses plaisirs. Un autre esprit dirigeait Tibère : toutefois il n’osait pas encore imposer de privations à des hommes accoutumés depuis tant d’années à un régime plus doux.
\subsection[{Arminius – Ségeste et Arminius}]{Arminius – Ségeste et Arminius}
\noindent \labelchar{LV.} Sous le consulat de Drusus César et de C. Norbanus, le triomphe fut décerné à Germanicus, quoique la guerre durât encore. Il se disposait à la pousser vigoureusement pendant l’été ; ce qui n’empêcha pas que, dès les premiers jours du printemps, il ne fît par avance une soudaine incursion chez les Chattes. Il comptait sur les divisions de l’ennemi, partagé entre Ségeste et Arminius, qui avaient tous deux signalé envers nous, l’un sa fidélité, l’autre sa perfidie. Arminius soufflait la révolte parmi les Germains : Ségeste en avait plus d’une fois dénoncé les apprêts. Même au dernier festin, après lequel on courut aux armes, il avait conseillé à Varus de s’emparer de lui Ségeste, d’Arminius et des principaux nobles : « La multitude n’oserait rien, privée de ses chefs ; et le général pourrait à loisir discerner l’innocent du coupable. » Mais Varus subit sa destinée, et tomba sous les coups d’Arminius. Ségeste, entraîné à la guerre par le soulèvement général du pays, n’en garda pas moins ses dissentiments, et des motifs personnels achevaient de l’aigrir. Sa fille, promise à un autre, avait été enlevée par Arminius, gendre odieux, qui avait son ennemi pour beau-père ; et ce qui resserre, quand on est d’intelligence, les nœuds de l’amitié, n’était pour ces cœurs divisés par la haine qu’un aiguillon de colère.
\subsection[{Cécina chez les Chattes}]{Cécina chez les Chattes}
\noindent \labelchar{LVI.} Germanicus donne à Cécina quatre légions, cinq mille auxiliaires et les milices levées à la hâte parmi les Germains qui habitent en deçà du Rhin. Il prend avec lui le même nombre de légions et le double de troupes alliées, relève sur le mont Taunus \footnote{Selon Malte-Brun, le mont Taunus est situé au nord de Francfort, et se nomme aujourd’hui \emph{die Hoehe} (la hauteur).} un fort que son père y avait jadis établi, et fond avec son armée sans bagages sur le pays des Chattes, laissant derrière lui L. Apronius, chargé d’entretenir les routes et les digues. Une sécheresse, rare dans ces climats, et le peu de hauteur des rivières, lui avaient permis d’avancer sans obstacles ; mais on craignait pour le retour les pluies et la crue des eaux. Son arrivée chez les Chattes fut si imprévue, que tout ce que l’âge et le sexe rendaient incapable de résistance fut pris ou tué dans un instant. Les guerriers avaient traversé l’Éder à la nage et voulaient empêcher les Romains d’y jeter un pont. Repoussés par nos machines et nos flèches, ayant essayé vainement d’entrer en négociation, quelques-uns passèrent du côté de Germanicus ; les autres, abandonnant leurs bourgades et leurs villages, se dispersèrent dans les forêts. César, après avoir brûlé Mattium, chef-lieu de cette nation, et ravagé le plat pays, tourna vers le Rhin. L’ennemi n’osa inquiéter la retraite, comme le font ces peuples lorsqu’ils ont cédé le terrain par ruse plutôt que par crainte. Les Chérusques avaient eu l’intention de secourir les Chattes ; mais Cécina leur fit peur en promenant ses armes par tout le pays. Les Marses eurent l’audace de combattre : une victoire les réprima.
\subsection[{Ségeste demande l’aide des Romains}]{Ségeste demande l’aide des Romains}
\noindent \labelchar{LVII.} Bientôt après, une députation de Ségeste vint implorer notre secours contre sa nation, qui le tenait assiégé. L’influence d’Arminius était alors la plus forte : il conseillait la guerre, et, chez les barbares, l’audace est un titre à la confiance ; son importance s’accroît des troubles qu’elle a suscités. Ségeste avait adjoint aux députés Ségimond son fils ; mais une conscience inquiète arrêtait le jeune homme : l’année où la Germanie se révolta, nommé prêtre à l’Autel des Ubiens, il arracha les bandelettes sacrées et s’enfuit aux rebelles. Rassuré toutefois par la clémence romaine, dont on flatta son espoir, il apporta le message de son père, reçut un bon accueil et fut envoyé avec une escorte sur la rive gauloise. L’occasion méritait que Germanicus retournât sur ses pas : on attaqua les assiégeants, et Ségeste fut enlevé de leurs mains avec une troupe nombreuse de ses clients et de ses proches. Dans ce nombre étaient de nobles femmes, parmi lesquelles l’épouse d’Arminius, fille de Ségeste, plus semblable par la fierté de son cœur à son mari qu’à son père, sans s’abaisser aux larmes, sans prononcer une parole suppliante, marchait les mains croisées sur sa poitrine, les yeux attachés sur le sein où elle portait un fils d’Arminius, Venaient ensuite les dépouilles de l’armée de Varus, échues dans le partage du butin à la plupart de ceux qui se remettaient alors en nos mains. Enfin Ségeste se reconnaissait à la hauteur de sa taille et à l’air assuré que lui donnait le souvenir d’une alliance fidèlement gardée. Voici comment il s’exprima :\par
\labelchar{LVIII.} « Cette journée n’est pas la première qui ait signalé ma fidélité et mon dévouement pour le peuple romain. Depuis que la faveur d’Auguste m’a mis au nombre de vos citoyens, j’ai toujours consulté dans le choix de mes amis et de mes ennemis le bien de votre empire : et je ne l’ai pas fait en haine de ma patrie (les traîtres sont odieux à ceux mêmes qu’ils servent) ; mais Rome et la Germanie me semblaient avoir les mêmes intérêts, et j’ai pensé que la paix valait mieux que la guerre. Aussi ai-je dénoncé à Varus, alors chef de vos légions, le ravisseur de ma fille, l’infracteur de vos traités, le perfide Arminius. Réduit, par les lenteurs de votre général, à ne plus rien espérer des lois, je le conjurai de nous saisir tous, Arminius, ses complices et moi-même : j’en atteste cette nuit fatale ; eh ! que n’a-t-elle été la dernière de mes nuits ! Déplorer les faits qui suivirent est plus facile que de les justifier. Du reste, Arminius a porté mes fers ; vaincu par sa faction, j’ai porté les siens. Enfin il nous est donné de vous voir, et aussitôt je renonce aux nouveautés pour l’ordre ancien, au trouble pour le repos. Puisse ce retour, entièrement désintéressé, m’absoudre du reproche de perfidie, et donner aux Germains un utile médiateur, s’ils aiment mieux se repentir que de se perdre ! Je demande grâce pour la jeunesse et l’erreur de mon fils. Je conviens que ma fille est conduite en ces lieux par la nécessité ; c’est à vous de juger si vous verrez en elle l’épouse d’Arminius ou la fille de Ségeste. » Germanicus lui répondit avec douceur, promettant sûreté à ses enfants et à ses proches, et à lui-même un établissement dans une de nos anciennes provinces. Il ramena son armée et reçut, de l’aveu de Tibère, le titre d’Imperator. La femme d Arminius mit au monde un fils qui fut élevé à Ravenne. Je dirai plus tard quelles vicissitudes tourmentèrent la destinée de cet enfant \footnote{La partie des \emph{Annales} où Tacite parle du fils d’Arminius est perdue.}.
\subsection[{Réaction d’Arminius}]{Réaction d’Arminius}
\noindent \labelchar{LIX.} La soumission de Ségeste et l’accueil fait à ce chef excitèrent chez les barbares l’espérance ou la douleur, selon que chacun redoutait ou désirait la guerre. Arminius, livré aux transports d’une violence que redoublaient encore son épouse enlevée et son enfant captif dès le sein maternel, parcourait le pays des Chérusques, demandant vengeance contre Ségeste, vengeance contre César. Sa fureur ne ménageait pas les invectives : « Quel tendre père ! Quel grand capitaine ! Quelle intrépide armée ! Tant de bras réunis pour emmener une femme ! Lui du moins, c’étaient trois légions, autant de généraux qu’il avait exterminés. Car ses ennemis n’étaient point des femmes enceintes, ni ses armes de lâches trahisons : il faisait une guerre ouverte à des hommes de guerre. Les enseignes romaines, consacrées par sa main aux dieux de la patrie, pendaient encore aux arbres des forêts germaniques. Ségeste pouvait habiter la rive des vaincus et rendre à son fils un vil sacerdoce : jamais de véritables Germains ne se croiraient absous d’avoir vu entre l’Elbe et le Rhin les verges, les haches et la toge. Heureuses les nations qui ne connaissaient point la domination romaine ! Elles n’avaient pas enduré les supplices, gémi sous les tributs. Puisque les Germains s’en étaient affranchis, et avaient renversé les projets de cet Auguste dont Rome a fait un dieu, de ce Tibère, dont elle a fait son maître, craindraient-ils un enfant dénué d’expérience et une armée de séditieux ? S’ils préféraient leur patrie, leurs parents à des tyrans, leur antique indépendance à ces colonies inconnues de leurs pères ; qu’ils suivissent Arminius dans le chemin de la gloire et de la liberté, plutôt que Ségeste, qui les menait à l’opprobre et à la servitude. »
\subsection[{Sur les traces des légions de Varus}]{Sur les traces des légions de Varus}
\noindent \labelchar{LX.} Il souleva par ces discours non seulement les Chérusques, mais encore les nations voisines, et entraîna dans la ligue son oncle Inguiomère, nom depuis longtemps estimé des Romains : César vit ce nouveau péril. Pour empêcher que tout le poids de la guerre ne pesât sur un seul point, et afin de diviser les forces de l’ennemi, il envoya Cécina vers l’Ems, par le pays des Bructères, avec quarante cohortes romaines. Le préfet Pédo conduisit la cavalerie par les confins de la Frise ; Germanicus lui-même s’embarqua sur les lacs \footnote{Les lacs de Batavie, dont la réunion, opérée par le temps et les invasions de la mer, a formé le Zuiderzee.} avec quatre légions ; et bientôt l’infanterie, la cavalerie et la flotte, se trouvèrent réunies sur le fleuve marqué pour rendez-vous. Les Chauques offrirent des secours et furent admis sous nos drapeaux. Les Bructères mettaient en cendres leur propre pays. L. Stertinius, envoyé par César avec une troupe légèrement équipée, les battit ; et, en continuant de tuer et de piller, il retrouva l’aigle de la dix-neuvième légion, perdue avec Varus. Ensuite l’armée s’avança jusqu’aux dernières limites des Bructères, et tout fut ravagé entre l’Ems et la Lippe, non loin de la forêt de Teutberg \footnote{Dans le voisinage de la petite ville de Horn, en Westphalie.}, où, disait-on, gisaient sans sépulture les restes de Varus et de ses légions.
\subsection[{Emotion}]{Emotion}
\noindent \labelchar{LXI.} César éprouva le désir de rendre les derniers honneurs au chef et aux soldats ; et tous les guerriers présents furent saisis d’une émotion douloureuse en songeant à leurs proches, à leurs amis, aux chances de la guerre et à la destinée des humains. Cécina est envoyé en avant pour sonder les profondeurs de la forêt, et construire des ponts ou des chaussées sur les marécages et les terrains d’une solidité trompeuse ; puis l’on pénètre dans ces lieux pleins d’images sinistres et de lugubres souvenirs. Le premier camp de Parus, à sa vaste enceinte, aux dimensions de sa place d’armes, annonçait l’ouvrage de trois légions. Plus loin un retranchement à demi ruiné, un fossé peu profond, indiquaient l’endroit où s’étaient ralliés leurs faibles débris. Au milieu de la plaine, des ossements blanchis ; épars ou amoncelés, suivant qu’on avait fui ou combattu, jonchaient la terre pêle-mêle avec des membres de chevaux et des armes brisées. Des têtes humaines pendaient au tronc des arbres ; et l’on voyait, dans les bois voisins, les autels barbares où furent immolés les tribuns et les principaux centurions. Quelques soldats échappés à ce carnage ou qui depuis avaient brisé leurs fers, montraient la place où périrent les lieutenants, où les aigles furent enlevées. « Ici Varus reçut une première blessure ; là son bras malheureux, tourné contre lui-même, le délivra de la vie. » Ils disaient « sur quel tribunal Arminius harangua son armée, combien il dressa de gibets, fit creuser de fosses pour les prisonniers ; par quelles insultes son orgueil outragea les enseignes et les aigles romaines. »
\subsection[{Hommage aux morts}]{Hommage aux morts}
\noindent \labelchar{LXII.} Ainsi les soldats présents sur le théâtre du désastre recueillaient, après six ans, les ossements de trois légions ; et, sans savoir s’ils couvraient de terre la dépouille d’un proche ou d’un étranger, animés contre l’ennemi d’une colère nouvelle, et la vengeance dans le cœur aussi bien que la tristesse, ils ensevelissaient tous ces restes comme ceux d’un parent ou d’un frère. On éleva un tombeau, dont César posa le premier gazon ; pieux devoir par lequel il honorait les morts et s’associait à la douleur des vivants. Toutes ces choses ne furent point approuvées de Tibère ; soit que Germanicus ne pût rien faire qu’il n’y trouvât du crime ; soit que l’image de tant de guerriers massacrés et privés de sépulture lui parût capable de refroidir l’armée pour les combats, et de lui inspirer la crainte de l’ennemi ; soit enfin qu’il pensât qu’un général, consacré par l’augurat et les rites les plus antiques, ne devait approcher ses mains d’aucun objet funèbre.
\subsection[{Sus à Arminius !}]{Sus à Arminius !}
\noindent \labelchar{LXIII.} Cependant Arminius s’enfonçait dans des lieux impraticables. Germanicus l’y suivit, et, dès qu’il put le joindre, il détacha sa cavalerie avec ordre d’enlever aux barbares une plaine qu’ils occupaient. Arminius se replie d’abord et se rapproche des forêts ; puis il fait tout à coup volte-face, et ordonne à ceux qu’il avait cachés dans les bois de s’élancer en avant. Cette nouvelle armée jette le trouble parmi les cavaliers ; des cohortes envoyées pour les soutenir sont entraînées dans leur fuite et augmentent le désordre. Elles allaient être poussées dans un marais connu du vainqueur, dangereux pour des étrangers, si Germanicus n’eût fait avancer ses légions en bataille. Ce mouvement porta la terreur chez l’ennemi, la confiance chez les nôtres, et l’on se sépara sans avantage décidé. Bientôt Germanicus ramena son armée vers l’Ems, et fit embarquer les légions sur la flotte. Une partie de la cavalerie eut ordre de regagner le Rhin en côtoyant l’Océan. Cécina marchait séparément ; et, quoiqu’il revînt par des routes connues, Germanicus lui conseilla de passer au plus tôt les Longs Ponts. On appelle ainsi une chaussée étroite, construite autrefois dans de vastes marais par L. Domitius. Des deux côtés on ne trouve qu’un limon fangeux, une vase épaisse, entrecoupée de ruisseaux. Tout autour, des bois s’élèvent en pente douce. Arminius les remplit de troupes ; il avait, par des chemins plus courts et une marche rapide, devancé nos soldats chargés d’armes et de bagages. Cécina, incertain comment il pourrait à la fois rétablir les ponts, ruinés par le temps, et repousser l’ennemi, résolut de camper en ce lieu et d’employer une partie de son armée au travail et l’autre au combat.
\subsection[{Avantage aux barbares}]{Avantage aux barbares}
\noindent \labelchar{LXIV.} Les barbares, essayant de forcer nos postes pour tomber sur les travailleurs, attaquent de front, en flanc, de tous les côtés ; les cris des ouvriers se mêlent aux cris des combattants. Tout se réunissait contre les Romains, une fange profonde et glissante, où le pied ne pouvait ni se tenir ni avancer, la pesanteur de leurs cuirasses, la difficulté de lancer les javelines au milieu des eaux. Les Chérusques avaient pour eux l’habitude de combattre dans les marais, une haute stature et la distance où atteignaient leurs longues piques. Nos légions commençaient à plier, quand la nuit vint les soustraire à un combat inégal. Le succès rendait les Germains infatigables : au lieu de prendre du repos, ils détournent toutes les eaux qui coulent des hauteurs environnantes, les versent dans la vallée, et, en noyant les ouvrages faits, doublent le travail du soldat. C’était la quarantième année que Cécina passait dans les camps, soit à obéir, soit à commander : l’expérience de la bonne et de la mauvaise fortune l’avait aguerri contre la crainte. Après avoir calculé toutes les chances, il ne trouva d’autre parti à prendre que de contenir l’ennemi dans les bois, tandis qu’il ferait passer d’abord les blessés et les bagages. Entre les collines et les marais s’allongeait une plaine étroite, où l’on pouvait ranger une armée sur peu de profondeur. Il choisit la cinquième légion pour former la droite ; il donne la gauche à la vingt et unième ; la première devait conduire la marche, et la vingtième la protéger par derrière.
\subsection[{L’horreur !}]{L’horreur !}
\noindent \labelchar{LXV.} La nuit fut sans repos des deux côtés ; mais les festins joyeux des barbares, leurs chants d’allégresse, leurs cris effrayants répercutés par l’écho des vallées et des bois, et, chez les Romains, des feux languissants, des soldats couchés auprès des palissades ou errant le long des tentes, moins occupés de veiller qu’incapables de dormir, faisaient un étrange contraste. Un songe affreux épouvanta le général : Quintilius Varus, tout couvert de sang, lui parut se lever du fond de ces marais ; il crut entendre, sans toutefois y obéir, sa voix qui l’appelait, et repousser sa main étendue vers lui. Au retour de la lumière, les légions envoyées sur les ailes, soit crainte, soit esprit de révolte, quittèrent leur poste et gagnèrent à la hâte un champ situé au-delà du marais. Arminius pouvait charger sans obstacle : il ne le fit point. Mais quand il vit les bagages embarrassés dans la fange et dans les fossés, et, tout autour, les soldats en désordre, les enseignes confondues, il profita de ce moment où chaque homme, tout entier au soin de sa conservation, n’entend plus la voix des chefs, pour donner aux Germains le signal de l’attaque : « Voilà Varus, s’écrie-t-il, voilà ses légions que leur fatalité nous livre une seconde fois. » Il dit ; et, avec l’élite de ses guerriers, il rompt notre ligne, et s’attache surtout à blesser les chevaux. Le pied manquait à ces animaux sur une terre glissante et mouillée de leur sang : ils renversent leurs cavaliers, dispersent tout devant eux, écrasent tout sur leurs pas. Les plus laborieux efforts se firent autour des aigles, qu’on ne pouvait ni porter à travers une grêle de traits, ni planter sur un sol fangeux. Cécina, en soutenant le courage des siens, eut son cheval tué sous lui. Il tomba et allait être enveloppé, sans la résistance de la première légion. L’avidité de l’ennemi, qui laissa le carnage pour courir au butin, permit aux légions d’atteindre, vers le soir, un terrain découvert et solide. Mais la fin de leurs maux n’était pas venue : il fallait élever des retranchements et en amasser les matériaux. Les instruments propres à remuer les terres et à couper le gazon étaient perdus en grande partie. On n’avait plus ni tentes pour les soldats, ni médicaments pour les blessés : pendant qu’on se partageait quelques vivres souillés de sang et de boue, l’horreur de cette nuit funeste, l’attente d’un lendemain qu’on croyait le dernier pour tant de milliers d’hommes, remplissaient le camp de lamentations.\par
\labelchar{LXVI.} Le hasard voulut qu’un cheval, ayant rompu ses liens et fuyant épouvanté par le bruit, renversât quelques hommes sur son passage. L’effroi devint général : on crut que les Germains avaient pénétré dans le camp ; et chacun se précipita vers les portes, principalement vers la décumane \footnote{Les camps romains étaient carrés et avaient une porte au milieu de chaque face. Celle qui était à la tête du camp, vis-à-vis de la tente du général, s’appelait la porte prétorienne : c’est par là que l’armée sortait pour la marche ou pour le combat. La décumane était du côté opposé : on la nommait ainsi, comme la plus voisine de la dixième cohorte de chaque légion.} qui étant du côté opposé à l’ennemi, paraissait la plus sûre pour la fuite. Cécina, qui avait reconnu que c’était une fausse alarme, essayait vainement d’arrêter les fuyards : ni ses ordres, ni ses prières, ni son bras, ne pouvaient les retenir. Enfin la pitié les retint : il se coucha en travers de la porte, et les soldats n’osèrent marcher sur le corps de leur général. En même temps les tribuns et les centurions les détrompèrent sur le sujet de leur frayeur.\par
\labelchar{LXVII.} Alors il les rassemble sur la place d’armes, et, après leur avoir ordonné de l’écouter en silence, il les avertit de ce qu’exigent le temps et la nécessité. « Ils n’ont de salut que dans les armes ; mais la prudence doit en régler l’usage : il faut rester dans le camp jusqu’à ce que les barbares, espérant le forcer, soient au pied des remparts ; alors ils sortiront de tous les côtés à la fois, et cette sortie les mène au Rhin. En fuyant, ils trouveraient de plus vastes forêts, des marais plus profonds, un ennemi féroce ; vainqueurs, la gloire et les distinctions les attendent. " Il invoque tour à tour les souvenirs de la famille et l’honneur militaire ; les revers, il n’en parle pas. Ensuite il fait amener les chevaux des lieutenants et des tribuns, en commençant par les siens ; et, sans rien considérer que le mérite, il les donne aux plus braves. Ceux-ci devaient charger d’abord, ensuite l’infanterie.
\subsection[{Déroute d’Arminius}]{Déroute d’Arminius}
\noindent \labelchar{LXVIII.} L’espoir, l’impatience, la lutte des opinions entre les chefs, ne tenaient pas les Germains dans une moindre agitation. Arminius voulait qu’on laissât partir les Romains, et que, pendant leur marche à travers des lieux difficiles et marécageux, on les enveloppât de nouveau. L’avis d’Inguiomère, plus violent et plus goûté des barbares, était de donner l’assaut. « La victoire serait prompte, les prisonniers plus nombreux, et l’on sauverait tout le butin. » Au lever du jour ils remplissent les fossés, jettent des claies, gravissent au haut des retranchements, où quelques soldats clairsemés semblaient immobiles de frayeur. Dès que Cécina les vit attachés à la palissade, il donna le signal aux cohortes. Clairons, trompettes, tout sonne à la fois ; bientôt un cri part, on s’élance et l’on enveloppe les Germains par derrière, en leur demandant où sont à présent leurs marais et leurs bois : « Ici tout est égal, le terrain et les dieux. » Les ennemis avaient cru trouver un pillage facile, une poignée d’hommes mal armés : le son des trompettes, l’éclat des armes, leur firent une impression de terreur d’autant plus profonde qu’elle était inattendue. Ils tombaient par milliers, aussi déconcertés dans la mauvaise fortune qu’impétueux dans la bonne. Les deux chefs abandonnèrent le combat, Arminius sain et sauf, Inguiomère grièvement blessé. On fit main basse sur la multitude, tant que dura la colère et le jour. La nuit ramena nos légions avec plus de blessures que la veille, et ne souffrant pas moins de la disette des vivres ; mais elles retrouvèrent tout dans la victoire, santé, vigueur, abondance.
\subsection[{Agrippine, « le général »}]{Agrippine, « le général »}
\noindent \labelchar{LXIX.} Cependant le bruit s’était répandu que l’armée avait été surprise, et que les Germains victorieux s’avançaient vers les Gaules ; et, si Agrippine n’eût empêché qu’on rompît le pont établi sur le Rhin, il se trouvait des lâches qui n’eussent pas reculé devant cette infamie. Mais cette femme courageuse remplit, pendant ces jours d’alarmes, les fonctions de général ; elle distribua des vêtements aux soldats pauvres, des secours aux blessés. Pline \footnote{Pline l’ancien.}, historien des guerres de Germanie, rapporte qu’elle se tint à la tête du pont, adressant aux légions, à mesure qu’elles passaient, des éloges et des remerciements. Ces actes furent profondément ressentis par Tibère. Selon lui, « tant de zèle n’était point désintéressé, et l’on enrôlait contre un autre ennemi que le barbare. Quel soin resterait donc aux empereurs, si une femme faisait la revue des cohortes, approchait des enseignes, essayait les largesses ? Comme si ce n’était pas assez se populariser que de promener en habit de soldat le fils d’un général, et de donner à un César le nom de Caligula ! Déjà le pouvoir d’Agrippine était plus grand sur les armées que celui des lieutenants, que celui des généraux : une femme avait étouffé une sédition contre laquelle le nom du prince avait été impuissant. " Séjan envenimait encore et aggravait ces reproches, semant, dans une âme qu’il connaissait à fond, des haines qui couveraient en silence, pour éclater quand l’orage serait assez grossi.
\subsection[{Tempête sur les rivages : deux légions en danger}]{Tempête sur les rivages : deux légions en danger}
\noindent \labelchar{LXX.} Cependant Germanicus, afin que sa flotte voguât plus légère parmi les bas-fonds ou s’échouât plus doucement à l’instant du reflux, débarqua la seconde et la quatorzième légions, et chargea Vitellius de les ramener par terre. Vitellius marcha d’abord sans obstacle sur une grève sèche ou à peine atteinte par la vague expirante. Bientôt, poussée par le vent du nord, une de ces marées d’équinoxe, où l’Océan s’élève à sa plus grande hauteur, vint assaillir et rompre nos bataillons. La terre se couvre au loin : mer, rivages, campagnes, tout présente un aspect uniforme. On ne distingue plus les fonds solides des sables mouvants, les gués des abîmes. Le soldat est renversé par la lame, noyé dans les gouffres, heurté par les chevaux, les bagages, les corps morts, qui flottent entre les rangs. Les manipules se confondent ; les hommes sont dans l’eau tantôt jusqu’à la poitrine, tantôt jusqu’au cou ; quelquefois, le sol manquant sous leurs pieds, ils sont engloutis ou dispersés. C’est en vain qu’ils s’encouragent de la voix et luttent contre les vagues. Le brave n’a aucun avantage sur le lâche, le sage sur l’imprudent, le conseil sur le hasard : tout est enveloppé dans l’inévitable tourmente. Enfin Vitellius parvint à gagner une éminence, où il rallia son armée. Ils y passèrent la nuit, sans provisions, sans feu, la plupart nus ou le corps tout meurtri, non moins à plaindre que des malheureux entourés par l’ennemi : ceux-là du moins ont la ressource d’un trépas honorable ; ici la mort était sans gloire. La terre repartit avec le jour, et l’on atteignit les bords de l’Hunsing \footnote{Rivière qui passe à Groningue.}, où Germanicus avait conduit sa flotte. Il y fit rembarquer les deux légions. Le bruit courait qu’elles avaient été submergées, et l’on ne crut à leur conservation qu’en voyant César et l’armée de retour.
\subsection[{Reddition de Ségimère}]{Reddition de Ségimère}
\noindent \labelchar{LXXI.} Déjà Stertinius, envoyé pour recevoir à discrétion Ségimére, frère de Ségeste, l’avait amené lui et son fils dans la cité des Ubiens. Tous deux obtinrent leur pardon, Ségimére facilement, son fils avec plus de peine : il avait, disait-on, insulté le cadavre de Varus. Au reste les Gaules, l’Espagne, l’Italie, rivalisèrent de zèle pour réparer les pertes de l’armée : chaque peuple offrit ce qu’il avait, des armes, des chevaux, de l’or. Germanicus loua leur empressement, et n’accepta que des hommes et des chevaux pour la guerre. Il secourut les soldats de sa bourse ; et, afin d’adoucir encore par ses manières affables le souvenir de leurs maux, il visitait les blessés, relevait leurs belles actions. En examinant les blessures, il encourageait celui-ci par l’espérance, celui-là par la gloire, tous par des paroles et des soins qui lui gagnaient les cœurs et les affermissaient pour l’heure des combats.
\subsection[{À Rome – Lois de lèse-majesté}]{À Rome – Lois de lèse-majesté}
\noindent \labelchar{LXXII.} On décerna cette année les ornements du triomphe \footnote{Le général honoré de cette distinction avait le droit de porter la robe triomphale à certains jours et dans certaines cérémonies ; et on lui érigeait une statue qui le représentait avec ce costume et couronné de laurier.} à Cécina, à L. Apronius et à C. Silius, pour la part qu’ils avaient eue aux succès de Germanicus. Tibère refusa le nom de Père de la patrie, dont le peuple s’obstinait à le saluer ; et, malgré l’avis du sénat, il ne permit pas qu’on jurât sur ses actes \footnote{Les triumvirs imaginèrent les premiers de jurer eux-mêmes et de faire jurer par les autres qu’ils regarderaient comme inviolables et sacrés les actes de Jules César. Ce serment eut lieu le 1er janvier 712. Le même jour de l’an 730, le sénat ratifia, par un serment pareil, tout ce qu’avait fait Auguste ; et l’usage s’établit de jurer ainsi, au renouvellement de l’année sur les actes de l’empereur régnant et de ses prédécesseurs.}, affectant de répéter « que rien n’est stable dans la vie, et que, plus on l’aurait placé haut, plus le poste serait glissant. » Et cependant cette fausse popularité n’en imposait à personne. Il avait remis en vigueur la loi de majesté ; loi qui chez les anciens, avec le même nom, embrassait des objets tout différents, trahisons à l’armée, séditions à Rome, atteinte portée par un magistrat prévaricateur à la majesté du peuple romain. On condamnait les actions, les paroles restaient impunies : Auguste le premier étendit cette loi aux libelles scandaleux, indigné de l’audace de Cassius Sévérus, dont les écrits insolents avaient diffamé des hommes et des femmes d’un rang illustre. Dans la suite Tibère, consulté, par le préteur Pompéius Macer, s’il fallait recevoir les accusations de lèse-majesté, répondit que les lois devaient être exécutées. Lui aussi avait été aigri par des vers anonymes qui coururent alors sur sa cruauté, son orgueil, et son aversion pour sa mère.\par
\labelchar{LXXIII.} Il ne sera pas inutile de rapporter ici quel essai fut tenté sur Falanius et Rubrius, simples chevaliers romains, de ces sortes d’accusations : on verra avec quelle adresse Tibère jeta au sein de la République les premiers germes d’un mal si funeste, et comment l’incendie, étouffé un instant, finit par éclater et par tout dévorer. L’accusateur reprochait à Falanius d’avoir reçu dans une de ces confréries que chaque maison réunissait alors pour le culte d’Auguste, un pantomime de mœurs infâmes, nommé Cassius, et d’avoir, en vendant ses jardins, livré en même temps la statue d’Auguste. Le crime imputé à Rubrius était d’avoir profané par un faux serment le nom de ce prince. Informé de ces accusations, Tibère écrivit aux consuls « que son père n’avait pas reçu l’apothéose pour la perte des citoyens ; que l’histrion Cassius avait coutume d’assister, avec d’autres hommes de sa profession, aux jeux que Livie célébrait en mémoire de son époux : qu’on pouvait, sans outrager la religion, comprendre la statue d’Auguste, comme celles des autres divinités, dans la vente des maisons et des jardins ; qu’à l’égard du parjure, il fallait le considérer comme si l’offense était faite à Jupiter, et laisser aux dieux le soin de venger les dieux. »\par
\labelchar{LXXIV.} Peu de temps après, Granius Marcellus, gouverneur de Bithynie, fut accusé de lèse-majesté par son propre questeur, Cépio Crispinus, auquel se joignit Romanus Hispo. Crispinus fut l’inventeur d’une industrie que le malheur des temps et l’effronterie des hommes mirent depuis fort en vogue. Pauvre, obscur, intrigant, il s’adressa d’abord, par des voies obliques et à l’aide de mémoires secrets, à la cruauté du prince. Bientôt il attaqua les plus grands noms ; et, puissant auprès d’un seul, abhorré de tous, il donna un exemple dont les imitateurs, devenus riches et redoutables d’indigents et méprisés qu’ils étaient, firent la perte d’autrui, et à la fin se perdirent eux-mêmes. Cépion reprochait à Marcellus d’avoir tenu sur Tibère des discours injurieux ; délation d’un succès infaillible : l’accusateur choisissait les traits les plus hideux de la vie du prince, et les mettait dans la bouche de l’accusé ; comme les faits étaient vrais, on croyait facilement aux paroles. Hispon ajouta « que la statue de Marcellus était placée plus haut que celles des Césars, et que, d’une autre statue, on avait ôté la tête d’Auguste pour y substituer celle de Tibère. » À ces mots Tibère éclate, et, sortant brusquement de son silence, il s’écrie « que, lui aussi, il donnera sa voix dans cette cause, et qu’il la donnera tout haut et avec serment. » C’était obliger les autres à en faire autant. Quelques accents restaient encore à la liberté mourante : « Apprends-nous, César, lui dit Cn. Piso, dans quel rang tu opineras. Si tu parles le premier, j’aurai sur qui me régler. Si tu ne parles qu’après nous, je crains d’être, sans le savoir, d’un autre avis que le tien. » Déconcerté par cette question, Tibère comprit qu’il s’était emporté trop loin, et, patient par repentir, il souffrit que Marcellus fût absous du crime de lèse-majesté. Restait celui de concussion, pour lequel on alla devant des récupérateurs \footnote{Commissaires donnés aux parties par le préteur ou, comme ici, par le sénat, pour estimer en argent une réparation d’injure ou une restitution de deniers.}.\par
\labelchar{LXXV.} Ce n’était pas assez pour Tibère des procédures sénatoriales : il assistait encore aux jugements ordinaires, assis dans un coin du tribunal, afin de ne pas déplacer le préteur de sa chaise curule ; et sa présence fit échouer, dans plus d’une affaire, les brigues et les sollicitations des grands ; mais, si cette influence profitait à la justice, c’était aux dépens de la liberté. Vers ce temps-là, le sénateur Pius Aurélius se plaignit que la construction d’un chemin et d’un aqueduc avait mis sa maison en danger de ruine, et recourut à la protection du sénat. Les préteurs de l’épargne \footnote{Auguste, en 726, chargea deux préteurs de l’administration du trésor public.} combattant sa demande, Tibère y pourvut et lui paya le prix de ses bâtiments. Ce prince aimait à faire un noble usage de ses trésors ; c’est une vertu qu’il conserva longtemps après avoir abjuré toutes les autres. Propertius Celer, ancien préteur, qui demandait à se retirer du sénat à cause de son indigence, reçut de sa générosité un million de sesterces \footnote{Cette somme, à la fin d’Auguste et au commencement de Tibère, équivalait à 198 798 F de notre monnaie.} ; c’était un fait connu que son père l’avait laissé sans fortune. D’autres aspirèrent aux mêmes faveurs : il leur enjoignit de faire approuver leurs motifs par le sénat ; tant l’esprit de sévérité rendait amer jusqu’au bien qu’il faisait ! Tous préférèrent la pauvreté et le silence à des bienfaits achetés par un pénible aveu.
\subsection[{Inondations – Combats de gladiateurs}]{Inondations – Combats de gladiateurs}
\noindent \labelchar{LXXVI.} Cette même année le Tibre, grossi par des pluies continuelles, avait inondé les parties basses de Rome, et entraîné, en se retirant, une grande quantité de ruines et de cadavres. Asinius Gallus voulait que l’on consultât les livres sibyllins : Tibère s’y opposa, aussi mystérieux en religion qu’en politique. Mais il fut décidé que L. Arruntius et Atéius Capito chercheraient les moyens de contenir le fleuve. L’Achaïe et la Macédoine imploraient une diminution des charges : on les délivra pour le moment du gouvernement proconsulaire, et on les remit aux mains de César. Drusus avait offert, au nom de Germanicus, son frère, et au sien, un combat de gladiateurs, il y présida et vit couler un sang, vil d’ailleurs, avec une joie trop marquée. Le peuple s’en alarma, et son père, dit-on, lui en fit des reproches. Celui-ci ne parut point à ce spectacle, et l’on interpréta diversement son absence. C’était, selon les uns, dégoût de réunions ; selon d’autres, tristesse d’humeur et crainte d’un fâcheux parallèle ; car Auguste se montrait à ces jeux de l’air le plus affable. Je ne puis croire qu’il eût voulu ménager à son fils l’occasion de mettre sa cruauté au grand jour et de s’aliéner les cœurs : toutefois cela fut dit aussi.
\subsection[{Mesures contre les désordres au théâtre}]{Mesures contre les désordres au théâtre}
\noindent \labelchar{LXXVII.} Les désordres du théâtre, qui avaient commencé l’année précédente, éclatèrent avec une nouvelle fureur. Des hommes furent tués parmi le peuple ; des soldats même et un centurion périrent, et un tribun prétorien fut blessé, en voulant apaiser le tumulte et faire respecter les magistrats. Un rapport fut fait au sénat sur cette sédition ; et l’on proposait de donner aux préteurs le droit de frapper de verges les histrions. Hatérius, tribun du peuple, s’y opposa et fut vivement combattu par Asinius Gallus, sans qu’il échappât un seul mot à Tibère : il aimait à laisser au sénat ces simulacres de liberté. Cependant l’opposition prévalut, parce qu’une ancienne décision d’Auguste mettait les histrions à l’abri des verges, et que les paroles d’Auguste étaient pour Tibère des lois inviolables. On fit plusieurs règlements pour borner le salaire des pantomimes et réprimer la licence de leurs partisans : les plus remarquables détendaient aux sénateurs d’entrer dans les maisons des pantomimes, aux chevaliers de leur faire cortège en public ; à eux-mêmes de donner des représentations ailleurs qu’au théâtre. Les préteurs furent autorisés à punir de l’exil tout spectateur qui troublerait l’ordre.
\subsection[{Constructions de temples pour Auguste – Refus de baisser les impôts}]{Constructions de temples pour Auguste – Refus de baisser les impôts}
\noindent \labelchar{LXXVIII.} La permission d’élever un temple à Auguste dans la colonie de Tarragone fut accordée aux Espagnols, et ce fut un exemple pour toutes les provinces. Le peuple demandait la suppression du centième imposé sur les ventes depuis les guerres civiles. Tibère déclara par un édit que ce revenu était la seule ressource du trésor militaire, et que même il ne suffirait pas, si la vétérance n’était reculée jusqu’à la vingtième année de service. Ainsi les concessions onéreuses arrachées par la dernière sédition, et qui fixaient le congé à seize ans, furent révoquées pour l’avenir.
\subsection[{Problème du débordement du Tibre}]{Problème du débordement du Tibre}
\noindent \labelchar{LXXIX.} Le sénat examina ensuite, sur le rapport d’Arruntius et d’Atéius, si, afin de prévenir les débordements du Tibre, on donnerait un autre écoulement aux lacs et aux rivières qui le grossissent. On entendit les députations des municipes et des colonies. Les Florentins demandaient en grâce que le Clanis ne fût pas détourné de son lit pour être rejeté dans l’Arno, ce qui causerait leur ruine. Ceux d’Intéramne \footnote{Terni, dans l’Ombrie, sur le Nar, aujourd’hui la Néra.} parlèrent dans le même sens : « On allait, disaient-ils, abîmer sous les eaux et changer en des marais stagnants les plus fertiles campagnes de l’Italie, si l’on ne renonçait pas au projet de diviser le Nar en petits ruisseaux. » Réate \footnote{Maintenant Riétin au pays des Sabins, près du lac Velinus.} ne se taisait pas sur le danger de fermer l’issue par où le lac Vélin se décharge dans le Nar : « Bientôt ce lac inonderait les plaines environnantes. La nature avait sagement pourvu aux intérêts des mortels, en marquant aux rivières leurs routes et leurs embouchures, le commencement et la fin de leur cours. Quelque respect aussi était dû à la religion des alliés, chez qui les fleuves de la patrie avaient un culte, des bois sacrés, des autels ; le Tibre lui-même, déshérité du tribut des ondes voisines, s’indignerait de couler moins glorieux. » Les prières des villes ou la difficulté des travaux ou enfin la superstition, firent prévaloir l’avis de Pison, qui conseillait de ne rien changer.
\subsection[{Nomination des gouverneurs de province}]{Nomination des gouverneurs de province}
\noindent \labelchar{LXXX.} Poppéus Sabinus fut continué dans le gouvernement de Mésie auquel on joignit l’Achaïe et la Macédoine. Ce fut une des maximes de Tibère de laisser longtemps l’autorité dans les mêmes mains ; et, sous lui, plus d’un gouverneur garda jusqu’à la mort son armée ou sa juridiction. On en donne différents motifs : les uns disent que, pour s’épargner l’ennui de nouveaux choix, il maintenait irrévocablement les premiers ; d’autres, que sa jalousie craignait de satisfaire trop d’ambitions. Quelques-uns pensent que la finesse de son esprit n’empêchait pas les perplexités de son jugement. Il ne recherchait point les vertus éminentes, et d’un autre côté il haïssait les vices ; il avait peur des gens de bien pour lui-même, des méchants pour l’honneur public. Cette irrésolution l’entraîna jusqu’à donner des provinces à des gouverneurs qu’il ne devait pas laisser sortir de Rome.
\subsection[{Les comices consulaires : une énigme}]{Les comices consulaires : une énigme}
\noindent \labelchar{LXXXI.} Il tint alors pour la première fois les comices consulaires. Je n’oserais rien affirmer sur cette élection ni sur celles qui la suivirent, tant je trouve de contradictions dans les historiens et dans les discours même du prince. Tantôt, sans dire le nom des candidats, il parlait de leur origine, de leur vie, de leurs campagnes, de manière à les faire reconnaître ; tantôt, supprimant jusqu’à cette désignation, il les exhortait à ne point troubler les comices par des brigues, et leur promettait de solliciter pour eux. Souvent il dit que les seuls qui eussent déclaré devant lui leurs prétentions étaient ceux dont il avait remis les noms aux consuls, que d’autres pouvaient encore se présenter, s’ils comptaient sur leur crédit ou sur leurs titres : paroles spécieuses, mais vaines ou perfides ; dehors trompeurs de liberté, dont se couvrait la tyrannie, pour éclater un jour avec plus de violence.
\section[{Livre second (16, 19)}]{Livre second (16, 19)}\renewcommand{\leftmark}{Livre second (16, 19)}

\noindent \labelchar{I.} Sous les consuls Sisenna Statilius Taurus et L. Libo, des troubles agitèrent les royaumes de l’Orient et les provinces romaines. Le signal fut donné par les Parthes, qui, sujets d’un roi qu’ils avaient demandé à Rome et reconnu volontairement, le méprisaient comme étranger, quoique du sang des Arsacides. Ce roi était Vonon qu’Auguste avait reçu en otage de Phraate. Car Phraate, quoiqu’il eût chassé nos armées et nos généraux, n’en avait pas moins rendu à Auguste tous les hommages du respect ; et, pour mieux s’assurer son amitié, il lui avait envoyé une partie de ses enfants, moins toutefois par crainte des Romains que par défiance des siens.\par
\labelchar{II.} Après la mort de Phraate et des rois ses successeurs, une ambassade vint à Rome au nom des grands du royaume, qui, las de voir couler le sang, redemandaient Vonon, le plus âgé de ses fils. Cette démarche flatta l’orgueil d’Auguste, qui renvoya le prince enrichi de ses dons. Les barbares l’accueillirent avec ces transports dont ils ont coutume de saluer leurs nouveaux maîtres. Mais bientôt, honteux de leur choix, ils s’accusent « d’avoir dégradé le nom des Parthes, en allant chercher dans un autre monde un roi corrompu par les arts de l’ennemi. Le trône des Arsacides était donc tenu et donné comme une province romaine ! Où était la gloire de ces héros qui avaient tué Crassus et chassé Marc Antoine, si un esclave de César, flétri par tant d’années de servitude, régnait sur les Parthes ? » Ainsi s’exprimait leur indignation, que Vonon achevait d’enflammer par son éloignement pour les usages des ancêtres, chassant rarement, aimant peu les chevaux, ne paraissant jamais dans les villes que porté en litière, et dédaignant les festins du pays. On tournait encore en dérision son cortège de Grecs et son cachet apposé sur les plus vils objets. Même son abord facile et son humeur prévenante, qualités ignorées de ces barbares, n’étaient pour eux que des vices nouveaux. Un air étranger rendait en lui le bien et le mal également odieux.\par
\labelchar{III.} Ils appellent donc Artaban, autre prince Arsacide, élevé chez les Dahes \footnote{Nation scythe, qui a donné son nom à la province appelée encore aujourd’hui Dahistan.} qui, vaincu dans un premier combat, retrouve des forces et s’empare du trône. Vonon fugitif se retira en Arménie, pays alors sans maître, et dont la foi partagée flottait entre les Parthes et les Romains depuis le crime d’Antoine, qui, après avoir, sous le nom d’ami, attiré dans un piège Artavasde, roi des Arméniens, le chargea de fers et finit par le tuer. Artaxias, fils de ce prince, ennemi de Rome à cause du souvenir de son père, se maintint, lui et son royaume, avec le secours des Arsacides. Artaxias ayant péri par la trahison de ses proches, Tigrane fut donné par Auguste aux Arméniens, et conduit dans ses États par Tibérius Nero \footnote{Depuis, l’empereur Tibère.}. Le trône ne resta pas longtemps à Tigrane, non plus qu’à ses enfants, quoique, selon l’usage des barbares, le frère et la sœur eussent associé leur lit et leur puissance. Un autre Artavasde fut imposé par Auguste, puis renversé, non sans perte pour nous.\par
\labelchar{IV.} C’est alors que Caïus César \footnote{Fils d’Agrippa et de Julie, petit-fils d’Auguste.} fut choisi pour pacifier l’Arménie. Il la donna au Mède Ariobarzane, dont les avantages extérieurs et le grand courage plaisaient aux Arméniens. Ariobarzane périt d’une mort fortuite, et sa race fut rejetée. Les Arméniens essayent alors du gouvernement d’une femme, nommée Érato, la chassent bientôt ; puis irrésolus, livrés à l’anarchie, moins libres que sans maître, ils placent enfin sur le trône le fugitif Vonon. Mais Artaban le menaçait, les Arméniens étaient peu capables de le défendre, et, si nous embrassions sa querelle, il fallait avoir la guerre avec les Parthes ; le gouverneur de Syrie, Créticus Silanus, l’attire dans sa province et le retient captif, en lui laissant le nom et l’appareil de roi. Nous dirons plus tard \footnote{Voy. ci-dessous, chap. LXVIII.} comment Vonon essaya d’échapper à cette à cette dérision.\par
\labelchar{V.} Tibère vit sans déplaisir les troubles de l’Orient : ils lui donnaient un prétexte pour enlever Germanicus à ses vieilles légions et le livrer, dans de nouvelles provinces, aux doubles attaques de la fortune et de la perfidie. Mais lui, d’autant plus occupé de hâter sa victoire, qu’il connaissait mieux le dévouement de ses troupes et la haine de son oncle, réfléchit à la conduite de la guerre et à ce qu’en trois ans d’expéditions il a éprouvé d’heureux ou de funeste. Il juge « que les Germains, toujours défaits en plaine et en bataille rangée, ont pour eux leurs bois et leurs marais, des étés courts, des hivers prématurés ; que les soldats souffrent moins du fer de l’ennemi que de la longueur des marches et de la perte de leurs armes, que la Gaule épuisée ne peut plus fournir de chevaux ; qu’une longue file de bagages est facile à surprendre, difficile à protéger ; que par mer, au contraire, l’invasion serait rapide, inattendue ; la campagne commencerait plus tôt ; les légions et les convois vogueraient ensemble ; et la cavalerie, en remontant les fleuves, arriverait toute fraîche, hommes et chevaux, au cœur de la Germanie. »\par
\labelchar{VI.} Il tourne donc ses vues de ce côté ; et, pendant que P. Vitellius et C. Antius vont régler le cens des Gaules, Silius, Antéius et Cécina sont chargés de construire une flotte. Mille vaisseaux parurent suffisants et furent bientôt achevés. Les uns étaient courts, étroits de poupe et de proue, larges de flancs, afin de mieux résister aux vagues ; les autres à carènes plates, pour pouvoir échouer sans péril ; la plupart à double gouvernail, afin qu’en changeant de manœuvre on les fît aborder à volonté par l’un ou l’autre bout ; un grand nombre pontés, pour recevoir les machines ou servir au transport des chevaux et des provisions ; tous bons voiliers, légers sous la rame, et montés par des soldats dont l’ardeur rendait cet appareil plus imposant et plus formidable. L’île des Bataves \footnote{Voy. \emph{Histoires}, IV, XII.} fut assignée pour rendez-vous, à cause de ses abords faciles et de la commodité qu’elle offre pour embarquer des troupes et envoyer la guerre sur un autre rivage. Car le Rhin, jusque-là contenu dans un seul lit ou n’embrassant que des îles de médiocre étendue, se partage, à l’entrée du territoire batave, comme en deux fleuves différents. Le bras qui coule le long de la Germanie conserve son nom et la violence de son cours jusqu’à ce qu’il se mêle à l’Océan. Plus large et plus tranquille, celui qui arrose la frontière gauloise reçoit des habitants le nom de Vahal, et le perd bientôt en se réunissant à la Meuse, avec laquelle il se décharge dans ce même océan par une vaste embouchure.\par
\labelchar{VII.} En attendant que sa flotte fût rassemblée, César envoya Silius avec un camp volant faire une incursion chez les Chattes. Lui-même, instruit qu’on assiégeait un fort établi sur la Lippe \footnote{Ce fort avait été construit par Drusus, père de Germanicus, à l’endroit où la rivière d’Aliso se jette dans la Lippe.}, y mena six légions. Silius, à cause des pluies qui survinrent, ne réussit qu’à enlever un peu de butin, avec la femme et la fille d’Arpus, chef des Chattes. Quant à César, les assiégeants ne lui fournirent pas l’occasion de combattre, s’étant dispersés à la nouvelle de son approche. Cependant ils avaient détruit le tombeau élevé dernièrement aux légions de Varus et un ancien autel consacré à Drusus. César releva l’autel, et rendit honneur à son père en défilant alentour à la tête des légions : il ne crut pas devoir rétablir le tombeau. Tout le pays situé entre le fort Aliso et le Rhin fut retranché et fermé par de nouvelles barrières.\par
\labelchar{VIII.} La flotte arrivée, Germanicus fait partir en avant les provisions, distribue les légionnaires et les alliés sur les navires, et entre dans le canal qui porte le nom de Drusus \footnote{Tout le monde convient, dit d’Anville, que ce canal est celui qui sort du Rhin sur la droite, au-dessous de la séparation du Vahal (ou Whaal), et qui se joint à l’Yssel près de Doesbourg. Il est appelé communément Nouvel-Yssel.}, en priant son père « d’être propice à un fils qui ose marcher sur ses traces, et de le soutenir par l’inspiration de ses conseils et l’exemple de ses travaux. » Ensuite une heureuse navigation le porta par les lacs et l’Océan jusqu’à l’Ems. Il laissa la flotte à Amisia 2), sur la rive gauche, et ce fut une faute de n’avoir pas remonté le fleuve : il fallut faire passer l’armée sur la rive droite, où elle devait agir, et plusieurs jours furent perdus à construire des ponts. La cavalerie et les légions franchirent en bon ordre les premiers courants, la mer ne montant pas encore. Il y eut de la confusion à l’arrière-garde, formée des auxiliaires : les Bataves, qui en faisaient partie, voulurent braver le flot et se montrer habiles nageurs, et quelques-uns furent noyés. César traçait son camp lorsqu’on lui annonça que les Ampsivariens s’étaient soulevés derrière lui. Stertinius, détaché aussitôt avec de la cavalerie et des troupes légères, punit cette perfidie par le fer et la flamme.\par
2. Amisia n’est pas Embden, puisque Embden est sur la rive droite de l’Ems. C’était quelque bourgade située sur la rive occidentale du fleuve, peut-être en face de la ville moderne.\par
\labelchar{IX.} Le Véser coulait entre les Romains et les Chérusques. Arminius parut sur la rive avec les autres chefs, et s’informa si César était présent. On lui répondit qu’il l’était ; alors il demanda la permission de s’entretenir avec son frère. Ce frère, nommé parmi nous Flavius, servait dans nos troupes avec une fidélité remarquable, et quelques années auparavant il avait perdu un œil en combattant sous Tibère. L’entrevue accordée, Flavius, s’avance ; Arminius le salue, renvoie son escorte, et demande que les archers qui bordaient notre rive s’éloignent pareillement. Quand ils se sont retirés, Arminius prie son frère de lui dire pourquoi il est ainsi défiguré. Flavius cite le lieu et le combat. Arminius veut savoir quelle a été sa récompense. Flavius énumère ce qu’il a reçu : une augmentation de paye, un collier, une couronne et d’autres présents militaires. Le Germain admira qu’on se fît esclave à si bon marché.\par
\labelchar{X.} Ensuite le débat s’engage. L’un fait valoir « la grandeur romaine, les forces de César, les châtiments terribles réservés aux vaincus, la clémence offerte à quiconque se soumet, enfin la femme et le fils d’Arminius généreusement traités. » L’autre invoque « les droits sacrés de la patrie, la liberté de leurs ancêtres, les dieux tutélaires des Germains, une mère qui se joint à ses prières et conjure son fils de ne pas aimer mieux, déserteur de ses proches, de ses alliés, de sa nation, les trahir que de les commander. » Peu à peu ils s’emportèrent jusqu’aux injures ; et, malgré le fleuve qui les séparait, ils allaient en venir aux mains, si Stertinius, accouru à la hâte, n’eût retenu Flavius, qui, bouillant de colère, demandait son cheval et ses armes. On voyait sur l’autre bord Arminius menaçant nous appeler au combat, car il jetait, parmi ses invectives, beaucoup de mots latins, qu’il avait appris lorsqu’il commandait dans nos armées un corps de Germains.\par
\labelchar{XI.} Le lendemain, les barbares parurent en bataille au-delà du Véser. César, persuadé qu’un général ne pouvait exposer ses légions sans avoir établi des ponts, avec des postes pour les défendre, fait passer à gué sa cavalerie. Stertinius et le primipilaire Émilius guidèrent le passage sur des points différents afin de partager l’attention de l’ennemi. Cariovalde, chef des Bataves, s’élança par l’endroit le plus rapide du fleuve. Les Chérusques, à l’aide d’une fuite simulée, l’attirent dans une plaine environnée de bois. Bientôt, sortis de leur embuscade, ils l’enveloppent ; et culbutent tout ce qui résiste, poursuivent tout ce qui plie. Les Bataves, s’étant formés en cercle, sont attaqués de près par les uns, harcelés de loin par les autres. Cariovalde, après avoir soutenu longtemps la violence du combat, exhorté les siens à percer en masse les bataillons ennemis, se jette lui-même à travers les rangs les plus serrés, et, accablé de traits, ayant eu son cheval tué, il tombe, et autour de lui beaucoup de nobles bataves : les autres furent sauvés par leur courage ou par la cavalerie de Stertinius et d’Émilius, qui vint les dégager.\par
\labelchar{XII.} César, ayant passé le Véser, apprit d’un transfuge qu’Arminius avait choisi son champ de bataille, que d’autres nations s’étaient réunies à lui dans une forêt consacrée à Hercule, et qu’ils tenteraient sur le camp romain une attaque nocturne. On crut à ce rapport : on voyait d’ailleurs les feux de l’ennemi ; et des éclaireurs, qui s’étaient plus avancés, annoncèrent des hennissements de chevaux et le bruit d’une immense et confuse multitude. À l’approche d’une affaire décisive, César voulut sonder les dispositions des soldats, et il réfléchissait aux moyens de rendre l’épreuve fidèle. Il connaissait « le penchant des tribuns et des centurions à donner plutôt de bonnes nouvelles que des avis certains, l’esprit servile des affranchis, la faiblesse des amis, trop enclins à flatter. Convoquer une assemblée n’était pas plus sûr : là, quelques voix commencent, toutes les autres répètent. Il fallait lire dans les âmes, lorsque les soldats, seuls, sans surveillants, au milieu des repas militaires, expriment librement leurs craintes et leurs espérances. »\par
\labelchar{XIII.} Au commencement de la nuit, il sort de l’augural par une porte secrète, ignorée des sentinelles, et, suivi d’un seul homme, les épaules couvertes d’une peau de bête, il parcourt les rues du camp, s’arrête auprès des tentes, et là, confident de sa renommée, il entend l’un vanter la haute naissance du général, l’autre sa bonne mine, la plupart sa patience, son affabilité, son humeur toujours la même dans les affaires et dans les plaisirs. Tous se promettent de le payer de ses bienfaits sur le champ de bataille, et d’immoler à sa vengeance et à sa gloire un ennemi parjure et infracteur des traités. En cet instant, un Germain qui parlait notre langue pousse son cheval jusqu’aux palissades, et d’une voix forte promet au nom d’Arminius, à tout déserteur, une femme, des terres, et cent sesterces par jour jusqu’à la fin de la guerre. Cette injure alluma la colère des légions : « Que le jour vienne, et qu’on livre bataille ; le soldat saura prendre les terres des Germains et emmener leurs femmes ; ils en acceptent l’augure ; les femmes et les trésors de l’ennemi seront le butin de la victoire. » Vers la troisième veille \footnote{Minuit. Les Romains comptaient douze heures du coucher au lever du soleil, et partageaient la nuit en quatre veilles de trois heures chacune.} les barbares insultèrent le camp, et se retirèrent sans avoir lancé un trait, lorsqu’ils virent les retranchements garnis de nombreuses cohortes et tous les postes bien gardés.\par
\labelchar{XIV.} Cette même nuit mêla d’une douce joie le repos de Germanicus. Il lui sembla qu’il faisait un sacrifice, et que, le sang de la victime ayant rejailli sur sa robe, il en recevait une plus belle d’Augusta, son aïeule. Encouragé par ce présage, que confirmaient les auspices, il convoque les soldats et leur donne, avec une sagesse prévoyante, ses instructions pour la bataille qui s’approche. « Il ne fallait pas croire, disait-il, que les plaines fussent seules favorables au soldat romain ; les bois et les défilés ne l’étaient pas moins, s’il profitait de ses avantages. Les immenses boucliers des barbares et leurs énormes piques n’étaient point, entre les arbres et au milieu des broussailles, d’un usage aussi commode que la javeline, l’épée et une armure serrée contre le corps. Il fallait presser les coups et chercher le visage avec la pointe des armes. Les Germains n’avaient ni cuirasses ni casques ; leurs boucliers mêmes, sans cuir ni fer qui les consolidât, n’étaient que de simples tissus d’osier ou des planches minces, recouvertes de peinture. Leur première ligne, après tout, était seule armée de piques ; le reste n’avait que des bâtons durcis au feu ou de très courts javelots. Et ces corps d’un aspect effroyable, vigoureux dans un choc de quelques instants, pouvaient-ils endurer une blessure ? Insensibles à la honte, sans nul souci de leurs chefs, ils lâchaient pied, ils fuyaient, tremblant dans les revers, bravant le ciel et la terre dans la prospérité. Si l’armée, lasse des marches et de la mer, désirait la fin de ses travaux, elle la trouverait sur ce champ de bataille. Déjà elle était plus près de l’Elbe que du Rhin ; et là cessait toute guerre, si, foulant avec lui les traces de son père et de son oncle \footnote{Drusus et Tibère.}, elle le rendait vainqueur sur ce théâtre de leur gloire. » L’ardeur des soldats répondit aux paroles du général, et le signal du combat fut donné.\par
\labelchar{XV.} De leur côté, Arminius et les autres chefs des Germains en attestent leurs guerriers : « Ces Romains, que sont-ils, sinon les fuyards de l’armée de Varus, qui, pour échapper à la guerre, se sont jetés dans la révolte ; qui, le dos chargé de blessures ou le corps tout brisé par les flots et les tempêtes, viennent s’exposer de nouveau au fer de l’ennemi et au courroux des dieux, sans apporter même l’espérance ? En effet, cachés dans leurs vaisseaux, ils ont cherché par mer des routes où nul homme ne pût ni les attendre ni les poursuivre ; mais, quand on se mesurera corps à corps, ce ne sont ni les vents ni les rames qui les tireront de nos mains. Guerriers, rappelez-vous leur avarice, leur cruauté, leur orgueil. Que vous reste-t-il, sinon de sauver la liberté ou de mourir avant de la perdre ? »\par
\labelchar{XVI.} Enflammés par ces discours et brûlant de combattre, ils descendent dans les champs qui portent le nom d’Idistavise \footnote{Il fut chercher Iditavise sur la rive droite du Véser. Brotier la suppose près de Halemn, non loin du lieu où le maréchal d’Estrées remporta en 1757 la célèbre victoire d’Hastembeck.}. C’est une plaine située entre le Véser et des collines, dont l’inégale largeur s’étend ou se resserre en suivant les sinuosités du fleuve et les saillies des montagnes. À l’extrémité s’élevait un bois de haute futaie, dont les arbres laissaient entre eux la terre dégarnie. L’armée des barbares se rangea dans la plaine et à l’entrée de ce bois. Les Chérusques seuls occupèrent les hauteurs, d’où ils devaient tomber sur les Romains au plus fort du combat. Voici l’ordre dans lequel marcha notre armée : les auxiliaires gaulois et germains en tête ; ensuite les archers à pied ; puis quatre légions, et Germanicus avec deux cohortes prétoriennes \footnote{Cohortes d’élite qui, dès le temps de la République, servaient de garde particulière au général dans ses campagnes.} et un corps de cavalerie d’élite ; enfin quatre autres légions, l’infanterie légère et les archers à cheval, avec le reste des alliés. Le soldat était attentif et prêt au signal, afin que l’ordre de marche se transformât tout à coup en ordre de bataille.\par
\labelchar{XVII.} Germanicus, voyant les bandes des Chérusques s’élancer, emportées par leur ardeur, commande à ses meilleurs escadrons d’attaquer en flanc, tandis que Stertinius, avec le reste de la cavalerie, tournerait l’ennemi et le chargerait en queue : lui-même promit de les seconder à propos. En ce moment, huit aigles furent vus se dirigeant vers la forêt, où ils pénétrèrent. Frappé d’un augure si beau, Germanicus crie aux soldats « de marcher, de suivre ces oiseaux romains, ces divinités des légions. » Aussitôt l’infanterie se porte en avant ; et déjà la cavalerie enfonçait les flancs et l’arrière-garde. On vit, chose étrange ! les deux parties d’une même armée se croiser dans leur fuite : ceux qui avaient occupé la forêt se sauvent dans la plaine, ceux de la plaine courent vers la forêt. Du haut de leurs collines, les Chérusques étaient précipités à travers cette mêlée. Parmi eux on distinguait Arminius blessé, qui, de son bras, de sa voix, de son sang, essayait de ranimer le combat. Il s’était jeté sur nos archers, tout prêt à les rompre, si les alliés Rhètes, Vindéliciens et Gaulois, ne lui eussent opposé leurs cohortes. Toutefois, par un vigoureux effort et l’impétuosité de son cheval, il se fit jour, la face couverte du sang de sa blessure pour n’être pas reconnu : quelques-uns prétendent que les Chauques, auxiliaires dans l’armée romaine, le reconnurent cependant et le laissèrent échapper. La même valeur ou la même trahison sauva Inguiomère ; le reste fut taillé en pièces. Un grand nombre, voulant passer le Véser à la nage, furent tués à coups de traits ou emportés par le courant ou abîmés dans l’eau par le poids l’un de l’autre et par l’éboulement des rives. Plusieurs cherchèrent sur les arbres un honteux refuge, et se cachèrent entre les branches : nos archers se firent un amusement de les percer de flèches ou bien l’arbre abattu les entraînait dans sa chute.\par
\labelchar{XVIII.} Cette victoire fut grande et nous coûta peu de sang. Massacrés sans relâche depuis la cinquième heure \footnote{Les Romains comptaient, du lever au coucher du soleil, douze heures, plus longues ou plus courtes suivant les saisons. Aux équinoxes, la cinquième heure serait, dans notre manière de mesurer le jour, onze heure du matin.} jusqu’à la nuit, les ennemis couvrirent de leurs armes et de leurs cadavres un espace de dix milles. On trouva, parmi les dépouilles, des chaînes qu’ils avaient apportées pour nos soldats ; tant ils se croyaient sûrs de vaincre. L’armée proclama Tibère \emph{Imperator} sur le champ de bataille. Elle éleva un tertre avec un trophée d’armes, où l’on inscrivit le nom des nations vaincues.\par
\labelchar{XIX.} Ni blessures, ni morts, ni ravages, n’avaient allumé dans le cœur des Germains autant de colère et de vengeance que la vue de ce monument. Ces hommes, qui tout à l’heure s’apprêtaient à quitter leurs foyers et à se retirer de l’autre côté de l’Elbe, veulent des combats, courent aux armes : jeunes, vieux, peuple, grands, tout se lève à la fois et trouble par des incursions subites la marche des Romains. Enfin ils choisissent pour champ de bataille une plaine étroite et marécageuse, resserrée entre le fleuve et des forêts. Les forêts elles-mêmes étaient entourées d’un marais profond, excepté d’un seul côté, où les Ampsivariens avaient construit une large chaussée qui servait de barrière entre eux et les Chérusques. L’infanterie se rangea sur cette chaussée ; la cavalerie se cacha dans les bois voisins pour prendre nos légions à dos, lorsqu’elles seraient engagées dans la forêt.\par
\labelchar{XX.} Aucune de ces mesures n’était ignorée de César. Projets, positions, résolutions publiques ou secrètes, il connaissait tout, et faisait tourner les ruses des ennemis à leur propre ruine. Il charge le lieutenant Séius Tubéro de la cavalerie et de la plaine ; il dispose les fantassins de manière qu’une partie entre dans la forêt par le côté où le terrain étant plat, tandis que l’autre emporterait d’assaut la chaussée. Il prend pour lui-même le poste le plus périlleux, et laisse les autres à ses lieutenants. Le corps qui avançait de plain-pied pénétra facilement. Ceux qui avaient la chaussée à gravir recevaient d’en haut, comme à l’attaque d’un mur, des coups meurtriers. Le général sentit que, de près, la lutte n’était pas égale ; il retire ses légions un peu en arrière, et ordonne aux frondeurs de viser sur la chaussée et d’en chasser les ennemis. En même temps les machines lançaient des javelots, dont les coups renversèrent d’autant plus de barbares que le lieu qu’ils défendaient les mettait plus en vue. Maître du rempart, Germanicus s’élance le premier dans la forêt à la tête des cohortes prétoriennes. Là on combattit corps à corps. La retraite était fermée à l’ennemi par le marais, aux Romains par le fleuve et les montagnes. De part et d’autre une position sans issue ne laissait d’espoir que dans le courage, de salut que dans la victoire.\par
\labelchar{XXI.} Égaux par la bravoure, les Germains étaient inférieurs par la nature du combat et par celle des armes. Resserrés dans un espace trop étroit pour leur nombre immense, ne pouvant ni porter en avant et ramener leurs longues piques, ni s’élancer par bonds et déployer leur agilité, ils étaient réduits à se défendre sur place, tandis que le soldat romain, le bouclier pressé contre la poitrine, l’épée ferme au poing, sillonnait de blessures leurs membres gigantesques et leurs visages découverts, et se frayait un passage en les abattant devant lui. Et déjà s’était ralentie l’ardeur d’Arminius, rebuté sans doute par la continuité des périls ou affaibli par sa dernière blessure. Inguiomère lui-même, qui volait de rang en rang, commençait à être abandonné de la fortune plutôt que de son courage ; et Germanicus, ayant ôté son casque pour être mieux reconnu, criait aux siens « de frapper sans relâche ; qu’on n’avait pas besoin de prisonniers ; que la guerre n’aurait de fin que quand la nation serait exterminée. » Sur le soir, il retira du champ de bataille une légion pour préparer le campement ; les autres se rassasièrent jusqu’à la nuit du sang des ennemis ; la cavalerie combattit sans avantage décidé.\par
\labelchar{XXII.} Germanicus après avoir publiquement félicité les vainqueurs, érigea un trophée d’armes avec cette inscription magnifique : « Victorieuse des nations entre le Rhin et l’Elbe, l’armée de Tibère César a consacré ce monument à Mars, à Jupiter, à Auguste. » II n’ajouta rien sur lui-même, soit crainte de l’envie, soit qu’il pensât que le témoignage de la conscience suffit aux belles actions. Il chargea Stertinius de porter la guerre chez les Ampsivariens ; mais ils la prévinrent par la soumission et les prières ; et, en ne se refusant à rien, ils se firent tout pardonner.\par
\labelchar{XXIII.} Cependant l’été s’avançait, et quelques légions furent renvoyées par terre dans leurs quartiers d’hiver. Germanicus fit embarquer le reste sur l’Ems, et regagna l’Océan. D’abord la mer, tranquille sous ces mille vaisseaux, ne retentissait que du bruit de leurs rames, ne cédait qu’à l’impulsion de leurs voiles. Tout à coup, d’un sombre amas de nuages s’échappe une effroyable grêle. Au même instant les vagues tumultueuses, soulevées par tous les vents à la fois, ôtent la vue des objets, empêchent l’action du gouvernail. Le soldat, sans expérience de la mer, s’épouvante ; et, en troublant les matelots ou les aidant à contre-temps, il rend inutile l’art des pilotes. Bientôt tout le ciel et toute la mer n’obéissent plus qu’au souffle du midi, dont la violence, accrue par l’élévation des terres de la Germanie, la profondeur de ses fleuves, les nuées immenses qu’il chasse devant lui, enfin par le voisinage des régions glacées du nord, disperse les vaisseaux, les entraîne au large ou les pousse vers des îles bordées de rocs escarpés ou de bancs cachés sous les flots \footnote{Sans doute les petites îles qui bordent la côte entre l’embouchure de l’Ems et celle du Véser ; car le vent du midi dut porter les vaisseaux vers le nord : peut-être échoua-t-on aussi entre l’Ems et le Rhin. Quoi qu’il en soit, il n’y a pas, dans ces îles, ni rochers à pic, ni côtes escarpées : c’est un rivage plat et sablonneux, ce qui explique comment les navires furent recueillis et ramenés après leur naufrage.}. On parvint à s’en éloigner un peu avec beaucoup d’efforts. Mais quand le reflux porta du même côté que le vent, il ne fut plus possible de demeurer sur les ancres, ni d’épuiser l’eau qui entrait de toutes parts. Chevaux, bêtes de somme, bagages, tout, jusqu’aux armes, est jeté à la mer pour soulager les navires, qui s’entrouvraient par les flancs ou s’enfonçaient sous le poids des vagues.\par
\labelchar{XXIV.} Autant l’Océan est plus violent que les autres mers, et le ciel de la Germanie plus affreux que les autres climats, autant ce désastre surpassa par sa grandeur et sa nouveauté tous les désastres semblables. On n’avait autour de soi que des rivages ennemis ou une mer si vaste et si profonde qu’on la regarde comme la limite de l’univers, et qu’on ne suppose pas de terres au-delà. Une partie des vaisseaux fut engloutie. Un plus grand nombre fut jeté sur des îles éloignées \footnote{Walther indique les Orcades, les îles de Shetland, et celles qui bordent la Norvège.}, où les soldats, ne trouvant aucune trace d’habitation humaine, périrent de faim ou se soutinrent avec la chair des chevaux échoués sur ces bords. La seule trirème de Germanicus prit terre chez les Chauques. Pendant tous les jours et toutes les nuits qu’il y passa, on le vit errer sur les rochers et sur les pointes les plus avancées, s’accusant d’être l’auteur de cette grande catastrophe ; et ses amis ne l’empêchèrent qu’avec peine de chercher la mort au sein des mêmes flots. Enfin la marée et un vent favorable ramenèrent le reste des navires, tout délabrés, presque sans rameurs, n’ayant pour voiles que des vêtements étendus, quelques-uns traînés par les moins endommagés. Germanicus les fit réparer à la hâte et les envoya visiter les îles. La plupart des naufragés furent ainsi recueillis. Les Ampsivariens, nouvellement soumis, en rachetèrent beaucoup dans l’intérieur des terres, et nous les rendirent. D’autres, emportés jusqu’en Bretagne, furent renvoyés par les petits princes du pays. Plus chacun revenait de loin, plus il racontait de merveilles, bourrasques furieuses, oiseaux inconnus, poissons prodigieux, monstres d’une forme indécise entre l’homme et la bête : phénomènes réels on fantômes de la peur.\par
\labelchar{XXV.} Le bruit semé que la flotte était perdue, en relevant l’espoir des ennemis, excita Germanicus à réprimer leur audace. Il envoya Silius contre les Chattes avec trente mille hommes de pied et trois mille chevaux. Lui-même entra chez les Marses avec une armée encore plus forte. Leur chef Mallovendus, qui s’était rendu depuis peu, déclara qu’une des aigles de Varus était enfouie dans un bois voisin, et gardée par une poignée d’hommes. Aussitôt un détachement est envoyé pour attirer les barbares en avant, tandis qu’un autre irait par derrière et enlèverait l’aigle. Tous deux réussirent. Animé par ce succès, Germanicus s’enfonce dans le pays, le dévaste et le pille, sans que l’ennemi osât en venir aux mains ou, s’il résistait quelque part, il était repoussé à l’instant, et jamais, au rapport des prisonniers, les Germains n’avaient ressenti une plus grande terreur. Ils disaient hautement « que les Romains étaient invincibles et à l’épreuve de tous les coups de la fortune, puisque, après le naufrage de leurs vaisseaux et la perte de leurs armes, lorsque les rivages étaient jonchés des cadavres de leurs hommes et de leurs chevaux, ils revenaient à la charge, aussi fiers, aussi intrépides, et comme multipliés. »\par
\labelchar{XXVI.} Le soldat fut ramené dans les quartiers d’hiver, satisfait d’avoir compensé par la victoire les malheurs de la navigation. César mit le comble à la joie par sa munificence, en payant à chacun ce qu’il déclarait avoir perdu. On ne doutait plus que les ennemis découragés ne songeassent à demander la paix, et qu’une nouvelle campagne ne terminât la guerre. Mais Tibère, par de fréquents messages, pressait Germanicus de revenir à Rome, où le triomphe l’attendait. « C’était assez d’événements, assez de hasards ; il avait livré d’heureux et mémorables combats ; mais devait-il oublier les vents et les flots, dont la fureur, qu’on ne pouvait reprocher au général, n’en avait pas moins causé de cruels et sensibles dommages ? Lui-même, envoyé neuf fois en Germanie par Auguste, avait dû plus de succès au conseil qu’à la force : c’était ainsi qu’il avait amené les Sicambres à se soumettre, enchaîné par la paie les Suèves et le roi Maroboduus. À présent que l’honneur de l’empire était vengé, on pouvait aussi abandonner à leurs querelles domestiques les Chérusques et les autres nations rebelles. » Germanicus demandait en grâce un an pour achever son ouvrage : Tibère livre à sa modestie une attaque plus vive, en lui offrant un deuxième consulat, dont il exercerait les fonctions en personne. Il ajoutait « que, s’il fallait encore faire la guerre, Germanicus devait laisser cette occasion de gloire à son frère Drusus, qui, faute d’un autre ennemi, ne pouvait qu’en Germanie mériter le nom d’\emph{Imperator} et cueillir de nobles lauriers. » Germanicus ne résista plus, quoiqu’il comprît que c’était un prétexte inventé par la jalousie pour l’arracher à une conquête déjà faite.\par
\labelchar{XXVII.} Vers le même temps, Libo Drusus, de la maison Scribonia, fut accusé de complots contre l’ordre établi. Je rapporterai en détail le commencement, la suite et la fin de cette affaire, parce qu’elle offre le premier exemple de ces intrigues qui furent tant d’années une des plaies de l’État. Firmius Catus, sénateur, intime ami de Libon, amena ce jeune homme imprévoyant et crédule à se fier aux promesses des Chaldéens \footnote{Les peuples de la Chaldée furent, selon Cicéron, les inventeurs de l’asrologie judiciaire. De là l’usage de donner le nom de Chaldéens à tous ceux qui se mêlaient de cet art chimérique.} et aux mystères de la magie ; il le poussa même chez des interprètes de songes. Sans cesse il montrait à ses yeux son bisaïeul Pompée, sa tante Scribonie, autrefois épouse d’Auguste, les Césars ses parents, et sa maison pleine d’illustres images l’engageant dans le luxe et les emprunts, s’associant à ses plaisirs, à ses liaisons, afin de multiplier les dépositions dont il enlacerait sa victime.\par
\labelchar{XXVIII.} Dès qu’il eut assez de témoins et qu’il put produire des esclaves instruits des mêmes faits, il sollicita une audience du prince, et lui fit connaître l’accusation et le nom de l’accusé par Flaccus Vescularius, chevalier romain, qui avait auprès de Tibère un accès plus facile. Tibère, sans repousser la délation, refuse l’audience, en disant qu’on pouvait communiquer par ce même Flaccus. Cependant il décore Libon de la préture, l’admet à sa table, sans jamais laisser voir (tant sa colère était renfermée) aucun mécontentement sur son visage, aucune émotion dans ses paroles. Maître de prévenir les discours et les actions du jeune homme, il préférait les épier. Enfin un certain Junius, que Libon priait d’évoquer par des enchantements les ombres des morts, en avertit Fulcinius Trio. Fulcinius était un accusateur célèbre et avide d’infamie : il saisit à l’instant cette proie, court chez les consuls, demande une instruction devant le sénat. Le sénat est convoqué ; l’édit portait qu’on aurait à délibérer sur une affaire grave et des faits atroces.\par
\labelchar{XXIX.} Cependant Libon, couvert d’habits de deuil, accompagné de femmes du premier rang, allait de maison en maison, implorant l’appui de ses proches et la voix d’un défenseur ; vaines prières, que tous repoussaient sous des prétextes divers, mais par le même motif, la peur. Le jour de l’assemblée, affaibli par l’inquiétude et les chagrins ou, selon quelques-uns, feignant d’être malade, il se fait conduire en litière jusqu’aux portes du sénat, et, appuyé sur le bras de son frère, il élève vers Tibère des mains et une voix suppliantes. Le prince l’écoute avec un visage immobile ; puis il lit les pièces et le nom des témoins, de ce ton mesuré qui évite également d’adoucir ou d’aggraver les charges.\par
\labelchar{XXX.} Aux accusateurs, Catus et Trio, s’étaient joints Fontéius Agrippa et C. Vibius. Tous quatre se disputaient à qui signalerait son éloquence contre l’accusé. Enfin Vibius, voyant que personne ne voulait céder, et que Libon était sans défenseur, déclare qu’il se bornerait à exposer l’un après l’autre les chefs d’accusation. Il produisit des pièces vraiment extravagantes : ainsi Libon s’était enquis des devins « s’il aurait un jour assez d’argent pour en couvrir la voie Appienne jusqu’à Brindes. » Les autres griefs étaient aussi absurdes, aussi frivoles, et, à le bien prendre, aussi dignes de pitié. Cependant une des pièces contenait les noms des Césars et des sénateurs, avec des notes, les unes hostiles, les autres mystérieuses, écrites, selon l’accusateur, de la main de Libon. Celui-ci les désavouant, on proposa d’appliquer à la question ceux de ses esclaves qui connaissaient son écriture ; et, comme un ancien sénatus-consulte défendait qu’un esclave fût interrogé à la charge de son maître, le rusé Tibère, inventeur d’une nouvelle jurisprudence, les fit vendre à un agent du fisc, afin qu’on pût, sans enfreindre la loi, les forcer à déposer contre Libon. Alors l’accusé demanda un jour de délai ; et, de retour chez lui, il chargea son parent, P. Quirinus, de porter à l’empereur ses dernières prières.\par
\labelchar{XXXI.} On lui répondit de s’adresser au Sénat. Cependant sa maison était environnée de soldats. Déjà on entendait le bruit qu’ils faisaient dans le vestibule : on pouvait même les apercevoir. En cet instant Libon, qui cherchait dans les plaisirs de la table une dernière jouissance, n’y trouvant plus qu’un nouveau supplice, demande la mort, saisit les mains de ses esclaves, y met son épée malgré eux. Ceux-ci reculent effrayés et renversent la lumière placée sur la table. Au milieu de ces ténèbres, qui furent pour lui celles du tombeau, Libon se porta deux coups dans les entrailles. Ses affranchis accoururent au cri qu’il poussa en tombant, et les soldats, le voyant mort, se retirèrent. L’accusation n’en fut pas poursuivie avec moins de chaleur dans le sénat, et Tibère jura qu’il aurait demandé la vie de l’accusé, tout coupable qu’il était, s’il ne se fût trop hâté de mourir.\par
\labelchar{XXXII.} Les biens de Libon furent partagés entre ses accusateurs, et des prétures extraordinaires données à ceux qui étaient de l’ordre du sénat \footnote{C’est-à-dire qu’ils furent créés préteurs par une nomination spéciale, et en sus du nombre ordinaire, qui était de douze.}. Cotta Messallinus \footnote{Cet homme, dont le nom reviendra plusieurs fois dans ces Annales, et qui fut un des opprobres de ce siècle, était fils de l’orateur M. Valérius Messala Corvinus.} vota pour que l’image de Libon ne pût être portée aux funérailles de ses descendants, Cn. Lentulus pour qu’aucun membre de la maison Scribonia ne prît désormais le surnom de Drusus. Plusieurs jours de supplications \footnote{Prières publiques adressées aux dieux pour les remercier d’une faveur éclatante.} furent décrétés sur la proposition de Pomponius Flaccus ; et, sur celle de L. Publius, d’Asinius Gallus, de Papius Mutilus et de L. Apronius, on résolut de consacrer des offrandes à Jupiter, à Mars, à la Concorde, et de fêter à l’avenir les Ides de septembre, jour où Libon s’était tué. J’ai rapporté ces bassesses et les noms de leurs auteurs, afin qu’on sache que l’adulation est un mal ancien dans l’État. D’autres sénatus-consultes chassèrent d’Italie les astrologues et les magiciens. Un d’entre eux, L. Pituavius, fut précipité de la roche Tarpéienne. Un autre, P. Marcius, conduit par ordre des consuls hors de la porte Esquiline, après que son jugement eut été proclamé à son de trompe, fut exécuté à la manière ancienne.\par
\bigbreak
\noindent \labelchar{XXXIII.} À la séance suivante, le consulaire Q. Hatérius et l’ancien préteur Octavius Fronto s’élevèrent avec force contre le luxe de Rome. La vaisselle d’or fut bannie des tables, et la soie interdite aux hommes, comme une parure dégradante. Fronton alla plus loin et demanda qu’on fixât ce que chacun pourrait avoir d’argenterie, de meubles, d’esclaves. Alors encore on voyait souvent les sénateurs, en opinant sur une question, proposer par surcroît tout ce qui leur paraissait utile. Asinius Gallus combattit le projet de Fronton. Selon lui, « les richesses particulières s’étaient accrues en même temps que l’empire ; et ce progrès n’était pas nouveau ; les plus vieilles mœurs s’en étaient ressenties : autre était la fortune des Fabricius, autre celle des Scipions ; tout se proportionnait à l’état de la République : pauvre, elle avait vu ses citoyens logés à l’étroit ; depuis qu’elle était parvenue à ce degré de splendeur, chacun s’était agrandi ; en fait d’esclaves, d’argenterie, d’ameublements, le luxe et l’économie se mesuraient sur la condition du possesseur : si la loi exigeait plus de revenu du sénateur que du chevalier, ce n’était pas que la nature eût mis entre eux aucune différence ; c’était afin qu’à la prééminence des fonctions, des dignités, des rangs, se joignissent tous les moyens de délasser l’esprit et d’entretenir la santé. Car on ne voudrait pas sans doute que ces grands citoyens, à qui sont imposés le plus de soins et de périls, fussent privés de ce qui peut en adoucir le poids et les inquiétudes. » On se rendit sans peine à l’avis de Gallus, faisant, sous des noms honnêtes, l’aveu des vices publics devant des hommes qui les partageaient. Tibère d’ailleurs avait ajouté « que ce n’était pas le temps de réformer les mœurs, qu’au premier signe de décadence elles ne manqueraient pas d’une voix qui vînt à leur secours. »\par
\labelchar{XXXIV.} Cependant L. Piso \footnote{Le même dont le procès et la mort sont racontés, au liv. IV, chap. XXI.} après s’être plaint des intrigues du Forum, de la corruption des juges, de la cruauté des orateurs, dont les accusations menaçaient toutes les têtes, protesta qu’il allait quitter Rome et ensevelir sa vie dans quelque retraite lointaine et ignorée ; et, en achevant ces mots, il sortait du sénat. Tibère, vivement ému, essaya de le calmer par de douces paroles ; il engagea même les parents de ce sénateur à employer, pour le retenir, leur crédit et leurs prières. Bientôt après, ce même Pison fit preuve d’une indignation non moins courageuse, en appelant en justice Urgulanie, que la faveur d’Augusta mettait au-dessus des lois. Urgulanie, au lieu de comparaître, se fit porter au palais de César, d’où elle bravait Pison, et celui-ci n’en continua pas moins ses poursuites, quoique Augusta se plaignît que c’était l’outrager elle-même et lui manquer de respect. Tibère, en prince citoyen, borna la condescendance pour sa mère à la promesse d’aller an tribunal du préteur et d’appuyer Urgulanie. Il sort du palais et ordonne aux soldats de le suivre de loin. On le voyait, au milieu d’un concours de peuple, s’avancer avec un visage composé, allongeant par différents entretiens le temps et le chemin ; lorsque enfin, Pison persistant malgré les représentations de ses proches, Augusta fit apporter la somme demandée. Ainsi finit un procès qui ne fut pas sans gloire pour Pison, et qui accrut la renommée de Tibère. Au reste le crédit d’Urgulanie était si scandaleux, qu’appelée en témoignage dans une cause qui s’instruisait devant le sénat, elle dédaigna de s’y rendre. Il fallut qu’un préteur allât chez elle recevoir sa déposition, quoique de tout temps celles des Vestales mêmes aient été entendues au Forum et devant le tribunal.\par
\labelchar{XXXV.} Il y eut cette année dans les affaires une interruption dont je ne parlerais pas, s’il n’était bon de connaître sur ce sujet les avis opposés de Cn. Piso et d’Asinius Gallus. Tibère avait annoncé qu’il serait absent quelque temps, et Pison voulait que pour cette raison même on redoublât d’activité : « Ce serait l’honneur du gouvernement, qu’en l’absence du prince le sénat et les chevaliers portassent également le poids de leurs fonctions. » Gallus, sans affecter une liberté dont Pison lui avait dérobé le mérite, soutint « que la présence de César était indispensable pour donner aux actes publics cet éclat qui convient à la majesté de l’empire, et que des discussions où l’Italie accourait, où affluaient les provinces, devaient être réservées à d’augustes regards. » Tibère écoutait en silence ces avis, qui furent débattus avec beaucoup de chaleur. Toutefois les affaires furent remises.\par
\labelchar{XXXVI.} Bientôt une discussion s’éleva entre Gallus et Tibère lui-même. Gallus était d’avis « qu’on élût à la fois les magistrats pour cinq ans ; que les lieutenants placés à la tête des légions avant d’avoir exercé la préture fussent de droit désignés préteurs ; enfin, que le prince nommât douze candidats pour chacune des cinq années. » Cette proposition couvrait évidemment des vues plus profondes, et touchait aux ressorts les plus cachés de l’empire. Tibère cependant, comme si elle avait dû accroître sa puissance, répondit « que sa modération serait gênée de choisir tant de concurrents et d’en ajourner tant d’autres : à peine, dans les élections annuelles, où une espérance prochaine consolait d’un refus, on évitait de faire des mécontents – que de haines soulèverait une exclusion de cinq ans ! Et comment prévoir quels changements pouvait apporter un si long avenir dans les intentions, dans les familles, dans les fortunes ? Entre désigné un an d’avance suffisait pour enfler l’orgueil ; que ne feraient pas cinq ans d’honneurs anticipés ? Ce serait enfin quintupler le nombre des magistrats, et renverser les lois qui fixaient une durée aux poursuites des prétendants, à la recherche et à la possession des dignités. » Par ce langage, populaire en apparence, Tibère sut retenir le pouvoir dans ses mains.\par
\labelchar{XXXVII.} Il augmenta le revenu de quelques sénateurs ; ce qui fit paraître plus étonnante la dureté avec laquelle il reçut la prière de M. Hortalus, jeune noble d’une pauvreté bien connue. Hortalus était petit-fils d’Hortensius l’orateur. Auguste, par le présent d’un million de sesterces \footnote{Cette somme répondait, sous Tibère, à 194 835 F 61 cent.}, l’avait engagé à se marier, afin de donner des rejetons à une famille illustre, qui allait s’éteindre. Ses quatre fils étaient debout à la porte du sénat, assemblés dans le palais. Quand son tour d’opiner fut venu, il se leva, et, portant ses regards tantôt sur l’image d’Hortensius, placée entre les orateurs, tantôt sur celle d’Auguste : « Pères conscrits, dit-il, ces enfants, dont vous voyez le nombre et le jeune âge, si je leur ai donné le jour, c’est uniquement par le conseil du prince ; et mes ancêtres, après tout, méritaient d’avoir des descendants : car pour moi, à qui l’inconstance du sort n’a permis de recevoir ou d’acquérir ni les richesses, ni la faveur du peuple, ni l’éloquence, ce patrimoine de notre maison, il me suffisait que ma pauvreté ne fût ni honteuse à moi-même, ni à charge à personne. L’empereur m’ordonna de prendre une épouse ; j’obéis. Voilà les rejetons et la postérité de tant de consuls, de tant de dictateurs ! Et ce langage n’est point celui du reproche ; c’est à votre pitié seule que je l’adresse. Ils obtiendront, César, sous ton glorieux empire, les honneurs qu’il te plaira de leur donner : en attendant, défends de la misère les arrière-petits-fils de Q. Hortensius, les nourrissons du divin Auguste. »\par
\labelchar{XXXVIII.} Le sénat paraissait favorable : ce fut une raison pour Tibère de s’opposer plus vivement ; ce qu’il fit à peu près en ces termes : « Si tous les pauvres s’habituent à venir ici demander de l’argent pour leurs enfants, la république s’épuisera sans rassasier jamais les particuliers. Quand nos ancêtres ont permis qu’un sénateur s’écartât quelquefois de l’objet sur lequel il vote, pour faire des propositions d’intérêt général, certes ils n’ont pas voulu que ce droit s’étendît aux affaires domestiques, et que nous vinssions, au profit de notre fortune, exposer le sénat et le prince à des censures inévitables, soit qu’ils accordent, soit qu’ils refusent. Non, ce n’est pas une prière, c’est une importunité, une surprise, que de se lever au milieu d’hommes réunis à tout autre fin, de violenter, avec le nombre et l’âge de ses enfants, la religion du sénat, d’exercer sur moi la même contrainte, et de forcer en quelque façon les portes du trésor, sans songer qu’il faudra le remplir par des crimes, si nous le vidons par complaisance. Auguste fut généreux envers toi, Hortalus, mais sans en être requis, mais sans faire une loi de te donner toujours. Ce serait ôter aux âmes leur ressort et mettre la paresse en honneur, que de souffrir que chacun plaçât hors de soi ses craintes et ses espérances, et, attendant avec sécurité des secours étrangers, vécût inutile à lui-même, onéreux à l’État. » Ce discours, applaudi par ces hommes que les princes trouvent toujours prêts à louer, également le bien et le mal, fut accueilli du plus grand nombre par un profond silence ou des murmures étouffés. Tibère s’en aperçut, et, reprenant la parole après quelques instants, il dit « qu’il avait répondu à Hortalus ; qu’au reste, si le sénat le jugeait à propos, il donnerait deux cent mille sesterces à chacun de ses enfants mâles. » Le sénat rendit grâces. Hortalus resta muet, soit qu’il fût retenu par la peur, soit qu’au sein de l’infortune il se ressouvint de la dignité de ses aïeux. Depuis ce temps, le cœur de Tibère, fermé à la pitié, laissa tomber la maison d’Hortensius dans une détresse humiliante.\par
\labelchar{XXXIX.} Cette même année l’audace d’un seul homme, si on ne l’eût promptement réprimée, allait plonger l’État dans les discordes et la guerre civile. Un esclave de Postumus Agrippa, nommé Clemens, en apprenant la mort d’Auguste, conçut un projet au-dessus de sa condition, celui de passer à l’île de Planasie, d’enlever son maître par force ou par ruse, et de le conduire aux armées de Germanie. Ce coup hardi manqua par la lenteur du vaisseau qui portait Clemens : on avait, dans l’intervalle, égorgé Postumus. L’esclave forme alors un dessein plus grand et plus périlleux : il dérobe les cendres du mort, se rend à Cosa \footnote{Petite ville à 11 milles de Rome. Les habitants des colonies et des municipes avaient apporté jusque-là le corps d’Auguste, mort à Nole. C’est à Boville que les chevaliers allèrent le prendre sur leurs épaules pour achever le voyage.}, promontoire d’Etrurie et se tient caché dans des lieux inconnus, assez longtemps pour laisser croître sa barbe et ses cheveux : il avait à peu près l’âge et les traits de son maître. Des émissaires, qu’il avait mis dans sa confidence, semèrent adroitement le bruit qu’Agrippa était vivant. D’abord c’est un secret qui se dit à voix basse, comme tout ce qui est illicite : bientôt la nouvelle vole de bouche en bouche, accueillie par la foule ignorante et par ces esprits turbulents qui ne désirent que révolutions. Clemens lui-même allait dans les villes, mais le soir, et sans paraître en public, sans prolonger nulle part son séjour. Convaincu que, si la vérité s’accrédite par le temps et l’examen, la précipitation et le mystère conviennent au mensonge, il devançait sa renommée ou s’y dérobait à propos.\par
1 Aujourd’hui Monte-Argentaro, près d’Orbitello. Dans le voisinage de ce promontoire était une colonie romaine, qui portait aussi le nom de Cosa.\par
\labelchar{XL.} Cependant on publiait dans l’Italie qu’un miracle des dieux avait sauvé Agrippa : on le croyait à Rome ; et déjà l’imposteur, débarqué à Ostie, avait été reçu par une multitude immense ; déjà dans Rome même il se trouvait à des réunions clandestines. Tibère éprouvait une vive anxiété, ne sachant s’il emploierait à réduire son esclave les armes des soldats ou s’il attendrait que l’illusion se dissipât d’elle-même. Persuadé tantôt que nul péril n’est à mépriser, tantôt qu’il ne faut pas s’alarmer de tout, combattu par la honte et par la crainte, il finit par s’en remettre à Crispus Sallustius. Celui-ci choisit deux de ses clients (quelques-uns disent deux soldats), et les charge de se présenter comme de nouveaux auxiliaires au faux Agrippa, et de lui offrir leur bourse, leur foi et leur épée. Ils font ce qui est commandé. Ensuite ils profitent d’une nuit où le fourbe n’était pas sur ses gardes, et, appuyés d’une force suffisante, ils le traînent lié et bâillonné au palais impérial. Là, interrogé par Tibère comment il était devenu Agrippa, on prétend qu’il répondit : « Comme toi César. » On ne put le contraindre à nommer ses complices. Tibère, n’osant hasarder en public le supplice de cet homme, ordonna qu’il fût tué dans un coin du palais, et que son corps fût emporté secrètement. Et, quoiqu’on assurât que beaucoup de personnes de la maison du prince, ainsi que des chevaliers et des sénateurs, l’avaient soutenu de leurs richesses ou aidé de leurs conseils, il ne se fit aucune recherche.\par
\labelchar{XLI.} À la fin de l’année on dédia un arc de triomphe, élevé près du temple de Saturne, en mémoire des aigles de Varus reconquises par les armes de Germanicus et sous les auspices de Tibère ; un temple de la déesse Fors Fortuna, bâti près du Tibre, dans les jardins légués par le dictateur César au peuple romain ; enfin, à Boville \footnote{Le pantomime Mnester.}, un sanctuaire consacré à la famille des Jules, et une statue de l’empereur Auguste. Sous le consulat de C. Cécilius et de L. Pomponius, le sept avant les calendes de juin, Germanicus César triompha des Chérusques, des Chattes, des Ampsivariens et des autres nations qui habitent jusqu’à l’Elbe. Les dépouilles, les captifs, les représentations des montagnes, des fleuves, des batailles, précédaient le vainqueur. On lui comptait comme finie cette guerre qu’un pouvoir supérieur l’avait seul empêché de finir. Ce qui attachait surtout les regards, c’était son air majestueux, et son char couvert de ses cinq enfants. Mais de tristes pressentiments venaient à la pensée, quand on se rappelait l’affection publique placée, avec peu de bonheur, sur son père Drusus ; son oncle Marcellus enlevé si jeune aux adorations de l’empire ; les amours du peuple romain si courtes et si malheureuses.\par
\labelchar{XLII.} Tibère donna au peuple trois cents sesterces par tête, au nom de Germanicus, et voulut être son collègue dans le consulat. Toutefois ces marques de tendresse n’en imposèrent à personne ; et bientôt il résolut de l’éloigner sous un prétexte honorable, dont il saisit l’occasion, s’il ne la fit pas naître. Archélaüs, qui depuis cinquante ans régnait en Cappadoce \footnote{La Cappadoce est une contrée de l’Asie-Mineure, située entre la Cilicie, l’Arménie et le Pont-Euxin. Lorsqu’elle devint province romaine, Mazaca, qui était la capitale, reçut le nom de Césarée, en l’honneur de Tibère. 2. Tacite a raconté ci-dessus, I, LXXVIII, que le peuple demandait l’abolition du centième sur les ventes publiques, et que Tibère n’avait pu l’accorder. Ici le prince, enrichi des revenus de la Cappadoce, diminue cet impôt de moitié. 3. Partie la plus septentrionale de la Syrie. La ville principale était Samosate, aujourd’hui Sémisat.}, était haï de Tibère, auquel il n’avait rendu aucun hommage lorsque ce prince vivait à Rhodes. Archélaüs ne s’en était point dispensé par orgueil, mais par le conseil des amis d’Auguste, qui, à l’époque de la faveur de Caïus César et de sa mission en Orient, ne croyaient pas sans péril l’amitié de Tibère. Quand la postérité des Césars fut détruite, et Tibère maître de l’empire, il chargea sa mère d’écrire au roi une lettre, où, sans dissimuler les ressentiments de son fils, elle lui offrait un pardon généreux s’il venait le demander. C’était un piège pour l’attirer dans Rome : Archélaüs ne le vit point ou, craignant la violence, il feignit de ne pas le voir et se hâta de venir. Reçu durement par Tibère, puis accusé devant le sénat, et accablé, non par les faits, qui étaient controuvés, mais par le chagrin, la vieillesse et l’abaissement, insupportable aux rois, pour qui l’égalité même est un état si nouveau, une mort, peut-être volontaire, mit bientôt fin à ses jours. Son royaume fut réduit en province romaine, et Tibère déclara qu’avec le revenu de ce pays on pouvait abaisser l’impôt du centième \footnote{Les Chérusques, vainqueurs de Varus. Les Lombards, qui s’étaient récemment soustraits à la domination de Marodobuus.}, qu’en effet il diminua de moitié. Vers le même temps la mort d’Antiochus, roi de Commagène \footnote{Apollonide, à moitié chemin entre Sardes et Pergame, à 300 stades de distance l’une de l’autre.}, et celle de Philopator, roi de Cilicie, avaient mis le trouble parmi ces nations, où les uns voulaient pour maîtres les Romains, les autres de nouveaux rois. Enfin les provinces de Syrie et de Judée, écrasées sous le poids des tributs, imploraient un soulagement.\par
\labelchar{LIII.} Tibère rendit compte au sénat de toutes ces affaires et de celles d’Arménie, dont j’ai parlé plus haut. « L’Orient ne pouvait ; disait-il, être pacifié que par la sagesse de Germanicus. Son âge à lui-même penchait vers le déclin, et celui de Drusus n’était pas encore assez mûr. » Alors un décret fut rendu, qui attribuait à Germanicus les provinces d’outre-mer, avec une autorité supérieure à celle des lieutenants du sénat et du prince, dans tous les lieux où il se trouverait. Cependant Tibère avait retiré de la Syrie Silanus Créticus, dont la fille, promise à Néron, fils aîné de Germanicus, unissait les deux pères par des liens de famille. Il avait mis à sa place Cnéius Piso, violent de caractère, incapable d’égards, héritier de toute la fierté de son père, cet autre Pison qui, dans la guerre civile, voyant le parti vaincu se relever en Afrique, s’y distingua parmi les ennemis les plus acharnés de César, combattit ensuite sous Brutus et Cassius, enfin, autorisé à revenir à Rome, s’abstint de demander les honneurs, jusqu’à ce qu’on allât le prier d’accepter un consulat que lui offrait Auguste. Cet orgueil héréditaire était accru par la naissance et les richesses de sa femme Plancine. À peine il cédait le pas à Tibère : il regardait les enfants de ce prince avec le dédain d’un homme beaucoup au-dessus d’eux, et il ne doutait pas qu’on ne l’eût donné pour gouverneur à la Syrie afin qu’il tînt en respect l’ambition de Germanicus. Quelques-uns même ont pensé qu’il avait reçu de Tibère de secrètes instructions ; et il est certain que Livie avait recommandé à Plancine d’humilier Agrippine par toutes les prétentions d’une rivale. Car la cour était divisée en deux partis, dont l’un penchait secrètement pour Drusus, l’autre pour Germanicus. Tibère préférait Drusus comme le fils né de son sang ; quant à Germanicus, l’aversion de son oncle lui donnait un titre de plus à l’amour des autres. D’ailleurs sa naissance était supérieure du côté maternel, où il avait Marc-Antoine pour aïeul et Auguste pour grand-oncle ; tandis que le bisaïeul de Drusus était un simple chevalier romain, Pomponius Atticus, dont l’image semblait déparer celle des Claudes. Enfin Agrippine, femme de Germanicus, effaçait par sa fécondité et sa bonne renommée Livie \footnote{Livia ou Livilla, sœur de Germanicus et de Claude.}, femme de Drusus. Toutefois les deux frères vivaient dans une admirable union, que les querelles de leurs proches n’altérèrent jamais.\par
\labelchar{XLIV.} Peu de temps après, Drusus fut envoyé dans l’Illyricum, afin qu’il apprît la guerre et se conciliât l’affection des troupes. Tibère pensait qu’un jeune homme passionné pour les plaisirs de la ville serait mieux dans les camps, et il se croyait plus en sûreté lui-même, si ses deux fils avaient des légions sous leurs ordres. Du reste, les Suèves fournirent un prétexte en demandant des secours contre les Chérusques. En effet, délivrés, par la retraite des Romains \footnote{La retraite de Germanicus et de son armée.}, de toute crainte étrangère, les barbares, fidèles à leur coutume et animés alors pas une rivalité de gloire, avaient tourné leurs armes contre eux-mêmes. La puissance des deux peuples, la valeur des deux chefs, allaient de pair ; mais Maroboduus était roi, et, à ce titre, haï de sa nation ; Arminius, défenseur de la liberté, était chéri de la sienne.\par
\labelchar{XLV.} Aussi Arminius ne vit-il pas seulement ses vieux soldats, les Chérusques et leurs alliés, embrasser sa querelle : du sein même des États de Maroboduus, les Semnones et les Lombards, peuples suèves, accoururent sous ses drapeaux. Ce renfort lui donnait l’avantage, si Inguiomère, suivi de ses clients, n’eût passé à l’ennemi, défection causée par la seule honte d’obéir à son neveu, et de soumettre sa vieillesse aux ordres d’un jeune homme. Les deux armées se rangèrent en bataille avec une égale espérance. Et ce n’étaient plus ces Germains accoutumés à charger au hasard et par bandes éparses : de longues guerres contre nous leur avaient appris à suivre les enseignes, à se ménager des réserves, à écouter la voix des chefs. Arminius, à cheval, courait de rang en rang, montrant à ses guerriers « la liberté reconquise, les légions massacrées, et ces dépouilles, et ces armes romaines, que beaucoup d’entre eux avaient encore dans leurs mains. Qu’était-ce, au contraire, que Maroboduus ? un fuyard, qui s’était sauvé sans combat dans la forêt Hercynienne, et, du fond de cet asile, avait mendié la paix par des présents et des ambassades ; un traître à la patrie, un satellite de César \footnote{Strabon, VII, 1, § 3, nous fournit l’explication de ce reproche : Maroboduus avait habité Rome pendant sa jeunesse, et avait reçu des bienfaits d’Auguste. C’est après son retour en Germanie que, de simple particulier, il se fit chef de sa nation, établie, suivant l’opinion la plus commune entre le Rhin, le Mein et le Danube, et la transplanta en Bohême.} qu’il fallait poursuivre avec cette même furie qui les animait quand ils tuèrent Varus. Qu’ils se souvinssent seulement de toutes ces batailles dont le succès, couronné enfin par l’expulsion des Romains montrait assez à qui était resté l’honneur de la guerre. »\par
\labelchar{XLVI.} Maroboduus n’était pas moins prodigue d’éloges pour lui-même, d’injures contre l’ennemi. Tenant Inguiomère par la main, « Voilà, disait-il, le véritable héros des Chérusques ; voilà celui dont les conseils ont préparé tout ce qui a réussi. » Puis il peignait Arminius comme « un furieux dénué d’expérience, qui se parait d’une gloire étrangère, pour avoir surpris, à force de perfidie, trois légions incomplètes et un chef trop confiant ; succès funeste à la Germanie et honteux à son auteur, dont la femme, dont le fils, subissaient encore l’esclavage \footnote{Voy. livre I, ch. LVIII.}. Lui, au contraire, menacé par douze légions ayant Tibère à leur tête, il avait conservé sans tache l’honneur des Germains et traité ensuite d’égal à égal : et certes il ne regrettait pas d’avoir mis son pays dans une position telle envers les Romains, qu’il pût choisir entre une guerre où ses forces seraient entières, et une paix qui n’avait point coûté de sang. » Outre l’effet de ces discours, des motifs particuliers aiguillonnaient encore les deux armées : les Chérusques et les Lombands combattaient pour une ancienne gloire ou une liberté récente (2), les Suèves pour étendre leur domination. Jamais choc ne fut plus violent, ni bataille plus indécise. De chaque côté l’aile droite fut mise en déroute. On s’attendait à une nouvelle action, quand Maroboduus se replia sur les hauteurs : ce fut le signe et l’aveu de sa défaite. Affaibli peu à peu par la désertion, il se retira chez les Marcomans et députa vers Tibère pour implorer des secours. On lui répondit « qu’il n’avait aucun droit d’invoquer les armes romaines contre les Chérusques, puisqu’il n’avait rien fait pour les Romains dans leurs guerres avec ce peuple. » Cependant Drusus, ainsi que nous l’avons dit, fut envoyé comme médiateur.\par
\labelchar{XLVII.} Cette même année, douze villes considérables de l’Asie furent renversées par un tremblement de terre qui eut lieu pendant la nuit, ce qui rendit le désastre plus imprévu et plus terrible, et l’on n’eut pas la ressource ordinaire en ces catastrophes de fuir dans la campagne, les terres entrouvertes n’offrant que des abîmes. On rapporte que de hautes montagnes s’affaissèrent, que des plaines s’élevèrent en collines, que des feux jaillirent du milieu de ce bouleversement. Sardes, la plus cruellement frappée, fut la plus généreusement secourue : César lui promit dis millions de sesterces, et la déchargea pour cinq ans de tout ce qu’elle payait à l’état ou au prince. Magnésie de Sipyle \footnote{Magnésie, située au pied du mont Sipyle, à la gauche de l’Hermus, aujourd’hui Magnia. Son surnom la distingue d’une autre Magnésie sur le Méandre.} reçut, après Sardes, le plus de dommage et de soulagement. Temnos, Philadelphie, Éges \footnote{Strabon, XIII, III, § 5, compte parmi les cités éoliques de l’Asie Éges et Temnos, patrie du rhéteur Hermagoras. Philadelphie, à l’orient de Sardes, au pied du mont Tmolus, avait été fondée par Attale Philadelphe, frère d’Eumène, roi de Pergame.}, Apollonide (3), Mostène \footnote{La ville de Mostène en Lydie est mentionnée dans les monuments et les géographes. – La plaine d’Hyrcanie avait reçu ce nom des Perses, à cause d’une colonie d’Hyrcaniens qu’ils y avaient amenée. Le surnom de \emph{Macedones} donné aux habitants semble indiquer que leur ville avait été fondée ou du moins agrandie par les Macédoniens.}, Hyrcanie la Macédonienne, Hiérocésarée \footnote{Ville de Lydie, célèbre par un temple de Diane persique.}, Myrine, Cymé, Tmolus \footnote{Myrine, ville maritime de l’Eolide, qui prenait le surnom de Sébastopolis. – Cymé, sur la même côte, à 9 milles de Myrine, était la plus puissante des colonies éoliques. D’Anville dit qu’on en a trouvé des vestiges dans un lieu appelé \emph{Nemourt}. – Tmolus, près de la montagne du même nom, d’où sort le Pactole, ce fleuve autrefois si renommé, mais qui, dès le temps de Strabon, ne roulait plus de paillettes d’or.}, furent exemptées de tributs pour le même temps ; et l’on décida qu’un sénateur irait sur les lieux examiner le mal et le réparer. On choisit un simple ex-préteur M. Alétus, de peur que, l’Asie étant gouvernée par un consulaire, il ne survînt entre deux hommes de même rang des rivalités qui nuiraient à la province.\par
\labelchar{XVLIII.} César couronna ces grandes libéralités publiques par des traits de générosité qui ne furent pas moins agréables. Les biens d’Émilia Musa, femme opulente, morte sans testament, étaient réclamés par le fisc : il les fit donner à Émilius Lépidus, parce qu’Émilie paraissait être de sa maison. Patuléius, riche chevalier romain, lui avait légué une partie de son héritage : il l’abandonna tout entier à M. Servilius, en faveur duquel un testament antérieur et non suspect en avait disposé. Il déclara que ces deux sénateurs avaient besoin de fortune pour soutenir leur naissance. Jamais il n’accepta de legs qu’il ne les eût mérités à titre d’ami : les inconnus, et ceux qui ne le nommaient dans un testament qu’en haine de leurs proches, furent toujours repoussés. Du reste, s’il soulagea la pauvreté honnête et vertueuse, il exclut du sénat ou laissa se retirer d’eux-mêmes, des hommes que la prodigalité et le vice avaient réduits à l’indigence, Vibidius Varro, Marius Népos, Appius Appianus, Cornélius Sylla et Q. Vitellius \footnote{Oncle de Vitellius, qui depuis fut empereur.}.\par
\labelchar{XLIX.} Vers la même époque il dédia quelques temples que le feu ou les ans avaient ruinés, et qu’Auguste avait commencé à rebâtir : celui de Bacchus, Cérès et Proserpine, près du grand Cirque, voué anciennement par le dictateur A. Postumius \footnote{L’an de Rome 257, avant la bataille du lac Régille.} ; celui de Flore, élevé au même endroit par les édiles Lucius et Marcus Publicius \footnote{L’an de Rome 513.} ; celui de Janus \footnote{Ne pas confondre ce temple avec celui qui avait été bâti par Numa, et dont les portes, ouvertes ou fermées, étaient le signe de la guerre ou de la paix.}, bâti près du marché aux légumes par Duillius, qui le premier illustra sur mer les armes romaines et mérita, par la défaite des Carthaginois, le triomphe naval. Le temple de l’Espérance fut inauguré par Germanicus : Atilius \footnote{Non pas le fameux Atilius Régulus, mais le consul Atilius Calatinus.} l’avait voué dans la même guerre.\par
\labelchar{L.} Cependant la loi de majesté prenait vigueur : elle fut invoquée contre Apuléia Varilia, petite-nièce d’Auguste, qu’un délateur accusait d’avoir fait de ce prince, de Tibère et d’Augusta, le sujet d’un injurieux badinage, et de souiller par l’adultère le sang des Césars. On jugea que l’adultère était assez réprimé par la loi Julia \footnote{Voy. liv. IV., chap. XLII.} : quant au crime de lèse-majesté, le prince demanda qu’on fît une distinction, et qu’en punissant les discours qui auraient outragé la divinité d’Auguste, on s’abstînt de rechercher ceux qui ne blessaient que lui. Prié par le consul de s’expliquer sur les propos contre sa mère imputés à Varilia il garda le silence ; mais à la séance suivante, il demanda aussi au nom d’Augusta que jamais, en quelques termes qu’on eût parlé d’elle, on ne fût accusé pour ce fait. Il déchargea Varilia du crime de lèse-majesté : il fit même, adoucir en sa faveur les peines de l’adultère, et fut d’avis que sa famille la reléguât, selon l’ancien usage, à deux cents milles de Rome. L’Italie et l’Afrique furent interdites à son complice Manlius.\par
\labelchar{LI.} Le choix d’un préteur pour remplacer Vipsanius Gallus, qui venait de mourir, excita quelques débats. Germanicus et Drusus (car ils étaient encore à Rome) soutenaient Hatérius Agrippa, parent de Germanicus. Un parti nombreux réclamait l’exécution de la loi d’après laquelle le candidat qui a le plus d’enfants doit être préféré. Tibère voyait avec plaisir le sénat balancer entre ses fils et les lois : la loi fut vaincue, et cela devait être ; mais elle ne le fut pas sans opposition ; elle ne le fut qu’à une faible majorité, comme les lois avaient coutume d’être vaincues dans le temps même de leur puissance.\par
\labelchar{LII.} Cette même année, la guerre commença en Afrique contre Tacfarinas. C’était un Numide, déserteur des armées romaines, où il avait servi comme auxiliaire. Il réunit d’abord, pour le vol et le butin, des bandes vagabondes, accoutumées au brigandage : bientôt il sut les discipliner, les ranger sous le drapeau, les distribuer en compagnies ; enfin, de chef d’aventuriers, il devint général des Musulames. Ce peuple puissant, qui confine aux déserts de l’Afrique, et qui alors n’avait point encore de villes, prit les armes et entraîna dans la guerre les Maures, ses voisins : ceux-ci avaient pour chef Mazippa. Les forces furent partagées : Tacfarinas se chargea de tenir dans des camps et d’habituer à l’obéissance et à la discipline les hommes d’élite, armés à la romaine, tandis que Mazippa, avec les troupes légères, porterait partout l’incendie, le carnage et la terreur. Déjà ils avaient forcé les Cinithiens, nation considérable, de se joindre à eux, lorsque Furius Camillus, proconsul d’Afrique, après avoir réuni sa légion et ce qu’il y avait d’auxiliaires sous les étendards, marcha droit à l’ennemi. C’était une poignée d’hommes, eu égard à la multitude des Numides et des Maures ; mais on évitait surtout d’inspirer à ces barbares une crainte qui leur eût fait éluder nos attaques : en leur faisant espérer la victoire, on réussit à les vaincre. La légion fut placée au centre, les cohortes légères et deux ailes de cavalerie sur les flancs. Tacfarinas ne refusa pas le combat. Les Numides furent défaits ; et la gloire des armes, après de longues années, rentra dans la maison des Furius. Car, depuis le libérateur de Rome et Camillus son fils, l’honneur de gagner des batailles était passé à d’autres familles : encore le Furius dont nous parlons n’était-il pas regardé comme un grand capitaine. Tibère en fit plus volontiers devant le sénat l’éloge de ses exploits. Les pères conscrits lui décernèrent les ornements du triomphe, distinction qui, grâce au peu d’éclat de sa vie, ne lui devint pas funeste.\par
\labelchar{LIII.} L’année suivante, Tibère fut consul pour la troisième fois, Germanicus pour la seconde. Germanicus prit possession du consulat à Nicopolis \footnote{Colonie romaine fondée par Auguste en mémoire de la bataille d’Actium. On en trouve des ruines près de Prevezza-Vecchia.}, ville d’Achaïe, où il venait d’arriver après avoir côtoyé l’Illyrie, vu en Dalmatie son frère Drusus, et essuyé sur la mer Adriatique et sur la mer Ionienne les traverses d’une navigation difficile : aussi employa-t-il quelques jours à réparer sa flotte. Pendant ce temps, il visita le golfe fameux par la victoire d’Actium, les monuments consacrés par Auguste et le camp de Marc-Antoine, l’imagination toute pleine de ses aïeux. Il était, comme je l’ai dit, petit-neveu d’Auguste, petit-fils d’Antoine, et ces lieux réveillaient en lui de grands souvenirs de deuil et de triomphe. De là il se rendit à Athènes, et, par égard pour une cité ancienne et alliée, il y parut avec un seul licteur. Les Grecs lui prodiguèrent les honneurs les plus recherchés, ayant soin, pour ajouter du prix à l’adulation, de mettre en avant les actions et les paroles mémorables de leurs ancêtres.\par
\labelchar{LIV.} D’Athènes, Germanicus passa dans l’île d’Eubée, puis dans celle de Lesbos, où Agrippine mit au monde Julie, son dernier enfant. Ensuite il longea les extrémités de l’Asie, visita dans la Thrace Périnthe \footnote{Périnthe, ville de Thrace, sur les bords de la Propontide ou mer de Marmara. Elle fut plus tard appelée Héraclée, nom qui subsiste encore dans celui d’Erékli. 2. Les mystères de Samothrace, île de la mer Égée, à la hauteur de la Chersonèse de Thrace, étaient célèbres dans tout l’univers. Ils passaient pour les plus anciens de la Grèce, et pour avoir donné naissance à ceux d’Éleusis.} et Byzance, et pénétra, par la Propontide, jusqu’à l’embouchure de l’Euxin, curieux de connaître ses lieux antiques et renommés. En même temps il soulageait les maux des provinces déchirées par la discorde ou opprimées par leurs magistrats. Il voulait, à son retour, voir les mystères de Samothrace \footnote{Un des principaux fleuves de la \emph{Cilicia campestris}. Il se nomme aujourd’hui Geihoun, ou plutôt Djihoun.} ; mais les vents du nord l’écartèrent de cette route. Après s’être donné à Ilion le spectacle des choses humaines, et avoir contemplé avec respect le berceau des Romains, il côtoie de nouveau l’Asie et aborde à Colophon, pour consulter l’oracle d’Apollon de Claros. L’interprète du dieu n’est point une femme, comme à Delphes : c’est un prêtre, choisi dans certaines familles et ordinairement à Milet. Il demande seulement le nombre et le nom des personnes qui se présentent : puis il descend dans une grotte, boit de l’eau d’une fontaine mystérieuse ; et cet homme, étranger le plus souvent aux lettres et à la poésie, répond en vers à la question que chacun lui fait par la pensée On a dit que celui-ci avait annoncé à Germanicus, dans le langage ambigu des oracles, une mort prématurée.\par
\labelchar{LV.} Cependant, afin de commencer plus tôt l’exécution de ses desseins, Pison, après avoir porté l’effroi dans Athènes par le fracas de son entrée, adressa aux habitants une sanglante invective, où il blâmait indirectement Germanicus « d’avoir, à la honte du nom romain, traité avec un excès déconsidération, non les Athéniens (après tant de désastres il n’en restait plus), mais une populace, vil ramas de toutes les nations, qui fut l’alliée de Mithridate contre Sylla, d’Antoine contre Auguste. » Il allait chercher aussi dans le passé leurs guerres malheureuses avec la Macédoine, les violences d’Athènes envers ses propres citoyens, et leur reprochait ces faits avec une animosité que redoublait encore un motif personnel, la ville lui refusant la grâce d’un certain Théophile, que l’Aréopage avait condamné comme faussaire. Ensuite, par une navigation rapide à travers les Cyclades, et en prenant les routes les plus courtes, il atteignit Germanicus à Rhodes. Celui-ci n’ignorait pas de quelles insultes il avait été l’objet ; mais telle était la générosité de son âme, que, voyant une tempête emporter sur des écueils le vaisseau de Pison, et pouvant laisser périr un ennemi dont la mort eût été attribuée au hasard, il envoya des galères à son secours et le sauva du danger. Loin d’être désarmé par ce bienfait, Pison contint à peine un seul jour son impatience : il quitte Germanicus, le devance ; et, arrivé en Syrie, il s’attache à gagner l’armée à force de largesses et de complaisances, prodigue les faveurs aux derniers des légionnaires, remplace les vieux centurions et les tribuns les plus fermes par ses clients ou par des hommes décriés, encourage l’oisiveté dans le camp, la licence dans les villes, laisse errer dans les campagnes une soldatesque effrénée ; corrupteur de la discipline à ce point que la multitude ne le nommait plus que le père des légions. Plancine, de son côté, oubliant les bienséances de son sexe, assistait aux exercices de la cavalerie, aux évolutions des cohortes, se répandait en injures contre Agrippine, contre Germanicus. Quelques-uns même des meilleurs soldats secondaient par zèle ces coupables menées, parce qu’un bruit sourd s’était répandu que rien ne se faisait sans l’aveu de l’empereur.\par
\labelchar{LVI.} Germanicus était instruit de tout ; mais son soin le plus pressant fut de courir en Arménie. De tout temps la foi de ce royaume fut douteuse, à cause du caractère des habitants et de la situation du pays, qui borde une grande étendue de nos provinces, et de l’autre côté s’enfonce jusqu’aux Mèdes. Placés entre deux grands empires, les Arméniens sont presque toujours en querelle, avec les Romains par haine, par jalousie avec les Parthes. Depuis l’enlèvement de Vonon, ils n’avaient point de roi ; mais le vœu de la nation se déclarait en faveur de Zénon. Ce prince, fils de Polémon, roi de Pont, en imitant dès son enfance les usages et la manière de vivre des Arméniens, leurs chasses, leurs festins, et tous les goûts des barbares s’était également concilié les grands et le peuple. Germanicus se rend donc dans la ville d’Artaxate, et, du consentement des nobles, aux acclamations de la multitude, il le ceint du bandeau royal. Le peuple se prosterna devant son nouveau maître et le salua du nom d’Artaxias, formé de celui de la ville. La Cappadoce, qui venait d’être réduite en province romaine, reçut pour gouverneur Q. Véranius, et l’on diminua quelque chose des tributs qu’elle payait à ses rois, afin qu’elle passât sous notre empire avec d’heureuses espérances. Q. Servéus fut mis à la tête de la Commagène, qui recevait pour la première fois un préteur.\par
\labelchar{LVII.} La paix si heureusement rétablie parmi les alliés ne donnait à Germanicus qu’une joie imparfaite, à cause de l’orgueil de Pison, auquel il avait commandé de mener en Arménie une partie de l’armée, soit en personne, soit par son fils, et qui s’était dispensé de le faire. Ils eurent enfin à Cyrrhe \footnote{Ville de Syrie, dans la Cyrrhestique, à deux journées d’Antioche.}, au camp de la dixième légion, une entrevue, où tous deux ce composèrent le visage, pour n’avoir pas l’apparence, Pison de la crainte, Germanicus de la menace. Celui-ci d’ailleurs était, comme je l’ai dit, naturellement doux ; mais ses amis, habiles à aigrir ses ressentiments, exagéraient les torts réels, en supposaient d’imaginaires, inculpaient de mille manières et Pison, et Plancine, et leurs enfants. L’entretien eut lieu en présence de quelques amis : Germanicus commença dans les termes que pouvaient suggérer la colère et la dissimulation ; Pison répondit par d’insolentes excuses, et ils se séparèrent la haine dans le cœur. Depuis ce temps, Pison parut rarement au tribunal de Germanicus ; et, s’il y siégeait quelquefois, c’était avec un air mécontent et un esprit d’opposition qu’il ne cachait pas. On l’entendit même, à un festin chez le roi des Nabatéens, où des couronnes d’or d’un grand poids furent offertes à César et à sa femme, de plus légères à Pison et aux autres, s’écrier que « c’était au fils du prince des Romains, et non à celui du roi des Parthes, que ce repas était donné. » En même temps il jeta sa couronne et se déchaîna contre le luxe. Ces outrages, tout cruels qu’ils étaient, Germanicus les dévorait cependant.\par
\labelchar{LVIII.} Sur ces entrefaites arrivèrent des ambassadeurs d’Artaban, roi des Parthes. Ils rappelèrent en son nom l’alliance et l’amitié qui unissait les deux empires, ajoutant « qu’il désirait les renouveler en personne, et que, par honneur pour Germanicus, il viendrait jusqu’au bord de l’Euphrate : il demandait, en attendant, qu’on éloignât Vonon de la Syrie, d’où, à la faveur du voisinage, ses émissaires excitaient à la révolte les grands du royaume. » Germanicus répondit avec une noble fierté sur l’alliance des Romains et des Parthes, avec une dignité modeste sur la déférence que le roi lui marquait en venant à sa rencontre. Vonon fut conduit à Pompéiopolis, ville maritime de Cilicie : c’était tout ensemble une satisfaction donnée au monarque, et un affront fait à Pison, auquel Vonon s’était rendu agréable par les soins et les présents qu’il prodiguait à Plancine.\par
\labelchar{LIX.} Sous le consulat de M. Silanus et de L. Norbanus, Germanicus partit pour l’Égypte, afin d’en visiter les antiquités : les besoins de la province lui servirent de prétexte. Il fit baisser le prix des grains en ouvrant les magasins, et charma les esprits par une conduite toute populaire, comme de marcher sans gardes, avec la chaussure et le vêtement grecs, à l’exemple de Scipion, qui, au plus fort de la guerre punique, en avait usé de même en Sicile. Tibère, après avoir blâmé en termes mesurés cette parure étrangère, se plaignit vivement de ce que, au mépris des lois d’Auguste, Germanicus était entré dans Alexandrie sans l’aveu du prince. Car Auguste, parmi d’autres maximes d’État, s’en fit une de séquestrer l’Égypte, en défendant aux sénateurs et aux chevaliers romains du premier rang d’y aller jamais qu’il ne l’eût permis. Il craignait que l’Italie ne fût affamée par le premier ambitieux qui s’emparerait de cette province, où, tenant les clefs de la terre et de la mer, il pourrait se défendre avec très peu de soldats contre de grandes armées.\par
\labelchar{LX.} Cependant Germanicus ignorait encore qu’on lui fit un crime de son voyage, et déjà il remontait le Nil, après s’être embarqué à Canope. Cette ville fut fondée par les Spartiates, en mémoire d’un de leurs pilotes, enseveli sur ces bords à l’époque où Ménélas, retournant en Grèce, fut écarté de sa route et poussé jusqu’aux rivages de Libye. De Canope, Germanicus était entré dans le fleuve par l’embouchure voisine, consacrée à Hercule, lequel, selon les Égyptiens, est né dans ce pays, et a précédé tous les autres héros émules de sa valeur et appelés de son nom. Bientôt il visita les grandes ruines de Thèbes. Des caractères égyptiens \footnote{Les hiéroglyphes. 2. Le même que Sésostris.}, tracés sur des monuments d’une structure colossale, attestaient encore l’opulence de cette antique cité. Un vieux prêtre, qu’il pria de lui expliquer ces inscriptions, exposait « que la ville avait contenu jadis sept cent mille hommes en âge de faire la guerre ; qu’à leur tête le roi Rhamsès (2) y avait conquis la Libye, l’Éthiopie, la Médie, la Perse, la Bactriane, la Scythie ; que tout le pays qu’habitent les Syriens, les Arméniens, et, en continuant par la Cappadoce, tout ce qui s’étend de la mer de Bithynie à celle de Lycie, avait appartenu à son empire. » On lisait, sur ces mêmes inscriptions, le détail des tributs imposés à tant de peuples, le poids d’or et d’argent, la quantité d’armes et de chevaux, les offrandes pour les temples, en parfums et en ivoire, le blé et les autres provisions que chaque nation devait fournir : tributs comparables par leur grandeur à ceux que lèvent de nos jours la monarchie des Parthes ou la puissance romaine.\par
\labelchar{LXI.} D’autres merveilles attirèrent encore les regards de Germanicus : il vit la statue en pierre de Memnon, qui, frappée des rayons du soleil, rend le son d’une voix humaine ; et ces pyramides, semblables à des montagnes, qu’élevèrent, au milieu de sables mouvants et presque inaccessibles, l’opulence et l’émulation des rois ; et ces lacs \footnote{Le lac Moeris. 2. Allusion aux conquêtes de Trajan en Arabie, en Mésopotamie et en Assyrie. Les anciens étendaient la dénomination de mer Rouge jusqu’à l’Océan indien.} creusés pour recevoir les eaux surabondantes du Nil débordé, et ailleurs ce même fleuve pressé entre ses rives et coulant dans un lit dont nul homme n’a jamais pu sonder la profondeur. De là il se rendit à Éléphantine et à Syène, où furent jadis les barrières de l’empire romain, reculées maintenant jusqu’à la mer Rouge (2).\par
\labelchar{LXII.} Pendant que Germanicus employait l’été à parcourir les provinces, Drusus se fit honneur par l’adresse avec laquelle il sut diviser les Germains, et susciter à Maroboduus, déjà si ébranlé, une guerre qui achevât de l’abattre. Il y avait parmi les Gothons un jeune homme d’une haute naissance, nommé Catualda, jadis obligé de fuir devant la puissance de Maroboduus, et que les malheurs de son ennemi enhardirent à se venger. Il entre en force chez les Marcomans ; et, soutenu des principaux de la nation, qu’il avait corrompus, il s’empare de la résidence royale et du château qui la défendait. Il y trouva du butin depuis longtemps amassé par les Suèves, ainsi que des vivandiers et des marchands de nos provinces, que la liberté du commerce, puis l’amour du gain, enfin l’oubli de la patrie, avaient arrachés à leurs foyers et fixés dans ces terres ennemies.\par
\labelchar{LXIII.} Maroboduus, abandonné de toutes parts, n’eut de ressource que dans la pitié de Tibère. Il passa le Danube, à l’endroit où ce fleuve borde la Norique, et il écrivit au prince, non comme un fugitif, ou un suppliant, mais en homme qui se souvenait de sa première fortune. « Beaucoup de nations, disait-il, appelaient à elles un roi naguère si fameux ; mais il avait préféré l’amitié des Romains. » César lui répondit « qu’un asile sûr et honorable lui était ouvert en Italie, tant qu’il y voudrait demeurer ; que, si son intérêt l’appelait ailleurs, il en sortirait aussi librement qu’il y serait venu. » Au reste, il dit dans le sénat, « que ni Philippe n’avait été aussi redoutable pour les Athéniens, ni Pyrrhus et Antiochus pour le peuple romain. » Son discours existe encore : il y relève la grandeur de Maroboduus, la force irrésistible des nations qui lui étaient soumises, le danger d’avoir si près de l’Italie un pareil ennemi, et les mesures qu’il avait prises pour amener sa chute. On plaça Maroboduus à Ravenne, d’où il servit à contenir l’insolence des Suèves, que l’on tenait perpétuellement sous la menace de son retour. Toutefois, il ne quitta pas l’Italie pendant les dix-huit ans qu’il vécut encore, et il vieillit dans cet exil, puni, par la perte de sa renommée, d’avoir trop aimé la vie. Catualda tomba comme lui, et, comme lui, eut recours à Tibère : chassé, peu de temps après son rival, par une armée d’Hermondures, sous les ordres de Vibillius, il fut accueilli dans l’empire et envoyé à Fréjus, colonie de la Gaule narbonnaise. De peur que les barbares venus à la suite des deux rois ne troublassent, par leur mélange avec les populations, la paix de nos provinces, ils furent établis au-delà du Danube, entre le Maros et le Cuse \footnote{La Morava ou March, en Moravie, et le Waag, en Hongrie.}, et reçurent pour roi Vannius, de la nation des Quades.\par
\labelchar{LXIV.} Comme on apprit dans le même temps qu’Artaxias venait d’être mis par Germanicus sur le trône d’Arménie, un sénatus-consulte décerna l’ovation à Germanicus et à Drusus ; et, des deux côtés du temple de Mars vengeur \footnote{Bâti par Auguste, en conséquence d’un vœu qu’il avait fait pendant qu’il combattait contre Brutus et Cassius pour venger la mort de son père.}, furent élevés des arcs de triomphes où l’on plaça leurs statues. Tibère était plus satisfait d’avoir assuré la paix par sa politique, que s’il eût terminé la guerre par des victoires. Aussi eut-il recours aux mêmes armes contre Rhescuporis, roi de Thrace. Rhémétalcès avait possédé seul tout ce royaume. À sa mort, Auguste le partagea entre Rhescuporis, son frère, et Cotys son fils. Cotys eut les terres cultivées, les villes, et ce qui touche à la Grèce ; les pays incultes, sauvages, voisins des nations ennemies, échurent à Rhescuporis : partage assorti au caractère des deux princes, l’un d’un esprit doux et agréable, l’autre farouche, ambitieux incapable de souffrir un égal. Ils vécurent d’abord dans une intelligence trompeuse : bientôt Rhescuporis franchit ses limites, entreprend sur les États de Cotys, et, si l’on résiste, il emploie la violence, avec hésitation sous Auguste, par qui tous deux régnaient, et qu’il n’osait braver dans la crainte de sa vengeance, mais plus hardiment depuis le changement de prince : alors il détachait des troupes de brigands, ruinait les forteresses, faisait tout pour amener la guerre.\par
\labelchar{LXV.} Tibère n’appréhendait rien tant que de voir la paix troublée quelque part. Il envoie un centurion défendre aux deux rois de vider leur querelle par les armes. Cotys congédie à l’instant les troupes qu’il avait rassemblées. Rhescuporis, avec une feinte modération, demande une entrevue : « Une seule conférence, pouvait, disait-il, terminer leurs débats. » On convint sans peine du temps, du lieu, et ensuite des conditions, la facilité d’une part et la perfidie de l’autre faisant tout accorder et tout accepter. Rhescuporis, sous prétexte de sceller la réconciliation, donna un festin, dont la joie, animée par le vin et la bonne chère, se prolongea bien avant dans la nuit. Cotys, sans défiance, s’aperçoit trop tard qu’il est trahi ; et, tout en invoquant le nom sacré de roi, les dieux de leur famille, les privilèges de la table hospitalière, il est chargé de fers. Son rival, en possession de toute la Thrace, écrivit à Tibère qu’un complot avait été formé contre sa personne, et qu’il en avait prévenu l’exécution. Et, alléguant une guerre contre les Bastarnes \footnote{Les Bastarnes habitaient au nord du Danube et s’étendaient jusqu’à l’embouchure de ce fleuve} et les Scythes, il se renforçait de nouvelles troupes d’infanterie et de cavalerie.\par
\labelchar{LXVI.} Tibère lui répondit avec ménagement « que, s’il avait agi sans fraude, il devait se reposer sur son innocence ; qu’au reste ni lui ni le sénat ne pourraient discerner qu’après un mûr examen le tort du bon droit ; qu’il livrât donc Cotys, et qu’en venant lui-même il détournât sur son adversaire le soupçon du crime. » Latinius Pandus, propréteur de Mésie, lui envoya cette lettre, avec des soldats chargés de recevoir Cotys. Rhescuporis, combattu quelque temps par la crainte et par la colère, aima mieux avoir à répondre d’un attentat consommé que d’être coupable à demi : il fait tuer Cotys, et publie qu’il s’est donné la mort. Cependant Tibère ne renonça pas à sa politique artificieuse : Pandus que Rhescuporis accusait d’être son ennemi personnel, venait de mourir ; il mit à sa place Pomponius Flaccus, homme éprouvé par de longs services, et qui, lié d’une étroite amitié avec le roi, en était plus propre à le tromper : c’est là surtout ce qui lui fit donner le gouvernement de la Mésie.\par
\labelchar{LXVII.} Flaccus passe dans la Thrace, et, calmant à force de promesses les craintes que donnait à Rhescuporis une conscience criminelle, il l’attire au milieu des postes romains. Là on l’entoure, comme par honneur, d’une garde nombreuse ; puis, à la persuasion des tribuns et des centurions, il s’engage plus avant ; et, tenu dans une captivité chaque jour moins déguisée, comprenant enfin qu’il ne peut plus reculer, il est traîné jusqu’à Rome. Il fut accusé devant le sénat par la veuve de Cotys, et condamné à rester en surveillance loin de son royaume. La Thrace fut partagée entre son fils Rhémétalcès, qui s’était opposé à ses desseins, et les enfants de Cotys. Ceux-ci étant très jeunes encore, on donna la régence de leurs États à Trébelliénus Rufus, ancien préteur, de même qu’autrefois on avait envoyé en Égypte M. Lépidus pour servir de tuteur aux enfants de Ptolémée \footnote{Immédiatement après la fin de la seconde guerre punique et avant la guerre de Macédoine.}. Rhescuporis fut conduit à Alexandrie, où une tentative d’évasion, réelle ou supposée, le fit mettre à mort.\par
\labelchar{LXVIII.} À la même époque, Vonon, relégué en Cilicie, comme je l’ai rapporté, corrompit ses gardiens et entreprit de se sauver en Arménie, de là chez les Albaniens et les Hénioques \footnote{Les Albaniens habitaient la partie orientale du Caucase, le long de la mer Caspienne. Les Hénioques étaient plus voisins du Pont-Euxin.}, enfin chez le roi ses Scythes, son parent. Sous prétexte d’une partie de chasse, il s’éloigne de la mer et s’enfonce dans les forêts : bientôt, courant de toute la vitesse de son cheval, il atteint le fleuve Pyrame (2). Les habitants, avertis de sa fuite, avaient rompu les ponts, et le fleuve n’était pas guéable. Arrêté sur la rive par Vibius Fronton, préfet de cavalerie \footnote{Le \emph{praefectus equitum} commandait une aile de cavalerie, et son grade répondait à celui de tribun dans une légion.}, Vonon est chargé de chaînes. Peu de temps après, un évocat \footnote{Les évocats formaient un corps particulier et portaient un cep de vigne comme les centurions.} nommé Remmius, qui gardait le roi avant son évasion, lui passa, comme par colère, son épée au travers du corps : on n’en fut que mieux persuadé qu’il était son complice, et qu’il l’avait tué pour prévenir ses révélations.\par
\labelchar{LXIX.} Cependant Germanicus, à son retour d’Égypte, trouva l’ordre qu’il avait établi dans les légions et dans les villes ou aboli ou remplacé par des règlements contraires. De là des reproches sanglants contre Pison, qui de son côté n’épargnait pas les offenses à César. Enfin Pison résolut de quitter la Syrie. Retenu par une maladie de Germanicus, à la nouvelle de son rétablissement, et pendant qu’on acquittait à Antioche les vœux formés pour la conservation de ce général, il fit renverser par ses licteurs l’appareil du sacrifice, enlever les victimes et disperser la multitude que cette fête avait rassemblée. Bientôt Germanicus eut une rechute, et Pison se rendit à Séleucie \footnote{On trouve dans la géographie ancienne treize villes nommées Séleucie. Celle où Pison s’embarqua était à quelques milles d’Antioche, près de l’embouchure de l’Oronte, et portait le surnom de \emph{Piera}, parce qu’elle était voisine d’une montagne à laquelle les Macédoniens avaient donné le nom de \emph{Pierus}.} pour en attendre les suites. Le mal, déjà violent, était aggravé par la persuasion où était César que Pison l’avait empoisonné. On trouvait aussi dans le palais, à terre et autour des murs, des lambeaux de cadavres arrachés aux tombeaux, des formules d’enchantements et d’imprécations, le nom de Germanicus gravé sur des lames de plomb, des cendres humaines à demi brûlées et trempées d’un sang noir, et d’autres symboles magiques, auxquels on attribue la vertu de dévouer les âmes aux divinités infernales. Enfin toutes les personnes envoyées par Pison étaient accusées de venir épier les progrès de la maladie.\par
\labelchar{LXX.} Ces noirceurs inspirèrent à Germanicus autant d’indignation que d’alarmes. « Si sa porte était assiégée, s’il lui fallait exhaler son dernier soupir sous les yeux de ses ennemis, que deviendrait sa malheureuse épouse ? Quel sort attendait ses enfants au berceau ? Le poison était donc trop lent ! On hâtait, on précipitait sa mort, afin d’être seul maître de la province et des légions. Mais Germanicus n’était pas encore délaissé à ce point, et le prix du meurtre ne resterait pas longtemps aux mains de l’assassin. » Il déclara, par lettres, à Pison, qu’il renonçait à son amitié. Plusieurs ajoutent qu’il lui ordonna de sortir de la province. Pison, sans tarder davantage, se mit en mer ; mais il s’éloignait avec une lenteur calculée, pour être plus tôt de retour si la mort de Germanicus lui ouvrait la Syrie.\par
\labelchar{LXXI.} César eut un rayon d’espérance qui le ranima quelques instants : ensuite ses forces l’abandonnèrent ; et, sentant approcher sa fin, il parla en ces termes à ses amis, rassemblés près de son lit : « Si je cédais à la loi de la nature, la plainte me serait encore permise, même envers les dieux, dont la rigueur prématurée m’enlèverait si jeune à mes parents, à mes enfants, à ma patrie : maintenant, frappé par le crime de Pison et de Plancine, je dépose dans vos cœurs mes dernières prières. Dites à mon père et à mon frère de quels traits cruels mon âme fut déchirée, quels pièges environnèrent mes pas, avant qu’une mort déplorable terminât la vie la plus malheureuse. Ceux que mes espérances ou les liens du sang intéressaient à mon sort, ceux même dont Germanicus vivant pouvait exciter l’envie, ne verront pas sans quelques larmes un homme jadis entouré de splendeur, échappé à tant de combats, périr victime des complots d’une femme. Vous aurez, vous, des plaintes à porter devant le sénat, les lois à invoquer. Le premier devoir de l’amitié n’est pas de donner à celui qui n’est plus de stériles regrets ; c’est de garder le souvenir de ce qu’il a voulu, d’accomplir ce qu’il a commandé. Les inconnus même pleureront Germanicus : vous, vous le vengerez, si c’était moi que vous aimiez plutôt que ma fortune. Montrez au peuple romain la petite-fille du divin Auguste, celle qui fut mon épouse ; nombrez-lui mes six enfants. La pitié sera pour les accusateurs ; et, quand le mensonge alléguerait des ordres impies, on refuserait de croire ou l’on ne pardonnerait pas. » Les amis de César lui jurèrent, en touchant sa main défaillante, de mourir avant de renoncer à le venger.\par
\labelchar{LXXII.} Alors, se tournant vers Agrippine, il la conjure, au nom de sa mémoire, au nom de leurs enfants, de dépouiller sa fierté, d’abaisser sous les coups de la fortune la hauteur de son âme, et, quand elle serait à Rome, de ne pas irriter par des prétentions rivales un pouvoir au-dessus du sien. À ces paroles, que tous purent entendre, il en ajouta d’autres en secret, et l’on croit qu’il lui révéla les dangers qu’il craignait de Tibère. Peu de temps après il expira, laissant dans un deuil universel la province et les nations environnantes. Les peuples et les rois étrangers le pleurèrent : tant il s’était montré affable aux alliés, clément pour les ennemis ; homme dont l’aspect et le langage inspiraient la vénération, et qui savait, dans un si haut rang, conserver cette dignité qui sied à la grandeur, et fuir l’orgueil qui la rend odieuse.\par
\labelchar{LXXIII.} Ses funérailles, sans images et sans pompe, furent ornées par l’éloge de sa vie et le souvenir de ses vertus. Plusieurs, trouvant dans sa figure, son âge, le genre de sa mort et le lieu même où il finit ses jours, le sujet d’un glorieux parallèle, comparaient sa destinée à celle du grand Alexandre. « Tous deux avaient eu en partage la beauté, la naissance, et tous deux, à peine sortis de leur trentième année, avaient péri par des trahisons domestiques, au milieu de nations étrangères. Mais Germanicus était doux envers ses amis, modéré dans les plaisirs, content d’un seul hymen et père d’enfants légitimes ; du reste non moins guerrier qu’Alexandre, bien qu’il fût moins téméraire, et qu’après tant de coups portés à la Germanie on l’eût empêché de la soumettre au joug. S’il eût été seul arbitre des affaires, s’il avait possédé le nom et l’autorité de roi, certes il aurait bien vite égalé, par la gloire des armes, le héros au-dessus duquel sa clémence, sa tempérance et ses autres vertus l’avaient tant élevé. » Son corps, avant d’être brûlé, fut exposé nu dans le Forum d’Antioche, lieu destiné à la cérémonie funèbre. Y parut-il quelque trace de poison ? Le fait est resté douteux : la pitié pour Germanicus, les préventions contraires ou favorables à Pison, donnèrent lieu à des conjectures tout opposées.\par
\labelchar{LXXIV.} Un conseil fut tenu entre les lieutenants et les sénateurs présents, pour décider à qui l’on confierait la Syrie. Vibius Marsus et Cn. Sentius partagèrent longtemps les suffrages, que les autres n’avaient que faiblement disputés. Vibius céda enfin à l’âge de son rival et à l’ardeur de sa poursuite. Il y avait dans la province une célèbre empoisonneuse, nommée Martina, fort aimée de Plancine : Sentius l’envoya à Rome, sur la demande de Vitellius, de Véranius et des autres amis de Germanicus, qui, sans attendre que leur accusation fût admise, préparaient déjà les moyens de conviction.\par
\labelchar{LXXV.} Agrippine, accablée de douleur, malade, et cependant impatiente de tout retardement qui différerait sa vengeance, s’embarque avec ses enfants et les cendres de Germanicus ; départ où l’on ne peut voir sans une émotion profonde cette femme, d’une si auguste naissance, parée naguère de l’éclat du plus noble mariage, naguère environnée de respects et d’adorations, porter maintenant dans ses bras des restes funèbres, incertaine si elle obtiendra justice, inquiète de sa destinée et malheureuse par sa fécondité même, qui l’expose tant de fois aux coups de la fortune. Pison apprit dans l’île de Cos que Germanicus avait cessé de vivre. À cette nouvelle, il ne se contient plus : il immole des victimes, court dans les temples, mêlant ses transports immodérés à la joie encore plus insolente de Plancine, qui, en deuil d’une sœur qu’elle avait perdue, reprit ce jour-là même des habits de fête.\par
\labelchar{LXXVI.} Les centurions arrivaient en foule, l’assurant du dévouement des légions, l’exhortant à reprendre une province qu’on n’avait pas eu le droit de lui ravir, et qui était sans chef. Il délibéra sur ce qu’il avait à faire, et son fils, Marcus Piso, fut d’avis qu’il se hâtât de retourner à Rome : « Il n’avait point jusqu’ici commis de crime inexpiable. Des soupçons vagues, de vaines rumeurs ne devaient point l’alarmer. Sa mésintelligence avec Germanicus pouvait lui mériter de la haine, mais non des châtiments. Par la perte de sa province, il avait satisfait à l’envie : s’il voulait y rentrer, la résistance de Sentius causerait une guerre civile. Quant aux centurions et aux soldats, il n’en fallait attendre qu’une foi peu durable, dont la mémoire récente de leur général et leur vieil attachement aux Césars triompheraient bientôt. "\par
\labelchar{LXXVII.} Domitius Céler, un de ses amis les plus intimes, dit au contraire « qu’il fallait profiter des conjectures ; que Pison, et non Sentius était gouverneur de Syrie ; qu’à lui seul avaient été donnés les faisceaux, l’autorité de préteur, le commandement des légions. S’il survenait une attaque de l’ennemi, à qui appartiendrait-il d’y opposer les armes, autant qu’à celui qui a reçu des pouvoirs directs et des instructions personnelles » Il faut laisser aux bruits les plus vains le temps de se dissiper : souvent l’innocence n’a pu résister aux premiers effets de la prévention. Mais Pison, à la tête d’une armée et accru de nouvelles forces, verra naître du hasard mille événements favorables, qu’on ne saurait prévoir. Nous presserons-nous d’arriver avec les cendres de Germanicus, afin que la tempête excitée par les gémissements d’Agrippine et les clameurs d’une multitude égarée vous emporte avant que votre voix ait pu se faire entendre ? Vous avez pour vous vos intelligences avec Augusta, la faveur de César ; mais c’est en secret, et nul ne pleure Germanicus avec plus d’ostentation que ceux à qui sa mort cause le plus de joie. »\par
\labelchar{LXXVIII.} Pison, qui aimait les partis violents, fut sans peine entraîné. Il écrivit à Tibère des lettres où il accusait Germanicus de faste et d’arrogance. « Chassé, ajoutait-il, pour que le champ restât libre à des projets ambitieux, la même fidélité qu’il avait montrée dans le commandement des légions lui avait fait un devoir de le reprendre. » En même temps il fit partir Domitius sur une trirème pour la Syrie, avec ordre d’éviter les côtes et de se tenir au large en passant devant les îles. Des déserteurs arrivaient de toutes parts : il les forme en compagnies, arme les valets d’armée, et, s’étant rendu avec sa flotte sur le continent, il intercepte un détachement de nouveaux soldats qui allait en Syrie, et mande aux petits souverains de la Cilicie de lui envoyer des secours. Le jeune Marcus, qui s’était prononcé contre la guerre, ne l’en secondait pas avec moins d’ardeur dans ces préparatifs.\par
\labelchar{LXXIX.} La flotte de Pison, en côtoyant les rivages de Lycie et de Pamphylie, rencontra les vaisseaux qui ramenaient Agrippine. Le premier mouvement, des deux côtés, fut d’apprêter ses armes ; et, des deux côtés, la crainte, plus forte que la colère, fit qu’on s’en tint aux injures. Marsus Vibius somma Pison de venir à Rome pour s’y justifier. Pison répondit avec ironie « qu’il y serait quand le préteur qui connaît des empoisonnements aurait fixé le jour à l’accusé et aux accusateurs. » Cependant Domitius avait abordé à Laodicée, ville de Syrie, et se rendait au camp de la sixième légion, qu’il croyait la plus disposée à servir ses desseins : il y fut prévenu par le lieutenant Pacuvius. Sentius annonça cette nouvelle à Pison, dans une lettre où il l’avertissait de ne plus attaquer l’armée par la corruption, la province par les armes : puis il rassemble tous ceux qu’il savait attachés à la mémoire de Germanicus et ennemis de ses persécuteurs ; et, invoquant la majesté de l’empereur, protestant que c’est à la république elle-même qu’on déclare la guerre, il se met en marche avec une troupe nombreuse et décidée à combattre.\par
\labelchar{LXXX.} Pison, qui voyait échouer ses tentatives, n’en prit pas moins les meilleures mesures que permît la circonstance : il s’empara d’un château très fort de Cilicie, nommé Célendéris. En mêlant les déserteurs, les recrues dernièrement enlevées, les esclaves de Plancine et les siens, aux troupes envoyées par les petits princes de Cilicie, il en avait formé l’équivalent d’une légion. Il attestait sa qualité de lieutenant de César. « C’était de César, disait-il, qu’il tenait sa province ; et il en était repoussé, non par les légions (elles-mêmes l’appelaient), mais par Sentius, qui cachait sous de fausses imputations sa haine personnelle. Qu’on se montrât seulement en bataille ; et les soldats de Sentius refuseraient de combattre dès qu’ils apercevraient Pison, que naguère ils nommaient leur père, Pison fort de son droit si l’on consultait la justice, assez fort de ses armes si l’on recourait à l’épée. » Il déploie ses manipules devant les remparts du château, sur une hauteur escarpée, du seul côté qui ne soit pas baigné par la mer. Les vétérans de Sentius s’avancèrent sur plusieurs lignes, et soutenus de bonnes réserves. Ici d’intrépides soldats ; là une position du plus rude accès, mais nul courage, nulle confiance, pas même d’armes, si ce n’est des instruments rustiques, ramassés à la hâte. Le combat, une fois engagé, ne dura que le temps nécessaire aux cohortes romaines pour gravir la colline : les Ciliciens prirent la fuite, et s’enfermèrent dans le château.\par
\labelchar{LXXXI.} Pison fit contre la flotte, mouillée à peu de distance, une entreprise qui n’eut pas de succès. Il rentra dans la place, et, du haut des murailles, tantôt se désespérant aux yeux des soldats, tantôt les appelant par leur nom, les engageant par des récompenses, il les excitait à la révolte. Déjà il avait ébranlé les esprits au point qu’un porte-enseigne de la sixième légion était passé à lui avec son drapeau. Alors Sentius fait sonner les trompettes et les clairons, ordonne qu’on marche au rempart, qu’on dresse les échelles, que les plus résolus montent à l’assaut, tandis que d’autres, avec les machines, lanceront des traits, des pierres, des torches enflammées. L’opiniâtreté de Pison fléchit à la fin, et il offrit de livrer ses armes, demandant seulement à rester dans le fort jusqu’à ce que César eût décidé à qui serait confiée la Syrie. Ces conditions furent rejetées ; et Pison n’obtint que des vaisseaux, et sûreté jusqu’en Italie.\par
\labelchar{LXXXII.} Cependant, lorsque le bruit de la maladie de Germanicus se répandit à Rome, avec les sinistres détails dont le grossissait l’éloignement des lieux, la douleur, l’indignation, les murmures éclatèrent de toutes parts : « Voilà donc pourquoi on l’a relégué au bout de l’univers, pourquoi la province a été livrée à Pison ; c’est là le secret des entretiens mystérieux d’Augusta et de Plancine. Les vieillards ne disaient que trop vrai en parlant de Drusus : les despotes ne pardonnent point à leurs fils d’être citoyens. Germanicus périt, comme son père, pour avoir conçu la pensée de rendre au peuple romain le règne des lois et de la liberté. » Sa mort, qu’on apprit au milieu de ces plaintes, en augmenta la violence ; et, avant qu’il parût ni édit des magistrats, ni sénatus-consulte, le cours des affaires fut suspendu. Les tribunaux sont déserts, les maisons fermées ; partout le silence ou des gémissements. Et rien n’était donné à l’ostentation : si l’on portait les signes extérieurs du deuil, le deuil véritable était au fond des cœurs. Sur ces entrefaites, des marchands, partis de Syrie lorsque Germanicus vivait encore, annoncèrent un changement heureux dans son état. La nouvelle est aussitôt crue, aussitôt publiée. Le premier qui l’entend court, sans examen, la répéter à d’autres, qui la racontent à leur tour, exagérée par la joie. La ville entière est en mouvement ; on force l’entrée des temples. La nuit aidait à la crédulité ; et, dans les ténèbres, on affirme avec plus de hardiesse. Tibère ne démentit point ces faux bruits ; mais le temps les dissipa de lui-même ; et le peuple, comme s’il eût perdu Germanicus une seconde fois, le pleura plus amèrement.\par
\labelchar{LXXXIII.} Chaque sénateur, suivant la vivacité de son amour ou de son imagination, s’évertua pour lui trouver des honneurs. On décréta que son nom serait chanté dans les hymnes des Saliens \footnote{Les Saliens ne chantaient que les dieux : insérer dans leurs hymnes le nom de Germanicus, c’était donc une sorte d’apothéose.} ; qu’il aurait, à toutes les places destinées aux prêtres d’Auguste, des chaises curules \footnote{C’était un honneur insigne d’avoir une chaise curule au Cirque et dans les théâtres. Cette distinction fut accordée à César pendant sa vie, au jeune Marcellus après sa mort. Ce siège vide rappelait incessamment le souvenir des personnes regrettées, et semblait, par une touchante illusion, témoigner qu’elles n’étaient qu’absentes.}, sur lesquelles on poserait des couronnes de chêne \footnote{Comme au sauveur des citoyens.} ; qu’aux jeux du Cirque son image en ivoire ferait partie de la pompe sacrée \footnote{Parmi les statues des dieux et des héros, que l’on portait en pompe dans le cortège solennel qui, partant du Capitole, traversait le Forum et se rendait au grand Cirque.} ; que nul ne lui succéderait comme augure ou comme flamine, s’il n’était de la maison des Jules. On ordonna qu’il lui fût élevé à Rome, sur le bord du Rhin, et sur le mont Amanus en Syrie, des arcs de triomphe qui porteraient inscrits ses exploits, avec la mention qu’il était mort pour la République ; un mausolée dans Antioche, où il avait été mis au bûcher ; un tribunal à Épidaphne\footnote{Faubourg d’Antioche ou plutôt village célèbre à quelque distance de cette ville, avec un bois très vaste d’oliviers et de cyprès consacré à Apollon.}, Soù il avait terminé sa vie. Il serait difficile de compter les statues qui lui furent érigées, les lieux où il fut honoré d’un culte. On proposait de le représenter, parmi les orateurs célèbres, sur un écusson en or \footnote{Écussons sur lesquels étaient sculptés les bustes des hommes illustres et que l’on suspendait dans la salle du sénat.}, d’une grandeur plus qu’ordinaire : Tibère déclara « qu’il lui en consacrerait un pareil à ceux des autres ; que l’éloquence ne se jugeait point d’après les rangs ; que c’était assez de gloire pour Germanicus d’être égalé aux anciens écrivains. » L’ordre équestre appela du nom de Germanicus l’escadron de la Jeunesse, et voulut que l’image de ce grand homme fût portée en tête de la cavalcade solennelle des ides de juillet \footnote{Il se faisait tous les ans, le 15 juillet, une cavalcade, dans laquelle les chevaliers romains se rendaient en pompe au Capitole, en partant du temple de Mars ou de celui de l’Honneur.}. La plupart de ces règlements sont restés en vigueur ; quelques-uns ne furent jamais suivis ou le temps les a fait oublier.\par
\labelchar{LXXXIV.} Le deuil de Germanicus durait encore, lorsque Livie sa sœur, mariée à Drusus, mit au monde deux fils jumeaux. Ce bonheur peu commun, et qui réjouit les plus modestes foyers, causa au prince un plaisir si vif, que, dans l’ivresse de sa joie, il se vanta devant le sénat d’être le premier Romain de ce rang qui eût vu naître à la fois deux soutiens de sa race : car il tirait vanité de tout, même des événements fortuits. En de pareilles circonstances, celui-ci fut pour le peuple un chagrin de plus : cette famille, accrue de nouveaux rejetons, semblait peser davantage sur celle de Germanicus.\par
\labelchar{LXXXV.} La même année le sénat rendit, contre les dissolutions des femmes, plusieurs décrets sévères. La profession de courtisane fut interdite à celles qui auraient pour aïeul, pour père ou pour mari, un chevalier romain. Vistilia, née d’une famille prétorienne, venait en effet de déclarer sa prostitution chez les édiles, d’après un usage de nos ancêtres, qui croyaient la femme impudique assez punie par l’aveu public de sa honte. Titidius Labéo, mari de Vistilia, fut recherché pour n’avoir pas appelé, sur une épouse manifestement coupable, la vengeance de la loi. Il répondit que les soixante jours accordés pour se consulter n’étaient pas révolus ; et le sénat crut faire assez en envoyant Vistilia cacher son ignominie dans l’île de Sériphe \footnote{Aujourd’hui Sefo ou Serfanto, petite île de l’Archipel, une des Cyclades.}. On s’occupa aussi de bannir les superstitions égyptiennes et judaïques. Un sénatus-consulte ordonna le transport en Sardaigne de quatre mille hommes, de la classe des affranchis, infectés de ces erreurs et en âge de porter les armes. Ils devaient y réprimer le brigandage ; et, s’ils succombaient à l’insalubrité du climat, la perte serait peu regrettable. II fut enjoint aux autres de quitter l’Italie, si, dans un temps fixé, ils n’avaient pas abjuré leur culte profane.\par
\labelchar{LXXXVI.} Tibère proposa ensuite d’élire une Vestale pour remplacer Occia, qui, pendant cinquante-sept ans, avait présidé aux rites sacrés avec une pureté de mœurs irréprochable. Il remercia Fontéius Agrippa et Domitius Pollio du zèle qu’ils montraient à l’envi pour la République en offrant leurs filles. On préféra la fille de Pollio, uniquement parce qu’il avait toujours conservé l’épouse dont elle était née ; car un divorce avait fait quelque tort à la maison d’Agrippa. Le prince consola, par une dot d’un million de sesterces, celle qui ne fut pas choisie.\par
\labelchar{LXXXVII.} Le peuple se plaignait de la cherté des vivres. César fixa le prix que l’acheteur payerait le blé, et promit au vendeur un dédommagement de deux sesterces par boisseau. Il n’en continua pas moins à refuser le titre de Père de la patrie, dont l’offre lui fut renouvelée ; et il réprimanda sévèrement ceux qui avaient appelé ses occupations, divines, et qui l’avaient salué du nom de Maître. Aussi ne restait-il au discours qu’un sentier étroit et glissant, sous un prince qui craignait la liberté et haïssait la flatterie.\par
\labelchar{LXXXVIII.} Je trouve, chez les auteurs contemporains, et dans les mémoires de quelques sénateurs, qu’on lut au sénat une lettre d’Adgandestrius, chef des Chattes, qui promettait la mort d’Arminius, si le poison nécessaire à son dessein lui était envoyé. On répondit « que le peuple romain ne se vengeait pas de ses ennemis par la fraude et les complots, mais ouvertement et à main armée », trait glorieux de ressemblance que Tibère se donnait avec ces anciens généraux qui empêchèrent l’empoisonnement du roi Pyrrhus et lui en dénoncèrent le projet. Au reste Arminius, après la retraite des Romains et l’expulsion de Maroboduus, voulut régner, et souleva contre lui la liberté de ses concitoyens. On prit les armes, et, après des succès divers, il périt par la trahison de ses proches. Cet homme fut sans contredit le libérateur de la Germanie ; et ce n’était pas, comme tant de rois et de capitaines, à Rome naissante qu’il faisait la guerre, mais à l’empire dans sa grandeur et sa force. Battu quelquefois, jamais il ne fut dompté. Sa vie dura trente-sept ans, sa puissance douze. Chanté encore aujourd’hui par les barbares, il est ignoré des Grecs, qui n’admirent d’autres héros que les leurs, et trop peu célèbre chez les Romains, qui, enthousiastes du passé, dédaignent tout ce qui est moderne.
\section[{Livre troisième (20, 22)}]{Livre troisième (20, 22)}\renewcommand{\leftmark}{Livre troisième (20, 22)}

\subsection[{Suite de la mort de Germanicus}]{Suite de la mort de Germanicus}
\noindent \labelchar{I.} Agrippine, dont l’hiver n’avait nullement interrompu la navigation, arrive à l’île de Corcyre, située vis-à-vis de la Calabre \footnote{Les anciens n’appliquaient pas le nom de Calabre au même pays que nous. Ils appelaient ainsi la pointe de l’Italie qui s’avance dans la mer ionienne, au sud-est de l’Apulie, et qui est aussi désignée par les noms de \emph{Messapia} et d’\emph{lapygia}. Quant à la Calabre actuelle, qui occupe la pointe la plus méridionale de l’Italie et se termine au détroit de Sicile, c’est ce que les Romains nommaient le pays des Bruttiens.}. Elle y resta quelques jours, afin de calmer les emportements d’une âme qui ne savait pas endurer son malheur. Cependant, au premier bruit de son retour, les amis les plus dévoués de sa famille, tous ceux qui avaient fait la guerre sous Germanicus, beaucoup d’inconnus même, accourus des cités voisines, les uns parce qu’ils croyaient plaire à César, les autres par esprit d’imitation, se précipitèrent dans Brindes le point le plus rapproché et le plus sûr où elle pût aborder. Aussitôt que la flotte fut aperçue dans le lointain, le port, le rivage, les remparts de la ville, les toits des maisons, tous les lieux d’où la vue s’étendait sur la mer, se couvrirent de spectateurs éplorés, qui se demandaient si l’on recevrait Agrippine en silence ou avec quelque acclamation. On doutait encore quel accueil serait le plus convenable, lorsque insensiblement la flotte toucha le port, dans un appareil où, au lieu de l’allégresse ordinaire des rameurs, tout annonçait la tristesse et le deuil. Au moment où, sortie du vaisseau avec deux de ses enfants, Agrippine parut, l’urne sépulcrale dans les mains, les yeux baissés vers la terre, il s’éleva un gémissement universel : et l’on n’eût pas distingué les parents des étrangers, les regrets des hommes de la désolation des femmes ; seulement le cortège d’Agrippine semblait abattu par une longue affliction, et la douleur du peuple, étant plus récente, éclatait plus vivement.\par
\labelchar{II.} Tibère avait envoyé deux cohortes prétoriennes, avec ordre aux magistrats de la Calabre, de l’Apulie et de la Campanie, de rendre à la mémoire de son fils les honneurs suprêmes. Les cendres étaient portées sur les épaules des tribuns et des centurions : devant elles marchaient les enseignes sans ornements et les faisceaux renversés. Quand on passait dans les colonies, le peuple vêtu de noir, les chevaliers en trabée, brûlaient, selon la richesse du lieu, des étoffes, des parfums, et tout ce qu’on offre aux morts pour hommage. Les habitants mêmes des villes éloignées de la route accouraient à l’envi, dressaient des autels, immolaient des victimes aux dieux mânes, et témoignaient leur douleur par des larmes et des acclamations funèbres. Drusus s’avança jusqu’à Terracine avec Claude, frère de Germanicus, et les enfants que celui-ci avait laissés à Rome. Les consuls M. Valérius et C. Aurélius, qui avaient déjà pris possession de leur charge, le sénat et une partie du peuple, se répandirent en foule sur la route. Ils marchaient sans ordre et chacun pleurait à son gré ; car l’adulation était loin de leur pensée, personne n’ignorant la joie mal déguisée que causait à Tibère la mort de Germanicus.\par
\labelchar{III.} Tibère et Augusta s’abstinrent de paraître en public soit qu’ils crussent au-dessous de la majesté suprême de donner leurs larmes en spectacle ; soit qu’ils craignissent que tant de regards, observant leurs visages, n’y lussent la fausseté de leurs cœurs. Pour Antonia, mère de Germanicus, je ne trouve ni dans les histoires ni dans les Actes journaliers\footnote{Véritables journaux manuscrits, qui circulaient non seulement à Rome, mais dans les provinces. On y racontait les nouvelles de la ville, les jeux publics, les supplices, etc.} de cette époque, qu’elle ait pris part à aucune cérémonie remarquable ; et cependant, avec Agrippine, Drusus et Claude, sont expressément nommés tous les autres parents. Peut-être fut-elle empêchée par la maladie ; peut-être, vaincue par la douleur, n’eut-elle pas la force d’envisager de ses yeux la grandeur de son infortune. Toutefois je croirais plutôt que Tibère et Augusta, qui ne sortaient pas du palais, l’y retinrent malgré elle, afin que l’affliction parût également partagée, et que l’absence de la mère justifiât celle de l’oncle et de l’aïeule.\par
\labelchar{IV.} Le jour où les cendres furent portées au tombeau d’Auguste, un vaste silence et des gémissements confus se succédèrent tour à tour. La multitude remplissait les rues ; des milliers de torches brillaient dans le champ de Mars. Là les soldats en armes, les magistrats dépouillés de leurs insignes, le peuple rangé par tribus, s’écriaient « que c’en était fait de la république, qu’il ne restait plus d’espérance. » À la vivacité, à la franchise de ces plaintes, on eût dit qu’ils avaient oublié leurs maîtres. Cependant rien ne blessa plus profondément Tibère que leur enthousiasme pour Agrippine : ils l’appelaient « l’honneur de la patrie, le véritable sang d’Auguste, l’unique modèle des anciennes vertus" ; puis, les yeux levés au ciel, ils priaient les dieux « de protéger ses enfants, et de les faire survivre à leurs persécuteurs. »\par
\labelchar{V.} Quelques-uns auraient désiré plus de pompe à des funérailles publiques : ils rappelaient ce que la magnificence d’Auguste avait fait pour honorer les obsèques de Drusus, père de Germanicus, « son voyage à Ticinum\footnote{Aujourd’hui Pavie.}, au plus fort de l’hiver, et son entrée dans Rome avec le corps, dont il ne s’était pas un instant séparé ; ces images des Claudes et des Jules environnant le lit funéraire ; les pleurs du Forum ; l’éloge prononcé du haut de la tribune ; tous les honneurs institués par nos ancêtres ou imaginés dans les âges modernes, accumulés pour Drusus : tandis que Germanicus n’avait pas reçu les plus ordinaires, ceux auxquels tout noble avait droit de prétendre. Qu’il eût fallu, à cause de l’éloignement des lieux, lui dresser dans une terre étrangère un vulgaire bûcher, n’était-ce pas une raison pour lui rendre avec usure ce que le sort lui avait dénié en ce premier moment ? Son frère n’était allé au-devant de lui qu’à une journée de Rome ; son oncle ne s’était pas même avancé jusqu’aux portes. Qu’étaient devenues les coutumes antiques, ce lit de parade où l’on plaçait l’effigie du mort, ces vers que l’on chantait à sa louange, ces panégyriques, ces larmes, symboles d’une douleur au moins apparente ? »\par
\labelchar{VI.} Tibère fut instruit de ces murmures : afin de les étouffer, il rappela par un édit « que beaucoup de Romains étaient morts pour la patrie, et que pas un n’avait excité une telle ardeur de regrets : regrets honorables sans doute et au prince et aux citoyens, pourvu qu’ils eussent des bornes, car la dignité interdisait aux chefs d’un grand empire et au peuple-roi ce qui était permis à des fortunes privées et à de petits États. Une douleur récente n’avait pas dû se refuser la consolation du deuil et des pleurs ; mais il était temps que les âmes retrouvassent leur fermeté : ainsi le divin Jules, privé de sa fille unique, ainsi le divin Auguste, après la mort de ses petits-fils, avaient dévoré leurs larmes. S’il fallait des exemples plus anciens, combien de fois le peuple romain n’avait-il pas supporté courageusement la défaite de ses armées, la perte de ses généraux, l’extinction de ses plus nobles familles ? Les princes mouraient ; la République était immortelle. On pouvait donc retourner aux devoirs accoutumés, et même aux plaisirs, qu’allaient ramener les jeux de la grande Déesse\footnote{Les fêtes de la grande Déesse (de la mère des Dieux) tombaient aux nones d’Avril.}. "\par
\labelchar{VII.} Alors le cours des affaires recommence, chacun reprend ses fonctions, et Drusus part pour l’Illyrie, laissant Rome dans une vive attente de la vengeance qu’on tirerait de Pison, et toute pleine de rumeurs contre « l’arrogance de cet homme, qui s’amusait à parcourir les beaux rivages de l’Asie et de la Grèce, et, avec ces délais perfidement calculés, anéantissait les preuves de ses crimes. » Le bruit s’était en effet répandu que l’empoisonneuse Martine, envoyée, comme je l’ai dit, par Sentius, était morte subitement à Brindes, et qu’on avait découvert du poison dans un nœud de ses cheveux, sans qu’il parût sur son corps aucun indice qu’elle s’en fût servie contre elle-même.\par
\labelchar{VIII.} Pison se fit précéder à Rome par son fils, avec des instructions pour adoucir le prince, et se rendit auprès de Drusus, dans lequel il espérait trouver moins d’exaspération que de bienveillance, le coup qui lui enlevait un frère le délivrant d’un rival. Tibère, afin de paraître exempt de prévention, reçut poliment le jeune homme, et lui donna les marques de générosité qui étaient d’usage envers les fils des grandes familles. Drusus répondit au père « que, si les bruits qu’on faisait courir étaient vrais, il serait le premier à demander vengeance ; qu’il désirait au reste que la fausseté en fût prouvée, et que la mort de Germanicus ne devînt funeste à personne. » Il parla ainsi publiquement, et il évita toute entrevue secrète. On crut reconnaître les leçons de Tibère, en voyant Drusus, malgré la franchise de son caractère et de son âge, déployer cette politique de vieillard.
\subsection[{Arrivée de Pison à Rome. Son procès}]{Arrivée de Pison à Rome. Son procès}
\noindent \labelchar{IX.} Pison, ayant traversé la mer de Dalmatie et laissé ses vaisseaux à Ancône, passa par le Picénum, et atteignit, sur la voie Flaminienne, une légion qui venait de Pannonie à Rome, d’où elle devait être conduite en Afrique. On s’entretint beaucoup de l’affectation avec laquelle il s’était montré, disait-on, aux soldats, sur la route et dans la marche. À Narni, pour écarter les soupçons ou par un effet des irrésolutions de la peur, il s’embarqua sur le Nar ; puis il descendit le Tibre, et mit le comble à l’indignation générale, en abordant près du tombeau des Césars. C’est de là qu’en plein jour, au moment où la rive était couverte de peuple, Pison, entouré de nombreux clients, Plancine, suivie d’un cortège de femmes, s’avancèrent, le front haut et radieux. Tout contribuait à irriter les haines, leur maison dominant sur le Forum, un appareil de fête, un banquet splendide, la publicité même, à laquelle rien n’échappe dans un lieu si fréquenté.\par
\labelchar{X.} Le lendemain, Fulcinius Trio déféra Pison aux consuls. De leur côté, Vitellius, Véranius et les autres amis de Germanicus soutenaient que Trio usurpait un rôle qui n’était pas le sien ; qu’ils venaient eux-mêmes, non comme accusateurs, mais comme témoins, pour révéler les faits et parler au nom de Germanicus. Trio renonça donc à la poursuite principale, et on lui abandonna la vie antérieure de l’accusé. Tibère fut prié d’instruire la cause en personne, et Pison même ne le récusait pas : effrayé des préventions du peuple et du sénat, il espérait tout d’un prince assez fort pour braver les clameurs, et d’un fils engagé dans les secrets de sa mère. « Un seul homme aussi discernait mieux la vérité de l’erreur ; la haine et l’envie triomphaient trop facilement devant une assemblée. » Tibère n’ignorait pas quel fardeau s’imposait le juge d’un tel procès, et à quelles imputations lui-même était en butte. Il écouta en présence de quelques familiers les menaces des accusateurs, les prières de Pison, et renvoya toute l’affaire au sénat.\par
\labelchar{XI.} Dans cet intervalle, Drusus, revenu d’Illyrie, ajourna l’ovation que le sénat lui avait décernée pour la soumission de Maroboduus et les succès de la dernière campagne, et entra sans pompe dans la ville. Cependant Pison demanda pour défenseurs L. Arruntius, T. Vinicius Asinius Gallus, Éserninus Marcellus et Sextus Pompéius, qui tous alléguèrent différentes excuses. M'. Lépidus, L. Piso et Livinéius Régulus se chargèrent de sa cause. Rome était impatiente de voir comment les amis de Germanicus lui garderaient leur foi, jusqu’où irait la confiance de l’accusé, si Tibère saurait maîtriser et comprimer ses vrais sentiments. Tout préoccupé de ces pensées, jamais le peuple ne se permit contre le prince plus de ces paroles qui se disent à voix basse, de ces réticences où perce le soupçon.\par
\labelchar{XII.} Le jour où le sénat fut assemblé, Tibère, dans un discours plein de ménagements étudiés, rappela « que Pison avait été l’ami et le lieutenant de son père, que lui-même, de l’aveu du sénat, l’avait placé auprès de Germanicus pour le seconder dans l’administration de l’Orient. Avait-il aigri le jeune César par des hauteurs et des contradictions ? S’était-il réjoui de sa mort ? L’avait-il hâtée par un crime ? Voilà ce qu’il fallait rechercher avec impartialité. Oui, pères conscrits, si un lieutenant est sorti des bornes du devoir, s’il a manqué de déférence pour son général, s’il a triomphé de sa mort et de ma douleur, je le haïrai, je lui fermerai ma maison, je vengerai mon injure privée et non celle du prince. Mais si un attentat, punissable quelle qu’en soit la victime, vous est révélé, c’est à votre justice à consoler les enfants de Germanicus de la perte d’un père et nous de celle d’un fils. Examinez en même temps si Pison a tenu, à la tête des armées, une conduite turbulente et factieuse, s’il a brigué ambitieusement la faveur des soldats, s’il est rentré de vive force dans la province ou bien si ce ne sont là que des faussetés grossies par les accusateurs, dont au surplus le zèle trop ardent a droit de m’offenser. Que servait-il en effet de dépouiller le corps de Germanicus, de l’exposer nu aux regards de la foule, et de répandre jusque chez les nations étrangères des bruits d’empoisonnement, si le fait, douteux encore, a besoin d’être éclairci ? Je pleure mon fils, il est vrai, je le pleurerai toujours ; mais je n’empêche pas l’accusé de produire tout ce qui peut appuyer son innocence, de dévoiler même les torts de Germanicus, s’il en eut quelques-uns. Que le douloureux intérêt que j’ai dans cette cause ne vous engage pas à prendre de simples allégations pour des preuves. Si le sang ou l’amitié donne à Pison des défenseurs, qu’ils emploient tout ce qu’ils ont d’éloquence et de zèle à le sauver du péril. J’exhorte les accusateurs aux mêmes efforts, à la même fermeté. Nous aurons accordé à Germanicus un seul privilège : c’est que le procès de sa mort soit instruit dans le sénat plutôt qu’au Forum, devant vous plutôt que devant les juges ordinaires. Que l’esprit d’égalité préside à tout le reste : ne voyez ni les larmes de Drusus, ni mon affliction, ni ce que la calomnie peut inventer contre nous. »\par
\labelchar{XIII.} On décida que les accusateurs auraient deux jours pour exposer leurs griefs, et qu’après un intervalle de six jours il en serait donné trois à la défense. Fulcinius commença par de vieux reproches d’avarice et d’intrigues dans le gouvernement de l’Espagne ; allégations futiles, qui, prouvées, ne pouvaient nuire à l’accusé, s’il détruisait les charges nouvelles, et, réfutées, ne l’absoudraient pas, s’il était convaincu de plus graves délits. Après lui Servéus, Véranius et Vitellius, tous trois avec une égale chaleur, le dernier avec une grande éloquence, soutinrent « qu’en haine de Germanicus et dans des vues de bouleversement, Pison, en autorisant la licence des troupes et l’oppression des alliés, avait gâté l’esprit des soldats au point d’être nommé, par ce qu’il y avait de plus méprisable, le père des légions ; tandis qu’il persécutait tous les hommes d’honneur, et principalement les compagnons et les amis de Germanicus. » Ils peignirent « les maléfices et le poison employés contre les jours de ce général, les actions de grâces de Pison et de Plancine et leurs sacrifices impies, la république attaquée par les armes d’un rebelle, et réduite à le vaincre pour l’amener en justice. »\par
\labelchar{XIV.} Excepté sur un point, la défense chancela : Pison ne pouvait nier ses ambitieuses complaisances pour le soldat, ni la province livrée en proie à des brigands, ni ses insultes envers son général. Le crime de poison fut le seul dont il parût s’être justifié, et les accusateurs aussi l’appuyaient de trop faibles preuves : selon eux, " Pison, invité à un repas chez Germanicus, et placé à table au-dessus de lui, avait de sa main empoisonné les mets. » Or, il paraissait incroyable qu’entouré d’esclaves qui n’étaient pas les siens, devant une foule de témoins, sous les yeux de Germanicus lui-même, il eût eu cette audace. L’accusé demandait d’ailleurs que ses propres esclaves et ceux qui avaient servi le repas fussent mis à la question. Mais les juges avaient chacun leurs motifs pour être inexorables : Tibère ne pardonnait point la guerre portée en Syrie ; les sénateurs ne pouvaient se persuader que le crime fût étranger à la mort de Germanicus. D’un autre côté on entendait le peuple crier, aux portes de l’assemblée, « qu’il ferait justice lui-même, si les suffrages du sénat épargnaient le coupable. » Déjà les statues de Pison, traînées aux Gémonies, allaient être mises en pièces, si le prince ne les eût fait protéger et remettre à leurs places. Pison remonta en litière et fut reconduit par un tribun des cohortes prétoriennes ; ce qui fit demander si cet homme le suivait pour garantir sa vie ou pour présider à sa mort.\par
\labelchar{XV.} Plancine, non moins odieuse, avait plus de crédit aussi ne savait-on pas jusqu’à quel point le prince serait maître de son sort. Elle-même, tant que Pison eut encore de l’espoir, protesta « qu’elle suivrait sa destinée, prête, s’il le fallait, à mourir avec lui. » Lorsque, par la secrète intercession de Livie, elle eut obtenu sa grâce, elle se détacha peu à peu de son époux et ne plaida plus que sa propre cause. L’accusé comprit ce que cet abandon avait de sinistre : incertain s’il tenterait un dernier effort, il cède aux exhortations de ses fils, s’arme de courage et reparaît dans le sénat. Là il entendit répéter l’accusation ; il essuya les invectives des sénateurs, leurs cris de haine et de vengeance ; et rien cependant ne l’effraya plus que de voir Tibère impassible, sans pitié, sans colère, fermant son âme à toutes les impressions. De retour chez lui, Pison feint de préparer une défense pour le lendemain, écrit quelques lignes, et les remet cachetées à un affranchi. Ensuite il donne à son corps les soins accoutumés, et, bien avant dans la nuit, sa femme étant sortie de l’appartement, il fait fermer la porte. Au lever du jour, on le trouva égorgé ; son épée était par terre à côté de lui.\par
\labelchar{XVI.} Je me souviens d’avoir entendu raconter à des vieillards qu’on vit plusieurs fois, dans les mains de Pison, des papiers dont il ne divulgua point le secret, mais qui, au dire de ses amis, contenaient des lettres et des instructions de Tibère contre Germanicus. « Il avait résolu, dit-on, de les lire en plein sénat et d’accuser le prince, si Séjan ne l’eut amusé par de vaines promesses. Enfin il ne se tua pas lui-même : un meurtrier lui fut dépêché. » Je ne garantis ni l’un ni l’autre de ces faits ; cependant je n’ai pas dû supprimer une tradition dont les auteurs vivaient encore dans ma jeunesse. Tibère, avec une tristesse affectée, se plaignit devant le sénat d’une mort qui avait pour but de lui attirer des haines ; ensuite il questionna beaucoup l’affranchi sur le dernier jour, sur la dernière nuit de Pison. À des réponses généralement prudentes cet homme mêlant quelques paroles indiscrètes, Tibère lut la lettre de Pison, qui était conçue à peu près en ces termes : « Accablé sous la conspiration de mes ennemis et sous le poids d’une odieuse et fausse imputation, puisque la vérité, puisque mon innocence ne trouvent accès nulle part, je prends les dieux à témoin, César, que ma fidélité envers toi fut toujours égale à mon pieux respect pour ta mère. Je vous implore tous deux en faveur de mes enfants. Cnéius, de quelque façon qu’on me juge, n’est pas lié à ma fortune, n’ayant point quitté Rome pendant ces derniers temps. Marcus m’avait dissuadé de rentrer en Syrie ; et plût aux dieux que j’eusse cédé à la jeunesse de mon fils, plutôt que lui à l’âge et à l’autorité de son père ! Je t’en conjure avec plus d’instances de ne pas le punir de mon erreur, dont il est innocent. C’est au nom de quarante-cinq ans de dévouement, au nom du consulat où nous fûmes collègues \footnote{Pison fut consul en 734 avec Auguste ; en 747 avec Tibère.}, au nom de l’estime dont m’honora le divin Auguste, ton père, qu’un ami, qui ne te demandera plus rien, te demande la grâce d’un fils infortuné. » La lettre ne disait pas un mot de Plancine.\par
\labelchar{XVII.} Tibère disculpa le jeune Marcus de la guerre civile : « C’étaient, disait-il, les ordres de son père ; un fils ne pouvait désobéir. » Ensuite il plaignit « la destinée d’une si noble famille, et la fin, méritée peut-être, mais déplorable, de Pison. » Quant à Plancine, il parla pour elle en homme confus et humilié de son rôle, alléguant les sollicitations de sa mère. Aussi était-ce surtout contre Livie que s’exhalait en secret l’indignation des gens de bien. « Il était donc permis à une aïeule d’envisager la femme qui tua son petit-fils, de lui adresser la parole, de l’arracher à la justice du sénat ! Ce que les lois obtenaient pour tout citoyen était refusé au seul Germanicus ! La voix de Vitellius et de Véranius demandait vengeance pour un César ; l’empereur et Augusta défendaient Plancine ! Sûre de ses poisons et d’un art si heureusement éprouvé, que tardait-elle à les tourner contre Agrippine, contre ses enfants, à rassasier une aïeule si tendre, un oncle si généreux, du sang de cette famille? » Deux jours furent employés encore à une ombre de procédure, pendant laquelle Tibère pressait les fils de Pison de défendre leur mère. Comme les accusateurs et les témoins parlaient à l’envi, sans que personne se levât pour répondre, la compassion faisait plus de progrès que la haine. Le consul Aurélius Cotta, interrogé le premier (car, lorsque c’était l’empereur qui ouvrait une délibération, les magistrats donnaient aussi leur suffrage), proposa « de rayer des fastes le nom de Pison, de confisquer une partie de ses biens, d’abandonner une autre à son fils Cnéius, qui changerait de prénom. Marcus, exclu du sénat, recevrait cinq millions de sesterces et serait relégué pour dix ans. La grâce de Plancine était accordée aux prières d’Augusta.\par
\labelchar{XVIII.} Le prince adoucit beaucoup la sévérité de cet avis. Il ne voulut pas que le nom de Pison fût rayé des fastes, puisqu’on y maintenait celui de Marc-Antoine, qui avait fait la guerre à la patrie, celui d’Iulus Antonius, qui avait porté le déshonneur dans la maison d’Auguste\footnote{Il fut puni de mort comme complice des débordements de Julie, pendant qu’elle était la femme de Tibère.}. Il sauva Marcus de l’ignominie, et lui laissa les biens paternels. J’ai déjà dit plusieurs fois que Tibère n’était point dominé par l’avarice ; et la honte d’avoir absous Plancine le disposait à la clémence. Valérius Messalinus proposait de consacrer une statue d’or dans le temple de Mars Vengeur, Cécina Sévérus d’élever un autel à la Vengeance ; César s’y opposa : « Ces monuments, disait-il, étaient faits pour des victoires étrangères ; les malheurs domestiques devaient être couverts d’un voile de tristesse. » Messalinus avait opiné aussi pour que Tibère, Augusta, Antonio, Drusus et Agrippine reçussent des actions de grâces comme vengeurs de Germanicus. Il n’avait fait aucune mention de Claude, et L. Asprénas lui demanda publiquement si cette omission était volontaire : alors le nom de Claude fut ajouté au décret. Pour moi, plus je repasse dans mon esprit de faits anciens et modernes, plus un pouvoir inconnu me semble se jouer des mortels et de leurs destinées. Certes, le dernier homme que la renommée, son espérance, les respects publics, appelassent à l’empire, était celui que la fortune tenait caché pour en faire un prince.\par
\labelchar{XIX.} Peu de jours après, Tibère fit donner par le sénat des sacerdoces à Vitellius, à Véranius, à Servéus. En promettant à Fulcinius de l’appuyer dans la recherche des honneurs, il l’avertit de prendre garde aux écarts d’une éloquence trop fougueuse. Là se bornèrent les expiations offertes aux mânes de Germanicus, dont la mort a été, non seulement chez les contemporains, mais dans les générations suivantes, un sujet inépuisable de controverse : tant sont enveloppés de nuages les plus grands événements, grâce à la crédulité qui accueille les bruits les moins fondés, au mensonge qui altère les faits les plus réels ; double cause d’une incertitude qui s’accroît avec le temps. Drusus, qui était sorti de Rome pour reprendre les auspices \footnote{Un général déposait le commandement en entrant dans Rome. Or Drusus, revenu d’Illyrie, y était entré et avait ajourné son ovation à cause des funérailles de son frère : il fallait donc, avant de la célébrer, qu’il sortît de nouveau, et qu’il reprît le commandement et par conséquent les auspices ; car on ne pouvait jouir ni du grand ni du petit triomphe sans être revêtu du pouvoir militaire.}, y rentra bientôt avec l’appareil de l’ovation. Au bout de quelques jours, il perdit sa mère Vipsanie, le seul des enfants d’Agrippa dont la mort ait été naturelle ; car, de tous les autres, l’un périt certainement par le fer, et le reste, si l’on en croit la renommée, par la faim ou par le poison.
\subsection[{Guerre en Afrique contre Tacfarinas}]{Guerre en Afrique contre Tacfarinas}
\noindent \labelchar{XX.} La même année Tacfarinas, battu l’été précédent par Camillus, ainsi que je l’ai dit, recommença la guerre en Afrique. Ce furent d’abord de simples courses, dont la vitesse le dérobait à toutes les poursuites. Bientôt il saccage les bourgades, entraîne après lui d’immenses butins, et finit par assiéger, près du fleuve Pagida \footnote{Broder écrit Pagyda, et conjecture que c’est la rivière d’Abeadh, dans la province de Constantine.}, une cohorte romaine. Le poste avait pour commandant Décrias, intrépide soldat, capitaine expérimenté, qui tint ce siège pour un affront. Après avoir exhorté sa troupe à présenter le combat en rase campagne, il la range devant les retranchements. Elle est repoussée au premier choc : Décrias, sous une grêle de traits, se jette à travers les fuyards, les arrête, crie aux porte-enseigne « qu’il est honteux que le soldat romain tourne le dos à une bande de brigands et de déserteurs. » Couvert de blessures, ayant un œil crevé, il n’en fait pas moins face à l’ennemi, et combat jusqu’à ce qu’il tombe mort, abandonné des siens.\par
\labelchar{XXI.} À la nouvelle de cet échec, L. Apronius, successeur de Camillus, plus indigné de la honte des Romains qu’alarmé du succès de l’ennemi, fit un exemple rare dans ces temps-là, et d’une sévérité antique : il décima la cohorte infâme, et tous ceux que désigna le sort expirèrent sous la verge. Cet acte de rigueur fut si efficace, qu’un corps de cinq cents vétérans défit seul les mêmes troupes de Tacfarinas, devant le fort de Thala\footnote{Ville de Numidie, voisine du désert, ruinée dans la guerre de César contre Juba.}, qu’elles venaient attaquer. Dans cette action, Helvius Rufus, simple soldat, eut la gloire de sauver un citoyen. Apronins lui donna la pique et le collier. Comme proconsul, il pouvait ajouter la couronne civique : il laissa ce mérite au prince, qui s’en plaignit plus qu’il n’en fut offensé. Tacfarinas, voyant ses Numides découragés et rebutés des sièges, court de nouveau la campagne, fuyant dès qu’on le presse, et bientôt revenant à la charge. Tant qu’il suivit ce plan, il se joua des efforts de l’armée romaine, qui se fatiguait vainement à le poursuivre. Lorsqu’il eut tourné sa course vers les pays maritimes, embarrassé de son butin, il lui fallut s’assujettir à des campements fixes. Alors Apronius Césianus, envoyé par son père avec de la cavalerie et des cohortes auxiliaires renforcées des légionnaires les plus agiles, battit les Numides et les rechassa dans leurs déserts.
\subsection[{Procès de Lépida}]{Procès de Lépida}
\noindent \labelchar{XXII.} Cependant à Rome, Lépida, en qui l’honneur d’avoir Sylla et Pompée pour bisaïeuls rehaussait l’éclat du nom Émilien, fut accusée d’avoir supposé un fruit de son mariage avec P. Quirinus, homme riche et sans enfants. On lui reprochait encore l’adultère, le poison, et des questions criminelles adressées aux astrologues sur la maison de César. Elle fut défendue par Manius Lépidus, son frère. Quoique décriée et coupable, l’acharnement de Quirinus à la poursuivre après l’avoir répudiée lui conciliait la pitié publique. Il serait difficile de discerner quels furent, dans ce procès, les vrais sentiments du prince, tant il sut varier et entremêler les signes de colère et de clémence. Il pria d’abord le sénat de ne point avoir égard à l’accusation de lèse-majesté ; ensuite il fit adroitement dénoncer par le consulaire M. Servilius, et par d’autres témoins, ce qu’il semblait avoir voulu taire. D’un autre côté, il transféra les esclaves de Lépida de la garde des soldats à celle des consuls \footnote{Trois modes de détention étaient en usage à Rome : 1° renfermer le détenu dans la prison publique ; 2° le confier à la garde d’un magistrat ; 3° le remettre à un soldat, qui répondait de sa personne.}, et il ne souffrit pas qu’à la question ils fussent interrogés sur ce qui touchait la famille impériale. Il ne voulut pas non plus que Drusus, consul désigné, opinât le premier ; exception où les uns virent une garantie donnée à la liberté des suffrages, et les autres une intention cruelle. Ceux-ci pensèrent que Drusus n’aurait pas cédé son rang, s’il n’eût fallu condamner.\par
\labelchar{XXIII.} Lépida, pendant les jeux \footnote{Les grands jeux, les jeux romains, qui se célébraient au cirque et au théâtre depuis le 5 jusqu’au 13 de septembre (des nones aux ides).} qui suspendirent le cours du procès, se rendit au théâtre, accompagnée de femmes du plus haut rang. Là, invoquant avec des cris lamentables le nom de ses ancêtres, celui du grand Pompée, fondateur de ce monument et dont les images frappaient de tous côtés les regards, elle excita une émotion si profonde que les spectateurs, fondant en larmes, chargèrent d’invectives et de malédictions Quirinus, « ce vieillard sans naissance, auquel on immolait, parce qu’il n’avait pas d’héritiers \footnote{L’avarice avait fini par gagner Tibère ; et il était bien aise de faire plaisir à un vieux riche sans héritiers, qui, par reconnaissance, ne manquerait pas de tester en sa faveur.}, une femme destinée jadis à être l’épouse de L. César et la bru d’Auguste. » Les esclaves révélèrent à la torture les débordements de leur maîtresse ; et l’on adopta l’avis de Rubellius Blandus, qui lui interdisait le feu et l’eau. Drusus s’y rangea lui-même, quoique d’autres en eussent ouvert de plus doux. En considération de Scaurus, qui avait une fille de Lépida, les biens ne furent pas confisqués. Le jugement prononcé, Tibère déclara savoir encore, par les esclaves de Quirinus, que Lépida avait essayé d’empoisonner leur maître.
\subsection[{Décimus Silanus}]{Décimus Silanus}
\noindent \labelchar{XXIV.} Les revers qui venaient d’accabler presque en même temps deux illustres maisons, en enlevant Pison aux Calpurnius, Lépida aux Emiles, eurent une compensation dans le rappel de Décimus Silanus, qui fut rendu à la famille Junia. Je reprendrai en peu de mots son histoire. La fortune, toujours fidèle à Auguste contre la république, le rendit malheureux dans sa vie privée, par les dérèglements de sa fille et de sa petite-fille. Il les chassa de Rome et punit leurs séducteurs de la mort ou de l’exil ; car, en donnant à une faute que les vices des deux sexes ont rendue si commune les noms aggravants de sacrilège et de lèse-majesté, il y appliquait des peines inconnues à la clémence de nos ancêtres et à ses propres lois. Mais je raconterai le châtiment des autres coupables et les événements de ce siècle, si, parvenu au terme de mon travail, il me reste assez de vie pour une tâche nouvelle. D. Silanus, à qui son commerce avec la petite-fille d’Auguste n’attira d’autre rigueur que la disgrâce du prince, se tint néanmoins pour averti d’aller en exil ; et ce ne fut que sous Tibère qu’il osa implorer le sénat et l’empereur. Il dut son rappel au crédit de Marcus Silanus, son frère, homme aussi éminent par son éloquence que par l’éclat de sa noblesse. Tibère, auquel Marcus adressait des remerciements, lui répondit en plein sénat, « qu’il se réjouissait avec lui de ce que son frère était revenu d’un long voyage ; que Décimus avait usé de son droit, puisque ni loi ni sénatus-consulte ne l’avaient banni ; que cependant il ne lui rendait pas l’amitié que son père lui avait retirée, et que les volontés d’Auguste n’étaient pas révoquées par le retour de Silanus. » Décimus acheva sa vie à Rome, sans parvenir aux honneurs.
\subsection[{Allégement des lois contre le célibat}]{Allégement des lois contre le célibat}
\noindent \labelchar{XXV.} On parla ensuite d’adoucir la loi Papia Poppea \footnote{Cette loi fut portée en 762, sous les consuls subrogés M. Papius Mutilus, et Q. Poppéus Sécundus.} qu’Auguste, déjà vieux, avait ajoutée aux lois Juliennes \footnote{La loi Julia, de \emph{Maritandis ordinibus}, fut portée par Auguste en 736, pour encourager les mariages et punir le célibat. Son principal but était de réparer la population épuisée par les guerres civiles, où il avait péri 80.000 hommes armés.}, pour assurer la punition du célibat et accroître les revenus du trésor public \footnote{Voy. chap. XXVIII.}. Cette loi ne faisait pas contracter plus de mariages ni élever plus d’enfants (on gagnait trop à être sans héritiers) ; mais elle multipliait les périls autour des citoyens, et, interprétée par les délateurs, il n’était pas de maison qu’elle ne bouleversât : alors les lois étaient devenues un fléau, comme autrefois les vices. Cette réflexion me conduit à remonter aux sources de la législation, et aux causes qui ont amené cette multitude infinie de lois différentes.
\subsection[{Digression de Tacite sur les lois}]{Digression de Tacite sur les lois}
\noindent \labelchar{XXVI.} Les premiers hommes, encore exempts de passions désordonnées, menaient une vie pure, innocente, et libre par là même de châtiments et de contrainte. Les récompenses non plus n’étaient pas nécessaires, puisqu’on pratiquait la vertu par instinct ; et comme on ne désirait rien de contraire au bon ordre, rien n’était interdit par la crainte. Quand l’égalité disparut, et qu’à la place de la modération et de l’honneur régnèrent l’ambition et la force, des monarchies s’établirent, et chez beaucoup de peuples elles se sont perpétuées. D’autres dès l’origine ou après s’être lassés de la royauté, préférèrent des lois. Elles furent simples d’abord et conformes à l’esprit de ces siècles grossiers. La renommée a célébré surtout celles que Minos donna aux Crétois, Lycurgue aux Spartiates, et plus tard Solon aux Athéniens : celles-ci sont déjà plus raffinées et en plus grand nombre. Chez nous, Romulus n’eut de règle que sa volonté. Numa, qui vint après, imposa au peuple le frein de la religion et des lois divines : quelques principes furent trouvés par Tullus et par Ancus ; mais le premier de nos législateurs fut Servius Tullius, aux institutions duquel les rois même devaient obéissance.\par
\labelchar{XXVII.} Après l’expulsion de Tarquin, le peuple, en vue d’assurer sa liberté et d’affermir la concorde, se donna, contre les entreprises des patriciens, de nombreuses garanties. Des décemvirs furent créés, qui, empruntant aux législations étrangères ce qu’elles avaient de meilleur, en formèrent les Douze Tables, dernières lois dont l’équité soit le fondement, car si celles qui suivirent eurent quelquefois pour but de réprimer les crimes, plus souvent aussi, nées de la division entre les ordres, d’une ambition illicite, de l’envie de bannir d’illustres citoyens ou de quelque motif également condamnable, elles furent l’ouvrage de la violence. De là les Gracques et Saturninus semant le trouble dans la multitude ; et Drusus non moins prodigue de concessions au nom du sénat ; et les alliés gâtés par les promesses, frustrés par les désaveux. Ni la guerre italique, ni la guerre civile, qui la suivit de près n’empêchèrent d’éclore une foule de lois, souvent contradictoires ; jusqu’à ce que L. Sylla, dictateur, après en avoir aboli, changé, ajouté un grand nombre, fît trêve aux nouveautés, mais non pour longtemps, car les séditieuses propositions de Lépidus \footnote{Lépidus, père de celui qui fut triumvir avec Marc-Antoine et Octave, voulut, après la mort de Sylla, faire revivre le parti de Marius et abolir les lois du dictateur.} éclatèrent aussitôt, et la licence ne tarda pas à être rendue aux tribuns d’agiter le peuple au gré de leur caprice. Alors on ne se borna plus à ordonner pour tous ; on statua même contre un seul, et jamais les lois ne furent plus multipliées que quand l’État fut le plus corrompu.\par
\labelchar{XXVIII.} Pompée, chargé dans son troisième consulat de réformer les mœurs, employa des remèdes plus dangereux que les maux ; et, premier infracteur des lois qu’il avait faites, il perdit par les armes un pouvoir qu’il soutenait par les armes. Puis succédèrent vingt années de discordes \footnote{Du troisième consulat de Pompée à la bataille d’Actium, en 723.} : plus de frein, plus de justice ; le crime restait impuni, et trop souvent la mort était le prix de la vertu. Enfin, consul pour la sixième fois, César Auguste, sûr de sa puissance, abolit les actes de son triumvirat, et fonda une constitution qui nous donnait la paix sous un prince. Dès ce moment les liens de l’autorité se resserrèrent, des gardiens veillèrent pour elle, et la loi Papia Poppéa les intéressa par des récompenses à ce que les héritages laissés à quiconque n’aurait pas les privilèges des pères fussent déclarés vacants, et dévolus au peuple romain, à titre de père commun. Mais la délation allait plus loin que la loi ; elle envahit Rome, l’Italie, tout l’empire : déjà beaucoup de fortunes avaient été renversées, et la terreur était dans toutes les familles, quand Tibère, pour arrêter ce désordre, fit désigner par le sort quinze sénateurs, dont cinq anciens préteurs et cinq consulaires, qui, en exceptant beaucoup de cas des gênes de la loi, ramenèrent pour le présent un peu de sécurité.
\subsection[{Néron, le fils de Germanicus nommé questeur, pontife et se marie avec Julie, fille de Drusus}]{Néron, le fils de Germanicus nommé questeur, pontife et se marie avec Julie, fille de Drusus}
\noindent \labelchar{XXIX.} Vers le même temps, le prince recommanda aux sénateurs Néron, l’un des fils de Germanicus, déjà sorti de l’enfance, et sollicita pour lui la dispense d’exercer le vigintivirat \footnote{C’est une dénomination collective, qui comprenait les \emph{triumviri capitales}, les \emph{triumviri monetales}, les \emph{quatuorviri viales}, et les \emph{decemviri litibus judicandis}. Les premiers étaient des magistrats inférieurs, chargés de surveiller la prison publique, et de faire exécuter les jugements criminels.}, et le droit de briguer la questure cinq ans avant l’âge légal ; demande qu’on ne pouvait guère écouter sans rire. Il alléguait que lui-même et son frère avaient obtenu la même faveur par l’intercession d’Auguste : mais dès lors une telle prière donna lieu sans doute a plus d’une raillerie secrète ; et cependant la grandeur des Césars ne faisait que de naître, les souvenirs de la république étaient plus rapprochés, et un beau-père tenait par des liens moins étroits aux enfants de sa femme qu’un aïeul à son petit-fils. Le sénat ajouta la dignité de pontife, et, le jour où Néron fit son entrée au Forum, des largesses furent distribuées au peuple, que la vue d’un fils de Germanicus arrivé à cet âge remplissait d’allégresse. La joie fut redoublée par le mariage de Néron avec Julie, fille de Drusus. Mais, si cette alliance eut l’approbation générale, on vit avec déplaisir Séjan destiné pour beau-père au fils de Claude. On jugea que Tibère souillait la noblesse de sa race, et qu’il élevait beaucoup trop un favori déjà suspect d’une ambition démesurée.
\subsection[{Morts de L. Volusius et de Sallustius Crispus}]{Morts de L. Volusius et de Sallustius Crispus}
\noindent \labelchar{XXX.} À la fin de l’année moururent deux hommes distingués, L. Volusius et Sallustius Crispus. La famille de Volusius, quoique ancienne n’avait jamais dépassé la préture : il y fit entrer le consulat. Il exerça même le pouvoir de la censure pour composer les décuries de chevaliers, et il accumula le premier les immenses richesses qui ont donné tant d’éclat à cette maison. Sallustius, d’origine équestre, avait pour aïeule une sœur de l’illustre historien C. Sallustius, par lequel il fut adopté. Quoique la carrière des honneurs fût ouverte devant lui, il prit Mécène pour modèle, et, sans être sénateur, il surpassait en crédit beaucoup de triomphateurs et de consulaires. Sa manière de vivre n’avait rien d’un ancien Romain ; les recherches de sa parure, le luxe de sa maison, le rendaient plus semblable à un riche voluptueux. Toutefois ces dehors cachaient une vigueur d’esprit capable des plus grandes affaires, et une âme d’autant plus active qu’il affectait davantage le sommeil et l’indolence. Le second dans la confiance du prince tant que vécut Mécène, il fut après lui le principal dépositaire des secrets du palais, et il eut part au meurtre d’Agrippa Postumus. Sur le déclin de son âge, il conserva l’apparence plutôt que la réalité de la faveur, et c’est aussi ce qui était arrivé à Mécène. Est-ce donc la destinée du pouvoir d’être rarement durable ? ou bien le dégoût s’empare-t-il des princes quand ils ont tout accordé, ou des favoris quand ils n’ont plus rien à prétendre ?
\subsection[{Tibère et son fils Drusus consuls en même temps. Une affaire de préséance.}]{Tibère et son fils Drusus consuls en même temps. Une affaire de préséance.}
\noindent \labelchar{XXXI.} Viennent ensuite le quatrième consulat de Tibère et le second de Drusus, remarquables en ce que le père eut son fils pour collègue. Il est vrai que, deux ans auparavant, Tibère avait partagé la même dignité avec Germanicus ; mais c’était à regret, et, après tout, il n’était que son oncle. Au commencement de l’année, Tibère, sous prétexte de rétablir sa santé, se retira dans la Campanie, soit que déjà il préludât à sa longue et continuelle absence \footnote{Ce fut cinq ans après cette époque que Tibère quitta Rome pour ne plus y rentrer. Voy. liv. IV, ch., LVII.}, soit pour laisser Drusus remplir seul et sans l’appui d’un père les fonctions du consulat. Une affaire peu importante, mais qui excita de grands débats, fournit en effet au jeune homme l’occasion de se faire honneur. Domitius Corbulo qui avait exercé la préture, se plaignit au sénat de ce qu’à un spectacle de gladiateurs un jeune noble, nommé Sylla, n’avait pas voulu lui céder sa place. Corbulon avait pour lui son âge, la coutume ancienne, la faveur des vieillards ; Sylla était soutenu par Mamercus Scaurus, L. Arruntius et ses autres parents. Il y eut, des deux côtés, des discours pleins de chaleur : on alléguait les exemples de nos ancêtres, qui avaient réprimé, par de sévères décrets, l’irrévérence de la jeunesse. Enfin Drusus fit entendre des paroles conciliantes, et satisfaction fut donnée à Corbulon par l’organe de Scaurus, oncle et beau-père de Sylla, et le plus fécond des orateurs de son temps. Le même Corbulon ne cessait de dénoncer le mauvais état des chemins, qui, par la fraude des entrepreneurs et la négligence des magistrats, étaient rompus et impraticables dans presque toute l’Italie. Il se chargea volontiers d’y pourvoir ; ce qui tourna moins à l’avantage du public qu’à la ruine de beaucoup de particuliers, auxquels il ôta la fortune et l’honneur par des condamnations et des ventes à l’encan.
\subsection[{Nouveaux troubles en Afrique}]{Nouveaux troubles en Afrique}
\noindent \labelchar{XXXII.} Peu de temps après, un message de Tibère informa les sénateurs que l’Afrique était de nouveau troublée par une incursion de Tacfarinas, et qu’il importait que leur choix désignât un proconsul qui sût la guerre, et dont la force de corps suffît à une telle expédition. Sextus Pompéius saisit cette occasion d’exhaler sa haine contre Manius Lépidus, et le peignit comme « un lâche, un indigent, opprobre de ses ancêtres, qu’il fallait écarter même du proconsulat d’Asie ", reproches désavoués par le sénat, qui trouvait Lépidus plus doué que nonchalant, et lui faisait un honneur plutôt qu’un crime d’une pauvreté héréditaire qui n’avait pas entaché sa noblesse. Lépidus fut donc envoyé en Asie ; et, quant à l’Afrique, on décida que César en choisirait lui-même le gouverneur.
\subsection[{Cécina essaie de faire passer une proposition interdisant à un magistrat d’une province d’y amener son épouse.}]{Cécina essaie de faire passer une proposition interdisant à un magistrat d’une province d’y amener son épouse.}
\noindent \labelchar{XXXIII.} Dans cette discussion, Cécina Sévérus demanda qu’il fût interdit à tout magistrat chargé d’une province d’y mener sa femme avec lui. Il déclara d’abord, à plusieurs reprises « qu’il avait une épouse d’une humeur assortie à la sienne, mère de six enfants, et que, ce qu’il exigeait des autres, il se l’était prescrit à lui-même, l’ayant toujours retenue en Italie, quoiqu’il eût fait quarante campagnes dans différentes provinces. Nos ancêtres avaient eu raison de ne pas vouloir qu’on traînât des femmes avec soi chez les alliés ou les nations étrangères. Une telle compagnie embarrassait en paix par son luxe, en guerre par ses frayeurs, et donnait à une armée romaine l’apparence d’une marche de barbares. Leur sexe n’était pas seulement faible et incapable de soutenir la fatigue : il devenait, quand on le laissait faire, cruel, ambitieux, dominateur. Elles se promenaient parmi les soldats ; les centurions étaient à leurs ordres. Une femme \footnote{Plancine, femme de Pison.} avait présidé naguère aux exercices des cohortes, à la revue des légions. Le sénat savait que, dans tous les procès de concussion, la femme était la plus accusée. C’était à l’épouse du gouverneur que s’attachaient d’abord les intrigants d’une province ; elle s’entremettait des affaires, elle les décidait. À elle aussi on faisait cortège en public ; elle avait son prétoire, et ses ordres étaient les plus absolus, les plus violents. Enchaînées jadis par la loi Oppia \footnote{La loi Oppia fut portée en 541, au plus fort de la seconde guerre punique, par le tribun C. Oppius. Elle défendait aux femmes d’avoir à leur usage plus d’une demi-once d’or, de porter des habits de diverses couleurs, de se faire voiturer à Rome ou à mille pas à la ronde, dans un char attelé de chevaux, si ce n’était pour se rendre aux sacrifices publics. Cette loi fut révoquée en 559, malgré l’opposition énergique du vieux Caton, alors consul.} et par d’autres non moins sages, les femmes, depuis que ces liens étaient rompus, régnaient dans les familles, dans les tribunaux et jusque dans les armées. »\par
\labelchar{XXXIV.} Ce discours eut peu d’approbateurs : on criait de toutes parts que ce n’était pas là le sujet de la délibération, qu’il fallait une autorité plus imposante que celle de Cécina pour une si grande réforme. Bientôt Valérius Messalinus, en qui l’on retrouvait une image de l’éloquence de son père Messala, répondit « que d’heureuses innovations avaient adouci en beaucoup de points la dureté des anciennes mœurs ; qu’en effet Rome n’avait plus, comme autrefois, la guerre à ses portes ou ses provinces pour ennemies ; qu’on faisait aux besoins des femmes certaines concessions, qui, loin d’être à charge aux alliés, ne l’étaient pas même à leurs époux ; qu’en tout le reste la communauté était entière, et que leur présence n’avait rien de gênant dans la paix. À la guerre sans doute il fallait être libre de tout embarras ; mais, au retour des travaux, quel délassement plus honnête que la société d’une épouse ? Quelques femmes peut-être avaient cédé à l’avarice ou à l’ambition, mais les magistrats eux-mêmes n’étaient-ils pas sujets à mille passions diverses ? Cependant on ne laissait pas pour cela les provinces sans gouverneurs. Souvent les vices des femmes avaient corrompu les maris : mais tous ceux qui n’avaient pas de femmes étaient-ils donc irréprochables ? Les lois Oppiennes, disait-il encore, ont été trouvées bonnes jadis, parce que le malheur des temps les rendait nécessaires ; d’autres convenances en ont fait depuis modérer la rigueur. En vain nous voulons déguiser notre faiblesse sous des noms empruntés ; c’est la faute du mari si la femme sort des bornes prescrites. Faut-il, pour un ou deux caractères pusillanimes, ravir aux maris la compagne de leurs plaisirs et de leurs peines ? On doit craindre aussi d’abandonner un sexe naturellement fragile, et de le livrer à son goût pour le luxe et aux passions d’autrui. À peine, sous les yeux surveillants d’un époux, la sainteté du mariage est-elle respectée : que sera-ce, si plusieurs années de séparation et presque de divorce en relâchent les nœuds ? Que l’on remédie aux abus des provinces, mais sans oublier les désordres de Rome. » Drusus ajouta quelques mots comme mari lui-même. Il dit « que le devoir des princes les appelait souvent aux extrémités de l’empire. Combien de fois Auguste n’avait-il pas visité l’Occident et l’Orient, accompagné de Livie ? Lui aussi était allé en Illyrie, et au besoin il irait dans d’autres contrées ; mais ce ne serait pas toujours de bon gré, si on le séparait d’une épouse chérie, qui l’avait rendu père de tant d’enfants. " Ainsi fut éludée la proposition de Cécina.
\subsection[{Tibère nomme deux candidats comme proconsuls en Afrique.}]{Tibère nomme deux candidats comme proconsuls en Afrique.}
\noindent \labelchar{XXXV.} Dans la séance suivante, on reçut une lettre de Tibère où, après avoir indirectement reproché au sénat de rejeter tous les soins sur le prince, il désignait deux candidats au proconsulat d’Afrique, M. Lépidus et Junius Blésus. Tous deux furent entendus : Lépidus s’excusa en termes pressants, alléguant une santé faible, l’âge de ses enfants, une fille à marier. Une raison qu’il ne disait pas, et que tout le monde devinait, c’est que Blésus était oncle de Séjan, titre certain à la préférence. Blésus feignit aussi de refuser, mais il insista beaucoup moins, et les flatteurs s’accordèrent à ne pas le soutenir.
\subsection[{L’image de l’empereur comme sauf-conduit}]{L’image de l’empereur comme sauf-conduit}
\noindent \labelchar{XXXVI.} Ensuite on donna cours à des plaintes renfermées jusqu’alors dans le secret des entretiens privés. Le dernier scélérat, pourvu qu’il tînt une image de l’empereur, était en possession de charger impunément les honnêtes gens d’outrages et d’invectives. L’affranchi même et l’esclave, en menaçant un maître, un patron, du geste ou de la voix, se faisaient redouter. Le sénateur C. Cestius représenta « qu’à la vérité les princes étaient comme des dieux, mais que les dieux n’écoutaient les prières que quand elles étaient justes, que personne ne se réfugiait dans le Capitole ou dans les autres temples pour faire de son asile le théâtre de ses crimes, que les lois étaient renversées, anéanties, depuis qu’Annia Rufilla, condamnée pour fraude à sa requête, venait en plein Forum, à la porte du sénat, l’insulter et le menacer, sans qu’il osât invoquer la justice : cette femme se couvrait d’une image de l’empereur \footnote{Les triumvirs élevèrent à Jules César un temple avec droit d’asile. Cet exemple eut des suites, et bientôt l’impunité fut assurée à tout malfaiteur qui se réfugiait auprès d’une statue de l’empereur régnant. Il paraît même qu’il suffisait, pour se rendre inviolable, de tenir une image du prince dans ses mains.}. Une foule de voix dénoncèrent des traits pareils ou de plus révoltants, et prièrent Drusus de faire un exemple. Rufilla fut mandée, convaincue et mise en prison.
\subsection[{Drusus se fait bien voir.}]{Drusus se fait bien voir.}
\noindent \labelchar{XXXVII.} En même temps Considius Aequus et Célius Cursor, chevaliers romains, qui avaient forgé contre le préteur Magius Cécilianus une accusation de lèse-majesté, furent punis, sur la demande du prince, par un décret du sénat. Ces deux actes tournèrent à la louange de Drusus. Vivant au milieu de Rome, se mêlant aux réunions, aux entretiens de la ville, il passait pour adoucir l’humeur concentrée de son père ; on pardonnait même volontiers à sa jeunesse le goût du plaisir : « Puisse-t-il, disait-on, se livrer à ce penchant, consumer les jours en spectacles, les nuits en festins, plutôt que d’entretenir, seul et loin de toutes les distractions, une vigilance chagrine et des soucis malfaisants ! »
\subsection[{Problèmes en Macédoine et en Thrace}]{Problèmes en Macédoine et en Thrace}
\noindent \labelchar{XXXVIII.} En effet, ni Tibère ni les accusateurs ne se lassaient. Ancharius Priscus avait dénoncé Césius Cordas, proconsul de Crète, comme coupable de concussion, crime auquel il ajoutait celui de lèse-majesté, alors complément nécessaire de toutes les accusations. Tibère, informé qu’Antistius Vétus, un des principaux de la Macédoine, venait d’être absous dans un procès d’adultère, réprimanda les juges, et, sous le même prétexte de lèse-majesté, le ramena devant la justice, comme un factieux, mêlé aux complots de Rhescuporis à l’époque où ce prince, après avoir tué Cotys son neveu, songeait à nous faire la guerre. L’eau et le feu furent interdits à Antistius, et l’on décida qu’il serait confiné dans une île qui ne fût à portée ni de la Thrace ni de la Macédoine. Car, depuis que la Thrace était partagée entre Rhémétalcès et les enfants de Cotys, auxquels on avait donné pour tuteur, à cause de leur bas âge, Trébelliénus Rufus, ces peuples, peu faits à notre présence, étaient mécontents, et ils n’accusaient pas moins Rhémétalcés que Trébelliénus de laisser leurs injures sans vengeance. Les Célètes, les Odruses \footnote{Les Célétes étaient divisés en \emph{majores} et \emph{minores}. Les grands Célétes habitaient au pied du mont Hémus, qui borne la Thrace vers le nord, et les petits au pied du mont Rhodope qui la traverse. – Les Odryses étaient plus voisins des sources de l’Hèbre, aujourd’hui la Maritza.} et d’autres nations puissantes, prirent les armes sous des chefs différents, égaux entre eux par leur obscurité ; ce qui, en tenant leurs forces désunies, nous préserva d’une guerre sanglante. Les uns soulèvent le pays, les autres franchissent le mont Hémus \footnote{Le Balkan.}, afin d’appeler à eux les populations éloignées ; les plus nombreux et les mieux disciplinés assiègent le roi dans Philippopolis \footnote{Maintenant Philippopoli, sur l’Hèbre, à environ 30 lieues ouest-nord-ouest d’Andrinople.}, ville bâtie par Philippe de Macédoine.\par
\labelchar{XXXIX.} À cette nouvelle P. Velléius, qui commandait l’armée la plus voisine, détache les auxiliaires à cheval et les cohortes légères contre les bandes qui couraient le pays pour le piller ou en tirer des renforts. Lui-même s’avance avec le gros de l’infanterie pour faire lever le siège. Tout réussit à la fois : les coureurs furent taillés en pièces ; la discorde éclata parmi les assiégeants, et le roi fit une sortie heureusement combinée avec l’arrivée de la légion. On ne saurait donner le nom de bataille ou de combat à une rencontre où fut massacrée, sans qu’il nous en coûtât de sang, une multitude éparse et mal armée.
\subsection[{Révolte en Gaule}]{Révolte en Gaule}
\noindent \labelchar{XL.} Cette même année les cités gauloises, fatiguées de l’énormité des dettes, essayèrent une rébellion, dont les plus ardents promoteurs furent, parmi les Trévires, Julius Florus, chez les Éduens, Julius Sacrovir, tous deux d’une naissance distinguée, et issus d’aïeux à qui leurs belles actions avaient valu le droit de cité romaine, alors que, moins prodigué, il était encore le prix de la vertu. Dans de secrètes conférences, où ils réunissent les plus audacieux de leurs compatriotes, et ceux à qui l’indigence ou la crainte des supplices faisait du crime un besoin, ils conviennent que Florus soulèvera la Belgique, et Sacrovir les cités les plus voisines de la sienne. Ils vont donc dans les assemblées, dans les réunions, et se répandent en discours séditieux sur la durée éternelle des impôts, le poids accablant de l’usure, l’orgueil et la cruauté des gouverneurs ; ajoutant « que la discorde est dans nos légions depuis la mort de Germanicus ; que l’occasion est belle pour ressaisir la liberté, si les Gaulois considèrent l’état florissant de la Gaule, le dénuement de l’Italie, la population énervée de Rome, et ces armées où il n’y a de fort que ce qui est étranger. »\par
\labelchar{XLI.} Il y eut peu de cantons où ne fussent semés les germes de cette révolte. Les Andécaves et les Turoniens \footnote{L’Anjou et la Touraine} éclatèrent les premiers. Le lieutenant Acilius Aviola fit marcher une cohorte qui tenait garnison à Lyon, et réduisit les Andécaves. Les Turoniens furent défaits par un corps de légionnaires que le même Aviola reçut de Visellius, gouverneur de la basse Germanie, et auquel se joignirent des nobles Gaulois, qui cachaient ainsi leur défection pour se déclarer dans un moment plus favorable. On vit même Sacrovir se battre pour les Romains, la tête découverte, afin, disait-il, de montrer son courage ; mais les prisonniers assuraient qu’il avait voulu se mettre à l’abri des traits en se faisant reconnaître. Tibère, consulté, méprisa cet avis, et son irrésolution nourrit l’incendie.\par
\labelchar{XLII.} Cependant Florus, poursuivant ses desseins, tente la fidélité d’une aile de cavalerie levée à Trèves et disciplinée à notre manière, et l’engage à commencer la guerre par le massacre des Romains établis dans le pays. Quelques hommes cédèrent à la corruption ; le plus grand nombre resta dans le devoir. Mais la foule des débiteurs et des clients de Florus prit les armes, et ils cherchaient à gagner la forêt d’Ardenne, lorsque des légions des deux armées de Visellius et de C. Silius, arrivant par des chemins opposés, leur fermèrent le passage. Détaché avec une troupe d’élite, Julius Indus, compatriote de Florus, et que sa haine pour ce chef animait à nous bien servir, dissipa cette multitude qui ne ressemblait pas encore à une armée. Florus, à la faveur de retraites inconnues, échappa quelque temps aux vainqueurs. Enfin, à la vue des soldats qui assiégeaient son asile, il se tua de sa propre main. Ainsi finit la révolte des Trévires.\par
\labelchar{XLIII.} Celle des Éduens fut plus difficile à réprimer, parce que cette nation était plus puissante, et nos forces plus éloignées. Sacrovir, avec des cohortes régulières s’était emparé d’Augustodunum \footnote{La même ville que Bibracte, maintenant Autun.}, leur capitale, où les enfants de la noblesse gauloise étudiaient les arts libéraux : c’étaient des otages qui attacheraient à sa fortune leurs familles et leurs proches. Il distribua aux habitants des armes fabriquées en secret. Bientôt il fut à la tête de quarante mille hommes, dont le cinquième était armé comme nos légionnaires : le reste avait des épieux, des coutelas et d’autres instruments de chasse. Il y joignit les esclaves destinés au métier de gladiateur, et que dans ce pays on nomme cruppellaires. Une armure de fer les couvre tout entiers, et les rend impénétrables aux coups, si elles les gêne pour frapper eux-mêmes. Ces forces étaient accrues par le concours des autres Gaulois, qui, sans attendre que leurs cités se déclarassent, venaient offrir leurs personnes, et parla mésintelligence de nos deux généraux, qui se disputaient la conduite de cette guerre. Varron, vieux et affaibli, la céda enfin à Silius, qui était dans la vigueur de l’âge.\par
\labelchar{XLIV.} Cependant à Rome ce n’étaient pas seulement les Trévires et les Éduens qu’on disait soulevés : à en croire les exagérations de la renommée, les soixante-quatre cités de la Gaule étaient en pleine révolte ; elles avaient entraîné les Germains, et l’Espagne chancelait. Les gens de bien gémissaient pour la république. Beaucoup de mécontents, en haine d’un régime dont ils désiraient la fin, se réjouissaient de leurs propres périls. Ils s’indignaient que Tibère consumât son temps à lire des accusations, quand le monde était en feu : « Citerait-il Sacrovir devant le sénat comme criminel de lèse-majesté. Il s’était enfin trouvé des hommes qui allaient arrêter par les armes le cours de ses messages sanglants. Mieux valait la guerre qu’une paie misérable. » Tibère, affectant plus de sécurité que jamais, passa ces jours d’alarmes dans ses occupations ordinaires, sans quitter sa retraite, sans changer de visage. Était-ce fermeté d’âme ? ou savait-il combien la voix publique grossissait le danger ?\par
\labelchar{XLV.} Pendant ce temps Silius s’avançait avec deux légions précédées d’un corps d’auxiliaires, et ravageait les dernières bourgades des Séquanes, qui, voisines et alliées des Éduens, avaient pris les armes avec eux. Bientôt il marche à grandes journées sur Augustodunum : les porte-enseigne disputaient de vitesse ; le soldat impatient ne voulait ni du repos accoutumé, ni des longues haltes de la nuit : « qu’il vît seulement l’ennemi, qu’il en fût aperçu, c’était assez pour vaincre. » À douze milles d’Augustodunum, on découvrit dans une plaine les troupes de Sacrovir. Il avait mis en première ligne ses hommes bardés de fer, ses cohortes sur les flancs, et par derrière des bandes à moitié armées. Lui-même, entouré des principaux chefs, parcourait les rangs sur un cheval superbe, rappelant les anciennes gloires des Gaulois, les coups terribles qu’ils avaient portés aux Romains, combien la liberté serait belle après la victoire, mais combien, deux fois subjugués, leur servitude serait plus accablante.\par
\labelchar{XLVI.} Il parla peu de temps et fut écouté avec peu d’enthousiasme : nos légions s’avançaient en bataille, et cette multitude sans discipline et sans expérience de la guerre ne pouvait plus rien voir ni rien entendre. De son côté Silius, à qui l’assurance du succès permettait de supprimer les exhortations, s’écriait cependant « qu’un ennemi comme les Gaulois devait faire honte aux conquérants de la Germanie. Une cohorte vient d’écraser le Turonien rebelle ; une aile de cavalerie a réduit les Trévires, et quelques escadrons de notre armée ont battu les Séquanes : plus riches et plus adonnés aux plaisirs, les Éduens sont encore moins redoutables. Vous êtes vainqueurs ; songez à poursuivre. » L’armée répond par un cri de guerre. La cavalerie investit les flancs de l’ennemi, l’infanterie attaque le front. I1 n’y eut point de résistance sur les ailes ; mais les hommes de fer, dont l’armure était à l’épreuve de l’épée et du javelot, tinrent quelques instants. Alors le soldat, saisissant la hache et la cognée comme s’il voulait faire brèche à une muraille, fend l’armure et le corps qu’elle enveloppe ; d’autres, avec des leviers ou des fourches, renversent ces masses inertes, qui restaient gisantes comme des cadavres, sans faire aucun effort pour se relever. Sacrovir se retira d’abord à Augustodunum ; ensuite, craignant d’être livré, il se rendit, avec les plus fidèles de ses amis, à une maison de campagne voisine. Là il se tua de sa propre main : les autres s’ôtèrent mutuellement la vie ; et la maison, à laquelle ils avaient mis le feu, leur servit à tous de bûcher.\par
\labelchar{XLVII.} Alors seulement Tibère écrivit au sénat pour lui annoncer le commencement et la fin de la guerre. Il en parlait sans rien taire, sans rien exagérer. « Du reste, ajoutait-il, le dévouement et le courage de ses lieutenants, et les mesures prescrites par lui-même, avaient triomphé de tout. » Il expliquait ensuite pourquoi ni lui ni Drusus n’étaient allés dans les Gaules, exaltant « la grandeur de l’empire, dont les chefs oublieraient leur dignité si, pour un ou deux cantons soulevés, ils quittaient la ville d’où partent les ordres qui régissent le monde. Maintenant que la crainte ne pouvait plus le conduire, il irait voir l’état du pays et consolider la paix. » Le sénat décréta des vœux pour son retour, des actions de grâces aux dieux, et d’autres honneurs où les convenances étaient gardées. Le seul Cornélius Dolabella, tombant par émulation de zèle dans une absurde flatterie, proposa que Tibère, à son retour de Campanie, entrât avec l’appareil de l’ovation. Aussi le prince ne tarda-t-il pas à écrire « qu’après avoir dompté les nations les plus belliqueuses, et reçu dans sa jeunesse ou refusé tant de triomphes, il ne se croyait pas assez dépourvu de gloire pour ambitionner à son âge cette vaine récompense d’une promenade aux portes de Rome. »
\subsection[{Mort de Sulpicius Quirinus}]{Mort de Sulpicius Quirinus}
\noindent \labelchar{XLVIII.} À peu près dans le même temps, il demanda au sénat que la mort de Sulpicius Quirinus fût honorée par des funérailles publiques. Quirinus, né à Lanuvium, n’était point de l’ancienne famille patricienne des Sulpicius ; mais sa bravoure à la guerre, et des commissions où il avait montré de l’énergie, lui avaient valu le consulat sous Auguste. Il avait obtenu les ornements du triomphe pour avoir enlevé aux Homonades \footnote{Peuple de la Cilicie Trachée, dont la capitale était Homonada, aujourd’hui Ermenek.}, nation de Cilicie, toutes leurs forteresses. Donné pour conseil à Caïus César, lorsque celui-ci commandait en Arménie, il n’en avait pas moins rendu des hommages à Tibère dans sa retraite de Rhodes. Le prince fit connaître ce fait au sénat, louant l’attachement de Quirinus pour sa personne, et accusant M. Lollius, aux suggestions duquel il attribuait l’injuste inimitié du jeune César. Mais la mémoire de Quirinus n’était point agréable aux sénateurs, tant à cause de ses persécutions contre Lépida, dont j’ai parlé plus haut, que de sa vieillesse avare et odieusement puissante.
\subsection[{Crime de lèse-majesté de C. Lutorius Priscus}]{Crime de lèse-majesté de C. Lutorius Priscus}
\noindent \labelchar{XLIX.} À la fin de l’année, C. Lutorius Priscus, chevalier romain, auteur d’un poème où il avait déploré avec succès la mort de Germanicus, et pour lequel il avait reçu du prince une gratification, tomba dans les mains d’un délateur. Son crime était d’avoir fait d’autres vers pendant une maladie de Drusus, dans l’espoir que, s’il mourait, ils seraient encore mieux récompensés. Lutorius avait eu l’indiscrète vanité de les lire chez P. Pétronius, devant Vitellia, belle-mère de ce dernier, et beaucoup de femmes de distinction. Quand celles-ci virent le fait dénoncé, la frayeur les saisit, et elles avouèrent tout. Vitellia seule protesta qu’elle n’avait rien entendu ; mais les dépositions qui perdaient l’accusé furent plus écoutées, et Hatérius Agrippa, consul désigné, opina pour le dernier supplice.\par
\labelchar{L.} M. Lépidus fut d’un avis contraire. « Pères conscrits, dit-il, si nous n’envisageons que l’horrible pronostic dont C. Lutorius a souillé son imagination et les oreilles qui l’ont entendu, ni le cachot, ni le lacet des criminels, ni les tortures des esclaves ne suffisent pour l’en punir. Mais si la modération du prince, si les exemples de vos ancêtres et les vôtres ont mis des bornes aux remèdes et aux châtiments, quand les désordres et les forfaits n’en ont point, s’il y a loin de l’indiscrétion au crime, des paroles aux attentats ; nous pouvons prononcer un arrêt tel que, sans laisser impunie la faute de Lutorius, nous n’ayons à nous reprocher ni trop d’indulgence ni trop de rigueur. J’ai plus d’une fois entendu le chef de cet empire se plaindre de ceux qui, par une mort volontaire, s’étaient dérobés à sa clémence : Lutorius est encore vivant, et sa vie ne peut être un danger, ni son supplice une leçon. Tout condamnable qu’est son délire, les œuvres en sont vaines et promptement oubliées. Quelle crainte sérieuse pourrait inspirer un insensé qui se trahit lui-même, et qui, n’osant s’adresser aux hommes, va mendier l’approbation de quelques femmes ? Toutefois qu’il s’éloigne de Rome ; qu’il perde ses biens, et qu’il soit privé du feu et de l’eau \footnote{C’était la formule de l’exil : cette interdiction du feu et de l’eau s’étendait à une certaine distance de Rome ou de l’Italie, distance au-delà de laquelle le condamné était libre de choisir sa résidence.}. Et cet avis, je le donne comme si la loi de majesté lui était réellement applicable. »\par
\labelchar{LI.} Seul de tous les sénateurs, le consulaire Rubellius Blandus partagea l’opinion de Lépidus : les autres se rangèrent à celle d’Agrippa. Lutorius fut conduit en prison et mis à mort sur-le-champ. Tibère, dans une lettre pleine de ses ambiguïtés ordinaires, en fit reproche au sénat, exaltant le zèle pieux avec lequel il vengeait les moindres injures du prince, le priant de ne pas punir avec tant de précipitation de simples paroles, louant Lépidus, ne blâmant point Agrippa. Alors il fut résolu que les décrets du sénat ne seraient portés au trésor qu’après dix jours \footnote{Dans l’origine, les sénatus-consultes étaient déposés dans la temple de Cérès, sous la garde des édiles plébéiens. Dans la suite, ils furent portés au trésor public, et ce n’était qu’après cette formalité qu’ils étaient exécutoires.}, et qu’on laisserait aux condamnés cette prolongation de vie. Mais ni les sénateurs n’avaient la liberté de révoquer leurs arrêts, ni le temps n’adoucissait Tibère.
\subsection[{Discours de Tibère sur le luxe à table}]{Discours de Tibère sur le luxe à table}
\noindent \labelchar{LII.} Le consulat suivant fut celui de C. Sulpicius et de Décimus Hatérius. L’année, sans troubles au dehors, ne fut pas au-dedans sans alarmes : on craignit des rigueurs contre le luxe, dont les prodigalités en tout genre ne connaissaient plus de mesure. Il est vrai qu’en dissimulant le prix des achats on tenait cachées les profusions les plus ruineuses, mais les folles dépenses de la table étaient le sujet de tous les entretiens, et l’on tremblait que le prince, d’une économie antique, ne les réprimât durement. Bibulus et, après lui, les autres édiles avaient représenté que la loi somptuaire était méprisée, que les prix illicites qu’on mettait aux denrées s’élevaient de jour en jour, et que des remèdes ordinaires seraient impuissants. Le sénat consulté remit l’affaire à la décision du prince. Tibère réfléchit longtemps s’il était possible d’arrêter cette licence effrénée, si la réforme ne serait pas plus dangereuse que l’abus, combien serait humiliante une tentative sans succès ou un succès qui couvrirait d’opprobre et d’infamie les premiers de l’état. Enfin il écrivit au sénat une lettre dont voici à peu près le sens :\par
\labelchar{LIII.} « Dans toute autre délibération, pères conscrits, le mieux serait peut-être que mon avis sur ce qui convient à l’intérêt public fût demandé et reçu de vive voix. Dans celle-ci, mon absence est préférable : au moins, s’il est des hommes coupables d’un luxe honteux, je ne les verrai pas, désignés par vos regards, rougir devant moi et me rendre témoin de leurs frayeurs. Si les édiles, dont j’estime le courage, en avaient d’abord conféré avec moi, peut-être leur aurais-je conseillé de laisser leur cours à des vices si anciens et si accrédités, plutôt que de nous mettre au hasard de montrer que nous ne pouvons rien contre certains désordres. Mais les édiles ont fait leur devoir comme je voudrais que tous les magistrats s’acquittassent du leur. Quant à moi, je ne puis me taire avec bienséance, et il m’est difficile de parler ; car mon langage ne peut être celui d’un édile ou d’un préteur ou d’un consul : on exige d’un prince des vues plus grandes et plus élevées ; et, quand chacun s’attribue l’honneur du bien qui s’opère, les fautes de tous retombent sur lui seul. Par où commencer la réforme, et que faut-il réduire d’abord à l’antique simplicité ? Sera-ce l’étendue sans limites de nos maisons de campagne ? Cette multitude ou plutôt ces nations d’esclaves ? Ces masses d’or et d’argent ? Ces bronzes précieux et ces merveilles du pinceau ? Ces vêtements qui nous confondent avec les femmes, et la folie particulière à ce sexe, ces pierreries pour lesquelles on transporte chez des peuples étrangers ou ennemis les trésors de l’empire ?\par
\labelchar{LIV.} « Je n’ignore pas que, dans les festins et dans les cercles, un cri général s’élève contre ces abus et en demande la répression : mais faites une loi, prononcez des peines, et les censeurs eux-mêmes s’écrieront que l’État est bouleversé, qu’on prépare la ruine des plus grandes familles, qu’il n’y aura plus personne d’innocent. Cependant, lorsque les maladies du corps sont opiniâtres et invétérées, un traitement sévère et rigoureux peut seul en arrêter le progrès ; ainsi, quand l’âme, à la fois corrompue et corruptrice, nourrit elle-même le feu qui la dévore, il faut, pour éteindre cette fièvre, des remèdes aussi forts que les passions qui l’allumèrent. Tant de lois, ouvrage de nos ancêtres, tant d’autres qu’institua la sagesse d’Auguste, tombées en désuétude ou, ce qui est plus honteux, abrogées par le mépris, n’ont fait qu’enhardir le luxe. Car le vice, encore libre du frein des lois, appréhende de s’y voir soumis ; mais, s’il a pu le briser impunément, ni crainte ni pudeur ne le retiendront plus. Pourquoi donc l’économie régnait-elle autrefois ? C’est que chacun réglait ses désirs ; c’est que nous étions tous citoyens d’une seule ville. Même quand notre domination embrassa l’Italie, la soif des plaisirs n’était pas irritée à ce point. Nos victoires lointaines nous ont appris à dissiper le bien d’autrui, les guerres civiles à prodiguer le nôtre. Qu’est-ce, après tout, que le mal dont se plaignent les édiles ? Combien il paraîtra léger, si l’on porte plus loin ses regards ! Eh ! Personne ne se lève pour nous dire que l’Italie attend sa subsistance de l’étranger ; que chaque jour de la vie du peuple romain flotte à la merci des vagues et des tempêtes ; que, si l’abondance des provinces ne venait au secours et des maîtres, et des esclaves, et de ces champs qui ne produisent plus, ce ne seraient pas sans doute nos parcs et nos maisons de plaisance qui fourniraient à nos besoins. Voilà, pères conscrits, les soins qui occupent le prince ; voilà ce qui, négligé un instant, entraînerait la chute de la république. Pour le reste, il faut chercher le remède en soi-même. Que l’honneur accomplisse notre réforme, la nécessité celle du pauvre, la satiété celle du riche. Ou si quelqu’un des magistrats nous promet assez d’habileté et de vigueur pour s’opposer au torrent, je le loue de son zèle, et je confesse qu’il me décharge d’une partie de mes travaux. Mais si, en voulant se donner le mérite d’accuser le vice, l’on soulève des haines dont on me laissera tout le poids, croyez, pères conscrits, que je suis aussi peu avide d’inimitiés que personne. J’en brave pour la république d’assez cruelles et, trop souvent, d’assez peu méritées ; mais celles qui seraient sans objet, et dont ni moi ni vous ne recueillerions aucun fruit, il est juste qu’on me les épargne. »\par
\labelchar{LV.} Après la lecture du message impérial, on dispensa les édiles du soin qu’ils prenaient, et le luxe de la table, qui, depuis la bataille d’Actium jusqu’à la révolution qui donna l’empire à Galba, s’était signalé par d’énormes profusions, est tombé peu à peu. Je crois à propos de rechercher les causes de ce changement. Autrefois les familles qui joignaient la richesse à la naissance ou à l’illustration s’abandonnaient sans réserve au goût de la magnificence. Alors encore il était permis de se concilier le peuple, les alliés, les rois, et d’en recevoir des hommages. L’opulence, une maison splendide, l’appareil de la grandeur, attiraient de la popularité, des clientèles, qui en rehaussaient l’éclat. Quand des flots de sang coulèrent et qu’une brillante renommée fut un arrêt de mort, le danger rendit les hommes plus sages. En outre, ces nouveaux sénateurs qu’on appelait chaque jour des municipes, des colonies et même des provinces, apportèrent à Rome l’économie de leur pays ; et, quoiqu’on vît la plupart d’entre eux, aidés par la fortune ou par le talent, parvenir à une vieillesse opulente, leur premier esprit se conservait toujours. Mais le principal auteur de la réforme fut Vespasien, qui, à sa table et dans ses vêtements, donnait l’exemple de la simplicité antique. Le désir de plaire au prince et l’empressement de l’imiter furent plus efficaces que la crainte des lois et des châtiments. Ou peut-être les choses humaines sont-elles sujettes à des vicissitudes réglées ; peut-être les mœurs ont-elles leurs périodes comme les temps. Tout ne fut pas mieux autrefois ; notre siècle aussi a produit des vertus et des talents dignes d’être un jour proposé pour modèles. Puisse durer toujours cette rivalité glorieuse avec nos ancêtres !
\subsection[{Drusus tribun du peuple}]{Drusus tribun du peuple}
\noindent \labelchar{LVI.} Tibère, après avoir fait applaudir sa modération en arrêtant les dénonciateurs prêts à tomber sur leur proie, écrivit au sénat une lettre où il demandait pour Drusus la puissance tribunitienne. C’est le titre qu’avait attaché au rang suprême la politique d’Auguste, qui, sans prendre le nom de roi ni de dictateur en voulait un cependant par lequel il dominât tous les autres pouvoirs. Auguste partagea d’abord cette dignité avec M. Agrippa, et, ce dernier étant mort, il y associa Tibère, afin qu’il ne restât sur son successeur aucune incertitude. Il croyait ainsi contenir des ambitions perverses, et il se reposait sur la fidélité de Tibère et sur sa propre grandeur. À son exemple, le prince approchait alors Drusus du trône impérial, après avoir, pendant la vie de Germanicus, tenu son choix indécis entre les deux frères. Il commençait sa lettre par prier les dieux de faire tourner ses desseins à la prospérité de la république ; ensuite il parlait modestement et sans exagération des qualités de son fils, ajoutant « qu’il avait une femme, trois enfants, et l’âge où lui-même fut appelé par Auguste à ces hautes fonctions ; que cette élévation d’ailleurs n’était point prématurée ; que Drusus, éprouvé par huit ans de travaux, ayant réprimé des séditions, terminé des guerres, mérité un triomphe et deux consulats, partagerait des soins qui lui étaient connus. »\par
\labelchar{LVII.} Les sénateurs s’attendaient à cette demande ; aussi l’adulation avait-elle étudié son langage. Elle n’imagina pourtant rien de plus que les honneurs accoutumés, des statues aux deux princes, des autels aux dieux, des temples, des arcs de triomphe. Seulement M. Silanus chercha dans l’avilissement du consulat un moyen d’honorer les empereurs : il fit la proposition que, au lieu de dater par les consuls de l’année, on inscrivît à l’avenir, sur les monuments et les actes soit publics, soit particuliers, les noms de ceux qui exerceraient la puissance tribunitienne. Q. Hatérius voulut aussi que les décrets de ce jour fussent gravés en lettres d’or dans le sénat ; et cette basse flatterie couvrit de ridicule un vieillard qui, à son âge, ne pouvait en recueillir que la honte.
\subsection[{Un flamine de Jupiter veut devenir proconsul}]{Un flamine de Jupiter veut devenir proconsul}
\noindent \labelchar{LVIII.} Cependant, Junius Blésus ayant été continué dans le gouvernement de l’Afrique, Servius Maluginensis, flamine de Jupiter, demanda la province d’Asie. Selon lui, « on avait tort de croire que les ministres de ce dieu ne pouvaient sortir de l’Italie : le droit n’était pas autre pour lui que pour les flamines de Mars et de Quirinus ; or, si ces derniers obtenaient des provinces, pourquoi ceux de Jupiter en seraient-ils privés ? Aucun décret du peuple, aucun livre sur les rites ne prononçait leur exclusion. Souvent les pontifes \footnote{Les pontifes avaient dans leurs attributions le culte de tous les dieux, tandis que les flamines étaient attachés à tel ou tel dieu en particulier. En outre, le collège des pontifes décidait souverainement de toutes les affaires qui intéressaient la religion.} avaient desservi les autels de Jupiter, lorsque la maladie ou des fonctions publiques en éloignaient le flamine. Après le meurtre de Cornélius Mérula \footnote{Après le retour de Marius, en 667, Cornélius Mérule se donna la mort au pied de l’autel de Jupiter, dont il était flamine, en priant le dieu que son sang retombât sur Cinna et sur tout son parti.}, ce sacerdoce avait vaqué soixante et douze ans, et l’on n’avait point vu les sacrifices interrompus. Si le culte du dieu s’était maintenu si longtemps, sans que l’on créât de prêtre, combien serait moins sensible l’absence d’une année qu’exigeait le proconsulat ! C’était pour satisfaire des ressentiments personnels que les grands pontifes avaient interdit les provinces au flamine : maintenant, grâce aux dieux, le premier des pontifes était aussi le premier des hommes ; ni rivalités, ni haines, ni aucune des passions de la condition privée, n’avaient d’empire sur lui. »\par
\labelchar{LIX.} À ces raisonnements l’augure Lentulus et d’autres sénateurs opposaient des réponses diverses, et l’on résolut enfin d’attendre la décision du souverain pontife. Tibère, différant cet examen, apporta quelques restrictions aux décrets rendus pour honorer la puissance tribunitienne de son fils. Il blâma spécialement la proposition de Silanus comme étrange, et dit que l’inscription en caractères d’or serait contraire aux anciens usages. Une lettre de Drusus, lue ensuite, quoique d’une modestie étudiée, fut trouvée pleine d’orgueil. « Voilà donc, disait-on, l’abaissement où Rome est descendue ! Un jeune homme est élevé au faîte des honneurs, et il ne daigne pas visiter les dieux de la cité, venir au sénat, inaugurer du moins sa nouvelle dignité sur le sol de la patrie. La guerre sans doute ou un voyage lointain le retiennent, lui qui parcourt à loisir les rivages et les lacs de la Campanie ! C’est ainsi qu’on forme le maître du monde ; voilà les premières leçons qu’il reçoit de son père : Qu’un vieil empereur fuie comme une vue importune l’aspect des citoyens, il a pour prétexte l’accablement de l’âge et ses travaux passés : mais Drusus, qui peut l’éloigner d’eux, si ce n’est l’arrogance ? »
\subsection[{L’anarchie dans les provinces}]{L’anarchie dans les provinces}
\noindent \labelchar{LX.} Cependant Tibère, content de fortifier dans ses mains les ressorts du pouvoir, offrait au sénat l’image des temps qui n’étaient plus, en renvoyant à sa décision les demandes des provinces. Les asiles se multipliaient sans mesure dans les villes grecques, et cet abus était enhardi par l’impunité. Les temples se remplissaient de la lie des esclaves ; ils servaient de refuge aux débiteurs contre leurs créanciers, aux criminels contre la justice. Point d’autorité assez forte pour réprimer les séditions du peuple, qui, par zèle pour les dieux, protégeait les attentats des hommes. Il fut résolu que chaque ville enverrait des députés avec ses titres. Quelques-unes renoncèrent d’elles-mêmes à des prérogatives usurpées. D’autres invoquaient d’anciennes croyances ou des services rendus au peuple romain. Ce fut un beau jour que celui où les bienfaits de nos ancêtres, les traités conclus avec nos alliés, les décrets mêmes des rois qui avaient eu l’empire avant nous, et le culte sacré des dieux, furent soumis à l’examen du sénat, libre comme autrefois de confirmer ou d’abolir.\par
\labelchar{LXI.} Les Éphésiens eurent audience les premiers. Ils représentèrent « que Diane et Apollon n’étaient point nés à Délos, comme le pensait le vulgaire ; qu’on voyait chez eux le fleuve Cenchrius et le bois d’Ortygie, où Latone, au terme de sa grossesse, et appuyée contre un olivier qui subsistait encore, avait donné le jour à ces deux divinités ; que ce bois avait été consacré par un ordre du ciel ; qu’Apollon lui-même, après le meurtre des Cyclopes, y avait trouvé un asile contre la colère de Jupiter ; que Bacchus victorieux avait épargné celles des Amazones qui s’étaient réfugiées au pied de l’autel ; que dans la suite Hercule, maître de la Lydie, avait accru les privilèges du temple, privilèges restés sans atteinte sous la domination des Perses, respectés par les Macédoniens, et maintenus par nous. »\par
\labelchar{LXII.} Immédiatement après, les Magnésiens \footnote{Magnésie, sur le Méandre.} firent valoir des ordonnances de L. Scipio \footnote{Scipion l’Asiatique, qui remporta sur Antiochus, près de Magnésie de Sipyle, la célèbre victoire qui soumit aux Romains toute l’Asie mineure.} et de L. Sylla, qui, vainqueurs l’un d’Antiochus l’autre de Mithridate, honorèrent le dévouement et le courage de ce peuple en déclarant le temple de Diane Leucophryne \footnote{Il y a plusieurs traditions sur ce surnom de Leucophryne donné à Diane. Il paraîtrait assez naturel de le rapporter à la ville de \emph{Leucophrys}, située dans les plaines du Méandre, où la déesse avait un temple fort révéré. Leucophrys était aussi l’ancien nom de l’île de Ténédos.} un asile inviolable. Les députés d’Aphrodisias \footnote{Aphrodisias, ville considérable de Carie.} et de Stratonice\footnote{Autre ville de Carie, fondée par les Séleucides, et qui tire son nom de Stratonice, femme d’Antiochus Soter.} présentèrent un décret du dictateur César, prix de services anciennement rendus à sa cause, et un plus récent de l’empereur Auguste : ces villes y étaient louées d’avoir subi une irruption des Parthes, sans que leur fidélité envers la république en fût ébranlée. Les Aphrodisiens défendirent les droits de Vénus, les Stratoniciens ceux de Jupiter et d’Hécate. Remontant plus haut, les orateurs d’Hiérocésarée \footnote{Voy. la note 1 de la page 76.} exposèrent que Diane Persique avait chez eux un temple dédié sous le roi Cyrus ; ils citèrent les noms de Perpenna, d’Isauricus \footnote{Perpenna ou Perperna, vainquit Aristonicus ; qui se prétendait héritier d’Attale, et le fit prisonnier dans Stratonice (an de Rome 624). – P. Servilius Isauricus fit la guerre aux pirates de Cilicie et subjugua la nation des Isauriens, d’où il tira son surnom (an de Rome 676).} et de plusieurs autres généraux, qui avaient étendu jusqu’à deux mille pas de distance la sainteté de cet asile. Les Cypriotes parlèrent pour trois temples, élevés, le plus ancien à Vénus de Paphos par Aérias, le second par Amathus, fils d’Aérias, à Vénus d’Amathonte, le troisième à Jupiter de Salamine par Teucer, fuyant la colère de son père Télamon.\par
\labelchar{LXIII.} On entendit aussi les députations des autres peuples. Fatigué de ces longues requêtes et des vifs débats qu’elles excitaient, le sénat chargea les consuls d’examiner les titres, et, s’ils y démêlaient quelque fraude, de soumettre de nouveau l’affaire à sa délibération. Outres les villes que j’ai nommées, les consuls firent connaître « qu’on ne pouvait contester à celle de Pergame son asile d’Esculape, mais que les autres cités ne s’appuyaient que sur de vieilles et obscures traditions. Ainsi les Smyrnéens alléguaient un oracle d’Apollon, en vertu duquel ils avaient dédié un temple à Vénus Stratonicienne ; ceux de Ténos \footnote{Ténos, île de la mer Égée, l’une des Cyclades, à 4 milles d’Andros, à 15 milles de Délos.} une réponse du même dieu, qui leur avait enjoint de consacrer une statue et un sanctuaire à Neptune. Sans remonter à des temps si reculés, Sardes se prévalait d’une concession d’Alexandre victorieux, Milet d’une ordonnance du roi Darius : ces deux villes étaient vouées l’une et l’autre au culte de Diane et d’Apollon. Enfin les Crétois formaient aussi leur demande pour la statue d’Auguste. » Des sénatus-consultes furent rédigés dans les termes les plus honorables, et restreignirent cependant toutes ces prétentions. On ordonna qu’ils seraient gravés sur l’airain et suspendus dans chaque temple, afin que la mémoire en fût consacrée, et que les peuples ne se créassent plus, sous l’ombre de la religion, des droits imaginaires.
\subsection[{Maladie d’Augusta}]{Maladie d’Augusta}
\noindent \labelchar{LXIV.} Vers le même temps, une maladie dangereuse d’Augusta mit le prince dans la nécessité de revenir promptement à Rome ; soit qu’une sincère union régnât encore entre la mère et le fils, soit que leur haine ne fût que déguisée. En effet, lors de la dédicace qu’elle avait faite récemment d’une statue d’Auguste, près du théâtre de Marcellus, Augusta n’avait inscrit le nom de Tibère qu’après le sien ; et l’on croyait que le prince, offensé de ce trait comme d’une insulte à sa majesté, en gardait au fond du cœur un vif ressentiment. Au reste, un sénatus-consulte ordonna des prières solennelles et la célébration des grands jeux \footnote{Les jeux du cirque.} dont on chargea les pontifes, les augures et les quindécemvirs, conjointement avec les septemvirs \footnote{On appelait \emph{septemviri epulones} des prêtres chargés spécialement de présider aux banquets sacrés que l’on offrait aux dieux dans les solennités religieuses. Les pontifes, les augures, les quindécemvirs et les septemvirs formaient ce qu’on appelait les quatre grands collèges de prêtres, \emph{sacerdoces summorum collegiorum}.} et les prêtres d’Auguste. L. Apronius avait proposé que les féciaux y présidassent aussi. Tibère lui opposa les attributions diverses des sacerdoces et l’autorité des exemples : il dit « que jamais les féciaux n’avaient été admis à de si hautes fonctions ; que, si l’on y appelait cette fois les prêtres d’Auguste, c’était comme attachés par leur institution à la famille pour laquelle s’acquittaient les vœux. »
\subsection[{Adulation et bassesse}]{Adulation et bassesse}
\noindent \labelchar{LXV.} Mon dessein n’est pas de rapporter toutes les opinions : je me borne à celles que signale un caractère particulier de noblesse ou d’avilissement, persuadé que le principal objet de l’histoire est de préserver les vertus de l’oubli, et d’attacher aux paroles et aux actions perverses la crainte de l’infamie et de la postérité. Au reste, dans ce siècle infecté d’adulation et de bassesse, la contagion ne s’arrêtait pas aux premiers de l’État, qui avaient besoin de cacher un nom trop brillant sous l’empressement de leurs respects : tous les consulaires, une grande partie des anciens préteurs, et même beaucoup de sénateurs obscurs, se levaient à l’envi pour voter les flatteries les plus honteuses et les plus exagérées. On raconte que Tibère, chaque fois qu’il sortait du sénat, s’écriait en grec : « O hommes prêts à tout esclavage ! » Ainsi, celui même qui ne voulait pas de la liberté publique ne voyait qu’avec dégoût leur servile et patiente abjection.
\subsection[{Procès de C. Silanus, proconsul d’Asie}]{Procès de C. Silanus, proconsul d’Asie}
\noindent \labelchar{LXVI.} De la bassesse, un insensible progrès les menait à la cruauté. C. Silanus, proconsul d’Asie, était dénoncé par la province comme concussionnaire ; le consulaire Mamercus Scaurus, le préteur Junius Otho, l’édile Brutidius Niger, s’emparent de cette victime, et l’accusent d’avoir offensé la divinité d’Auguste, manqué de respect à la majesté de Tibère. Mamercus, s’autorisant d’illustres exemples, citait L. Cotta, accusé par Scipion l’Africain, Serv. Galba par Caton le Censeur, P. Rutilius par M. Scaurus ; comme si c’étaient des crimes de cette espèce qu’eussent poursuivis et Scipion, et Caton, et cet ancien Scaurus que son arrière-petit-fils Mamercus, l’opprobre de ses aïeux, déshonorait par l’infamie de ses œuvres. Junius Otho avait été d’abord maître d’école : devenu sénateur par le crédit de Séjan, il cherchait à pousser, à force d’impudence et d’audace, une fortune sortie du néant. Brutidius, rempli de belles qualités, pouvait, en suivant le droit chemin, arriver à la situation la plus brillante ; mais une impatiente ambition le sollicitait à surpasser d’abord ses égaux, puis ceux d’un rang supérieur, enfin ses propres espérances. Et la même cause a fait la ruine de bien des hommes, d’ailleurs estimables, qui, dédaignant une élévation tardive et sans péril, courent, au risque de se perdre, à des succès prématurés.\par
\labelchar{LXVII.} Gellius Publicola et M. Paconius grossirent le nombre des accusateurs : le premier était questeur de Silanus, l’autre son lieutenant. Il ne paraissait pas douteux que ce proconsul ne fût coupable d’exactions et de violences ; mais l’orage amassé sur sa tête eût fait trembler l’innocence elle-même. À tant de sénateurs ligués contre lui, aux plus habiles orateurs de l’Asie entière, choisis pour l’accuser, il fallait qu’il répondît seul, sans connaître l’art de la parole, et cela dans un danger personnel, circonstance qui intimide l’éloquence la mieux exercée. Et Tibère l’accablait encore de sa voix, de ses regards, le pressant de questions multipliées, sans qu’il lui fût permis de rien éluder, de rien combattre : souvent même il était contraint d’avouer, pour que le prince n’eût pas interrogé vainement. En outre, un agent du fisc avait acheté les esclaves de Silanus, afin qu’on pût les mettre à la torture ; et, de peur qu’un ami généreux ne vînt à son secours, l’accusation de lèse-majesté, supplément aux autres griefs, enchaînait le zèle et faisait du silence une nécessité. Aussi, après avoir demandé une remise de quelques jours, Silanus abandonna sa propre défense : il hasarda toutefois une lettre à César, où il mêlait la plainte aux prières.\par
\labelchar{LXVIII.} Tibère, afin de pallier, à la faveur d’un exemple, l’odieux du traitement qu’il préparait à Silanus, fit lire un mémoire de l’empereur Auguste au sujet de Volésus Messala, aussi proconsul d’Asie, et un sénatus-consulte rendu contre ce magistrat. Ensuite il demanda l’avis de Lucius Piso. Celui-ci, après un éloge pompeux de la clémence du prince, proposa d’interdire à l’accusé le feu et l’eau, et de le reléguer dans l’île de Gyare. Les autres furent du même avis : seulement Cn. Lentulus ajouta que, Silanus étant né d’une mère sans reproche, il était juste d’excepter de la confiscation ses biens maternels et de les rendre à son fils ; ce qui fut approuvé de Tibère. Cornelius Dolabella voulut pousser plus loin l’adulation : il commença par censurer les mœurs de Silanus ; puis il demanda que nul homme d’une vie scandaleuse et d’une réputation décriée ne pût obtenir un gouvernement, exclusion dont le prince serait juge. « En effet, disait-il, si les lois punissent les délits, n’y aurait-il pas bienveillance pour les candidats, avantage pour les provinces, à faire en sorte qu’il ne s’en commît point ? »\par
\labelchar{LXIX.} Tibère combattit cette proposition. Il dit « qu’il n’ignorait pas ce que la renommée publiait de Silanus ; mais que la renommée ne devait pas être la règle de nos jugements ; que beaucoup de gouverneurs avaient agi, dans leurs provinces, autrement qu’on ne l’avait craint ou espéré ; que certaines âmes s’élevaient avec la fortune, et devenaient meilleures où d’autres perdaient leur vertu ; qu’il était impossible que le prince connût tout par lui-même, dangereux qu’il se laissât guider aux passions d’autrui ; que les lois avaient pour objet les faits accomplis, parce que les actes futurs étaient incertains ; qu’ainsi l’avaient voulu nos ancêtres : le délit d’abord, ensuite la peine ; qu’il ne fallait pas changer des institutions sages et consacrées par le temps ; que les princes avaient assez de devoirs, assez même de puissance ; que la justice perdait tout ce que gagnait le pouvoir, et que l’autorité n’avait rien à faire où les lois conservaient leur action. » Des paroles si généreuses sortaient rarement de la bouche de Tibère : celles-ci en furent accueillies avec plus de joie. Le prince, qui savait modérer les sévérités, quand il n’était pas animé par des ressentiments personnels, ajouta « que Gyare était une île sauvage et déserte ; qu’on devait à la famille des Junius, à un homme qui avait été sénateur, de le reléguer plutôt à Cythnos ; que la sœur de Silanus, Torquata, Vestale d’une vertu antique, demandait aussi cette grâce. » Cet avis fut adopté.\par
\labelchar{LXX.} On donna ensuite audience aux Cyrénéens ; et Césius Cordus, accusé par Ancharius Priscus, fut condamné pour concussion. L. Ennius, chevalier romain, était dénoncé comme coupable de lèse-majesté, pour avoir converti en argenterie une statue du prince, et Tibère ne voulait pas qu’on admît l’accusation : il fut hautement combattu par Atéius Capito, qui, avec une fausse indépendance, s’écria « qu’on ne devait pas enlever au sénat sa juridiction, ni laisser un si grand forfait impuni. Que César mît, s’il le voulait, de la mollesse à poursuivre ses injures personnelles ; mais qu’il ne fût pas généreux au préjudice de la vengeance publique. » Tibère prit ces paroles pour ce qu’elles étaient, et persista dans son opposition. Quant à Capito, son ignominie fut d’autant plus éclatante, que, profondément versé dans les lois divines et humaines, il déshonorait un grand mérite d’homme d’État et de belles qualités domestiques.\par
\labelchar{LXXI.} Un doute s’éleva sur le temple où l’on placerait une offrande vouée par les chevaliers romains à la Fortune Equestre pour le rétablissement d’Augusta. La déesse avait des sanctuaires en plusieurs endroits de Rome, mais dans aucun elle n’était adorée sous ce titre. On trouva qu’un temple ainsi nommé existait à Antium, et qu’il n’était point en Italie d’institution religieuse, de lieu sacré, d’image des dieux qui ne fût sous la juridiction suprême du peuple romain ; et le don fut porté à Antium. Pendant qu’on s’occupait de religion, le prince fit connaître sa réponse, différée jusqu’alors, sur l’affaire de Servius Maluginensis, flamine de Jupiter. Il lut un décret des pontifes qui autorisait le ministre de ce dieu à s’absenter plus de deux nuits, pour cause de maladie et avec le consentement du grand pontife, pourvu que ce ne fût point dans le temps des sacrifices publics, ni plus de deux fois par an. Ce règlement, établi sous Auguste, prouvait assez que les prêtres de Jupiter ne pouvaient être absents une année entière, ni gouverner les provinces : on citait même l’exemple d’un grand pontife, L. Métellus. qui avait retenu à Rome le flamine Aulus Postumius \footnote{L’an de Rome 512, pendant la première guerre punique, le consul A. Postumius Albinus se préparait à partir pour la Sicile. Le grand pontife Métellus l’en empêcha, parce qu’il était flamine du dieu Mars.}. Ainsi l’Asie fut donnée au consulaire le plus ancien après Maluginensis.
\subsection[{Embellissement de Rome}]{Embellissement de Rome}
\noindent \labelchar{LXXII.} À la même époque, Lépidus demanda la permission de réparer et d’embellir à ses frais la basilique de Paulus, ouvrage des Emiles et monument de leur nom. Alors encore la munificence privée s’exerçait au profit du public ; et Auguste n’avait pas empêché Taurus, Philippe, Balbus, de consacrer à l’ornement de Rome et à l’illustration de leur postérité les dépouilles ennemies et le superflu d’une immense fortune. C’est dans le même esprit que Lépidus, quoiqu’il eût peu de richesses, voulut renouveler les titres de sa maison. Quant au théâtre de Pompée, qu’un incendie avait réduit en cendres, Tibère déclara qu’aucun membre de la famille ne pouvant suffire aux dépenses de sa reconstruction, il le rebâtirait lui-même, et n’en laisserait pas moins subsister le nom du fondateur. Il fit aussi un grand éloge de Séjan, dont les efforts et la vigilance avaient su, disait-il, borner à un seul édifice les ravages de la flamme. Les sénateurs décernèrent à Séjan une statue, qui serait placée dans le théâtre de Pompée ; et, peu de temps après, Tibère, en décorant des insignes du triomphe Junius Blésus, proconsul d’Afrique, dit qu’il le faisait par estime pour Séjan, dont Blésus était l’oncle.
\subsection[{De nouveau Tacfarinas}]{De nouveau Tacfarinas}
\noindent \labelchar{LXXIII.} Cependant les exploits de Blésus méritaient que cet honneur lui fût personnel. Tacfarinas, souvent chassé par nos troupes, et toujours revenu du fond de l’Afrique avec de nouvelles forces, avait enfin poussé l’insolence jusqu’à envoyer à César une ambassade, qui demandait un établissement pour lui et pour son armée, on menaçait d’une guerre interminable. On rapporte que jamais insulte à l’empereur et au peuple romain n’indigna Tibère comme de voir un déserteur et un brigand ériger en puissance ennemie. « Il n’avait pas été donné à Spartacus lui-même, lorsque, après la défaite de tant d’armées consulaires, il saccageait impunément l’Italie, lorsque les grandes guerres de Sertorius et de Mithridate ébranlaient la république, d’obtenir un traité qui lui garantît le pardon ; et l’empire au faîte de la puissance se rachèterait, par la paix et par des concessions de territoire, des brigandages de Tacfarinas !"Il chargea Blésus d’offrir l’impunité à ceux qui mettraient bas les armes, mais de s’emparer, du chef à quelque prix que ce fût.\par
\labelchar{LXXIV.} Beaucoup de rebelles profitèrent de l’amnistie bientôt, aux ruses du Numide, on opposa le genre de guerre dont il donnait l’exemple. Comme ses troupes, moins fortes que les nôtres, et meilleures pour les surprises que pour le combat, couraient par bandes détachées, attaquant tour à tour ou éludant les attaques et dressant des embuscades, l’armée romaine se mit en marche dans trois directions et sur trois colonnes. Le lieutenant Cornélius Scipio ferma les passages par où l’ennemi venait piller le pays de Leptis \footnote{On connaît, dans l’antiquité, deux villes de Leptis, la grande, aujourd’hui \emph{Lebeda}, dans le pays de Tripoli et la petite, beaucoup plus à l’ouest, dans la province que les Romains nommaient proprement Afrique.} et se sauvait ensuite chez les Garamantes : du côté opposé, le fils de Blésus alla couvrir les bourgades dépendantes de Cirta : au milieu, le général lui-même, avec un corps d’élite, établissait dans les lieux convenables des postes fortifiés ; de sorte que les barbares, serrés, enveloppés de toutes parts, ne faisaient pas un mouvement sans trouver des Romains en face, sur leurs flancs, souvent même sur leurs derrières. Beaucoup furent tués ainsi ou faits prisonniers. Alors Blésus subdivisa ses trois corps en plusieurs détachements, dont il donna la conduite à des centurions d’une valeur éprouvée ; et, l’été fini, au lieu de retirer ses troupes suivant la coutume, et de les mettre en quartier d’hiver dans notre ancienne province, il les distribua dans des forts qui cernaient, pour ainsi dire, le théâtre de la guerre. De là, envoyant à la poursuite de Tacfarinas des coureurs qui connaissaient les routes de ces déserts, il le chassait de retraite en retraite. Il ne revint qu’après s’être emparé du frère de ce chef ; et ce fut encore trop tôt pour le bien des alliés, puisqu’il laissait derrière lui des ennemis prêts à recommencer la lutte. Tibère la considéra cependant comme terminée, et permit que Blésus fût salué par ses légions du nom d’imperator : c’est un titre que les armées, dans l’enthousiasme et les acclamations de la victoire, donnaient jadis aux généraux qui avaient bien mérité de la république. Plusieurs en étaient revêtus à la fois, sans cesser d’être les égaux de leurs concitoyens. Auguste l’avait même accordé à quelques-uns : Blésus le reçut alors de Tibère, et nul ne l’obtint après lui.
\subsection[{Mort d’un grand légiste}]{Mort d’un grand légiste}
\noindent \labelchar{LXXV.} La mort enleva cette année deux hommes d’un grand nom, Asinius Saloninus et Atéius Capito. Petit-fils de M. Agrippa et d’Asinius Pollio, Saloninus était de plus frère de Drusus, et l’empereur lui destinait une de ses petites-filles. Capito, dont j’ai parlé déjà \footnote{Voy. chap. LXX.}, s’était placé par ses vastes connaissances au rang des premiers citoyens. Du reste il avait pour aïeul un centurion de Sylla, et pour père un simple préteur. Auguste l’éleva de bonne heure au consulat, afin que l’éclat de cette dignité lui donnât la prééminence sur Antistius Labéo, son rival dans la science des lois. Car le même siècle vit briller ces deux ornements de la paix : mais Labéo, d’une liberté inflexible, avait une renommée plus populaire ; Capito, habile courtisan, était plus avant dans la faveur du maître. L’un ne parvint qu’à la préture, et cette injustice accrut sa considération ; l’autre fut consul, et l’opinion jalouse s’en vengea par la haine.
\subsection[{Des rancuniers}]{Des rancuniers}
\noindent \labelchar{LXXVI.} Ce fut cette même année, la soixante-quatrième après la bataille de Philippes \footnote{La bataille de Philippes eut lieu l’an de Rome 712.}, que Junie, sœur de Brutus, veuve de Cassius et nièce de Caton, finit sa carrière. Son testament fut le sujet de mille entretiens, parce que, étant fort riche, et mentionnant honorablement dans ses legs presque tous les grands de Rome, elle avait omis l’empereur. Tibère prit cet oubli en citoyen, et n’empêcha pas que l’éloge fût prononcé à la tribune, que la pompe accoutumée décorât les funérailles. On y porta les images de vingt familles illustres : les Manlius, les Quintius y parurent, avec une foule de Romains d’une égale noblesse ; mais Cassius et Brutus, qui n’y furent pas vus, les effaçaient tous par leur absence même.
\section[{Livre quatrième (23, 28)}]{Livre quatrième (23, 28)}\renewcommand{\leftmark}{Livre quatrième (23, 28)}

\subsection[{Ambition de Séjan}]{Ambition de Séjan}
\subsection[{Son portrait}]{Son portrait}
\noindent \labelchar{I.} Sous le consulat de C. Asinius et C. Antistius, Tibère voyait, pour la neuvième année, la république paisible et sa maison florissante (car il comptait la mort de Germanicus au nombre de ses prospérités), quand la fortune commença tout à coup à troubler ce repos. Le prince devint cruel, ou prêta des forces à la cruauté d’autrui. Ce fut l’ouvrage d’Elius Séjanus, préfet des cohortes prétoriennes. J’ai déjà parlé de son crédit : je vais retracer son origine, ses mœurs, et le crime par lequel il tenta de s’élever au pouvoir suprême. Né à Vulsinie \footnote{Ville d’Étrurie, maintenant Bolséna, bourg des États de l’Église.} de Séius Strabo, chevalier romain, il s’attacha dans sa première jeunesse à Caïus César\footnote{Fils d’Agrippa et de Julie, fille d’Auguste.}, petit-fils d’Auguste, et certains bruits l’accusèrent de s’être prostitué pour de l’argent au riche et prodigue Apicius. Bientôt, à force d’artifices, il enchaîna si bien Tibère qu’il rendit confiant et ouvert pour lui seul ce cœur impénétrable à tout autre : ce qu’il faut attribuer moins à l’adresse de Séjan, vaincu dans la suite par des ruses semblables qu’à la colère des dieux sur les Romains, à qui furent également funestes sa puissance et sa chute. Son corps était infatigable, son âme audacieuse. Habile à se déguiser et à noircir les autres, rampant et orgueilleux tout ensemble, il cachait sous les dehors de la modestie le désir effréné des grandeurs ; affectant, pour y parvenir, quelquefois la générosité et le faste, plus souvent la vigilance et l’activité, non moins pernicieuses quand elles servent de masque à l’ambition de régner.
\subsection[{Ses intrigues}]{Ses intrigues}
\noindent \labelchar{II.} Avant lui, la préfecture du prétoire donnait une autorité médiocre ; pour l’accroître, il réunit dans un seul camp les cohortes jusqu’alors dispersées dans Rome. Il voulait qu’elles reçussent ses ordres toutes à la fois, et que leur nombre, leur force, leur vue mutuelle, inspirassent à elles plus de confiance, aux autres plus de terreur. Ses prétextes furent la licence de soldats épars ; les secours contre un péril soudain, plus puissants par leur ensemble ; la discipline, plus sévère entre des remparts, loin des séductions de la ville. Le campement achevé, il s’insinua peu à peu dans l’esprit des soldats par sa familiarité et ses caresses. En même temps il choisissait lui-même les centurions et les tribuns, et n’oubliait pas de se ménager des appuis dans le sénat, en donnant à ses créatures les dignités et les gouvernements ; toujours avoué par la facile complaisance de Tibère, qui, non seulement dans ses entretiens, mais au sénat et devant le peuple, aimait à proclamer Séjan le compagnon de ses travaux, et souffrait que ses images fussent honorées au théâtre, dans les places, et à la tête des légions.
\subsection[{Son adultère avec la femme de Drusus}]{Son adultère avec la femme de Drusus}
\noindent \labelchar{III.} Toutefois, la maison impériale remplie de Césars, un fils dans la force de l’âge, des petits-fils déjà sortis de l’enfance, reculaient le terme de son ambition. Frapper tant de têtes à la fois était dangereux ; et la ruse, plus sûre que la force, demandait un intervalle entre les crimes. Il préféra cependant les voies sourdes de la ruse, et résolut de commencer par Drusus, contre qui l’animait une colère toute récente. Drusus, incapable de souffrir un rival et impétueux de caractère, ayant un jour pris querelle avec Séjan, l’avait menacé de la main, et celui-ci, dans un mouvement pour avancer, avait été frappé au visage. De toutes les vengeances, Séjan trouva que la plus prompte était de s’adresser à la femme de son ennemi. C’était Livie, sœur de Germanicus, qui, dépourvue dans son enfance des agréments de la figure, avait acquis avec l’âge une rare beauté. En feignant pour elle un ardent amour, il commença par la séduire. Une femme qui a sacrifié sa pudeur n’a plus rien à refuser. Quand il eut sur elle les droits du premier crime, il lui mit en tête l’espérance du mariage, le partage du trône, l’assassinat de son époux. Ainsi la nièce d’Auguste, la bru de Tibère, la mère des enfants de Drusus sans respect ni d’elle-même, ni de ses aïeux, ni de ses descendants, se prostituait à un vil étranger, et sacrifiait une grandeur présente et légitime à des espérances criminelles et incertaines. On mit dans la confidence Eudémus, ami et médecin de Livie, et qui, sous prétexte de son art, la voyait souvent sans témoins. Séjan pour ôter tout ombrage à sa complice, répudia sa femme Apicata, dont il avait trois enfants. Toutefois l’énormité de l’attentat multipliait les craintes, les délais, lés résolutions contradictoires.
\subsection[{Tableau de l’empire romain}]{Tableau de l’empire romain}
\subsection[{Drusus prend la toge virile. Projet de visite des provinces}]{Drusus prend la toge virile. Projet de visite des provinces}
\noindent \labelchar{IV.} Ce fut au commencement de cette année que Drusus, un des enfants de Germanicus, prit la robe virile. Tous les décrets rendus en l’honneur de son frère Néron \footnote{sur la mer Tyrrhénienne, prés de Naples.} furent renouvelés pour lui. Tibère prononça en outre un discours où il relevait par de grands éloges la bienveillance paternelle de son fils pour ceux de Germanicus. Car, quoique la puissance et la concorde habitent rarement ensemble, Drusus passait pour aimer ses neveux, ou du moins pour ne pas les haïr. Ensuite le prince remit en avant le projet tant de fois annoncé et toujours feint de visiter les provinces. Il prétexta le grand nombre des vétérans et les levées à faire pour compléter les armées ; ajoutant que les enrôlements volontaires manquaient, ou ne donnaient que des soldats sans courage et sans discipline, parce qu’il ne se présentait guère pour servir que des indigents et des vagabonds. Il fit à ce sujet l’énumération rapide des légions et des provinces qu’elles avaient à défendre. Je crois à propos de dire aussi ce que Rome avait alors de forces militaires, quels rois étaient ses alliés, et combien l’empire était moins étendu qu’aujourd’hui.\par
\labelchar{I.} Voy. liv. III, XXIX.
\subsection[{Dénombrement des forces romaines}]{Dénombrement des forces romaines}
\noindent \labelchar{V.} Deux flottes, l’une à Misène(1), l’autre à Ravenne \footnote{Sur L’Adriatique}, protégeaient l’Italie sur l’une et l’autre mer ; et des galères qu’Auguste avait prises à la bataille d’Actium et envoyées à Fréjus gardaient, avec de bons équipages, la partie des Gaules la plus rapprochée. Mais la principale force était sur le Rhin, d’où elle contenait également les Germains et les Gaulois ; elle se composait de huit légions. Trois légions occupaient l’Espagne, dont on n’avait que depuis peu achevé la conquête. Juba régnait sur la Mauritanie, présent du peuple romain. Le reste de l’Afrique était gardé par deux légions, l’Égypte par deux autres ; quatre suffisaient pour tenir en respect les vastes contrées qui, à partir de la Syrie, s’étendent jusqu’à l’Euphrate et confinent à l’Albanie, à l’Ibérie \footnote{L’Albanie s’étend au levant de l’Ibérie, le long de la mer Caspienne, jusqu’au Cyrus ou Kur. Les Turcs l’appellent Dagh-istan, ou pays de montagnes.}, et à d’autres royaumes dont la grandeur romaine protège l’indépendance. La Thrace était sous les lois de Rhémétalcès et des enfants de Cotys. Deux légions en Pannonie, deux en Mésie, défendaient la rive du Danube. Deux autres, placées en Dalmatie, se trouvaient, par la position de cette province, en seconde ligne des précédentes, et assez près de l’Italie pour voler à son secours dans un danger soudain. Rome avait d’ailleurs ses troupes particulières, trois cohortes urbaines et neuf prétoriennes, levées en général dans l’Étrurie, l’Ombrie, le vieux Latium, et dans les plus anciennes colonies romaines. Il faut ajouter les flottes alliées, les ailes et les cohortes auxiliaires, distribuées selon le besoin et la convenance des provinces. Ces forces étaient presque égales aux premières ; mais le détail en serait incertain, puisque, suivant les circonstances, elles passaient d’un lieu dans un autre, augmentaient ou diminuaient de nombre.
\subsection[{L’administration}]{L’administration}
\noindent \labelchar{VI.} Il sera bon de jeter aussi un regard sur les autres parties de l’administration, et de voir quels principes les dirigèrent, jusqu’à l’année qui apporta dans le gouvernement de Tibère un funeste changement. Et d’abord les affaires publiques et les plus importantes des affaires particulières se traitaient dans le sénat. Les principaux de cet ordre discutaient librement, et, s’ils tombaient dans la flatterie, le prince était le premier à les arrêter. Dans la distribution des honneurs, il avait égard à la noblesse des aïeux, à la gloire militaire, à l’éclat des talents civils. On convenait généralement qu’il n’aurait pu faire de meilleurs choix. Les consuls, les préteurs, conservaient l’extérieur de leur dignité ; les magistrats subalternes exerçaient sans obstacle l’autorité de leurs charges. Les lois, si l’on excepte celle de majesté, étaient sagement appliquées. Les blés de la république, les impôts et les autres revenus de l’État étaient affermés à des compagnies de chevaliers romains. Quant à ses intérêts privés, le prince en chargeait les hommes les plus intègres, quelques-uns sans les connaître et sur la foi de leur renommée. Son choix fait, il y était fidèle, même jusqu’à l’excès ; et la plupart vieillissaient dans leur emploi. Le peuple souffrait de la cherté des grains ; mais ce n’était pas la faute du prince. Tibère n’épargna même ni soins ni dépenses pour remédier à la stérilité de la terre et aux accidents de mer. Il veillait à ce que de nouvelles charges ne portassent point l’effroi dans les provinces, et il empêchait que les anciennes ne fussent aggravées par l’avarice ou la cruauté des magistrats : on ne parlait ni de punitions corporelles ni de confiscations.
\subsection[{L’irrésistible ascension de Séjan}]{L’irrésistible ascension de Séjan}
\subsection[{Empoisonnement de Drusus}]{Empoisonnement de Drusus}
\noindent \labelchar{VII.} Les domaines du prince en Italie étaient peu nombreux, ses esclaves retenus, sa maison bornée à quelques affranchis. Était-il en différend avec un particulier ? On allait au Forum, et la justice prononçait. Sans doute il lui manquait des manières affables, et son air repoussant n’inspirait guère que la terreur. Toutefois il retint ces sages maximes, jusqu’à ce qu’elles fussent renversées par la mort de Drusus ; car elles régnèrent tant que celui-ci vécut. Séjan voulait signaler par de bons conseils son pouvoir naissant. Il craignait aussi dans Drusus un vengeur qui ne déguisait pas sa haine et se plaignait souvent que, l’empereur ayant un fils, un autre fût appelé le compagnon de ses travaux. « Et à quoi tenait-il encore qu’il ne fût nommé son collègue ? Les premières espérances de l’ambition étaient d’un haut et difficile abord ; ce degré franchi, elle trouvait un parti, des ministres. Déjà un camp avait été construit au gré du préfet ; on avait mis des soldats dans ses mains ; son image brillait au milieu des monuments de Pompée \footnote{Une statue de bronze avait été dédiée à Séjan dans le théâtre de Pompée.} ; le sang des Drusus allait communiquer sa noblesse aux petits-fils de Séjan. Il était temps après cela d’implorer sa modération, pour qu’il daignât se borner ! » Et ce n’était ni rarement, ni devant un petit nombre d’amis qu’il tenait ces discours. Ses paroles les plus secrètes étaient révélées d’ailleurs par son infidèle épouse.\par
\labelchar{VIII.} Séjan pensa donc qu’il fallait se hâter. Il choisit un poison dont l’action lente et insensible imitât les progrès d’une maladie naturelle. Il le fit donner à Drusus par l’eunuque Lygdus, comme on le reconnut huit ans après. Pendant la maladie de son fils, Tibère, sans inquiétude ou par ostentation de courage, alla tous les jours au sénat. Il y alla même entre sa mort et ses funérailles. Les consuls, en signe d’affliction, s’étaient assis parmi les simples sénateurs : il les fit souvenir de la place qui appartenait à leur dignité ; et pendant que l’assemblée s’abandonnait aux larmes, seul étouffant ses gémissements, il la releva par un discours suivi. Il dit qu’on le blâmerait peut-être de se montrer aux regards du sénat, dans ces premiers moments de douleur où l’on se refuse même aux entretiens de ses proches, où l’on supporte à peine la lumière du jour ; que, sans accuser de faiblesse un sentiment si naturel aux affligés, il avait cherché dans les bras de la république des consolations plus dignes d’une âme forte. Ensuite, après quelques réflexions douloureuses sur l’extrême vieillesse de sa mère, sur le bas âge de ses petits-fils, et sur ses propres années, qui penchaient vers leur déclin, il demanda qu’on fît entrer les fils de Germanicus, unique adoucissement aux maux qui l’accablaient. Les consuls sortent, adressent à ces jeunes hommes des paroles d’encouragement, et les amènent devant l’empereur. Celui-ci les prenant par la main : « Pères conscrits, dit-il, quand la mort priva ces enfants de leur père, je les confiai à leur oncle, et, quoique lui-même eût des fils, je le priai de les chérir, de les élever comme s’ils étaient son propre sang, de les former pour lui et pour sa postérité. Maintenant que Drusus nous est ravi, c’est à vous que j’adressa mes prières. Je vous en conjure, en présence des dieux et de la patrie, adoptez les arrière-petits-fils d’Auguste, les rejetons de tant de héros ; soyez leurs guides ; remplissez auprès d’eux votre place et la mienne. Et vous, Néron et Drusus \footnote{Des trois fils de Germanicus, Tibère ne recommanda aux sénateurs que les deux plus âgés, Néron et Drusus. Le seul dont il ne parla point est celui qui régna.}, voilà ceux qui vous tiendront lieu de pères. Dans le rang où vous êtes nés, vos biens et vos maux intéressent la république. »\par
\labelchar{IX.} À ce discours beaucoup de larmes coulèrent, beaucoup de vœux s’élevèrent au ciel ; et, si Tibère eût ensuite gardé le silence, il laissait tous les cœurs pénétrés d’attendrissement et remplis de sa gloire. En retombant sur le propos vain et usé par le ridicule, de remettre l’empire et d’en charger ou les consuls ou quelque autre chef, il décrédita même ce qu’il pouvait avoir dit de paroles sincères et généreuses. La mémoire de Drusus reçut tous les honneurs déjà rendus à Germanicus, et beaucoup d’autres qu’y ajouta la flatterie, toujours pressée d’enchérir sur elle-même. Des images sans nombre décorèrent la pompe de ses funérailles. Énée, tige de la maison des Jules, tous les rois Albains, Romulus, fondateur de la ville, puis les nobles effigies de la branche sabine, Attus Clausus et tous les Claudius, ses descendants, formaient un long et majestueux cortège.\par
\labelchar{X.} Dans ce récit de la mort de Drusus, j’ai rapporté les faits attestés par les auteurs les plus nombreux et les plus dignes de foi. Cependant je ne puis omettre un bruit tellement accrédité alors qu’il n’a pas encore perdu toute créance. Séjan, dit-on, après avoir engagé Livie dans le crime par le déshonneur, s’assura, par une liaison non moins infâme, de l’eunuque Lygdus, chéri de son maître pour sa jeunesse et sa beauté, et chargé dans la maison de Drusus d’un des premiers emplois. Ensuite, quand le jour et le lieu de l’empoisonnement furent arrêtés entre les complices, il eut l’audace de donner le change ; et, accusant Drusus en termes couverts de méditer un parricide, il avertit le prince d’éviter la première coupe qui lui serait présentée à la table de son fils. Dupe de cet artifice, le vieillard remit à Drusus la coupe qui lui fut offerte au commencement du repas, et le jeune homme sans défiance la vida d’un seul trait, ce qui fortifia les soupçons : on crut que, dans son effroi et sa honte, il s’était condamné lui-même à la mort qu’il destinait à son père.\par
\labelchar{XI.} Tels étaient les bruits populaires ; mais aucune autorité ne les confirme, et le moindre examen les réfute. Quel homme en effet, je ne dirai pas instruit comme Tibère par ce que l’expérience a de plus hautes leçons, mais quel homme de bon sens, aurait pu, sans entendre son fils, lui présenter la mort, et cela de sa propre main, au risque de se préparer d’inutiles regrets ? Neût-il pas plutôt mis à la torture l’esclave qui offrait le poison ? Ne serait-il pas remonté à la source du crime ? Aurait-il renoncé, pour un fils unique jusqu’alors irréprochable, à cette lenteur circonspecte dont il usait même envers des étrangers ? Mais l’opinion qu’il n’était pas un forfait dont Séjan ne fût capable, la faiblesse du prince pour ce favori, la haine dont l’un et l’autre étaient l’objet, accréditaient jusqu’aux fables les plus monstrueuses ; et la renommée se plaît à entourer la mort des princes de tragiques circonstances. D’ailleurs Apicata, femme de Séjan, révéla toute l’intrigue ; et les aveux d’Eudème et de Lygdus, à la torture, la mirent dans le plus grand jour. Aussi de tous les écrivains qui, en haine de Tibère, ont recherché et grossi tous ses torts, pas un n’a chargé sa mémoire de ce trait odieux. J’ai voulu le rapporter et le combattre, afin de confondre, par un exemple éclatant, les traditions mensongères, et d’engager ceux dans les mains de qui tombera ce fruit de mon travail à ne point préférer des récits incroyables, avidement reçus par la multitude, à des faits réels, et que n’a point altérés l’amour du merveilleux.
\subsection[{Séjan s’attaque à Agrippine}]{Séjan s’attaque à Agrippine}
\noindent \labelchar{XII.} Tibère prononça du haut de la tribune un éloge de son fils, que le sénat et le peuple accueillirent avec les démonstrations de la douleur plutôt qu’avec une émotion véritable. On pensait à Germanicus, et l’on voyait avec une joie secrète se relever sa maison. Mais cette popularité naissante et les espérances trop peu déguisées de sa veuve Agrippine en hâtèrent la chute. Quand Séjan vit que la mort de Drusus n’était ni vengée sur ses assassins ni pleurée des Romains, emporté par l’audace du crime et l’ivresse d’un premier succès, il ne songea plus qu’aux moyens de détruire les enfants de Germanicus, qui devaient naturellement succéder à l’empire. On ne pouvait leur donner à tous trois du poison ; la fidélité de leurs gouverneurs et la vertu de leur mère formaient autour d’eux un impénétrable rempart. Il prend le parti d’accuser de révolte la fierté d’Agrippine ; il arme contre elle la haine invétérée d’Augusta et les nouveaux intérêts de sa complice Livie, afin que toutes deux la dénoncent au prince comme une femme orgueilleuse de sa fécondité, appuyée sur la faveur populaire, et insatiable de domination. Il employait en outre d’adroits calomniateurs, au nombre desquels il avait choisi, comme l’instrument le plus propre à ses desseins, Julius Postumus, amant de Mutilie, et devenu par ce commerce adultère un des familiers d’Augusta, auprès de laquelle Mutilie était toute puissante. Alarmant ainsi la vieillesse d’une femme jalouse de son pouvoir, il rendait l’aïeule intraitable pour sa bru. L’intrigue trouvait même auprès d’Agrippine des complices, dont les perfides suggestions exaspéraient de son côté ce caractère altier.
\subsection[{Affaires provinciales}]{Affaires provinciales}
\subsection[{Prières des alliés}]{Prières des alliés}
\noindent \labelchar{XIII.} Cependant Tibère, sans interrompre un instant ses travaux accoutumés, et cherchant sa consolation dans les soins de l’empire, réglait les droits des citoyens, écoutait les prières des alliés. Les villes de Cibyre \footnote{Cibyre, ville considérable de Phrygie, qui paraît sous le nom de Buruz dans les annales turques.} en Asie, d’Égium en Achaïe \footnote{D’Anville croit qu’Aegium est remplacé par la ville moderne de Vostitza.} avaient été ruinées par des tremblements de terre. Des sénatus-consultes, rendus à la demande du prince, les déchargèrent pour trois ans de l’impôt. Vibius Sérénus, proconsul, de l’Espagne ultérieure, condamné pour son excessive dureté d’après la loi sur la violence publique, fut déporté dans l’île d’Amorgos \footnote{Ile de l’Archipel grec, connue encore aujourd’hui sous le même nom.}. Carsidius Sacerdos, accusé d’avoir fourni du blé à Tacfarinas, ennemi de l’empire, fut absous. Caïus Gracchus, poursuivi pour la même cause, le fut également. Gracchus avait partagé dès l’enfance l’exil de son père Sempronius qui l’avait emmené avec lui dans l’île de Cercine. Élevé parmi des bannis, dans toute l’ignorance de ce pays barbare, il n’avait pour subsister d’autre ressource que d’échanger en Sicile et en Afrique quelques viles marchandises : obscure condition qui ne put le dérober aux périls des grandes fortunes. Si Elius Lamia et L. Apronius, anciens gouverneurs d’Afrique, n’eussent protégé son innocence, l’éclat d’un nom malheureux et l’influence des destins paternels le perdaient à son tour
\subsection[{Députations grecques}]{Députations grecques}
\noindent \labelchar{XIV.} Des villes grecques envoyèrent encore cette année des députations. Les habitants de Samos demandaient pour le temple de Junon, ceux de Cos pour le temple d’Esculape, la confirmation d’un ancien droit d’asile. Les Sauriens s’appuyaient sur un décret des Amphictyons, juges suprêmes de toutes les affaires au temps où les Grecs, par les villes qu’ils avaient fondées en Asie, régnaient sur toutes les côtes de cette mer. Ceux de Cos produisaient des titres d’une égale antiquité, et leur temple avait des droits à notre reconnaissance : ils l’avaient ouvert aux citoyens romains, pendant qu’on les égorgeait, par ordre de Mithridate, dans toutes les îles et toutes les cités de l’Asie. Ensuite les préteurs renouvelant, contre la licence des histrions, des plaintes longtemps inutiles, le prince soumit enfin cette affaire au sénat. Il représenta que ces bouffons troublaient la tranquillité publique et portaient le déshonneur dans les familles ; que les vieilles scènes des Osques \footnote{C’étaient les mêmes scènes qu’on appelait atellanes, d’Atella, ville des Osques où ces jeux avaient pris naissance.} sans procurer au peuple beaucoup d’amusement, étaient devenues l’occasion de tant d’audace et de scandales, qu’il fallait pour les réprimer toute l’autorité du sénat. Les histrions furent chassés d’Italie.
\subsection[{Condamnation de Capito, procurateur d’Asie}]{Condamnation de Capito, procurateur d’Asie}
\noindent \labelchar{XV.} La même année mit de nouveau le prince en deuil, en lui ravissant un des jumeaux de Drusus, et un ami dont la perte ne l’affligea pas moins. C’était Lucilius Longus, le compagnon de sa bonne et de sa mauvaise fortune, et le seul des sénateurs qui l’eût suivi dans sa retraite de Rhodes. Aussi, quoique Lucilius fût un homme nouveau, un sénatus-consulte lui décerna, aux frais du trésor, des funérailles solennelles et une statue dans le forum d’Auguste. Car toutes les affaires se traitaient encore dans le sénat. C’est même par ce corps que fut jugé Lucilius Capito, procurateur d’Asie, accusé par la province. Tibère protesta hautement qu’il ne lui avait donné de pouvoir que sur ses esclaves et sur ses domaines particuliers ; que, si son intendant s’était arrogé les droits d’un gouverneur et avait employé la force militaire, c’était au mépris de ses ordres ; qu’ainsi on écoutât les plaintes des alliés. Le procès fut instruit et Capito condamné. Reconnaissantes de cet acte de justice, et de la vengeance qu’elles avaient obtenue l’année précédente contre Silanus, les villes d’Asie décernèrent un temple à l’empereur, à sa mère et au sénat. On leur permit de l’élever, et Néron adressa pour elles au sénat et à son aïeul des actions de grâces qui excitèrent dans tous les cœurs de douces émotions. La mémoire de Germanicus était encore présente ; on croyait le voir, on croyait l’entendre. La modestie du jeune homme, son air noble et digne d’un si beau sang, ajoutaient à l’illusion et s’embellissaient de tout ce que la haine trop connue de Séjan lui préparait de dangers.
\subsection[{Affaires religieuses}]{Affaires religieuses}
\subsection[{Un nouveau flamine}]{Un nouveau flamine}
\noindent \labelchar{XVI.} Vers le même temps mourut le flamine de Jupiter, Servius Maluginensis. Tibère, en consultant le sénat sur le choix de son successeur, proposa de changer la loi qui réglait cette élection. Il dit que l’ancien usage de nommer d’abord trois patriciens nés de parents unis par confarréation \footnote{La confarréation était un acte religieux, que l’on accomplissait en présence de dix témoins et avec des paroles solennelles. On offrait un sacrifice où l’on employait un gâteau fait avec l’espèce de blé nommée \emph{far}. La confarréation rendait l’union de l’homme et de la femme indissoluble, et le divorce impossible.}, et d’élire parmi eux le flamine, était devenu d’une pratique difficile, En effet, la confarréation était abolie, ou ne se conservait que dans un petit nombre de familles. Il en donnait plusieurs causes : d’abord l’insouciance des deux sexes ; ensuite les difficultés mêmes de la cérémonie, que l’on aimait à s’épargner ; enfin l’intérêt de la puissance paternelle, dont le flamine de Jupiter et sa femme étaient affranchis. Il était d’avis qu’on adoucît par un sénatus-consulte la rigueur de l’usage, ainsi qu’Auguste avait accommodé aux nouvelles mœurs plusieurs institutions d’une sévérité trop antique. Ce point de religion soigneusement éclairci, on résolut de ne rien innover à l’égard du flamine lui-même ; mais une loi ordonna que l’épouse du flamine serait sous la puissance de son mari pour ce qui regarde le culte de Jupiter, et, que, pour le reste, elle demeurerait soumise au droit commun des femmes. Le fils de Malugnensis fut substitué à son père. Afin de relever la dignité des sacerdoces et d’exciter pour le service des autels plus de zèle et d’empressement, on assigna deux millions de sesterces à la vestale Cornélie, élue pour remplacer Scantia ; et il fut décidé que désormais Augusta s’assoirait parmi les vestales, toutes les fois qu’elle irait au théâtre.
\subsection[{Affaires intérieures}]{Affaires intérieures}
\subsection[{Tibère jaloux de Néron et Drusus, les enfants de Germanicus}]{Tibère jaloux de Néron et Drusus, les enfants de Germanicus}
\noindent \labelchar{XVII.} Sous le consulat de Cornélius Céthégus et de Visellius Varro, les pontifes, et à leur exemple les autres prêtres, offrant des vœux pour le prince, recommandèrent aux mêmes dieux Néron et Drusus, moins par tendresse pour eux que par esprit de flatterie ; et, dans un État corrompu l’absence et l’excès de la flatterie sont également dangereux. Tibère n’avait jamais aimé la famille de Germanicus ; mais voir honorer des enfants à l’égal de sa vieillesse lui causa un dépit dont il ne fut pas maître. Il fit venir les pontifes et leur demanda si c’était aux prières ou aux menaces d’Agrippine qu’ils avaient accordé ce triomphe. Ils s’en défendirent ; et cependant ils furent censurés, mais avec ménagement, car ils étaient tous les parents de l’empereur ou les premiers de Rome. Au reste, dans un discours au sénat, le prince recommanda pour l’avenir de ne point enorgueillir par des honneurs prématurés de jeunes et mobiles esprits. Séjan animait sa colère. Il lui montrait la république divisée comme par une guerre civile ; le nom de parti d’Agrippine prononcé par des hommes qui se vantaient d’en être. « Et ce parti grossira si on ne l’étouffe ; le seul moyen d’arrêter les progrès de la discorde est de frapper une ou deux des têtes les plus séditieuses. »
\subsection[{Procès contre C. Silius}]{Procès contre C. Silius}
\noindent \labelchar{XVIII.} Il dirigea ses attaques contre C. Silius et Titius Sabinus. L’amitié de Germanicus leur fut fatale à tous deux, et ce n’était pas le seul crime de Silius. Il avait commandé sept ans une puissante armée, mérité en Germanie les ornements du triomphe, vaincu le rebelle Sacrovir. C’était une grande victime qui, tombant avec cette masse de gloire, répandrait par sa chute une profonde terreur. Plusieurs pensèrent que son indiscrétion aggravait ses dangers. Il répétait avec trop de jactance que ses légions étaient restées fidèles, quand toutes les autres se soulevaient, et que l’empire aurait changé de maître, si l’esprit de révolte avait gagné son armée. De tels souvenirs semblaient détrôner Tibère, et sa fortune se sentait accablée sous le poids d’un si grand service : car le bienfait conserve son mérite, tant que l’on croit pouvoir s’acquitter ; quand la reconnaissance n’a pas de prix assez haut, on le paye par la haine.\par
\labelchar{XIX.} Silius avait pour femme Sosia Galla, aimée d’Agrippine, et à ce titre haïe de Tibère. On résolut de les frapper tous deux, et d’ajourner Sabinus ; et l’on mit en avant le consul Marron, qui, sous prétexte de venger son père, servait honteusement les passions de Séjan. L’accusé demandait un court délai, jusqu’à ce que l’accusateur fût sorti de charge. Le prince s’y opposa. « De tout temps, selon lui, les magistrats avaient cité en justice des hommes privés, et il ne fallait pas attenter aux droits du consul, sur la vigilance duquel reposait le salut de la république. » Ce fut le crime de Tibère d’emprunter au passé son langage, pour déguiser des forfaits tout nouveaux. Il assemble donc le sénat avec des protestations hypocrites, comme si les lois eussent été intéressées au jugement de Silius, comme si Marron eût été un consul, ou le gouvernement de Tibère une république. L’accusé se tut ; ou, s’il hasarda quelques mots pour sa défense, il ne laissa pas ignorer quelle haine l’accablait. On l’accusait d’avoir longtemps dissimulé, par une connivence coupable, la trahison de Sacrovir, déshonoré sa victoire par des rapines, et toléré les excès de sa femme. Et certes, l’un et l’autre se seraient difficilement justifiés du reproche de concussion ; mais tout le procès roula sur le crime de lèse-majesté, et Silius prévint par une mort volontaire une condamnation inévitable.\par
\labelchar{XX.} On sévit cependant contre ses biens ; et ce ne fut pas pour rendre l’argent aux peuples tributaires : personne ne le redemandait. Mais on reprit les libéralités d’Auguste, et l’on supputa rigoureusement tout ce qui pouvait retourner au fisc. Ce fut la première fois que Tibère regarda le bien d’autrui d’un œil intéressé. Sosia fut, sur l’avis d’Asinius Gallus, condamnée à l’exil. Gallus voulait que la moitié de ses biens fût confisquée, l’autre moitié laissée à ses enfants. Mais M. Lépidus fit donner le quart aux accusateurs, pour obéir à la loi, et les enfants conservèrent le reste. Je trouve que Lépidus fut, pour ces temps malheureux, un homme sage et ferme. J’en juge par tant d’arrêts cruels, que l’adulation dictait aux autres et qu’il fit adoucir. Et cependant sa conduite ne manquait pas de ménagement, puisqu’il conserva jusqu’à la fin son influence et l’amitié de Tibère. C’est ce qui me fait douter si l’ascendant irrésistible qui règle notre sort destine aussi, dès la naissance, aux uns la faveur des princes, aux autres leur disgrâce ; ou si la sagesse humaine ne peut pas, entre la résistance qui se perd et la servilité qui se déshonore, trouver une route exempte à la fois de bassesse et de périls. Messalinus Cotta, d’une naissance non moins illustre, mais d’un caractère différent, proposa de décréter que tout magistrat dont la femme serait accusée par la province, fût-il innocent lui-même et eût-il ignoré le crime, serait puni cependant comme s’il en était l’auteur.
\subsection[{Procès contre Calpurnius Pison}]{Procès contre Calpurnius Pison}
\noindent \labelchar{XXI.} On s’occupa ensuite de Calpurnius Pison, ce noble renommé par la fierté de son esprit. C’est lui qui, s’élevant contre les manœuvres des délateurs, avait protesté en plein sénat, qu’il sortirait de Rome, et qui, bravant le pouvoir d’Augusta, n’avait pas craint de traîner en justice Urgulanie et de l’arracher du palais de César. Tibère respecta pour le moment cette liberté républicaine. Mais, dans une âme qui se repliait sur les offenses passées, en vain la blessure avait été légère ; le souvenir l’aggravait. Q. Granius accusa Pison de discours tenus secrètement contre la majesté du prince. Il lui reprochait en outre d’avoir chez lui du poison, et de venir au sénat armé d’une épée : imputations qui tombèrent, décréditées par leur gravité même. Sur les autres griefs, qu’on accumulait en grand nombre, l’accusation fut reçue, mais non poursuivie : Pison mourut à propos. On entendit aussi un rapport sur Cassius Sévérus, déjà exilé. Cet homme, d’une basse origine, d’une vie malfaisante, mais puissant par la parole, avait soulevé contre lui tant de haines, qu’un arrêt du sénat, rendu sous la religion du serment, l’avait relégué en Crète. Là, continuant ses habitudes perverses, il s’attira de nouvelles inimitiés et réveilla les anciennes. Dépouillé de ses biens et privé du feu et de l’eau, il vieillit sur le rocher de Sériphe.
\subsection[{La femme d’un préteur défenestrée}]{La femme d’un préteur défenestrée}
\noindent \labelchar{XXII.} Vers le même temps, le préteur Silvanus avait, pour un motif qu’on ignore, précipité d’une fenêtre sa femme Apronia. Traîné devant César par son beau-père Apronius, il répondit avec trouble : il feignait un profond sommeil, pendant lequel sa femme, sûre qu’il ne la voyait pas, s’était elle-même donné la mort. Tibère court à l’instant dans la maison, visite l’appartement, et y trouve des signes certains de violence et de résistance. Il fit son rapport au sénat, et des juges furent nommés. Mais Urgulanie, aïeule de Silvanus, envoya un poignard à son petit-fils. On crut que c’était le prince qui lui avait donné ce conseil, à cause de sa liaison avec Augusta. Silvanus, après avoir vainement essayé le fer, se fit ouvrir les veines. Bientôt Numantina, sa première femme, accusée d’avoir, par des philtres et des enchantements, aliéné sa raison, fut déclarée innocente.
\subsection[{Politique extérieure}]{Politique extérieure}
\subsection[{Fin de Tacfarinas en Afrique}]{Fin de Tacfarinas en Afrique}
\noindent \labelchar{XXIII.} Cette année délivra enfin le peuple romain de la longue guerre du Numide Tacfarinas. Jusqu’alors nos généraux, contents d’obtenir les ornements du triomphe, laissaient reposer l’ennemi dès qu’ils croyaient les avoir mérités. Déjà trois statues couronnées de laurier s’élevaient dans Rome, et Tacfarinas mettait encore l’Afrique au pillage. Il s’était accru du secours des Maures, qui, abandonnés par la jeunesse insouciante de Ptolémée, fils de Juba, au gouvernement de ses affranchis, s’étaient soustraits par la guerre à la honte d’avoir des esclaves pour maîtres. Receleur de son butin et compagnon de ses ravages, le roi des Garamantes, sans marcher avec une armée, envoyait des troupes légères, que la renommée grossissait en proportion de l’éloignement. Du sein même de la province \footnote{La partie de l’Afrique septentrionale dont les Romains avaient fait une province.}, tous les indigents, tous les hommes d’une humeur turbulente, couraient sans obstacle sous les drapeaux du Numide. En effet, Tibère, croyant l’Afrique purgée d’ennemis par les victoires de Blésus, en avait rappelé la neuvième légion ; et le proconsul de cette année, P. Dolabella, n’avait osé la retenir : il redoutait les ordres de César encore plus que les périls de la guerre.\par
\labelchar{XXIV.} Cependant Tacfarinas, ayant semé le bruit que la puissance romaine, entamée déjà par d’autres nations, se retirait peu à peu de l’Afrique, et qu’on envelopperait facilement le reste des nôtres, si tous ceux qui préféraient la liberté à l’esclavage voulaient fondre sur eux, augmente ses forces, campe devant Thubusque et investit cette place. Aussitôt Dolabella rassemble ce qu’il a de soldats ; et, grâce à la terreur du nom romain, jointe à la faiblesse des Numides en présence de l’infanterie, il chasse les assiégeants par sa seule approche, fortifie les postes avantageux, et fait trancher la tête à quelques chefs musulans qui préparaient une défection. Puis, convaincu par l’expérience de plusieurs campagnes qu’une armée pesante et marchant en un seul corps n’atteindrait jamais des bandes vagabondes, il appelle le roi Ptolémée avec ses partisans, et forme quatre divisions qu’il donne à des lieutenants ou à des tribuns. Des officiers maures choisis conduisaient au butin des troupes légères ; lui-même dirigeait tous les mouvements.\par
\labelchar{XXV.} Bientôt on apprit que les Numides, réunis près des ruines d’un fort nommé Auzéa, qu’ils avaient brûlé autrefois, venaient d’y dresser leurs huttes et de s’y établir, se fiant sur la bonté de cette position tout entourée de vastes forêts. À l’instant, des escadrons et des cohortes, libres de tout bagage et sans savoir où on les mène, courent à pas précipités. Au jour naissant, le son des trompettes et un cri effroyable les annonçaient aux barbares à moitié endormis. Les chevaux des Numides étaient attachés ou erraient dans les pâturages. Du côté des Romains, tout était prêt pour le combat, les rangs de l’infanterie serrés, la cavalerie à son poste. Chez les ennemis, rien de prévu : point d’armes, nul ordre, nul mouvement calculé ; ils se laissent traîner, égorger, prendre comme des troupeaux. Irrité par le souvenir de ses fatigues, et joyeux d’une rencontre désirée tant de fois et tant de fois éludée, le soldat s’enivrait de vengeance et, de sang. On fit dire dans les rangs de s’attacher à Tacfarinas, connu de tous après tant de combats ; que, si le chef ne périssait, la guerre n’aurait jamais de fin. Mais le Numide, voyant ses gardes renversés, son fils prisonnier, les Romains débordant de toutes parts, se précipite au milieu des traits, et se dérobe à la captivité par une mort qu’il fit payer cher. La guerre finit avec lui.\par
\labelchar{XXVI.} Le général demanda les ornements du triomphe et ne les obtint pas. Tibère eût craint de flétrir les lauriers de Blésus, oncle de son favori. Mais Blésus n’en fut pas plus illustre, et la gloire de Dolabella s’accrut de l’honneur qui lui était refusé. Avec une plus faible armée, il avait fait des prisonniers de marque, tué le chef ennemi, mérité le renom d’avoir terminé la guerre. À sa suite arrivèrent des ambassadeurs des Garamantes, spectacle rarement vu dans Rome. Effrayée de la chute de Tacfarinas, et n’ignorant pas ses propres torts, cette nation les avait envoyés pour donner satisfaction au peuple romain. Sur le compte qui fut rendu des services de Ptolémée pendant cette guerre, on renouvela un usage des premiers temps : un sénateur fut député pour lui offrir le sceptre d’ivoire, la toge brodée, antiques présents du sénat, et le saluer des noms de roi, d’allié et d’ami.
\subsection[{Révolte servile en Italie}]{Révolte servile en Italie}
\noindent \labelchar{XXVII.} Ce même été, le hasard étouffa en Italie les germes d’une guerre d’esclaves. Le chef de la révolte, T. Curtisius, autrefois soldat prétorien, avait d’abord tenu à Brindes et dans les villes voisines des assemblées secrètes, et maintenant, par des proclamations publiquement affichées, il appelait à la liberté les pâtres grossiers et féroces de ces forêts lointaines, lorsque arrivèrent, comme par une faveur des dieux, trois birèmes destinées à protéger la navigation de cette mer. Le questeur Curtius Lupus, auquel était échue la surveillance des pâturages, de tout temps réservée à la questure, se trouvait aussi dans ces contrées. Il se mit à la tête des soldats de marine, et dissipa cette conjuration au moment même où elle éclatait. Bientôt le tribun Staïus, envoyé à la hâte par Tibère avec un fort détachement, traîne à Rome le chef et ses plus audacieux complices. L’alarme y était déjà répandue, à cause de la multitude des esclaves qui croissait sans mesure, pendant que la population libre diminuait chaque jour.
\subsection[{Politique intérieure}]{Politique intérieure}
\subsection[{Procès contre Vibius Serenus}]{Procès contre Vibius Serenus}
\noindent \labelchar{XXVIII.} Sous les mêmes consuls, on vit un exemple horrible des misères et de la cruauté de ces temps, un père accusé, un fils accusateur. Tous deux, nommés Vibius Sérénus, furent introduits dans le sénat. Arraché de l’exil, le père, dans un triste et hideux appareil, écoutait enchaîné le discours de son fils. Le jeune homme, élégamment paré, le visage rayonnant, tout à la fois dénonciateur et témoin, parlait de complots formés contre le prince, d’émissaires envoyés dans les Gaules pour y souffler la révolte. C’était, ajoutait-il, l’ancien préteur Cécilius Cornutus qui avait fourni l’argent. Cornutus, pour abréger ses inquiétudes, et persuadé que le péril était la mort, se hâta de mourir. Quant à l’accusé, rien n’abattit son courage. Tourné vers son fils, il secouait ses chaînes, invoquait les dieux vengeurs, afin qu’ils lui rendissent un exil où il ne verrait pas de telles mœurs, et que leur justice atteignît quelque jour un fils dénaturé. Il protestait que Cornutus était innocent et victime de fausses terreurs ; qu’on en aurait la preuve en exigeant le nom des autres complices : car sans doute deux hommes n’avaient pas conjuré seuls la mort du prince et le renversement de l’État.\par
\labelchar{XXIX.} Alors l’accusateur nomma Cn. Lentulus et Séius Tubéro ; à la grande confusion de César, qui voyait les premiers de Rome, ses plus intimes amis, Lentulus, d’une extrême vieillesse, Tubéro, d’une santé languissante, accusés d’avoir appelé la guerre étrangère et conspiré contre la république. Tous deux furent aussitôt déchargés. On mit à la question les esclaves du père, et leurs dépositions confondirent le dénonciateur. Celui-ci, égaré par le délire du crime, effrayé des clameurs du peuple, qui le menaçait du cachot fatal \footnote{Dans la prison publique bâtie sur le penchant du mont Capitolin, vis-à-vis du Forum, était un cachot souterrain où l’on exécutait les criminels condamnés à mort. C’est aujourd’hui la chapelle souterraine d’une petite église qu’on appelle \emph{San Pietro in carcere}, parce que saint Pierre y fut mis en prison.}, de la roche tarpéienne, ou du supplice des parricides, s’enfuit de la ville. Ramené par force de Ravenne, il fut contraint de continuer sa poursuite. Tibère ne cachait pas sa vieille haine contre l’exilé Sérénus. Après la condamnation de Libon, celui-ci s’était plaint, dans une lettre à l’empereur, d’être le seul dont le zèle fût resté sans récompense ; et il avait ajouté quelques paroles trop fières pour ne pas blesser des oreilles superbes et délicates. Tibère, après huit ans, rappela ses griefs, et chargea le temps intermédiaire d’imputations diverses, « toutes certaines, disait-il, quoique la torture n’eût arraché aucun aveu à l’opiniâtreté des esclaves. »\par
\labelchar{XXX.} Plusieurs sénateurs furent d’avis que Sérénus fût puni à la manière de nos ancêtres \footnote{Peut-être par l’expression « punir à la manière des ancêtres, " faut-il entendre en général punir de mort.}. Tibère s’y opposa, pour diminuer l’odieux de cette affaire. Gallus Asinius voulait qu’on l’enfermât à Gyare ou à Donuse. Il s’y opposa encore, parce que ces deux îles manquaient d’eau, et qu’on devait laisser les moyens de vivre lorsque l’on accordait la vie. Sérénus fut reconduit à l’île d’Armogos. Comme Cornutus s’était donné la mort, on parla de supprimer les récompenses des accusateurs, lorsqu’un homme, poursuivi pour lèse-majesté, se serait lui-même privé de l’existence avant la fin du procès. On allait se ranger à cet avis, si Tibère, contre sa coutume, se déclarant ouvertement pour les accusateurs, ne se fût plaint avec dureté « que les lois perdaient leur sanction, que la république était au bord du précipice. Mieux valait renverser tous les droits que d’ôter les gardiens qui veillaient à leur maintien. » Ainsi l’on faisait un appel aux délateurs, et cette race d’hommes, née pour la ruine publique, et que nul châtiment ne réprima jamais assez, était encouragée par des récompenses.
\subsection[{Clémence et sévérité de Tibère}]{Clémence et sévérité de Tibère}
\noindent \labelchar{XXXI.} Cette triste succession d’événements douloureux fut interrompue par un moment de joie. Un chevalier romain, C. Cominius, convaincu d’avoir fait contre l’empereur des vers satiriques, obtint sa grâce ; César l’accorda aux prières de son frère, qui était sénateur. Étrange contradiction ! Tibère voyait le bien, il connaissait la gloire attachée à la clémence, et il préférait la rigueur ! Car ce n’était pas faute de lumières qu’il s’égarait ; il en faut peu, d’ailleurs, pour discerner si l’enthousiasme qu’excitent les actions des princes est feint ou véritable. Lui-même n’étudiait pas toujours son langage ; et ses paroles, ordinairement pénibles et embarrassées, coulaient plus faciles et plus abondantes quand il prêtait sa voix au malheur. Du reste, il fut inflexible pour Suilius, ancien questeur de Germanicus, convaincu d’avoir pris de l’argent dans un procès où il était juge. On le bannissait d’Italie : le prince voulut qu’il fût relégué dans une île ; et telle fut la chaleur avec laquelle il soutint son avis, qu’il affirma par serment que le bien public l’exigeait. Cette sévérité, mal reçue d’abord, devint un sujet d’éloges après le retour de Suilius, que la génération suivante vit, également puissant et vénal, jouir longtemps de l’amitié de Claude et toujours en abuser. La même peine fut proposée contre le sénateur Catus Firmius, pour avoir intenté faussement à sa sœur une accusation de lèse-majesté. C’est Catus qui avait, comme je l’ai dit \footnote{Voy. livre II, chap. XXVII et suiv.}, attiré Libon dans le piège, pour le dénoncer ensuite et le perdre. Tibère reconnut ce service en lui faisant remettre, sous d’autres prétextes, la peine de l’exil. Il le laissa cependant exclure du sénat.
\subsection[{Digression de Tacite}]{Digression de Tacite}
\noindent \labelchar{XXXII.} Peut-être la plupart des faits que j’ai rapportés et de ceux que je rapporterai encore sembleront petits et indignes de l’histoire, je le sais ; mais on ne doit pas comparer ces \emph{Annales} aux monuments qu’ont élevés les historiens de l’ancienne république. De grandes guerres, des prises de villes, des rois vaincus et captifs, et, au-dedans, les querelles des tribuns et des consuls, les lois agraires et frumentaires, les rivalités du peuple et des nobles, offraient à leurs récits une vaste et libre carrière. La mienne est étroite et mon travail sans gloire : une paix profonde ou faiblement inquiétée, Rome pleine de scènes affligeantes, un prince peu jaloux de reculer les bornes de l’empire. Toutefois il ne sera pas inutile d’observer des faits indifférents au premier aspect, mais d’où l’on peut souvent tirer de grandes leçons.\par
\labelchar{XXXIII.} En effet, chez toutes les nations, dans toutes les villes, c’est le peuple, ou les grands, ou un seul, qui gouverne. Une forme de société, composée de mélange heureusement assorti des trois autres, est plus facile à louer qu’à établir ; et, fût-elle établie, elle ne saurait être durable. Rome vit autrefois le peuple et le sénat faire la loi tour à tour ; et alors il fallait connaître le caractère de la multitude, et savoir par quels tempéraments on peut la diriger ; alors qui avait étudié à fond l’esprit du sénat et des grands, possédait le renom de sage et d’habile politique. Aujourd’hui que tout est changé, et que Rome ne diffère plus d’un État monarchique, la recherche et la connaissance des faits que je rapporte acquièrent de l’utilité. Peu d’hommes, en effet, distinguent par leurs seules lumières ce qui avilit de ce qui honore, ce qui sert de ce qui nuit : les exemples d’autrui sont l’école du plus grand nombre. Au reste, si ces détails sont utiles, j’avoue qu’ils offrent peu d’agrément. La description des pays, les scènes variées des combats, les morts fameuses des chefs, voilà ce qui attache, ce qui ranime l’attention. Mais moi, dans cet enchaînement d’ordres barbares, de continuelles accusations, d’amitiés trompeuses, d’innocents condamnés, et de procès qui tous ont une même issue, je ne rencontre qu’une monotone et fatigante uniformité. Ajoutez que les anciens écrivains trouvent peu de censeurs passionnés. Et qu’importe au lecteur qu’on relève plus ou moins la gloire des armées romaines ou carthaginoises ? Mais beaucoup de ceux qui, sous Tibère, subirent le supplice ou l’infamie, ont une postérité ; et, en supposant même leurs familles éteintes, il y aura toujours des hommes qui, se reconnaissant dans vos peintures, croiront que vous leur reprochez les bassesses d’autrui. La vertu même offense quelquefois, et les gloires trop récentes paraissent accuser ce qui ne leur ressemble pas. Mais je reviens à mon sujet.
\subsection[{Procès contre Crémutius Cordus}]{Procès contre Crémutius Cordus}
\noindent \labelchar{XXXIV.} Sous les consuls Cornélius Cossus et Asinius Agrippa, Crémutius Cordus fut l’objet d’une accusation nouvelle et jusqu’alors sans exemple : « Il avait publié des \emph{Annales}, où il louait Brutus et appelait Cassius le dernier des Romains. » Les accusateurs étaient Satrius Sécundus et Pinarius Natta, clients de Séjan. Ce fut la perte de l’accusé, prononcée d’ailleurs par la colère qui se peignait sur le visage du prince en écoutant sa défense. Résolu de quitter la vie, Crémutius parla en ces termes : « Pères conscrits, on accuse mes paroles, tant mes actions sont innocentes : mais ces paroles mêmes n’attaquent ni César ni sa mère, les seuls qu’embrasse la loi de majesté. J’ai loué, dit-on, Brutus et Cassius ! Beaucoup d’autres ont écrit leur histoire, et personne n’a parlé d’eux sans éloge. Tite-Live, signalé entre les auteurs par son éloquence et sa véracité, a donné tant de louanges à Pompée, qu’Auguste l’appelait le Pompéien ; et leur amitié n’en fut point affaiblie. Scipion, Afranius \footnote{Scipion Métellus, qui, après la bataille de Pharsale, se rendit en Afrique, et y continua la guerre, de concert avec Caton, Varus, Afranius, Pétréius, Labiénus, et les autres chefs du parti pompéien, aidés des secours de Juba roi de Mauritanie.}, Cassius lui-même et Brutus, n’ont jamais reçu de Tite Live les noms de brigands et de parricides qu’on leur prodigue aujourd’hui. Souvent même il en parle comme de personnages illustres. Les écrits d’Asinius Pollion \footnote{C’est celui auquel Virgile adresse sa fameuse églogue \emph{Sicelides musae}, etc., et Horace la première ode du second livre, \emph{Motum ex Metello consule}, etc.} ne retracent d’eux que de nobles souvenirs ; Messala Corvinus \footnote{M. Valérius Messala Corvinus avait composé un ouvrage sur les familles romaines, cité par Pline, mais perdu.} appelait hautement Cassius son général : et cependant Messala et Pollion vécurent au sein de l’opulence et des honneurs. Cicéron fit un livre où il élevait Caton jusqu’au ciel. Quelle vengeance en tira le dictateur César ? Il répondit par un autre livre, comme s’il eût plaidé devant des juges. Les lettres d’Antoine, les harangues de Brutus, contiennent des invectives, fausses, il est vrai, mais sanglantes, contre Auguste. Dans Bibaculus \footnote{M. Furius Bibaculus, poète satirique, dont il ne reste que deux fragments très courts, cités par Suétone.}, dans Catulle, on lit une foule de vers où les Césars sont outragés. Et ces dieux de l’empire, les Jules, les Auguste, souffrirent ces offenses et les dédaignèrent. Gloire en soit rendue à leur sagesse, autant peut-être qu’à leur modération ! Car une satire méprisée tombe d’elle-même ; en témoigner de la colère, c’est accepter le reproche. »\par
\labelchar{XXXV.} « Je ne parle pas des Grecs : chez eux la licence même n’eut pas plus de frein que la liberté ; ou, si jamais des paroles furent punies, ce fut par des paroles. Mais certes on peut toujours, librement et sans crime, exprimer sa pensée sur ceux que la mort a soustraits à la haine et à la faveur. Brutus et Cassius couvrent-ils donc de leurs bataillons armés les plaines de Philippes, tandis qu’orateur séditieux j’excite le peuple à la guerre civile ? ou ne sont-ils pas morts depuis soixante-dix ans ? et, quand on peut contempler leurs traits sur des images respectées même du vainqueur, serait-il défendu à l’histoire de conserver aussi leur souvenir ? La postérité rend à chacun l’honneur qui lui est dû. Si je suis condamné, on n’oubliera pas Cassius et Brutus, et quelques-uns peut-être se souviendront de moi. » Après ce discours, il sortit de l’assemblée et mit fin à sa vie en se privant de nourriture. Le sénat enjoignit aux édiles de brûler son ouvrage ; mais l’ouvrage subsista, caché, puis reproduit : tant la tyrannie est insensée de croire que son pouvoir d’un moment étouffera jusque dans l’avenir le cri de la vérité ! Persécuter le génie, c’est en augmenter l’influence ; et ni les rois étrangers, ni ceux qui à leur exemple ont puni les talents, n’ont rien obtenu que honte pour eux-mêmes et gloire pour leurs victimes.
\subsection[{Les provinces}]{Les provinces}
\subsection[{Cyzique}]{Cyzique}
\noindent \labelchar{XXXVI.} Au reste, de continuelles accusations remplirent tellement cette année, que, même pendant les féries latines \footnote{Quarante-sept peuples latins étaient liés par une confédération religieuse, et célébraient chaque année, sur le mont Albain, une fête en l’honneur de \emph{Jupiter latiaris}, à laquelle présidaient les Romains. Tous les magistrats romains y assistaient. Pour que la ville ne restât point livrée à l’anarchie pendant leur absence, on créait un magistrat temporaire, sous le nom de préfet de Rome à cause des féries latines. C’est cette charge qui est ici confiée à Drusus, fils de Germanicus.}, au moment où Drusus, préfet de Rome, était monté sur son tribunal pour prendre possession de sa charge, Calpurnius Salvianus vint lui dénoncer Sext. Marius : trait odieux qui, publiquement censuré par Tibère, valut l’exil à son auteur. La ville de Cyzique fut accusée d’avoir négligé le culte d’Auguste et usé de violence envers des citoyens romains, Elle perdit la liberté, qu’elle avait méritée dans la guerre de Mithridate, lorsque, assiégée par ce prince, elle dut sa délivrance à son courage autant qu’aux armes de Lucullus. Fontéius Capito, ancien proconsul d’Asie, fut absous. On reconnut fausses les imputations alléguées contre lui par Vibius Sérénus. Toutefois Sérénus ne porta pas la peine de sa calomnie : la haine publique lui servait de sauvegarde ; car tout accusateur un peu redoutable devenait en quelque sorte une personne sacrée : les délateurs sans nom et sans conséquence étaient seuls punis.
\subsection[{L’Espagne}]{L’Espagne}
\noindent \labelchar{XXXVII.} Vers le même temps, l’Espagne ultérieure envoya des députés au sénat pour demander la permission d’élever, à l’exemple de l’Asie, un temple à César et à sa mère. L’âme de Tibère avait cette force qui fait mépriser les honneurs. Il crut d’ailleurs que c’était l’occasion de réfuter les bruits qui l’accusaient de s’être plié aux faiblesses de la vanité, et il tint à peu près ce discours : « Je sais, pères conscrits, que mon caractère a paru se démentir lorsque, les villes d’Asie ayant fait dernièrement la même demande, je ne l’ai pas combattue. Je vais exposer à la fois les raisons de mon silence passé, et ce que j’ai résolu pour l’avenir. Quand Pergame \footnote{C’est sans doute un lieu peu considérable de l’Anatolie, nommé Bergamo.} voulut consacrer un temple à Auguste et à Rome, ce prince immortel ne s’y opposa pas ; et, comme toutes ses actions et toutes ses paroles sont pour moi des lois inviolables, j’ai suivi d’autant plus volontiers un exemple déjà donné, que le sénat devait partager avec moi la vénération des peuples. Mais si c’est une chose excusable d’avoir accepté une fois, laisser dans toutes les provinces adorer nos images parmi celles des dieux, serait vanité, orgueil. Le culte d’Auguste s’avilira d’ailleurs, si l’adulation le prodigue sans mesure.\par
\labelchar{XXXVIII.} « Oui, pères conscrits, je suis mortel ; les devoirs que je remplis sont ceux d’un mortel, et c’est assez pour moi d’être placé au rang suprême ; vous m’en êtes témoins, et je veux que la postérité s’en souvienne : trop heureux si elle pense un jour que je fus digne de mes ancêtres, attentif à vos intérêts, ferme dans les périls, prêt à braver toutes les inimitiés pour servir l’État. Mes temples, mes statues, je veux les avoir dans vos cœurs ; voilà les plus beaux, les plus durables des monuments : ceux qu’on élève en marbre sont méprisés comme de vils sépulcres, si la haine de la postérité révoque l’apothéose. Puissent donc les alliés, les citoyens, les dieux mêmes, entendre ma prière ! Que ceux-ci m’accordent, jusqu’à la fin de ma vie, la paix de l’âme et l’intelligence des lois divines et humaines ; et quand j’aurai payé tribut à la nature, puissent les autres donner quelques éloges à ma mémoire et prononcer mon nom avec reconnaissance ! » Il continua depuis à repousser, jusque dans ses plus secrets entretiens, un culte dont il serait l’objet. Les uns voulaient que ce fût modestie, d’autres défiance de lui-même, d’autres enfin bassesse d’âme. « Les grands hommes, en effet, aspiraient aux grandes récompenses. C’était par là que Bacchus et Hercule chez les Grecs, Quirinus chez les Romains, avaient pris place parmi les dieux. Honneur à Auguste, qui sut espérer ! Les princes possèdent tous les autres biens : un seul leur reste à conquérir, et ils en doivent être insatiables : c’est une immortalité glorieuse. Qui méprise la gloire, méprise aussi la vertu. »
\subsection[{Séjan}]{Séjan}
\subsection[{Demande en mariage Livie}]{Demande en mariage Livie}
\noindent \labelchar{XXXIX.} Séjan, tout à la fois enivré de sa fortune et stimulé par l’impatience d’une femme (car Livie demandait instamment le mariage promis), présente un mémoire à César : c’était l’usage alors de s’adresser par écrit à l’empereur, même présent. « Comblé, lui disait-il, des bontés d’Auguste, son père, honoré par lui-même de tant de preuves d’estime, il avait pris l’habitude de confier ses espérances et ses vœux aux oreilles des princes, avant de les porter aux dieux. Et jamais il n’avait désiré l’éclat des dignités : veiller et travailler, comme le dernier des soldats, pour la sûreté de l’empereur, était plus de son goût. Cependant de tous les honneurs il avait obtenu le plus grand, celui d’être jugé digne d’allier sa famille à celle de César. C’était là l’origine de son espérance ; et, comme il avait entendu dire qu’Auguste, voulant donner un époux à sa fille, avait un moment jeté les yeux sur de simples chevaliers romains, il priait César, s’il en cherchait un pour Livie, de ne pas oublier un ami qui dans cette alliance n’envisageait que la gloire : car il ne voulait pas se décharger des devoirs imposés à son zèle ; il lui suffisait d’affermir sa maison contre les injustes ressentiments d’Agrippine. Encore n’avait-il besoin d’appui que pour ses enfants ; car, pour lui, des jours dont il aurait rempli la mesure au service d’un tel prince seraient assez longs. »
\subsection[{Refus de Tibère}]{Refus de Tibère}
\noindent \labelchar{XL.} Tibère, après avoir loué dans sa réponse l’attachement de Séjan, et parlé avec modestie de ses propres bienfaits, demanda du temps comme pour en délibérer, et ajouta « qu’il n’en était pas des princes comme des autres hommes : ceux-ci n’avaient à consulter que leur intérêt ; mais les princes devaient surtout considérer l’opinion. Ainsi, laissant à part des réponses qui se présentaient d’elles-mêmes, il ne lui dirait pas que c’était à Livie à décider si, après Drusus, elle voulait avoir un époux ou rester dans le veuvage ; que d’ailleurs elle avait une mère, une aïeule, ses conseils naturels. Il parlerait avec plus de franchise. Et d’abord, les haines d’Agrippine ne deviendraient-elles pas plus ardentes, si le mariage de Livie partageait la maison des Césars comme en deux factions ? Déjà la rivalité de ces femmes éclatait par des chocs dont ses petits-fils ressentaient la secousse ; que serait-ce si un tel mariage venait animer le combat ? Car tu te trompes, Séjan, si tu penses que cette union te laisserait dans le rang où tu es, et que la veuve de Caïus César et de Drusus consentirait à vieillir femme d’un chevalier. Quand je le permettrais, ceux qui ont vu son frère, son père, tous nos aïeux, au faîte des honneurs, voudraient-ils le souffrir ? Sans doute, tu n’ambitionnes pas un plus haut rang que le tien. Mais ces magistrats, ces grands, qui, malgré toi, pénètrent chez toi et te consultent sur toutes les affaires, trouvent que tu es élevé depuis longtemps au-dessus de l’ordre équestre, et que les amitiés de mon père n’égalèrent jamais la faveur dont tu jouis. Ils ne s’en cachent pas ; et, par jalousie contre toi, ils m’accusent moi-même. Auguste eut, dit-on, la pensée de donner sa fille à un chevalier romain. Est-il étonnant qu’un prince occupé de soins infinis, et qui savait à quel comble de grandeur allait monter celui qu’une telle alliance aurait mis hors de pair, ait nommé dans ses entretiens C. Proculéius, et d’autres citoyens connus par la tranquillité de leur vie et un éloignement absolu des affaires publiques ? Mais si nous sommes touchés de l’incertitude d’Auguste, combien le serons-nous plus de son choix, qui fut pour Agrippa d’abord, ensuite pour moi ? Voilà ce que mon amitié n’a pas dû te cacher. Au reste, je ne m’opposerai ni à tes projets ni à ceux de Livie. Quant aux secrets desseins que j’ai formés sur toi, et aux nouveaux liens dont je veux t’unir plus étroitement à ma personne, je m’abstiendrai d’en parler en ce moment. Je me borne à te dire qu’il n’est rien de si élevé où tes vertus et tes sentiments pour moi ne te donnent droit de prétendre. Quand il sera temps, soit au sénat, soit devant le peuple, je ne m’en tairai pas. »
\subsection[{Séjan demande à Tibère de quitter Rome}]{Séjan demande à Tibère de quitter Rome}
\noindent \labelchar{XLI.} Séjan écrivit encore, mais sans parler de mariage. Ses craintes portaient plus loin, et il cherchait à conjurer les soupçons de la défiance, la malignité du vulgaire, les menaces de l’envie. Éloigner de sa maison la cour assidue qui la remplissait, c’eût été affaiblir sa puissance ; continuer de la recevoir, c’était donner des armes à ses accusateurs : il résolut d’engager Tibère à vivre loin de Rome, dans quelque riant asile. Il voyait à cela beaucoup d’avantages. Les avenues du palais s’ouvriraient que par lui ; les lettres même, portées par des soldats, seraient en grande partie à sa discrétion ; le prince, déjà sur le déclin de l’âge, et amolli par la retraite, lui abandonnerait plus facilement les rênes de l’empire ; enfin l’envie serait moins acharnée, quand elle ne verrait plus autour de lui cette foule d’adorateurs ; et, ce qu’il perdrait en représentation, il le gagnerait en pouvoir. Il s’élève donc insensiblement contre les embarras de la ville, et ce concours de peuple, ces flots de courtisans, dont on y est assiégé, exaltant les douceurs du repos et de la solitude, où, à l’abri des ennuis et des mécontentements, on peut traiter à loisir les plus grandes affaires.\par
\labelchar{XLII.} Tibère balançait. Le procès de Votiénus Montanus, qu’on instruisit alors, acheva de le décider à éviter les assemblées du sénat, où souvent des vérités dures retentissaient à son oreille. Montanus, homme célèbre par son esprit, était accusé d’avoir tenu contre l’empereur des discours injurieux. Le témoin Émilius, militaire, pour mieux appuyer sa déposition, répéta mot pour mot ces discours. En vain on cherchait à étouffer sa voix ; il en insistait avec plus de force, et Tibère entendit les malédictions dont on le chargeait secrètement. Il en fut si troublé, qu’il voulait, s’écriait-il, se justifier sur-le-champ ou par une instruction formelle, et que les prières de ses voisins, les adulations de tous, eurent peine à calmer son esprit. Montanus fut puni d’après la loi de majesté. Tibère, voyant qu’on lui reprochait sa rigueur envers les accusés, s’y attacha plus opiniâtrement. Aquilia était poursuivie comme coupable d’adultère avec Varius Ligur, et le consul désigné Lentulus la condamnait aux peines de la loi Julia \footnote{Loi contre l’adultère, portée par Auguste, l’an de Rome 737.} : le prince la punit de l’exil. Apidius Mérula n’ayant pas juré sur les actes d’Auguste, il le raya du tableau des sénateurs.
\subsection[{Dispute entre Lacédémone et Messène pour un temple}]{Dispute entre Lacédémone et Messène pour un temple}
\noindent \labelchar{XLIII.} On entendit ensuite les députés de Lacédémone et de Messène, au sujet du temple de Diane Limnatide \footnote{Il y avait, sur les confins de la Laconie et de la Messénie, au bourg de Limnae (en grec = les marais), un temple de Diane, où les deux pays offraient en commun des sacrifices. Telle est l’origine du surnom de Limnatide donné à cette déesse.}, dont ces deux villes se disputaient la propriété. Les Lacédémoniens affirmaient, sur la foi des historiens et des poètes, qu’il avait été bâti par leurs ancêtres et sur leur territoire ; qu’à la vérité Philippe de Macédoine le leur avait enlevé, dans une guerre, par la force des armes ; mais qu’une décision de Jules César et d’Antoine les en avait remis en possession. Les Messéniens, de leur côté, faisaient valoir un ancien partage du Péloponnèse entre les descendants d’Hercule. Selon eux, « le champ de Denthélie, où est ce temple, était échu à leur roi ; et d’antiques inscriptions, gravées sur la pierre et sur l’airain, attestaient encore ce fait. S’il fallait invoquer le témoignage de l’histoire et de la poésie, des monuments plus nombreux et plus authentiques déposaient pour eux. La décision de Philippe était un acte de sa justice et non de son pouvoir : le roi Antigone, le général romain Mummius, avaient prononcé comme lui ; les Milésiens, pris pour arbitres, et en dernier lieu Atidius Géminus, préteur d’Achaïe, avaient confirmé cet arrêt. » On jugea en faveur des Messéniens. Le temple de Vénus, sur le mont Éryx, était tombé de vétusté. Les Ségestains demandèrent qu’on le rebâtît, rappelant, sur son origine, ce que les traditions connues ont de flatteur pour le prince. Tibère se chargea volontiers de ce soin, comme d’un devoir de famille. On s’occupa ensuite d’une requête des Marseillais, et le précédent de P. Rutilius fut l’autorité qui décida la réponse. Banni par les lois, Rutilius avait reçu le droit de cité chez les Smyrnéens ; et c’est à son imitation que Vulcatius Moschus, exilé comme lui et devenu citoyen de Marseille, avait légué ses biens à sa nouvelle patrie.
\subsection[{Deux décès}]{Deux décès}
\noindent \labelchar{XLIV.} Dette année moururent deux hommes d’un haut rang, Cn. Lentulus et L. Domitius. Honoré du consulat et des décorations triomphales, méritées en combattant les Gétules, Lentulus eut encore une autre gloire : après avoir soutenu la pauvreté sans bassesse, il jouit sans orgueil d’une grande fortune légitimement acquise. Domitius brillait de l’éclat de son père, qui fut maître de la mer dans la guerre civile, jusqu’à ce qu’il se joignît au parti d’Antoine, puis enfin à celui de César. Son aïeul périt à la bataille de Pharsale, pour la cause des grands. Lui-même fut choisi pour être l’époux de la jeune Antonia, fille d’Octavie. À la tête d’une armée, il passa l’Elbe et pénétra dans la Germanie plus avant qu’aucun de ses prédécesseurs : succès qui lui valurent les ornements du triomphe. Dans ce temps mourut aussi L. Antonius, héritier d’un nom illustre, mais malheureux. Il était très jeune lorsque son père Julus Antonius fut puni de mort pour son commerce adultère avec Julie ; et Auguste envoya cet enfant, petit-fils de sa sœur, dans la cité de Marseille, où ses études servirent de prétexte à un véritable exil. Cependant on honora ses funérailles, et sa cendre fut portée, par ordre du sénat, dans le tombeau des Octaves.\par
\labelchar{XLV.} Sous les mêmes consuls, un crime atroce fut commis dans l’Espagne citérieure, par un paysan termestin. Le gouverneur de la province, L. Piso, voyageait avec toute la sécurité de la paix. L’assassin l’attaque brusquement sur la route et le tue d’un seul coup ; puis, s’enfuyant d’une course rapide, il gagne les bois, quitte son cheval, et s’enfonce dans des lieux coupés et sans chemins, ou l’on perdit sa trace. Mais il ne put échapper longtemps. On s’empara du cheval ; et, en le conduisant dans les villages voisins, on apprit quel en était le maître. Celui-ci fut découvert et mis à la question : mais, au lieu de nommer ses complices, il s’écria de toutes ses forces, dans la langue du pays, qu’on l’interrogeait vainement ; que ses compagnons pouvaient accourir et regarder ; que jamais la douleur, si forte qu’elle fût n’arracherait la vérité de sa bouche. Le lendemain, comme on le ramenait à la torture, il s’échappa tout à coup des mains de ses gardiens, et se jeta la tête contre une pierre avec tant de violence, qu’il mourut à l’instant. On croit que le meurtre de Pison fut une vengeance concertée par les Termestins, parce qu’il poursuivait les détenteurs de deniers publics avec une rigueur que des barbares ne savent pas supporter.
\subsection[{Révolte chez les Thraces}]{Révolte chez les Thraces}
\noindent \labelchar{XLVI.} Sous les consuls Lentulus Gétulicus et G. Galvisius, les ornements du triomphe furent décernés à Poppéus Sabinus, pour avoir réduit des nations de la Thrace, que la vie sauvage des montagnes entretenait dans une farouche indépendance. Outre le caractère de ce peuple, la révolte eut pour cause sa répugnance à souffrir les levées de soldats et à donner à nos armées l’élite de sa jeunesse. Accoutumée à n’obéir même à ses rois que par caprice, à ne leur envoyer de troupes qu’avec des officiers de son choix, à ne faire la guerre que sur ses frontières, cette nation crut, sur des bruits alors répandus, qu’on allait l’arracher à ses foyers, la mêler à d’autres peuples et la disperser dans des contrées lointaines. Toutefois, avant de prendre les armes, ils envoyèrent des députés pour rappeler leur fidélité, leur soumission, et déclarer qu’ils resteraient les mêmes tant que de nouvelles charges ne tenteraient point leur patience ; mais que, si on leur imposait l’esclavage comme à des vaincus, ils avaient du fer, des guerriers, et ce courage qui sait vouloir la liberté ou la mort. En même temps ils montraient, sur la cime des rochers, les forteresses où ils avaient réuni leurs parents et leurs femmes, et nous menaçaient d’une guerre rude, sanglante, hérissée d’obstacles.\par
\labelchar{XLVII.} Poppéus, pour avoir le temps de rassembler une armée, répondit par des paroles conciliantes. Lorsque Pomponius Labéo fut arrivé avec une des légions de Mésie, et le roi Rhémétalcès avec des secours que fournirent les Thraces restés fidèles, le général ajouta ce qu’il avait de forces, et marcha droit aux rebelles. Ils étaient déjà postés dans des gorges, au milieu des bois. D’autres, plus hardis, se montraient sur des collines découvertes. Poppéus y monte en bon ordre et les chasse sans peine. Les barbares perdirent peu de monde, ayant leur refuge tout près. Ensuite Poppéus se retranche dans ce lieu même, et occupe, avec un fort détachement, une montagne dont la croupe étroite, mais unie et continue, s’étendait jusqu’à une première forteresse gardée par de nombreux défenseurs, soldats ou multitude. Pendant que les plus ardents s’agitaient devant les remparts, avec leurs chants et leurs danses sauvages, il envoya contre eux l’élite de ses archers. Tant que ceux-ci combattirent de loin, ils firent beaucoup de mal sans en recevoir. S’étant avancés plus près, une brusque sortie les mit en désordre. Ils furent soutenus par une cohorte de Sicambres \footnote{Nation germanique que Tibère soumit l’an de Rome 746, et qu’il transporta sur la rive gauche du Rhin.}, que le général avait placée à quelque distance ; troupe intrépide, et non moins effrayante que les Thraces par ses chants guerriers et le fracas de ses armes.\par
\labelchar{XLVIII.} Ensuite Poppéus alla camper en face de l’ennemi, et laissa dans ses premiers retranchements les Thraces auxiliaires dont j’ai parlé. Il leur fut permis de ravager, de brûler de piller, pourvu que leurs courses finissent avec le jour, et que la nuit, renfermés dans le camp, ils y fissent bonne garde. Cet ordre fut observé d’abord. Bientôt, prenant le goût de la débauche et enrichis par le pillage, ils cessent de garder les postes. Ce ne sont plus que festins désordonnés, que soldats tombant d’ivresse et de sommeil. Les rebelles, instruits de leur négligence, se divisent en deux corps. L’un devait fondre sur ces pillards, l’autre assaillir le camp romain, non dans l’espérance de le prendre, mais afin que leurs cris, leurs traits, enfin le danger personnel, attirant toute l’attention des nôtres, leur dérobassent le bruit de l’autre combat. Ils choisirent la nuit, pour augmenter la frayeur. Ceux qui attaquèrent le camp des légions furent aisément repoussés. La soudaine irruption des autres jeta l’effroi parmi les Thraces auxiliaires, dont une partie dormait le long des palissades, tandis qu’un plus grand nombre errait dans la campagne. Ils furent massacrés avec d’autant plus de fureur, qu’on les regardait comme des transfuges et des traîtres, qui se battaient pour leur esclavage et celui de la patrie.\par
\labelchar{XLIX.} Le lendemain, Poppéus déploya son armée hors des retranchements, pour essayer si les barbares, animés par le succès de la nuit, hasarderaient une bataille. Voyant qu’ils ne quittaient point leur fort ou les hauteurs voisines, il en commença le siège en élevant de distance en distance de fortes redoutes, qu’il unit ensuite par un fossé et des lignes dont le circuit embrassait quatre mille pas. Peu à peu, pour ôter aux assiégés l’eau et le fourrage, il resserra son enceinte et les enferma plus étroitement. Quand on fut assez près, on construisit une terrasse d’où on lançait des pierres, des feux, des javelines. Mais rien ne fatiguait l’ennemi autant que la soif. Il ne restait qu’une seule fontaine pour une si grande multitude de combattants et de peuple. Les chevaux, les troupeaux, enfermés avec eux suivant la coutume des barbares, périssaient faute de nourriture. À côté de ces animaux gisaient les cadavres des hommes que les blessures ou la soif avaient tués. Tout était infecté par la corruption, l’odeur, le contact de la mort. À tant de calamités se joignit, pour dernier fléau, la discorde. Les uns parlaient de se rendre, les autres de mourir en se frappant mutuellement. Il s’en trouva qui, au lieu d’une mort sans vengeance, conseillèrent une sortie désespérée ; résolution noble aussi, quoique différente.\par
\labelchar{L.} Dinis, un des chefs, à qui son grand âge et sa longue expérience avaient appris à connaître la force et la clémence de Rome, soutenait l’avis de mettre bas les armes, comme le seul remède en de telles extrémités. Lui-même vint le premier, avec sa femme et ses enfants, se livrer au vainqueur. Il fut suivi de ceux que leur âge ou leur sexe condamne à la faiblesse, et de ceux qui aimaient la vie plus que la gloire. La jeunesse était partagée entre Tarse et Turésis. Tous deux voulaient périr avec la liberté ; mais Tarse s’écriait qu’il fallait hâter leur fin, et trancher d’un seul coup les craintes et les espérances. Il donna l’exemple en se plongeant son épée dans le sein, et sa mort ne manqua pas d’imitateurs. Turésis attendit la nuit avec sa troupe, non toutefois à l’insu de notre général. Aussi, tous les postes furent garnis de renforts nombreux. Avec la nuit s’était élevée une affreuse tempête, et l’ennemi, par des cris effroyables, suivis tout à coup d’un vaste silence, avait jeté l’incertitude parmi les assiégeants. Poppéus parcourt aussitôt toute sa ligne : il exhorte les soldats à ne pas ouvrir de chance aux barbares, en se laissant attirer par un bruit trompeur, ou surprendre par un calme perfide ; mais à rester immobiles à leurs postes, et à ne lancer leurs traits qu’à coup sûr.\par
\labelchar{LI.} Cependant les barbares, descendant par pelotons, jettent sur nos retranchements des pierres, des pieux durcis au feu, des tronçons d’arbres ; d’autres remplissent les fossés de fascines, de claies, de cadavres. Quelques-uns, munis de ponts et d’échelles, les appliquent au rempart, saisissent, arrachent les palissades, et luttent corps à corps avec ceux qui les défendent. Nos soldats les renversent à coups de traits, les poussent du bouclier ou leur envoient d’énormes javelines, et roulent sur eux des monceaux de pierres. Chez les nôtres, la victoire qu’ils tiennent dans les mains, et qui rendra, si elle échappe, la honte plus éclatante ; chez les barbares l’idée que ce combat est leur dernier espoir, les cris lamentables de leurs femmes et de leurs mères, qui les suivent dans la mêlée, échauffent les courages. La nuit accroît l’audace des uns, grossit aux autres le danger. Les coups volent au hasard, arrivent inattendus ; amis, ennemis, on ne distingue personne. L’écho de la montagne, dont nos soldats entendaient le retentissement derrière eux, acheva de tout confondre. Ils crurent les retranchements forcés et en abandonnèrent une partie. Cependant les ennemis ne les traversèrent qu’en petit nombre. Les plus braves furent tués ou blessés ; et, au point du jour, le reste fut poursuivi jusqu’au sommet du rocher, où ils furent, à la fin, contraints de se rendre. Les bourgades voisines se soumirent d’elles-mêmes. L’hiver rigoureux et prématuré du mont Hémus empêcha que les autres ne fussent réduites par la force ou par des sièges.
\subsection[{Agrippine}]{Agrippine}
\subsection[{Condamnation de sa cousine}]{Condamnation de sa cousine}
\noindent \labelchar{LII.} À Rome, de violentes secousses agitaient la maison de César. Pour préluder aux coups dont Agrippine devait un jour être atteinte, on attaque sa cousine Claudia. L’accusateur fut Domitius Afer. Cet homme sortait de la préture, avec peu de considération, et prêt à tout faire pour acquérir une prompte célébrité. Il reprochait à Claudia une vie déréglée, un commerce adultère avec Furnius des maléfices et des enchantements contre le prince. Agrippine, toujours emportée, et qu’enflammait encore le danger de sa parente, court chez Tibère, et le trouve occupé d’un sacrifice à Auguste. Tirant de cette vue le sujet d’une invective amère, elle s’écrie « qu’on ne devrait pas immoler des victimes au divin Auguste, quand on persécute ses enfants. Ce n’est pas dans de muettes images que réside l’esprit de ce dieu ; son image vivante, celle qui est formée de son sang immortel, comprend ses dangers ; elle se couvre de deuil, pendant qu’on encense les autres. On accuse Claudia : vain subterfuge ! Claudia périt pour avoir follement adressé son culte à la malheureuse Agrippine, sans songer que le même crime a perdu Sosie. » Ces plaintes arrachèrent à la dissimulation de Tibère un de ces mots si rares dans sa bouche. Il lui répliqua sévèrement, par un vers grec, que ses droits n’étaient pas lésés de ce qu’elle ne régnait point. Claudia et Furnius furent condamnés. Afer prit place parmi les hommes les plus éloquents : ce procès venait de révéler son génie ; et le prince avait mis le sceau à sa réputation en disant que le titre d’orateur lui appartenait de plein droit. Il continua d’accuser et de défendre ; carrière où il fit plus admirer son talent qu’estimer son caractère. Et ce talent même perdit beaucoup dans son dernier âge, où, malgré l’affaiblissement de son esprit, il ne put se résigner au silence.
\subsection[{Demande d’un nouveau mari}]{Demande d’un nouveau mari}
\noindent \labelchar{LIII.} Cependant Agrippine tomba malade et reçut la visite de César. Opiniâtre en sa colère, elle pleura longtemps sans rompre le silence. Enfin, exhalant son dépit avec ses prières, elle le conjure « d’avoir pitié de sa solitude ; de lui donner un époux : elle est jeune encore, et une femme vertueuse ne peut demander de consolations qu’à l’hymen ; Rome a des citoyens qui daigneront sans doute recevoir la veuve de Germanicus avec ses enfants. » Tibère sentit les conséquences politiques de cette demande. Toutefois, pour ne pas laisser éclater son mécontentement ou ses craintes, il sortit sans répondre, malgré les instances d’Agrippine. Ce fait n’est pas rapporté dans les annales du temps. Je le trouve dans les \emph{Mémoires} où Agrippine, sa fille et mère de Néron, a transmis à la postérité l’histoire de sa propre vie et les malheurs de sa famille.
\subsection[{Séjan met la zizanie}]{Séjan met la zizanie}
\noindent \labelchar{LIV.} Bientôt Séjan, abusant de sa douleur et de son imprévoyance pour lui porter un coup plus fatal, lui fit donner l’avis perfidement officieux qu’on voulait l’empoisonner ; qu’elle se défiât des festins de son beau-père. Agrippine ne savait point dissimuler. Un jour elle était à table, près de l’empereur, silencieuse, le visage immobile, ne touchant à aucun mets. Tibère s’en aperçut, soit par hasard, soit qu’il fût averti ; et, pour mieux pénétrer sa pensée, il loua des fruits qu’on venait de servir, et en présenta lui-même à sa bru. Les soupçons d’Agrippine s’en accrurent. Elle remit les fruits aux esclaves, sans en avoir goûté. Tibère cependant ne lui adressa pas une parole ; mais, se tournant vers sa mère, il dit que ce ne serait pas une chose étonnante qu’il fût un peu sévère pour une femme qui l’accusait d’empoisonnement. Aussitôt le bruit se répandit que la perte d’Agrippine était résolue, et que l’empereur, craignant les regards des Romains, cherchait la solitude pour consommer ce crime.
\subsection[{Qui, en Asie, aura son temple de Tibère ?}]{Qui, en Asie, aura son temple de Tibère ?}
\noindent \labelchar{LV.} Le prince, pour détourner ces rumeurs, allait au sénat plus assidûment que jamais. Il entendit pendant plusieurs jours les députés de l’Asie, qui disputaient entre eux où serait construit le temple de Tibère. Onze villes d’un rang inégal soutenaient leurs prétentions avec une égale ardeur. Toutes vantaient, à peu près dans les mêmes termes, l’ancienneté de leur origine, leur zèle pour le peuple romain pendant les guerres de Persée, d’Aristonicus et des autres rois. Tralles, Hypèpes, Laodicée et Magnésie \footnote{Hypèpe petite ville de Lydie, sur le penchant du Tmolus. Elle n’existe plus. – Tralles, ville considérable du même pays, dont on voit les ruines sur une hauteur, non loin du Méandre. – Laodicée, ville de Phrygie, dont les restes sont encore appelés Ladik.- Magnésie, au pied du mont Sipyle, à la gauche de l’Hermus, aujourd’hui Magnisa.}, furent d’abord exclues, comme d’un rang trop inférieur. Ilion même allégua vainement que Troie était le berceau de Rome : elle n’avait d’autre titre que son antiquité. On pencha un moment en faveur d’Halicarnasse \footnote{Capitale de la Carie. D’Anville croit qu’elle était au lieu où se trouve aujourd’hui un château nommé Bodroun.}. Pendant douze siècles aucun tremblement de terre n’avait ébranlé les demeures de ses habitants, et ils promettaient d’asseoir sur le roc vif les fondements de l’édifice. Pergame faisait valoir son temple d’Auguste : on jugea qu’il suffisait à sa gloire. Vouées tout entières au culte, l’une de Diane et l’autre d’Apollon, Ephèse et Milet parurent ne plus avoir de place pour un culte nouveau. C’est donc entre Sardes et Smyrne qu’il restait à délibérer. Les Sardiens lurent un décret par lequel les Étrusques les reconnaissaient pour frères. On y voyait qu’autrefois Tyrrhénus et Lydus, fils du roi Atys, se partagèrent la nation, devenue trop nombreuse. Lydus resta dans son ancienne patrie ; Tyrrhénus alla en fonder une nouvelle ; et ces deux chefs donnèrent leur nom à deux peuples, l’un en Italie, l’autre en Asie. Dans la suite, les Lydiens, ayant encore augmenté leur puissance, envoyèrent des colonies dans cette partie de la Grèce qui doit son nom à Pélops. Sardes produisait en outre des lettres de nos généraux et des traités faits avec nous pendant les guerres de Macédoine ; enfin elle n’oubliait pas la beauté de ses fleuves, la douceur de son climat, la richesse de ses campagnes.\par
\labelchar{LVI.} Smyrne, après avoir rappelé sa haute antiquité, soit qu’elle eût pour fondateur Tantale, fils de Jupiter, ou Thésée, également issu d’une race divine, ou l’une des Amazones, se hâta d’exposer des titres plus réels, les services qu’elle avait rendus au peuple romain en lui fournissant des vaisseaux, non seulement pour les guerres du dehors, mais même pour celles d’Italie. Elle ajouta « qu’elle avait, la première, érigé un temple à la ville de Rome, sous le consulat de Marcus Porcius \footnote{Sous le consulat de Caton l’Ancien, l’an de Rome 569, avant J. C. 195.}, dans un temps où le peuple romain, quoique déjà très puissant, n’était pas encore maître de l’univers ; car alors Carthage subsistait, et de grands monarques régnaient en Asie. » Elle prit à témoin le dictateur Sylla, « dont elle avait secouru l’armée, réduite à une affreuse détresse par la rigueur de l’hiver et le manque de vêtements. La nouvelle de nos besoins avait été apportée à Smyrne au moment où le peuple était assemblé, et aussitôt tous les citoyens s’étaient dépouillés de leurs habits pour les envoyer à nos légions. » Les sénateurs allèrent aux voix, et Smyrne obtint la préférence. Vibius Marsus proposa de donner à M. Lépidus, nommé gouverneur d’Asie, un lieutenant extraordinaire pour veiller à la construction du temple. Lépidus refusant par modestie de le choisir lui-même, on eut recours au sort, qui désigna l’ancien préteur Valérius Naso.
\subsection[{Tibère quitte Rome}]{Tibère quitte Rome}
\subsection[{Départ pour la Campanie}]{Départ pour la Campanie}
\noindent \labelchar{LVII.} Cependant Tibère, exécutant à la fin un projet médité longtemps et tant de fois différé, partit pour la Campanie, sous prétexte de dédier un temple de Jupiter à Capoue et un d’Auguste à Nole, mais avec la résolution de vivre loin de Rome. J’ai suivi la tradition la plus accréditée en attribuant son départ aux artifices de Séjan. Mais, comme il vécut encore six ans dans la solitude après le supplice de cet homme, peut-être, sans chercher ses motifs hors de lui-même, les trouverait-on dans le besoin d’un séjour qui cachât ce que ses actions affichaient, ses vices et sa cruauté. Plusieurs ont pensé que, dans sa vieillesse, son extérieur même lui causait quelque honte. Sa haute taille était grêle et courbée, son front chauve, son visage semé de tumeurs malignes, et souvent tout couvert d’emplâtres. D’ailleurs il s’était accoutumé, dans sa retraite de Rhodes, à fuir les réunions et à renfermer ses débauches. On dit encore que ce fut l’humeur impérieuse de sa mère qui le chassa de Rome. Il en coûtait à son orgueil de l’admettre au partage du pouvoir, et il n’osait l’en exclure, parce que ce pouvoir était un présent de ses mains. Car Auguste avait eu l’idée de laisser l’empire à Germanicus, petit-fils de sa sœur et objet des louanges universelles. Mais, vaincu par les prières de sa femme, il prit Tibère pour fils et donna Germanicus à Tibère. Voilà ce que reprochait Augusta, ce qu’elle redemandait.\par
\labelchar{LVIII.} Une suite peu nombreuse accompagna le prince : un seul sénateur, homme consulaire et profond jurisconsulte, Coccéius Nerva\footnote{L’aïeul de l’empereur Nerva.} ; Séjan, et un autre chevalier romain du premier rang, Curtius Atticus ; enfin quelques gens de lettres, la plupart Grecs, dont l’entretien amuserait ses loisirs. Les astrologues prétendirent que Tibère était sorti de Rome sous des astres qui ne lui promettaient pas de retour ; prédiction fatale à plusieurs, qui crurent sa fin prochaine et en semèrent le bruit. Ils étaient loin de prévoir la chance, incroyable en effet, que, de son plein gré, il se priverait onze ans de sa patrie. La suite fit voir combien dans cet art l’erreur est près de la science, et de quels nuages la vérité s’y montre enveloppée. L’annonce qu’il ne rentrerait plus dans la ville n’était pas vaine ; le reste trompa tous les calculs, puisque, habitant tour à tour quelque campagne ou quelque rivage près de Rome, souvent même établi au pied de ses murailles, il parvint à une extrême vieillesse.
\subsection[{Séjan sauve la vie de Tibère}]{Séjan sauve la vie de Tibère}
\noindent \labelchar{LIX.} Vers ce temps-là, un accident qui mit sa vie en danger accrédita ces frivoles conjectures et augmenta sa confiance dans l’amitié et l’intrépidité de Séjan. Ils soupaient dans une grotte naturelle, à Spélunca\footnote{Aujourd’hui Sperlonga, au royaume de Naples, près de Fondi sur le bord de la mer.}, entre la mer d’Amycle\footnote{Ville du Latium, entre Gaète et Terracine} et les montagnes de Fondi. L’entrée de la grotte, s’écroulant tout à coup, écrasa quelques esclaves. La peur saisit tous les autres, et les convives s’enfuient. Séjan, appuyé sur un genou, les bras tendus, les yeux attachés sur Tibère, oppose son corps aux masses qui tombaient. Les soldats accourus au secours le trouvèrent dans cette attitude. Il en devint plus puissant ; et, quelque pernicieux que fussent ses conseils, l’idée qu’il s’oubliait lui-même leur donnait de l’autorité. Il affectait à l’égard des enfants de Germanicus l’impartialité d’un juge, tandis que ses affidés les accusaient pour lui, et s’acharnaient surtout contre Néron, le plus proche héritier. Il est vrai que ce jeune homme, d’ailleurs sage et modeste, ne se souvenait pas toujours des ménagements qu’exigeait sa fortune. Ses clients et ses affranchis, impatients d’acquérir du pouvoir, l’excitaient à montrer une âme élevée et confiante : « C’était la volonté du peuple romain, le désir des armées ; alors tomberait l’audace de Séjan, qui maintenant bravait également la patience d’un vieillard et la faiblesse d’un jeune homme ».
\subsection[{Séjan met la brouille}]{Séjan met la brouille}
\noindent \labelchar{LX.} Animé par de tels discours, Néron, sans former aucune pensée coupable, laissait échapper quelquefois des paroles d’une hardiesse imprudente, que des espions placés près de lui ne manquaient pas de recueillir, d’envenimer et de dénoncer, sans qu’il lui fût permis de se défendre. D’un autre côté, les alarmes se multipliaient autour de lui sous toutes les formes. L’un évitait sa rencontre ; les autres, après lui avoir rendu le salut, se détournaient aussitôt ; plusieurs commençaient à lui parler et le quittaient brusquement, tandis que des amis de Séjan restaient pour insulter à son humiliation. Quant à Tibère, il le recevait avec un visage menaçant ou un sourire faux. Que le jeune homme parlât, qu’il se tût, on trouvait du crime à son silence, du crime à ses discours. La nuit même, ses dangers ne cessaient pas ; s’il veillait, s’il dormait, s’il poussait un soupir, sa femme en rendait compte à Livie sa mère, et celle-ci à Séjan. Cet homme gagna même Drusus, frère de sa victime, en le flattant du rang suprême, s’il en écartait un frère aîné déjà presque abattu. À la soif du pouvoir, et à ces haines si violentes entre frères, l’âme passionnée de Drusus joignait tous les emportements de la jalousie, à cause des préférences de leur mère Agrippine pour Néron. Au reste, Séjan, tout en caressant Drusus, ne laissait pas de préparer de loin les coups qui devaient le frapper aussi ; trop sûr que son caractère fougueux le livrerait sans défense aux embûches de la trahison.
\subsection[{Morts d’Asinius Agrippa et de Quintus Hatérius}]{Morts d’Asinius Agrippa et de Quintus Hatérius}
\noindent \labelchar{LXI.} À la fin de l’année moururent deux hommes distingués Asinius Agrippa, d’une famille moins ancienne qu’illustre, dont sa vie ne ternit pas la noblesse ; et Q. Hatérius, d’une maison sénatoriale, et, tant qu’il vécut, orateur célèbre. Les monuments qui nous restent de son talent ne répondent pas à sa renommée. C’est qu’il avait plus de chaleur que d’art. Aussi, tandis que la gloire des ouvrages que vivifient le travail et la méditation s’accroît d’âge en âge, l’éloquence harmonieuse et rapide de Q. Hatérius s’est éteinte avec lui.
\subsection[{Désastres}]{Désastres}
\subsection[{Ecroulement de l’amphithéâtre de Fidène}]{Ecroulement de l’amphithéâtre de Fidène}
\noindent \labelchar{LXII.} Sous le consulat de M. Licinius et de L. Calpurnius, un malheur imprévu égala seul les calamités des plus grandes guerres. Le même instant vit commencer et consommer le désastre. Un certain Atilius, affranchi d’origine, voulant donner à Fidène un spectacle de gladiateurs, avait construit son amphithéâtre sans en assurer les fondements, ni en consolider par des liens assez forts la vaste charpente ; aussi n’était-ce pas la surabondance des richesses ni l’ambition de se populariser dans sa ville, mais un sordide intérêt, qui lui avait suggéré cette entreprise. Là courut, avide de spectacles et sevrée de plaisirs sous un prince comme Tibère, une multitude de tout sexe, de tout âge, dont la proximité où Fidène est de Rome augmentait l’affluence. La catastrophe en fut plus terrible. L’édifice entièrement rempli, ses flancs se déchirent ; il s’écroule en dedans, se renverse en dehors, entraînant dans sa chute ou couvrant de ses ruines la foule innombrable qui regardait les jeux ou se pressait à l’entour. Heureux, dans un tel malheur, ceux qui dès le premier instant moururent écrasés, ceux-là du moins échappèrent aux souffrances. Les plus à plaindre furent ceux qui, tout mutilés, conservaient un reste de vie, et qui, le jour, avaient sous les yeux, la nuit, reconnaissaient à leurs cris lamentables leurs femmes et leurs enfants. Au premier bruit de l’événement, on accourt de toutes parts : l’un redemande en pleurant un frère ou un parent, l’autre son père ou sa mère. Ceux même dont les amis et les proches étaient absents pour d’autres causes ne sont pas sans alarmes. Tant qu’on ignora, quelles victimes le sort avait frappées, les craintes furent sans bornes, comme l’incertitude.\par
\labelchar{LXIII.} Cependant on écarte les débris, et chacun se précipite autour des morts, les embrasse, les couvre de baisers. Souvent trompés par l’âge et par la taille ; plusieurs se disputent des restes défigurés, que l’œil ne peut reconnaître. Cinquante mille personnes furent estropiées ou écrasées par ce funeste accident. Pour en prévenir le retour, un sénatus-consulte défendit de donner des spectacles de gladiateurs, à moins d’avoir quatre cent mille sesterces de revenu, et d’élever aucun amphithéâtre que la solidité du terrain n’eût été constatée. Atilius fut envoyé en exil. Au reste, dans cette calamité, les maisons des grands furent ouvertes ; on trouva partout des secours et des médecins : et pendant ces premiers jours l’aspect de Rome, tout morne qu’il était, rappela ces temps antiques, ou, après de grandes batailles, les citoyens prodiguaient aux blessés leurs soins et leurs richesses.
\subsection[{Incendie à Rome}]{Incendie à Rome}
\noindent \labelchar{LXIV.} Rome pleurait encore ce malheur, quand elle fut ravagée par un incendie dont la violence extraordinaire mit en cendres tout le mont Célius. Chacun disait que cette année était sinistre, et que Tibère avait formé sous de funestes auspices le projet de son absence ; car c’est la coutume du peuple d’imputer aux hommes les torts de la fortune. Mais l’empereur apaisa les murmures en distribuant des sommes proportionnées aux pertes. La reconnaissance publique eut dans le sénat d’illustres interprètes, et la renommée, organe du peuple, vanta cette munificence, qui, sans brigue, sans sollicitations de cour, appelait même des inconnus au partage de ses dons. On proposa que le mont Célius fût nommé désormais le mont Auguste, parce qu’au milieu de l’embrasement général la seule statue de Tibère, placée dans la maison du sénateur Junius, avait été respectée par le feu. « Ce même prodige était, disait-on, arrivé jadis pour Claudia Quinta ; et sa statue, échappée deux fois à la fureur des flammes, avait été consacrée par nos ancêtres dans le temple de la Mère des dieux. La race des Claudes était sainte et chérie du ciel ; et il fallait attacher de nouveaux respects au lieu où les Immortels avaient rendu au prince un si éclatant honneur. »\par
\labelchar{LXV.} Il n’est pas hors de propos de remarquer que ce mont fut d’abord appelé Querquétulanus, à cause du grand nombre de chênes dont il était couvert. Il fut ensuite nommé Célius, de Célès Vibenna, chef étrusque, qui, appelé au secours de Rome avec un corps de sa nation, fut établi en cet endroit par Tarquin l’Ancien ou quelque autre de nos rois : car les historiens diffèrent sur ce point ; du reste, tous conviennent que ces étrangers, fort nombreux, s’étendirent même au bas de la montagne et jusqu’auprès du Forum. C’est d’eux que la rue Toscane a tiré son nom.
\subsection[{À Rome}]{À Rome}
\subsection[{Accusations contre Quintilius Varus}]{Accusations contre Quintilius Varus}
\noindent \labelchar{LXVI.} Mais si l’humanité des grands et les largesses du prince avaient adouci des calamités fortuites, il n’était point de remède contre la rage des accusateurs, qui se déchaînait plus forte chaque jour et plus acharnée. Domitius Afer s’était saisi de Quintilius Varus, riche, parent de César, et dont il avait déjà fait condamner la mère Claudia Pulchra. Personne ne fut surpris que Domitius, longtemps pauvre, et qui avait dissipé follement un premier salaire, courût à de nouvelles bassesses. Mais on s’étonna de voir P. Dolabella s’associer à cette délation, et qu’un homme issu de nobles ancêtres, allié de Varus, dégradât sa noblesse et devînt le bourreau de son propre sang. Le sénat résista cependant, et fut d’avis qu’on attendît l’empereur, seul et passager refuge contre les maux les plus pressants.
\subsection[{Vers Capri}]{Vers Capri}
\noindent \labelchar{LXVII.} Après la dédicace des temples de Campanie, Tibère, non content d’avoir défendu par un édit qu’on vînt troubler son repos, et de s’être entouré de soldats qui repoussaient loin de lui le concours des habitants, prit en haine les villes municipales, les colonies, tous les lieux situés sur le continent, et se cacha dans l’île de Caprée, que sépare du promontoire de Surrentum un canal de trois mille pas. Je suis porté à croire que cette solitude lui plut, parce que l’île, sans aucun port offre à peine quelques lieux de refuge aux bâtiments légers, et qu’on ne peut y aborder sans être aperçu par les gardes. Une montagne, qui l’abrite des vents froids, y entretient pendant l’hiver une douce température ; et l’aspect du couchant, la libre étendue de la mer, y rafraîchissent délicieusement les étés. L’œil découvrait du côté de la terre le plus bel horizon, avant que l’éruption du Vésuve changeât la face des lieux. Les Grecs possédèrent, dit-on, ces rivages, et des Téléboens \footnote{Les Téléboens, suivant Strabon, étaient un peuple d’Acarnanie. Sous Auguste, l’île de Caprée appartenait à la cité de Naples. Auguste l’acquit de cette république, à laquelle il donna en échange d’autres petites îles. Caprée était donc une propriété particulière de l’empereur.} habitèrent Caprée. Tibère, maintenant, venait d’y bâtir douze maisons de plaisance, dont les noms et les constructions l’avaient envahie tout entière. C’est là que ce prince, si occupé jadis des affaires publiques, ensevelit ses dissolutions honteuses et son oisiveté malfaisante. Car il lui restait cette crédulité soupçonneuse, que Séjan avait nourrie dans Rome, et que chaque jour il inquiétait davantage. Déjà cet homme ne se bornait plus contre Néron et sa mère à d’obscures intrigues. Un soldat était sans cesse attaché à leurs pas. Messages, visites, démarches publiques ou secrètes, il inscrivait tout comme dans des annales. Des gens apostés leur conseillaient en même temps de se réfugier auprès des années de Germanie, ou de courir, au moment où la foule se presse dans le Forum, embrasser la statue d’Auguste, et implorer la protection du sénat et du peuple. Ils repoussaient en vain de tels projets ; on les accusait de les avoir formés.
\subsection[{Accusations contre Titius Sabinus}]{Accusations contre Titius Sabinus}
\noindent \labelchar{LXVIII.} L’année des consuls Junius Silanus et Silius Nerva s’ouvrit sous d’affreux auspices : on la commença par traîner au cachot fatal un chevalier romain du premier rang, Titius Sabinus, coupable d’attachement à Germanicus. Il n’avait cessé d’honorer sa veuve et ses fils, les visitant dans leur maison, les accompagnant en public, resté seul après tant de clients, et, à ce titre, loué des bons, odieux aux pervers. Latinius Latiaris, Porcius Cato, Pétilius Rufus et M. Opsius, anciens préteurs, se liguent pour le perdre. Ils voulaient le consulat, auquel on n’arrivait que par Séjan, et l’on n’achetait l’appui de Séjan que par le crime. Il fut convenu entre eux que Latiaris, qui avait quelques relations avec Sabinus, tendrait le piège, que les autres seraient témoins, et qu’ensuite on intenterait l’accusation. Latiaris commence par des propos vagues et indifférents. Bientôt louant la constance de Sabinus, il le félicite de ce qu’ami d’une maison florissante, il ne l’a pas, comme les autres, abandonnée dans ses revers. En même temps, il parlait honorablement de Germanicus et déplorait le sort d’Agrippine. Les malheureux s’attendrissent facilement : Sabinus versa des larmes, se plaignit à son tour. Alors Latiaris attaque plus hardiment Séjan, sa cruauté, son orgueil, son ambition. Tibère même n’est pas épargné dans ses invectives. Ces entretiens, comme des confidences séditieuses, formèrent entre eux l’apparence d’une étroite amitié. Bientôt Sabinus fut le premier à chercher Latiaris, à le visiter, à lui confier ses douleurs comme à l’ami le plus sûr.\par
\labelchar{LXIX.} Les hommes que j’ai nommés délibérèrent sur le moyen de faire entendre ses discours par plus d’un témoin. Il fallait que le lieu du rendez-vous parût solitaire. S’ils se tenaient derrière la porte, ils avaient à craindre quelque regard, le bruit, un soupçon que le hasard ferait naître. L’espace qui, sépare le toit du plafond fut l’ignoble théâtre d’une ruse détestable. C’est là que se cachèrent trois sénateurs, l’oreille attachée aux trous et aux fentes. Cependant Latiaris, ayant trouvé Sabinus dans la rue, feint d’avoir des secrets tout nouveaux à lui apprendre, l’entraîne chez lui, le conduit dans sa chambre. Là, le passé et le présent lui fournissent une abondante matière, qu’il grossit des terreurs de l’avenir. Sabinus s’abandonne aux mêmes plaintes, et plus longtemps encore ; car la douleur, une fois qu’elle s’exhale, ne sait plus s’arrêter. L’accusation est dressée à l’instant. Les traîtres écrivent à César, et, avec le détail de l’intrigue, ils lui racontent leur propre déshonneur. Jamais plus de consternation et d’alarmes ne régnèrent dans Rome. On tremble devant ses plus proches parents ; on n’ose ni s’aborder ni se parler ; connue, inconnue, toute oreille est suspecte. Même les choses muettes et inanimées inspirent de la défiance : on promène sur les murs et les lambris des regards inquiets.\par
\labelchar{LXX.} Le jour des calendes de janvier, Tibère adressa un message au sénat pour le renouvellement de l’année. Après les vœux ordinaires, il en vint à Sabinus, qu’il accusait d’avoir corrompu quelques-uns de ses affranchis pour attenter à ses jours. Il demandait vengeance en termes non équivoques, et cette vengeance fut prononcée à l’instant. Sabinus condamné, on le traînait à la mort, la gorge serrée, la voix étouffée avec ses vêtements ; et, en cet état, ramassant toutes ses forces : « C’était donc ainsi, s’écriait-il, que l’on commençait l’année ! Voilà quelles victimes tombaient en l’honneur de Séjan ! » Partout où il portait ses regards, où arrivaient ses paroles, ce n’était plus que fuite et solitude ; les rues, les places étaient abandonnées. Quelques-uns pourtant revenaient sur leurs pas et se montraient de nouveau, épouvantés de leur propre frayeur. On se demandait « quel jour vaqueraient les supplices, si, parmi les sacrifices et les vœux, quand l’usage défendait jusqu’aux paroles profanes, on voyait mettre les chaînes, serrer le cordon fatal ? Non, ce n’était pas sans dessein que Tibère avait affronté l’odieux d’un tel exemple. Sa cruauté réfléchie avait voulu que les Romains s’attendissent à voir désormais les nouveaux magistrats ouvrur le cachot aussi bien que les temples. » Le prince écrivit bientôt au sénat pour le remercier d’avoir fait justice d’un ennemi de la république. Il ajouta que sa vie était pleine d’alarmes, qu’il redoutait d’autres complots. Il ne nommait personne ; mais on ne douta pas qu’il n’eût en vue Agrippine et Néron.\par
\labelchar{LXXI.} Si mon plan ne m’obligeait à suivre l’ordre des années, je céderais à l’impatience d’anticiper sur les événements, et de raconter ici comment finirent Latiaris, Opsius, et les autres artisans de cette trame exécrable, non seulement quand l’empire fut aux mains de Caïus, mais déjà même du vivant de Tibère. Car s’il ne voulait pas voir briser par d’autres les instruments de ses crimes, il s’en lassa plus d’une fois ; et, quand des ministres nouveaux s’offrirent pour les mêmes services, il sacrifia ceux dont l’ancienneté lui pesait. Mais je raconterai ces châtiments et ceux des autres coupables, quand le temps sera venu. Asinius Gallus, dont les enfants avaient Agrippine pour tante maternelle, fut d’avis qu’on priât l’empereur d’avouer le sujet de ses craintes et de permettre au sénat de l’en délivrer. Parmi ce que Tibère croyait ses vertus, il n’estimait rien à l’égal de la dissimulation ; aussi fut-il offensé de voir qu’on révélât ce qu’il cachait. Séjan le calma, non par intérêt pour Gallus, mais pour laisser mûrir la vengeance du prince. Il savait que Tibère n’éclatait qu’après de longues réflexions, mais qu’alors des actes cruels suivaient de près les paroles menaçantes. Dans le même temps mourut Julie, petite-fille d’Auguste. Convaincue d’adultère et condamnée pour ce crime, son aïeul l’avait jetée dans l’île de Trimète \footnote{Le nom actuel est Trémiti. C’est une des îles que les anciens appelaient \emph{Diomedeae insulae}. Elles sont dans la mer Adriatique, non loin des côtes de la Capitanate, au royaume de Naples.}, non loin des côtes d’Apulie. Elle y endura vingt ans les rigueurs de l’exil, subsistant des libéralités d’Augusta, qui, après avoir renversé par de sourdes intrigues la maison de son époux, étalait pour ses malheureux débris une fastueuse pitié.
\subsection[{Révolte des Frisons}]{Révolte des Frisons}
\noindent \labelchar{LXXII.} La même année, la paix fut troublée chez les Frisons, au-delà du Rhin, plutôt par notre avarice que par l’indocilité de ce peuple. La nation était pauvre, et Drusus ne lui avait imposé d’autre tribut qu’un certain nombre de cuirs de bœufs pour l’usage de nos troupes. Personne ne les avait inquiétés sur la grandeur et la force de ces cuirs, jusqu’au primipilaire Olennius, qui, chargé du commandement de la Frise, choisit des peaux d’aurochs pour modèle de celles qu’on recevrait. Cette condition, dure partout ailleurs, était impraticable en Germanie, où les animaux domestiques sont petits, tandis que les forêts en nourrissent d’énormes. Ils furent réduits à livrer d’abord les bœufs mêmes, ensuite leurs champs, enfin à donner comme esclaves leurs enfants et leurs femmes. De là l’indignation, les plaintes, et la guerre, dernier remède à des maux dont on n’obtenait point le soulagement. Ils saisissent les soldats qui levaient le tribut, et les mettent en croix. Olennius dut son salut à la fuite. Il trouva un asile dans le château de Flève, d’où un corps assez nombreux de Romains, et d’alliés observait les côtes de l’Océan.\par
\labelchar{LXXIII.} À cette nouvelle, L. Apronius, propréteur de la basse Germanie, fait venir de la province supérieure des détachements des légions et l’élite de l’infanterie et de la cavalerie auxiliaire. Avec ces troupes réunies aux siennes, il s’embarque sur le Rhin et descend chez les Frisons. Les rebelles avaient déjà levé le siège du château pour courir à la défense de leurs foyers. Des lagunes arrêtaient la marche d’Apronius ; il y construisit des chaussées et des ponts, pour assurer le passage du gros de l’armée. Pendant ce temps ayant trouvé un gué, il détache une aile de Canninéfates\footnote{Les Canninéfates habitaient la partie occidentale de l’île des Bataves.}, et ce qu’il avait sous ses drapeaux d’infanterie germaine, avec ordre de tourner l’ennemi. Celui-ci, déjà rangé en bataille, repousse les escadrons alliés et la cavalerie légionnaire envoyée pour les soutenir. Alors on fait partir trois cohortes légères, ensuite deux, et quelque temps après toute la cavalerie auxiliaire, forces suffisantes, si elles eussent donné toutes ensemble ; arrivant par intervalles, non seulement elles ne rendirent point le courage à ceux qui pliaient, mais la terreur et la fuite des autres les entraînèrent elles-mêmes. Le général donne à Céthégus Labéo, lieutenant de la cinquième légion, ce qui lui restait de troupes alliées. Ce nouveau renfort pliait aussi, et Céthégus, placé dans une position critique, dépêchait courrier sur courrier, pour implorer le secours des légions. Elles s’élancent, la cinquième en tête, et, après un combat opiniâtre, elles repoussent l’ennemi et ramènent les cohortes et la cavalerie chargées de blessures. Le général romain ne songea point à la vengeance et n’ensevelit pas les morts, quoiqu’on eût perdu beaucoup de tribuns, de préfets, et les premiers centurions. On sut bientôt par les transfuges que neuf cents Romains avaient péri auprès du bois de Baduhenne, après avoir prolongé le combat jusqu’au lendemain, et que quatre cents autres, voulant se défendre dans une maison dont le maître, nommé Cruptorix, avait servi dans nos armées, avaient craint d’être trahis, et s’étaient mutuellement donné la mort.
\subsection[{Tibère se montre en Campanie}]{Tibère se montre en Campanie}
\noindent \labelchar{LXXIV.} Depuis ce temps le nom des Frisons fut célèbre parmi les Germains. Tibère dissimula nos pertes, pour ne pas confier à quelqu’un la conduite d’une guerre. Quant au sénat, il s’inquiétait peu si le nom romain était déshonoré aux extrémités de l’empire. La peur des maux domestiques préoccupait les esprits, et l’on y cherchait un remède dans l’adulation. Ainsi l’on interrompit une délibération commencée, pour voter un autel à la Clémence, et un autre à l’Amitié, entouré des statues de Tibère et de Séjan. On implorait, par des sollicitations redoublées, la faveur de les voir. Toutefois ils ne vinrent ni à Rome ni dans le voisinage. Ils crurent faire assez de sortir de leur île et de se montrer à l’entrée de la Campanie. Là coururent sénateurs, chevaliers, une grande partie du peuple, tous en peine d’arriver à Séjan, dont l’accès plus difficile ne s’ouvrait qu’à la brigue ou à la complicité. On s’accorde à dire que son arrogance fut accrue par le spectacle d’une servitude si honteusement étalée. À Rome, les yeux sont accoutumés au mouvement, et la grandeur de la ville ne permet pas de savoir quel intérêt dirige les pas des citoyens. Ici, c’est dans la plaine ou sur le rivage que cette multitude, étendue pêle-mêle, passe les jours et les nuits, pour subir, à la porte du favori, les dédains ou la protection de ses esclaves. Bientôt on leur ôte même ce droit ; et ils reviennent à Rome, les uns désespérés de ce qu’il ne les a pas jugés dignes d’une parole, d’un regard, les autres follement enivrés d’une amitié qui leur prépare de sinistres destins.
\subsection[{Mariage d’Agrippine la jeune}]{Mariage d’Agrippine la jeune}
\noindent \labelchar{LXXV.} Cependant Tibère maria la jeune Agrippine, fille de Germanicus, à Cn. Domitius. Après les avoir unis lui-même, il voulut que les noces fussent célébrées à Rome. Le prince avait choisi en Domitius le rejeton d’une antique famille et un parent des Césars. Ce Romain avait Octavie pour aïeule, et par elle il était petit-neveu d’Auguste.
\section[{Livre cinquième (29, 31)}]{Livre cinquième (29, 31)}\renewcommand{\leftmark}{Livre cinquième (29, 31)}

\subsection[{Mort de Livie, femme d’Auguste}]{Mort de Livie, femme d’Auguste}
\noindent \labelchar{I.} Sous les consuls Rubellius et Fufius, surnommés tous deux Géminus, mourut, dans un âge très avancé, Julia Augusta \footnote{C’est le nom que portait Livie, depuis que, par le testament d’Auguste, elle avait été adoptée dans la famille des Jules.}, héritière de la noblesse des Claudes, réunie par adoption à celle des Livius et des Jules. Elle fut mariée d’abord à Tibérius Néro \footnote{Lorsque César Octavien se fit céder Livie par son mari Tibérius Claudius Néro, elle était déjà mère de Tibère, et, de plus, enceinte de Drusus.} qui s’enfuit de sa patrie dans la guerre de Pérouse \footnote{Contre L. Antonius, frère du triumvir.} et y revint lorsque la paix fut faite entre Sext. Pompée et les triumvirs. Déjà mère et enceinte de nouveau, César, épris de sa beauté, l’enleva à son mari (on ne sait si ce fut malgré elle) ; et, dans son impatience, il en fit son épouse, sans attendre même qu’elle fût accouchée. Il n’eut pas d’enfants de ce dernier mariage ; mais l’union d’Agrippine et de Germanicus mêla sen sang à celui d’Auguste (1), et lui donna des arrière-petits-fils communs avec ce prince. Elle fut pure dans ses mœurs comme aux anciens jours, prévenante au-delà de ce qui semblait permis aux femmes d’autrefois, mère impérieuse, épouse complaisante, le caractère enfin le mieux assorti à la politique de son époux, à la dissimulation de son fils. Ses funérailles furent modestes, son testament longtemps négligé. Elle fut louée à la tribune par Caïus César, son arrière-petit-fils, qui depuis parvint à l’empire.\par
4. Germanicus était petit-fils de Livie, et Agrippine, petite-fille d’Auguste : les enfants qui naquirent d’eux descendaient donc au même degré d’Auguste et de Livie.\par
\labelchar{II.} Tibère, qui n’avait point interrompu le cours de ses plaisirs pour rendre à sa mère les derniers devoirs, s’en excusa, par lettre, sur la grandeur des affaires ; et, parmi les honneurs dont le sénat s’était montré libéral pour la mémoire d’Augusta, il retrancha les uns, comme par modestie, reçut un très petit nombre des autres, ajoutant qu’on s’abstînt de décerner l’apothéose ; que telle était la volonté de sa mère. Il s’éleva même dans un endroit de sa lettre contre ces amitiés qu’on lie avec les femmes ; censure indirecte qui s’adressait au consul Fufius, dont la fortune était l’ouvrage d’Augusta. Fufius était doué des agréments qui attirent ce sexe ; du reste, diseur de bons mots, et se permettant sur Tibère de ces plaisanteries mordantes dont les hommes puissants conservent un long souvenir.
\subsection[{Tibère attaque}]{Tibère attaque}
\noindent \labelchar{III.} Depuis ce moment, la domination devint emportée et violente. Du vivant d’Augusta, il restait encore un refuge : le prince gardait à sa mère un respect d’habitude, et Séjan n’osait opposer son crédit à l’autorité maternelle. Délivrés de ce frein, ils s’abandonnèrent à leur rage. Une lettre fut adressée au sénat contre Agrippine et Néron. On crut qu’envoyée depuis longtemps elle avait été arrêtée par Augusta ; car elle fut lue peu de jours après sa mort. Elle contenait des expressions d’une amertume étudiée. Au reste, il n’y était question ni de révolte, ni de complots. Tibère imputait à son petit-fils des amours infâmes et l’oubli de sa propre pudeur. Quant à sa bru, n’osant même calomnier ses mœurs, il lui reprochait un air hautain et une âme rebelle. La peur et le silence régnaient dans le sénat. Enfin, quelques-uns de ces hommes qui n’attendent rien des moyens honnêtes (et l’ambition particulière sait tourner à son profit les malheurs publics) demandèrent qu’on délibérât. Déjà Messalinus Cotta, plus empressé que les autres, proposait un avis cruel : mais le reste des grands tremblait, et surtout les magistrats ; car Tibère, malgré la violence de son invective, avait laissé sa volonté douteuse.
\subsection[{Attaques contre Séjan}]{Attaques contre Séjan}
\noindent \labelchar{IV.} Un sénateur, nommé Junius Rusticus, était chargé par le prince de tenir le journal des actes du sénat, et on le croyait initié aux pensées de Tibère. Cet homme, entraîné sans doute par un mouvement involontaire (car il n’avait donné jusqu’alors aucune preuve de courage), ou par une fausse politique qui, l’aveuglant sur un danger présent, l’effrayait d’un avenir incertain, se joint aux indécis, et engage les consuls à ne pas commencer la délibération. Il représente qu’un instant peut tout changer, et que, par respect pour le nom de Germanicus, il faut laisser au vieux prince le temps de se repentir. Cependant le peuple, portant les images d’Agrippine et de Néron, entoure le sénat, et, au milieu de ses acclamations et de ses vœux pour Tibère, il ne cesse de crier que la lettre est fausse et que c’est contre la volonté du prince qu’on trame la perte de sa maison. Aucune résolution cruelle ne fut donc prise ce jour-là. On fit même circuler, sous le nom de quelques consulaires, de prétendues opinions prononcées contre Séjan : satires où des auteurs inconnus exerçaient sans contrainte la malignité de leur esprit. La colère du favori en devint plus violente, et ses calomnies eurent un prétexte de plus : « Le sénat, selon lui, méprisait les douleurs du prince. Le peuple était en pleine révolte ; déjà on entendait, on lisait les harangues et les sénatus-consultes d’un nouveau gouvernement. Que leur restait-il à faire, sinon de tirer l’épée, et de choisir pour chefs et pour empereurs ceux dont les images leur servaient d’étendards ? »\par
\labelchar{V.} Tibère renouvela donc ses invectives contre son petit-fils et sa bru. II blâma le peuple par un édit, et se plaignit au sénat que les conseils perfides d’un seul homme eussent attiré un affront public à la majesté impériale. Il demanda cependant que tout fût réservé à sa décision. Le sénat ne balança plus, non pas à ordonner les dernières rigueurs (on l’avait défendu), mais à déclarer que prêt à venger l’empereur, il était retenu par sa volonté suprême \footnote{Ici commence une lacune qui embrasse le reste de l’année courante la suivante tout entière, et au moins dix mois de la troisième}…
\subsection[{Ce qui se passe après la mort de Séjan}]{Ce qui se passe après la mort de Séjan}
\noindent \labelchar{VI.} On entendit à ce sujet \footnote{Le golfe Corinthique, dans la mer Ionienne.} quarante-quatre discours, dont quelques-uns étaient dictés par la crainte, un plus grand nombre par l’habitude de flatter… « J’ai pensé que ce serait attirer la honte sur moi ou l’envie sur Séjan…… La fortune est changée, et celui qui avait choisi cet homme pour collègue et pour gendre \footnote{Aujourd’hui Preveza Vecchia, sur le golfe de l’Aria.} se pardonne son erreur ; les autres, après lui avoir prodigué un vil encens, lui déclarent une guerre impie… Est-on plus à plaindre, accusé à cause de l’amitié, que dénonciateur de son ami ? Je ne le déciderai pas. Du reste, je n’éprouverai ni la rigueur ni la clémence de personne. Libre et jouissant de ma propre estime, je préviendrai le danger. Et vous qui m’entendez, au lieu de donner des pleurs à ma mémoire, bénissez mes destins, et mettez-moi au nombre de ceux qui, par une fin honorable, ont échappé aux malheurs publics. »\par
1.  Probablement la conjuration de Séjan.\par
2.  Tibère avait fait Séjan consul avec lui, et l’avait trompé par l’espoir d’une alliance.\par
\labelchar{VII.} Ensuite il passa une partie du jour à s’entretenir avec ses amis, permettant à chacun de se retirer quand il voulait ou de rester auprès de lui. Ils l’entouraient encore en grand nombre et admiraient l’intrépidité de son visage, sans penser que l’heure suprême dût arriver sitôt, lorsqu’il se laissa tomber sur une épée qu’il avait cachée sous sa robe. Tibère ne flétrit sa mémoire d’aucune imputation, quoiqu’il eût cruellement outragé celle de Blésus.\par
\labelchar{VIII.} On instruisit ensuite le procès de P. Vitellius et de Pomponius Sécundus. Le premier était accusé d’avoir offert à la conjuration les clefs de l’épargne, dont il était préfet, ainsi que le trésor de la guerre. L’ancien préteur Considius reprochait au second l’amitié d’Élius Gallus, qui, après le supplice de Séjan, avait choisi les jardins de Pomponius comme l’asile le plus sûr où il pût se réfugier. Les accusés ne trouvèrent d’appui que dans le dévouement de leurs frères, qui se firent leurs cautions. L’affaire fut souvent remise, et Vitellius, également fatigué d’espérer et de craindre, demanda un canif comme s’il eût voulu écrire, et s’en piqua légèrement les veines. Quelque temps après, le chagrin termina sa vie. Pomponius, qui joignait une grande élégance de mœurs à un esprit distingué, supporta courageusement l’infortune et survécut à Tibère.\par
\labelchar{IX.} On résolut ensuite de sévir contre les derniers enfants de Séjan, quoique la colère du peuple commençât à s’amortir, et que les premiers supplices eussent calmé les esprits. On les porte à la prison : le fils prévoyait sa destinée ; la fille la soupçonnait si peu que souvent elle demanda quelle était sa faute, en quel lieu on la traînait, ajoutant qu’elle ne le ferait plus, qu’on pouvait la châtier comme on châtie les enfants. Les auteurs de ce temps rapportent que l’usage semblant défendre qu’une vierge subît la peine des criminels, le bourreau la viola auprès du lacet fatal. Puis il les étrangla l’un et l’autre, et les corps de deux enfants furent jetés aux Gémonies !
\subsection[{Un faux Drusus}]{Un faux Drusus}
\noindent \labelchar{X.} Vers le même temps, une alarme assez vive, mais qui dura peu, effraya l’Asie et l’Achaïe. Le bruit courut que Drusus, fils de Germanicus, avait été vu aux îles Cyclades, puis sur le continent. Il y parut en effet un jeune homme à peu près de son âge, que quelques affranchis de Tibère feignaient de reconnaître, et qu’ils accompagnaient par ruse. D’autres le suivaient de bonne foi, séduits par l’éclat de son nom et cet amour du merveilleux et de la nouveauté si naturel aux Grecs. Échappé de sa prison, il allait disait-on, rejoindre les armées de son père et s’emparer de l’Égypte et de la Syrie : et les inventeurs de cette fable y croyaient les premiers. Déjà il voyait la jeunesse accourir sur ses pas, et les villes lui adresser des hommages publics, succès qui l’enivraient de chimériques espérances lorsque la nouvelle de ce mouvement parvint à Poppéus Sabinus. Ce général, occupé alors en Macédoine, n’en veillait pas moins sur l’Achaïe. Vraies ou fausses, il voulut aller au-devant des prétentions de cet homme : il passe rapidement les golfes de Torone et de Thermes, l’île d’Eubée dans la mer Égée, le Pirée dans l’Attique, côtoie le rivage de Corinthe, traverse l’Isthme, et, se rembarquant sur une autre mer (1), il arrive à Nicopolis (2), colonie romaine, où il apprend que, pressé par d’adroites questions, l’imposteur s’était dit fils de M. Silanus, et que, abandonné de presque tous ses partisans, il était monté sur un vaisseau comme pour aller en Italie. Sabinus en instruisit Tibère. Du reste, je n’ai pu découvrir ni l’origine ni l’issue de cette entreprise.
\subsection[{Discorde entre les deux consuls}]{Discorde entre les deux consuls}
\noindent \labelchar{XI.} À la fin de l’année, la mésintelligence des consuls, longtemps accrue dans le silence, éclata. Trion, exercé aux combats de la parole et toujours prêt à défier les haines, avait indirectement accusé Régulus de négligence à poursuivre les complices de Séjan. Régulus, modéré quand on ne le provoquait pas, non seulement repoussa l’attaque ; mais il voulut poursuivre son collègue comme complice lui-même de la conjuration. En vain beaucoup de sénateurs les priaient de calmer des inimitiés qui tourneraient à leur ruine : ils continuèrent de se haïr et de se menacer jusqu’à la fin de leur magistrature.
\section[{Livre sixième (32, 37)}]{Livre sixième (32, 37)}\renewcommand{\leftmark}{Livre sixième (32, 37)}

\subsection[{Le vieux vicieux}]{Le vieux vicieux}
\noindent \labelchar{I.} Cn. Domitius et Camillus Scribonianus avaient pris possession du consulat, quand Tibère, franchissant le détroit qui sépare Caprée de Surrentum, se mit à côtoyer les rivages de Campanie, incertain s’il entrerait à Rome ou décidé peut-être à n’y pas entrer, et, par cette raison, faisant croire à sa prochaine arrivée. Il descendit même plusieurs fois dans les environs, et visita ses jardins situés près du Tibre. Ensuite, regagnant ses rochers, il cacha de nouveau, dans la solitude des mers, des crimes et des dissolutions dont il était honteux. L’ardeur de la débauche l’emportait à ce point, qu’à l’exemple des rois il souillait de ses caresses les jeunes hommes libres. Et ce n’étaient pas seulement les grâces et la beauté du corps qui allumaient ses désirs ; il aimait à outrager dans ceux-ci une enfance modeste, dans ceux-là les images de leurs ancêtres. Alors furent inventés les noms auparavant inconnus de \emph{sellarii}, de \emph{spintriae}, qui rappelaient des lieux obscènes ou de lubriques raffinements. Des esclaves affidés lui cherchaient, lui traînaient des victimes, récompensant la bonne volonté, effrayant la résistance ; et si un parent, si un père défendait sa famille, ils exerçaient sur elle la violence, le rapt, toutes les brutalités d’un vainqueur sur ses captifs.
\subsection[{Idée saugrenue de Togonius Gallus}]{Idée saugrenue de Togonius Gallus}
\noindent \labelchar{II.} A Rome, au commencement de cette année, comme si les crimes de Livie, punis depuis longtemps, eussent été récemment découverts, on proposait encore des arrêts flétrissants contre ses images et sa mémoire. On voulait aussi que les biens de Séjan fussent enlevés au trésor public et donnés à celui du prince, comme si le choix importait. Ces avis, les Scipions, les Silanus, les Cassius, qui ne faisaient guère que répéter les paroles l’un de l’autre, les exprimaient avec beaucoup de feu lorsque Togonius Gallus, en essayant d’associer son obscurité à de si grands noms, se couvrit de ridicule. Il priait Tibère de choisir des sénateurs, dont vingt, désignés par le sort et armés d’un glaive, veilleraient à sa sûreté toutes les fois qu’il viendrait au sénat. Togonius avait sans doute cru sincère une lettre où le prince demandait l’escorte d’un des consuls, pour venir sans péril de Caprée à Rome. Tibère, qui savait mêler le sérieux à la dérision, rendit grâces au sénat de sa bienveillance ; « mais qui exclure ? qui choisir ? Prendrait-on toujours les mêmes ou de nouveaux tour à tour ? des vieillards qui eussent passé par les charges ou des jeunes gens ? des hommes privés ou des magistrats ? D’ailleurs, quel spectacle que celui de sénateurs mettant l’épée à la main pour entrer au conseil ! Il estimait peu la vie, s’il fallait la défendre par les armes. » C’est ainsi qu’il combattit Togonius en termes très mesurés, et conseillant seulement de rejeter sa proposition.
\subsection[{Autre idée de Junius Gallio}]{Autre idée de Junius Gallio}
\noindent \labelchar{III.} Quant à Junius Gallio, qui avait proposé que les prétoriens vétérans eussent le droit de s’asseoir, au théâtre, sur les quatorze rangs de sièges destinés aux chevaliers, il lui fit une violente réprimande, lui demandant, comme s’il eût été devant lui, « ce qu’il avait de commun avec les soldats ; s’il était juste que ceux-ci reçussent les ordres de l’empereur et leurs récompenses d’un autre. Pensait-il donc avoir trouvé dans son génie quelque chose qui eût échappé à la prévoyance d’Auguste ? ou plutôt, complice de Séjan, ne cherchait-il pas une occasion de discorde et de trouble, en soufflant dans des esprits grossiers une ambition qui ruinerait la discipline ? " Voilà ce que valut à Gallion cette recherche de flatterie. Chassé sur-le-champ du sénat, ensuite de l’Italie, on trouva que son exil serait trop doux dans l’île célèbre et agréable de Lesbos, qu’il avait choisie pour retraite ; on l’en tira pour l’emprisonner à Rome, dans les maisons des magistrats \footnote{Les sénateurs et les citoyens de quelque distinction n’étaient point mis dans la prison publique. On les donnait en garde à un magistrat qui les enfermait chez lui.}. Par la même lettre, Tibère frappa l’ancien préteur Sextius Paconianus d’une accusation qui remplit de joie le sénat. C’était un homme audacieux, malfaisant, épiant les secrets de toutes les familles, et que Séjan avait choisi pour préparer la ruine de Caïus César. Au récit de cette intrigue, les haines amassées depuis longtemps éclatèrent. On condamnait Paconianus au dernier supplice, s’il n’eût promis une révélation.
\subsection[{Latiaris et Hatérius Agrippa}]{Latiaris et Hatérius Agrippa}
\noindent \labelchar{IV.} Lorsqu’il eut prononcé le nom de Latiaris, ce fut un agréable spectacle de voir aux prises un accusateur et un accusé également odieux. Latiaris, principal auteur du complot contre Sabinus, fut aussi le premier qui en porta la peine. Sur ces entrefaites, Hatérius Agrippa, s’attaquant aux consuls de l’année précédente, demanda « où en étaient leurs menaces d’accusation mutuelle, et d’où venait maintenant leur silence. Il concevait qu’une communauté de craintes et de remords les eût réconciliés ; mais le sénat devait-il taire ce qu’il avait entendu ? » Régulus répondit que le temps restait à sa vengeance, et qu’il la poursuivrait en présence de l’empereur. Trion soutint que ces rivalités entre collègues, et des imputations échappées à la colère, ne méritaient que l’oubli. Hatérius insistait, le consulaire Sanquinius Maximus pria le sénat de ne point aigrir les chagrins du prince, en lui cherchant de nouvelles amertumes ; que César saurait lui-même prescrire le remède. Ainsi fut sauvé Régulus, et fut différée la perte de Trion. Pour Hatérius, il en devint plus odieux. On s’indignait de voir qu’un homme énervé par le sommeil ou par des veilles dissolues, et protégé par son abrutissement contre toutes les cruautés du prince, méditât, au milieu de l’ivresse et des plaisirs honteux, la ruine des plus illustres Romains.
\subsection[{Messalinus Cotta}]{Messalinus Cotta}
\noindent \labelchar{V.} Messalinus Cotta, auteur des avis les plus sanguinaires et objet d’une haine invétérée, fut chargé, dès que l’occasion s’en offrit, d’accusations nombreuses. Il avait appelé Caïus César, \emph{Caïa}, comme pour lui reprocher des mœurs infâmes. Assistant à un banquet donné par les prêtres pour célébrer le jour natal d’Augusta, il avait traité ce repas de banquet funèbre. Un jour qu’il se plaignait de L. Arruntius et de M’. Lépidus, avec lesquels il avait une discussion d’intérêt : « Si le sénat est pour eux, avait-il ajouté, j’ai pour moi mon petit Tibère. » Et sur tous ces points les premiers de Rome confondaient ses dénégations. Pressé par leurs témoignages, il en appelle à César ; et bientôt arrive une lettre en forme de plaidoyer, où le prince, après avoir rappelé l’origine de son amitié avec Cotta, et les preuves nombreuses d’attachement qu’il avait reçues de lui, priait le sénat de ne pas tourner en crimes des paroles mal interprétées et quelques plaisanteries échappées dans la gaieté d’un repas.
\subsection[{Lettre de Tibère au sénat}]{Lettre de Tibère au sénat}
\noindent \labelchar{VI.} Le début de cette lettre parut remarquable. Tibère la commençait par ces mots : « Que vous écrirai-je, pères conscrits ? Comment vous écrirai-je ? Ou que dois-je en ce moment ne pas vous écrire ? Si je le sais, que les dieux et les déesses me tuent plus cruellement que je ne me sens périr tous les jours. » Tant ses forfaits et ses infamies étaient devenus pour lui-même un affreux supplice. Ce n’est pas en vain que le prince de la sagesse avait coutume d’affirmer que, si l’on ouvrait le cœur des tyrans, on le verrait déchiré de coups et de blessures, ouvrage de la cruauté, de la débauche, de l’injustice, qui font sur l’âme les mêmes plaies que fait sur le corps le fouet d’un bourreau. Ni le trône ni la solitude ne préservaient Tibère d’avouer les tourments de sa conscience et les châtiments par lesquels il expiait ses crimes.
\subsection[{Délation}]{Délation}
\noindent \labelchar{VII.} Libre de prononcer à son gré sur le sénateur Cécilianus, qui avait produit contre Cotta les charges les plus nombreuses, le sénat le punit de la même peine qu’Aruséius et Sanquinius, accusateurs d’Arruntius ; et c’est le plus grand honneur qu’ait jamais reçu Cotta, noble, il est vrai, mais ruiné par le luxe et décrié par ses bassesses, d’avoir paru digne d’une vengeance qui égalait ses vices aux vertus d’Arruntius. Q. Servéus et Minucius Thermus comparurent ensuite : Servéus, ancien préteur, et compagnon de Germanicus dans ses campagnes ; Minucius, de l’ordre équestre ; tous deux ayant joui, sans en abuser, de l’amitié de Séjan, ce qui excitait pour eux une pitié plus vive. Mais Tibère, après les avoir traités comme les principaux instruments du crime, enjoignait à C. Cestius, le père, de déclarer devant le sénat ce qu’il avait écrit au prince ; et Cestius se chargea de l’accusation. Ce fut le plus triste fléau de ces temps malheureux, que les premiers sénateurs descendissent même aux plus basses délations. On accusait en public ; plus encore en secret. Nulle distinction de parents ou d’étrangers, d’amis ou d’inconnus. Le fait le plus oublié comme le plus récent, une conversation indifférente au Forum ou dans un repas, tout devenait crime. C’était à qui dénoncerait le plus vite et ferait un coupable, quelques-uns pour leur sûreté, le plus grand nombre par imitation, et comme atteints d’une fièvre contagieuse. Minucius et Servéus, condamnés, se joignirent aux délateurs, et firent éprouver le même sort à Julius Africanus, né en Saintonge, dans les Gaules, et à Séius Quadratus, dont je n’ai pu savoir l’origine. Je n’ignore pas que la plupart des écrivains ont omis beaucoup d’accusations et de supplices, soit que leur esprit fatigué ne pût suffire au nombre ; soit que, rebutés de tant de scènes affligeantes, ils aient voulu épargner aux lecteurs le dégoût qu’eux-mêmes en avaient éprouvé. Pour moi, j’ai rencontré beaucoup de faits dignes d’être connus, bien que laissés par d’autres dans le silence et l’oubli.
\subsection[{M. Terentius se défend devant le sénat : il est fier d’avoir été l’ami de Séjan}]{M. Terentius se défend devant le sénat : il est fier d’avoir été l’ami de Séjan}
\noindent \labelchar{VIII.} Ainsi, lorsque chacun reniait sans pudeur l’amitié de Séjan, un chevalier romain, M. Térentius, accusé d’y avoir eu part, osa s’en faire honneur devant le sénat. « Pères conscrits, dit-il, peut-être conviendrait-il mieux à ma fortune de repousser l’accusation que de la reconnaître. Mais, quel que puisse être le prix de ma franchise, je l’avouerai, je fus l’ami de Séjan, j’aspirai à le devenir ; je fus joyeux d’y avoir réussi. Je l’avais vu commander avec son père les cohortes prétoriennes ; je le voyais remplir à la fois les fonctions civiles et militaires. Ses proches, ses alliés, étaient comblés d’honneurs ; son amitié était le titre le plus puissant à celle de César ; sa haine plongeait dans les alarmes et le désespoir quiconque l’avait encourue. Je ne prends personne pour exemple : je défendrai à mes seuls périls tous ceux qui, comme moi, furent innocents de ses derniers complots. Non, ce n’était pas à Séjan de Vulsinies que s’adressaient nos hommages ; c’était à la maison des Claudes et des Jules, dont une double alliance l’avait rendu membre \footnote{Allusion à l’union arrêtée anciennement entre la fille de Séjan et le fils de Claude, et à celle que Tibère avait promise à Séjan lui-même avec une femme de sa famille, sans doute avec sa bru Livie.} ; c’était à ton gendre, César, à ton collègue dans le consulat, au dépositaire de ton autorité. Ce n’est pas à nous, d’examiner qui tu places sur nos têtes, ni quels sont tes motifs. A toi les dieux ont donné la souveraine décision de toutes choses ; obéir est la seule gloire qui nous soit laissée. Or, nos yeux sont frappés de ce qu’ils ont en spectacle ; ils voient à qui tu dispenses les richesses, les honneurs, où se trouve la plus grande puissance de servir ou de nuire. Cette puissance, ces honneurs, on ne peut nier que Séjan ne les ait possédés. Vouloir deviner les secrètes pensées du prince et ses desseins cachés, est illicite, dangereux ; le succès d’ailleurs manquerait à nos recherches. Pères conscrits, ne considérez pas le dernier jour de Séjan ; pensez plutôt à seize ans de sa vie. A cause de lui, Satrius même et Pomponius obtinrent nos respects. Être connu de ses affranchis, des esclaves qui veillaient à sa porte, fut réputé un précieux avantage. Que conclure de ces réflexions ? qu’elles donnent également l’innocence à tous les amis de Séjan ? non, sans doute ; il faut faire une juste distinction : que les complots contre la république et les attentats à la vie du prince soient punis ; mais qu’une amitié qui a fini, César, en même temps que la tienne, nous soit pardonnée comme à toi. »
\subsection[{Lèse-majesté}]{Lèse-majesté}
\noindent \labelchar{IX.} La fermeté de ce discours, et la joie de trouver un homme dont la bouche proclamât ce qui était dans toutes les âmes, eurent tant de pouvoir, que ses accusateurs, dont on rappela en même temps les crimes passés, furent punis par la mort ou l’exil. Tibère écrivit ensuite contre Sext. Vestilius, ancien préteur, autrefois cher à Drusus son frère, et à ce titre admis par lui-même dans sa familiarité. La disgrâce de Vestilius fut causée par un écrit injurieux aux mœurs de Caïus, qu’il avait composé ou dont peut-être on l’accusa faussement. Banni pour ce crime de la table du prince, ce vieillard essaya sur lui-même un fer mal assuré, puis se referma les veines, écrivit une lettre suppliante, et, n’ayant reçu qu’une réponse dure, se les ouvrit de nouveau. Après lui furent poursuivis en masse, pour lèse-majesté, Annius Pollio, auquel on joignait son fils Vinicianus, Appius Silanus, Mamercus Scaurus, et Sabinus Calvisius, tous illustres par leur naissance, quelques-uns par l’éclat des premières dignités. Les sénateurs tremblèrent d’épouvante. Lequel d’entre eux ne tenait pas, soit par l’alliance, soit par l’amitié, à tant d’accusés d’un si haut rang ? Toutefois, l’un des dénonciateurs, Celsus, tribun d’une cohorte urbaine, en sauva deux, Appius et Calvisius. Tibère différa le procès de Pollio, de Vinicianus et de Scaurus, se réservant de l’instruire lui-même avec le sénat. Sa lettre contenait sur Scaurus quelques mots d’un sinistre augure.
\subsection[{L’hécatombe}]{L’hécatombe}
\noindent \labelchar{X.} Les femmes même n’étaient pas exemptes de danger. Ne pouvant leur imputer le dessein d’usurper l’empire, on accusait leurs larmes. La mère de Fufius Géminus, Vitia, d’un âge très avancé, fut tuée pour avoir pleuré la mort de son fils. Tels étaient les actes du sénat. Le prince, de son côté, faisait périr Vescularius Atticus et Julius Marinus, deux de ses plus anciens amis, qui l’avaient suivi à Rhodes et ne le quittaient point à Caprée. Vescularius avait prêté son entremise au complot contre Libon. Séjan s’était servi de Marinus pour perdre Curtius Atticus. Aussi vit-on avec joie la délation s’emparer contre eux de leurs propres exemples. Dans le même temps mourut le pontife L. Piso, et, ce qui était rare dans une si haute fortune, sa mort fut naturelle. Jamais la servilité n’eut en lui un organe volontaire, et, toutes les fois que la nécessité lui imposa une opinion, il l’adoucit par de sages tempéraments. J’ai dit qu’il était fils d’un père honoré de la censure. Sa carrière s’étendit jusqu’à l’âge de quatre-vingts ans. Il avait mérité en Thrace les ornements du triomphe ; mais il acquit sa principale gloire comme préfet de Rome, par les ménagements admirables avec lesquels il usa d’un pouvoir dont la perpétuité récente pesait à des esprits neufs pour l’obéissance.
\subsection[{Le praefectus urbis}]{Le praefectus urbis}
\noindent \labelchar{XI.} Autrefois, quand les rois, et après eux les magistrats, s’éloignaient de la ville, afin qu’elle ne restât point livrée à l’anarchie, un homme, choisi pour le temps de leur absence, était chargé de rendre la justice et de pourvoir aux besoins imprévus. Romulus donna, dit-on, cette magistrature passagère à Denter Romulius, Tullus Hostilius à Marcius Numa, et Tarquin le Superbe à Sp. Lucrétius. Dans la suite, le choix appartint aux consuls. On conserve une image de cette institution, en nommant encore aujourd’hui, pour les féries latines, un préfet chargé des fonctions consulaires. Auguste, pendant les guerres civiles, avait confié à Cilnius Mécénas, chevalier romain, l’administration de Rome et de toute l’Italie. Devenu maître de l’empire, et considérant la grandeur de la population, la lenteur des secours qu’on trouve dans les lois, il chargea un consulaire de contenir les esclaves, et cette partie du peuple dont l’esprit remuant et audacieux ne connaît de frein que la crainte. Messala Corvinus reçut le premier ce pouvoir, qui lui fut retiré bientôt comme au-dessus de ses forces. Statilius Taurus, quoique d’un âge avancé, en soutint dignement le poids. Après lui, Pison l’exerça vingt ans avec un succès qui ne se démentit jamais. Le sénat lui décerna des funérailles publiques.
\subsection[{Un nouveau livre sibyllin}]{Un nouveau livre sibyllin}
\noindent \labelchar{XII.} Le tribun du peuple Quintilianus soumit ensuite à la délibération du sénat un nouveau livre sibyllin, que le quindécemvir Caninius Gallus voulait faire admettre par un sénatus-consulte. Le décret, rendu au moyen du partage \footnote{Si les opinions étaient suffisamment formées, les partisans de la proposition passaient d’un côté de la salle, les adversaires se rangeaient de l’autre. Si la chose était encore douteuse, on demandait les voix individuellement, et chacun prononçait son avis en le motivant. Le premier mode s’appelait \emph{per discessionem} le second, \emph{per exquisitas sententias}, ou encore \emph{per relationem} 2. Ville célèbre d’Ionie, vis-à-vis de l’île de Chio, aujourd’hui Eréthri.}, fut blâmé par une lettre du prince. Tibère y faisait au tribun une légère réprimande, accusant sa jeunesse d’ignorer les anciens usages. Plus sévère pour Gallus, il s’étonnait qu’un homme vieilli dans la science religieuse eût accueilli l’ouvrage d’un auteur incertain, sans consulter son collège, sans le faire lire et juger, suivant la coutume, par les maîtres des rites, et l’eût proposé aux suffrages d’une assemblée presque déserte. Il rappelait en outre une ordonnance d’Auguste, qui, voyant de prétendus oracles publiés chaque jour sous un nom accrédité, fixa un terme pour les porter chez le préteur de la ville, et défendit que personne en pût garder entre ses mains. Un décret semblable avait été rendu chez nos ancêtres après l’incendie du Capitole, au temps de la guerre sociale. Alors on recueillit à Samos, à Ilium, à Érythrée \footnote{Ce fleuve doit être au sud-est de la mer Caspienne.}, en Afrique même, en Sicile, et dans les villes d’Italie, tous les livres sibyllins (soit qu’il ait existé une ou plusieurs Sibylles), et on chargea les prêtres de reconnaître, autant que des hommes pouvaient le faire, quels étaient les véritables. Celui de Gallus fut également soumis à l’examen des quindécemvirs.
\subsection[{Nouvelles exécutions}]{Nouvelles exécutions}
\noindent \labelchar{XIV.} A la fin de l’année périrent Géminius, Celsus et Pompéius, chevaliers romains, accusés d’avoir eu part à la conjuration. De grandes dépenses et une vie voluptueuse avaient lié Géminius avec Séjan, mais leur amitié n’eut jamais un objet sérieux. Celsus était tribun : il s’étrangla dans sa prison en tirant sur sa chaîne, qu’il trouva moyen d’allonger assez pour se la passer autour du cou. On donna des gardiens à Rubrius Fabatus, sous prétexte que, désespérant de Rome, il fuyait chez les Parthes pour y chercher la pitié. Il est vrai que, surpris vers le détroit de Sicile et ramené par un centurion, il ne put alléguer aucune raison plausible d’un voyage lointain. Cependant on lui laissa la vie, par oubli plutôt que par clémence.
\subsection[{Mariage des deux filles de Germanicus}]{Mariage des deux filles de Germanicus}
\noindent \labelchar{XV.} Sous les consuls Serv. Galba et L. Sylla, Tibère, longtemps incertain sur le choix des époux qu’il donnerait à ses petites-filles, et voyant que leur âge ne permettait plus de retard, se décida pour L. Cassius et M. Vinicius. D’origine municipale, et sorti de Calès, Vinicius était fils et petit-fils de consulaires, et toutefois d’une famille équestre ; du reste, esprit doué, élégant orateur. Cassius, originaire de Rome, était d’une maison plébéienne, mais ancienne et décorée ; et, quoique élevé par son père sous une austère discipline, il se recommandait plutôt par la facilité de ses mœurs que par l’énergie de son âme. Il reçut en mariage Drusille, et Vinicius Julie, toutes deux filles de Germanicus. Tibère en écrivit au sénat, avec quelques mots d’éloge pour les époux. Dans la même lettre, après avoir allégué sur son absence de vagues et frivoles excuses, il passait à de plus graves objets, les haines qu’il encourait pour la république, et demandait que le préfet Macron et un petit nombre de centurions et de tribuns pussent entrer avec lui toutes les fois qu’il irait au sénat. Un décret fut rendu dans les termes les plus favorables, sans fixation de l’espèce ni du nombre des gardes ; et Tibère, loin de venir jamais au conseil public, ne vint pas même dans Rome. Tournant autour de sa patrie, presque toujours par des routes écartées, il semblait à la fois la chercher et la fuir.
\subsection[{Lutte contre l’usure}]{Lutte contre l’usure}
\noindent \labelchar{XVI.} Cependant une légion d’accusateurs se déchaîna contre ceux qui s’enrichissaient par l’usure, au mépris d’une loi du dictateur César sur la proportion des créances et des possessions en Italie , loi depuis longtemps mise en oubli par l’intérêt particulier, auquel le bien public est toujours sacrifié. L’usure fut de tout temps le fléau de cette ville, et une cause sans cesse renaissante de discordes et de séditions. Aussi, même dans des siècles où les mœurs étaient moins corrompues, on s’occupa de la combattre. Les Douze Tables réduisirent d’abord à un pour cent l’intérêt, qui, auparavant, n’avait de bornes que la cupidité des riches. Ensuite un tribun le fit encore diminuer de moitié ; enfin on défendit tout prêt à usure, et de nombreux plébiscites furent rendus pour prévenir les fraudes de l’avarice, qui, tant de fois réprimées, se reproduisaient avec une merveilleuse adresse. Le préteur Gracchus, devant qui se faisaient les poursuites dont nous parlons ici, fut effrayé du grand nombre des accusés et consulta le sénat. Les sénateurs alarmés (car pas un ne se sentait irréprochable) demandèrent grâce au prince. Leur prière fut entendue, et dix-huit mois furent donnés à chacun pour régler ses affaires domestiques comme la loi l’exigeait.\par
\labelchar{XVII.} Des remboursements qui remuaient à la fois toutes les dettes, et la perte des biens de tant de condamnés, qui accumulait dans le fisc ou dans l’épargne les espèces monnayées, rendirent l’argent rare. Ajoutez un décret du sénat qui enjoignait aux prêteurs de placer en biens-fonds situés dans l’Italie les deux tiers de leurs créances. Or ceux-ci les exigeaient en entier ; et les débiteurs, requis de payer, ne pouvaient sans honte rester au-dessous de leurs engagements. En vain ils courent, ils sollicitent ; le tribunal du préteur retentit bientôt de demandes. Les ventes et les achats, où l’on avait cru trouver un remède, augmentèrent le mal. Plus d’emprunts possibles ; les riches serraient leur argent pour acheter des terres. La multitude des ventes en fit tomber le prix ; et plus on était obéré, plus on avait de peine à trouver des acheteurs. Beaucoup de fortunes étaient renversées, et la perte des biens entraînait celle du rang et de la réputation. Enfin Tibère soulagea cette détresse en faisant un fonds de cent millions de sesterces \footnote{– 19 483 561 F.}, sur lesquels l’État prêtait sans intérêt, pendant trois ans, à condition que le débiteur donnerait une caution en biens-fonds du double de la somme empruntée. Ainsi l’on vit renaître le crédit, et peu à peu les particuliers même prêtèrent. Quant aux achats de biens, on ne s’en tint pas à la rigueur du sénatus-consulte ; et c’est le sort de toutes les réformes, sévères au commencement, à la fin négligées.
\subsection[{La terreur}]{La terreur}
\noindent \labelchar{XVIII.} Bientôt on retomba dans les anciennes alarmes, en voyant Considius Proculus accusé de lèse-majesté. Il célébrait tranquillement le jour de sa naissance, lorsqu’il fut tout à coup, et dans le même instant, traîné au sénat, condamné, mis à mort. Sancia, sa sœur, fut privée du feu et de l’eau. L’accusateur était Q. Pomponius, esprit turbulent, qui couvrait ses bassesses du désir de gagner les bonnes grâces du prince, afin de sauver son frère Pomponius Sécundus. L’exil fut aussi prononcé contre Pompéia Macrina, dont Tibère avait déjà frappé le mari Argolicus et le beau-père Laco, deux des principaux de l’Achaïe. Son père, chevalier romain du premier rang, et son frère, ancien préteur, menacés d’une condamnation inévitable, se tuèrent eux-mêmes. Un de leurs crimes était l’amitié qui avait uni à Pompée leur bisaïeul Théophane de Mitylène, et les honneurs divins décernés à ce même Théophane par l’adulation des Grecs \footnote{Théophane fut l’ami et l’historiographe du grand Pompée, qui, à sa prière, rendit aux Lesbiens la liberté qu’ils avaient perdue pour avoir embrassé le parti de Mithridate. C’est en reconnaissance de ce bienfait que les Lesbiens décernèrent à Théophane les honneurs divins.}.\par
\labelchar{XIX.} Après eux, Sext. Marius, le plus riche des Espagnols, fut accusé d’inceste avec sa propre fille, et précipité de la roche Tarpéienne. Pour qu’on ne doutât point que ses richesses étaient la cause de sa perte, Tibère garda pour lui ses mines d’or, bien qu’elles fussent confisquées au profit de l’État. Bientôt, les supplices irritant sa cruauté, il fit mettre à mort tous ceux qu’on retenait en prison comme complices de Séjan. La terre fut jonchée de cadavres ; et tous les âges, tous les sexes, des nobles, des inconnus, gisaient épars ou amoncelés. Les parents, les amis, ne pouvaient en approcher, les arroser de larmes, les regarder même trop longtemps. Des soldats, postés à l’entour, épiaient la douleur, suivaient ces restes misérables lorsque, déjà corrompus, on les traînait dans le Tibre. Là, flottant sur l’eau, ou poussés vers la rive, les corps restaient abandonnés sans que personne osât les brûler, osât même les toucher. La terreur avait rompu tous les liens de l’humanité ; et plus la tyrannie devenait cruelle, plus on se défendait de la pitié.
\subsection[{Mariage de Caligula}]{Mariage de Caligula}
\noindent \labelchar{XX.} A peu près dans le même temps, Caïus César, qui avait accompagné son aïeul à Caprée, reçut en mariage Claudia, fille de M. Silanus. Caïus, sous une artificieuse douceur, cachait une âme atroce. La condamnation de sa mère, l’exil de ses frères, ne lui arrachèrent pas une plainte. Chaque jour il se composait sur Tibère ; c’était le même visage, presque les mêmes paroles. De là ce mot si heureux et si connu de l’orateur Passiénus, « qu’il n’y eut jamais un meilleur esclave ni un plus méchant maître. » Je n’omettrai pas une prédiction de Tibère au consul Servius Galba. Il le fit venir, et, après un entretien dont le but était de le sonder, il lui dit en grec : « Et toi aussi, Galba, tu goûteras quelque jour à l’empire ;" allusion à sa tardive et courte puissance, révélée à Tibère par sa science dans l’art des Chaldéens. Rhodes lui avait offert, pour en étudier les secrets, du loisir et un maître nommé Thrasylle, dont il éprouva l’habileté de la façon que je vais dire.
\subsection[{Tibère et l’astrologie}]{Tibère et l’astrologie}
\noindent \labelchar{XXI.} Toutes les fois qu’il voulait consulter sur une affaire, il choisissait une partie élevée de sa maison, et prenait pour confident un seul affranchi. Cet homme, d’une grossière ignorance, d’une grande force de corps, menait, par un sentier bordé de précipices (car la maison est sur la mer, au haut d’un rocher), l’astrologue dont Tibère se proposait d’essayer le talent. Au moindre soupçon de charlatanisme ou de fraude, le guide, en revenant, précipitait le devin dans les flots, afin de prévenir ses indiscrétions. Thrasylle fut, comme les autres, amené par cette route escarpée. Tibère, vivement frappé de ses réponses, qui lui promettaient le rang suprême et lui dévoilaient habilement les secrets de l’avenir, lui demanda s’il avait aussi fait son horoscope, et à quel signe étaient marqués pour lui cette année et ce jour même. Thrasylle, observant l’état du ciel et la position des astres, hésite d’abord ; ensuite il pâlit, et, plus il poursuit ses calculs, plus il semble agité de surprise et de crainte. Il s’écrie enfin que le moment est critique, et que le dernier des dangers le menace de près. Alors Tibère, l’embrassant, le félicite d’avoir vu le péril, le rassure ; et, regardant comme des oracles les prédictions qu’il venait de lui faire, il l’admet dés ce jour dans sa plus intime confiance.
\subsection[{Digression de Tacite : un homme qui doute}]{Digression de Tacite : un homme qui doute}
\noindent \labelchar{XXII.} Ces exemples et d’autres semblables me font douter si les choses humaines sont régies par des lois éternelles et une immuable destinée, ou si elles roulent au gré du hasard. Les plus sages d’entre les anciens et leurs modernes sectateurs professent sur ce point des doctrines opposées. Beaucoup sont imbus de l’opinion que notre commencement, que notre fin, que les hommes, en un mot, ne sont pour les dieux le sujet d’aucun soin, et que de là naissent deux effets trop ordinaires, les malheurs de la vertu et les prospérités du vice. D’autres subordonnent les événements à une destinée. Mais, indépendante du cours des étoiles, ils la voient dans les causes premières et l’enchaînement des faits qui deviennent causes à leur tour. Toutefois ils nous laissent le choix de notre vie ; mais, ajoutent-ils, « ce choix entraîne, dès qu’il est fait, une suite de conséquences inévitables. D’ailleurs les biens et les maux ne sont pas ce que pense le vulgaire : plusieurs semblent accablés par l’adversité, sans en être moins heureux ; et un grand nombre sont malheureux au sein de l’opulence, parce que les uns supportent courageusement la mauvaise fortune, ou que les autres usent follement de la bonne. » Au reste, la plupart des hommes ne peuvent renoncer à l’idée que le sort de chaque mortel est fixé au moment de sa naissance ; que, si les faits démentent quelquefois les prédictions, c’est la faute de l’imposture, qui prédit ce qu’elle ignore ; qu’ainsi se décrédite un art dont la certitude a été démontrée, et dans les siècles anciens et dans le nôtre, par d’éclatants exemples. Et en effet, le fils de ce même Thrasylle annonça d’avance l’empire de Néron, comme je le rapporterai dans la suite : ce récit m’entraînerait maintenant trop loin de mon sujet.
\subsection[{Mort de Drusus, fils de Germanicus}]{Mort de Drusus, fils de Germanicus}
\noindent \labelchar{XXIII.} Sous les mêmes consuls, on apprit la mort d’Asinius Gallus. Personne ne doutait qu’elle ne fût l’ouvrage de la faim ; mais on ignora si elle était volontaire ou forcée. Tibère, à qui on demanda la permission de lui rendre les derniers devoirs, ne rougit pas de l’accorder, tout en se plaignant du sort qui enlevait un accusé avant qu’il fût publiquement convaincu : comme si trois ans n’avaient pas suffi pour qu’un vieillard consulaire, et père de tant de consuls, parût devant ses juges ! Drusus \footnote{– Fils de Germanicus.} mourut ensuite, réduit à ronger la bourre de son lit, affreuse nourriture, avec laquelle il traîna sa vie jusqu’au neuvième jour. Il était en prison dans le palais. Quelques-uns rapportent que Macron avait ordre de l’en tirer, et de le mettre à la tête du peuple, si Séjan recourait aux armes. Bientôt, le bruit s’étant répandu que Tibère allait se réconcilier avec sa bru et son petit-fils, il aima mieux être cruel que de paraître se repentir.\par
\labelchar{XXIV.} Il poursuivit Drusus jusque dans le tombeau, lui reprochant d’infâmes prostitutions, une haine mortelle pour sa famille, un esprit ennemi de la république. Il fit lire le journal qu’on avait tenu de ses actions, de ses moindres paroles. Ce fut le comble de l’horreur de voir combien d’années des gens placés autour de lui avaient épié son visage, ses gémissements, ses soupirs les plus secrets ; de penser qu’un aïeul avait pu entendre ces détails, les lire, les produire au grand jour. On en croyait à peine ses oreilles, si les lettres du centurion Aetius et de l’affranchi Didyme n’eussent désigné par leurs noms les esclaves qui, chaque fois que Drusus voulait sortir de sa chambre, l’avaient repoussé de la main, épouvanté du geste. Le centurion répétait même des mots pleins de cruauté dont il faisait gloire. Il citait les paroles du mourant, qui, dans un faux délire, s’était livré d’abord contre Tibère aux emportements d’une raison égarée, et bientôt, privé de tout espoir, l’avait chargé d’imprécations étudiées et réfléchies, souhaitant à l’assassin de sa bru, de son neveu, de ses petits-fils, au bourreau de toute sa maison, un supplice qui vengeât à la fois ses aïeux et sa postérité. Le sénat, par ses murmures, semblait protester contre de pareils vœux mais la peur descendait au fond des âmes, avec l’étonnement qu’un homme, si rusé jadis et si attentif à envelopper ses crimes de ténèbres, en fût venu à cet excès d’impudence, de faire en quelque sorte tomber les murailles, et de montrer son petit-fils sous la verge d’un centurion, frappé par des esclaves, implorant, pour soutenir un reste de vie, des aliments qui lui sont refusés.
\subsection[{Mort d’Agrippine}]{Mort d’Agrippine}
\noindent \labelchar{XXV.} Ces impressions douloureuses n’étaient pas encore effacées lorsqu’on apprit la mort d’Agrippine. Sans doute qu’après le supplice de Séjan, soutenue par l’espérance, elle consentit à vivre, puis se laissa mourir, quand elle vit que la tyrannie n’adoucissait point ses rigueurs. Peut-être aussi la priva-t-on d’aliments, pour ménager à l’imposture la supposition d’une mort volontaire. Ce qui est certain, c’est que Tibère éclata contre sa mémoire en reproches outrageants. C’était, à l’en croire, une femme adultère, que la mort de son amant Asinius Gallus avait jetée dans le dégoût de la vie. Mais Agrippine, d’un caractère ambitieux et dominateur, en revêtant les passions des hommes, avait dépouillé les vices de son sexe. Tibère remarqua qu’elle était morte le jour même où, deux ans plus tôt, Séjan avait expié sa trahison ; fait dont il voulut que l’on conservât la mémoire. Il se fit un mérite de ce qu’elle n’avait été ni étranglée ni jetée aux Gémonies. Des actions de grâces lui en furent rendues, et on décréta que le quinze avant les calendes de novembre, jour où Agrippine et Séjan avaient péri, on consacrerait tous les ans un don à Jupiter.
\subsection[{Mort de Coccéius Nerva et de Plancine}]{Mort de Coccéius Nerva et de Plancine}
\noindent \labelchar{XXVI.} Peu de temps après, Coccéius Nerva, ami inséparable du prince, profondément versé dans les lois divines et humaines, jouissant d’une fortune prospère, exempt d’infirmités, résolut de mourir. Instruit de ce dessein, Tibère ne quitte plus ses côtés, le presse de questions, a recours aux prières, lui avoue enfin quel poids ce sera pour sa conscience, quelle injure pour sa renommée, que son ami le plus intime ait fui la vie sans aucune raison de vouloir la mort. Nerva, sourd à ces représentations, s’abstint dès lors de toute nourriture. Les confidents de ses pensées disaient que, voyant de plus près que personne les maux de la république, c’était par colère et par crainte qu’il avait cherché une fin honorable, avant que sa gloire et son repos fussent attaqués. Au reste, la perte d’Agrippine, ce qu’on croirait à peine, entraîna celle de Plancine. Mariée autrefois à Cn. Pison, cette femme avait publiquement triomphé de la mort de Germanicus. Quand Pison tomba, protégée par les prières de Livie, elle ne le fut pas moins par la haine d’Agrippine. Dès que la haine et la faveur cessèrent, la justice prévalut. Accusée de crimes manifestes, elle s’en punit de sa main, châtiment plus tardif que rigoureux.
\subsection[{Remariage de Julie, fille de Drusus}]{Remariage de Julie, fille de Drusus}
\noindent \labelchar{XXVII.} Pendant que toutes ces morts mettaient Rome en deuil, ce fut un surcroît de douleur de voir Julie, fille de Drusus, autrefois épouse de Néron, passer par le mariage dans la maison de Rubellius Blandus, petit-fils d’un homme que plusieurs se souvenaient d’avoir connu à Tibur simple chevalier romain. A la fin de l’année, la mort d’Elius Lamia fut honorée par des funérailles solennelles. Elius, délivré enfin du vain titre de gouverneur de Syrie, avait été préfet de Rome. Sa naissance était distinguée ; sa vieillesse fut pleine de vigueur, et le gouvernement dont on l’avait privé le relevait encore dans l’estime publique. On lut ensuite, à l’occasion de la mort de Pomponius Flaccus, propréteur de Syrie, une lettre de Tibère. Il se plaignait de ce que les hommes les plus illustres et les plus capables de commander les armées refusaient cet emploi ; refus qui le contraignait d’avoir recours aux prières pour déterminer quelques-uns des consulaires à se charger des gouvernements. Il avait oublié que depuis dix ans il empêchait Arruntius de se rendre en Espagne. La même année, mourut M. Lépidus, dont j’ai assez fait connaître la modération et la sagesse dans les livres précédents. Il est inutile de parler longuement de sa noblesse : la maison des Émiles fut toujours féconde en grands citoyens ; et, dans cette famille, ceux même qui n’eurent pas de vertus fournirent encore une carrière brillante \footnote{L’auteur fait allusion surtout au triumvir Lépide et au père du triumvir, Emilius Lépidus, qui, étant consul après la mort de Sylla, réunit les débris du parti de Marius, recommença la guerre civile, et fut battu par son collègue Catulus, d’abord sous les murs de Rome, puis en Étrurie.}.
\subsection[{Le phénix}]{Le phénix}
\noindent \labelchar{XXVIII.} Sous le consulat de Paulus Fabius et de L. Vitellius, parut en Égypte, après un long période de siècles, le phénix, oiseau merveilleux qui fut pour les savants grecs et nationaux le sujet de beaucoup de dissertations. Je rapporterai les faits sur lesquels ils s’accordent, et un plus grand nombre qui sont contestés et qui pourtant méritent d’être connus. Le phénix est consacré au soleil. Ceux qui l’ont décrit conviennent unanimement qu’il ne ressemble aux autres oiseaux, ni par la forme, ni par le plumage. Les traditions diffèrent sur la durée de sa vie. Suivant l’opinion la plus accréditée, elle est de cinq cents ans. D’autres soutiennent qu’elle est de quatorze cent soixante et un. Le phénix parut, dit-on, pour la première fois sous Sésostris, ensuite sous Amasis, enfin sous Ptolémée, le troisième des rois macédoniens ; et chaque fois il prit son vol vers Héliopolis, au milieu d’un cortège nombreux d’oiseaux de toute espèce, attirés par la nouveauté de sa forme. Mais de telles antiquités sont pleines de ténèbres. Entre Ptolémée et Tibère, on compte moins de deux cent cinquante ans. Aussi quelques-uns ont-ils cru que ce dernier phénix n’était pas le véritable, qu’il ne venait pas d’Arabie, et qu’on ne vit se vérifier en lui aucune des anciennes observations. On assure, en effet, qu’arrivé au terme de ses années, et lorsque sa mort approche, le phénix construit dans sa terre natale un nid auquel il communique un principe de fécondité, d’où doit naître son successeur. Le premier soin du jeune oiseau, le premier usage de sa force, est de rendre à son père les devoirs funèbres. La prudence dirige son entreprise. D’abord il se charge de myrrhe, essaye sa vigueur dans de longs trajets, et, lorsqu’elle suffit à porter le fardeau et à faire le voyage, il prend sur lui le corps de son père, et va le déposer et le brûler sur l’autel du soleil. Ces récits sont incertains, et la fable y a mêlé ses fictions. Néanmoins on ne doute pas que cet oiseau ne paraisse quelquefois en Égypte.
\subsection[{Suicides}]{Suicides}
\noindent \labelchar{XXIX.} Cependant à Rome, où le sang ne cessait de couler, Pomponius Labéo, le même que la Mésie avait eu pour gouverneur, s’ouvrit les veines et abandonna la vie. Sa femme Paxéa suivit son exemple. La crainte du bourreau multipliait ces morts volontaires. Les condamnés étaient d’ailleurs privés de sépulture et leurs biens confisqués ; on gagnait au contraire à disposer de soi-même et à se hâter de mourir : les honneurs du tombeau et le respect des testaments étaient à ce prix. Au reste, Tibère écrivit au sénat que l’usage de nos ancêtres était d’interdire leur maison, en signe de rupture, à ceux dont ils voulaient cesser d’être amis ; qu’il en avait usé de la sorte avec Labéo, et que cet homme, accablé par les preuves d’une administration infidèle et de plusieurs autres attentats, avait couvert sa honte de l’intérêt qu’inspire une victime. Quant à sa femme, elle s’était faussement alarmée quoique coupable, elle n’avait rien à craindre. Ensuite fut attaqué de nouveau Mamercus Scaurus, distingué par sa noblesse et son éloquence, infâme par ses mœurs. Ce ne fut point l’amitié de Séjan qui le perdit : ce fut la haine de Macron, non moins mortelle à qui l’avait encourue. Macron continuait, mais avec plus de mystère, les pratiques de son prédécesseur. Il dénonça le sujet d’une tragédie composée par Scaurus, et indiqua les vers dont le sens détourné s’appliquait au prince. Mais Servilius et Cornélius chargés de l’accusation, alléguèrent un commerce adultère avec Livie, et des sacrifices magiques. Scaurus, avec un courage digne des Émiles, ses aïeux, prévint le jugement, à la persuasion de sa femme Sextia, qui partagea sa mort après l’avoir conseillée.
\subsection[{Relégations}]{Relégations}
\noindent \labelchar{XXX.} Les accusateurs étaient punis à leur tour, quand l’occasion s’en présentait. Ainsi Cornélius et Servilius, qu’avait honteusement signalés la perte de Scaurus, furent relégués dans une île, avec interdiction du feu et de l’eau, pour avoir fait payer leur silence à Varius Ligur, qu’ils menaçaient d’une dénonciation ; et Abudius Ruso, ancien édile, ayant voulu faire un crime à Lentulus Gétulicus, sous lequel il avait commandé une légion, d’avoir choisi pour gendre le fils de Séjan, fut condamné lui-même et chassé de Rome. Gétulicus commandait alors les légions de la haute Germanie, et s’était acquis auprès d’elles une merveilleuse popularité, prodigue de grâces, avare de châtiments, et, par son beau-père Apronius, agréable même à l’armée voisine. C’est une tradition accréditée qu’il osa écrire au prince que, « s’il avait pensé à l’alliance de Séjan, c’était par le conseil de Tibère ; qu’il avait pu se tromper aussi bien que César ; que la même erreur ne devait pas être pour l’un sans reproche, pour les autres sans pardon ; que sa foi, inviolable jusqu’alors, le serait toujours, si sa sûreté n’était pas menacée ; qu’il regarderait l’envoi d’un successeur comme un arrêt de mort ; qu’ils pouvaient conclure une espèce de traité, par lequel le prince, maître du reste de l’empire, laisserait au général sa province. » Ce fait, tout surprenant qu’il est, parut croyable, quand on vit que, de tous les alliés de Séjan, Gétulicus seul conservait sa vie et sa faveur. Chargé de la haine publique et affaibli par les années, Tibère comprit que l’opinion, plus que la force, soutenait sa puissance.
\subsection[{Révolte des Parthes}]{Révolte des Parthes}
\noindent \labelchar{XXXI.} Sous le consulat de C. Cestius et de M. Servilius, quelques grands de la nation des Parthes vinrent à Rome, à l’insu de leur roi Artaban. Fidèle aux Romains et juste envers les siens tant qu’il craignit Germanicus, ce prince ne tarda pas ensuite à braver notre empire et à tyranniser ses peuples. Des guerres faites avec succès aux nations voisines avaient enflé son orgueil ; il méprisait, comme faible et désarmée, la vieillesse de Tibère, et il convoitait l’Arménie. Ce pays ayant perdu son roi Artaxias, Artaban lui imposa l’aîné de ses fils, nommé Arsace ; et, joignant l’insulte à l’usurpation, il envoya réclamer les trésors laissés par Vonon dans la Syrie et la Cilicie. En même temps il parlait des anciennes limites des Perses et des Macédoniens, et menaçait, avec une insolente jactance, de reprendre tout ce qu’avaient possédé Cyrus et Alexandre. Le Parthe dont les conseils contribuèrent le plus à l’envoi d’une députation secrète, fut Sinnacès, également distingué par sa naissance et par ses richesses, et après lui l’eunuque Abdus. Chez les barbares, la qualité d’eunuque n’entraîne point le mépris ; elle conduit même quelquefois au pouvoir. Ces deux hommes s’associèrent d’autres nobles ; et, comme ils ne pouvaient placer sur le trône aucun prince du sang d’Arsace, la plupart ayant été tués par Artaban, et les autres n’étant pas encore sortis de l’enfance, ils demandèrent à Rome Phraate, fils du roi Phraate. « Il ne leur fallait, disaient-ils, qu’un nom et l’aveu de César. Qu’il fût permis à un Arsacide de se montrer sur les bords de l’Euphrate, c’était assez. »\par
\labelchar{XXXII.} Ce plan entrait dans les vues de Tibère. Fidèle à sa maxime d’employer dans les affaires du dehors la ruse et la politique, sans y engager ses armées, il envoie Phraate, enrichi de présents, à la conquête du trône paternel. Pendant ce temps, Artaban, instruit de ces complots, était tantôt retenu par la crainte, tantôt embrasé du feu de la vengeance : et, pour les barbares, différer est d’un esclave ; exécuter à l’instant, c’est agir en roi. Toutefois l’intérêt prévalut. Il invite Abdus à un repas, en signe d’amitié, et s’assure de lui par un poison lent. Il dissimule avec Sinnacès, et l’enchaîne par des présents et des emplois. Quant à Phraate, accoutumé depuis tant d’années à la vie des Romains, il la quitta en Syrie pour reprendre celle des Parthes ; et, trop faible pour des mœurs qui n’étaient plus les siennes, il fut emporté par une maladie. Tibère n’en poursuivit pas moins ses desseins. Il donne pour rival à Artaban Tiridate, prince du même sang, choisit l’Ibérien Mithridate pour reconquérir l’Arménie, le réconcilie avec Pharasmane, son frère, qui régnait en Ibérie, héritage de la famille, et charge Vitellius de diriger toutes les révolutions qui se préparaient en Orient. Je n’ignore pas que ce consulaire a laissé à Rome une mémoire décriée, et que mille traits sont racontés à sa honte. Mais il porta dans le gouvernement des provinces les vertus antiques. Après son retour, la crainte de Caïus, l’amitié de Claude, l’abaissèrent à une honteuse servilité, et on le cite aujourd’hui comme le modèle de la plus abjecte adulation. Sa fin a démenti ses commencements, et une vieillesse couverte d’opprobre a flétri une jeunesse honorable.\par
\labelchar{XXXIII.} Parmi les princes d’Orient, Mithridate, s’adressant le premier à Pharasmane, lui persuade de le seconder par la ruse et par la force. On trouva des corrupteurs, qui, avec beaucoup d’or, poussèrent au crime les serviteurs d’Arsace ; et en même temps les Ibériens, avec des troupes nombreuses, envahirent l’Arménie et s’emparèrent de la ville d’Artaxate. A cette nouvelle, Artaban confie sa vengeance à son fils Orode, lui donne une armée de Parthes, et envoie au dehors acheter des auxiliaires. Pharasmane, de son côté, se ligue avec les Albaniens, et appelle les Sarmates, dont les princes, payés par les deux partis, se vendirent, suivant l’usage de leur nation, aux deux causes opposées. Mais les Ibériens, maîtres du pays, ouvrirent les portes Caspiennes \footnote{L’espace qui sépare la mer Caspienne du Pont-Euxin forme une espèce d’isthme au travers duquel le Caucase s’élève comme une muraille immense. Les divers passages de cette montagne ont reçu des anciens le nom de portes : ce sont les portes Caucasiennes, Albaniennes, Ibériennes. Le nom de portes Caspiennes est appliqué vaguement par les Romains à plusieurs de ces défilés, quoiqu’il appartienne proprement à un passage beaucoup plus au sud, dans le mont Caspius, entre la Médie et le pays des Parthes.}, et inondèrent l’Arménie de leurs Sarmates. Ceux qui arrivaient aux Parthes étaient au contraire facilement arrêtés : l’ennemi occupait tous les passages ; le seul qui restât entre la mer et les dernières montagnes d’Albanie était impraticable à cause de l’été, car alors les vents étésiens submergent cette côte. C’est en hiver seulement, lorsque le vent du midi refoule les eaux et fait rentrer, la mer dans son lit, que les grèves sont découvertes.\par
\labelchar{XXXIV.} Orode était ainsi privé de secours. Pharasmane, appuyé de ses auxiliaires, lui présente la bataille, et, voyant qu’il l’évitait, il le harcèle, insulte son camp, massacre ses fourrageurs ; souvent même il l’environne d’une ceinture de postes et le tient comme assiégé. Enfin les Parthes, peu faits à souffrir l’outrage, entourent leur roi et lui demandent le combat. Toute leur force consistait en cavalerie. Pour Pharasmane, il avait aussi des gens de pied. Car les Ibériens et les Albaniens habitant un pays de montagnes, supportent mieux une vie dure et des travaux pénibles. Ils se disent issus de ces Thessaliens qui suivirent Jason, lorsque après avoir enlevé Médée et en avoir eu des enfants il revint, à la mort d’Éétès, occuper son palais désert et donner un maître à Colchos. Le nom de ce héros se retrouve partout dans le pays, et l’oracle de Phrixus y est révéré. On n’oserait y sacrifier un bélier, animal sur lequel ils croient que Phrixus passa la mer, ou dont peut-être l’image décorait son vaisseau. Les deux armées rangées en bataille, le Parthe vante à ses guerriers l’éclat des Arsacides, et demande ce que peuvent, contre une nation maîtresse de l’Orient, l’Ibérien sans gloire et ses vils mercenaires. Pharasmane rappelle aux siens qu’ils n’ont jamais subi le joug des Parthes ; que, plus leur entreprise est grande, plus elle offre de gloire au vainqueur, de honte et de péril au lâche qui fuirait. Et il leur montre, de son côté, des bataillons hérissés de fer, du côté de l’ennemi, des Mèdes chamarrés d’or ; ici des soldats, là une proie à saisir.\par
\labelchar{XXXV.} Mais ce n’était pas la voix seule du chef qui animait les Sarmates. Ils s’excitent l’un l’autre à ne pas engager l’action avec leurs flèches, mais à s’élancer et à fondre inattendus sur l’armée ennemie. La bataille offrit un spectacle varié. Le Parthe, également exercé à poursuivre et à fuir, se débande, prend de l’espace pour mesurer ses coups. Les Sarmates, renonçant à leurs arcs, dont la portée est moins longue, courent la pique en avant ou l’épée à la main. Tantôt, comme dans un combat de cavalerie, c’est une alternative de charges et de retraites ; tantôt c’est une mêlée où les armes s’entrechoquent, les hommes se poussent et se repoussent. Enfin les Albaniens et les Ibériens saisissent leurs ennemis, les démontent, les mettent dans un double péril, entre les coups dont les accable d’en haut le bras des cavaliers, et ceux que le fantassin leur assène de plus près. Pharasmane et Orode couraient partout, secondant ou ranimant les courages. Ils se reconnaissent aux marques de leur dignité ; et leurs cris, leurs traits, leurs coursiers se croisent à l’instant. Pharasmane était le plus impétueux : il perça le casque d’Orode ; mais, emporté par son cheval, il ne put redoubler, et le blessé fut couvert par les plus intrépides de ses gardes. Toutefois le bruit faussement répandu qu’Orode était mort effraya les Parthes, et ils cédèrent la victoire.\par
\labelchar{XXXVI.} Artaban remua, pour venger cette défaite, toutes les forces de son empire. Les Ibériens, connaissant mieux le pays, eurent un nouvel avantage ; et cependant il ne se rebutait pas encore, si Vitellius, en rassemblant ses légions et en semant le bruit d’une invasion dans la Mésopotamie, ne lui eût fait peur des armes romaines. Alors Artaban quitta l’Arménie, et ses affaires allèrent en décadence. Vitellius sollicitait les Parthes d’abandonner un roi qui était leur fléau, dans la paix par sa cruauté, dans la guerre par ses revers. J’ai déjà dit que Sinnacès était l’ennemi d’Artaban. Il entraîne à la révolte son père Abdagèse et d’autres mécontents dont cette suite de désastres avait encouragé les secrets desseins. Le parti se grossit peu à peu de tous ceux qui, plus soumis par crainte que par attachement, avaient repris de l’audace en se voyant des chefs. Artaban n’avait pour toute ressource que quelques étrangers dont il formait sa garde, vil ramas de bannis, qui n’ont ni intelligence du bien, ni souci du mal, mercenaires qu’on nourrit pour être les instruments du crime. Il part avec eux et s’enfuit rapidement jusqu’aux frontières de la Scythie : il croyait y trouver du secours, ayant des liaisons de famille avec les Hyrcaniens et les Carmaniens ; et même il fondait quelque espoir sur l’inconstance des Parthes, aussi prompts à regretter leurs rois qu’à les trahir.\par
\labelchar{XXXVII.} Vitellius, voyant Artaban en fuite et les esprits disposés à un changement de maître, exhorte Tiridate à prendre possession de sa conquête, et mène aux rives de l’Euphrate l’élite des légions et des auxiliaires. Là, suivant l’usage des Romains, il offrait aux dieux un suovétaurile\footnote{De \emph{sus, ovis, taurus}, un porc, un bélier, un taureau.}, et Tiridate immolait un cheval en l’honneur du fleuve. Tout à coup les habitants annoncèrent que, de lui-même et sans la moindre pluie l’Euphrate venait de s’élever outre mesure, et que l’écume blanchissante formait à la surface de l’eau des cercles qui semblaient autant de diadèmes. Ce fut pour les uns l’augure d’un heureux passage ; d’autres, par une interprétation plus subtile, en conclurent que la fortune, favorable d’abord, ne le serait pas longtemps. Selon eux, « les phénomènes du ciel et de la terre parlaient sans doute un langage plus sûr ; mais les fleuves, dans leur éternelle mobilité, ne faisaient que montrer et emporter le présage. » Cependant on fit un pont de bateaux, et l’armée passa sur l’autre rive. Ornospade vint le premier s’y joindre avec plusieurs milliers d’hommes à cheval. Exilé jadis, Ornospade se distingua sous Tibère, qui achevait la guerre de Dalmatie, et ses services lui valurent le droit de cité romaine. Depuis, rentré en grâce auprès du roi, et comblé de distinctions, il eut le gouvernement des plaines immenses qui, enfermées entre les deux célèbres fleuves du Tigre et de l’Euphrate, ont reçu le nom de Mésopotamie. Peu après, Sinnacès amena de nouvelles troupes ; et Abdagèse, le soutien de ce parti, livra les trésors et toutes les décorations de la grandeur royale. Vitellius, persuadé qu’il suffisait d’avoir montré les armes romaines, engage Tiridate et les grands, l’un à ne pas oublier qu’il est le petit-fils de Phraate et le nourrisson de César, deux titres si glorieux pour lui, les autres à demeurer toujours soumis à leur roi, respectueux envers nous, fidèles à l’honneur et au devoir. Ensuite il revient en Syrie avec ses légions.
\subsection[{Morts et suicides}]{Morts et suicides}
\noindent \labelchar{XXXVIII.} Ces événements furent l’ouvrage de deux étés. Je les ai réunis pour me délasser du spectacle des malheurs domestiques. Car trois ans s’étaient vainement écoulés depuis la mort de Séjan : le temps, les prières, la satiété, qui adoucissent les cœurs les plus aigris, n’avaient point désarmé Tibère. Il poursuivait des faits douteux et oubliés, comme des crimes récents et irrémissibles. Averti par ces rigueurs, Fulcinius Trio ne voulut pas subir l’outrage d’une accusation. Dans l’écrit dépositaire de ses dernières pensées, il entassa mille invectives contre Macron et les principaux affranchis du palais ; il n’épargna pas même l’empereur, que l’âge avait, disait-il, privé de sa raison, et dont la retraite sans fin n’était qu’un exil. Les héritiers cachaient ce testament : Tibère en ordonna la lecture, affectant de souffrir la liberté d’autrui, et bravant sa propre infamie, ou curieux peut-être, après avoir ignoré si longtemps les crimes de Séjan, de les entendre publier à quelque prix que ce fût, et d’apprendre, au moins par l’injure, la vérité qu’étouffe l’adulation. Quelques jours après, le sénateur Granius Martianus, accusé de lèse-majesté par C. Gracchus, se donna la mort ; et Tatius Gratianus, ancien préteur, poursuivi sous le même prétexte, fut condamné au dernier supplice.\par
\labelchar{XXXIX.} De semblables trépas finirent les jours de Trébelliénus Rufus et de Sextius Paconianus. Le premier se tua de sa main ; l’autre avait fait dans sa prison des vers contre le prince, il y fut étranglé. Et ce n’était plus de l’autre côté de la mer et par de lointains messages que Tibère recevait ces nouvelles. Établi près de Rome \footnote{Tantôt à Tusculum, tantôt dans le territoire d’Albe.}, et répondant le jour même ou après l’intervalle d’une nuit aux lettres des consuls, il regardait, en quelque sorte, le sang couler à flots dans les maisons, et les bourreaux à l’ouvrage. A la fin de l’année mourut Poppéus Sabinus, d’une naissance médiocre, honoré, par l’amitié des princes, du consulat et des décorations triomphales, et placé vingt-quatre ans à la tête des plus grandes provinces, non qu’il fût doué de qualités éminentes, mais parce que sa capacité suffisait aux affaires, sans s’élever au-dessus.\par
\labelchar{XL.} L’année suivante eut pour consuls Q. Plautius et Sext. Papinius. Le supplice de L. Aruséius, dans une ville accoutumée au spectacle de ses maux, fut une cruauté à peine remarquée ; mais la terreur fut au comble, quand on vit le chevalier romain Vibulénus, après avoir entendu jusqu’au bout ses accusateurs, tirer du poison de dessous sa robe et l’avaler en plein sénat. Il tomba mourant : les licteurs le saisirent à la hâte, le traînèrent dans la prison, et les étreintes du lacet pressèrent un cadavre. Tigrane même, autrefois souverain d’Arménie et alors accusé, ne put échapper au supplice ; roi, il périt comme les citoyens. Le consulaire C. Galba et les deux Blésus finirent volontairement leurs jours : Galba, sur une lettre sinistre où l’empereur lui défendait de se présenter au partage des provinces ; les Blésus, parce que des sacerdoces promis à chacun d’eux pendant la prospérité de leur maison, ajournés depuis ses malheurs, venaient enfin d’être donnés d’autres comme des dignités vacantes. C’était un arrêt de mort ; ils le comprirent et l’exécutèrent. Émilia Lépida, dont j’ai rapporté le mariage avec le jeune Drusus, et qui fut l’accusatrice acharnée de son époux, vécut abhorrée, et toutefois impunie, tant que son père Lépidus vit le jour. Quand il fut mort, les délateurs s’emparèrent d’elle, pour cause d’adultère avec un esclave. On ne doutait nullement du crime ; aussi, renonçant à se défendre, elle mit fin à sa vie.
\subsection[{Révolte en Cappadoce}]{Révolte en Cappadoce}
\noindent \labelchar{XLI.} Pendant ce même temps, la nation des Clites, soumise au Cappadocien Archélaüs \footnote{La Cappadoce avait été réduite en province romaine à la mort de son roi Archélaüs sans doute père de celui-ci. Les Clites habitaient la partie montagneuse de la Cilicie. 2. Nicéphorium, ville de Mésopotamie, sur le bord de l’Euphrate, bâtie par ordre d’Alexandre, aujourd’hui Racca. – Anthémusiade, ville de l’Osmène, dans la même contrée, entre l’Euphrate et le Tigre. 3. Halus, ville d’Assyrie, aujourd’hui Galoula, selon d’Anville. D’Anville dit que la position d’Artémite, cette ancienne cité, tombe sur un lieu nommé Dascara et surnommé El-Melik, la Royale, parce que Khosroès II, roi de Perse, y habita vingt-quatre ans ; elle s’appelait alors Dastagerda.}, et mécontente d’être assujettie, comme nos tributaires, au cens et aux impôts, se retira sur les hauteurs du mont Taurus, où l’avantage des lieux la soutenait contre les troupes mal aguerries du roi. Enfin le lieutenant M. Trébellius y fut envoyé par Vitellius, gouverneur de Syrie, avec quatre mille légionnaires et l’élite des alliés. Les barbares occupaient deux collines, la moins haute nommée Cadra, l’autre Davara. Il les environna d’une circonvallation et tailla en pièces ceux qui hasardèrent des sorties ; la soif obligea les autres à se rendre. Thridace cependant, reconnu volontairement par les Parthes, prit possession de Nicéphorium, d’Anthémusiade (2) et des autres villes qui, fondées par les Macédoniens, ont reçu des noms grecs ; il prit aussi deux villes parthiques, Artémite et Halus (3) ; et partout éclatait l’enthousiasme des peuples, qui, détestant pour sa cruauté Artaban, nourri chez les Scythes, espéraient de Tiridate, élève de la civilisation romaine, un caractère plus doux.\par
\labelchar{XLII.} L’adulation se signala particulièrement à Séleucie \footnote{Séleucie était située sur la rive droite du Tigre, à quelques lieues au-dessous de la position actuelle de Bagdad. Sur la rive opposée du même fleuve, et pour contre-balancer la puissance de Séleucie, les Parthes bâtirent Ctésiphon.}. C’est une ville puissante, environnée de murailles, et qui au milieu de la barbarie, a gardé l’esprit de son fondateur Séleucus. Trois cents citoyens, choisis d’après leur fortune ou leurs lumières, lui composent un sénat. Le peuple a sa part de pouvoir. Quand ces deux ordres sont unis, on ne craint rien du Parthe ; s’ils se divisent, chacun cherche de l’appui contre ses rivaux, et l’étranger, appelé au secours d’un parti, les asservit tous deux. C’est ce qui venait d’avoir lieu sous Artaban, dont la politique livra le peuple à la discrétion des grands ; il savait que le gouvernement populaire est voisin de la liberté, tandis que la domination du petit nombre ressemble davantage au despotisme d’un roi. A l’arrivée de Tiridate, on lui prodigua tous les honneurs dont jouirent les anciens monarques, avec ceux qu’y avait encore ajoutés l’adulation moderne ; et en même temps on maudissait le nom d’Artaban, qui, disait-on, « ne tenait que par sa mère à la famille d’Arsace, et n’était du reste qu’un rejeton bâtard. » Tiridate remit le pouvoir aux mains du peuple. Ensuite, comme il délibérait sur le jour où il prendrait solennellement les marques de la royauté, il reçut de Phraate et d’Hiéron, gouverneurs des deux principales provinces, des lettres où ils le priaient de les attendre quelques jours. Il crut devoir cet égard à des hommes si puissants. Dans l’intervalle, il se rendit à Ctésiphon, siège de l’empire. Mais, comme ils demandaient chaque jour un nouveau délai, Suréna, suivant l’usage du pays, et aux acclamations d’un peuple immense, ceignit du bandeau royal le front de Tiridate.\par
\labelchar{XLIII.} Si en ce moment il avait pénétré plus avant et s’était montré au reste des provinces, il s’emparait des volontés indécises, et tout se ralliait sous les mêmes étendards. En assiégeant un château où Artaban avait renfermé ses trésors et ses concubines, il laissa le temps d’oublier les promesses. Phraate, Hiéron, et tous ceux dont le concours avait manqué à la solennité du jour où il prit le diadème, redoutant sa colère, ou jaloux d’Abdagèse, qui gouvernait la cour et le nouveau roi, se tournèrent du côté d’Artaban. Ce prince fut trouvé en Hyrcanie, couvert de haillons, et n’ayant que son arc pour fournir à ses besoins. Il crut d’abord qu’on lui tendait un piège et conçut des craintes. Bientôt, sur l’assurance qu’on était venu pour lui rendre son trône, il reprend courage et demande quel est donc ce changement soudain. Alors Hiéron se déchaîne contre Tiridate, qu’il appelle un enfant. « Non, l’empire n’était pas aux mains d’un Arsacide ; ce lâche, corrompu par la mollesse étrangère, ne possédait qu’un vain titre ; la puissance était dans la maison d’Abdagèse. »\par
\labelchar{XLIV.} L’expérience du vieux roi comprit que, si leur amour était faux, leurs haines ne l’étaient pas. Il ne différa que le temps de rassembler chez les Scythes des troupes auxiliaires, et s’avança rapidement pour prévenir et les ruses de ses ennemis, et l’inconstance de ses amis. Il avait conservé ses haillons, afin d’émouvoir la pitié de la multitude. Artifices, prières, il n’omit rien pour gagner les indécis, affermir les zélés. Déjà il s’approchait en force de Séleucie, et Tiridate apprit à la fois la marche et l’arrivée d’Artaban. A ce coup subit, il demeure incertain s’il ira le combattre, ou s’il traînera la guerre en longueur. Ceux qui étaient d’avis de livrer bataille et de brusquer la fortune voulaient qu’on profitât du désordre et de la fatigue d’une longue route, et du peu de temps qu’avaient eu pour se rattacher au devoir des soldats traîtres naguère et rebelles au maître qu’ils servaient maintenant. Mais le conseil d’Abdagèse était qu’on se retirât en Mésopotamie. Là, couverts par le fleuve, on ferait lever derrière soi les Arméniens, les Elyméens et les autres nations ; puis, accrus de ces renforts et de ceux qu’enverrait le général romain, on tenterait le sort des armes. Cet avis prévalut, grâce à l’ascendant d’Abdagèse et à la faiblesse de Tiridate en présence du danger. Mais la retraite eut l’air d’une fuite. La désertion commence par les Arabes, et bientôt chacun regagne sa demeure, ou va grossir l’armée d’Artaban. Enfin Tiridate retourne lui-même en Syrie avec une poignée d’hommes, et sauve à tous la honte d’une trahison.
\subsection[{Incendie à Rome – Machinations de Macron}]{Incendie à Rome – Machinations de Macron}
\noindent \labelchar{XLV.} La même année, un violent incendie éclata dans Rome, et consuma la partie du cirque qui touche au mont Aventin et tout le quartier bâti sur cette colline. Tibère fit tourner ce désastre à sa gloire en payant le prix des maisons détruites. Cent millions de sesterces furent employés à cet acte de munificence dont on lui sut d’autant plus de gré, que pour lui-même il dépensait peu en bâtiments. Du reste, il ne construisit non plus que deux édifices publics, un temple à Auguste et la scène du théâtre de Pompée ; et même, quand ces ouvrages furent achevés, soit mépris des applaudissements, soit vieillesse, il négligea d’en faire la dédicace. L’estimation des pertes causées par l’incendie fut confiée aux quatre gendres de l’empereur \footnote{Les maris de ses petites-filles.}, Cn. Domitius, Cassius Longinus, M. Vinicius et Rubellius Blandus, auxquels fut adjoint, sur le choix des consuls, P. Pétronius, On décerna au prince tous les honneurs que put inventer le génie de l’adulation. On ignore ceux qu’il agréa ou refusa : sa mort suivit de trop près. Ce fut en effet au bout d’assez peu de temps que les derniers consuls du règne de Tibère, Cn. Acerronius et C. Pontius, entrèrent en charge. Déjà Macron jouissait d’un pouvoir excessif. Il n’avait jamais négligé l’amitié de Caïus César, et de jour en jour il la cultivait avec plus d’empressement. Après la mort de Claudia, qui avait été, comme je l’ai dit, mariée à Caïus, Macron fit servir à ses vues sa femme Ennia, que lui-même envoyait auprès du jeune homme avec mission de le séduire, et de l’enchaîner par une promesse de mariage. Celui-ci se prêtait à tout pour arriver au trône ; car, malgré la violence de son caractère, il avait appris, à l’école de son aïeul, les ruses de la dissimulation.
\subsection[{Quel successeur ?}]{Quel successeur ?}
\noindent \labelchar{XLVI.} Le prince le savait ; aussi balança-t-il sur le choix du maître qu’il donnerait à l’empire. De ses deux petits-fils, la tendresse et le sang parlaient pour celui dont Drusus était le père ; mais il n’était pas encore sorti de l’enfance. Le fils de Germanicus, déjà dans la force de l’âge, était chéri du peuple et par conséquent haï de son aïeul. Restait Claude, d’un âge mûr, désirant naturellement le bien, mais faible d’esprit : Tibère n’y songea qu’un instant. De chercher un successeur hors de sa maison, il craignait que ce ne fût livrer la mémoire d’Auguste et le nom des Césars à l’insulte et aux outrages car, si l’opinion contemporaine le touchait peu, l’avenir n’était pas indifférent à sa vanité. Enfin, l’esprit irrésolu, le corps affaissé, il abandonna au destin une délibération dont il n’était pas capable. Toutefois, des paroles tombées de sa bouche témoignèrent qu’il en prévoyait l’issue. Il fit à Macron le reproche clairement allégorique de quitter le couchant pour regarder l’orient. Il prédit à Caïus, qui dans une conversation se moquait de Sylla, qu’il aurait tous les vices de ce dictateur et pas une de ses vertus. Comme il embrassait, en pleurant beaucoup, le plus jeune de ses petits-fils, il surprit à Caïus un regard sinistre : « Tu le tueras \footnote{Caïus fit mourir, en effet, le jeune Tibère, dés la première année de son règne.}, lui dit-il, et un autre te tuera. » Cependant sa santé s’affaiblissait de jour en jour, sans qu’il renonçât à aucune de ses débauches ; patient pour paraître fort, accoutumé d’ailleurs à se railler de la médecine et de ceux qui, passé trente ans, avaient besoin, pour connaître ce qui leur était bon ou mauvais, de conseils étrangers.
\subsection[{Dernières purges}]{Dernières purges}
\noindent \labelchar{XLVII.} A Rome, on préparait les voies à des assassinats qui devaient avoir leur cours même après Tibère. Lélius Balbus avait accusé de lèse-majesté Acutia, qui avait été femme de P. Vitellius. Acutia condamnée, comme on décernait une récompense à l’accusateur, Junius Otho, tribun du peuple, opposa son intervention. Ce fut entre eux une source de haines, que l’exil d’Otho suivit de près. Une femme décriée par le nombre de ses amants, Albucilla, qui avait eu pour mari Satrius Sécundus, dénonciateur de Séjan, fut déférée comme impie envers le prince. On lui donnait pour complices d’impiété et d’adultères Cn. Domitius \footnote{Le gendre même de Tibère, le mari d’Agrippine, mère de Néron. Voy. Suétone, \emph{Néron}, chap. v.}, Vibius Marsus, L. Arruntius. J’ai parlé de l’illustration de Domitius. Marsus joignait aussi à d’anciens honneurs l’éclat des talents. Les pièces envoyées au sénat portaient que Macron avait présidé à l’interrogatoire des témoins et à la torture des esclaves. Le prince d’ailleurs n’ayant point écrit contre les accusés, on soupçonnait Macron d’avoir abusé de son état de faiblesse, et forgé peut-être à son insu la plupart des griefs, en haine d’Arruntius, dont on le savait ennemi.\par
\labelchar{XLVIII.} Domitius, en préparant sa défense, Marsus, en feignant de se laisser mourir de faim, prolongèrent leur vie. Pressé par ses amis de temporiser comme eux, Arruntius répondit « que les convenances n’étaient pas les mêmes pour tous ; qu’il avait assez vécu ; que tout son regret était d’avoir traîné, parmi les affronts et les périls, une vieillesse tourmentée, odieux longtemps à Séjan, maintenant à Macron, toujours à la puissance du moment et cela sans autre tort que son horreur pour le crime. Sans doute il pouvait échapper aux derniers jours d’un prince expirant ; mais comment éviter la jeunesse du maître qui menaçait l’empire ? Si Tibère, avec sa longue expérience, n’avait pas tenu contre cet enivrement du pouvoir qui change et bouleverse les âmes, qu’attendre de Caïus, à peine sorti de l’enfance, ignorant de toutes choses, ou nourri dans la science du mal ? Entrerait-il dans de meilleures voies, sous la conduite d’un Macron, qui, pire que Séjan et, à ce titre, choisi pour l’accabler, avait déchiré la république par plus de forfaits encore ? Déjà il voyait s’avancer un plus dur esclavage, et il fuyait à la fois le passé et l’avenir. » Après ces mots, prononcés avec un accent prophétique, il s’ouvrit les veines. La suite prouvera qu’Arruntius fit sagement de mourir. Albucilla, blessée par sa propre main d’un coup mal assuré, fut portée dans la prison par ordre du sénat. Les ministres de ses débauches furent condamnés, Carsidius Sacerdos, ancien préteur, à être déporté dans une île, Pontius Frégellanus à perdre le rang de sénateur. On prononça les mêmes peines contre Lélius Balbus, et ce fut avec plaisir : Balbus passait pour un orateur d’une éloquence farouche, toujours prête à se déchaîner contre l’innocence.\par
\labelchar{XLIX.} Pendant ces mêmes jours, Sext. Papinius, d’une famille consulaire, choisit un trépas aussi affreux que soudain ; il se précipita. La cause en fut imputée à sa mère, dont les caresses et les séductions, longtemps repoussées, avaient, disait-on, réduit enfin ce jeune homme à une épreuve d’où il ne pouvait sortir que par la mort. Accusée devant le sénat, en vain elle se jeta aux pieds de ses juges, attestant la douleur que cause à tous les hommes la perte d’un fils, douleur plus vive en un sexe plus faible ; en vain, pour augmenter la pitié, elle fit entendre de longues et déchirantes lamentations : elle n’en fut pas moins bannie de Rome pour dix ans, en attendant que le second de ses fils eût passé l’âge où les pièges sont à craindre.
\subsection[{Mort de Tibère}]{Mort de Tibère}
\noindent \labelchar{L.} Déjà le corps, déjà les forces défaillaient chez Tibère, mais non la dissimulation. C’était la même inflexibilité d’âme, la même attention sur ses paroles et ses regards, avec un mélange étudié de manières gracieuses, vains déguisements d’une visible décadence. Après avoir plusieurs fois changé de séjour, il s’arrêta enfin auprès du promontoire de Misène, dans une maison qui avait eu jadis Lucullus pour maître. C’est là, qu’on sut qu’il approchait de ses derniers instants, et voici de quelle manière. Auprès de lui était un habile médecin nommé Chariclès, qui, sans gouverner habituellement la santé du prince, lui donnait cependant ses conseils. Chariclès, quittant l’empereur sous prétexte d’affaires particulières, et lui prenant la main pour la baiser en signe de respect, lui toucha légèrement le pouls. Il fut deviné ; car Tibère, offensé peut-être et n’en cachant que mieux sa colère, fit recommencer le repas d’où l’on sortait, et le prolongea plus que de coutume, comme pour honorer le départ d’un ami. Le médecin assura toutefois à Macron que la vie s’éteignait, et que Tibère ne passerait pas deux jours. Aussitôt tout est en mouvement, des conférences se tiennent à la cour, on dépêche des courriers aux armées et aux généraux. Le dix-sept avant les calendes d’avril, Tibère eut une faiblesse, et l’on crut qu’il avait terminé ses destins. Déjà Caïus sortait, au milieu des félicitations, pour prendre possession de l’empire, lorsque tout à coup on annonce que la vue et la parole sont revenues au prince, et qu’il demande de la nourriture pour réparer son épuisement. Ce fut une consternation générale : on se disperse à la hâte ; chacun prend l’air de la tristesse ou de l’ignorance. Caïus était muet et interdit, comme tombé, d’une si haute espérance, à l’attente des dernières rigueurs. Macron, seul intrépide, fait étouffer le vieillard sous un amas de couvertures, et ordonne qu’on s’éloigne. Ainsi finit Tibère, dans la soixante-dix-huitième année de son âge.
\subsection[{Oraison funèbre}]{Oraison funèbre}
\noindent \labelchar{LI.} Il était fils de Tibérius Néro, et des deux côtés issu de la maison Claudia, quoique sa mère fût passée par adoption dans la famille des Livius, puis dans celle des Jules. II éprouva dès le berceau les caprices du sort. De l’exil, où l’avait entraîné la proscription de son père, il passa, comme beau-fils d’Auguste, dans la maison impériale. Là, de nombreux concurrents le désespérèrent, tant que dura la puissance de Marcellus, d’Agrippa, et ensuite des Césars Caïus et Lucius. Il eut même dans son frère Drusus un rival heureux de popularité. Mais sa situation ne fut jamais plus critique que lorsqu’il eut reçu Julie en mariage, forcé qu’il était de souffrir les prostitutions de sa femme ou d’en fuir le scandale. Revenu de Rhodes, il remplit douze ans le vide que la mort avait fait dans le palais du prince, et régla seul, près de vingt-trois autres années, les destins du peuple romain. Ses mœurs eurent aussi leurs époques diverses : honorable dans sa vie et sa réputation, tant qu’il fut homme privé ou qu’il commanda sous Auguste ; hypocrite et adroit à contrefaire la vertu, tant que Germanicus et Drusus virent le jour ; mêlé de bien et de mal jusqu’à la mort de sa mère ; monstre de cruauté, mais cachant ses débauches, tant qu’il aima ou craignit Séjan, il se précipita tout à la fois dans le crime et l’infamie, lorsque, libre de honte et de crainte, il ne suivit plus que le penchant de sa nature.
\section[{Livre onzième (47, 48)}]{Livre onzième (47, 48)}\renewcommand{\leftmark}{Livre onzième (47, 48)}

\bigbreak
\subsection[{L’accusateur Suilius – Attaques contre Valérius Asiaticus et son ancienne maîtresse Poppéa}]{L’accusateur Suilius – Attaques contre Valérius Asiaticus et son ancienne maîtresse Poppéa}
\noindent \labelchar{I.} Messaline crut que Valérius Asiaticus, deux fois consul, avait été autrefois l’amant de Poppéa. D’ailleurs elle convoitait ses jardins, commencés par Lucullus, et qu’il embellissait avec une rare magnificence. Elle déchaîna contre l’un et l’autre l’accusateur Suilius. Chargé de le seconder, Sosibius, précepteur de Britannicus, avertissait Claude, avec une hypocrite sollicitude, de se mettre en garde contre une audace et un crédit menaçants pour les princes ; « qu’Asiaticus, premier auteur du meurtre de Caïus \footnote{Caligula.}, n’avait pas craint d’avouer ce forfait dans l’assemblée du peuple romain, et de s’en faire une gloire criminelle ; que, depuis ce temps, son nom était célèbre dans Rome, répandu dans les provinces ; qu’il se disposait à partir pour les armées de Germanie, et qu’étant né à Vienne, et soutenu d’une parenté nombreuse et puissante, il soulèverait sans peine des peuples dont il était le compatriote. » Claude, sans rien approfondir, et comme s’il s’agissait d’étouffer une guerre naissante, envoie à la hâte Crispinus, préfet du prétoire, avec un détachement de soldats. Asiaticus fut trouvé à Baies, chargé de fers, et traîné à Rome.\par
\labelchar{II.} On ne lui permit pas de se justifier devant le sénat. Il fut entendu dans l’appartement de Claude, en présence de Messaline. Suilius le peignit comme un corrupteur des soldats, qu’il avait, disait-il, achetés au crime par ses largesses et ses impudicités. Il l’accusa ensuite d’adultère avec Poppéa ; enfin il lui reprocha de dégrader son sexe. A ce dernier outrage, sa patience vaincue lui échappe : « Interroge tes fils, dit-il à Suilius, ils avoueront que je suis un homme. » Les paroles qu’il prononça pour sa défense émurent vivement Claude, et arrachèrent des larmes à Messaline elle-même. En sortant pour les essuyer, elle avertit Vitellius de prendre garde que l’accusé n’échappât. Pour elle, tournant ses soins à la perte de Poppéa, elle aposta des traîtres, qui la poussèrent, par la peur du cachot, à se donner la mort. Ce fut tellement à l’insu du prince, que, peu de jours après, ayant reçu à sa table Scipion, mari de Poppéa, Claude lui demanda pourquoi il était venu sans sa femme. Scipion répondit qu’elle avait fini sa destinée.\par
\labelchar{III.} Claude délibéra s’il absoudrait Asiaticus. Alors Vitellius, après avoir rappelé en pleurant son ancienne amitié avec l’accusé, les respects qu’ils avaient rendus ensemble à Antonia, mère du prince puis les services d’Asiaticus envers la république, ses exploits récents contre la Bretagne, enfin tout ce qui semblait capable de lui concilier la pitié, conclut à lui laisser le choix de sa mort ; et Claude se déclara aussitôt pour la même clémence. Les amis d’Asiaticus l’exhortaient à sortir doucement de la vie en s’abstenant de nourriture : il les remercie de leur bienveillance ; puis il se livre à ses exercices accoutumés, se baigne, soupe gaiement ; et, après avoir dit qu’il eût été plus honorable de périr victime de la politique de Tibère ou des fureurs de Caïus, que des artifices d’une femme et de la langue impure de Vitellius, il se fait ouvrir les veines. Il avait auparavant visité son bûcher, et ordonné qu’on le changeât de place, de peur que l’ombrage de ses arbres ne fût endommagé par la flamme : tant il envisageait tranquillement son heure suprême !
\subsection[{Attaques contre deux chevaliers}]{Attaques contre deux chevaliers}
\noindent \labelchar{IV.} On convoque ensuite le sénat, et Suilius, continuant ses poursuites, accuse deux chevaliers romains du premier rang, surnommés Pétra. La cause de leur mort fut d’avoir prêté leur maison aux entrevues de Poppéa et d’Asiaticus. Le prétexte fut un songe où l’un d’eux avait cru voir Claude ceint d’une couronne d’épis renversés, image qu’il avait interprétée, disait-on, comme le pronostic d’une famine. Quelques-uns rapportent que la couronne était de pampres blanchissants, et que l’accusé en avait conclu que le prince mourrait au déclin de l’automne. Un point qui n’est pas douteux, c’est qu’un songe, quel qu’il soit, causa la perte des deux frères. Quinze cent mille sesterces \footnote{– 292 253 francs de notre monnaie.} et les ornements de la préture furent décernés à Crispinus. Vitellius fit ajouter un million de sesterces pour Sosibius, en récompense des services qu’il rendait à Britannicus par ses leçons, à Claude par ses conseils. Scipion ne fut pas dispensé de donner son avis. « Je pense comme tout le monde, dit-il, sur les liaisons de Poppéa ; supposez donc que je parle aussi comme tout le monde :" tempérament ingénieux entre l’amour du mari et ce que la nécessité commandait au sénateur.
\subsection[{Suicide du chevalier Samius}]{Suicide du chevalier Samius}
\noindent \labelchar{V.} Depuis ce temps Suilius continua d’accuser sans relâche ni pitié, et les imitateurs ne manquèrent pas à son audace. Le prince, en attirant à lui toute la puissance des magistrats et des lois, avait ouvert un vaste champ à la cupidité. Nulle marchandise publiquement étalée ne fut plus à vendre que la perfidie des avocats. Ainsi un chevalier romain distingué, Samius, après avoir donné à Suilius quatre cent mille sesterces, reconnut qu’il le trahissait et se perça de son épée dans la maison de ce défenseur infidèle. Cependant, à la voix de C. Silius, consul désigné, dont je raconterai en leur temps la fortune et la chute, les sénateurs se lèvent et demandent l’exécution de l’ancienne loi Cincia, qui défend de recevoir, pour plaider une cause, ni argent, ni présents.
\subsection[{La loi Cincia : les avocats ne doivent pas se faire payer}]{La loi Cincia : les avocats ne doivent pas se faire payer}
\noindent \labelchar{VI.} Ceux à la honte desquels on invoquait cette loi éclataient en murmures. Silius, ennemi personnel de Suilius, insiste avec force, rappelant l’exemple des anciens orateurs, qui regardaient l’estime de la postérité comme le plus digne salaire de l’éloquence. II ajoute que, « penser autrement, c’est profaner par un vil trafic le plus noble des arts ; qu’il n’est plus de garantie contre la trahison, quand la grandeur des profits est comptée pour quelque chose ; que, si la plaidoirie n’enrichissait personne, il y aurait moins de procès ; que les inimitiés, les accusations, les haines, les injustices, étaient encouragées par les avocats, qui trouvaient dans cette plaie du barreau, comme les médecins dans les maladies, une source de fortune ; qu’on se souvînt d’Asinius, de Messala, et, dans des temps plus voisins, d’Arruntius et d’Éserninus, qui tous étaient montés au faite des honneurs par une vie sans reproche et une éloquence désintéressée. » Ainsi parlait le consul désigné ; et, le plus grand nombre partageant son avis, on préparait un décret pour soumettre les coupables à la loi sur les concussions, lorsque Suilius, Cossutianus et d’autres, qui voyaient décréter, non leur accusation (ils étaient convaincus d’avance), mais leur châtiment, entourent le prince et implorent l’oubli du passé. Enhardis par son consentement, ils essayent de répondre.
\subsection[{Réponse des avocats}]{Réponse des avocats}
\noindent \labelchar{VII.} Ils demandent quel est l’homme assez présomptueux pour se promettre l’immortalité. Selon eux, « l’éloquence a un objet plus utile et plus réel : c’est un appui ménagé à la faiblesse, pour qu’elle ne soit pas, faute de défenseurs ; à la merci de la force. Et cependant ce talent ne s’acquiert pas sans qu’il en coûte. L’orateur néglige ses affaires pendant qu’il se dévoue à celles d’autrui. Le guerrier vit de son épée, le laboureur de sa charrue ; nul n’embrasse un état sans en avoir auparavant calculé les avantages. Asinius et Messala, enrichis par la guerre dans les querelles d’Antoine et d’Auguste, Èserninus et Arruntius, héritiers de familles opulentes, avaient pu aisément se parer de magnanimité ; mais d’autres exemples attestaient à quel prix les Clodius, les Curion, mettaient leur éloquence. Pour eux, simples sénateurs, vivant sous un gouvernement tranquille, ils n’aspiraient à rien, qu’à jouir des fruits de la paix. Que sera-ce du peuple, s’il en est dans cet ordre qui se distinguent au barreau ? Oui, c’en est fait des talents, si l’on supprime les récompenses. » Si ces réflexions étaient peu nobles, le prince ne les trouva pas sans fondement. Il fixa des bornes aux honoraires, et permit de recevoir jusqu’à dix mille sesterces \footnote{– 1943 francs de notre monnaie.}, au delà desquels on serait coupable de concussion.
\subsection[{A l’extérieur – L’Arménie et Mithridate}]{A l’extérieur – L’Arménie et Mithridate}
\noindent \labelchar{VIII.} Vers le même temps, Mithridate, ce roi d’Arménie qui fut amené devant Caïus, ainsi que je l’ai raconté, retourna dans son royaume par le conseil de Claude et sur la foi qu’il avait aux secours de Pharasmane. Celui-ci, roi d’Ibérie et frère de Mithridate, annonçait que les Parthes étaient en proie à la discorde, et que, l’empire étant divisé sur le choix d’un maître, il ne restait plus de soins pour de moindres intérêts. Las en effet des cruautés de Gotarzès, et craignant tout d’un homme qui avait immolé son propre frère Artaban, avec la femme et le fils de ce frère, les Parthes avaient appelé Bardane. Ce prince, actif et audacieux, franchit trois mille stades en deux jours, surprend Gotarzès et le chasse épouvanté. Puis, sans perdre un instant, il s’empare des provinces voisines. Seuls dans tout le pays, les Séleuciens refusaient de se soumettre. Bardane, irrité contre un peuple dont son père avait aussi éprouvé la rébellion, écoute plus la colère que la politique, et s’engage dans les longueurs d’un siège contre une ville puissante, protégée par un fleuve, munie de remparts et d’approvisionnements. Cependant Gotarzès, fortifié du secours des Dahes et des Hyrcaniens, renouvelle la guerre ; et Bardane, contraint d’abandonner Séleucie, va camper dans les plaines de la Bactriane.\par
\labelchar{IX.} Pendant que cette querelle partageait l’Orient et tenait les nations indécises, Mithridate trouva l’occasion d’envahir l’Arménie, dont la valeur romaine lui conquérait les hauteurs et les forteresses, tandis que les troupes d’Ibérie infestaient la campagne. Car les Arméniens ne résistèrent plus, après la défaite de leur gouverneur Démonax, qui avait hasardé un combat. Le succès fut un peu retardé par Cotys, roi de la petite Arménie, auquel plusieurs grands étaient venus se joindre. Une lettre de Claude suffit pour le contenir, et tout se soumit à Mithridate, qui montra une dureté peu habile au commencement d’un règne. Quant aux Parthes, les deux chefs rivaux se préparaient au combat, lorsqu’à la nouvelle d’une conspiration de leurs sujets, que Gotarzès découvrit à son frère, ils se réconcilièrent tout à coup. Ils eurent une entrevue, où, après quelques moments d’hésitation ils se donnèrent la main, et se promirent, sur les autels des dieux, de punir la perfidie de leurs ennemis et de s’accorder entre eux sur leurs prétentions. Ils jugèrent que le sceptre serait mieux placé dans les mains de Bardane ; et, pour ne pas donner d’ombrage, Gotarzès se retira au fond de l’Hyrcanie. Quand Bardane reparut, Séleucie ouvrit ses portes. C’était la septième année que durait sa révolte, non sans honte pour les Parthes, dont une seule ville avait bravé si longtemps la puissance.\par
\labelchar{X.} Bardane se saisit ensuite des provinces les plus importantes, et il se préparait à reconquérir l’Arménie, si Vibius Marsus, gouverneur de Syrie, ne l’eût arrêté en le menaçant de la guerre. De son côté, regrettant la couronne qu’il avait cédée, et rappelé par la noblesse dont la paix rend toujours l’esclavage plus dur, Gotarzès lève des troupes. Les deux rivaux se rencontrèrent près du fleuve Ërinde \footnote{Tacite est le seul. auteur qui nomme ce fleuve. Ryckius croit que c’est le même que Ptolémée place entre l’Hyrcanie et la Médie, sous le nom de Charondas.}, dont le passage fut vivement disputé. Bardane resta vainqueur, et par une suite de combats heureux, il soumit toutes les nations jusqu’au Sinde (2), qui sépare les Dahes et les Aries. Ce fut là le terme de ses succès ; car les Parthes, quoique vainqueurs, se refusaient à des guerres si lointaines. Il érigea des monuments en mémoire de sa conquête, et pour attester que nul Arsacide avant lui n’avait imposé tribut à ces nations ; puis il revint glorieux dans ses États. Mais son orgueil s’accrut avec sa gloire, et le rendit de plus en plus insupportable à ses sujets. Ils formèrent un complot contre sa vie et le tuèrent pendant qu’il se livrait sans défiance au plaisir de la chasse. Ainsi mourut Bardane à la fleur de l’âge, mais avec un nom que peu de rois vieillis sur le trône auraient surpassé, s’il eût été aussi jaloux d’être aimé de ses peuples que d’être craint de ses ennemis. Sa mort remit le trouble chez les Parthes, incertains quel nouveau maître ils se donneraient. Beaucoup penchaient pour Gotarzès ; quelques-uns pour Méherdate, descendant de Phraate, et qui était en otage à Rome. Gotarzès l’emporta. Mais une fois sur le trône, sa cruauté et ses débauches forcèrent les Parthes d’adresser à l’empereur une prière secrète pour que Méherdate leur fût rendu et vînt reprendre le sceptre de ses pères.
\subsection[{A Rome – Les jeux séculaires}]{A Rome – Les jeux séculaires}
\noindent \labelchar{XI.} Sous les mêmes consuls, huit cents ans après la fondation de Rome, soixante-quatre ans après les jeux séculaires \footnote{Les jeux séculaires furent institués suivant quelques-uns, l’an de Rome 245, après l’expulsion des rois, et, suivant d’autres, l’an 353. Un oracle sibyllin ordonnait de les célébrer tous les cent dix ans, et ils duraient trois jours et trois nuits.} d’Auguste, Claude renouvela cette solennité. Je ne dirai pas quels calculs suivirent les deux princes, je les ai fait connaître dans l’histoire de Domitien ; car cet empereur donna aussi des jeux séculaires. J’y assistai même très-exactement : j’étais revêtu alors du sacerdoce des quindécemvirs et préteur en exercice ; ce que je ne rapporte pas ici par vanité, mais parce que le soin de ces jeux appartint de tout temps au collège des quindécemvirs, et que les magistrats étaient chargés des principales cérémonies. Aux jeux du cirque, où Claude était présent, les jeunes nobles exécutaient à cheval les courses troyennes \footnote{C’est ce jeu que Virgile décrit dans l’Enéide, V, 545 et sv.}, ayant avec eux Britannicus fils du prince, et L. Domitius, qui bientôt après devint par adoption héritier de l’empire, et fut appelé Héron. Les acclamations du peuple, plus vives en faveur de Domitius, furent regardées comme un présage. On publiait aussi que des dragons avaient paru auprès de son berceau, comme pour le garder ; prétendu prodige emprunté aux fables étrangères. Héron lui-même, qui n’était pas accoutumé à se rabaisser, a souvent raconté qu’un seul serpent avait été vu dans sa chambre.
\subsection[{Messaline prend un amant}]{Messaline prend un amant}
\noindent \labelchar{XII.} Cette prédilection du peuple était un reste de son attachement à Germanicus, dont ce jeune homme était le dernier descendant mâle. L’intérêt qu’inspirait Agrippine sa mère croissait d’ailleurs avec la cruauté de Messaline. Cette implacable ennemie, plus acharnée que jamais, n’eût pas tardé à lui trouver des crimes et des accusateurs, si un amour nouveau et voisin de la frénésie ne l’eût préoccupée. Elle s’était enflammée pour C. Silius, le plus beau des Romains, d’une si violente ardeur, qu’afin de le posséder sans partage elle chassa de son lit une épouse du plus haut rang, Junia Silana. Silius ne se déguisait ni le crime ni le danger ; mais, avec la certitude de périr s’il refusait, une vague espérance de tromper Claude, et de grandes récompenses, il attendait l’avenir, jouissait du présent, et se consolait ainsi. Elle, dédaignant de se cacher, traînait chez lui tout son cortège, ne quittait pas sa maison, s’attachait partout à ses pas, lui prodiguait honneurs et richesses ; enfin, comme si déjà l’empire eût changé de mains, les esclaves du prince, ses affranchis, les ornements de son palais, étaient vus publiquement chez l’amant de sa femme
\subsection[{Parmi d’autres mesures, Claude modifie l’alphabet}]{Parmi d’autres mesures, Claude modifie l’alphabet}
\noindent \labelchar{XIII.} Claude, qui, sans voir les désordres de sa maison, exerçait les fonctions de censeur, réprima par des édits sévères la licence du théâtre, où le peuple avait outragé le consulaire Pomponius, auteur de poèmes destinés à la scène, et plusieurs femmes illustres. Il mit un frein à la cruauté des usuriers, en leur défendant de prêter aux fils de famille des sommes remboursables à la mort de leurs pères. Il détourna les eaux des monts Simbruins et les amena dans Rome. Il ajouta de nouveaux caractères à l’écriture, et les fit adopter, se fondant sur ce que l’alphabet grec n’était pas non plus sorti complet des mains de ses inventeurs.\par
\labelchar{XIV.} Les Égyptiens surent les premiers représenter la pensée avec des figures d’animaux, et les plus anciens monuments de l’esprit humain sont gravés sur la pierre. Ils s’attribuent aussi l’invention des lettres. Les Phéniciens, disent-ils, plus puissants sur mer, les portèrent dans la Grèce, et eurent le renom d’avoir trouvé ce qu’ils avaient reçu. La tradition veut en effet que Cadmus, arrivé sur une flotte de Phénicie, les ait enseignées aux Grecs encore barbares. Quelques-uns prétendent que Cécrops l’Athénien, ou Linus le Thébain, ou, au temps de la guerre de Troie, Palamède d’Argos, en inventèrent seize, et que d’autres ensuite, principalement Simonide, ajoutèrent le reste. En Italie, les Étrusques les reçurent du Corinthien Démarate, et les Aborigènes de l’Arcadies Évandre ; et l’on voit que nos lettres ont la forme des plus anciens caractères grecs. Au commencement aussi nous en eûmes peu ; le nombre fut augmenté plus tard. Claude d’après cet exemple, en ajouta trois, qui employées sous son règne et tombées depuis en désuétude, se voient encore aujourd’hui sur les tables d’airain posées dans les temples et les places pour donner à tous la connaissance des actes publics.
\subsection[{Le collège des aruspices}]{Le collège des aruspices}
\noindent \labelchar{XV.} Il appela ensuite la délibération du sénat sur le collège des aruspices. « Il ne fallait pas, disait-il, laisser périr par négligence le plus ancien des arts cultivés en Italie. Souvent, dans les calamités publiques, on y avait eu recours ; et les cérémonies sacrées, rétablies à la voix des aruspices, avaient été plus religieusement observées. Les premières familles d’Etrurie, soit d’elles-mêmes, soit par le conseil du sénat romain, avaient gardé et transmis à leurs descendants le dépôt de cette science ; zèle bien refroidi maintenant par l’indifférence du sicle pour les connaissances utiles, et par l’invasion des superstitions étrangères \footnote{Le culte de Sérapis, le judaïsme, et cette religion plus pure et plus spirituelle qui devait détrôner tous les dieux de l’empire.} Sans doute l’état présent de l’empire était florissant ; mais c’était une reconnaissance justement due à la bonté des dieux, de ne pas mettre en oubli dans la prospérité les rites, pratiqués dans les temps difficiles. » Un sénatus-consulte chargea les pontifes de juger ce qu’il fallait conserver et affermir dans l’institution des aruspices.
\subsection[{À l’extérieur – Les Chérusques}]{À l’extérieur – Les Chérusques}
\noindent \labelchar{XVI.} La même année, les Chérusques nous demandèrent un roi. Leur noblesse avait péri dans les guerres civiles, et il ne restait de la race royale qu’un seul rejeton, nommé Italicus, que l’on gardait à Rome. Ce prince avait pour père Flavius, frère d’Arminius. Sa mère était fille de Cattumère, chef des Canes. Bien fait de sa personne, il savait manier les armes et monter à cheval, à la manière de son pays aussi bien qu’à la nôtre. Claude lui donne de l’argent, des gardes, et l’exhorte à reprendre, avec une noble fierté, le rang de ses ancêtres. « Il sera le premier qui, né à Rome, et l’ayant habitée comme citoyen, non comme otage, en soit sorti pour occuper un trône étranger. » Son arrivée fit d’abord la joie de la nation, d’autant plus que, n’étant prévenu d’aucun esprit de parti, il régnait avec une impartiale équité. On se pressait à sa cour ; il était entouré de respects ; tantôt se montrant affable et tempérant, ce qui ne déplaît à personne, plus souvent encore se livrant au vin et aux autres excès, ce qui plaît tant aux barbares. Déjà les contrées limitrophes, déjà même les pays éloignés retentissaient de son nom, lorsque, jaloux de sa puissance, les factieux qui avaient brillé à la faveur des troubles se retirent chez les peuples voisins et les prennent à témoin que c’en est fait de l’antique liberté des Germains, et que Rome triomphe. « Leur patrie n’avait donc pas enfanté un homme qui fût digne du rang suprême ! Il fallait que le fils d’un espion, d’un Flavius, fût imposé à tant de braves ! En vain on invoquait le nom d’Arminius : ah ! le propre fils de ce héros, nourri dans une terre ennemie \footnote{Le fils d’Arminius avait été réellement élevé à Ravenne.}, vint-il pour régner, on devait craindre un homme que l’éducation, la servitude, le genre de vie, enfin tout eût infecté des poisons de l’étranger. Mais si c’était l’esprit de son père qu’Italicus apportait sur le trône, qui fut, plus que ce père, l’implacable ennemi de sa patrie et de ses dieux domestiques ? »\par
\labelchar{XVII.} A la faveur de tels discours, ils parvinrent à rassembler des forces nombreuses. Italicus n’avait pas moins de partisans ; « car enfin, disaient-ils, ce n’était pas de vive force qu’il était entré dans le pays : on l’avait appelé. Puisqu’il était le premier par la naissance, il fallait au moins éprouver sa valeur, et voir s’il serait digne de son oncle Arminius, de son aïeul Cattumère. Pourquoi rougirait-on de son père ? Engagé envers Rome par la volonté des Germains, qu’avait-il fait que de garder sa foi ? La liberté n’était qu’un vain prétexte dont se couvraient des factieux qui, nés pour être la honte de leur famille et le fléau de leur patrie, n’avaient d’espoir que dans la discorde. » La multitude applaudissait avec transport à ces discours. Un grand combat fut livré entre les barbares, et le roi demeura vainqueur. Bientôt l’orgueil de la bonne fortune en fit un tyran. Chassé par les siens, rétabli par le secours des Lombards, ses succès et ses revers affaiblirent également la puissance des Chérusques.
\subsection[{Les Cauques}]{Les Cauques}
\noindent \labelchar{XVIII.} Dans le même temps, les Cauques, exempts de troubles intérieurs et enhardis par la mort de Sanquinius, firent, avant l’arrivée de Corbulon, des incursions dans la basse Germanie. Leur chef était Gannascus, Canninéfate d’origine, qui, longtemps auxiliaire dans nos armées, ensuite transfuge, exerçait la piraterie avec des vaisseaux légers, et ravageait surtout les côtes habitées par les Gaulois, qu’il savait riches et peu guerriers. Mais Corbulon entra dans la province. Là, déployant un zèle payé bientôt par la gloire (car cette campagne fut le commencement de sa renommée), il fit descendre les galères par le Rhin, et les autres navires par les canaux et les lacs, suivant la navigation à laquelle ils étaient propres. Après avoir submergé les barques ennemies et chassé au loin Gannascus, voyant la paix suffisamment rétablie, il s’occupa de ramener à l’ancienne discipline les légions, qui ne connaissaient plus les travaux ni la fatigue, et ne prenaient plaisir qu’au pillage. Il défendit de s’écarter dans les marches, de combattre sans son ordre. Garde, veilles, service de jour et de nuit, désormais tout se fit en armes. Deux soldats furent, dit-on, punis de mort pour avoir travaillé aux retranchements, l’un sans être armé, l’autre armé seulement du poignard : exemples d’une rigueur excessive, controuvés peut-être, mais dont le récit n’en atteste pas moins la sévérité du général. Certes il dut être ferme et inexorable pour les grandes fautes, celui qu’on supposait si rigoureux pour les plus légères.\par
\labelchar{XIX.} Au reste, la terreur de cette discipline produisit sur nous et sur l’ennemi deux effets opposés : elle accrut le courage des Romains ; elle brisa l’orgueil des barbares. Les Frisons, toujours rebelles ou prêts à le devenir depuis la révolte qui avait commencé par la défaite d’Apronius, donnèrent des otages et se renfermèrent dans le territoire qui leur fut assigné par Corbulon. Ce général établit chez eux un sénat, des magistrats, des lois ; et, pour s’assurer de leur obéissance, il éleva une forteresse dans le pays. En même temps, il envoyait des émissaires chez les grands Cauques pour les solliciter à se soumettre, et pour tramer secrètement la perte de Gannascus. La ruse fut employée avec, succès et sans honte contre un déserteur et un parjure. Mais la mort de ce chef irrita les Cauques ; et Corbulon jetait parmi eux des semences de révolte. Aussi, reçues par le plus grand nombre avec enthousiasme, ces nouvelles inspiraient à quelques-uns des réflexions sinistres : « Pourquoi provoquer l’ennemi ? les revers pèseront sur la république, les succès sur leur auteur. La paix s’effraye des noms éclatants, et un prince sans courage en est importuné. » Claude empêcha si bien toute entreprise nouvelle contre la Germanie, qu’il fit ramener les garnisons en deçà du Rhin.\par
\labelchar{XX.} Corbulon campait déjà sur le territoire ennemi, lorsqu’il reçut cet ordre. A ce coup soudain, l’esprit combattu de mille pensées diverses, et craignant tout ensemble la colère de l’empereur, le mépris des barbares, les railleries des alliés, il ne prononça pourtant que ce peu de mots : « Heureux autrefois les généraux romains ! » et il donna le signal de la retraite. Toutefois, pour arracher les soldats à l’oisiveté, il fit creuser entre la Meuse et le Rhin un canal de vingt-trois milles, destiné à donner une issue aux débordements de l’océan. Claude, qui lui avait refusé l’occasion de vaincre, lui accorda cependant les ornements du triomphe. Bientôt après, Curtius Rufus obtint le même honneur pour avoir fait ouvrir une mine d’argent dans le pays de Mattium \footnote{Dans la Germanie, au delà du Rhin.}, entreprise qui rapporta peu et pendant peu de temps, mais qui coûta cher aux légions, condamnées à creuser des tranchées souterraines et à faire dans ces abîmes des travaux déjà pénibles à la clarté des cieux. Rebutés de tant de fatigues, et voyant qu’on endurait les mêmes maux dans d’autres provinces, les soldats composèrent secrètement une lettre, par laquelle l’empereur était prié, au nom des armées, d’accorder d’avance aux généraux qu’il nommerait les décorations triomphales.
\subsection[{À Rome – Curtius Rufus}]{À Rome – Curtius Rufus}
\noindent \labelchar{XXI.} Je me tairai sur l’origine de Curtius Rufus, que quelques-uns font naître d’un gladiateur. Je craindrais de répéter des mensonges ; et le vrai même, j’ai quelque honte à le dire. Au sortir de l’adolescence, ayant suivi notre questeur en Afrique, il se trouvait dans la ville d’Adrumète. C’était vers le milieu du jour, et il se promenait seul sous les portiques déserts, lorsqu’une femme d’une taille plus qu’humaine apparut à ses yeux, et lui dit ces paroles : « C’est toi, Rufus, qui viendras un jour dans cette province comme proconsul. » Cette prédiction enfle ses espérances. Il retourne à Rome, et, par la faveur de ses amis et sa propre activité, il parvient à la questure. Bientôt, préféré aux plus nobles candidats, il est créé préteur par le suffrage du prince. Tibère, pour voiler la bassesse de sa naissance, dit que Curtius Rufus était le fils de ses œuvres. Il parvint depuis à une longue vieillesse ; et, tristement servile auprès des grands, hautain envers les petits, capricieux avec ses égaux, il obtint le consulat, les ornements du triomphe, et enfin le gouvernement de l’Afrique. Il acheva, en y mourant, de vérifier l’annonce de sa destinée.
\subsection[{Problème de la questure}]{Problème de la questure}
\noindent \labelchar{XXII.} Cependant, à Rome, un chevalier nommé Cn. Novius, sans motif connu alors ou qu’on ait pu découvrir depuis, fut trouvé avec un poignard dans la foule de ceux qui venaient saluer le prince. Déchiré par la torture, il s’avoua coupable, sans révéler de complices ; on ignore s’il en avait dont il cachât les noms. Sous les mêmes consuls, P. Dolabella proposa qu’il fût donné tous les ans un spectacle de gladiateurs aux frais de ceux qui obtiendraient la questure. Du temps de nos ancêtres, cette dignité était le prix de la vertu, et tout citoyen qui se sentait du mérite pouvait demander les magistratures. L’âge même n’était pas fixé, et rien n’empêchait que, dès la première jeunesse, on ne fût consul ou dictateur. La questure fut instituée sous les rois, comme on le voit par la loi curiate \footnote{On appelait loi curiate l’acte par lequel le peuple, assemblé en curies, confirmait un testament ou une adoption, et celui par lequel il attribuait aux magistrats le commandement militaire, \emph{imperium} ; acte sans lequel ils ne possédaient que l’autorité civile, \emph{potestas}. Il s’agit ici de la loi qui réglait le pouvoir des rois et qui était renouvelée à chaque règne ; Brutus la renouvela aussi, afin de conférer aux consuls les mêmes pouvoirs qu’avaient eus les rois auxquels ils succédaient.} que Brutus renouvela. Le droit d’élire à cette charge demeura aux consuls, jusqu’aux temps où le peuple la conféra comme les autres honneurs. Les premiers questeurs qu’il nomma furent Valérius Potitus et Marhercus Émilius, soixante-trois ans après l’expulsion des Tarquins ; ils devaient accompagner les généraux à la guerre. Les affaires se multipliant chaque jour, on en ajouta deux pour la ville. Le nombre en fut doublé lorsqu’aux tributs que payait déjà l’Italie se joignirent les revenus des provinces. Sylla, par une loi, le porta jusqu’à vingt, afin qu’ils servissent à recruter le sénat, auquel il avait attribué les jugements \footnote{L’an de Rome 672, Sylla rendit une loi qui admettait les seuls sénateurs à siéger comme juges dans les tribunaux.}. Plus tard, les jugements furent rendus aux chevaliers. Mais toujours la questure, qu’elle fût obtenue par le mérite ou accordée par la faveur, était donnée gratuitement, jusqu’à l’époque où, sur la proposition de Dolabella, on commença de la vendre.
\subsection[{Des Gaulois veulent devenir sénateurs}]{Des Gaulois veulent devenir sénateurs}
\noindent \labelchar{XXIII.} Sous le consulat d’Aulus Vitellius et de L. Vipstanus, il fut question de compléter le sénat. Les principaux habitants de la Gaule chevelue \footnote{On appelait ainsi la Gaule transalpine, à cause de l’usage où étaient les habitants de porter les cheveux longs. La Gaule cisalpine était nommée \emph{togata}, parce qu’on y avait adopté la toge romaine.}, qui depuis longtemps avaient obtenu des traités et le titre de citoyens, désiraient avoir dans Rome le droit de parvenir aux honneurs. Cette demande excita de vives discussions et fut débattue avec chaleur devant le prince. On soutenait « que l’Italie n’était pas assez épuisée pour ne pouvoir fournir un sénat à sa capitale. Les seuls enfants de Rome, avec les peuples de son sang, y suffisaient jadis ; et certes on n’avait pas à rougir de l’ancienne république : on citait encore les prodiges de gloire et de vertu qui, sous ces mœurs antiques, avaient illustré le caractère romain. Était-ce donc peu que des Vénètes et des Insubriens eussent fait irruption dans le sénat ; et fallait-il y faire entrer en quelque sorte la captivité elle-même avec cette foule d’étrangers ? A quels honneurs pourraient désormais prétendre ce qui restait de nobles et les sénateurs pauvres du Latium ? Ils allaient tout envahir, ces riches dont les aïeuls et les bisaïeuls, à la tête des nations ennemies, avaient massacré nos légions, assiégé le grand César auprès d’Alise. Ces injures étaient récentes : que serait-ce si on se rappelait le Capitole et la citadelle presque renversés par les mains de ces mêmes Gaulois ? Qu’ils jouissent, après cela, du nom de citoyens ; mais les décorations sénatoriales, mais les ornements des magistratures, qu’ils ne fussent pas ainsi prostitués. »
\subsection[{Discours de Claude sur l’admission d’étrangers au sénat}]{Discours de Claude sur l’admission d’étrangers au sénat}
\noindent \labelchar{XXIV.} Le prince fut peu touché de ces raisons. Il y répondit sur-le-champ ; et, après avoir convoqué le sénat, il les combattit encore par ce discours \footnote{Le discours même de Claude existe presque entier, gravé sur des tables de bronze découvertes à Lyon en 1528, et que l’on conserve dans cette ville.} : « Mes ancêtres, dont le plus ancien, Clausus, né parmi les Sabins, reçut tout à la fois et le droit de cité romaine et le titre de patricien, semblent m’exhorter à suivre la même politique en transportant ici tout ce qu’il y a d’illustre dans les autres pays. Je ne puis ignorer qu’Albe nous a donné les Jules, Camérie les Coruncanius, Tusculum les Porcius, et, sans remonter si haut, que l’Étrurie, la Lucanie, l’Italie entière, ont fourni des sénateurs. Enfin, en reculant jusqu’aux Alpes les bornes de cette contrée, ce ne sont plus seulement des hommes, mais des nations et de vastes territoires que Rome a voulu associer à son nom. La paix intérieure fut assurée, et notre puissance affermie au dehors, quand les peuples d’au delà du Pô firent partie de la cité, quand la distribution de nos légions dans tout l’univers eut servi de prétexte pour y admettre les meilleurs guerriers des provinces, et remédier ainsi à l’épuisement de l’empire. Est-on fâché que les Balbus soient venus d’Espagne, et d’autres familles non moins illustres, de la Gaule narbonnaise ? Leurs descendants sont parmi nous, et leur amour pour cette patrie ne le cède point au nôtre. Pourquoi Lacédémone et Athènes, si puissantes par les armes, ont-elles péri, si ce n’est pour avoir repoussé les vaincus comme des étrangers ? Honneur à la sagesse de Romulus notre fondateur, qui tant de fois vit ses voisins en un seul jour ennemis et citoyens ! Des étrangers ont régné sur nous. Des fils d’affranchis obtiennent les magistratures : et ce n’est point une innovation, comme on le croit faussement ; l’ancienne république en a vu de nombreux exemples. Nous avons combattu, dit-on, avec les Sénonais. Jamais sans doute les Èques et les Volsques ne rangèrent contre nous une armée en bataille ! Nous avons été pris par les Gaulois. Mais nous avons donné des otages aux Étrusques, et nous avons passé sous le joug des Samnites. Et cependant rappelons-nous toutes les guerres ; aucune ne fut plus promptement terminée que celle des Gaulois, et rien n’a depuis altéré la paix. Déjà les mœurs, les arts, les alliances, les confondent avec nous ; qu’ils nous apportent aussi leurs richesses, et leur or, plutôt que d’en jouir seuls. Pères conscrits, les plus anciennes institutions furent nouvelles autrefois. Le peuple fut admis aux magistratures après les patriciens, les Latins après le peuple, les autres nations d’Italie après les Latins. Notre décret vieillira comme le reste, et ce que nous justifions aujourd’hui par des exemples servira d’exemple à son tour. »
\subsection[{On admet des Gaulois comme sénateurs}]{On admet des Gaulois comme sénateurs}
\noindent \labelchar{XXV.} Un sénatus-consulte fut rendu sur le discours du prince, et les Éduens reçurent les premiers le droit de siéger dans le sénat. Cette distinction fut accordée à l’ancienneté de leur alliance, et au nom de frères des Romains, qu’ils prennent seuls parmi tous les Gaulois. A la même époque, le prince éleva les sénateurs les plus anciens, ou dont les pères s’étaient le plus illustrés, à la dignité de patriciens. Il restait peu des familles patriciennes de première et de seconde création, instituées par Romulus et par Brutus ; et celles qu’avaient ajoutées le dictateur César par la loi Cassia, l’empereur Auguste par la loi Sénia, étaient aussi presque éteintes. Cette partie des fonctions de la censure avait quelque chose de populaire, et Claude s’en acquittait avec joie. Plus inquiet sur les moyens de purger le sénat des membres déshonorés, il eut recours à un tempérament doux et nouvellement imaginé, plutôt qu’à la sévérité des anciens temps. Il dit que c’était à chacun d’interroger sa conscience et de demander à n’être plus sénateur ; que cette faculté s’obtiendrait sans peine, et qu’il présenterait les expulsions sans les distinguer des retraites volontaires, afin que la justice des censeurs, confondue avec celle qu’on se ferait à soi-même, en devint moins flétrissante. A cette occasion, le consul Vipstanus proposa de décerner à Claude le nom de père du sénat. Celui de père de la patrie était à son gré devenu trop vulgaire, et des services nouveaux voulaient être honorés par des titres nouveaux. Mais le prince arrêta lui-même le zèle du consul : il trouva que c’était pousser trop loin la flatterie. Il fit la clôture du lustre, où l’on compta six millions neuf cent quarante-quatre mille citoyens. C’est vers ce temps qu’il cessa d’ignorer la honte de sa maison : il fut forcé d’ouvrir les yeux et de punir les débordements de sa femme, en attendant qu’un autre mariage mît l’inceste en son lit.
\subsection[{Messaline – Débordements de Messaline}]{Messaline – Débordements de Messaline}
\noindent \labelchar{XXVI.} Dégoûtée de l’adultère, dont la facilité émoussait le plaisir, déjà Messaline courait à des voluptés inconnues, lorsque de son côté Silius, poussé par un délire fatal, ou cherchant dans le péril même un remède contre le péril, la pressa de renoncer à la dissimulation. « Ils n’en étaient pas venus à ce point, lui disait-il, pour attendre que le prince mourût de vieillesse : l’innocence pouvait se passer de complots ; mais le crime, et le crime public, n’avait de ressource que dans l’audace. Des craintes communes leur assuraient des complices ; lui-même, sans femme, sans enfants, offrait d’adopter Britannicus en épousant Messaline : elle ne perdrait rien de son pouvoir, et elle gagnerait de la sécurité, s’ils prévenaient Claude, aussi prompt à s’irriter que facile à surprendre."Elle reçut froidement cette proposition, non par attachement à son mari, mais dans la crainte que Silius, parvenu au rang suprême, ne méprisât une femme adultère, et, après avoir approuvé le forfait au temps du danger, ne le payât bientôt du prix qu’il méritait. Toutefois le nom d’épouse irrita ses désirs, à cause de la grandeur du scandale, dernier plaisir pour ceux qui ont abusé de tous les autres. Elle n’attendit que le départ de Claude, qui allait à Ostie pour un sacrifice, et elle célébra son mariage avec toutes les solennités ordinaires.
\subsection[{Messaline se marie !}]{Messaline se marie !}
\noindent \labelchar{XXVI.} Sans doute il paraîtra fabuleux que, dans une ville qui sait tout et ne tait rien, l’insouciance du péril ait pu aller à ce point chez aucun mortel, et, à plus forte raison, qu’un consul désigné ait contracté avec la femme du prince, à un jour marqué, devant des témoins appelés pour sceller un tel acte, l’union destinée à perpétuer les familles ; que cette femme ait entendu les paroles des auspices, reçu le voile nuptial, sacrifié aux dieux, pris place à une table entourée de convives ; qu’ensuite soient venus les baisers, les embrassements, la nuit, enfin, passée entre eux dans toutes les libertés de l’hymen. Cependant je ne donne rien à l’amour du merveilleux : les faits que je raconte, je les ai entendus de la bouche de nos vieillards ou lus dans les écrits du temps.\par
\labelchar{XXVIII.} A cette scène, la maison du prince avait frémi d’horreur. On entendait surtout ceux qui, possédant le pouvoir, avaient le plus à craindre d’une révolution, exhaler leur colère, non plus en murmures secrets, mais hautement et à découvert. « Au moins, disaient-ils, quand un histrion \footnote{« A s’en tenir à l’étymologie, dit Tite Live, I, XLIV, le mot \emph{pomoerium} signifie ce qui est derrière les murs ; mais on l’emploie pour désigner cet espace vide que les Étrusques consacraient en bâtissant une ville, et qui régnait tout à l’entour, tant en dedans qu’en dehors. Il n’était permis ni de le cultiver ni d’y bâtir. »} foulait insolemment la couche impériale, s’il outrageait le prince, il ne le détrônait pas. Mais un jeune patricien, distingué par la noblesse de ses traits, la force de son esprit, et qui bientôt sera consul, nourrit assurément de plus hautes espérances. Eh ! qui ne voit trop quel pas reste à faire après un tel mariage ? " Toutefois ils sentaient quelques alarmes en songeant à la stupidité de Claude, esclave de sa femme, et aux meurtres sans nombre commandés par Messaline. D’un autre côté, la faiblesse même du prince les rassurait : s’ils la subjuguaient une fois par le récit d’un crime si énorme, il était possible que Messaline fût condamnée et punie avant d’être jugée. Le point important était que sa défense ne fût pas entendue, et que les oreilles de Claude fussent fermées même à ses aveux.
\subsection[{Comment apprendre à Claude que sa femme s’est mariée en son absence ?}]{Comment apprendre à Claude que sa femme s’est mariée en son absence ?}
\noindent \labelchar{XXIX.} D’abord Calliste, dont j’ai parlé à l’occasion du meurtre de Caius, Narcisse, instrument de celui d’Appius \footnote{Messaline avait jeté un œil incestueux sur Appius Silanus, mari de sa mère. Repoussée comme elle devait l’être elle résolut de se venger.}, et Pallas, qui était alors au plus haut période de sa faveur, délibérèrent si, par de secrètes menaces, ils n’arracheraient pas Messaline à son amour pour Silius, en taisant d’ailleurs tout le reste. Ensuite, dans la crainte de se perdre eux-mêmes, Pallas et Calliste abandonnèrent l’entreprise, Pallas par lâcheté, Calliste par prudence : il avait appris à l’ancienne cour que l’adresse réussit mieux que la vigueur, à qui veut maintenir son crédit. Narcisse persista ; seulement il eut la précaution de ne pas dire un mot qui fit pressentir à Messaline l’accusation ni l’accusateur, et il épia les occasions. Comme le prince tardait à revenir d’Ostie, il s’assure de deux courtisanes qui servaient habituellement à ses plaisirs ; et, joignant aux largesses et aux promesses l’espérance d’un plus grand pouvoir quand il n’y aurait plus d’épouse, il les détermine à se charger de la délation.
\subsection[{Claude apprend qu’il a été “répudié”}]{Claude apprend qu’il a été “répudié”}
\noindent \labelchar{XXX.} Calpurnie (c’était le nom d’une de ces femmes), admise à l’audience secrète du prince, tombe à ses genoux, et s’écrie que Messaline est mariée à Silius. Puis elle s’adresse à Cléopatre, qui, debout près de là, n’attendait que cette question, et lui demande si elle en est instruite. Sur sa réponse qu’elle le sait, Calpurnie conjure l’empereur d’appeler Narcisse. Celui-ci, implorant l’oubli du passé et le pardon du silence qu’il a gardé sur les Titius, les Vectius, les Plautius, déclare « qu’il ne vient pas même en ce moment dénoncer des adultères, ni engager le prince à redemander sa maison, ses esclaves, tous les ornements de sa grandeur ; ah ! plutôt, que le ravisseur jouît des biens, mais qu’il rendît l’épouse, et qu’il déchirât l’acte de son mariage. Sais-tu, César, que tu es répudié ? Le peuple, le sénat, l’armée, ont vu les noces de Silius, et, si tu ne te hâtes, le mari de Messaline est maître de Rome. »
\subsection[{Messaline en Bacchante}]{Messaline en Bacchante}
\noindent \labelchar{XXXI.} Alors Claude appelle les principaux de ses amis ; et d’abord il interroge le préfet des vivres Turranius, ensuite Lusius Géta, commandant du prétoire. Enhardis par leur déposition, tous ceux qui environnaient le prince lui crient à l’envi qu’il faut aller au camp, s’assurer des cohortes prétoriennes, pourvoir à sa sûreté avant de songer à la vengeance. C’est un fait assez constant que Claude, dans la frayeur dont son âme était bouleversée, demanda plusieurs fois lequel de lui ou de Silius était empereur ou simple particulier. On était alors au milieu de l’automne : Messaline, plus dissolue et plus abandonnée que jamais, donnait dans sa maison, un simulacre de vendanges. On eût vu serrer les pressoirs, les cuves se remplir ; des femmes vêtues de peaux bondir comme les bacchantes dans leurs sacrifices ou dans les transports de leur délire ; Messaline échevelée, secouant un thyrse, et près d’elle Silius couronné de lierre, tous deux chaussés du cothurne, agitant la tète au bruit d’un chœur lascif et tumultueux. On dit que, par une saillie de débauche, Vectius Valens étant monté sur un arbre très-haut quelqu’un lui demanda ce qu’il voyait, et qu’il répondit : « Un orage furieux du côté d’Ostie ;" soit qu’un orage s’élevât en effet, ou qu’une parole jetée au hasard soit devenue le présage de l’événement.
\subsection[{Messaline sent le vent tourner}]{Messaline sent le vent tourner}
\noindent \labelchar{XXXII.} Cependant ce n’est plus un bruit vague, mais des courriers arrivant de divers côtés, qui annoncent que Claude, instruit de tout, accourt pour se venger. Messaline se retire aussitôt dans les jardins de Lucullus ; Silius, pour déguiser ses craintes, alla vaquer aux affaires du Forum. Comme les autres se dispersaient à la hâte, des centurions surviennent et les chargent de chaînes, à mesurent qu’ils les trouvent dans les rues ou les découvrent dans leurs retraites. Messaline, malgré le trouble où la jette ce revers de fortune, prend la résolution hardie, et qui l’avait sauvée plus d’une fois, d’aller au-devant de son époux et de s’en faire voir. Elle ordonne à Britannicus et à Octavie de courir dans les bras de leur père, et elle prie Vibidia, la plus ancienne des vestales, de faire entendre sa voix au souverain pontife et d’implorer sa clémence. Elle-même, accompagnée en tout de trois personnes (telle est la solitude qu’un instant avait faite), traverse à pied toute la ville, et, montant sur un de ces chars grossiers dans lesquels on emporte les immondices des jardins, elle prend la route d’Ostie : spectacle qu’on vit sans la plaindre, tant l’horreur de ses crimes étouffait la pitié.
\subsection[{Inquiètude de Claude}]{Inquiètude de Claude}
\noindent \labelchar{XXXIII.} L’alarme n’était pas moindre du côté de César. Il se fiait peu au préfet Géta, esprit léger, aussi capable de mal que de bien. Narcisse, d’accord avec ceux qui partageaient ses craintes, déclare que l’unique salut de l’empereur est de remettre, pour ce jour-là seul, le commandement des soldats à l’un de ses affranchis, et il offre de s’en charger ; puis, craignant que, sur la route, les dispositions de Claude ne soient changées par L. Vitellius et Largus Cécina, il demande et prend une place dans la voiture qui les portait tous trois.
\subsection[{Rencontre de Claude et de Messaline}]{Rencontre de Claude et de Messaline}
\noindent \labelchar{XXXIV.} On a souvent raconté, depuis, qu’au milieu des exclamations contradictoires du prince, qui tantôt accusait les dérèglements de sa femme, tantôt s’attendrissait au souvenir de leur union et du bas âge de leurs enfants, Vitellius ne dit jamais que ces deux mots : « O crime ! ô forfait ! » En vain Narcisse le pressa d’expliquer cette énigme et d’énoncer franchement sa pensée, il n’en put arracher que des réponses ambiguës et susceptibles de se prêter au sens qu’on y voudrait donner. L’exemple de Vitellius fut suivi par Cécina. Déjà cependant Messaline paraissait de loin, conjurant le prince à cris redoublés d’entendre la mère d’Octavie et de Britannicus ; mais l’accusateur couvrait sa voix en rappelant Silius et son mariage. En même temps, pour distraire les yeux de Claude, il lui remit un mémoire où étaient retracées les débauches de sa femme. Quelques moments après, comme le prince entrait dans la ville, on voulut présenter à sa vue leurs communs enfants ; mais Narcisse ordonna qu’on les fit retirer. Il ne réussit pas à écarter Vibidia, qui demandait, avec une amère énergie, qu’une épouse ne fût pas livrée à la mort sans avoir pu se défendre. Narcisse répondit que le prince l’entendrait, et qu’il lui serait permis de se justifier ; qu’en attendant la vestale pouvait retourner à ses pieuses fonctions.
\subsection[{Mise à mort du “mari” et de ses complices}]{Mise à mort du “mari” et de ses complices}
\noindent \labelchar{XXXV.} Claude gardait un silence étrange en de pareils moments. Vitellius semblait ne rien savoir. Tout obéissait à l’affranchi. Narcisse fait ouvrir la maison du coupable et y mène l’empereur. Dès le vestibule, il lui montre l’image de Silius le père, conservée au mépris d’un sénatus-consulte ; puis toutes les richesses des Nérons et des Drusus, devenues le prix de l’adultère. Enfin, voyant que sa colore allumée éclatait en menaces, il le transporte au camp, où l’on tenait déjà les soldats assemblés. Claude, inspiré par Narcisse, les harangue en peu de mots ; car son indignation, quoique juste, était honteuse de se produire. Un long cri de fureur part aussitôt des cohortes elles demandent le nom des coupables et leur punition. Amené devant le tribunal, Silius, sans chercher à se défendre ou à gagner du temps, pria qu’on hâtât sa mort. La même fermeté fit désirer un prompt trépas à plusieurs chevaliers romains d’un rang illustre. Titius Proculus, auquel Silius avait confié la garde de Messaline, Vectius Valens, qui avouait tout et offrait des révélations, deux complices, Pompéius Urbicus et Saufellus Trogus, furent traînés au supplice par l’ordre de Claude. Décius Calpurnianus, préfet des gardes nocturnes, Sulpicius Rufus, intendant des jeux, et le sénateur Jutions Virgilianus, subirent la même peine.\par
\labelchar{XXXVI.} Le seul Mnester donna lieu à quelque hésitation. II criait au prince, en déchirant ses vêtements, « de regarder sur son corps les traces des verges ; de se souvenir du commandement exprès par lequel lui-même l’avait soumis aux volontés de Messaline ; que ce n’était point, comme d’autres, l’intérêt ou l’ambition, mais la nécessité, qui l’avait fait coupable ; qu’il eût péri le premier, si l’empire frit tombé aux mains de Silius » Ému par ces paroles, Claude penchait vers la pitié. Ses affranchis lui persuadèrent qu’après avoir immolé de si grandes victimes on ne devait pas épargner un histrion ; que, volontaire ou forcé, l’attentat n’en était pas moins énorme. On n’admit pas même la justification du chevalier romain Traulus Montanus. C’était un jeune homme de murs honnêtes, mais d’une beauté remarquable, que Messaline avait appelé chez elle et chassé dès la première nuit, aussi capricieuse dans ses dégoûts que dans ses fantaisies. On fit grâce de la vie à Suilius Césoninus et à Plautius Latéranus. Ce dernier dut son salut aux services signalés de son oncle. Césoninus fut protégé par ses vices : il avait joué le rôle de femme dans cette abominable fête.
\subsection[{Narcisse ordonne la mort de Messaline}]{Narcisse ordonne la mort de Messaline}
\noindent \labelchar{XXXVII.} Cependant Messaline, retirée dans les jardins de Lucullus, cherchait à prolonger sa vie et dressait une requête suppliante, non sans un reste d’espérance, et avec des retours de colère ; tant elle avait conservé d’orgueil en cet extrême danger. Si Narcisse n’eût hâté sa mort, le coup retombait sur l’accusateur. Claude, rentré dans son palais, et charmé par les délices d’un repas dont on avança l’heure, n’eut pas plus tôt les sens échauffés par le vin, qu’il ordonna qu’on allât dire à la malheureuse Messaline (c’est, dit-on, le terme qu’il employa) de venir le lendemain pour se justifier. Ces paroles firent comprendre que la colère refroidie faisait place à l’amour ; et, en différant, on redoutait la nuit et le souvenir du lit conjugal. Narcisse sort brusquement, et signifie aux centurions et au tribun de garde d’aller tuer Messaline ; que tel est l’ordre de l’empereur. L’affranchi Évodus fut chargé de les surveiller et de presser l’exécution. Évodus court aux jardins, et, arrivé le premier, il trouve Messaline étendue par terre, et Lépida, sa mère, assise auprès d’elle. Le cœur de Lépida, fermé à sa fille tant que celle-ci fut heureuse, avait été vaincu par la pitié en ces moments suprêmes. Elle lui conseillait de ne pas attendre le fer d’un meurtrier, ajoutant que la vie avait passé pour elle, et qu’il ne lui restait plus qu’à honorer sa mort. Mais cette âme, corrompue par la débauche, était incapable d’un effort généreux. Elle s’abandonnait aux larmes et à des plaintes inutiles, quand les satellites forcèrent tout à coup la porte. Le tribun se présente en silence ; l’affranchi, avec toute la bassesse d’un esclave, se répand en injures.
\subsection[{Mort de Messaline}]{Mort de Messaline}
\noindent \labelchar{XXXVIII.} Alors, pour la première fois, Messaline comprit sa destinée. Elle accepta un poignard, et, pendant que sa main tremblante l’approchait vainement de sa gorge et de son sein, le tribun la perça d’un coup d’épée. Sa mère obtint que son corps lui fût remis. Claude était encore à table quand on lui annonça que Messaline était morte, sans dire si c’était de sa main ou de celle d’un autre. Le prince, au lieu de s’en informer, demande à boire et achève tranquillement son repas. Même insensibilité les jours qui suivirent : il vit, sans donner un signe de haine ni de satisfaction, de colère ni de tristesse, et la joie des accusateurs, et les larmes de ses enfants. Le sénat contribua encore à effacer Messaline de sa mémoire, en ordonnant que son nom et ses images fussent ôtés de tous les lieux publics et particuliers. Narcisse reçut les ornements de la questure, faible accessoire d’une fortune qui surpassait celle de Calliste et de Pallas. Ainsi fut consommée une vengeance, juste sans doute, mais qui eut des suites affreuses, et ne fit que changer la scène de douleur qui affligeait l’empire.
\section[{Livre douzième (49, 54)}]{Livre douzième (49, 54)}\renewcommand{\leftmark}{Livre douzième (49, 54)}

\subsection[{Suites du meurtre de Messaline – Une nouvelle épouse pour Claude}]{Suites du meurtre de Messaline – Une nouvelle épouse pour Claude}
\noindent \labelchar{I.} Le meurtre de Messaline bouleversa le palais. Les affranchis se disputaient à qui choisirait une épouse à Claude, impatient du célibat, et mari toujours dépendant. L’ambition des femmes n’était pas moins ardente. Naissance, beauté, richesse, elles faisaient tout valoir, et chacune étalait ses titres à un si noble hymen. Mais le choix indécis flottait surtout entre Lollia Paullina, fille du consulaire M. Lollius, et Agrippine, dont Germanicus était le père. Celle-ci avait Pallas pour appui ; l’autre était soutenue par Calliste. Narcisse protégeait une troisième rivale, Elia Pétina, de la famille des Tubérons. Le prince penchait tantôt pour l’une, tantôt pour l’autre, suivant la dernière impression qu’il avait reçue. Voyant enfin qu’ils ne pouvaient s’accorder, il les réunit en conseil, enjoignant à chacun de dire son avis et de le motiver.
\subsection[{On présente Agrippine}]{On présente Agrippine}
\noindent \labelchar{II.} Narcisse alléguait en faveur de Pétina son ancien mariage avec le prince et le gage qui en restait (car Antonia était sa fille), ajoutant que le palais ne s’apercevrait d’aucun changement au retour d’une épouse déjà connue, qui certes n’aurait point pour Britannicus et Octavie, liés si étroitement à son propre sang, la haine d’une marâtre. Calliste soutint qu’un long divorce l’avait condamnée pour toujours, et ne ferait qu’enfler son orgueil si elle rentrait au palais ; qu’il valait beaucoup mieux y appeler Lollia, qui, sans enfants, et par conséquent sans jalousie ; servirait de mère à ceux de son époux. Pallas louait surtout, dans Agrippine, l’avantage d’associer à la famille impériale un petit-fils de Germanicus, bien digne d’une si haute fortune. Elle serait d’ailleurs le noble lien qui réunirait tous les descendants des Claudius, et une femme jeune encore, d’une fécondité éprouvée, ne porterait pas dans une autre maison l’illustration des Césars.
\subsection[{Agrippine veut marier son fils Domitius avec Octavie fille de Claude}]{Agrippine veut marier son fils Domitius avec Octavie fille de Claude}
\noindent \labelchar{III.} Cet avis prévalut, appuyé des séductions d’Agrippine, qui, profitant de son titre de nièce pour visiter à chaque instant son oncle, prit sur lui un tel empire que, préférée à ses rivales, et sans avoir encore le nom d’épouse, elle en exerçait déjà l’autorité. Une fois sûre de son mariage, elle porte ses vues plus loin, et songe à en conclure un second entre Domitius, qu’elle avait eu de Cn. Ahénobarbus, et Octavie, fille de l’empereur. Ce projet ne pouvait s’accomplir sans un crime car Octavie était fiancée à Silanus ; et Claude, ajoutant à l’illustration dont brillait déjà ce jeune homme les ornements du triomphe et la magnificence d’un spectacle de gladiateurs, l’avait désigné d’avance à la faveur publique. Mais rien ne paraissait difficile avec un prince qui n’avait ni affection ni haine qui ne lui fût suggérée ou prescrite.
\subsection[{Problème : Silanus}]{Problème : Silanus}
\noindent \labelchar{IV.} Vitellius, couvrant son artificieuse servilité du nom de censeur, et habile à pressentir l’avènement des puissances nouvelles, s’engagea dans les intrigues d’Agrippine, afin de gagner ses bonnes grâces. Il se fit le calomniateur de Silanus, dont la sœur Julia Calvina, belle, il est vrai, et libre en ses manières, avait été peu auparavant épouse de son fils. Ce fut le fondement de l’accusation : il tourna en crime un amour fraternel innocent, mais indiscret. Claude prêtait l’oreille : sa tendresse pour sa fille le rendait facile à prévenir contre son gendre. Silanus, alors préteur, ignorait le complot, lorsque tout à coup il est chassé du sénat par édit de Vitellius, quoique le choix des sénateurs et la clôture du lustre fussent achevés depuis longtemps. Claude, de son côté, rompt l’alliance conclue, et Silanus est forcé d’abdiquer la préture. Il lui restait un jour d’exercice, qui fut rempli par Éprius Marcellus.
\subsection[{49 – A Rome – Une nièce peut-elle être l’épouse de son oncle ?}]{49 – A Rome – Une nièce peut-elle être l’épouse de son oncle ?}
\noindent \labelchar{V.} Sous le consulat de C. Pompéius et de Q. Véranius, le mariage arrêté entre Claude et Agrippine avait déjà reçu la sanction de la publicité et d’un amour illicite. Toutefois ils n’osaient pas encore célébrer la cérémonie nuptiale, parce qu’il était sans exemple qu’une nièce fût devenue l’épouse de son oncle. On s’effrayait même de l’inceste, et on craignait, en bravant ce scrupule, d’attirer sur l’État quelque grand malheur. L’incertitude ne cessa que lorsque Vitellius eut pris sur lui de terminer l’affaire par un coup de son génie. Il demande à l’empereur s’il cédera aux ordres du peuple, à l’autorité du sénat ; et, sur sa réponse qu’un citoyen ne peut résister seul à la volonté de tous, il le prie d’attendre dans son palais. Lui-même se rend au sénat ; et, protestant qu’il s’agit des plus grands intérêts de la république, il obtient la permission de parler le premier. Aussitôt il expose « que les immenses travaux de César, travaux qui embrassent l’univers, lui rendent nécessaire un appui sur lequel il se repose des soins domestiques, pour veiller uniquement au bien général. Or, où l’âme d’un censeur trouverait-elle un délassement plus honnête que dans la société d’une épouse qui partage ses soucis et ses joies, à laquelle puisse ouvrir son cœur et confier ses jeunes enfants un prince qui n’a jamais connu les excès ni les plaisirs, mais qui, dès sa première jeunesse, s’est fait un devoir d’obéir aux lois ? »\par
\labelchar{VI.} Après cet exorde insinuant, qui fut reçu par les sénateurs avec un applaudissement universel, Vitellius, reprenant la parole, ajouta « que, puisque toutes les voix conseillaient le mariage du prince, il fallait lui choisir une femme distinguée par sa noblesse, sa fécondité, sa vertu ; qu’Agrippine avait sans contredit une naissance supérieure à toute autre ; qu’elle avait donné des preuves de fécondité, et que ses vertus répondaient à ce double avantage. Mais c’était, selon lui, une faveur signalée des dieux qu’elle fût veuve \footnote{Agrippine était veuve de l’orateur Crispus Passiénus qu’elle avait épousé après la mort de Cn. Domitius, père de Néron.} : elle s’unirait libre à un prince qui n’avait jamais attenté aux droits d’un autre époux. Leurs pères avaient vu, ils avaient vu eux-mêmes des Césars enlever arbitrairement des femmes à leurs maris : combien cette violence était loin de la modération présente ! Il était bon de régler par un grand exemple comment le prince devait recevoir une épouse. L’union entre l’oncle et la nièce est, dira-t-on, nouvelle parmi nous. Mais elle est consacrée chez d’autres nations, et aucune loi ne la défend. Longtemps aussi les mariages entre cousins germains furent inconnus ; ils ont fini par devenir fréquents. Les coutumes varient selon les intérêts ; et la nouveauté d’aujourd’hui demain sera un usage. »
\subsection[{Oui !}]{Oui !}
\noindent \labelchar{VII.} Il ne manqua pas de sénateurs qui se précipitèrent à l’envi hors de l’assemblée, en protestant que, si César balançait, ils emploieraient la force. Une multitude confuse s’attroupe aussitôt, et répète à grands cris que le peuple romain forme les mêmes vœux. Claude, sans différer davantage, se présente aux félicitations du Forum ; puis, s’étant rendu au sénat, il demande un décret qui autorise à l’avenir le mariage des oncles avec les filles de leurs frères. Un seul homme se rencontra cependant qui désirât former une telle union, T. Allédius Sévérus, chevalier romain : encore a-t-on dit que c’était pour plaire à Agrippine. Dès ce moment la révolution fut complète. Tout obéissait à une femme ; mais cette femme n’était plus Messaline, faisant de la chose publique le jouet de ses caprices : on crut sentir la main d’un homme qui ramenait à soi les rênes de l’autorité. Agrippine portait au dehors un visage sévère et plus souvent hautain. Au dedans, ses mœurs n’outrageaient point la pudeur, à moins que ce ne fût au profit de l’ambition. Une soif insatiable de l’or se couvrait du prétexte de ménager des ressources au pouvoir.
\subsection[{Mariage, mort de Silanus, rappel de Sénèque}]{Mariage, mort de Silanus, rappel de Sénèque}
\noindent \labelchar{VIII.} Le jour du mariage, Silanus se donna la mort, soit qu’il eût conservé jusqu’à ce moment l’espérance de vivre, ou qu’il cherchât dans le choix de cette journée un contraste odieux. Sa sœur Calvina fut chassée d’Italie. Claude fit ajouter au décret que les pontifes célébreraient les cérémonies instituées par le roi Tullus, et feraient des expiations dans le bois sacré de Diane : grand sujet de risée, de voir quel temps on choisissait pour expier et punir un inceste. Cependant Agrippine, afin de ne pas se signaler uniquement par le mal, obtint pour Sénèque le rappel de l’exil et la dignité de préteur, persuadée que cet acte serait généralement applaudi à cause de l’éclat de ses talents, et bien aise que l’enfance de Domitius grandît sous un tel maître, dont les conseils pourraient d’ailleurs leur être utiles à tous deux pour arriver à la domination : car on croyait Sénèque dévoué à Agrippine par le souvenir du bienfait, ennemi de Claude par le ressentiment de l’injure.
\subsection[{Fiançailles de Néron et d’Octavie}]{Fiançailles de Néron et d’Octavie}
\noindent \labelchar{IX.} On résolut au reste de ne pas différer ; et à force de promesses on engagea le consul désigné, Memmius Pollio, à proposer un sénatus-consulte par lequel Claude serait prié de fiancer Octavie à Domitius. Leur âge ne s’y opposait pas, et c’était un chemin ouvert à de plus grands desseins. Pollio répète à peu près ce qu’avait dit Vitellius au sujet d’Agrippine. Octavie est fiancée, et Domitius, joignant à ses premiers titres ceux d’époux et de gendre, marche désormais l’égal de Britannicus, grâce aux intrigues de sa mère et à la politique des accusateurs de Messaline, qui craignaient que son fils ne la vengeât un jour.
\subsection[{Les Parthes – Méherdate}]{Les Parthes – Méherdate}
\noindent \labelchar{X.} Dans le même temps les ambassadeurs des Parthes, venus, comme je l’ai dit, pour demander Méherdate, furent admis à l’audience du sénat. Ils exposèrent « qu’ils n’ignoraient pas nos traités et qu’ils ne venaient point comme rebelles à la famille des Arsacides ; qu’ils recouraient au fils de Vonon, au petit-fils de Phraate, contre la domination de Gotarzès, également insupportable à la noblesse et au peuple ; que, non content d’avoir assassiné frères, parents, étrangers, Gotarzès immolait maintenant les femmes enceintes et les enfants au berceau, tyran imbécile dans la paix, malheureux dans la guerre, qui voulait faire oublier sa lâcheté par ses barbaries. » Ils ajoutaient « que leur alliance avec nous était ancienne et contractée au nom de la nation ; que nous devions secourir des amis qui, étant nos rivaux en force, nous cédaient par respect ; que, s’ils nous donnaient en otage les enfants de leurs rois, c’était afin de pouvoir, quand l’oppression lasserait leur patience, recourir au sénat et au prince, et leur demander un maître formé à notre école et plus digne de régner. »\par
\labelchar{XI.} Tel fut à peu près le discours des ambassadeurs. Claude, à son tour, parla de la grandeur romaine et des hommages qu’elle recevait des Parthes, s’égalant à Auguste, auquel ils avaient déjà demandé un roi, et sans faire mention de Tibère, qui cependant leur avait aussi envoyé des souverains. Puis, s’adressant à Méherdate, qui était présent, il lui conseilla de voir, dans lui-même et dans son peuple, non un maître et des esclaves, mais un chef et des citoyens, et de pratiquer la clémence et la justice, vertus que leur nouveauté même ferait chérir des barbares. Enfin il se tourna vers les députés et fit l’éloge de Méherdate. « C’était, disait-il, un élève de Rome, et sa modération ne s’était pas encore démentie. Du reste il fallait supporter le caractère dés rois, et l’on ne gagnait rien à en changer trop souvent. Rassasiée de gloire, Rome pouvait désormais souhaiter le repos même aux peuples étrangers. » C. Cassius, gouverneur de Syrie, fut chargé de conduire le jeune prince jusqu’aux rives de l’Euphrate.
\subsection[{Cassius ramène Méherdate}]{Cassius ramène Méherdate}
\noindent \labelchar{XII.} Cassius était alors le premier des Romains dans la science des lois. Je ne dis rien des talents militaires : on ne les connaît point dans cette inaction de la paix, qui tient au même rang l’homme de cœur et le lâche. Toutefois, autant qu’il était possible sans guerre, il faisait revivre l’ancienne discipline, exerçait continuellement les troupes, aussi actif, aussi vigilant que s’il eût eu l’ennemi en présence ; c’est ainsi qu’il honorait ses ancêtres et le nom des Cassius, déjà célèbre parmi ces nations \footnote{Cassius, qui fut depuis l’un des meurtriers de César, avait défendu la Syrie contre les Parthes, après la défaite de Crassus, dont il était questeur. 2. Le mot grec Zeugma veut dire pont, et plusieurs auteurs rapportent qu’Alexandre en fit construire un en cet endroit pour passer l’Euphrate : une ville bâtie à côté en emprunta le nom.}. Il appelle tous ceux qui avaient voulu qu’on demandât le nouveau roi, et campe près de Zeugma (2), lieu où le passage du fleuve est le plus facile. Lorsque les principaux d’entre les Parthes, et Acbare, roi des Arabes, furent arrivés, il avertit Méherdate que le zèle des barbares, d’abord impétueux, languit si l’on diffère, ou se change en perfidie ; qu’il fallait donc presser l’entreprise. Cet avis fut méprisé par la faute d’Acbare ; et ce traître, abusant de l’inexpérience d’un jeune homme qui plaçait la grandeur dans les plaisirs, le retint longtemps à Édesse. En vain Carrhène les appelait et leur promettait un succès infaillible s’ils arrivaient promptement : au lieu d’aller droit en Mésopotamie, ils firent un détour et gagnèrent l’Arménie, alors peu praticable parce que l’hiver commençait.
\subsection[{Gotarzès}]{Gotarzès}
\noindent \labelchar{XIII.} Après de grandes fatigues au milieu des neiges et des montagnes, ils approchaient enfin des plaines, lorsqu’ils se joignirent aux troupes de Carrhéne. Ils passent le Tigre et traversent l’Adiabénie, dont le roi Izatès, en apparence allié de Méherdate, penchait secrètement pour Gotarzès et le servait de meilleure foi. On prit, chemin faisant, Ninive, ancienne capitale de l’Assyrie, et Arbèle, château fameux par cette dernière bataille entre Darius et Alexandre, où la puissance des Perses fut abattue. Cependant Gotarzès était sur le mont Sambulos, offrant des vœux aux divinités du lieu. Le culte principal est celui d’Hercule. Ce dieu, à de certaines époques, avertit ses prêtres, pendant leur sommeil, de tenir auprès du temple des chevaux équipés pour la chasse. Sitôt qu’on a mis sur ces chevaux des carquois garnis de flèches, ils se répandent dans les bois, et à l’entrée de la nuit ils reviennent tout hors d’haleine, rapportant les carquois vides. Le dieu, par une nouvelle apparition nocturne, indique les chemins qu’il a parcourus dans la forêt, et l’on y trouve les animaux étendus de côté et d’autre.
\subsection[{Gotarzès contre Méherdate}]{Gotarzès contre Méherdate}
\noindent \labelchar{XIV.} Au reste Gotarzès, dont l’armée n’était pas encore assez nombreuse, se couvrait du fleuve Corma comme d’un rempart. Là, malgré les insultes et les défis par lesquels on le provoquait au combat, il temporisait, changeait de positions, envoyait des corrupteurs acheter la trahison dans les rangs ennemis. Bientôt Izatès, et ensuite Acbare, se retirèrent avec les Adiabéniens et les Arabes : telle est l’inconstance de ces peuples ; et l’expérience a prouvé d’ailleurs que les barbares aiment à nous demander des rois bien plus qu’à les garder. Méherdate, privé de si puissants auxiliaires et craignant la défection des autres, prit le seul parti qui lui restât, celui de s’en remettre à la fortune et de hasarder une bataille. Gotarzès, enhardi par l’affaiblissement de l’ennemi, ne la refusa point. Le choc fut sanglant et le succès douteux, jusqu’au moment où Carrhène, ayant renversé tout ce qui était devant lui, se laissa emporter trop loin et fut enveloppé par des troupes fraîches. Alors tout fut désespéré ; et Méherdate, s’étant fié aux promesses de Parrhace, client de son père, fut enchaîné par cet ami perfide, et livré au vainqueur. Celui-ci, après l’avoir désavoué pour son parent, pour un Arsacide, et traité d’étranger et de Romain, lui fait couper les oreilles et le laisse vivre pour être un monument de sa clémence et de notre honte. Gotarzès mourut ensuite de maladie, et Vonon, alors gouverneur des Mèdes, fut appelé au trône. Ni prospérités ni revers n’ont rendu célèbre le nom de ce nouveau roi. Son règne fut court et sans gloire, et la couronne des Parthes fut donnée après lui à son fils Vologèse.
\subsection[{Mithridate}]{Mithridate}
\noindent \labelchar{XV.} Mithridate, roi détrôné du Bosphore, errait de pays en pays, lorsqu’il sut que le général romain Didius était absent avec l’élite de son armée, et qu’il ne restait pour garder le Bosphore que le nouveau roi Cotys, jeune homme sans expérience, et un petit nombre de cohortes commandées par un simple chevalier romain, Julius Aquila. Plein de mépris pour ces deux clefs, Mithridate appelle aux armes les nations voisines, attire des transfuges ; enfin, parvenu à former une armée, il chasse le roi des Dandarides \footnote{Strabon compte les Dandari parmi les Méotes, peuples Sauromates ou Sarmates, qui habitaient sur la côte orientale de la mer d’Azof (les Palus-Méotides), entre le Kuban et le Don ou Tanaïs. Il place dans les mêmes contrées les Aorses et les Siraques, répandus vers le midi, jusqu’aux monts Caucasiens.} et s’empare de ses États. A cette nouvelle, qui menaçait le Bosphore d’une prochaine invasion, Aquila et Cotys, se défiant de leurs forces, et voyant que Zorsinès, roi des Siraques, avait recommencé les hostilités, cherchèrent aussi des appuis au dehors : ils députèrent vers Eunone, chef de la nation des Aorses. L’alliance ne fut pas difficile à conclure : Eunone avait à choisir entre la puissance romaine et le rebelle Mithridate. On convint qu’il fournirait de la cavalerie, et que les Romains assiégeraient les villes.\par
\labelchar{XVI.} Alors on s’avance en bon ordre, ayant en tête et en queue les Aorses, et au centre nos cohortes avec les troupes du Bosphore, armées à la romaine. L’ennemi est repoussé, et l’on arrive à Soza, ville de la Dandarique, abandonnée par Mithridate, où, à cause des dispositions équivoques des habitants, on laissa garnison. Marchant ensuite contre les Siraques, on passe le fleuve Panda ; et l’on investit la ville d’Uspé, située sur une éminence et défendue par des fossés et des murs. Mais ces murs, construits, au lieu de pierre, avec de la terre soutenue des deux côtés de claies et de branchages, ne pouvaient tenir contre un assaut. Nos tours, plus élevées, lançaient des torches et des javelines qui jetaient le désordre parmi les assiégés, et, si la nuit n’eût mis fin au combat, le siège eût été entrepris et achevé en un jour.\par
\labelchar{XVII.} Le lendemain, des députés vinrent demander grâce pour les personnes libres et offrir dix mille esclaves. Les vainqueurs rejetèrent cette proposition : massacrer des gens reçus à merci eût été barbare ; garder tant de prisonniers était difficile. On aima mieux qu’ils périssent par le droit de la guerre. Déjà les soldats avaient escaladé les murs : on leur donna le signal du carnage. Le sac d’Uspé intimida les autres villes. Elles ne voyaient plus de rempart assuré contre un vainqueur que n’arrêtaient ni armes ni retranchements, ni bois ni montagnes, ni fleuves ni murailles. Zorsinès réfléchit longtemps s’il risquerait le trône de ses pères pour la cause désespérée de Mithridate. Enfin l’intérêt de sa maison prévalut : il donna des otages et vint se prosterner devant la statue de César ; à la grande gloire de l’armée romaine, qui, par une suite de victoires non sanglantes, était parvenue jusqu’à trois journées du Tanaïs. Le retour fut moins heureux : quelques-uns des navires qui rapportaient les troupes par mer furent jetés sur le rivage de la Tauride, et enveloppés par les barbares, qui tuèrent un préfet de cohorte et plusieurs centurions.\par
\labelchar{XVIII.} Cependant Mithridate, qui n’attendait plus rien des armes, délibérait à qui demander de la pitié. Traître, puis ennemi, son frère Cotys ne lui donnait que des craintes. Il n’y avait dans le pays aucun Romain d’une assez haute considération pour qu’on pût s’assurer dans les promesses qu’il ferait. Il se tourne vers Eunone, exempt à son égard de haine personnelle, et fort auprès de nous du crédit que donne une amitié récente. Il prend donc l’air et l’habit le plus conforme à sa fortune, entre dans le palais d’Eunone, et tombant à ses genoux : « Tu vois, dit-il, ce Mithridate que les Romains cherchent depuis tant d’années sur terre et sur mer : il se remet lui-même en tes mains. Dispose à ton gré du descendant du grand Achéménès \footnote{Mithridate, roi du Bosphore, étant issu du grand Mithridate, septième du nom, sa famille remontait jusqu’à Mithridate Ier, satrape de la Cappadoce maritime, pays plus connu dans la suite sous le nom du royaume de Pont. Or, Mithridate Ier descendait d’un certain Artabaze, regardé par quelques historiens comme un fils de Darius Hystaspes, roi de Perse ; et la tige des rois de Perse était Achéménès, aïeul (ou bisaïeul) de Cambyse, père de Cyrus.} : ce titre est le seul bien que mes ennemis ne m’aient pas ravi. »\par
\labelchar{XIX.} Le nom éclatant de Mithridate, l’inconstance des choses humaines, la dignité de cette prière, émurent Eunone. Il relève le suppliant, et le loue d’avoir choisi la nation des Aorses et l’intercession de leur roi pour demander son pardon. Aussitôt il envoie des députés vers Claude avec une lettre dont le sens était « que les premières alliances entre les empereurs romains et les monarques des plus puissantes nations avaient eu pour base leur commune grandeur ; qu’il y avait entre Claude et lui un lien de plus, celui de la victoire ; que c’était finir glorieusement la guerre, que de la terminer en pardonnant ; qu’ainsi on n’avait rien ôté à Zorsinès vaincu ; que, Mithridate étant plus coupable, Eunone ne demandait pour lui ni puissance ni trône, mais la vie et la faveur de n’être pas mené en triomphe. »\par
\labelchar{XX.} Les grandeurs étrangères trouvaient facilement grâce devant Claude. Il délibéra cependant s’il devait recevoir à merci un tel prisonnier, ou le réclamer les armes à la main. Le ressentiment et la vengeance conseillaient ce dernier parti. Mais on objecta mille inconvénients : « d’abord, la guerre dans un pays sans routes et sur une mer sans ports ; ensuite des rois intrépides, des peuples errants, un sol stérile ; enfin les ennuis de la lenteur, les dangers de la précipitation ; peu de gloire si l’on était vainqueur, beaucoup de honte si l’on était repoussé. Pourquoi ne pas saisir ce qui était offert, et garder en exil un captif dont le supplice serait d’autant plus grand, que sa vie, dénuée de tout se prolongerait davantage ? » Convaincu par ces raisons, Claude écrivit à Eunone « que Mithridate avait mérité les dernières rigueurs, et que la force ne manquait pas aux Romains pour faire un grand exemple ; mais qu’ils avaient appris de leurs ancêtres à montrer autant de clémence envers les suppliants que de vigueur contre les ennemis ; qu’à l’égard du triomphe, on ne le gagnait que sur des peuples ou des rois qui ne fussent pas déchus. »\par
\labelchar{XXI.} Mithridate, livré alors et conduit à Rome par Junius Cilo, procurateur du Pont, montra, dit-on, en parlant à Claude, une fierté plus haute que sa fortune. Voici ses paroles telles que la renommée les publia : « Je n’ai point été renvoyé vers toi, j’y suis revenu ; si tu ne le crois pas, laisse-moi partir, et tâche de me reprendre. » Son visage conserva toute son intrépidité, lorsque, placé près des rostres et entouré de gardes, il fut offert aux regards du peuple. Les ornements consulaires furent décernés à Cilo, ceux de la préture à Julius Aquila.
\subsection[{A Rome – Agrippine s’attaque à une rivale}]{A Rome – Agrippine s’attaque à une rivale}
\noindent \labelchar{XXII.} Sous les mêmes consuls, Agrippine, implacable en ses haines, et mortelle ennemie de Lollia, qui lui avait disputé la main de Claude, lui chercha des crimes et un accusateur. Elle avait, disait-on, interrogé des astrologues et des magiciens, et consulté l’oracle d’Apollon de Claros sur le mariage du prince. Claude, sans entendre l’accusée, prononce son avis dans le sénat. Après un long exorde sur l’illustration de cette femme, qui était nièce de L. Volusius, petite-nièce de Messalinus Cotta, et qui avait eu Memmius Régulus pour époux (car il omettait à dessein son mariage avec l’empereur Caïus), il ajouta qu’il fallait réprimer des complots funestes à la république, et ôter au crime ses moyens de succès. Il proposa donc la confiscation des biens et le bannissement hors de l’Italie. Lollia fut exilée, et, sur son immense fortune, on lui laissa cinq millions de sesterces \footnote{– 974 178 francs de notre monnaie.}. Calpurnie, femme du premier rang, fut frappée à son tour, parce que le prince avait loué sa figure ; éloge indifférent toutefois, où l’amour n’entrait pour rien : aussi la colère d’Agrippine n’alla-t-elle pas aux dernières violences. Quant à Lollia, un tribun fut envoyé pour la forcer à mourir. On condamna encore Cadius Rufus en vertu de la loi sur les concussions : il était accusé par les Bithyniens.
\subsection[{Quelques mesures de Claude – le pomerium}]{Quelques mesures de Claude – le pomerium}
\noindent \labelchar{XXIII.} La Gaule narbonnaise, distinguée par son respect envers le sénat, reçut en récompense un privilège réservé jusqu’alors à la Sicile : il fut permis aux sénateurs de cette province d’aller visiter leurs biens sans demander la permission du prince. Les Ituréens et les Juifs, dont les rois, Sobémus et Agrippa, venaient de mourir, furent réunis au gouvernement de Syrie. L’augure de Salut \footnote{Espèce de divination qu’on employait, lorsque la république était dans une paix complète, pour savoir si les dieux approuvaient qu’on leur on demandât la continuation.} y était négligé depuis vingt-cinq ans : on ordonna qu’il fût pris de nouveau et continué dans la suite. Claude étendit le pomérium (2), d’après un ancien usage qui donnait à ceux qui avaient reculé les bornes de l’empire le droit d’agrandir aussi l’enceinte de la ville ; droit dont cependant aucun des généraux romains n’avait usé, même après les plus vastes conquêtes, si ce n’est Sylla et Auguste.\par
\labelchar{XXIV.} Quelle fut, à cet égard, ou la vanité ou la gloire des rois, c’est un point sur lequel les traditions varient. Mais je ne crois pas inutile de connaître en quel lieu furent bâtis les premiers édifices, et quel pomérium fut marqué par Romulus. Le sillon tracé pour désigner l’enceinte de la place partait du marché aux bœufs, où nous voyons un taureau d’airain (à cause de la charrue traînée par cet animal), et ce sillon embrassait le grand autel d’Hercule. Ensuite, des pierres placées de distance en distance, en suivant le pied du mont Palatin, allaient d’abord à l’autel de Consus \footnote{C’est ce dieu qu’on adorait aussi sous le nom de Neptune Équestre, et dont la fête servit de prétexte à l’enlèvement des Sabines.}, puis aux anciennes Curies \footnote{Ces curies étaient des édifices où les membres de chacune des curies qui composaient le peuple romain offraient des sacrifices et prenaient des repas en commun, à certains jours réglés. On appelait vieilles les Curies qu’avait bâties Romulus.}, enfin au petit temple des Lares et au forum Romanum. Quant au Capitole, on croit que c’est Tatius, et non Romulus, qui l’a enfermé dans la ville. Depuis, l’enceinte de Rome s’est accrue avec sa fortune. Les limites figées par Claude sont faciles à connaître : elles sont marquées dans les actes publics.
\subsection[{50 – A Rome – Adoption de Domitius – son nouveau nom : Néron}]{50 – A Rome – Adoption de Domitius – son nouveau nom : Néron}
\noindent \labelchar{XXV.} Sous le consulat de C. Antistius et de M. Suilius, on employa le crédit de Pallas à hâter l’adoption de Domitius. Lié doublement aux intrigues d’Agrippine et par son mariage, dont il avait été le négociateur, et par l’adultère, où elle l’avait engagé depuis, l’affranchi pressait Claude « de songer aux intérêts de l’empire, de donner un appui à l’enfance de Britannicus. Ainsi l’empereur Auguste, quoiqu’il eût des petits-fils pour soutiens de sa maison, avait approché de son trône les enfants de sa femme ; ainsi Tibère, ayant déjà un héritier de son sang, avait adopté Germanicus. Claude, à leur exemple, devait s’appuyer d’un jeune homme qui partageât les soins du rang suprême. » Vaincu par ces discours, Claude préfère à son propre fils Domitius, plus âgé de deux ans, et va répéter au sénat les raisons que son affranchi venait de lui donner. Les habiles remarquèrent qu’il n’y avait eu jusqu’alors aucune adoption dans la branche patricienne des Claudius, et que, depuis Attus Clausus, elle s’était perpétuée sans mélange.\par
\labelchar{XXVI.} On adressa au prince des actions de grâces où la flatterie épuisa tous ses raffinements pour Domitius. Une loi fut rendue pour le faire passer dans la famille Claudia et l’appeler Néron ; Agrippine fut décorée du surnom d’Augusta. Ces actes consommés, il n’y eut pas de cœur si dur que le sort de Britannicus ne touchât de pitié. Délaissé peu à peu, jusqu’à n’avoir plus un esclave pour le servir, il tournait en dérision les soins importuns de sa marâtre, dont il comprenait l’hypocrisie : car on prétend que son esprit ne manquait pas de vivacité, soit que la chose fût vraie, ou qu’il doive à la recommandation du malheur une renommée qu’il n’eut pas le temps de se justifier.
\subsection[{À l’extérieur – Problèmes en Germanie – les Cattes}]{À l’extérieur – Problèmes en Germanie – les Cattes}
\noindent \labelchar{XXVII.} Agrippine voulut aussi étaler son pouvoir aux yeux des peuples alliés. Elle obtint qu’on envoyât dans la ville des Ubiens, où elle était née, des vétérans et une colonie, à laquelle on donna son nom, Par une rencontre du hasard, c’était son aïeul Agrippa qui, à l’époque où cette nation passa le Rhin, l’avait reçue dans notre alliance. Vers le même temps, une irruption des Cattes, accourus pour piller, jeta l’alarme dans la haute Germanie. Aussitôt le lieutenant L. Pomponius détache les cohortes des Vangions et des Némètes \footnote{Nations venues de la Germanie transrhénane, et qui occupaient les pays de Worms et de Spire.}, soutenues par des cavaliers auxiliaires, avec ordre de prévenir les pillards, ou de tomber à l’improviste sur leurs bandes éparses. Les soldats secondèrent habilement les vues du général ; ils se divisèrent en deux corps, dont l’un prit à gauche, et trouva les barbares nouvellement revenus du butin. La débauche où ils s’étaient plongés et l’accablement du sommeil les rendirent faciles à envelopper. La joie fut accrue par la délivrance de quelques soldats de Varus, arrachés, après quarante ans, à la servitude.\par
\labelchar{XXVIII.} Ceux qui s’étaient avancés à droite et par des chemins plus courts, rencontrant un ennemi qui osa combattre, en firent un plus grand carnage. Tous, chargés de gloire et de butin, revinrent au mont Taunus, où le général les attendait avec les légions, dans l’espoir que les Cattes, animés par la vengeance, lui fourniraient l’occasion de livrer une bataille. Ceux-ci, craignant d’être enfermés d’un côté par les Romains, de l’autre par les Chérusques, leurs éternels ennemis, envoyèrent à Rome des députés et des otages. Pomponius reçut les ornements du triomphe, et c’est, auprès de la postérité, le moindre titre d’une gloire dont il doit à ses vers la plus belle partie.
\subsection[{Les Suèves}]{Les Suèves}
\noindent \labelchar{XXIX.} A la même époque, le roi Vannius, imposé aux Suèves par Drusus César, fut chassé de ses États. Les premières années de son règne avaient été glorieuses et populaires. L’orgueil vint avec le temps, et arma contre lui la haine de ses voisins et les factions domestiques. Les auteurs de sa perte furent Vangion et Sidon, tous deux fils de sa sœur, et Vibillius, roi des Hermondures. Aucune prière ne put décider Claude à interposer ses armes dans cette querelle entre barbares. Il promit à Vannius un asile s’il était chassé ; et il écrivit à P. Atellius Hister, gouverneur de Pannonie, d’occuper la rive du Danube avec sa légion et des auxiliaires choisis dans le pays même, afin de protéger les vaincus et de tenir les vainqueurs en respect, de peur qu’enorgueillis par le succès ils ne troublassent aussi la paix de notre empire. Car une multitude innombrable de Lygiens \footnote{Les Lygiens habitaient sur la Vistule. 2. Au nord des Palus-Méotides, entre le Tanaïs et le Borysthène.} accourait avec d’autres nations, attirées par le bruit des trésors que Vannius, pendant trente ans d’exactions, avait accumulés dans ce royaume. Vannius, avec l’infanterie qu’il avait à lui et la cavalerie que lui fournissaient les Sarmates Iazyges \footnote{Les Brigantes, au nord des Canges et des Ordoviques, s’étendaient d’une mer à l’autre, dans les comtés de Lancastre, de Cumberland, de Durham et d’York.}, était faible contre tant d’ennemis. Aussi résolut-il de se défendre dans ses places et de traîner la guerre en longueur.\par
\labelchar{XXX.} Mais les Sarmates ne pouvaient souffrir l’ennui d’être assiégés. En courant les campagnes voisines, ils attirèrent de ce côté les Lygiens et les Hermondures, et le combat-devint inévitable. Vannius quitte ses forteresses et perd une bataille, revers qui lui valut au moins l’éloge d’avoir payé de sa personne et reçu d’honorables blessures. Il gagna la flotte qui l’attendait sur le Danube. Bientôt après ses vassaux le suivirent, et reçurent dans la Pannonie des terres et un établissement. Vangion et Sidon se partagèrent le royaume, et nous gardèrent une foi inaltérable ; très-aimés des peuples avant qu’ils fussent leurs maîtres, et (dirai-je par la faute de leur caractère, ou par le malheur de la domination ?) encore plus haïs quand ils le furent devenus.
\subsection[{Désordres en Bretagne : Caractacus – Le préteur Ostorius en Bretagne}]{Désordres en Bretagne : Caractacus – Le préteur Ostorius en Bretagne}
\noindent \labelchar{XXXI.} C’est le temps ou le propréteur P. Ostorius arrivait dans la Bretagne, qu’il trouva pleine de troubles. Les ennemis avaient fait sur les terres de nos alliés une incursion d’autant plus furieuse qu’ils ne s’attendaient pas qu’un nouveau général avec une armée inconnue, et déjà en hiver, marcherait contre eux. Ostorius, qui savait combien les premiers événements ôtent ou donnent de confiance, vole avec les cohortes, tue ce qui résiste, poursuit les autres dispersés ; puis, dans la crainte qu’ils ne se rallient, et afin de se prémunir contre une paix hostile et trompeuse qui ne laisserait de repos ni au général ni aux soldats, il s’apprête à désarmer les peuplades suspectes, et à les contenir, par une ligne de postes fortifiés, au delà des rivières d’Auvone et de Sabrine \footnote{La Saverne et probablement le Non ou Nyne, qui passe à Northampton, et se jette dans la mer du Nord.}. La résistance commença par les Icéniens, nation puissante et que les combats n’avaient point mutilée, parce qu’elle avait d’elle-même embrassé notre alliance. Soulevés par eux, les peuples d’alentour choisissent un champ de bataille entouré d’une terrasse rustique, avec une entrée si étroite que la cavalerie n’y pouvait pénétrer. Le général romain n’avait point amené les légions, cette force d’une armée : il entreprit toutefois, avec les seuls auxiliaires, d’emporter ces retranchements. Il distribue les postes aux cohortes, et tient la cavalerie elle-même prête à combattre à pied. Le signal donné, on fait brèche au rempart, et l’ennemi, emprisonné dans ses propres fortifications, est mis en désordre. Pressés par la conscience de leur rébellion, jointe à l’impossibilité de fuir, les barbares firent des prodiges de valeur. Dans ce combat, M. Ostorius, fils du général, mérita la couronne civique.
\subsection[{Chez les Brigantes}]{Chez les Brigantes}
\noindent \labelchar{XXXII.} Le désastre des Icéniens contint ceux qui balançaient entre la paix et la guerre, et l’armée fut conduite chez les Canges \footnote{Les Canges habitaient dans le nord du pays de Galles, près des Ordoviques.}. Les champs furent dévastés et l’on ramassa beaucoup de butin, sans que l’ennemi osât en venir aux mains, ou, s’il essaya par surprise d’entamer nos colonnes, on l’en fit repentir. Déjà on approchait de la mer qui est en face de l’Hibernie, lorsque des troubles survenus chez les Brigantes (2) rappelèrent le général, inébranlable dans la résolution de ne point tenter de nouvelles conquêtes qu’il n’eût assuré les anciennes. Le supplice d’un petit nombre de rebelles armés, et le pardon accordé aux autres, pacifièrent les Brigantes. Quant aux Silures \footnote{Les Silures habitaient le midi du pays de Galles, entre la Saverne et la mer d’Irlande.}, ni rigueur ni clémence ne put les ramener : ils continuèrent la guerre, et il fallut que des légions, campées au milieu d’eux, les pliassent au joug. Pour y mieux réussir, on conduisit à Camulodunum, \footnote{Plusieurs pensent que c’est aujourd’hui Colchester.}, sur les terres enlevées à l’ennemi, une forte colonie de vétérans. C’était un boulevard contre les rebelles, et une école où les alliés apprendraient à respecter les lois.
\subsection[{Contre les Silures}]{Contre les Silures}
\noindent \labelchar{XXXIII.} On marcha ensuite contre les Silures, dont l’intrépidité naturelle était doublée par leur confiance aux ressources de Caractacus, guerrier que beaucoup de revers, beaucoup de succès, avaient élevé si haut, qu’il éclipsait tous les autres chefs de la Bretagne. Il avait pour lui ses ruses et les pièges du terrain, mais non la force des soldats : en conséquence, il transporte la guerre chez les Ordoviques, se recrute de tous ceux qui redoutaient la paix que nous donnons, et hasarde une action décisive, après avoir choisi un champ de bataille où l’accès, la retraite, tout fût danger pour nous, avantage pour les siens. Il occupait des montagnes escarpées, et, partout où la pente était plus douce, il avait entassé des pierres en forme de rempart. Au-devant coulait un fleuve dont les gués n’étaient pas sûrs, et des bataillons armés bordaient les retranchements.\par
\labelchar{XXXIV.} Cependant les chefs de chaque nation parcourent les rangs, exhortent, encouragent, atténuant le danger, exagérant l’espérance, n’oubliant rien de ce qui peut animer au combat. Pour Caractacus, il volait de tous les côtés, s’écriant que ce jour, que cette bataille allait commencer l’affranchissement de la Bretagne ou son éternelle servitude. II nommait aux guerriers ces héros leurs ancêtres, qui avaient chassé le dictateur César, et par qui, sauvés des haches et des tributs, ils conservaient à l’abri de l’outrage leurs femmes et leurs enfants. Pendant qu’ils parlaient de la sorte, l’armée applaudissait à grand bruit, et chacun jurait, par les dieux de sa tribu, que ni fer ni blessures ne le feraient reculer.\par
\labelchar{XXXV.} Cet enthousiasme intimida le général romain. Un fleuve à traverser, un rempart à franchir, ces monts escarpés, ces lieux où l’œil ne découvrait que du fer et des soldats, tout ébranlait son courage. Mais l’armée demandait le combat : tous s’écriaient à l’envi qu’il n’est rien dont la valeur ne triomphe ; et les préfets, les tribuns, tenant le même langage, échauffaient encore leur ardeur. Ostorius, ayant reconnu ce qui est accessible, ce qui ne l’est point, les fait avancer ainsi animés, et passe facilement la rivière. Parvenus au rempart, tant que l’on combattit avec des armes de trait, les blessés et les morts furent plus nombreux de notre côté ; mais lorsque, à l’abri de la tortue, on eut démoli cet amas informe de pierres amoncelées, et que les deux armées furent aux prises sur le même terrain, les barbares reculèrent vers le sommet de leurs montagnes. Mais les troupes légères et l’infanterie pesamment armée y coururent après eux, celles-là en les harcelant à coups de traits, celles-ci en pressant, par une marche serrée, leurs bataillons rompus et en désordre. Car les Bretons n’avaient pour se couvrir ni casque ni cuirasse ; et, s’ils essayaient de résister aux auxiliaires, ils tombaient sous l’épée et le javelot du légionnaire ; s’ils faisaient face aux légions, le sabre et les javelines des auxiliaires jonchaient la terre de leurs corps. Cette victoire fut éclatante : on prit la femme et la fille de Caractacus, et ses frères se rendirent à discrétion.
\subsection[{Caractacus trahi}]{Caractacus trahi}
\noindent \labelchar{XXXVI.} Le malheur appelle la trahison : Caractacus avait cru trouver un asile chez Cartismandua, reine des Brigantes ; il fut chargé de fers et livré aux vainqueurs. C’était la neuvième année que la guerre durait en Bretagne. La renommée de ce chef, sortie des îles où elle était née, avait parcouru les provinces voisines et pénétré jusqu’en Italie. On était impatient de voir quel était ce guerrier qui, depuis tant d’années, bravait notre puissance. A Rome même le nom de Caractacus n’était pas sans éclat ; et le prince, en voulant rehausser sa gloire, augmenta celle du vaincu. On convoque le peuple comme pour un spectacle extraordinaire ; les cohortes prétoriennes sont rangées en armes dans la plaine qui est devant leur camp. Alors paraissent les vassaux du roi barbare, avec les ornements militaires, les colliers, les trophées conquis par lui sur les peuples voisins ; viennent ensuite ses frères, sa femme et sa fille ; enfin lui-même est offert aux regards. Les autres s’abaissèrent par crainte à des prières humiliantes ; lui, sans courber son front, sans dire un mot pour implorer la pitié, arrivé devant le tribunal, parla en ces termes :
\subsection[{Discours de Caractacus devant le Sénat}]{Discours de Caractacus devant le Sénat}
\noindent \labelchar{XXXVII.} « Si ma modération dans la prospérité eût égalé ma naissance et ma fortune, j’aurais pu venir ici comme ami, jamais comme prisonnier ; et toi-même tu n’aurais pas dédaigné l’alliance d’un prince issu d’illustres aïeux et souverain de plusieurs nations. Maintenant le sort ajoute à ta gloire tout ce qu’il ôte à la mienne. J’ai eu des chevaux, des soldats, des armes, des richesses : est-il surprenant que je ne les aie perdus que malgré moi ? Si vous voulez commander à tous, ce n’est pas une raison pour que tous acceptent la servitude. Que je me fusse livré sans combat, ni ma fortune ni ta victoire n’auraient occupé la renommée : et même aujourd’hui mon supplice serait bientôt oublié. Mais si tu me laisses la vie, je serai une preuve éternelle de ta clémence » Claude lui pardonna, ainsi qu’à sa femme et à ses frères. Dégagés de leurs fers, ils allèrent vers Agrippine, qu’on voyait assise à une petite distance sur un autre tribunal, et lui rendirent les mêmes hommages et les mêmes actions de grâce qu’à l’empereur ; chose nouvelle assurément et opposée à l’esprit de nos ancêtres, de voir une femme siéger devant les enseignes romaines : ses aïeux avaient conquis l’empire ; elle en revendiquait sa part.
\subsection[{Nouveaux troubles en Bretagne}]{Nouveaux troubles en Bretagne}
\noindent \labelchar{XXXVIII.} Le sénat fut ensuite convoqué, et l’on y fit de pompeux discours sur la prise de Caractacus, que l’on comparait aux anciennes prospérités du peuple romain, lorsque Scipion, Paul-Émile et les autres généraux montraient à ses regards Siphax, Persée et d’autres rois, captifs et enchaînés. Les ornements du triomphe furent décernés à Ostorius. Il n’avait eu jusqu’alors que des succès : bientôt sa fortune se démentit ; soit que, délivré de Caractacus, et croyant la guerre terminée, il laissât la discipline se relâcher parmi nous ; soit que les ennemis, touchés du malheur d’un si grand roi, courussent à la vengeance avec un redoublement d’ardeur. Un préfet de camp et plusieurs cohortes légionnaires, restés chez les Silures pour y construire des forts, furent enveloppés ; et, si l’on ne fût promptement accouru des villages et des postes voisins, le massacre eût été général. Malgré ce secours, le préfet, huit centurions, et les plus braves soldats périrent. Peu de temps après, nos fourrageurs et la cavalerie envoyée pour les soutenir furent mis en déroute.
\subsection[{Mort d’Ostorius}]{Mort d’Ostorius}
\noindent \labelchar{XXXIX.} Ostorius fit sortir alors de l’infanterie légère ; et cependant la fuite ne s’arrêtait pas encore. Il fallut que les légions soutinssent le combat. Leur masse plus solide rétablit l’égalité et bientôt nous donna l’avantage. Les ennemis s’enfuirent sans beaucoup de perte, parce que le jour baissait. Ce ne furent, depuis ce moment, que rencontres fortuites ou cherchées, et dort la plupart ressemblaient à des attaques de brigands. On se battait dans les bois, dans les marais, tumultuairement ou avec méthode, par vengeance ou pour faire du butin, par l’ordre des chefs ou à leur insu. Les plus acharnés étaient les Silures, qu’une parole du général romain, publiquement répétée, enflammait de colère. Il avait dit, en les comparant aux Sicambres, exterminés jadis et transportés dans la Gaule, qu’il fallait anéantir aussi jusqu’au nom des Silures. Deux cohortes, conduites par des préfets trop avides, pillaient sans précaution : ils les enlevèrent ; et, en partageant avec les autres nations, les dépouilles et les prisonniers, il les entraînaient toutes à la révolte, lorsque, dévoré d’ennuis et d’inquiétudes, Ostorius mourut. Les ennemis s’en réjouirent, satisfaits de voir qu’à défaut de leur épée la guerre du moins eût consumé les jours d’un général qui n’était nullement à mépriser.
\subsection[{Un nouveau général : Didius}]{Un nouveau général : Didius}
\noindent \labelchar{XL.} Quand l’empereur eut appris la mort de son lieutenant, pour ne pas laisser la province sans chef, il mit à sa place A. Didius. Celui-ci, malgré sa diligence, ne trouva pas les choses dans l’état où Ostorius les avait laissées. Une légion gourmandée par Manlius Valens avait été battue dans l’intervalle ; échec que les Bretons grossissaient pour effrayer le nouveau général, et dont lui-même exagéra l’importance, afin de se ménager ou plus de gloire, s’il le réparait, ou une excuse plus légitime, si l’ennemi conservait l’avantage. C’étaient encore les Silures qui nous avaient porté ce coup ; et jusqu’à ce que Didius, accouru à la hâte, les eût repoussés, ils infestèrent au loin le pays. Depuis la prise de Caractacus, les barbares n’avaient pas de meilleur capitaine que Vénusius. J’ai déjà dit qu’il était de la nation des Brigantes. Fidèle à notre empire et défendu par nos armes tant qu’il fut l’époux de la reine Cartismandua, il ne fut pas plus tôt séparé d’elle par le divorce, ensuite par la guerre, qu’il devint aussitôt notre ennemi. La lutte fut d’abord entre eux seuls, et Cartismandua, par un adroit stratagème, fit prisonnier le frère et les parents de Vénusius. Indignés et redoutant l’ignominie d’obéir à une femme, les ennemis armèrent leur plus brave jeunesse, et fondirent sur les États de la reine. Nous l’avions prévu, et des cohortes envoyées à son secours livrèrent un rude combat, où la fortune, d’abord indécise, finit par nous être prospère. Une légion combattit avec le même succès sous les ordres de Césius Nasica : car Didius, appesanti par l’âge et rassasié d’honneurs, faisait la guerre par ses officiers, et se bornait à repousser l’ennemi. Ces événements eurent lieu en plusieurs années sous deux propréteurs, Ostorius et Didius. Je les ai réunis, de peur que séparés ils ne laissassent un souvenir trop fugitif. Je reviens à l’ordre des temps.
\subsection[{51 – À Rome – Néron ascendant, Britannicus dans la tourmente}]{51 – À Rome – Néron ascendant, Britannicus dans la tourmente}
\noindent \labelchar{XLI.} Sous le consulat de Cornélius Orphitus et le cinquième de Claude, on donna prématurément la robe virile à Néron, afin qu’il parût en état de prendre part aux affaires publiques. Le prince accorda facilement aux adulations des sénateurs que Néron prît possession du consulat à vingt ans, que jusque-là il eût le titre de consul désigné et le pouvoir proconsulaire hors de Rome, enfin qu’il fût nommé prince de la jeunesse. On fit de plus en son nom des libéralités au peuple et aux soldats ; et, dans les jeux du cirque qui furent donnés pour lui gagner l’affection de la multitude, Britannicus parut avec la prétexte, et Néron avec la robe triomphale. Ainsi le peuple romain put les contempler tous deux, revêtus, l’un des habits de l’enfance, l’autre des attributs du commandement, et pressentir à cette vue leurs futures destinées. Quelques centurions et quelques tribuns plaignaient le sort de Britannicus : on les éloigna par des motifs supposés, ou sous prétexte d’emplois honorables. On écarta même le peu d’affranchis qui lui eussent gardé jusqu’alors une foi incorruptible, et voici à quelle occasion. Un jour, les deux frères se rencontrant, Néron salua Britannicus par son nom, et celui-ci appela Néron, Domitius. Agrippine dénonce ce mot à son époux comme un signal de discorde, et s’en plaint amèrement. « On méprise, selon elle, une auguste adoption ; on abroge dans l’intérieur du palais un acte conseillé par le sénat, ordonné par le peuple. Si l’on ne réprime la méchanceté des maîtres qui donnent ces leçons de haine, elle enfantera quelque malheur public. » Ces invectives furent pour Claude des accusations capitales. II bannit ou fit mourir les plus vertueux instituteurs de son fils, et plaça près de lui des surveillants du choix de sa marâtre.
\subsection[{Montée en puissance d’Agrippine}]{Montée en puissance d’Agrippine}
\noindent \labelchar{XLII.} Toutefois Agrippine n’osait tenter les dernières entreprises, tant que les gardes prétoriennes resteraient confiées aux soins de Crispinus et de Géta, qu’elle croyait attachés à la mémoire et aux enfants de Messaline. Elle représente donc que la rivalité inévitable entre deux chefs divise les cohortes, et que, sous l’autorité d’un seul, la discipline serait plus ferme. Claude suivit le conseil de sa femme, et le prétoire fut mis sous les ordres de Burrus Afranius, guerrier distingué, mais qui savait trop de quelle main il tenait le commandement. Agrippine rehaussait de plus en plus l’éclat de sa propre grandeur. On la vit entrer au Capitole sur un char suspendu, privilège réservé de tout temps aux prêtres et aux images des dieux, et qui ajoutait aux respects du peuple pour une femme de ce rang, la seule jusqu’à nos jours qui ait été fille d’un César \footnote{Germanicus. 2. Agrippine était sœur de Caligula, femme de Claude, mère de Néron.}, sœur, épouse et mère d’empereurs \footnote{Les décuries des scribes ou greffiers des magistrats.}. Cependant le plus zélé de ses partisans, Vitellius, dans toute la force de son crédit, à la fin de sa carrière (tant la fortune des grands est incertaine), fut frappé d’une accusation. Le sénateur Junius Lupus le dénonçait comme coupable de lèse-majesté, et lui reprochait de convoiter l’empire. Claude eût prêté l’oreille, si les menaces encore plus que les prières d’Agrippine n’avaient changé ses dispositions, au point qu’il prononça contre l’accusateur l’interdiction du feu et de l’eau ; c’est tout ce que Vitellius avait exigé.
\subsection[{Présages et famine}]{Présages et famine}
\noindent \labelchar{XLIII.} Cette année fut fertile en prodiges. On vit des oiseaux sinistres perchés sur le Capitole. De nombreux tremblements de terre renversèrent des maisons, et, dans le désordre que produisait la crainte de désastres plus étendus, les personnes les plus faibles furent écrasées par la foule. La disette de grains et la famine qu’elle causa furent aussi regardées comme des présages funestes. On ne se borna pas à de secrets murmures. Pendant que Claude rendait la justice, le peuple l’environna tout à coup avec des cris tumultueux. Il fut poussé jusqu’à l’extrémité du Forum, et on l’y pressait vivement, lorsqu’à l’aide d’un gros de soldats il perça cette multitude irritée. C’est un fait certain qu’il ne restait dans Rome que pour quinze jours de vivres, et il fallut la bonté signalée des dieux et un hiver sans orages pour la préserver des derniers malheurs. Étrange vicissitude ! jadis l’Italie envoyait ses productions dans les provinces les plus éloignées : la terre n’est pas plus stérile aujourd’hui ; mais nous cultivons de préférence l’Afrique et l’Égypte, et la vie du peuple romain est abandonnée aux hasards de la mer.
\subsection[{À l’extérieur – Guerre entre les Arméniens et les Ibères – Pharasmane roi d’Ibérie et Mithridate roi d’Arménie}]{À l’extérieur – Guerre entre les Arméniens et les Ibères – Pharasmane roi d’Ibérie et Mithridate roi d’Arménie}
\noindent \labelchar{XLIV.} La même année, une guerre survenue entre les Arméniens et les Ibères s’étendit aux Romains et aux Parthes, et donna lieu entre eux à de grands mouvements. Vologèse, né d’une concubine grecque, régnait, du consentement de ses frères, sur la nation des Parthes. Pharasmane tenait l’Ibérie de ses ancêtres, et Mithridate, son frère, devait à la protection de Rome le trône d’Arménie. Pharasmane avait un fils nommé Rhadamiste, d’une taille majestueuse, d’une force de corps extraordinaire, habile dans tous les exercices de son pays, et célèbre jusque chez les peuples voisins. Ce jeune homme trouvait que la vieillesse de son père gardait longtemps le petit royaume d’Ibérie, et il le répétait si souvent et d’un ton si animé, qu’on ne pouvait se méprendre sur ses désirs. Pharasmane, craignant pour ses années déjà sur le déclin une ambition jeune, impatiente, et soutenue par l’attachement des peuples, lui offrit un autre appât dans la conquête de l’Arménie. « Lui-même, disait-il, l’avait arrachée aux Parthes et donnée à Mithridate : toutefois il fallait différer l’emploi de la force ; la ruse était plus sûre, et on accablerait Mithridate sans qu’il fût sur ses gardes. » Alors Rhadamiste, feignant d’avoir encouru la disgrâce de son père et de céder aux haines d’une marâtre, se retire chez son oncle. Reçu par lui comme un fils, et traité avec la bonté la plus généreuse, il excite à la révolte les grands du royaume ; intrigue ignorée de Mithridate, qui le comblait chaque jour de nouveaux bienfaits.
\subsection[{Rhadamiste, fils du roi d’Ibérie convoite l’Arménie}]{Rhadamiste, fils du roi d’Ibérie convoite l’Arménie}
\noindent \labelchar{XLV.} Retourné chez son père sous prétexte d’une réconciliation, il lui annonce que tout ce qu’on pouvait attendre de la ruse est préparé, que c’est aux armes à faire le reste. Pharasmane invente alors un sujet de rupture. Il suppose qu’étant en guerre avec le roi d’Albanie, et appelant les Romains à son secours, il avait trouvé dans son frère un obstacle à ses desseins, injure dont il prétend se venger par la ruine de Mithridate. En même temps il donne à son fils des troupes nombreuses. Celui-ci, par une soudaine irruption, épouvante l’ennemi, le chasse de la campagne, et le pousse jusque dans le fort de Gornéas, défendu à la fois par sa position et par une garnison romaine sous les ordres du préfet Célius Pollio et du centurion Caspérius. Rien de plus inconnu aux barbares que l’usage des machines et fart des sièges, rien au contraire où nous excellions davantage. Aussi Rhadamiste, après avoir tenté plusieurs attaques sans succès ou avec perte, investit la place, et achète de l’avarice du préfet ce qu’il n’attend plus de la force. En vain Caspérius demandait avec instance qu’un roi allié, que le royaume d’Arménie, présent du peuple romain, ne fussent pas sacrifiés au crime et vendus pour de l’or. Pollion alléguait le grand nombre des ennemis, Rhadamiste les ordres de son père. Enfin le centurion convient d’une trêve, et part dans l’intention de décider Pharasmane à cesser la guerre, ou d’instruire le gouverneur de Syrie Ummidius Quadratus de l’état de l’Arménie.
\subsection[{Intervention des Romains}]{Intervention des Romains}
\noindent \labelchar{XLVI.} Le préfet, délivré ainsi d’un surveillant importun, presse Mithridate de traiter sans retard. Il lui rappelle les nœuds sacrés de la fraternité, l’âge plus avancé de Pharasmane, les autres liens qui l’unissent à ce prince comme époux de sa fille et beau-père de Rhadamiste. Il fait valoir la modération des Ibériens, qui ne refusent point la paix malgré leurs succès, la perfidie trop connue des Arméniens, le peu de ressources qu’offre un château dépourvu de vivres, enfin les avantages d’une capitulation qui épargnerait le sang. Mithridate n’osait se fier au préfet, qui avait séduit une de ses concubines, et qu’on croyait, pour de l’or, prêt à toutes les bassesses. Pendant qu’il hésitait, Caspérius arrive chez Pharasmane et demande que les Ibériens lèvent le siège. Le roi lui donne en public des réponses équivoques ; souvent même il feint de consentir, tandis que ses émissaires avertissent Rhadamiste de hâter par tous les moyens possibles la prise de la forteresse. On augmente le salaire du crime, et Pollion, corrupteur de ses propres soldats, les pousse secrètement à demander la paix, si l’on ne veut qu’ils abandonnent la place. Vaincu par la nécessité, Mithridate accepte une entrevue où le traité doit se conclure, et sort du château.
\subsection[{Rhadamiste tue Mithridate et s’empare de l’Arménie}]{Rhadamiste tue Mithridate et s’empare de l’Arménie}
\noindent \labelchar{XLVII.} A son arrivée, Rhadamiste se jette dans ses bras, lui prodigue les marques de respect, les noms de père et de beau-père. Il ajoute le serment de n’employer contre lui ni le fer ni le poison ; puis il l’entraîne dans un bois voisin où il avait, disait-il, ordonné les apprêts d’un sacrifice, afin que la paix fût scellée en présence des dieux. L’usage de ces rois, quand ils font une alliance, est de se prendre mutuellement la main droite et de s’attacher ensemble les pouces par un nœud très-serré. Lorsque le sang est venu aux extrémités, une légère piqûre le fait jaillir, et chacun des contractants suce celui de l’autre. Cette consécration du sang leur paraît donner au traité une force mystérieuse. Celui qui était chargé d’appliquer le lien feignit de tomber, et, saisissant les genoux de Mithridate, le renversa lui-même. Aussitôt ce prince est environné, chargé de chaînes, et entraîné les fers aux pieds, ce qui est chez les barbares le dernier des opprobres. Le peuple, traité durement sous son règne, s’en vengea par des injures et des gestes menaçants. Il en était aussi dont cette grande vicissitude de la fortune excitait la pitié. Sa femme suivait avec ses enfants en bas âge, et faisait retentir l’air de ses lamentations. On les enferma séparément dans des chariots couverts, jusqu’à ce qu’on eût pris les ordres de Pharasmane. Un frère et une fille n’étaient rien pour ce barbare auprès d’une couronne, et son âme était disposée à tous les crimes. Cependant, par un reste de pudeur, il ne les fit pas tuer devant lui. De son côté, Rhadamiste se souvint de son serment : il n’employa, contre son oncle et sa sœur, ni le fer ni le poison ; mais il les fit étendre par terre, et étouffer sous un amas d’étoffes pesantes. Les fils même de Mithridate furent égorgés pour avoir pleuré en voyant périr les auteurs de leurs jours.
\subsection[{Réaction des Romains}]{Réaction des Romains}
\noindent \labelchar{XLVIII.} A la nouvelle de la trahison qui avait mis le royaume de Mithridate au pouvoir de ses meurtriers, Quadratus assemble son conseil, expose les faits, met en délibération s’il en tirera vengeance. L’honneur public eut peu de défenseurs. Le plus grand nombre, inclinant pour le parti le plus sûr, soutinrent que ces crimes étrangers devaient faire notre joie, qu’il fallait même jeter parmi les barbares des semences de haine, comme avaient fait plusieurs fois les empereurs romains en donnant cette même Arménie moins comme un présent que comme un sujet de discordes. « Que Rhadamiste jouisse de son injuste conquête, pourvu qu’il en jouisse odieux et décrié ; elle servirait moins bien les intérêts de Rome, si elle était plus glorieuse » Cet avis prévalut. Cependant, pour ne point paraître approuver un crime, et dans la crainte que Claude ne donnât des ordres contraires, on fit sommer Pharasmane d’abandonner l’Arménie et d’en rappeler son fils.
\subsection[{Julius Pélignus le corrompu}]{Julius Pélignus le corrompu}
\noindent \labelchar{XLIX.} La Cappadoce avait pour procurateur Julius Pélignus, homme à qui les difformités de son corps, autant que la lâcheté de son âme, attiraient le mépris, mais l’un des familiers de Claude, à l’époque où celui-ci, encore simple particulier, amusait avec des bouffons ses stupides loisirs. Pélignus lève dans sa province un corps d’auxiliaires, comme pour reconquérir l’Arménie ; et, après avoir pillé les alliés plutôt que les ennemis, abandonné des siens, assailli par les barbares, dépourvu de ressources, il se rend chez Rhadamiste. Gagné par l’or de ce prince, il l’exhorta le premier à ceindre le diadème, et, satellite d’un ennemi, il autorisa cette cérémonie par sa présence. Quand cette honteuse nouvelle fut divulguée, pour montrer que tous les Romains n’étaient pas des Pélignus, on envoya le lieutenant Helvidius Priscus à la tête d’une légion, avec pouvoir de remédier au désordre selon les circonstances. Helvidius franchit rapidement le mont Taurus ; et déjà, par la douceur plus que par la force, il avait commencé à rétablir le calme, lorsqu’il reçut l’ordre de rentrer en Syrie, de peur d’occasionner une guerre avec les Parthes.
\subsection[{Les Parthes arrivent}]{Les Parthes arrivent}
\noindent \labelchar{L.} Car Vologèse, croyant le moment arrivé de reprendre l’Arménie, possédée jadis par ses ancêtres et devenue par un crime la proie de l’étranger, avait rassemblé des troupes, et se préparait à placer sur ce trône Tiridate, son frère, afin que sa famille ne comptât que des rois. L’arrivée des Parthes suffit, même sans combat, pour chasser les Ibères, et les villes arméniennes d’Artaxate et de Tigranocerte acceptèrent le joug. Ensuite un hiver rigoureux, le défaut de vivres, dû peut-être à l’imprévoyance, et les maladies produites par cette double cause, forcèrent Vologèse de quitter pour le moment sa conquête. Voyant l’Arménie abandonnée, Rhadamiste y rentra plus terrible que jamais : il avait une rébellion à punir, et il en craignait une nouvelle. En effet, les Arméniens, quoique faits à la servitude, éclatèrent enfin, et coururent en armes investir le palais.
\subsection[{Fuite de Rhadamite – sa femme Zénobie}]{Fuite de Rhadamite – sa femme Zénobie}
\noindent \labelchar{LI.} Rhadamiste n’eut d’autre ressource que la vitesse de ses chevaux, sur lesquels il s’enfuit accompagné de sa femme. Celle-ci était enceinte : toutefois la crainte de l’ennemi et la tendresse conjugale lui donnèrent des forces, et elle supporta le mieux qu’elle put les premières fatigues. Bientôt, les continuelles secousses d’une course prolongée lui déchirant les entrailles, elle conjure son époux de la soustraire par une mort honorable aux outrages de la captivité. Rhadamiste l’embrasse, la soutient, l’encourage, passant tour à tour de l’admiration pour son héroïsme à la crainte de la laisser au pouvoir d’un autre. Enfin, transporté de jalousie, habitué d’ailleurs aux grands attentats, il tire son cimeterre, l’en frappe, et, l’ayant traînée au bord de l’Araxe, il l’abandonne au courant du fleuve, pour que son corps même ne puisse être enlevé. Pour lui, il gagne précipitamment les États de son père. Cependant Zénobie (c’était le nom de cette femme) flotta doucement jusque sur la rive, respirant encore et donnant des signes manifestes de vie. Des bergers l’aperçurent ; et, jugeant à la noblesse de ses traits qu’elle n’était pas d’une naissance commune, ils bandent sa plaie, y appliquent les remèdes connus aux champs ; ensuite, instruits de son nom et de son aventure, ils la portent dans la ville d’Artaxate. De là elle fut conduite, par les soins des magistrats, à la cour de Tiridate, qui la reçut avec bonté et la traita en reine.
\subsection[{Année 52 – À Rome}]{Année 52 – À Rome}
\noindent \labelchar{LII.} Pendant le consulat de Faustus Sylla et de Salvius Otho, Furius Scribonianus fut exilé sous prétexte qu’il avait interrogé des astrologues sur l’époque de la mort du prince. On lui reprochait en outre les plaintes de sa mère, bannie elle-même, et qui, disait-on, supportait impatiemment sa disgrâce. Le père de Scribonianus, Camille, avait essayé une révolte en Dalmatie, et Claude se piqua de clémence en épargnant pour la seconde fois une race ennemie. Au reste, l’exilé ne jouit pas longtemps de la vie qui lui était laissée. Il mourut, naturellement suivant les uns, suivant d’autres par le poison. On rendit, pour chasser les astrologues d’Italie, un sénatus-consulte rigoureux, mais sans effet. Ensuite le prince loua dans un discours les sénateurs qui, à cause de la médiocrité de leur fortune, se retiraient volontairement du sénat, et il en exclut ceux qui, s’obstinant à y rester, ajoutaient l’impudence à la pauvreté.
\subsection[{Sanctions contre les rapports sexuels matrones-esclaves – Pallas félicité}]{Sanctions contre les rapports sexuels matrones-esclaves – Pallas félicité}
\noindent \labelchar{LIII.} On délibère ensuite sur la punition des femmes qui auraient commerce avec des esclaves. Il fut décidé qu’elles seraient elles-mêmes tenues pour esclaves, si elles s’étaient ainsi dégradées à l’insu du maître ; pour affranchies, si c’était de son aveu. Claude ayant déclaré que l’idée de ce règlement était due à Pallas, le consul désigné, Baréa Soranus, proposa de lui décerner les ornements de la préture et quinze millions de sesterces. Cornélius Scipion voulut en outre qu’on le remerciât, au nom de l’État, de ce qu’étant issu des rois d’Arcadie il sacrifiait au bien public une très-ancienne noblesse, et consentait à être compté parmi les serviteurs du prince. Claude assura que Pallas, content de l’honneur, voulait rester dans sa pauvreté ; et un sénatus-consulte fut gravé sur l’airain et publiquement affiché, où un affranchi, possesseur de trois cents millions de sesterces \footnote{Prés de trois millions de notre monnaie (2 922 534 fr.)}, était loué comme le parfait modèle de l’antique désintéressement.
\subsection[{Félix, frère de Pallas – Les Galiléens et les Samaritains}]{Félix, frère de Pallas – Les Galiléens et les Samaritains}
\noindent \labelchar{LIV.} II était en effet désintéressé, en comparaison de son frère surnommé Félix, depuis longtemps procurateur en Judée, et qui, soutenu de l’énorme crédit de Pallas, croyait l’impunité assurée d’avance à tous ses crimes. Il est vrai que les Juifs avait donné des signes de rébellion en se soulevant contre l’ordre de placer dans, leur temple la statue de Caïus. Caïus était mort, et l’ordre resté sans exécution, mais la crainte qu’un autre prince n’en donnât un pareil subsistait tout entière. De son côté, Félix aigrissait le mal par des remèdes hors de saison, et Ventidius Cumanus n’imitait que trop bien ses excès. Cumanus administrait une partie de la province : il avait sous ses ordres les Galiléens, Félix les Samaritains, nations de tout temps ennemies, et dont les haines, sous des chefs méprisés, éclataient sans contrainte. Chaque jour on voyait ces deux peuples se piller mutuellement, envoyer l’un chez l’autre des troupes de brigands, se dresser des embuscades, se livrer même de véritables combats, et rapporter aux procurateurs les dépouilles et le butin. Ceux-ci s’en réjouirent d’abord : bientôt, alarmés des progrès de l’incendie, ils voulurent l’arrêter avec des soldats, et les soldats furent taillés en pièces. La guerre eût embrasé la province, si Quadratus, gouverneur de Syrie, ne fût venu la sauver. Les Juifs qui avaient eu l’audace de massacrer nos soldats ne donnèrent pas lieu à une longue délibération : ils payèrent ce crime de leur tête. Cumanus et Félix embarrassèrent davantage le général : car le prince, informé des causes de la révolte, lui avait donné pouvoir de prononcer même sur ses procurateurs. Mais Quadratus montra Félix parmi les juges, et, en le faisant asseoir sur son tribunal, il étouffa les voix prêtes à l’accuser. Cumanus fut condamné seul pour les crimes que deux avaient commis, et le calme fut rendu à la province.
\subsection[{Révolte en Cilicie}]{Révolte en Cilicie}
\noindent \labelchar{LV.} Peu de temps après, les tribus sauvages de Cilicie, connues sous le nom de Clites, et qui déjà s’étaient soulevées plus d’une fois, se révoltèrent de nouveau, conduites par Trosobore, et campèrent sur des montagnes escarpées. De là, elles descendaient sur les côtes et jusque dans les villes, et enlevaient les habitants ou les laboureurs, mais surtout les marchands et les maîtres de navires. La ville d’Anémur fut assiégée par ces barbares, et des cavaliers envoyés de Syrie avec le préfet Curtius Sévérus, pour la secourir, furent mis en déroute, à cause de l’âpreté du terrain, qui était favorable à des gens de pied, tandis que la cavalerie n’y pouvait combattre. Enfin le roi de ce pays, Antiochus, en flattant la multitude et en trompant le chef, parvint à désunir les forces des rebelles ; et, après avoir fait mourir Trosobore et quelques autres des plus marquants, il ramena le reste par la clémence.
\subsection[{Grands travaux}]{Grands travaux}
\noindent \labelchar{LVI.} Vers le même temps, on acheva de couper la montagne qui sépare le lac Fucin \footnote{Aujourd’hui le lac de Célano, dans l’Abruzze ultérieure. 2. Le Garigliano.} du Liris (2) ; et, afin que la magnificence de l’ouvrage eût plus de spectateurs, on donna sur le lac même un combat naval, comme avait fait Auguste sur un bassin construit en deçà du Tibre. Mais Auguste avait employé des vaisseaux plus petits et moins de combattants. Claude arma des galères à trois et quatre rangs de rames, qui furent montées par dix-neuf mille hommes. Une enceinte de radeaux fermait tout passage à la fuite, et embrassait cependant un espace où pouvaient se déployer la force des rameurs, l’art des pilotes, la vitesse des navires, et toutes les manœuvres d’un combat. Sur les radeaux étaient rangées des troupes prétoriennes, infanterie et cavalerie, et devant elles on avait dressé des parapets d’où l’on pût faire jouer les catapultes et les balistes. Les combattants, sur des vaisseaux pontés, occupaient le reste du lac. Les rivages, les collines, le penchant des montagnes, formaient un vaste amphithéâtre, où se pressait une foule immense, accourue des villes voisines et de Rome même, par curiosité ou pour plaire à César. Claude, revêtu d’un habit de guerre magnifique, et non loin de lui Agrippine, portant aussi une chlamyde tissue d’or, présidèrent au spectacle. Le combat, quoique entre des criminels, fut digne des plus braves soldats. Après beaucoup de sang répandu, on les dispensa de s’entr’égorger.\par
\labelchar{LVII.} Le spectacle achevé, on ouvrit passage aux eaux, et alors parut à découvert l’imperfection de l’ouvrage : le canal destiné à la décharge du lac ne descendait pas à la moitié de sa profondeur. On prit du temps pour creuser davantage ; et, afin d’attirer de nouveau la multitude, on donna un combat de gladiateurs sur des ponts construits à ce dessein. Un repas fut même servi près du lieu où le lac devait se verser dans le canal, et devint l’occasion d’une terrible épouvante. Cette masse d’eau violemment élancée entraîna tout sur son passage, et ce qu’elle n’atteignit pas fut ébranlé par la secousse ou effrayé par le fracas et le bruit. Agrippine, profitant de la terreur du prince pour l’animer contre Narcisse, directeur de ces travaux, l’accusa de cupidité et de vol. Narcisse ne manqua pas d’accuser à son tour le caractère impérieux de cette femme et son ambition démesurée.
\subsection[{53 – Mariage de Néron et d’Octavie}]{53 – Mariage de Néron et d’Octavie}
\noindent \labelchar{LVIII.} Sous les consuls D. Junius et Q. Hatérius, Néron, âgé de seize ans, reçut en mariage Octavie, fille de Claude. Afin d’illustrer sa jeunesse par un emploi honorable du talent et par les succès de l’éloquence, on le chargea de la cause d’Ilium. Après avoir rappelé dans un brillant discours l’origine troyenne des Romains, Énée, père des Jules, et d’autres traditions qui touchent de près à la fable, il obtint que les habitants d’Ilium fussent exemptés de toutes charges publiques. A la demande du même orateur, la colonie de Bologne, ruinée par un incendie, reçut un secours de dix millions de sesterces ; la liberté fut rendue aux Rhodiens \footnote{Les Rhodiens avaient perdu la liberté neuf ans auparavant, pour avoir mis en croix des citoyens romains.}, qui l’avaient souvent perdue ou recouvrée, selon qu’ils nous avaient servis dans nos guerres ou offensés par leurs séditions ; enfin le tribut fut remis pour cinq ans à la ville d’Apamée, renversée par un tremblement de terre.
\subsection[{Convoitise d’Agrippine}]{Convoitise d’Agrippine}
\noindent \labelchar{LIX.} Cependant les artifices d’Agrippine poussaient Claude aux plus odieuses cruautés. Statilius Taurus avait de grandes richesses : elle convoita ses jardins, et, afin de le perdre, elle lui suscita pour accusateur Tarquitius Priscus. Cet homme avait été lieutenant de Taurus, proconsul en Afrique. A leur retour, il l’accusa de concussion, mais en alléguant peu de griefs ; il lui reprochait surtout des superstitions magiques. Taurus ne supporta pas longtemps les impostures de la calomnie et le rôle humiliant d’accusé. Il se donna la mort avant la décision du sénat. Tarquitius fut cependant chassé de cet ordre : les sénateurs, indignés, de sa délation, remportèrent ce triomphe sur les intrigues d’Agrippine.
\subsection[{Pouvoirs accrus pour les procurateurs}]{Pouvoirs accrus pour les procurateurs}
\noindent \labelchar{LX.} Dans le cours de cette année, on entendit Claude répéter souvent que les jugements de ses procurateurs \footnote{Les procurateurs étaient les intendants des domaines et des revenus du prince. Là se bornaient leurs fonctions dans les provinces gouvernées par des proconsuls ou des propréteurs. Mais eux-mêmes tenaient lieu de gouverneurs dans certaines provinces moins importantes, comme les deux Mauritanies, la Rhétie, la Norique, la Thrace, etc.} devaient avoir la même force que si c’était lui qui les eût prononcés ; et, afin qu’on ne prît pas ces paroles pour un propos sans conséquence, un sénatus-consulte y pourvut par une concession plus formelle et plus étendue que jamais. Déjà l’empereur Auguste avait donné aux chevaliers qui gouvernaient l’Égypte l’administration de la justice, et avait voulu que leurs décisions fussent aussi respectées que si elles émanaient des magistrats romains. Bientôt furent ainsi partagées, dans les autres provinces et à Rome même, des attributions qui anciennement n’appartenaient qu’aux préteurs. Enfin Claude livra tout entier un droit qui donna lieu jadis à tant de séditions ou de combats, lorsque les lois semproniennes mettaient l’ordre équestre en possession des jugements, ou qu’à leur tour les lois serviliennes les rendaient au sénat ; un droit qui, plus que tout le reste, arma l’un contre l’autre Sylla et Marius. Mais alors c’était une lutte entre les ordres de l’État, et le parti vainqueur dominait à titre de puissance publique. C. Oppius et Cornélius Balbus furent les premiers que la volonté d’un homme, le dictateur César, érigea en négociateurs de la paix et en arbitres de la guerre. Il n’est pas besoin de citer après eux les Matius, les Médius, et tant d’autres chevaliers fameux par leur immense pouvoir, quand on voit Claude égaler à lui-même et aux lois les affranchis qu’il avait chargés de ses affaires domestiques.
\subsection[{Claude favorise l’île de Cos}]{Claude favorise l’île de Cos}
\noindent \labelchar{LXI.} Le prince fit ensuite la proposition d’exempter de tributs l’île de Cos, et s’étendit beaucoup sur l’antiquité du peuple qui l’habite. Il dit « que les Argiens, ou Céus, père de Latone, y avaient les premiers établi leur séjour ; qu’ensuite Esculape y avait apporté l’art de la médecine, art cultivé avec éclat par ses descendants," dont il cita les noms et fixa les époques. Il ajouta « que Xénophon, à la science duquel lui-même avait ordinairement recours, était issu de cette famille ; qu’il fallait accorder à ses prières une immunité qui fît de l’île de Cos une terre sacrée à jamais, et vouée sans partage au culte de son dieu. » Nul doute que cette nation n’eût des titres à la reconnaissance du peuple romain, et l’on aurait pu citer plusieurs de nos victoires auxquelles son courage l’avait associée. Mais Claude, avec sa facilité irréfléchie, négligea d’appuyer sur des raisons politiques une faveur qui était toute personnelle.
\subsection[{Supplications des Byzantins}]{Supplications des Byzantins}
\noindent \labelchar{LXII.} Les Byzantins, admis à l’audience du sénat ; implorèrent une diminution des charges qui pesaient sur eux, et n’omirent aucun de leurs titres. Ils rappelèrent d’abord le traité de paix qu’ils avaient fait avec nous, dans le temps de notre guerre contre ce roi de Macédoine, qui, usurpant une origine illustre, reçut le nom de faux Philippe. Ils parlèrent ensuite des troupes qu’ils nous avaient fournies contre Antiochus, Persée, Aristonicus ; de leur zèle à seconder Antoine contre les pirates ; des secours qu’ils avaient offerts à Sylla, à Lucullus, à Pompée ; enfin des services plus récents qu’avait rendus aux Césars une ville placée si avantageusement pour le passage, soit par terre, soit par mer, de nos armées et de nos généraux, ainsi que pour le transport des approvisionnements.\par
\labelchar{LXIII.} En effet, c’est aux lieux où l’Europe et l’Asie sont séparées par le plus petit intervalle, que les Grecs ont fondé Byzance, à l’endroit même où l’Europe finit. Ils avaient consulté sur l’emplacement de leur ville Apollon Pythien, et l’oracle leur avait répondu de chercher une demeure en face de la terre des aveugles. Ce nom mystérieux désignait les Chalcédoniens, qui, arrivés les premiers sur ces côtes, et pouvant choisir la meilleure position, avaient pris la plus mauvaise. Prés de Byzance, la terre et la mer sont également fécondes. Une quantité innombrable de poissons \footnote{Le thon.} qui se jettent hors de l’Euxin, apercevant sous l’eau une barre de rochers, s’éloignent effrayés de la côte d’Asie, et refluent vers ce port. Ce fut pour les Byzantins une source de commerce et d’opulence. Des charges énormes les accablèrent ensuite : ils en sollicitaient alors la fin ou la diminution ; le prince appuya leur demande, en disant qu’ils étaient épuisés par les dernières guerres de Thrace et du Bosphore, et qu’il était juste de les soulager. Les tributs leur furent remis pour cinq ans.
\subsection[{54 – À Rome – Prodiges – Rivalité entre Agrippine et Lépida}]{54 – À Rome – Prodiges – Rivalité entre Agrippine et Lépida}
\noindent \labelchar{LXIV.} Sous le consulat de M. Asinius et de M. Acilius, des prodiges nombreux annoncèrent dans l’État de funestes changements. Des enseignes militaires et des tentes furent brûlées par le feu du ciel ; un essaim d’abeilles alla se poser au faite du Capitole ; on débita que des femmes avaient donné le jour à des monstres, et qu’un porc était né avec des serres d’épervier. On comptait encore au nombre des présages sinistres la diminution qu’éprouvèrent dans leur nombre tous les collèges de magistrats, un questeur, un édile, un tribun, un préteur, un consul, étant morts dans l’espace de quelques mois. Mais Agrippine était plus que personne tourmentée par la crainte. Une parole échappée à Claude dans l’ivresse la faisait trembler : il avait dit que sa destinée était de supporter les désordres de ses femmes et de les punir ensuite. C’est pourquoi elle résolut d’agir, et d’agir au plus tôt. Mais elle immola d’abord à la vanité de son sexe Domina Lépida. Fille d’Antonia la jeune, petite-nièce d’Auguste, cousine germaine du père d’Agrippine, et sœur de son premier mari Domitius, Lépida se croyait son égale du côté de la noblesse. La beauté, l’âge, les richesses différaient peu entre l’une et l’autre. Toutes deux impudiques, déshonorées, violentes, elles étaient rivales de vices autant que de fortune. Mais la grande querelle était à qui, de la mère ou de la tante, aurait le plus d’ascendant sur Néron. Lépida enchaînait ce jeune cœur par les présents et les caresses. Agrippine, au contraire, ne lui montrait qu’un visage sévère et menaçant : elle voulait bien donner l’empire à son fils, elle ne pouvait souffrir qu’il en exerçât les droits.
\subsection[{Lépida condamnée à mort malgré l’opposition de Narcisse}]{Lépida condamnée à mort malgré l’opposition de Narcisse}
\noindent \labelchar{LXV.} Au reste, Lépida fut accusée d’avoir essayé, contre l’hymen du prince, des enchantements sacrilèges, et d’entretenir en Calabre des légions d’esclaves dont l’indiscipline troublait la paix de l’Italie. L’arrêt de mort fut prononcé, malgré l’opposition de Narcisse, qui, se défiant de plus en plus d’Agrippine, s’en ouvrit avec ses amis les plus intimes, et leur dit « que sa perte était certaine, soit que Britannicus, soit que Néron succédât à l’empire ; mais que la reconnaissance lui faisait une loi de s’immoler pour le service de Claude ; qu’il avait convaincu Messaline et Silius ; que les mêmes raisons d’accuser se présentaient de nouveau, accusation qui le perdrait si Néron venait à régner, et ne le sauverait pas si c’était Britannicus ; que cependant les intrigues d’une marâtre bouleversaient tout le palais, et qu’il y aurait plus de honte à se taire qu’il n’y en aurait eu à dissimuler les impudicités de la précédente épouse ; qu’au reste la pudeur n’était pas moins outragée par celle qui se prostituait à Pallas : elle témoignait assez, par cet avilissement d’elle-même, que la décence, que l’honneur, que rien enfin n’était sacré pour son ambition. » En tenant ces discours et d’autres semblables, il embrassait Britannicus ; il priait les dieux de hâter pour lui l’âge de la force ; il tendait les mains tantôt vers le ciel, tantôt vers le jeune homme, et lui souhaitait de croître, de chasser les ennemis de son père, dût-il punir aussi les meurtriers de sa mère.
\subsection[{Empoisonnement de Claude}]{Empoisonnement de Claude}
\noindent \labelchar{LXVI.} En proie à de si graves soucis, Narcisse tomba malade et se rendit à Sinuesse, dans l’espoir que la douce température de l’air et la salubrité des eaux rétabliraient ses forces. Agrippine, dont le crime, résolu depuis longtemps, avait des ministres tout prêts, saisit avidement l’occasion. Le choix du poison l’embarrassait un peu : trop soudain et trop prompt, il trahirait une main criminelle ; si elle en choisissait un qui consumât la vie dans une langueur prolongée, Claude, en approchant de son heure suprême, pouvait deviner le complot et revenir à l’amour de son fils. Il fallait un venin d’une espèce nouvelle, qui troublât la raison, sans trop hâter la mort. On jeta les yeux sur une femme habile en cet art détestable, nommée Locusta, condamnée depuis peu pour empoisonnement, et qui fut longtemps, pour les maîtres de l’empire, un instrument de pouvoir. Le poison fut préparé par le talent de cette femme, et donné par la main de l’eunuque Halotus, dont la fonction était de servir les mets et de les goûter.\par
\labelchar{LXVII.} Tous les détails de ce crime devinrent bientôt si publics que les écrivains du temps n’en omettent aucun. Le poison fut mis dans un ragoût de champignons, mets favori du prince. La stupidité de Claude, ou peut-être l’ivresse, en déguisèrent l’effet pendant quelque temps. La nature, en soulageant ses entrailles, parut même l’avoir sauvé. Agrippine effrayée, et bravant tout parce qu’elle avait tout à craindre, s’adressa au médecin Xénophon, dont elle s’était assuré d’avance la complicité. Celui-ci, sous prétexte d’aider le vomissement, enfonça, dit on, dans le gosier de Claude une plume imprégnée d’un poison subtil, bien convaincu que, s’il y a du péril à commencer les plus grands attentats, on gagne à les consommer.\par
\labelchar{LXVIII.} Cependant le sénat s’assemblait, les consuls et les prêtres offraient des vœux pour la conservation du prince, tandis que son corps déjà sans vie était soigneusement enveloppé dans son lit, où l’on affecta de lui prodiguer des soins, jusqu’à ce que le pouvoir de Néron fût établi sans retour. Dès le premier instant, Agrippine, feignant d’être vaincue par la douleur et de chercher des consolations, courut auprès de Britannicus. Elle le serrait dans ses bras, l’appelait la vivante image de son père, empêchait par mille artifices qu’il ne sortit de son appartement. Elle retint de même ses sueurs Antonia et Octavie. Des gardes fermaient par ses ordres toutes les avenues du palais, et elle publiait de temps en temps que la santé du prince allait mieux, afin d’entretenir l’espérance des soldats et d’attendre le moment favorable marqué par les astrologues.\par
\labelchar{LXIX.} Enfin, le trois avant les ides d’octobre, à midi, les portes du palais s’ouvrent tout à coup, et Néron, accompagné de Burrus, s’avance vers la cohorte qui, suivant l’usage militaire, faisait la garde à ce poste. Au signal donné par le préfet, Néron est accueilli avec des acclamations et placé dans une litière. Il y eut, dit-on, quelques soldats qui hésitèrent, regardant derrière eux, et demandant où était Britannicus. Mais, comme il ne s’offrait point de chef à la résistance, ils suivirent l’impulsion qu’on leur donnait. Porté dans le camp, Néron fit un discours approprié aux circonstances, promit des largesses égales à celles de son père, et fut salué empereur. Cet arrêt des soldats fut confirmé par les actes du sénat ; il n’y eut aucune hésitation dans les provinces. Les honneurs divins furent décernés à Claude, et ses funérailles célébrées avec la même pompe que celles d’Auguste ; car Agrippine fut jalouse d’égaler la magnificence de sa bisaïeule Livie. Toutefois on ne lut pas le testament, de peur que l’injustice d’un père qui sacrifiait son fils au fils de sa femme ne révoltât les esprits et ne causât quelque trouble.
\section[{Livre treizième (54, 58)}]{Livre treizième (54, 58)}\renewcommand{\leftmark}{Livre treizième (54, 58)}

\subsection[{A Rome – Mort de Junius Silanus et de Narcisse}]{A Rome – Mort de Junius Silanus et de Narcisse}
\noindent \labelchar{I.} Le premier meurtre du nouveau règne fut celui de Junius Silanus, proconsul d’Asie, que préparèrent, à l’insu de Néron, les intrigues d’Agrippine. Ce n’est pas qu’il eût provoqué sa destinée par la violence de son humeur : c’était un homme indolent, et tenu sous les maîtres précédents en un si grand dédain, que Caïus l’appelait souvent la brebis d’or \footnote{Caïus se moquait des richesses de Silanus et de son indolence.} ; mais Agrippine, qui avait tramé la perte de son frère Lucius, craignait un vengeur ; et la voix publique préférait hautement à Néron, à peine sorti de l’enfance et parvenu à l’empire par un crime, un homme irréprochable, d’un âge mur, d’un nom illustre, et, ce qu’alors on eût considéré, du sang des Césars. Car Silanus descendait aussi au quatrième degré de l’empereur Auguste : ce fut la cause de sa mort. Le chevalier romain P. Celer et l’affranchi Hélius, tous deux intendants des domaines du prince en Asie, en furent les instruments. Ils empoisonnèrent le proconsul à table, et avec si peu de précaution que personne ne s’y méprit. La perte de Narcisse, affranchi de Claude, dont j’ai rapporté les invectives contre Agrippine, ne fut pas moins précipitée : une prison rigoureuse, puis un ordre fatal, le forcèrent de se tuer. Sa mort affligea Néron, dont les vices, encore cachés, s’accordaient merveilleusement avec l’avarice et la prodigalité de cet affranchi.
\subsection[{D’un côté Sénèque et Burrus, de l’autre Pallas}]{D’un côté Sénèque et Burrus, de l’autre Pallas}
\noindent \labelchar{II.} On allait se précipiter dans les meurtres, si Burrus et Sénèque ne s’y fussent opposés. Ces deux hommes, qui gouvernaient la jeunesse de l’empereur avec un accord peu commun dans un pouvoir partagé, exerçaient, à des titres divers, une égale influence : Burrus par ses talents militaires et la sévérité de ses mœurs, Sénèque par ses leçons d’éloquence et les grâces dont il parait la sagesse ; travaillant de concert à sauver le prince des périls de son âge, et, si la vertu l’effarouchait, à le contenir au moins par des plaisirs permis. Ils n’avaient l’un et l’autre à combattre que la violence d’Agrippine, qui, tourmentée de tous les délires d’un pouvoir malfaisant, était soutenue de Pallas, auteur du mariage incestueux et de la funeste adoption par lesquels Claude s’était perdu lui-même. Il est vrai que Néron n’était pas de caractère à plier sous un esclave ; et Pallas, par son humeur triste et hautaine, sortant des bornes de sa condition, s’était rendu à charge. Toutefois on accumulait publiquement les honneurs sur Agrippine : un tribun, suivant l’usage militaire, étant venu à l’ordre, Néron lui donna pour mot, " la meilleure des mères. » Le sénat lui décerna deux licteurs, avec le titre de prêtresse de Claude, et à Claude des funérailles solennelles, ensuite l’apothéose.
\subsection[{Eloge funèbre de Claude composé par Sénèque}]{Eloge funèbre de Claude composé par Sénèque}
\noindent \labelchar{III.} Le jour des obsèques, Néron prononça l’éloge funèbre. Tant qu’il vanta dans Claude l’ancienneté de sa race, les consulats et les triomphes de ses ancêtres, l’attention de l’auditoire soutint l’orateur. On se prêta mème à l’entendre louer ses connaissances littéraires, et rappeler que, sous son règne, la république n’avait essuyé aucun échec au dehors ; mais, quand il en vint à la sagesse et à la prévoyance de Claude, personne ne put s’empècher de rire. Cependant le discours, ouvrage de Sénèque, était paré de tous les ornements de l’éloquence ; on sait combien cet ëcrivain avait un esprit agréable et assorti au goût de ses contemporains. Les vieillards, qui s’amusent à rapprocher le passé du présent, remarquaient que Néron était le premier des maîtres de l’empire qui eût eu besoin de recourir au talent d’autrui. Le dictateur César ne le cédait pas aux plus grands orateurs. Auguste avait l’élocution abondante ; et facile qui convient à un prince ; Tibère savait, de plus, peser ses expressions avec un art merveilleux, donnant de la force à sa pensée, ou l’enveloppant à dessein. Dans Caïus même, le désordre de la raison ne détruisit pas l’énergie de la parole ; et Claude, toutes les fois qu’il avait préparé ses discours, ne manquait pas d’une certaine élégance. Néron, dès son enfance, tourna d’un autre côté la vivacité de son esprit : il s’exerçait à graver, à peindre, à chanter ou à conduire des chars. Quelquefois aussi des poésies de sa composition prouvèrent qu’il avait au moins une teinture des lettres.\par
\labelchar{IV.} Quand on eut donné à l’imitation de la douleur ce que demande l’usage, Néron fit son entrée au sénat. Après avoir fondé son droit sur l’autorité de cet ordre et le vœu unanime des soldats, il ajouta « qu’il avait, pour bien gouverner, tout ce qu’il faut de conseils et d’exemples ; que ni guerres civiles ni querelles domestiques n’avaient aigri sa jeunesse ; qu’il n’apportait au rang supréme ni haine, ni offenses reçues, ni désir de vengeance ». Puis il traça le plan de son règne futur, écartant surtout les abus dont l’odieux souvenir était encore présent. « Ainsi, on ne le verrait point, juge de tous les procès, enfermer dans le secret du palais l’accusation et la défense, afin que le pouvoir de quelques hommes y triomphât sans obstacle. Si la vénalité ni la brigue ne pénétreraient à sa cour ; sa maison et l’État seraient deux choses distinctes ; le sénat pouvait reprendre ses antiques fonctions, l’ltalie et les provinces du peuple romain s’adresser au tribunal des consul : par eux, on aurait accès auprès des pères conscrits ; lui, chargé des armées, leur réservait tous ses soins. »\par
\labelchar{V.} Ces paroles ne furent pas vaines, et l’indépendance du sénat parut dans plusieurs décisions : ainsi l’on défendit aux orateurs de faire acheter leurs services par des présents ou de l’argent ; et les questeurs désignés furent dispensés de donner des combats de gladiateurs. En vain Agrippine prétendit que c’était renverser les actes de Claude ; le vœu des sénateurs prévalut. Les séances se tenaient au palais, afin qu’à la faveur d’une porte dérobée elle pût y assister derrière un voile, qui l’empéchait d’être vue sans l’empêcher d’entendre. Elle fit plus : un jour que des ambassadeurs arméniens plaidaient devant Néron la cause de leur pays, elle se préparait à monter sur le tribunal de l’empereur et à siéger près de lui, si, bravant la crainte qui tenait les autres immobiles, Sénèque n’eût averti le prince d’aller au-devant de sa mère. Ainsi le respect filial servit de prétexte pour prévenir un déshonneur public.\par
\labelchar{VI.} A la fin de l’année, de sinistres rumeurs annoncèrent une nouvelle irruption des Parthes, l’envahissement de l’Arménie et l’expulsion de Rhadamiste, qui, souvent maître de ce royaume et souvent fugitif, avait alors renoncé même à la guerre. Rome est avide d’entretiens ; elle se demandait « comment un prince à peine âgé de dix-sept ans pourrait soutenir un pareil fardeau ou s’en délivrer. Qu’attendre d’un enfant gouverné par une femme ? Ses précepteurs dirigeraient-ils aussi les combats, les sièges et toutes les opérations de la guerre ? " D’autres se félicitaient « que ce ne fùt pas Claude, un vieillard imbécile, qui fût appelé aux travaux guerriers, pour les conduire au gré de ses esclaves. Burrus, après tout, et Sénèque n’avaient-ils pas donné mille preuves de capacité ? et l’empereur même, que manquait-il à la force de son àge, puisque Pompée à dix-huit ans, Octavien à dix-neuf ans avaient soutenu le poids des guerres civiles ? Les auspices et la politique d’un prince font plus que son épée et son bras. Neron montrera clairement s’il place bien on mal son amitié, selon qu’il saura, en dépit de l’envie, choisir un habile capitaine, ou que, cédant à. la brigue, il préférera quelque riche en faveur."\par
\labelchar{VII.} Ainsi s’expliquaient hautement les opinions diverses ; et cependant Néron faisait venir, pour compléter les légions d’Orient, des troupes levées dans les provinces voisines, et ordonnait aux légions elles-mémes de se rapprocher de l’Arménie. Il manda, en outre, aux deux anciens rois Antiochus et Agrippa de tenir leurs troupes prêtes à entrer sur le territoire des Parthes. Des ponts sont jetés sur l’Euphrate ; l’Arménie mineure est donnée avec les ornements de la royauté à Aristobule, et le pays de Sophène à Sohémus. Enfin, un rival s’élève à propos contre Vologèse dans la personne de son fils Vardanes, et les Parthes quittent l’Arménie, en hommes qui ne font que différer la guerre.\par
\labelchar{VIII.} L’importance de ces événements fut exagérée dans le sénat par le vote de plusieurs jours d’actions de grâces, auquel il fut ajouté que Néron porterait pendant leur durée la robe triomphale ; qu’il entrerait dans Rome avec les bonheurs de l’ovation, et que des statues d’une grandeur égale à celle de Mars Vengeur lui seraient érigées dans le temple de ce dieu. A l’esprit d’adulation s’unissait la joie de voir Corbulon choisi pour sauver l’Arménie, et l’idée que la carrière était ouverte au mérite. Les troupes d’Orient furent ainsi divisées : une partie des auxiliaires et deux légions restèrent en Syrie sous le commandement d’Ummidius Quadratus, gouverneur de cette province ; un pareil nombre de Romains et d’étrangers furent donnés à Corbulon, avec les cohortes et la cavalerie qui étaient en quartier d’hiver dans la Cappadoce. Les rois alliés eurent ordre d’obéir à l’un ou à l’autre, suivant les besoins de la guerre ; mais leur zèle inclinait de préférence vers Corbulon. Ce général, pour se concilier la renommée, dont l’influence est décisive au commencement de toute entreprise, fait une marche rapide et arrive à Egée, ville de Cilicie. Il y trouva Quadratus, qui s’était avancé jusque-là, dans la crainte que, si Corbulon entrait en Syrie pour y prendre son armée, il n’attirât sur lui tous les regards, imposant par sa taille, magnifique dans son langage, et joignant à l’expérience et au talent ces vains dehors qui ont aussi leur puissance.\par
\labelchar{IX.} Au reste, nos deux généraux conseillaient, par des envoyés, au roi Vologèse de préférer la paix à la guerre, et d’imiter, en donnant des otages au peuple romain, la déférence de ses prédécesseurs. Vologèse, soit pour faire à loisir ses préparatifs, soit pour écarter, sous le nom d’otages, ceux dont il craignait la rivalité, livra les plus nobles des Arsacides. Ils furent reçus par le centurion Histéius, qui, envoyé par Quadratus et arrivé le premier, eut à ce sujet une entrevue avec le roi. A cette nouvelle, Corbulon ordonne au préfet de cohorte Arrius Varus d’aller les retirer de ses mains. Une querelle s’engagea entre le préfet et le centurion ; et, afin de ne pas donner plus longtemps ce spectacle aux barbares, on prit pour arbitres les otages eux-mêmes et les ambassadeurs qui les conduisaient. Ceux-ci, par respect pour une gloire récente, et cédant, quoique ennemis, à un secret penchant, préférèrent Corbulon. La discorde se mit alors entre les deux chefs : Quadratus se plaignit qu’on lui enlevait le fruit de ses négociations ; Corbulon protestait au contraire « que les Parthes n’avaient songé à offrir des otages qu’au moment où, choisi pour leur faire la guerre, il avait changé en crainte leurs espérances. » Néron, pour les mettre d’accord, fit publier « qu’en l’honneur des succès de Quadratus et de Corbulon, les faisceaux de l’empereur seraient ornés de lauriers. » Une partie de ces faits anticipe sur le consulat, suivant ; je les ai réunis.
\subsection[{Piété filiale et débuts prometteurs.}]{Piété filiale et débuts prometteurs.}
\noindent \labelchar{X.} La même année, le prince demanda au sénat une statue pour son père, Cn. Domitius, et les ornements consulaires pour Asconius Labéo, qui avait été son tuteur. On lui offrait à lui-même des statues d’argent ou d’or massif : il les refusa ; et, quoique les sénateurs eussent émis le vœu que désormais le nouvel an s’ouvrit au mois de décembre, où était né le prince, il conserva aux calendes de janvier leur solennel et antique privilège de commencer l’année. Il ne voulut pas qu’on mît en jugement le sénateur Carinas Céler, accusé par un esclave, ni Julius Densus, chevalier romain, auquel on faisait un crime de son attachement à Britannicus.
\subsection[{55}]{\textsc{55}}
\noindent \labelchar{XI.} Sous le consulat de Néron et de L. Antistius, comme les magistrats juraient sur les actes des princes, Néron défendit à son collègue de jurer sur les siens : modestie à laquelle le sénat prodigua les éloges, afin que ce jeune cœur, animé par la gloire qui s’attachait aux plus petites choses, s’élevât jusqu’aux grandes. Ce trait fut suivi d’un exemple de douceur envers Plautius Latéranus, chassé du sénat comme coupable d’adultère avec Messaline : Néron le rendit à son ordre, engageant solennellement sa clémence, dans de fréquentes harangues que Sénèque, pour attester la sagesse de ses leçons ou pour faire briller son génie, publiait par la bouche du prince.
\subsection[{Néron amoureux d’Acté}]{Néron amoureux d’Acté}
\noindent \labelchar{XII.} Cependant le pouvoir d’Agrippine fut ébranlé peu à peu par l’amour auquel son fils s’abandonna pour une affranchie nommée Acté, et l’ascendant que prirent deux jeunes et beaux favoris qu’il mit dans sa confidence, Othon, issu d’une famille consulaire, et Sénécion, fils d’un affranchi du palais. Leur liaison avec le prince, ignorée d’abord, puis vainement combattue par sa mère, était née au sein des plaisirs, et avait acquis, dans d’équivoques et mystérieuses relations, une intimité chaque jour plus étroite. Au reste, ceux même des amis de Néron qui étaient plus sévères ne mettaient pas d’obstacle à son penchant pour Acté ; ce n’était après tout qu’une femme obscure, et les désirs du prince étaient satisfaits sans que personne eût à se plaindre. Car son épouse Octavie joignait en vain la noblesse à la vertu : soit fatalité, soit attrait plus puissant des voluptés défendues, il n’avait que de l’aversion pour elle ; et il était à craindre que, si on lui disputait l’objet de sa fantaisie, il ne portât le déshonneur dans les plus illustres maisons.
\subsection[{Agrippine jalouse}]{Agrippine jalouse}
\noindent \labelchar{XIII.} Mais Agrippine, avec toute l’aigreur d’une femme offensée ; se plaint qu’on lui donne une affranchie pour rivale, une esclave pour bru. Au lieu d’attendre le repentir de son fils ou la satiété, elle éclate en reproches, et plus elle l’en accable, plus elle allume sa passion. Enfin Néron, dompté par la violence de son amour, dépouille tout respect pour sa mère, et s’abandonne à Sénèque. Déjà un ami de ce dernier, Annéus Sérénus, feignant d’aimer lui-même l’affranchie, avait prêté son nom pour voiler la passion naissante du jeune prince ; et les secrètes libéralités de Néron passaient en public pour des présents de Sérénus. Alors Agrippine change de système, et emploie pour armes les caresses : c’est son appartement, c’est le sein maternel, qu’elle offre pour cacher des plaisirs dont un si jeune âge et une si haute fortune ne sauraient se passer. Elle s’accuse même d’une rigueur hors de saison ; et ouvrant son trésor, presque aussi riche que celui du prince, elle l’épuise en largesses ; naguère sévère à l’excès pour son fils, maintenant prosternée à ses pieds. Ce changement ne fit pas illusion à Néron. D’ailleurs les plus intimes de ses amis voyaient le danger, et le conjuraient de se tenir en garde contre les pièges d’une femme toujours implacable, et alors implacable à la fois et dissimulée. Il arriva que vers ce temps Néron fit la revue des ornements dont s’étaient parées les épouses et les mères des empereurs, et choisit une robe et des pierreries qu’il envoya en présent à sa mère. Il n’avait rien épargné : il offrait les objets les plus beaux, et ces objets, que plus d’une femme avait désirés, il les offrait sans qu’on les demandât. Mais Agrippine s’écria : « que c’était moins l’enrichir d’une parure nouvelle que la priver de toutes les autres, et que son fils lui faisait sa part dans un héritage qu’il tenait d’elle tout entier. » On ne manqua pas de répéter ce mot et de l’envenimer.
\subsection[{Disgrâce de Pallas – Agrippine hystérique}]{Disgrâce de Pallas – Agrippine hystérique}
\noindent \labelchar{XIV.} Irrité contre ceux dont s’appuyait cet orgueil d’une femme, le prince ôte à Pallas la charge qu’il tenait de Claude \footnote{Entre autres, Sénèque et Burrus. 2. Il s’agit sans doute d’Antonia, l’aïeule de Néron.}, et qui mettait en quelque sorte le pouvoir dans ses mains. On rapporte qu’en le voyant se retirer suivi d’un immense cortège, Néron dit assez plaisamment que Pallas allait abdiquer : il est certain que cet affranchi avait fait la condition que le passé ne donnerait lieu contre lui à aucune recherche, et qu’il serait quitte envers la république. Cependant Agrippine, forcenée de colère, semait autour d’elle l’épouvante et la menace ; et, sans épargner même les oreilles du prince, elle s’écriait « que Britannicus n’était plus un enfant ; que c’était le véritable fils de Claude, le digne héritier de ce trône, qu’un intrus et un adopté n’occupait que pour outrager sa mère. Il ne tiendrait pas à elle que tous les malheurs d’une maison infortunée ne fussent mis au grand jour, à commencer par l’inceste et le poison. Grâce aux dieux et à sa prévoyance, son beau-fils au moins vivait encore : elle irait avec lui dans le camp ; on entendrait d’un côté la fille de Germanicus, et de l’autre l’estropié Burrus et l’exilé Sénèque, venant, l’un avec son bras mutilé, l’autre avec sa voix de rhéteur, solliciter l’empire de l’univers. » Elle accompagne ces discours de gestes violents, accumule les invectives, en appelle à la divinité de Claude, aux mânes des Silanus, à tant de forfaits inutilement commis.1. Pallas était maître des comptes et trésorier de Claude ; indépendamment des revenus particuliers de l’empereur, il administrait encore les finances de l’État.
\subsection[{Essai d’empoisonnement sur Britannicus}]{Essai d’empoisonnement sur Britannicus}
\noindent \labelchar{XV.} Néron, alarmé de ces fureurs, et voyant Britannicus près d’achever sa quatorzième année, rappelait tour à tour à son esprit et les emportements de sa mère, et le caractère du jeune homme, que venait de révéler un indice léger, sans doute, mais qui avait vivement intéressé en sa faveur. Pendant les fêtes de Saturne, les deux frères jouaient avec des jeunes gens de leur âge, et, dans un de ces jeux, on tirait au sort la royauté ; elle échut à Néron. Celui-ci, prés avoir fait aux autres des commandements dont ils pouvaient s’acquitter sans rougir, ordonne à Britannicus de se lever, de s’avancer et de chanter quelque chose. Il comptait faire rire aux dépens d’un enfant étranger aux réunions les plus sobres, et plus encore aux orgies de l’ivresse. Britannicus, sans se déconcerter, chanta des vers dont le sens rappelait qu’il avait été précipité du rang suprême et du trône paternel. On s’attendrit, et l’émotion fut d’autant plus visible que la nuit et la licence avaient banni la feinte. Néron comprit cette censure, et sa haine redoubla. Agrippine par ses menaces en hâta les effets. Nul crime dont on pût accuser Britannicus, et Néron n’osait publiquement commander le meurtre d’un frère : il résolut de frapper en secret, et fit préparer du poison. L’agent qu’il choisit fut Julius Pollio, tribun d’une cohorte prétorienne, qui avait sous sa garde Locuste, condamnée pour empoisonnement, et fameuse par beaucoup de forfaits. Dés longtemps on avait eu soin de ne placer auprès de Britannicus que des hommes pour qui rien ne fût sacré : un premier breuvage lui fut donné par ses gouverneurs trop faible, soit qu’on l’eût mitigé, pour qu’il ne tuât pas sur-le-champ. Néron, qui ne pouvait souffrir cette lenteur dans le crime, menace le tribun, ordonne le supplice de l’empoisonneuse, se plaignant, que, pour prévenir de vaines rumeurs et se ménager une apologie, ils retardaient sa sécurité. Ils lui promirent alors un venin qui tuerait aussi vite que le fer : il fut distillé auprès de la chambre du prince, et composé de poisons d’une violence éprouvée.
\subsection[{La seconde fois est toujours la bonne}]{La seconde fois est toujours la bonne}
\noindent \labelchar{XVI.} C’était l’usage que les fils des princes mangeassent assis avec les autres nobles de leur âge, sous les yeux de leurs parents, à une table séparée et plus frugale. Britannicus était à l’une de ces tables. Comme il ne mangeait ou ne buvait rien qui n’eût été goûté par un esclave de confiance, et qu’on ne voulait ni manquer à cette coutume, ni déceler le crime par deux morts à la fois, voici la ruse qu’on imagina. Un breuvage encore innocent, et goûté par l’esclave, fut servi à Britannicus ; mais la liqueur était trop chaude, et il ne put la boire. Avec l’eau dont on la rafraîchit, on y versa le poison, qui circula si rapidement dans ses veines qu’il lui ravit en même temps la parole et la vie. Tout se trouble autour de lui : les moins prudents s’enfuient ; ceux dont la vue pénètre plus avant demeurent immobiles, les yeux attachés sur Néron. Le prince, toujours penché sur son lit et feignant de ne rien savoir, dit que c’était un événement ordinaire, causé par l’épilepsie dont Britannicus était attaqué depuis l’enfance ; que peu à peu la vue et le sentiment lui reviendraient. Pour Agrippine, elle composait inutilement son visage : la frayeur et le trouble de son âme éclatèrent si visiblement qu’on la jugea aussi étrangère à ce crime que l’était Octavie, sueur de Britannicus : et en effet, elle voyait dans cette mort la chute de son dernier appui et l’exemple du parricide. Octavie aussi, dans un âge si jeune, avait appris à cacher sa douleur, sa tendresse, tous les mouvements de son âme. Ainsi, après un moment de silence, la gaieté du festin recommença.
\subsection[{Funérailles de Britannicus}]{Funérailles de Britannicus}
\noindent \labelchar{XVII.} La même nuit vit périr Britannicus et allumer son bûcher. L’apprêt des funérailles était fait d’avance ; elles furent simples : toutefois ses restes furent ensevelis au Champ-de-Mars ; il tombait une pluie si violente, que le peuple y vit un signe de la colère des dieux contre un forfait que bien des hommes ne laissaient pas d’excuser, en se rappelant l’histoire des haines fraternelles et en songeant qu’un trône ne se partage pas. Presque tous les écrivains de ce temps rapportent que, les derniers jours avant l’empoisonnement, Néron déshonora par de fréquents outrages l’enfance de Britannicus. Ainsi, quoique frappé à la table sacrée du festin, sous les yeux de son ennemi, et si rapidement qu’il ne put même recevoir les embrassements d’une sœur, on ne trouve plus sa mort ni prématurée, ni cruelle, quand on voit l’impureté souiller, avant le poison, ce reste infortuné du sang des Claudius. Néron excusa par un édit la précipitation des obsèques. « C’était, disait-il, la coutume de nos ancêtres, de soustraire aux yeux les funérailles du jeune âge, sans en prolonger l’amertume par une pompe et des éloges funèbres. Quant à lui, privé de l’appui d’un frère, il n’avait plus d’espérance que dans la république ; nouveau motif pour le sénat et le peuple d’entourer de leur bienveillance un prince qui restait seul d’une famille née pour le rang suprême. » Ensuite il combla de largesses les principaux de ses amis.
\subsection[{Néron se méfie de sa mère}]{Néron se méfie de sa mère}
\noindent \labelchar{XVIII.} On ne manqua pas de trouver étrange que des hommes qui professaient une morale austère (1) se fussent, dans un pareil moment, partagé comme une proie des terres et des maisons. Quelques-uns pensèrent qu’ils y avaient été forcés par le prince, dont la conscience coupable espérait se faire pardonner son crime, en enchaînant par des présents ce qu’il y avait de plus accrédité dans l’Etat. Mais aucune libéralité n’apaisa-le courroux de sa mère : elle serre Octavie dans ses bras ; elle a de fréquentes et secrètes conférences avec ses amis ; à son avarice naturelle parait se joindre une autre prévoyance, et elle ramasse de l’argent de tous côtés, accueillant d’un air gracieux tribuns et centurions, honorant les noms illustres et les vertus que Rome possède encore, comme si elle cherchait un chef et des partisans. Agrippine conservait, comme mère de l’empereur, la garde qu’elle avait eue en qualité d’épouse : Néron, instruit de ses manœuvres, ordonna qu’elle en fût privée, ainsi que des soldats germains qu’il y avait ajoutés par surcroît d’honneur. Pour éloigner d’elle la foule des courtisans, il sépara leurs deux maisons et transporta sa mare dans l’ancien palais d’Antonia (2). Lui-même n’y allait jamais qu’escorté de centurions, et il se retirait après un simple baiser.
\subsection[{Complot d’Agrippine}]{Complot d’Agrippine}
\noindent \labelchar{XIX.} Rien au monde n’est aussi fragile et aussi fugitif qu’un renom de pouvoir qui n’est pas appuyé sur une force réelle. Le seuil d’Agrippine est aussitôt désert ; personne ne la console, personne ne la visite, si ce n’est quelques femmes qu’attire l’amitié, ou la haine peut-être. Parmi elles était Junia Silana, que Messaline avait chassée, comme je l’ai raconté plus haut, du lit de Silius. Silana, célèbre par sa naissance, sa beauté, la licence de ses mœurs, fut longtemps chérie d’Agrippine. De secrètes inimitiés avaient rompu leur intelligence, depuis qu’Agrippine, à force de répéter que c’était une femme dissolue et surannée, avait dégoûté de sa main un jeune noble, Sextius Africanus ; non sans doute en vue de se réserver Sextius pour elle-même, mais afin d’empêcher les biens de Silana, riche et sans enfants, de tomber au pouvoir d’un mari. Celle-ci crut tenir l’occasion de se venger : elle suscite parmi ses clients deux accusateurs, Iturius et Calvisius. Sans s’arrêter aux reproches tant de fois renouvelés de pleurer Britannicus, de divulguer les chagrins d’Octavie, ce qu’elle dénonce est plus grave : « Agrippine médite une révolution en faveur de Rubellius Plautus, descendant d’Auguste par les femmes au même degré que Néron ; ensuite, par le partage de son lit et de son trône, elle envahira de nouveau la puissance suprême. » Iturius et Calvisius révèlent ces projets à un affranchi de Domitia, tante de Néron, nommé Atimétus. Joyeux de cette confidence (car il régnait entre Agrippine et Domitia une mortelle jalousie), Atimétus détermine un autre affranchi de Domina, l’histrion Paris, à courir chez le prince et à présenter la dénonciation sous les plus noires couleurs.
\subsection[{L’histion Paris dénonce le complot à Néron}]{L’histion Paris dénonce le complot à Néron}
\noindent \labelchar{XX.} La nuit était avancée, et Néron prolongeait les heures de la débauche, quand Paris se présenta. C’était le moment où il avait coutume de venir chez le prince, afin d’animer ses plaisirs. L’air triste qu’il avait pris cette fois, et les complots dont il fit le détail, effrayèrent tellement Néron, que sa première idée fut de tuer sa mère et Plautus. Il voulait même ôter à Burrus le commandement du prétoire, sur le soupçon que, tenant tout d’Agrippine, il la payait de retour. Si l’on en croit Fabius Rusticus, un ordre fut écrit, qui transportait cette charge à Tuscus Cécina \footnote{Cécina Tuscus était fils de la nourrice de Néron. 2. Cluvius Rufus, sous Galba, était gouverneur d’Espagne. On pense que c’est le même qui écrivit dans la suite l’histoire de son temps.} ; mais le crédit de Sénèque sauva cet affront à Burrus. Pline et Cluvius (2) disent qu’il ne s’éleva aucun doute sur la fidélité du préfet. Il est certain que Fabius incline à louer Sénèque, auteur de sa fortune : pour moi, l’accord des écrivains me sert de règle ; quand ils diffèrent, je rapporte les faits sous leur nom. Néron, troublé par la peur et impatient de se délivrer de sa mère, ne consentit à différer que quand Burrus lui eut promis qu’elle mourrait si elle était convaincue. « Mais tout accusé, une mère surtout, avait droit de se défendre. Où étaient les accusateurs ? La seule voix qui s’élevât partait d’une maison ennemie : et que de choses devaient mettre en défiance, les ténèbres, les veilles d’une nuit de plaisir, tant de causes d’erreur et de surprise ! »
\subsection[{Agrippine se défend}]{Agrippine se défend}
\noindent \labelchar{XXI.} La frayeur du prince fut un peu calmée, et au retour de la lumière on alla chez Agrippine, afin que, l’accusation entendue, elle se justifiât ou fût punie. Burrus porta la parole en présence de Sénèque : quelques affranchis assistaient comme témoins de l’entretien. Après avoir exposé les griefs et nommé les dénonciateurs, Burrus prit le ton de la menace. Alors Agrippine, rappelant toute sa fierté : « Je ne m’étonne pas, dit-elle, que Silana, qui n’eut jamais d’enfants, ne connaisse point le cœur d’une mère ; non, une mère ne change pas de fils comme une prostituée d’amants. Si Calvisius et Iturius, après avoir dévoré leur fortune, n’ont d’autre ressource que de vendre à une vieille courtisane leurs délations mercenaires, faut-il que j’encoure le soupçon d’un parricide, ou que César en subisse le remords ? Quant à Domitia, je rendrais grâce à sa haine, si elle disputait avec moi de tendresse pour mon cher Néron. Mais la voilà qui arrange avec son favori Atimétus et l’histrion Paris des scènes de théâtre. Elle construisait à Baïes ses magnifiques réservoirs, tandis que Néron, adopté, revêtu de la puissance proconsulaire, désigné consul, voyait tomber par mes soins toutes les barrières qui le séparaient du trône. Qu’une voix s’élève et me convainque d’avoir sollicité une cohorte dans Rome, ébranlé la fidélité des provinces, corrompu des esclaves ou des affranchis. Hélas ! pouvais-je espérer de vivre, si Britannicus eût régné ? Et maintenant, que Plautus ou tout autre s’empare du pouvoir et devienne mon juge, manquerai-je d’accusateurs prêts à me reprocher, non des paroles indiscrètes, échappées à une tendresse jalouse, mais des crimes dont mon fils seul peut absoudre sa mère ? " Ceux qui étaient présents furent vivement émus et cherchèrent à calmer ses transports. Elle demanda alors une entrevue avec son fils : elle n’y parla ni de son innocence, dont elle eût paru se défier, ni de ses bienfaits, ce qui eût semblé un reproche ; mais elle obtint la punition de ses dénonciateurs, et des récompenses pour ses amis.
\subsection[{Châtiment}]{Châtiment}
\noindent \labelchar{XXII.} La préfecture des vivres fut donnée à Fénius Rufus ; Arruntius Stella fut chargé des jeux que préparait César, et C. Balbillus eut le gouvernement de l’Égypte. La Syrie fut promise à P. Antéius ; mais on éluda son départ sous différents prétextes, et il fut enfin retenu à Rome. Silana fut envoyée en exil, Iturius et Calvisius relégués, et Atimétus livré au supplice. Paris était trop nécessaire aux plaisirs du prince pour être puni ; quant à Plautus, on ne parla pas de lui pour le moment.
\subsection[{Attaques contre Pallas et Burrus}]{Attaques contre Pallas et Burrus}
\noindent \labelchar{XXIII.} Bientôt Pallas et Burrus furent accusés d’avoir fait un complot pour donner l’empire à Cornélius Sylla, né d’une race illustre, et honoré de l’alliance de Claude, dont l’hymen d’Antonia l’avait rendu gendre. Cette délation était l’ouvrage d’un certain Pétus, solliciteur odieusement célèbre de confiscations et d’enchères \footnote{II s’était perdu de réputation en recherchant les biens sur lesquels le trésor avait des droits, et qui avaient échappé à la confiscation.}, et qui fut alors convaincu d’imposture. L’innocence de Pallas fit moins de plaisir que son orgueil ne révolta. En entendant nommer quelques-uns de ses affranchis, qu’on lui donnait pour complices, il répondit « que jamais il n’avait commandé chez lui que des yeux ou du geste, et que, s’il fallait de plus longues explications, il écrivait, pour ne pas prostituer ses paroles. » Burrus, quoique accusé, opina parmi les juges. L’accusateur fut puni de l’exil, et l’on brûla des registres où il faisait revivre des créances du trésor anciennement éteintes.
\subsection[{Néron fait retirer les forces de l’ordre dans les jeux publics}]{Néron fait retirer les forces de l’ordre dans les jeux publics}
\noindent \labelchar{XXIV.} A la fin de l’année, la cohorte qui faisait la garde aux jeux publics en fut retirée, afin que la liberté parût plus entière, et que le soldat, cessant d’être mêlé à la licence du théâtre, en fût moins corrompu. On voulait voir encore si le peuple serait paisible quand il n’aurait plus de surveillants. Le prince, sur une réponse des aruspices, purifia la ville, parce que la foudre était tombée sur les temples de Jupiter et de Minerve.
\subsection[{56 – A Rome – Néron le jeune débauché}]{56 – A Rome – Néron le jeune débauché}
\noindent \labelchar{XXV.} Le consulat de Q. Volusius et de P. Scipion vit au dehors une paix profonde, au dedans les plus scandaleux désordres. Néron parcourait les rues de la ville, les lieux de débauche, les tavernes, déguisé en esclave, et accompagné de gens qui pillaient les marchandises et blessaient les passants. On le reconnaissait si peu, que lui-même recevait des coups dont il porta les marques au visage. Quand on sut que l’auteur de ces violences était César, les outrages se multiplièrent contre les hommes et les femmes du premier rang. Une fois la licence autorisée par le nom du prince, d’autres commirent impunément, avec leurs bandes, de semblables excès, et Rome offrait chaque nuit l’image d’une ville prise. Julius Montanus, de l’ordre sénatorial, mais qui n’était pas encore parvenu aux honneurs, rencontra Néron dans les ténèbres, et repoussa vivement son attaque ; il le reconnut ensuite, fit des excuses qu’on prit pour des reproches, et fut contraint de se tuer. Néron cependant, devenu plus timide, s’entoura de soldats et de gladiateurs. Tant que la lutte n’était pas trop violente, ils la traitaient comme une querelle privée et laissaient faire ; si la résistance était un peu vigoureuse, ils interposaient leurs armes. La licence du théâtre et les cabales en faveur des histrions furent aussi encouragées par l’impunité et les récompenses : Néron en fit presque des combats, dont il jouissait sans être vu, et que plus souvent encore il contemplait publiquement. Enfin la discorde allumée parmi le peuple fit craindre de plus dangereux mouvements, et l’on ne trouva d’autre remède que de chasser les histrions d’Italie, et de placer de nouveau des soldats au théâtre.
\subsection[{Problème des affranchis}]{Problème des affranchis}
\noindent \labelchar{XXVI.} Vers le même temps, des plaintes s’élevèrent dans le sénat contre les trahisons des affranchis, et l’on demanda, que les patrons eussent le droit de punir l’ingratitude en révoquant la liberté. Beaucoup de sénateurs étaient prêts à donner leur avis ; mais le prince n’était pas prévenu, et les consuls n’osèrent ouvrir la délibération : toutefois ils lui transmirent par écrit le vœu du sénat. Néron délibéra dans son conseil s’il autoriserait ce règlement. Les opinions furent partagées : quelques-uns s’indignaient des excès où s’emportait l’insolence enhardie par la liberté. « C’était peu que l’affranchi fût l’égal de son maître ; déjà il osait lever sur lui un bras menaçant, et cette violence restait impunie, ou la punition faisait rire le coupable. Quelle vengeance était permise en effet au patron offensé, que de reléguer son affranchi au delà du vingtième mille, aux beaux rivages de Campanie ? Dans tout le reste, nulle différence entre eux devant les tribunaux. Il fallait aux maîtres une arme qu’on ne pût braver. Il en coûterait peu aux affranchis de conserver la liberté comme ils l’avaient acquise, par de justes égards. Quant aux auteurs de crimes manifestes, ils méritaient bien de rentrer dans l’esclavage : ainsi les âmes insensibles aux bienfaits seraient contenues par la crainte. »\par
\labelchar{XXVII.} D’autres soutinrent « que les coupables devaient porter la peine de leurs fautes, sans que, pour un petit nombre, on attaquât les droits de tous ; que ce corps était répandu dans toute la société ; qu’il servait à recruter les tribus \footnote{Le peuple romain était divisé en trente-cinq tribus, dont trente et une de la campagne, et quatre de la ville. Celles-ci comprenaient les prolétaires, les capite censi, enfin tout le menu peuple ; aussi étaient-elles moins honorables que les autres. C’est par cette raison qu’on y faisait entrer les affranchis.}, les décuries (2), les cohortes même de la ville \footnote{Les gardes nocturnes, établies par Auguste et composées d’abord d’affranchis.} ; qu’on en tirait les officiers des magistrats et des prêtres ; que la plupart des chevaliers et beaucoup de sénateurs n’avaient pas une autre origine ; que, si l’on faisait des affranchis une classe séparée, la disette de citoyens nés libres paraîtrait à découvert. Non, ce n’est pas en vain que nos pères, en faisant à chacun des ordres sa part de dignité, laissèrent la liberté commune et indivise ; ils instituèrent même deux sortes d’affranchissement, afin qu’on eût le temps, ou de changer d’avis, ou de confirmer son bienfait par un autre. L’esclave que son maître n’a pas rendu libre dans la forme solennelle tient encore à la servitude par une dernière chaîne. C’est à chacun de peser le mérite, et de ne pas accorder légèrement un don irrévocable. » Cet avis prévalut. Le prince écrivit au sénat d’examiner les plaintes des patrons contre les affranchis toutes les fois qu’il s’en présenterait, mais de ne rien statuer de général. Peu de temps après, la tante de Néron se vit enlever, par un abus du droit civil, son affranchi Paris, non sans honte pour le prince, qui fit prononcer par jugement que Paris était né libre.
\subsection[{Dispute entre tribun et préteur}]{Dispute entre tribun et préteur}
\noindent \labelchar{XXVIII.} Toutefois, il subsistait encore un fantôme de république. Une contestation s’éleva entre le préteur Vibullius et Antistius, tribun du peuple, au sujet de quelques séditieux arrêtés par le préteur pour leur violence dans les cabales du théâtre, et relâchés par ordre du tribun. Le sénat blâma cet ordre comme un excès de pouvoir, et se déclara pour Vibullius. En même temps on défendit aux tribuns d’usurper la juridiction des préteurs ou des consuls, ou de citer devant eux aucune personne d’Italie contre laquelle les voies légales seraient ouvertes. L. Pison, consul désigné, fit ajouter qu’ils ne prononceraient dans leur maison aucune condamnation ; que nulle amende imposée par eux ne serait portée sur les registres publics par les questeurs de l’épargne, qu’après un délai de quatre mois ; que, pendant ce temps, on pourrait en appeler, et que les consuls statueraient sur l’appel. On restreignit aussi le pouvoir des édiles, et l’on détermina ce que les édiles curules, ce que les édiles plébéiens pourraient prendre de gages \footnote{Le citoyen qui ne se rendait pas à la citation d’un magistrat, le sénateur qui, dûment convoqué, ne venait pas à l’assemblée, étaient contraints par une saisie que l’on exerçait sur leurs meubles. C’est ce que l’on appelait pignus capere. Ce gage répondait de l’amende à laquelle était condamné celui qui ne justifiait pas son absence par un motif légitime.} ou infliger d’amende. Helvidius Priscus, tribun du peuple, profita de ce moment pour satisfaire ses ressentiments particuliers contre Obultronius Sabinus, questeur de l’épargne, qu’il accusait d’aggraver sans pitié le droit de saisie contre les pauvres. Bientôt le prince ôta aux questeurs les registres du trésor pour les confier à des préfets.
\subsection[{Les préfets}]{Les préfets}
\noindent \labelchar{XXIX.} Cette partie de l’administration publique changea souvent de forme. Auguste laissa d’abord au sénat le soin d’élire des préfets ; ensuite on craignit la brigue, et l’on substitua des préteurs, pris au sort parmi ceux de l’année. Cet usage ne dura pas non plus, parce que le sort s’égarait sur des hommes peu capables. Alors Claude rendit l’épargne aux questeurs ; et, pour encourager leur sévérité contre la crainte de déplaire, il leur promit les honneurs par privilège. Mais, comme c’était leur première magistrature, il leur manquait la maturité de l’âge ; Néron choisit donc d’anciens préteurs, dont l’expérience offrit une garantie.
\subsection[{Condamnations et suicides}]{Condamnations et suicides}
\noindent \labelchar{XXX.} Sous les mêmes consuls, Vipsanius Lénas fut condamné pour ses rapines dans le gouvernement de la Sardaigne. Accusé de concussion, Cestius Proculus fut absous, sur le désistement de ses accusateurs. Clodius Quirinalis, préfet des galères stationnées à Ravenne, qui s’était conduit en Italie comme chez la dernière des nations, et l’avait désolée par sa débauche et sa cruauté, prévint son jugement en prenant du poison. Caninius Rébilus, un des premiers de Rome par son habileté et par ses immenses richesses, se déroba, en s’ouvrant les veines, aux tourments d’une vieillesse infirme : c’est un courage qu’on n’attendait pas d’un homme dont les mœurs infâmes faisaient mentir son sexe. L. Volusius mourut aussi, mais environné de l’estime publique : une carrière de quatre-vingt-treize ans, de grands biens légitimement acquis, tant de règnes tyranniques traversés sans disgrâce, tel fut le partage de Volusius.
\subsection[{57 – Rien à signaler}]{57 – Rien à signaler}
\noindent \labelchar{XXI.} Peu d’événements mémorables signalèrent l’année où Néron, consul pour la seconde fois, eut L. Pison pour collègue ; à moins qu’on ne veuille employer des volumes à vanter les fondements et la charpente du vaste amphithéâtre que le prince fit construire au Champ-de-Mars. Mais la dignité du peuple romain ne veut dans un livre d’annales que des faits éclatants ; elle laisse ces détails aux journaux de la ville. Les colonies de Capoue et de Nucérie reçurent un renfort de vétérans. Quatre cents sesterces par tête furent distribués au peuple à titre de largesse, et quarante millions furent portés au trésor public, pour assurer le crédit de l’empire. Le vingt-cinquième dû sur les achats d’esclaves fut supprimé, suppression plus apparente que réelle ; car le vendeur, obligé de payer cet impôt, élevait d’autant le prix de la vente. Un édit de César défendit aux magistrats et aux procurateurs de donner dans leurs provinces ni spectacles de gladiateurs, ni combats d’animaux, ni jeux d’aucune espèce. Auparavant, de telles libéralités n’étaient pas moins que leurs rapines un fléau pour les sujets, en mettant sous la protection de la popularité les crimes de l’avarice.\par
\labelchar{XXXII.} Un sénatus-consulte, tout ensemble de vengeance et de sécurité, ordonna que, si un maître était tué par ses esclaves, ceux qu’il aurait affranchis par son testament subiraient comme les autres le dernier supplice, s’ils habitaient sous le même toit. On rendit au sénat le consulaire Lucius Varius, condamné autrefois comme concussionnaire. Une femme de la première distinction, Pomponia Grécina, épouse de Plautius, auquel ses exploits en Bretagne avaient mérité l’ovation, fut accusée de se livrer à des superstitions étrangères, et abandonnée au jugement de son mari. Arbitre de la vie et de l’honneur de sa femme, Plautius, d’après l’ancien usage, instruisit son procès devant un conseil de famille, et la déclara innocente. Pomponia vécut longtemps et toujours dans les larmes : car, après que les intrigues de Messaline eurent fait périr Julie, fille de Drusus, pendant quarante ans elle ne porta que des habits de deuil, ne s’occupa que de sa douleur ; constance impunie sous Claude, et qui fut après lui un titre de gloire.\par
\labelchar{XXIII.} La même année vit plusieurs accusations, entre autres celle de P. Celer, que dénonçait la province d’Asie. Néron, ne pouvant l’absoudre, traîna le procès en longueur jusqu’à ce que l’accusé mourût de vieillesse. Céler avait empoisonné, comme je l’ai déjà dit, le proconsul Silanus, et la grandeur de ce crime couvrait tous les autres. Cossutianus Capito était poursuivi par les Ciliciens comme un infâme chargé de souillures, et dont l’audace s’était arrogé dans la province les mêmes droits qu’elle avait usurpés à Rome. Lassé par la persévérance des accusateurs, il renonça enfin à se défendre, et fut condamné d’après la loi sur la concussion. Éprius Marcellus, attaqué en restitution par les Lyciens, dut à la brigue un succès plus heureux : son crédit fut assez fort pour faire exiler quelques-uns des accusateurs, sous prétexte qu’ils avaient mis en péril un innocent.
\subsection[{58 – À l’extérieur – Les Parthes et Corbulon}]{58 – À l’extérieur – Les Parthes et Corbulon}
\noindent \labelchar{XXXIV.} Néron, dans son troisième consulat, eut pour collègue Valérius Messala, dont quelques vieillards se ressouvenaient encore d’avoir vu le bisaïeul, l’orateur Corvinus, exercer cette magistrature avec Auguste, trisaïeul de Néron. L’éclat de cette noble famille fut accru par le don qu’on offrit à Messala de cinq cent mille sesterces par an, pour l’aider à soutenir son honorable pauvreté. Aurélius Cotta et Hatérius Antoninus reçurent aussi du prince un revenu annuel, quoiqu’ils eussent dissipé dans les prodigalités du luxe les richesses de leurs pères. Au commencement de cette année, la guerre entre les Parthes et les Romains pour la possession de l’Arménie, mollement engagée et traînée jusqu’alors en longueur, éclata vivement. Vologèse ne voulait pas que son frère Tiridate fût privé d’un trône qu’il tenait de ses mains, ni qu’il le possédât comme le don d’une puissance étrangère. De son côté, Corbulon croyait digne de la grandeur romaine de recouvrer les conquêtes de Lucullus et de Pompée. Enfin, la foi indécise des Arméniens appelait tour à tour les deux partis. Toutefois ce peuple, par la position des lieux, ainsi que par les mœurs, se rapprochait des Parthes ; et, confondu avec eux par les mariages, ignorant d’ailleurs la liberté, c’est d’eux qu’une préférence naturelle le portait à recevoir des maîtres.
\subsection[{Réorganisation des légions romaines}]{Réorganisation des légions romaines}
\noindent \labelchar{XXXV.} Mais la perfidie de l’ennemi donna moins d’embarras à Corbulon que la lâcheté de ses troupes. Amollies par une longue paix, les légions appelées de Syrie supportaient impatiemment les travaux du soldat romain. On tint pour constant qu’il y avait dans cette armée des vétérans qui n’avaient jamais ni veillé, ni monté la garde ; la vue d’un fossé et d’un retranchement les étonnait comme un spectacle nouveau. Sans casques, sans cuirasses, occupés de se parer ou de s’enrichir, c’était dans les villes qu’ils avaient accompli le temps de leur service. Corbulon congédia ceux que l’âge ou les infirmités avaient affaiblis, et demanda des recrues. Des levées se firent dans la Galatie et dans la Cappadoce. Il lui vint en outre une légion de Germanie, ayant avec elle ses auxiliaires tant à pied qu’à cheval. Toute l’armée fut retenue sous la tente, malgré les rigueurs de l’hiver le plus rude. La terre était si durcie par la glace, qu’il fallait la creuser avec le fer pour y enfoncer les pieux. Beaucoup de soldats eurent les membres gelés, et plusieurs moururent en sentinelle. On en remarqua un qui, en portant une fascine, eut les mains tellement roidies par le froid, qu’elles s’attachèrent à ce fardeau et tombèrent de ses bras mutilés. Corbulon, vêtu légèrement, la tête nue, se multipliait dans les marches, dans les travaux, louant l’activité, consolant la faiblesse, donnant l’exemple à tous. Cependant la dureté du climat et celle du service rebutèrent le soldat, et beaucoup désertaient : on eut recours alors à la sévérité. Dans les autres armées, on pardonnait une première, une seconde faute ; sous Corbulon, quiconque abandonnait son drapeau était sur-le-champ puni de mort. Cette rigueur fut salutaire, et l’on reconnut qu’elle valait mieux que la clémence ; car il y eut moins de désertions à punir dans ce camp que dans ceux où l’on faisait grâce.
\subsection[{Pactius désobéit aux ordres}]{Pactius désobéit aux ordres}
\noindent \labelchar{XXXVI.} Corbulon tint ses légions campées jusqu’aux premiers beaux jours du printemps, et distribua ses cohortes auxiliaires dans des positions avantageuses, avec défense de hasarder aucune attaque. Le commandement de ces détachements fut confié à Pactius Orphitus, qui avait été primipilaire. En vain Pactius écrivit que la négligence des barbares offrait des chances dont on pouvait profiter ; il lui fut enjoint de rester dans ses retranchements et d’attendre de plus grandes forces. Mais il enfreignit cet ordre ; et, renforcé de quelques escadrons qui arrivaient des postes voisins, et qui demandaient imprudemment le combat, il en vint aux mains et fut mis en déroute. Effrayés par sa défaite, ceux qui devaient le soutenir s’enfuirent en désordre chacun dans leur camp. Corbulon, indigné, réprimanda Pactius, ainsi que les officiers et les soldats, et les condamna tous à camper hors des retranchements \footnote{C’était une punition militaire usitée depuis les temps les plus anciens. On condamnait les troupes coupables à rester hors du camp, quelquefois même sans tentes, et sans pouvoir s’entourer de fossés et de palissades.} ; ils subirent cette humiliation, et n’en furent relevés qu’à la prière de l’armée tout entière.
\subsection[{Tiridate attaque l’Arménie}]{Tiridate attaque l’Arménie}
\noindent \labelchar{XXXVII.} Cependant Tiridate joignait au parti qu’il avait lui-même l’appui de Vologèse son frère ; et ce n’était plus par des attaques furtives, mais par une guerre ouverte qu’il désolait l’Arménie, pillant ceux qu’il croyait attachés à notre cause, éludant la rencontre des troupes envoyées contre lui, enfin voltigeant de tous côtés, et causant plus de terreur par le bruit de ses courses que par la force de ses armes. Corbulon, après avoir longtemps cherché le combat, frustré dans son attente, et contraint de porter, à l’exemple de l’ennemi, la guerre en vingt endroits, divise ses troupes, afin que ses lieutenants et ses préfets attaquent sur plusieurs points à la fois ; il avertit en outre le roi Antiochus \footnote{Antiochus, roi de Commagène.} d’entrer dans les provinces de son voisinage. De son côté, Pharasmane venait de tuer, comme traître à sa personne, son fils Rhadamiste ; et, afin de nous prouver sa fidélité, il assouvissait avec un redoublement d’ardeur sa vieille haine contre les Arméniens. Enfin, une nation distinguée par son attachement aux Romains, les Insiques, attirés alors pour la première fois dans notre alliance, parcouraient les lieux les plus impraticables de l’Arménie. Ainsi étaient déconcertés les plans de Tiridate : il envoya des ambassadeurs demander, en son nom et au nom des Parthes, « pourquoi, lorsqu’on venait de livrer des otages et qu’une amitié renouvelée semblait annoncer aussi des bienfaits nouveaux, on le dépouillait d’une ancienne possession. Il ajoutait que, si Vologèse n’agissait pas encore, c’était parce qu’ils aimaient mieux discuter leurs droits que de recourir à la force ; mais que, si l’on s’obstinait à la guerre, les Arsacides retrouveraient cette valeur et cette fortune que les défaites des Romains signalèrent plus d’une fois. » Corbulon savait qu’une révolte des Hyrcaniens occupait Vologèse : pour toute réponse, il conseille à Tiridate d’employer auprès de César les prières pour armes ; « il peut s’assurer une puissance durable et un trône qui ne coûtera pas de sang, si, au lieu de lointaines et tardives espérances, il en poursuit de plus prochaines et de plus sûres. »
\subsection[{Proposition de rencontre entre Tiridate et Corbulon}]{Proposition de rencontre entre Tiridate et Corbulon}
\noindent \labelchar{XXXVIII.} Ensuite, comme l’échange des courriers n’avançait point la conclusion de la paix, on proposa un rendez-vous où les deux chefs conféreraient en personne. Tiridate voulait s’y trouver escorté de mille chevaux : « il ne fixait à Corbulon ni le nombre ni l’espèce des troupes qui l’accompagneraient, pourvu qu’elles vinssent dans un appareil pacifique, sans casques ni cuirasses. » Personne, et encore moins un vieux et prudent capitaine, ne se fût laissé prendre à cette ruse du barbare. « Ce nombre, borné pour l’un des chefs, illimité pour l’autre, cachait un piège. A quoi servirait la multitude des soldats, si on les offrait découverts à une cavalerie si habile à lancer des flèches ? " Toutefois Corbulon, comme s’il n’eût rien soupçonné, répondit que des affaires qui intéressaient les deux peuples seraient discutées plus dignement en présence des deux armées. Puis il choisit un lieu dont une partie, s’élevant en pente douce, était propre à recevoir les lignes de l’infanterie, et l’autre, s’étendant en plaine, permettait à la cavalerie de se développer. Au jour convenu, Corbulon, arrivé le premier, plaça sur les ailes les cohortes auxiliaires et les troupes des rois alliés ; il mit au centre la sixième légion, renforcée de trois mille hommes de la troisième, qu’il avait tirés, pendant la nuit, d’un autre camp. Il ne laissa qu’une seule aigle, pour n’offrir l’apparence que d’une seule légion. Le jour baissait déjà quand Tiridate parut, mais à une distance d’où il était plus facile de le voir que de l’entendre. La conférence n’eut pas lieu, et le général romain fit rentrer ses soldats chacun dans leur camp.
\subsection[{Victoire romaine}]{Victoire romaine}
\noindent \labelchar{XXXIX.} Le roi, soit pour éviter les embûches que lui fit craindre la marche de nos troupes dans plusieurs directions, soit pour intercepter les convois qui nous arrivaient de l’Euxin et de Trébizonde, se retira précipitamment ; mais il ne put enlever les convois, parce qu’ils cheminaient par des montagnes garnies de troupes romaines. Et d’un autre côté, pour empêcher que la guerre ne se prolongeât sans fruit, et réduire les Arméniens à la nécessité de se défendre, Corbulon résolut de détruire leurs places. La plus forte de cette province se nommait Volande : il se charge lui-même d’en faire le siège, et confie celui des moins importantes au lieutenant Cornelius Flaccus, et au préfet de camp Instéius Capito. Après avoir visité l’enceinte et tout préparé pour un assaut, il anime ses soldats contre « un ennemi vagabond qui ne veut ni de la paix ni du combat, et qui fait, en fuyant, l’aveu de sa lâcheté et de sa perfidie. » Il les exhorte à lui ôter ses retraites, et leur montre à la fois la gloire et le butin. Ensuite il divise son armée en quatre parties : l’une forme la tortue et s’approche pour saper la muraille ; une autre reçoit l’ordre de dresser les échelles ; un grand nombre, de lancer avec les machines des javelots et des torches ; enfin un poste est assigné aux frondeurs pour envoyer de loin une grêle de balles : ainsi, également menacé partout, l’ennemi ne pourrait porter de secours nulle part. L’ardeur du soldat fut telle qu’avant le tiers du jour les murs étaient balayés, les portes enfoncées, les fortifications prises par escalade, tous les adultes passés au fil de l’épée ; et nous n’avions que peu de blessés, pas un mort. La foule inhabile aux combats fut vendue comme esclave, et le reste du butin abandonné aux vainqueurs. Le lieutenant et le préfet eurent le même succès ; et trois places emportées en un jour entraînèrent, par la terreur ou la bonne volonté des habitants, la reddition de toutes les autres. Dès lors Corbulon se crut assez fort pour attaquer Artaxate, capitale du pays : toutefois il n’y conduisit pas directement son armée. L’Araxe coule au pied des murailles ; et, en le passant sur un pont, il aurait mis ses légions sous les coups de l’ennemi : on traversa le fleuve plus loin, par un gué assez large.
\subsection[{Hésitations de Tiridate}]{Hésitations de Tiridate}
\noindent \labelchar{XL.} Tiridate flottait entre la honte et la crainte : laisser faire le siège, c’était avouer son impuissance ; et il ne pouvait l’empêcher sans s’engager peut-être, lui et sa, cavalerie, dans des lieux impraticables. Il résolut de se montrer en bataille, et d’attendre le point du jour, soit pour combattre en effet, soit pour nous attirer dans quelque piège par une fuite simulée. Les barbares se répandent donc tout à coup autour de l’armée romaine, mais sans surprendre le général, qui avait tout disposé et pour la marche et pour le combat. La troisième légion s’avançait à la droite, la sixième à la gauche, l’élite de la dixième au centre ; les bagages étaient placés entre les lignes, et mille chevaux formaient l’arrière-garde, avec ordre de tenir ferme si l’on chargeait, mais de ne jamais poursuivre. Les cohortes, les archers et le reste de la cavalerie garnissaient les deux ailes : la gauche se prolongeait davantage en suivant le pied des collines, afin que, si l’ennemi essayait de pénétrer, il fût reçu par une attaque de front et de flanc tout à la fois. Tiridate nous harcelait de son côté, sans approcher cependant jusqu’à la portée des traits, et affectant tour à tour la menace ou la frayeur, dans l’espoir de désunir nos lignes et de fondre sur nos corps isolés : mais la témérité ne fit aucun désordre ; seulement un centurion de cavalerie, emporté par son audace, tomba percé de flèches. Sa mort fut pour les autres une leçon de discipline, et aux approches de la nuit l’ennemi se retira.
\subsection[{Prise d’Artaxate, capitale de l’Arménie}]{Prise d’Artaxate, capitale de l’Arménie}
\noindent \labelchar{XLI.} Corbulon campa sur le lieu même, et songea d’abord à profiter de la nuit pour aller avec ses légions sans bagages investir Artaxate \footnote{Capitale de toute l’Arménie.}, où il croyait que le roi s’était retiré ; mais ayant appris par les éclaireurs que Tiridate s’éloignait, sans qu’on sût s’il allait en Médie ou en Albanie, il différa jusqu’au jour, et fit partir en avant ses cohortes légères, avec ordre d’environner la place et de commencer l’attaque de loin. Mais, les habitants ouvrirent leurs portes, et s’abandonnèrent aux Romains avec ce qu’ils possédaient. Cette soumission sauva ; leurs personnes ; la ville fut livrée aux flammes et détruite de fond en comble : il eût fallu, pour la conserver, une forte garnison, à cause de la grandeur de l’enceinte ; et nous n’avions pas assez de troupes pour les partager entre la guerre active et la garde d’une telle place. D’un autre côté, la laisser debout sans s’en assurer la possession, c’était perdre la gloire et le fruit de cette conquête. On ajoute que la volonté du ciel s’était manifestée par un prodige : un soleil brillant éclairait tous les dehors de la ville, lorsqu’en un moment tout ce qu’enfermaient les murailles se couvrit d’un nuage épais et sillonné d’éclairs. On en conclut que les dieux irrités la livraient à sa perte. Néron, pour ce succès, fut salué \emph{imperator} ; un sénatus-consulte décerna des actions de grâces aux dieux, et au prince des statues, des arcs de triomphe, le consulat pour plusieurs années. On proposa de consacrer par des fêtes les jours où la victoire, avait été remportée, connue à Rome, annoncée au sénat ; sans compter mille autres flatteries si excessives que Cassius, en votant pour le reste, déclara « que, si la reconnaissance publique devait égaler ses hommages aux bienfaits du ciel, toute l’année ne suffirait pas aux actions de grâces ; qu’il fallait des jours de travail ainsi que des jours sacrés, afin d’honorer les dieux sans entraver les affaires des hommes. »
\subsection[{À Rome – Suilius contre Sénèque}]{À Rome – Suilius contre Sénèque}
\noindent \labelchar{XLII.} Un accusé, qui éprouva longtemps des fortunes diverses et mérita bien des haines, fut condamné ensuite, non toutefois sans qu’il en rejaillît de l’odieux sur Sénèque. C’était Suilius, orateur vénal et redouté sous Claude, tombé, par le changement des temps, moins bas que ses ennemis n’auraient voulu, et qui préférait le rôle de coupable à celui de suppliant. On attribuait au seul dessein de le perdre la proposition de renouveler par un sénatus-consulte les peines de la loi Cincia contre les orateurs qui rendaient leurs services. Suilius ne ménageait ni plaintes ni reproches, violent par caractère, et trop prés du tombeau pour n’être pas libre. Sénèque était l’objet de ses invectives. « Cet homme se vengeait, selon lui, sur les amis de Claude, du juste exil qu’il avait subi sous ce prince. Accoutumé aux études mortes de l’école et habile devant une jeunesse ignorante, il était jaloux de ceux qui consacraient à la défense des citoyens une vive et saine éloquence. Il avait été, lui, le questeur de Germanicus, et Sénèque le séducteur de sa fille. Était-ce donc un plus grand crime de recevoir le prix offert par la reconnaissance à un travail honorable, que de souiller la couche des princesses ? Quelle sagesse, quelles leçons de philosophie, avaient instruit Sénèque à entasser, en quatre ans de faveur, trois cents millions de sesterces \footnote{Trois cents millions de sesterces valaient sous Néron 55 142 940 fr. de notre monnaie.} ? Rome, où il surprenait les testaments et attirait dans ses pièges les vieillards sans héritiers, l’Italie et les provinces, qu’il épuisait à force d’usures, ne le savaient que trop ! Pour lui, de pénibles travaux ne lui avaient procuré que des biens modiques ; et il subirait accusation, périls, tout, plutôt que d’humilier, devant cette fortune soudaine, sa longue et ancienne considération. »\par
\labelchar{XLIII.} Des bouches ne manquèrent pas pour faire à Sénèque un rapport, fidèle ou envenimé, de ces discours. On trouva des dénonciateurs qui accusèrent Suilius d’avoir pillé les alliés et volé le trésor public pendant qu’il gouvernait l’Asie. Un an leur fut donné pour recueillir les preuves ; mais bientôt ils jugèrent plus court de chercher à Rome même des crimes dont ils eussent les témoins tout prêts. Pomponius jeté dans la guerre civile par la violence de ses accusations, Julie, fille de Drusus, et Poppéa Sabina contraintes de mourir, Valérius Asiaticus, Lusius Saturninus, Cornélius Lupus perdus par ses intrigues, enfin des légions de chevaliers romains condamnées, et toutes les cruautés de Claude, voilà ce qu’ils lui reprochèrent. L’accusé répondit « qu’il n’avait rien fait de son propre mouvement, qu’il avait obéi à César. » Mais Néron lui ferma la bouche en déclarant que son père n’avait jamais ordonné une accusation ; qu’il en trouvait la preuve dans les tablettes de ce prince. Alors il mit en avant les ordres de Messaline, et la défense chancela. « Pourquoi, en effet, avait-il été choisi plutôt qu’un autre pour prêter sa voix aux fureurs d’une prostituée ? Il fallait punir ces exécuteurs d’ordres barbares, qui, après avoir reçu le salaire du crime, rejetaient le crime sur autrui. » Dépouillé de la moitié de ses biens (car la moitié fut laissée à son fils et à sa petite-fille, avec ce qu’ils tenaient par testament de leur mère ou aïeule), il fut relégué dans les îles Baléares \footnote{Deux îles de la Méditerranée, vis-à-vis l’Espagne ; aujourd’hui Majorque et Minorque.}, sans que, ni pendant son procès, ni après sa condamnation, l’on vît fléchir son orgueil. On dit qu’il consola, par une vie molle et voluptueuse, l’ennui de cet exil. Les accusateurs attaquèrent, en haine de lui, son fils Nérulinus, sous prétexte de concussion. Le prince les arrêta, en disant qu’on avait assez fait pour la vengeance.
\subsection[{Un crime passionnel}]{Un crime passionnel}
\noindent \labelchar{XLIV.} Vers le même temps, Octavius Sagitta, tribun du peuple, épris pour une femme mariée, nommée Pontia, d’un violent amour, achète l’adultère à force de présents. Bientôt il la décide à quitter son mari, s’engage à l’épouser, et reçoit sa promesse. Mais Pontia, une fois libre, remettait de jour en jour, opposait la volonté de son pire ; séduite enfin par l’espérance d’un plus riche mariage, elle retira sa parole. Octavius se plaint, menace, atteste sa réputation perdue, sa fortune épuisée, offrant à Pontia de prendre jusqu’à sa vie, le seul bien qui lui reste. Toujours repoussé, il demande pour consolation une dernière nuit, dont les douceurs lui rendront l’empire sur ses sens. La nuit est fixée : Pontia donne la garde de sa chambre à une suivante qui était dans la confidence ; Octavius, accompagné d’un seul affranchi, entre avec un fer caché sous sa robe. On sait tout ce qu’inspirent la colère et l’amour, querelles, prières, reproches, raccommodement ; le plaisir eut aussi dans les ténèbres ses moments privilégiés. Tout à coup, saisi d’une fureur à laquelle Pontia était loin de s’attendre, Octavius la perce de son poignard. L’esclave accourt ; il l’écarte d’un second coup, et s’élance hors de la chambre. Le lendemain le meurtre fut public, et l’on n’avait aucun doute sur le meurtrier : le séjour d’Octavius chez Pontia était avéré. Mais l’affranchi se déclara seul coupable : il avait, disait-il, vengé l’injure de son patron ; et la noblesse de cet aveu ébranlait quelques esprits, lorsque la suivante, guérie de sa blessure, découvrit la vérité. Octavius, en sortant du tribunat \footnote{On ne pouvait mettre en jugement un magistrat qu’après l’expiration de sa charge.}, fut cité devant les consuls par le père de sa victime, et condamné par le sénat, d’après la loi sur les assassins.
\subsection[{Poppée, la femme fatale}]{Poppée, la femme fatale}
\noindent \labelchar{XLV.} Une impudicité non moins scandaleuse signala cette année, et fut pour la république le commencement de grands malheurs. Il y avait à Rome une femme nommée Sabina Poppéa : fille de T. Ollius, elle avait pris le nom de son aïeul maternel Poppéus Sabinus, dont la mémoire, plus illustre, brillait des honneurs du consulat et du triomphe ; car Ollius n’avait pas encore rempli les hautes dignités, quand l’amitié de Séjan le perdit. Rien ne manquait à Poppée, si ce n’est une âme honnête. Sa mère, qui surpassait en beauté toutes les femmes de son temps, lui avait transmis tout ensemble ses traits et l’éclat de son nom. Ses richesses suffisaient à son rang ; son langage était poli, son esprit agréable. Cachant, sous les dehors de la modestie, des mœurs dissolues, elle paraissait rarement en public, et toujours à demi voilée, soit pour ne pas rassasier les regards, soit qu’elle eût ainsi plus de charmes. Prodigue de sa renommée, elle ne distingua jamais un amant d’un époux ; indépendante de ses affections comme de celles d’autrui, et portant, où elle voyait son intérêt, ses changeantes amours. Elle était mariée au chevalier romain Rufius Crispinus, dont elle avait un fils, lorsqu’ Othon la séduisit par sa jeunesse, son faste, et la réputation qu’il avait d’être le favori le plus aimé de Néron. L’adultère fut bientôt suivi du mariage.
\subsection[{Néron éloigne le mari Othon}]{Néron éloigne le mari Othon}
\noindent \labelchar{XLVI.} Othon ne cessait de vanter à Néron la beauté et les grâces de son épouse, indiscret par amour, ou voulant peut-être allumer les désirs du prince, dans l’idée que la possession de la même femme serait un nouveau lien qui assurerait son crédit. Souvent un l’entendit répéter, en quittant la table de César, « qu’il allait revoir ce trésor accordé à sa flamme, cette noblesse, cette beauté, l’objet des vœux de tous, la joie des seuls favoris du sort. » De telles amorces eurent bientôt produit leur effet. Admise au palais, Poppée établit son empire par les caresses et la ruse : elle feint de ne pouvoir maîtriser son ardeur, d’être éprise de la figure de Néron ; puis quand elle voit que la passion du prince est assez vive, elle prend de la fierté ; s’il veut la retenir plus d’une ou deux nuits, elle représente « qu’elle a un époux, et qu’elle ne peut renoncer à son mariage. Othon tient son cœur enchaîné par un genre de vie que personne n’égale ; c’est lui dont l’âme est grande, le train magnifique, c’est chez lui qu’elle voit un spectacle digne du rang suprême ; tandis que Néron, amant d’une vile esclave et captif sous les lois d’Acté, n’a retiré de ce commerce ignoble rien que de bas et de servile. » Othon fut exclus d’abord de l’intimité du prince, puis de sa cour et de sa suite ; enfin, pour éloigner de Rome un rival importun, on l’envoya gouverner la Lusitanie. Il y resta jusqu’à la guerre civile, et fit oublier par une vie pure et irréprochable ses premiers scandales ; sans frein dans la condition privée, plus maître de lui dans le pouvoir.
\subsection[{Exil de Cornélius Sylla}]{Exil de Cornélius Sylla}
\noindent \labelchar{XLVII.} Depuis ce temps Néron ne chercha plus à voiler ses débauches et ses crimes. Il se défiait surtout de Cornélius Sylla \footnote{Mari d’Antonia, fille de Claude, le même auquel Panas et Burrus furent accusés de vouloir donner l’empire.}, dont l’indolence, changeant de nature à ses yeux, lui paraissait finesse et dissimulation. Graptus, affranchi du palais, à qui son grand âge et une expérience commencée dès le règne de Tibère avaient appris la cour, redoubla ses craintes par le mensonge que voici. Le pont Milvius était alors, pour les plaisirs nocturnes, un rendez-vous célèbre : le prince y allait souvent, afin de donner, hors de Rome, une plus libre carrière à ses dissolutions. Un jour, à en croire Graptus, on l’attendit au retour sur la voie Flaminienne, et, s’il évita l’embuscade, c’est qu’heureusement il revint par une autre route aux jardins de Salluste ; et l’auteur de ce complot était Sylla. Il est vrai que des serviteurs du prince avaient rencontré en revenant des jeunes gens qui, par une licence très-ordinaire alors, s’étaient fait un jeu de les effrayer. On n’avait reconnu parmi eux aucun des esclaves ni des clients de Sylla ; et son caractère, méprisé de tout le monde et incapable d’une pensée hardie, réfutait l’accusation. Cependant, comme s’il eût été convaincu, il reçut ordre de quitter sa patrie, et de se confiner dans les murs de Marseille.
\subsection[{Députations de Pouzzoles}]{Députations de Pouzzoles}
\noindent \labelchar{XLVIII.} Sous les mêmes consuls, on entendit les députations envoyées séparément au sénat romain par le sénat et le peuple de Pouzzoles. Le premier accusait les violences de la multitude ; le second s’élevait contre l’avarice des magistrats st des grands. Déjà des pierres lancées, des menaces d’incendie, marquaient le progrès de la sédition, et appelaient le massacre et les armes : C. Cassius fut choisi pour y porter remède ; mais sa sévérité révolta les esprits ; et, sur sa propre demande, on mit à sa place les deux frères Scribonius, auxquels on donna une cohorte prétorienne. La terreur qu’elle inspira, jointe à quelques supplices, rétablit la concorde.
\subsection[{Le sénat s’occupe même des jeux des Syracusains}]{Le sénat s’occupe même des jeux des Syracusains}
\noindent \labelchar{XLIX.} Je ne rapporterais pas un sénatus-consulte d’aussi peu d’intérêt que celui qui permit aux Syracusains d’excéder dans les jeux le nombre prescrit de gladiateurs, si Thraséas, en le combattant, n’eût donné à ses détracteurs l’occasion de censurer son vote. « Car enfin, s’il croyait la liberté du sénat si nécessaire à la république, pourquoi s’attacher à de telles frivolités ? Que ne consacrait-il sa voix à discuter la paix ou la guerre, les impôts, les lois, tout ce qui touche à la grandeur romaine ? Tout sénateur, chaque fois qu’il recevait le droit d’opiner, était libre d’exposer ses vœux et de requérir une délibération. Était-ce donc la seule réforme à faire que d’empêcher que Syracuse ne dépensât trop en spectacles ? et régnait-il dans tout le reste un aussi bel ordre que si Thraséas tenait, à la place de Néron, le gouvernail de l’État ? Si l’on gardait un silence prudent sur les choses importantes, combien plus devait-on se taire sur des bagatelles ! " Thraséas répondait à ses amis, qui de leur côté lui demandaient ses motifs : « que, s’il s’opposait à de pareils décrets, ce n’était pas faute de connaître la situation des affaires ; c’était pour sauver l’honneur du sénat, en faisant voir à tous que des yeux ouverts sur des objets si frivoles ne se fermeraient pas sur les grands intérêts de l’empire. »
\subsection[{Néron veut supprimer les taxes à Rome}]{Néron veut supprimer les taxes à Rome}
\noindent \labelchar{L.} La même année, touché des instances réitérées du peuple, et de ses plaintes contre la tyrannie des publicains, Néron eut la pensée d’abolir toutes les taxes \footnote{Les douanes, les droits d’entrée et de péage et les taxes sur les consommations.}, et de faire ainsi au genre humain le plus magnifique des présents. Mais les sénateurs, après avoir beaucoup loué la générosité du prince, en arrêtèrent l’élan. Ils lui représentèrent « que c’en était fait de l’empire, si l’on diminuait les revenus qui soutenaient sa puissance ; que, les péages supprimés, on ne manquerait pas de demander aussi la suppression du tribut ; que la plupart des fermes publiques avaient été établies par les consuls et les tribuns du peuple, quand la liberté romaine était encore dans toute sa vigueur ; qu’on n’avait fait depuis que pourvoir aux moyens d’égaler les recettes aux dépenses ; qu’on réprimât, à la bonne heure, l’avarice des traitants, afin que des charges supportées sans murmure depuis tant d’années ne fussent pas changées, par des rigueurs nouvelles, en d’odieuses vexations. »\par
\labelchar{LI.} Le prince ordonna donc par un édit « que les lois qui réglaient chaque impôt, tenues secrètes jusqu’alors, fussent affichées ; que ce qu’on n’aurait pas demandé dans l’année, on ne pût l’exiger plus tard ; qu’à Rome le préteur, et dans les provinces le propréteur ou le proconsul, connussent extraordinairement de toute plainte contre les publicains ; que les soldats conservassent leur immunité, excepté pour les objets dont ils feraient trafic ;" et plusieurs autres dispositions très-sages, qui furent observées quelque temps, ensuite méprisées. Cependant il nous reste encore l’abolition du quarantième et du cinquantième \footnote{Droits de douane.}, et de quelques autres perceptions illégales, inventées sous des noms divers par d’avides exacteurs. On rendit moins onéreux pour les provinces d’outre-mer le transport des blés, et l’on régla que les navires ne seraient pas comptés dans le cens des négociants, ni sujets au tribut.\par
\labelchar{LII.} Deux accusés, qui avaient exercé en Afrique le pouvoir proconsulaire, Sulpicius Camérinus et Pomponius Silvanus, furent absous par Néron. Camérinus n’avait pour adversaires qu’un petit nombre de particuliers, qui lui reprochaient des actes de rigueur plutôt que des concussions. Silvanus était assailli par une foule d’accusateurs. Ceux-ci demandaient du temps pour faire venir des témoins ; l’accusé voulait se justifier à l’instant même. Il l’emporta, parce qu’il était riche, sans héritiers, et vieux ; vieillesse qui ne l’empêcha pas de survivre à ceux dont la brigue l’avait sauvé.
\subsection[{À l’extérieur – La Germanie}]{À l’extérieur – La Germanie}
\noindent \labelchar{LIII.} Jusqu’à cette époque tout avait été tranquille en Germanie, par la politique de nos généraux, qui, voyant prodiguer les décorations triomphales, espéraient trouver dans le maintien de la paix un honneur moins vulgaire. Paullinus Pompéius et L. Vétus avaient alors le commandement des armées. Afin de ne pas laisser le soldat oisif, Paullinus acheva la digue commencée depuis soixante-trois ans par Drusus, pour contenir le Rhin. Vétus se disposait à joindre la Moselle et la Saône par un canal au moyen duquel les troupes, après avoir traversé la Méditerranée, remonté le Rhône et la Saône, seraient entrées dans la Moselle, puis dans le Rhin, qu’elles auraient descendu jusqu’à l’Océan. On eût évité par là des marches difficiles, et la navigation aurait uni les rivages du nord à ceux du couchant. Élius Gracilis, lieutenant de Belgique, lui envia l’honneur de cette entreprise, en le détournant de conduire ses légions dans une province qui n’était pas la sienne, et de chercher dans les Gaules une popularité qui alarmerait l’empereur ; crainte qui fait échouer souvent les plus louables desseins.\par
\labelchar{LIV.} Du reste, la longue inaction des armées fit croire que nos généraux n’avaient plus le droit de les mener à l’ennemi. Aussi les Frisons s’approchèrent du Rhin, la jeunesse guerrière par les bois et les marais, le reste par les lacs, et occupèrent des terres vacantes, réservées pour l’usage des troupes : l’entreprise avait pour chefs Verritus et Malorix, rois de cette nation, si l’on peut dire que les Germains aient des rois. Déjà ils avaient construit des maisons, ensemencé les champs, et ils cultivaient ce sol comme un héritage de leurs pères, lorsque Dubius Avitus, successeur de Paullinus, les menaça des armes romaines s’ils ne retournaient dans leur ancien séjour, ou s’ils n’obtenaient de César ces nouvelles demeures, et décida Verritus et Malorix à recourir aux prières. Ils se rendirent à Rome : là, en attendant que Néron, occupé d’autres soins, leur donnât audience, on étalait à leurs yeux les merveilles de la ville : on les mena un jour au théâtre de Pompée, afin qu’ils vissent l’immensité du peuple réuni dans ce lieu. Le spectacle n’offrait à leur ignorance aucun intérêt : ils passaient le temps à s’enquérir de la composition de l’assemblée, comment l’on distinguait les ordres, quels étaient les chevaliers, où s’asseyaient les sénateurs. Ils remarquèrent parmi ces derniers des spectateurs en costume étranger \footnote{Ces étrangers étaient des ambassadeurs parthes et arméniens.}. Ils demandèrent qui ils étaient, et, apprenant qu’on donnait ces places d’honneur aux députés des nations les plus distinguées par leur courage et par leur fidélité à l’empire : « Aucun peuple, s’écrient-ils, n’est plus brave ni plus fidèle que les Germains," et à l’instant ils descendent et vont s’asseoir entre les sénateurs ; hardiesse qu’on accueillit avec bienveillance, comme la saillie d’une franchise antique, et l’effet d’une utile émulation. Néron les reçut tous deux au nombre des citoyens : les Frisons eurent ordre de quitter le pays ; comme ils s’y refusaient, la cavalerie auxiliaire, envoyée subitement contre eux, leur fit une nécessité d’obéir, en tuant ou faisant prisonniers les plus opiniâtres.\par
\labelchar{LV.} Ces mêmes champs furent envahis par les Ansibariens, plus redoutables que les Frisons, à cause de leur nombre et de la pitié qu’ils trouvèrent chez les nations voisines. Chassés par les Cauques, sans terre où se fixer, ils imploraient un exil tranquille. Un homme célèbre parmi ces peuples, et fidèle à notre empire, nommé Boiocalus, appuyait leur demande, en représentant « que, dans 1a révolte des Chérusques, Arminius l’avait chargé de fers ; qu’ensuite il avait porté les armes sous Tibère et Germanicus. Il venait, à cinquante ans d’obéissance, ajouter un nouveau service, en mettant sa nation sous nos lois. De ces champs inutiles, combien était petite la partie sur laquelle on transportait quelquefois les troupeaux de l’armée ! qu’on leur réservât, à la bonne heure, l’espace que l’homme abandonne partout aux animaux ; mais pourquoi préférer le voisinage d’un désert à celui d’un peuple ami ? Ce territoire avait appartenu jadis aux Chamaves, puis aux Tubantes, enfin aux Usipiens. La terre fut donnée aux mortels, comme le ciel aux dieux : les places vides sont un domaine public. » Ensuite regardant le soleil, s’adressant à tous les astres, comme s’ils eussent été devant lui, il leur demandait « s’ils voudraient éclairer un sol inhabité. Ah ! qu’ils versassent plutôt les eaux de l’Océan sur les ravisseurs de la terre. »\par
\labelchar{LVI.} Offensé de ce discours, Avitus répondit « qu’il fallait subir la loi du plus digne ; que ces dieux dont ils attestaient la puissance avaient fait Rome maîtresse de donner ou d’ôter, sans reconnaître d’autre juge qu’elle-même," Telle fut sa réponse publique aux Ansibariens : quant à Boiocalus, il lui dit qu’en mémoire de sa longue amitié il lui donnerait des terres ; et le Germain repoussa cette offre comme le prix de la trahison. « La terre, ajouta-il peut nous manquer pour vivre ; elle ne peut nous manquer pour mourir ;" et les deux partis se séparèrent également irrités. Les Ansibariens appelaient à leur secours les Bructères, les Tenctères, et même des nations plus éloignées. Avitus écrivit à Curtilius Mancia, général de l’armée du Haut-Rhin, de passer le fleuve et de se montrer sur les derrières des barbares. De son côté, il conduisit ses légions chez les Tenctères, et menaça de tout saccager s’ils ne renonçaient à la ligue : ils obéirent. La même terreur désarma les Bructères, et chacun désertant les périls d’une querelle qui n’était pas la sienne, les Ansibariens, restés seuls, reculèrent jusque chez les Usipiens et les Tubantes. Chassés de ces cantons, ils fuient chez les Cattes, puis chez les Chérusques ; et, après des courses longues et vagabondes, étrangers, manquant de tout, reçus en ennemis, les hommes jeunes et armés périrent par le fer, loin du sol natal ; le reste fut partagé comme une proie.\par
\labelchar{LVII.} Un combat sanglant se livra, le même été, entre les Hermondures et les Cattes. Ils se disputaient un fleuve dont l’eau fournit le sel en abondance \footnote{Eckard pense que cette rivière est la Saale ou Sala.}, et qui arrose leurs communes limites. A la passion de tout décider par l’épée, se joignait la croyance religieuse « que ces lieux étaient le point le plus voisin du ciel, et que nulle part les dieux n’entendaient de plus près les prières des hommes. C’était pour cela que le sel, donné par une prédilection divine à cette rivière et à ces forêts, ne naissait pas, comme en d’autres pays, des alluvions de la mer lentement évaporées. On versait l’eau du fleuve sur une pile d’arbres embrasés ; et deux éléments contraires, la flamme et l’onde, produisaient cette précieuse matière. » La guerre, heureuse pour les Hermondures, fut d’autant plus fatale aux Cattes, que les deux partis avaient dévoué à Mars et à Mercure l’armée qui serait vaincue, vœu suivant lequel hommes, chevaux, tout est livré à l’extermination. Ici du moins les menaces de nos ennemis tournaient contre eux-mêmes : bientôt un fléau inattendu frappa les Ubiens, nos amis. Des feux sortis de terre ravageaient les fermes, les champs cultivés, les villages, et s’avançaient jusqu’aux murs de la colonie nouvellement fondée. Rien ne pouvait les éteindre, ni l’eau du ciel, ni celle des rivières, ni aucun autre liquide. Enfin, de colère contre un mal où ils ne trouvaient point de remède, quelques paysans lancent de loin des pierres dans les flammes, et, les voyant s’affaisser, ils approchent et les chassent, comme on chasse des animaux, avec des bâtons et des fouets. Enfin ils se dépouillent de leurs vêtements et les jettent sur le feu : plus l’étoffe était sale et usée, plus elle l’étouffait aisément.
\subsection[{Le figuier Ruminal}]{Le figuier Ruminal}
\noindent \labelchar{LVIII.} La même année, le figuier Ruminal, qu’on voyait au Comice, et qui, plus de huit cent quarante ans auparavant, avait ombragé l’enfance de Rémus et de Romulus, perdit ses branches, et son tronc se dessécha, ce qui parut d’un sinistre augure ; mais il poussa de nouveaux rejetons.
\section[{Livre quatorzième (59, 62)}]{Livre quatorzième (59, 62)}\renewcommand{\leftmark}{Livre quatorzième (59, 62)}

\subsection[{A Rome – Poppée contre Agrippine}]{A Rome – Poppée contre Agrippine}
\noindent \labelchar{I.} Sous le consulat de C. Vipstanus et de Fontéius, Néron ne différa plus le crime qu’il méditait depuis longtemps. Une longue possession de l’empire avait affermi son audace, et sa passion pour Poppée devenait chaque jour plus ardente. Cette femme, qui voyait dans la vie d’Agrippine un obstacle à son mariage et au divorce d’Octavie, accusait le prince et le raillait tour à tour, l’appelant un pupille, un esclave des volontés d’autrui, qui se croyait empereur et n’était pas même libre. « Car pourquoi différer leur union ? Sa figure déplaît apparemment, ou les triomphes de ses aïeux, ou sa fécondité et son amour sincère ? Ah ! l’on craint qu’une épouse, du moins, ne révèle les plaintes du sénat offensé et la colère du peuple, soulevée contre l’orgueil et l’avarice d’une mère. Si Agrippine ne peut souffrir pour bru qu’une ennemie de son fils, que l’on rende Poppée à celui dont elle est la femme : elle ira, s’il le faut, aux extrémités du monde ; et, si la renommée lui apprend qu’on outrage l’empereur, elle ne verra pas sa honte, elle ne sera pas mêlée à ses périls. » Ces traits, que les pleurs et l’art d’une amante rendaient plus pénétrants, on n’y opposait rien : tous désiraient l’abaissement d’Agrippine, et personne ne croyait que la haine d’un fils dût aller jamais jusqu’à tuer sa mère.
\subsection[{L’inceste ?}]{L’inceste ?}
\noindent \labelchar{II.} Cluvius rapporte qu’entraînée par l’ardeur de conserver le pouvoir, Agrippine en vint à ce point, qu’au milieu du jour, quand le vin et la bonne chère allumaient les sens de Néron, elle s’offrit plusieurs fois au jeune homme ivre, voluptueusement parée et prête à l’inceste. Déjà des baisers lascifs et des caresses, préludes du crime, étaient remarqués des courtisans, lorsque Sénèque chercha, dans les séductions d’une femme, un remède aux attaques de l’autre, et fit paraître l’affranchie Acté. Celle-ci, alarmée tout à la fois pour elle-même et pour l’honneur de Néron, l’avertit « qu’on parlait publiquement de ses amours incestueuses ; que sa mère en faisait trophée, et qu’un chef impur serait bientôt rejeté des soldats. » Selon Fabius Rusticus, ce ne fut point Agrippine, mais Néron, qui conçut un criminel désir ; et la même affranchie eut l’adresse d’en empêcher le succès. Mais Cluvius est ici d’accord avec les autres écrivains, et l’opinion générale penche pour son récit ; soit qu’un si monstrueux dessein fût éclos en effet dans l’âme d’Agrippine, soit que ce raffinement inouï de débauche paraisse plus vraisemblable chez une femme que l’ambition mit, encore enfant, dans les bras de Lépide, que la même passion prostitua depuis aux plaisirs d’un Pallas, et que l’hymen de son oncle avait instruite à ne rougir d’aucune infamie.
\subsection[{Comment tuer sa mère ?}]{Comment tuer sa mère ?}
\noindent \labelchar{III.} Néron évita donc de se trouver seul avec sa mère, et, quand elle partait pour ses jardins et pour ses campagnes de Tuscule et d’Antium, il la louait de songer au repos. Elle finit, en quelque lieu qu’elle fût, par lui peser tellement, qu’il résolut sa mort. Il n’hésitait plus que sur les moyens, le poison, le fer, ou tout autre. Le poison lui plut d’abord ; mais, si on le donnait à la table du prince, une fin trop semblable à celle de Britannicus ne pourrait être rejetée sur le hasard ; tenter la foi des serviteurs d’Agrippine paraissait difficile, parce que l’habitude du crime lui avait appris à se défier des traîtres ; enfin, par l’usage des antidotes, elle avait assuré sa vie contre l’empoisonnement. Le fer avait d’autres dangers : une mort sanglante ne pouvait être secrète, et Néron craignait que l’exécuteur choisi pour ce grand forfait ne méconnût ses ordres. Anicet offrit son industrie : cet affranchi, qui commandait la flotte de Misène, avait élevé l’enfance de Néron, et haïssait Agrippine autant qu’il en était haï. Il montre « que l’on peut disposer un vaisseau de telle manière, qu’une partie détachée artificiellement en pleine mer la submerge à l’improviste. Rien de plus fertile en hasards que la mer : quand Agrippine aura péri dans un naufrage, quel homme assez injuste imputera au crime le tort des vents et des flots ? Le prince donnera d’ailleurs à sa mémoire un temple, des autels, tous les honneurs où peut éclater la tendresse d’un fils. »
\subsection[{Le crime se fera sur la mer}]{Le crime se fera sur la mer}
\noindent \labelchar{IV.} Cette invention fut goûtée, et les circonstances la favorisaient. L’empereur célébrait à Baïes les fêtes de Minerve ; il y attire sa mère, à force de répéter qu’il faut souffrir l’humeur de ses parents, et apaiser les ressentiments de son cœur : discours calculés pour autoriser des bruits de réconciliation, qui seraient reçus d’Agrippine avec cette crédulité de la joie, si naturelle aux femmes. Agrippine venait d’Antium ; il alla au-devant d’elle le long du rivage, lui donna la main, l’embrassa et la conduisit à Baules \footnote{Baules est une maison de campagne qui avait appartenu à l’orateur Hortensius. 2. La face des lieux ayant été changée par l’invasion de la mer, le lac de Baïes ne subsiste plus.} ; c’est le nom d’une maison de plaisance, située sur une pointe et baignée par la mer, entre le promontoire de Misène et le lac de Baïes \footnote{L’eau Marcia est un des plus célèbres aqueducs de l’ancienne Rome. La source était dans les montagnes des Péligniens. On voit encore, près de Rome, des ruines imposantes de cet aqueduc.}. Un vaisseau plus orné que les autres attendait la mère du prince, comme si son fils eût voulu lui offrir encore cette distinction ; car elle montait ordinairement une trirème, et se servait des rameurs de la flotte : enfin, un repas où on l’avait invitée donnait le moyen d’envelopper le crime dans les ombres de la nuit. C’est une opinion assez accréditée que le secret fut trahi, et qu’Agrippine, avertie du complot et ne sachant si elle y devait croire, se rendit en litière à Baies. Là, les caresses de son fils dissipèrent ses craintes ; il la combla de prévenances, la fit place, à table au-dessus de lui. Des entretiens variés, où Néron affecta tour à tour la familiarité du jeune âge et toute la gravité d’une confidence auguste, prolongèrent le festin. Il la reconduisit à son départ, couvrant de baisers ses yeux et son sein ; soit qu’il voulût mettre le comble à sa dissimulation, soit que la vue d’une mère qui allait périr attendrit en ce dernier instant cette âme dénaturée.
\subsection[{Le navire coule}]{Le navire coule}
\noindent \labelchar{V.} Une nuit brillante d’étoiles, et dont la paix s’unissait au calme de la mer, semblait préparée par les dieux pour mettre le crime dans toute son évidence. Le navire n’avait pas encore fait beaucoup de chemin. Avec Agrippine étaient deux personnes de sa cour, Crépéréius Gallus et Acerronie. Le premier se tenait debout prés du gouvernail ; Acerronie, appuyée sur le pied du lit où reposait sa maîtresse, exaltait, avec l’effusion de la joie, le repentir du fils et le crédit recouvré par la mère. Tout à coup, à un signal donné, le plafond de la chambre s’écroule sous une charge énorme de plomb. Crépéréius écrasé reste sans vie. Agrippine et Acerronie sont défendues par les côtés du lit qui s’élevaient au-dessus d’elles, et qui se trouvèrent assez forts pour résister au poids. Cependant le vaisseau tardait à s’ouvrir, parce que, dans le désordre général, ceux qui n’étaient pas du complot embarrassaient les autres. Il vint à l’esprit des rameurs de peser tous du même côté, et de submerger ainsi le navire. Mais, dans ce dessein formé subitement, le concert ne fut point assez prompt ; et une partie, en faisant contre-poids, ménagea aux naufragés une chute plus douce. Acerronie eut l’imprudence de s’écrier « qu’elle était Agrippine, qu’on sauvât la mère du prince ;" et elle fut tuée à coups de crocs, de rames, et des autres instruments qui tombaient sous la main. Agrippine, qui gardait le silence, fut moins remarquée, et reçut cependant une blessure à l’épaule. Après avoir nagé quelque temps, elle rencontra des barques qui la conduisirent dans le lac Lucrin, d’où elle se fit porter à sa maison de campagne.
\subsection[{Agrippine saine et sauve}]{Agrippine saine et sauve}
\noindent \labelchar{VI.} Là, rapprochant toutes les circonstances, et la lettre perfide, et tant d’honneurs prodigués pour une telle fin, et ce naufrage près du port, ce vaisseau qui, sans être battu par les vents ni poussé contre un écueil, s’était rompu par le haut comme un édifice qui s’écroule ; songeant en même temps au meurtre d’Acerronie, et jetant les yeux sur sa propre blessure, elle comprit que le seul moyen d’échapper aux embûches était de ne pas les deviner. Elle envoya l’affranchi Agérinus annoncer à son fils « que la bonté des dieux et la fortune de l’empereur l’avaient sauvée d’un grand péril ; qu’elle le priait, tout effrayé qu’il pouvait être du danger de sa mère, de différer sa visite ; qu’elle avait en ce moment besoin de repos. » Cependant, avec une sécurité affectée, elle fait panser sa blessure et prend soin de son corps. Elle ordonne qu’on recherche le testament d’Acerronie, et qu’on mette le scellé sur ses biens : en cela seulement elle ne dissimulait pas.
\subsection[{Panique de Néron}]{Panique de Néron}
\noindent \labelchar{VII.} Néron attendait qu’on lui apprît le succès du complot, lorsqu’il reçut la nouvelle qu’Agrippine s’était sauvée avec une légère blessure, et n’avait couru que ce qu’il fallait de danger pour ne pouvoir en méconnaître l’auteur. Éperdu, hors de lui même, il croit déjà la voir accourir avide de vengeance. « Elle allait armer ses esclaves, soulever les soldats, ou bien se, jeter dans les bras du sénat et du peuple, et leur dénoncer son naufrage, sa blessure, le meurtre de ses amis : quel appui restait-il au prince, si Burrus et Sénèque ne se prononçaient ? » Il les avait mandés dés le premier moment : on ignore si auparavant ils étaient instruits. Tous deux gardèrent un long silence, pour ne pas faire des remontrances vaines ; ou peut-être croyaient-ils les choses arrivées à cette extrémité, que, si l’on ne prévenait Agrippine, Néron était perdu. Enfin Sénèque, pour seule initiative, regarda Burrus et lui demanda s’il fallait ordonner le meurtre aux gens de guerre. Burrus répondit « que les prétoriens, attachés à toute la maison des Césars, et pleins du souvenir de Germanicus, n’oseraient armer leurs bras contre sa fille. Qu’Anicet achevât ce qu’il avait promis. » Celui-ci se charge avec empressement de consommer le crime. A l’instant Néron s’écrie « que c’est en ce jour qu’il reçoit l’empire, et qu’il tient de son affranchi ce magnifique présent ; qu’Anicet parte au plus vite et emmène avec lui des hommes dévoués. » De son côté, apprenant que l’envoyé d’Agrippine, Agérinus, demandait audience, il prépare aussitôt une scène accusatrice. Pendant qu’Agérinus expose son message, il jette une épée entre les jambes de cet homme ; ensuite il le fait garrotter comme un assassin pris en flagrant délit, afin de pouvoir feindre que sa mère avait attenté aux jours du prince, et que, honteuse de voir son crime découvert, elle s’en était punie par la mort.
\subsection[{Mort d’Agrippine}]{Mort d’Agrippine}
\noindent \labelchar{VIII.} Cependant, au premier bruit du danger d’Agrippine, que l’on attribuait au hasard, chacun se précipite vers le rivage. Ceux-ci montent sur les digues ; ceux-là se jettent dans des barques ; d’autres s’avancent dans la mer, aussi loin qu’ils peuvent ; quelques-uns tendent les mains. Toute la côte retentit de plaintes, de vœux, du bruit confus de mille questions diverses, de mille réponses incertaines. Une foule immense était accourue avec des flambeaux : enfin l’on sut Agrippine vivante, et déjà on se disposait à la féliciter, quand la vue d’une troupe armée et menaçante dissipa ce concours. Anicet investit la maison, brise la porte, saisit les esclaves qu’il rencontre, et parvient à l’entrée de l’appartement. Il y trouva peu de monde ; presque tous, à son approche, avaient fui épouvantés. Dans la chambre, il n’y avait qu’une faible lumière, une seule esclave, et Agrippine, de plus en plus inquiète de ne voir venir personne de chez son fils, pas même Agérinus. La face des lieux subitement changée, cette solitude, ce tumulte soudain, tout lui présage le dernier des malheurs. Comme la suivante elle-même s’éloignait : « Et toi aussi, tu m’abandonnes," lui dit-elle : puis elle se retourne et voit Anicet, accompagné du triérarque Herculéus et d’Oloarite, centurion de la flotte. Elle lui dit « que, s’il était envoyé pour la visiter, il pouvait annoncer qu’elle était remise ; que, s’il venait pour un crime, elle en croyait son fils innocent ; que le prince n’avait point commandé un parricide. » Les assassins environnent son lit, et le triérarque lui décharge le premier un coup de bâton sur la tête. Le centurion tirait son glaive pour lui donner la mort. « Frappe ici," s’écria-t-elle en lui montrant son ventre, et elle expira percée de plusieurs coups.
\subsection[{Commentaire de Tacite}]{Commentaire de Tacite}
\noindent \labelchar{IX.} Voilà les faits sur lesquels on s’accorde. Néron contempla-t-il le corps inanimé de sa mère, en loua-t-il la beauté ? les uns l’affirment, les autres le nient. Elle fut brûlée la nuit même, sur un lit de table, sans la moindre pompe ; et, tant que Néron fut maître de l’empire, aucun tertre, aucune enceinte ne protégea sa cendre. Depuis, des serviteurs fidèles lui élevèrent un petit tombeau sur le chemin de Misène, prés de cette maison du dictateur César, qui, située à l’endroit le plus haut de la côte, domine au loin tout le golfe. Quand le bûcher fut allumé, un de ses affranchis, nommé Mnester, se perça d’un poignard, soit par attachement à sa maîtresse soit par crainte des bourreaux. Telle fut la fin d’Agrippine, fin dont bien des années auparavant elle avait cru et méprisé l’annonce. Un jour qu’elle consultait sur les destins de Néron, les astrologues lui répondirent qu’il régnerait et qu’il tuerait sa mère : « Qu’il me tue, dit-elle, pourvu qu’il règne. »
\subsection[{“Tistesse” de Néron}]{“Tistesse” de Néron}
\noindent \labelchar{X.} C’est quand Néron eut consommé le crime qu’il en comprit la grandeur. Il passa le reste de la nuit dans un affreux délire : tantôt morne et silencieux, tantôt se relevant avec effroi, il attendait le retour de la lumière comme son dernier moment. L’adulation des centurions et des tribuns, par le conseil de Burrus, apporta le premier soulagement à son désespoir. Ils lui prenaient la main, le félicitaient d’avoir échappé au plus imprévu des dangers, aux complots d’une mère. Bientôt ses amis courent aux temples des dieux, et, l’exemple une fois donné, les villes de Campanie témoignent leur allégresse par des sacrifices et des députations. Néron, par une dissimulation contraire, affectait la douleur ; il semblait haïr des jours conservés à ce prix, et pleurer sur la mort de sa mère. Mais les lieux ne changent pas d’aspect comme l’homme de visage, et cette mer, ces rivages, toujours présents, importunaient ses regards. L’on crut même alors que le son d’une trompette avait retenti sur les coteaux voisins, et des gémissements, dit-on, furent entendus au tombeau d’Agrippine. Néron prit le parti de se retirer à Naples, et écrivit une lettre au sénat.
\subsection[{Lettre de justification au sénat}]{Lettre de justification au sénat}
\noindent \labelchar{XI.} « Un assassin, disait-il, Agérinus, affranchi d’Agrippine et l’un de ses plus intimes confidents, avait été surpris avec un poignard ; et elle-même, cédant au cri de sa conscience, s’était infligé la peine comme elle avait ordonné le crime. » A cette accusation, il en ajoutait de plus anciennes. « Elle avait rêvé le partage de l’empire ; elle s’était flattée que les cohortes prétoriennes jureraient obéissance à une femme, et que le sénat et le peuple subiraient le même déshonneur. Trompée dans ses désirs, elle s’en était vengée sur les sénateurs, le peuple et les soldats, en s’opposant aux largesses du prince, et en amassant les dangers sur les plus illustres têtes. Avec quelle peine ne l’avait-il pas empêchée de forcer les portes du sénat, et de donner ses réponses aux nations étrangères ! » Il remontait jusqu’au temps de Claude, dont il fit la satire indirecte, rejetant sur sa mère tous les crimes de ce règne, et attribuant sa mort à la fortune de Rome : car il parlait aussi du naufrage, sans songer qu’il n’y avait personne d’assez stupide pour le croire fortuit, ou pour s’imaginer qu’une femme, échappée des flots, eût envoyé un homme à travers les cohortes et les flottes de l’empereur, afin que seul, avec une épée, il brisât ce rempart. Aussi ce n’était plus sur Néron que tombait la censure publique ; sa barbarie était trop au-dessus de toute indignation : c’était sur Sénèque, auquel on reprochait d’avoir tracé dans ce discours un horrible aveu.
\subsection[{Prodiges – Clémence de Néron}]{Prodiges – Clémence de Néron}
\noindent \labelchar{XII.} On vit toutefois parmi les grands une merveilleuse émulation de bassesse. Des actions de grâces sont ordonnées dans tous les temples ; et des jeux annuels ajoutés aux fêtes de. Minerve pour célébrer la découverte du complot. On vote à la, déesse une statue d’or, qui sera placée dans le sénat, et auprès de laquelle on verra l’image du prince ; enfin le jour où naquit Agrippine est mis au nombre des jours néfastes. Pétus Thraséas, qui laissait passer les adulations ordinaires sans autre protestation que le silence, ou une adhésion froidement exprimée, sortit alors du sénat, ce qui lui attira des dangers, sans que les autres en devinssent plus libres. Des prodiges nombreux furent vus dans ce temps, et n’eurent pas plus d’effet. Une femme accoucha d’un serpent ; une autre fut tuée par la foudre dans les bras de son mari ; le soleil s’éclipsa tout à coup ; et le feu du ciel tomba dans les quatorze quartiers de Rome. Mais ces phénomènes annonçaient si peu l’intervention des dieux, qu’on vit se prolonger encore bien des années le règne et les crimes de Néron. Au reste, pour faire à sa victime une mémoire plus odieuse, et prouver que sa clémence était plus grande depuis que sa mère n’y mettait plus obstacle, il rendit à leur patrie deux femmes du premier rang, Junie et Calpurnie, et deux anciens préteurs, Valérius Capito et Licinius Gabolus, tous bannis autrefois par Agrippine. Il permit aussi qu’on rapportât les cendres de Lollia Paullina, et qu’on lui élevât un tombeau. Il fit grâce à Iturius et à Calvisius, que lui-même avait relégués depuis peu. Quant à Silana, elle avait fini ses jours à Tarente : elle y était revenue d’un exil plus éloigné, lorsque Agrippine, dont la haine avait causé sa chute, chancelait à son tour ou s’était adoucie.
\subsection[{Néron rentre à Rome}]{Néron rentre à Rome}
\noindent \labelchar{XIII.} Néron parcourait lentement les villes de Campanie, inquiet sur son retour à Rome, et craignant de n’y plus retrouver le dévouement du sénat et l’affection du peuple. Mais tous les pervers (et jamais cour n’en réunit davantage) l’assuraient « que le nom d’Agrippine était abhorré, et que sa mort avait redoublé pour lui l’enthousiasme populaire. Qu’il allât donc sans crainte, et qu’il essayât la vertu de sa présence auguste. » Eux-mêmes demandent à le précéder, et trouvent un empressement qui passait leurs promesses, les tribus accourant au-devant de lui, le sénat en habits de fête, des troupes de femmes et d’enfants, rangées suivant l’âge et le sexe, et, sur tout son passage, des amphithéâtres qu’on avait dressés comme pour voir un triomphe. Fier et vainqueur de la servilité publique, Néron monta au Capitole, rendit grâce aux dieux, et s’abandonna au torrent de ses passions, mal réprimées jusqu’alors, mais dont l’ascendant d’une mère, quelle qu’elle fût, avait suspendu le débordement.
\subsection[{Néron dans l’arène}]{Néron dans l’arène}
\noindent \labelchar{XIV.} Il avait depuis longtemps à cœur de conduire un char dans la carrière ; et par une fantaisie non moins honteuse, on le voyait souvent, tenant une lyre, imiter à table les chants du théâtre. « Des rois, disait-il, d’anciens généraux l’avaient fait avant lui. Cet art était célébré par les poëtes et servait à honorer les dieux. Le chant n’était-il pas un attribut sacré d’Apollon ? et n’était-ce pas une lyre à la main que, dans les temples de Rome, aussi bien que dans les villes de la Grèce, on représentait ce dieu, l’un des plus grands de l’Olympe, le dieu des oracles ? » Déjà rien ne pouvait plus le retenir, quand Sénèque et Burrus résolurent de lui céder une victoire, pour éviter qu’il en remportât deux. On établit dans la vallée du Vatican une enceinte fermée où il pût guider un char sans se prodiguer aux regards de la foule : bientôt le peuple romain fut appelé à ce spectacle et applaudit avec transport, avide de plaisir, comme l’est toute multitude, et joyeux de retrouver ses penchants dans le prince. On avait cru que la publicité de la honte en amènerait le dégoût ; elle ne fut qu’un aiguillon nouveau : se croyant moins flétri, plus il en flétrirait d’autres, il dégrada les fils de plusieurs nobles familles, en traînant sur la scène leur indigence vénale. Tout morts qu’ils sont, je ne les nommerai pas, par respect pour leurs ancêtres : le plus déshonoré, après tout, est celui qui emploie son or à payer l’infamie plutôt qu’à la prévenir. Des chevaliers romains d’un nom connu descendirent même dans l’arène : il les engagea pour ce honteux service à force de présents ; mais les présents de qui peut commander ne sont-ils pas une véritable contrainte ?
\subsection[{Les Juvénales}]{Les Juvénales}
\noindent \labelchar{XV.} Cependant, pour ne pas se prostituer encore sur un théâtre public, il institua la fête des Juvénales \footnote{Suivant Dion, Néron institua ces jeux à l’occasion de la première barbe, dont il consacra les poils à Jupiter Capitolin, après les avoir fait enchâsser dans une boîte d’or.}. C’est ainsi qu’il appela des jeux nouveaux, où les citoyens s’enrôlèrent en foule. Ni la noblesse ni l’âge ne retinrent personne : on vit d’anciens magistrats exercer l’art d’un histrion grec ou latin, se plier à des gestes, moduler des chants indignes de leur sexe. Des femmes même, d’une haute naissance, étudièrent des rôles indécents. Dans le bois qu’Auguste avait planté autour de sa naumachie, furent construites des salles et des boutiques où tout ce qui peut irriter les désirs était à vendre. On y distribuait de l’argent, que chacun dépensait aussitôt, les gens honnêtes par nécessité, les débauchés par vaine gloire. De là une affreuse contagion de crimes et d’infamie ; et jamais plus de séductions qu’il n’en sortit de ce cloaque impur n’assaillirent une société dès longtemps corrompue. Les bons exemples maintiennent à peine les bonnes mœurs ; comment, dans cette publique émulation de vices, eût-on sauvé le moindre sentiment de pudicité, de modestie, d’honneur ? Enfin Néron monta lui-même sur la scène, touchant les cordes d’une lyre, et préludant avec une grâce étudiée. Ses courtisans étaient près de lui, et, avec eux, une cohorte de soldats, les centurions, les tribuns et Burrus, qui gémissaient tout en applaudissant. Alors fut créé ce corps de chevaliers romains qu’on appela les Augustans, tous vigoureux et brillants de jeunesse, attirés les uns par un esprit de licence, les autres par des vues ambitieuses. Le jour entendait leurs acclamations ; ils en faisaient retentir les nuits, cherchant à la voix, et à la beauté du prince des noms parmi les dieux : ils avaient à ce prix ce qu’on mérite par la vertu, les honneurs et l’illustration.
\subsection[{Néron poète}]{Néron poète}
\noindent \labelchar{XVI.} Toutefois, afin que la gloire de l’empereur ne fût pas bornée aux arts de la scène, il ambitionna encore le nom de poëte. Il réunissait chez lui les jeunes gens qui avaient quelque talent pour les vers : là, chacun étant assis, leur tâche était de lier et d’assortir les morceaux que Néron avait apportés ou qu’il improvisait, et de remplir les mesures imparfaites, en conservant, bonnes ou mauvaises, ses propres expressions on s’en aperçoit au style de ces poésies, dénuées d’inspiration et de verve, et qui ne semblent pas couler d’une même source. Enfin il donnait aussi aux philosophes quelques moments après le repas ; et, remarquant l’opposition de leurs doctrines, il se plaisait à les mettre aux prises : car il s’en trouva plus d’un qui fut bien aise qu’on le vît, avec son maintien grave et son visage austère, servir aux passe-temps du maître.
\subsection[{Émeutes à l’amphithéâtre}]{Émeutes à l’amphithéâtre}
\noindent \labelchar{XVII.} Vers la même époque, une dispute légère fut suivie d’un horrible massacre entre les habitants de deux colonies romaines, Nucéria et Pompéi. Livinéius Régulus, que j’ai dit avoir été chassé du sénat, donnait un spectacle de gladiateurs. De ces railleries mutuelles où s’égaye la licence des petites villes, on en vint aux injures, puis aux pierres, enfin aux armes. La victoire resta aux Pompéiens, chez qui se donnait la fête. Beaucoup de Nucériens furent rapportés chez eux le corps tout mutilé ; un grand nombre pleuraient la mort d’un fils ou d’un père. Le prince renvoya le jugement dé cette affaire au sénat, et le sénat aux consuls. Le sénat, en ayant été saisi de nouveau, défendit pour dix ans à la ville de Pompéi ces sortes de réunions, et supprima les associations qui s’y étaient formées aux mépris des lois. Livinéius et les autres auteurs de la sédition furent punis de l’exil.
\subsection[{Procès}]{Procès}
\noindent \labelchar{XVIII.} Pédius Blésus perdit aussi le rang de sénateur, accusé par les Cyrénéens d’avoir violé le trésor d’Esculape, et cédé, dans la levée des soldats, à la double corruption de la brigue et de l’or. Le même peuple poursuivait Acilius Strabo ancien préteur, envoyé par Claude pour régler la propriété de plusieurs domaines possédés autrefois par le roi Apion \footnote{Le roi Apion, descendant des Lagides, dernier souverain d’une partie de la Libye, avait légué ses États au peuple romain l’an de Rome 880. Les principales villes étaient Bérénice Ptolémaïs et Cyrène.}, et que ce prince avait laissés, avec ses États, au peuple romain. Les propriétaires voisins les avaient envahis, et ils se prévalaient d’une usurpation longtemps tolérée, comme d’un titre légitime. En prononçant contre eux, le juge souleva les esprits contre lui-même. Le sénat répondit aux Cyrénéens qu’il ignorait les ordres de Claude, et qu’il fallait consulter le prince, Néron, approuvant le jugement d’Acilius, écrivit néanmoins que, par égard pour les alliés, il leur faisait don de ce qu’ils avaient usurpé.
\subsection[{Fin de carrière pour deux maîtres du barreau}]{Fin de carrière pour deux maîtres du barreau}
\noindent \labelchar{XIX.} Bientôt après, deux hommes du premier rang, Domitius Afer et M. Servilius, terminèrent une carrière qui avait brillé de tout l’éclat des honneurs et de l’éloquence, Tous deux furent célèbres au barreau ; Servilius le devint doublement en écrivant l’histoire romaine, et il le fut encore par une élégance de mœurs à laquelle la vie toute différente de son rival de génie donnait un nouveau lustre.
\subsection[{60 – les jeux quinquennaux}]{60 – les jeux quinquennaux}
\noindent \labelchar{XX.} Sous le quatrième consulat de Néron, qui eut pour collègue Cornelius Cassus, des jeux quinquennaux, institués à Rome à l’imitation des combats de la Grèce, donnèrent lieu, comme toutes les nouveautés, à des réflexions diverses. Selon les uns, « Pompée lui-même avait encouru le blâme des vieillards en établissant un théâtre permanent ; car avant lui la scène et les gradins, érigés pour le besoin présent, ne duraient pas plus que les jeux et même, si l’on remontait plus haut, le peuple y assistait debout ; assis, on eût craint qu’il ne consumât des journées entières dans l’oisiveté du théâtre. Au moins fallait-il s’en tenir aux spectacles anciens, tels que les donnaient encore les préteurs, où nul citoyen n’était obligé de disputer le prix. Les mœurs de la patrie, altérées peu à peu, allaient périr entièrement par cette licence importée. Ainsi tout ce qui peut au monde recevoir et donner la corruption serait vu dans Rome ! ainsi dégénérerait, énervée par des habitudes étrangères, une jeunesse dont les gymnases, le désœuvrement et d’infâmes amours se partageraient la vie ; et cela par la volonté du prince et du sénat, qui, non contents de tolérer le vice, en faisaient une loi. Que les grands de Rome allassent donc, sous le nom de poëtes et d’orateurs se dégrader sur la scène. Que leur restait-il à faire, sinon de jeter leurs vêtements, de prendre le ceste, et de renoncer, pour les combats de l’arène, à la guerre et aux armes ? En seraient-ils des augures plus savants et les chevaliers en rempliraient-ils mieux les nobles fonctions de juges, pour avoir entendu en connaisseurs des voix mélodieuses et des chants efféminés ? Les nuits mêmes étaient ajoutées aux heures du scandale, afin que pas un instant ne fût laissé à la pudeur, et que, dans ces confus rassemblements ce que le vice aurait convoité pendant le jour, il l’osât au milieu des ténèbres. »\par
\labelchar{XXI.} C’était cette licence même qui plaisait au plus grand nombre, et cependant ils couvraient leur secrète pensée de prétextes honnêtes. « Nos ancêtres, disaient-ils, ne s’étaient pas refusé plus que nous le délassement des spectacles, et ils en avaient de conformes à leur fortune : c’est ainsi que des Étrusques ils avaient pris les histrions, des Thuriens \footnote{Thurium, bâtie après la destruction de Sybaris et non loin de ses ruines, était située entre les rivières de Crathis et de Sybaris, prés du golfe de Tarente.} les courses de chevaux. Maîtres de la Grèce et de l’Asie, ils avaient donné plus de pompe à leurs jeux, sans qu’aucun Romain de naissance honnête se fût abaissé jusqu’aux arts de la scène, pendant les deux siècles écoulés depuis le triomphe de Mummius, qui le premier avait montré à Rome ces spectacles nouveaux. C’était au reste par économie qu’on avait bâti un théâtre fixe et durable, au lieu de ces constructions éphémères que chaque année voyait s’élever à grands frais. Plus de nécessité aux magistrats d’épuiser leur fortune à donner des spectacles grecs, plus de motifs aux cris du peuple pour en obtenir des magistrats, lorsque l’Etat ferait cette dépense. Les victoires des poëtes et des orateurs animeraient les talents et quel juge, enviant à son oreille un plaisir légitime, serait fâché d’assister à ces nobles exercices de l’esprit ? C’était à la joie, bien plus qu’à la licence, que l’on consacrait quelques nuits en cinq ans, nuits éclairées de tant de feux, qu’elles n’auraient plus d’ombres pour cacher le désordre. » Il est certain que cette fête passa sans laisser après elle aucune éclatante flétrissure. Le peuple même ne se passionna pas un instant. C’est que les pantomimes, quoique rendus à la scène, n’étaient pas admis dans les jeux sacrés. Personne ne remporta le prix de l’éloquence ; mais Néron fut proclamé vainqueur. L’habillement grec, avec lequel beaucoup de personnes s’étaient montrées pendant la durée des fêtes, fut quitté aussitôt.
\subsection[{A la recherche d’un successeur}]{A la recherche d’un successeur}
\noindent \labelchar{XXII.} Il parut dans ce temps une comète, présage, aux yeux du peuple, d’un règne qui va finir. A cette vue, comme si Néron eût été déjà renversé du trône, les pensées se tournèrent vers le choix de son successeur. Toutes les voix proclamaient Rubellius Plautus, qui par sa mère tirait sa noblesse de la famille des Jules. Attaché aux maximes antiques, Plautus avait un extérieur austère ; sa maison était chaste, sa vie retirée ; et plus il s’enveloppait d’une prudente obscurité, plus la renommée le mettait en lumière. Les conjectures non moins vaines auxquelles donna lieu un coup de tonnerre accrurent encore ces rumeurs : comme Néron soupait auprès des lacs Simbruins, dans le lieu nommé Sublaqueum \footnote{Tacite, liv. XI, ch. XI, a fait mention des monts Simbruins. Pline parle de trois lacs fort agréables formés par l’Anio, ou Téveron, qui ont donné le nom au lieu appelé Sublaqueum.}, les mets furent atteints de la foudre, et la table fracassée ; or, cet événement étant arrivé sur les confins de Tibur, d’où Plautus tirait son origine paternelle, on en conclut que la volonté des dieux le destinait à l’empire. Il eut même des courtisans parmi ces hommes qu’une politique intéressée et souvent trompeuse hasarde les premiers au devant des fortunes naissantes. Néron alarmé écrivit à Plautus « de pourvoir au repos de la ville, et de se dérober à la méchanceté de ses diffamateurs ; qu’il avait en Asie des domaines héréditaires, où, loin des dangers et du trouble, il jouirait en paix de sa jeunesse. » Plautus partit avec sa femme Antistia et quelques amis. A la même époque, une recherche indiscrète de plaisir valut à Néron infamie et péril : il avait nagé dans la fontaine d’où l’eau Marcia (2) est amenée à Rome, et l’on croyait qu’en y plongeant son corps il avait profané une source sacrée, et violé la sainteté du lieu. Une maladie qui vint à la suite parut un témoignage de la colère céleste.
\subsection[{À l’extérieur – Corbulon en Arménie}]{À l’extérieur – Corbulon en Arménie}
\noindent \labelchar{XXIII.} Cependant Corbulon, qui venait de raser Artaxate, voulut profiter d’une première impression de terreur pour s’emparer de Tigranocerte, afin de redoubler l’effroi des ennemis en détruisant cette ville, ou de s’acquérir, en la conservant, un renom de clémence. Il y marcha donc, mais d’une marche inoffensive, pour ne pas porter le désespoir devant lui, et toutefois sans négliger les soins de la prudence, à cause de l’humeur changeante de ces peuples, lents et craintifs à l’aspect du danger, toujours prêts à l’heure de la trahison. Les barbares, chacun selon son caractère, se présentent en suppliants, ou abandonnent leurs hameaux et fuient loin des routes pratiquées. Il y en eut même qui se cachèrent dans des cavernes avec ce qu’ils avaient de plus cher. Le général romain, ménageant habilement sa conduite, faisait grâce aux prières, poursuivait la fuite avec rapidité. Impitoyable pour ceux qui occupaient des retraites souterraines, il leur ferma toutes les issues avec des sarments et des broussailles, et les brilla dans leurs repaires. Comme il longeait les frontières des Mardes \footnote{Au pied des monts Gordyens.}, cette nation, exercée au brigandage et défendue par des monts inaccessibles, le harcela de ses incursions. Il envoya les Ibériens ravager leur pays, et l’audace de cet ennemi fut punie aux dépens d’un sang étranger.
\subsection[{Prise de Tigranocerte}]{Prise de Tigranocerte}
\noindent \labelchar{XXIV.} Mais si Corbulon et son armée ne perdaient rien par le combat, toute leur vigueur pliait sous le faix des travaux et de la misère. Réduits pour unique nourriture à la chair des bestiaux, le manque d’eau, un été brillant, de longues marches, mettaient le comble à leurs souffrances, que la seule patience du général adoucissait un peu : car lui-même endurait plus de maux que le dernier des soldats. On arriva ensuite dans des lieux cultivés, et l’on fit la moisson. De deux forteresses où les Arméniens s’étaient réfugiés, l’une fut prise d’assaut ; celle qui repoussa la première attaque fut forcée par un siège. On passa de là dans le pays des Taurannites, où Corbulon sortit heureusement d’un péril inattendu. Un barbare de distinction, surpris non loin de sa tente avec un poignard et mis à la torture, s’avoua l’auteur d’une conspiration, dont il découvrit le plan et les complices. Les traîtres qui, sous le masque de l’amitié, tramaient un assassinat, furent convaincus et punis. Bientôt après, Tigranocerte annonça par une députation que ses portes étaient ouvertes, et qu’elle était prête à recevoir des ordres. En même temps elle envoyait une couronne d’or, gage d’hospitalité. Corbulon reçut les députés avec honneur et n’ôta rien à la ville, dans l’espoir qu’une obéissance plus zélée serait lé prix de ce bienfait.\par
\labelchar{XXV.} Mais la citadelle, où se tenait enfermée une jeunesse intrépide, ne fut pas réduite sans combat. Ils affrontèrent au pied de leurs murs les hasards d’une bataille, et, repoussés derrière les remparts, ils ne cédèrent qu’à l’extrémité, lorsque déjà l’on forçait la place. Ces succès étaient facilités par la guerre d’Hyrcanie, qui occupait les Parthes. Les Hyrcaniens avaient même envoyé vers l’empereur pour lui demander son alliance, faisant valoir, comme une preuve de leur amitié, l’occupation qu’ils donnaient à Vologèse. A leur retour, les députés risquaient d’être surpris, de l’autre côté de l’Euphrate par les détachements de l’ennemi : Corbulon leur donna une escorte et les fit accompagner jusqu’aux bords de la mer Rouge, d’où, en évitant les frontières des Parthes, ils retournèrent dans leur patrie.
\subsection[{Rome installe Tigranes comme roi d’Arménie}]{Rome installe Tigranes comme roi d’Arménie}
\noindent \labelchar{XXVI.} Tiridate essayait de pénétrer en Arménie par le pays des Mèdes. Le général détache aussitôt le lieutenant Vérulanus avec les auxiliaires, le suit rapidement à la tête des légions, et force le barbare de fuir au loin et de renoncer à ses projets de guerre. Enfin, ayant désolé par le fer et la flamme ceux qu’il savait animer, à cause du roi, de sentiments hostiles, il était en pleine possession de l’Arménie, lorsque parut Tigranes, choisi par Néron pour souverain de cette contrée. Tigranes, né d’un sang illustre en Cappadoce, était petit-fils du roi Archélaüs ; mais retenu longtemps comme otage à Rome, il en avait rapporté l’esprit lâche et rampant d’un esclave. Il ne fut pas reçu sans opposition. Les Arsacides régnaient encore dans quelques âmes ; mais le plus grand nombre, révolté de l’orgueil des Parthes, préférait un roi donné par les Romains. On laissa auprès de Tigranes un détachement de mille légionnaires, trois cohortes alliées et deux ailes de cavalerie ; et, afin qu’il maintînt plus facilement son pouvoir naissant, on soumit aux rois Pharasmane, Polémon, Aristobule et Antiochus, les parties de l’Arménie voisines de leurs États. Corbulon se retira dans la Syrie, privée de gouverneur par la mort de Quadratus, et confiée à ses soins.
\subsection[{Tremblement de terre à Laocidée – Vétérans à Tarente}]{Tremblement de terre à Laocidée – Vétérans à Tarente}
\noindent \labelchar{XXVII.} La même année, un tremblement de terre renversa Laodicée \footnote{Laodicée de Phrygie, dont le nom subsiste encore dans celui de Ladik.} l’une des cités les plus célèbres de l’Asie : elle se releva par elle-même et sans notre concours. En Italie, l’ancienne ville de Pouzzoles obtint de Néron les droits et le surnom de colonie romaine. Des vétérans furent désignés pour habiter Antium et Tarente, et ne remédièrent point à la dépopulation de ces villes : ils se dispersèrent presque tous, et chacun regagna la province où il avait achevé son service. Étrangers d’ailleurs à l’usage de se marier et d’élever des enfants, ils ne laissaient dans leurs maisons désertes aucune postérité. Car ce n’étaient plus ces légions que jadis on établissait tout entières, tribuns, centurions, soldats de mêmes manipules, et qui, unies d’esprit et de cœur, ne tardaient pas à former une cité : c’étaient des hommes inconnus entre eux, tirés de différents corps, sans chef, sans affection mutuelle, qui tous venaient comme d’un autre monde, et dont le soudain assemblage formait une multitude plutôt qu’une colonie.
\subsection[{Mesures diverses}]{Mesures diverses}
\noindent \labelchar{XXVIII.} L’élection des préteurs, ordinairement abandonnée au sénat, fut agitée par des brigues plus vives que de coutume : le prince y ramena la paix, en mettant à la tête d’une légion trois candidats qui excédaient le nombre des charges. Il releva la dignité des sénateurs, en ordonnant que ceux qui, des juges particuliers, appelleraient au sénat, consigneraient la même somme que ceux qui appelaient à César. Auparavant, les appels à cet ordre étaient libres et francs de toute amende. A la fin de l’année, Vibius Sécundus, chevalier romain. accusé de concussion par les Maures, fut condamné et chassé d’Italie. Le crédit de son frère Crispus le sauva seul d’une peine plus sévère.
\subsection[{61 – À l’extérieur – En Bretagne}]{61 – À l’extérieur – En Bretagne}
\noindent \labelchar{XXIX.} Sous le consulat de Césonius Pétus et de Pétronius Turpilianus, l’empire essuya en Bretagne un sanglant désastre. J’ai déjà dit que le lieutenant Aulus Didius s’était contenté d’y maintenir nos conquêtes. Véranius, son successeur, fit quelques incursions chez les Silures, et, surpris par la mort, il ne put porter la guerre plus loin. Cet homme, à qui la renommée attribua toute sa vie une austère indépendance, laissa voir, dans les derniers mots de son testament, l’esprit d’un courtisan : il y prodiguait mille flatteries à Néron, ajoutant que, s’il eût vécu encore deux années, il lui aurait soumis la province tout entière. Après lui, les Bretons eurent pour gouverneur Suétonius Paullinus, que ses talents militaires et la voix publique, qui ne laisse jamais le mérite sans rival, donnaient pour émule à Corbulon. Lui-même songeait à l’Arménie reconquise, et brûlait d’égaler un exploit si glorieux en domptant les rebelles. L’île de Mona \footnote{Anglesey.}, déjà forte par sa population, était encore le repaire des transfuges : il se dispose à l’attaquer, et construit des navires dont la carène fût assez plate pour aborder sur une plage basse et sans rives certaines. Ils servirent à passer les fantassins ; la cavalerie suivit à gué ou à la nage, selon la profondeur des eaux.\par
\labelchar{XXX.} L’ennemi bordait le rivage : à travers ses bataillons épais et hérissés de fer, couraient, semblables aux Furies, des femmes échevelées, en vêtements lugubres, agitant des torches ardentes ; et des druides, rangés à l’entour, levaient les mains vers le ciel avec d’horribles prières. Une vue si nouvelle étonna les courages, au point que les soldats, comme si leurs membres eussent été glacés, s’offraient immobiles aux coups de l’ennemi. Rassurés enfin par les exhortations du général, et s’excitant eux-mêmes à ne pas trembler devant un troupeau fanatique de femmes et d’insensés, ils marchent en avant, terrassent ce qu’ils rencontrent, et enveloppent les barbares de leurs propres flammes. On laissa garnison chez les vaincus, et l’on coupa les bois consacrés à leurs atroces superstitions ; car ils prenaient pour un culte pieux d’arroser les autels du sang des prisonniers, et de consulter les dieux dans des entrailles humaines. Au milieu de ces travaux, Suétonius apprit que la province venait tout à coup de se révolter.\par
\labelchar{XXXI.} Le roi des Icéniens, Prasutagus, célèbre par de longues années d’opulence, avait nommé l’empereur son héritier, conjointement avec ses deux filles. Il croyait que cette déférence mettrait à l’abri de l’injure son royaume et sa maison. Elle eut un effet tout contraire : son royaume, en proie à des centurions, sa maison, livrée à des esclaves, furent ravagés comme une conquête. Pour premier outrage, sa femme Boadicée est battue de verges, ses filles déshonorées : bientôt, comme si tout le pays eût été donné en présent aux ravisseurs, les principaux de la nation sont dépouillés des biens de leurs aïeux, et jusqu’aux parents du roi sont mis en esclavage. Soulevés par ces affronts et par la crainte de maux plus terribles (car ils venaient d’être réduits à l’état de province), les Icéniens courent aux armes et entraînent dans leur révolte les Trinobantes\footnote{Les Trinobantes habitaient entre les Icéniens au nord et la Tamise au sud : maintenant les comtés de Middlesex et d’Essex.} et d’autres peuples, qui, n’étant pas encore brisés à la servitude, avaient secrètement conjuré de s’en affranchir. L’objet de leur haine la plus violente étaient les vétérans, dont une colonie, récemment conduite à Camulodunum, chassait les habitants de leurs maisons, les dépossédait de leurs terres, en les traitant de captifs et d’esclaves, tandis que les gens de guerre, par une sympathie d’état et l’espoir de la même licence, protégeaient cet abus de la force. Le temple élevé à Claude offensait aussi les regards, comme le siège et la forteresse d’une éternelle domination ; et ce culte nouveau engloutissait la fortune de ceux qu’on choisissait pour en être les ministres. Enfin il ne paraissait pas difficile de détruire une colonie qui n’avait point de remparts, objet auquel nos généraux avaient négligé de pourvoir, occupés qu’ils étaient de l’agréable avant de songer à l’utile.\par
\labelchar{XXXII.} Dans ces conjonctures, une statue de la Victoire, érigée à Camulodunum, tomba sans cause apparente et se trouva tournée en arrière, comme si elle fuyait devant l’ennemi. Des femmes agitées d’une fureur prophétique annonçaient une ruine prochaine. Le bruit de voix étrangères entendu dans la salle du conseil, le théâtre retentissant de hurlements plaintifs, l’image d’une ville renversée vue dans les flots de la Tamise, l’Océan couleur de bang, et des simulacres de cadavres humains abandonnés par le reflux, tous ces prodiges que l’on racontait remplissaient les vétérans de terreur et les Bretons d’espérance. Comme Suétonius était trop éloigné, on demanda du secours au procurateur Catus Décianus. Il n’envoya pas plus de deux cents hommes mal armés, et la colonie n’avait qu’un faible détachement de soldats. On comptait sur les fortifications du temple, et d’ailleurs de secrets complices de la rébellion jetaient le désordre dans les conseils ; aussi on ne s’entoura ni de fossés ni de palissades, on n’éloigna point les vieillards et les femmes pour n’opposer à l’ennemi que des guerriers. La ville, aussi mal gardée qu’en pleine paix, est envahie subitement par une nuée de barbares. Tout fut en un instant pillé ou mis en cendres ; le temple seul, où s’étaient ralliés les soldats, soutint un siège et fut emporté le second jour. Pétilius Cérialis, lieutenant de la neuvième légion, arrivait au secours ; les Bretons victorieux vont au-devant de lui et battent cette légion. Ce qu’il y avait d’infanterie fut massacré ; Cérialis, avec la cavalerie, se sauva dans son camp et fut protégé par ses retranchements. Alarmé de cette défaite et haï de la province, que son avarice avait poussée à la guerre, le procurateur Catus se retira précipitamment dans la Gaule.\par
\labelchar{XXXIII.} Mais Suétonius, avec un courage admirable, perce au travers des ennemis, et va droit à Londinium, ville qui, sans être décorée du nom de colonie, était l’abord et le centre d’un commerce immense. Il délibéra s’il choisirait ce lieu pour théâtre de la guerre. Mais, voyant le peu de soldats qui était aux environs et la terrible leçon qu’avait reçue la témérité de Cérialis, il résolut de sacrifier une ville pour sauver la province. En vain les habitants en larmes imploraient sa protection ; inflexible à leurs gémissements, il donne le signal du départ, et emmène avec l’armée ceux qui veulent la suivre. Tout ce que retint la faiblesse du sexe, ou la caducité de l’âge, ou l’attrait du séjour, tout fut massacré par l’ennemi. Le municipe de Vérulam \footnote{Ancienne ville, dont l’illustration a été renouvelée par le titre de baron de Vérulam, donné au célèbre chancelier Bacon. C’est aujourd’hui Saint-Albans, dans le comté d’Hertford.} éprouva le même sort ; car les Bretons laissaient de côté les forts et les postes militaires, courant, dans la joie du pillage et l’oubli de tout le reste, aux lieux qui promettaient les plus riches dépouilles et le moins de résistance. On calcula que soixante-dix mille citoyens ou alliés avaient péri dans les endroits que j’ai nommés. Faire des prisonniers, les vendre, enfin tout trafic de guerre, eût été long pour ces barbares : les gibets, les croix, le fer, le feu, servaient mieux leur fureur ; on eût dit qu’ils s’attendaient à l’expier un jour, et qu’ils vengeaient par avance leurs propres supplices.\par
\labelchar{XXXIV.} Suétonius réunit à la quatorzième légion les vexillaires de la vingtième et ce qu’il y avait d’auxiliaires dans le voisinage. Il avait environ dix mille hommes armés, lorsque, sans temporiser davantage, il se dispose au combat. Il choisit une gorge étroite et fermée par un bois, bien sûr auparavant qu’il n’avait d’ennemis qu’en face, et que la plaine, unie et découverte, ne cachait point d’embûches. C’est là qu’il s’établit, la légion au centre et les rangs serrés, les troupes légères rangées à l’entour, la cavalerie ramassée sur les ailes. Quant aux Bretons, leurs bandes à pied et à cheval se croisaient et voltigeaient tumultueusement, plus nombreuses qu’en aucune autre bataille, et animées d’une audace si présomptueuse, que, afin d’avoir jusqu’aux femmes pour témoins de la victoire, elles les avaient traînées à leur suite, et placées sur des chariots qui bordaient l’extrémité de la plaine.
\subsection[{Boadicée}]{Boadicée}
\noindent \labelchar{XXXV.} Boadicée, montée sur un char, ayant devant elle ses deux filles, parcourait l’une après l’autre ces nations rassemblées, en protestant « que, tout accoutumés qu’étaient les Bretons à marcher à l’ennemi conduits par leurs reines, elle ne venait pas, fière de ses nobles aïeux, réclamer son royaume et ses richesses ; elle venait, comme une simple femme, venger sa liberté ravie, son corps déchiré de verges, l’honneur de ses filles indignement flétri. La convoitise romaine, des biens, était passée aux corps, et ni la vieillesse ni l’enfance n’échappaient à ses souillures. Mais les dieux secondaient enfin une juste vengeance : une légion, qui avait osé combattre, était tombée tout entière ; le reste des ennemis se tenait caché dans son camp, ou ne songeait qu’à la fuite. Ils ne soutiendraient pas le bruit même et le cri de guerre, encore moins le choc et les coups d’une si grande armée. Qu’on réfléchît avec elle au nombre des combattants et aux causes de la guerre, on verrait qu’il fallait vaincre en ce lieu ou bien y périr. Femme, c’était là sa résolution : les hommes pouvaient choisir la vie et l’esclavage. »\par
\labelchar{XXXVI.} Suétonius ne se taisait pas non plus en ce moment décisif. Plein de confiance dans la valeur de ses troupes, il les exhortait cependant, il les conjurait « de mépriser ce vain fracas et ces menaces impuissantes de l’armée barbare : on y voyait plus de femmes que de soldats ; cette multitude sans courage et sans armes lâcherait pied sitôt qu’elle reconnaîtrait, tant de fois vaincue, le fer et l’intrépidité de ses vainqueurs. Beaucoup de légions fussent-elles réunies, c’était encore un petit nombre de guerriers qui gagnait les batailles ; et ce serait pour eux un surcroît d’honneur d’avoir prouvé qu’une poignée de braves valait une grande armée. Ils devaient seulement se tenir serrés, lancer leurs javelines, puis, frappant de l’épée et du bouclier, massacrer sans trêve ni relâche, et ne pas s’occuper du butin : la victoire livrerait tout en leurs mains. » Telle fut l’ardeur qui éclatait à chacune de ces paroles, et l’air dont balançaient déjà leurs redoutables javelines ces vieux soldats éprouvés dans cent batailles, que Suétonius, assuré du succès, donna aussitôt le signal du combat.\par
\labelchar{XXXVII.} Immobile d’abord, et se faisant un rempart de la gorge étroite où elle était postée, la légion attendit que l’ennemi s’approchât, pour lui envoyer des coups plus sûrs. Quand elle eut épuisé ses traits, elle s’avança rapidement en forme de coin. Les auxiliaires chargent en même temps, et les cavaliers, leurs lances en avant, rompent et abattent ce qui résiste encore. Le reste fuyait ou plutôt essayait de fuir à travers la haie de chariots qui fermait les passages. Le soldat n’épargna pas môme les femmes ; et jusqu’aux bêtes de somme tombèrent sous les traits et grossirent les monceaux de cadavres. Cette journée fut glorieuse et comparable à nos anciennes victoires : quelques-uns rapportent qu’il n’y périt guère moins de quatre-vingt mille Bretons. Quatre cents soldats environ furent tués de notre côté ; il n’y eut pas beaucoup de blessés. Boadicée finit sa vie par le poison. Quand Pénius Postumus, préfet de camp de la deuxième légion, apprit le succès de la quatorzième et de la vingtième, désespéré d’avoir privé la sienne d’une gloire pareille en se refusant, contre les lois de la discipline, aux ordres du général, il se perça de son épée.\par
\labelchar{XXXVIII.} Toute l’armée fut ensuite réunie et tenue sous la tente, pour éteindre les derniers restes de la guerre. L’empereur la renforça en envoyant de Germanie deux mille légionnaires, huit cohortes alliées et mille chevaux. Les soldats légionnaires servirent à compléter la neuvième légion ; les cohortes et la cavalerie furent placées dans des cantonnements nouveaux, et toutes les nations qui s’étaient montrées indécises ou ennemies en furent punies par le fer et la flamme. Mais aucun fléau ne les désolait autant que la famine : comme elles avaient compté sur nos magasins, tous les âges s’étaient tournés vers la guerre, sans qu’on se mît en peine d’ensemencer les champs. Toutefois ces peuples opiniâtres tardaient à déposer les armes, parce que Julius Classicianus, successeur de Catus et ennemi du général, opposait au bien public ses haines particulières. Il faisait débiter qu’il fallait prendre un nouveau chef, qui, n’ayant ni la colère d’un ennemi ni l’orgueil d’un vainqueur, userait de clémence envers la soumission. En même temps il écrivait à Rome que la lutte ne finirait jamais tant que Suétonius ne serait pas remplacé, attribuant les revers à sa mauvaise conduite et les succès à la fortune de l’empire.\par
\labelchar{XXXIX.} Néron envoya l’affranchi Polyclète pour reconnaître l’état de la Bretagne : il avait un grand espoir que son ascendant rétablirait la concorde entre le général et le procurateur, et que même il ramènerait à la paix les esprits rebelles des barbares. Polyclète ne manqua pas d e traîner au delà de l’Océan ce cortège immense dont il avait foulé l’Italie et la Gaule, et de marcher redoutable à nos soldats eux-mêmes : mais il fut la risée des Bretons ; la liberté vivait encore dans leurs âmes, et ils ne connaissaient pas alors cette puissance des affranchis. Leur étonnement était grand de voir le général et l’armée qui venaient d’achever une guerre si terrible obéir à des esclaves. Au reste, l’état des choses fut présenté à Néron sous un jour favorable, et Suétonius laissé à la tête des affaires. Depuis, ayant perdu sur le rivage quelques navires avec leurs rameurs, il reçut ordre, comme si la guerre eût encore duré, de remettre l’armée à Pétronius Turpilianus, déjà sorti du consulat. Celui-ci, sans provoquer l’ennemi ni en être inquiété, décora du nom de paix sa molle inaction.
\subsection[{À Rome – Deux crimes}]{À Rome – Deux crimes}
\noindent \labelchar{XL.} La même année, deux crimes fameux signalèrent à Rome l’audace d’un sénateur et celle d’un esclave. Il y avait un ancien préteur, nommé Domitius Balbus, riche, sans enfants, et qu’une longue vieillesse livrait aux pièges de la cupidité. Un de ses parents, Valérius Fabianus, destiné à la carrière des honneurs, lui supposa un testament, de concert avec Vinicius Rufinus et Térentius Lentinus, chevaliers romains. Ceux-ci mirent dans le complot Antonius Primus \footnote{C’est ce même Antonius qui joue un si grand rôle dans la guerre entre Vitellius et Vespasien.} et Asinius Marcellus, le premier d’une audace à tout entreprendre, le second brillant du lustre de son bisaïeul Asinius Pollio, et jusqu’alors estimé pour ses mœurs, si ce n’est qu’il regardait la pauvreté comme le dernier des maux. Fabianus fit sceller l’acte faux par ceux que je viens de dire et par d’autres d’un rang moins élevé, et il en fut convaincu devant le sénat. Lui et Antonius furent condamnés, avec Rufinus et Térentius, aux peines de la loi Cornélia \footnote{Exil, déportation dans une île, ou exclusion du sénat.}. Quant à Marcellus, la mémoire de ses ancêtres et les prières de César le sauvèrent du châtiment plutôt que de l’infamie.
\subsection[{Condamnation}]{Condamnation}
\noindent \labelchar{LI.} Le même jour vit frapper aussi Pompéius Élianus, jeune homme qui avait été questeur, et qu’on jugea instruit des bassesses de Fabianus. Le séjour de l’Italie, ainsi que de l’Espagne, où il était né, lui fut interdit. Valérius Ponticus subit la même flétrissure, parce que, afin de soustraire les coupables à la justice du préfet de Rome, il les avait déférés au préteur, couvrant d’un prétendu respect des lois une collusion ménagée pour éluder leur vengeance. Il fut ajouté au sénatus-consulte que quiconque aurait acheté ou vendu de telles connivences serait soumis aux mêmes peines\footnote{Ces peines étaient l’infamie, le talion, l’exil, la relégation dans une île, ou l’exclusion de l’ordre auquel on appartenait.} que le calomniateur condamné par un jugement public
\subsection[{Faut-il supplicier tous les esclaves lors d’un crime ?}]{Faut-il supplicier tous les esclaves lors d’un crime ?}
\noindent \labelchar{XLII.} Peu de temps après, le préfet de Rome Pédanius Sécundus fut tué par un de ses esclaves, soit qu’il eût refusé de l’affranchir après être convenu du prix de sa liberté, soit que l’esclave, jaloux de ses droits sur le complice d’un vil amour, ne pût souffrir son maître pour rival. Lorsque, d’après un ancien usage, il fut question de conduire au supplice tous les esclaves qui avaient habité sous le même toit, la pitié du peuple, émue en faveur de tant d’innocents, éclata par des rassemblement qui allèrent jusqu’à la sédition. Dans le sénat même un parti repoussait avec chaleur cette excessive sévérité, tandis que la plupart ne voulaient aucun changement. Parmi ces derniers, C. Cassius, quand son tour d’opiner fut venu, prononça ce discours :\par
\labelchar{XLIII.} « Souvent, pères conscrits, j’ai vu soumettre à vos délibérations des demandes qui allaient à contredire par des règlements nouveaux les principes et les lois de nos pères, et je ne les ai pas combattues. Non que je doutasse qu’en toutes choses la prévoyance des anciens n’eût été mieux inspirée que la nôtre, et qu’innover dans ses décrets, ce ne fût changer le bien en mal ; mais je craignais que trop d’attachement` aux coutumes antiques ne fût attribué au désir de relever la science que je cultive ; et de plus, je ne voulais pas affaiblir, par une opposition habituelle, l’autorité que peuvent avoir mes paroles, afin de la trouver entière au moment où la république aurait besoin de conseils. Ce moment est venu, aujourd’hui qu’un consulaire est assassiné dans ses foyers, par la trahison d’un esclave, trahison que pas un des autres n’a ni prévenue ni révélée, quoique aucune attaque n’eût encore ébranlé le sénatus-consulte qui les menaçait tous du dernier supplice. Décrétez maintenant l’impunité : qui de nous trouvera dans sa dignité de maître une sauvegarde que le préfet de Rome n’a pas trouvée dans sa place ? qui s’assurera en de nombreux serviteurs, lorsque quatre cents n’ont pas sauvé Pédanius ? A qui porteront secours des esclaves que déjà la crainte de la mort n’intéresse pas à nos dangers ? Dira-t-on, ce que plusieurs n’ont pas rougi de feindre, que le meurtrier avait des injures à venger ? Apparemment il avait hérité de son père l’argent de sa rançon, ou l’esclave qu’on lui enlevait était un bien de ses aïeux ! Faisons plus : prononçons que, s’il a tué son maître, il en avait le droit.\par
\labelchar{XLIV.} Veut-on argumenter sur des questions résolues par de plus sages que nous ? Eh ! bien, si nous avions celle-ci à décider pour la première fois, croyez-vous qu’un esclave ait conçu le dessein d’assassiner son maître, sans qu’il lui soit échappé quelque parole menaçante, sans qu’une seule indiscrétion ait trahi sa pensée ? Je veux qu’il l’ait enveloppée de secret, que personne ne l’ait vu aiguiser son poignard : pourra-t-il traverser les gardes de nuit, ouvrir la chambre, y porter de la lumière, consommer le meurtre, à l’insu de tout le monde ? Mille indices toujours précèdent le crime. Si nos esclaves le révèlent, nous pourrons vivre seuls au milieu d’un grand nombre, sûrs de notre vie parmi des gens inquiets pour la leur ; enfin, entourés d’assassins, si nous devons périr, ce ne sera pas sans vengeance. Nos ancêtres redoutèrent toujours l’esprit de l’esclavage, alors même que, né dans le champ ou sous le toit de son maître, l’esclave apprenait à le chérir en recevant le jour. Mais depuis que nous comptons les nôtres par nations \footnote{Les troupes immenses d’esclaves que possédaient quelques Romains étaient divisées selon leur pays, leur couleur, leur âge : c’est-à-dire qu’on mettait respectivement ensemble les Thraces, les Phrygiens, les Africains, etc.}, dont chacune a ses mœurs et ses dieux, non, ce vil et confus assemblage ne sera jamais contenu que par la crainte. Quelques innocents périront. Eh ! lorsqu’on décime une armée qui a fui, le sort ne peut-il pas condamner môme un brave à expirer sous le bâton ? Tout grand exemple est mêlé d’injustice, et le mal de quelques-uns est racheté par l’avantage de tous. »\par
\labelchar{XLV.} A cet avis de Cassius, que personne n’osa combattre individuellement, cent voix confuses répondaient en plaignant le nombre, l’âge, le sexe de ces malheureux, et, pour la plupart, leur incontestable innocence. Le parti qui voulait le supplice prévalut cependant. Mais la multitude attroupée, et qui s’armait déjà de pierres et de torches, arrêtait l’exécution. Le prince réprimanda le peuple par un édit, et borda de troupes tout le chemin par où les condamnés furent conduits à la mort. Cingonius Varro avait proposé d’étendre la punition aux affranchis qui demeuraient sous le même toit, et de les déporter hors de l’Italie. Le prince s’y opposa, pour ne pas aggraver par de nouvelles rigueurs un usage ancien que la pitié n’avait pas adouci.
\subsection[{Autres accusations}]{Autres accusations}
\noindent \labelchar{XLVI.} Sous les mêmes consuls, Tarquitius Priscus, accusé par les Bithyniens, fut condamné aux peines de la concussion ; à la grande joie des sénateurs, qui se souvenaient de l’avoir vu accuser lui-même son proconsul Statilius Taurus. Il y eut dans les Gaules un recensement des biens. Q. Volusius et Sextius Africanus, qui le firent avec Trébellius Maximus, étaient divisés par des prétentions de naissance. Pendant qu’ils se disputaient le premier rang, leurs communs dédains y placèrent Trébellius.
\subsection[{Mort de Memmius Régulus}]{Mort de Memmius Régulus}
\noindent \labelchar{XLVII.} La même année, mourut Memmius Régulus, dont le crédit, le caractère, la renommée, eurent autant d’éclat qu’il est permis d’en avoir sous l’ombre du trône impérial. Un jour, Néron, malade et entouré de flatteurs qui lui disaient que c’en était fait de l’empire si le destin ne préservait ses jours, répondit qu’un appui restait à la république. On lui demanda lequel : « C’est, dit-il, Memmius Régulus. » Régulus survécut cependant, protégé par le silence de sa vie : il n’était d’ailleurs ni d’une maison anciennement illustre, ni d’une opulence à tenter les envieux. Néron fit cette année la dédicace d’un gymnase, et, par une libéralité toute grecque, il fournit l’huile aux chevaliers et aux sénateurs.
\subsection[{62 – À Rome – Lèse-majesté}]{62 – À Rome – Lèse-majesté}
\noindent \labelchar{XLVIII.} Sous le consulat de P. Marius et de L. Asinius, le préteur Antistius, qui, étant tribun du peuple, s’était signalé, comme je l’ai dit, par l’abus de son pouvoir, composa des vers injurieux pour le prince, et les lut devant de nombreux convives, à un souper chez Ostorius Scapula. Aussitôt il fut accusé de lèse-majesté par Cossutianus Capito, qui, à la prière de Tigelli, son beau-père, avait recouvré depuis peu le rang de sénateur. C’était la première fois que la loi de majesté fût remise en vigueur ; on croyait môme que le but de ce procès était moins la perte de l’accusé que la gloire du prince, et que, lorsque Antistius aurait été condamné par le sénat, Néron userait de sa puissance tribunitienne pour le sauver de la mort. Appelé en témoignage, Ostorius déclara n’avoir rien entendu : on crut de préférence les témoins qui accusaient. Junius Marullus, désigné consul, opina pour que le coupable fût destitué de la préture, et mis à mort suivant la coutume de nos ancêtres. Chacun approuvant cet avis, Thraséas se lève à son tour, et, après un hommage éclatant rendu à César, et une vive censure d’Antistius, il ajoute : « que, sous un si bon prince, et quand le sénat n’est enchaîné par aucune nécessité, ses arrêts ne doivent pas ordonner tout ce que le criminel mériterait de souffrir ; que le bourreau et le lacet fatal sont depuis longtemps oubliés ; qu’il existe des châtiments établis par les lois, et qu’on peut infliger des peines qui n’attestent pas la cruauté des juges et la honte du siècle. Oui, relégué dans une île et dépouillé de ses biens, plus Antistius y traînera longtemps sa coupable existence, plus il sentira cruellement ses misères privées, sans cesser d’être un grand exemple de la clémence publique. »\par
\labelchar{XLIX.} La liberté de Thraséas arracha les autres à leur asservissement, et, le consul ayant autorisé le partage, tous passèrent du côté de ce grand homme, excepté quelques flatteurs, entre lesquels A. Vitellius \footnote{Celui même qui fut empereur.} se distingua par l’empressement de sa bassesse, attaquant de ses invectives les plus gens de bien, et, comme font les lâches ; restant muet à la première réponse. Toutefois les consuls, n’osant rédiger le décret du sénat, écrivirent au prince le vœu de cet ordre. Néron balança d’abord entre la honte et la colère : enfin il répondit « que, sans être provoqué par aucune injure, Antistius s’était permis contre le prince les paroles les plus outrageantes ; que vengeance en avait été demandée au sénat ; qu’il eût été juste de proportionner la peine à la grandeur du crime ; mais que, résolu par avance d’arrêter l’effet de la sévérité, il ne s’opposerait pas à la clémence ; qu’ils prononçassent ce qu’ils voudraient ; qu’au nombre de leurs pouvoirs était même celui d’absoudre. » Cette lettre, où chaque mot décelait une âme offensée, fut lue sans que les consuls changeassent rien à la délibération, ou que Thraséas renonçât à son avis, ou que les autres désavouassent ce qu’ils avaient approuvé. Les uns craignaient qu’on ne leur prêtât l’intention de rendre le prince odieux ; la plupart se confiaient en leur nombre ; Thraséas ne consultait que la fermeté de son âme et les intérêts de sa gloire.\par
\labelchar{L.} Une accusation du même genre causa la ruine de Fabricius Véiento. Il avait composé, sous le nom de Codicille, un livre rempli d’invectives contre les sénateurs et les prêtres. L’accusateur, Talius Géminus, lui reprochait encore d’avoir trafiqué des faveurs du prince, et vendu le droit de parvenir aux honneurs ; circonstance qui décida Néron à évoquer à lui cette affaire. Véiento fut convaincu et chassé d’Italie. L’ouvrage, condamné aux flammes, fut recherché et lu avidement, tant qu’il y eut péril à se le procurer ; dès que tout le monde put t’avoir, il tomba dans l’oubli.
\subsection[{Mort de Burrus}]{Mort de Burrus}
\noindent \labelchar{LI.} Cependant l’État perdait ses appuis à mesure que ses maux s’aggravaient. Burrus cessa de vivre ; par la maladie ou par le poison, c’est ce qu’on ne put savoir. Une enflure au dedans de la gorge, qui, s’accroissant peu à peu, lui ôta la vie avec la respiration, semblait annoncer une mort naturelle ; mais on assurait plus généralement qu’une main guidée par Néron lui avait, sous le nom de remède, humecté le palais de sucs meurtriers. Burrus, ajoute-t-on, s’aperçut de ce crime ; et, Néron étant venu le visiter, il détourna les yeux, et, pour toute réponse à ses questions, lui dit qu’il se trouvait bien. Cette grande perte excita des regrets, que nourrirent longtemps le souvenir des vertus de Burrus et le choix de ses successeurs, l’un d’une probité molle et nonchalante, l’autre ardent pour le crime et tout souillé d’adultères ; car le prince avait donné deux chefs aux cohortes prétoriennes, Fénius Rufus, désigné par la faveur populaire à cause de son désintéressement dans l’administration des vivres, et Sophonius Tigellinus, qui avait pour titres l’impureté de ses mœurs et une longue infamie. Leur destinée répondit à leur caractère : Tigellin fut tout-puissant sur l’esprit de Néron, et confident de ses débauches les plus secrètes ; Fénius, estimé du peuple et des soldats, en eut moins de droits aux bonnes grâces du maître.
\subsection[{Attaques contre Sénèque}]{Attaques contre Sénèque}
\noindent \labelchar{LII.} La mort de Burrus brisa la puissance de Sénèque : le parti de la vertu était affaibli d’un de ses chefs, et Néron d’ailleurs penchait pour les méchants. Ceux-ci commencent l’attaque par mille imputations diverses. Selon eux, « Sénèque, dont les immenses richesses excédaient la mesure d’une condition privée, travaillait à s’enrichir encore ; il recherchait une ambitieuse popularité ; bientôt il surpasserait l’empereur par l’agrément de ses jardins et la magnificence de ses maisons de campagne. » Ils lui reprochaient encore de s’arroger à lui seul la gloire de l’éloquence, de faire des vers plus fréquemment, depuis que Néron avait pris le goût de la poésie. « Censeur injuste et public des amusements du prince, il lui refuse le mérite de bien conduire un char ; il rit de ses accents, toutes les fois qu’il chante. Quand donc tout ce qui se fait de glorieux dans l’État cessera-t-il de paraître inspiré par cet homme ? Certes, l’enfance de Néron est finie, et l’âge de la force est venu pour lui. Qu’il s’affranchisse d’une odieuse discipline : n’a-t-il pas d’autres maîtres, et d’assez grands, ses aïeux ? "\par
\labelchar{LIII.} Sénèque, averti, par quelques hommes encore sensibles à l’honneur, des crimes qu’on lui prêtait, voyant d’ailleurs le prince repousser de plus en plus son intimité, demande un entretien, et, l’ayant obtenu, il parle ainsi : « Il y a quatorze ans, César, que je fus placé auprès du berceau de ta future grandeur ; il y en a huit que tu règnes. Pendant ce temps, tu as accumulé sur moi tant d’honneurs et de richesses, qu’il ne manque rien à ma félicité que d’avoir des bornes. Je citerai de grands exemples, et je les prendrai non dans mon rang, mais dans le tien. Ton trisaïeul Auguste permit que M. Agrippa se retirât à Mitylène, et que Mécène, sans quitter Rome, s’y reposât comme dans une lointaine retraite. L’un, compagnon de ses guerres, l’autre, éprouvé à Rome par des travaux de toute espèce, avaient reçu des récompenses, magnifiques saris doute, mais achetées par d’immenses services. Moi, quels titres ai-je apportés à ta munificence, si ce n’est des études nourries, pour ainsi dire, dans l’ombre, et qui empruntent tout leur éclat de ce que je parais avoir dirigé les essais de ta jeunesse, prix déjà si haut de si faibles talents ? Mais toi, César, tu mas environné d’un crédit sans bornes, de richesses infinies, au point que souvent je me dis à moi-même : Qui ? moi, né simple chevalier, au fond d’une province \footnote{Sénèque était né à Cordoue, en Espagne, d’une famille de chevaliers.}, je suis compté parmi les premiers de l’État ! ma nouveauté s’est fait jour entre tant de noms décorés d’une longue illustration ! Où est cette philosophie si bornée dans ses désirs ? est-ce elle qui embellit ces jardins, qui promène son faste dans ces maisons de plaisance, qui possède ces vastes domaines, ces inépuisables revenus ? Une seule excuse se présente : je n’ai pas dû repousser tes bienfaits.\par
\labelchar{LIV.} Mais nous avons tous deux comblé la mesure, toi de ce qu’un prince peut donner à son ami, moi de ce qu’un ami peut recevoir de son prince. Plus de bontés irriteraient l’envie : je sais qu’elle rampe, avec le reste des choses humaines, bien au-dessous de ta grandeur ; mais elle pèse sur moi ; c’est moi qu’il faut soulager. Soldat épuisé de travaux ou voyageur fatigué de la route, je demanderais un appui : de même, en ce chemin de la vie, où, vieux et succombant aux moindres soins, je ne puis porter plus loin le fardeau de mes richesses, j’implore une main secourable. Ordonne qu’elles soient régies par tes intendants, reçues dans ton domaine impérial. Je ne me réduirai pas ainsi à la pauvreté ; je déposerai des biens dont l’éclat m’importune, et tout le temps que me ravit le soin de ces jardins et de ces terres, je le rendrai à mon esprit. La force surabonde en toi, et de longues années ont assuré dans tes mains le gouvernail de l’empire. Nous, tes vieux amis, nous pouvons maintenant acquitter notre dette par le repos. Cela même doit tourner à ta gloire, que tu aies élevé aux grandeurs des hommes capables de soutenir la médiocrité. »\par
\labelchar{LV.} Néron lui répondit à peu près ainsi : « Si je réplique sur-le-champ à un discours préparé, c’est un premier avantage que je te dois, puisque tu m’as appris à parler également, que le sujet fût prévu ou qu’il ne le fût pas. Mon trisaïeul Auguste, après les grands travaux d’Agrippa et de Mécène, leur permit le repos ; mais il était d’un âge dont l’autorité mettait cette faveur, quelle qu’elle fût, à l’abri de la censure ; et cependant il ne dépouilla ni l’un ni l’autre des récompenses qu’ils avaient reçues de lui. Ils les avaient méritées par la guerre et les périls ! c’est que la jeunesse d’Auguste se passa dans les périls et la guerre. Certes, ton bras et ton épée ne m’auraient pas manqué non plus, si j’avais eu les armes à la main. Mais tu as fait ce que les temps demandaient : tes lumières, tes conseils, tes préceptes, ont formé mon enfance, cultivé ma jeunesse. Et les biens dont tu m’as enrichi dureront impérissables, tant que durera ma vie : ceux que tu tiens de moi, jardins, trésors, maisons de campagne, sont sujets aux caprices du sort ; et, tout grands qu’ils paraissent, combien d’hommes, fort au-dessous de ton mérite, en ont possédé davantage ! J’ai honte de citer des affranchis qui étalent une tout autre opulence. Je rougis même que, le premier dans mon cœur, tu ne sois pas encore au-dessus de tous par la fortune.\par
\labelchar{LVI.} « Mais ton âge plein de vigueur suffit toujours et aux travaux, et aux jouissances que les travaux procurent ; et moi je fais mes premiers pas dans la carrière du gouvernement. Sans doute tu ne te mets pas au-dessous de Vitellius, qui fut trois fois consul, ni moi au-dessous de Claude ; et cependant Volusius a plus acquis de biens par de longues épargnes, que ne peut t’en donner ma générosité. Tu sais combien la pente de la jeunesse est glissante ; si elle m’entraîne, sois près de moi pour me retenir. Soutiens cette raison que as ornée ; gouverne ma force avec plus de soin que jamais. Ce n’est pas ta modération, si tu renonces à tes biens, ni ton amour du repos, si tu quittes le prince, c’est mon avarice, c’est la crainte supposée de ma cruauté, qui seront dans toutes les bouches : mais dût la voix publique célébrer ton désintéressement, jamais il ne sera digne d’un sage de sacrifier à sa gloire la réputation d’un ami. » A ces paroles, il ajoute des embrassements et des baisers ; formé par la nature, exercé par l’habitude à voiler sa haine sous d’insidieuses caresses. Sénèque lui rendit grâce, conclusion ordinaire des entretiens avec les puissances ; mais il changea les habitudes d’une faveur qui n’était plus. Il écarte cette foule qui s’empressait à le visiter ; il évite qu’on lui fasse cortège, se montre peu dans la ville, alléguant tour à tour qu’une santé faible ou des études philosophiques le retenaient chez lui.
\subsection[{Intrigues de Tigellin}]{Intrigues de Tigellin}
\noindent \labelchar{LVII.} Sénèque abattu, il ne fut pas difficile d’ébranler Fénius : son crime était l’amitié d’Agrippine. Tigellin devenait donc plus fort de jour en jour. Persuadé que ses vices, unique fondement de son crédit, seraient mieux reçus du prince, si une société de crimes resserrait leur union, il épie ses défiances ; et, s’étant assuré qu’il ne craignait personne autant que Plautus et Sylla, relégués depuis peu, Plautus en Asie, et Sylla dans la Gaule narbonnaise, il lui parle de leur naissance, de leur séjour auprès des armées, l’un d’Orient, l’autre de Germanie. A l’en croire, « il ne nourrit pas, lui, comme faisait Burrus, de doubles espérances ; la sûreté de Néron est tout ce qui l’occupe. Le prince a contre les complots du dedans une sauvegarde telle quelle, sa présence ; mais les gouvernements lointains, quel moyen de les réprimer ? Le nom de Sylla, ce nom dictatorial, tient la Gaule attentive ; et un petit-fils de Drusus porte au milieu de l’Asie une illustration qui ne la rend pas moins suspecte. Sylla est pauvre, ce qui accroît son audace ; il feint l’indolence, en attendant l’heure de tenter les hasards. Plautus, maître d’une grande fortune, n’affecte pas même le désir du repos. Il se pare d’une ambitieuse imitation des vieux Romains. II prend jusqu’à l’arrogance des stoïciens, et l’esprit d’une secte qui fait des intrigants et des séditieux. » On ne perdit pas un moment : en six jours, des meurtriers sont rendus à Marseille, et avant le premier soupçon, le moindre bruit du danger, Sylla est tué en se mettant à table. Sa tête, rapportée à Néron, excita ses railleries ; il la trouva blanchie avant le temps.
\subsection[{Mort de Plautus}]{Mort de Plautus}
\noindent \labelchar{LVIII.} La mort de Plautus ne put être préparée avec, le même secret : plus de personnes s’intéressaient à sa conservation ; et la distance des lieux, les délais d’un long voyage de terre et de mer, donnèrent l’éveil à la renommée. On supposa qu’il s’était rendu auprès de Corbulon, qui avait alors de grandes armées sous son commandement, et qui était un des premiers menacés, si la gloire et l’innocence étaient des arrêts de mort. On ajoutait que l’Asie avait pris les armes en faveur du jeune homme, que le nombre ou la résolution avait manqué aux soldats envoyés pour consommer le crime, et que, n’ayant pu exécuter leurs ordres, ils avaient embrassé les intérêts nouveaux. Ces fictions, comme tous les bruits publics, étaient grossies par une oisive crédulité. Au reste, un affranchi de Plautus, favorisé par les vents, prévint les meurtriers, et lui apporta les paroles d’Antistius, son beau-père : il lui disait « de ne pas abandonner lâchement sa vie ; qu’un secours lui restait, la haine du tyran et l’intérêt qui s’attache à un grand nom ; que les gens de bien viendraient à lui ; qu’il appellerait les audacieux ; qu’en attendant aucune ressource n’était à dédaigner ; que, s’il repoussait soixante soldats (c’est le nombre qui était en route), il faudrait du temps pour que la nouvelle en fût portée à Néron, pour qu’une autre troupe passât la mer, et que sa résistance, aidée par mille événements, pouvait devenir une guerre ; qu’enfin, ou cette résolution le sauverait, ou le courage ne lui attirerait pas un danger de plus que la faiblesse. "\par
\labelchar{LIX.} Ces raisons ne décidèrent point Plautus : soit qu’il ne comptât sur aucun secours, exilé et sans armes ; soit ennui de vivre entre l’espoir et la crainte ; soit amour de sa femme et de ses enfants, qui trouveraient peut-être Néron plus exorable, lorsqu’aucune alarme n’aurait troublé son repos. Quelques-uns rapportent qu’un autre message de son beau-père lui annonça qu’il n’avait rien à craindre ; et que deux philosophes, le Grec Céranus, et Musonius, Toscan d’origine, lui conseillèrent d’attendre courageusement la mort, plutôt que de mener une vie précaire et agitée. Il est certain qu’on le trouva, au milieu du jour, nu et se livrant à des exercices du corps. C’est en cet état que le centurion le tua, en présence de l’eunuque Pélagon, qui, par l’ordre du prince, commandait au chef et aux soldats, comme le ministre d’un roi à des satellites. La tête de Plautus fut apportée à Rome. « Eh ! bien, dit le prince en la voyant (et je cite ses propres paroles), Néron doit être rassuré : que ne hâte-t-il les noces de Poppée, retardées jusqu’ici par toutes ces terreurs, et le renvoi d’Octavie, cette épouse sage, il est vrai, mais que le nom de son père et l’attachement du peuple lui rendent insupportable ? " Il envoya une lettre au sénat, où, sans rien avouer du meurtre de Sylla et de Plautus, il les peignait comme des esprits turbulents, ajoutant qu’il veillait avec soin au salut de la république. On décréta que des actions de grâces auraient lieu dans les temples, et que Sylla et Plautus seraient chassés du sénat ; dérision toutefois plus insultante que funeste.
\subsection[{Néron répudie Octavie et épouse Poppée}]{Néron répudie Octavie et épouse Poppée}
\noindent \labelchar{LX.} Néron n’eut pas plus tôt reçu le décret du sénat, que, voyant tous ses crimes érigés en vertus, il chasse Octavie sous prétexte de stérilité ; ensuite il s’unit à Poppée. Cette femme, longtemps sa concubine, et toute-puissante sur l’esprit d’un amant devenu son époux, suborne un des gens d’Octavie, afin qu’il l’accuse d’aimer un esclave : on choisit, pour en faire le coupable, un joueur de flûte, natif d’Alexandrie, nommé Eucérus. Les femmes d’Octavie furent mises à la question, et quelques-unes, vaincues par les tourments, avouèrent un fait qui n’était pas ; mais la plupart soutinrent constamment l’innocence de leur maîtresse. Une d’elles, pressée par Tigellin, lui répondit qu’il n’y avait rien sur le corps d’Octavie qui ne fût plus chaste que sa bouche. Octavie est éloignée cependant, comme par un simple divorce, et reçoit, don sinistre, la maison de Burrus et les terres de Plautus. Bientôt elle est reléguée en Campanie, où des soldats furent chargés de sa garde. De là beaucoup de murmures ; et, parmi le peuple, dont la politique est moins fine, et l’humble fortune sujette à moins de périls, ces murmures n’étaient pas secrets. Néron s’en émut ; et, par crainte bien plus que par repentir, il rappela son épouse Octavie.
\subsection[{Poppée attaquée se défend}]{Poppée attaquée se défend}
\noindent \labelchar{LXI.} Alors, ivre de joie, la multitude monte au Capitole et adore enfin la justice des dieux ; elle renverse les statues de Poppée ; elle porte sur ses épaules les images d’Octavie, les couvre de fleurs, les place dans le Forum et dans les temples. Elle célèbre même les louanges du prince et demande qu’il s’offre aux hommages publics. Déjà elle remplissait jusqu’au palais de son affluence et de ses clameurs, lorsque des pelotons de soldats sortent avec des fouets ou la pointe du fer en avant, et la chassent en désordre. On rétablit ce que la sédition avait déplacé, et les honneurs de Poppée sont remis dans tout leur éclat. Cette femme, dont la haine, toujours acharnée, était encore aigrie par la peur de voir ou la violence du peuple éclater plus terrible, ou Néron, cédant au vœu populaire, changer de sentiments, se jette à ses genoux, et s’écrie « qu’elle n’en est plus à défendre son hymen, qui pourtant lui est plus cher que la vie ; mais que sa vie même est menacée par les clients et les esclaves d’Octavie, dont la troupe séditieuse, usurpant le nom de peuple, a osé en pleine paix ce qui se ferait à peine dans la guerre ; que c’est contre le prince qu’on a pris les armes ; qu’un chef seul a manqué, et que, la révolution commencée, ce chef se trouvera bientôt : qu’elle quitte seulement la Campanie et vienne droit à Rome, celle qui, absente, excite à son gré les soulèvements ! Mais Poppée elle-même, quel est donc son crime ? qui a-t-elle offensé ? Est-ce parce qu’elle donnerait aux Césars des héritiers de leur sang, que le peuple romain veut voir plutôt les rejetons d’un musicien d’Égypte assis sur le trône impérial ? Ah ! que le prince, si la raison d’État le commande, appelle de gré plutôt que de force une dominatrice, ou qu’il assure son repos par une juste vengeance ! Des remèdes doux ont calmé les premiers mouvements ; mais, si les factieux désespèrent qu’Octavie soit la femme de Néron, ils sauront bien lui donner un époux. »
\subsection[{Anicet prétend qu’il est l’amant d’Octavie}]{Anicet prétend qu’il est l’amant d’Octavie}
\noindent \labelchar{LXII.} Ce langage artificieux, et calculé pour produire la terreur et la colère, effraya tout à la fois et enflamma le prince. Mais un esclave était mal choisi pour asseoir les soupçons, et d’ailleurs l’interrogatoire des femmes les avait détruits. On résolut donc de chercher l’aveu d’un homme auquel on pût attribuer aussi le projet d’un changement dans l’État. On trouva propre à ce dessein celui par qui Néron avait tué sa mère, Anicet, qui commandait, comme je l’ai dit, la flotte de Misène. Peu de faveur, puis beaucoup de haine, avait suivi son crime ; c’est le sort de qui prote son bras aux forfaits d’autrui : sa vue est un muet reproche. Néron fait venir Anicet et lui rappelle son premier service : « lui seul avait sauvé la vie du prince des complots de sa mère ; le moment était venu de mériter une reconnaissance non moins grande, en le délivrant d’une épouse ennemie. Ni sa main ni son épée n’avaient rien à faire ; qu’il s’avouât seulement l’amant d’Octavie. » Il lui promet des récompenses, secrètes d’abord, mais abondantes, des retraites délicieuses, ou, s’il nie, la mort. Cet homme, pervers par nature, et à qui ses premiers crimes rendaient les autres faciles, ment au delà de ce qu’on exigeait, et se reconnaît coupable devant plusieurs favoris, dont le prince avait formé une sorte de conseil. Relégué en Sardaigne, il y soutint, sans éprouver l’indigence, un exil que termina sa mort.
\subsection[{Exil d’Octavie}]{Exil d’Octavie}
\noindent \labelchar{LXIII.} Cependant Néron annonce par un édit, que, dans l’espoir de s’assurer de la flotte, Octavie en a séduit le commandant ; et, sans penser à la stérilité dont il l’accusait naguère, il ajoute que, honteuse de ses désordres, elle en a fait périr le fruit dans son sein. Il a, dit-il, acquis la preuve de ces crimes ; et il confine Octavie dans l’île de Pandataria. Jamais exilée ne tira plus de larmes des yeux témoins de son infortune. Quelques-uns se rappelaient encore Agrippine, bannie par Tibère ; la mémoire plus récente de Julie, chassée par Claude, remplissait toutes les âmes. Toutefois, l’une et l’autre avaient atteint la force de l’âge ; elles avaient vu quelques beaux jours, et le souvenir d’un passé plus heureux adoucissait les rigueurs de leur fortune présente. Mais Octavie, le jour de ses noces fut pour elle un jour funèbre : elle entrait dans une maison où elle ne devait trouver que sujets de deuil, un père, puis un frère, empoisonnés coup sur coup, une esclave plus puissante que sa maîtresse, Poppée ne remplaçant une épouse que pour la perdre, enfin une accusation plus affreuse que le trépas.
\subsection[{Mort d’Octavie}]{Mort d’Octavie}
\noindent \labelchar{LXIV.} Ainsi une faible femme, dans la vingtième année de son âge, entourée de centurions et de soldats, et déjà retranchée de la vie par le pressentiment de ses maux, ne se reposait pourtant pas encore dans la paix de la mort. Quelques jours s’écoulèrent, et elle reçut l’ordre de mourir. En vain elle s’écrie qu’elle n’est plus qu’une veuve, que la sœur du prince \footnote{Octavie était fille de Claude par la nature, Néron fils par l’adoption. Répudiée comme épouse, elle n’était donc plus que la sœur du prince} ; en vain elle atteste les Germanicus, leurs communs aïeul \footnote{Claude, père d’Octavie, et Drusus, père de Claude, portaient tous deux le surnom de Germanicus. D’un autre côté, Néron était petit-fils, par sa mère Agrippine, du grand Germanicus, qui lui-même était frère de Claude et fils de Drusus. Le premier qui prit le titre de Germanique était donc aïeul d’Octavie et bisaïeul de Néron.}, et jusqu’au nom d’Agrippine, du vivant de laquelle, épouse malheureuse, elle avait du moins échappé au trépas : on la lie étroitement, et on lui ouvre les veines des bras et des jambes. Comme le sang, glacé par la frayeur, coulait trop lentement, on la mit dans un bain très-chaud, dont la vapeur l’étouffa ; et, par une cruauté plus atroce encore, sa tête ayant été coupée et apportée à Rome, Poppée en soutint la vue. Des offrandes pour les temples furent décrétées à cette occasion ; et je le remarque, afin que ceux qui connaîtront, par mes récits ou par d’autres, l’histoire de ces temps déplorables, sachent d’avance que, autant le prince ordonna d’exils ou d’assassinats, autant de fois on rendit grâce aux dieux, et que ce qui annonçait jadis nos succès signalait alors les malheurs publics. Je ne tairai pas cependant les sénatus-consultes que distinguerait quelque adulation neuve, ou une servilité poussée au dernier terme.
\subsection[{Assassinat de deux affranchis}]{Assassinat de deux affranchis}
\noindent \labelchar{LXV.} La même année Néron, à ce que l’on crut, tua par le poison ses principaux affranchis, Doryphore, pour s’être opposé à l’hymen de Poppée, Pallas, parce que sa longue vieillesse retenait sans fin des trésors immenses. Romanus avait secrètement accusé Sénèque de liaisons suspectes avec Pilon ; mais Sénèque rejeta victorieusement l’accusation sur son auteur. Pison conçut des craintes ; et de là naquit plus tard une conspiration redoutable, mais malheureuse, contre le prince.
\section[{Livre quinzième (63, 65)}]{Livre quinzième (63, 65)}\renewcommand{\leftmark}{Livre quinzième (63, 65)}

\subsection[{A l’extérieur – Les Parthes}]{A l’extérieur – Les Parthes}
\noindent \textbf{I.} Cependant le roi des Parthes, Vologèse, instruit des succès de Corbulon, et voyant l’étranger Tigrane placé sur le trône d’Arménie, voulait d’abord venger la gloire des Arsacides, outragée par l’expulsion de son frère Tiridate. Puis la pensée de la grandeur romaine et le respect d’une ancienne et constante alliance arrêtaient son esprit combattu. Naturellement temporiseur, il était encore retenu par la révolte des Hyrcaniens, nation puissante, et par les guerres sans nombre où cette défection l’avait engagé. Pendant qu’il flottait indécis, l’annonce d’une insulte nouvelle vint aiguillonner sa lenteur. Tigrane, sorti de l’Arménie, avait désolé l’Adiabénie, contrée limitrophe, par des ravages trop longs et trop étendus pour n’être qu’un simple brigandage. Les grands de ces nations s’en plaignaient avec amertume : « A quel abaissement étaient-ils donc descendus, pour se voir envahis, non pas même par un général romain, mais par l’audace téméraire d’un otage, confondu tant d’années parmi de vils esclaves ? » Monobaze, qui gouvernait l’Adiabénie, aigrissait leurs ressentiments en demandant à quels secours il devait implorer et à qui s’adresser. Déjà on avait cédé l’Arménie, et le reste suivrait, si les Parthes n’en prenaient la défense. Se soumettre aux Romains valait mieux que d’être conquis : on y gagnait un esclavage plus doux. » Tiridate, détrôné et fugitif, faisait, par le silence ou des plaintes mesurées, une impression plus forte encore : « Non, ce n’était point par la lâcheté que les grands empires se soutenaient ; il fallait des hommes, des armes, des combats. Entre puissances, l’équité, c’est la force. Conserver ce qui est ; à soi, suffit à un particulier ; combattre pour ce qui est à d’autres, c’est la gloire d’un roi. »\par
\textbf{II.} Entraîné par tous ces motifs, Vologèse assemble son conseil, place Tiridate auprès de lui, et parle en ces termes : « Ce prince, né du même père que moi, m’ayant, à cause de mon âge, cédé la couronne la plus noble, je l’ai mis en possession de l’Arménie, troisième trône de notre famille ; car Pacorus occupait déjà celui des Mèdes. Je croyais avoir ainsi préservé notre maison des haines et des rivalités qui de tout temps régnèrent entre frères. Les Romains s’y opposent ; et la paix, qu’ils ne troublèrent jamais impunément, ils la rompent encore aujourd’hui pour leur perte. Je l’avouerai, c’est par l’équité plutôt que par le sang, par les négociations plutôt que par les armes, que j’ai voulu, d’abord conserver les conquêtes de mes ancêtres. Si ce délai fut une faute, mon courage la réparera. Votre force, du moins, et votre gloire sont entières ; et vous avez de plus l’honneur de la modération, que les mortels les plus grands ne doivent pas dédaigner, et qui a son prix chez les dieux. » Ensuite il ceint du diadème le front de Tiridate, donne à Monèse, un des nobles, sa garde à cheval avec les auxiliaires de l’Adiabénie, et commande qu’on chasse Tigrane de l’Arménie : lui-même, après s’être réconcilié avec les Hyrcaniens, lève au cœur de ses États une armée formidable, et menace les provinces romaines.\par
\textbf{III.} Corbulon, instruit de ces faits par des rapports certains, envoie au secours de Tigrane Vérulanus Sévérus et Vettius Bolanus à la tête de deux légions, avec l’ordre secret de mettre dans leurs mouvements plus de précaution que de rapidité ; car il aimait mieux avoir la guerre que de la faire. Il avait même écrit à l’empereur qu’il fallait un chef particulier pour défendre l’Arménie ; que la Syrie menacée par Vologèse, était dans un danger plus pressant. En attendant, il place le reste de ses légions sur la rive de l’Euphrate, arme un corps levé à la hâte dans la province, ferme avec des troupes les passages par où l’ennemi pouvait pénétrer, et, comme le pays est presque sans eau, il s’assure des sources en y élevant des forts ; il ensevelit aussi quelques ruisseaux sous des amas de sable.\par
\textbf{IV.} Pendant que Corbulon mettait ainsi la Syrie à rouvert, Monèse voulut, par une marche rapide, devancer jusqu’au bruit de son approche, et n’en trouva pas moins Tigrane prévenu et sur ses gardes. Ce prince s’était jeté dans Tigranocerte, ville également forte par ses défenseurs et par la hauteur de ses murailles. En outre le fleuve Nicéphore \footnote{Selon d’Anville, c’est le Khabour, et il passe prés d’une ville nommée Séred, qui, dit ce géographe, pourrait tenir la place de l’ancienne Tigranocerte.}, d’un assez large cours, environne une partie des remparts, et un vaste fossé défend ce que le fleuve eût trop peu garanti. Des soldats romains étaient dans la place, et on l’avait munie d’approvisionnements. Quelques-uns des hommes chargés de ce soin s’étant emportés trop avant, l’ennemi les avait subitement enveloppés, et cette perte avait inspiré aux autres plus de colère que de crainte. D’ailleurs le Parthe réussit mal dans les sièges, faute d’audace pour attaquer de près : il lance au hasard quelques flèches, qui trompent ses efforts et n’effrayent point un ennemi retranché. Les Adiabéniens, ayant approché des échelles et des machines, furent aisément renversés, et les nôtres, dans une brusque sortie, les taillèrent en pièces.\par
\textbf{V.} Corbulon, persuadé, malgré ces heureux succès, qu’il fallait user modérément de la fortune, députa vers Vologèse pour se plaindre qu’on eût attaqué sa province, qu’on tînt assiégés un roi allié et ami et des cohortes romaines. Il l’avertissait de lever le siège, ou lui-même irait camper sur les terres ennemies. Le centurion Caspérius chargé de cette mission trouva le roi dans Nisibe \footnote{Ville forte de l’ancienne Mygdonie, partie de la Mésopotamie : il n’en reste que de faibles traces dans le bourg de Nesbin.}, à trente-sept milles de Tigranocerte, et lui exposa fièrement ses ordres. Vologèse avait depuis longtemps pour maxime invariable d’éviter les armes romaines. D’un autre côté, ses affaires prenaient un cours malheureux : le siège était sans résultat ; Tigrane ne manquait ni de soldats ni de vivres ; un assaut venait d’être repoussé ; des légions étaient entrées en Arménie, et d’autres, sur les frontières de Syrie, n’attendaient que le signal d’envahir ses États : lui, cependant, n’avait qu’une cavalerie épuisée par le manque de fourrages ; car une multitude de sauterelles avait dévoré tout ce qu’il y avait dans le pays d’herbes et de feuilles. Il renferme donc ses craintes, et, prenant un langage modéré, il répond qu’il va envoyer une ambassade à l’empereur des Romains pour lui demander l’Arménie et affermir la paix. Il ordonne à Monèse d’abandonner Tigranocerte, et lui-même se retire.\par
\textbf{VI.} La plupart, attribuant cette retraite aux craintes du roi et aux menaces de Corbulon, en parlaient avec enthousiasme. D’autres supposaient un accord secret par lequel, la guerre cessant des deux côtés, et Vologèse retirant ses troupes, Tigrane quitterait aussi l’Arménie. « Car pourquoi avoir rappelé l’armée romaine de Tigranocerte ? Pourquoi abandonner dans la paix ce qu’on avait défendu par la guerre ? Avait-on plus commodément passé l’hiver au fond de la Cappadoce, sous des huttes construites à la hâte, que dans la capitale d’un royaume qu’on venait de sauver ? Non, ce n’était qu’une trêve consentie par Vologèse pour avoir en tète un autre ennemi que Corbulon, par Corbulon pour ne plus exposer une gloire, ouvrage de tant d’années » J’ai dit en effet que ce général avait demandé pour l’Arménie un chef particulier, et l’on parlait de l’arrivée prochaine de Césennius Pétus. Il parut bientôt, et tes troupes furent ainsi divisées : la quatrième et la douzième légion, avec la cinquième, appelée récemment de Mésie, ainsi que les auxiliaires du Pont, de la Galatie et de la Cappadoce, obéirent à Pétun. La troisième, la sixième, la dixième et les anciens soldats de Syrie restèrent à Corbulon. Du reste, ils devaient, suivant les circonstances, unir ou partager leurs forces. Mais Corbulon ne souffrait pas de rival ; et Pétus, à qui l’honneur du second rang aurait dû suffire, rabaissait les exploits de ce chef. Il ne cessait de dire « qu’il n’avait ni tué d’ennemis ni enlevé de butin ; que les villes qu’il avait forcées se réduisaient à de vains noms ; qu’il saurait, lui, imposer aux vaincus des lois, des tributs, et, au lieu d’un fantôme de roi, la domination romaine. »\par
\textbf{VII.} Vers le même temps, les ambassadeurs que Vologèse avait, comme je l’ai dit, envoyés vers le prince, revinrent sans avoir rien obtenu, et les Parthes commencèrent ouvertement la guerre. Pétus ne refusa pas le défi ; il prend avec lui deux légions, la quatrième, commandée alors par Funisulanus Vettonianus, la douzième, par Calavius Sabinus, et entre en Arménie sous de sinistres auspices. Au passage de l’Euphrate, qu’il traversait sur un pont, le cheval qui portait les ornements consulaires prit l’effroi sans cause apparente, et s’échappa en retournant sur ses pas. Pendant qu’on fortifiait un camp, une victime, debout près des travaux, rompit les palissades à moitié terminées et se sauva hors des retranchements. Enfin les javelots des soldats jetèrent des flammes, prodige d’autant plus frappant que c’est avec des armes de trait que combattent les Parthes.\par
\textbf{VIII.} Pétus méprisa ces présages, et, sans avoir achevé ses fortifications, sans avoir pourvu aux subsistances, il entraîna l’armée au delà du mont Taurus, afin, disait-il, de reprendre Tigranocerte et de ravager des pays que Corbulon avait laissés intacts. Il prit en effet plusieurs forts, et il eût remporté quelque gloire et quelque butin, s’il eût su chercher l’une avec mesure et prendre soin de l’autre. Après avoir parcouru de vastes espaces qu’on ne pouvait garder, et détruit les provisions qu’on avait enlevées, pressé par l’approche de l’hiver, il ramena ses troupes, et adressa au prince une lettre où, supposant la guerre terminée, il cachait le vide des choses sous la magnificence des paroles.\par
\textbf{IX.} Pendant ce temps, Corbulon, qui n’avait pas un moment négligé la rive de l’Euphrate, la garnit de postes plus rapprochés que jamais ; et, afin que les bandes ennemies, qui déjà voltigeaient avec un appareil redoutable dans la plaine opposée, ne pussent l’empêcher de jeter un pont, il fait avancer sur le fleuve de très-grands bateaux, liés ensemble avec des poutres et surmontés de tours. De là, il repousse les barbares au moyen de balistes et de catapultes, d’où les pierres et les javelines volaient à une distance que ne pouvait égaler la portée de leurs flèches. Le pont est ensuite achevé, et les collines de l’autre rive occupées par les cohortes auxiliaires, ensuite par le camp des légions, avec une telle promptitude et un déploiement de forces si imposant, que les Parthes renoncèrent à envahir la Syrie, et tournèrent vers l’Arménie toutes leurs espérances.\par
\textbf{X.} Pétus, sans prévoir l’orage qui s’approchait de lui, tenait au loin dans le Pont la cinquième légion, et avait affaibli les autres en prodiguant les congés, lorsqu’il apprit que Vologèse accourait avec une armée nombreuse et menaçante. Il appelle la douzième légion, et ce qui devait faire croire ses forces augmentées ne fit que trahir sa faiblesse. On pouvait toutefois conserver le camp, et déconcerter, en temporisant, les desseins des Parthes, si Pétus avait su marcher en ses conseils ou en ceux d’autrui d’un pas plus constant. Mais à peine des hommes habiles dans la guerre l’avaient-ils fortifié contre un péril imminent, que, afin de paraître n’avoir pas besoin de lumières étrangères, il changeait tout pour faire plus mal. C’est ainsi qu’il abandonna ses quartiers, en s’écriant que ce n’était pas un fossé et des retranchements, mais des hommes et du fer qu’on lui avait donnés contre l’ennemi, et fit avancer ses légions comme pour combattre. Ensuite, ayant perdu un centurion et quelques soldats qu’il avait envoyés reconnaître les troupes barbares, il revint avec précipitation. Mais le peu d’ardeur que Vologèse avait mis à le poursuivre lui rendit sa folle confiance, et il plaça trois mille fantassins d’élite sur le sommet le plus voisin du mont Taurus, afin d empêcher le passage du roi. Des Pannoniens qui faisaient la force de sa cavalerie furent confinés dans une partie de la plaine ; enfin il cacha sa femme et son fils dans un château nommé Arsamosate \footnote{Place considérable, dont, selon d’Anville, on retrouve le nom sous la forme de Simsat ou Shimsliat. On croit que cette ville avait été fondée par Arsamés, qui régnait en Arménie vers 245 avant J. C.}, sous la garde d’une cohorte. Il dispersait ainsi son armée, qui, réunie, eût mieux résisté à des bandes vagabondes. On ne le détermina, dit-on, qu’avec peine à faire à Corbulon l’aveu de sa détresse ; et celui-ci ne se pressait pas non plus de le secourir, afin que, le péril devenant plus grave, il y eût plus de gloire à l’en délivrer. Il ordonna cependant que mille hommes de chacune de ses trois légions, huit cents cavaliers, et un pareil nombre de soldats auxiliaires, se tinssent prêts à partir.\par
\textbf{XI.} Vologèse, informé que les passages étaient gardés, ici par la cavalerie, là par l’infanterie de Pétus, n’en suivit pas moins son dessein ; et, joignant la force aux menaces, il effraya les hommes à cheval, écrasa les fantassins. Un seul centurion, Tarquitius Crescens, osa défendre une tour confiée à sa garde ; il fit plusieurs sorties, tailla en pièces ceux des barbares qui approchaient le plus prés, jusqu’à ce que des feux lancés du dehors l’enveloppassent de toutes parts. Ceux qui étaient sans blessures se sauvèrent loin des routes pratiquées ; les blessés regagnèrent le camp, faisant de la valeur du roi, du nombre et de la férocité de ces peuples, mille récits exagérés par la crainte et facilement accueillis par une crainte semblable. Le général lui-même ne luttait plus contre ce cours fâcheux d’événements. Il avait abandonné tous les soins de la guerre, et conjuré Corbulon, par un second message, « de venir au plus tôt, de sauver les étendards, les aigles, le nom presque anéanti d’une armée malheureuse. Eux, en attendant, feraient leur devoir jusqu’au dernier soupir. »\par
\textbf{XII.} Corbulon, sans s’effrayer, laisse une partie de ses troupes en Syrie pour garder les fortifications construites sur l’Euphrate ; et, prenant le chemin qui était le moins long et offrait le plus de ressources, il traverse la Commagène, puis la Cappadoce, et entre en Arménie. Il menait avec l’armée, outre l’attirail ordinaire de guerre, une grande quantité de chameaux chargés de blé, afin de repousser à la fois la famine et l’ennemi. Le premier des fuyards qu’il trouva sur la route fut le primipilaire Pactius, et après lui beaucoup de soldats. Aux prétextes dont ils s’efforçaient de couvrir leur fuite, il répondait en leur conseillant « de retourner aux drapeaux et d’essayer la clémence de Pétus ; que, pour sa part, il fallait vaincre, ou il était sans pitié. » Ensuite il parcourt ses légions, les encourage, les fait souvenir de leurs premiers exploits, leur montre une gloire nouvelle. « Ce n’étaient plus des bourgades ou de petites villes d’Arménie, mais un camp romain, et, dans ce camp, deux légions assiégées, qui allaient être le prix de leurs travaux. Si chaque soldat recevait de la main du général une couronne particulière pour le citoyen qu’il aurait sauvé, combien serait glorieux le jour où il y aurait autant de couronnes civiques à distribuer qu’il y avait eu de citoyens en péril ! " Par ces paroles et d’autres semblables, animés pour la cause commune d’une ardeur que doublait chez quelques-uns le danger particulier d’un parent ou d’un frère, ils hâtaient jour et nuit leur marche non interrompue.\par
\textbf{XIII.} Vologèse n’en pressait que plus vivement les assiégés, insultant tour à tour le camp des légions et le château où l’on gardait ceux que l’âge rend inhabiles à la guerre. Il s’approchait même plus qu’il n’est ordinaire aux Parthes, dans l’espoir que cette témérité attirerait ses ennemis au combat. Mais on avait peine à les arracher de leurs tentes, et ils se bornaient à la défense des retranchements, les uns pour obéir au général, les autres par lâcheté, alléguant qu’ils attendaient Corbulon, et prêts à faire valoir, si l’attaque devenait trop violente, les exemples de Numance et des fourches Caudines. « Et combien moins redoutables étaient les Samnites, peuple d’Italie, et les Carthaginois, quoique rivaux de notre empire ! Oui, cette glorieuse antiquité avait aussi, dans les périls extrêmes, mis le salut avant tout. » Vaincu par le désespoir de son armée, le général écrivit à Vologèse une première lettre qui n’avait rien de suppliant. Il s’y plaignait au contraire que le roi nous fît la guerre pour l’Arménie, « de tout temps possédée par les Romains ou soumise à un prince du choix de l’empereur. » Il ajoutait « que la paix serait utile aux deux partis ; que Vologèse ne devait pas seulement envisager le présent qu’il était venu contre deux légions avec toutes les forces de son royaume, mais qu’il restait aux Romains l’univers pour soutenir leur querelle. »\par
\textbf{XIV.} Vologèse, sans rien discuter, répondit : « qu’il était obligé d’attendre ses frères Pacorus et Tiridate ; que ce lieu même et ce temps étaient désignés pour un conseil où ils prononceraient sur le sort de l’Arménie, et (puisque les justes dieux donnaient ce triomphe au sang d’Arsace) où ils fixeraient de plus le destin des légions romaines. » Pétus députa vers le roi pour lui demander un entretien : celui-ci envoya Vasacès, commandant de sa cavalerie. Alors le général parla des Lucullus, des Pompée, de tous les actes des Césars, soit pour garder, soit pour donner l’Arménie. Vasacès soutenait que, si nous avions l’image de ce pouvoir, les Parthes en avaient la réalité. Après de longs débats, Monobaze d’Adiabénie fut appelé le lendemain comme témoin de leur accord. On convint que le siège du camp serait levé, que tous les soldats sortiraient de l’Arménie, que les forts et les approvisionnements seraient livrés aux Parthes, et que, toutes ces choses accomplies, on donnerait le temps à Vologèse d’envoyer au prince des ambassadeurs.\par
\textbf{XV.} Cependant Pétus jeta un pont sur le fleuve Arsanias \footnote{Fleuve aujourd’hui nommé Arsen, qui traverse la Sophène et se rend dans l’Euphrate, après avoir passé par Arsamosate. (D’Anville.)}, qui coulait près du camp ; il feignit d’en avoir besoin pour son passage ; mais les Parthes avaient imposé ce travail en preuve de leur victoire, car ce fut à eux qu’il servit : les nôtres prirent la route opposée. La renommée ajouta que les légions avaient subi l’infamie du joug, et d’autres ignominies vraisemblables en de tels revers, et dont les Parthes se donnèrent le spectacle simulé ; car ils entrèrent dans le camp avant que l’armée romaine en fût sortie, et à son départ, ils se placèrent des deux côtés de la route, reconnaissant et emmenant des esclaves et des bêtes de somme depuis longtemps entre nos mains. Des habits même furent enlevés, des armes retenues, et le soldat tremblant n’osait s’y opposer, de peur d’être obligé de combattre. Vologèse, pour constater notre défaite, fit amonceler les armes et les corps des hommes tués ; du reste il se refusa à la vue de nos légions en fuite : son orgueil rassasié aspirait aux honneurs de la modération. Il affronta le courant de l’Arsanias monté sur un éléphant, et ceux qui étaient près de lui le traversèrent à cheval, parce que le bruit s’était répandu que le pont romprait sous le faix par la fraude des constructeurs ; mais ceux qui osèrent y passer reconnurent qu’il était solide et ne cachait aucun piège.\par
\textbf{XVI.} Au reste, il demeura constant que les assiégés étaient si bien pourvus de vivres qu’ils mirent le feu à des magasins de blé ; tandis qu’au rapport de Corbulon les Parthes, dénués de ressources, et voyant leurs fourrages épuisés, allaient abandonner le siège, et que lui-même n’était plus qu’à trois jours de marche. Corbulon ajouta que Pétus avait juré au pied des enseignes devant les envoyés de Vologèse, présents comme témoins, qu’aucun Romain n’entrerait en Arménie, jusqu’à ce qu’un message de l’empereur annonçât s’il consentait à la paix. Si ces récits furent arrangés en vue d’aggraver l’infamie, il est d’autres faits d’une évidence incontestable : c’est que Pétus fit quarante milles en un jour, laissant les blessés sur les chemins, et qu’une déroute en face de l’ennemi n’eût pas étalé un spectacle plus affreux que cette fuite précipitée. Corbulon, qui les rencontra au bord de l’Euphrate, ne voulut pas que son armée leur fît voir, dans l’éclat de ses armes et de ses décorations, un contraste humiliant. Tristes et plaignant le sort de leurs malheureux compagnons, les soldats ne purent même retenir leurs larmes : à peine, au milieu des pleurs, pensèrent-ils à donner et à rendre le salut. Ce n’était plus cette rivalité de courage, cette ambition de gloire, passions faites pour les cœurs heureux : la pitié régnait seule, plus vive dans les rangs moins élevés.\par
\textbf{XVII.} Les deux chefs eurent ensemble un court entretien. Corbulon se plaignit d’avoir inutilement fatigué son armée, tandis que la guerre pouvait finir par la fuite des Parthes. Pétus répondit que rien n’était perdu ni pour l’un ni pour l’autre ; qu’ils n’avaient qu’à porter leurs aigles en avant, et à fondre tous deux sur l’Arménie, affaiblie par la retraite de Vologèse. Corbulon répliqua qu’il n’avait pas l’ordre de César ; que le danger seul des légions l’avait tiré de sa province ; que, dans l’incertitude de ce que voulaient faire les Parthes, il allait retourner en Syrie ; qu’encore il lui faudrait implorer la bonne fortune, pour qu’une infanterie épuisée par de si longues marches n’y fût pas devancée par des cavaliers alertes, dont de vastes plaines facilitaient la course. Pétus alla passer l’hiver dans la Cappadoce. Bientôt Vologèse envoya sommer Corbulon de retirer les postes qu’il avait au delà de l’Euphrate, afin que le fleuve séparât comme autrefois les deux empires. Corbulon demandait à son tour que les garnisons des Parthes sortissent de l’Arménie : le roi finit par y consentir. Les ouvrages élevés par Corbulon de l’autre côté de l’Euphrate furent démolis, et les Arméniens restèrent sans maîtres.
\subsection[{À Rome – Trophées pour la défaite des Parthes – Néron fait jeter le blé usagé dans le Tibre}]{À Rome – Trophées pour la défaite des Parthes – Néron fait jeter le blé usagé dans le Tibre}
\noindent \textbf{XVIII.} Cependant, à Rome, on érigeait des trophées pour la défaite des Parthes, et, sur le penchant du mont Capitolin, s’élevaient des arcs de triomphe ordonnés par le sénat quand les chances de la guerre étaient entières, et continués malgré nos revers, pour flatter les yeux en dépit de la conscience publique. Afin de mieux dissimuler ses inquiétudes sur les affaires du dehors, Néron fit plus : une partie des blés destinés au peuple étaient vieux et gâtés ; il les jeta dans le Tibre, comme sûr de l’abondance ; et quoiqu’une tempête eût submergé dans le port même prés de deux cents navires, et qu’un incendie en eût consumé cent autres qui avaient déjà remonté le fleuve, le prix des vivres ne fut point augmenté. Le prince confia ensuite les revenus publics à trois consulaires, L. Pison, Ducennius Géminus et Pompéius Paulinus, en blâmant ses prédécesseurs « d’avoir, par l’énormité de leurs dépenses, excédé la mesure des recettes : lui, au contraire, faisait à la république un présent annuel de soixante millions de sesterces. »
\subsection[{Adoptions simulées}]{Adoptions simulées}
\noindent \textbf{XIX.} Une coutume des plus condamnables s’était établie vers ce temps. A l’approche des comices, ou lorsqu’on était près de tirer au sort les provinces, beaucoup de gens sans enfants se donnaient des fils par de feintes adoptions \footnote{La loi Papia Poppéa, rendue sous Auguste, l’an de Rome 702, qui renouvelait et complétait la loi Julia, portée vingt-cinq ans plus tôt, accordait ou confirmait certains privilèges aux citoyens mariés et qui avaient des enfants. Ainsi, ils étaient préférés pour les magistratures et le gouvernement des provinces, et, entre plusieurs candidats, celui qui avait le plus d’enfants devait l’emporter ; ils pouvaient aspirer aux dignités avant l’âge légal, etc.}, et à peine avaient-ils concouru, à titre de pères, au partage des prétures et des gouvernements, qu’ils émancipaient ceux qu’ils venaient d’adopter. Des plaintes amères furent portées au sénat ; on fit valoir « les droits de la nature, les soins de l’éducation, contre des adoptions frauduleuses, calculées, éphémères. N’était-ce pas assez de privilèges pour les hommes sans enfants, de voir, exempts de soucis et de charges, toutes les routes du crédit et des honneurs ouvertes à leurs désirs ? Fallait-il que les promesses de la loi, si longtemps attendues, fussent enfin éludées, et que le prétendu père d’enfants qu’il possède sans inquiétude et perd sans regret vînt tout à coup balancer les vœux longs et patients d’un père véritable ? » Un sénatus-consulte prononça que les adoptions simulées ne donneraient aucun droit aux fonctions publiques, et n’autoriseraient pas même à recevoir des héritages.
\subsection[{Procès du Crétois Timarchus}]{Procès du Crétois Timarchus}
\noindent \textbf{XX.} Ensuite on instruisit le procès du Crétois Timarchus. Outre ces injustices que la richesse orgueilleuse et puissante fait éprouver aux faibles dans toutes les provinces, on lui reprochait une parole dont l’injurieuse atteinte pénétrait jusqu’au sénat : il avait affecté de dire « qu’il dépendait de lui que les gouverneurs de la Crète reçussent, ou non, des actions de grâces. » Thraséas, faisant tourner cette occasion au profit de la chose publique, vota d’abord l’exil du coupable hors de la province de Crète, ensuite il ajouta : « L’expérience prouve, pères conscrits, que les bonnes lois, les actes faits pour servir d’exemple, sont inspirés aux gens de bien par les vices des méchants. Ainsi doivent naissance à la licence des orateurs la loi Cincia, aux brigues des candidats les lois Juliennes \footnote{Portées par Auguste pour réprimer la brigue.}, aux magistrats avares les plébiscites Calpurniens \footnote{L’an de Rome 605, le tribun L. Calpurnius Piso fit rendre la première loi contre les concussionnaires : elle donnait aux habitants des provinces le droit de poursuivre à Rome la restitution des sommes extorquées par les magistrats, et un tribunal permanent fut établi pour en connaître.} ; car, dans l’ordre des temps, la faute précède la peine, et la réforme vient après l’abus. Prenons aussi, contre cet orgueil nouveau des hommes de province, une résolution digne de la justice et de la gravité romaine, et qui, sans rien diminuer de la protection due aux alliés, nous désabuse de l’erreur qu’un Romain a d’autres juges de sa réputation que ses concitoyens.\par
\textbf{XXI.} « Jadis ce n’était pas seulement un préteur ou un consul qu’on envoyait dans les provinces : des particuliers même allaient quelquefois s’assurer de la soumission de chacun, afin d’en rendre compte, et des nations entières attendaient en tremblant le jugement d’un seul homme. Maintenant nous caressons les étrangers, nous rampons devant eux ; et si, d’un geste, ils disposent ici des remerciements, plus facilement encore ils y dictent les accusations. Accusons donc à leur voix, et laissons aux habitants des provinces ce moyen d’étaler leur puissance. Mais que toute louange fausse et mendiée soit interdite aussi sévèrement que la calomnie, que la cruauté. Souvent on commet plus de fautes en obligeant qu’en offensant ; il est même des vertus dont la haine est le prix ; telles sont une sévérité inflexible, une âme que la faveur ne peut vaincre. Aussi les commencements de nos magistrats sont-ils généralement meilleurs ; la fin dégénère, parce que ce ne sont plus que des candidats qui cherchent des suffrages. Empêchons ce désordre, et les provinces seront gouvernées avec une équité plus égale et plus ferme. Car, si la crainte des poursuites a mis un frein à l’avarice, la prohibition des actions de grâces préviendra les ménagements intéressés. »\par
\textbf{XXII.} Cet avis fut reçu avec un applaudissement universel. Toutefois le sénatus-consulte ne put être rendu, parce que les consuls refusèrent de le mettre en délibération. Bientôt après, sur la proposition du prince, un décret défendit que jamais on parlât dans le conseil des alliés de remerciements à demander au sénat pour les préteurs ou les proconsuls, et que personne vînt en députation pour cet objet. Sous les mêmes consuls, le feu du ciel brûla le Gymnase, et la statue de Néron qui s’y trouvait fut fondue en un bronze informe. Un tremblement de terre renversa en grande partie Pompéi, ville considérable de la Campanie. Enfin la vestale Lélia mourut et fut remplacée par Cornélia, de la branche de Cossus.
\subsection[{Naissance et mort d’un fils de Poppée}]{Naissance et mort d’un fils de Poppée}
\noindent \textbf{XXIII.} Sous le consulat de Memmius Régulus et de Virginius Rufus, Néron reçut, avec les démonstrations d’une joie plus qu’humaine, une fille qui lui naquit de Poppée ; il l’appela Augusta, et donna en même temps ce surnom à la mère. Les couches se firent dans la colonie d’Antium, où lui-même était né. Déjà le sénat avait recommandé aux dieux la grossesse de Poppée et décrété des vœux solennels ; de nouveaux furent ajoutés, et on les accomplit tous. On décerna en outre des prières publiques, un temple à la Fécondité, des combats semblables aux jeux sacrés d’Actium. On ordonna que les images en or des deux Fortunes \footnote{Les Antiates adoraient la Fortune sous deux noms divers, la Fortune équestre et la Fortune prospère.} seraient placées sur le trône de Jupiter Capitolin, et que les jeux du Cirque, établis à Boville en l’honneur de la maison des Jules, seraient également donnés à Antium, au nom des Domitius et des Claudes ; institutions oubliées aussitôt, l’enfant étant mort avant l’âge de quatre mois. Ce furent alors de nouvelles adulations : on vota l’apothéose, le coussin sacré, un temple avec un prêtre. Pour Néron, sa douleur ne fut pas moins démesurée que sa joie. On fit la remarque qu’à la nouvelle de la naissance, le sénat s’étant précipité tout entier à Antium, Thraséas ne fut pas reçu, et qu’il soutint sans s’émouvoir cet affront, avant-coureur d’un prochain arrêt de mort. Bientôt le prince se vanta, dit-on, à Sénèque, de s’être réconcilié avec Thraséas, et Sénèque en félicita le prince : franchise qui augmentait tout ensemble la gloire et les périls de ces deux grands hommes.
\subsection[{Ambassade Parthe}]{Ambassade Parthe}
\noindent \textbf{XXIV.} Au commencement du printemps arrivèrent les ambassadeurs des Parthes, avec des instructions de Vologèse et une lettre conçue dans le même sens. « Il se tairait, disait-il, sur la question tant de fois débattue de la souveraineté de l’Arménie, puisque les dieux, arbitres des nations les plus puissantes, avaient livré aux Parthes, non sans honte pour les Romains, la possession de ce royaume. Dernièrement il avait tenu Tigrane enfermé dans une place ; plus tard, pouvant écraser Pétus et ses légions, il les avait renvoyés sans aucun mal. Déjà sa force s’était assez fait connaître ; il venait de prouver également sa clémence. Tiridate ne refuserait pas d’aller à Rome pour y recevoir le diadème, s’il n’était retenu par les devoirs sacrés du sacerdoce \footnote{Tiridate était mage.}. Il irait auprès des étendards et des images du prince ; et là, en présence des légions, se ferait l’inauguration de sa royauté. »
\subsection[{On recommence la guerre contre les Parthes}]{On recommence la guerre contre les Parthes}
\noindent \textbf{XXV.} Comme cette lettre de Vologèse était en contradiction avec celles de Pétus, qui laissaient croire que rien n’était encore décidé, on interrogea sur l’état de l’Arménie le centurion venu avec les ambassadeurs. Il répondit que tous les Romains l’avaient quittée. Alors on sentit l’ironie des barbares, qui demandaient ce qu’ils avaient pris ; et Néron délibéra, avec les premiers de Rome, sur le choix à faire entre une guerre hasardeuse et une paix déshonorante : on ne balança pas à préférer la guerre ; et, Corbulon connaissant par une longue expérience le soldat et l’ennemi, on lui en remit la conduite, de peur que l’ignorance d’un autre Pétus n’amenât encore des fautes et des regrets. Les ambassadeurs furent donc renvoyés sans avoir rien obtenu, et toutefois avec des présents, afin qu’il restât l’espérance que Tiridate ne demanderait pas en vain, s’il apportait sa prière en personne. L’administration de la Syrie fut confiée à Cincius, les forces militaires à Corbulon. On y ajouta la quinzième légion, qui lui fut amenée de Panonie par Marius Celsus. On écrivit aux tétrarques et aux rois, aux préfets et aux procurateurs, enfin à ceux des préteurs qui gouvernaient les provinces voisines, d’obéir aux ordres de Corbulon, dont le pouvoir, ainsi augmenté, égalait presque celui que Pompée avait reçu du peuple romain pour faire la guerre aux pirates. Pétus, de retour, craignait un traitement sévère : le prince, bornant son châtiment à quelques railleries, lui dit à peu prés « qu’il se hâtait de lui pardonner, de peur qu’un homme aussi prompt à s’alarmer que lui ne tombât malade d’inquiétude. »\par
\textbf{XXVI.} La perte des plus braves soldats et le découragement des autres rendait la quatrième et la douzième légion peu propres au combat. Corbulon les transporta en Syrie, et, de cette province, il conduisit en Arménie la sixième et la troisième, troupes fraîches et aguerries par beaucoup de travaux et de succès ; il y ajouta la cinquième légion, qui, restée dans le Pont, n’avait point eu part au désastre, ainsi que la quinzième, récemment arrivée, des vexillaires choisis d’Illyrie et d’Égypte, ce qu’il avait d’auxiliaires à pied et à cheval enfin les troupes des rois alliés, réunies en un seul corps à Mélitène \footnote{Aujourd’hui Malatié. Méliténe n’était alors qu’un camp romain.}, où il se proposait de passer l’Euphrate. Là, il rassembla son armée après les lustrations d’usage, et, promettant sous les auspices de César de brillantes prospérités, rappelant ses propres exploits, imputant les revers à l’inexpérience de Pétus, il parla aux soldats avec cet ascendant qui, dans un tel guerrier, tenait lieu d’éloquence.\par
\textbf{XXVII.} Ensuite il prend le chemin frayé autrefois par Lucullus, et rouvre les passages que le temps avait fermés. Des ambassadeurs de Tiridate et de Vologèse étant venus pour traiter de la paix, loin de les repousser, il envoie avec eux des centurions qui portaient des paroles conciliantes : « On n’en était pas réduit à la nécessité d’un combat à outrance. Beaucoup d’événements avaient été heureux pour les Romains, quelques-uns pour les Parthes ; c’était une leçon contre l’orgueil. Il convenait aux intérêts de Tiridate de recevoir en présent un royaume qui ne fût pas ravagé ; et Vologèse servirait mieux la nation des Parthes par son alliance avec Rome, que par des hostilités mutuellement funestes. Le général n’ignorait pas leurs discordes intestines, et quels peuples indomptables le roi gouvernait. Son empereur au contraire jouissait partout d’une paix profonde, et n’avait que cette seule guerre. » Aux conseils ajoutant la terreur, il chasse de leurs habitations les grands d’Arménie qui avaient commencé la révolte, et il rase leurs châteaux. Plaines et hauteurs, puissants et faibles, il remplit tout d’une égale consternation.\par
\textbf{XXVIII.} Le nom de Corbulon n’inspirait aux barbares mêmes aucune prévention, encore moins cette haine qu’on ressent pour un ennemi : aussi eurent-ils foi à ses conseils ; et Vologèse, qui ne repoussait pas un accommodement, demanda une trêve pour plusieurs de ses provinces. Tiridate désira une entrevue. Le temps fut fixé à un jour prochain : le lieu fut celui où Pétus avait été naguère assiégé avec ses légions. Les barbares le choisirent à cause du succès qu’il leur rappelait, et Corbulon ne l’évite pas, dans l’idée que le contraste rehausserait sa gloire. Le mauvais renom de Pétus le touchait peu d’ailleurs : il en donna une preuve éclatante en chargeant le fils même de Pétus, tribun des soldats, d’aller avec un détachement et d’ensevelir les restes de la dernière défaite. Au jour convenu, Tibérius Alexander \footnote{Le même qui depuis fut préfet d’Égypte et fit le premier reconnaître Vespasien comme empereur.}, chevalier romain du premier rang, donné à Corbulon pour l’aider dans cette guerre, et Vivianus Annius, gendre de ce général, trop jeune encore pour être sénateur, mais placé, avec les fonctions de lieutenant, à la tète de la cinquième légion, se rendirent dans le camp de Tiridate pour faire honneur à ce prince, et le rassurer, par un tel gage, contre toute crainte d’embûches. Les deux chefs prirent chacun vingt cavaliers. A la vue de Corbulon, le roi descendit le premier de cheval : Corbulon l’imita aussitôt, et l’un et l’autre, s’avançant à pied, se donnèrent la main.\par
\textbf{XXIX.} Alors le Romain loua le jeune prince de ce que, au lieu de se précipiter dans les hasards, il revenait aux conseils de la prudence. Celui-ci parla beaucoup de sa noble origine ; puis, avec plus de modestie, il ajouta « qu’ainsi donc il irait à Rome, et porterait à César un triomphe inconnu jusqu’alors, un Arsacide suppliant, quand les Parthes n’étaient pas vaincus. » On convint que Tiridate déposerait devant l’effigie de l’empereur le bandeau royal, et ne le reprendrait que de la main de Néron. Ensuite ils s’embrassent et se séparent. Après quelques jours d’intervalle, on vit se déployer, dans un appareil également imposant, d’un côté les cavaliers parthes, rangés par escadrons et parés des décorations de leurs pays, de l’autre les légions romaines, à la tête desquelles brillaient les aigles, les enseignes, et les images des dieux, dont l’aspect donnait à ce lieu la majesté d’un temple. Au centre s’élevait un tribunal, surmonté d’une chaise curule où était placée la statue de Néron. Tiridate, après avoir, suivant l’usage, immolé des victimes, s’avance, détache le diadème de sa tète, et le dépose aux pieds de la statue ; spectacle qui remua profondément toutes les âmes, et dont l’impression fut d’autant plus vive, qu’on avait encore devant les yeux le massacre ou le siège des armées romaines. « Mais combien était changé le cours des destins ! Tiridate allait se montrer aux nations ; et que manquait-il pour que ce fût en captif ? "\par
\textbf{XXX.} Aux soins de la gloire, Corbulon joignit les attentions de la politesse et donna un festin. Le roi, à chaque objet nouveau qui frappait ses regards, lui en demandait l’explication : « Pourquoi un centurion annonçait-il le commencement des veilles d’où venait l’usage de se lever de table au son de la trompette, d’aller, avec une torche, allumer le feu sur un autel construit devant l’augural ? » Corbulon, par des réponses où les paroles agrandissaient les choses, le remplit d’admiration pour nos anciennes coutumes. Le lendemain, Tiridate demanda que, avant d’entreprendre un si long voyage, il lui fût permis d’aller voir ses frères et sa mère. En attendant, il laissa sa fille en otage, avec une lettre suppliante pour Néron.\par
\textbf{XXXI.} Il part, et trouve Pacorus chez les Mèdes, Vologèse à Ecbatane \footnote{Ecbatane, capitale de la Grande-Médie, maintenant Ramadan, ville considérable de l’Irak-Adjemi.}. Ce roi n’oubliait pas son frère. Il avait même, par des envoyés particuliers, demandé à Corbulon « qu’on lui épargnât toutes les formes de la servitude, qu’il ne rendit point son épée, qu’il fût admis à embrasser les gouverneurs de nos provinces, dispensé d’attendre à leur porte, traité à Rome avec la même distinction que lest consuls. » C’est que Vologèse, accoutumé à l’orgueil des cours étrangères, ne connaissait pas l’esprit des Romains, pour qui la réalité du pouvoir est tout, ses vanités peu de chose.
\subsection[{Quelques mesures de Néron en 63}]{Quelques mesures de Néron en 63}
\noindent \textbf{XXXII.} La même année, le prince étendit aux nations des Alpes maritimes le droit du Latium \footnote{« Le droit de \emph{Latium} dit Gibbon, était d’une espèce particulière : dans les villes qui jouissaient de cette faveur, les magistrats seulement prenaient, à l’expiration de leurs offices, la qualité de citoyen romain ; mais comme ils étaient annuels, les principales familles se trouvaient bientôt revêtues de cette dignité. »}. Il assigna aux chevaliers romains des places dans le cirque, en avant de celles du peuple ; car jusqu’alors ces deux ordres y assistaient confondus, la loi Roscia n’ayant statué que sur les quatorze premiers rangs du théâtre. Enfin il donna des spectacles de gladiateurs aussi magnifiques que les précédents ; mais trop de sénateurs et de femmes distinguées se dégradèrent sur l’arène.
\subsection[{64 – Néron fait du théâtre}]{64 – Néron fait du théâtre}
\noindent \textbf{XXXIII.} Sous le consulat de C. Lécanius et de M. Licinius, Néron était pressé d’un désir chaque jour plus ardent de monter sur les théâtres publics. Car jusqu’alors il n’avait chanté que dans son palais ou dans ses jardins, aux Juvénales, où les spectateurs étaient trop peu nombreux à son gré, et la scène trop étroite pour une voix si belle. N’osant toutefois faire ses débuts à Rome, il choisit Naples, en qualité de ville grecque : « là il préluderait, pour aller ensuite recueillir dans la Grèce ces brillantes couronnes que l’opinion des siècles a consacrées, et revenir avec une réputation qui enlèverait les applaudissements du peuple romain. » Bientôt la population de Naples rassemblée en foule, les curieux qu’attira des villes voisines le bruit de cette nouveauté, les courtisans du prince, et ceux que leur service attachait à sa suite, enfin jusqu’à des compagnies de soldats, remplirent le théâtre.
\subsection[{Vatinius}]{Vatinius}
\noindent \textbf{XXXIV.} Il y arriva un événement, sinistre aux yeux de la plupart, mais où Néron vit une providence attentive et des dieux bienveillants : quand les spectateurs furent sortis, le théâtre s’écroula ; et, comme il était vide, personne ne fut blessé. Néron composa des chants pour remercier les dieux et retracer l’histoire de cette mémorable aventure. Dans le dessein de traverser la mer Adriatique, il s’arrêta, chemin faisant, à Bénévent, où Vatinius donnait un brillant spectacle de gladiateurs. Vatinius fut une des plus hideuses monstruosités de cette cour. Élevé dans une boutique de cordonnier, les difformités de son corps et la bouffonnerie de son esprit le firent appeler d’abord pour servir de risée : il se poussa par la calomnie et acquit, aux dépens des gens de bien, un crédit, une fortune, un pouvoir de nuire, dont les plus pervers pouvaient être jaloux.
\subsection[{Mort de Torquatus Silanus}]{Mort de Torquatus Silanus}
\noindent \textbf{XXXV.} Pendant que Néron assistait à ces jeux, les plaisirs même ne suspendaient pas le cours des assassinats. C’est précisément à cette époque que Torquatus Silanus fut contraint de mourir, parce qu’à l’illustration de la famille Junia il joignait le crime d’avoir Auguste pour trisaïeul. Les accusateurs eurent ordre de lui reprocher des largesses dont la prodigalité ne lui laissait d’espérance que dans une révolution. On l’accusait même d’avoir chez lui des hommes qu’il qualifiait de secrétaire, de maîtres des requêtes, de trésoriers, préludant par l’usurpation des titres à celle du pouvoir. Alors les affranchis de son intime confiance sont enlevés et chargés de fers. Lui-même, voyant approcher l’instant de sa condamnation, se coupa les veines des bras, et Néron ne manqua pas de dire, suivant sa coutume, que, quel que fût le crime de Torquatus et sa juste défiance dans une défense impossible, il aurait vécu cependant, s’il avait attendu la clémence de son juge.
\subsection[{Un voyage en Égypte avorté}]{Un voyage en Égypte avorté}
\noindent \textbf{XXXVI.} Bientôt Néron, sans qu’on ait su pourquoi, renonça pour le moment à la Grèce et revint à Rome, l’imagination secrètement occupée des provinces d’Orient et surtout de l’Égypte. Il déclara enfin par un édit que son absence ne serait pas longue, et que le repos et la prospérité de l’État n’en seraient point altérés. Puis, à l’occasion de son départ, il monta au Capitole et adora les dieux. Étant allé ensuite au temple de Vesta, il se mit subitement à trembler de tous ses membres, effrayé sans doute par la présence de la déesse, ou par le souvenir de ses forfaits, qui ne le laissait pas un moment sans crainte. Dès lors, il abandonna son dessein, en protestant « qu’aucun soin ne balançait dans son cœur l’amour de la patrie. Il avait vu les visages abattus des citoyens, il entendait leurs plaintes secrètes : ce voyage entraînerait si loin d’eux celui dont ils ne supportaient pas même la plus courte absence, accoutumés qu’ils étaient à se rassurer par la vue du prince contre les coups de la fortune ! Dans les affections de famille, l’homme préférait ce qui tenait à lui de plus près : à ce même titre, le peuple romain avait les premiers droits sur Néron ; et puisqu’il le retenait, il fallait obéir » Ce langage fut agréable à la multitude, avide de plaisirs, et, ce qui est son principal souci, inquiète des subsistances si le prince s’éloignait. Pour le sénat et les grands, ils ne savaient si on devait, absent ou présent, le redouter davantage : à la fin, comme dans toutes les grandes alarmes, l’événement qui arriva fut estimé le pire.
\subsection[{Luxure}]{Luxure}
\noindent \textbf{XXXVII.} Pour accréditer l’opinion que le séjour de Rome faisait ses délices, Néron donnait des festins dans les lieux publics, et il semblait que la ville entière fût son palais. De tous ces repas, aucun n’égala en luxe et en célébrité celui qu’ordonna Tigellin, et que je citerai pour exemple, afin de n’avoir pas à raconter cent fois les mêmes profusions. On construisit sur l’étang d’Agrippa un radeau qui, traîné par d’autres bâtiments, portait le mobile banquet. Les navires étaient enrichis d’or et d’ivoire ; de jeunes infâmes, rangés selon leur âge et leurs lubriques talents, servaient de rameurs. On avait réuni des oiseaux rares, des animaux de tous les pays, et jusqu’à des poissons de l’Océan. Sur les bords du lac s’élevaient des maisons de débauche remplies de femmes du premier rang, et, vis-à-vis, l’on voyait des prostituées toutes nues. Ce furent d’abord des gestes et des danses obscènes ; puis, à mesure que le jour disparut, tout le bois voisin, toutes les maisons d’alentour, retentirent de chants, étincelèrent de lumières. Néron, souillé de toutes les voluptés que tolère ou que proscrit la nature, semblait avoir atteint le dernier terme de la corruption, si, quelques jours après, il n’eut choisi, dans cet impur troupeau, un certain Pythagoras auquel il se maria comme une femme, avec toutes les solennités de noces véritables. Le voile des épouses fut mis sur la tête de l’empereur : auspices, dot, lit nuptial, flambeaux de l’hymen, rien ne fut oublié. Enfin, on eut en spectacle tout ce que, même avec l’autre sexe, la nuit cache de son ombre.
\subsection[{L’incendie de Rome}]{L’incendie de Rome}
\noindent \textbf{XXXVIII.} Le hasard, ou peut-être un coup secret du prince (car l’une et l’autre opinion a ses autorités), causa le plus grand et le plus horrible désastre que Rome eût jamais éprouvé de la violence des flammes. Le feu prit d’abord à la partie du Cirque qui tient au mont Palatin et au mont Célius. Là, des boutiques remplies de marchandises combustibles lui fournirent un aliment, et l’incendie, violent dès sa naissance et chassé par le vent, eut bientôt enveloppé toute la longueur du Cirque ; car cet espace ne contenait ni maisons protégées par un enclos, ni temples ceints de murs, ni rien enfin qui pût en retarder les progrès. Le feu vole et s’étend, ravageant d’abord les lieux bas, puis s’élançant sur les hauteurs, puis redescendant, si rapide que le mal devançait tous les remèdes, et favorisé d’ailleurs par les chemins étroits et tortueux, les rues sans alignement de la Rome d’autrefois. De plus, les lamentations des femmes éperdues, l’âge qui ôte la force aux vieillards et la refuse à l’enfance, cette foule où chacun s’agite pour se sauver soi-même ou en sauver d’autres, où les plus forts entraînent ou attendent les plus faibles, où les uns s’arrêtent, les autres se précipitent, tout met obstacle aux secours. Souvent, en regardant derrière soi, on était assailli par devant ou par les côtés : on se réfugiait dans le voisinage, et il était envahi par la flamme ; on fuyait encore, et les lieux qu’on en croyait le plus loin s’y trouvaient également en proie. Enfin, ne sachant plus ce qu’il fallait ou éviter ou chercher, toute la population remplissait les rues, gisait dans les campagnes. Quelques-uns, n’ayant pas sauvé de toute leur fortune de quoi suffire aux premiers besoins de la vie, d’autres, désespérés de n’avoir pu arracher à la mort les objets de leur tendresse, périrent quoiqu’ils pussent échapper. Et personne n’osait combattre l’incendie : des voix menaçantes défendaient de l’éteindre ; des inconnus lançaient publiquement des torches, en criant qu’ils étaient autorisés ; soit qu’ils voulussent piller avec plus de licence, soit qu’en effet ils agissent par ordre.\par
\textbf{XXXIX.} Pendant ce temps, Néron était à Antium et n’en revint que quand le feu approcha de la maison qu’il avait bâtie pour joindre le palais des Césars aux jardins de Mécène. Toutefois on ne put empêcher l’embrasement de dévorer et le palais, et la maison, et tous les édifices d’alentour. Néron, pour consoler le peuple fugitif et sans asile, ouvrit le Champ de Mars, les monuments d’Agrippa et jusqu’à ses propres jardins. Il fit construire à la hâte des abris pour la multitude indigente ; des meubles furent apportés d’Ortie et des municipes voisins, et le prix du blé fut baissé jusqu’à trois sesterces \footnote{Le prix indiqué ici est celui du \emph{modius}, qu’on traduit ordinairement par boisseau, et qui égalait 40 litres 1/10 de nos mesures. Or 3 sesterces représentaient sous Néron 54 centimes 3/4, ce qui porterait l’hectolitre à 5 fr. 42 c.}. Mais toute cette popularité manqua son effet, car c’était un bruit général qu’au moment où la ville était en flammes il était monté sur son théâtre domestique et avait déclamé la ruine de Troie, cherchant, dans les calamités des vieux âges, des allusions au désastre présent.\par
\textbf{XL.} Le sixième jour enfin, on arrêta le feu au pied des Esquilies, en abattant un nombre immense d’édifices, afin d’opposer à sa contagion dévorante une plaine nue et pour ainsi dire le vide des cieux. La terreur n’était pas encore dissipée quand l’incendie se ralluma, moins violent, toutefois, parce que ce fut dans un quartier plus ouvert : cela fit aussi que moins d’hommes y périrent ; mais les temples des dieux, mais les portiques destinés à l’agrément, laissèrent une plus vaste ruine. Ce dernier embrasement excita d’autant plus de soupçons, qu’il était parti d’une maison de Tigellin dans la rue Émilienne. On crut que Néron ambitionnait la gloire de fonder une ville nouvelle et de lui donner son nom. Rome est divisée en quatorze régions : quatre restèrent intactes ; trois étaient consumées jusqu’au sol ; les sept autres offraient à peine quelques vestiges de bâtiments en ruine et à moitié brûlés.\par
\textbf{XLI.} Il serait difficile de compter les maisons, les \emph{îles} \footnote{On appela d’abord \emph{insula} un quartier plus ou moins grand, compris entre quatre rues, ce qu’on nomme encore \emph{île} dans plusieurs villes du midi de la France ; ce nom s’étendit peu à peu à chacune des maisons qui formaient cet assemblage.}, les temples qui furent détruits. Les plus antiques monuments de la religion, celui que Servius Tullius avait dédié à la Lune, le Grand autel et le temple consacrés par l’Arcadien Évandre à Hercule vivant et présent, celui de Jupiter Stator, voué par Romulus, le palais de Numa Pompilius et le sanctuaire de Vesta, avec les Pénates du peuple romain, furent la proie des flammes. Ajoutez les richesses conquises par tant de victoires, les chefs-d’ œuvre des arts de la Grèce, enfin les plus anciens et les plus fidèles dépôts des conceptions du génie, trésors dont les vieillards gardaient le souvenir, malgré la splendeur de la ville renaissante, et dont la perte était irréparable. Quelques-uns remarquèrent que l’incendie avait commencé le quatorze avant les kalendes d’août, le jour même où les Sénonais avaient pris et brûlé Rome. D’autres poussèrent leurs recherches jusqu’à supputer autant d’années, de mois et de jours de la fondation de Rome au premier incendie, que du premier au second.
\subsection[{Reconstruction}]{Reconstruction}
\noindent \textbf{XLII.} Néron mit à profit la destruction de sa patrie, et bâtit un palais où l’or et les pierreries n’étaient pas ce qui étonnait davantage ; ce luxe est depuis longtemps ordinaire et commun mais il enfermait des champs cultivés, des lacs, des solitudes artificielles, bois, esplanades, lointains. Ces ouvrages étaient conçus et dirigés par Céler et Sévérus, dont l’audacieuse imagination demandait à l’art ce que refusait la nature, et se jouait capricieusement des ressources du prince. Ils lui avaient promis de creuser un canal navigable du lac Averne à l’embouchure du Tibre, le long d’un rivage aride ou sur un sol traversé de montagnes. On ne rencontrait d’eaux que celles des marais Pontins ; le reste du pays était sec ou escarpé dût-on venir à bout de vaincre les obstacles, le travail était excessif, l’utilité médiocre. Néron cependant voulait de l’incroyable : il essaya de percer les hauteurs voisines de l’Averne, et l’on voit encore des traces de son espérance déçue.\par
\textbf{XLIII.} Au reste, ce que l’habitation d’un homme laissa d’espace à la ville, ne fut pas, comme après l’incendie des Gaulois, rebâti au hasard et sans ordre. Les maisons furent alignées, les rues élargies, les édifices réduits à une juste hauteur. On ouvrit des cours, et l’on éleva des portiques devant la façade des bâtiments. Néron promit de construire ces portiques à ses frais, et de livrer aux propriétaires les terrains nettoyés, ajoutant, pour ceux qui auraient achevé leurs constructions dans un temps qu’il fixa, des récompenses proportionnés à leur rang et à leur fortune. Les marais d’Ostie furent destinés à recevoir les décombres ; on en chargeait, à leur retour vers la mer, les navires qui avaient remonté le Tibre avec du blé. Une partie déterminée de chaque édifice fut bâtie sans bois, mais seulement avec des pierres d’Albe ou de Gabie, qui sont à l’épreuve du feu. L’eau, que des particuliers détournaient à leur usage, fut rendue au public ; et des gardiens furent chargés de veiller à ce qu’elle coulât plus abondante et en plus de lieux divers : chacun fut obligé de tenir toujours prêt et sous la main ce qu’il faut pour arrêter le feu ; enfin les murs mitoyens furent interdits, et l’on voulut que chaque maison eût son enceinte séparée. Ces règlements contribuèrent à l’embellissement non moins qu’à l’utilité de la nouvelle ville. Quelques-uns crurent cependant que l’ancienne forme convenait mieux pour la salubrité, parce que, les rues étant étroites et les toits élevé, le soleil y dardait moins de feu, tandis que, maintenant, il embrase de toutes ses ardeurs ces vastes espaces que ne défend aucune ombre.
\subsection[{Les Chrétiens}]{Les Chrétiens}
\noindent \textbf{XLIV.} La prudence humaine avait ordonné tout ce qui dépend de ses conseils : on songea bientôt à fléchir les dieux, et l’on ouvrit les livres sibyllins. D’après ce qu’on y lut, des prières furent adressées à Vulcain, à Cérès et à Proserpine : des dames romaines implorèrent Junon, premièrement au Capitole, puis au bord de la mer la plus voisine, où l’on puisa de l’eau pour faire des aspersions sur les murs du temple et la statue de la déesse ; enfin les femmes actuellement mariées célébrèrent des sellisternes\footnote{Dans certaines solennités religieuses, ordonnées pour remercier ou apaiser le ciel, on couvrait les autels des mets les plus somptueux, et comme si l’on eût invité les dieux à un festin, on rangeait leurs statues à l’entour, celles des dieux sur des lits, \emph{lectos, pulvinaria}, celles des déesses sur des sièges, \emph{sellas} ; d’où \emph{lectisternia} et \emph{sellisternia}.} et des veillées religieuses. Mais aucun moyen humain, ni largesses impériales, ni cérémonies expiatoires ne faisaient taire le cri public qui accusait Néron d’avoir ordonné l’incendie. Pour apaiser ces rumeurs, il offrit d’autres coupables, et fit souffrir les tortures les plus raffinées à une classe d’hommes détestés pour leurs abominations et que le vulgaire appelait chrétiens. Ce nom leur vient de Christ, qui, sous Tibère, fut livré au supplice par le procurateur Pontius Pilatus. Réprimée un instant, cette exécrable superstition se débordait de nouveau, non-seulement dans la Judée, où elle avait sa source, mais dans Rome même, où tout ce que le monde enferme d’infamies et d’horreurs afflue et trouve des partisans. On saisit d’abord ceux qui avouaient leur secte ; et, sur leurs révélations, une infinité d’autres, qui furent bien moins convaincus d’incendie que de haine pour le genre humain. On fit de leurs supplices un divertissement : les uns, couverts de peaux de bêtes, périssaient dévorés par des chiens ; d’autres mouraient sur des croix, ou bien ils étaient enduits de matières inflammables, et, quand le jour cessait de luire, on les brûlait en place de flambeaux. Néron prêtait ses jardins pour ce spectacle, et donnait en même temps des jeux au Cirque, où tantôt il se mêlait au peuple en habit de cocher, et tantôt conduisait un char. Aussi, quoique ces hommes fussent coupables et eussent mérité les dernières rigueurs, les cœurs s’ouvraient à la compassion, en pensant que ce n’était pas au bien public, mais à la cruauté d’un seul, qu’ils étaient immolés.
\subsection[{Pillage – Sénèque se porte malade}]{Pillage – Sénèque se porte malade}
\noindent \textbf{XLV.} Cependant, pour remplir le trésor, on ravageait l’Italie, on ruinait les provinces, les peuples alliés, les villes qu’on appelle libres. Les dieux mêmes furent enveloppés dans ce pillage : on dépouilla les temples de Rome, et on en retira tout l’or votif ou triomphal que le peuple romain, depuis son origine, gavait consacré dans ses périls ou ses prospérités. Mais en Asie, mais en Grèce, avec les offrandes, on enlevait encore les statues des dieux ; mission dignement remplie par Acratus et Sécundus Carinas, envoyés dans ces provinces. Acratus était un affranchi capable de tous les crimes ; Carinas, exercé dans la philosophie grecque, en avait les maximes à la bouche, sans que la morale eût pénétré jusqu’à son âme. Le bruit courut que Sénèque, pour échapper à l’odieux de tant de sacrilèges, avait demandé à se retirer dans une terre éloignée, et que, sur le refus du prince, il avait feint d’être malade de la goutte et n’était plus sorti de son appartement. Quelques-uns rapportent que du poison fut préparé pour lui par un de ses affranchis nommé Cléonicus, qui en avait l’ordre de Néron, et que Sénèque fut sauvé, soit par la révélation de l’affranchi, soit par sa propre défiance et son extrême frugalité, ne se nourrissant que de fruits sauvages, et se désaltérant avec de l’eau courante.
\subsection[{Désastre naval}]{Désastre naval}
\noindent \textbf{XLVI.} Dans le même temps, des gladiateurs qui étaient à Préneste essayèrent de rompre leurs fers, et furent contenus par les soldats chargés de les garder. Déjà les imaginations effrayées voyaient renaître Spartacus et tous les malheurs anciens : tant le peuple désire à la fois et redoute les nouveautés. Peu de temps après, on reçut la nouvelle d’un désastre naval, causé non par la guerre (jamais la paix ne fut plus profonde), mais par les ordres absolus de Néron, qui avait fixé un jour pour que la flotte fût rendue en Campanie, et n’avait pas excepté les hasards de la mer. Les pilotes obéissants partirent de Formies, malgré la tempête, et pendant qu’ils s’efforçaient de doubler le promontoire de Misène, un vent violent d’Afrique les poussa sur les rivages de Cumes, où ils perdirent la plupart des trirèmes et beaucoup de petits bâtiments.
\subsection[{Prodiges}]{Prodiges}
\noindent \textbf{XLVII.} A la fin de l’année, on ne s’entretint que de prodiges, avant-coureurs de calamités prochaines : coups de foudre plus réitérés qu’à aucune autre époque ; apparition d’une comète, sorte de présage que Néron expia toujours par un sang illustre ; embryons à deux têtes, soit d’hommes, soit d’animaux, jetés dans les chemins, ou trouvés dans les sacrifices où l’usage est d’immoler des victimes pleines. Enfin, dans le territoire de Plaisance, un veau naquit, dit-on, près de la route, avec la tête à la cuisse ; et les aruspices en conclurent qu’on voulait donner à l’empire une autre tête, mais qu’elle ne serait pas forte, ni le secret bien gardé, parce que l’accroissement de l’animal avait été arrêté dans le ventre de la mère, et qu’il était né sur la voie publique.
\subsection[{65 – A Rome – Conjuration de Pison}]{65 – A Rome – Conjuration de Pison}
\noindent \textbf{XLVIII.} Silius Nerva et Atticus Vestinus prirent possession du consulat au moment d’une conjuration, puissante aussitôt que formée, dans laquelle s’étaient jetés à l’envi des sénateurs, des chevaliers, des soldats, des femmes même, autant par haine contre le prince que par inclination pour C. Pison. Issu des Calpurnius, et tenant, par la noblesse du sang paternel, à beaucoup d’illustres familles, Pison devait à ses vertus, ou à des dehors qui ressemblaient aux vertus, une grande popularité. Consacrant son éloquence à défendre les citoyens, généreux envers ses amis, affable et prévenant même pour les inconnus, il avait encore ce que donne le hasard, une haute taille et une belle figure ; mais nulle gravité dans les mœurs, nulle retenue dans les plaisirs : il menait une vie douce, amie du faste, dissolue quelquefois ; et c’était un titre de plus aux suffrages de tous ceux qui, séduits par les charmes du vice, ne veulent pas dans le pouvoir suprême trop de contrainte ni de sévérité.
\subsection[{Conjuration de Pison – Mort de Lucain}]{Conjuration de Pison – Mort de Lucain}
\noindent \textbf{XLIX.} Le complot ne naquit point de son ambition, et toutefois j’aurais peine à dire quel en fut le premier auteur, et sous l’inspiration de qui se forma un dessein qui eut tant de complices. Les plus ardents, comme le prouva la fermeté de leur mort, furent Subrius Flavius, tribun d’une cohorte prétorienne, et le centurion Sulpicius Asper. Lucain et le consul désigné Plautius Latéranus y portèrent toute la vivacité de la haine. Un ressentiment personnel animait Lucain : Néron, pour étouffer sa réputation poétique, lui avait défendu de montrer ses vers, dont il avait la vanité d’être jaloux. Quant à Latéranus, élu consul, il n’avait aucun motif de vengeance ; l’amour seul de la patrie en fit un conjuré. Deux sénateurs, Flavius Scévinus et Afranius Quinctianus, démentirent leur renommée, en embrassant, dès le commencement, une si hasardeuse entreprise : car Scévinus avait l’âme énervée parla débauche, et sa vie languissait dans l’assoupissement ; Quinctianus, décrié pour l’impureté de ses mœurs, et diffamé par Néron dans des vers satiriques, pensait à venger son injure.\par
\textbf{L.} Ces hommes donc, par les propos qu’ils tenaient entre eux ou avec leurs amis sur les crimes du prince, la fin prochaine de l’empire, la nécessité de choisir un chef qui le sauvât de sa ruine, associèrent à leurs vues Tullius Sénécio, Cervarius Proculus, Vulcatius Araricus, Julius Tugurinus, Munatius Gratus, Antonius Natalis, Martius Festus, tous chevaliers romains. D’une intime familiarité avec le prince, il restait à Sénécion les semblants de l’amitié, et plus de périls en menaçaient sa tête. Natalis était le confident de tous les secrets de Pison ; le reste fondait sur une révolution d’ambitieuses espérances. Avec Subrius et Sulpicius, que j’ai déjà nommés, d’autres gens d’épée promirent encore leurs bras, Granius Silvanus et Statius Proximus, tribuns dans les cohortes prétoriennes, Maximus Scaurus et Vénétus Paulus, centurions. Mais la force principale semblait être dans le préfet du prétoire Fénius Rufus, homme estimé pour sa conduite et ses mœurs, que Tigellin, cruel, impudique, et, à ce titre, placé bien plus avant dans le cœur du prince, poursuivait de ses délations. Même il l’avait plus d’une fois mis en péril, sous prétexte d’amours criminelles avec Agrippine, que Fénius regrettait, selon lui, et qu’il voulait venger. Quand les conjurés virent un préfet du prétoire engagé dans leur parti, et qu’ils en eurent plusieurs fois reçu l’assurance de sa bouche, ils commencèrent à délibérer plus hardiment sur le lieu et le temps de l’exécution. On dit que Subrius avait déjà eu la pensée d’attaquer Néron pendant qu’il chantait sur la scène, ou lorsque, dans l’incendie du palais, il courait çà et là, de nuit et sans gardes. Ici la solitude, là tout un peuple témoin d’un coup si glorieux, aiguillonnaient ce généreux courage ; mais il fut retenu par le désir de l’impunité, écueil ordinaire des grands desseins.\par
\textbf{LI.} Pendant que les conjurés indécis reculaient le terme de leurs espérances et de leurs craintes, une femme nommée Épicharis, qui était entrée dans le secret sans qu’on ait su comment (rien d’honnête jusqu’alors n’avait occupé sa pensée), les animait par ses exhortations et ses reproches. Enfin, ennuyée de leurs lenteurs, et se trouvant en Campanie auprès de la flotte de Misène, elle essaye d’en ébranler les chefs et de les lier au parti par la complicité. Voici le commencement de cette intrigue : un des chiliarques \footnote{Un commandant de mille hommes.} de la flotte, Volusius Proculus, avait eu part à l’attentat de Néron contre les jours de sa mère, et se croyait peu récompensé pour un crime de cette importance. Soit qu’Épicharis le connût auparavant, ou qu’une amitié récente les unit, il lui parle des services qu’il avait rendus à Néron et du peu de fruit qu’il en recueillait. Les plaintes qu’il ajoute, sa résolution de se venger, s’il en avait le pouvoir, donnèrent à Épicharis l’espérance de l’entraîner, et, par lui, beaucoup d’autres. La flotte eût été d’un grand secours et aurait offert de fréquentes occasions, le prince aimant beaucoup à se promener sur mer à Pouzzoles et Misène. Épicharis poursuit donc l’entretien et passe en revue tous les forfaits de Néron : « Oui, le sénat était anéanti, mais on avait pourvu à ce que le destructeur de la république expiât ses crimes : que Proculus se tînt prêt seulement à seconder l’entreprise et tâchât d’y gagner les plus intrépides soldats ; il recevrait un digne prix de ses services » Elle tut cependant le nom des conjurés : aussi les révélations de Proculus furent-elles sans effet, quoiqu’il eût rapporté à Néron tout ce qu’il avait entendu. Épicharis, appelée et confrontée avec le délateur, réfuta sans peine ce que n’appuyait aucun témoin. Toutefois, elle fut retenue en prison, Néron soupçonnant que des faits dont la vérité n’était pas démontrée pouvaient encore n’être pas faux.\par
\textbf{LII.} Alarmés cependant par la crainte d’une trahison, les conjurés furent d’avis de hâter le meurtre et de le consommer à Baïes, dans la maison de campagne de Pison ; car l’empereur, charmé des agréments de ce lieu, s’y rendait souvent, et s’y livrait aux plaisirs du bain et de la table, sans garde et débarrassé de l’attirail de sa puissance. Mais Pison s’y refusa : il trouvait odieux d’ensanglanter par le meurtre d’un prince, quel qu’il fût, la table sacrée du festin et les dieux hospitaliers. « C’était au sein de Rome, dans ce palais abhorré, bâti des dépouilles des citoyens, c’était au moins dans un lieu public, qu’il fallait accomplir un dessein conçu pour l’avantage de tout le peuple. » Voilà ce qu’il disait tout haut ; mais sa crainte secrète était que Silanus, qu’une illustre naissance et une âme formée par les soins et sous la discipline de C. Cassius semblaient porter à toutes les grandeurs, ne s’emparât du pouvoir ; entreprise que s’empresseraient d’appuyer tous les hommes étrangers à la conjuration, et ceux qui plaindraient en Néron la victime d’une perfidie. Plusieurs crurent que Pison redoutait aussi le génie entreprenant du consul Vestinus, qui aurait pu songer à la liberté, ou choisir un prince qui reçût l’empire comme un don de sa main. Car Vestinus n’était pas de la conjuration, bien qu’elle ait servi de prétexte à Néron pour l’immoler, malgré son innocence, à de vieilles inimitiés.\par
\textbf{LIII.} Enfin ils résolurent d’exécuter leur projet pendant les jeux du Cirque, le jour consacré à Cérès. Néron, dont les sorties étaient rares, et qui se tenait renfermé dans son palais ou dans ses jardins, ne manquait pas de venir à ces fêtes, et la gaieté du spectacle rendait auprès de lui l’accès plus facile. Voici comme ils avaient concerté leur attaque. Latéranus, sous prétexte d’implorer pour les besoins de sa maison la générosité du prince, devait tomber à ses genoux d’un air suppliant, le renverser adroitement et le tenir sous lui ; car Latéranus était d’un courage intrépide et d’une stature colossale. Ainsi terrassé et fortement contenu, les tribuns, les centurions et les plus déterminés des autres complices accourraient pour le tuer. Scévinus sollicitait l’honneur de frapper le premier : il avait enlevé un poignard du temple de la déesse Salus en Étrurie, ou, suivant quelques-uns, du temple de la Fortune à Férentum, et il le portait sans cesse, comme une arme vouée à quelque grand exploit. On était convenu que Pison attendrait dans le temple de Cérès, où Fénius et les autres iraient le prendre pour le conduire au camp : Antonia, fille de l’empereur Claude, devait l’y accompagner, afin de lui concilier la faveur des soldats ; c’est au moins ce que rapporte Pline. Quoi qu’il en soit de cette tradition, je n’ai pas cru devoir la négliger, malgré le peu d’apparence qu’Antonia, sur un espoir chimérique, eût prêté son nom, et hasardé ses jours, ou que Pison, connu pour aimer passionnément sa femme, se fût lié par la promesse d’un autre mariage ; si ce n’est pourtant qu’il n’y a pas de sentiment que n’étouffe l’ambition de régner.\par
\textbf{LIV.} Il est étonnant combien, entre tant de conjurés, différents d’âge, de rang, de naissance, hommes, femmes, riches, pauvres, tout fut renfermé dans un impénétrable secret. Enfin la trahison partit de la maison de Scévinus. La veille de l’exécution, après un long entretien avec Natalis, il rentra chez lui et scella son testament ; puis, tirant du fourreau le poignard dont je viens de parler, et se plaignant que le temps l’eût émoussé, il ordonna qu’on en avivât le tranchant sur la, pierre et qu’on y fit une pointe bien acérée ; il confia ce soin à l’affranchi Milichus. En même temps il fit servir un repas plus somptueux qu’à l’ordinaire, et donna la liberté à ses esclaves favoris, de l’argent aux autres. Du reste, il était sombre, et une grande pensée le préoccupait visiblement, quoiqu’il s’égarât en des propos divers où il affectait la gaieté. Enfin il charge ce même Milichus d’apprêter ce qu’il faut pour bander des plaies et arrêter le sang ; soit que cet affranchi connut la conjuration et eût été fidèle jusqu’alors, soit qu’il ignorât un secret dont le premier soupçon lui serait venu à cet instant même, comme la suite l’a fait dire à plusieurs. Quand cette âme servile eut calculé le prix de la perfidie, ne rêvant plus que trésors et puissance, elle oublia le devoir, la vie d’un patron, la liberté reçue. Milichus avait pris d’ailleurs les conseils de sa femme, conseils lâches et pervers. Elle lui remplissait l’esprit de frayeurs : « Beaucoup d’esclaves, lui disait-elle, beaucoup d’affranchis avaient vu les mêmes choses que lui ; le silence d’un seul homme ne sauverait rien ; mais un seul homme aurait toutes les récompenses, quand il aurait donné le premier avis. »\par
\textbf{LV.} Au point du jour, Milichus courut donc aux jardins de Servilius. D’abord on lui en refusa l’entrée ; mais à force de répéter qu’il apportait un avis de la nature la plus grave, la plus effrayante, il se fit introduire chez Épaphrodite, affranchi de Néron. Conduit par celui-ci devant le prince, il lui dénonce un péril imminent, de redoutables complots, enfin tout ce qu’il a entendu, tout ce qu’il a conjecturé. Il lui montre même le poignard aiguisé pour le tuer, et demande qu’on fasse venir celui qu’il accuse. Enlevé aussitôt par des soldats, Scévinus paraît et cherche à se justifier : « Le fer dont on lui faisait un crime était l’objet d’un culte héréditaire dans sa famille ; il le gardait dans sa chambre, d’où son perfide affranchi l’avait dérobé. Déjà plus d’une fois il avait scellé son testament, sans avoir pour cet acte des jours de préférence : plus d’une fois aussi il avait donné à des esclaves ou de l’argent ou la liberté ; s’il s’était montré plus généreux en cette occasion, c’est que, devenu moins riche et pressé par ses créanciers, il avait des craintes pour son testament. De tout temps sa table avait été libéralement servie, et il aimait à se donner des douceurs que les juges sévères n’avaient pas toujours approuvées. D’appareils pour panser les blessures, il n’en avait point commandé ; mais le calomniateur, sentant qu’il incriminait vainement des faits publics, en avait supposé un dont il fût tout à la fois le dénonciateur et le témoin. » Une intrépide fermeté soutenait ce langage : il traite à son tour l’affranchi de monstre exécrable, chargé de tous les crimes, et cela d’un ton et d’un visage si assurés que la délation tombait, si Milichus n’eut été averti par sa femme que Natalis avait eu avec Scévinus une longue et secrète conférence, et que tous deux étaient intimes amis de Pison.\par
\textbf{LVI.} On fait venir Natalis et on les interroge séparément sur la nature et l’objet de cet entretien. La diversité de leurs réponses ayant fait naître des soupçons, on les chargea de fers. Leur constance ne tint point contre l’aspect et la menace des tortures. Ce fut pourtant Natalis qui parla le premier. Mieux instruit de tout le complot et accusateur plus adroit, il nomma d’abord Pison ; puis il ajouta Sénèque, soit qu’en effet il eût servi de négociateur entre Pison et lui, soit dans la vue de plaire à Néron, qui haïssait mortellement Sénèque et cherchait tous les moyens de le perdre. Quand Scévinus connut les dépositions de Natalis, succombant à la même faiblesse, ou peut-être croyant tout découvert et ne voyant plus d’avantage à garder le silence, il révéla les autres complices. Trois d’entre eux, Lucain, Quinctianus et Sénécion, nièrent longtemps. Enfin, corrompus par la promesse de l’impunité, ils voulurent se faire pardonner la lenteur de leurs aveux, et Lucain prononça le nom d’Atilla, sa propre mère, Quinctianus celui de Glitius Gallus, Sénécion celui d’Annius Pollio, leurs meilleurs amis.\par
\textbf{LVII.} Néron cependant se ressouvint qu’Épicharis était détenue sur la dénonciation de Proculus, et, persuadé qu’un sexe si faible ne résisterait pas à la douleur, il donna ordre qu’on la déchirât de tortures. Mais ni le fouet, ni les feux, ni la rage des bourreaux, qui redoublaient d’acharnement pour ne pas être bravés par une femme, ne purent lui arracher un aveu. C’est ainsi que le premier jour elle triompha de la question. Le lendemain, comme on la traînait au même supplice, assise dans une chaise à porteur (car ses membres tout brisés ne pouvaient plus la soutenir), elle défit le vêtement qui lui entourait le sein, et, avec le lacet, forma un nœud coulant qu’elle attacha au haut de la chaise ; puis elle y passa son cou, et, pesant sur ce nœud de tout le poids de son corps, elle s’ôta le souffle de vie qui lui restait : courage admirable dans une affranchie, dans une femme, qui, soumise à une si redoutable épreuve, protégeait de sa fidélité des étrangers, presque des inconnus ; tandis que des hommes de naissance libre, d’un sexe fort, des chevaliers romains, des sénateurs, n’attendaient pas les tortures pour trahir à l’envi ce qu’ils avaient de plus cher. Car Lucain même, et Sénécion, et Quinctianus, dénonçaient encore et révélaient complices sur complices, au grand effroi de Néron, qui tremblait de plus en plus, malgré les, gardes sans nombre dont il s’était environné.\par
\textbf{LVIII.} Et c’était peu de ses gardes : il couvrit de troupes les murailles, en assiégea la mer et le fleuve ; on eût dit qu’il, tenait Rome même prisonnière. De tous côtés voltigeaient sur les places, dans les maisons et jusque dans les campagnes et dans les villes voisines, des fantassins, des cavaliers entremêlés de Germains, qui avaient la confiance du prince à titre d’étrangers. Ils ramenaient par longues files des légions d’accusés ; qu’on entassait aux portes des jardins. Quand on les introduisait pour être jugés, malheur à celui qui avait souri à un conjuré, qui lui avait parlé par hasard, qui l’avait seulement rencontré, ou qui s’était trouvé avec lui dans un repas ou à quelque spectacle : c’étaient autant de crimes. Durement interrogés par Tigellin et Néron, Fénius les pressait encore avec, acharnement : personne ne l’avait nommé jusqu’alors ; mais, en traitant impitoyablement ses complices, il se ménageait un moyen de faire croire qu’il avait tout ignoré. Subrius, présent aux interrogatoires, eut l’idée de tirer son glaive et de frapper Néron sur l’heure même ; il fit signe à Fénius, qui par un signe contraire arrêta son mouvement ; déjà le tribun portait la main à la garde de son épée.\par
\textbf{LIX.} Quand le secret de la conjuration fut trahi, pendant qu’on entendait Milichus et que Scévinus balançait encore, quelques amis de Pison le pressèrent de se rendre au camp, ou de monter à la tribune aux harangues et d’essayer les dispositions du peuple ou des soldats. « Si les confidents de leurs desseins accouraient à ce signal, ils entraîneraient tout le reste ; et quelle impression ne ferait pas ce premier coup porté, impression si puissante en toute nouvelle entreprise ? Néron n’avait rien de préparé contre cette attaque. Une surprise déconcertait jusqu’aux braves : irait-il, ce comédien, accompagné sans doute de Tigellin avec ses concubines, opposer la force à la force ? On voyait souvent réussir à l’épreuve ce qu’un esprit timide aurait cru impossible. En vain Pison espérait-il de tant de complices silence et fidélité : ou les âmes, ou les corps pouvaient faillir ; il n’était pas de secret que ne pénétrassent les tortures ou les récompenses. On viendrait donc l’enchaîner à son tour, et le traîner à une mort ignominieuse. Combien il serait plus beau de périr dans les bras de la république, en élevant le drapeau de la liberté ! Dussent les soldats manquer à son courage, dût-il être abandonné du peuple ah ! que du moins, si la vie lui était arrachée, il honorât sa mort aux yeux de ses ancêtres et de ses descendants ! " Insensible à ces exhortations, il sortit un instant, puis se renferma chez lui, fortifiant son âme contre le moment suprême. Bientôt parut une troupe de satellites que Néron avait composée de recrues ou de gens ayant peu de service : il craignait que l’esprit des vieux soldats ne fût gagné au parti. Pison mourut en se faisant couper les veines des bras. Il laissa un testament rempli pour Néron de basses flatteries : c’était par faiblesse pour son épouse, femme dégradée, sans autre mérite que sa beauté, et qu’il avait enlevée à la couche d’un de ses amis. Elle se nommait Arria Galla, le premier mari Domitius Silius ; tous deux ont à jamais flétri la renommée de Pison, lui par d’infâmes complaisances, elle par son impudicité.
\subsection[{Mort de Sénèque}]{Mort de Sénèque}
\noindent \textbf{LX.} La première mort qui suivit fut celle du consul désigné Plautius Latéranus ; elle fut si précipitée que Néron ne lui permit ni d’embrasser ses enfants, ni de jouir de ce peu de moments qu’il laissait à d’autres pour choisir leur trépas. Traîné au lieu réservé pour le supplice des esclaves, il est égorgé par la main du tribun Statius et meurt plein d’une silencieuse constance, et sans reprocher au tribun sa propre complicité. A cette mort succéda celle de Sénèque, plus agréable au prince que toutes les autres : non que rien prouvât qu’il eût eu part au complot ; mais Néron voulait achever par le fer ce qu’il avait en vain tenté par le poison. Natalis seul avait nommé Sénèque, et il s’était borné à dire « que, celui-ci étant malade, il avait eu mission de le visiter et de se plaindre que sa porte fût fermée à Pison, quand ils devraient plutôt cultiver leur amitié, en se voyant familièrement. A quoi Sénèque avait répondu que des visites mutuelles et de fréquents entretiens ne convenaient ni à l’un ni à l’autre ; qu’au reste ses jours étaient attachés à la conservation de Pison. » Granius Silvanus, tribun d’une cohorte prétorienne, fut chargé de communiquer cette déposition à Sénèque, et de lui demander s’il reconnaissait les paroles de Natalis et sa propre réponse. Soit hasard, soit dessein Sénèque était arrivé ce jour-là de Campanie, et il s’était arrêté dans une maison de plaisance, la quatrième pierre milliaire. Le tribun s’y rendit vers le soir, et entoura la maison de soldats. Sénèque était à table avec sa femme Pompéia Paullina et deux de ses amis, quand il lui exposa le message de l’empereur.\par
\textbf{LXI.} Il répondit « que Natalis était venu chez lui se plaindre, au nom de Pison, que ce dernier ne fût pas admis à lui rendre visite, et que pour excuse il avait allégué sa santé et son amour du repos ; que du reste il n’avait aucune raison de préférer les jours d’un particulier à sa propre conservation ; qu’il n’avait pas l’esprit enclin à la flatterie ; que Néron le savait mieux que personne, ayant plus souvent trouvé en lui un homme libre qu’un esclave. » Quand Silvanus eut rapporté ces paroles à Néron, en présence de Poppée et de Tigellin, les conseillers intimes de ses cruautés, le prince demanda si Sénèque se disposait à quitter la vie. Le tribun assura qu’il n’avait remarqué en lui aucun signe de frayeur, que rien de triste n’avait paru dans ses discours ni sur son visage. A l’instant il reçut l’ordre de retourner et de lui signifier son arrêt de mort. Fabius Rusticus raconte que Silvanus ne prit pas le chemin par où il était venu, mais qu’il se détourna pour aller chez Fénius, et que, après lui avoir exposé les volontés du prince, il lui demanda s’il devait obéir, ce que le préfet lui conseilla de faire. Étrange concours de lâcheté ! Silvanus aussi était de la conjuration, et il grossissait le nombre des crimes dont il avait conspiré la vengeance. Il eut toutefois la pudeur de ne pas se montrer ; et un centurion entra par son ordre pour notifier à Sénèque la sentence fatale.\par
\textbf{LXII.} Sénèque, sans se troubler, demande son testament, et, sur le refus du centurion, il se tourne vers ses amis, et déclare « que, puisqu’on le réduit à l’impuissance de reconnaître leurs services, il leur laisse le seul bien qui lui reste, et toutefois le plus précieux, l’image de sa vie ; que, s’ils gardent le souvenir de ce qu’elle eut d’estimable, cette fidélité à l’amitié deviendra leur gloire. » Ses amis pleuraient : lui, par un langage tour à tour consolateur et sévère, les rappelle à la fermeté, leur demandant « ce qu’étaient devenus les préceptes de la sagesse, où était cette raison qui se prémunissait depuis tant d’années contre tous les coups du sort. La cruauté de Néron était-elle donc ignorée de quelqu’un ? et que restait-il à l’assassin de sa mère et de son frère, que d’être aussi le bourreau du maître qui éleva son enfance ? "\par
\textbf{LXIII}. Après ces exhortations, qui s’adressaient à tous également, il embrasse sa femme, et, s’attendrissant un peu en ces tristes instants, il la prie, il la conjure « de modérer sa douleur ; de ne pas nourrir des regrets éternels ; de chercher plutôt, dans la contemplation d’une vie toute consacrée à la vertu, de nobles consolations à la perte d’un époux. » Pauline proteste qu’elle aussi est décidée à mourir ; et elle appelle avec instance la main qui doit frapper. Sénèque ne voulut pas s’opposer à sa gloire ; son amour d’ailleurs craignait d’abandonner aux outrages une femme qu’il chérissait uniquement. « Je t’avais montré, lui dit-il, ce qui pouvait te gagner à la vie : tu préfères l’honneur de la mort ; je ne t’envierai pas le mérite d’un tel exemple. Ce courageux trépas, nous le subirons l’un et l’autre d’une constance égale ; mais plus d’admiration consacrera ta fin. » Ensuite le même fer leur, ouvre les veines des bras. Sénèque, dont le corps affaibli par les années et par l’abstinence laissait trop lentement échapper le sang, se fait aussi couper les veines des jambes et des jarrets. Bientôt, dompté par d’affreuses douleurs, il craignit que ses souffrances n’abattissent le courage de sa femme, et que lui-même, en voyant les tourments qu’elle endurait, ne se laissât aller à quelque faiblesse ; il la pria de passer dans une chambre voisine. Puis, retrouvant jusqu’en ses derniers moments toute son éloquence, il appela des secrétaires et leur dicta un assez long discours. Comme on l’a publié tel qu’il sortit de sa bouche, je m’abstiendrai de le traduire en des termes différents.\par
\textbf{LXVI.} Néron, qui n’avait contre Pauline aucune haine personnelle, et qui craignait de soulever les esprits par sa cruauté, ordonna qu’on l’empêchât de mourir. Pressés par les soldats, ses esclaves et ses affranchis lui bandent les bras et arrêtent le sang. On ignore si ce fut à l’insu de Pauline ; car (telle est la malignité du vulgaire) il ne manqua pas de gens qui pensèrent que, tant qu’elle crut Néron inexorable, elle ambitionna le renom d’être morte avec son époux, mais qu’ensuite, flattée d’une plus douce espérance, elle se laissa vaincre aux charmes de la vie. Elle la conserva quelques années seulement, gardant une honorable fidélité à la mémoire de son mari, et montrant assez, par la pâleur de son visage et la blancheur de ses membres, à quel point la force vitale s’était épuisée en elle. Quant à Sénèque, comme le sang coulait péniblement et que la mort était lente à venir, il pria Statius Annéus, qu’il avait reconnu par une longue expérience pour un ami sûr et un habile médecin, de lui apporter le poison dont il s’était pourvu depuis longtemps, le même qu’on emploie dans Athènes contre ceux qu’un jugement public a condamnés à mourir \footnote{Ce poison est la ciguë.}. Sénèque prit en vain ce breuvage : ses membres déjà froids et ses vaisseaux rétrécis se refusaient à l’activité du poison. Enfin il entra dans un bain chaud, et répandit de l’eau sur les esclaves qui l’entouraient, en disant : « J’offre cette libation à Jupiter Libérateur. » Il se fit ensuite porter dans une étuve, dont la vapeur le suffoqua. Son corps fut brûlé sans aucune pompe il l’avait ainsi ordonné par un codicille, lorsque, riche encore et tout-puissant, il s’occupait déjà de sa fin.\par
\textbf{LXV.} Le bruit courut que Subrius, de concert avec les centurions, avait décidé secrètement, mais non pourtant à l’insu de Sénèque, qu’une fois Néron tué par la main de Pison, Pison serait tué à son tour, et l’empire donné à Sénèque, comme à un homme sans reproche, appelé au rang suprême par l’éclat de ses vertus. On débitait même une parole de Subrius : Opprobre pour opprobre, qu’importe un musicien ou un acteur de tragédies ? » car, si Néron jouait de la lyre, Pison déclamait en habit de tragédien.
\subsection[{Autres morts}]{Autres morts}
\noindent \textbf{LXVI.} La complicité des gens de guerre fut enfin découverte. On se lassa de voir Férius poursuivre un crime auquel il avait pris part, et l’indignation lui fit des accusateurs. Pressé par ses questions menaçantes, Scévinus répondit, avec un sourire expressif, que personne n’en savait plus que lui ; et il l’exhorta vivement à payer de reconnaissance un si bon prince. A ces mots Fénius ne trouve plus de voie et pourtant ne veut pas se taire ; il bégaye quelques paroles entrecoupées, qui trahissent sa peur : bientôt il est confondu par les autres et surtout par Cervarius Proculus, chevalier romain, qui l’accusent à l’envi ; et l’empereur donne ordre au soldat Cassius, qu’il tenait près de lui à cause de sa force extraordinaire, de le saisir et de le garrotter.\par
\textbf{LXVII.} Le tribun Subrius périt, dénoncé par les mêmes complices. Il allégua d’abord la différence de ses mœurs et l’impossibilité que lui, homme de guerre, eût concerté un projet si hardi avec des lâches et des efféminés. Ensuite, comme on le pressait, il embrassa la gloire d’un généreux aveu. Interrogé par Néron sur la cause qui avait pu l’entraîner à oubli de son serment : « Je te haïssais, répondit-il : aucun soldat, ne te fut plus fidèle tant que tu méritas d’être aimé ; j’ai commencé à te haïr depuis que tu es devenu assassin de ta mère et de ta femme, cocher, histrion, incendiaire. » J’ai rapporté ses propres paroles, parce qu’elles ne furent pas publiques comme celles de Sénèque, et que cette courageuse et naïve saillie d’une âme guerrière ne méritait pas moins d’être connue. Rien, dans toute cette conjuration, ne blessa plus sensiblement les oreilles du prince, toujours prêt à commettre des crimes, mais peu fait à les entendre appeler par leur nom. Le supplice de Subrius fut confié au tribun Véianus Niger. Comme il faisait creuser la fosse dans un champ voisin, Subrius la trouva trop étroite et trop peu profonde : « Cela même, dit-il aux soldats qui l’entouraient, ils ne savent pas le faire dans les règles. » Averti de tenir la tête ferme : « Que ta main, répliqua-t-il, soit aussi ferme que ma tête ! » En effet, Niger tout tremblant put à peine la lui abattre en deux coups ; il s’en fit auprès de Néron un mérite barbare, en disant qu’il avait tué Subrius une fois et demie.\par
\textbf{LXVIII.} Après Subrius, nul ne montra plus d’intrépidité que le centurion Sulpicius Asper. Néron lui demandant pourquoi il avait conspiré contre sa vie, il dit, pour toute réponse : « qu’on ne pouvait secourir autrement un homme souillé de tant de forfaits," et il marcha au supplice. Les autres centurions subirent aussi la mort sans que leur fermeté se démentît. Mais Fénius n’eut pas le même courage, et il déposa ses lamentations jusque dans son testament. Néron attendait qu’on enveloppât dans l’accusation le consul Vestinus, qu’il regardait comme un homme violent et ennemi de sa personne. Mais les conjurés ne s’étaient point ouverts à Vestinus, quelques-uns à cause d’anciennes inimitiés, le plus grand nombre parce qu’ils ne voyaient en lui qu’un esprit fougueux et insociable. Au reste, la haine de Néron contre Vestinus était née d’une étroite liaison ; où ils avaient appris, l’un à connaître et à mépriser la bassesse du prince, l’autre à craindre la fierté d’un ami dont il avait souvent essuyé les mordantes plaisanteries : or ces jeux d’esprit, quand ils tiennent trop de la vérité, laissent après eux de vifs ressouvenirs. A ces causes de haine s’en joignait une récente : Vestinus venait d’épouser Statilia Messallina, quoiqu’il n’ignorât pas que l’empereur était un de ses amants.\par
\textbf{LXIX.} Comme il ne se découvrait ni crime ni accusateur, Néron, ne pouvant se donner l’apparence d’un juge, eut recours à la force d’un maître. Il envoya le tribun Gérélanus, à la tête d’une cohorte, avec ordre « de prévenir les desseins du consul, d’enlever sa forteresse, de désarmer sa milice. » Vestinus avait en effet une maison qui dominait le Forum, et une troupe d’esclaves bien faits et d’âges assortis. Il avait rempli ce jour-là toutes les fonctions consulaires, et il donnait un festin, sans rien craindre de fâcheux, ou pour mieux dissimuler ses craintes, lorsque les soldats entrèrent et dirent que le tribun le demandait. Il se lève sans tarder, et tout s’achève en un moment. Il est enfermé dans une chambre, un médecin s’y trouve et lui coupe les veines ; encore plein de vie, il est porté au bain et plongé dans l’eau chaude, sans avoir proféré un seul mot où il plaignit son destin. Les convives, environnés de soldats, ne furent rendus à la liberté que bien avant dans la nuit, après que Néron, se représentant l’état de ces malheureux qui attendaient la mort au sortir de table, et riant de leur frayeur, eut dit qu’ils avaient payé assez cher l’honneur de souper chez un consul.\par
\textbf{LXX.} Le prince ordonne ensuite le meurtre de Lucain. Pendant que le sang coulait de ses veines, ce poëte, sentant se refroidir ses pieds et ses mains, et la vie se retirer peu à peu des extrémités, tandis que le cœur conservait encore la chaleur et le sentiment, se ressouvint d’un passage où il avait décrit, avec les mîmes circonstances, la mort d’un soldat blessé, et se mit à réciter les vers : ce furent ses dernières paroles. Sénécion mourut ensuite, puis Quinctianus, puis Scévinus, mieux que ne promettait la mollesse de leur vie. Les autres conjurés périrent à leur tour, sans avoir rien dit ni rien fait de mémorable.\par
\textbf{LXXI.} Cependant la ville se remplissait de funérailles et le Capitole de victimes. A mesure que l’un perdait un fils, l’autre un frère, un parent, un ami, ils rendaient grâce aux dieux, ornaient leurs maisons de laurier, tombaient aux genoux du prince, et fatiguaient sa main de baisers. Néron, qui prenait ces démonstrations pour de la joie, récompensa par l’impunité les promptes révélations de Natalis et de Cervarius. Milichus, comblé de richesses, se décora d’un nom grec qui veut dire Sauveur \footnote{Il prit le surnom de Soter.}. Un des tribuns, Silvanus, quoique absous, se tua de sa main ; un autre, Statius Proximus, avait reçu son pardon de l’empereur : il mourut pour braver sa clémence. Pompéius, Cornélius Martialis, Flavius Népos, Statius Domitius, furent dépouillés du tribunat, sous prétexte que, s’ils n’étaient pas les ennemis du prince, ils passaient pour l’être. Novius Priscus avait été l’ami de Sénèque ; Glitius Gallus et Annius Pollio étaient plus compromis que convaincus : on leur assigna des exils. Priscus y fut suivi d’Antonia Flaccilla sa femme, Gallus d’Égnatia Maximilla. Celle-ci possédait de grands biens qu’on lui laissa d’abord, et qu’on finit par lui ôter ; deux circonstances qui relevèrent également sa gloire. Rufius Crispinus fut aussi exilé : la conjuration servit de prétexte ; le vrai motif, c’est que Néron ne lui pardonnait pas d’avoir été le mari de Poppée. Virginius et Rufus durent leur bannissement à l’éclat de leur nom. Virginius, par ses leçons d’éloquence, Musonius Rufus, en enseignant la philosophie, entretenaient parmi les jeunes gens une émulation suspecte. Cluvidiénus Quiétus, Julius Agrippa, Blitius Catullinus, Pétronius Priscus, Julius Altinus, allèrent, comme une colonie, peupler les îles de la mer Égée. Cadicia, femme de Scévinus, et Césonius Maximus, chassés d’Italie, n’apprirent que par la punition qu’on les avait accusés. Atilla, mère de Lucain, ne fut ni justifiée ni punie : on ne fit pas mention d’elle.\par
\textbf{LXXII.} Toutes ces choses accomplies, Néron fit assembler les soldats et leur distribua deux mille sesterces\footnote{En monnaie actuelle, 367 fr. 62 cent.} à chacun en ordonnant de plus que le blé, qu’ils avaient payé jusqu’alors au prix du commerce, leur fût livré gratuitement. Ensuite, comme s’il eût eu à rendre compte de quelque exploit guerrier, il convoque le sénat, et il donne les ornements du triomphe au consulaire Pétronius Turpilianus, à Coccéius Nerva \footnote{Le même qui depuis fut empereur. Entre cette dignité suprême et les honneurs qu’il reçoit de Néron, il éprouva les malheurs de l’exil : il fut relégué à Tarente par Domitien qui le soupçonnait de conspirer contra lui.}, préteur désigné, au préfet du prétoire Tigellin ; si prodigue d’honneurs pour Tigellin et Nerva, qu’outre les statues triomphales qui leur furent érigées au Forum, il plaça encore leurs images dans le palais. Nymphidius reçut les décorations consulaires. Comme il s’offre pour la première fois dans mes récits, j’en dirai quelques mots ; car ce sera aussi l’un des fléaux de Rome. Né d’une affranchie qui prostitua sa beauté aux esclaves et aux affranchis des princes, il se prétendait fils de l’empereur Caïus, parce que le hasard lui avait donné sa haute stature et son regard farouche : ou peut-être Caïus, qui descendait jusques aux courtisanes, avait-il porté chez la mère de cet homme ses brutales fantaisies.\par
\textbf{LXXIII.} Non content d’avoir assemblé le sénat et harangué les pères conscrits, Néron adressa au peuple un édit, auquel était joint le recueil de toutes les dépositions, avec les aveux des condamnés. C’était une réponse aux bruits populaires qui l’accusaient d’avoir tué des innocents par envie ou par crainte. Au reste, qu’une conjuration ait été formée, mûrie, étouffée, ceux qui cherchaient la vérité de bonne foi n’en doutèrent pas alors ; et les exilés revenus à Rome après la mort de Néron l’ont eux-mêmes reconnu. Dans le sénat, tout le monde était aux pieds du prince, et les plus affligés flattaient plus que les autres. Comme Junius Gallio, effrayé de la mort de Sénèque, son frère, demandait grâce pour lui-même, Saliénus Clémens, se déchaînant contre lui, le traita d’ennemi et de parricide. Il fallut, pour arrêter Saliénus, que le sénat tout entier le conjurât « de ne pas laisser croire qu’il abusait des malheurs publics au profit de ses haines particulières, et de ne pas ramenez sous le glaive ce que la clémence du prince avait couvert de la paix ou de l’oubli. »\par
\textbf{LXXIV.} On décerna ensuite des offrandes et des actions de grâces aux dieux, avec des hommages particuliers au Soleil, qui a, prés du Cirque, où devait s’exécuter le crime, un ancien temple et dont la providence avait dévoilé ces mystérieux complots. II fut décidé qu’aux jeux célébrés dans le Cirque en l’honneur de Cérès on ajouterait de nouvelles courses de chars, que le nom de Néron serait donné au mois d’avril, et un temple élevé à la déesse Salus, au lieu même d’où Scévinus avait tiré son poignard. Néron consacra cette arme dans le Capitole, avec l’inscription : A JUPITER VINDEX. Ces mots ne furent pas remarqués d’abord : après le soulèvement de Julius Vindex, on y vit l’annonce et le présage d’une future vengeance. Je trouve dans les actes du sénat que le consul désigné Cérialis Anicius avait opiné pour que l’État fît au plus tôt bâtir à ses frais un temple au dieu Néron, hommage qu’il lui décernait sans doute comme à un héros élevé au-dessus de la condition humaine et digne de l’adoration des peuples, mais qu’on pouvait un jour interpréter comme un pronostic de sa mort : car on ne rend aux princes les honneurs des dieux que quand ils ont cessé d’habiter parmi les hommes.
\section[{Livre seizième (66)}]{Livre seizième (66)}\renewcommand{\leftmark}{Livre seizième (66)}

\subsection[{Le trésor de Didon}]{Le trésor de Didon}
\noindent \labelchar{I.} Bientôt la fortune se joua de Néron, abusé par sa propre crédulité et par les promesses du Carthaginois Césellius Bassus. Cet homme, d’une imagination mal réglée, avait pris pour un oracle infaillible l’illusion d’un songe. Il vient à Rome, achète une audience du prince, lui expose « qu’il a découvert dans son champ un souterrain d’une profondeur immense, renfermant une grande quantité d’or non monnayé, dont les masses brutes annonçaient la plus antique origine. C’étaient d’un côté d’énormes lingots entassés par terre, tandis que de l’autre côté l’or s’élevait en colonnes : trésors enfouis depuis tant de siècles pour accroître les prospérités de l’âge présent. Nul doute, au reste, que ce ne fût la Phénicienne Didon qui, après sa fuite de Tyr et la fondation de Carthage, avait caché ces richesses, de peur qu’un peuple naissant ne fût amolli par trop d’opulence, ou que les rois numides, déjà ses ennemis, ne fussent entraînés par la soif de l’or à s’armer contre elle. »\par
\labelchar{II.} Néron, sans examiner quelle foi méritait fauteur de ce récit ou le récit même, et sans charger personne d’aller reconnaître si on lui annonçait la vérité, accrédite le premier cette nouvelle, et envoie chercher une proie qu’il croit déjà tenir. Afin d’accélérer le voyage, il donne des galères avec des équipages choisis. Ce fut, dans ce temps-là, l’unique objet des crédules entretiens de la foule et des réflexions toutes contraires des gens éclairés. Comme on célébrait alors les secondes Quinquennales, les orateurs tirèrent de ce fonds les principaux ornements de leurs panégyriques : « C’était peu, disaient-ils, que la terre se couvrît de moissons et engendrât ces minerais où l’or est enveloppé ; elle ouvrait les sources d’une fécondité nouvelle, et les biens s’offraient d’eux-mêmes, apportés par la main des dieux ; " serviles inventions, qu’avec beaucoup d’éloquence et non moins de bassesse ils variaient à l’infini, sûrs de trouver auprès de Néron une croyance facile.\par
\labelchar{III.} Cependant, sur ce frivole espoir, le luxe allait croissant, et l’on épuisait les anciens trésors, dans l’idée qu’il s’en offrait un nouveau qui suffirait aux profusions d’un grand nombre d’années. Néron donnait même déjà sur ce fonds, et l’attente des richesses fut une des causes de la pauvreté publique. Bossus fouilla son champ et tous ceux d’alentour, assurant qu’à telle place, puis à telle autre, était la caverne promise, et suivi non-seulement d’une troupe de soldats, mais de tout le peuple des campagnes, appelé à ce travail. Enfin, revenu de son délire, et ne pouvant concevoir que ses rêves, jusqu’alors infaillibles, l’eussent une fois trompé, il se déroba par une mort volontaire à la honte et à la crainte. Quelques-uns rapportent qu’il fut mis en prison, puis relâché, et qu’on lui prit ses biens pour tenir lieu des trésors de Didon.
\subsection[{Néron aux jeux quinquennaux}]{Néron aux jeux quinquennaux}
\noindent \labelchar{IV.} A l’approche des jeux quinquennaux, le sénat, pour sauver l’honneur, offrit au prince la victoire du chant ; il y joignait la couronne de l’éloquence, qui devait couvrir là honte d’une palme théâtrale ; mais Néron déclara « que ni la brigue ni l’autorité du sénat ne lui étaient nécessaires ; que tout serait égal entre lui et ses rivaux, et qu’il devrait à la religion des juges le triomphe qu’il aurait mérité. » Il commence par réciter des vers sur la scène. Bientôt, pressé par la multitude « de faire jouir le public de tous ses talents » (ce furent leurs expressions), il s’avance sur le théâtre, en obéissant à toutes les lois prescrites pour les combats de la lyre, comme de ne pas s’asseoir, quelque fatigué qu’il pût être, de n’essuyer la sueur de son front qu’avec la robe qu’il portait, de ne point cracher ni se moucher à la vue des spectateurs. Enfin il fléchit le genou, et, saluant respectueusement de la main une telle assemblée, il attendait avec une feinte anxiété la décision des juges. La populace de Rome, prodigue d’encouragements même pour des histrions, faisait entendre des acclamations notées et applaudissait en mesure. On eût dit qu’elle était joyeuse, et peut-être se réjouissait-elle en effet, dans sa profonde insouciance du déshonneur public.\par
\labelchar{V.} Mais ceux qui étaient venus des villes éloignées, où l’on retrouve encore la sévère Italie avec ses mœurs antiques, et ceux qu’une mission publique ou leurs affaires particulières avaient amenés du fond des provinces, où une telle licence est inconnue, ne pouvaient ni soutenir cet aspect, ni suffire à cette indigne tâche. Leurs mains ignorantes tombaient de lassitude et troublaient les habiles. Aussi étaient-ils souvent frappés par les soldats qui, debout entre les gradins, veillaient à ce qu’il n’y eût pas un moment d’inégalité ni de relâche dans les acclamations. C’est un fait constant que beaucoup de chevaliers, en voulant traverser les flots impétueux de la multitude, furent écrasés dans les passages trop étroits, et que d’autres, à force de rester jour et nuit sur leurs sièges, furent atteints de maladies mortelles : mais ce danger les effrayait moins que celui de l’absence, des gens apostés, les uns publiquement, un plus grand nombre en secret, remarquant les noms et les visages, la tristesse ou la gaieté de chaque spectateur. Les accusés vulgaires étaient aussitôt livrés au supplice : avec ceux d’un plus haut rang, la haine, un moment dissimulée, était une dette qui se payait plus tard. Vespasien, dit-on, durement réprimandé par l’affranchi Phébus, sous prétexte que le sommeil lui fermait les yeux, fut sauvé, non sans peine, par les prières des gens de bien, et, s’il échappa depuis à la perte qui l’attendait, c’est qu’un destin plus fort le protégea.
\subsection[{La mort de Poppée}]{La mort de Poppée}
\noindent \labelchar{VI.} Après la fin des jeux mourut Poppée, victime d’un emportement de son époux, dont elle reçut, étant enceinte, un violent coup de pied ; car je ne crois pas au poison, dont plusieurs écrivains ont parlé, moins par conviction que par haine : Néron désirait des enfants et il avait le cœur vivement épris de sa femme. Le corps de Poppée ne fut point consumé par le feu, suivant l’usage romain ; il fut embaumé à la manière des rois étrangers, et porté dans le tombeau des Jules. On lui fit cependant des funérailles publiques, et le prince, du haut de la tribune, loua la beauté de ses traits, la divinité de l’enfant dont elle avait été mère, et les autres dons de la fortune, ses uniques vertus.
\subsection[{Cassius et Silanus}]{Cassius et Silanus}
\noindent \labelchar{VII.} L’odieux de la mort de Poppée (qui d’ailleurs, pleurée en public, inspirait une joie secrète à cause de la barbarie et de l’impudicité de cette femme) fut porté au comble par la défense que Néron fit à C. Cassius de paraître aux funérailles. Ce fut le premier signe de l’orage, qui ne tarda pas à éclater. Silanus fut associé à sa perte. Tout leur crime était de briller entre les Romains, Cassius par son opulence héréditaire et la gravité de ses mœurs, Silanus par une naissance illustre et une jeunesse sagement réglée. Néron, dans un discours envoyé au sénat, exposa qu’il fallait soustraire la république à l’influence de ces deux hommes. Il reprochait au premier d’honorer parmi les images de ses aïeux celle de l’ancien Cassius, qui portait cette inscription : LE CHEF DU PARTI. « Cassius, disait-il, jetait ainsi des semences de guerre civile, et appelait la révolte contre la maison des Césars. Et, non content de réveiller la mémoire d’un nom ennemi pour allumer la discorde, il s’était associé Silanus, jeune homme d’une naissance noble et d’un esprit aventureux, afin de le montrer à la rébellion. »\par
\labelchar{VIII.} Passant à Silanus lui-même, Néron l’accusa, comme son oncle Torquatus, « de préluder ambitieusement aux soins de l’empire ; de faire ses affranchis trésoriers, secrétaires, maîtres des requêtes :" imputations aussi fausses que frivoles ; car la crainte tenait Silanus sur ses gardes, et le malheur de son oncle lui avait, par une leçon terrible, enseigné la prudence. Néron fit paraître de prétendus témoins qui chargèrent Lépida, femme de Cassius, tante paternelle de Silanus, d’un inceste avec son neveu et de sacrifices magiques. On lui donnait pour complices Vulcatius Tullinus et Marcellus Cornélius, sénateurs, Calpurnius Fabatus \footnote{Aïeul de la femme de Pline le Jeune.}, chevalier romain. Ceux-ci éludèrent, par un appel au prince, la condamnation qui les menaçait ; et Néron, occupé de forfaits plus importants, négligea de si obscures victimes.\par
\labelchar{IX.} Cependant le sénat prononce l’exil de Cassius et de Silanus, et renvoie au prince le jugement de Lépida. Cassius fut déporté dans l’île de Sardaigne ; on se reposait du reste sur sa vieillesse. Silanus, sous prétexte qu’on devait l’embarquer pour Naxos, est conduit à Ostie, puis enfermé dans un municipe d’Apulie nommé Barium. II y supportait en sage l’indignité de son sort, lorsqu’un centurion, envoyé pour le tuer, le saisit tout à coup. Cet homme l’engageant à se laisser ouvrir les veines, il lui répondit « que sa mort était résolue dans le fond de son âme, mais qu’il n’en donnerait pas la gloire à la main d’un bourreau. » Il était sans armes, et toutefois sa vigueur et son air, où se peignait plus de colère que de crainte, intimidèrent le centurion, qui donna ordre aux soldats de se jeter sur lui. Silanus ne cessa de se défendre et de frapper lui-même autant qu’il le pouvait de ses bras désarmés, jusqu’au moment où, accablé par le centurion de blessures toutes reçues par devant, il tomba comme sur un champ de bataille.
\subsection[{Mort de L. Vétus}]{Mort de L. Vétus}
\noindent \labelchar{X.} Ce ne fut pas avec moins de courage que L. Vétus, sa belle-mère Sextia, et Pollutia sa fille, subirent le trépas. Ils étaient odieux au prince, auquel leur vie semblait reprocher le meurtre de Rubellius Plautus, gendre de Vétus. Mais cette haine, pour éclater, attendait une occasion ; l’affranchi Fortunatus la fournit, en accusant son maître, après l’avoir ruiné. Il se fit appuyer de Claudius Démianus, que Vétus, étant proconsul d’Asie, avait emprisonné pour ses crimes, et que Néron mit en liberté pour prix de sa délation. Vétus, apprenant qu’il était accusé, et qu’il fallait lutter d’égal à égal avec son affranchi, se retira dans sa terre de Formies, où des soldats le gardèrent secrètement à vue. Sa fille était avec lui, doublement exaspérée par le danger présent et par la longue douleur qui obsédait son âme depuis qu’elle avait vu les assassins de son mari Plautus, et qu’elle avait reçu dans ses bras sa tête ensanglantée. Ce sang et les vêtements qu’il avait arrosés, elle les conservait précieusement, veuve inconsolable, toujours enveloppée de deuil, et ne prenant de nourriture que pour ne pas mourir. A la prière de son père, elle se rend à Naples : là, privée de tout accès auprès de Néron, elle épiait ses sorties et lui criait à son passage « d’entendre un innocent, de ne pas livrer à la merci de son esclave un homme qui fut consul avec le prince. » Elle continua ses cris, tantôt avec l’accent d’une femme au désespoir, tantôt avec une énergie toute virile et d’une voix indignée, jusqu’à ce qu’elle vit qu’il n’était ni larmes ni reproches qui pussent émouvoir Néron.\par
\labelchar{XI.} Elle revient donc, et annonce à son père qu’il faut abandonner l’espérance et se soumettre à la nécessité. On apprend en même temps que le procès va s’instruire devant le sénat, et qu’un arrêt cruel se prépare. Plusieurs conseillèrent à Vétus de nommer le prince héritier d’une grande partie de sa fortune, afin d’assurer le reste à ses petits-fils. Vétus s’y refusa, pour ne pas flétrir, par une fin servile, une vie passée avec quelque indépendance. Il distribue ce qu’il avait d’argent à ses esclaves et leur ordonne de partager entre eux ce qui peut s’emporter, réservant seulement trois lits pour autant de funérailles. Alors tous trois, dans la même chambre, avec le même fer, s’ouvrent les veines, et aussitôt, couverts pour la décence d’un seul vêtement chacun, ils se font porter au bain. Là, tenant les yeux attachés, le père et l’aïeule sur leur fille, la fille sur son aïeule et son père, ils souhaitaient à l’envi que leur âme achevât promptement de s’exhaler, afin de laisser les objets de leur tendresse encore vivants, quoique si près de mourir. Le sort garda l’ordre de la nature : la plus âgée expira d’abord ; la plus jeune s’éteignit la dernière. Mis en jugement après leurs funérailles, ils furent condamnés au genre de supplice usité chez nos ancêtres, et Néron, intervenant, leur permit un trépas de leur choix : c’est ainsi qu’à des meurtres consommés on ajoutait la dérision.\par
\labelchar{XII.} L. Gallus, chevalier romain, intime ami de Fénius, et qui n’avait pas été sans liaisons avec Vétus, fut puni par l’interdiction du feu et de l’eau. L’affranchi et l’accusateur eurent, pour salaire, une place au théâtre parmi les viateurs des tribuns \footnote{Les fonctions du viateur consistaient particulièrement à accompagner les tribuns et les édiles.}. Le mois d’avril portait déjà le nom de Néron ; on donna celui de Claudius à mai, celui de Germanicus à juin. Cornélius Orfitus, qui proposa ces changements, protestait que, s’il ne voulait pas qu’un mois s’appelât juin, c’était parce que déjà deux Torquatus, mis à mort pour leurs crimes, avaient rendu sinistre le nom de Junius.
\subsection[{Épidémies et catastrophes}]{Épidémies et catastrophes}
\noindent \labelchar{XIII.} Cette année souillée de tant de forfaits, les dieux la signalèrent encore par les tempêtes et les épidémies. La Campanie fut ravagée par un ouragan qui emporta métairies, arbres, moissons. Ce fléau promena sa violence jusqu’aux portes de Rome, tandis qu’au dedans une affreuse contagion étendait ses ravages sur tout ce qui respire. On ne voyait aucun signe de corruption dans l’air, et cependant les maisons se remplissaient de cadavres, les rues de funérailles : ni sexe, ni âge n’échappait au péril ; la multitude, esclave ou libre, était moissonnée avec une égale rapidité ; ils expiraient au milieu des lamentations de leurs femmes et de leurs enfants, qui, frappés à leur chevet, atteints en pleurant leur trépas, étaient souvent brûlés sur le même bûcher. Les morts des chevaliers et des sénateurs, quoique aussi nombreuses, étaient moins déplorables : la mortalité commune semblait les dérober à la cruauté du prince. La même année on fit des levées dans la Gaule narbonnaise, dans l’Asie et dans l’Afrique, afin de recruter les légions d’Illyrie, d’où l’on congédia les soldats fatigués par l’âge où les infirmités. Le prince soulagea le désastre de Lyon par le don de quatre millions de sesterces \footnote{La colonie de Lugdunum (Lyon) fut fondée par Munatius Plancus, l’an de Rome 711, sur la hauteur de Fourvière, qui n’est aujourd’hui que la moindre partie de cette grande cité. Cent ans après, l’an 811, elle fut entièrement détruite par un incendie, qui fait le sujet de la Lettre XCIe de Sénèque. Il s’était donc écoulé sept ans entre le désastre de cette ville et le moment où Néron vint à son secours. – Quatre millions de sesterces équivalent à 735 239 fr. 20 c.}, qu’il fit à la ville pour relever ses ruines ; les Lyonnais nous avaient eux-mêmes offert cette somme dans des temps malheureux.
\subsection[{Délation}]{Délation}
\noindent \labelchar{XIV.} C. Suétonius et L. Télésinus étant consuls, Antistius Sosianus, exilé, comme je l’ai dit, pour avoir fait contre Néron des vers satiriques, fut tenté par les récompenses prodiguées aux délateurs et la facilité du prince à répandre le sang. Il y avait dans le même lieu un autre exilé, Pammène, fameux dans l’art des Chaldéens, et, à ce titre, engagé dans une infinité de liaisons. Sosianus, esprit remuant, et que l’occasion trouvait toujours prêt, tira parti de la conformité de leur sort pour gagner sa confiance. Persuadé que ce n’était pas sans quelque motif qu’à chaque instant Pammène recevait des messages, donnait des consultations, il apprit en outre que P. Antéius lui faisait une pension annuelle. Or, il n’ignorait pas qu’Antéius était haï de Néron comme ami d’Agrippine, que son opulence était faite pour éveiller la cupidité, et qu’une cause pareille était fatale à beaucoup d’autres. Il intercepta donc une lettre d’Antéius, et déroba des papiers secrets où Pamméne avait tracé l’horoscope de cet homme ; il trouva aussi les calculs de l’astrologue sur la naissance et la vie d’Ostorius Scapula. Aussitôt il écrit à Néron « que, s’il veut suspendre un moment son exil, il lui révélera des secrets importants, où la sûreté de sa personne est intéressée ; qu’Antéius et Ostorius ont des vues sur l’empire ; qu’ils s’enquièrent de leurs destinées et de celles du prince. » Des galères sont envoyées, et Sosianus amené en toute hâte. Sa délation fut à peine connue que déjà on voyait dans Antéius et Ostorius moins des accusés que des condamnés. Personne n’aurait même scellé le testament d’Antéius, si Tigellin n’eut autorisé cette hardiesse : il l’avait averti auparavant de ne pas différer ses dernières dispositions. Antéius prit du poison ; puis, fatigué d’en attendre l’effet, il hâta son trépas en se coupant les veines.\par
\labelchar{XV.} Ostorius était alors dans une terre éloignée, sur la frontière de Ligurie. Un centurion fut chargé d’aller sans retard lui porter la mort. Cette précipitation avait ses motifs : environné d’une grande réputation militaire, et décoré d’une couronne civique méritée en Bretagne, Ostorius intimidait Néron par la force prodigieuse de son corps et son adresse à manier les armes. Le prince croyait déjà s’en voir assailli, de tout temps sujet à la peur, mais plus effrayé que jamais depuis la dernière conjuration. Le centurion, après avoir fermé toutes les issues de la maison d’Ostorius, lui signifie un ordre de l’empereur, Celui-ci tourne alors contre lui-même un courage si souvent éprouvé contre les ennemis. Comme ses veines, quoiqu’elles fussent ouvertes, laissaient couler peu de sang, il eut recours à un esclave, dont il exigea, pour tout service, qu’il tint d’un bras ferme un poignard levé ; puis il lui saisit la main et de sa gorge alla chercher le fer.
\subsection[{Digression de Tacite}]{Digression de Tacite}
\noindent \labelchar{XVI.} Quand même des guerres étrangères et des morts courageusement reçues pour la république seraient l’objet de mes récits, la constante uniformité des événements m’aurait lassé moi-même, et je n’attendrais du lecteur qu’un dédaigneux ennui, à la vue de ces honorables mais tristes et continuels trépas. Combien plus cette soumission passive et ces flots de sang perdus en pleine paix fatiguent l’âme et serrent péniblement le cœur Un mot sera ma seule apologie : que ceux qui liront ces pages me permettent de ne pas haïr des victimes si lâchement résignées. La colère des dieux sur les Romains se déclarait par des exemples qu’on ne peut, comme la défaite d’une armée ou la prise d’une ville, raconter une fois et passer outre. Accordons ce privilège aux rejetons des grandes familles, que, si la pompe de leurs funérailles les distingue de la foule, l’histoire consacre aussi à leurs moments suprêmes une mention particulière.
\subsection[{Délation (suite)}]{Délation (suite)}
\noindent \labelchar{XVII.} Dans l’espace de peu de jours, tombèrent coup sut coup Annéus Mella, Cérialis Anicius, Rufius Crispinus et C. Pétronius. Mella et Crispinus étaient deux chevaliers romains de rang sénatorial \footnote{Chevaliers romains qui avaient le cens nécessaire pour devenir sénateurs et le droit de porter le laticlave.}. Crispinus, ancien préfet des gardes prétoriennes et décoré des ornements consulaires, venait d’être relégué en Sardaigne, comme complice de Pison. En apprenant l’ordre de sa mort, il se tua lui-même. Mella, né des mêmes parents que Gallion et Sénèque, s’était abstenu de briguer les honneurs ; ambitieux à sa manière, et voulant égaler, simple chevalier romain, le crédit des consulaires : il croyait d’ailleurs que l’administration des biens du prince était, pour aller à la fortune, le chemin le plus court. C’était lui qui avait donné le jour à Lucain, ce qui ajoutait beaucoup à l’éclat de son nom. Après la mort de celui-ci, la recherche exacte et empressée qu’il fit de ses biens lui attira un accusateur, Fabius Romanus, intime ami du poëte. On supposa le père initié par son fils au secret de la conjuration ; et l’on produisit une fausse lettre de Lucain. Néron, après l’avoir lue, ordonna qu’elle fût portée à Mella, dont il convoitait les richesses. Mella choisit, pour mourir, la voie que tout le monde prenait alors : il se coupa les veines, après avoir fait un codicille où il léguait à Tigellin et au gendre de Tigellin, Cossutianus Capito, une grande somme d’argent, afin de sauver le reste. Une phrase fut ajoutée par laquelle on lui faisait dire, comme pour accuser l’injustice de son sort, « qu’il périssait le moins coupable des hommes, tandis que Rufius Crispinus et Anicius Cérialis jouissaient de la vie, quoique ennemis du prince. » On crut ce trait forgé contre Crispinus, parce qu’il était mort, contre Cérialis, afin qu’il mourût ; car peu de temps après il mit fin à ses jours, moins plaint toutefois que les autres : on se souvenait qu’il avait livré à Caïus le secret d’une conjuration.
\subsection[{Mort de Pétrone}]{Mort de Pétrone}
\noindent \labelchar{XVIII.} Je reprendrai d’un peu plus haut ce qui regarde Pétrone. Il consacrait le jour au sommeil, la nuit aux devoirs et aux agréments de la vie. Si d’autres vont à la renommée par le travail, il y alla par la mollesse. Et il n’avait pas la réputation d’un homme abîmé dans la débauche, comme la plupart des dissipateurs, mais celle d’un voluptueux qui se connaît en plaisirs. L’insouciance même et l’abandon qui paraissait dans ses actions et dans ses paroles leur donnait un air de simplicité d’où elles tiraient une grâce nouvelle. On le vit cependant, proconsul en Bithynie et ensuite consul, faire preuve de vigueur et de capacité. Puis retourné aux vices, ou à l’imitation calculée des vices, il fut admis à la cour parmi les favoris de prédilection. Là, il était l’arbitre du bon goût : rien d’agréable, rien de délicat, pour un prince embarrassé du choix, que ce qui lui était recommandé par le suffrage de Pétrone. Tigellin fut jaloux de cette faveur : il crut avoir un rival plus habile que lui dans la science des voluptés. Il s’adresse donc à la cruauté du prince, contre laquelle ne tenaient jamais les autres passions, et signala Pétrone comme ami de Scévinus : un délateur avait été acheté parmi ses esclaves, la plus grande partie des autres jetés dans les fers, et la défense interdite à l’accusé.\par
\labelchar{XIX.} L’empereur se trouvait alors en Campanie, et Pétrone l’avait suivi jusques à Cumes, où il eut ordre de rester. Il ne soutint pas l’idée de languir entre la crainte et l’espérance ; et toutefois il ne voulut pas rejeter brusquement la vie. Il s’ouvrit les veines, puis les referma, puis les ouvrit de nouveau, parlant à ses amis et les écoutant à leur tour : mais dans ses propos, rien de sérieux, nulle ostentation de courage ; et, de leur côté, point de réflexions sur l’immortalité de l’âme et les maximes des philosophes ; il ne voulait entendre que des vers badins et des poésies légères. Il récompensa quelques esclaves, en fit châtier d’autres ; il sortit même ; il se livra au sommeil, afin que sa mort, quoique forcée, parût naturelle. Il ne chercha point, comme la plupart de ceux qui périssaient, à flatter par son codicille ou Néron, ou Tigellin, ou quelque autre des puissants du jour. Mais, sous les noms de jeunes impudiques et de femmes perdues, il traça le récit des débauches du prince, avec leurs plus monstrueuses recherches, et lui envoya cet écrit cacheté : puis il brisa son anneau, de peur qu’il ne servît plus tard à faire des victimes.
\subsection[{Autres meurtres}]{Autres meurtres}
\noindent \labelchar{XX.} Néron cherchait comment avaient pu être divulgués les mystères de ses nuits. Silia s’offrit à sa pensée : épouse d’un sénateur, ce n’était point une femme inconnue ; elle servait d’instrument à la lubricité du prince, et d’étroites liaisons l’avaient unie à Pétrone. Elle fut exilée, comme n’ayant pas su taire ce qu’elle avait vu et enduré. Néron l’avait sacrifiée à sa propre haine : il livra Minucius Thermus, ancien préteur, aux ressentiments de Tigellin, contre lequel un affranchi de Thermus avait hasardé quelques accusations. L’affranchi expia son indiscrétion par d’horribles tortures, et le maître, qui en était innocent, la paya de sa tête.
\subsection[{Et maintenant Thraséas}]{Et maintenant Thraséas}
\noindent \labelchar{XXI.} Après avoir massacré tant d’hommes distingués, Néron voulut à la fin exterminer la vertu même, en immolant Pétus Thraséas et Baréa Soranus. Tous deux il les haïssait depuis longtemps ; mais Thraséas avait à sa vengeance des titres particuliers. Il était sorti du sénat, comme je l’ai dit, pendant la délibération qui suivit la mort d’Agrippine. Aux représentations des Juvénales, il n’avait pas fait voir un zèle assez empressé, offense d’autant plus sensible à Néron, que le même Thraséas, étant à Padoue, sa patrie, aux jeux du Ceste institués par le Troyen Anténor \footnote{Tout le monde connaît la tradition qui attribuait su Troyen Anténor la fondation de Patavium, ou Padoue. Les jeux qu’on y célébrait se rattachaient donc aux plus anciens souvenirs de la patrie, et avaient quelque chose de national et de religieux à la fois.}, avait chanté sur la scène en costume tragique. De plus, le jour où le préteur Antistius allait être condamné à mort pour une satire contre le prince, il avait proposé et fait prévaloir un avis plus doux. Enfin, lorsqu’on décerna les honneurs divins à Poppée, il s’était absenté volontairement, et n’avait point paru aux funérailles. C’étaient autant de souvenirs que ne laissait pas tomber Cossutianus Capito, esprit naturellement pervers, et de plus ennemi de Thraséas, dont le suffrage avait entraîné sa condamnation, quand les députés ciliciens vinrent l’accuser de rapines.\par
\labelchar{XXII.} Il lui trouvait encore d’autres crimes : « Au commencement de l’année, Thraséas évitait le serment solennel ; il n’assistait pas aux vœux pour l’empereur, quoiqu’il fût revêtu du sacerdoce des quindécemvirs ; jamais il n’avait offert de victimes pour le salut du prince ni pour sa voix céleste. Assidu jadis et infatigable à défendre ou à repousser les moindres sénatus-consultes, il n’avait pas, depuis trois ans, mis le pied dans le sénat. Dernièrement encore, pendant qu’on y courait à l’envi pour réprimer les complots de Silanus et de Vétus, il avait donné la préférence aux intérêts particuliers de ses clients. N’était-ce pas là une scission prononcée, un parti, et, si beaucoup imitaient cette audace, une guerre ? Il fut un temps où c’était de Caton et de César, aujourd’hui, Néron, c’est de Thraséas et de toi que s’entretient un peuple avide de discordes. Thraséas a des sectateurs, ou plutôt des satellites, qui, sans se permettre encore ses votes séditieux, copient déjà son air et son maintien ; gens qui se font rigides et austères, afin de te reprocher une vie dissolue. Lui seul n’a pas une pensée pour ta conservation, pas un hommage pour tes talents. Les succès du prince excitent ses mépris : faut-il encore que son deuil et ses douleurs ne puissent rassasier sa haine ? Ne pas croire à la divinité de Poppée vient du même esprit que de ne pas jurer sur les actes des dieux Jules et Auguste. Il brave la religion, anéantit les lois. Les armées, les provinces, lisent les journaux du peuple romain avec un redoublement de curiosité, afin de savoir ce que Thraséas n’a pas fait. De deux choses l’une : embrassons les maximes qu’il professe, si elles valent mieux que les nôtres ; ou ôtons aux partisans des nouveautés leur chef et leur instigateur. Cette secte a produit les Tubéron et les Favonius, noms qui déplurent même à l’ancienne république. Pour renverser le pouvoir, ils parlent de liberté ; le pouvoir abattu, ils attaqueront la liberté elle-même. Vainement tu as éloigné Cassius, si tu laisses les émules des Brutus se multiplier et marcher tête levée. Enfin, César, n’écris rien toi-même sur Thraséas : que le sénat soit juge entre lui et moi. » Le prince échauffe encore l’esprit de Capiton, si animé déjà par la colère, et lui adjoint Éprius Marcellus, orateur d’une fougueuse énergie.
\subsection[{Et puis Soranus}]{Et puis Soranus}
\noindent \labelchar{XXIII.} Quant à Baréa Soranus, son accusation avait été retenue par Ostorius Sabinus, chevalier romain, lorsqu’à peine il sortait du proconsulat d’Asie, où plusieurs causes avaient augmenté contre lui les mécontentements du prince ; d’abord sa justice et son activité, puis le soin qu’il avait pris d’ouvrir le port d’Éphèse, enfin l’impunité laissée à la ville de Pergame, qui avait empêché par la force Acratus, affranchi de l’empereur, d’enlever ses statues et ses tableaux. Le prétexte de l’accusation fut l’amitié de Plautus et la recherche d’une séditieuse popularité. On choisit, pour les deux condamnations, le temps où Tiridate venait recevoir la couronne d’Arménie, afin que, tout entière aux choses étrangères, l’attention publique remarquât moins ce crime domestique : ou peut-être Néron, en frappant des têtes illustres, croyait-il faire éclater, par des coups dignes d’un roi, la grandeur impériale.\par
\labelchar{XXIV.} Au moment où toute la population se précipitait hors de Rome, pour aller au-devant de César et contempler le monarque, Thraséas reçut défense d’approcher. Son courage n’en fut point abattu : il écrivit à Néron, lui demandant quel était son crime, et promettant de se justifier pleinement, si on lui donnait connaissance de l’accusation et liberté de répondre, Néron lut avidement cette lettre, dans l’espérance que Thraséas effrayé aurait écrit des choses où triompherait la vanité du prince, et qui déshonoreraient leur auteur. Trompé dans son attente, il redouta lui-même les regards, la fierté, la libre franchise d’un innocent, et fit assembler le sénat. Thraséas, entouré de ses amis, délibéra s’il essayerait ou dédaignerait de se défendre : les avis furent partagés.\par
\labelchar{XXV.} Ceux qui lui conseillaient d’aller au sénat, « étaient, disaient-ils, sûrs de sa fermeté. Il ne prononcerait pas une parole qui n’augmentât sa gloire. Les faibles seuls et les timides environnaient de secret leurs derniers moments. Il fallait que le peuple vît un homme courageux en face de la mort ; que le sénat entendit les oracles d’une voix plus qu’humaine. Ce prodige pouvait ébranler jusqu’à Néron ; mais, s’obstinât-il dans sa cruauté, la postérité distinguerait au moins le brave qui honore son trépas, du lâche qui périt en silence. »\par
\labelchar{XXVI.} D’autres voulaient qu’il attendît chez lui.."De son courage, ils n’en doutaient pas ; mais que d’outrages et d’humiliations il aurait à subir ! Ils lui conseillaient de dérober son oreille à l’invective et à l’injure. Marcellus et Capiton n’étaient pas seuls voués au crime ; trop de méchants étaient capables de se jeter sur lui dans leur brutale furie ; et la peur n’entraînait-elle pas jusqu’aux bons ? Ah ! que plutôt il épargnât au sénat, dont il avait été l’ornement, la honte d’un si grand forfait, et qu’il laissât incertain ce qu’auraient décidé les pères conscrits à la vue de Thraséas accusé ! Croire que Néron pût rougir de ses crimes, c’était se flatter d’un chimérique espoir. Combien plus il fallait craindre que la femme de Thraséas, sa famille, tous les objets de sa tendresse, ne périssent à leur tour ! Qu’il finît donc ses jours, sans qu’aucun affront eût profané sa vertu ; et que la gloire des sages dont les exemples et les maximes avaient guidé sa vie éclatât en sa mort. » A ce conseil assistait Rusticus Arulénus \footnote{Il était préteur lorsque la sanglante querelle entre Vitellius et Vespasien se vida aux portes de Rome et dans le sein même de la ville (Hist., liv. III, ch. LXXX). Il écrivit la vie de Thraséas, qu’il se faisait gloire de prendre pour modèle, et ce courage lui valut la mort : il fut condamné sous Domitien ; et le délateur Régulus, non content d’avoir contribué à sa perte, insulta sa mémoire dans un écrit public, où il le traitait de singe des stoïciens.}, jeune homme ardent, qui, par amour de la gloire, offrit de s’opposer au sénatus-consulte ; car il était tribun du peuple. Thraséas retint son élan généreux, et le détourna d’une entreprise vaine, et qui, sans fruit pour l’accusé, serait fatale au tribun. Il ajouta « que sa carrière était achevée, et qu’il ne pouvait abandonner les principes de toute sa vie ; que Rusticus, au contraire, débutait dans les magistratures, et que tout l’avenir était à lui ; qu’il se consultât longtemps sur la route politique où, dans un tel siècle, il lui convenait d’entrer. » Quant à la question s’il devait aller au sénat, il se réserva d’y songer encore.\par
\labelchar{XXVII.} Le lendemain, au lever du jour, deux cohortes prétoriennes sous les armes investirent le temple de Vénus Génitrix ; un gros d’hommes en toge, avec des épées qu’ils ne cachaient même pas, assiégeait l’entrée du sénat ; enfin des pelotons de soldats étaient distribués sur les places et dans les basiliques. Ce fut en essuyant les menaces et les regards de ces satellites, que les sénateurs se rendirent au conseil. Un discours du prince fut lu par son questeur. Sans prononcer le nom de personne, il accusait les sénateurs d’abandonner les fonctions publiques et d’autoriser par leur exemple l’insouciance des chevaliers. « Fallait-il s’étonner qu’on ne vint pas des provinces éloignées, lorsque, après avoir obtenu des consulats et des sacerdoces, la plupart ne songeaient qu’à l’embellissement de leurs jardins ? » Ces paroles furent comme une arme que saisirent les accusateurs.\par
\labelchar{XXVIII.} Capiton attaqua le premier ; puis Marcellus, plus violent encore, s’écria : « qu’il s’agissait du salut de l’État ; que la révolte des inférieurs aigrissait un chef naturellement doux ; que c’était, de la part du sénat, un excès d’indulgence d’avoir laissé jusqu’à ce jour un Thraséas, déserteur de la chose publique, un Helvidius Priscus, gendre de cet homme et complice de ses fureurs, un Paconius Agrippinus, héritier de la haine de son père contre les princes, un Curtius Montanus, auteur de vers abominables, braver impunément sa justice ; qu’il voulait voir au sénat un consulaire, un prêtre aux vœux publics, un citoyen au serment annuel ; à moins qu’au mépris des institutions et des rites antiques Thraséas ne se fût ouvertement déclaré traître et ennemi. Qu’il vienne donc, ce sénateur zélé, ce protecteur de quiconque ose calomnier le prince, qu’il vienne reprendre son rôle, nous dire quelle réforme, quel changement il exige : on supportera plutôt des censures qui attaquent tout en détail, qu’un silence qui condamne tout en masse. Est-ce la paix de l’univers, ou ces victoires qui ne coûtent point de sang à nos armées, qui lui déplaisent ? Si le bonheur public le désespère, si nos places, nos théâtres, ne sont pour lui que d’odieuses solitudes, s’il nous menace chaque jour de son exil, ne comblons pas ses détestables vœux. Il ne reconnaît ni vos décrets, ni vos magistrats ; pour lui Rome même n’est plus Rome : qu’il brise en mourant ses derniers liens avec une cité depuis longtemps bannie de son cœur, aujourd’hui insupportable à sa vue. »\par
\labelchar{XXIX.} Pendant ces invectives, que Marcellus, naturellement farouche et menaçant, débita d’une voix animée, le visage et les yeux tout en feu, il régnait parmi les sénateurs non cette tristesse que des périls de chaque jour avaient tournée en habitude, mais une terreur inconnue et que rendait plus profonde la vue des soldats et des glaives. La figure vénérable de Thraséas s’offrait en même temps à leur pensée. Quelques-uns plaignaient aussi Helvidius, qu’on allait punir d’une alliance qui n’avait rien de coupable. Et Agrippinus, quel était son crime, sinon la triste destinée d’un père innocent comme lui, et victime de Tibère ? Et Montanus, jeune homme vertueux dont les vers ne diffamaient personne, on l’exilait donc pour avoir montré du talent !\par
\labelchar{XXX.} Cependant Ostorius Sabinus, accusateur de Soranus, entre et parle à son tour. Il lui reproche « ses liaisons avec Plautus, et son proconsulat d’Asie, où, plus soigneux de lui-même et de sa popularité que de l’intérêt public, il a entretenu dans les villes l’esprit de sédition. » Ces griefs étaient vieux : il en impute un plus récent à la fille de Soranus, qu’il associe au danger de son père « pour avoir prodigué de l’argent à des devins. » Servilie (c’était son nom) avait eu en effet ce malheur, et la piété filiale en était cause. Sa tendresse pour son père, l’imprudence de son âge, l’avaient conduite chez les devins, uniquement toutefois pour savoir ce que sa maison devait espérer ; si Néron se laisserait fléchir ; si le sénat prononcerait un arrêt qui ne fût pas sinistre. Servilie fut appelée à l’instant ; et l’on vit debout, devant le tribunal des consuls, d’un côté un père chargé d’années ; en face de lui, sa fille à peine âgée de vingt ans, condamnée déjà, par l’exil récent d’Annius Pollio son mari, au veuvage et à la solitude, et n’osant pas même lever les yeux sur son père, dont elle semblait avoir aggravé les périls.\par
\labelchar{XXXI.} Interrogée par l’accusateur si elle n’avait pas vendu ses présents de noces et le collier dont elle était parée, pour en employer l’argent à des sacrifices magiques, elle se jette par terre et ne répond d’abord que par un long silence et d’abondantes larmes. Ensuite, embrassant les autels : « Non, s’écria-t-elle, je n’ai point invoqué d’affreuses divinités ni formé de vœux impies ; tout ce que j’ai demandé par ces prières malheureuses, c’est d’obtenir de toi, César, et de vous, pères conscrits, le salut du meilleur des pères. Mes pierreries, mes robes, les décorations de mon rang, je les ai données comme j’aurais donné mon sang et ma vie s’ils l’eussent exigé. C’est à ces hommes, inconnus de moi jusqu’alors, à répondre du nom qu’ils portent et de l’art qu’ils exercent. Quant au prince, je ne le nommai jamais qu’entre les dieux. Et cependant mon malheureux père ignore tout : si un crime fut commis, moi seule en suis coupable. »\par
\labelchar{XXXII.} Elle parlait encore ; Soranus l’interrompt et s’écrie, « qu’elle ne l’a pas suivi dans sa province ; qu’à son âge elle n’a pu être connue de Plautus ; qu’elle n’a pas été nommés dans les faits reprochés à son époux ; que trop de piété fut tout son crime. Ah ! qu’on sépare leur cause, et que lui-même subisse le destin qu’on voudra. » En même temps, il se précipitait dans les bras de sa fille, élancée vers lui ; mais des licteurs se jetèrent entre deux et les retinrent. Bientôt on entendit les témoins ; et, autant la cruauté de l’accusation avait ému les cœurs de pitié, autant la déposition de P. Égnatius les souleva d’indignation. Cet homme, client de Soranus, et acheté pour être l’assassin de son ami, se parait de l’extérieur imposant du stoïcisme ; habile à exprimer dans son air et son maintien l’image de la probité, du reste, perfide, artificieux, recélant au fond de son cœur l’avarice et la débauche. Quand l’or eut mis à découvert cet abîme de vices, on eut la preuve qu’il ne faut pas se défier uniquement des méchants enveloppés de fraude et souillés d’opprobres, mais que, sous de plus beaux dehors, il est aussi de fausses vertus et de trompeuses amitiés.\par
\labelchar{XXXIII.} Le même jour vit cependant une action généreuse ; Cassius Asclépiodotus, Bithynien distingué par ses grandes richesses, après avoir honoré la fortune de Soranus, n’abandonna pas sa disgrâce ; trait qui lui valut l’exil et la perte de tous ses biens : ainsi la justice des dieux opposait un bon exemple à un mauvais. Thraséas, Soranus, Servilie, eurent le choix de leur mort. Helvidius et Paconius furent chassés d’Italie. La grâce de Montanus fut accordée à son père, à condition que le jeune homme renoncerait aux honneurs. Les accusateurs Marcellus et Capiton reçurent chacun cinq millions de sesterces \footnote{De notre monnaie, 919 049 fr. – 2. 220 671 fr. 76 c.}, Ostorius douze cent mille (2), avec les insignes de la questure.\par
\labelchar{XXXIV.} Thraséas était alors dans ses jardins, où le questeur du consul lui fut envoyé sur le déclin du jour. Il avait réuni un cercle nombreux d’hommes et de femmes distingués, et il s’entretenait particulièrement avec Démétrius, philosophe de l’école cynique. A en juger par l’expression de sa figure, et quelques mots prononcés un peu plus haut que le reste, il s’occupait de questions sur la nature de l’âme et sa séparation d’avec le corps ; lorsque Domitius Cécilianus, un de ses intimes amis, arrive et lui expose ce que vient d’ordonner le sénat. A cette nouvelle, tous pleurent, tous gémissent : Thraséas les presse de s’éloigner au plus tôt, et de ne pas lier imprudemment leur fortune à celle d’un condamné. Arria voulait, à l’exemple de sa mère \footnote{Arria, belle-mère de Thraséas, était femme de Pétus Cécina, qui prit part à la révolte de Scribonianus contre Claude, et qui, forcé de mourir, reçut d’elle l’exemple du courage : elle se perça la première, et, lui présentant le poignard qu’elle venait de retirer de son sein : « Pétus, lui dit-elle, cela ne fait pas de mal : \emph{Paete, non dolet}. »}, partager le destin de son époux : il la conjura de vivre et de ne pas ravir à leur fille son unique soutien.\par
\labelchar{XXXV.} Puis il s’avance sous le portique de sa maison, où arriva bientôt le questeur. Il le reçut d’un air presque joyeux, parce qu’il venait d’apprendre que son gendre Helvidius n’était que banni d’Italie. Quand on lui eut remis l’arrêt du sénat, il fit entrer Helvidius et Démétrius dans sa chambre, et présenta au fer ses deux bras à la fois. Aussitôt que le sang coula, il en répandit sur la terre, et, priant le questeur d’approcher : « Faisons, dit-il, cette libation à Jupiter Libérateur. Regarde, jeune homme, et puissent les dieux détourner ce présage ! mais tu es né dans des temps où il convient de fortifier son âme par des exemples de fermeté. » La mort était lente à venir, et Thraséas souffrait de cruelles douleurs ; se tournant vers Démétrius…
 


% at least one empty page at end (for booklet couv)
\ifbooklet
  \pagestyle{empty}
  \clearpage
  % 2 empty pages maybe needed for 4e cover
  \ifnum\modulo{\value{page}}{4}=0 \hbox{}\newpage\hbox{}\newpage\fi
  \ifnum\modulo{\value{page}}{4}=1 \hbox{}\newpage\hbox{}\newpage\fi


  \hbox{}\newpage
  \ifodd\value{page}\hbox{}\newpage\fi
  {\centering\color{rubric}\bfseries\noindent\large
    Hurlus ? Qu’est-ce.\par
    \bigskip
  }
  \noindent Des bouquinistes électroniques, pour du texte libre à participation libre,
  téléchargeable gratuitement sur \href{https://hurlus.fr}{\dotuline{hurlus.fr}}.\par
  \bigskip
  \noindent Cette brochure a été produite par des éditeurs bénévoles.
  Elle n’est pas faîte pour être possédée, mais pour être lue, et puis donnée.
  Que circule le texte !
  En page de garde, on peut ajouter une date, un lieu, un nom ; pour suivre le voyage des idées.
  \par

  Ce texte a été choisi parce qu’une personne l’a aimé,
  ou haï, elle a en tous cas pensé qu’il partipait à la formation de notre présent ;
  sans le souci de plaire, vendre, ou militer pour une cause.
  \par

  L’édition électronique est soigneuse, tant sur la technique
  que sur l’établissement du texte ; mais sans aucune prétention scolaire, au contraire.
  Le but est de s’adresser à tous, sans distinction de science ou de diplôme.
  Au plus direct ! (possible)
  \par

  Cet exemplaire en papier a été tiré sur une imprimante personnelle
   ou une photocopieuse. Tout le monde peut le faire.
  Il suffit de
  télécharger un fichier sur \href{https://hurlus.fr}{\dotuline{hurlus.fr}},
  d’imprimer, et agrafer ; puis de lire et donner.\par

  \bigskip

  \noindent PS : Les hurlus furent aussi des rebelles protestants qui cassaient les statues dans les églises catholiques. En 1566 démarra la révolte des gueux dans le pays de Lille. L’insurrection enflamma la région jusqu’à Anvers où les gueux de mer bloquèrent les bateaux espagnols.
  Ce fut une rare guerre de libération dont naquit un pays toujours libre : les Pays-Bas.
  En plat pays francophone, par contre, restèrent des bandes de huguenots, les hurlus, progressivement réprimés par la très catholique Espagne.
  Cette mémoire d’une défaite est éteinte, rallumons-la. Sortons les livres du culte universitaire, cherchons les idoles de l’époque, pour les briser.
\fi

\ifdev % autotext in dev mode
\fontname\font — \textsc{Les règles du jeu}\par
(\hyperref[utopie]{\underline{Lien}})\par
\noindent \initialiv{A}{lors là}\blindtext\par
\noindent \initialiv{À}{ la bonheur des dames}\blindtext\par
\noindent \initialiv{É}{tonnez-le}\blindtext\par
\noindent \initialiv{Q}{ualitativement}\blindtext\par
\noindent \initialiv{V}{aloriser}\blindtext\par
\Blindtext
\phantomsection
\label{utopie}
\Blinddocument
\fi
\end{document}
