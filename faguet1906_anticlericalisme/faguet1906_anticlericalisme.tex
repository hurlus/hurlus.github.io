%%%%%%%%%%%%%%%%%%%%%%%%%%%%%%%%%
% LaTeX model https://hurlus.fr %
%%%%%%%%%%%%%%%%%%%%%%%%%%%%%%%%%

% Needed before document class
\RequirePackage{pdftexcmds} % needed for tests expressions
\RequirePackage{fix-cm} % correct units

% Define mode
\def\mode{a4}

\newif\ifaiv % a4
\newif\ifav % a5
\newif\ifbooklet % booklet
\newif\ifcover % cover for booklet

\ifnum \strcmp{\mode}{cover}=0
  \covertrue
\else\ifnum \strcmp{\mode}{booklet}=0
  \booklettrue
\else\ifnum \strcmp{\mode}{a5}=0
  \avtrue
\else
  \aivtrue
\fi\fi\fi

\ifbooklet % do not enclose with {}
  \documentclass[french,twoside]{book} % ,notitlepage
  \usepackage[%
    papersize={105mm, 297mm},
    inner=12mm,
    outer=12mm,
    top=20mm,
    bottom=15mm,
    marginparsep=0pt,
  ]{geometry}
  \usepackage[fontsize=9.5pt]{scrextend} % for Roboto
\else\ifav
  \documentclass[french,twoside]{book} % ,notitlepage
  \usepackage[%
    a5paper,
    inner=25mm,
    outer=15mm,
    top=15mm,
    bottom=15mm,
    marginparsep=0pt,
  ]{geometry}
  \usepackage[fontsize=12pt]{scrextend}
\else% A4 2 cols
  \documentclass[twocolumn]{report}
  \usepackage[%
    a4paper,
    inner=15mm,
    outer=10mm,
    top=25mm,
    bottom=18mm,
    marginparsep=0pt,
  ]{geometry}
  \setlength{\columnsep}{20mm}
  \usepackage[fontsize=9.5pt]{scrextend}
\fi\fi

%%%%%%%%%%%%%%
% Alignments %
%%%%%%%%%%%%%%
% before teinte macros

\setlength{\arrayrulewidth}{0.2pt}
\setlength{\columnseprule}{\arrayrulewidth} % twocol
\setlength{\parskip}{0pt} % classical para with no margin
\setlength{\parindent}{1.5em}

%%%%%%%%%%
% Colors %
%%%%%%%%%%
% before Teinte macros

\usepackage[dvipsnames]{xcolor}
\definecolor{rubric}{HTML}{800000} % the tonic 0c71c3
\def\columnseprulecolor{\color{rubric}}
\colorlet{borderline}{rubric!30!} % definecolor need exact code
\definecolor{shadecolor}{gray}{0.95}
\definecolor{bghi}{gray}{0.5}

%%%%%%%%%%%%%%%%%
% Teinte macros %
%%%%%%%%%%%%%%%%%
%%%%%%%%%%%%%%%%%%%%%%%%%%%%%%%%%%%%%%%%%%%%%%%%%%%
% <TEI> generic (LaTeX names generated by Teinte) %
%%%%%%%%%%%%%%%%%%%%%%%%%%%%%%%%%%%%%%%%%%%%%%%%%%%
% This template is inserted in a specific design
% It is XeLaTeX and otf fonts

\makeatletter % <@@@


\usepackage{blindtext} % generate text for testing
\usepackage[strict]{changepage} % for modulo 4
\usepackage{contour} % rounding words
\usepackage[nodayofweek]{datetime}
% \usepackage{DejaVuSans} % seems buggy for sffont font for symbols
\usepackage{enumitem} % <list>
\usepackage{etoolbox} % patch commands
\usepackage{fancyvrb}
\usepackage{fancyhdr}
\usepackage{float}
\usepackage{fontspec} % XeLaTeX mandatory for fonts
\usepackage{footnote} % used to capture notes in minipage (ex: quote)
\usepackage{framed} % bordering correct with footnote hack
\usepackage{graphicx}
\usepackage{lettrine} % drop caps
\usepackage{lipsum} % generate text for testing
\usepackage[framemethod=tikz,]{mdframed} % maybe used for frame with footnotes inside
\usepackage{pdftexcmds} % needed for tests expressions
\usepackage{polyglossia} % non-break space french punct, bug Warning: "Failed to patch part"
\usepackage[%
  indentfirst=false,
  vskip=1em,
  noorphanfirst=true,
  noorphanafter=true,
  leftmargin=\parindent,
  rightmargin=0pt,
]{quoting}
\usepackage{ragged2e}
\usepackage{setspace} % \setstretch for <quote>
\usepackage{tabularx} % <table>
\usepackage[explicit]{titlesec} % wear titles, !NO implicit
\usepackage{tikz} % ornaments
\usepackage{tocloft} % styling tocs
\usepackage[fit]{truncate} % used im runing titles
\usepackage{unicode-math}
\usepackage[normalem]{ulem} % breakable \uline, normalem is absolutely necessary to keep \emph
\usepackage{verse} % <l>
\usepackage{xcolor} % named colors
\usepackage{xparse} % @ifundefined
\XeTeXdefaultencoding "iso-8859-1" % bad encoding of xstring
\usepackage{xstring} % string tests
\XeTeXdefaultencoding "utf-8"
\PassOptionsToPackage{hyphens}{url} % before hyperref, which load url package

% TOTEST
% \usepackage{hypcap} % links in caption ?
% \usepackage{marginnote}
% TESTED
% \usepackage{background} % doesn’t work with xetek
% \usepackage{bookmark} % prefers the hyperref hack \phantomsection
% \usepackage[color, leftbars]{changebar} % 2 cols doc, impossible to keep bar left
% \usepackage[utf8x]{inputenc} % inputenc package ignored with utf8 based engines
% \usepackage[sfdefault,medium]{inter} % no small caps
% \usepackage{firamath} % choose firasans instead, firamath unavailable in Ubuntu 21-04
% \usepackage{flushend} % bad for last notes, supposed flush end of columns
% \usepackage[stable]{footmisc} % BAD for complex notes https://texfaq.org/FAQ-ftnsect
% \usepackage{helvet} % not for XeLaTeX
% \usepackage{multicol} % not compatible with too much packages (longtable, framed, memoir…)
% \usepackage[default,oldstyle,scale=0.95]{opensans} % no small caps
% \usepackage{sectsty} % \chapterfont OBSOLETE
% \usepackage{soul} % \ul for underline, OBSOLETE with XeTeX
% \usepackage[breakable]{tcolorbox} % text styling gone, footnote hack not kept with breakable


% Metadata inserted by a program, from the TEI source, for title page and runing heads
\title{\textbf{ L’anticléricalisme }}
\date{1906}
\author{Émile Faguet}
\def\elbibl{Émile Faguet. 1906. \emph{L’anticléricalisme}}
\def\elsource{Émile Faguet, \emph{{\itshape L’anticléricalisme}}, 1906. }

% Default metas
\newcommand{\colorprovide}[2]{\@ifundefinedcolor{#1}{\colorlet{#1}{#2}}{}}
\colorprovide{rubric}{red}
\colorprovide{silver}{lightgray}
\@ifundefined{syms}{\newfontfamily\syms{DejaVu Sans}}{}
\newif\ifdev
\@ifundefined{elbibl}{% No meta defined, maybe dev mode
  \newcommand{\elbibl}{Titre court ?}
  \newcommand{\elbook}{Titre du livre source ?}
  \newcommand{\elabstract}{Résumé\par}
  \newcommand{\elurl}{http://oeuvres.github.io/elbook/2}
  \author{Éric Lœchien}
  \title{Un titre de test assez long pour vérifier le comportement d’une maquette}
  \date{1566}
  \devtrue
}{}
\let\eltitle\@title
\let\elauthor\@author
\let\eldate\@date


\defaultfontfeatures{
  % Mapping=tex-text, % no effect seen
  Scale=MatchLowercase,
  Ligatures={TeX,Common},
}


% generic typo commands
\newcommand{\astermono}{\medskip\centerline{\color{rubric}\large\selectfont{\syms ✻}}\medskip\par}%
\newcommand{\astertri}{\medskip\par\centerline{\color{rubric}\large\selectfont{\syms ✻\,✻\,✻}}\medskip\par}%
\newcommand{\asterism}{\bigskip\par\noindent\parbox{\linewidth}{\centering\color{rubric}\large{\syms ✻}\\{\syms ✻}\hskip 0.75em{\syms ✻}}\bigskip\par}%

% lists
\newlength{\listmod}
\setlength{\listmod}{\parindent}
\setlist{
  itemindent=!,
  listparindent=\listmod,
  labelsep=0.2\listmod,
  parsep=0pt,
  % topsep=0.2em, % default topsep is best
}
\setlist[itemize]{
  label=—,
  leftmargin=0pt,
  labelindent=1.2em,
  labelwidth=0pt,
}
\setlist[enumerate]{
  label={\bf\color{rubric}\arabic*.},
  labelindent=0.8\listmod,
  leftmargin=\listmod,
  labelwidth=0pt,
}
\newlist{listalpha}{enumerate}{1}
\setlist[listalpha]{
  label={\bf\color{rubric}\alph*.},
  leftmargin=0pt,
  labelindent=0.8\listmod,
  labelwidth=0pt,
}
\newcommand{\listhead}[1]{\hspace{-1\listmod}\emph{#1}}

\renewcommand{\hrulefill}{%
  \leavevmode\leaders\hrule height 0.2pt\hfill\kern\z@}

% General typo
\DeclareTextFontCommand{\textlarge}{\large}
\DeclareTextFontCommand{\textsmall}{\small}

% commands, inlines
\newcommand{\anchor}[1]{\Hy@raisedlink{\hypertarget{#1}{}}} % link to top of an anchor (not baseline)
\newcommand\abbr[1]{#1}
\newcommand{\autour}[1]{\tikz[baseline=(X.base)]\node [draw=rubric,thin,rectangle,inner sep=1.5pt, rounded corners=3pt] (X) {\color{rubric}#1};}
\newcommand\corr[1]{#1}
\newcommand{\ed}[1]{ {\color{silver}\sffamily\footnotesize (#1)} } % <milestone ed="1688"/>
\newcommand\expan[1]{#1}
\newcommand\foreign[1]{\emph{#1}}
\newcommand\gap[1]{#1}
\renewcommand{\LettrineFontHook}{\color{rubric}}
\newcommand{\initial}[2]{\lettrine[lines=2, loversize=0.3, lhang=0.3]{#1}{#2}}
\newcommand{\initialiv}[2]{%
  \let\oldLFH\LettrineFontHook
  % \renewcommand{\LettrineFontHook}{\color{rubric}\ttfamily}
  \IfSubStr{QJ’}{#1}{
    \lettrine[lines=4, lhang=0.2, loversize=-0.1, lraise=0.2]{\smash{#1}}{#2}
  }{\IfSubStr{É}{#1}{
    \lettrine[lines=4, lhang=0.2, loversize=-0, lraise=0]{\smash{#1}}{#2}
  }{\IfSubStr{ÀÂ}{#1}{
    \lettrine[lines=4, lhang=0.2, loversize=-0, lraise=0, slope=0.6em]{\smash{#1}}{#2}
  }{\IfSubStr{A}{#1}{
    \lettrine[lines=4, lhang=0.2, loversize=0.2, slope=0.6em]{\smash{#1}}{#2}
  }{\IfSubStr{V}{#1}{
    \lettrine[lines=4, lhang=0.2, loversize=0.2, slope=-0.5em]{\smash{#1}}{#2}
  }{
    \lettrine[lines=4, lhang=0.2, loversize=0.2]{\smash{#1}}{#2}
  }}}}}
  \let\LettrineFontHook\oldLFH
}
\newcommand{\labelchar}[1]{\textbf{\color{rubric} #1}}
\newcommand{\milestone}[1]{\autour{\footnotesize\color{rubric} #1}} % <milestone n="4"/>
\newcommand\name[1]{#1}
\newcommand\orig[1]{#1}
\newcommand\orgName[1]{#1}
\newcommand\persName[1]{#1}
\newcommand\placeName[1]{#1}
\newcommand{\pn}[1]{\IfSubStr{-—–¶}{#1}% <p n="3"/>
  {\noindent{\bfseries\color{rubric}   ¶  }}
  {{\footnotesize\autour{ #1}  }}}
\newcommand\reg{}
% \newcommand\ref{} % already defined
\newcommand\sic[1]{#1}
\newcommand\surname[1]{\textsc{#1}}
\newcommand\term[1]{\textbf{#1}}

\def\mednobreak{\ifdim\lastskip<\medskipamount
  \removelastskip\nopagebreak\medskip\fi}
\def\bignobreak{\ifdim\lastskip<\bigskipamount
  \removelastskip\nopagebreak\bigskip\fi}

% commands, blocks
\newcommand{\byline}[1]{\bigskip{\RaggedLeft{#1}\par}\bigskip}
\newcommand{\bibl}[1]{{\RaggedLeft{#1}\par\bigskip}}
\newcommand{\biblitem}[1]{{\noindent\hangindent=\parindent   #1\par}}
\newcommand{\dateline}[1]{\medskip{\RaggedLeft{#1}\par}\bigskip}
\newcommand{\labelblock}[1]{\medbreak{\noindent\color{rubric}\bfseries #1}\par\mednobreak}
\newcommand{\salute}[1]{\bigbreak{#1}\par\medbreak}
\newcommand{\signed}[1]{\bigbreak\filbreak{\raggedleft #1\par}\medskip}

% environments for blocks (some may become commands)
\newenvironment{borderbox}{}{} % framing content
\newenvironment{citbibl}{\ifvmode\hfill\fi}{\ifvmode\par\fi }
\newenvironment{docAuthor}{\ifvmode\vskip4pt\fontsize{16pt}{18pt}\selectfont\fi\itshape}{\ifvmode\par\fi }
\newenvironment{docDate}{}{\ifvmode\par\fi }
\newenvironment{docImprint}{\vskip6pt}{\ifvmode\par\fi }
\newenvironment{docTitle}{\vskip6pt\bfseries\fontsize{18pt}{22pt}\selectfont}{\par }
\newenvironment{msHead}{\vskip6pt}{\par}
\newenvironment{msItem}{\vskip6pt}{\par}
\newenvironment{titlePart}{}{\par }


% environments for block containers
\newenvironment{argument}{\itshape\parindent0pt}{\vskip1.5em}
\newenvironment{biblfree}{}{\ifvmode\par\fi }
\newenvironment{bibitemlist}[1]{%
  \list{\@biblabel{\@arabic\c@enumiv}}%
  {%
    \settowidth\labelwidth{\@biblabel{#1}}%
    \leftmargin\labelwidth
    \advance\leftmargin\labelsep
    \@openbib@code
    \usecounter{enumiv}%
    \let\p@enumiv\@empty
    \renewcommand\theenumiv{\@arabic\c@enumiv}%
  }
  \sloppy
  \clubpenalty4000
  \@clubpenalty \clubpenalty
  \widowpenalty4000%
  \sfcode`\.\@m
}%
{\def\@noitemerr
  {\@latex@warning{Empty `bibitemlist' environment}}%
\endlist}
\newenvironment{quoteblock}% may be used for ornaments
  {\begin{quoting}}
  {\end{quoting}}

% table () is preceded and finished by custom command
\newcommand{\tableopen}[1]{%
  \ifnum\strcmp{#1}{wide}=0{%
    \begin{center}
  }
  \else\ifnum\strcmp{#1}{long}=0{%
    \begin{center}
  }
  \else{%
    \begin{center}
  }
  \fi\fi
}
\newcommand{\tableclose}[1]{%
  \ifnum\strcmp{#1}{wide}=0{%
    \end{center}
  }
  \else\ifnum\strcmp{#1}{long}=0{%
    \end{center}
  }
  \else{%
    \end{center}
  }
  \fi\fi
}


% text structure
\newcommand\chapteropen{} % before chapter title
\newcommand\chaptercont{} % after title, argument, epigraph…
\newcommand\chapterclose{} % maybe useful for multicol settings
\setcounter{secnumdepth}{-2} % no counters for hierarchy titles
\setcounter{tocdepth}{5} % deep toc
\markright{\@title} % ???
\markboth{\@title}{\@author} % ???
\renewcommand\tableofcontents{\@starttoc{toc}}
% toclof format
% \renewcommand{\@tocrmarg}{0.1em} % Useless command?
% \renewcommand{\@pnumwidth}{0.5em} % {1.75em}
\renewcommand{\@cftmaketoctitle}{}
\setlength{\cftbeforesecskip}{\z@ \@plus.2\p@}
\renewcommand{\cftchapfont}{}
\renewcommand{\cftchapdotsep}{\cftdotsep}
\renewcommand{\cftchapleader}{\normalfont\cftdotfill{\cftchapdotsep}}
\renewcommand{\cftchappagefont}{\bfseries}
\setlength{\cftbeforechapskip}{0em \@plus\p@}
% \renewcommand{\cftsecfont}{\small\relax}
\renewcommand{\cftsecpagefont}{\normalfont}
% \renewcommand{\cftsubsecfont}{\small\relax}
\renewcommand{\cftsecdotsep}{\cftdotsep}
\renewcommand{\cftsecpagefont}{\normalfont}
\renewcommand{\cftsecleader}{\normalfont\cftdotfill{\cftsecdotsep}}
\setlength{\cftsecindent}{1em}
\setlength{\cftsubsecindent}{2em}
\setlength{\cftsubsubsecindent}{3em}
\setlength{\cftchapnumwidth}{1em}
\setlength{\cftsecnumwidth}{1em}
\setlength{\cftsubsecnumwidth}{1em}
\setlength{\cftsubsubsecnumwidth}{1em}

% footnotes
\newif\ifheading
\newcommand*{\fnmarkscale}{\ifheading 0.70 \else 1 \fi}
\renewcommand\footnoterule{\vspace*{0.3cm}\hrule height \arrayrulewidth width 3cm \vspace*{0.3cm}}
\setlength\footnotesep{1.5\footnotesep} % footnote separator
\renewcommand\@makefntext[1]{\parindent 1.5em \noindent \hb@xt@1.8em{\hss{\normalfont\@thefnmark . }}#1} % no superscipt in foot
\patchcmd{\@footnotetext}{\footnotesize}{\footnotesize\sffamily}{}{} % before scrextend, hyperref


%   see https://tex.stackexchange.com/a/34449/5049
\def\truncdiv#1#2{((#1-(#2-1)/2)/#2)}
\def\moduloop#1#2{(#1-\truncdiv{#1}{#2}*#2)}
\def\modulo#1#2{\number\numexpr\moduloop{#1}{#2}\relax}

% orphans and widows
\clubpenalty=9996
\widowpenalty=9999
\brokenpenalty=4991
\predisplaypenalty=10000
\postdisplaypenalty=1549
\displaywidowpenalty=1602
\hyphenpenalty=400
% Copied from Rahtz but not understood
\def\@pnumwidth{1.55em}
\def\@tocrmarg {2.55em}
\def\@dotsep{4.5}
\emergencystretch 3em
\hbadness=4000
\pretolerance=750
\tolerance=2000
\vbadness=4000
\def\Gin@extensions{.pdf,.png,.jpg,.mps,.tif}
% \renewcommand{\@cite}[1]{#1} % biblio

\usepackage{hyperref} % supposed to be the last one, :o) except for the ones to follow
\urlstyle{same} % after hyperref
\hypersetup{
  % pdftex, % no effect
  pdftitle={\elbibl},
  % pdfauthor={Your name here},
  % pdfsubject={Your subject here},
  % pdfkeywords={keyword1, keyword2},
  bookmarksnumbered=true,
  bookmarksopen=true,
  bookmarksopenlevel=1,
  pdfstartview=Fit,
  breaklinks=true, % avoid long links
  pdfpagemode=UseOutlines,    % pdf toc
  hyperfootnotes=true,
  colorlinks=false,
  pdfborder=0 0 0,
  % pdfpagelayout=TwoPageRight,
  % linktocpage=true, % NO, toc, link only on page no
}

\makeatother % /@@@>
%%%%%%%%%%%%%%
% </TEI> end %
%%%%%%%%%%%%%%


%%%%%%%%%%%%%
% footnotes %
%%%%%%%%%%%%%
\renewcommand{\thefootnote}{\bfseries\textcolor{rubric}{\arabic{footnote}}} % color for footnote marks

%%%%%%%%%
% Fonts %
%%%%%%%%%
\usepackage[]{roboto} % SmallCaps, Regular is a bit bold
% \linespread{0.90} % too compact, keep font natural
\newfontfamily\fontrun[]{Roboto Condensed Light} % condensed runing heads
\ifav
  \setmainfont[
    ItalicFont={Roboto Light Italic},
  ]{Roboto}
\else\ifbooklet
  \setmainfont[
    ItalicFont={Roboto Light Italic},
  ]{Roboto}
\else
\setmainfont[
  ItalicFont={Roboto Italic},
]{Roboto Light}
\fi\fi
\renewcommand{\LettrineFontHook}{\bfseries\color{rubric}}
% \renewenvironment{labelblock}{\begin{center}\bfseries\color{rubric}}{\end{center}}

%%%%%%%%
% MISC %
%%%%%%%%

\setdefaultlanguage[frenchpart=false]{french} % bug on part


\newenvironment{quotebar}{%
    \def\FrameCommand{{\color{rubric!10!}\vrule width 0.5em} \hspace{0.9em}}%
    \def\OuterFrameSep{\itemsep} % séparateur vertical
    \MakeFramed {\advance\hsize-\width \FrameRestore}
  }%
  {%
    \endMakeFramed
  }
\renewenvironment{quoteblock}% may be used for ornaments
  {%
    \savenotes
    \setstretch{0.9}
    \normalfont
    \begin{quotebar}
  }
  {%
    \end{quotebar}
    \spewnotes
  }


\renewcommand{\headrulewidth}{\arrayrulewidth}
\renewcommand{\headrule}{{\color{rubric}\hrule}}

% delicate tuning, image has produce line-height problems in title on 2 lines
\titleformat{name=\chapter} % command
  [display] % shape
  {\vspace{1.5em}\centering} % format
  {} % label
  {0pt} % separator between n
  {}
[{\color{rubric}\huge\textbf{#1}}\bigskip] % after code
% \titlespacing{command}{left spacing}{before spacing}{after spacing}[right]
\titlespacing*{\chapter}{0pt}{-2em}{0pt}[0pt]

\titleformat{name=\section}
  [block]{}{}{}{}
  [\vbox{\color{rubric}\large\raggedleft\textbf{#1}}]
\titlespacing{\section}{0pt}{0pt plus 4pt minus 2pt}{\baselineskip}

\titleformat{name=\subsection}
  [block]
  {}
  {} % \thesection
  {} % separator \arrayrulewidth
  {}
[\vbox{\large\textbf{#1}}]
% \titlespacing{\subsection}{0pt}{0pt plus 4pt minus 2pt}{\baselineskip}

\ifaiv
  \fancypagestyle{main}{%
    \fancyhf{}
    \setlength{\headheight}{1.5em}
    \fancyhead{} % reset head
    \fancyfoot{} % reset foot
    \fancyhead[L]{\truncate{0.45\headwidth}{\fontrun\elbibl}} % book ref
    \fancyhead[R]{\truncate{0.45\headwidth}{ \fontrun\nouppercase\leftmark}} % Chapter title
    \fancyhead[C]{\thepage}
  }
  \fancypagestyle{plain}{% apply to chapter
    \fancyhf{}% clear all header and footer fields
    \setlength{\headheight}{1.5em}
    \fancyhead[L]{\truncate{0.9\headwidth}{\fontrun\elbibl}}
    \fancyhead[R]{\thepage}
  }
\else
  \fancypagestyle{main}{%
    \fancyhf{}
    \setlength{\headheight}{1.5em}
    \fancyhead{} % reset head
    \fancyfoot{} % reset foot
    \fancyhead[RE]{\truncate{0.9\headwidth}{\fontrun\elbibl}} % book ref
    \fancyhead[LO]{\truncate{0.9\headwidth}{\fontrun\nouppercase\leftmark}} % Chapter title, \nouppercase needed
    \fancyhead[RO,LE]{\thepage}
  }
  \fancypagestyle{plain}{% apply to chapter
    \fancyhf{}% clear all header and footer fields
    \setlength{\headheight}{1.5em}
    \fancyhead[L]{\truncate{0.9\headwidth}{\fontrun\elbibl}}
    \fancyhead[R]{\thepage}
  }
\fi

\ifav % a5 only
  \titleclass{\section}{top}
\fi

\newcommand\chapo{{%
  \vspace*{-3em}
  \centering % no vskip ()
  {\Large\addfontfeature{LetterSpace=25}\bfseries{\elauthor}}\par
  \smallskip
  {\large\eldate}\par
  \bigskip
  {\Large\selectfont{\eltitle}}\par
  \bigskip
  {\color{rubric}\hline\par}
  \bigskip
  {\Large TEXTE LIBRE À PARTICPATION LIBRE\par}
  \centerline{\small\color{rubric} {hurlus.fr, tiré le \today}}\par
  \bigskip
}}

\newcommand\cover{{%
  \thispagestyle{empty}
  \centering
  {\LARGE\bfseries{\elauthor}}\par
  \bigskip
  {\Large\eldate}\par
  \bigskip
  \bigskip
  {\LARGE\selectfont{\eltitle}}\par
  \vfill\null
  {\color{rubric}\setlength{\arrayrulewidth}{2pt}\hline\par}
  \vfill\null
  {\Large TEXTE LIBRE À PARTICPATION LIBRE\par}
  \centerline{{\href{https://hurlus.fr}{\dotuline{hurlus.fr}}, tiré le \today}}\par
}}

\begin{document}
\pagestyle{empty}
\ifbooklet{
  \cover\newpage
  \thispagestyle{empty}\hbox{}\newpage
  \cover\newpage\noindent Les voyages de la brochure\par
  \bigskip
  \begin{tabularx}{\textwidth}{l|X|X}
    \textbf{Date} & \textbf{Lieu}& \textbf{Nom/pseudo} \\ \hline
    \rule{0pt}{25cm} &  &   \\
  \end{tabularx}
  \newpage
  \addtocounter{page}{-4}
}\fi

\thispagestyle{empty}
\ifaiv
  \twocolumn[\chapo]
\else
  \chapo
\fi
{\it\elabstract}
\bigskip
\makeatletter\@starttoc{toc}\makeatother % toc without new page
\bigskip

\pagestyle{main} % after style

  \section[{Chapitre premier. L’irréligion nationale.}]{Chapitre premier.\\
L’irréligion nationale.}\renewcommand{\leftmark}{Chapitre premier.\\
L’irréligion nationale.}

\noindent Je vais étudier une des maladies de la race française, la plus répandue et l’une des plus profondes à la fois et des plus aiguës. Je pense apporter de l’impartialité dans cette étude, n’appartenant à aucune confession religieuse, ni, ce qui est peut-être plus important encore dans l’espèce, à aucun parti politique. Mon intention très arrêtée est d’étudier cette affection comme si j’écrivais sur un sujet de l’antiquité, comme si cléricalisme, anticléricalisme, catholicisme et France même avaient disparu depuis dix siècles ; et comme si tous ces objets faisaient partie du domaine de la philologie ou de l’archéologie. Je m’en crois capable, encore qu’en pareille matière on ne saurait répondre de rien ; mais mon dessein, très ferme, est très exactement ce que je viens de dire. On me saura gré  d’avoir au moins pris la plume dans cet état d’esprit.\par

\astertri

\noindent Nietzsche a dit, dans la même phrase je crois, que le Français est essentiellement religieux et qu’il est essentiellement irréligieux. Il n’a pas tort, à la condition seulement qu’on mesure l’étendue des manifestations religieuses des Français et l’étendue aussi des manifestations contraires. Le Français, ce me semble, a des dispositions naturelles essentiellement irréligieuses ; seulement, et précisément à cause de cela, par réaction des esprits nés religieux contre leurs entours, il y a eu des groupements pénétrés de l’esprit religieux le plus intense ; il y a eu des îlots religieux singulièrement nets et pour ainsi dire aigus, comme il y a des îlots granitiques au milieu des pays calcaires, qui tranchent vigoureusement avec tout ce qui les entoure et se font remarquer d’autant.\par
Cela, ce me semble, à toutes les époques : vaudois, cathares, huguenots, jansénistes. La race française, étant ardente, devait produire quelques foyers d’ardent sentiment religieux, çà et là, sous l’influence d’un esprit dominateur ou d’une âme apostolique ; sous l’influence, aussi, des entours mêmes, poussant et provoquant au sentiment religieux les âmes douées de l’esprit de contradiction.\par
 Mais le fond de la race française, la généralité des Français me semble toujours avoir été peu capable d’embrasser et d’entretenir l’esprit religieux et le sentiment religieux. Il ne faut pas que nos guerres religieuses, assez nombreuses et assez longues, nous fassent illusion sur ceci. La religion, dans ces guerres, a été pour un cinquième cause et pour quatre cinquièmes prétexte, à calculer approximativement, comme on est bien forcé de faire en pareille matière. — Est-il quelqu’un qui estime aujourd’hui que la croisade des Albigeois ait été une manifestation de foi religieuse ? Est-il quelqu’un qui conteste qu’elle ait été une {\itshape ruée} de pillards avides sur des contrées riches, prenant prétexte dans le mot d’ordre d’un pape, qui lui-même obéissait à des idées politiques en le donnant ?\par
De même les guerres du \textsc{xvi}\textsuperscript{e} siècle entre protestants et catholiques français ont été surtout des guerres de féodaux contre la royauté ; et les guerres du \textsc{xvii}\textsuperscript{e} siècle entre catholiques et protestants français ont été surtout des guerres de républicains plus ou moins conscients et prenant plus ou moins leur mot d’ordre à Genève et en Hollande, contre la royauté française devenant absolutiste.\par
Et je dis même que chez les plus simples et qui ne se savaient, ni se sentaient ni féodaux à  telle époque, ni républicains à telle autre, la religion entrait moins en jeu que le simple goût de lutte pour la lutte et de guerre pour la guerre. Le Français est essentiellement homme de guerre civile. Il est batailleur de village à village, de province à province, de quartier à quartier. Ce peuple, qui a été amené à l’unité nationale par la persévérance étonnante d’une maison royale énergique et tenace et qui, livré à lui-même, semble retourner peu à peu au particularisme, ne me paraît jamais avoir montré par lui-même un désir d’unité et une volonté de concentration.\par
Bien plus fort a toujours été en lui l’instinct de guerre intérieure et intestine, le désir, soit de province à province, soit de parti à parti, d’écraser l’adversaire, sans se demander très précisément pourquoi il est l’adversaire et tout simplement parce qu’il faut bien se haïr et parce qu’il faut bien se battre.\par
L’enfant, en France, est élevé par ses parents dans la haine d’une certaine catégorie de Français ; et la première chose, presque, qu’on lui désigne, c’est un ennemi, très proche, quelqu’un, à côté de lui, qu’il faut s’habituer à détester et à injurier sans motif très précis ; mais pour montrer qu’on est le fils de son père.\par
Je crois que cela est « dans le sang ». Ce que  sont les partis politiques au \textsc{xx}\textsuperscript{e} siècle, les partis religieux l’étaient au \textsc{xvi}\textsuperscript{e} et au \textsc{xvii}\textsuperscript{e} siècle. Les guerres de religion n’ont guère été chez nous une manifestation de foi, d’un côté ou de l’autre ; elles ont été, avant tout, une forme du besoin gratuit de guerre civile.\par
Faut-il creuser ? On le pourrait, je crois. Faut-il se demander d’où vient lui-même ce besoin chez le Français de se battre contre le voisin, s’il a un couvre-chef d’une autre couleur que la nôtre ; et même ce besoin de n’adopter une autre couleur que la sienne que pour pouvoir se battre contre lui ? On le pourrait, je crois. Le Français est actif de corps et paresseux d’esprit. Il est nerveux et il est de cerveau nonchalant. Il sent le besoin d’agir et il n’aime pas à se donner beaucoup de peine pour trouver un motif d’agir, c’est-à-dire une idée. Il s’ensuit que sur un simple prétexte, sur une ombre d’idée adoptée en courant, il se jette en campagne et il frappe. Les premiers coups échangés sont ensuite un motif suffisant de continuer, par rancune, indéfiniment. Le Français peut donc livrer et soutenir une longue guerre sans avoir jamais eu un motif vrai de l’entreprendre, et être soutenu lui-même pendant cette longue lutte acharnée par une idée qu’en vérité il n’a jamais eue, et qui, à le bien prendre, n’existait pas.\par
 C’est ce qui a bien trompé les philosophes, très légers eux-mêmes, du \textsc{xviii}\textsuperscript{e} siècle. Ayant constaté qu’on sortait à peine des guerres religieuses, et de guerres religieuses épouvantables, ils se sont imaginé que c’était la religion qui était la cause de ces guerres et de ces épouvantements, et ils ont maudit et dénoncé les religions de tout leur cœur. Mais, après eux, on s’est battu pour d’autres idées, que du reste on ne comprenait pas davantage et qui n’étaient également que des prétextes ; et il a été suffisamment prouvé qu’autre chose que l’esprit religieux pouvait mettre aux hommes les armes à la main.\par
Où ils s’étaient trompés, c’était à croire que la religion fut la véritable cause de la guerre et que, la religion réduite à l’impuissance, il n’y aurait plus de guerre civile. La véritable cause des guerres civiles, c’était l’amour de la guerre civile elle-même et l’instinct même, impérieux et impatient, de guerre civile.\par
Des guerres de religion françaises ne concluons donc nullement que le Français soit très religieux ni qu’il l’ait jamais été. Il a, simplement, aimé l’échange des coups sous un prétexte ou sous un autre. La vérité, c’est que depuis qu’il existe il a eu un fond d’esprit irréligieux que les circonstances ont longtemps comprimé, que les circonstances  ont ensuite comme libéré et affranchi. Sans remonter aux satires et gouailleries anticléricales du moyen âge, lesquelles, d’abord ne sont qu’anticléricales et ne sont point antireligieuses ; lesquelles, ensuite, partent presque toujours d’auteurs qui sont clercs eux-mêmes et par conséquent ne sont que manières de plaisanteries de collège ou de divertissements de couvent et se ramènent à la règle : {\itshape maledicere de priore} ; on voit très bien, à partir du moment où la nationalité française est constituée, que l’esprit moyen, l’esprit bourgeois a comme une inclination naturelle à l’impiété. Tout, vers le \textsc{xvi}\textsuperscript{e} siècle, tend de ce côté ou y mène, exceptions réservées, qui ne laissent pas d’être rares.\par
L’esprit de la Réforme et l’esprit de la Renaissance, si différents, si contraires, du reste, en leur dernière influence sur l’esprit des classes moyennes françaises, mènent également à un détachement relativement à la foi. Que la moitié de l’Europe rompe avec l’empire spirituel de Rome, cela ne mène pas du tout les classes moyennes françaises à le secouer elles-mêmes. Non ; mais cela leur fait dire : « Rome n’est donc pas inébranlable et universelle ; et il y a donc deux religions qui se contrebalancent et dont il n’est pas certain que l’une soit la vraie ? Peut-être nous trompons-nous ou  sommes-nous trompés. » Le {\itshape qui sait ?} entrait dans les esprits et dans les mœurs, le {\itshape qui sait ?} subconscient et insidieux, qui est plus fort comme destructeur, comme rongeur, que toutes les affirmations et que toutes les argumentations du monde.\par
D’autre part, la Renaissance apprenait aux classes moyennes de la nation que de très grands peuples (ou au moins un) avaient vécu, et brillamment et glorieusement, et produit une civilisation extrêmement belle, sans avoir connu le Christianisme et guidés peut-être par ceux-là mêmes qui n’avaient aucune religion, mais seulement une philosophie toute personnelle. Platon, Zénon, Cicéron et Sénèque, ou plutôt ceci que Platon, Zénon, Cicéron et Sénèque ont existé, a eu une extraordinaire influence sur le bourgeois lettré ou à demi lettré du \textsc{xvi}\textsuperscript{e} siècle.\par
Il s’est formé ainsi un état d’esprit, sinon général, du moins assez répandu, qui était à base de scepticisme ou qui acceptait le scepticisme comme un de ses éléments. La Réforme et la Renaissance n’ont aucunement créé l’esprit irréligieux en France ; elles l’ont libéré, elles l’ont dégagé, elles lui ont donné occasion et raison de se manifester et de se déclarer plus ou moins. Il était latent, il est devenu découvert ; il était subconscient,  il a pris conscience de lui ; il était amorphe, et il a pris une certaine forme. C’est à partir de ce moment qu’il faut le considérer dans ses traits généraux.\par

\astertri

\noindent Le Français est irréligieux ou peu religieux d’abord en raison d’une de ses qualités, et c’est à savoir parce qu’il a l’esprit clair et le goût de la clarté. Religion et clarté ne sauraient aller ensemble, puisque la religion est surtout et peut être avant tout le sens du mystère. Or le sens du mystère est chose qui n’est pas très souvent connue du Français et qui presque toujours ne laisse pas de lui paraître assez ridicule. Le Français croit avoir tout dit, quand il a dit : « je ne comprends pas », et c’est une chose qu’il dit extrêmement vite. Le mot galimatias exerce sur le Français un ascendant extraordinaire et il l’applique avec une incroyable facilité.\par
Les esprits, je parle même des grands, qui ont eu, et tout de suite, sur le Français une grande influence, sont ceux qui sont prompts, vifs et clairs. C’est Montaigne, c’est Molière, c’est Voltaire. La moitié, au moins, de l’empire exercé par Descartes et de la grande fortune qu’il a faite si vite est due à ceci qu’il a donné pour criterium du vrai l’évidence, c’est-à-dire la vision claire de quelque  chose. Tout le monde s’est écrié : « A la bonne heure ! » Vico s’est beaucoup moqué de « l’évidence » de Descartes. Il n’a pas si grand tort. Quand on a un peu réfléchi on s’aperçoit qu’il n’y a qu’une chose qui soit évidente, c’est que rien n’est évident. Descartes a dit (au moins la foule l’a entendu ainsi) : « Ce qui est vrai, c’est ce qui est clair. » Notre meilleur professeur de philosophie de l’École normale, en présence d’un de ces systèmes qui en deux mots rendent compte de tout, disait doucement : « Oui, oui ; … c’est trop clair pour être vrai. » Il n’était pas très cartésien ce jour-là.\par
Remarquez-vous que le grand irréligieux et le grand immoraliste allemand, c’est Nietzsche sans doute que je veux dire, s’écrie avec l’accent du triomphe, quelque part : « Enfin nous devenons clairs ! » — et ailleurs : « Ce peuple qui se grise de bière et pour qui l’obscurité est une vertu. » Il y a un très grand sens dans ces boutades. Ce que Nietzsche sait bien, c’est que la passion de la clarté est une chose excellente contre la religion, contre la métaphysique et même contre la morale ; et c’est encore que ne pas avoir horreur de l’obscurité peut être favorable à ces choses exécrées et à ces états d’esprit jugés funestes.\par
La vérité est que le criterium de l’évidence et  le criterium de la clarté ne s’appliquent pas à tout objet. La vérité est qu’il y a des choses qui se comprennent lumineusement et qu’il y en a d’autres qui s’entendent aussi, mais qui ne peuvent pas se démêler dans une lumière si éclatante. Le plus beau mot de Hégel est celui-ci : « Il faut comprendre l’inintelligible… Eh oui ! Il faut le comprendre comme inintelligible. » Oui, encore il faut s’en rendre compte au moins ainsi ; et non pas l’exclure de tout examen et de toute préoccupation parce qu’il n’est pas clair. Qu’il ne soit pas objet de connaissance, nous le voulons bien, et nous l’appellerons, si l’on veut, l’inconnaissable ; mais qu’il ne soit pas objet d’étude, c’est ce que veut le principe de l’évidence, et c’est ce que nous ne pouvons pas vouloir.\par
Les choses qui sont claires, on les constate, et les choses qui ne sont pas claires, on en raisonne. De ce qu’il entre de l’hypothèse dans un raisonnement, cela prouve qu’on n’a pas fait le tour de son objet ; cela ne prouve pas qu’il ne fallût pas même essayer de raisonner sur cet objet. De ce que l’on comprend l’inintelligible comme inintelligible, ne concluez pas qu’on a eu tort d’essayer de le comprendre ; on a très bien fait d’essayer de savoir ce qu’il fallait en penser ; et en penser quelque chose,  sans doute ce n’est pas le connaître, mais encore c’est l’ignorer moins.\par
La métaphysique est un devoir de l’homme envers soi-même, du moins de l’homme cultivé. Elle consiste à mesurer les forces de son esprit et à chercher où est le point (qui n’est pas le même pour tous les hommes) où précisément l’évidence cesse ; et où est le point (qui n’est pas le même non plus pour tous les hommes) où la probabilité cesse aussi et où l’hypothèse commence à être, non plus rationnelle, mais tout imaginative. Oui, c’est vraiment un devoir pour l’homme cultivé de faire cet effort et cet essai.\par
Et il est bien certain que celui-ci ne les fera point qui se sera juré de ne pas faire un pas au-delà de la pleine clarté et de l’évidence éblouissante. Je n’ai pas besoin de dire que si Descartes a fait ce serment, il s’est bien donné de garde de le tenir. Mais, pour y revenir, c’est précisément ce serment que les Français, d’ordinaire, même quand ils ne l’ont pas fait, tiennent toujours. Le clair, le très clair, le clair tout de suite, ils ne veulent pas sortir de cela, et, à plus forte raison, ils ne veulent pas s’en éloigner. Il y a en France comme un préjugé contre le complexe, aussi bien en philosophie qu’en politique. Le Français a même à cet égard comme un flair particulier, comme un  sixième sens. Il sait d’avance que vous allez n’être pas clair ou que vous vous engagez dans une voie au bout de laquelle vous ne le serez pas. Il vous fait des procès de tendances à l’obscurité. Que peut-il cependant savoir d’avance en cela ? Cette clarté, qu’il aime tant, ne va-t-il pas la trouver à la fin du raisonnement que vous commencez et, à se dérober, ne reste-t-il pas dans l’obscur au lieu d’aller vers la lumière ? N’importe. Il se défie extrêmement et il ne s’embarque pas vers une plus grande clarté, de peur d’abandonner la demi-clarté où il est, ou croit être.\par
Au fond, il y a en cela une énorme paresse d’esprit. Rien n’est plus éloigné du Français que cette passion que s’attribuait Renan, que cette inquiétude, qui, la vérité étant trouvée, la lui faisait chercher encore. Le Français aime ne pas chercher, avant même d’avoir trouvé quelque chose. Il fait du {\itshape bon sens} un cas énorme, du bon sens, qui n’est pas une mauvaise chose, à la condition qu’il ne soit pas le nom que l’on donne à la nonchalance. Le bon sens est précisément cette demi-clarté dont se contente le Français en l’appelant évidence et en déclarant qu’il n’y a pas lieu de chercher et de tracasser au-delà. Personne, sur la terre, plus vite que le Français, ne dit : « Il est évident que »… et : « cela tombe sous le sens. »\par
 Les métaphysiques et les religions lui sont donc des ennemies naturelles, puisqu’elles essayent de sonder les grands mystères, c’est-à-dire tout simplement les questions les plus générales, et d’en donner ou une explication ou une vue.\par
Quelquefois — et c’est cela qui a bien trompé Auguste Comte — quelquefois le Français semble donner dans la métaphysique. Mais ce n’est que parce qu’il croit que la métaphysique va détrôner la religion, la détruire et — tant elle la détruira — la remplacer. Alors il s’échauffe pour une métaphysique laïque de tout son cœur ou plutôt de toute son humeur antireligieuse. Mais ce beau feu ne se soutient pas, et cette « période métaphysique », dont Auguste Comte faisait une époque de l’humanité, un âge du genre humain, a bien duré pour nous un siècle et demi tout au plus.\par
Ce beau feu métaphysique s’éteint très naturellement, parce que le Français s’aperçoit assez vite que sa chère clarté ne se trouve pas plus dans les métaphysiques que dans les religions et ne saurait s’y trouver, pour cette raison suffisante qu’une religion n’est qu’une métaphysique organisée, et qu’une métaphysique n’est qu’une religion {\itshape en devenir} ; et que, sauf cette différence, elles sont même chose, ayant exactement les mêmes objets. En conséquence, le Français renvoie la métaphysique  au même lieu d’exil où il avait relégué la religion, et dit de celle-là comme de celle-ci : « Tout cela n’est pas clair et donne beaucoup de peine à comprendre » ; ce qui est très vrai.\par
Notez ceci que les religions, comme, du reste, les métaphysiques, mais plus peut-être, parce qu’elles passionnent davantage, ne gagnent pas en clarté à mesure qu’elles durent, mais, au contraire, s’enfoncent dans l’obscurité, parce qu’elles s’expliquent sans cesse, et finissent par s’ensevelir sous les commentaires. Elles brillent, en leurs commencements, de la lumière éclatante d’un principe nouveau qu’elles apportent au monde, et leur force, une partie au moins de leur force, est dans leur clarté même. Mais bientôt elles cherchent à rendre compte de tout et s’encombrent d’une métaphysique complète, soit croyant s’appuyer sur elle, soit croyant y mener et y aboutir.\par
Il ne faut pas leur en vouloir précisément, puisque les religions étant faites pour donner aux hommes une règle de vie, elles sont bien presque contraintes de donner aux hommes une réponse relativement à ce qui les préoccupe, c’est-à-dire relativement à tout. Notre vie dépend de notre place dans le monde et par conséquent dépend du monde ; et donc, pour me montrer ce que je dois être, dites-moi ce qu’est tout ! De là cette quasi  nécessité pour une religion de se revêtir ou de se mêler d’une métaphysique. En attendant, cette religion s’est surchargée d’obscurités dont elle s’est rendue responsable. Ces obscurités augmentent de siècle en siècle par les explications toujours nouvelles et s’accumulant les unes sur les autres qu’on en donne ; et ainsi, aux yeux d’un peuple qui aime la clarté, plus une religion dure plus elle se détruit, parce que plus elle dure plus elle s’explique, et plus elle s’explique plus elle s’obnubile.\par
Une des forces d’une religion, c’est qu’elle est vieille ; une des faiblesses d’une religion, c’est aussi qu’elle est vieille, parce qu’elle s’est compliquée terriblement.\par
Je ne mets donc pas en doute qu’une des raisons de l’hostilité d’un grand nombre de Français contre la religion de leurs pères n’ait été « la clarté française ». La clarté française a du bon. Elle a aussi de très grands dangers. Elle nous persuade trop vite que tout est résolu, décidé, tranché, réfuté, ou irréfutable. L’intrépidité de certitude est un défaut français par excellence. Celui-là n’est guère français qui cherche en gémissant ; mais celui-là est très français qui affirme fermement ce qu’il n’a pas approfondi, ou qui nie en riant ce qu’il fait le ferme propos de ne pas approfondir.\par
Qui guérirait les Français, non pas de leur clarté  d’esprit, mais d’un certain excès dans la passion de voir clair trop vite, ne leur rendrait pas, peut-être, un maigre service. Ce n’est que dans les prairies que les « petits ruisseaux clairs » font de grands fleuves. Dans un peuple ils ne confluent pas, et ils ne forment qu’un concert de murmures agréables ; ou ils ne sont que de nombreux petits miroirs limpides où l’on se regarde soi-même avec un extrême contentement.\par
Le trop grand goût de clarté, c’est ce qu’on a appelé {\itshape le simplisme}. Le simplisme est le défaut des Français en politique ; il est aussi leur défaut en choses religieuses. « Ce qui n’est pas simple n’est pas vrai. » Axiome français. Or rien n’est simple, excepté le superficiel. Se condamner à n’admettre que le très simple, c’est se condamner à ne rien approfondir. Les Français ont une tendance à repousser les métaphysiques et les religions, qui n’est qu’une forme de leur horreur de creuser les questions.\par

\astertri

\noindent A cela leur légèreté naturelle contribue beaucoup, à ce point que ces choses se confondent et que l’on pourrait presque dire qu’au fond la « clarté française » n’est que de la légèreté. Par légèreté française il faut entendre l’impossibilité de s’occuper longtemps de la même chose, l’impossibilité  de s’obstiner, le manque de ténacité. Le Français n’est pas l’homme des œuvres de longue haleine et des entreprises à long terme. Il aime commencer, aime peu à continuer, finit rarement et voudrait avoir terminé très peu de temps après avoir commencé.\par
Ce n’est point paresse, à proprement parler. Le Français est extrêmement actif. Seulement le Français a une paresse active, ou si l’on veut une activité paresseuse. Il a une activité qui s’accommode de mille besognes courtes, et une paresse qui s’accommode de mille changements d’occupation comme d’autant de repos. Il est l’homme des « Expositions universelles » bâclées en deux ans, étalées six mois, démolies en trois semaines. Il fut l’homme des expéditions d’Italie, chevauchées rapides, retours hâtifs, nulles comme fondations, et dont il avait périodiquement comme la démangeaison d’abord et le dégoût ensuite. Il est l’homme des systèmes philosophiques tracés en grandes lignes brillantes, sans consistance, montant au ciel comme des fusées et retombant de même, après une illumination d’un instant. Il est l’homme des révolutions audacieuses et violentes et des réactions ou plutôt des dépressions profondes, qui succèdent presque immédiatement, et qui sont telles qu’il n’y a plus la moindre ressemblance apparente entre  l’homme d’hier et le même homme d’aujourd’hui. Le Français est la personne humaine dont il est le plus difficile d’établir l’identité.\par
Cette légèreté se marque dans sa vie privée, dans l’inconstance de ses amours, souvent très vives, rarement longues et persévérantes ; profondes, en vérité, puisqu’elles le prennent tout entier et quelquefois le brisent et le tuent, obstinées et tenaces, non pas, ou très rarement, et telles que, volant d’objets en objets, elles reviennent quelquefois vers le premier. Du Don Juan de Molière ôtez la méchanceté, et le Français n’est point du tout méchant ; au Don Juan de Molière ajoutez une véritable {\itshape sincérité} dans l’amour, et le Français est sincère à chaque fois, même quand il dit {\itshape toujours} ; du Don Juan de Molière retenez l’inconstance fondamentale et comme constitutionnelle, le désir de conquêtes, l’éternel besoin de plaire, l’éternel besoin d’être aimé promptement et légèrement, l’oubli rapide, l’impatience de toute obligation et de tout joug, l’incapacité de comprendre que l’amour est un contrat par lui-même et un lien qui ne peut se rompre que du consentement des deux parties : vous avez le Français dans le domaine des choses de l’amour.\par
La légèreté française est faite d’intelligence vive et vite fatiguée par la contemplation du même  objet ; de sensibilité vive et vite fatiguée de la possession du même objet.\par
Or, cette légèreté est à peu près incompatible avec la gravité religieuse, puisque le sentiment religieux est la contemplation d’un objet éternel. Ce qui fait la gravité religieuse, ce qui fait, du reste, le sentiment religieux lui-même, c’est la communion qui veut s’établir, qui s’établit en vérité, par une sorte de paradoxe sublime, entre un être d’un jour et un être d’éternité. Le sentiment religieux est la soif du permanent et de ce qui ne change pas. C’est un effort de l’être éphémère pour participer de l’éternel et de l’immuable.\par
Cette soif, le Français, par définition, l’éprouve peu, et cet effort, par définition, il est rarement tenté de le faire. Il n’est pas étranger à ce sentiment ; car il est homme ; mais il est, de tous les hommes peut-être, le plus étranger à ce sentiment. Il n’y a rien de plus vrai, quand on y réfléchit, que le mot magnifique de Lamartine : « L’homme se compose de deux éléments, le temps et l’éternité. » Il se compose de ces deux éléments, en ce sens qu’il est dans l’un et qu’il tend à l’autre, et comme il est fait également de ses conditions et de ses tendances, il est très vrai qu’il se « compose » de ces deux éléments, puisqu’il est dans l’un, puisqu’il  tend vers l’autre et vit en définitive de tous les deux. De ces deux éléments, il est certain que le Français vit trop dans le moment présent pour qu’il ne sacrifie pas, ou, tout au moins, pour qu’il n’oublie pas très souvent le second.\par
Il vit à court terme, d’une vie énergique, parfois « intense » ; mais comme saccadée et à brèves étapes. « Les Français, a dit Napoléon, sont des machines nerveuses. » Il n’y a rien, à ce qu’il peut paraître, de moins nerveux que l’éternel. Le Français et le divin ne sont évidemment pas très bien faits pour s’entendre. Je ne sais plus si Sainte-Beuve a dit de son cru ou dit en citant quelqu’un : « Dieu n’est pas français. » Il y a du vrai.\par

\astertri

\noindent Aussi bien, voyez comme ils comprennent Dieu quand ils {\itshape le font}, j’entends quand ils ne le {\itshape reçoivent pas}, quand ils n’en reçoivent pas l’idée toute faite d’une religion antérieure, antique, qu’ils ont acceptée ; mais quand, libres de tout lien dogmatique et affranchis de toute pensée traditionnelle et héritée, ils se font un Dieu à leur guise et à leur gré. Pour Montesquieu Dieu est la plus froide des abstractions. C’est la Loi des Lois, c’est \emph{l’esprit des Lois}, je l’ai dit sans aucune plaisanterie ; c’est « la raison primitive… qui agit selon les règles qu’elle a faites, qui les connaît parce qu’elle les a faites  et qui les a faites parce qu’elles ont du rapport avec sa sagesse et sa puissance » ; c’est la Loi des Lois, intelligente, personnelle — personnelle autant qu’il faut l’être pour être intelligente ; — mais rien de plus. Le Dieu de Montesquieu est un Dieu personnel qui est à peine une personne. Il n’y a pas un atome de sentiment religieux dans la religion de Montesquieu.\par
Pour Voltaire Dieu est un Lieutenant surnaturel de police, dont il a besoin pour que « la canaille » soit maintenue dans une crainte salutaire et pour que M. de Voltaire ne soit pas assassiné par ses domestiques.\par
Pour Rousseau, quoique Rousseau ait quelques traits vagues d’une âme religieuse, Dieu est, comme pour Voltaire, en dernière analyse, une idée qui importe à la conservation de l’état social et que l’État doit imposer par la force aux citoyens ; Dieu est un article important du Contrat social.\par
Et enfin, pour Diderot, Dieu n’existe pas et n’est qu’une invention de ces sophistes oppresseurs, qui, pour dominer l’homme, ont créé un « homme artificiel ».\par
Voilà comme les Français, quand ils sont dégagés de la trame des liens traditionnels, {\itshape inventent} Dieu, conçoivent Dieu. Ils l’inventent, ils le conçoivent de la façon la plus superficielle du monde.  Le profond sentiment religieux leur est à peu près inconnu.\par
Je n’ai pas besoin, ou à peine, de dire que les réactions religieuses (toujours mis à part la religion traditionnelle et le domaine où elle agit et l’empire qu’elle exerce, et ici comme plus haut je ne parle que des Français {\itshape inventant} Dieu) sont aussi superficielles que les théodicées philosophiques. Les poètes qui, de 1802 à 1850, ont exprimé des idées religieuses, ont fait preuve d’un sentiment religieux extrêmement inconsistant et débile. Tous ont été frappés des « beautés » de la religion, et non de sa grandeur, et non du besoin, en quelque sorte, constitutionnel, que l’homme en a. Eux-mêmes semblent en avoir eu besoin pour leurs œuvres et non pour leurs cœurs. La religion fut pour eux un excellent répertoire de thèmes poétiques. Leur génie fut plus religieux que leur cœur, et même ce fut leur art qui fut plus religieux que leur génie. Musset, peut-être seul, et un seul jour, eut un cri où se sent le véritable, profond, absolument sincère sentiment religieux, ou besoin de sentiment religieux. — La légèreté française, même chez les plus grands Français, est décidément un obstacle assez fort à la pénétration du sentiment religieux ; et l’état d’âme religieux n’est chose française que par exception et par accident.\par
 Outre le besoin de clarté apparente ou de clarté provisoire, outre la légèreté d’esprit, la vanité, très répandue chez les Français, ne va pas sans les écarter beaucoup des voies religieuses ou des chemins qui pourraient mener à la religion.\par
La vanité française est chose très différente de l’orgueil tel qu’on le peut voir et constater dans d’autres nations. L’orgueil est national et la vanité est individuelle. L’Anglais, l’Allemand, l’Américain sont extrêmement orgueilleux ; mais ils le sont surtout d’être américains, allemands et anglais. L’orgueil est volontiers collectif et, à être collectif, il est une force plutôt qu’une faiblesse. L’orgueil sous sa forme française, c’est à savoir la vanité, est tout à fait individualiste. Le Français s’admire en soi, de tout son cœur. Il se fait centre naturellement et se persuade très volontiers que la circonférence doit l’admirer.\par
Les mots d’enfants recueillis par Taine sont bien à leur place ici : une petite fille à son oncle qui lui demande ce qu’elle est en train de faire : « Ouvre les yeux, mon oncle, tu le verras. » Elle n’est pas fâchée d’indiquer à son oncle qu’elle le considère comme un imbécile. — Une autre enfant remarque entre son père et elle-même un trait de ressemblance : « Tu tiens de moi. » Elle rapporte  déjà tout à elle et a tendance naturelle à se tenir pour une cause et non pour un effet.\par
Tout vieux professeur a remarqué des inclinations toutes pareilles chez les jeunes Français. Ils n’ont presque aucune méchanceté ; mais leur amour-propre est immensurable. Deux traits essentiels : ils sont moqueurs et ils ne peuvent supporter la moquerie. C’est la vanité sous ses deux aspects. Ils se vantent à tout propos, ou, plus avertis, laissent seulement voir qu’ils se tiennent en haute estime. Tout petits, ils cherchent continuellement à attirer l’attention et ils semblent ne vivre que de l’attention qu’ils veulent attirer et ne vivre que quand ils l’attirent. Si parmi eux il en est un qui semble vivre d’une vie intérieure, qui soit réfléchi et méditatif et ne soit pas toujours comme en scène, il est traité de « sournois ».\par
Plus tard, adolescent, jeune homme, le Français, très souvent, est proprement insupportable. Il semble toujours avoir conquis le monde ou être sur le point de le conquérir, pourvu qu’il le veuille et qu’il fasse un geste pour cela. Il est tranchant, décisif et décisionnaire. Son opinion est la seule opinion qu’il considère et qui soit digne de considération. Il s’étonne qu’il puisse y en avoir une autre et qu’on fasse quelque attention à celles qu’il n’a pas. Surtout il croit toujours avoir inventé les  idées qu’il a ou qu’il pense avoir. Il expose des opinions très connues comme des découvertes qu’il vient de faire et qui attendaient sa venue au monde pour y paraître elles-mêmes. Tout au plus, il associe un nom célèbre au sien, non pas comme le nom d’un maître à celui d’un disciple, mais comme celui d’un égal à celui d’un égal. Il dit : « Comte et moi. » Il dit : « Renan et moi. » Se mettre de pair avec un grand homme est jusqu’où sa modestie puisse atteindre.\par
Tout jeune Français a inventé une philosophie, créé un art nouveau et improvisé un genre littéraire qui était inconnu.\par
C’est ce qui fait qu’il n’y a pas de peuple au monde, excepté peut-être le peuple grec, où il y ait autant de déclassés. Le déclassé est un homme qui se sent tellement né pour les grandes choses qu’il ne peut prendre sur lui de faire les petites ou les moyennes. Reste qu’il ne fait rien et traîne d’expédients en expédients en rêvant toujours des grands rôles auxquels il était destiné et que les circonstances l’ont empêché de jouer. La France contient beaucoup de ces dévoyés qu’un peu de connaissance de soi aurait préservés.\par
On s’étonne qu’il n’y en ait pas davantage ; car tous les Français, à bien peu près, sont terriblement vains. Mais il faut reconnaître que, tout à  côté de leur vanité, ils ont un certain sens pratique. S’ils ne se rendent pas compte d’eux-mêmes, ils se rendent assez bien compte des choses. Ils reconnaissent qu’il faut se résigner à la médiocrité de situation, tout en gardant la haute opinion de soi-même, qui console et qui réconforte. « Le mérite console de tout », disait Montesquieu. C’est la devise de la plupart des bourgeois français.\par
De là tant de petits employés qui sont des poètes et qui lisent leurs vers en famille et à leurs amis, en se disant qu’il ne leur a manqué que quelque loisir et une petite fortune indépendante pour être des Lamartine ; qui ont un système politique et toute une sociologie et qui gémissent de l’obscurité où ce système reste enseveli avec eux ; qui font des romans et des pièces de théâtre et poursuivent toute leur vie le rêve d’être imprimés ou d’être joués ; du reste, ponctuels à leur bureau, sinon zélés, et acceptant en maugréant, mais relativement avec patience, la vie terne que l’injuste destin leur a faite. La France est pleine de grands hommes inconnus de tous, mais très manifestes à eux-mêmes, que le silence accable, et de très honnêtes petits employés des contributions indirectes qui se répètent sans cesse à eux-mêmes : « {\itshape Qualis artifex pereo} ! »\par
Peut-être cela ne va-t-il point sans quelque inconvénient  pour le métier que ces honnêtes gens exercent et les fonctions que l’État leur confie, et ce serait un point à considérer, mais qui nous écarterait de notre sujet.\par

\astertri

\noindent Pour n’en point sortir, considérez et ces hommes dont leur vanité fait des déclassés, ou des demi-déclassés, ou des {\itshape inquiets} et des neurasthéniques, légion qui en France est une armée ; et ces autres hommes, moins privés du sens du réel et qui, à cause de cela, se classent, mais tout aussi vains, tout aussi prétentieux, tout aussi enivrés de {\itshape sens propre} ; et voyez, au point de vue religieux, ce qu’ils peuvent être.\par
Ils ont comme une tendance instinctive à repousser ce sentiment. Le sentiment religieux en général, le christianisme en particulier, le catholicisme plus particulièrement encore, est avant tout humilité. Il est avant tout reconnaissance et confession du peu que nous sommes pour connaître et pour agir. Et, pour ce qui est de la connaissance et pour suppléer à notre impuissance en cet ordre, les religions ont inventé la révélation. Et pour ce qui est de l’action et pour suppléer à notre débilité en cette matière, elles ont inventé la grâce. L’acte d’humilité est la première démarche religieuse ; le premier mot de l’homme qui est attiré  vers Dieu est : « Ma substance n’est rien devant vous. »\par
Si l’humilité est le principe de toute religion, nos Français sont bien peu nés pour être religieux, et la religion, comme l’a très bien vu Pascal, est quelque chose comme leur ennemie personnelle. Ils la voient comme un personnage doucement ironique qui rabat leur superbe et qui se moque de leurs prétentions. Chacun la voit comme quelqu’un qui lui dirait : « Votre plus cher entretien est de vous croire quelque chose et vous n’êtes rien du tout. » Le Français est l’être du monde entier qui aime le moins qu’on lui dise cela. Contester à un Français sa puissance de savoir, de connaître et de décider, cela, pour lui, ne se peut souffrir. Lui contester son infaillibilité, dont, au fond, et même quand il évite de la proclamer, il est toujours convaincu, cela ne vous fait pas de lui un ami. Tout Français est un Voltaire ou un Rousseau, moins le génie, qui n’est pas loin de croire qu’il est impossible qu’il n’ait pas toujours raison, qui n’est pas loin d’estimer que les autres hommes sont imbéciles dans la proportion où ils le contredisent et qu’il n’y a besoin ni d’autre signe ni d’autre mesure.\par
La religion, selon cette façon de juger, ne peut qu’avoir tort. Ils l’accusent à la fois de présomption et de bassesse, de superbe et de lâcheté. De superbe, parce qu’elle prétend imposer ses décisions ; de  lâcheté parce que, si elle nie chez les autres la capacité de connaître, elle se la refuse à elle-même et ne se croit en possession de la vérité que parce que celle-ci lui a été révélée. Rien ne peut être à la fois plus insupportable à la vanité du Français et plus propre à exciter sa raillerie, pleine de vanité encore, que cette double formule : « Vous ne savez rien et êtes incapable de rien savoir ; et, du reste, nous sommes exactement dans les mêmes conditions ».\par
Il y a beaucoup d’analogie entre la situation de l’Église en face des Français et celle de Socrate en face des Athéniens : « Je sais que je ne sais rien, disait Socrate ; et vous, vous ne savez rien en croyant savoir quelque chose. »\par
« Fausse humilité, répondaient les Athéniens, pour ce qui est de ce que tu dis de toi-même ; insolence pour ce qui est de ce que tu dis de nous. Et dans ton humilité, insolence encore, car tu ne dis que tu ne sais rien que pour faire entendre à quel point nous sommes vains de croire savoir quelque chose quand Socrate ne sait rien et confesse ne rien savoir. »\par
De même le Français en veut autant à l’Eglise de sa négation du savoir humain qui le blesse, que de son humilité qui elle-même l’humilie. Vanité française et humilité ecclésiastique, que cette humilité  soit par l’Église commandée aux autres ou qu’elle soit pratiquée par elle, ne peuvent faire bon ménage ensemble.\par
Remarquez, comme j’en ai déjà dit quelque chose, que l’orgueil est plus compatible avec la religion que la vanité, pour cette simple raison que la vanité est individuelle et que l’orgueil peut être collectif. C’est même ici la vraie distinction, ou l’une au moins des vraies distinctions entre ces deux sentiments. Sans doute, l’orgueil opposé à la vanité, c’est surtout un sentiment puissant opposé à un sentiment mesquin : l’orgueil est une exaltation, la vanité est une démangeaison ; l’orgueil est une grandeur fausse, la vanité est une petitesse ; l’orgueil ne se satisfait que des grands succès et dédaigne les médiocres jouissances, la vanité se repaît de tout et ne dédaigne rien ; l’orgueil n’atteint jamais son but, tant il le met haut ; la vanité, quoique insatiable, atteint ou manque son but tous les jours et presque à chaque heure, le mettant partout.\par
Tout cela est vrai ; mais on n’a pas assez remarqué qu’une des distinctions, et très considérable et très significative, entre l’orgueil et la vanité, c’est que la vanité est individuelle et que l’orgueil peut être collectif. Il est souvent individuel lui-même, mais il {\itshape peut être} collectif. La vanité ne peut  être qu’individuelle. Elle consiste à vouloir se distinguer, à tous moments et par toutes sortes de choses, de tous les autres, et à montrer que l’on est un être tout particulièrement privilégié. Elle est individuelle par définition. Elle n’est même que l’individualisme lui-même, qu’un individualisme enfantin et naïf. Elle dit sans cesse : « Moi… Moi, au contraire… Tandis que moi… » L’orgueil souvent dit : « Moi. » Mais il peut dire : « Nous. » C’est une sensible différence.\par
L’orgueil peut se satisfaire et presque se remplir dans la contemplation d’une grande œuvre accomplie en commun. L’orgueil romain fut collectif ; l’orgueil anglais, l’orgueil allemand sont collectifs. On ne peut guère dire : « la vanité nationale », et l’on dit très bien : « l’orgueil national ». L’orgueil ne fait jamais abstraction du moi ; mais, précisément parce qu’il est un sentiment grand et fort, il peut sentir le moi s’exprimer, se déployer et triompher dans une grande œuvre faite à plusieurs. L’homme vain dit : « Moi. » L’orgueilleux peut très bien dire : « {\itshape civis romanus sum} », ou : « je suis anglais ; je suis allemand », et trouver à le dire une immense satisfaction de son orgueil même.\par
Un des phénomènes de l’histoire de France est précisément ceci que certains hommes ont trouvé le moyen de transformer la vanité des Français  en orgueil : Louis XIV, Napoléon. Sous l’empire de l’un et de l’autre, le Français a cessé d’être vain pour devenir orgueilleux. Il a confondu sa personnalité dans l’ensemble de la communauté française ; et, dans les succès et dans la grandeur de cette communauté, il s’est enorgueilli de telle sorte qu’il a presque oublié les sollicitations de sa vanité individuelle. — Mais ceci n’est, pour ainsi parler, que de l’orgueil intermittent. Le véritable orgueil n’a pas besoin du succès et de la gloire pour être entier, pour être sans défaillance et pour être actif. Aux heures de deuil et même d’écrasement, il demeure ferme, et conçoit et il prépare les revanches, les relèvements et les restaurations futures. La vanité française ne devient orgueil collectif qu’assez rarement et sous l’impulsion d’une volonté puissante et dans l’exaltation d’une grande gloire acquise. A l’état normal, elle est simplement vanité individuelle.\par
Or, c’est la vanité qui est incompatible avec la religion et non pas l’orgueil ; ou la vanité est beaucoup plus incompatible que l’orgueil avec la religion. L’orgueil, sans doute, peut mépriser la religion, et ce n’est pas à tort que la religion a fait de l’orgueil un péché ; mais la vanité la méprise bien plus encore, ou se sent beaucoup plus atteinte et mortellement blessée par elle. L’orgueil peut  s’accommoder de cette œuvre collective qu’est la religion et même y trouver son compte et sa satisfaction. On peut être fier d’être chrétien, comme on est fier d’être romain, ou comme on est fier d’être anglais ou allemand. On peut être fier d’appartenir à une institution qui a transformé l’humanité. On peut être fier d’appartenir à une collectivité qui a comme substitué un genre humain à un autre genre humain. Personne n’ignore que l’orgueil, s’il est qualifié de péché par l’Église, est précisément un péché très ecclésiastique. — Mais comment veut-on que la vanité puisse supporter la religion ? Elle consiste précisément à repousser tout ce qui est collectif ; elle consiste précisément en ceci qu’un homme est secrètement convaincu « qu’il n’y a que lui ». Elle consiste à se traiter intimement d’excellence et d’éminence. Elle consiste à ne guère admettre qu’un autre que vous puisse avoir complètement raison, ou qu’un autre que vous réalise pleinement en lui l’humanité.\par
Tout ce qui est collectif répugne donc comme naturellement à l’homme vain. Il en serait plutôt comme jaloux. Il voit en une collectivité des hommes qui, contrairement à lui, font abstraction de leur personnalité ; et différence engendre haine. Il voit en une collectivité des gens, aussi, qui, par la  force que donne l’union, peuvent l’offusquer, lui, et l’éclipser, et il en prend ombrage ; et jalousie engendre haine.\par
L’homme vain est donc anticollectif par définition, et, par parenthèse, le furieux individualisme des Français qui les rend ennemis de toute caste, de toute classe, de toute corporation, n’est qu’une forme de leur vanité. Or, si le Français, de par sa vanité, est {\itshape déjà} ennemi de toute collectivité, dans quels sentiments voulez-vous qu’il soit à l’égard d’une collectivité qui, d’abord est une collectivité, et qui ensuite est une collectivité qui recommande et commande l’humilité comme la première des vertus humaines ? Non, il est très difficile que l’homme vain soit religieux ; et il est très facile que l’homme vain soit ennemi de la religion, ou, tout au moins, ait à son égard quelque impatience.\par

\astertri

\noindent Ce qui suit n’est qu’un autre aspect de ce qui précède et n’est, au fond, qu’à très peu près la même chose. Pour ces mêmes raisons de vanité et de fanfaronnade, le Français a horreur {\itshape de la tradition}. Que quelque chose, institution, loi, maxime publique, mœurs, idée généralement répandue, ait régné jusqu’au jour où il naît, et semble avoir fait la grandeur de sa nation ou y avoir contribué, ce lui est une raison pour n’y pas tenir et pour la  repousser instinctivement plus ou moins fort. Il y a des peuples pour qui le mot « antiquité » a un grand prestige ; pour le Français « antiquité » est « vieillerie », et vieillerie est ridicule et absurdité.\par
Pour beaucoup de Français, la nouveauté d’une idée est preuve qu’elle est juste. Une idée vraie, c’est une idée nouvelle : il ne faut pas chercher davantage : le criterium est aisé. La plupart des Français sont parfaitement convaincus que l’on n’a commencé à faire usage de la raison qu’à dater du moment où ils ont eu dix-huit ou vingt ans et que tout ce qui a précédé cette époque ne fut que ténèbres. C’est une illusion assez naturelle à la jeunesse, et qui même ne laisse pas d’avoir sa part d’utilité ; mais c’est une illusion assez forte cependant, et qui a, somme toute, plus d’inconvénients que d’avantages.\par
Il en résulte une chose que l’on n’a peut-être pas assez remarquée : c’est l’effet tout particulier que l’éducation a en France. L’éducation, en France, a pour effet de convaincre la génération éduquée juste à l’inverse de la génération éducatrice ; de suggérer à la génération éduquée toutes les idées contraires à celles de la génération éducatrice. Le mépris des fils pour les pères et des élèves pour les maîtres est, en France, très général, et il semble  très légitime. Ceux-là ne sont-ils pas jeunes et ceux-ci ne sont-ils pas vieux ? Que faut-il davantage ?\par
Tout n’est pas mauvais en cela ; car il faut certainement que chaque génération cherche par elle-même et ne s’endorme pas sur la parole d’autrui ; et ce n’est pas sans quelque raison, forme paradoxale mise à part, qu’un père me disait : « Oui, les enfants méprisent les pères : c’est providentiel. » Mais il faudrait que cet esprit d’indépendance ne fût pas accompagné d’une grande légèreté, d’une grande vanité, ou plutôt n’eût pas légèreté et vanité pour ses sources mêmes.\par
Toujours est-il que les choses vont ainsi ; et tout père français peut être sûr que son fils a pour lui une douce pitié, tempérée par un assez faible respect ; et tout professeur français peut croire aux mêmes sentiments chez ses élèves, chez ceux, du moins, de ses élèves qui n’ont pas pour lui une complète indifférence.\par
La nation entière a un peu le même caractère, et voici pourquoi. Si l’amour des nouveautés paradoxales n’était qu’une maladie de jeunesse dont chacun guérit vers la trentaine, il n’en serait que cela, et le corps même de la nation resterait sain et resterait ferme dans des idées générales traditionnelles qui feraient sa force morale et sa vigueur  intellectuelle. Mais l’empêchement à cela, c’est que le Français n’a pas de passion plus violente que celle qui consiste à ne point vouloir paraître vieux, et à ne point s’avouer qu’il le soit. Il en résulte que les idées des jeunes gens, ces idées qu’ils se sont données en prenant juste le contrepied des idées de leurs pères, leurs pères eux-mêmes les prennent à leur tour, gauchement, maladroitement, lourdement ; mais enfin ils les prennent, en tout ou en partie, pour ne point paraître démodés, surannés et tout encombrés de « vieilleries ». — « Nous aussi, nous sommes modernes. Nous aussi, nous marchons avec notre temps. » Ils marchent, si suivre peut s’appeler marcher.\par
C’est ainsi que, même sans souci de popularité, même sans souci de rester en bons termes avec les gouvernements nouveaux et de se tenir toujours du côté du manche, même sans tout cela, tel vieux brave homme de 1906 exprimera, étalera en ses propos un mélange prodigieux de ses idées de 1869, de ses idées de 1872 et de ses idées, pour ainsi parler, de 1906, et offrira aux regards une synthèse bizarre de Duruy, de Gambetta, de Jules Ferry et de M. Jaurès. Je rencontre un brave homme de ce genre, plus ou moins éloquent, plus ou moins spécieux, plus ou moins brillant, et, mon Dieu, toujours aussi sincère, ou à peu près, toutes les  fois que je me promène. Cela est si délicieux de ne point vieillir, de croire que l’on ne vieillit pas et de croire persuader aux autres qu’on ne vieillit point ! Presque tout Français de soixante ans est un vieux beau qui suit la mode, qui en a le respect et aussi comme une espèce de terreur, et qui serait désespéré si l’on pouvait soupçonner qu’il n’est plus homme à la comprendre.\par
C’est ainsi que, chez d’autres, chez les outranciers et les inquiets et les nerveux, le désir de n’être pas {\itshape dépassés} va jusqu’à être plus {\itshape avancés}, non seulement qu’ils n’ont jamais été dans leur jeunesse, mais que ne sont même les jeunes gens qui les entourent et dont ils veulent passionnément être entourés toujours. Ils voudraient les effrayer par leur jeunesse. Ils multiplient les audaces, les témérités et les bravades. Ils sont iconoclastes avec une frénésie croissante. Ils brisent davantage à mesure qu’ils ont les mains plus faibles. De ceux-ci la ligne de vie est assez bizarre. Ils ont été sages et modérés, quelquefois d’un doux scepticisme, dans leur jeunesse ; ils sont affirmatifs et ils sont révolutionnaires dans leur âge avancé. Les années, au lieu de les mûrir, les rendent acides. Ils semblent avoir marché à reculons dans le sentier de la vie et avoir cheminé, à partir de vingt ans, d’une douce et sereine vieillesse à une jeunesse  inquiète et à une adolescence tumultueuse.\par
Remarquez que le phénomène n’est pas nouveau en France et que c’est (surtout) à partir de son âge mûr que Voltaire, et lui aussi, ce semble bien, pour suivre la mode et courir après la popularité, est devenu partiellement révolutionnaire, avec le tempérament le plus conservateur du monde, poussant toujours de ce côté avec plus d’âpreté et de violence. Il faut dire seulement, à l’éloge de son bon sens et aussi, relativement, de son courage, il faut dire, pour être juste, que cependant il n’a pas eu tout à fait la terreur d’être dépassé et a laissé d’autres aller plus loin que lui en disant nettement qu’il n’allait pas jusqu’où ils allaient.\par
Tant y a que le Français, par une horreur bien naturelle de la vieillesse, n’aime les vieilleries ni quand il est jeune ni quand il est vieux, s’en détourne passionnément quand il est jeune, tend à n’en pas être soupçonné quand il est vieux, a donc toujours tendance à s’attacher aux nouveautés, sans les examiner et pour ce mérite seul qu’elles sont nouvelles, mérite incomparable aux yeux de sa vanité, de sa légèreté et de sa « jeunesse ».\par
Une vieillerie comme la religion ne saurait donc lui plaire et il s’en détourne, communément, avec une sorte de dédain et de hauteur. Le Français est « esprit fort » dans l’âme, de par tous  les défauts qui se trouvent ordinairement en lui.\par

\astertri

\noindent On me dira sans doute que cette tendance même qui fait que chaque génération française tient essentiellement à prendre le contre-pied de la génération précédente doit avoir pour effet qu’une génération religieuse succède à une génération d’esprits forts et une génération d’esprits forts à une génération religieuse, et ainsi de suite, de telle sorte que les gains et les pertes finissent par se contrebalancer et que le parti religieux reste, en définitive, à très peu près sur son terrain.\par
Partiellement, c’est bien en effet ce qui arrive ; et, précisément par suite de ce goût qu’ont les Français de désapprendre ce qu’ils ont appris, par cet effet de l’éducation chez les Français qui est de les faire réagir contre l’éducateur. C’est ainsi que le mouvement de renaissance religieuse de 1800 à 1840 environ est très évidemment l’effet des fureurs antireligieuses de la seconde moitié du \textsc{xviii}\textsuperscript{e} siècle. C’est ainsi que l’anticléricalisme de 1840 à 1870 est une réaction contre la « religiosité » de Chateaubriand et de ses disciples et aussi contre les manifestations religieuses du parti ecclésiastique pendant la Restauration. C’est ainsi que les catholiques ou simplement les libéraux attendent  beaucoup en ce moment de la stupidité et de l’outrecuidance de la plupart des instituteurs français et ne doutent point que leurs élèves ne soient un jour, et prochainement, ou grands amis de l’Église ou très déférents à son égard.\par
Tout cela est assez vrai ; mais il faut remarquer d’abord, comme je l’ai montré, que le besoin, dans l’esprit de chaque génération, de mépriser la génération précédente n’est pas la seule cause, mais une des causes seulement de l’anticléricalisme français, ce qui fait que quand cette tendance agit en faveur du sentiment religieux, les autres causes d’anticléricalisme subsistent et ont leur effet compensatoire. Et il faut remarquer ensuite que, de par son antiquité, la religion est en quelque sorte la vieillerie par excellence, la plus vieille des vieilleries, défaut dont le temps ne la guérit pas ; et, par conséquent, même devant une génération qui serait assez portée à l’accepter par réaction contre ce qu’on lui a enseigné, a toujours ce tort grave d’être ce qu’on a cru autrefois et pendant si longtemps qu’il y a vraiment quelque honte à y croire encore.\par
Tous les réformateurs religieux de ces derniers siècles, quoiqu’ils aient échoué, avaient raisonné assez juste. Ils s’étaient dit : « Les Français ne s’éprendront d’une religion que si elle est nouvelle.  Inventons donc une religion inédite. » Ce n’était pas mal pensé. Le malheur, c’est que ces religions inédites ressemblaient singulièrement à l’ancienne et ne pouvaient pas ne point lui ressembler beaucoup ; et c’est pourquoi, pour les traditionistes ayant le tort d’être nouvelles et pour les curieux de choses vraiment nouvelles ayant le ridicule d’être des vieilleries dissimulées, elles n’ont eu de succès auprès de personne.\par
La légèreté française sous cette forme : inconstance, versatilité, horreur de la tradition, mépris de l’enseignement reçu, tendance antidomestique, est donc un très grand obstacle à l’influence du sentiment religieux. L’Église avait l’habitude de se faire appeler « notre sainte mère l’Église ». Relativement aux Français, c’était une faute : « C’est une mère. Comment voulez-vous que je la respecte ? »\par

\astertri

\noindent L’{\itshape immoralité française} — je m’expliquerai tout à l’heure sur ce gros mot — est chose encore qui contrarie l’influence du sentiment religieux. Le Français n’est pas immoral. Du moins, il ne l’est pas plus que les hommes des autres peuples, et peut-être l’est-il moins. J’ai tendance, sans m’avancer, en une matière où les statistiques, même à demi exactes, sont impossibles, à croire qu’il l’est  moins, à cause de sa légèreté même, du peu de violence de ses passions ; à cause aussi d’un certain sentiment d’élégance en toutes choses, qu’il a toujours, même un peu dans les classes inférieures, et qui, certes, n’est pas du tout la moralité, mais n’est pas sans y contribuer et assez fort ; à cause aussi de sa bonté, qui est réelle et qui est un frein à la basse débauche : le grand débauché est toujours cruel ; à cause enfin de ceci que le Français est le seul peuple du monde (avec le peuple américain) qui se laisse mener par les femmes, au lieu de les traiter durement et despotiquement.\par
Cette dernière tendance a d’immenses inconvénients : une partie de la force des peuples européens autres que la France consiste en ce que les hommes sont maîtres chez eux et que les femmes n’ont aucune influence sur leurs décisions et sur leurs volontés ; une partie de notre faiblesse vient de ce que les femmes ont un immense empire sur nous et nous efféminent. Mais, au point de vue de la moralité, qui ne voit qu’un peuple où les femmes dominent ne peut pas être très immoral ? Les femmes, je parle de la majorité des femmes, ne le permettent pas. Le peuple immoral est celui où les femmes sont considérées comme des choses ou comme des êtres inférieurs et se sont habituées à être considérées ainsi et sont passives et s’abandonnent  aux désirs avec une sorte d’inertie. Le peuple où la femme est forte, sans être d’une moralité absolue, ni même extraordinaire, est maintenu, par la dignité de la femme, très loin de la basse immoralité.\par
De là ce foyer français, que les étrangers connaissent très bien, qu’ils prennent plaisir à nier ou à moquer, sur la foi de nos stupides romanciers galants, mais sur lequel ils ne se trompent point et dont ils connaissent très bien les mérites ; ce foyer français qui a quelques ridicules, où la femme est trop maîtresse et où l’homme n’est souvent que l’aîné des enfants, ce qui, du reste, est délicieux ; mais ce foyer français, presque toujours très chaste, très honnête, très fermé et très jaloux de son intimité, et de son secret, et de son bonheur.\par
Non, je ne crois pas le peuple français plus immoral qu’un autre peuple, et même je crois, avec toutes sortes de raisons pour le croire, qu’il est à un très haut degré, dans l’échelle de la moralité, parmi les nations.\par
Seulement le Français a une manie, qui est de rougir de la moralité, et de croire que la moralité est ridicule, et de ne point vouloir avouer qu’il est moral, et d’être un fanfaron de vices ou tout au moins de libertinage.\par
Je crois bien que la France est le seul peuple du  monde où la chasteté soit un ridicule. Elle en est un chez lui. Le moindre courtaud de boutique, laid, gauche, lourdaud et imbécile, se flatte et se vante de ses succès féminins, d’autant plus qu’il en a moins, et pour se poser avantageusement dans le monde. L’adolescent le plus timide n’a peur que d’une chose, c’est qu’on le croie vierge. Il vaincra sa timidité avec des efforts de volonté surhumaine pour prouver qu’il ne l’est pas ; ou il mentira violemment pour assurer qu’il ne l’est point.\par
L’homme mûr n’est pas exempt de cette maladie, et si le bel air chez les jeunes gens, tout au moins de la bourgeoisie et du peuple, est de se donner pour corrompus, le bel air chez les hommes qui sont au milieu de la vie est de se laisser soupçonner d’être infidèles à leurs femmes et de n’être point liés à cet égard par de sots scrupules.\par
Les vieillards mêmes ne détestent pas laisser planer sur eux une petite légende de libertinage. C’est une élégance ; et ils sont extrêmement flattés qu’on les considère comme atteints de cette turpitude et de cette fâcheuse maladie.\par
C’est placer sa vanité d’une façon un peu singulière. Mais c’est ainsi. Où les autres peuples mettent leur hypocrisie, nous mettons notre fanfaronnade, et où ils mettent leur pudeur, nous mettons  notre ostentation. Ce qu’ils font en se cachant, nous nous en vantons en ne le faisant pas. Ce dont ils rougissent, c’est ce dont nous sommes fiers, et ce dont ils se flattent, faussement, du reste, c’est ce que nous ne pouvons pas prendre sur nous d’avouer. La chasteté est une vertu dont ils se targuent sans l’avoir et dont nous avons la plus grande honte du monde, même quand nous la pratiquons. Le libertinage en France est une tradition nationale que nous sommes beaucoup plus enclins à proclamer qu’à soutenir.\par
Il est bien évident que ceci est encore un effet de notre vanité. Mais pourquoi avons-nous mis notre vanité en ceci et non ailleurs ? Je ne sais trop. J’ai souvent pensé que c’était peut-être précisément parce que nous sommes bons et faibles et très faciles à nous laisser mener par les femmes. Précisément à cause de cela et parce que nous le savons, nous ne voulons pas en convenir, et c’est ici que se place notre vanité et qu’elle agit. Ne voulant pas convenir que nous nous laissons mener par les femmes, nous nous attribuons ce rôle de séducteurs, de conquérants et de Don Juan qui, en effet, est bien celui des hommes qui méprisent les femmes, les dominent, se moquent d’elles et sont leurs maîtres au lieu d’être leurs serviteurs. Oh ! que ce rôle nous agrée, en ce qu’il montre  bien et prouve d’une manière éclatante que nous ne subissons pas l’empire des femmes, de la nôtre surtout, et que nous sommes des hommes forts ! Comme il nous agrée, en ce qu’il nous donne justement comme le contraire de ce que nous sommes et de ce que nous n’aimons pas à convenir que nous puissions être ! Comme il nous agrée, en ce qu’il dissimule notre faiblesse ! Comme il nous agrée en ce qu’il nous déguise !\par
Voilà mon explication, une, du moins, des explications que je me suis données de ce travers français, qui est le plus distinctif et le plus caractéristique de nos travers.\par
Quoi qu’il en soit de cette hypothèse psychologique, le travers existe, et je me résumerai en disant : le Français n’est pas immoral ; mais il tient infiniment et il prend un plaisir infini à passer pour l’être.\par
Or, au point de vue des choses religieuses, être immoral ou en jouer le rôle a exactement les mêmes effets. L’homme immoral ou l’homme qui tient à passer pour l’être s’écarte naturellement d’une religion qui recommande la chasteté et qui la considère comme une grande force et une grande vertu. Il fait partie du rôle du jeune homme qui se flatte de mépriser la chasteté, tout comme un autre, de s’écarter avec grand soin de la société  des prêtres et de toute pratique religieuse. Toute l’économie du rôle qu’il joue devant ses amis et {\itshape devant lui-même} serait ruinée par cette contradiction déplorable et ridicule. M. Homais, qui est la raison même en ces matières, ne manque pas de dire, à propos d’un jeune homme de province qui va faire son droit à Paris : « Il ne fera pas de sottises ; c’est un jeune homme sérieux. Il faudra bien pourtant qu’il fréquente un peu les filles pour n’avoir pas l’air d’un jésuite. » Voilà la vérité. Le Français affiche l’immoralité par amour-propre et s’écarte de la religion pour soutenir son bon renom d’homme immoral ; et, réciproquement, il est immoral, sans en avoir la moindre envie, pour ne point passer pour être clérical. Ces choses sont cause et effet, et une fois l’engrenage en mouvement, le Français devient d’autant plus anticlérical qu’il tient plus à passer pour libertin, et d’autant plus libertin, au moins en apparence et d’enseigne, qu’il s’efforce d’être anticlérical.\par
La langue ne s’y est pas trompée, et c’est comme un contrôle. Elle a appelé {\itshape libertin} : d’abord « l’esprit fort », et ensuite le débauché et, du reste, concurremment l’un et l’autre ; mais d’abord surtout « l’esprit fort », et ensuite surtout le débauché. Cela est naturel. Ce qui a frappé d’abord, c’est l’{\itshape affranchissement} qu’affichaient certains philosophes  à l’égard des dogmes religieux ; de là {\itshape libertin} dans le sens de libre penseur. Ce qui a frappé ensuite, c’est l’attitude arrogante des débauchés se faisant gloire de leurs mauvaises mœurs. Le peuple s’est dit, très finement : « Ceux-ci sont surtout des hommes qui sont fiers de ne subir aucune règle et qui veulent le montrer par leur genre de vie. Ce sont des {\itshape libertins} en acte. Ce sont essentiellement des libertins. Le nom leur en restera. »\par
Et cela veut dire, non pas que la libre pensée mène à l’immoralité ; mais que l’affectation de l’immoralité mène à l’affectation de la libre pensée et se couvre de la libre pensée comme d’un beau manteau philosophique.\par
Molière non plus ne s’y est pas trompé et il a fait de son Don Juan à la fois un débauché et un athée : d’abord pour peindre un homme d’une certaine classe de son temps ; ensuite, comme toujours, pour peindre un homme éternel en France. Don Juan est dédaigneux de morale et dédaigneux de religion en même temps, également, et l’un et l’autre surtout par vanité et « par air », à la française. Une de ses joies, car il en a d’autres, et je n’oublie pas qu’il est très complexe, est de scandaliser le peuple, les simples, représentés par Sganarelle. C’est ce qu’il a de particulièrement français.\par
Le Français fait tout, ou il s’en faut de peu, par  vanité. La vanité est son grand ressort. Toute l’immoralité qu’il peut avoir ou qu’il peut affecter tient à cela. L’étranger ne déteste pas être libertin sans qu’on le sache ; le Français aimerait mieux se priver de voluptés de toute sa vie et passer pour un mauvais sujet, que posséder toutes les femmes et être aimé d’elles et être tenu pour coquebin. — Il y aurait un joli sujet de comédie : \emph{Don Juan blanc}. Ce Don Juan-là, pour une raison qu’il faudrait trouver, ou même sans raison, n’aurait jamais aucun succès de galanterie ; mais il aurait la réputation, habilement entretenue, soit par lui-même, soit par un autre qui aurait intérêt à cela, d’être logé à l’enseigne du « grand vainqueur » et de tenir toutes les promesses de son affiche. Cet homme-là, né français, serait le plus heureux des hommes.\par
Dans ces conditions, on comprend assez que la moralité, encore qu’elle soit en usage chez les Français, n’y soit pas en honneur, et peut-être y soit aussi peu en honneur qu’elle y est en usage. Peut-être y a-t-il proportion juste, et s’il était ainsi, on ne saurait croire à quel point les Français seraient moraux ; ils le seraient presque avec excès.\par
Toujours est-il que l’affectation d’immoralité et toutes les habitudes d’esprit, de conduite et d’attitude que cette affectation entraîne détachent les  Français de la religion, du sentiment religieux et de l’état d’esprit religieux. De très bonne heure et comme tout de suite, longtemps avant d’être devenu formellement catholique, l’illustre philosophe M. Brunetière avait horreur de la grivoiserie française, de la gauloiserie, et la considérait comme une plaie honteuse de la littérature française. C’était comme d’instinct, et celui qui avait en lui comme les germes et les semences de l’esprit religieux sentait bien que grivoiserie et gauloiserie n’étaient pas autre chose qu’à la fois les effets et les causes de l’irréligion et devaient être traitées non comme un travers désobligeant, mais comme une maladie profonde et d’autant plus funeste qu’elle est endémique.\par

\astertri

\noindent Telles sont, à ma connaissance, les causes psychologiques les plus générales de l’anticléricalisme en France. Sans atteindre la nation tout entière, elles sont répandues, et depuis très longtemps, inégalement, du reste, mais universellement, dans toutes les classes de la nation et dans toutes les régions du pays.\par
Il peut être utile maintenant de considérer l’anticléricalisme français dans la suite de son histoire.
 \section[{Chapitre II. L’anticléricalisme au XVIIe siècle.}]{Chapitre II.\\
L’anticléricalisme au XVII\textsuperscript{e} siècle.}\renewcommand{\leftmark}{Chapitre II.\\
L’anticléricalisme au XVII\textsuperscript{e} siècle.}

\noindent D’une part, l’anticléricalisme a existé en France au \textsc{xvii}\textsuperscript{e} siècle ; mais il y a été très faible et très peu répandu ; d’autre part, aucun siècle n’a plus préparé l’anticléricalisme en France que le \textsc{xvii}\textsuperscript{e} siècle.\par
Ce sont ces trois propositions que nous examinerons dans ce chapitre.\par
L’anticléricalisme a existé en France au \textsc{xvii}\textsuperscript{e} siècle, surtout dans le premier tiers de ce siècle et dans le troisième. Dans le premier tiers il était représenté par un certain nombre d’écrivains, et c’étaient les Théophile de Viau et les Cyrano de Bergerac, nourris de Montaigne, mais plus audacieux que Montaigne ; nourris de Lucrèce, mais moins systématiques que lui et se plaisant dans la négation pure et simple ; qui, pour se plaire à eux-mêmes d’abord et pour plaire ensuite à quelques grands seigneurs licencieux, leurs protecteurs, faisaient comme marcher de pair le licencieux et  l’incrédulité et flattaient ainsi deux passions basses assez répandues alors dans les hautes classes.\par
Il ne faut pas oublier qu’à cette époque, le gouvernement étant aux mains de prêtres, Richelieu, Père Joseph, si peu prêtre, du reste, que fût l’un d’eux, c’était faire acte d’opposition ou d’indépendance ou de taquinerie, choses très chères aux Français, que d’affecter l’incrédulité et le libertinage. L’anticléricalisme a été, vers 1630, une attitude aristocratique ; l’anticléricalisme a été, vers 1630, très grand seigneur.\par
Il était grossier, du reste, et impudent. Il était immonde. Il se roulait dans les fanges des \emph{Parnasses satyriques}. Il était, en très parfaite exactitude, à pieds de satyre. Il conduisait à la place de Grève en réalité ou en effigie. Il était en horreur à la majorité de la nation.\par

\astertri

\noindent Il fut à la fois plus décent et plus scientifique, un peu, dans le troisième tiers du \textsc{xvii}\textsuperscript{e} siècle, plus prudent aussi. L’influence de Gassendi fut très faible, je crois, et je ne serais pas très éloigné de penser qu’elle fut nulle, parce qu’elle ne trouva pas un homme de talent pour se mettre à son service et pour illustrer de littérature les idées du philosophe provençal. En France le génie réussit peu, ou tardivement ; la force de pensée ne réussit pas  sans le talent ; mais la moindre chose réussit quand le talent s’y joint et se mêle de lui faire un succès. Gassendi est, toutes proportions gardées, du reste, un Auguste Comte qui n’a pas trouvé d’Hippolyte Taine.\par
Mais trois influences, dans cette fin du \textsc{xvii}\textsuperscript{e} siècle, ont eu une certaine importance au point de vue de l’anticléricalisme, celle de Descartes, celle de Bayle et celle de Molière.\par
Je place l’influence antireligieuse de Descartes au troisième tiers du \textsc{xvii}\textsuperscript{e} siècle, et je crois que je ne me trompe pas extrêmement. Dans le milieu du \textsc{xvii}\textsuperscript{e} siècle Descartes {\itshape tout entier} est trop présent aux esprits pour qu’il ait une influence antireligieuse. Il n’est pas assez loin pour qu’on ne se rappelle point que personnellement il est chrétien, très chrétien, aussi chrétien que possible, homme qui fait des vœux et des pèlerinages ; que, comme philosophe, il a un système qui est tout plein de Dieu, jusque-là qu’on peut dire et, pour mon compte, c’est ce que j’ai dit, que Dieu est la pierre de fondation et la pierre clef de voûte de tout son édifice et que sans l’idée de Dieu le système de Descartes n’existe absolument pas. Descartes, en 1660, n’est pas assez loin pour qu’on ne se rappelle pas tout cela.\par
Mais à mesure qu’il s’éloigne et recule dans le  passé, ce qu’on se rappelle plus distinctement et peu à peu uniquement ; d’abord parce que c’est en quoi il se distingue nettement de l’enseignement religieux traditionnel, et ce à quoi il a attaché son nom ; ensuite parce que c’est une idée très simple, très précise et très accessible au moindre esprit ; ce qu’on se rappelle plus distinctement et peu à peu uniquement, c’est sa méthode et le premier principe de sa méthode : il ne faut croire qu’à ce qui est absolument évident à l’esprit et ensuite à ce qui, par raisonnements justes, s’appuie sur cette première évidence.\par
Peu à peu l’on ne se souvient plus que de cela. Or cela est le rationalisme pur et simple. Cela écarte et élimine le merveilleux, le mystérieux et le miraculeux. Cela est positiviste au premier chef. Et cela est cartésien et se revêt en quelque sorte de l’immense autorité de Descartes. Ce principe et l’autorité dont il se pare sont des éléments considérables d’incrédulité. La destinée curieuse de ce philosophe consiste en ceci qu’on a oublié son système pour ne se souvenir que de sa méthode et qu’on a pris sa méthode pour son système.\par
Cela s’est passé surtout au \textsc{xix}\textsuperscript{e} siècle ; mais cela n’a pas laissé, ce me semble, de commencer au \textsc{xvii}\textsuperscript{e} siècle, vers 1680. Remarquez que le très pieux  Malebranche en est déjà, malgré sa piété, à affirmer très énergiquement, à maintes reprises, que Dieu n’agit jamais par des volontés particulières. Il trouve le moyen, ou il croit le trouver, de concilier cela avec la croyance aux miracles, de {\itshape sauver le miracle}, par je ne sais plus quel tour de force de dialectique ; mais enfin il affirme et proclame que Dieu ne peut pas agir par volontés particulières.\par
Or c’est un cartésien, et un cartésien fieffé, qui pose ce principe, et, à vrai dire, c’est bien un principe cartésien, si l’on veut ; car le miracle n’est jamais {\itshape évident}, et il faut toujours de la foi pour y croire ; et que Dieu n’agisse point par des volontés particulières, c’est une idée assez {\itshape évidente} aux yeux de la raison, aux yeux du moins de la raison systématique.\par
On voit la pente. Descartes se transforme, vers 1680, en philosophe rationaliste, et le cartésianisme se fait rationalisme d’après le premier principe, non de lui-même, mais de la méthode qu’il a prétendu suivre et que, du reste, il n’a pas suivie du tout.\par
Mais il est bien vrai qu’il avait ouvert cette avenue, et qu’il était naturel, le souvenir du cartésianisme en son ensemble un peu effacé, qu’on passât par elle et qu’on allât peu à peu où elle  menait. Je tiens l’influence de Descartes, à partir de 1680 environ, pour antireligieuse. {\itshape Sic vos, adversus vos.} Il est rare, comme a dit à peu près Bossuet, avec une hautaine et belle mélancolie, que la pensée humaine ne travaille pas pour des fins qui non seulement la dépassent, mais qui sont le contraire même de son dessein.\par

\astertri

\noindent Je n’ai pas besoin de dire longuement que l’influence antireligieuse de Bayle se fit sentir dès le troisième tiers du \textsc{xvii}\textsuperscript{e} siècle avec une certaine force. Le scepticisme insinuant de Bayle a dû même avoir plus d’effets que le rationalisme latent de Descartes. Il convenait à merveille au tempérament modéré (surtout à cette époque) des Français et à leur façon souriante et moqueuse, non renfrognée ou grimaçante, d’être incrédules. La discrète irréligion de Bayle est éminemment accommodée à la complexion française.\par
Je ferai remarquer seulement ici, parce que j’ai oublié de le dire ailleurs, qu’une des forces de Bayle a été de {\itshape continuer} quelqu’un qui n’avait pas cessé d’occuper l’esprit des Français. Montaigne avait été adoré des Français du \textsc{xvii}\textsuperscript{e} siècle. Il n’avait pas quitté leurs mains. Bossuet le savait bien et sentait bien qu’il y avait là pour la religion un péril extrême, et c’est pourquoi il est revenu si  souvent, par allusions épigrammatiques ou éloquentes, à attaquer ou réfuter ce très vivant adversaire. Malebranche aussi le sentait bien, et c’est la raison de la spirituelle, incisive et prolongée mauvaise humeur qu’il a montrée à l’égard de l’auteur des \emph{Essais}.\par
Or Bayle continuait Montaigne, et à la fois bénéficiait de la fortune de son prédécesseur et aussi lui donnait comme une nouveauté, comme un renouvellement et un regain. Il continuait Montaigne, avec moins de talent, avec plus de connaissances variées, avec plus d’études et de recherches fureteuses dans les anciens philosophes, les anciennes croyances et les anciennes superstitions. Il répandait le scepticisme absolument de la même manière, à petits coups mesurés et goutte à goutte, par prétéritions et par sous-entendus, par échappées et par inadvertances très calculées, avec tous les agréments nouveaux, du reste, des faits du jour et des actualités intéressantes, amusantes ou instructives. Il renouvelait Montaigne en mettant comme des notes en marge des \emph{Essais}.\par
Bayle ramenait à Montaigne ou y aurait ramené s’il eût été besoin de cela, et aussi Montaigne introduisait Bayle. A eux deux, ils entretenaient très proprement le scepticisme dans les esprits, malgré les parties dogmatiques de Montaigne et malgré  le mépris de Bayle pour la négation affirmative à son tour et arrogante. Montaigne renouvelé par Bayle et Bayle introduit et comme soutenu de dessous par Montaigne ont dû avoir quelque influence sur les esprits dans le troisième tiers du \textsc{xvii}\textsuperscript{e} siècle, puisque, aussi bien, ils ont eu un succès de lecture, l’un persistant, l’autre conquérant, pour ainsi parler, en cette époque de notre histoire intellectuelle.\par

\astertri

\noindent Et enfin, pour ce qui est de Molière, je ne saurais dire à quel point je le considère comme un des pères de l’anticléricalisme français.\par
Qu’il l’ait été consciemment et volontairement, il n’est rien moins que certain, et la question sera, je crois, toujours débattue. Nous ne savons rien et sommes, ce me semble, destinés à ne savoir jamais rien des opinions personnelles de Molière sur la religion. D’abord il n’en a rien dit personnellement. C’est un auteur dramatique et il reste toujours caché derrière ses personnages, qu’il fait parler chacun selon son caractère, et il n’est responsable de rien de ce qu’il dit, puisque ce n’est pas lui qui parle. C’est le privilège de l’auteur dramatique qu’on ne puisse jamais lui faire qu’un procès de tendances.\par
Ensuite, à vouloir saisir et surprendre sa pensée  personnelle dans le langage de tel de ses personnages qu’il semble bien être et qui vraiment est donné évidemment comme le truchement de l’auteur lui-même, on peut se tromper encore, ce personnage, le Cléante de {\itshape Tartuffe} par exemple, pouvant bien n’être qu’une précaution prise par l’auteur, et, non un drapeau, mais un paratonnerre.\par
Cherche-t-on quelque lumière dans l’{\itshape esprit général} de l’œuvre ? D’abord ce sont toujours des lumières douteuses que celles qu’on tire de l’examen de « l’esprit général », et il ne faudrait s’y lier que sur un bon garant qui ici nous manque.\par
Ensuite l’esprit général de l’œuvre de Molière c’est, il me semble bien, l’esprit modéré, l’esprit tempéré, l’esprit moyen terme et, en un mot, l’esprit bourgeois.\par
Molière est le plus grand bourgeois de notre littérature. Toutes les idées chères au bourgeois français du \textsc{xvii}\textsuperscript{e} siècle et un peu des siècles suivants, il les a eues, il les a chéries et il les a recommandées en les illustrant : supériorité de l’homme sur la femme, subordination de la femme, instruction sommaire et rudimentaire de la femme ; se tenir dans sa sphère et ne pas aspirer à en sortir ; ne guère croire à la science, se défier des médecins  et se soigner soi-même ; mépriser les hommes de lettres, {\itshape excepté} ceux qui tiennent à la cour et qui ont reçu comme une estampille officielle ; respect du gouvernement et conviction que rien ne lui échappe et que c’est sur lui qu’il faut compter comme {\itshape Deus ex machina} qui tire les honnêtes gens des filets des coquins ; mépris des vieillards ou tout au moins tendance à ne les considérer que comme maniaques et figures à nasardes.\par
La plupart au moins des idées chères au bourgeois français et des sentiments qui lui sont familiers forment l’esprit général du théâtre de Molière, et ici encore nous ne pouvons guère savoir si cet esprit général est son esprit à lui ou s’il se le donne pour plaire à son public et pour le servir selon son goût ; car, plus que tout écrivain, beaucoup plus, l’auteur dramatique a le public pour principal collaborateur et pour inspirateur essentiel ; mais encore l’esprit général du théâtre de Molière est bien celui-là.\par
Or à supposer, pour faire court, que cet esprit fût celui de Molière lui-même, qu’en faudrait-il conclure relativement au cléricalisme ou à l’anticléricalisme de Molière ?\par
Rien du tout ; car, au \textsc{xvii}\textsuperscript{e} siècle, le bourgeois est en général religieux, et aussi au \textsc{xvii}\textsuperscript{e} siècle le bourgeois est souvent à tendances anticléricales.  Personnellement à quel groupe appartenait Molière ? A celui des bourgeois d’esprit religieux, à celui des bourgeois très tièdes sur la religion et déjà frondeurs ? On ne peut rien en savoir. Tout au plus pourrait-on dire que, comme comédien, il ne pouvait pas avoir grande tendresse pour l’Église, qui n’en avait aucune pour sa corporation ; mais personnellement il n’avait nullement à se plaindre de l’Église, qui ne lui a jamais cherché querelle, qui baptisait ses enfants très honorablement ; et il n’est pas probable qu’il ait prévu qu’elle lui refuserait les honneurs suprêmes. Non, on ne peut vraiment rien savoir et l’on ne peut honnêtement rien affirmer sur les idées et sentiments religieux de Molière. On ne peut pas assurer qu’il ait été consciemment et volontairement un des pères de l’anticléricalisme.\par
Mais qu’il l’ait été {\itshape en fait} et le plus illustre et peut-être le plus puissant, je crois que c’est une tout autre affaire et je crois que c’est incontestable.\par
On peut d’abord faire remarquer, quoique je ne considère pas cette considération comme très importante, que l’œuvre de Molière en son ensemble est étrangère {\itshape essentiellement} à toute idée religieuse. On se moquera de moi là-dessus et l’on me demandera comment je voudrais que des comédies  et farces fussent empreintes de sentiment religieux et révélassent des préoccupations religieuses chez leur auteur. Ce n’est point cela que je veux dire, mais seulement que, si l’œuvre de Molière en son ensemble ne révèle aucun principe religieux, ce qui est assez naturel, elle ne laisse pas d’en indiquer d’autres, qui sont contraires au sentiment religieux.\par
Très évidemment Molière a confiance, je ne dirai pas en la {\itshape nature} et en l’{\itshape instinct naturel}, ce qui a été beaucoup trop affirmé, et ce que, vraiment, je ne crois pas du tout, mais confiance dans le {\itshape bon sens} purement humain. Il est rationaliste à sa manière, et c’est-à-dire qu’il croit que la raison moyenne, la raison de Chrysale et de Cléante, constatant les faits avec sang-froid et tranquillité et raisonnant un peu sur ces faits, sans subtilité et sans profondeur, suffit très bien à l’humanité, assure son bonheur relatif et est enfin ce à quoi elle doit se tenir, sans voir plus loin ni plus haut.\par
C’est cela Molière, c’est précisément cela, à mon avis.\par
Or rien n’est plus contraire, sans hostilité, sans la moindre hostilité, peut-être, mais cependant rien n’est plus contraire au sentiment religieux et en général et particulièrement à l’influence de  l’Église, en ce que cela donne l’{\itshape habitude} de penser, de sentir et de vivre sans avoir le moindre besoin de religion, de métaphysique, de philosophie ni même de morale un peu élevée.\par
Et cela fait illusion ; car les honnêtes gens de Molière sont assez honnêtes gens en effet pour donner suffisamment envie de les prendre pour modèles ; et, à les prendre pour modèles, on se passera de tout ce que je viens de dire le plus aisément du monde et avec l’approbation pleine et entière de son bon sens et en se tenant pour un sage et pour un homme honnête autant qu’on peut l’être ; et cela mène très bien à l’élimination de toute préoccupation religieuse et de toute religion.\par
Ne vous paraîtrait-il pas naturel qu’un honnête homme, comme Chrysale, eût, à un moment donné, et je dis en passant, sans une insistance qui serait parfaitement mal à propos, un mot qui indiquerait qu’il a reçu une éducation religieuse et qu’il en a gardé des traces ? Ce serait très naturel. Ce mot, il ne l’a jamais.\par

\begin{center}
\noindent Former aux bonnes mœurs l’esprit de ses enfants.\par
\end{center}

\noindent « Aux bonnes mœurs », c’est tout. « Leur faire craindre Dieu » ou « leur faire aimer Dieu », ce qui serait si naturel dans la bouche d’un bourgeois du \textsc{xvii}\textsuperscript{e} siècle, non. « Les bonnes mœurs » ; c’est tout.  Le bon Chrysale est, par prétérition, tenant de la morale laïque.\par
Ne vous paraîtrait-il pas naturel et même d’{\itshape observation juste} qu’un {\itshape honnête homme distingué}, comme Philinte ou comme Alceste, eût, à un moment donné et en passant, un mot point du tout de dévot, mais d’homme ayant un fond religieux, soit Philinte pour consoler Alceste, soit Alceste pour se consoler dans son infortune ? Il n’y aurait rien de plus juste, qui fût plus {\itshape du temps} et en vérité qui fût plus attendu. Ils n’ont jamais, ni l’un ni l’autre, un seul mot, un seul petit mot de ce genre. « De mon temps, on avait Dieu », dit le marquis d’Auberive à sa femme. On dirait, dans Molière, qu’au \textsc{xvii}\textsuperscript{e} siècle on n’avait pas Dieu.\par
Encore une fois, je n’attache pas une grande importance à une observation si générale et d’ordre, pour ainsi parler, négatif ; mais enfin que Dieu, qui pouvait y tenir une place, si petite qu’elle fût et qu’elle dût être, soit absolument absent du théâtre courant de Molière, du théâtre de Molière, les deux pièces où la question religieuse est abordée mises à part ; et que tout ce théâtre courant soit dominé par la seule idée du bon sens humain se suffisant à lui-même et seul appui et seul recours : c’est une chose qui ne laisse pas d’avoir  peut-être un peu de signification et qu’il fallait considérer un instant.\par
Et maintenant, venons aux deux pièces où Molière a abordé la question religieuse. J’ai fait remarquer plus haut que Molière, qui voit très juste, qui sait son siècle et qui sait l’humanité, n’a pas manqué de faire son Don Juan à la fois athée et immoral et débauché, et il faut lui rendre cette justice qu’il a très bien vu ainsi les rapports qui existent entre le libertinage dans un sens de ce mot et le libertinage dans l’autre sens de ce terme. Et il faut certainement remarquer que ceci aurait pu et pourrait passer pour être à tendances religieuses. Molière semble dire : « Voyez que le libertinage de la croyance mène au libertinage des mœurs, ou celui-ci à celui-là, et qu’en tout cas l’un et l’autre ont ensemble étroit parentage, connexion intime, lien naturel et lien rationnel. Don Juan est débauché parce qu’il ne croit pas, et il ne croit pas, aussi, parce qu’étant débauché, il a intérêt à ne pas croire. »\par
Et si Molière ne dit pas cela, si sa pièce ne le dit pas, (et, en vérité, que dit-elle, sinon cela ?) encore est-il que certainement Molière nous présente un athée débauché et qu’il ne l’aime pas. Oh ! pour cela, c’est certain. Don Juan n’est pas personnage sympathique. Molière le déteste bien d’une haine  très probablement personnelle et où il entre de la rancune. « Le grand seigneur méchant homme » et corrupteur de femmes est franchement détesté par Molière. — La pièce, au premier regard, serait donc plutôt à tendances religieuses.\par
Il est vrai ; mais remarquez deux choses assez significatives en sens contraire à ce qui précède. D’abord ils n’avaient pas tout le tort, les ennemis de Molière qui faisaient observer que Don Juan, quand il attaque Dieu, a le beau rôle ; que Don Juan, en tant qu’athée, a les rieurs de son côté, sans les y mettre à la vérité, mais tout naturellement, d’après le texte, ce qui est peut-être encore plus grave ; que Don Juan nie Dieu et que le défenseur de Dieu, l’avocat de Dieu, est un imbécile qui ne dit rien qui vaille, qui est ridicule, qui fait rire en effet, qui tombe par terre en voulant plaider et dont « l’argument se casse le nez ».\par
Sophisme de polémique, dira-t-on. Il en reste cependant quelque chose comme observation, et l’observation est juste. Don Juan ne fait pas de profession irréligieuse, et celui qui lui reproche son irréligion est ridicule. Qu’est-ce à dire, sinon que Molière semble avoir voulu épargner à Don Juan l’odieux qu’une profession de foi irréligieuse aurait attiré sur lui et n’a pas voulu épargner  au croyant imbécile le ridicule de son imbécillité largement étalée ? Il y a apparence au moins. Molière ménage singulièrement Don Juan en tant qu’athée : cela me paraît difficile à contester.\par
Je ne tirerai aucun parti de la fameuse « scène du pauvre », qui fit scandale à l’époque. Je ne puis voir dans cette scène qu’une {\itshape chose vraie}, où chacun parle et agit selon son naturel, l’homme du peuple étant religieux avec héroïsme ; Don Juan étant corrupteur à son ordinaire ; puis, je ne dirai pas généreux, mais homme ne tenant pas à l’argent, comme il est naturel qu’il le soit ; puis disant : « Je te le donne par amour de l’humanité », sans grand dessein philosophique, tout simplement parce qu’il ne peut pas dire : « pour l’amour de Dieu » et que cependant il veut dire : « je te le donne gratis ». — Non, je ne tirerai aucun parti de la « scène du pauvre » dont on a abusé, ce me semble, dans un sens ou dans un autre et dans laquelle je ne vois qu’un incident ressortissant à l’{\itshape idée la plus générale} de l’ouvrage : montrer qu’un pauvre diable de mendiant peut se trouver bien au-dessus d’un grand seigneur, quand ce grand seigneur est méchant homme, et avoir en quelque sorte, relativement à celui-ci, les honneurs de la scène. Je ne vois pas autre chose dans la « scène du pauvre ».\par
 Mais songez à la fin de Don Juan selon Molière. La fin de Don Juan consiste à devenir hypocrite de religion. Ceci est très significatif. Qu’est-ce qu’il signifie, sinon, d’une part, que la méchanceté, le libertinage, la débauche, mènent premièrement à l’athéisme et secondement à l’hypocrisie religieuse ; sinon, d’autre part, que le parti religieux se recrute parmi les Tartuffe, ce qui sera démontré plus tard, parmi les imbéciles comme Sganarelle, et {\itshape aussi} parmi les athées débauchés, corrupteurs et scélérats quand ils sont devenus prudents ?\par
N’est-ce point cela ? N’est-ce point cela, je ne veux pas dire que Molière a voulu faire entendre ; car je n’en sais rien ; mais que le public de Molière peut comprendre, doit sans doute comprendre et est presque forcé de conclure ? Il me semble ainsi, ou j’en ai peur.\par
Un paradoxal ou un malintentionné dirait sans doute : « Molière est tellement irréligieux qu’ayant à présenter un personnage profondément immoral il le donne comme athée, ne pouvant pas faire autrement, puisque c’est la vérité et que Molière est parfaitement esclave de la vérité ; mais qu’en même temps, en tant qu’athée, il le ménage et lui donne ou lui laisse presque le beau rôle ; et qu’en même temps, il trouve le moyen de le faire entrer encore dans le parti religieux ; tant il est impossible  à Molière de concevoir un coquin qui ne soit pas religieux par quelque côté et qui ne ressortisse pas, en fin de compte, d’une manière ou d’une autre, au parti que Molière déteste ; et plus il a, comme forcé par la vérité, par l’observation, par l’expérience, représenté son scélérat comme athée, d’autant plus, comme s’il prenait sa revanche, il l’a fait plus noir et plus hideux dans le rôle de clérical que dans le rôle d’athée, et c’est seulement quand il le considère sous ce nouvel aspect que Molière fait éclater toute la haine qu’il professe à son endroit. »\par
Voilà ce que dirait un paradoxal ou un malintentionné. Et ce n’est pas ce que je dis ; et si l’on me crie : « Eh ! que dis-tu donc, traître ? » je ferai observer que c’est seulement de l’effet possible et probable du \emph{Don Juan} sur le public que je m’occupe, et qu’examinant cet effet possible et probable, j’estime que \emph{Don Juan} a été pour le public une pièce antiathéistique un peu, mais une pièce anticléricale beaucoup, et que le public a dû y puiser des sentiments peu sympathiques à la religion et au monde religieux, quelque intention, dessein ou tendance involontaire que, du reste, Molière ait pu y mettre.\par
Pour ce qui est de \emph{Tartuffe}, la tendance anticléricale est encore plus forte et sans mélange,  ou — encore une fois — l’effet produit en ce sens est encore plus certain. Ceux-là ont très bien jugé de \emph{Tartuffe} qui en ont dit : « Ce n’est pas une pièce contre Tartuffe, c’est une pièce contre Orgon, puisque Tartuffe {\itshape n’y est qu’odieux} et qu’Orgon y est ridicule. C’est une pièce destinée à tourner en ridicule le dévot, l’homme entêté de religion et à qui la religion fait faire sottise sur sottise, et à qui la religion ôte toute sensibilité et toute humanité, qu’en un mot la religion rend bête et méchant. Toute l’essence de \emph{Tartuffe} est dans ces vers, qui sont, à tout égard, dignes de Lucrèce :\par


\begin{verse}
Il m’enseigne à n’avoir d’affection pour rien ;\\
De tout attachement il détache mon âme,\\
Et je verrais mourir mère, enfants, frère, femme,\\
Que je m’en soucierais autant que de cela.\\
\end{verse}


\begin{quoteblock}
\noindent « {\itshape Tantum relligio potuit suadere malorum.} »
\end{quoteblock}

\noindent Et si l’on nous dit qu’Orgon a cependant quelques belles qualités, que ce dévot a été et est resté un bon citoyen et que, sauf son engouement, il est encore sur le pied d’homme sage ; si l’on attire notre attention sur cette dualité du caractère et du rôle d’Orgon ; nous répondrons, comme il est facile de le prévoir et comme sans doute on s’y attend, que c’est précisément là qu’on saisit le  fond de la pensée de Molière et son dessein très net et très précis.\par
Qu’aurait-il prouvé s’il avait fait d’Orgon un simple imbécile ? Que les imbéciles sont facilement et volontiers dévots. C’était quelque chose, et, du reste, \emph{Tartuffe} ne laisse pas de diriger vers cette conclusion (rôle de M\textsuperscript{me} Pernelle). Mais il a voulu prouver bien plus. Il a voulu prouver que d’hommes intelligents, sages, droits et généreux la monomanie religieuse fait des imbéciles, des dupes, des niais, des benêts et des figures à nasardes.\par
Ne voyez-vous pas qu’il s’adresse à ces bourgeois du parterre ou des loges et qu’il leur dit : « Remarquez qu’Orgon, c’est vous, hommes de mérite, hommes de valeur, hommes de grand poids, et observez ce que devient un homme tel par monomanie religieuse. {\itshape Et nunc intelligite et erudimini.} » Voilà ce que dit Molière aux bourgeois de Paris. Voilà sa leçon.\par
Il est clair qu’il ne leur conseille pas de ne point être des Tartuffes ; c’est fort inutile ; il leur conseille de n’être pas des Orgons. Et le vrai moyen c’est de commencer par n’avoir aucun commerce avec l’Église. \emph{Tartuffe} est une pièce contre Orgon et non contre Tartuffe, puisque c’est Orgon qui est ridicule, tandis que Tartuffe n’est qu’odieux.\par
Nous dira-t-on qu’ils sont ridicules tous les  deux et que, par conséquent, la pièce peut passer pour être à la fois contre Orgon et contre Tartuffe ? D’abord Tartuffe est {\itshape surtout} odieux ; ensuite, si Tartuffe est ridicule, c’est d’abord de par ce souci qu’a toujours eu Molière, et dont il faut le louer, de ne jamais quitter le ton de la comédie, qu’il eût quitté s’il avait fait de Tartuffe un simple coquin habile ; c’est ensuite, peut-être bien, pour montrer à quel point la monomanie religieuse, l’imbécillité religieuse a de la puissance sur les hommes et est dangereuse, puisque la contagion en peut venir à un homme comme Orgon par le canal d’un homme comme Tartuffe.\par
Est-ce que Molière ne semble pas dire encore : « Et non seulement un homme de forte tête peut être assoté par le commerce avec les hommes d’Église ; mais il ne faut pas un Bossuet ou un Bourdaloue pour l’abêtir. Fréquentez les hommes d’église. Vous les trouverez sots, grossiers, gourmands, vulgaires, plats, gueux, et ne vous en défierez point. Prenez garde ! Ils ont en eux une telle puissance, malgré tout, à cause de ce au nom de quoi ils parlent et par ce qu’ils vous disent, relativement à certaine « méchante affaire », que, si vous avez la foi, ce qui est leur prise sur vous, vous serez bientôt entre leurs mains comme de tout petits garçons qu’on fouette. »\par
 N’est-ce pas là ce que dit \emph{le Tartuffe} aux bourgeois de Paris qui l’écoutent ? Et n’est-ce point une invitation suffisante, par la mise en jeu de l’amour-propre, à l’irréligion et l’incrédulité ? Par tous ses aspects, par toutes ses tendances, par tout ce qu’il donne à entendre, sans qu’on sollicite les textes, \emph{le Tartuffe} est la pièce antireligieuse, tout au moins la pièce anticléricale par excellence.\par
Le réquisitoire est fort, je ne puis le nier. J’ai cherché moi-même à le réfuter, au moins partiellement. J’ai dit, avec raison, je crois : oui, c’est surtout d’Orgon que Molière se moque dans \emph{le Tartuffe} ; mais cela tient à ce que l’office de l’auteur comique est de se moquer, non des coquins, mais des honnêtes gens.\par
Sans aucun doute. On ne se moque pas des coquins ; on les dénonce et on les flétrit, et si la comédie se mêlait de poursuivre les coquins, elle deviendrait autre chose que ce qu’elle est. Elle deviendrait la satire ou l’éloquence judiciaire. Et c’est bien pour cela, précisément, que telles tirades du \emph{Tartuffe}, celles qui sont contre Tartuffe, ont le caractère de la satire ou de l’éloquence de ministère public ou de tribun.\par
Mais la comédie en elle-même, à ne pas sortir de son domaine, de sa définition et de son office, la comédie en elle-même se moque des travers des  honnêtes gens pour les corriger. Elle se moque de la parcimonie, de la vanité du bourgeois gentilhomme, de la manie du bel esprit, des chimères et folles terreurs du malade imaginaire. Voilà son domaine véritable.\par
Remarquez l’axiome antique : {\itshape Castigat ridendo… vitia} ? point du tout : {\itshape mores}. Elle corrige non les vices, incorrigibles ; mais les mœurs moyennes en ce qu’elles ont de mauvais ; c’est-à-dire qu’elle corrige, non les vices, mais les travers.\par
Et surtout (du reste encore pour corriger ces mêmes travers), elle avertit les honnêtes gens des périls où ces travers les engagent, des ennemis, par exemple, qui se rendront maîtres des honnêtes gens en exploitant habilement leurs défauts. Voilà le point. Elle se moque de Philaminte, surtout pour l’avertir que sa manie du bel esprit peut la mettre aux mains d’un écornifleur qui flattera cette manie ; de Jourdain, surtout pour lui montrer que ses prétentions au bel air le livreront pieds et poings liés aux professeurs de belles manières et de beaux-arts et aux chevaliers d’industrie et aux comtesses de contrebande ; de Dandin surtout pour lui montrer que d’épouser une fille de famille où le ventre anoblit fait du paysan gentilhomme ce qu’il n’est pas besoin de dire ;  d’Arnolphe, surtout pour lui montrer que de vouloir à quarante ans épouser une fille de seize met un homme en fâcheuse posture ; d’Harpagon, même, pour lui montrer qu’il se trouvera tel intendant flattant sa manie avaricieuse et poursuivant sa pointe et ses secrets desseins dans la maison, sous ce couvert.\par
Non, ce n’est point les vices que la comédie poursuit, c’est les défauts, et elle met surtout en lumière ceci que par leurs défauts les honnêtes gens ou demi-honnêtes gens sont à la merci et tombent sous la prise des criminels ou des intrigants. Et ce qu’elle fait partout, elle l’a fait dans le \emph{Tartuffe} et elle n’a pas fait autre chose.\par
Voilà une défense de Molière que naturellement je n’ai aucune raison de trouver mauvaise ; mais encore on pourra toujours dire : « Sans doute ; mais pourquoi Molière a-t-il choisi, avec quelque prédilection, on l’avouera, ce genre de travers qui est la monomanie religieuse ; beaucoup moins grave (ne l’avouera-t-on point ?) que ceux qu’il a attaqués d’ordinaire ; plus respectable aussi ; et pourquoi a-t-il rendu un homme qui se trompe sur le choix d’un directeur, car il n’y a que cela, aussi ridicule et de temps en temps aussi odieux et plus odieux que l’avare, le bourgeois gentilhomme et autres ? Et pourquoi l’a-t-il fait tomber dans des  malheurs ou l’a-t-il amené au seuil de malheurs plus grands que ceux où tombent ou que ceux dont approchent tous les autres ? »\par
« Il y a bien dans \emph{le Tartuffe} un Molière plus irrité et plus cruel qu’ailleurs, et n’est-ce point qu’il se sent en face, soit du crime qu’il déteste le plus, voilà pour Tartuffe ; soit du défaut, du travers ou de la stupidité qu’il a le plus en horreur, voilà pour Orgon ; et peu importe qui soit celui des deux à qui particulièrement il en veut, pour ce qui est de la chose qu’il attaque et qu’il bafoue. »\par
Et surtout on pourra toujours dire : « Laissons de côté Molière lui-même et ses intentions et desseins et pensées de derrière la tête et les haines que l’on peut supposer qu’il ait eues. L’effet produit n’a pu être qu’un mouvement de pitié pour les hommes qui ont commerce avec les gens d’Église et un mouvement d’horreur contre les hypocrites de religion ; et la conclusion de gros bon sens, la conclusion un peu vulgaire, mais très naturelle, la conclusion bourgeoise de ce public bourgeois, n’a pu être que celle-ci : « {\itshape Tout compte fait}, il y a beaucoup d’hypocrites, d’imposteurs et d’écornifleurs dans le monde religieux, et ceux qui s’entêtent de religion sont des bêtes, ou, ce qui est pire, le deviennent. Le plus sûr est donc de n’avoir point commerce avec les  gens d’Église et de n’avoir qu’une religion très tempérée, un minimum de religion, la religion de Valère qui « ne hante point les églises ». — Voilà très certainement la conclusion que tirera du \emph{Tartuffe} le public de Molière ; car enfin si la conclusion des \emph{Femmes savantes} est bien clairement : « Fermez votre porte aux gens de lettres », il faut bien que celle du \emph{Tartuffe} soit : « Ne l’ouvrez pas aux gens d’Église. »\par
Et certainement, pour ce qui est de l’effet produit, ce qui précède est peu contestable.\par
On peut donc dire que la vogue prodigieuse de Molière doit être comptée dès le \textsc{xvii}\textsuperscript{e} siècle comme influence anticléricale.\par

\astertri

\noindent Donc l’anticléricalisme au \textsc{xvii}\textsuperscript{e} siècle a existé, surtout dans le premier tiers de ce siècle et dans le troisième. Mais il a été extrêmement faible. La masse de la nation, l’immense majorité de la nation n’en a pas été touchée. Cela se voit à la grande popularité des livres religieux, à la grande popularité des prédicateurs, aussi à la religion très fervente des esprits les plus disposés par leurs inclinations naturelles à l’indépendance, à l’irrévérence et au sarcasme.\par
La Bruyère est très bon chrétien ; il fait un  chapitre contre les esprits forts. Voyez-vous La Bruyère naissant trente ans plus tard ; est-ce que vous le vous figurez très chrétien ? Est-ce que vous le vous figurez autre que libre penseur ou tout au moins très détaché, à la manière des \emph{Lettres persanes} ? Que La Bruyère soit très chrétien, c’est une preuve que tout le monde l’était alors et jusqu’aux esprits naturellement satiriques et impertinents.\par
Songez encore que toute la bourgeoisie et tout le peuple, sauf les protestants, ont applaudi, tous jusqu’à un innocent païen comme La Fontaine, à l’exécrable révocation de l’Édit de Nantes. Il n’y a pas de signe plus frappant que celui-ci. La France de 1685 est profondément religieuse et profondément catholique. On n’a pas besoin de me prier pour me faire dire qu’elle l’est beaucoup trop.\par
Que ceci soit donc retenu. L’anticléricalisme existe au \textsc{xvii}\textsuperscript{e} siècle. Il est très faible. On peut aller jusqu’à dire qu’il est imperceptible, parce qu’il est encore latent.\par

\astertri

\noindent Seulement aucun siècle plus que le \textsc{xvii}\textsuperscript{e} siècle n’a préparé merveilleusement et comme couvé l’anticléricalisme. Voici pourquoi et voici comment.\par
 Au \textsc{xvii}\textsuperscript{e} siècle, tout au moins à partir de Louis XIV, le protestantisme est définitivement vaincu. Il n’est plus un péril ; il n’est plus même une opposition gênante ; il est absolument inoffensif. C’est le moment, bien entendu, que l’on prend et que l’on saisit pour le combattre avec sauvagerie et férocité. C’est d’ordre commun, particulièrement en France. Avoir vaincu cela ne donne que l’envie d’écraser. Plus on est sûrement maître, plus on veut être tyran. Car, je vous le demande, à quoi servirait-il d’être maître ?\par
De même de nos jours, le catholicisme n’ayant plus ni ongles ni dents, que reste-t-il ? A le tuer. Et c’est ce qu’on fait avec allégresse.\par
Donc, au \textsc{xvii}\textsuperscript{e} siècle, le protestantisme étant vaincu, on le combattit, et l’on fit le ferme propos de l’exterminer.\par
On y parvint à très peu près, dans toute la mesure du possible. La France en fut affaiblie ; mais pour un parti il ne s’agit jamais de la France ; et Louis XIV, en cette affaire, chose honteuse pour un roi, ne fut pas autre chose qu’un chef de parti. Il eut tout juste la largeur d’esprit et la portée d’intelligence d’un Combes. Il est honorable pour M. Combes de ressembler à Louis XIV ; il est moins honorable pour Louis XIV de ressembler à M. Combes.\par
 Donc on persécuta avec acharnement le protestantisme désarmé et inoffensif.\par
On persécuta plus inoffensif et plus désarmé encore, puisqu’on persécuta des gens qui, non seulement n’avaient plus d’armes, mais qui n’en avaient jamais eu. On traqua et l’on chassa à courre le janséniste, on sait avec quelle vigueur et avec quelle persévérance. C’est ici que se voit bien, peut-être, la pensée maîtresse du gouvernement de cette époque. Que l’on combatte à mort un parti qui a été puissant, cela se comprend encore. Ce peut être une illusion. Il se peut qu’on le croie encore vivant. Mais que l’on combatte une opinion qui comme parti n’a jamais existé, cela montre bien que c’est l’opinion que l’on combat, la manière de voir et rien de plus, et que l’on a pour principe que personne dans tout le pays n’a le droit de penser autrement que vous, et que tout homme dans le pays a le devoir strict de penser exactement comme vous pensez.\par
Je m’étonne que Louis XIV ait pu dire que Boileau s’entendait en vers mieux que lui. Au fond, je n’en crois pas un mot. Ou, s’il est vrai, on voit assez à quel point cela a stupéfié les contemporains, puisqu’ils ont rapporté cela comme un trait de libéralisme absolument extraordinaire.\par
En tout cas Louis XIV n’admettait pas que quelqu’un  s’entendît en théologie mieux que lui, ni que quelqu’un s’y entendît d’une autre façon que la sienne.\par
Il est possible et même probable qu’il y eût autre chose encore dans la pensée ou dans l’arrière-pensée du gouvernement d’alors. Pour le gouvernement de Louis XIV, les protestants étaient des républicains et les jansénistes étaient des demi-protestants, et la guerre aux protestants et aux jansénistes c’était le royalisme qui se défendait, et les persécutions contre les protestants et contre les jansénistes c’était une expédition de Hollande à l’intérieur.\par
Tout n’était pas faux dans cette arrière-pensée du gouvernement ; mais ceci est caractéristique d’une autre manie, très analogue du reste à la précédente, des gouvernements despotiques. Ils ont la manie de se chercher des ennemis et de tellement les chercher qu’ils en créent pour en trouver. Les gouvernements despotiques sont des femmes jalouses. Celles-ci veulent absolument que leurs maris les trompent ; je veux dire qu’elles en ont tant peur qu’elles se figurent toujours que la chose est, et qu’elles semblent vouloir qu’elle soit, et qu’à force de la redouter il semble qu’elles la désirent.\par
Le gouvernement despotique veut avoir des  ennemis ; et il les suppose pour en avoir ; et il en arrive ainsi, par procès de tendances à très longue trajectoire, à considérer comme républicains des gens qui le sont en puissance, c’est-à-dire qui pourraient l’être ; et des gens aussi qui ne le sont point du tout, mais qui ressemblent un peu, à d’autres égards, à ceux qui pourraient le devenir. Le procès de tendances consiste à poursuivre des personnes, pour opinions, d’abord, et ensuite pour des opinions qu’elles n’ont pas, mais qu’il ne serait pas impossible qu’elles eussent si elles en avaient d’autres que celles qu’elles ont.\par
Tel est l’état d’esprit des gouvernements despotiques. Et c’est-à-dire, comme l’a démontré Platon dans une jolie page, que ce ne sont pas des gouvernements, mais des « factions ». Ce sont des partis qui ont besoin d’avoir des ennemis ; ce qui est précisément le propre des partis ; et qui sentent continuellement ce besoin comme une condition et comme une nécessité de leur existence ; et qui ont besoin d’opprimer quelqu’un pour se prouver à eux-mêmes qu’ils existent et qui, par conséquent, inventent des ennemis pour pouvoir se battre et des {\itshape oppressibles} pour pouvoir être oppresseurs.\par
Ce ne sont donc pas des gouvernements, puisqu’ils ne gouvernent pas à proprement parler, mais mènent les citoyens à la bataille les uns  contre les autres ; ce sont des factions au pouvoir. Louis XIV, pendant toute une partie de son règne, a été un factieux.\par
Quoi qu’il en soit, tel a été le \textsc{xvii}\textsuperscript{e} siècle au point de vue religieux. Il a été persécuteur de gens désarmés et conculcateur de gens à terre. Or un tel siècle prépare mieux à l’irréligion qu’un siècle de guerres civiles proprement dites.\par
Un homme naissant après le \textsc{xvi}\textsuperscript{e} siècle, en France, peut se dire : « On s’est battu. On s’est battu pour cause de religion et sous prétexte de religion. On s’est battu pour que la messe fût dite en français et pour conquérir le pouvoir. C’est épouvantable. On ne devrait être que Français. Mais encore, on se battait à armes égales ou qui semblaient l’être, rendant coups pour coups et ne frappant que par souvenir d’avoir été frappé ou crainte de l’être. C’était la guerre. Que les religions soient cause de cela ou mêlées très intimement à cela, c’est très regrettable ; mais encore ce n’est pas une raison pour détester toute religion ou se tenir éloigné de toute religion. La preuve n’est pas faite que les religions soient éternellement et indéfiniment persécutrices. Pour ce qu’elles ont de bon, on peut les garder, chacun la sienne et, tout compte fait, je garde celle dans laquelle on m’a élevé. »\par
 Oui, un homme naissant après le \textsc{xvi}\textsuperscript{e} siècle, en France, pouvait raisonner à peu près de cette façon.\par
Mais un homme de la fin du \textsc{xvii}\textsuperscript{e} siècle était frappé de ceci que les discordes religieuses survivaient à leurs grandes causes, à leurs grandes causes morales, nationales, ethniques et politiques ; qu’elles se continuaient et prolongeaient comme par elles-mêmes ; qu’elles se multipliaient, du reste, en se subdivisant ; qu’il ne suffisait plus que protestants et catholiques se combattissent d’un bout de l’Europe à l’autre ; mais qu’il fallait que les catholiques se partageassent en jansénistes et ultramontains et les protestants en orthodoxes et en libéraux, et que c’était d’une part l’Église de France proscrivant les jansénistes et que c’était d’autre part Jurieu poursuivant Bayle d’une haine implacable et le dénonçant furieusement à tous les tribunaux comme athée et comme criminel, si bien que Bayle écrivait à un protestant de France : « Si vous voulez rester fidèle à votre religion, vivez dans le pays où elle est persécutée, et Dieu vous garde de vivre dans celui où, étant maîtresse, elle est persécutrice. »\par
Devant ce spectacle, l’homme de la fin du \textsc{xvii}\textsuperscript{e} siècle en venait à se dire que la cause des querelles et des violences entre les hommes était la  religion elle-même, quelle qu’elle fût, et qu’il fallait détruire toute religion.\par
Surtout l’intervention du pouvoir civil dans les querelles religieuses, alors que le pouvoir civil n’était menacé en rien et n’avait nullement affaire, sous la secte religieuse, à un parti politique, surtout cela amenait comme naturellement un homme d’esprit moyen à se dire que les religions étaient les mauvais démons des pouvoirs civils et leur donnaient de détestables inspirations, et c’était droit au mauvais démon qu’il poussait, et le mauvais démon qu’il dénonçait et voulait détruire.\par
Il ne savait pas dire aux gouvernements : « Ne vous mêlez jamais d’affaires religieuses et laissez les religions se quereller par la parole et se disputer les populations par la parole ; et n’intervenez que comme chef de police quand elles déchaînent la guerre civile, et alors avec une parfaite impartialité ; et, en d’autres termes, soyez neutres tant qu’on parle ; et, quand on agit, n’intervenez que pour qu’on cesse d’agir : et réprimez la guerre civile, ne la faites pas. »\par
Il ne savait pas dire cela aux gouvernements ; mais sachant, non sans raison historique, que les gouvernements intervenaient toujours, soit pour une religion, soit pour une autre, il se disait plutôt : « Ce qu’il faudrait, c’est qu’il n’y eût plus de  religion du tout ; ce qu’il faudrait, c’est que la cause pour laquelle les gouvernements font des guerres à l’intérieur disparût. »\par
A se déchirer les unes les autres, les religions ont fait souhaiter que toutes disparussent ; à soutenir les sectes religieuses les unes contre les autres, les gouvernements ont fait souhaiter que toutes les religions cessassent d’être.\par
Quand on proscrit, sans la moindre utilité démontrée, successivement protestants, jansénistes et quiétistes, en définitive, ce sont des athées que l’on fait.\par
La prodigieuse rapidité avec laquelle, sinon la France, du moins la classe dite éclairée, en France, est devenue irréligieuse, ou indifférente en matière de religion, ou sarcastique à l’égard des religions, dès le commencement du \textsc{xviii}\textsuperscript{e} siècle, s’explique, à mon avis, par ce qu’il y avait de prodigieusement inutile, de prodigieusement dénué de raison et même de prétexte et de prodigieusement stupide dans les longues persécutions religieuses du \textsc{xvii}\textsuperscript{e} siècle.
 \section[{Chapitre III. L’anticléricalisme au XVIIIe siècle.}]{Chapitre III.\\
L’anticléricalisme au XVIII\textsuperscript{e} siècle.}\renewcommand{\leftmark}{Chapitre III.\\
L’anticléricalisme au XVIII\textsuperscript{e} siècle.}

\noindent L’anticléricalisme au \textsc{xviii}\textsuperscript{e} siècle fut plus bruyant qu’il ne fut profond. Comme le prouve tout ce qu’on connaît des cahiers de 1789, il ne pénétra que fort peu dans les couches dites inférieures de la nation. Comme tendent à le prouver quelques procès célèbres du \textsc{xviii}\textsuperscript{e} siècle où les choses religieuses sont mêlées, la population aussi bien du midi que du nord était encore très catholique et très cléricale. C’est Voltaire et c’est du reste tout ce qui nous est rapporté par tout le monde sur les affaires Calas, Sirven et La Barre qui nous sont témoins que la population de Toulouse et de la province de Toulouse, que la population d’Abbeville et de la région d’Abbeville étaient « unanimes » contre Calas, contre Sirven et contre La Barre. Les passions catholiques étaient tout aussi fortes dans la bourgeoisie et dans le peuple au \textsc{xviii}\textsuperscript{e} siècle qu’au \textsc{xvii}\textsuperscript{e}.\par
 M. Cruppi l’a dit et, du reste, rien n’est plus évident, si le jury eût existé au \textsc{xviii}\textsuperscript{e} siècle, Calas, Sirven, La Barre et d’Etallonde eussent été condamnés ; Calas et La Barre eussent été suppliciés tout comme ils l’ont été par l’arrêt des juges. La chose seulement eût été plus certaine dès le premier moment de l’affaire. Il n’y a aucun doute sur ce point.\par
Quant à la magistrature, elle était en immense majorité catholique ; mais elle l’était d’une façon particulière. Elle était toute janséniste. Elle lutta, depuis le commencement du siècle jusqu’en 1771, contre les évêques et les curés ultramontains et dominés par l’influence des Jésuites, qui refusaient les sacrements aux jansénistes. Elle était janséniste, gallicane et antipapiste ; elle voyait, non sans raison, dans les jansénistes des hommes indépendants qui ne se croyaient pas obligés de penser exactement en religion et en autres choses comme le roi voulait qu’on pensât ; mais elle était profondément catholique et d’autant plus sérieusement, d’autant plus intimement, d’un sentiment d’autant plus réfléchi et d’autant plus passionné que, précisément, elle était janséniste et de la religion de Pascal.\par
Or la magistrature, c’était la bourgeoisie ; c’était la grande bourgeoisie française ; c’était la bourgeoisie  française assez riche, fort instruite et fort éclairée, très patriote, catholique gallicane et catholique libérale, antiprotestante, à tendances ou à sympathies jansénistes, adversaire, généralement, de la noblesse et du haut clergé, adversaire du despotisme, dévouée au roi, mais indépendante à son égard et voulant qu’il fût respectueux des « lois fondamentales ». Il y eut accord presque parfait entre la bourgeoisie française et la magistrature jusqu’aux approches de la Révolution de 1789.\par
On peut donc dire qu’au \textsc{xviii}\textsuperscript{e} siècle l’anticléricalisme ne pénétra pas très profondément. Il n’atteignit ni le peuple, ni la petite bourgeoisie, ni la grande. Il fut encore très nettement en minorité et en minorité très faible.\par
Mais il fut bruyant et très brillant, parce qu’il fut très répandu parmi les hommes de lettres, qui étaient devenus comme une classe dans la nation.\par
On peut dire que ce fut le \textsc{xvii}\textsuperscript{e} siècle qui fut encore cause de cela et que le \textsc{xvii}\textsuperscript{e} siècle contribua de loin, très indirectement et très involontairement, à la cause de l’anticléricalisme, en ce sens que c’est sa gloire littéraire qui fit des hommes de lettres une classe, et une classe très considérable, et qu’il se trouva que les hommes de lettres, après lui, furent anticléricaux.\par
 Imaginez, après Balzac, Descartes, Corneille, Molière, La Rochefoucauld, Sévigné, Bossuet, Racine, Boileau, La Bruyère et le retentissement de ces grands noms dans toute l’Europe et la diffusion, grâce à eux, de la langue française dans toute l’Europe, et la gloire européenne de la France, gloire qu’elle sent qu’elle doit principalement à ses hommes de lettres, imaginez bien ce que c’est qu’un homme de lettres en 1700.\par
C’est un homme qui fait partie d’une classe mal déterminée, mais illustre ; et cette classe, ceci encore est à noter, contient de petits bourgeois, de grands bourgeois, des hommes nobles, des femmes nobles, des grands seigneurs et des princes de l’Église. Et elle les réunit, par libre choix, à titre d’égaux, dans une sorte de conseil supérieur qui s’appelle l’Académie française. Elle est mal définie ; mais elle est constituée ; elle est visible et en grande lumière ; c’est bien une classe de la nation. On prendra l’habitude, et ceci, sous l’ancien régime, est un signe très caractéristique, quand on emprisonnera les hommes de lettres, de les enfermer, non dans la première prison venue, à Bicêtre ou au For l’Évêque, mais dans la prison aristocratique. L’homme de lettres a droit à la Bastille. C’est reconnaître qu’il fait partie d’une classe.\par
Or cette classe des hommes de lettres, au  \textsc{xvii}\textsuperscript{e} siècle et dès le commencement du \textsc{xviii}\textsuperscript{e} siècle, fut en majorité anticléricale et même anticatholique et même antichrétienne.\par
Pourquoi cela ? Parce qu’elle était une classe, parce qu’elle avait pris conscience qu’elle en était une et parce qu’elle était laïque.\par
Instruite, douée de talent et d’éloquence, très en vue, très écoutée, presque organisée, elle a eu l’idée très naturelle d’avoir de l’influence sur les hommes et de les diriger. Donc elle a tout de suite vu des rivaux dans ceux qui jusqu’alors avaient de l’influence sur les hommes et les dirigeaient, c’est-à-dire dans les églises.\par
Toute classe veut devenir un pouvoir. La classe des hommes de lettres a eu, dès 1700 ou 1720, l’idée sourde de devenir le pouvoir spirituel. Or le pouvoir spirituel était occupé : elle a considéré ceux qui l’occupaient comme ses adversaires. Le cléricalisme, c’est la concurrence.\par
Ajoutez, quoique ceci soit beaucoup moins important, mais Auguste Comte l’a signalé avec quelque raison, que, même depuis le \textsc{xvi}\textsuperscript{e} siècle, l’homme de lettres se considère comme l’héritier direct de l’antiquité. La littérature, c’est l’antiquité qui {\itshape renaît}. La renaissance de ce qui n’a pas connu le christianisme exclut le christianisme ; l’histoire recommence en deçà du christianisme et suit son  cours sans s’occuper de lui, en faisant abstraction de lui et en s’appuyant sur son passé à elle, sans avoir à tenir compte de ce qu’il a apporté dans le monde. Il y a quelque chose de cela dans la pensée, comme on le verra, de quelques-uns des hommes de lettres du \textsc{xviii}\textsuperscript{e} siècle.\par
En tout cas, la littérature au \textsc{xviii}\textsuperscript{e} siècle est un groupe nombreux et vigoureux, une classe devenue adulte, qui est tourmentée sourdement par la pensée plus ou moins précise que le christianisme constitué et directeur d’âme est un concurrent, un rival et un obstacle. La raison est là de l’anticléricalisme des hommes de lettres au \textsc{xviii}\textsuperscript{e} siècle.\par
Cet anticléricalisme ne fut pas le même chez tous les « philosophes » du \textsc{xviii}\textsuperscript{e} siècle. Il prit, naturellement, la couleur de l’esprit et de la complexion de chacun d’eux\footnote{Sur ce qui suit voir plus de détail et voir les textes dans ma \emph{Politique comparée de Montesquieu, Rousseau et Voltaire}.}.\par
Montesquieu, pour commencer par lui, a varié sur ce point. Dans les \emph{Lettres persanes} il est spirituellement et violemment anticatholique à tendances protestantes. Il se montre effrayé pour l’avenir de la nation du célibat ecclésiastique et des biens de mainmorte et il se répand en plaisanteries sur les dogmes de la religion même chrétienne.\par
 Dans l’\emph{Esprit des Lois} il en est arrivé au moins à comprendre deux très grandes choses : la première que la religion chrétienne est {\itshape essentiellement} antidespotique, parce qu’elle a dit que quelque chose de l’individu n’appartient pas à l’État, à savoir son âme, ce qui est le fondement même des \emph{Droits de l’homme} ; la seconde que la religion chrétienne est {\itshape pratiquement} antidespotique, parce qu’elle forme, contre le pouvoir central ou en face de lui, une de ces barrières qui, pour Montesquieu, sont absolument nécessaires en un État bien constitué.\par
Mais il reste, on le sent plus qu’on ne le voit, anticatholique. Il n’est religieux qu’en tant qu’il est libéral et, certes, c’est une façon d’être au moins sympathique à la religion ; mais n’étant religieux qu’en tant qu’il est libéral, une religion affirmant les droits de la conscience, d’une part, et, d’autre part, une religion organisée en dehors de l’État pour pouvoir servir de limite au pouvoir central, lui suffit. Il serait donc volontiers protestant ; il accepterait volontiers un protestantisme n’ayant pas pour chef le chef de l’État. Il admettrait volontiers une religion nationale, sans célibat des prêtres et sans moines, ayant des chefs nommés par elle et indépendante du gouvernement ; une religion, si l’on me permet de parler ainsi, nationale,  {\itshape laïque} et autonome. On retrouvera quelque chose de cela au temps de la Révolution.\par
Au fond, Montesquieu, ici comme partout ailleurs, est un magistrat du \textsc{xviii}\textsuperscript{e} siècle ; c’est un robin qui a du génie. Il n’aime pas le pouvoir absolu et il est janséniste. Il est janséniste sans être chrétien ; mais il est janséniste comme tous ses confrères. Il aimerait une religion gallicane, indépendante et du Saint-Siège et du gouvernement de Versailles et qui ne serait pas sans analogies avec le protestantisme. Un janséniste est un demi-protestant. Montesquieu est janséniste en ce sens qu’il est à demi protestant dans la conception de la religion qui aurait ses sympathies, sinon religieuses, du moins politiques.\par
Voltaire est plus simple. Il est purement et simplement despotiste et, par conséquent, comme il accepte une religion pour le peuple (la formule est de lui), il veut une religion qui soit tout entière dans la main du gouvernement et des prêtres qui soient des officiers de morale commandés, gouvernés et soldés par le gouvernement. Autant Montesquieu veut plusieurs puissances dans l’État et, par conséquent, accepte très volontiers une puissance spirituelle, autant Voltaire n’en veut qu’une et tient ferme pour que tout, dans l’État, soit dans la main du souverain, la religion comme le reste et  plus que le reste. C’est le fond et c’est comme le tout de ses idées sur cette question.\par
Un de ses griefs contre les Parlements, c’est que les Parlements se mêlent de questions religieuses en dehors de l’initiative et contre le gré du pouvoir royal et, « puissance » autonome eux-mêmes, ou voulant l’être, sont auxiliaires, alliés ou collaborateurs d’une « puissance » encore, en tant qu’elle veut être indépendante et autonome.\par
Il n’y a qu’une solution et qu’un bon ordre, c’est que le souverain gouverne souverainement temporel, spirituel et judiciaire, comme il gouverne le militaire et l’administratif.\par
Du reste, il est bon, ne fût-ce, sans aller plus loin, que pour arriver à ce résultat, de décréditer le pouvoir spirituel et de montrer qu’il n’a jamais que commis des crimes et fait des infamies. De là cette démonstration mille fois répétée que les guerres religieuses et les persécutions religieuses ne sont connues de l’humanité que depuis le christianisme et ont été inventées par le christianisme, à l’imitation des Juifs ; que ni les Grecs ni les Romains n’ont été persécuteurs et que les plus grands malheurs que l’humanité ait commis ont été déchaînés sur elle par les disciples du Christ.\par
Ce qu’il s’agit de démontrer, c’est que le despotisme  temporel est inoffensif et bienfaisant et que la puissance spirituelle une fois englobée, pour ainsi dire, et absorbée par le despotisme temporel, elle aura perdu tout son venin et, gouvernée, inspirée et réprimée par le despotisme temporel, d’abord n’aura que des effets très limités et ensuite n’aura que de bons effets.\par
Mais encore ce despotisme temporel, par qui sera-t-il inspiré lui-même ? Par les sages, les philosophes et les hommes de lettres. Personne, plus que Voltaire, n’a eu, ce me semble, cette conception, que j’indiquais plus haut, d’une « classe » d’hommes de lettres, aristocratie spirituelle rangée autour de la Royauté, la conseillant respectueusement, l’illustrant et la décorant, du reste, et l’aidant dans la partie intellectuelle de sa tâche. Un roi philosophe, antichrétien et ami des philosophes, il a cherché cela toute sa vie et a mis toute sa vie à s’apercevoir qu’il ne l’avait pas trouvé et à déplorer de ne l’avoir trouvé jamais.\par
Et encore cette idée que c’est l’antiquité qu’il faut faire revivre, le rêve de Julien l’Apostat, cette idée que l’on doit rebrousser en deçà du christianisme et continuer le chemin selon les lumières des philosophes de l’antiquité et que les hommes de lettres modernes sont les héritiers et successeurs des sages antiques et ne doivent être que  cela, cette idée est encore celle de Voltaire, subconsciemment au commencement de sa carrière, très clairement à la fin, et je serais assez porté à croire que, sans s’en rendre compte encore, c’est pour cela que tout jeune, il {\itshape réagissait contre la réaction} qui s’attaquait aux hommes de lettres et aux poètes du \textsc{xvii}\textsuperscript{e} siècle et cherchait, non sans succès, à renouer le fil, à maintenir la tradition, celle-ci du moins.\par
Il devait, sans y songer très précisément, raisonner ainsi : « Hommes de lettres du \textsc{xvii}\textsuperscript{e} siècle, religieux, chrétiens, catholiques, il est vrai. « Siècle de grands talents plutôt que de lumières. » Oui. Mais ces hommes n’en sont pas moins les successeurs des hommes de la Renaissance, lesquels relèvent directement de l’antiquité. L’antiquité païenne, qu’il faut appeler l’antiquité philosophique, s’oppose naturellement, historiquement et fatalement à l’antiquité judéo-chrétienne, et c’est cela qui tuera ceci ou aidera à le tuer. Maintenons la tradition, maintenons la suite de notre ascendance. Nous pourrons ensuite (et c’est ce qu’il a fait plus tard) opposer comme un bloc homogène tout l’art hellénique, romain et européen depuis la Renaissance à l’art chrétien, et montrer combien celui-ci est pâle, inélégant et ridicule par comparaison à celui-là, et cette comparaison  sera de très grand effet et de très grande influence sur l’esprit des hommes et, par répercussion, sur leur conscience. »\par
Tel est l’état d’esprit de Voltaire. Il veut le christianisme décrédité et dégradé dans l’esprit des hommes, maintenu cependant, « pour le peuple », mais mis entre les mains du gouvernement central comme l’administration et comme l’armée, asservi pour être inoffensif, rudement tenu en laisse et fouaillé ; — et il veut un gouvernement despotique, absolument souverain, mais aidé des lumières des philosophes et des hommes de lettres antichrétiens. Au fond, un Marc-Aurèle, vertueux, nourri de sagesse antique, ami des philosophes et les consultant, chef de son clergé à lui et persécutant les hommes qui adorent Dieu d’une autre façon que lui : c’est la pensée complète de Voltaire en fait de choses religieuses.\par
Rousseau est beaucoup plus compliqué. Il a le sentiment religieux. Il est anticatholique forcené. Il est despotiste autant que Voltaire et plus durement, plus cruellement si l’on peut ainsi dire. C’est une espèce de Calvin jacobin.\par
Il a le sentiment religieux. Il l’a tellement qu’il ne saurait comprendre qu’un homme dénué du sentiment religieux, non seulement puisse être un bon citoyen, mais puisse être un citoyen. Le sentiment  religieux et la foi religieuse font pour lui partie du civisme. La foi est le premier élément social, l’élément social fondamental. On reconnaît là, chose curieuse, l’esprit genevois de 1550 conservé aussi pur que si l’on était encore en 1550. C’est un phénomène de persistance, c’est un phénomène d’immobilité tout à fait extraordinaire.\par
Qu’un libre penseur puisse rester dans la cité, et que la cité puisse subsister si elle conserve dans son sein un seul libre penseur, c’est ce que Rousseau n’admet pas et ne peut pas comprendre.\par
D’autre part, Jean-Jacques Rousseau est anticatholique radical. On peut admettre dans la cité toutes les religions excepté le catholicisme. La raison en est claire. « Il est impossible de vivre en paix avec des gens que l’on croit damnés. Il faut absolument qu’on les ramène ou qu’on les tourmente. Donc, quiconque ose dire : {\itshape Hors de l’Église point de salut}, doit être chassé de l’État ». Tout État bien constitué doit faire une révocation de l’édit de Nantes contre les catholiques, parce que le catholicisme est antisocial au premier chef, parce que le catholicisme est comme par lui-même la guerre civile en permanence. Il faut chasser le catholique de l’État {\itshape a priori} et sans examen, sur la simple constatation qu’il est catholique.\par
Et enfin, Jean-Jacques Rousseau est despotiste  radical. Il ne l’est pas de la même façon que Voltaire, mais il l’est autant et même plus violemment. Il donne la souveraineté à la majorité de la nation et il n’assigne à cette souveraineté aucune limite. Il appelle même {\itshape liberté} l’oppression de la minorité de la nation par la majorité et il ne voit pas la liberté ailleurs et il ne la conçoit pas autrement. En un mot, il est despotiste démocrate, ou plutôt il est démocrate dans le sens précis du mot.\par
Or, étant despotiste démocrate d’une part et d’autre part ayant un profond sentiment religieux et croyant que la religion est élément social par excellence, il ne se peut point qu’il n’arrive pas à la conception d’une religion d’État. Il y arrive très vite ou plutôt la religion d’État était en lui en quelque sorte, sans qu’il eût besoin d’y aller ; elle était dans la combinaison même de son sentiment religieux et de sa conviction despotiste.\par
Donc il y aura une religion d’État, une espèce de {\itshape minimum} de croyances, que le citoyen devra avoir, sous peine d’être chassé de l’État, sous peine, aussi, d’être mis à mort si, après avoir déclaré qu’il a ces croyances, il se conduit comme s’il ne les avait pas. Ce minimum de croyances, c’est du reste toute une religion ; c’est la croyance en Dieu, la croyance en la Providence, la croyance en  l’immortalité de l’âme, la croyance à la punition future des méchants et au bonheur futur des justes, la croyance en la sainteté du contrat social et des lois. Telle est la religion qu’il faut avoir et pratiquer sous peine d’exil et de mort, parce que si on ne l’a pas, on n’est pas un citoyen ; on n’est pas pénétré des principes sur lesquels la société s’appuie et dont elle a besoin pour exister ; on est par conséquent un élément antisocial dans la société, et donc un ennemi qu’il faut supprimer.\par
L’État devra donc, s’il veut vivre : 1º exiler {\itshape a priori} tous les catholiques : « Quiconque ose dire : Hors de l’Église point de salut doit être chassé de l’État » ; 2º exiler tous ceux qui déclareront ne point croire à Dieu, à l’immortalité de l’âme, à la Providence, aux récompenses et aux peines futures ou à l’un quelconque de ces points ; 3º punir de mort ceux qui, ayant adhéré à cette religion, se conduiraient de manière à montrer qu’ils n’y croient pas.\par
Cela paraît exorbitant au premier abord ; mais ce n’est, en somme, que l’exclusion des catholiques, des libres penseurs et des hommes de mauvaises mœurs. C’est le gouvernement de Genève en son temps glorieux. Rousseau est un « citoyen de Genève », comme on le sait et comme il le dit assez  souvent pour qu’on le sache. Il est un citoyen de Genève qui a conservé en toute leur pureté les traditions de son pays. Il n’y a pas autre chose.\par
On pourrait seulement lui faire observer, dans un esprit de modération qu’il n’accepterait pas, mais qui ferait peut-être quelque impression sur lui, qu’il n’est peut-être pas nécessaire d’exiler et de tuer ; que le catholique, le libre penseur et l’homme de mauvaises mœurs, étant des membres antisociaux de la société, il suffirait peut-être de les diminuer de la tête seulement, dans le sens latin, de les priver de tout droit politique et de tout droit civil. Dès lors la société, gouvernée uniquement par des protestants, des déistes et des hommes vertueux, sans intervention, dans le gouvernement ni dans la législation, des catholiques, des libres penseurs et des pécheurs, pourrait, ce nous semble, être une société assez bonne. L’essentiel est que les pécheurs, les libres penseurs et surtout les catholiques soient des parias ou des ilotes. C’est le suffisant et le nécessaire. Il n’est pas indispensable de leur couper le cou.\par
Je crois que, dans la pratique, c’est à cette solution libérale que Jean-Jacques Rousseau se serait ramené ou résigné.\par
Quoi qu’il en soit, malgré ses sentiments religieux  très profonds, très vifs et même exaltés, Jean-Jacques Rousseau doit être compté, je crois, au nombre des ennemis du catholicisme au \textsc{xviii}\textsuperscript{e} siècle.\par
Il me paraît, j’entends comme anticlérical, remarquez bien, n’avoir eu qu’une influence assez restreinte. Il a inspiré ceux des Français qui ont été à la fois anticléricaux et religieux. C’est une espèce rare. Il a inspiré Robespierre, Chaumette et peut-être Edgar Quinet. Il a inspiré vaguement tous ceux qui ont poursuivi la chimère de fonder une religion nationale plus ou moins détachée ou éloignée de toutes les autres, théophilanthropes, etc. Mais il a eu, au point de vue religieux, peu d’influence sur la masse des Français, qui sont volontiers radicaux en cette affaire, qui ne s’arrêtent pas aux moyens termes et qui, quand ils ne sont pas catholiques, ne croient à rien ; je parle de la généralité.\par
Tout compte fait, je dis toujours à ne le considérer qu’au point de vue religieux, il a été cause de la mort de Robespierre, et c’est tout ce qu’il a fait à cet égard.\par
Mais la grande influence, à mon avis, la plus prolongée surtout, et peut-être plus profonde que celle de Voltaire, au point de vue anticlérical, a été celle de Diderot. Diderot est athée ; il est {\itshape naturiste} ;  il est immoraliste. Cet étourdi, qui ne laissa pas, à ses heures, d’être un homme prudent, n’a pas toujours, tant par étourderie que par prudence, proclamé formellement ni soutenu énergiquement ces trois doctrines. Il en a même soutenu d’autres à l’occasion et assez souvent. Mais son fond, pour qui l’a bien lu, c’est l’athéisme, le naturisme et l’immoralisme.\par
Il est athée parce que, du reste dénué de la foi, il n’est sensible à aucune preuve de l’existence de Dieu. Le sentiment général de l’humanité jusqu’à lui, le {\itshape consensus communis} ne lui impose pas ; car il est très orgueilleux ; et se sentant (avec raison) très supérieur comme philosophe à tous les penseurs contemporains, et d’ailleurs croyant au progrès intellectuel et estimant son siècle supérieur aux siècles précédents, que tous les philosophes jusqu’à lui aient cru en Dieu, cela n’est pas pour l’intimider : ils vivaient dans des siècles d’obscurité et de tâtonnements ; le siècle de Diderot est un siècle de lumières ; dans ce siècle Diderot est supérieur comme penseur à tous les hommes qui se mêlent de raisonner ; l’opinion de Diderot, quoique étant contraire à celle de tout le genre humain jusqu’à lui, peut donc, malgré cela, être la vraie.\par
Il n’est pas sensible aux arguments philosophiques,  cause première et causes finales, précisément parce qu’il est philosophe et qu’il est comme blasé sur ces raisonnements et qu’il sait bien, ou se persuade, que les plus forts, les plus graves et les plus convaincants peuvent bien facilement se ployer, se tordre et se retourner contre eux-mêmes. Ce qui paraît irréfutable à Voltaire : l’horloge qui n’aurait pas d’horloger, par exemple, paraît à Diderot, sinon tout à fait un enfantillage, du moins une de ces choses qu’on démolit d’un tournemain en argumentation métaphysique ou qui se volatilisent entre des doigts philosophiques.\par
C’est le propre de ceux qui causent beaucoup, discutent beaucoup et discutent bien, que ce qui est preuve, ce qui est argument, en vient à se dégrader à leurs yeux, à perdre sa valeur, par la trop grande connaissance qu’ils en ont et l’abus qu’ils en ont fait. Ce sont choses dont on joue ; elles n’ont point d’empire sur l’esprit ; il les fait trop caracoler pour qu’elles l’envahissent ; il les dirige trop pour qu’elles le dirigent. L’argument tombant dans l’esprit du silencieux et du méditatif creuse, s’enracine et se développe. Dans un esprit de cette sorte l’argument devient sentiment (Kant). Dans l’esprit du discuteur l’argument, si souvent envoyé, reçu et renvoyé, devient une chose tout extérieure qui ne tient plus à vous et à qui l’on ne tient plus.  Aucune preuve philosophique de l’existence de Dieu ne peut s’imposer à Diderot, même, ce qui lui arrive, quand il l’administre.\par
Quant aux {\itshape sentiments} qui mènent à Dieu par d’autres routes que celles du raisonnement, Diderot était l’homme du monde qui les éprouvait le moins. Ni le spectacle des beautés et des sublimités de la nature qui, tout raisonnement à part, mettent certaines âmes en état religieux ; ni la présence au fond de nous de la conscience morale n’étaient très capables d’avoir influence sur Diderot, et le mot de Kant : « Deux choses donnent l’idée de Dieu, la voûte étoilée au-dessus de nos têtes et la conscience au fond de nos cœurs », s’il eût pu le connaître, l’aurait peu ému. On ne voit pas Diderot contemplant les étoiles et on ne le voit pas non plus écoutant sa conscience, et, à l’écouter, s’en faisant une.\par
On ne songe pas assez qu’il n’y a que les gens réfléchis qui aient une conscience, puisque la conscience est une réflexion de l’esprit sur l’acte, réflexion qui devient peu à peu préalable et apprend à s’exercer sur l’acte à faire, après s’être longtemps exercée sur l’acte fait ; mais réflexion toujours. Or, Diderot est l’être le moins réfléchi qui ait existé et le plus continuellement impulsif qu’on ait connu. C’est un perpétuel improvisateur de pensées et  d’actions. Ces gens-là n’ont pas de conscience, ou je veux bien accorder qu’ils en ont une ; mais elle reste à l’état rudimentaire ou à l’état latent, et c’est ce qu’on peut appeler une conscience inconsciente.\par
En définitive, je ne vois pas par où l’idée de Dieu aurait pu entrer dans Diderot, ou comment il aurait pu l’inventer. Or, il aimait assez ne croire qu’à ce qu’il inventait lui-même et à ne pas accepter tout fait ce qu’inventaient les autres. Et s’il n’y avait aucune raison pour qu’il inventât lui-même Dieu, il y en avait d’autres, comme on le verra plus loin, pour qu’il le repoussât.\par
Diderot est {\itshape naturiste}. J’entends par là qu’il croit la nature bonne et inspiratrice de bonnes choses. J’entends par là qu’il a confiance dans les instincts humains non rectifiés, dans les instincts humains à leur état naturel ; et « naturel » ne signifie rien, puisqu’il est sans doute dans notre nature aussi de rectifier nos instincts ; mais enfin dans les instincts humains moins la civilisation qui les a modifiés.\par
Par là il rejoint Rousseau, que très probablement, du reste, il a inspiré.\par
Ce qui frappe les naturistes, c’est l’immense distance — très faible, à mon avis, mais qui peut paraître immense, et ces choses ne sont pas pour être mesurées sûrement — qui sépare l’homme  civilisé de l’homme… peu civilisé ; car l’homme naturel n’existe pas ; et la distance immense aussi qui sépare {\itshape en chacun de nous} l’homme tel que nous sentons qu’il serait s’il n’avait pas été dressé, de l’homme tel qu’il est dans la réalité de tous les jours après éducation et dressage social ; et ils appellent celui-là l’homme naturel et l’autre l’homme altéré.\par
Aux chrétiens — et déjà un peu aux païens, il faudrait s’en souvenir — cette différence a paru si grande qu’ils ont pensé que l’homme avait été écarté de sa nature, et (pour eux) qu’il avait été élevé au-dessus de sa nature, par la révélation et par la grâce, en d’autres termes éclairé par une lumière supérieure à ses lumières et soutenu par une force supérieure à ses forces. Et ils opposent la nature à la révélation et la nature à la grâce.\par
Pour ceux qui ne veulent que constater les faits, il existe une manière de révélation et une manière de grâce. Seulement elles ne sont qu’humaines. Pour ceux-ci l’homme avait reçu une nature grossière et la faculté de voir qu’elle était grossière et une sourde et puissante aspiration à la modifier ; et il avait reçu une nature grossière et la force de la modifier peu à peu, à tel point qu’il dût finir par ne point la reconnaître. L’homme au cours du temps se révèle lui-même à lui-même et au cours  du temps se verse à lui-même une grâce efficace dont il avait comme la source dans sa nature même.\par
Dans les deux conceptions il y a bien l’homme de la nature et l’homme modifié ; l’homme de la nature et l’homme qui lutte victorieusement contre sa nature.\par
Pour le naturiste ces deux hommes existent aussi ; mais le second se trompe ou a été trompé. La nature, c’est-à-dire les instincts, sont bons et l’on a eu tort de tant lutter contre eux pour détruire quelque chose qui était salutaire et pour le remplacer par quelque chose qui est funeste. C’est la nature qui a raison et c’est la prétendue raison qui a tort. Soyons optimistes. Ayons confiance en nous-mêmes.\par

\begin{itemize}[itemsep=0pt,]
\item  — Mais il y a deux {\itshape nous}.
\item  — Ayons confiance au {\itshape nous} qui est sans doute le vrai, puisqu’il ressemble à la nature entière et puisque, à ne vouloir connaître que celui-ci, nous évitons ce paradoxe monstrueux qui consiste à croire que dans l’immense nature il n’y a qu’un être qui ait pour vocation, pour mission et pour devoir de ne pas lui ressembler, de ne pas être selon ses lois et d’être le contraire de ce qu’elle est. Ayons confiance en l’homme en tant que semblable au reste de la nature animée.
\end{itemize}

\noindent  Et c’est ici que l’athéisme, naturel — il m’a semblé — à Diderot, rencontre comme sa confirmation et se renforce. Ne voit-on pas que Dieu est un expédient et une invention pour expédient ? Les hommes qui à la fois ont remarqué que l’homme échappait à sa nature et l’ont approuvé d’y échapper et ont voulu qu’il y échappât, ces hommes ont inventé Dieu, comme étant celui qui a indiqué à l’homme les moyens d’échapper à sa nature et celui qui lui a donné la force de la dépasser ; comme un être supérieur à la nature et à l’homme, qui était capable et seul capable de tirer l’homme au-dessus de la nature.\par
Pourquoi l’ont-ils inventé ainsi ? A la fois pour donner une explication du chemin immense qu’avait parcouru l’homme et de son ascension ; et pour lui persuader de continuer ce chemin et cette escalade. Ils ont donné à l’erreur de l’homme se détachant de la nature l’autorité d’une volonté et d’une intervention surnaturelles.\par
Et ce qui était, dans leurs discours, explication, fortifiait ce qui était dans leurs discours exhortation. Ils disaient : « Détachez-vous de la nature. Pourquoi ? Parce que Dieu le commande. Et vous voyez bien que Dieu le commande, puisque vous vous en êtes détachés déjà ; et comment l’auriez-vous pu faire, comment auriez-vous pu sauter  au-delà de votre ombre et vous élever au-dessus de vous-mêmes si une puissance supérieure à vous ne vous avait soutenus et soulevés ? » — Ce qui était explication préparait et confirmait ce qui était exhortation, et tout se tenait.\par
Dieu a donc été inventé, très habilement, par ceux qui prétendaient améliorer l’homme et voulaient qu’il s’améliorât. Mais celui qui trouve que les hommes, en se modifiant, ne se sont pas améliorés, d’abord n’a pas besoin de Dieu ; et ensuite il le rencontre comme un adversaire, comme le collaborateur très puissant, quoique n’existant pas, de ceux qui veulent dénaturer l’homme ; et donc à la fois il n’a aucun besoin d’y croire et il a un très grand intérêt à dire qu’il n’y croit point et qu’il n’existe pas.\par
C’est ainsi que le naturisme de Diderot rejoint son athéisme et le confirme, ce que, du reste, il n’était pas très nécessaire de démontrer si longuement.\par
Enfin Diderot est immoraliste, ce qui n’est qu’un nouvel aspect, mais très important, des idées précédentes. Il croit que la morale elle-même, comme la religion, comme le déisme, est une invention de politiques habiles qui ont voulu {\itshape dénaturer} l’homme pour l’améliorer, peut-être, mais surtout pour l’asservir. Il existait, dit-il, « un homme  naturel » qui était excellent ; on a créé un homme « artificiel » qui est comme garrotté d’obligations et de devoirs et qui ne sait plus qu’obéir. Secouez l’homme artificiel et marchez dans votre liberté primitive. L’homme artificiel, c’est l’homme moral. Diderot dénonce la morale comme il a dénoncé la croyance en Dieu, pour les mêmes raisons et au même titre. C’est ici qu’il se distingue et se sépare plus qu’ailleurs de Voltaire et de Rousseau. Voltaire, du reste déiste, on a vu comment et pourquoi, tient extrêmement à la morale et la fonde, comme il peut, sur l’idée de solidarité, sur cette idée que l’homme ne doit chercher que le bonheur, mais ne peut le trouver que par le dévouement de l’homme à l’homme. Le moyen de bonheur pour l’humanité, c’est l’humanité. Il faut être juste, modéré, tempérant et charitable par seul amour du bonheur ; mais bien savoir que le bonheur ne s’atteint pas autrement. Toute la morale de Voltaire est eudémonique ; mais encore il y a une morale de Voltaire.\par
Rousseau, du reste déiste, on a vu comment et pourquoi, est plus près de Diderot, en ce sens que, comme je l’ai dit, il croit très bien que l’humanité s’est trompée de chemin et que c’est l’homme primitif et naturel qui était dans le vrai ; mais il sauve la morale en la retrouvant dans l’homme  primitif et naturel, en prétendant l’y retrouver et en assurant qu’elle y est. Il croit qu’on a changé l’homme et créé un homme artificiel ; mais il ne met pas la morale au nombre des choses qu’on lui a apprises, qu’on a introduites en lui pour le changer ; et, tout au contraire, il considère la morale comme la chose qui est la plus naturelle à l’homme et comme une des choses dont la civilisation le dépouille et le vide. L’homme est né moral et il est devenu corrompu. Rousseau, tout en croyant à « l’homme artificiel », est donc moraliste très convaincu et, du reste, comme on sait, très passionné et très éloquent.\par
Diderot, non ; et la morale, « cette Circé des philosophes », comme dit Nietzsche, ne l’a pas enchanté. Il a été jusqu’au bout, devançant d’Holbach, devançant Stirner, devançant Nietzsche, et il a rangé la morale au nombre des préjugés. C’est qu’il a bien vu, comme Nietzsche l’a vu plus tard, qu’il est très probable qu’on ne vient à bout de la religion, des religions, de l’esprit religieux, qu’en venant à bout de la morale, et que si l’on n’a pas fait cela, il n’y a rien de fait, {\itshape nil actum reputans si quid superesset agendum}.\par
Car enfin, selon les cas, la religion crée la morale, ou la morale crée la religion, et quand ce n’est pas l’une qui crée l’autre, c’est celle-ci qui crée celle-là,  et il paraît quelquefois qu’elles se créent réciproquement l’une l’autre, et il semble, à certains moments, qu’elles ne sont pas autre chose que le même aspect de la même idée ou du même sentiment ou du même besoin.\par
La religion enfante la morale. Les hommes, inquiets, émus et effrayés des grandes forces de la nature qui les entourent, prennent ses forces pour des êtres puissants et redoutables. Ils cherchent à se les concilier. Ils les apaisent par des sacrifices et des dons et des offrandes. Puis ils s’imaginent que ces êtres exigent quelque chose d’eux, leur donnent des ordres qu’il s’agit de comprendre. Et cela est assez naturel, puisqu’eux-mêmes commandent certaines choses à ceux qui sont au-dessous d’eux, à leurs femmes, à leurs enfants, à leurs animaux domestiques. Ils se voient naturellement, relativement aux dieux, comme leurs femmes, leurs enfants, leurs serviteurs, leurs animaux domestiques sont relativement à eux. Ils se persuadent donc qu’ils doivent, non seulement plaire aux dieux, comme tout à l’heure, mais leur obéir, et que leur obéir est sans doute la meilleure façon de leur plaire.\par
Mais ces dieux, que commandent-ils ? Probablement et sûrement ce que les hommes commandent à ceux qui sont sous leur pouvoir. Fidélité, loyauté,  ne pas mentir, ne pas dérober, ne pas frapper, ne pas tuer, concourir à l’ordre de la maison, ne pas avoir de passions égoïstes ou les réprimer. Et voilà toute la morale qui est constituée. Elle est née de la terreur des dieux. Elle est née de la croyance aux dieux, de la crainte des dieux, et de cette tendance bien naturelle d’imaginer les dieux semblables à des hommes, et de ce raisonnement élémentaire qu’ils doivent commander aux hommes ce que les hommes commandent à ceux qu’ils dominent et qui les craignent. La religion enfante la morale.\par
D’autre part, la morale enfante la religion. L’homme veut qu’on lui obéisse autour de lui ; il veut maintenir le bon ordre autour de lui. Rien de plus jusqu’à présent. Pour cela il a recours d’abord à la force, puis à la persuasion, puis à l’exemple, et ce sont peut-être trois stades, très longs, de l’humanité primitive. — La force ne suffit pas, elle s’épuise ; elle rencontre des résistances dans les faiblesses coalisées qui sont des forces à leur tour ; elle ne suffit pas.\par
La persuasion vient au secours. L’homme s’efforce de démontrer que ce qu’il commande est le bien général et que non seulement lui, mais tous en profitent. Cela ne laisse pas de réussir : mais cela s’épuise aussi, rencontre des résistances dans  le sophisme ou l’ironie ou l’indifférence ; la persuasion ne suffit pas.\par
L’homme s’avise que l’exemple est d’une grande force et même est la seule force morale un peu sérieuse. Il plie les autres à l’obéissance en obéissant lui-même, c’est-à-dire, ce qu’il veut qui soit fait il en fait une règle permanente à laquelle il obéit le premier. Cela a une très grande influence ; mais ne suffit pas ; d’abord parce que la force de l’exemple est combattue par l’individualisme, par les passions personnelles ; ensuite parce qu’aucun homme ou presque aucun ne peut donner un bon exemple permanent, indéfectible, auquel il soit lui-même absolument et éternellement fidèle. Ni la force, ni la persuasion, ni l’exemple ne suffisent.\par
Alors l’homme s’avise de chercher une autorité en dehors de lui. Ces règles qu’il a prescrites, il les donne comme venant de plus grand et de plus fort et de plus haut que lui. Il s’appuie d’une autorité supérieure. Il dit : « Ce que je commande, ce sont ces forces qui vous entourent et qui sont si redoutables, qui vous l’ordonnent et qui l’ordonnent à moi-même ; car s’il n’en était pas ainsi, pourquoi obéirais-je moi-même à ce que je vous prescris ? L’ordre du père de famille, du chef de tribu, c’est l’ordre du ciel. »\par
 Et remarquez qu’en disant cela, il n’est pas fourbe. Ce n’est pas un mensonge diplomatique. Ce serait un mensonge diplomatique s’il n’avait, pour se faire obéir, employé que la force. Mais il a employé la persuasion et le bon exemple, c’est-à-dire qu’il est entré dans l’ordre des forces morales et c’est-à-dire des forces mystiques. Il a été éloquent et il a été consciencieux. Il s’est fait un instrument d’argumentation et il s’est fait une conscience. Dès que l’homme emploie autre chose que la force, il sent un Dieu en lui. {\itshape Habitat Deus.} Il sent en lui quelque chose qui ne se mesure pas, dont il ne connaît pas les limites, qui est spirituel et qu’il ne sent pas absolument à lui comme son poing ou son bras. Il se sent inspiré. Il croît l’être. Et ce Dieu qu’il invoque comme autorité appuyant sa parole et sa pensée, il y croit comme à l’inspirateur secret qui a dicté sa pensée et sa parole.\par
La religion est née. Elle est née de la morale. Elle est née de la nécessité de l’ordre ici-bas. Cet ordre, pour se faire respecter, a pris toutes les armes : il a pris la force, il a pris la persuasion, il a pris le bon exemple. Dès qu’il a pris le bon exemple pour arme, il est devenu conscience. Dès qu’il est devenu conscience, Dieu est né, car la conscience s’appuie sur Dieu comme sur son autorité et le suppose comme son auteur, et la divinité  est pour elle comme une conscience universelle. La morale enfante la religion.\par
Et rien ne dit que ces choses soient chronologiques, qu’elles se soient succédé dans le temps, que : ou la religion primitive ait peu à peu créé la morale ; ou la morale primitive, de par la nécessité de l’ordre, se soit peu à peu formée elle-même et ait enfin créé la religion. Ces choses peuvent avoir coexisté et s’être créées l’une l’autre réciproquement, la morale créant la religion pour ses besoins et la religion en même temps créant la morale par son seul développement ; la religion n’ayant pas besoin de la morale mais la suggérant, comme on a vu plus haut, et la morale sentant le besoin de la religion pour s’assurer, mais, du reste, la supposant presque nécessairement, comme je l’ai montré ; et toutes les deux s’enfantant l’une l’autre dès le commencement et se complétant l’une l’autre à travers les siècles ; et ces deux suites d’événements que je décrivais séparément pour la clarté de l’analyse, on peut les considérer comme jointes et s’entrelaçant, et, de tout temps, non seulement parallèles, mais comme engrenées. C’est même ce que je suis porté à croire comme étant le vrai.\par
En tout cas, la religion et la morale ont de tels liens, de telles connexions qu’il est très difficile et d’atteindre l’une sans toucher à l’autre et, détruisant  l’une, de sauver l’autre, et, maintenant l’une, de ne pas donner à l’autre un involontaire mais puissant secours. Le péril est donc grand, pour qui veut détruire Dieu, de prétendre garder la morale ; comme il serait grand pour qui voudrait détruire la morale de prétendre garder Dieu ; et les religions qui ont été immorales et qui n’ont pas suivi la morale dans son développement et dans son progrès ont dû périr et ont péri.\par
Au fond, la religion et la morale n’ont pas toujours été, et il s’en est fallu, les deux aspects de la même idée ; mais elles sont {\itshape devenues} les deux aspects de la même idée. La preuve, c’est qu’elles se convertissent l’une en l’autre. La religion devient une morale et la morale devient une religion. — La religion devient une morale. Avez-vous remarqué que le croyant passionné, exalté, n’a point de morale ? A proprement parler, il n’en a pas ; car il ne se croit obligé qu’envers Dieu, et cela, c’est de la religion et non pas de la morale. Il ne se croit obligé qu’envers Dieu, ne respire que Dieu, ne vit que par Dieu et pour Dieu. Mais ce Dieu, que naturellement il imagine sur le modèle perfectionné de l’homme, il le croit bon, juste, miséricordieux, secourable, aimé des hommes, et il croit qu’il faut l’imiter ; et, à cause de cela, il agit {\itshape comme s’il avait} une morale proprement dite.\par
 Chez cet homme la religion est devenue une morale. Dans un cerveau mal fait et qui se figurerait, qui s’imaginerait un Dieu méchant, la religion ne deviendrait pas une morale, il est vrai, et cela n’a pas laissé de se produire quelquefois ; mais dans un cerveau normal, ressortissant à la moyenne de l’humanité, la quantité d’humain que nous sommes comme forcés de mettre dans notre conception de Dieu, fait que, par cela seul que nous aimons Dieu, nous sommes moraux sans avoir de morale et en n’ayant que de la religion. C’est la religion qui est devenue une morale ou qui en tient lieu, et c’est la même chose.\par
A l’inverse, la morale devient une religion chez celui qui sent fortement la morale, sans avoir, du reste, de la religion. L’homme qui se croit {\itshape obligé}, qui est et qui veut être esclave de sa conscience, qui a le scrupuleux respect du devoir, qui s’y sacrifie ; c’est un homme qui obéit à un Dieu, à un Dieu intérieur, mais à un Dieu ; à un Dieu qu’il n’extériorise pas, mais à un Dieu. Il obéit à quelque chose qui n’exerce pas de contrainte physique ; il obéit à quelque chose de spirituel, il obéit, sans autre raison que la vénération dont il est rempli à l’égard de cela, à quelque chose de mystique qui ne dit point ses raisons et qui s’impose.\par
Ce quelque chose qu’est-ce donc, sinon un Dieu ?  Cet homme est parfaitement {\itshape en état religieux}. Se sentir obligé, c’est adorer. L’impératif catégorique de Kant, la conscience, le devoir, sont des dieux.\par
Il ne s’en faut que de rien ; à savoir il ne s’en faut que d’une métaphore. Impératif, conscience, devoir, sont des dieux sur lesquels n’a point passé l’opération métaphorique qui d’une idée fait un être. Ce sont des dieux, on pourrait même dire, qui n’ont pas été dégradés par une matérialisation que l’on peut juger grossière ou enfantine. A cela près, et c’est-à-dire à rien près, ce sont des dieux. Il n’y a pas d’homme plus religieux que l’homme qui, sans religion, est passionné de morale.\par

\begin{itemize}[itemsep=0pt,]
\item  — Mais il peut y avoir une morale sans obligation, et, dans ce cas, point d’assimilation possible entre moralité et religion, et la morale sans obligation ne peut pas devenir une religion. Le raisonnement est juste ; mais c’est précisément qu’il puisse y avoir une morale sans idée d’obligation que je ne crois point. J’espère le démontrer un jour et que la morale sans obligation est un pur rien, parce qu’elle n’a aucune force et que la théorie de la morale sans obligation n’est qu’un détour prudent ou une illusion honnête de ceux qui n’ont pas osé nier tout simplement la morale et l’attaquer de front ; mais c’est une démonstration que je n’ai pas à faire  pour le moment, ne voulant que montrer que jusqu’à présent l’humanité n’a pas trouvé le moyen d’être religieuse sans être en même temps morale, ni d’être morale sans être en même temps religieuse ; ne voulant que montrer que la religion se métamorphose en morale et la morale en religion, l’une et l’autre comme naturellement ; ne voulant que montrer qu’il y a eu jusqu’à présent entre elles des liens intimes de création réciproque et de substitution et même d’identité, sinon primitive, du moins acquise ; ne voulant que montrer enfin que qui veut détruire l’une doit s’attacher à détruire les deux.
\end{itemize}

\noindent C’est ce que n’avaient fait, ni voulu faire, ni Montesquieu, ni Voltaire, ni Rousseau. C’est ce que, prudemment encore, obliquement le plus souvent, mais non sans adresse et non sans force, s’est attaché à faire Diderot. De tous ceux qui s’attelaient à l’œuvre anticléricale, c’est lui qui avait l’œil le plus juste.\par
Il a eu beaucoup d’influence, non pas au \textsc{xviii}\textsuperscript{e} siècle, ses œuvres les plus athéistiques et immorales n’ayant du reste été connues qu’au \textsc{xix}\textsuperscript{e} ; et d’autre part le \textsc{xviii}\textsuperscript{e} siècle, comme la première moitié du \textsc{xix}\textsuperscript{e}, étant resté assez fermement « déiste » ; mais il a eu une assez forte influence sur le \textsc{xix}\textsuperscript{e} siècle et particulièrement sur  la génération qui est venue au jour vers 1850. Il s’est rencontré alors avec les athées et les immoralistes modernes dont il a semblé être le précurseur et dont, au fait, il avait prévu et devancé les idées et les démarches générales. On doit le compter parmi les plus importants facteurs de l’anticléricalisme contemporain.\par

\astertri

\noindent En résumé, car je néglige les d’Holbach, les Helvétius et autres disciples et hommes à la suite, le \textsc{xviii}\textsuperscript{e} siècle, à le considérer dans ses écrivains et dans ses directeurs d’esprit, a été, à divers titres, très antireligieux ; mais, comme j’en ai prévenu, on se tromperait à croire que l’esprit irréligieux, que même la répulsion à l’endroit du catholicisme aient été très répandus dans les masses. A la veille même de la Révolution la France était travaillée de passions antireligieuses et particulièrement anticatholiques ; mais elle était très religieuse et très catholique encore dans son ensemble.
 \section[{Chapitre IV. L’anticléricalisme pendant la période révolutionnaire.}]{Chapitre IV.\\
L’anticléricalisme pendant la période révolutionnaire.}\renewcommand{\leftmark}{Chapitre IV.\\
L’anticléricalisme pendant la période révolutionnaire.}

\noindent En immense majorité, d’après tout ce que l’on en connaît, les cahiers de 1789 furent très favorables à la religion catholique. Ils ne réclamèrent que la tolérance, c’est-à-dire le droit pour les protestants et les juifs d’exercer librement leurs cultes et d’être les égaux des catholiques devant l’État civil. Mais la religion catholique, maintenue comme religion d’État, était encore l’idée dominante et presque universelle. A la vérité, la Constituante n’en eut pas une autre ; et même elle eut cette pensée-là à l’état d’idée maîtresse et d’idée fixe. Elle voulut tellement que la religion catholique fût une religion d’État qu’elle voulut que la religion catholique fût une religion nationale.\par
La Constituante était catholique gallicane. C’était un contresens dans les termes mêmes ; mais  c’était son principe, dont aucune observation ne put la détacher. Selon elle, il devait y avoir une {\itshape Église catholique de France}, professant les dogmes catholiques, mais autonome, indépendante, séparée entièrement ou presque entièrement de Rome, n’en relevant pas, ne lui obéissant pas et n’en recevant qu’une inspiration générale\footnote{Il est défendu à toute église ou paroisse de France et à tout citoyen français de reconnaître en aucun cas et sous quelque prétexte que ce soit l’autorité d’un évêque dont le siège serait établi sous la domination d’une puissance étrangère… le tout sans préjudice de l’unité de foi et de communion qui sera entretenue avec le chef visible de l’Église universelle. (\emph{Constitution civile du clergé}, I, 4.) — Le nouvel évêque ne pourra s’adresser au pape pour en obtenir aucune confirmation ; mais il lui écrira comme au chef visible de l’Église universelle en témoignage de l’unité de foi et de la communion qu’il doit entretenir avec lui (II, 19).}.\par
Cette Église catholique particulariste devait être organisée démocratiquement. Les évêques et les curés devaient être élus par la population et, chose remarquable, par les mêmes électeurs que ceux qui nommaient les membres de l’assemblée départementale et les membres de l’assemblée du district, c’est-à-dire par des catholiques, des protestants, des juifs et des athées.\par
Il n’y a rien de plus logique et de plus naturel, quand on y réfléchit, que cette absurdité. Elle montre précisément le fond de la pensée de l’Assemblée constituante. Les Constituants considéraient  tellement l’Église qu’ils établissaient comme une Église nationale et comme une Église d’État qu’ils trouvaient tout juste et tout rationnel d’appeler à la constituer tous les citoyens actifs de France sans aucune distinction. C’est la France entière qui choisit, {\itshape parmi les catholiques}, ceux en qui elle a confiance pour diriger religieusement les catholiques. L’Église n’a un caractère véritablement national que précisément à cette condition.\par
Ce qui présidait à cette constitution civile du clergé, ce n’était ni l’idée de Montesquieu, ni l’idée de Voltaire, ni l’idée de Rousseau. C’était une idée confusément janséniste. D’après les idées de Montesquieu, il me semble qu’on aurait séparé l’Église de l’État et fait de l’Église un de ces corps intermédiaires autonomes destinés à servir de barrières à la souveraineté du gouvernement central. Il me semble ainsi, et il ne faudrait pas beaucoup me pousser pour me faire dire que j’en suis sûr. Cependant il n’est pas impossible que, magistrat à tendances jansénistes et semi-protestantes, il ne se fût rallié à la constitution civile du clergé en demandant seulement qu’évêques et curés ne fussent nommés que par les catholiques et surtout que les évêques fussent nommés par les curés ; car c’est seulement ainsi que se peut constituer un « corps » autonome. Il est possible.\par
 D’après les idées de Voltaire, on aurait tout simplement fait nommer les évêques et les curés par le ministre de l’intérieur, avec, pour le ministre de l’intérieur, droit de révocation {\itshape ad nutum}.\par
Enfin, d’après Rousseau, on aurait tout simplement exilé tous les catholiques.\par
L’idée de la Constituante était donc une idée janséniste mêlée d’idée démocratique : l’Église de France est autonome ; elle est élue par tous les Français à quelque confession qu’ils appartiennent et quelques idées qu’ils professent, comme les assemblées politiques. C’est bien une constitution {\itshape civile} de l’Église.\par
Je n’ai pas besoin de dire que cette invention, en son principe même, choquait les catholiques français de telle sorte qu’il leur était impossible de l’accepter. De là sont venus tous les désordres intérieurs et aussi toutes les fureurs qui suivirent. Le monde catholique français, qui avait été en majorité très favorable à la Révolution de 1789, se retourna tout entier contre elle, beaucoup moins à cause de la confiscation des biens du clergé qu’à cause d’une constitution de l’Église qui obligeait celle-ci à renoncer à son chef et à n’entretenir avec lui que des rapports de politesse. Ce n’est pas, comme le croyaient les Constituants, une mesure  sans importance à l’égard de catholiques que de prétendre en faire des protestants.\par
La guerre du catholicisme contre la Révolution était déclarée, à dater du 24 août 1790.\par
Mon intention n’est point du tout d’en retracer les épisodes tragiques ; mais, faisant surtout l’histoire des idées, j’attirerai l’attention sur ce point qu’en dehors de la guerre au couteau faite par les catholiques contre les hommes qui détruisaient leur Église et par les parlementaires contre des hommes qui étaient devenus leurs adversaires électoraux, la pensée {\itshape philosophique} de la Convention à l’égard de l’Église catholique était toujours la même que celle de la Constituante.\par
La Convention {\itshape complétait} l’invention de la Constituante par une série de mesures, vexatoires aussi ; mais — c’est à cela que je m’attache pour le moment — qui montraient encore très bien l’état d’esprit des parlementaires sur ce sujet. Elle voulait si bien, elle encore, que l’Église fût nationale, que l’Église fût « d’État », qu’elle légiférait dans l’Église, qu’elle faisait des lois religieuses et ecclésiastiques que curés et évêques devaient appliquer.\par
Par exemple, elle prétendait forcer les curés à marier des gens qui n’étaient pas baptisés, qui ne s’étaient pas confessés, qui avaient divorcé, qui  étaient prêtres. Elle tenait surtout à ce dernier point et décrétait (juillet 1793) que « les évêques qui apporteraient, soit directement, soit indirectement, quelque obstacle au mariage des prêtres seraient déportés ou remplacés ».\par
On conçoit très bien que de semblables prétentions révoltassent les prêtres assermentés, les prêtres constitutionnels eux-mêmes. Ce n’était cependant que la suite très logique de l’idée qui avait présidé à la Constitution civile elle-même. L’Église est d’État, elle est, comme nous-mêmes, nommée par le corps de la nation. Donc ce ne sont pas ses idées, ses principes et ses dogmes qu’elle doit appliquer, mais ceux de la nation, c’est-à-dire les nôtres. La Constitution civile du clergé était, dans l’esprit des révolutionnaires, en apparence, une libération, en réalité {\itshape un transfert d’obéissance}. L’Église catholique obéissait autrefois au concile et au pape ; en la faisant nationale, les révolutionnaires n’entendaient qu’une chose, c’est que désormais elle leur obéît et que, relativement à l’Église, l’assemblée des députés français fût le concile.\par
C’était la religion catholique elle-même qui était supprimée, sans qu’ils eussent l’air de s’en douter.\par
Ce malentendu formidable amena peu à peu, assez vite du reste, à l’idée de la séparation de  l’Église et de l’État. Cette séparation fut accomplie par le décret du 29 septembre 1795, qui peut être résumé ainsi : l’État n’empêche l’exercice d’aucun culte ; il n’en salarie aucun ; il empêche que qui que ce soit trouble l’exercice des cultes ; il défend qu’on célèbre aucun culte en dehors des locaux déclarés comme affectés à l’exercice d’un culte ; il croit devoir punir d’une façon particulièrement sévère les attaques au gouvernement qui seraient faites par les ministres d’un culte dans l’enceinte affectée aux cérémonies religieuses ; il exige des ministres de tous les cultes la déclaration qu’ils reconnaissent la souveraineté du peuple et qu’ils se soumettent aux lois de l’État.\par
C’était la séparation absolue, et c’était, à mon avis, la loi la plus sensée qu’on ait jamais faite sur cet objet. Mais c’était une loi qui, libérant l’Église catholique, la faisait très forte ; qui, laissant à l’Église catholique strictement l’autorité qu’elle pourrait tirer d’elle-même, lui en donnait une immense et très redoutable pour le gouvernement d’alors ; car l’Église, en 1795, était forte de toute l’autorité qu’elle avait gardée sur les consciences catholiques et, en outre, de tout le prestige que les persécutions récentes lui avaient donné et de toute l’horreur qu’une partie de la France éprouvait pour les terroristes.\par
 Aussi la loi de séparation, ou, en d’autres termes, la loi de neutralité ne fut nullement appliquée, ni en sa lettre ni en son esprit, par le gouvernement du Directoire. La persécution des catholiques et particulièrement des prêtres catholiques, plus ou moins déguisée, plus ou moins violente aussi, il faut le noter, fut continuelle de 1795 à 1800. En 1796, armé des lois très compréhensives sur les rebelles et insurgés, armé d’une loi aussi élastique que possible contre « toute provocation à la dissolution du gouvernement républicain et tout crime attentatoire à la sûreté publique », le Directoire faisait fusiller par ses « colonnes mobiles » de l’Ouest les prêtres estimés complices des « brigands », ou il les faisait juger et guillotiner. Une trentaine, à ce qu’estime M. Debidour\footnote{\emph{Histoire des rapports de l’Église et de l’État en France de 1789 à 1870}.}, périrent ainsi en 1796. C’était bien peu, comme le fait remarquer l’auteur, « si l’on compare ce chiffre à celui des prêtres exécutés pendant la Terreur », et certainement le Directoire était un gouvernement trop modéré ; mais enfin il continuait la tradition, un peu gêné par les deux Conseils où l’esprit de 1793 n’était presque plus représenté. Il usait des lois en vigueur contre les prêtres insermentés  qui s’obstinaient à célébrer le culte, et il ne cessait pas d’en déporter autant qu’il pouvait.\par
Cela traîna ainsi jusqu’au 18 fructidor ; mais, à partir de cette date, la persécution, qui n’avait jamais cessé, reprit avec une vigueur toute nouvelle. Le gouvernement dictatorial de Fructidor s’était, dès le premier jour (19 fructidor), accordé par loi spéciale le droit de déporter sans jugement et par simples arrêtés individuels les ecclésiastiques « qui troubleraient la tranquillité publique ». C’était purement et simplement mettre hors la loi tous les prêtres de France et faire dépendre leur liberté et leur vie (car la déportation était le plus souvent la mort, et on l’appelait « la guillotine sèche ») du seul caprice d’un gouvernement qui les détestait.\par
Le gouvernement ne se priva point d’appliquer cette loi de proscription. Il décréta la déportation en masse de six mille prêtres de Belgique. Il déporta, surtout à partir de prairial (1799), un nombre difficile à calculer de prêtres français. Dans l’Ouest, dans la Normandie, dans le Midi, les catholiques, très approuvés, même par les évêques constitutionnels et républicains, répondirent par la guerre civile. On peut dire sans exagération qu’à la veille du coup d’État de Bonaparte quiconque en France était catholique pratiquant était,  non seulement un suspect, mais un proscrit. C’était, à peu près, le rêve de Jean-Jacques Rousseau réalisé.\par
Il ne faut pas s’y tromper : c’était le résultat naturel et presque forcé, je dis en France, de la loi de séparation de l’Église et de l’État. Il n’y a rien de plus sensé et de plus juste que la séparation de l’Église et de l’État. Dans le monde moderne c’est la solution vraie, c’est la vérité. Ni le gouvernement, dans un pays partagé entre protestants, juifs, catholiques et libres penseurs, ne peut avoir une religion d’État ; ni il ne peut, sans de grands inconvénients, partager le gouvernement de l’Église catholique avec un chef spirituel qui est un étranger ; ni il ne peut se mêler de légiférer ecclésiastiquement et imposer à l’Église catholique des lois religieuses selon son goût à lui et contre son goût à elle. Il doit considérer l’Église comme une association spirituelle indépendante où il n’a rien à voir et à l’égard de laquelle il n’a que des fonctions de simple police à l’effet de maintenir l’ordre matériel. L’État n’empêche la célébration d’aucun culte, il n’en salarie aucun, il n’en gouverne ni en réglemente aucun : voilà la vérité, laquelle avait été lumineusement définie par la loi de séparation, c’est-à-dire par la loi de liberté de 1795.\par
 {\itshape Mais} précisément la loi de séparation est une loi de liberté. Et d’abord une idée de liberté entre très difficilement dans l’esprit d’un Français ; et ensuite une loi de liberté donne à une Église aussi ancienne que l’Église catholique en France et aussi enracinée, une puissance énorme, une puissance qu’il est difficile de mesurer, mais que je ne serais pas étonné qui fût plus grande ou devînt plus grande que celle dont l’Église jouissait sous l’ancien régime, sous le régime des concordats.\par
A cela un catholique dit : « Tant mieux ! » A cela un libéral dit : « Soit ! Il n’y a rien à dire. On n’a pas le droit d’empêcher une force toute spirituelle d’être forte ; on n’a pas le droit d’empêcher une idée d’avoir de l’influence. Combattez l’idée par l’idée. Faites une association de libres penseurs et d’athées qui recrute autant de partisans que l’Église catholique. Le gouvernement n’a pas à s’occuper de cela. »\par
Mais le Français raisonne rarement ainsi, et dès qu’il s’est aperçu que par une loi de liberté il a fortifié l’Église ou l’a mise en état de se fortifier, il prend peur. Il voudrait d’une loi de séparation qui fût contre l’Église et qui ne contînt rien qui fût pour elle. Il voudrait les bénéfices pour lui d’une loi de séparation, sans aucun bénéfice, pour l’Église, de cette même séparation.\par
 Dès lors, ou il maintient la séparation, mais en compensant tout ce qu’elle peut avoir d’avantageux pour l’Église par des mesures de persécution et d’oppression contre l’Église, et c’est ce qui est arrivé de 1795 à 1800 ; ou il se remet à rêver d’un retour en arrière, d’un nouveau concordat, par exemple, disposé de telle sorte qu’il replace l’Église sous la main du pouvoir central, et c’est ce qui est arrivé de 1800 à 1804.\par
Dans les dispositions d’esprit où étaient les révolutionnaires de la fin du \textsc{xviii}\textsuperscript{e} siècle, la séparation de l’Église et de l’État ne pouvait être qu’une {\itshape occasion} de persécuter plus que jamais les catholiques et qu’un {\itshape motif} de les opprimer plus que jamais. Il est probable que toute séparation de l’Église et de l’État aura toujours en France les mêmes effets. Mais n’anticipons pas et voyons comment le Consulat et l’Empire ont compris le problème.
 \section[{Chapitre V. L’anticléricalisme sous le Consulat et l’Empire.}]{Chapitre V.\\
L’anticléricalisme sous le Consulat et l’Empire.}\renewcommand{\leftmark}{Chapitre V.\\
L’anticléricalisme sous le Consulat et l’Empire.}

\noindent Les révolutionnaires, relativement au problème religieux, avaient été, un peu tâtonnant et trébuchant, des idées de Montesquieu à celles de Rousseau et à celles des jansénistes. Napoléon Bonaparte, étant un Frédéric II, alla tout droit à celle de Voltaire : les prêtres doivent être des grenadiers à vêtements longs et à idées courtes. Ils doivent être sous la main du gouvernement, soldés par lui, gouvernés et dirigés par lui, enseigner la morale, un peu de dogme, s’ils y tiennent, la fidélité et l’obéissance au gouvernement ; et avoir de la tenue. Il n’y a pas autre chose dans la question religieuse.\par
Et c’est pour cela qu’il fit le Concordat de 1802.\par
Il est assez probable que presque personne en France ne le désirait. Comme M. Debidour l’a  fort lumineusement démontré, qui aurait pu le désirer ? Ni le clergé constitutionnel, qui n’avait besoin que d’un gouvernement fort ; ni le clergé « réfractaire », qui n’avait besoin que d’un gouvernement libéral qui ne le fusillât point ; ni la masse des fidèles, qui n’avait besoin de rien, sinon que les Églises fussent ouvertes et qu’on ne la fouettât point quand elle y allait.\par
Ce qu’on désirait, c’était la fin des persécutions et la liberté religieuse assurée par une bonne police, et c’est-à-dire que ce que l’on désirait, c’était la loi de 1795 avec un gouvernement qui la fît respecter ; ce que l’on voulait, c’était que la loi de séparation fût désormais une vérité.\par
M\textsuperscript{me} de Staël semble dans le vrai, elle qui connaît fort bien l’opinion publique de cette époque, en disant : « A l’époque de l’avènement de Bonaparte, les partisans les plus sincères du catholicisme, après avoir été si longtemps victimes de l’inquisition politique, n’aspiraient qu’à une parfaite liberté religieuse. Le vœu général de la nation se bornait à ce que toute persécution cessât désormais contre les prêtres et que l’on n’exigeât plus d’eux aucune espèce de serment, enfin que l’autorité politique ne se mêlât plus en rien des opinions religieuses de personne. Le gouvernement consulaire eût contenté l’opinion en maintenant  en France la tolérance telle qu’elle existe en Amérique. »\par
C’est précisément ce que le Premier Consul donna à la France pendant deux ans, laissant rouvrir celles des églises qui n’avaient pas encore été rouvertes ; laissant sonner les cloches, ce qui pour les campagnes fut le signal officiel de la liberté rendue au culte ; ne déportant plus, rappelant les déportés, permettant la célébration du dimanche comme jour férié, substituant à l’ancien serment exigé des prêtres celui-ci : « Je promets fidélité à la Constitution » ; garantissant aux insurgés de l’Ouest la pleine liberté dans l’exercice de leur culte.\par
On peut et on doit dire que le régime de la séparation de l’Église et de l’État ne fut appliqué en son esprit et en sa lettre en France que pendant deux ans et demi, c’est à savoir de 1799 à 1802.\par
Mais l’application du régime de la séparation n’était pas ce que désirait Napoléon Bonaparte ; et ce dont se contentait la France n’était pas et ne pouvait pas être ce qu’il estimât lui suffire. Il voulait, comme Louis XIV, comme Frédéric II et comme Voltaire, que le chef de la nation fût le chef de l’Église dans son pays et qu’il s’en servît comme d’une armée à ses ordres. Il voulait, lui  chef de la France, être le maître de l’Église française. Louis XIV n’a jamais cessé d’être devant ses yeux et dans son esprit comme modèle et comme idéal.\par
Comment voudrait-on, du reste, qu’un despote raisonnât autrement qu’un démocrate ? Pour le démocrate, tout le monde doit obéir en toutes choses à la volonté de la majorité ; et le croyant doit avoir la croyance de la majorité ; et le prêtre doit enseigner au croyant la croyance de la majorité ; et il ne saurait y avoir rien de plus ni rien autre. Pour le despote, toute la nation doit obéir en toute chose à lui et croire ce qu’il croit et penser ce qu’il pense. Il n’y a aucune raison pour que le despotisme individuel et le despotisme collectif ne raisonnent pas exactement de la même façon.\par
Donc Bonaparte ne pouvait pas entendre parler d’un clergé indépendant. La conclusion logique devait donc être celle des constituants, ou à peu près : une Église nationale, séparée de Rome, gouvernée par un patriarche sous l’autorité supérieure du chef de l’État.\par
Évidemment ; mais, averti par le mauvais succès de la Constitution civile du clergé, échec qui avait démontré que le premier principe des catholiques français était d’obéir, en matière de religion, au pape et aux conciles ; qui avait démontré  que le premier principe des catholiques français était d’être catholiques ; il dut écarter tout de suite l’idée d’une Église schismatique et se ramener à celle-ci : être le maître de l’Église catholique en France {\itshape autant qu’on pourra l’être en laissant au pape autorité sur l’Église de France}.\par
Cette idée contenait le Concordat ; cette idée, c’était le Concordat lui-même. Bonaparte dut être concordataire dès le 18 brumaire an VIII.\par
De fait, il commença à négocier avec Pie VII dès 1800. Ces négociations, après bien des discussions et bien des péripéties et même bien des luttes, aboutirent au Concordat de 1802, ou, pour l’appeler par son nom officiel, à la \emph{Convention entre le gouvernement français et Sa Sainteté Pie VII} du 18 germinal an X.\par
Cette convention, pour la résumer en ses lignes générales, reconnaissait la religion catholique comme la religion de la grande majorité des Français ; portait que la religion catholique serait librement et publiquement exercée en France en se conformant aux règlements de police que le gouvernement jugerait nécessaires pour la tranquillité publique ; que, sous le nouveau gouvernement, comme sous l’ancien, les évêques seraient nommés {\itshape et} par le gouvernement français {\itshape et} par le  pape, en ce sens qu’ils seraient nommés par le gouvernement et recevraient du pape l’institution canonique et ne pourraient exercer qu’après cette double investiture ; que les curés seraient nommés par les évêques, mais à la condition d’être agréés par le gouvernement français ; que le clergé français serait payé par l’État français et que les catholiques français auraient le droit de faire des fondations en faveur des Églises ; enfin que les évêques et les curés prêteraient au gouvernement un serment de fidélité et d’obéissance, serment allant jusqu’à l’engagement de « dénoncer au gouvernement les conspirations ou les trames pouvant lui porter préjudice. »\par
C’était l’Église de France, non détachée, sans doute, de Rome ; mais attachée au gouvernement français plus étroitement qu’au gouvernement du Saint-Siège.\par
Cela ne suffisait point, cependant, au Premier Consul qui, de sa grâce et de son chef, compléta le {\itshape Concordat} par des \emph{Articles organiques}, lesquels, quoique n’ayant jamais été acceptés par le Saint-Siège, eurent toujours force de loi en France. Les articles organiques dans lesquels figurent, non sans intention sans doute, les mots d’Église gallicane, pour les résumer en leurs dispositions essentielles, portaient que : aucune bulle, aucun bref, rescrit,  décret, mandat ni autres expéditions de la Cour de Rome, même ne concernant que des particuliers, ne pourraient être reçus, publiés, ni imprimés en France sans la permission du gouvernement ; que les décrets des conciles ne pourraient être publiés en France sans l’examen et sans la permission du gouvernement ; que la déclaration faite par le clergé français en 1682 (libertés de l’Eglise gallicane) serait souscrite par les professeurs des séminaires, qu’ils s’engageraient à l’enseigner et qu’ils l’enseigneraient en effet ; que tous les ecclésiastiques français auraient pour costume de ville l’habit à la française ; que les cloches des églises ne devraient sonner que pour l’appel des fidèles au service divin et ne devraient sonner pour autre cause qu’avec permission de la police locale ; que tout autre établissement ecclésiastique que les séminaires serait interdit et que par conséquent les ordres monastiques, l’Église « régulière », demeuraient abolis.\par
C’était la mise en tutelle de l’Église de France aux mains du gouvernement. M. Debidour n’exagère point en écrivant que c’était « l’asservissement » de l’Église de France. L’Église de France, par le Concordat et par les Articles organiques, était soumise au gouvernement nouveau beaucoup plus qu’elle ne l’avait jamais été au gouvernement royal.\par
 Je ne retracerai pas les longs démêlés, si dramatiques, entre le pape et l’empereur jusqu’en 1813. Ils n’entraînèrent aucune modification importante dans le régime de l’Église française, et ce n’est que l’histoire de ce régime que j’écris.\par
1802 est la date la plus importante de toute l’histoire de l’Église de France. L’Église de France avait été un corps de l’État ; elle avait cessé de l’être et le Concordat ne la replaçait pas, et tant s’en faut, dans cette situation.\par
Elle avait été, quelques années, une Église indépendante, traquée, persécutée, mais, selon la loi, indépendante et qui pouvait rester telle, moins les persécutions et vexations, et devenir une des forces libres de la nation.\par
Elle devenait une administration, un agrégat de fonctionnaires dirigés par un ministre, très analogue à l’Université ; et, d’une part, tout ce qui pouvait la relier à l’Église universelle était soigneusement limité et presque aboli ; d’autre part, tout ce qui pouvait constituer, à côté d’elle, une Église autonome, non fonctionnaire, non soldée, non assermentée, non domestiquée (ordres monastiques), était absolument interdit.\par
De 1790 à 1795 et de 1795 à 1802, l’Église de France avait été de chute en chute et de diminution en diminution.\par
 Ceux qui se sont étonnés du rétablissement, malgré toutes les lois restrictives, des ordres monastiques en France, au cours du \textsc{xix}\textsuperscript{e} siècle, et de leur multiplication, et de leurs rapides succès, et de leur rapide progrès, et de leur enrichissement, n’ont pas compris que c’était la suite naturelle de 1802 et du déclassement de l’Église « séculière ». Une Église n’a une véritable influence sur les âmes que si elle est séparée de l’État ou plutôt du gouvernement, {\itshape soit par sa puissance, soit par sa liberté}.\par
Corps de l’État, « ordre de l’État », l’Église d’avant 1790 était séparée du gouvernement ; elle n’en dépendait pas ; elle traitait avec lui de puissance à puissance ; elle était un gouvernement spirituel à côté d’un gouvernement d’administration, d’armée et de police. Elle pouvait, elle devait avoir de l’influence sur les âmes et de l’empire sur les consciences.\par
L’Église de 1795, dont, au reste, on ne peut guère tirer des conclusions bien précises, parce qu’elle a trop peu duré, étant séparée du gouvernement et n’étant pas une puissance, était une association libre, pleine de feu et de zèle, enivrée d’esprit de propagande, ardente de passion désintéressée, analogue aux premières Églises du christianisme primitif. Elle pouvait, elle devait avoir un très grand empire sur les esprits.\par
 Ce que les âmes croyantes et pieuses ne peuvent aimer que d’une affection tiède, c’est une Église confondue avec l’État et soldée et dirigée par le gouvernement, une Église de fonctionnaires timides, craintifs, pliés aux habitudes bureaucratiques et à l’obéissance, et, sinon terrorisés, du moins assagis par la considération des honoraires à garder ou à perdre. Ces prêtres-là peuvent être de très dignes « officiers de morale » ; ils peuvent être de très honnêtes et très dignes débitants de sacrements ; ils peuvent, ne chicanons point, être de très bons prêtres ; ils ne peuvent guère être des apôtres.\par
Et c’est pourquoi, sans mépriser aucunement l’Église « séculière », la partie ardente, la partie passionnée, la partie vivante de la population catholique, en France, s’est portée vers l’Église « régulière », vers les moines, vers cette Église libre, autonome, sans attache et sans soumission au pouvoir civil, qui, après tout, avait les véritables caractères d’une Église, et a fait à cette Église, en si peu de temps, une si grande fortune. La foi vive ne s’attachera jamais qu’à une Église qui ne sera pas le gouvernement.\par
1802 a préparé la fortune des nouveaux moines d’Occident.\par
Il paraît bien qu’il le prévoyait ; car il leur interdisait  de reparaître, et c’était bien vu. Tout gouvernement qui s’attachera à déchristianiser la France devra : 1º domestiquer énergiquement l’Église officielle ; 2º interdire, éliminer, proscrire et exterminer infatigablement toute Église extra-officielle.\par

\begin{itemize}[itemsep=0pt,]
\item  — C’est-à-dire faire du despotisme de deux manières.
\item  — Sans doute, et l’Empire ne comprenait pas la politique autrement que par le despotisme de toutes les manières.
\end{itemize}

\noindent Quoi qu’il en soit, pour résumer, Napoléon Bonaparte a fait de l’Église un corps de fonctionnaires soumis ; il a empêché que toute autre Église ou quasi Église existât en France ; il a préparé ainsi les voies, sous des gouvernements moins autoritaires, à une résurrection ou à un rajeunissement très puissant des corps religieux non assermentés au gouvernement.
 \section[{Chapitre VI. L’anticléricalisme sous la Restauration.}]{Chapitre VI.\\
L’anticléricalisme sous la Restauration.}\renewcommand{\leftmark}{Chapitre VI.\\
L’anticléricalisme sous la Restauration.}

\noindent Napoléon avait dit : « Les Bourbons, quand ils reviendront aux Tuileries, seront bien avisés de se coucher dans mon lit. Il est très bien fait. » Les Bourbons n’y manquèrent point, autant qu’ils purent. Après un essai malheureux de remise en vigueur du Concordat de François I\textsuperscript{er}, la Restauration vécut {\itshape provisoirement} pendant quinze ans sous le régime du Concordat de 1802.\par
Au fond, elle ne le trouvait pas mauvais. Aucun gouvernement ne trouve mauvais un instrument de despotisme. Seulement ce fut sous la Restauration que ce régime commença de produire les effets que j’ai annoncé plus haut qu’il devait sortir. L’Église officielle commença à baisser et l’Église extra-gouvernementale, l’Église à côté de l’Église, l’Église latérale commença à se développer et à grandir. Cette Église latérale se composait, d’une part des ordres monastiques et compagnies ecclésiastiques  (jésuites et autres) ; d’autre part des associations non seulement « séculières », mais laïques.\par
Les jésuites rentrèrent sous différents noms ; les autres ordres religieux se montrèrent aussi et enfin des « congrégations » s’établirent, civiles et militaires, dans le dessein de répandre la foi et l’esprit religieux.\par
C’était inévitable. C’était tout simplement l’esprit religieux qui créait son organe, c’était une Église nouvelle qui se formait, au service du reste de la foi ancienne ; mais c’était une Église nouvelle qui se formait ; parce qu’il est de l’essence d’une Église, là où il y a un esprit religieux, d’être libre et de se sentir gênée dans les cadres et dans l’enrégimentation gouvernementale.\par
Napoléon avait dit ou au moins pensé : « L’Église sera une caserne, ou elle ne sera pas. » La Restauration pensait, à très peu près, de la même façon. L’esprit religieux répondait : « L’Église sortira de la caserne : si elle y restait, elle ne serait pas une Église. »\par
Deux partis, très différents du reste entre eux, ne comprirent pas cela ou ne voulurent pas y entendre ; c’est à savoir le parti purement gouvernemental et le parti révolutionnaire.\par
Le parti gouvernemental, quoique s’appuyant  sur la population religieuse, ou voulant s’appuyer sur elle, ne voulait pas cependant d’une Église latérale, parce qu’il était gouvernement et gouvernement français et à ce titre pénétré beaucoup plus qu’il ne le croyait d’idées napoléoniennes ; et ainsi, tiré et sollicité en divers sens, il était très embarrassé.\par
Le parti révolutionnaire se serait accommodé à peu près d’une Église officielle, c’est-à-dire sans ressort ; mais il était épouvanté de la formation d’une Église latérale, c’est-à-dire vivante. Payez l’Église ; mais n’en souffrez pas qui soit gratuite. Ce sont ceux qu’on ne paie pas qui sont les plus dangereux. Le raisonnement, instinctif ou médité, était très juste.\par
Aussi, c’est sous la Restauration que s’est créé définitivement le parti anticlérical, parce que c’est sous la Restauration qu’une Église vivante, beaucoup plus vivante que celle de l’ancien régime, s’est formée.\par

\begin{itemize}[itemsep=0pt,]
\item  — « Autrement dit, font remarquer les anticléricaux, le parti anticlérical s’est fondé sous la Restauration, parce que sous la Restauration s’est fondé le parti clérical. »
\end{itemize}

\noindent Certainement, répondrai-je ; et une religion, dans un pays qui n’est pas tout entier religieux, ne peut être qu’un parti. Je serais assez curieux de savoir  ce qu’elle pourrait être. Toujours est-il que le parti anticlérical dénonçait avec fureur et à grands cris l’immense péril que le « parti prêtre » faisait courir à la France.\par
Quel était ce péril ? Le « parti prêtre » voulait répandre ses idées par la prédication, par la propagande et par l’enseignement, exactement comme vous ou moi, comme le premier citoyen venu peut le désirer. C’était, ce semble, son droit ; et s’il y avait péril, il était facile à conjurer. Ils veulent répandre leurs idées par la prédication : ne les écoutez pas ; par la propagande : ne les fréquentez pas ; par l’enseignement : ne leur envoyez pas vos enfants.\par
Mais le Français raisonne rarement ainsi. Il en appelle toujours au gouvernement et il disait au gouvernement d’alors, comme il a continué de dire : « Forcez-moi à ne pas les entendre ; forcez-moi à ne pas les fréquenter ; forcez-moi à ne pas envoyer mes enfants chez eux.\par

\begin{itemize}[itemsep=0pt,]
\item  — Pourquoi ces mesures de contrainte contre vous-mêmes ? aurait pu répondre le gouvernement.
\item  — Parce que, auraient pu répliquer les anticléricaux, ces gens-là, étant organisés, sont une force, et on ne peut lutter contre une force que par une force. J’en appelle à la vôtre.
\end{itemize}

\noindent  — Eh bien, si ces messieurs ont organisé une force, organisez-en une.\par

\begin{itemize}[itemsep=0pt,]
\item  — C’est bien pénible. Nous ne sommes pas très actifs. Nous n’aimons pas nous gouverner nous-mêmes. Nos principes, même, nous le défendent. A nous organiser, nous aurions l’air de faire un État dans l’État, ce qui est infâme. Forcez-nous à ne pas entendre ces messieurs ; forcez-nous à ne pas avoir commerce avec eux ; forcez-nous à ne pas leur envoyer nos enfants. Nous sommes le parti libéral. »
\end{itemize}

\noindent Le gouvernement, pour les raisons que j’ai dites, accéda partiellement à ce désir. Le grand effort du parti religieux et de l’Église latérale portant du côté de l’enseignement et les récriminations du parti « libéral » portant principalement sur la même question, le gouvernement de Charles X, plus autoritaire que l’ancien régime, s’inquiéta de réprimer l’invasion cléricale dans l’enseignement. — Il faut bien s’entendre : toujours fidèle aux idées napoléoniennes, il voulait bien de l’ingérence de l’Église officielle dans l’enseignement officiel, et cette ingérence il l’avait assurée et continuait de la maintenir par toute une série de mesures très favorables à l’Église officielle ; mais que l’Église latérale enseignât elle-même ou qu’elle s’introduisît dans les écoles de l’Église  officielle (séminaires), voilà ce qu’il voyait de très mauvais œil et ce sur quoi il était d’accord avec le parti révolutionnaire et ce qu’il voulait enrayer.\par
De là les célèbres ordonnances de 1828.\par
Par l’une, dont le titre est très significatif : \emph{Ordonnance sur les écoles secondaires ecclésiastiques} (petits séminaires) {\itshape et sur l’immixtion des congrégations dans la direction de ces écoles}, il était défendu aux congréganistes de diriger les petits séminaires et d’y enseigner ; et il était enjoint à tout professeur, soit de l’Université, soit des écoles secondaires ecclésiastiques, d’affirmer par écrit qu’il n’appartenait à aucune congrégation religieuse.\par
Par l’autre, qui restreignait les droits mêmes de l’Église officielle, le nombre des élèves des petits séminaires était limité sévèrement ; défense était faite à ces établissements de recevoir des élèves externes ; ordre leur était donné d’habiller leurs élèves en ecclésiastiques dès l’âge de quatorze ans ; le baccalauréat était interdit aux élèves des petits séminaires et remplacé par un diplôme particulier, lequel ne pouvait se transformer en diplôme de bachelier que quand celui qui le détiendrait serait entré dans les ordres.\par
En un mot, on voulait que l’Église, même officielle, n’enseignât, très limitativement, que de  futurs prêtres. La Restauration en venait à défendre énergiquement le monopole de l’Université napoléonienne. Le principe était le même : l’État ne reconnaît que l’Église de l’État, soumise à ses ordres ; l’État enseigne et ne veut pas que d’autres que lui puissent enseigner.
 \section[{Chapitre VII. L’anticléricalisme sous Louis-Philippe.}]{Chapitre VII.\\
L’anticléricalisme sous Louis-Philippe.}\renewcommand{\leftmark}{Chapitre VII.\\
L’anticléricalisme sous Louis-Philippe.}

\noindent La Révolution de 1830 fut une victoire pour le parti religieux, quelque paradoxale que puisse paraître tout d’abord cette assertion.\par
Elle fut une victoire, d’abord parce que, comme le dit très bien M. Debidour, la majorité de la nation, satisfaite d’avoir renversé les Bourbons, ne s’acharna pas sur l’Église, qu’elle détestait beaucoup moins et que même elle ne détestait pas. « Satisfaite d’avoir brisé le trône, elle ne songea pas à briser l’autel. »\par
Et la Révolution de 1830 fut une victoire pour le parti religieux parce qu’elle inscrivit dans la Charte la liberté de l’enseignement, ce qui autorisait toutes les revendications du parti religieux en ce sens, ce qui liait les mains au gouvernement, ou tout au moins l’embarrassait fort au cas où il voulût revenir à la conception napoléonienne ; ce qui enfin faisait du droit d’enseigner  une loi constitutionnelle de l’État et du monopole de l’Université un abus contraire à la constitution et condamné par elle.\par
Au point de vue religieux, toute l’histoire de la monarchie de Juillet, c’est la lutte du parti religieux réclamant le droit d’enseigner en s’appuyant sur la Constitution et du parti révolutionnaire s’insurgeant contre la Constitution en refusant aux non-universitaires le droit d’enseigner.\par
Au point de vue religieux, l’histoire de la monarchie de Juillet, c’est ceux qui avaient été vaincus en apparence en 1830 criant : « Vive la Charte ! » et ceux qui avaient fait la Charte de 1830 criant : « Violons la Charte ! »\par
La solution, c’était la séparation libérale et loyale de l’Église et de l’État. Elle était conforme à la Charte ; car il n’y avait rien qui pût mieux assurer la liberté de l’enseignement que l’Église livrée à elle-même et enseignant à sa guise, à ses risques et à ses périls, au gré de la confiance des pères de famille.\par
Elle eût été, je crois, un principe d’apaisement ; car, à cette époque surtout, où les passions antireligieuses n’étaient pas encore ou n’étaient plus à leur dernier degré de violence, avec la séparation de l’Église et de l’État il y aurait eu fusion nécessaire entre l’Église officielle et l’Église « latérale »,  et cette fusion eût été salutaire, l’Église hier officielle modérant l’Église latérale, et celle-ci vivifiant l’Église hier officielle, et l’Église latérale cessant d’être aussi batailleuse et agressive qu’elle l’était et surtout qu’elle le devint quelques années plus tard.\par
Mais la séparation ne fut demandée à peu près par personne.\par
Gouvernement et hommes du juste milieu en étaient toujours à la conception napoléonienne légèrement modifiée : payer l’Église, la tenir en bride, la ménager et caresser.\par
Les hommes du parti religieux en étaient toujours à leur erreur séculaire : maintenir à l’Église son caractère, sinon d’ordre de l’État, du moins de corps de l’État, pour lui conserver son prestige.\par
Seuls, d’une part, quelques républicains demandèrent la séparation comme mesure vexatoire contre l’Église ; et, d’autre part, Lamennais la demandait passionnément comme principe et comme condition de la régénération de l’Église.\par
A mon avis, Lamennais seul avait raison.\par
Le Concordat fut donc maintenu. Concordat maintenu et promesse faite par la Charte de la liberté d’enseignement, c’est sur ce terrain qu’on se battit pendant dix-huit ans.\par
 On se battit fort. L’Église latérale, multipliant ses associations, élargissait sa propagande et créait plus ou moins subrepticement autant de maisons d’instruction et d’éducation qu’elle pouvait ; et du reste, forte du texte de la Charte, d’une part réclamait une organisation régulière de la liberté d’enseignement, d’autre part affirmait que d’ores et déjà toutes les créations de maisons religieuses d’instruction étaient en conformité avec l’esprit de la Constitution.\par
Les partis avancés faisaient, de leur côté, une guerre acharnée à l’esprit clérical, au « parti prêtre », au « jésuite » de robe longue ou de « robe courte » par le pamphlet, par le livre, par le roman, par le cours public ; et inventaient cet argument sur lequel ils ont vécu jusqu’à nos jours, qu’il ne doit pas y avoir de liberté pour les ennemis de la liberté et que, par conséquent, le libéral ne doit accorder la liberté qu’à lui-même.\par
Quant au gouvernement, il atermoyait. Il atermoya pendant dix-huit ans. Pendant dix-huit ans il reconnut que la liberté d’enseignement était dans la Charte et s’engagea à la faire passer dans la loi au premier jour. Il fut renversé avant d’avoir commencé de mettre ce projet à exécution.
 \section[{Chapitre VIII. L’anticléricalisme sous la seconde République et le second Empire.}]{Chapitre VIII.\\
L’anticléricalisme sous la seconde République et le second Empire.}\renewcommand{\leftmark}{Chapitre VIII.\\
L’anticléricalisme sous la seconde République et le second Empire.}

\noindent L’avènement brusque du suffrage universel changea les choses. Il porta aux assemblées législatives des hommes qui en majorité étaient catholiques, ou croyaient que la religion est une chose bonne. Il apparut très vite que si 1830 avait été une victoire pour les catholiques, ce que j’ai dit, mais ce qui peut être contesté, 1848 en était certainement une autre.\par
M. Debidour fait remarquer avec douleur, mais avec raison, combien la Constitution de 1848 fut cléricale. Elle fut « {\itshape placée sous l’invocation de Dieu} » ! Elle « tint à déclarer dans son préambule {\itshape qu’il existe des droits et des devoirs antérieurs aux lois positives} » ! Elle « tint à déclarer {\itshape que le citoyen doit être protégé dans sa religion} » ! Elle repoussa la séparation de l’Église et de l’État. La Charte de 1830 n’avait que « promis » la liberté  d’enseignement, la constitution de 1848 la « proclama » !\par
Pour ce qui est de « la liberté d’association, de pétitionnement, de la liberté de la presse, elle les assurait largement « {\itshape à tous} », et l’idée « {\itshape ne lui vint pas de les restreindre au préjudice des catholiques} » ! — En un mot, elle violait formellement tous les principes des vrais républicains.\par
Quant au Concordat, après quelques tentatives timides et mal coordonnées dans le dessein d’arriver à le modifier, il fut purement et simplement maintenu pendant toute la durée de la seconde République.\par
Les faits suivirent, en conformité avec les idées régnantes : expédition de la République française en Italie pour relever le pouvoir temporel du pape et « expédition de Rome à l’intérieur », comme on disait alors, c’est-à-dire organisation de la liberté d’enseignement.\par
L’expédition de Rome ne se justifiait à mon avis nullement, l’intérêt de la France n’étant pas d’intervenir dans les affaires du peuple italien, si ce n’est pour y contrebalancer l’influence autrichienne ; mais on pouvait la contrebalancer tout autrement et sans jouer en Italie précisément le rôle de l’Autriche ; et l’expédition de Rome ne fut qu’un acte de l’ambition personnelle du prince-président  désirant s’appuyer en France sur le parti conservateur.\par
Quant à l’organisation de la liberté d’enseignement, elle n’était que l’exécution du programme libéral, que l’exécution des promesses de la Charte de 1830 et de la Constitution de 1848 et une réaction contre le régime autocratique de Napoléon I\textsuperscript{er}. Elle rétablissait en France une liberté qui avait existé sous l’ancien régime, une liberté qui était indiquée très nettement dans la \emph{Déclaration des Droits de l’homme}, une liberté qui avait été inscrite dans les deux constitutions de 1830 et de 1848 ; et elle n’était en opposition qu’avec les idées napoléoniennes.\par
Cette organisation de la liberté d’enseignement (loi Falloux, 1850) admettait d’une part des écoles « publiques », écoles d’État, écoles dont les chefs et les professeurs seraient nommés par le gouvernement, d’autre part des écoles « libres », dirigées soit par des particuliers, soit par des associations. Ces dernières étaient encore soumises à l’État, en ce sens qu’elles devaient être inspectées par les agents du gouvernement et surveillées par eux, tout particulièrement au point de vue politique ; car il était spécifié que l’inspection devait avoir pour objet « la moralité, l’hygiène, la salubrité et ne porter sur l’enseignement que pour vérifier s’il  n’était pas contraire à la morale, {\itshape à la constitution et aux lois} ».\par
C’était donc une liberté très limitée encore ; car un gouvernement autoritaire aurait pu faire fermer, conformément à cet article, toute école libre où l’on n’aurait pas enseigné le culte du gouvernement, toute école libre fréquentée par les enfants des familles de l’opposition et où ces enfants auraient fait à M. l’inspecteur des réponses conformes à leurs sentiments, réponses qu’on aurait supposées dictées par les professeurs.\par
Il l’aurait pu ; car il aurait traduit le directeur devant le conseil académique, lequel avait le droit d’interdire l’exercice de la profession à tout délinquant ; et les conseils académiques étaient composés de telle sorte que les fonctionnaires ou membres désignés par le gouvernement y étaient en majorité.\par
Il faut reconnaître du reste que la loi Falloux était favorable au clergé en ce sens qu’elle permettait d’enseigner, sans brevet de capacité, à tout ministre d’un culte reconnu par l’État ou à tout religieux qui aurait fait un stage de trois ans dans un établissement libre. C’était dispenser les religieux, non seulement du brevet de capacité, mais de capacité. Il est vrai ; mais encore de quel droit l’État prétendrait-il me défendre de confier mon  fils à un ignorant ou prétendu tel ? C’est à moi d’en juger. Si l’État a le droit et même le devoir d’interdire la profession de médecin à un non-diplômé parce que ceci est de salubrité publique ; il n’a aucunement le droit d’interdire l’enseignement à qui que ce soit. En matière d’enseignement, les diplômes qu’il décerne ne sont que des indications : « Je désigne monsieur un tel comme ayant été jugé par moi apte à enseigner. » Je ferai peut-être bien, moi, particulier, de me fier à cette indication ; mais j’ai le droit de n’en avoir cure et de confier mon fils à un homme que j’ai jugé, moi, apte à enseigner mon fils ; et c’est un abus énorme que de prétendre m’obliger à ne le confier qu’à celui que vous avez estampillé. Je vous remercie de l’indication que vous me donnez et j’en pourrai tenir compte, mais je vous en remercie à la condition que je conserverai mon droit de n’en point profiter.\par
On juge bien, du reste, que l’intérêt de l’Église était que ses instituteurs et professeurs fussent aussi bons que les instituteurs et professeurs de l’État, et c’était précisément un bienfait de la liberté qu’elle établît la concurrence.\par
De fait, pendant le quart de siècle qui suivit, si les professeurs religieux de l’enseignement secondaire, Jésuites et autres, furent (peut-être)  inférieurs aux professeurs universitaires de l’enseignement secondaire, les frères des Écoles chrétiennes furent incomparablement plus instruits que les instituteurs de l’État. Et, encore une fois, le citoyen prétendu libre d’un État prétendu libre a le droit de prendre pour l’aider à élever son fils qui il veut, et l’État ne doit avoir sur les professeurs qu’un droit d’inspection strictement relatif à l’hygiène du local et à la moralité de l’enseignement et de l’enseignant.\par
La loi Falloux était donc certainement favorable au clergé, mais elle était formellement conforme aux principes du libéralisme. C’était une loi à la marque de 1789. Ceux qui la firent pouvaient dire aux républicains : « Nous vous combattons, certainement ; mais nous vous servons selon vos principes, selon les principes que vous invoquez toujours quand vous êtes les plus faibles, et que vous oubliez toujours, comme il est naturel, quand vous êtes les plus forts. Mais précisément, en ce moment, vous êtes les plus faibles. De quoi donc vous plaignez-vous ? »\par
Et en effet, les législateurs de 1850 auraient pu « faire du despotisme » réactionnaire, supprimer l’Université et confier l’enseignement à l’Église. Ils « faisaient de la liberté » ; et ce qu’on leur demandait, c’était de « faire du despotisme » républicain.  En vérité, c’était leur demander trop.\par
Remarquez bien que, même, ils ne « faisaient pas de la liberté » pure et simple. La liberté pure et simple consiste en ceci : l’enseignement n’est pas une affaire de l’État ; enseigne qui veut ; l’État n’enseigne pas ; l’État surveille les maisons d’enseignement au point de vue de l’hygiène et de la moralité. — Ce qu’instituaient les législateurs de 1850 était très loin de cela. Ils maintenaient l’enseignement d’État ; et, {\itshape à côté de lui}, ils permettaient que les particuliers enseignassent librement. Ce n’était qu’une demi-liberté de l’enseignement.\par
Oui, ce n’était qu’une demi-liberté de l’enseignement ; car remarquez bien que permettre une entreprise commerciale, — ceci n’est qu’une comparaison, — la permettre, et puis en faire soi-même une du même genre à laquelle on force tous les contribuables à coopérer de leurs deniers, c’est permettre à l’entreprise libre de vivre, mais lui faire la vie extrêmement dure, et c’est presque la condamner à mort au moment même qu’on lui donne le droit de naître. Je l’ai dit, je crois, quelque part, c’est comme si l’État, qui a une ligne de chemin de fer de Paris à Bordeaux, permettait, sans doute, à la Compagnie d’Orléans d’avoir une ligne de Paris à Bordeaux, mais forçait tous les  voyageurs de la Compagnie d’Orléans à payer à l’État une redevance pour entretenir la ligne d’État Paris-Bordeaux. Il y aurait quelque chance pour qu’on ne voyageât plus que sur la ligne de l’État.\par
D’autant plus que les moyens pour l’État enseignant de faire concurrence à l’enseignement libre sont illimités. A un universitaire qui était partisan — naturellement — du monopole universitaire, je disais, il y a trente-cinq ans : « Il y a un moyen bien simple d’assurer en pratique le monopole universitaire, tout en affirmant qu’on ne monopolise rien et que la liberté d’enseignement est pleine et entière. Donnez tout l’enseignement pour rien, primaire, secondaire, professionnel, supérieur, tout pour rien. Tous les établissements libres seront ruinés.\par

\begin{itemize}[itemsep=0pt,]
\item  — Non, ils ne le seraient pas complètement. Il y aura toujours des gens qui aimeront mieux payer deux fois, une fois comme contribuables pour nous entretenir, une fois comme parents à la caisse des professeurs libres, que de venir à nous. Il n’y a que le monopole qui vaille. »
\end{itemize}

\noindent Peut-être ; mais enfin ceux qui maintenaient l’enseignement de l’État comme concurrence redoutable et pouvant devenir quasi mortelle à l’enseignement libre ; ceux qui n’admettaient qu’un enseignement libre très surveillé par l’État et luttant  contre l’enseignement de l’État à armes très inégales ; ceux qui faisaient « payer deux fois », comme s’ils leur imposaient une amende, les parents usant de l’enseignement libre ; ceux, donc, qui accordaient une liberté très limitée en maintenant sinon le monopole, du moins le privilège de l’enseignement de l’État ; ceux-là non seulement ne pouvaient être incriminés de livrer l’enseignement à l’Église ; non seulement ne pouvaient être incriminés de faire balance égale à l’Église et à l’État ; non seulement ne pouvaient être incriminés d’accorder la pleine liberté d’enseignement ; mais ils pouvaient l’être de n’accorder à cet égard qu’une tolérance légale ; et ce n’est pas autre chose, en réalité, qu’ils avaient fait.\par
Les réclamations et accusations de la part des « vrais républicains » n’en furent pas moins bruyantes et furieuses, comme il est naturel, puisque le tempérament du démocrate est d’être purement et simplement absolutiste ; et c’est à propos de cette loi que Victor Hugo ne manqua point de dire qu’elle constituait « un {\itshape monopole} au profit de la sacristie et du confessionnal ».\par
Les effets du nouveau régime furent les suivants. La bonne intelligence de l’Église et du gouvernement d’abord présidentiel, puis impérial, dura jusqu’en 1859. Extérieurement au moins,  l’Église catholique eut le prestige officiel qu’elle avait eu sous le premier Empire et sous la Restauration. M. Debidour note avec regret « les processions se déroulant dans les villes avec participation des fonctionnaires et de l’armée, le travail suspendu le dimanche dans les chantiers publics et les cabarets fermés pendant les offices » !\par
Les moines rentrèrent en foule, y compris les Jésuites, qui n’étaient du reste jamais sortis complètement ; l’Université fut intimidée et molestée ; les professeurs anticléricaux mal notés ; une véritable réaction cléricale, encouragée, il faut le dire, et ce que je blâme parfaitement, par le gouvernement, qui doit toujours en ces matières n’être absolument d’aucun côté, mena assez durement le pays.\par
On ne comprend pas très bien comment un gouvernement qui avait pour lui la quasi unanimité de la nation, qui était sorti d’un plébiscite où il avait obtenu huit millions de voix contre un million, qui dans la Chambre élective avait trois cent cinquante partisans dévoués et trois ou cinq opposants, sentit le besoin de s’appuyer sur le parti clérical et de l’appuyer.\par
Il était persuadé sans doute que la France, parce qu’elle détestait les révolutionnaires de 1848, ce qui était vrai, était cléricale, ce qui était faux. La  France de 1848 à 1859, et même plus tard, ne voulait que l’ordre et la répression, sans rigueurs du reste, des désordonnés.\par
L’état d’esprit du gouvernement présidentiel et impérial fut cependant celui que je viens de dire jusqu’en 1859, par cette aberration sans doute qui consiste à « grossir l’ennemi » et à ne trouver jamais que l’on a assez de soutiens ; et, non satisfait d’avoir ces deux appuis formidables, le peuple et l’armée, ce gouvernement voulait encore s’appuyer sur l’Église.\par
A partir de 1859, sa politique fut toute différente, mais cette fois encore plus inintelligible. Il fit la guerre d’Italie, qui était avant tout une guerre anticléricale, qui déchaînait la révolution dans la Péninsule et qui fut saluée avec enthousiasme par tout le parti anticlérical français, toujours plus soucieux, naturellement, des intérêts de l’anticléricalisme que des intérêts de la France ; — et d’autre part il prétendit soutenir et défendre le pouvoir, tant spirituel que temporel, du souverain pontife.\par
Je ne sais pas quel était son dessein, à travers cette incohérence, ni s’il en avait un ; mais les effets furent ceci : à l’extérieur, l’unité de l’Italie, qui ne « contenait » point du tout, comme on a trop dit, l’unité de l’Allemagne, et les deux « grandes  fautes » sont indépendantes l’une de l’autre ; mais qui était en elle-même un échec pour la France, celle-ci n’ayant aucun intérêt à créer sur ses flancs une grande puissance susceptible de devenir l’alliée d’un de nos ennemis ; — à l’intérieur, le parti bonapartiste coupé en deux et par là sensiblement affaibli et n’offrant plus un appui, une base aussi solide qu’auparavant.\par
C’est à partir de 1859, en effet, qu’il y eut des bonapartistes cléricaux et des bonapartistes anticléricaux. — Il y eut des bonapartistes cléricaux, gens à la manière et à la mode de 1850, conservateurs et cléricaux comme fond permanent, ralliés à l’Empire comme à un pouvoir fort, dompteur de révolutionnaires. — Il y eut des bonapartistes anticléricaux, gens à la mode de 1810 ou de 1820, nouveaux exemplaires des « libéraux » de la Restauration, autoritaires et despotistes comme fond permanent, mais désirant que l’on fût despotique en dehors de l’Église et un peu contre elle.\par
C’est ainsi qu’en 1860, une commission composée du « ministre d’État », du président du conseil d’État, du ministre de l’intérieur, du ministre de l’instruction et du garde des sceaux, fit un rapport qui n’eut pas de suite, mais où était signalé le danger de la liberté d’enseignement avec tous les arguments dont en 1903 les républicains, comme  il va de soi, se sont servis à leur tour : « La liberté d’enseignement, qui {\itshape semble} consacrer un grand principe d’équité, a cet immense inconvénient de perpétuer dans notre pays, par la diversité de l’éducation donnée à la jeunesse, toutes les divisions sociales et politiques qui s’effaceraient avec le temps dans l’unité de l’enseignement de l’État. Les établissements religieux sont le refuge des enfants appartenant à des familles qui n’adoptent ni les principes de 89 ni le gouvernement impérial. L’instruction qui s’y distribue est conforme à ces regrettables tendances. » — C’était l’argument des « deux Frances ». — Quant à l’argument des « deux Églises », il y était aussi, et cette ancienne volonté napoléonienne qu’il n’y eût qu’une Église, officielle, domestiquée et obéissante au gouvernement français, se retrouve dans ce document : « Les congrégations religieuses visent, en multipliant leurs noviciats et leurs couvents, à remplacer notre clergé séculier, c’est-à-dire les curés et les desservants qui sortent de nos séminaires, qui sont originaires de notre pays et qui reconnaissent la direction de leur évêque attaché lui-même au pays et à l’empereur par sa nationalité et par son serment. Or, le clergé régulier est tout simplement une milice romaine, secouant le joug de l’Ordinaire, n’ayant ni patrie ni personnalité, obéissant,  {\itshape perinde ac cadaver}, au gouvernement absolu d’un étranger… »\par
Ce rapport est, suppose-t-on, de M. Baroche. On le dirait rédigé par M. Georges Clémenceau.\par
C’est ainsi encore qu’à un certain moment, un peu plus tard, en 1862, on vit des préfets ultrabonapartistes protéger énergiquement ou imposer la représentation du \emph{Fils de Giboger} sur les théâtres de province et y applaudir avec ostentation.\par
Les bonapartistes anticléricaux étaient les uns très convaincus, il faut toujours faire cette part, les autres très malins. Ces derniers recueillaient le double bénéfice d’être fort bien avec le pouvoir comme bonapartistes et d’être populaires et d’être qualifiés de « libéraux » comme anticléricaux. « J’en ai connu de ces saints-là ! » L’attitude était excellente, point très difficile à soutenir et assez rémunératrice.\par
Cependant le parti bonapartiste était bifurqué et ce lui était une faiblesse. Les bonapartistes anticléricaux voyaient de mauvais œil les bonapartistes cléricaux et déploraient les services rendus encore au Saint-Père, l’occupation de Rome par nos troupes, le maintien des Jésuites en France, etc., et, — je parle des convaincus, — se détachaient un peu de l’Empire pour incliner vers le républicanisme.\par
Les bonapartistes cléricaux, se prétendant détenteurs  de la tradition de 1850 et véritables appuis, soutiens et même fondateurs du second Empire, regardaient avec colère les bonapartistes anticléricaux, déploraient l’expédition d’Italie, rendaient, avec raison du reste, le gouvernement responsable de l’unité de l’Italie et du triomphe de la Révolution en Italie, et se refroidissaient singulièrement à l’égard de l’Empire, et inclinaient vers la légitimité ou l’orléanisme.\par
A partir de 1859, l’Empire eut « deux armées au lieu d’une », comme on dit pour se consoler quand on a vu son armée coupée en deux par l’ennemi ; mais il eut deux armées peu solides, défiantes l’une de l’autre, et défiantes même de lui, et qui ne valaient pas son armée unique d’auparavant.\par
Je suis de ceux qui sont persuadés que l’Empire était encore infiniment fort en 1870 et que ce n’est que la guerre de 1870 qui l’a tué ; mais enfin il était un peu plus faible de 1859 à 1870 que de 1850 à 1859, et il fallait l’indiquer pour être complet.\par
Quant aux effets de la loi sur l’enseignement, vous pouvez demander à tout anticlérical, de 1860 environ à 1906, quels ils ont été. Vous aurez cette réponse {\itshape ne varietur} et qui n’a pas varié et ne variera point : « La liberté de l’enseignement a fait {\itshape deux Frances}, et tous nos malheurs viennent de là. » Un sénateur français qui, du reste, est très intelligent,  mais qui, lorsqu’il s’agit de cléricalisme, paraît l’être moins, disait récemment : « Avant 1850 l’unité morale de la France existait, depuis elle n’existe plus. » Et l’on sait ce qu’était l’unité morale de la France au \textsc{xviii}\textsuperscript{e} siècle, sous la Révolution, sous la Restauration et sous Louis-Philippe. Elle était figurée par le mot d’Horace : « {\itshape Tot capita, tot sensus.} »\par
Je ne vois que deux époques où il y ait eu en France une « unité morale », très relative encore ; c’est l’époque de Louis XIV et l’époque de Napoléon I\textsuperscript{er}. Sous Louis XIV, malgré les jansénistes et les protestants, on peut dire à la rigueur qu’il y a une unité morale, que toute la France, à très peu près, est réunie dans un même sentiment : le culte du roi et le désir d’extension du territoire. — Sous Napoléon, malgré Chateaubriand, M\textsuperscript{me} de Staël et quelques émigrés à l’extérieur ou à l’intérieur, on peut dire, à la rigueur, que la France est réunie dans un même sentiment : l’idolâtrie de l’empereur et le désir de conquêtes et de gloire. Sauf ces deux époques, où il a existé une unité morale, qui, du reste, n’est pas du tout de mon goût, l’unité morale de la France n’a jamais été.\par
Et c’est un bien. C’est un bien évidemment dans une certaine limite que j’indiquerai tout à l’heure ; mais c’est un bien. La diversité des sentiments et  des idées, c’est la vie même, intellectuelle et morale, d’un peuple. Celui qui a dit : « Il faut qu’il y ait des hérésies », a dit une des paroles les plus profondes qui aient été dites dans ce monde. Il faut qu’il y ait des hérésies, c’est-à-dire il est bon qu’on pense, et la seule chose qui indique qu’un peuple pense, c’est ceci qu’il pense de différentes manières. Un peuple qui tout entier penserait exactement la même chose — rêve de nos démocrates autoritaires et unitaires — c’est qu’il ne penserait pas du tout, ce qui peut-être n’est pas très sain. La multiplicité des sectes religieuses prouve la vitalité du sentiment religieux, et la multiplicité des écoles et partis politiques, philosophiques, économiques, prouve simplement qu’un peuple s’occupe de philosophie, de politique et d’économie, et n’est pas absolument abruti.\par
Il y a une limite, sans doute, et je crois la connaître. Il peut arriver qu’un homme, et ceci ne serait rien, mais il peut arriver qu’un groupe considérable soit tellement entêté de son idée, religieuse, philosophique, économique, qu’il la préfère à la patrie. Un catholique peut être papiste avant d’être français ; un protestant peut être genevois plutôt que français ; un socialiste peut être prolétaire cosmopolite et n’être point français du tout. Des groupes plus catholiques que français ou plus  protestants que français, à ce point que les uns et les autres appelaient l’étranger à leur secours sur le sol français, cela s’est vu au \textsc{xvi}\textsuperscript{e} siècle. Des groupes plus socialistes que français, à ce point qu’ils n’appartiennent qu’au prolétariat universel et non à la France, qu’ils sont d’une classe et qu’ils ne sont point d’un pays, cela se voit de nos jours.\par
Par parenthèse, on se demande toujours quand le patriotisme a commencé en France. Je le crois très ancien, mais j’estime qu’il a commencé très nettement, très précisément et sans rien qui restât confus, qu’il a commencé définitivement, si l’on me permet de parler ainsi, qu’il a été constitué, le jour où cela a paru un crime de préférer son parti à la France, ou de préférer ses droits de caste à la France ; le jour où cela a paru un crime qu’un protestant ou un catholique appelât l’étranger sur le territoire français ; le jour où cela a paru un crime qu’un grand seigneur passât à l’étranger et combattît contre les Français pour venger {\itshape même une injustice} commise à son égard ; et j’estime donc que le patriotisme français a été constitué au commencement du \textsc{xvii}\textsuperscript{e} siècle. La {\itshape catholicisation} d’Henri IV me paraît un des actes les plus sérieux, les plus philosophiques, les plus profondément conçus par une grande intelligence, qui aient jamais été. « Paris vaut bien une messe » est une  boutade de Béarnais qui doit se traduire ainsi : « Je suis l’État, ou tout au moins je le représente, et il se résume et il s’exprime en moi. Or cet État est en majorité catholique. Je dois donc être officiellement catholique, non pas, et je le prouverai, pour n’admettre que les catholiques comme citoyens français ; mais pour prouver qu’aucun de nous ne doit préférer ses façons particulières de penser au bien de l’État. Particulier, je resterais protestant et bon français ; chef de l’État, je serais soupçonné de n’être qu’un parti au pouvoir et de préférer ce parti à l’ensemble de la nation, si je restais protestant. Je me fais catholique pour prouver que le particularisme protestant n’est plus mon fait. Je me fais catholique, non pas, au vrai, pour me faire catholique, mais pour déclarer que, politiquement, je ne suis plus huguenot, ce qui est précisément mon devoir ». L’acte d’Henri IV fut une déclaration de patriotisme ; ce fut la {\itshape nationalisation} d’Henri IV.\par
Il y a donc, en effet, une limite à la diversité intellectuelle et morale. La diversité intellectuelle et morale est un bien en soi ; elle devient un mal lorsqu’elle va jusqu’à préférer une idée particulière, ou de groupe, à la patrie. Peuples mauvais et qui ont un germe de mort : d’une part, ceux qui n’ont qu’une façon de penser et qui, par conséquent,  ne pensent point ; d’autre part, ceux dans lesquels des groupes considérables préfèrent leur façon de penser à la patrie. Peuples excellents (Angleterre, États-Unis, Allemagne), ceux qui pensent beaucoup et par conséquent de façons diverses, mais dans lesquels aucun groupe ne préfère sa pensée à la patrie.\par
Et c’est bien précisément pour cela qu’il ne faut pas violenter les façons de penser et que la liberté de penser, de parler, d’écrire, d’enseigner est une mesure patriotique et une mesure de salut public. Comment cela ? Mais parce que ces émigrations, ces sécessions, ces renoncements à la patrie, ces déterminations extrêmes de l’intelligence et de la conscience qui consistent à préférer son idée à son pays, {\itshape vous les créez}, en gênant les libertés de pensée et de croyance. Un sentiment peut être fort quand il est libre, mais il est violent quand il est contrarié et opprimé. En accordant la liberté de penser et de répandre sa pensée, vous maintenez donc tel sentiment dans la mesure où il est actif, mais non véhément et où il ne se préfère pas à l’idée de patrie et à l’amour de la patrie. En réprimant, en proscrivant la liberté de pensée, vous poussez tel sentiment à ce degré d’acuité où il se préfère à l’idée de patrie et s’insurge contre une patrie qui l’opprime. En persécutant  un catholique, vous en faites naturellement et nécessairement un ultramontain ; en persécutant un protestant, vous en faites naturellement un genevois ou un hollandais ; en persécutant M\textsuperscript{me} de Staël, vous en faites une… européenne ; en supprimant la liberté de pensée en France, vous faites d’un simple libéral un homme qui aimerait autant être américain ou être belge.\par
Voilà ce qui me fait appeler la liberté une mesure de salut public. Il est très vrai que la diversité de pensée peut aller jusqu’à rompre le faisceau national ; mais il est aussi vrai que « l’unité morale » imposée pousse tout simplement les pensées indépendantes à devenir antipatriotiques et rompt le faisceau national bien plus complètement. La liberté est consolidatrice du patriotisme ; le despotisme est destructeur de l’idée de patrie. En un mot, la seule unité morale qu’il faille désirer et à quoi il faille tenir, c’est l’unité patriotique ; et cette unité morale, celle-ci, la liberté ne la détruit pas ; elle la confirme. Voilà pour moi la vérité absolue.\par
Or — revenons — les « deux Frances » que la liberté d’enseignement était prétendue avoir créées étaient-elles véritablement deux Frances ennemies l’une de l’autre ? L’une des deux était-elle « étrangère » ou amie de l’étranger, ou cosmopolite ? La diversité des pensées allait-elle jusqu’à seulement  détendre et relâcher le lien national ? Pas le moins du monde et d’abord pour la raison que j’ai dite. Les Français catholiques, élevés très librement dans des maisons catholiques, n’avaient aucune raison d’être froids à l’égard d’un pays qui ne violentait ni ne contrariait leurs idées et sentiments et qui ne leur demandait que de rester Français. J’ai connu beaucoup de jeunes gens élevés dans ces maisons. Ils étaient les meilleurs Français du monde ; et c’est depuis les violences de la République autoritaire qu’ils le sont moins, ce que, sans l’approuver, je ne puis m’empêcher de comprendre et de trouver assez naturel.\par
Je n’ai pas besoin de rappeler ce qui a été reconnu par tous et proclamé par tous, à savoir qu’il n’y a eu aucune différence entre le patriotisme des anciens élèves des maisons religieuses et celui des anciens élèves de l’État pendant la guerre de 1870 ; et je ne sache pas qu’à l’heure où j’écris les antipatriotes soient d’anciens élèves ou d’anciens professeurs des maisons religieuses.\par
Remarquez de plus que les enfants et jeunes gens qui étaient confiés aux éducateurs religieux appartenaient en majorité à ce qu’on appelle plus ou moins exactement les hautes classes. Or tout le monde, certes, a intérêt à être patriote, mais particulièrement les citoyens appartenant aux classes  hautes ; par la très bonne raison, qu’il ne leur est pas difficile de comprendre ou de sentir, qu’en cas de conquête de la patrie, ce sont eux qui seraient remplacés comme dirigeants et comme personnages considérables par des citoyens du peuple vainqueur. Ce n’est pas, par exemple, à un futur saint-cyrien ou polytechnicien, ou à son père, qu’il faudrait venir dire que France ou Allemagne c’est indifférent et que l’on doit avoir des objets de préoccupation plus considérables. Les hautes classes d’une nation sont, comme les pays frontières, non pas plus patriotes, mais patriotes avec plus d’inquiétude, comme premières proies de l’invasion. Si tout se réglait par l’intérêt, les hautes classes seraient très patriotes et le peuple le serait peu. Comme tout se règle par l’intérêt et le sentiment mêlés, ce n’est pas ainsi que vont les choses ; mais il reste encore que les hautes classes sont patriotes plus sûrement, parce qu’elles le sont forcément.\par
Ce n’est donc pas aux enfants des classes dites supérieures que des professeurs, religieux ou autres, pourraient, le voulussent-ils, enseigner l’antipatriotisme ou un patriotisme très tempéré. Aussi ne l’ont-ils pas fait et ne le feront-ils jamais. Les deux Frances, au point de vue de « l’unité patriotique », n’ont donc jamais existé.\par
 Qu’elles aient existé au point de vue de la fameuse « unité morale », je ne le nie point. On n’enseignait pas dans les écoles religieuses l’amour de la Révolution et on l’enseignait dans les écoles de l’État. C’est incontestable. Les professeurs des lycées, ou étaient secrètement républicains, ou appartenaient à ce bonapartisme anticlérical dont j’ai parlé plus haut. Par contrepartie, les professeurs des maisons religieuses ou étaient légitimistes, ou appartenaient à ce bonapartisme clérical que j’ai indiqué plus haut également. Je demandais, en Bretagne, vers 1858, à un élève de petit séminaire : « Êtes-vous d’accord en politique ?\par

\begin{itemize}[itemsep=0pt,]
\item  — Non ! Il y a deux camps, les blancs et les bleus.
\item  — Qu’est-ce qu’ils sont, les blancs ?
\item  — Ils sont pour Henri V.
\item  — Et les bleus ?
\item  — Ils sont pour l’empereur.
\item  — Et des républicains ?
\item  — Il n’y en a pas. »
\end{itemize}

\noindent Républicains et bonapartistes anticléricaux, c’était la population des lycées ; légitimistes et bonapartistes cléricaux, c’était la population des maisons ecclésiastiques. Voilà, au vrai, les « deux Frances » de 1850 à 1870.\par
Voilà ce qui exaspérait les partis « avancés ». Et  il n’y a pas de quoi s’exaspérer. D’abord parce que, comme on l’a bien vu, cela n’empêchait pas tous ces jeunes gens d’être de très bons Français, ce que j’avoue qui me suffit ; ensuite parce que le péril est, précisément, comme je l’ai dit, de faire de mauvais Français en tracassant les gens ; enfin parce que, non seulement la concurrence matérielle, pour ainsi parler, est une bonne chose, et quand il y a deux enseignements dans un pays ils sont bons tous deux ; mais encore et enfin parce que la concurrence morale, si je puis dire ainsi, est excellente.\par
Ces jeunes gens, tous bons Français, tous patriotes, ayant ainsi un lien commun, nécessaire et suffisant, ayant ainsi une « unité » nécessaire et suffisante, se rencontraient dans la vie, discutaient, remuaient des idées et se faisaient, par la discussion, des idées personnelles, et il n’y a que les idées personnelles qui soient des « idées-forces », et des idées fortes, et des idées fécondes.\par
Supposez, par impossible du reste, du moins dans le monde vraiment civilisé, un peuple à qui des éducateurs n’ayant qu’une manière de penser inculqueraient indéfiniment cette conception générale ; et qui du reste n’aurait pas une force intellectuelle suffisante pour réagir ; c’est le peuple que semblent rêver les partisans de l’unité  morale, et ce peuple, je n’en pense qu’une chose, c’est qu’il deviendrait promptement idiot.\par
Tels furent, avec leurs inconvénients peut-être, avec celui surtout d’être désagréables à ceux qui veulent mener la France comme un éternel enfant, tels furent, beaucoup meilleurs à mon avis que mauvais, les effets de la liberté de l’enseignement en France sous le second Empire.\par

\astertri

\noindent La politique générale de la fin du second empire fut ce que l’on sait trop. Au point de vue des choses religieuses, il y a un point assez important à éclaircir. Les anticléricaux, croyant ou feignant de croire que tous les maux et toutes les catastrophes viennent du catholicisme et ne peuvent venir d’autre source, ont assuré que les désastres de 1870 sont imputables aux catholiques. Ils ne pouvaient y manquer. Quand on lit dans un ouvrage de M. Edme Champion : « La pente qui éloigne de Voltaire aboutit à Sedan », on ne comprend pas très bien, parce que c’est une sorte de résumé de philosophie historique plutôt qu’un réquisitoire précis. Mais les journalistes et les historiens du parti sont entrés dans le détail de cette considération et l’ont exposée minutieusement.\par
Voici comment ils raisonnent, je puis dire tous ;  car j’ai retrouvé l’argumentation un peu partout presque dans les mêmes termes.\par
L’Empire s’est obstiné jusqu’au bout à maintenir et soutenir le pape à Rome, à défendre les restes du pouvoir temporel, à refuser Rome aux Italiens. Aux approches de la guerre de 1870, l’Empire chercha des alliés. Il en trouva un, c’est à savoir l’Autriche ; mais l’Autriche ne voulait entrer en campagne contre la Prusse qu’avec le concours de l’Italie, pour n’être pas attaquée par celle-ci, une dixième fois, pendant qu’elle serait occupée ailleurs. Mais l’Italie ne voulait entrer dans cette triple alliance que si on lui laissait Rome. L’alliance franco-autrichienne dépendait donc de l’agrément de l’Italie, et l’agrément de l’Italie dépendait de l’abandon de Rome aux Italiens.\par
Jusqu’au dernier moment, Napoléon III tergiversa, recula. A la veille de l’entrée en campagne, ne s’agissant plus d’alliance austro-française, mais seulement d’alliance italienne, il hésita encore à abandonner le Saint-Père par un engagement formel : l’alliance n’eut pas lieu. La France fut vaincue. « C’est ainsi, dit éloquemment M. Debidour, que, conduit à Sedan par la justice immanente des choses, Napoléon III paya, au bout de plus de vingt ans, le tort de s’être abandonné à l’Église par ambition et fit, du même coup, payer à la France la  faiblesse qu’elle avait eue de s’abandonner à lui. Son alliance avec le pape l’avait élevé au trône ; elle contribuait maintenant à l’en faire descendre. Quant à la France, elle lui avait valu dix-huit ans de servitude, elle lui valait maintenant d’être envahie en attendant d’être démembrée. »\par
Cette argumentation et ces conclusions sont assez douteuses. Laissons de côté la servitude intérieure : la France a été asservie à l’Empire, non pas à cause de l’alliance de l’Empire avec le catholicisme, mais parce qu’elle a voulu l’être, depuis 1849 ; il n’y a pas d’exemple d’asservissement volontaire plus net et plus éclatant ; la France de 1849 à 1870 était bonapartiste, et cela suffit pour qu’on soit asservi à un Bonaparte.\par
Quant aux alliances {\itshape in extremis}, je crois qu’on n’aurait jamais eu l’alliance ni de l’Autriche ni de l’Italie, même en livrant Rome aux Italiens, et que jamais ces deux puissances n’ont été sérieusement disposées à soutenir la France. L’Autriche l’a montré précisément si elle a dit : « J’en serai si l’Italie en est », ce qui est un atermoiement et même une défaite.\par
Quant à l’Italie, qui ne voit que son intérêt en tout état de cause était d’attendre, pour se mettre ensuite du côté du vainqueur ? Qu’avait-elle à gagner à entrer dans une guerre où elle ne pouvait que  très peu secourir la France, où la France avait autant de chances d’être vaincue avec le concours de l’Italie que sans ce concours, et où la France vaincue, l’Italie se fût trouvée compromise du côté du victorieux ? L’Italie, possesseur de Rome, qu’a-t-elle à gagner à entrer en guerre contre la Prusse en notre faveur ? Je ne le vois pas. Je ne le vois d’aucun côté. Je dirai presque au contraire ; car l’Italie étant possesseur de Rome, c’est du côté du Trentin, de Trieste, etc., que son ambition tourne les yeux. Au point de vue de cette ambition, ce qu’il lui faut, c’est une Autriche affaiblie ; or la Prusse victorieuse fait une Autriche faible. Une fois détentrice de Rome, c’est plutôt de la Prusse que de la France que l’Italie doit désirer le succès. Cela pourrait se soutenir et n’est point du tout déraisonnable.\par
Contentons-nous de dire, sans vouloir trop prouver, que, possédant Rome ou ne le possédant pas, l’intérêt de l’Italie était certainement d’attendre et de voir venir les événements. Une guerre engagée par l’Italie en faveur de la France eût été une guerre purement sentimentale, ce dont l’Italie a peu l’habitude.\par
L’Autriche, elle, avait certainement un intérêt à nous soutenir pour empêcher que la Prusse n’absorbât tous les pays allemands et ne relevât à son  profit l’empire d’Allemagne ; mais il faut songer que l’Autriche est en partie pays allemand elle-même, et que faire marcher des Allemands contre des Allemands {\itshape en faveur des Français} est chose très difficile, sinon impossible. Les Allemands peuvent se battre les uns contre les autres ; c’est ce qu’ils ont fait à Sadowa ; mais mener des Allemands contre des Allemands {\itshape pour des Français}, ce serait presque une folie que de le vouloir entreprendre.\par
Il me paraît donc certain qu’il n’y a eu, de 1868 à 1870, {\itshape aucune espèce de bonne intention}, même conditionnelle, à notre égard, ni de la part de l’Autriche ni de la part de l’Italie, et qu’il n’y en aurait eu ni plus ni moins si nous avions abandonné Rome aux Italiens. Il y a eu des sollicitations de la part des Tuileries auprès des cours de Vienne et de Florence. Ces sollicitations ont été reçues avec politesse ; ces politesses, mêlées de considérations diverses et de prétextes allégués, ont pu être prises pour des promesses conditionnelles ; mais il me semble bien qu’il n’y a pas eu autre chose et que l’intention ferme de l’Autriche et de l’Italie a été, quoi qu’il arrivât du côté de Rome, de ne pas secourir la France.\par
Pour les raisons que j’ai dites plus haut, la {\itshape triplice}, la triple alliance prusso-italo-autrichienne est naturelle et, pour ces raisons, on peut dire {\itshape qu’elle  existait} en 1869-1870 à l’état latent. Elle ne dépendait pas du tout de la question romaine et ne s’y rattachait point. Si nous avons échoué en 1869-1870 à trouver un allié, c’est que nous nous sommes heurtés {\itshape déjà} contre la triple alliance. Abandonner Rome, ce que du reste je crois qu’on aurait dû faire beaucoup plus tôt, aurait été à cette époque, au moment où nous étions en posture de solliciteurs, un manque de dignité très inutile.\par
En tout cas, la question romaine n’a pesé d’aucun poids dans les événements de cette époque. Pie IX et le parti clérical responsables de Sedan est quelque chose que l’on doit considérer comme une légende anticléricale sans aucun caractère scientifique. Nous avons été vaincus en 1870 parce que notre organisation militaire était la plus faible, parce que nous n’étions pas prêts, ce que je crois que nous sommes destinés à n’être jamais, et parce que nous ne pouvions pas, quelque sacrifice que nous consentissions, avoir d’alliés : voilà le vrai.\par

\begin{itemize}[itemsep=0pt,]
\item  — Mais peut-on concevoir un malheur de la France qui ne vienne pas des Jésuites, le peut-on ?
\item  — Je reconnais que pour beaucoup de Français cette conception est absolument impossible.
\end{itemize}

 \section[{Chapitre IX. L’anticléricalisme sous la troisième République jusqu’en 1904.}]{Chapitre IX\\
L’anticléricalisme sous la troisième République jusqu’en 1904.}\renewcommand{\leftmark}{Chapitre IX\\
L’anticléricalisme sous la troisième République jusqu’en 1904.}

\noindent Les élections de 1871 furent une troisième victoire pour le cléricalisme. Après les désastres de 1870, une Assemblée, la plus remarquable du reste, comme réunion de talents et de lumières, que nous ayons jamais eue, avait été nommée, uniquement, dans la pensée des électeurs, pour faire la paix et pourvoir aux nécessités urgentes. Elle voulut, de plus, faire une constitution et des lois générales. C’était son droit absolu ; car il est évident qu’après une révolution la première Assemblée qui est nommée par la nation a tous les droits politiques et est constituante et législative par le fait même de sa nomination. Il n’y a personne qui ait qualité pour lui contester ces droits et pour limiter sa souveraineté. Mais peut-être l’Assemblée de 1871 eût-elle bien fait de se limiter elle-même et, la paix conclue, d’appeler la nation à de nouvelles élections  qui eussent été faites sur une nouvelle plateforme, sur des questions constitutionnelles et sur des questions politiques.\par
Elle ne le voulut point, à tort ou à raison, et elle fit des lois (et plus tard une constitution), comme si elle avait été nommée pour en faire et comme si, en la nommant, la nation avait eu en son esprit la considération de ces lois. Ce fut une faute, à mon avis, parce que, étant monarchique, quand la nation, comme le prouvèrent tout de suite les élections complémentaires de juillet 1871, tendait à devenir républicaine, toutes les lois de l’Assemblée de 1871 devaient plus tard prendre dans l’opinion un caractère monarchique et comme une mauvaise odeur monarchique et devenir suspectes à la nation.\par
La réaction anticléricale de la troisième république a pour un de ses fondements le souvenir de l’Assemblée monarchique de 1871 et le souvenir des tentatives monarchiques du 24 mai 1873 et du 16 mai 1877.\par
Quoi qu’il en soit, l’Assemblée de 1871 élabora une nouvelle loi de liberté de l’enseignement. La seconde République avait établi la liberté de l’enseignement primaire et la liberté de l’enseignement secondaire. Elle n’avait rien fait à l’égard de l’enseignement supérieur. L’Assemblée  de 1871 établit et organisa la liberté de l’enseignement supérieur. Le droit fut reconnu aux particuliers ou aux associations de donner l’enseignement supérieur à qui voudrait le recevoir d’eux. La liberté d’enseignement était enfin établie en France à tous les degrés (17 juin 1875).\par
Les effets, excellents à mon avis, de cette nouveauté, ne se firent pas attendre. L’enseignement supérieur tel qu’il existait à l’état de monopole, depuis le premier Empire jusqu’en 1875, était une des hontes de la France. Il végétait. Il était inactif et presque amorphe. Professeurs fatigués ou nonchalants, public rare et endormi, d’étudiants point. Sous l’aiguillon de la concurrence, l’enseignement supérieur français, qui a rencontré du reste des directeurs intelligents, zélés et hardis dans l’innovation, est devenu le plus actif, le plus laborieux et le plus illustre peut-être de l’Europe entière. Je ne puis pas m’empêcher de croire que la loi de 1875 soit une des causes au moins de cet heureux changement.\par
Mais l’anticléricalisme veillait, et vigilante aussi restait cette pensée de Louis XIV que l’on trouve dans toutes les pièces relatives à la révocation de l’Édit de Nantes et qui est le {\itshape Credo} même de tout le parti démocratique : « {\itshape Un seul troupeau sous un  seul pasteur}. » Le parti démocratique et anticlérical français ne peut pas concevoir la France autrement que comme un troupeau.\par
Aussi, dès 1880, cinq ans après la loi de liberté de l’enseignement, Jules Ferry lançait, très inopinément du reste, son projet de loi sur l’enseignement supérieur contenant le fameux article VII. L’article VII, c’était l’expulsion des Jésuites et des congrégations enseignantes non autorisées. L’article VII n’ayant pas été adopté par le Sénat, le gouvernement, en vertu d’anciennes lois remontant au \textsc{xviii}\textsuperscript{e} siècle, décréta contre les congrégations non autorisées et exécuta violemment les décrets, de juin à novembre 1880.\par
Rien de plus légal, puisqu’on peut toujours exécuter une loi qui n’a pas été abrogée ; mais, d’une part, c’était montrer quelque désinvolture à l’égard du Parlement que de substituer une loi désuète à une loi qu’on lui avait présentée et dont il n’avait pas voulu, pour arriver à un but qu’on tenait à atteindre et qu’il avait suffisamment indiqué qu’il ne voulait pas qu’on poursuivît ; et en ce sens c’était un véritable coup d’État ; — d’autre part, c’était une première atteinte à la liberté d’enseignement, le principe de la liberté d’enseignement étant que tout Français a le droit d’enseigner, s’il est honnête homme, si son enseignement n’est pas immoral,  si son enseignement n’est pas dirigé contre les lois.\par
La campagne Ferry fit, du reste, beaucoup plus de bruit qu’elle n’eut de grands effets. Jésuites et congréganistes ne furent expulsés ou inquiétés que pour la forme et restèrent, à très peu près, plus ou moins ouvertement, sur leurs positions. Les temps n’étaient pas venus. La France était peu convaincue de l’immensité du péril catholique. M. Jules Ferry lui-même n’avait mené cette campagne que pour se faire une popularité et eut dans la suite à s’occuper de beaucoup d’autres choses qui, du reste, la lui firent perdre.\par

\astertri

\noindent Jusqu’en 1900, il n’y a rien à relever de très important pour ce qui est de l’histoire de l’anticléricalisme en France.\par
En 1900 la campagne anticléricale recommença avec vigueur, et ce qui prouve, chose du reste qu’il est naïf de faire remarquer, que cette guerre n’est inspirée que par des intérêts électoraux, ce fut à propos d’ingérence du monde religieux dans les élections que les hostilités reprirent. Les moines assomptionnistes s’étaient mêlés des élections de 1898. Il était difficile de leur pardonner cela.  Ils furent poursuivis pour violation de l’article du Code pénal interdisant les réunions de plus de vingt personnes. — Comme sous l’Empire ? — Naturellement, comme sous l’Empire.\par
Mais ce que le procureur {\itshape impérial} leur reprocha, ce ne fut pas d’être plus de vingt, mais d’avoir favorisé des candidats, du reste républicains pour la plupart, mais d’un républicanisme qui n’agréait pas au gouvernement. Le procureur Bulot cita dans son réquisitoire les noms de trente et un députés qu’il affirmait qui avaient dû leur élection à l’influence des Pères de l’Assomption. Les députés crièrent et les Assomptionnistes furent condamnés et leur congrégation dissoute comme illicite.\par
Ce fut le premier coup de cloche.\par
L’année suivante, le gouvernement (ministère Waldeck-Rousseau) apporta un projet de loi sur les associations. Ce projet était très confus et n’a jamais été très complètement élucidé. Au fond, M. Waldeck-Rousseau voulait régulariser la situation des congrégations en France. Il voulait que les congrégations d’ores et déjà autorisées restassent dans la situation où elles étaient ; que les congrégations non autorisées montrassent leurs statuts, demandassent l’autorisation, fussent autorisées par une loi ou ne le fussent pas ; que celles qui seraient autorisées subsistassent et que celles  qui ne le seraient pas disparussent. Il voulait, semble-t-il, faire ce qu’on a spirituellement appelé un « concordat congréganiste ».\par
Mais ce qui était à prévoir, c’était que le parti antireligieux profitât précisément de cette occasion pour refuser toute autorisation aux congrégations, surtout aux congrégations enseignantes, et admît la proposition d’un « concordat congréganiste », précisément pour ne pas le faire, pour le rejeter, et du même coup pour supprimer par mesure légale toute liberté de congrégation.\par
Il le pouvait d’autant plus qu’il lui était loisible, pour ainsi agir, de s’appuyer sur le texte même du projet du gouvernement. Ce projet commençait par déclarer « illicites » toutes les congrégations religieuses possibles, sous prétexte : que les membres de ces associations vivent en commun, qu’ils font vœu de chasteté, de célibat et de pauvreté, et que l’article 1118 du Code civil déclare que « seules les choses qui sont dans le commerce peuvent faire l’objet d’une convention » et que la communauté, la chasteté, la pauvreté et le célibat ne sont pas choses qui sont dans le commerce.\par
Cette argumentation était comme fourmillante de sophismes. D’abord elle transportait dans le Code pénal une disposition du Code civil et elle faisait un crime de ce qui n’est qu’une incapacité judiciaire :  le contractant qui a fait un contrat sur une chose qui n’est pas dans le commerce ne peut exiger judiciairement l’exécution de ce contrat si son co-contractant s’y refuse, et voilà tout ce que signifie l’article 1118, et il n’y a aucune pénalité contre un homme qui a fait un contrat non conforme à l’article 1118 du Code civil.\par
A raisonner ainsi, le {\itshape mariage} serait illicite ; car il est une convention d’obéissance, de protection et de fidélité entre deux personnes, et ni l’obéissance, ni la protection, ni la fidélité ne sont choses qui sont dans le commerce, et donc le mariage est contraire à l’article 1118 du Code civil.\par
Mais est illicite aussi toute convention qui est contraire aux bonnes mœurs. Sans doute ; mais il est difficile de considérer comme contraire aux bonnes mœurs le fait de vivre en commun, le fait de s’engager à la pauvreté, au célibat et à la chasteté.\par
Et enfin il y avait dans cette position de la question une confusion volontaire entre la convention et le vœu. Un vœu n’est pas un contrat, c’est une résolution que l’on prend et dans laquelle on persiste. D’aucune façon, l’article 1118 du Code civil n’a rien à voir dans la question des associations et congrégations.\par
Voilà par où commençait le projet de loi ; mais  il continuait en affirmant que, — nonobstant le caractère illicite de toute association religieuse si énergiquement proclamé plus haut, — les associations religieuses pourraient exister et subsister si elles obtenaient l’autorisation de l’État par une loi. C’était dire : « Les associations religieuses sont {\itshape par elles-mêmes} illégales et criminelles. Mais vous pouvez autoriser à vivre ces associations qui sont criminelles et illégales par le seul fait d’exister. Elles sont, toutes, contraires aux lois ; mais vous pouvez permettre à certaines d’entre elles de vivre contrairement aux lois et en insurrection contre les lois. » C’était une nouveauté juridique un peu étrange.\par
C’était surtout, en même temps qu’on invitait le pouvoir législatif à autoriser certaines congrégations, lui fournir des arguments détestables à la vérité, mais lui fournir des arguments, pour les proscrire toutes ; c’était, en même temps qu’on établissait les bases d’un concordat congréganiste, inciter par des considérations générales le pouvoir législatif à repousser tout concordat et toute concorde et à frapper les congréganistes comme des criminels ; c’était, en même temps qu’on affichait l’intention de faire entrer régulièrement les congréganistes dans la société, crier du haut de sa tête qu’ils étaient des ennemis sociaux.\par
 Incohérence involontaire, ou incohérence volontaire ? Inadvertance ou perfidie ? M. Waldeck-Rousseau « le faisait-il exprès » ou le faisait-il par mégarde ? On ne sait ; car rien jamais ne fut plus lucide que la parole de M. Waldeck-Rousseau et ne fut plus obscur et plus difficile à comprendre que sa conduite.\par
Mais ce qui est clair, c’est que rien n’était plus facile que de repousser la seconde partie de son projet en s’appuyant sur la première, et de conclure logiquement de la première partie de son projet à une condamnation absolue de la seconde.\par
On n’y manqua point. Le parti antireligieux ne voulut voir en M. Waldeck-Rousseau que l’homme qui mettait tous les congréganistes hors la loi en les considérant comme y étant et le salua comme l’ennemi acharné de {\itshape tous} les congréganistes, ce qu’il n’était pas et ce que, bon gré mal gré, il se donnait comme étant.\par
Quant à la partie antireligieuse de la Chambre et du Sénat, elle saisit avec empressement l’occasion de recommencer l’œuvre traditionnelle, l’œuvre de la {\itshape déchristianisation} de la France. Elle limitait encore la portée de ses desseins et de ses espérances. Elle restait, en général, sur le terrain napoléonien. Elle prétendait n’en vouloir qu’à  l’Église latérale, qu’à l’Église irrégulière, c’est-à-dire à l’Église « régulière ». Elle restait concordataire. Tout le monde en ce temps-là se prétendait concordataire et s’appuyait sur le Concordat et s’enfermait dans le Concordat comme dans un fort pour tirer sur les « réguliers ». M. Waldeck-Rousseau, le premier, se réclamait du Concordat, il prenait la défense du clergé paroissial contre le clergé monastique, et de « l’église contre la chapelle » ; mais en même temps il montrait les congrégations comme l’armée de la contre-Révolution et conviait à voter son projet tous ceux qui voulaient assurer « la paix et le développement régulier de la Révolution française ».\par
Et c’était toujours la même {\itshape duplicité} volontaire ou inconsciente : n’attaquer que {\itshape les uns}, mais avec des arguments qui peuvent s’appliquer à {\itshape tous} et qui peuvent être invoqués pour proscrire tous. Dire que les congrégations sont l’armée de la contre-Révolution et que l’existence des congrégations est incompatible avec le développement pacifique et régulier de la Révolution, c’est placer la question en dehors de toute considération juridique et de toute considération de justice ; c’est dire : « frappez tous ceux qui vous paraîtront contre-révolutionnaires » ; en langage pratique, c’est dire : « frappez tous vos adversaires électoraux » ;  et des paroles de M. Waldeck-Rousseau on pouvait tirer une raison suffisante de proscrire toute l’Église séculière comme toute l’Église régulière, quelque attaché au Concordat que M. Waldeck-Rousseau prétendît être.\par
De même qu’en prétendant ne toucher qu’aux moines factieux, M. Waldeck-Rousseau instituait une théorie qui frappait tout moine quel qu’il fût comme étant hors la loi et contre la loi ; de même, en affectant de ne suspecter que les congréganistes, il donnait un argument aux termes duquel tous les ecclésiastiques et même tous les catholiques étaient déclarés suspects.\par
En attendant, il s’affirmait concordataire.\par
Concordataire aussi était M. Camille Pelletan, qui raisonnait ainsi : Nous sommes profondément attachés au Concordat. Or il n’y a rien dans le Concordat qui concerne les moines. Donc c’est être conformes au Concordat que de proscrire les moines, et ce serait violer le Concordat que de les tolérer. « La suppression des ordres religieux est la contre-partie des garanties accordées à l’Eglise. » Si, donc, on rétablissait les ordres religieux, le Concordat n’existerait plus. — D’où il faudrait conclure à l’heure où nous sommes que, le Concordat n’existant plus, on doit rétablir les ordres religieux. J’ignore si cette conclusion, qui s’impose en bonne  logique, est acceptée à l’heure où nous sommes par M. Camille Pelletan.\par
Au milieu de tous ces sophismes, je n’ai pas besoin de dire que le groupe libéral, surtout par l’éloquent organe de M. Ribot, posa les vrais principes : l’association religieuse est une association comme une autre, les hommes ayant le droit de vivre en commun, d’être pauvres, d’être célibataires et d’être chastes ; il ne faut que limiter l’accumulation des biens de mainmorte dans les associations religieuses comme dans les autres associations ; il faut surveiller les biens et laisser tranquilles les personnes.\par
La loi fut votée à peu près telle que M. Waldeck-Rousseau l’avait présentée. Toutes les congrégations non autorisées devaient demander l’autorisation d’exister, ou disparaître. Par disposition législative, les congrégations seraient autorisées à exister ou ne le seraient point.\par
Certaines congrégations (Jésuites, Bénédictins, etc.) ne jugèrent pas expédient de demander l’autorisation et se dispersèrent tout de suite. D’autres prirent leurs dispositions pour demander à être autorisées. On verra plus loin ce qu’il en advint.\par
Subsidiairement la loi sur les congrégations avait porté une grave atteinte à la liberté de l’enseignement.  Dans le projet primitif il n’y avait rien de spécial pour ou contre cette liberté ; il n’y avait que ceci, implicitement, à savoir qu’une congrégation enseignante, si elle n’était pas autorisée, n’enseignerait plus, naturellement, puisqu’elle aurait cessé d’être. Mais la gauche avait exigé davantage. Elle avait voulu qu’il fût interdit d’enseigner à toute personne appartenant à une congrégation même dissoute, c’est-à-dire à toute personne ayant appartenu à une congrégation, c’est-à-dire à tout congréganiste, même qui ne le serait plus. Autrement dit, cet article nouveau, très nouveau en effet, proclamait la perpétuité des vœux, alors que la loi ne reconnaît pas de vœux perpétuels. Il disait : « La loi ne reconnaît pas de vœux perpétuels, mais moi, non seulement je les reconnais, mais je les impose. Vous avez été congréganiste ; vous prétendez ne plus l’être, vous voulez ne plus l’être ; mais moi je veux que vous le soyez. Du reste, j’ai dissous votre congrégation ; mais quand il s’agit de vous, je la tiens pour existant encore, afin que vous en soyez, et je vous défends d’être professeur parce que vous en êtes. » Cela devenait absolument fou. Ce fut dénoncé comme tel, en même temps que comme tyrannique, et ce fut cinglé très vigoureusement par plusieurs orateurs  du centre, notamment par M. Aynard.\par
Mais les orateurs de gauche répondirent par l’argument irrésistible des « deux jeunesses », des « deux Frances », par cette affirmation de Danton et probablement de Lycurgue, que les enfants appartiennent à la République avant d’appartenir à leurs parents, et par cette raison sans réplique que « la tolérance n’est pas due aux intolérants ». D’où il suit que la tolérance ne doit exister nulle part sur la terre, les intolérants étant intolérants et les tolérants devant être intolérants pour réprimer ceux qui le sont ou qui le seraient s’ils pouvaient l’être. On a vu du reste plus haut que cette idée est de Jean-Jacques Rousseau lui-même.\par
L’article était absurde en soi ; mais comme il était despotique, il passa très facilement.\par

\astertri

\noindent De quelle façon cette loi sur les associations fut appliquée, c’est une chose que le cabinet Waldeck-Rousseau aurait pu prévoir et que peut-être il prévoyait ; car lorsque l’arbitraire est déjà dans la loi il se déchaîne dans l’application de la loi beaucoup plus violemment encore. Les élections de 1902 ayant renforcé quelque peu la partie antireligieuse de la Chambre des  députés, M. Waldeck-Rousseau, fatigué et peut-être voyant qu’après l’avoir mené déjà plus loin qu’il ne lui plaisait d’aller, on le mènerait encore beaucoup plus outre, se retira volontairement du ministère.\par
Mais ce qui laisse à soupçonner qu’il n’était point fâché que les iniquités qu’il ne voulait pas commettre, fussent commises pourvu qu’elles le fussent par un autre — et on trouvera toujours du « je ne sais quoi » dans la conduite de M. Waldeck-Rousseau, et l’on ne saura jamais s’il fut plus ingénu ou plus perfide, — c’est qu’il désigna pour lui succéder le plus borné et le plus violent des hommes politiques, M. Combes, à qui M. Loubet, qui ne pouvait pas le souffrir, mais qui a passé sa vie à faire ce qu’il désapprouvait et à blâmer dans ses discours la politique qu’il signait au bas de tous ses décrets, s’empressa de confier la présidence du conseil et la direction du gouvernement.\par

\astertri

\noindent Ce fut un gouvernement de combat. Les établissements congréganistes furent fermés par décrets, précipitamment, comme au hasard, sans respect des délais spécifiés par la loi, contrairement à des  jugements de la magistrature, qui, dans les commencements, fit quelque résistance au gouvernement et désira que la loi fût exécutée au moins comme elle était rédigée. — Des exécutions furent faites contre des établissements religieux, même ne comprenant qu’un seul membre d’une congrégation et encore d’une congrégation autorisée. Les tribunaux furent matés par l’intimidation et surtout par ce fait que tout jugement émané d’eux, non conforme au bon plaisir du gouvernement, était annulé par un « arrêté de conflit » qui leur enlevait le droit de s’occuper de l’affaire dont s’agissait.\par
Le gouvernement, sûr de la Chambre et encore plus, s’il était possible, du Sénat, allait droit de l’avant, ne gardant aucune mesure dans les actes, comme aucune forme, même hypocrite, dans les paroles.\par
Il y eut une insurrection, oratoire du moins, des républicains libéraux et même des républicains les plus authentiquement radicaux. Non seulement M. Charles Benoist, mais M. Gabriel Monod, M. René Goblet protestèrent. M. Gabriel Monod écrivit : « Ceux qui, comme moi, sont partisans d’une liberté absolue d’association et en même temps de la séparation de l’Église et de l’État, persuadés qu’alors c’est l’Église même qui imposerait des limites au développement indéfini des ordres religieux, sont  effrayés et navrés de voir les anticléricaux d’aujourd’hui manifester à l’égard de l’Église catholique des sentiments et des doctrines identiques à ceux que les catholiques ont manifestés naguère à l’égard des protestants et des hérétiques de tout ordre. On lit aujourd’hui dans certains journaux qu’il n’est pas possible de laisser l’Église continuer à élever la jeunesse française dans l’erreur ; j’ai même lu « qu’il n’était pas possible d’admettre la liberté de l’erreur ». Comme si la liberté de l’erreur n’était pas l’essence même de la liberté ! Et dire que ceux qui écrivent ces phrases protestent contre le \emph{Syllabus}, tout en le copiant ! Sommes-nous condamnés à être éternellement ballottés entre deux incohérences ? et le cri de « Vive la liberté ! » ne sera-t-il jamais que le cri des oppositions persécutées, au lieu d’être la devise des majorités triomphantes ? »\par
De son côté, M. Goblet écrivait : « … Je ne souhaite même pas la suppression complète des congrégations enseignantes, non seulement parce qu’il n’existe pas actuellement assez d’écoles et de maîtres laïques pour recueillir tous les enfants qui reçoivent l’instruction congréganiste ; mais parce que je suis un partisan déterminé de la liberté d’enseignement et que, tout en demandant que l’État ouvre aussi largement que possible ses établissements  à tous les enfants, je ne lui reconnais pas le droit d’empêcher les parents de faire donner, s’ils le préfèrent, l’instruction à leurs enfants dans des établissements privés, même tenus par des religieux… Comment donc est-ce que j’entends qu’on peut combattre le cléricalisme ? D’abord en faisant ce qui a toujours été un des articles essentiels du programme républicain : la séparation des Églises et de l’État ; en enlevant aux Églises la force qu’elles tirent de leur union avec l’État et les ressources qu’elles puisent dans le budget, et en laissant aux associations le soin de subvenir aux besoins des différents cultes… En second lieu, je voudrais qu’on laissât les congrégations libres de se former moyennant une simple déclaration ; mais en réservant le droit d’inspection de l’État et en limitant strictement leur faculté d’acquérir et de posséder… Et je persiste à penser que le régime de liberté, joint à l’exacte application des lois scolaires, servirait infiniment mieux la cause de la République et de la laïcité que le système de contrainte, je ne veux pas dire de persécution, irritant autant qu’inefficace, dans lequel je vois avec regret le parti républicain s’engager. »\par
Ces avertissements, comme il arrive quand on a affaire à de certaines gens, ne servirent qu’à pousser  plus vivement le parti despotiste dans le « système de contrainte » et dans la « manière forte ». Les fermetures de couvents se multiplièrent pendant toute l’année 1902 ; et M. Combes, dans ses discours-manifestes, étalait avec orgueil les chiffres de 22.000 ou 23.000 couvents mis sous scellés et annonçait le jour prochain où il n’y aurait plus un moine en France, ce qui ferait de la France la première nation du monde.\par
Cependant restait la question des demandes d’autorisation. D’après la loi de 1901, il fallait une disposition législative pour décider de l’autorisation à accorder ou à refuser à une congrégation qui l’aurait demandée. M. Combes, qui avait déjà montré par ses actes qu’il estimait la loi de 1901 beaucoup trop libérale, trouva que discuter chaque demande d’autorisation, alors qu’on voulait n’autoriser personne, était une grande perte de temps. Il fit donc décider par le Parlement qu’on n’examinerait point du tout les demandes d’autorisation, je veux dire qu’on ne les examinerait point séparément, ce qui était sans doute la seule manière de les examiner, mais qu’on répartirait en trois groupes les congrégations demandant à être autorisées : congrégations enseignantes, congrégations prédicantes, congrégations commerçantes ; puis que, sur chacun de ces groupes, pris en bloc, on prendrait  une détermination. C’était les trois charrettes, et à la façon dont on les aménageait et attelait, il n’était pas difficile de voir où l’on voulait qu’elles conduisissent.\par
Cette application de la loi de 1901 parut une violation de la loi de 1901 à l’auteur de la loi de 1901. M. Waldeck-Rousseau protesta contre la manière de sa créature, M. Combes. Il dit au Sénat : « L’application de la loi de 1901 soulève, à l’égard de toutes les congrégations en instance d’autorisation, une même question. Il faut considérer les garanties qu’elles présentent, leur utilité au point de vue matériel et moral. {\itshape C’est là un examen individuel dont aucune ne doit être dispensée et dont aucune ne peut être exclue.} La loi de 1901, étant une loi de procédure en même temps qu’une loi de principe, ce serait la méconnaître que d’opposer à une demande d’autorisation {\itshape une sorte de question préalable}. Ce serait la méconnaître aussi que d’admettre l’autorisation sans examiner, comme on n’a jamais manqué de le faire sous le régime antérieur à 1901, quel est le véritable caractère de la congrégation et si elle est en mesure de réaliser son objet… »\par
Mais M. Waldeck-Rousseau était depuis longtemps dépassé ; il n’avait plus qu’une faible autorité ; il avait ouvert les outres ; et si M. Combes  effrayait M. Waldeck — {\itshape patrem suus conterruit infans} — la majorité du parti républicain ne suivait plus que M. Combes, que méconnaissait l’œil de son père. Successivement le Sénat et la Chambre adoptèrent la « manière forte » de M. Combes.\par
A la Chambre ce fut surtout M. Ferdinand Buisson qui exposa en toute sa précision la théorie despotiste. Pour lui, au fond, la loi de 1901 n’existait pas. Ce qui était toujours en vigueur c’était la législation de 1792, qui avait supprimé toutes les congrégations, quelles qu’elles fussent. La loi de 1901 avait supposé, sans doute, que des congrégations pouvaient être autorisées ; mais elle n’obligeait nullement à en autoriser une seule. Et, de fait, il n’en fallait pas autoriser une seule, parce que toutes étaient « en dehors de la vie familiale et de la vie sociale ».\par
Et c’était un argument à faire frémir les célibataires, lesquels, vivant en dehors de la vie familiale, pouvaient s’attendre à être chassés du territoire français pour cause de conduite antisociale.\par
Pour ce qui est de la liberté du père de famille, M. Buisson répondait, en bon platonicien, que « l’enfant n’appartient pas aux parents ; mais à l’État, que l’État est son tuteur et doit le défendre comme il doit défendre tous les faibles. »\par
 M. Buisson était ainsi, en bon éclectique, familial dans la première partie de son argumentation et antifamilial dans la seconde. Dans la première partie il disait aux religieux : « Vous ne pouvez pas enseigner, parce que vous n’êtes pas pères ». Dans la seconde il disait aux pères de famille : « Vous ne pouvez pas choisir l’enseignement à donner à vos enfants, parce que vous êtes pères ». Dans la première partie de l’argumentation, n’être pas père ôtait un droit ; dans la seconde, être père n’en donnait aucun.\par
Logique au fond, malgré les contradictions formelles, était cette théorie ; puisque, selon les démocrates, personne n’a aucun droit, personne, excepté le gouvernement.\par
M. Combes reproduisit cette argumentation avec moins d’éclat, mais non moins de force. Il assura, chose peut-être étonnante au premier abord, qu’interdire l’enseignement à un congréganiste ou l’interdire à un homme n’ayant pas de grades universitaires, c’était absolument la même chose : « Du moment qu’on admet la légitimité des garanties, ne fût-ce que celle des grades, et des précautions, ne fût-ce que celle de la surveillance de l’État, {\itshape il n’y a pas de raison} pour que l’État, à certaines époques, ne puisse interdire l’enseignement à certaines catégories de personnes… Les motifs dont  l’État peut étayer cette interdiction sont {\itshape de même nature} que ceux en vertu desquels il interdit l’enseignement à ceux qui ne remplissent pas les conditions voulues de grades, de stage, de moralité… »\par
Il eût été plus simple de dire : « Nous n’admettons à enseigner que ceux qui ont des grades, une moralité reconnue et qui nous plaisent. » Et M. Combes ne disait pas autre chose ; seulement il s’efforçait d’établir une identité singulière entre le fait d’être bachelier et le fait de plaire au gouvernement, choses qui semblent bien n’être pas tout à fait « de même nature ».\par
Ces puissantes argumentations convainquirent parfaitement la majorité, qui refusa en bloc toutes les autorisations demandées. Les trois charrettes étaient arrivées à destination.\par

\astertri

\noindent Parallèlement le gouvernement atteignait encore la liberté d’enseignement en faisant supprimer par le Sénat la loi Falloux, la loi de 1850. A la vérité, de cette loi il ne restait que peu de chose, après un certain nombre de modifications de détail antérieures même à l’année 1900, et après la loi de 1901, et particulièrement après l’article de cette loi qui  interdisait l’enseignement à tout membre d’une congrégation même dissoute. Mais il importait, sans doute pour le principe, que la liberté d’enseignement fût attaquée directement et de face ; et que la charte de la liberté d’enseignement en France fût déchirée avec une certaine solennité ; et que les républicains de 1903 déclarassent très nettement qu’ils n’avaient rien de commun avec ceux de 1850.\par
Une proposition déposée par M. Béraud traînait au Sénat depuis 1901, sur le sujet de l’abrogation de l’article de la loi Falloux, subsistant encore, qui reconnaissait la liberté de l’enseignement secondaire. Ce projet Béraud exigeait le vote d’une loi spéciale pour autoriser à l’avenir l’ouverture de quelque établissement d’enseignement privé que ce fût. Le gouvernement ne fit pas sien ce projet. Il en apporta un autre, moins liberticide, quoique très restrictif encore de la liberté d’enseignement. Ce projet maintenait comme de droit général la liberté de fonder un établissement d’enseignement privé. Mais il exigeait un certificat d’aptitude qui laissait à qui devait le donner une singulière et inquiétante latitude d’arbitraire et la quasi liberté de le refuser par bon plaisir. Il organisait du reste l’inspection des établissements d’enseignement privé, et la réglementation des petits  séminaires, et la surveillance de l’enseignement secondaire, etc.\par
C’était ce projet, extrêmement défiant à l’égard de l’enseignement libre, qui révoltait la gauche, parce qu’au moins en principe il maintenait ou paraissait maintenir la liberté d’enseignement.\par
La commission sénatoriale fut pour le projet Béraud, et la droite, faute de mieux, pour le projet du gouvernement, dit projet Chaumié. M. Charles Dupuy fit un discours général, très chaleureux, en faveur de la liberté d’enseignement. Il n’y avait pas selon lui péril du côté de l’enseignement congréganiste, puisque la loi de 1901, avec « application » de 1903, venait de supprimer les congrégations. Donc ce que voulait la commission, c’était le régime antérieur à 1850, c’est-à-dire le monopole de l’État. Pourquoi ? Pour établir la fameuse « unité morale » du pays. Mais cette unité morale est une chimère qui inspirait autrefois la révocation de l’Édit de Nantes, qui a inspiré quelques autres sottises nationales et dont on devrait bien abandonner la poursuite.\par
M. Béraud répondit que Rome était l’unique objet de son ressentiment et qu’il s’agissait d’arracher la jeunesse française à l’étreinte jésuitique.\par
M. Eugène Lintilhac, avec citation d’Aristote, reproduisit l’argumentation de M. Ferdinand  Buisson, affirma que l’État avait tous les droits, que le libéralisme était une utopie dangereuse et que le père de famille n’avait pas le droit monstrueux de mettre son fils en travers de la route que suit l’humanité.\par
M. Thézard, rapporteur de la commission, mit une fois de plus en avant « l’unité morale », en affirmant qu’avant 1850 l’unité morale existait en France ; mais que depuis 1850 elle n’existait plus, ce qui était d’un généralisateur hardi et d’un artiste amoureux de la symétrie, mais d’un historien peut-être insuffisamment informé.\par
M. Clémenceau osa se moquer de M. Lintilhac, dire que la citation d’Aristote, rapportée du reste sous forme de rébus, lui avait paru être de Loyola. Il osa peut-être plus, c’est à savoir qu’il se permit de dire que les anticléricaux ne faisaient que « transférer la puissance spirituelle du pape à l’État ». Et ainsi, « pour éviter la Congrégation, nous faisons de la France une immense congrégation… Nos pères ont cru qu’ils faisaient la Révolution pour s’affranchir ; nullement : c’était pour changer de maîtres… Aujourd’hui, quand nous avons détrôné les rois et les papes, on veut que nous fassions l’État roi et pape. Je ne suis ni de cette politique ni de cette philosophie. » — {\itshape En conséquence}, et cette logique particulière n’étonnera aucun de  ceux qui sont familiers avec la mentalité anticléricale, M. Clémenceau concluait à interdire l’enseignement à tout congréganiste, tout congréganiste étant un « morceau de la société romaine et non de la société française ».\par
M. Waldeck-Rousseau vint une fois de plus, peut-être avec je ne sais quoi qui ressemblait un peu à du ridicule, gémir sur l’éclosion de l’oiseau qu’il avait couvé. L’intention (que venait de manifester le président du conseil) d’interdire l’enseignement à tout congréganiste était contraire au texte et à l’esprit de la loi de 1901. Celle-ci n’avait pas indiqué ni entendu que certaines congrégations fussent autorisées sauf déduction du droit d’enseigner. On s’appuyait sur la loi de 1901 pour en tirer des conséquences qui allaient contre elle.\par
Cruellement, M. Clémenceau répliqua que M. Waldeck-Rousseau avait tort de s’en prendre aux autres quand il n’avait à s’en prendre qu’à lui-même : qu’il avait ouvert la voie de telle sorte qu’il était difficile qu’on n’allât point jusqu’au bout du chemin, et au-delà des prévisions que M. Waldeck-Rousseau avait eu le tort de ne point avoir ; que, si M. Combes avait mal appliqué la loi de M. Waldeck, la faute en était à M. Waldeck qui avait confié à M. Combes le soin d’appliquer la loi de M. Waldeck, alors que rien ni personne n’avait empêché  M. Waldeck de l’appliquer lui-même. — A quoi M. Waldeck trouva certainement quelque chose à répondre, mais ne répondit rien.\par
Finalement le projet Chaumié, très transformé et très aggravé, contenant notamment un paragraphe exigeant que tout Français voulant ouvrir un établissement libre d’instruction doit produire la déclaration qu’il n’appartient pas à une congrégation, fut voté par le Sénat. La question de savoir si le droit d’enseigner serait accordé ou refusé aux membres du clergé {\itshape séculier} était réservée.\par
Du reste, {\itshape ce qui restait de liberté} dans la loi votée par le Sénat en première lecture en 1903 n’était garanti par rien, puisqu’il avait été stipulé par cette loi même que le gouvernement peut fermer toute école libre, {\itshape même contre l’avis du conseil supérieur}, si l’enseignement de cette école lui paraît contraire à la morale, à la constitution et aux lois. Or aucun homme sensé ne veut qu’une école puisse donner un enseignement contraire à la constitution, à la morale et aux lois ; {\itshape mais} tout homme sensé veut que ce soit un tribunal ou un arbitre quelconque, indépendant du gouvernement, et non le gouvernement lui-même, qui soit juge en cette question.\par
La loi revint en seconde lecture au Sénat en 1904. Le débat fut court. Il porta sur l’interdiction d’enseigner  faite à tout congréganiste, et cette interdiction fut maintenue. M. Chaumié, s’appropriant les idées philosophiques de M. Buisson, assura que les congréganistes n’étaient point mis hors la loi à cause de leurs sentiments religieux, ce qui serait contraire à la liberté de conscience, laquelle est sacrée ; ni à cause de leur incapacité, car il y en a de bien intelligents ; mais parce qu’ils sont dociles et qu’ils obéissent à leurs chefs et qu’ils ne sont pas « des êtres d’évolution ». Qu’ils cessent d’être congréganistes et alors ils seront des êtres d’évolution et pourront enseigner. — La capacité d’enseigner se mesurant à l’indocilité, on peut espérer que le gouvernement républicain donnera un avancement rapide à tout professeur de l’Université enseignant des choses désagréables au gouvernement.\par
Le débat porta encore sur l’article qui permettait au gouvernement de fermer les établissements d’enseignement libre, {\itshape même contre l’avis du conseil supérieur}, en cas d’enseignement contraire à la morale, à la constitution et aux lois. Cet article donnait pleine faculté d’arbitraire et de bon plaisir au gouvernement. Mais M. Thézard ayant déclaré, comme s’il l’avait su, que le gouvernement ne fermerait qu’{\itshape exceptionnellement} un établissement contrairement à l’avis du conseil supérieur, le Sénat fut pleinement rassuré et l’article fut maintenu.  Bref, la loi fut votée en seconde lecture telle qu’elle l’avait été en première.\par

\astertri

\noindent En 1904, nouveau {\itshape progrès}. M. Combes, au cours de la discussion devant le Sénat de la loi abrogatrice de la loi Falloux, s’était engagé à déposer un projet global et définitif supprimant décidément tout enseignement congréganiste de quelque ordre et de quelque nature qu’il fût, et si autorisée qu’eût été et que fût encore la congrégation à laquelle il appartenait. C’était le dernier tour de vis. Cette loi fut immédiatement déposée à la Chambre des députés, et l’ouverture de la discussion eut lieu fin février et la discussion se prolongea jusqu’au 28 mars.\par
Elle ne pouvait que rééditer toutes les argumentations qui s’étaient produites depuis quatre ans. M. Buisson ne manqua pas de dire qu’il était permis d’être moine isolé, mais non d’être moine associé, parce que, quand une association est religieuse, elle perd tout droit à la liberté d’association.\par
M. Combes ne manqua point d’affirmer que les moines ne sont pas des citoyens et ne peuvent revendiquer les droits de l’homme, ni celui d’enseigner ni celui de s’associer.\par
M. Ribot, mieux placé que ne l’avait été M. Waldeck-Rousseau  lui-même, pour montrer combien les lois de 1903 et 1904 étaient contraires à la loi de 1901, mesura les pas de géant que l’anticléricalisme avait faits depuis trois ou quatre ans et montra que ce qu’il avait reconstitué de toutes pièces, c’était le pur et simple arbitraire, et que rien de pareil n’avait été rêvé ni par M. Waldeck-Rousseau ni par ceux qui l’avaient suivi.\par
Élargissant la question, il s’en prit à la formule « sécularisation complète de l’État » et demanda si les États-Unis étaient un État sécularisé, eux qui accordent la liberté la plus complète à tous les citoyens sans leur demander quelle robe ils portent ni quels vœux ils ont faits. Il fit remarquer spirituellement qu’il y a en France, sinon une grande science théologique, du moins un sentiment théologique poussé à une singulière véhémence, à savoir la haine théologique, {\itshape invidia theologica}, et que « beaucoup de libres penseurs n’ont pas d’autre conception de la libre pensée que de prendre l’envers du cléricalisme qu’ils combattent avec tant d’énergie, si bien que, gardant toutes les habitudes d’esprit qu’ils reprochent à leurs adversaires, ils ne sont que des cléricaux à rebours ».\par
Il termina ainsi, en donnant comme le programme, non seulement du libéralisme français, mais de la  politique religieuse de tout État moderne et civilisé : « Je m’inquiète et je m’attriste de voir que dans notre pays il y ait cette tendance à revenir toujours vers le passé, à ne pouvoir sortir des ornières où nous nous sommes traînés, à ne pouvoir renouveler nos idées et nos conceptions de la liberté moderne. Nous retardons singulièrement sur beaucoup d’autres peuples, et je ne sais pas s’il y a deux parlements en Europe où des discussions comme celles auxquelles nous avons assisté puissent s’ouvrir. Ce qui m’inquiète et m’attriste aussi, c’est qu’à mesure que ces vieilles idées reparaissent et que se renouvellent ces vieilles pratiques si souvent condamnées par nos chefs, chaque jour déclinent ces grandes idées libérales qui sont l’essence même de cette République française {\itshape qui n’est rien si elle n’est pas la liberté organisée}. On commence à aimer dans ce pays l’usage de la force, même et surtout quand elle est accompagnée d’un peu de brutalité : on aime les coups de majorité. Permettez-moi de vous dire que cela, c’est l’affaiblissement, l’oubli du véritable esprit républicain et que, sous prétexte de défendre la République, on aboutit à abolir ce qui est notre honneur et ce qui est notre force : l’esprit de large tolérance, l’esprit d’équité, le respect de tous les droits… Quant à nous, libéraux, nous n’avons pas à dire ce que nous  sommes et ce que nous voulons. Nous avons toujours, et dès la première heure, défendu nos idées à cette tribune. Nous y restons fidèles. Nous savons, parce que nous avons étudié l’histoire, parce que nous avons jeté les yeux en dehors de chez nous, vers les pays qui marchent, qui progressent, — tandis que nous nous attardons dans les luttes, dans les querelles stériles, — nous savons où va le progrès humain et nous avons la conviction que nous sommes dans la bonne voie. Le présent peut nous réserver encore quelques tristesses et quelques déceptions ; mais l’avenir nous donnera raison. »\par
La loi fut votée. Elle décidait, en somme, que toute congrégation enseignante, qu’elle eût été autorisée ou qu’elle n’eût pas été autorisée, devait avoir disparu dans le délai de dix années.\par
Ce qui résume peut-être le mieux l’œuvre de cette dernière loi et de toutes les lois précédentes, c’est cette déclaration que M. Henri Maret lut avant le vote définitif de la loi : « Je ne voterai pas cette loi pour plusieurs raisons. La première, c’est que la loi est une loi contre la liberté. C’est une loi de combat, et toutes les lois de ce genre finissent toujours par se retourner contre leurs auteurs. Ensuite vous faites une loi un peu jésuitique. Vous faites une loi contre les personnes, puisque vous laissez  subsister l’enseignement ; vous ne l’interdisez qu’à une certaine catégorie de personnes. En troisième lieu, vous faites une loi inutile ; car l’enseignement congréganiste subsistera sans les congrégations. Enfin la loi que vous avez votée porte une telle atteinte à la liberté d’enseignement que cette liberté ne sera plus qu’un leurre, surtout pour les pauvres. »\par

\astertri

\noindent Mais déjà (commencement de 1904), une autre campagne anticléricale d’un tout autre genre avait commencé. Pendant quelques années, le parti anticlérical s’était tenu très ferme sur le terrain du Concordat, en tirant même des arguments du Concordat contre « l’Église latérale », l’Église régulière, l’Église congréganiste, ainsi que nous avons vu. A partir de 1904, il aiguilla vers la suppression du Concordat, vers la séparation de l’Église et de l’État. Il n’y eut du reste que contradiction apparente entre ces deux démarches, puisqu’au fond le dessein du parti anticlérical était de combattre et de détruire en France la religion chrétienne elle-même, et puisque, l’Église latérale détruite ou muselée, ce qui restait, c’était de foudroyer et pulvériser l’Église séculière elle-même.\par
C’est ici que les républicains despotistes se  séparaient de Napoléon I\textsuperscript{er}, celui-ci ayant tenu au maintien d’une Église asservie à son gouvernement et les républicains despotistes voulant la suppression de l’Église quelle qu’elle fût et, du reste, de la religion chrétienne elle-même.\par
C’est ce dessein que n’avait point caché M. Jaurès dans un discours qui à première vue avait pu paraître étrange, mais qui, intentionnellement ou non, livrait le secret, lequel, du reste, n’était un mystère pour personne. Dans la discussion sur la loi interdisant l’enseignement à tout congréganiste, M. Jaurès, pour conclure simplement contre la capacité éducatrice du moine, avait fait son procès au christianisme tout entier depuis ses origines jusqu’à nos jours : « Au moment où le christianisme est apparu à la surface du monde, au moment où l’idée divine a tenu, selon la religion catholique, à se manifester dans une personne humaine, à ce moment la face du monde a changé. Un double et contradictoire effet allait naître de cet événement historique ; il allait en résulter, à la fin, une concentration, un assainissement de la pensée humaine et l’exaltation même de cette pensée ; et ce contraste montre combien court et misérable doit rester, dans le prolongement des événements, un effort, si puissant soit-il, de la logique abstraite. Un Dieu était venu sauver les  hommes ; l’infini divin s’était mêlé un instant aux contingences humaines ; la loi qu’il avait dictée à quelques disciples allait être et devait rester l’intangible enseignement qu’aucune autre doctrine ne pourrait venir remplacer, qu’aucune autre leçon ne devait même atténuer ou expliquer. Le \emph{Syllabus} est en germe dans l’Évangile : il fallait que l’homme aveugle recueillît la clarté révélée un jour par le Dieu qui était venu la lui apporter et ne la perdît jamais de vue, dût-on la lui imposer de force, dût-on proscrire tout autre enseignement. Il fallait répandre cette vérité de cerveau à cerveau ; et pour cela se réclamer de la liberté, puisque seule elle permettait de la propager. Mais il fallait aussi refuser cette liberté aux autres, puisque le cerveau qui en était un jour tout éclairé [de cette vérité] ne devait plus jamais rechercher d’autre lumière. Voilà comment le christianisme devait asservir la pensée humaine ; et en même temps il allait l’exalter en lui donnant de son origine, puisqu’un Dieu était venu pour le sauver, en lui donnant de son but dans une autre existence, la plus haute idée. L’homme avait le droit de s’exalter, et c’est cette double loi, condensée dans le christianisme naissant, qui allait se développer au cours de l’histoire. C’est elle qui explique l’élan passionné du fidèle agenouillé dans la ferveur  mystique — et les abominables crimes de l’Inquisition. C’est elle qui fait comprendre toute une longue période de notre histoire qui resterait incompréhensible. C’est elle qui nous montre, dans les nuits éloignées et troubles du moyen âge, la douce lueur de l’étoile du matin vers laquelle on prie et les flammes sinistres des bûchers autour desquels on tue. Mais par l’accomplissement de cette double loi le christianisme a épuisé sa force. Il a ruiné le droit de la personne humaine, et c’est la personne humaine qui, affranchie aujourd’hui, veut d’autres enseignements. C’est ainsi que la communauté laïque a été conduite à intervenir pour inculquer à la jeunesse les principes de la raison. Ces principes, nous avons le devoir d’en faire une réalité, et c’est dans cet esprit que nous voterons la loi. »\par
Ce discours était d’une clarté douteuse et il semblait dépasser singulièrement son objet, si attaquer tout le christianisme, attaquer l’Évangile — et y trouver le \emph{Syllabus} — pour aboutir à écarter de l’enseignement les frères des Écoles chrétiennes, est d’une dialectique disproportionnée à son objet. Mais, au fond, le discours était pertinent, et s’il était nébuleux dans l’exposition, il était clair en son dessein. Il était même très loyal. Il voulait dire : « Point de surprise ! Soyez prévenus. Ceci  n’est qu’un épisode. Nous frappons aujourd’hui les congréganistes parce qu’à chaque jour suffit sa proscription ; mais c’est au christianisme tout entier que nous en voulons et même à toute religion, puisque, ce que nous voulons qui disparaisse, c’est l’exaltation religieuse, c’est le sentiment religieux. »\par
Ce jour-là, et il fut très bien compris, M. Jaurès dirigeait le parti anticlérical vers la séparation de l’Église et de l’État, {\itshape considérée elle-même comme un épisode de la guerre sans merci à la religion chrétienne}.\par
A la vérité, la séparation de l’Église et de l’État était un très ancien article du programme républicain. La Convention, comme on a vu plus haut, l’avait établie, par lassitude, il est vrai, de la Constitution civile du clergé ; elle avait été réclamée en 1830 et en 1848 par une fraction du parti républicain ; et, sous l’Empire, surtout par esprit d’opposition à l’occupation de Rome par les troupes françaises, elle était à son rang dans les manifestes du parti démocratique. Sous la troisième République, un grand nombre de républicains et quelques hommes aussi qui ne l’étaient point, tenaient la séparation pour souhaitable. Peu de propositions parlementaires, cependant, à ma connaissance, furent faites en ce sens. La première que je connaisse  et, en tout cas, la première qui ait eu quelque retentissement ou plutôt à qui l’on ait fait attention, est celle de M. Holtz, député de la Seine, en janvier 1901. Il déposa un projet de résolution portant qu’aussitôt après la promulgation de la loi sur les associations (loi Waldeck), la Chambre poursuivrait la séparation des Églises et de l’État. Mais le temps n’était pas venu. Le parti anticlérical était alors concordataire ou affectait de l’être, ayant appris quelque peu l’art de « sérier les questions ». Le parti radical resta insensible. M. Brisson, M. Trouillot s’abstinrent. M. Léon Bourgeois vota même contre la proposition, qui n’obtint que 146 voix contre 328.\par
Il est vrai qu’immédiatement après, M. Gauthier (de Clagny), voulant faire comme confirmer et renforcer ce vote et ayant proposé une résolution par laquelle l’Assemblée s’engageait à « maintenir le Concordat », sa proposition fut repoussée par 261 voix contre 246. Il n’y avait pas là précisément incohérence et il ne faudrait pas dire que la Chambre manifesta le même jour son désir de maintenir le Concordat et de le supprimer. La vérité est que l’Assemblée repoussait une manifestation de gauche et une manifestation de droite, ce qui était une manière de rester et d’indiquer qu’elle voulait rester dans le {\itshape statu  quo} ; et cette attitude était encore concordataire, sans l’être passionnément ; et tel était bien l’état d’âme de l’Assemblée de 1898-1902.\par
Celle de 1902 eut un esprit un peu différent. Non pas que les élections eussent marqué aucunement un esprit anticoncordataire. A peine une centaine de candidats — et qui ne furent pas tous élus — avaient mis la séparation de l’Église et de l’État dans leurs professions de foi. Mais l’assemblée populaire de 1902 était un peu plus radicale que celle de 1898 et, de plus, M. Waldeck-Rousseau, très concordataire et même très conservateur, avait choisi M. Combes pour la diriger.\par
A la vérité, M. Combes n’était pas anticoncordataire, lui non plus, en 1902, et il n’avait pas soufflé mot dans son discours-programme de la séparation de l’Église et de l’État ; mais il est de ces hommes qui vont devant eux beaucoup plus loin qu’ils ne veulent aller, par emportement et colère, comme M. Waldeck était de ceux qui vont beaucoup plus loin qu’ils ne veulent aller, parce qu’ils ne savent pas où ils veulent aller, sachant seulement où ils veulent parvenir.\par
Tant y a qu’une certaine agitation s’étant produite dans quelques églises de Paris à propos d’anciens congréganistes qui y prêchaient, les anticléricaux ayant d’abord frappé les fidèles, puis ayant  été vigoureusement corrigés à leur tour, et le gouvernement ayant supprimé le traitement des curés de ces églises, il y eut à la Chambre interpellation et proposition, faite par M. Massé et M. Hubbard, de séparation de l’Église et de l’État (mai 1903).\par
Les interpellateurs eux-mêmes, MM. Gayraud, républicain catholique, M. X. Reille, M. de Grandmaison, M. Grousseau (de même nuance) reconnaissaient que la conduite du gouvernement depuis un an menait droit à la séparation et que, sans en être partisans, ils s’y résignaient sans crainte. M. Combes parla. Pour la première fois, il laissa entrevoir la séparation comme pouvant entrer dans les prévisions des hommes politiques. Il se plaignit de la rébellion de l’Église ; il exprima cette idée que le Concordat n’avait pas donné à l’État des armes suffisantes contre l’Église et il envisagea la séparation comme possible en s’exprimant ainsi : « Devant le spectacle de cette rébellion, l’opinion publique s’interroge avec inquiétude et, {\itshape pour peu que ce spectacle se prolonge}, elle sera fatalement amenée à rejeter sur le Concordat la responsabilité d’un état de choses où les écarts de conduite sont encouragés par l’insuffisance même des moyens de répression. L’opinion sera ainsi amenée à considérer que le Concordat de 1801 a fait son temps, et à envisager une de ces deux solutions :  ou bien la séparation de l’Église et de l’État… »\par
On l’interrompit, on lui cria : « des Églises ! » Il en profita pour bien montrer que ce qu’il désirerait, le cas échéant, c’était bien la séparation d’avec l’État de l’Église catholique seule, de telle sorte que l’Église protestante et l’Église juive devinssent religions d’État et l’Église catholique religion privée ; car il reprit ainsi : « … de l’Église catholique et de l’État ; ou bien, si elle pense que cette séparation n’a pas été suffisamment préparée, à une révision sérieuse et efficace des règlements sur la police des cultes. »\par
La Chambre n’accepta point la proposition tendant à la séparation et vota un ordre du jour conciliateur de M. Étienne ; mais la question était posée et le gouvernement, pour la première fois, avait eu un sourire favorable pour la séparation.\par
S’il n’avait pas été plus loin, c’est que le parti républicain, {\itshape comme tous les autres}, du reste, était divisé sur cette question et d’ailleurs, {\itshape comme tous les autres}, l’est encore. Il y a des républicains antiséparatistes ; il y a des libéraux antiséparatistes et il y a des réactionnaires séparatistes. Les républicains séparatistes sont des autoritaires qui ne prennent la séparation que pour une mesure de spoliation à l’égard de l’Église (suppression du budget des cultes) et pour une mesure qui permettra  de traiter l’Église séculière comme on a traité l’Église régulière, en s’appuyant sur les mêmes arguments. Pour eux, la séparation est une préface de l’extermination.\par
Les républicains antiséparatistes (M. Thézard par exemple, dans un remarquable discours prononcé devant ses commettants en août 1903 et qu’il a repris en le développant devant le Sénat en novembre 1905) sont des autoritaires aussi, fidèles à la conception napoléonienne et qui se disent que la séparation, c’est, quoi qu’on fasse, la liberté et qu’avec la liberté on ne sait jamais ce qui peut arriver ; que la séparation, quelque complétée qu’elle puisse être par toutes les mesures possibles d’extermination, commence toujours par mettre l’Église hors des mains de l’État et que cela seul est déjà dangereux ; que mieux vaut garder un esclave enchaîné que l’affranchir en se promettant de l’assommer ensuite. Ces considérations n’ont jamais été sans faire hésiter un peu les plus séparatistes des républicains.\par
Les libéraux antiséparatistes (M. Ribot) sont des hommes qui, d’une part, n’aiment jamais les résolutions extrêmes et qui, d’autre part, savent très bien qu’une séparation ne pourrait être libérale que faite par eux, et que faite par des radicaux, elle n’est qu’une mesure de guerre ajoutée à d’autres et devant  être suivie par d’autres indéfiniment ; et que, par conséquent, il n’y a que des raisons théoriques et il n’y a aucune raison pratique à affranchir l’esclave pour qu’il soit assommé le lendemain.\par
Les libéraux séparatistes, dont je suis, n’ont pas la naïveté de croire qu’une séparation faite par des républicains despotistes puisse être à arrière-pensée libérale et sont parfaitement convaincus qu’elle est toujours à arrière-pensée d’écrasement. Seulement ils croient que, même organisée contre la religion et destinée à être complétée dans le même sens, la séparation vaut mieux que la domestication, parce qu’elle est la liberté, la liberté très menacée, mais encore la liberté. Ils disent : « {\itshape Malo periculosam libertatem.} » Ils croient que dans cette liberté même périlleuse, que dans cette liberté de combat, l’Église puisera des forces nouvelles et qui sont celles mêmes qui font qu’une Église est vivante. En un mot, ils croient qu’une Église est une force autonome, indépendante de l’État, ou qu’elle n’est rien. Ils croient encore que l’Église indépendante de l’État, même tourmentée, même persécutée, suscitera à l’État, sans même le vouloir, mais parce que l’État en France sera toujours autoritaire, de tels embarras, lui donnera de telles impatiences, que lui-même en reviendra,  comme en 1801, à désirer passionnément un Concordat, ce qu’ils ne souhaitent pas, mais ce qu’ils prévoient. En somme, ils raisonnent exactement, pour conclure à la séparation, comme les républicains antiséparatistes pour la repousser.\par
Et enfin les réactionnaires séparatistes raisonnent comme les libéraux séparatistes, pendant que les réactionnaires antiséparatistes, timorés ou prudents, traitent leurs amis séparatistes de « risquons-tout » et de « casse-cou ».\par
Voilà quel est encore, voilà quel était en mai 1903 l’état des partis dans cette question.\par
Il y avait donc hésitation un peu partout. Que les républicains en sortissent, cela dépendait du gouvernement ; car M. Combes inspirait une extraordinaire confiance à un parti peu intelligent, peu réfléchi et que la violence d’attitudes et de paroles séduit presque toujours. Or M. Combes, très peu séparatiste au début de son ministère, très peu séparatiste encore en mai 1903, comme on vient de le voir, inclina très rapidement vers la séparation, comme un homme que mènent les circonstances interprétées par un tempérament colérique. Il eut avec la cour de Rome quelques démêlés qu’un gouvernement conciliant ou simplement maître de soi, non seulement aurait résolus en quelques minutes, mais  même aurait considérés comme n’existant pas ; car en vérité ils n’étaient que de très légers dissentiments. Deux évêques, M. Geay, évêque de Laval, M. Le Nordez, évêque de Dijon, étaient agréables au gouvernement français et suspects, soit pour leur conduite privée, soit pour leur administration, à la Curie. De Rome on avertit M. Le Nordez de se démettre de ses fonctions. La lettre romaine fut communiquée par l’évêque au gouvernement français, qui protesta auprès du Saint-Siège, alléguant que, d’après le Concordat, les nominations des évêques devant être faites par le gouvernement français, sauf institution canonique réservée au Saint-Siège, il en devait être des révocations comme des nominations et que le Saint-Siège n’avait pas le droit de déposer un évêque français. Exactement la même procédure avait été suivie à l’égard de M. Geay et exactement les mêmes protestations, relativement à l’affaire de M. Geay, étaient faites par le gouvernement français. En même temps, ordre était donné par le gouvernement français et à M. Geay et à M. Le Nordez d’avoir à ne pas quitter leurs postes.\par
Le sous-secrétaire d’État romain répondit qu’autre chose était une déposition d’évêque et un avertissement donné à un évêque de se démettre  pour un temps de ses fonctions et de venir s’expliquer et se justifier devant la Curie romaine ; que de pareils avertissements étaient du droit du Saint-Siège, devant qui les évêques, canoniquement institués par lui, étaient toujours responsables. La question était discutable et évidemment, avec un peu de diplomatie et de temporisation, était susceptible d’arrangement.\par
Le gouvernement français fut cassant, brusqua les choses, rappela l’ambassadeur français et remit ses passeports au nonce. C’était la guerre déclarée.\par
Les deux évêques, qui avaient eu jusqu’alors à choisir entre l’obéissance à l’égard du gouvernement français et l’obéissance à l’égard du Saint-Siège, se décidèrent pour celle-ci, partirent furtivement pour Rome, se soumirent à la Curie et donnèrent leur démission d’évêques français.\par
M. Combes vit dans tout cela des motifs suffisants, non seulement pour rompre toutes relations diplomatiques avec le Saint-Siège, mais encore pour dénoncer le Concordat et pour séparer l’Église de l’État, en rejetant formellement — il l’a fait vingt fois — toute la responsabilité de ces graves mesures sur le gouvernement pontifical.\par
Aussi, dès le 4 septembre, discourant à Auxerre, il fit cette déclaration importante qu’il croyait  sincèrement « que le parti républicain, éclairé {\itshape enfin} pleinement par l’expérience des deux dernières années, accepterait sans répugnance la pensée du divorce entre l’Église et l’État ». La proposition de séparation de l’Église et de l’État était déposée devant le peuple.\par
Elle le fut deux mois après (10 novembre) sur le bureau de la Chambre des députés. Quelque opinion que l’on puisse avoir sur le fond de la question, il y a certainement à affirmer que la responsabilité de la séparation ne doit pas être rejetée sur Rome et que cette mesure a été engagée sur les plus futiles motifs, s’il ne faut pas dire sans motifs, et avec une précipitation qui sent la colère infantile.\par
Ce n’est pas que les avertissements fermes en même temps que respectueux eussent manqué à M. Combes et à M. Delcassé, ministre des affaires étrangères. A propos d’un premier rappel de notre ambassadeur accrédité auprès du pape, M. Ribot avait {\itshape approuvé} cette mesure en tant que provisoire, mais il avait montré les dangers d’une rupture qui fût définitive avec le Saint-Siège et, envisageant cette escarmouche comme le prélude de la séparation de l’Église et de l’État, il avait manifesté ses inquiétudes patriotiques : « Les gouvernements étrangers… peuvent chercher à  profiter d’une brouille un peu prolongée entre le Saint-Siège et le gouvernement français pour se faire accorder quelques avantages ou opérer quelque demi-réconciliation à nos dépens. Je ne saurais, quant à moi, leur en vouloir : ils font leur métier de gouvernements. C’est à nous à ne point nous prêter bénévolement à ce que ce malentendu, cette brouille, qui, je l’espère, sera aussi courte que possible, puisse donner à nos concurrents dans le monde des avantages qu’il vaut mieux garder pour nous… J’ai écouté avec beaucoup d’intérêt le discours de l’honorable M. Briand. Il m’a charmé par certains côtés, je le dis sans ironie. M. Briand fait en moment-ci un travail que je serai le premier à discuter dans un esprit très large, parce que notre collègue est en train de découvrir et de nous montrer les difficultés d’une question qu’on présente depuis trente ans sous une forme trop simplifiée — il l’a dit lui-même — aux électeurs, en risquant ainsi de les tromper. Il est en train de découvrir les difficultés du problème de la séparation de l’Église et de l’État. Nous avons peut-être sur M. Briand cet avantage que nous les avons découvertes avant lui… Quand le moment sera venu, je montrerai à quelles conditions on peut s’acheminer vers une indépendance plus grande de l’Église et de l’État, et nous discuterons à fond sur  tous ces points. La seule chose que je retienne et sur laquelle nous sommes d’accord maintenant ( ?), c’est que ce serait une folie et une folie criminelle de vouloir procéder à une rupture violente avec le Saint-Siège et de décréter la séparation sans avoir préparé les esprits par toutes les mesures nécessaires. Gambetta, quand on lui parlait de la séparation de l’Église et de l’État, avait coutume de dire : « Oh ! Ce serait la fin de tout. » Il me semble que vous commencez à comprendre que ce serait au moins la fin d’une foule de choses auxquelles je tiens et auxquelles nous devons tous tenir. Ce serait la fin de ce qui reste de paix religieuse dans ce pays, et c’est pourquoi vous êtes prudents, vous n’êtes que prudents, en demandant des délais, en ne voulant pas tirer des conséquences imprévues, en ne demandant pas la rupture définitive avec le Saint-Siège… » (27 mai 1904.)\par
Et sur la rupture, qui précisément devait être définitive, M. Ribot parlait ainsi cinq mois après. Il s’attachait surtout à démontrer que c’était un sophisme que de prétendre que le Saint-Siège était responsable de la rupture : « … Vous pouviez prendre un de ces moyens qui, tout en maintenant avec la dernière fermeté les droits de la France, ne compromettent pas tout, ne brisent pas tout en une heure. Mais, vous venez de le dire vous-même  à la tribune, vous avez donné vingt quatre heures au pape pour vous répondre. {\itshape Vous avez voulu la rupture.} J’ai le droit de le dire, la précipitation avec laquelle vous l’avez opérée ne laisse aucun doute sur le but que vous poursuiviez. Vous vouliez la séparation de l’Église et de l’État ; c’était votre nouvelle politique, qui avait pris naissance il y a quelques jours à peine, et il vous fallait comme préface et comme prétexte à cette séparation dont vous étiez désormais le partisan, il vous fallait une rupture publique, officielle avec Rome, et alors vous avez tout pressé, vous avez tout brisé, vous n’avez pas laissé le temps de vous répondre. Vous porterez devant le pays et devant l’histoire la responsabilité des conséquences de votre conduite. C’est une singulière préface à la séparation que cette rupture totale avec le Saint-Siège ! Elle la rend singulièrement dangereuse ! Elle contribue à lui donner ce caractère qui suffirait à lui seul à empêcher beaucoup de nos collègues de la voter… La séparation pourra se faire le jour où l’état des esprits le permettra ; elle se fera comme une mesure de pacification. Mais si elle est faite en pleine guerre contre l’Église, elle prend un tout autre caractère et elle doit faire reculer les plus hardis dans cette Chambre… Quand même nous serions d’accord sur  le papier, quand vous auriez fait le plan de cette cité future où l’État et l’Église seront complètement séparés, vous n’auriez pas résolu la plus grande difficulté, qui est de faire passer ce nouveau régime dans les mœurs d’un pays aussi vieux que le nôtre, de donner la liberté totale à un clergé qui a été tenu en tutelle jusqu’à ce jour {\itshape et de faire comprendre à vos amis et à tous les citoyens de ce pays, habitués à réclamer contre l’ingérence du clergé dans la politique, que tout est changé et que désormais il faut qu’ils s’y résignent} !… Fussions-nous en état, demain, de faire la séparation de l’Église et de l’État, ce serait encore une faute impardonnable d’avoir, à la veille de cette séparation, demandée par vous, rompu violemment toutes relations avec le Saint-Siège. »\par
Il est très certain que l’opération, très grave, je le reconnais, de la séparation de l’Église et de l’État, a été engagée précisément dans les conditions qui lui donnaient et qui lui laissent le caractère qu’elle ne devrait pas avoir.\par
La séparation a été une mesure de combat au cours d’une bataille. Elle a été un coup de canon répondant à une attitude de mécontentement, assez justifiée du reste. Elle a été une mesure de représailles et de vengeance.\par
Dans ces conditions, il est possible qu’il en sorte,  mais il est difficile et il est infiniment peu probable qu’il en sorte une exécution, une pratique dominée par des sentiments libéraux, dominée, même, par des sentiments de loyauté. Qui ne voit que, en raison, en bon sens froid, le Concordat est un régime régulier et pacifique et la séparation un autre régime pacifique et régulier ; que, par conséquent, la préface à une séparation doit être exactement la même qu’une préface à un concordat ; qu’une séparation doit être faite, pour être bien faite et sortir de bons effets, par des négociations longues, prudentes et réciproquement respectueuses et bienveillantes, entre le pouvoir spirituel et le pouvoir temporel ; qu’elle doit être faite après un examen bilatéral et une discussion bilatérale de toute la question de principe, de toute la question de forme et de toute la question de conséquences ; et que c’est précisément d’une façon toute contraire à celle que nous venons d’indiquer que la séparation a été engagée et qu’elle a été accomplie ?\par
Si le Concordat, régime du reste détestable à mon avis, je l’ai dit assez, a été cependant un régime viable, tout au moins, et sur lequel on a pu vivre régulièrement et se tenir à peu près en équilibre, c’est qu’il avait été débattu entre les deux parties contractantes, sinon avec beaucoup de  sagesse, sinon, même, avec une extrême bonne foi, du moins avec une certaine prudence, du moins avec cet avantage, qu’a toujours une tractation contradictoire, que tous les côtés et tous les aspects de la question avaient été envisagés.\par
Qu’on ne me dise point : « Eh ! Il faut délibérer ensemble et finalement s’entendre pour contracter ; mais c’est bien inutile pour décider qu’on n’est plus contractants. Il est besoin d’un contrat pour un mariage ; il n’en est pas besoin pour un divorce. Il est besoin de convenir pour savoir comment on vivra ensemble ; il n’est pas besoin de convenir pour savoir comment on vivra séparés et chacun de son côté, sans jamais se voir. »\par
Je répondrais : Pardon ! Dans un pays où est établie la séparation de l’Église et de l’État, l’Église et l’État sont indépendants l’un de l’autre, mais ils vivent ensemble ; ils ne sont séparés que de biens ; ils habitent la même maison. Il faut donc qu’il y ait des conventions entre eux. Ces conventions sont fondées sur la séparation, c’est-à-dire sur l’indépendance réciproque, au lieu de l’être sur le concordat, c’est-à-dire sur un partage des pouvoirs ; mais il faut encore qu’il y ait conventions. Il faut qu’il y ait conventions passées une fois pour toutes et auxquelles on se conforme, et il faut qu’il y ait tout le temps relations régulières et pacifiques  pour régler les points de détail et circonstanciels. C’est donc précisément quand on fait la séparation qu’il faudrait la faire après délibération bilatérale et après conventions prises de commun accord ; et c’est, aussi, précisément quand on a fait la séparation qu’il faut entretenir de constantes relations diplomatiques avec le chef de l’Église ; et ce sont {\itshape surtout} les pays de séparation de l’Église et de l’État qui ont besoin d’être, d’une façon ou d’une autre, d’ailleurs, en conversation et commerce continuels avec le souverain pontife.\par
C’est donc et dans les pires conditions du monde que la séparation a été faite en France et dans les pires conditions du monde qu’elle existe, et ce début et cette situation ne peuvent qu’être féconds en embarras inextricables et ne peuvent être féconds qu’en cela ; et il y aurait à soupçonner, si l’on inclinait à être méfiant, que c’est justement dans cet esprit, dans ces prévisions et dans ce dessein que la séparation a été faite.\par
En tout cas, elle est destinée à porter toujours ou très longtemps la marque, le poids et la peine de son origine.
 \section[{Chapitre X. La situation actuelle.}]{Chapitre X.\\
La situation actuelle.}\renewcommand{\leftmark}{Chapitre X.\\
La situation actuelle.}

\noindent La séparation est faite maintenant. Elle est faite par la loi de 1905, dite « loi Briand ».\par
Cette loi, imitée en partie de la loi de 1795, est, à mon avis, acceptable pour les libéraux et pour les catholiques. Elle est beaucoup plus libérale qu’on n’aurait pu l’attendre de la majorité qui l’a votée, ce qui fait honneur et à M. Briand lui-même, qui a dû la défendre contre les attaques de ses propres amis, et à M. Ribot, qui l’a discutée pied à pied et point par point avec un incomparable talent.\par
Elle est plus libérale, tout compte fait, que la loi de la Convention. Elle permet à une Église catholique de se former et de se développer en France en autorisant des « associations cultuelles », à organiser le culte, à recueillir des cotisations et des offrandes, à percevoir des fonds par quêtes et location de bancs et chaises, en un  mot, à posséder et à administrer, même en s’aidant les uns les autres, ce qui établit une cohésion et un organisme de l’Église catholique française.\par
La restriction par laquelle les associations cultuelles ne peuvent consacrer leur argent, exclusivement, qu’aux frais du culte, est une précaution assez juste prise contre les biens de mainmorte, est conforme aux idées traditionnelles de toute l’école libérale et semble directement inspirée du mot heureux de M. Ribot : « Surveiller les biens, laisser libres les personnes. »\par
D’autres restrictions qui ne me plaisent point du tout peuvent, à la rigueur, être considérées comme des mesures de transition qui seraient destinées à disparaître avec le temps, {\itshape si la loi était destinée elle-même à être appliquée dans un esprit libéral}. Par exemple (art. 35), si un discours prononcé ou un écrit affiché ou distribué publiquement dans les lieux où s’exerce le culte « {\itshape tend} à soulever ou armer une partie des citoyens contre les autres », le ministre du culte qui s’en sera rendu coupable sera puni d’un emprisonnement de trois mois à deux ans. Débarrassé de la phraséologie législative, ce texte veut dire que tout propos politique tenu dans l’église par un prêtre sera puni de l’emprisonnement.\par
 Or on peut, en raison pure et en logique pure, raisonner ainsi : « C’est monstrueux ! S’il est naturel et même obligatoire d’interdire au prêtre rémunéré par l’État et, par conséquent, sinon fonctionnaire, du moins subordonné au gouvernement, tout propos politique, comment peut-on interdire toute parole de politique à un homme qui est chez lui, absolument chez lui, avec ses amis et qui n’est rémunéré, s’il l’est, que par ses amis, lesquels, autour de lui, sont également chez eux, absolument chez eux ? »\par
Rien de plus juste ; mais on peut répondre : « Pendant un certain temps, le prêtre devenu prêtre libre conservera, même malgré lui, aux yeux des populations, quelque chose du caractère du prêtre officiel ; et, pendant ce temps, à ce caractère pseudo-officiel les paroles qu’il peut prononcer devant emprunter une autorité particulière, il faut, pendant ce temps et pour ce temps seulement, interdire au prêtre ce qui est permis à un simple citoyen dans une réunion publique. »\par
Comme mesure de transition cette entrave peut donc, à la rigueur, être acceptée.\par
Cette loi est donc relativement libérale. Elle met l’Église catholique à peu près, je ne dis pas à très peu près, dans la situation de l’Église  catholique américaine. Je crois qu’il y a dans cette loi pour l’Église catholique non seulement faculté d’exister, mais {\itshape principe} ou {\itshape occasion} d’une véritable rénovation et d’un magnifique rajeunissement. C’est ce que, selon le parti dont on est, il est possible, d’après cette loi, ou d’espérer ou de craindre.\par
Au fond, il me semble bien que cette loi n’a véritablement inquiété que ceux qui l’ont faite. Les républicains despotistes y ont été amenés au gré de circonstances interprétées tout de travers par des frénétiques, et, maintenant qu’ils l’ont faite, ils ne sont pas tout à fait sûrs qu’elle ne soit pas contre eux. — Et de même ceux qui l’ont repoussée ne sont pas tout à fait sûrs qu’elle ne leur soit pas favorable et se demandent un peu s’ils n’auraient pas dû la proposer. En somme, on ne l’a guère repoussée à droite que parce qu’elle venait de la gauche, et on l’aurait certainement repoussée à gauche si elle était venue du côté droit. La raison des résistances qui se sont produites a été le {\itshape Timeo Danaos et dona ferentes}, et l’on n’a suspecté le don qu’à cause des donateurs.\par
A ne regarder que le don seulement, comme un homme qui s’obstine à n’appartenir à aucun parti, je ne puis pas dire que je ne sois point relativement satisfait. En somme, c’est bien là une  séparation de l’Église et de l’État rationnelle, sinon généreuse, et pacifique, en soi, sinon bienveillante. Or la séparation est pour moi la vérité, et je crois que, comme j’en ai toujours été, je serai toujours de l’avis de Lamartine en 1831 et en 1843. En 1831, il écrivait, en son \emph{Mémoire sur la politique rationnelle} : « La séparation de l’Église et de l’État est l’heureuse et incontestable nécessité d’une époque où le pouvoir appartient à tous et non à quelques-uns : incontestable, car sous un gouvernement universel et libre, un culte ne peut être exclusif et privilégié ; heureuse, car la religion n’a de force et de vertu que dans la conscience. Si l’État s’interpose, la religion devient pour l’homme {\itshape quelque chose de palpable et de matériel}, qu’on lui jette ou qu’on lui retire au caprice de toutes les tyrannies. Elle participe [est-ce assez vrai à considérer l’état de l’opinion de 1815 à 1830 ?] de l’amour ou de la haine que le pouvoir humain inspire ; c’est le feu sacré de l’autel alimenté avec les corruptions des cours et les immondices des places publiques. »\par
Et il écrivait en 1843, avec cette horreur du régime napoléonien qu’il eut toujours et qui, s’il n’est pas le commencement de toute sagesse politique, est le commencement de tout libéralisme et de toute conception généreuse de la chose politique,  il écrivait dans sa brochure \emph{l’État, l’Église et l’Enseignement} : « Napoléon, ce grand destructeur de toutes les œuvres de la philosophie, s’est hâté de renverser cette liberté, {\itshape fondement même de toutes les autres}. Il a fondé de nouveau l’Église dans l’État, l’État dans l’Église ; il a fait subir un sacre au pouvoir civil ; il a fait un Concordat : il a déclaré une Église nationale et par là même [ou plutôt parallèlement] un enseignement aussi… Il a vendu à faux poids son peuple à l’Église et l’Église ensuite à son peuple. Cet acte a reculé d’un siècle peut-être le règne de la liberté des âmes qui approchait. »\par
Il me semble que ces fortes paroles du grand homme, que personne ne peut incriminer ou soupçonner d’hostilité, ni même de malveillance à l’égard de la religion catholique, sont à méditer aujourd’hui et portent à considérer comme un progrès toute mesure qui, même maladroitement, même parcimonieusement, délivre l’Église des chaînes lourdes et peu dorées de l’État.\par

\astertri

\noindent Quoi qu’il en puisse être, la situation actuelle est celle-ci.\par
Au point de vue des congrégations religieuses,  despotisme absolu, proscription absolue. C’est à peine si quelques ordres hospitaliers sont tolérés encore.\par
Au point de vue de la liberté de l’enseignement, interdiction absolue à tout congréganiste d’enseigner quoi que ce soit. Les prêtres séculiers peuvent enseigner encore.\par
Au point de vue du culte, de la prédication et de l’administration des sacrements, Église libre, relativement, assez largement, non payée par l’État, vivant des ressources qu’elle se créera par le mécanisme des associations cultuelles.\par
Dans cette situation y a-t-il solution trouvée et acquise ? En d’autres termes, la question du cléricalisme et de l’anticléricalisme est-elle fermée ou reste-t-elle ouverte ?\par
Elle reste ouverte pleinement. Car ce qui reste à reconquérir pour les catholiques, à les considérer comme libéraux, le voici :\par
C’est la liberté d’association pour les religieux, lesquels ont parfaitement le droit de vivre en commun et de posséder en commun, sous réserve de précautions à prendre contre l’accroissement des biens de mainmorte.\par
C’est la liberté d’enseignement pour les religieux, lesquels ont parfaitement le droit d’enseigner, ou plutôt c’est la liberté pour les pères de  famille de faire enseigner leurs enfants par qui ils veulent, pourvu que l’enseignement ne soit ni immoral ni dirigé contre la patrie ou contre les lois.\par
C’est enfin, quoique beaucoup moins important, la liberté pour le prêtre séculier, qui désormais est un prêtre libre, de dire, dans la chaire libre où il parlera désormais, tout ce qu’il voudra, sauf des choses immorales ou des choses contre la patrie ; de dire même des choses contre les lois, car il est permis de discuter les lois entre hommes libres dans le dessein de faire amender celles qu’on trouve mauvaises, pourvu qu’on n’insulte pas le législateur.\par
Et, d’autre part, ce que les républicains à la fois autoritaires et anticléricaux ont à conquérir : c’est la dispersion, proscription et destruction des dernières congrégations religieuses encore épargnées, si rares soient-elles.\par
C’est l’interdiction d’enseigner à tout prêtre séculier ; car exactement les mêmes raisons existent d’empêcher un prêtre, {\itshape chaste, pauvre et docile}, d’enseigner quoi que ce soit, que d’empêcher un moine, {\itshape chaste, pauvre et docile}, d’enseigner quoi que ce puisse être.\par
C’est enfin la dispersion, destruction et suppression de l’Église libre elle-même, quand on s’apercevra,  ce dont on ne pourra pas manquer de s’apercevoir, qu’elle est un élément de liberté ; qu’elle ne dit pas exactement et littéralement ce que le gouvernement pense et veut qu’on pense et que, par conséquent, elle rompt « l’unité morale » de la France et est contraire à la formule de Louis XIV, reprise par M. Combes : « Une loi, une foi ».\par
Or pourquoi les catholiques, à les considérer comme libéraux, ce qu’ils sont pour le moment obligés d’être, et à les tenir pour actifs et énergiques, ce qu’ils sont, renonceraient-ils à ce qu’ils ont à reconquérir ; et pourquoi les républicains despotistes et anticatholiques renonceraient-ils aux conquêtes si vastes qu’ils ont encore à faire ? Il n’y a aucune raison ni pour ceci ni pour cela.\par
Je laisse de côté cette partie du sujet : les choses que les catholiques ou simplement les libéraux ont à reconquérir ; cela n’a pas besoin de démonstration. Je m’attache à cette partie du sujet : ce que les républicains despotistes ont à conquérir encore et les raisons pourquoi ils ne voudront ni ne pourront renoncer à la poursuite de cette conquête.\par
Qu’ils aient à conquérir encore, on vient de le voir suffisamment ; qu’ils veuillent conquérir  encore, ils l’ont dit d’une façon générale dans la déclaration de l’extrême gauche avant le vote de la loi de 1905 à la Chambre des députés : « La loi n’est que {\itshape provisoire} : elle marque seulement une étape nécessaire dans la marche de la laïcité intégrale » ; et ils l’ont dit en détail mille fois, ainsi qu’on verra plus loin. — Et qu’ils soient désormais comme obligés de conquérir encore, c’est de quoi l’on verra toutes les raisons, ou du moins les plus évidentes, dans les pages qui vont suivre.\par
Il est très évident et il n’est pas besoin de déduire longuement que les républicains despotistes voudront détruire et voudront ensuite éternellement empêcher de renaître ce qui reste encore des congrégations, puisque pour eux, ils l’ont assez dit, le congréganiste est un être, non seulement antisocial, mais antihumain. L’homme, ou la femme, docile, pauvre et chaste, est pour le républicain despotiste, pour le démocrate, un être qui ne doit plus enseigner. Et pourquoi ne doit-il pas enseigner ? Parce qu’il est un être contre nature. C’est leur argument mille fois répété. Il est clair que l’argument va plus loin qu’à interdire l’enseignement au congréganiste ; il va jusqu’à interdire au congréganiste d’exister. Si le congrégatisme fait du congréganiste un être  contre nature, le congrégatisme est un délit, est un crime. Si permettre à un tel être d’enseigner est monstrueux, lui permettre d’être est au moins illicite, est au moins indigne.\par
Songez qu’il y a même identité entre le crime d’enseigner quand on est dans certaines conditions et le crime d’exister quand on est dans ces mêmes conditions. Car on enseigne par l’enseignement, mais tout autant (peut-être plus) par l’exemple. Être ou enseigner, au fond, c’est donc la même chose. Celui qui donne, dans la société moderne, l’exemple de la docilité, de la pauvreté et de la chasteté est donc et aussi dangereux et aussi coupable que s’il donnait l’éducation proprement dite. Au fond, le crime, d’après l’argumentation des anticléricaux, ce n’est pas d’enseigner, étant moine ; c’est d’être moine.\par
Et les anticléricaux n’ont pas tort, à se placer à leur point de vue ; car le moine, à être docile, à être pauvre et à être chaste, donne l’exemple de vertus — il les appelle ainsi — qui sont contraires au développement et à l’affermissement de la société démocratique. Une société démocratique, telle que l’entendent les démocrates modernes, est une société où toute force individuelle est supprimée. Rien ne donne force individuelle comme le désintéressement, et  celui-là est une personnalité très forte qui n’est l’esclave ni du désir de posséder, ni du désir de commander personnellement, ni du désir de jouir. Un tel homme se soustrait en quelque sorte à l’État par le peu de besoin ou par le nul besoin qu’il a de l’État. Homme dangereux par ceci qu’il est indépendant. Vertueux peut-être, mais de vertus qui, étant antisociales, peuvent et doivent être appelées des vices sociaux.\par
Un tel homme n’a pas de place en régime despotique et, par conséquent, il doit être proscrit de la République telle que la comprennent la plupart des républicains. Ils y mettent moins de dialectique ; mais l’instinct a sa logique qui, pour être confuse, n’en est pas moins sûre. Le congréganiste est, de par un instinct qui ne se trompe aucunement, un être monstrueux et redoutable pour le démocrate. Il doit, non seulement ne plus enseigner, mais cesser d’être. Il n’y a aucun doute que les républicains despotistes ne poursuivent et ne consomment la destruction totale de toute espèce de congrégation, quelle qu’elle puisse être et si anodine en apparence qu’elle soit.\par
Je suis même persuadé que si la démocratie se développe dans le sens où elle se dirige, elle inventera un ostracisme, contre le citoyen, même isolé,  qui par son indépendance, son obéissance à une loi {\itshape personnelle}, sa pauvreté, sa sobriété, sa chasteté, sera un homme qui ne présentera plus de prises, pour ainsi dire, à l’État et sera une espèce de protestation contre la servilité générale que l’État sera parvenu à établir.\par
Croit-on que la démocratie déteste et redoute l’homme riche parce qu’il est riche ? Point du tout. Elle le déteste et elle le redoute, parce qu’il est fort ; et parce qu’étant fort, il n’a aucun besoin de se soumettre à elle et lui échappe. Or le désintéressement, l’abnégation, c’est une force ; c’est même la plus grande force qui soit : la démocratie ne peut donc pas ou ne pourra donc pas souffrir le désintéressement et l’abnégation.\par
Les républicains despotistes poursuivront de même la destruction complète de la liberté d’enseignement et, progressivement, ils interdiront d’enseigner 1º aux prêtres séculiers, 2º aux laïques non enregimentés dans l’Université, 3º aux universitaires qui montreront quelque indépendance, soit dans leur enseignement, soit en dehors de leur enseignement.\par
La raison en est simple. C’est qu’à chaque destruction partielle et limitée de la liberté d’enseignement, ils s’apercevront qu’ils n’ont rien fait  qui soit adéquat, ni même qui soit accommodé, à leur dessein et qu’il leur reste tout à faire [{\itshape nil actum reputans si quid superesset agendum}], et ainsi ils seront comme acculés au monopole, eussent-ils commencé par n’en pas vouloir, et au monopole exercé strictement selon leurs idées.\par
On commence par ne pas vouloir du jésuite et on le chasse. Qu’y gagne-t-on ? Le jésuite est remplacé par l’oratorien, par le mariste ou par tel autre éducateur qui donne le même enseignement, à peu de chose près, on en conviendra, que le jésuite.\par
On chasse le mariste et l’oratorien. Il est remplacé par le prêtre séculier qui donne le même enseignement que le mariste ou l’oratorien.\par
On interdit l’enseignement au prêtre séculier. Il est remplacé par le laïque catholique qui a exactement les mêmes idées que le prêtre séculier.\par
On chasse le laïque catholique et l’on arrive au monopole, auquel on a été comme adossé.\par
C’est à la liberté qu’on ne fait pas sa part, surtout en choses intellectuelles. Elle reste entière tant qu’elle n’est pas nulle. Cela est tellement senti par tous les républicains despotistes que tous ou le proclament ou en conviennent, soit par leur silence, soit par leur embarras à en sortir.  M. Ferdinand Buisson a toujours été ou a toujours cru être l’adversaire et l’adversaire indigné du monopole. Il a dit un jour : « Le rétablissement du monopole universitaire, c’est un aveu d’impuissance dans la lutte contre les congrégations. C’est le recours à un biais pour éviter la lutte directe. C’est une diversion pour masquer une défaite. C’est l’abandon de la politique anticléricale que l’on remplacerait par la politique antilibérale… Je suis prêt à parler du monopole, à en examiner l’utilité, l’opportunité, l’efficacité ; mais après la suppression réelle des congrégations enseignantes, le jour où la République ayant laïcisé pour tout de bon écoles publiques et écoles privées, ayant dissous les congrégations, dispersé leurs membres et fermé leurs noviciats, il sera démontré que nous nous retrouvons en présence, comme on l’assure, du même péril clérical. »\par
On lui a répondu : mais, sans aucun doute, vous vous retrouverez en présence du même péril clérical. « M. Buisson abolit toutes les congrégations enseignantes. Il les abolit parce que congrégations. Entendu ! Abolit-il aussi pour les prêtres le droit d’enseigner ? Abolit-il pour les moines le droit de se séculariser et d’endosser la soutane ? S’il n’abolit pas tous ces droits, ce sera comme s’il  n’avait rien fait. M. Buisson détruirait-il l’enseignement clérical parce qu’il détruirait l’enseignement congréganiste ? Il me semble que nous ne serions pas mieux lotis quand le même enseignement serait distribué par des ecclésiastiques en robe séculière ou par d’anciens moines qui auraient fait simplement l’emplette d’un veston. L’enseignement congréganiste n’est qu’une forme de l’enseignement clérical, la forme la plus visible ; mais il n’est pas l’enseignement clérical lui même. » (\emph{Dépêche de Toulouse}.) — Le raisonnement ne me paraît pas très facilement réfutable.\par
De même, M. Ferdinand Buisson se félicitait de ce que la Chambre avait rejeté un amendement interdisant l’enseignement {\itshape aussi bien} aux prêtres séculiers qu’aux congréganistes ; et, pour démontrer qu’il avait raison de se réjouir, il employait des arguments qui précisément, en s’appliquant aux congréganistes, s’appliquaient tout aussi bien aux séculiers. Il disait : « Nous voulons seulement enlever aux maîtres catholiques le privilège de se grouper dans des conditions exceptionnelles qui les transforment en une masse militairement constituée. »\par

\begin{itemize}[itemsep=0pt,]
\item  — « Fort bien, lui répondait-on (Sigismond Lacroix, {\itshape Radical}) ; mais qui ne voit que les prêtres séculiers sont, aussi eux, groupés dans des conditions  exceptionnelles, qu’ils forment eux aussi une masse militairement constituée ? »
\end{itemize}

\noindent Il disait : « Il faut et il suffit que la société laïque retire à l’enseignement clérical la seule chose qui en fasse une force abusive, à savoir le droit d’enrégimentation, grâce auquel les maîtres catholiques forment une masse homogène pesant de son poids mort sur la société laïque. »\par
On lui répondait : « Mais le clergé séculier est enrégimenté comme l’autre. »\par
Il disait… la même chose en d’autres termes.\par
On lui répondait : « S’il est juste et nécessaire d’interdire l’enseignement aux membres des congrégations, il est non moins juste et nécessaire de prendre la même précaution à l’égard des prêtres séculiers » ; car l’Église est une congrégation, ou qu’est-ce qu’elle est donc ?\par
Et l’on concluait en disant très justement que M. Buisson avait donné des arguments « pour l’avenir », c’est-à-dire contre les prêtres séculiers, et des arguments démontrant très exactement le contraire de la thèse soutenue par lui.\par
M. Buisson disait : Ce n’est pas à cause de leur enseignement que je proscris les moines ; car je suis partisan de la liberté d’enseignement et je reconnais à quiconque « le droit d’enseigner ce qu’il voudra comme il voudra ». Si je proscris les  moines, 1º c’est « parce qu’une société monastique donne à ses membres un idéal trop différent d’une société démocratique » ; 2º c’est parce que les moines « exercent sur l’enfant une pression qui est de nature à compromettre le développement normal de son esprit ».\par
On lui répondait : Mais c’est une partie de leur enseignement que leur respect et leur culte pour l’idéal auquel ils sont soumis et qui est trop différent de la société démocratique ; et donc, quand même ils ne le voudraient pas, ils enseignent la contre-démocratie. Et c’est bien aussi (ou de quoi parlons-nous ?) une partie de leur enseignement que la pression qu’ils exercent sur l’enfant de manière à compromettre le développement normal de son esprit. C’est probablement leur méthode même. C’est donc bien {\itshape en raison de leur enseignement} que vous leur interdisez d’enseigner. « Eh bien ! si c’est pour la qualité de leur enseignement que M. Buisson frappe les congrégations enseignantes, pourquoi tolère-t-il le même enseignement de la part de l’ancien congréganiste, de la part du prêtre séculier, de la part du tiers ordre ? Ce qui est détestable de la part d’une congrégation est-il moins détestable de la part des individus ? En supportant chez les individus ce qui est condamné chez les congrégations, ce sera absolument comme si rien n’était fait contre  le danger de ces dernières. Si vous n’excluez de l’enseignement que les communautés religieuses et si vous n’en excluez pas {\itshape tous les éducateurs qui sont à leur image}, qu’aurez-vous fait pour barrer la route à l’enseignement clérical ?… L’abolition de l’enseignement congréganiste n’aura servi de rien si cet enseignement doit se survivre {\itshape sous une forme quelconque, soit séculière, soit laïque}. » Pourquoi il faut combattre l’enseignement clérical, c’est parce qu’il est « partial » ; c’est parce qu’il est « exclusif ». « Cet argument nous donne le moyen de mater l’{\itshape Église tout entière, de ses moines jusqu’à ses prêtres et de ses prêtres jusqu’à ses dévots}. » (\emph{Dépêche de Toulouse}.)\par
Enfin M. Ferdinand Buisson, se réfutant lui-même ou plutôt se renonçant et s’abandonnant et se laissant aller jusqu’où sa doctrine conduit tout homme logique et par conséquent devait le conduire lui-même, M. Ferdinand Buisson, {\itshape dans le même article} (\emph{Temps} du 17 septembre 1902) d’une part, après avoir marqué que seule la congrégation, par sa constitution même, était inapte à enseigner, déniait ensuite ce droit tout autant au prêtre séculier qu’au congréganiste ; — d’autre part, quand il en arrivait au prêtre séculier, confessait que c’était parfaitement en raison de la qualité de son enseignement qu’il déniait le droit d’enseigner au simple prêtre.\par
 En effet, il disait d’abord :\par
« La société a le droit de dire au congréganiste : « Vous réclamez la liberté de vous retirer du monde, de vous enfermer au cloître, après avoir juré par tout ce que vous avez de plus sacré de renoncer au mariage, à la propriété de vos biens, à la liberté de votre personne. Je vous y autorise. Mais, à peine en possession de cette autorisation, vous réapparaissez, réclamant vos droits d’homme et de citoyen et tout d’abord celui d’instruire et d’élever les enfants destinés à vivre au sein de cette société dont vous êtes isolé. Voilà qui est étrange. Vous ne sortez du monde que pour y mieux rentrer ; vous ne renoncez aux charges et aux conditions ordinaires de la vie de vos concitoyens que pour prendre aussitôt par l’éducation la direction de la société de demain, sinon de celle d’aujourd’hui. Choisissez. L’une de ces deux libertés suppose que l’on renonce à l’autre. Vivez si bon vous semble cette vie exceptionnelle du couvent, vie extra-familiale et extra-sociale. Mais tant qu’il vous plaira d’y rester, trouvez bon que je ne vous charge pas de préparer nos enfants à la vie familiale et sociale de tout le monde ; … qu’en tout cas je ne vous autorise pas, sans plus ample examen, à vous dédoubler ainsi en deux hommes, dont l’un vit en dehors de ce monde et dont l’autre aspire à le gouverner. »\par
 Voilà ce que M. Buisson disait d’abord, et c’était bien la thèse Buisson nº 1, cette thèse qui consiste à soutenir que le congréganiste vivant en congrégation est inapte à enseigner et indigne d’enseigner {\itshape à cause de sa manière de vivre}.\par
Sans doute, cette thèse est fausse. Elle repose sur ce principe qu’on ne peut préparer les enfants à vivre dans le monde qu’à la condition de lui ressembler trait pour trait ; qu’on ne peut préparer les enfants à vivre dans le monde qu’à la condition de n’être pas meilleur que lui : vous êtes docile, vous ne pouvez pas élever mon enfant ; vous méprisez l’argent, vous ne pouvez pas élever mon enfant ; vous êtes chaste, vous ne pouvez pas élever mon enfant. L’idée est contestable.\par
Et, d’autre part, cette thèse repose sur cette idée qu’user d’une liberté « suppose que l’on renonce à l’autre » ; qu’user de la liberté d’association implique que l’on renonce à la liberté d’enseignement. Si je vis associé, évidemment je ne puis être professeur ; si je suis professeur il est évident, il va de soi que je ne puis être associé à d’autres professeurs. Voilà une nouveauté. Cette incompatibilité entre les divers droits de l’homme n’avait pas encore été relevée. Faut-il dire aussi que si j’use du droit de propriété je renonce par cela même à mon droit de liberté individuelle, et que si j’use du droit  d’aller et devenir je ne puis plus être propriétaire ? Ce serait tout aussi juste. L’incompatibilité du droit d’association et du droit à enseigner est une trouvaille bien remarquable.\par
La thèse est donc fausse, et quelques-uns pourraient aller jusqu’à la trouver ridicule. Mais encore c’est bien la thèse Buisson nº 1 : le congréganiste est inhabile à enseigner à cause de son genre de vie ; et cette thèse réserve et respecte le droit à enseigner du prêtre séculier, lequel n’a pas du tout le même genre de vie que le congréganiste.\par
Or, dans le même article, comme j’ai dit, cent cinquante lignes plus loin, M. Buisson en vient à refuser le droit d’enseigner au prêtre séculier, au prêtre qui vit dans le monde, au prêtre mondain, et c’est la thèse nº 2, celle des écrivains qui étaient tout à l’heure les adversaires assez mordants de M. Buisson, et celle qui, comme les adversaires de M. Buisson l’avaient parfaitement vu, est bien forcée de dénier le droit d’enseigner, non en raison du genre de vie de ceux qui enseignent, mais en raison de la qualité de l’enseignement de ceux qui prétendent enseigner :\par
« Envisageons donc de sang-froid le moment, prochain peut-être, où la conscience publique, s’étant ressaisie, acceptera, approuvera, décrétera une {\itshape dernière et non moins naturelle incompatibilité} :  celle des fonctions religieuses et des fonctions enseignantes. Le prêtre, et encore plus le moine, est l’homme de la foi ; le professeur est l’homme de la raison, par conséquent, du libre examen. S’engager à être professeur c’est s’engager à penser librement et à faire penser librement. C’est promettre d’éveiller et d’exercer le sens critique, l’habitude de la discussion, l’esprit de recherche sans limite et sans réserve ; c’est déclarer que, quelle que soit la vérité, on l’acceptera le jour où la science la fera éclater, dût-elle renverser toutes les théories reçues. Peut-on soutenir que cet état d’esprit soit celui d’un prêtre ? »\par
Et voilà la thèse nº 2, et voilà les « {\itshape incompatibilités naturelles} » qui se multiplient. Il y avait incompatibilité naturelle entre le droit d’association et le droit d’enseigner. Il y a maintenant incompatibilité naturelle entre, non pas l’état de vie, mais {\itshape l’état d’âme} du prêtre et le droit d’enseigner. Pour parler net, et non pas plus net que M. Buisson, mais plus court, {\itshape on n’a pas le droit d’enseigner quand on a la foi} ; on n’a le droit d’enseigner que quand on est libre penseur ; les libres penseurs seuls peuvent enseigner et en ont le droit ; la liberté d’enseignement existera et nul n’en est plus partisan que M. Buisson ; mais, {\itshape naturellement}, elle ne peut exister que pour les libres penseurs.\par
 Donc on la refusera à tout prêtre catholique, bien entendu. Pour ce qui est des prêtres protestants, c’est très délicat. Comme c’est une question d’état d’âme et comme il est très difficile de savoir quel est l’état d’âme d’un prêtre protestant, il faudra voir et il y aura matière à examen. Comme le prêtre protestant, selon qu’il est plus ou moins près de Calvin, est plus homme de foi ou plus homme de libre examen, mais est toujours en partie foi, et en partie libre examen, il faudra lui demander si décidément il a encore la foi, ou s’il ne l’a plus du tout et s’il est sans réserve et sans retour libre penseur comme M. Berthelot. S’il a encore un peu de foi, on lui dira : « Perdez cela, et vous pourrez être professeur. »\par
Il en serait de même du prêtre israélite et, remarquez-le bien, il ne pourra pas en être différemment du laïque catholique, du laïque protestant et du laïque juif. Du moment que « l’incompatibilité naturelle », c’est-à-dire l’incapacité d’enseigner, est fondée et qu’on avoue qu’on la fonde sur l’état d’âme du prétendant éducateur et sur l’enseignement qu’à cause de cet état d’âme il doit donner, non seulement tout prêtre qui a la foi, mais tout homme qui a la foi ou qui est suspect de l’avoir est proscrit de l’enseignement. Il n’y a absolument aucune raison pour qu’un laïque à  sentiments religieux, qu’il soit, du reste, catholique, protestant ou juif, soit plus apte à enseigner qu’un prêtre catholique, protestant ou juif.\par
L’incompatibilité signalée par M. Buisson avec une netteté qui ne laisse rien à souhaiter, c’est l’incompatibilité du sentiment religieux et du droit d’enseigner. Le premier acte du futur professeur ou du futur instituteur doit donc être une abjuration. La première déclaration que devra faire le futur professeur ou le futur instituteur sera, {\itshape non seulement} : « Je ne serai pas docile, je ne serai pas pauvre, je ne serai pas chaste », ce qui va de soi et est indispensable ; mais encore : « Je n’ai aucune foi religieuse ; s’il m’en vient une je démissionnerai », ce qui, d’après la doctrine, est indispensable tout autant et ni plus ni moins.\par
On voit donc bien que si, par la force des choses, les républicains despotistes doivent être obligés d’en venir à la suppression de toute liberté d’enseignement et au pur et simple monopole, s’apercevant qu’à interdire l’enseignement aux congréganistes ils n’ont rien gagné tant que les prêtres séculiers pourront enseigner et enseigneront, et qu’à interdire l’enseignement aux prêtres séculiers ils n’auront rien gagné tant que les laïques croyants pourront enseigner et enseigneront ; selon la doctrine, aussi, la liberté d’enseignement  s’effondre et doit s’effondrer dès que la doctrine prend conscience d’elle-même et s’aperçoit qu’elle vise non pas telle ou telle façon de vivre de tels hommes, mais leur état d’âme lui-même, leur conception de la vie et du monde, leur mentalité et leur conscience, et dès qu’elle s’aperçoit qu’elle est en son fond la libre pensée pure et simple opposée à toute espèce de foi et à toute espèce de croyance, et la libre pensée qui refuse toute liberté de pensée à tout ce qui n’est pas elle-même.\par
Entre la liberté d’enseignement et le pur et simple et rude monopole de l’enseignement aux mains de l’État il n’y a rien du tout, rien, si ce n’est des subtilités d’opportunisme ou de polémique et des sophismes un peu ridicules dont ne sont pas dupes ceux-là même qui les alignent et auxquels ils renoncent sitôt, ou qu’ils sont vivement pressés, ou qu’ils prennent plus nettement conscience et maîtrise de leur pensée même.\par
Et non seulement les républicains despotistes seront acculés par la force des choses et conduits tout droit par leurs principes autant que par leurs passions au monopole de l’enseignement ; mais encore il serait étonnant qu’ils ne s’avisassent point un jour et prochainement qu’ils doivent interdire aux prêtres même l’enseignement {\itshape religieux} et qu’il n’est que logique et aussi qu’il est  nécessaire, après avoir fermé l’école au prêtre, de lui fermer l’église et de l’empêcher de parler de quelque façon que ce soit et sur quoi que ce puisse être.\par
C’est ce que le pasteur Farel a très lumineusement démontré, chose qui n’était pas très difficile, dans une lettre à la \emph{Dépêche de Toulouse}. Il écrivait, sans se fâcher, et très pertinemment, en homme qui sait que le logique et intégral Calvin n’a pas interdit aux catholiques seulement l’enseignement général, mais tout enseignement et {\itshape surtout} l’enseignement de leur religion : « Monsieur, voici ce que je lis dans votre article du 7 novembre 1903 : « Ce que nous voulons mater, ce n’est pas l’enseignement en soutane. C’est l’enseignement sectaire de l’Église, {\itshape sous quelque vêtement qu’il se cache}, c’est l’enseignement confessionnel… Nous entendons protéger l’enfant contre toutes les mainmises, plus ou moins dissimulées, plus ou moins hypocrites, de l’Église. » J’aurais pu relever des citations de même nature dans d’autres articles de vous sur la question de l’enseignement. Vous voulez donc pour l’État le monopole complet de l’enseignement et vous le voulez dans l’intérêt de la liberté de l’enfant que vous désirez soustraire à l’influence de l’Église. Non sans doute que vous pensiez que l’Église a une grammaire ou une  arithmétique différentes de celles des écoles de l’État, mais parce que vous craignez que, à la faveur de la grammaire et de l’arithmétique, ou concurremment avec elles, l’Église ne glisse son enseignement religieux. Je me demande alors comment vous faites — avec les arguments dont vous appuyez le droit et le devoir de l’État d’accaparer l’enseignement — pour ne pas aboutir à l’interdiction à l’Église de donner son enseignement religieux lui-même à la jeunesse. Vous lui contestez le droit de donner l’enseignement que nous appellerons « profane » ; vous le lui contestez par défiance, évidemment, de l’enseignement religieux, que l’Église pourrait donner sous le couvert de l’autre… Cet enseignement vous paraîtra-t-il moins dangereux quand il sera donné tout seul, que lorsqu’il est donné concurremment avec d’autres leçons ? Je voudrais bien savoir, en un mot, comment vous faites, avec vos principes, pour respecter la liberté de l’enseignement religieux. »\par
Et voilà qui est raisonné et à quoi il n’y a rien à répondre, et il serait plus logique de dire aux Églises : « Comme c’est de votre enseignement religieux que nous nous défions, vous enseignerez tout ce que vous voudrez, {\itshape excepté} votre religion », que de leur dire : « Comme c’est de votre enseignement  religieux que nous nous défions, vous n’enseignerez rien du tout, {\itshape excepté} votre religion. »\par
Et le plus logique encore est de leur dire : « Comme c’est de votre enseignement religieux que nous nous défions, et comme nous ne l’aimons ni lui-même ni se glissant sous le couvert de l’enseignement de l’histoire, de la physique ou de la trigonométrie, vous n’enseignerez ni votre religion ni autre chose. »\par
Et c’est inévitablement, d’après leurs principes et par les nécessités de leur dessein, ce que les anticléricaux en arriveront à dire en effet.\par
Comme le proclamait sans fard la \emph{Lanterne} du 17 mars 1903 : « Un jour viendra où les sociétés policées poursuivront les marchands de messes comme des malfaiteurs. »\par
Ajoutez ceci, qui n’a que peu d’importance, tant que les socialistes sont en minorité assez faible ; mais ce qui pourra être considérable un jour, et qui dès à présent doit compter : les socialistes sont dans les mêmes idées au point de vue religieux que les radicaux, et cela par conformité et par fidélité à leurs principes les plus profonds et à leurs idées les plus générales.\par
La grande différence entre le radical et le socialiste, c’est que pour le radical il n’y a qu’une question,  la question religieuse, et qu’un dessein à poursuivre, l’écrasement des religions ; et que, dès qu’il est question de socialisme, le radical ne comprend plus et, du reste, se méfie et, du reste, a peur.\par
Mais si le radical n’est pas socialiste (et malgré certains essais d’alliance et certaines étiquettes de caractère tout électoral, personne n’est plus antisocialiste que le radical), si le radical n’est pas socialiste, le socialiste est radical, c’est-à-dire anticlérical et antireligieux.\par
Pourquoi cela ? Très conformément à ses principes, comme je le disais. C’est parce qu’il voit très bien que le monopole de l’enseignement aux mains de l’État et, d’une façon plus générale, le monopole de toute pensée attribué à l’État est du même ordre et du même système que le monopole de toute propriété aux mains de l’État et est un acheminement à l’établissement de ce monopole de propriété. Le socialisme, c’est le retour à Louis XIV. Or, Louis XIV prétendait tout aussi bien être le propriétaire de tous les biens de ses sujets qu’il prétendait que tous ses sujets n’eussent qu’une religion et qu’une manière de penser, à savoir la sienne.\par
Ce rapport parfaitement juste entre des monopolisations et des socialisations du reste si différentes, est à remarquer et à méditer. M. Jean  Izoulet, radical sans doute, a dit ce mot, que je cite souvent en lui demandant la permission de le trouver abominable : « Ce ne sont pas les biens qu’il faut socialiser ; ce sont les personnes. » Et voilà ce que disent les radicaux. Socialisons les personnes en faisant de telle sorte qu’elles n’aient qu’une pensée, celle que l’État leur dictera. Le socialiste dit, lui : « Il faut socialiser les biens {\itshape et} les personnes, et qu’on socialise les personnes avant les biens, nous en sommes, parce que, tout au moins, la socialisation des personnes achemine à la socialisation des biens et, en attendant, en rend l’idée familière et habitue à la conception de l’État souverain. L’essentiel, après tout, c’est que la liberté recule. Qu’elle recule ici ou là, c’est toujours gain. »\par
C’est ce que M. Brousse établissait naguère dans un article qu’il faut lire lentement ; car M. Brousse n’est pas lumineux ; mais qui n’est ni sans portée, ni sans profondeur, ni sans justesse : « Inouï que des amis aussi proches de nous que les citoyens Aulard et Buisson soient amenés à défendre, sous couleur de liberté d’enseigner, la suprématie des éternels adversaires de la République et de la Démocratie !… {\itshape Leur liberté d’enseigner ressemble étrangement à la fameuse liberté de travail de Messieurs les économistes…} Je le proclame ici sans  ambages, le père Combes parviendrait-il à nettoyer le territoire de toutes congrégations, en aurait-il fini avec les comédies de sécularisations, je demeurerais partisan du monopole de l’enseignement de l’État, à tous les degrés, parce que c’est uniquement avec un service public de l’enseignement qu’il sera loisible d’assurer à tous les gens capables le droit d’enseigner. On croira que je tiens une gageure ; mais je ne sais pas si, à tout prendre, il ne vaudrait pas mieux pour la Démocratie subir à perpétuité la suprématie de l’enseignement congréganiste… que l’enseignement victorieux de telle grande compagnie mômière ou laïque, telle que les sceptiques à la Jules Lemaître seraient fort capables d’en créer… Ce n’est pas le caractère confessionnel seulement de la Congrégation qui constitue un danger, mais le fait même du monopole [religieux], et je le répète, il pourrait se former tel monopole laïque [à caractère religieux] dont l’enseignement serait pour notre avenir plus dangereux encore que le congréganiste. {\itshape Le clergé régulier, par la création d’un monopole de l’enseignement, joue le même rôle social que celui qui est dévolu dans le monde capitaliste aux grandes compagnies industrielles.} Et de même qu’on n’assurera à tous le droit au travail que par la nationalisation des moyens de production, de même on ne  pourra garantir à quiconque le droit d’enseigner que par le monopole de renseignement [aux mains de l’État]. Que demain, dans le monde économique, quelqu’un fasse table rase des fortunes et que partout l’intégrale concurrence s’exerce : avant un quart de siècle, de vastes monopoles privés, de grandes compagnies industrielles se seront créées sur l’écrasement des plus faibles dans la lutte pour la vie… Il en serait de même le jour où la liberté intégrale de l’enseignement serait proclamée. A défaut de la corporation congréganiste, il se formerait un grand nombre de monopoles privés désireux de conduire la jeunesse à leur gré. Et un seul remède serait efficace : l’expropriation et la nationalisation de tout cet enseignement particulier, aux vues égoïstes… »\par
La question, au point de vue socialiste, est très bien posée dans cet article, quoique lourdement. Plus habile à écrire en français et à trouver la formule nette, M. Jaurès a dit, de son côté, exactement la même chose d’une façon plus forte. Il a dit, et le mot doit rester : « {\itshape Le monopole universitaire, c’est le collectivisme de l’enseignement.} » — Certainement cette étroite parenté entre la socialisation intellectuelle et la socialisation matérielle, cette connexité indiscutable entre la socialisation des personnes et la socialisation  des propriétés, sera peu du goût des radicaux quand, l’un des deux termes acquis, on passera à l’autre ; certainement ils diront : « ce n’est pas du tout la même chose » ; certainement M. Jaurès a raison de dire, en souriant dans sa barbe : « {\itshape Peut-être} les radicaux, qui demandent aujourd’hui avec nous le collectivisme de l’enseignement, seront-ils embarrassés un jour pour combattre le collectivisme de la production. Peut-être, après avoir démontré et proclamé que la nation enseignante ne menace dans une démocratie aucune liberté, devront-ils reconnaître que la même nation possédante ne menacerait dans une démocratie aucune initiative et aucun droit. » Mais j’en suis à mettre en lumière, d’une part que les socialistes, conséquents avec eux-mêmes, seront les alliés des radicaux dans l’œuvre de la destruction de la liberté de l’enseignement tant que de cette liberté il restera un vestige ou une ombre ; d’autre part que toutes les socialisations sont solidaires comme toutes les libertés se tiennent, et que par le chemin du monopole de l’enseignement on va tout droit à la « nation possédante », c’est-à-dire à l’abolition de la propriété et à la confiscation des propriétés.\par
Et qu’est-ce, en effet, que le droit de penser, le droit de parler, le droit d’écrire, le droit d’enseigner, si ce n’est propriétés intellectuelles ?\par
 
\asterism

\noindent Et enfin j’ai dit qu’après avoir interdit aux congréganistes d’enseigner, après avoir interdit aux prêtres séculiers d’enseigner, pour les mêmes raisons qui s’appliquaient aux congréganistes ; après avoir interdit aux laïques à sentiments religieux d’enseigner, pour les mêmes raisons qui s’appliquaient aux prêtres séculiers — voir l’argumentation de M. Brousse ; voir le mot de la \emph{Dépêche} : « Aussi bien à leurs dévots qu’à leurs prêtres » — après avoir établi le monopole universitaire absolu, les anticléricaux en viendraient, dans le monopole même, dans l’Université, dans {\itshape leur} Université même, à établir et à faire régner un despotisme absolu et une absolue conformité et servilité à leur manière de voir, au {\itshape credo} qu’ils dicteraient.\par
C’est ce que l’on conteste beaucoup. Une formule assez bien trouvée, assez spécieuse, a eu beaucoup de succès et a été répétée, avec variantes négligeables, par tous les ennemis secrets ou déclarés de la liberté de l’enseignement, depuis M. Buisson jusqu’à la \emph{Dépêche de Toulouse}, et depuis la \emph{Dépêche} jusqu’à M. Brousse : « A la liberté de l’enseignement nous voulons substituer la liberté {\itshape dans}  l’enseignement. Nous ne voulons pas de la liberté {\itshape de} l’enseignement ; nous voulons la liberté {\itshape dans} l’enseignement ; le seul enseignement libre, c’est l’enseignement monopolisé par l’État, {\itshape mais} où la liberté régnera. »\par
Je ne veux jamais suspecter la sincérité de mes adversaires ; mais en laissant à d’autres de déclarer que cette formule est hypocrite, j’affirme qu’il n’en est pas de plus décevante, ni de plus creuse. Soit ; vous avez parfaitement, en toute sincérité et même en toute conviction, {\itshape en y tenant}, l’intention, d’abord de n’avoir en France qu’une Université d’État, ensuite de laisser les professeurs de cette Université d’État enseigner « ce qu’ils voudront, comme ils voudront ». Soit ; c’est bien votre intention et, si cela vous peut plaire que je vous le dise, je reconnaîtrai que je suis moi-même assez persuadé que vous laisserez une certaine latitude de doctrine et d’enseignement dans l’Université qui sera à vous, malgré la tentation bien naturelle d’imposer ses idées aux gens qu’on paie. Soit donc. Mais cette latitude, vous ne la laisserez évidemment que dans certaines limites…\par

\begin{itemize}[itemsep=0pt,]
\item  — Non ; sans limites !
\item  — Comment ! Mais alors pourquoi faites-vous votre Université et lui donnez-vous le monopole ? Ce n’est donc pas pour arracher la jeunesse à l’influence  cléricale ? Ce n’est donc pas pour qu’il n’y ait plus « deux jeunesses » et « deux Frances » ? Ce n’est donc pas pour établir « l’Unité morale » ? Si c’est pour cela, et vous ne pouvez pas dire, après toutes vos déclarations, que ce ne soit pas pour cela ; si c’est pour cela, ne voyez-vous pas que tout ce qui détruit l’unité morale, que tout ce qui fait deux jeunesses, que tout ce qui fait deux Frances, que l’influence contre-révolutionnaire, que l’influence religieuse, que l’influence cléricale va rentrer dans votre Université et y sévir et que vous n’aurez fait que transporter chez vous ce que vous aurez voulu détruire ailleurs ?
\end{itemize}

\noindent Doutez-vous que vos ennemis, leurs écoles détruites, n’entrent dans les vôtres, précisément parce que détruites auront été les leurs, comme ils ont mis soin et ardeur à entrer à l’École polytechnique et à l’École Saint-Cyr ? Vous les connaissez ; vous ne doutez pas de cela.\par
Eh bien alors, qu’aurez-vous fait et quoi de gagné ? La liberté {\itshape de} l’enseignement, c’était la France à moitié « infestée » ; la liberté {\itshape dans} l’enseignement, ce sera l’Université à moitié « infestée », et la France, par suite, à moitié « infestée » comme auparavant.\par
On pourrait même dire : « un peu plus » ; car la moitié des professeurs de l’Université cléricalisant  la jeunesse, les cléricalisera {\itshape avec plus d’autorité} que des professeurs libres, parce qu’ils auront comme l’estampille et l’apostille de l’État et feront comme partie du gouvernement ; et l’on aura ce spectacle curieux d’un gouvernement « affranchissant » et « libéralisant » la France par une moitié de ses professeurs et la christianisant et cléricalisant par l’autre moitié.\par
Avouez que ce spectacle et ce résultat, vous n’en voudrez pas. Vous ne pourrez pas en vouloir, puisque précisément ce que vous aurez voulu éviter en monopolisant l’enseignement, vous vous trouverez le faire vous-même par votre enseignement monopolisé et parce que vous l’aurez monopolisé. On ne réussit pas à ce point contre son dessein et contre toutes les raisons de son dessein, sans regimber contre soi-même et sans dire : « Ah ! Cependant ! Ah ! mais non ! »\par
Et dès lors vous serez amenés à imposer un {\itshape credo}, c’est-à-dire à supprimer la liberté {\itshape dans} l’enseignement, après avoir supprimé la liberté {\itshape de} l’enseignement.\par
Ce {\itshape credo}, j’admets que vous ne l’imposerez pas par un programme, par une déclaration, par une bulle, par un \emph{Syllabus} ; je l’admets ; mais vous l’imposerez par vos inspecteurs, proviseurs, doyens, directeurs et les avertissements qu’ils donneront  aux professeurs hérétiques ou dissidents, et ce sera exactement la même chose qu’un {\itshape credo} affiché sur les murailles ou inséré à l’\emph{Officiel}.\par
Je reconnais encore que ce {\itshape credo} aura un caractère particulier : il aura un caractère négatif. Vous admettrez très bien une certaine liberté de penser en dehors de la mentalité chrétienne et de la mentalité contre-révolutionnaire. Qu’un professeur enseigne Kant ou enseigne Spencer, cela vous sera à peu près indifférent ; qu’un professeur enseigne Danton ou enseigne Robespierre, vous n’y regarderez pas de très près. Mais qu’un professeur enseigne la foi prouvée par la raison ; ou la nécessité de la foi, la raison étant infirme ; ou, comme veut M. Bourget, la « destruction méthodique de l’œuvre de la Révolution » ; il est clair comme le jour que vous ne le laisserez pas se livrer à ces exercices.\par
Votre {\itshape credo} sera donc négatif. En sera-t-il moins impérieux, moins exclusif, moins tyrannique ? Pas le moins du monde. Il sera comme celui de l’Église, qui laisse toute liberté de penser et d’enseigner dans certaines limites, celles au-delà desquelles les hérésies commencent. La liberté que vous laisserez sera celle d’être libre penseur comme on voudra et révolutionnaire comme on l’entendra ;  elle ne sera jamais celle d’être ancien-régime ou d’être croyant.\par
Cela veut dire que le seul moyen d’avoir la liberté {\itshape dans} l’enseignement, c’est d’avoir la liberté {\itshape de} l’enseignement, et qu’en dehors de la liberté {\itshape de} l’enseignement il n’y a plus de liberté du tout. Et, tout au fond, vous le savez bien. « Je ne veux pas de la liberté des autres ; je veux être libéral moi-même. » Naïveté ou hypocrisie, c’est un joli mot de comédie, que personne ne prendra un instant au sérieux.\par
Pour ce qui est de la liberté d’enseignement, ce qui reste encore à faire aux républicains despotistes et ce qu’ils sont condamnés à faire, c’est supprimer la liberté d’enseignement pour les prêtres, supprimer la liberté d’enseignement pour les laïques croyants, établir le monopole universitaire, exclure toute liberté véritable de l’Université monopolisée. Donc la bataille continue.\par

\astertri

\noindent Elle continuera également sur la question de l’existence même de l’Église catholique libre, sur la question de l’existence de l’Église catholique, quelque séparée qu’elle ait été de l’État. Il ne faut se faire aucune illusion là-dessus. La loi de séparation,  la loi de 1905, n’a satisfait personne, sans doute, ni les hommes de droite ni les hommes de gauche ; mais ce sont surtout les hommes de gauche qu’elle n’a pas satisfaits.\par
La question, avant la loi de 1905, se posait ainsi : on séparera l’Église de l’État ; mais, une fois séparée, lui appliquera-t-on tout simplement le droit commun, ou lui imposera-t-on un régime exceptionnel ? Les républicains libéraux, groupe insignifiant dans l’armée républicaine, répondaient : « On la mettra simplement dans le droit commun ». Les républicains despotistes répondaient : « Jamais de la vie ! On lui imposera un régime exceptionnel et aussi dur que possible. »\par
Les républicains libéraux avaient leur représentant le plus net et leur interprète le plus précis en la personne du regretté M. Goblet. M. Goblet disait, dans les \emph{Annales de la Jeunesse laïque} (1903) : « Si je reste fidèle à l’idée de la séparation, dont j’ai toujours été le partisan convaincu, c’est avant tout pour affranchir l’État d’un lien qui lui est plus nuisible qu’à la religion et aux Églises elles-mêmes ; mais c’est aussi sous cette réserve que l’État, en reprenant sa liberté, devra respecter celle des croyances religieuses et aussi celle des Églises. Son rôle est, suivant moi, {\itshape de les ignorer}.  Le jour où l’État aura cessé de subventionner les Églises et de leur communiquer la force qu’elles tirent de leur union avec lui, il n’aurait plus à les considérer que comme des {\itshape associations ordinaires} soumises à la loi commune. Les sectes diverses qui ne manqueraient pas de se former auraient bientôt réduit l’autorité de l’Église catholique à ce qu’est aujourd’hui celle des Églises protestantes. Les unes et les autres pourraient bien exercer encore, et peut-être même plus qu’aujourd’hui, une influence morale que je ne veux nullement leur enlever ; elles auraient perdu l’influence politique que seule il importe de détruire. Je ne vois même pas, sous ce régime, la nécessité d’une loi sur la police des cultes ; car les mandements des évêques, comme les prédications des membres du clergé, n’auraient pas plus de valeur alors que les articles de journaux ou les discours de réunions publiques. Les dispositions de la loi pénale suffiraient pour les réprimer. »\par
Voilà la pure doctrine libérale en cette matière, la doctrine libérale intégrale et absolue. Mais de cette doctrine les républicains despotistes, les républicains autoritaires et même beaucoup de républicains modérés étaient aussi éloignés que possible, comme on peut croire, et tenaient la doctrine de M. Goblet pour une doctrine ultra-cléricale.\par
 Dans le même temps, M. Aulard écrivait : « J’entends bien dire que c’est violer les principes du républicanisme que de refuser à quiconque le bénéfice de la liberté et du droit commun. Mais je réponds que l’Église catholique n’est pas {\itshape quiconque}. Cette Église internationale, dirigée par un monarque étranger, par un monarque tout-puissant et qui se dit infaillible, par un monarque auquel ses sujets font profession de soumettre toute leur conscience, toute leur personne morale ; cette Église organisée en une solide et serrée hiérarchie despotique ; cette Église qui prétend être elle-même une cité, une société, un État, l’État parfait, l’État dans lequel devraient s’absorber tous les États ; cette Église qui affecte un rôle mondial, à la fois politique et social, le rôle de conductrice de peuples, et qui conduit en effet les peuples à un idéal opposé à celui des sociétés modernes, affichant la haine et le mépris de la civilisation actuelle, de la liberté de conscience, de toutes les libertés, de la raison ; cette Église, enfin, qui complote ouvertement la destruction de l’édifice politique et social élevé par la Révolution française et l’abolition des \emph{Droits de l’homme} qu’elle appelle sataniques ; comment cette Église pourrait-elle se réclamer du droit commun ? Quel droit a-t-elle au droit commun, puisqu’elle n’existe,  ne parle et n’agit que pour renverser ce droit commun ? »\par
Cette déclaration, qui, moins l’éloquence, se ramène à cette formule : « Je n’accorde la liberté qu’à ceux qui sont si faibles qu’ils me sont inoffensifs et qu’à ceux qui ont le même idéal que moi », était l’expression même de l’esprit général du parti républicain ; je n’ai pas besoin, après tout ce que j’ai rapporté dans ce volume, de le prouver, ni même de le dire.\par
Elle concluait à un régime exceptionnel et très rigoureux pour cette association tout à fait exceptionnelle qui s’appelle l’Église catholique.\par
De son côté, le rédacteur ordinaire de la \emph{Dépêche de Toulouse}, avec les mêmes arguments et avec d’autres, prenant plus précisément à partie M. Goblet, écrivait : « … les associations catholiques ne sont pas et ne seront jamais des associations comme les autres. Elles obéissent, elles sont tenues d’obéir à un mot d’ordre étranger. Cela suffit, et amplement, à leur donner un caractère exceptionnel. Elles ne peuvent donc être soumises qu’à un régime d’exception. M. de Pressensé l’a compris. Avec beaucoup de prévoyance, il limite {\itshape leur développement} d’abord et ensuite leurs richesses. M. de Pressensé n’a pas tort. Ce développement, ces richesses, si on ne leur assignait des limites,  pourraient précisément devenir des instruments de règne de l’Église temporelle. Ils pourraient être des facteurs de cette influence politique que, de l’aveu même de M. Goblet, il importe de détruire. »\par
Et allant, cette fois, droit au point, avec un remarquable esprit de précision, le rédacteur ajoutait : « Le législateur (de 1903) a pris le plus grand soin de surveiller et d’endiguer les associations religieuses constituées par les congrégations. {\itshape N’aura-t-il pas les mêmes motifs de surveiller et d’endiguer les associations religieuses que le clergé séculier constituera au lendemain de la séparation} ? »\par
Voilà précisément le fond des choses. Les associations religieuses, les associations cultuelles qui seront la contexture même de l’Église après la séparation, ces associations, à très peu près, seront la même chose que ce qu’étaient les associations de congréganistes ; et il y aura pour les républicains, d’après leurs idées et leurs passions, exactement les mêmes raisons de surveiller, puis d’endiguer, puis de détruire les associations cultuelles, qu’il y a eu pour surveiller, puis endiguer, puis détruire les associations congréganistes.\par
Pourquoi oui, cela saute aux yeux ; et pourquoi non, il m’est impossible de l’entrevoir.\par
 Cherchant, pour conclure, une formule qui exprimât au plus juste la mentalité républicaine en matière ecclésiastique, le rédacteur s’arrêtait à ceci : « Pour ce qui est de la propagande, Église libre dans l’État neutre ; mais pour ce qui est de l’association, Église libre dans l’État souverain. » C’est-à-dire que l’Église pourra {\itshape dire} ce qu’elle voudra ; mais que, {\itshape pour exister}, l’association étant désormais son seul mode possible d’existence, elle sera libre dans l’État ayant tout droit de la supprimer.\par
Et tel me semble bien être depuis une dizaine d’années l’esprit général et presque universel du parti républicain.\par
{\itshape Or}, depuis ces échanges de vue, la loi de 1905 a été faite. Cette loi, comme nous l’avons vu, s’est placée entre les deux doctrines exposées ci-dessus. Elle s’est placée entre le droit commun et le régime exceptionnel dur ; et elle est, à mon avis, quoique établissant un régime d’exception, plus près du droit commun que du régime exceptionnel rigoureux. Cela ne peut aucunement satisfaire le parti républicain et, s’il l’a déjà inquiété au cours des discussions de la loi, l’irritera et lui sera insupportable dans la pratique. Il n’y a pas de raison pour que son esprit change et il y a toutes sortes de raisons pour que les faits qui  doivent sortir du régime nouveau établi par la loi de 1905, vus l’exaspèrent, puisque, seulement prévus, ils l’ont alarmé.\par
Oui, sans doute, il se trouvera en face d’associations cultuelles qui lui paraîtront des foyers de réaction et des antres d’obscurantisme. Lui qui n’a pas pu supporter jadis la Société de Saint-Vincent-de-Paul, comment pourrait-il supporter des associations qui, avec un maniement de fonds, limité, sans doute, mais encore considérable, auront clientèle, subordonnés, alliés, hiérarchie, seront ce que les républicains appellent tout de suite des « États dans l’État » et ce que, à ce titre, ils détestent d’une haine sauvage et d’une horreur qui leur ôte tout usage de la raison ?\par
Remarquez encore que ce que le Concordat avait interdit, à savoir les communications, non contrôlées par le gouvernement français, entre le Saint-Siège et l’Église française, n’est plus interdit par la loi nouvelle. La loi de séparation ne connaît pas le Saint-Siège, elle l’ignore, et, parce qu’elle l’ignore, elle le passe sous silence, et il le faut bien, car si elle en parlait, elle serait une manière de Concordat ; elle serait un Concordat unilatéral, si l’on peut parler ainsi ; mais elle serait une manière de Concordat, en ce qu’elle connaîtrait des relations entre le Saint-Siège et l’Église française  et les réglerait. Les communications du Saint-Siège à l’Église de France, sous le régime nouveau, ne tombent plus que sous l’article 34 de la loi de séparation : comme tous les autres « discours prononcés, {\itshape lectures faites}, écrits distribués, affiches apposées », elles ne sont poursuivies que si elles outragent ou diffament un citoyen chargé d’un service public, ou si elles constituent une provocation directe à résister à l’exécution des lois, ou si elles tendent à soulever ou armer une partie des citoyens contre les autres. Mais, sauf ces cas, elles sont permises.\par
De quel œil les républicains verront-ils des communications du Saint-Siège aux fidèles de France lues en chaire sans avoir passé par la censure du gouvernement français ? Ils trouveront évidemment que la loi nouvelle a désarmé la France, la République française, l’Unité morale de la France, devant le Saint-Siège, devant la Rome pontificale. Ils regretteront le Concordat, « la digue » du Concordat.\par
Remarquez encore le mot, parfait pour moi, inquiétant pour eux, de M. Goblet : « Les sectes diverses [lisez : associations cultuelles, d’esprits différents peut-être] pourraient bien exercer encore et {\itshape peut-être plus qu’aujourd’hui} une influence morale que je ne veux nullement leur enlever ;  elles auraient perdu l’influence politique. »\par
Ceci est si juste, probablement, ceci est si vraisemblable qu’un article de la \emph{Semaine religieuse de Paris}, dû peut-être à la plume et assurément à l’inspiration de l’archevêque de Paris, se rencontre absolument avec ces quelques lignes de M. Goblet : « Ne se pourrait-il pas que l’obligation où nous allons nous trouver de recourir à l’association pour sauvegarder les intérêts de l’Église de France {\itshape tourne en définitive à l’avantage de nos paroisses} et que nous retrouvions par là {\itshape cette cohésion que l’organisation trop administrative du Concordat} nous a fait perdre en nous déchargeant de trop de soucis ? Qui peut dire s’il n’y aura pas là un vaste champ ouvert à des initiatives jusque-là comprimées et si ce ne sera pas, pour bien des {\itshape vocations laïques}, en particulier, l’occasion de se révéler ? S’il devait en être ainsi et qu’on pût arriver dans chaque paroisse à grouper dans un faisceau unique, sous l’autorité du curé, les œuvres devenues plus nombreuses ; si l’on pouvait, ensuite, constituer, sous l’autorité de l’évêque, une union de tous {\itshape ces organismes bien vivants} qui échapperaient ainsi à l’individualisme et centupleraient par là leur action ; ne serait-on pas en droit d’espérer qu’après les tristesses de demain des jours meilleurs pourraient se lever pour l’Église de France ? »\par
 Je suis persuadé, pour mon compte, encore que je puisse me tromper, qu’en brisant le Concordat et en permettant les associations cultuelles, les républicains français ont {\itshape rapproché les fidèles de leurs pasteurs}, créé entre eux un lien et une communication étroite qui manquait et qui ne pouvait que manquer sous le régime du Concordat et que le Concordat avait été fait pour détruire.\par
Je suis persuadé que la loi Briand a rétabli l’Église dans ses véritables conditions de vie, c’est-à-dire dans les conditions où il faut qu’elle soit pour qu’elle soit vivante ; et si je l’ai dit longtemps avant, ce n’est pas une raison suffisante pour que je ne le pense pas après.\par
Je suis persuadé que si « l’Église latérale », l’Église congréganiste, a été si puissante et si riche n’ayant pour soutien que les fidèles, l’Église officielle devenue l’Église libre trouvera dans les fidèles le même appui et le même viatique ; et je ne vois pas les raisons pourquoi il en serait autrement.\par
Je suis persuadé que la loi Briand, telle qu’elle est, est encore un bienfait pour l’Église et qu’elle sera regardée comme une faute par les républicains et qu’ils ne tarderont pas beaucoup à l’appeler la loi des dupes.\par
Mais, précisément à cause de cela, la persécution  à l’endroit de l’Église va recommencer et ne peut que recommencer plus vive et plus ardente. Le procédé en quelque sorte automatique des révolutionnaires à l’égard de l’Église catholique est celui-ci : spolier l’Église ; puis, en compensation de la mesure spoliatrice, lui accorder certains avantages ; puis supprimer ces avantages, sans revenir, bien entendu, au régime précédent.\par
L’Église était possédante : on lui prend ses biens et en compensation on lui donne le budget des cultes garanti par un Concordat.\par
Elle a un budget des cultes garanti par le Concordat : on supprime le Concordat et le budget des cultes, et en compensation on donne à l’Église la liberté, en lui disant : « cela vaut mieux » ; ce que, du reste, je crois.\par
Demain, si l’on voit que cela vaut mieux, et d’autant plus que l’on constatera que cela vaut mieux, et même, du reste, si cela valait moins, et sauf le cas où cela ne vaudrait rien du tout, on supprimera la liberté de l’Église.\par

\astertri

\noindent Autre aspect : il y a deux Églises, l’Église officielle et « l’Église latérale ». On supprime l’Église latérale et l’on dit à l’Église officielle : « C’est un cadeau  que nous vous faisons, car tout l’argent qui allait à l’Église latérale ira à vous et toute l’influence dont elle jouissait, c’est vous qui l’aurez. » Mais quand l’Église latérale est supprimée on ne permet pas à l’Église officielle de posséder autant que cela était permis à l’Église latérale, et quant à son influence, on surveillera celle qu’elle pourra prendre et on la « matera », c’est le mot constamment employé, dans la mesure précisément de l’influence prise.\par
D’ailleurs la munificence accordée, toujours révocable, doit se mesurer, pour ce qu’elle doit devenir, aux idées générales et aux principes du bienfaiteur. Les républicains accordent à l’Église la liberté, ou à peu près ; mais les républicains, neuf sur dix, sont gens à qui la liberté est odieuse, insupportable et du reste inintelligible. Le seul cadeau que les républicains démocrates ne puissent pas faire de telle manière qu’on y puisse croire, c’est la liberté, et il y a soit ironie, soit distraction, soit hypocrisie, soit, et en tout cas, inconséquence de leur part à l’accorder ou à prétendre qu’ils la donnent. La liberté accordée par des démocrates, c’est un serment de fidélité prêté par Don Juan.\par
Autre aspect : c’est sous forme d’association, et ce ne pouvait pas être autrement, que les républicains ont accordé la liberté à l’Église. Or l’association  est toujours pour les républicains un objet de défiance invincible, depuis Rousseau jusqu’à M. Jaurès. Ce sont eux, je l’ai montré bien souvent, qui ont inventé le contresens qui consiste à appeler « aristocratie » tout ce qui est association. Une vraie aristocratie, c’est une association qui gouverne, qui a des soldats, des juges, des gendarmes et des collecteurs d’impôts. Mais pour les républicains français est « corps aristocratique » toute association, sans soldats, sans juges, sans gendarmes et sans publicains, par cela seul qu’elle est association et se distingue un peu de la vaste association qui est l’État. Le mot « État dans l’État », qui fait frémir tout démocrate français, n’a pas d’autre sens. Il veut dire « association particulière au sein de l’association générale » ; et c’est cela qui fait horreur aux républicains.\par
Ceux-ci, en donnant licence à l’Église de s’organiser en association, en l’organisant eux-mêmes, pour ainsi dire, en association, ont donc fait quelque chose qui est si contraire à leurs passions et à leurs principes qu’en le faisant ils ont comme promis et juré de ne pas le faire, ou qu’en l’établissant ils ont comme juré de n’y point persévérer.\par
Royer-Collard, je crois, discutant un projet de loi qu’il estimait abominable, déclarait : « Si vous  faites cette loi, je jure de lui désobéir. » Les républicains de 1905, en faisant la loi Briand, ont dit, par tout leur passé, par toutes leurs idées mille fois exprimées et par toute leur coutume : « Je fais cette loi avec le ferme propos de la détruire. »\par
C’est ce que M. Bepmale a déclaré formellement ; c’est ce que presque tous les républicains ont pensé.\par
Donc, sur ce terrain encore, la bataille continue.\par
Le parti républicain achèvera les congrégations, si tant est que quelqu’une de ces blessées respire encore.\par
Il détruira ce qui reste de la liberté d’enseignement et en arrivera à l’établissement pur et simple du monopole universitaire, qui est son idéal.\par
Il détruira pièce par pièce la loi Briand considérée comme trop libérale et tenue pour seconde loi Falloux, et il réduira l’Église catholique à la situation qui était celle de l’Église protestante au \textsc{xviii}\textsuperscript{e} siècle, comme c’est aussi son idéal.\par
« Le catholique est toujours un factieux ; il faut le cantonner légalement dans l’état de factieux », voilà le fond de la pensée républicaine. « Le cléricalisme, c’est l’ennemi ; il faut et le traiter en ennemi et le forcer à n’être dans l’État qu’un ennemi », voilà, depuis Gambetta, le principe  même de la « société laïque » et de la « société moderne ».\par

\astertri

\noindent Remarquez, du reste, ce qui est le plus important, que si, par leurs principes, idées et passions, les républicains despotistes détestent tout catholique et même tout chrétien croyant, en pratique, quand bien même ils n’auraient ni ces idées ni ces passions, ils seraient à peu près forcés de se conduire comme s’ils les avaient et de faire la guerre au catholicisme comme s’ils l’exécraient en effet.\par
Car c’est, en vérité, leur seule ressource électorale, et ils ne vivent que d’élections. Il s’agit pour le parti républicain, ou plutôt pour le syndicat des politiciens républicains, d’{\itshape être populaires}. Or, ils n’ont aucun autre moyen de l’être que l’anticléricalisme.\par
Ils ne peuvent pas l’être par des réformes utiles, salutaires et fécondes. D’abord parce qu’ils sont peu en état de les imaginer et de les accomplir, presque aucun n’étant homme d’État. Ensuite parce que le système parlementaire tel qu’il est organisé en France, extrêmement lent pour l’élaboration d’une réforme quelconque et extrêmement prompt à remettre le député en face de ses électeurs,  ne permet pas au député d’être réformateur d’une façon utile pour lui, de façon qu’il recueille en popularité le fruit de la réforme à laquelle il se sera attelé.\par
Supposez un député qui étudie une bonne réforme à faire, consciencieusement, sérieusement. Avant qu’elle soit discutée, les quatre ans de la législature seront épuisés et notre homme se retrouvera en présence de ses électeurs qui lui diront : « Vous n’avez rien fait. » Cela n’encourage pas ; cela dégoûte. Il faudrait être héroïque, en France, pour s’occuper d’une réforme utile quand on est député. Tous ceux, à peu d’exceptions près, en France, qui s’occupent de réformes, sont étrangers au Parlement. C’est quand on n’a pas affaire aux électeurs qu’on peut être, non un politicien, mais un homme politique studieux.\par
Cette popularité, qui est le pain dont ils vivent, les politiciens se l’assureront-ils tout simplement, tout humblement, par une bonne administration du pays ? Ceci regarde surtout le gouvernement et aussi les parlementaires en tant que contrôlant, surveillant et inspirant le travail administratif du gouvernement.\par
Or, non, la popularité ne s’acquiert pas, en France, par une bonne administration, parce que le pays ne fait presque aucune attention à la façon dont il est  administré. Nul peuple au monde n’est plus indifférent à cet égard. Ce n’est pas un peuple pratique ; ce n’est pas un peuple réaliste, c’est un peuple uniquement préoccupé d’idées générales, d’idées générales très simples et très grossières, comme l’égalité, l’abolition des supériorités, le nivellement des fortunes, la destruction des religions ; mais enfin d’idées générales et uniquement d’idées générales.\par
Il semble qu’il vive de cela. Toutes les conversations que vous écoutez roulent sur des théories. Jamais ou presque jamais vous n’entendez, au café, en chemin de fer, sur le pont d’un bateau, parler d’une question administrative, d’une amélioration pratique, d’un meilleur aménagement de la maison commune. Les Français sont un peuple qui ne parle que de politique générale ou de femmes.\par
Ce n’est pas absolument toujours ainsi. On a remarqué et j’ai beaucoup, non sans complaisance, insisté sur ce fait, qu’en 1789 les Français par leurs « cahiers » n’ont presque absolument demandé qu’une chose : être bien administrés, être administrés d’une façon régulière ; et que le grand succès du 18 Brumaire est venu de ce qu’il apportait en 1799 justement ce que les Français avaient réclamé et presque uniquement réclamé dix ans auparavant.\par
 Mais, précisément, remarquez aussi qu’une révolution qui avait été faite pour obtenir des réformes administratives, a roulé tout entière sur des idées générales, n’a poursuivi clairement, à travers ses convulsions sanglantes, que deux desseins : l’égalité et la souveraineté du peuple ; et a été aussi « abstraite » et aussi « idéaliste » que possible en son esprit.\par
Et, précisément, remarquez aussi que le 18 Brumaire, agréable à la nation française parce qu’il satisfaisait ses désirs de 1789 et répondait à son état d’âme de 1789, a inauguré un régime tout inspiré encore d’idées générales : grandeur et gloire de la France, diffusion par les armes, à travers le monde, des principes révolutionnaires, égalité, souveraineté du peuple, antiaristocratisme, anticléricalisme, etc.\par
Et, précisément, remarquez encore que le second Empire, accueilli favorablement en France tout simplement parce qu’il muselait l’anarchie et par conséquent dans un esprit tout réaliste et tout pratique, a presque immédiatement eu besoin d’une idée générale, parfaitement contraire du reste aux intérêts matériels de la France, c’est à savoir du « principe des nationalités », pour maintenir sa popularité, ou plutôt pour se faire une popularité d’un genre nouveau, la première s’étant épuisée  parce qu’avaient disparu les circonstances qui l’avaient produite.\par
D’abondant remarquez encore que le seul gouvernement français, au \textsc{xix}\textsuperscript{e} siècle, qui n’ait pas été du tout populaire est le gouvernement de 1830-1848, parce que, sans aucun idéal, il ne s’est occupé absolument que de bien administrer et d’assurer la prospérité matérielle du pays.\par
En France, bien gouverner la maison est une duperie, et ce que le Français demande le moins à un gouvernement, c’est d’être une bonne ménagère.\par
Les hommes politiques français feraient donc une pure folie et montreraient une colossale naïveté s’ils cherchaient la popularité par la pratique d’une bonne administration intérieure, soit, ministres, en administrant bien, soit, parlementaires, en exigeant des ministres qu’ils administrent sagement et en contrôlant sévèrement leur gestion.\par
Comment donc les hommes politiques s’assureront-ils la popularité dont ils ont besoin comme d’air et de nourriture ? Uniquement en exploitant une idée générale et, selon les temps, celle-ci ou celle-là, celle qui, à un moment donné, a les faveurs de la foule ou d’une partie au moins très considérable du pays. Or, en ce moment, j’entends depuis 1871 jusqu’à nos jours et jusqu’aux jours qui  vont venir, quelle est l’idée générale à exploiter ?\par
Est-ce l’idée de souveraineté du peuple ? Non ; elle n’est plus à exploiter, parce qu’elle est acquise. Le peuple se sent souverain très suffisamment. On pourrait, sans doute, lui faire remarquer qu’il ne l’est pas, qu’il ne le serait que sans députés et sans juges nommés par le gouvernement ; qu’il ne le serait que par le gouvernement direct et la magistrature élue. Le moment viendra peut-être où l’on exploitera cette idée générale très spécieuse et très exploitable ; mais il n’est pas venu ; cette idée, quand on y touche, ne « rend pas » ; il est de fait que le peuple français se sent très suffisamment souverain, en quoi on peut reconnaître qu’il n’a pas absolument tort.\par
Les hommes politiques, pour se faire une popularité ou pour entretenir celle qu’ils ont, exploiteront-ils l’idée d’égalité ? Il en est qui font ainsi. Ce sont les socialistes. Ils s’attachent à démontrer, et ils n’ont aucune peine à démontrer, que l’égalité n’existe point du tout, puisqu’il y a des inégalités de fortune, et énormes, et puisqu’aucune égalité légale et juridique n’est que leurre et ombre pour proie, tant qu’il y a inégalité de fortunes. Ils ont raison, admis le principe d’où ils partent, et ils sont très écoutés. Seulement il y a trop de possédants en France pour que la France soit socialiste  en majorité, et l’on ne se fait, avec le socialisme, qu’une popularité locale, qu’une popularité de minorité, en définitive, ce qui n’est jamais tout à fait du goût d’un homme politique. L’idée générale de l’égalité, de l’égalité « réelle », n’est pas d’un très bon rapport.\par
Les hommes politiques exploiteront-ils les idées de grandeur et gloire de la France, de diffusion des principes révolutionnaires à travers le monde, ou la théorie des nationalités ? Depuis 1871, tout cela est cruellement démodé et hors d’usage. La France se sent nation de second rang, ne rêve plus d’aucune conquête, ne se sent plus appelée par les peuples asservis, ou ne peut pas raisonnablement s’imaginer qu’elle soit appelée par eux ; et quant au principe des nationalités, quelque idéaliste qu’elle soit, il lui a été trop terriblement funeste pour qu’elle ne l’ait pas quelque peu écarté de son cœur.\par
En d’autres termes, il n’y a plus ici de popularité à se faire avec la politique étrangère. La conséquence principale des événements de 1870 a été que la France ne s’est plus occupée du tout de politique étrangère. C’est à partir de 1871 qu’elle aurait dû s’en occuper plus que jamais ; mais ne s’intéressant qu’à ce qui est glorieux et non à ce qui est utile, elle ne s’est plus appliquée à la politique  étrangère du moment que la politique étrangère n’était plus matière de gloire et entretien de vastes desseins.\par
Il y a eu à cet égard comme une dépression intellectuelle et morale en France. Après avoir été la nation mégalomane, la France est devenue la nation, non seulement prudente, en quoi elle aurait bien raison, mais timorée et parlant bas. On se faisait une popularité vers 1840 en agitant les souvenirs de l’Empire et en jetant toujours, par métaphore, l’épée de la France au-delà des frontières. M. Mauguin, bien oublié, s’était fait une spécialité de ce jeu-là et y avait récolté presque de la gloire.\par
On se faisait une popularité, vers 1859 et un peu plus tard, en appelant l’Italie à l’indépendance et à l’unité et en réclamant le guerre contre la Russie pour délivrer la Pologne (Havin et Guéroult).\par
Tout cela est absolument passé. Il n’y a aucune popularité à se faire avec la politique étrangère. Un ministre des affaires extérieures, en France, est un ministre souterrain.\par
On voit donc bien qu’un homme politique qui veut être populaire, et aucun ne peut ne pas le vouloir, est comme forcé de se rabattre sur l’anticléricalisme comme sur son unique ressource. Le député ne peut même mettre en coupe réglée que  cela d’une façon sérieuse et lucrative. Les services rendus aux électeurs, les faveurs gouvernementales obtenues pour l’arrondissement ne font presque que blanchir, tant, si multipliés qu’ils soient, ils sont toujours infiniment disproportionnés à la demande, et tant, à chaque bienfait qu’il dispense, il peut dire comme Louis XIV : « Je me fais cent ennemis et un ingrat. »\par
C’est même précisément pour se faire pardonner le peu de services généraux ou particuliers qu’il a pu rendre, puisqu’on trouve toujours qu’il en a rendu trop peu, que le député se montre, d’autant, anticlérical résolu et opiniâtre, pour pouvoir dire : « Il est vrai, je n’ai pas rendu tous les services que j’aurais voulu rendre ; je n’ai pas donné autant que j’aurais désiré ; je n’ai pas fait à ma circonscription tout le bien que je souhaitais lui faire ; mais j’ai été si anticlérical ! »\par
L’anticléricalisme est la tarte à la crème que l’on prodigue quand on ne peut pas en donner une autre.\par
Quant au gouvernement, il n’a, pour se soutenir contre ses ennemis et pour étayer ou réparer sa popularité, rien autre chose que l’anticléricalisme. Hors de la guerre à l’Église, pas de salut. Ne pouvant donner au peuple ni la gloire, ni la prospérité, ni, jusqu’à nouvel ordre, les propriétés de la  bourgeoisie, ni une reconstitution sociale où le peuple se trouverait plus à l’aise, cette œuvre, peut-être impossible, souffrant au moins de très grandes difficultés ; il ne peut lui donner que des satisfactions de haine assouvie, et il faut bien qu’il les lui donne : il bat le clergé devant lui. Ce sont les {\itshape circenses} de notre temps. C’est le recours des Césars modernes.\par
Voyez-les tous, successivement, quand ils sentent le terrain chancelant, se diriger vers ce fort et, quand ils se sentent démunis, ramasser cette arme. M. Jules Ferry, très en faveur avant 1870, mais impopulaire depuis le siège de Paris, sentant du reste que son caractère difficile augmentait de jour en jour dans le Parlement le nombre de ses ennemis, brusquement, sans antécédent, sans entente, du reste, avec ses collègues du ministère, invente le fameux article VII et fait la campagne des décrets pour reconquérir d’un coup toute la popularité qu’il avait perdue, pour parcourir la France entière « à la Gambetta », et pour faire crier sur son passage : « Vive l’article VII », même « par les petits enfants » (c’est un mot de lui).\par
M. Waldeck-Rousseau, très mal accueilli à la Chambre lors de la constitution de son ministère, par suite de la double bizarrerie qu’il avait eue de mettre dans son ministère un socialiste pour  irriter le centre et le général de Galliffet pour exaspérer la gauche, entreprend tout aussitôt sa campagne anticongréganiste pour se faire une popularité et s’en fait une, en effet, en moins d’un instant ; et un ministère qui avait l’air de devoir durer deux semaines dure trois ans, uniquement sur la question anticléricale.\par
M. Combes, enfin, considéré unanimement comme borné, choisi, on ne sait dans quel dessein secret, par M. Waldeck-Rousseau, peut-être pour que le président du conseil ne fût pas, le cas échéant, un concurrent sérieux à la présidence de la République ; M. Combes, subi, on ne sait par quelle faiblesse, par M. Loubet, qui n’avait pour lui que le contraire de la sympathie ; M. Combes, ministre incapable, de l’avis et de l’aveu de tous, se maintient au pouvoir aussi longtemps, plus longtemps que M. Waldeck-Rousseau, malgré faute sur faute, malgré des collaborateurs aussi incapables que lui, malgré la délation employée systématiquement comme instrument de règne, uniquement parce qu’il est anticlérical résolu, entêté et brutal, que rien ne l’arrête dans la poursuite furieuse de ce dessein et précisément parce que, comme il l’a dit lui-même, « il n’a pris le pouvoir que pour cela » et qu’il est absolument incapable de voir autre chose dans le gouvernement  de la France et dans toute l’histoire moderne.\par
Et, non seulement il dure trois ans, mais il n’est jamais renversé, non plus que M. Waldeck-Rousseau, et c’est spontanément qu’il se retire, et personne ne peut assurer qu’en s’en allant il n’ait fait que prendre les devants et anticiper sur une disgrâce. « Ah ! qu’un anticlérical est dur à abattre ! »\par
Les gouvernements savent cela et que faire une campagne anticléricale, c’est prendre une assurance sur la vie et une assurance contre les accidents de voyage.\par

\begin{itemize}[itemsep=0pt,]
\item  — Mais cela n’est pas une ressource indéfinie.
\item  — Si, précisément, c’est une ressource éternelle. M. Henry Maret a dit spirituellement : « Radicaux, mes frères, ne solutionnez jamais la question cléricale ; vous vous ôteriez le pain de la bouche. » Le mot est piquant ; mais il est faux ; parce qu’on n’épuise jamais la question cléricale, attendu qu’elle est inépuisable.
\end{itemize}

\noindent J’ai montré qu’elle durera, {\itshape et avec une vivacité et une intensité toujours croissantes}, tant qu’il y aura un catholique en France, comme la lutte contre les Maures en Espagne a duré tant qu’il y a eu un Maure dans la Péninsule.\par
Il y aura un péril clérical en France tant qu’il y aura un clergé, puis tant qu’il y aura des croyants,  tant que la « mentalité romaine » ne sera pas éteinte. Et tant qu’il y aura ou qu’on affectera de croire qu’il y a un péril clérical, les campagnes anticléricales continueront et se succéderont les unes aux autres.\par
Il y aura des accalmies, comme il y en a eu, par cette seule raison qu’en France surtout on ne peut pas dire toujours la même chose, ni faire toujours la même chose, et que même l’anticléricalisme a son point de saturation ; mais le moment reviendra toujours où un gouvernement dans l’embarras, et j’ai montré qu’ils sont destinés à y être tous, viendra dire : « Le péril clérical renaît ; la réaction cléricale relève la tête » ; et ce sera toujours vrai ou toujours tenu pour exact.\par
Après les congréganistes on poursuivra les prêtres séculiers ; après les prêtres séculiers, les laïques croyants, tenus pour « jésuites de robe courte » ; après les jésuites de robe courte, tout père de famille qui aura trouvé ou cherché le moyen de faire donner à son fils une éducation autre qu’antireligieuse et athée ; après ceux-ci, les pères de famille qui auront donné eux-mêmes à leurs enfants une éducation de couleur désagréable au gouvernement.\par
J’ai dit en ne plaisantant qu’à moitié : la seule solution efficace de la question cléricale, c’est  d’interdire aux hommes qui ne pensent pas comme le ministre de l’instruction publique d’avoir des enfants. Et, en effet, la question cléricale n’étant pour les anticléricaux qu’une question électorale, ce que les anticléricaux poursuivront toujours, infatigablement, ce sont ceux, quels qu’ils puissent être et quels qu’ils doivent être, qui leur prépareront des électeurs adverses.\par
Dans ces conditions, la guerre anticléricale est éternelle ou, du moins, de nature à se prolonger au-delà de toutes les prévisions possibles. Un anticlérical me disait : « Je ne crois pas, tout de même, que nous en ayons pour plus d’un siècle. »\par
Il a été dit, au cours de la discussion sur la loi de la séparation, que si les socialistes donnaient dans cette loi et avec ardeur, ce n’était pas seulement par conviction ; mais parce que, la question cléricale étant résolue, les radicaux, leurs ennemis non avoués, mais réels, ne pourraient plus amuser le peuple avec la bataille anticléricale et seraient acculés aux réformes sociales et obligés enfin de se prononcer sur elles, obligés enfin à devenir socialistes pratiques ou à devenir impopulaires.\par
Si les socialistes ont raisonné ainsi, ils ont raisonné très mal, ou ils n’ont raisonné qu’à moitié bien. Ils n’ont raisonné juste que pour un temps  très court, infiniment court, ce qui est une manière encore de prendre l’ombre pour la proie. Oui, sans doute, pour quelque temps, et encore je ne sais, et encore je ne crois pas, la séparation de l’Église et de l’État {\itshape paraîtra} une solution et permettra aux socialistes de dire aux radicaux : « Et maintenant, voyons si, l’anticléricalisme ôté, vous avez quelque chose dans votre sac. » Mais, d’une part, les essais de socialisation tentés par les socialistes avec concours timide, gauche et de mauvaise grâce des radicaux, étant probablement destinés à mal aboutir, et, d’autre part, le cléricalisme « renaissant », puisqu’il renaît toujours d’une façon ou d’une autre ; les radicaux, enchantés de s’évader, crieront tout du haut de leur tête : « le péril clérical est toujours là, il y est plus que jamais depuis que la séparation a donné à l’Église une vie nouvelle, et nous nous occupons d’autre chose ! Courons au péril clérical ! »\par
Et il n’y a aucun doute, étant donnée la mentalité française, que les radicaux ne soient écoutés et que cette diversion ne réussisse presque immédiatement.\par
Sans doute, les socialistes, qui comprendront très bien le secret du jeu et qui le comprennent déjà parfaitement, ne manqueront pas de dire, pour sauver leur influence : « Mais, faisons, certes, les  deux choses à la fois ; et d’un côté matons l’Église et de l’autre opérons de profondes réformes sociales ; il n’y a aucune incompatibilité entre ces deux œuvres ». Seulement, un peu partout, et surtout en France, il n’y a jamais mouvement général pour faire deux choses à la fois. Un gouvernement peut faire plusieurs choses à la fois ; un peuple ne poursuit pas à la fois plusieurs objets ; et longtemps, peut-être toujours, la diversion sur le cléricalisme fera son effet et amusera les masses.\par
Voilà les raisons pourquoi et partis politiques et gouvernements seront longtemps encore comme forcés de revenir périodiquement à l’anticléricalisme comme à une ressource précieuse et comme à une condition d’existence. Ce {\itshape dada} est le cheval de Troie ; on s’y cache pour vaincre et même on s’y cache parce qu’on ne peut pas faire autre chose.\par
Et remarquez que ce ne sont pas seulement les gouvernements et les politiciens qui ont le plus grand intérêt du monde à tenir toujours pendante, toujours actuelle, la question du cléricalisme, à la tenir, pour ainsi dire, toujours sur le feu. C’est la bourgeoisie française elle-même, non politicienne, non politiquante et qui aimerait à ne s’occuper aucunement de politique.\par
Je ne comprends pas du tout comment un certain  nombre de publicistes, dont le dernier en date est M. Paul Seippel (\emph{Les Deux Frances et leurs origines historiques}) s’obstinent à répéter cette vieille vérité que la bourgeoisie française s’est jetée entre les bras de l’Église pour y trouver une défense contre la Révolution, pour appuyer aussi et soutenir ceux qui peuvent museler la Révolution et l’endormir.\par
C’est une vieille vérité et c’est une erreur d’aujourd’hui. Ç’a pu être vrai, mais c’est faux actuellement plus qu’à moitié, plus qu’aux trois quarts. La bourgeoisie française tout entière, depuis ses plus hautes régions jusqu’à celles de la toute petite bourgeoisie possédante, n’a qu’une peur et n’a qu’une antipathie. Et le seul objet de cette antipathie et de cette peur, c’est le socialisme. Par conséquent, elle n’a ou ne croit avoir qu’un intérêt : détourner le peuple des préoccupations socialistes, dériver les passions populaires aussi loin que possible du socialisme, amuser le peuple avec quelque chose qui ne soit pas le socialisme.\par
Or, avec quoi l’amuser ? Avec une de ses passions. Il n’en a que deux, l’abolition de la propriété individuelle et la haine du curé. C’est donc exclusivement, et l’on n’a pas le choix, avec la haine du curé qu’il faut détourner son attention et le divertir. La bourgeoisie secoue la robe noire  devant le peuple comme le toréador secoue la cape rouge devant le taureau.\par
C’est le procédé du Sénat romain. Dès que le peuple demandait un peu impatiemment une réforme sociale, le Sénat lui montrait un peuple à conquérir ou un roi étranger menaçant. De même, M. Waldeck-Rousseau montrait au peuple le milliard des congréganistes à partager, milliard qui s’est réduit en définitive à rien, si l’on en juge par ce fait que les impôts n’ont pas diminué, mais se sont accrus ; et de même M. Clémenceau dénonce tous les matins Rome — « Rome l’unique objet de son ressentiment » — comme la dangereuse puissance étrangère qui est toujours sur le point de conquérir la France, de l’opprimer et de la réduire en servitude.\par
« Plaise à Dieu, dit soir et matin la bourgeoisie française, que Rome soit longtemps, soit toujours l’unique objet du ressentiment du peuple français. C’est elle qui est entre le peuple et nous. C’est elle qui nous couvre. Tant que le peuple regardera avec colère du côté du monastère et du côté de l’église, il ne regardera pas trop du côté de nos propriétés ou du côté du grand livre. Enivrons-le d’anticléricalisme. C’est un stupéfiant qui ne nous coûte rien et qui nous donne la sécurité. Anticléricalisme, amusement du peuple et tranquillité des bourgeois. »\par
 Cela ne peut pas durer indéfiniment, dira-t-on. Non, sans doute ; mais c’est bien précisément parce que cela ne peut pas durer toujours que la bourgeoisie tient à ce que cela dure le plus possible ; et c’est bien précisément pour cela qu’elle « fera de l’anticléricalisme » jusqu’à épuisement absolu de la question cléricale et pour ainsi dire par-delà ; c’est bien précisément pour cela qu’elle la fera renaître de ses cendres et qu’elle affirmera toujours qu’elle existe toujours, et plus grave et plus redoutable que jamais ; c’est bien précisément pour cela qu’après avoir dénoncé comme péril clérical les congrégations, qu’après avoir dénoncé comme péril clérical le Concordat, qu’après avoir dénoncé comme péril clérical le clergé séculier ultramontain, elle dénoncera comme péril clérical le clergé séculier le plus particulariste et le plus gallican du monde ; elle dénoncera comme péril clérical les laïques les plus laïques, pourvu qu’ils aient des sentiments religieux, et les libres penseurs, oui, les libres penseurs les plus libres, pourvu qu’ils émettent la prétention de faire élever leurs enfants comme ils voudront, et les libres penseurs sans enfants, les plus libres — croyez que j’en sais quelque chose — pourvu qu’ils soient libres penseurs de telle manière — manière assurément très rare — qu’ils  revendiquent la liberté pour tout le monde.\par
L’anticléricalisme, mais voyez donc que c’est les dernières cartouches de la bourgeoisie. Elle en usera tant que la cartouchière ne sera pas vide, et ensuite elle fabriquera des cartouches tant qu’il y aura quelque chose à sa portée qui ressemblera à de la poudre.\par
Comprenez donc la suite des choses : La bourgeoisie lance le peuple contre un ennemi autre qu’elle tant que cet autre ennemi existe ou qu’elle peut faire croire qu’il est. Très longtemps, pendant trois quarts de siècle, elle a excité, soulevé et lancé le peuple contre l’aristocratie, contre « les nobles ». L’aristocratie n’existait plus ; les nobles existaient, mais n’avaient absolument aucun privilège et ne constituaient aucun danger. N’importe : en vertu de cette loi qui fait qu’une génération poursuit de sa haine les descendants de ceux dont a souffert la génération d’il y a cent ans, la bourgeoisie obtenait du peuple qu’il détestât les descendants ou les pseudo-descendants des Lauzun ou des Montmorency. C’étaient « les morts qui parlaient » par la bouche de leurs arrière-neveux. Tel un protestant de 1905 ne peut pas voir un catholique sans avoir la fièvre de la Saint-Barthélemy et sans croire obscurément que ce catholique qu’il rencontre a pris part à ce massacre.\par
 Cependant l’aristocratie et les nobles et les ci-devant, cela a fini par s’user. Par habitude, les bourgeois daubent encore sur le théâtre la noblesse française ; mais cela a peu de retentissement. Il fallait chercher ou retenir autre chose.\par
Le clergé catholique et le catholicisme en général étaient la dernière ressource ; ils sont encore la dernière ressource. Exploiter contre eux le ressentiment séculaire du peuple contre les abbés trop gras et les évêques trop riches du \textsc{xviii}\textsuperscript{e} siècle, c’est le dernier jeu que peut jouer la bourgeoisie pour détourner les regards du peuple loin des gras d’aujourd’hui et loin des riches de maintenant. Littéralement, pour la bourgeoisie française actuelle, hors de la haine de l’Église point de salut.\par
On peut donc croire qu’elle ne cessera de suivre cette tactique qu’à la dernière extrémité.\par
Quand ? Comme pour l’aristocratie et la noblesse : lorsqu’il n’y aura plus d’Église et plus de catholiques, comme il n’y a plus maintenant, bien visiblement, à n’en pas douter, d’aristocratie ni de nobles. La limite aura été la même dans les deux cas.\par
Or il y aura une Église, et peut-être plus forte, plus inquiétante comme probabilité de survie, tant qu’on n’aura pas tout simplement interdit le culte catholique en France ; et il y aura des catholiques  toujours « menaçants », c’est-à-dire toujours désireux de s’entendre entre eux, c’est-à-dire de créer une Église et toujours sur le point de la fonder, tant qu’il y aura des catholiques et tant qu’on ne les aura pas chassés du territoire français, comme le voulait Rousseau ; tout au moins tant qu’on ne leur aura pas ôté toute liberté, non seulement d’enseignement, mais d’association, de parole, de presse et d’écriture.\par
On voit comme se prolonge devant nous, sans qu’on exagère rien et tout simplement en calculant ce que devra forcément vouloir la bourgeoisie et ce qu’elle fera si le peuple la suit, la période de lutte, de bataille et de persécution sans merci contre le catholicisme de France.\par
A mon sens, les choses se passeront à peu près de la manière suivante : abolition de l’Église catholique considérée comme un « État dans l’État », précisément parce qu’elle ne sera plus liée à l’État, comme une association trop cohérente et trop forte, c’est-à-dire comme une « aristocratie », analogue à la franc-maçonnerie, mais moins agréable.\par
Puis longue persécution contre les églises locales, fragmentaires, isolées, les églises « au désert », qui auront réussi tant bien que mal à se former et qui seront considérées comme des sociétés  secrètes, ce que, du reste, elles seront parfaitement, par impossibilité d’être autre chose.\par
Puis longue persécution contre les catholiques isolés qui auront survécu à toutes les vexations et rigueurs et qui seront la dernière proie ou le dernier ennemi signalé au peuple par la bourgeoisie cramponnée à sa dernière ressource.\par
Et ce sera, réalisé, le rêve du doux Edgar Quinet, qui n’a jamais rien vu autre chose en politique et dans toute l’histoire moderne que le catholicisme à exterminer par tous les mêmes moyens que le christianisme vainqueur avait employés pour exterminer la religion païenne ; et cette politique est d’une grande simplicité, encore qu’elle ne soit pas évangélique.\par
Peut-être avant cette troisième période, peut-être avant la seconde, la bourgeoisie aura été balayée par le peuple. Mais, pour se placer dans cette hypothèse, il faut supposer le peuple français devenu collectiviste, ce qu’il devient, sans doute ; mais avec une extrême lenteur, pour toutes sortes de raisons, et ce qu’il ne sera peut-être jamais.\par
Il est à croire que le stade anticlérical qui reste devant nous et qu’on peut présumer qui sera parcouru tout entier, est extrêmement long encore.\par
 
\asterism

\noindent A le parcourir, la France continuera à s’affaiblir de plus en plus. L’anticléricalisme lui a fait un mal énorme ; il continuera à lui en faire un qui est difficilement calculable.\par
A parler en général, d’abord, l’anticléricalisme a sur une nation, et sur une nation particulièrement nerveuse, tous les effets d’une passion exclusive sur un homme nerveux. Elle le détourne tout simplement de tout. L’homme passionné pour les femmes ou pour le jeu ou pour l’alcool, ne songe à rien qu’à l’alcool, au jeu ou aux femmes. Il ne néglige pas ses affaires : il ne les connaît plus ; elles n’existent plus pour lui. Il est hypnotisé.\par
Chose très remarquable et très vraie, que j’ai parfaitement observée, l’homme passionné, non seulement est en proie matériellement à sa passion et n’y peut point résister ; mais il s’en fait une espèce de religion : il la spiritualise et il la {\itshape mystifie}, si l’on me permet d’employer le mot dans son acception étymologique, ou, si vous voulez, il la {\itshape mysticise}. Propos de Don Juan au premier acte de la pièce de Molière, propos du Joueur dans Regnard, propos de Thausettes dans la \emph{Denise} de Dumas fils : « Oh ! la sensation !… » — J’ai connu  des alcooliques qui parlaient des ivrognes avec vénération, de leurs souvenirs d’orgies avec tendresse, des libations avec une sorte d’extase, et qui disaient : « Cet homme n’aime rien : il n’aime même pas boire ! »\par
L’anticléricalisme n’est pas une passion d’une autre nature ; il est une passion ; il est d’ordre pathologique. J’ai entendu un homme du peuple, qui n’était pas méchant, peut-être, mais qui avait sa conception de la liberté, dire : « La liberté ! Elle est jolie, la liberté ! On n’a pas seulement le droit de tuer un curé. » La passion parle là toute pure.\par
Eh bien, la passion anticléricale rend la France insensible à tout, excepté à l’anticléricalisme. Elle lui ferme les yeux sur ses intérêts, sur ses affaires, sur ses droits (je n’ai pas besoin de dire sur ses devoirs) et sur ses moindres commodités. Elle l’aveugle et la paralyse. A l’étranger on dit : « Qu’est-ce que c’est que la France ? — C’est un pays où l’on ne s’occupe que du Vatican. »\par
Un mot de M. Ribot est admirable, ou du moins excellent. Il parlait sur une question très importante. On lui cria : « Et l’affaire Dreyfus ? » L’affaire Dreyfus était précisément une affaire très mêlée d’anticléricalisme. M. Ribot, personnellement, était plutôt du côté des « Dreyfusistes ». Il  répondit : « J’ai mon opinion sur l’affaire dont vous me parlez ; mais je n’admets pas que toute la politique de la France pivote sur l’affaire Dreyfus. » C’est précisément la sottise de la France que pour elle toute la politique roule sur le cléricalisme et qu’elle n’admette pas qu’il y ait autre chose dont on ait à s’occuper.\par
J’ai entendu souhaiter que le catholicisme disparût, pour que la France, n’ayant plus à se passionner pour ou contre lui, s’avisât de s’occuper de ses affaires.\par
Le malheur, c’est que pendant que nous cédons à ce travers très français et à cette habitude très française de mener notre barque comme si nous étions seuls au monde, les étrangers profitent de nos fautes et de notre aveuglement et de notre hypnotisation pour faire leurs affaires à notre détriment. Il est excellent pour eux qu’il y ait un peuple en Europe qui se conduise comme celui qui a les yeux bandés au jeu de colin-maillard, pendant que tous les autres ont les yeux ouverts. En 1905 nous fûmes tout près d’avoir la guerre avec l’Allemagne. Qu’il fût expédient de la faire ou très inopportun de l’entreprendre ou de l’accepter, c’est ce qui peut être matière à discussion.\par
Un parti, en France, la désirait, considérant que l’Allemagne était très isolée et qu’elle pouvait  ne l’être pas toujours, et c’était la politique personnelle du ministre des affaires étrangères, peut-être du président de la République.\par
Un parti la repoussait de toutes ses forces, considérant, non sans raison, que peut-être, malgré nos bons rapports avec l’Italie et l’Autriche, l’Allemagne était moins isolée qu’on ne croyait et qu’à coup sûr notre principal allié, la Russie, nous manquait absolument, et qu’enfin l’Angleterre ne nous servirait pas à grand’chose et qu’en résumé c’était plutôt le bon moment pour l’Allemagne que pour la France.\par
Mais là n’était pas vraiment la question. La question essentielle était de savoir si nous étions prêts. Or, on peut le dire maintenant, de tous les propos qui s’échangèrent au conseil supérieur de la guerre, de tous les rapports qui parvinrent des commandements au gouvernement, de toutes les confidences qui furent faites, et j’en ai reçu, il résulta que nous n’étions pas prêts le moins du monde.\par
A la vérité, nous ne le sommes jamais, c’est une chose à dire pour essayer de secouer l’incroyable force d’apathie et l’incroyable incurie dont nous sommes affligés depuis un demi-siècle. Mais, si nous ne sommes jamais prêts, en 1905 nous l’étions moins que jamais ; nous étions le contraire comme  jamais nous ne l’avions été ; nous l’étions à souhait.\par
Pourquoi ? Parce que nous avions eu pendant trois ou quatre ans un ministre de la guerre et un ministre de la marine qui en fait de marine et de guerre n’avaient songé qu’à « combattre la réaction et le cléricalisme » dans les armées de terre et de mer et qui n’avaient été choisis que pour cela.\par
Que l’anticléricalisme désorganise une armée, désorganise une marine, tarisse les approvisionnements, démunisse les magasins de munitions, laisse en déplorable état les forts de l’Est, découvre et ouvre la frontière, au premier abord cela paraît singulier. C’est un fait pourtant, parfaitement indiscutable, et c’est tout naturel.\par
Si Louis XIV, à l’époque de la révocation de l’édit de Nantes, avait disgracié Louvois comme insuffisamment partisan des Dragonnades et l’avait remplacé par un homme n’ayant pour tout mérite que d’être un jésuite, peut-être l’administration de l’armée aurait-elle souffert de cette mesure.\par
Comme toute passion, l’anticléricalisme est terriblement exclusif, et toute passion, quelle qu’elle soit, fait d’un homme ou d’un peuple un être qui perd jusqu’à l’instinct de sa conservation, de sa défense et de sa persévérance dans l’être. Le fanatisme antireligieux produit les mêmes effets que le fanatisme religieux pourrait produire et certainement  produirait. Tant que nous serons dévorés d’anticléricalisme, nous ne pourrons pas faire la guerre, même attaqués.\par

\astertri

\noindent En attendant, même en paix, nous nous diminuons de jour en jour. Nous perdons notre influence et notre prépondérance, au moins morale, en Orient, comme protecteurs reconnus des chrétiens d’Orient, et c’est merveille et il est très significatif comme M. Combes est indifférent à la perte de notre influence en Orient : « D’abord les deux questions [séparation de l’Église et de l’État et Protectorat des chrétiens d’Orient] ne sont pas nécessairement liées ensemble, l’une concernant nos rapports avec la Papauté, l’autre nos relations diplomatiques avec d’autres puissances. Mais je veux, sans m’arrêter à cette considération, envisager directement l’éventualité dont on cherche à nous effrayer. Si la croyance des siècles passés a attaché au protectorat une idée de pieux dévouement à la grandeur chrétienne ; si elle a servi notre influence à une époque de foi ; il s’est trouvé alors aussi, qu’on ne l’oublie pas, d’autres motifs, très positifs et très humains, qui ont contribué largement à faire attribuer à l’ancienne France un  privilège, glorieux, j’en conviens, dans l’esprit de ce temps, mais toutefois encore plus embarrassant que glorieux. Il fallait, pour l’exercer, une puissance militaire et navale de premier ordre. La France réunissait cette double condition. Notre pays a rempli honorablement les obligations découlant des capitulations et des traités, et il peut s’étonner à bon droit de la menace dont il est l’objet. Mais la Papauté s’abuse si elle s’imagine nous amener, par ce procédé comminatoire, à quelque acte de résipiscence. Nous n’avons plus la même prétention au titre de fille aînée de l’église, dont la monarchie se faisait un sujet d’orgueil pour la France, et nous avons la conviction absolue que notre considération et notre ascendant dépendent exclusivement aujourd’hui de notre puissance matérielle, ainsi que des principes d’honneur, de justice et de solidarité humaine qui ont valu à la France moderne, héritière des grandes maximes sociales de la Révolution, une place à part dans le monde. »\par
Impossible de mieux dire, ou peut-être de plus mal dire, mais de dire plus formellement : « Le protectorat de la France sur les chrétiens d’Orient, au fond je suis ravi d’en être débarrassé. Cela était bon, et encore c’était « embarrassant » d’une part quand nous étions chrétiens, d’autre part  quand nous étions forts. Mais vous ne tenez ni à être chrétiens ni à être forts. Donc advienne que pourra du protectorat ! Qu’il passe à une nation qui ait cette force militaire et cette force navale à laquelle nous ne tenons pas. Nous restons, nous, en contemplation devant les immortels Principes de 89. »\par
Il est difficile de perdre plus gaiement un protectorat.\par
Pendant ce temps-là, l’Allemagne, que nous nous obstinons sottement à considérer comme une puissance protestante, et qui est une puissance moitié catholique, moitié protestante, se tient en bons termes avec Rome, comme c’est son devoir et son intérêt, et s’achemine, et le chemin est à moitié fait, vers la conquête du protectorat des chrétiens d’Orient, qui est une affaire d’extension commerciale, autant qu’une affaire d’extension d’influence morale. Mais que ne sacrifierait-on pas à l’impérieux devoir de faire la guerre au Vatican ?\par
L’Orient, pour des Français, est bien loin ; mais à nos portes, il y a l’Alsace. Or, le résultat le plus clair de notre généreux anticléricalisme a été de nous faire perdre l’Alsace une seconde fois. Nous ne l’avions perdue que matériellement en 1870. Cela ne nous suffisait pas. Nous nous sommes attachés de tout notre cœur à la perdre moralement,  à nous l’aliéner. L’Alsace est profondément catholique, et de plus, ce qui n’est pas tout à fait la même chose et ce qui, dans l’espèce, est plus important, le clergé catholique a sur elle un très grand empire. Le curé alsacien est un petit roi dans son village. Et enfin le clergé catholique alsacien était profondément patriote et antiallemand.\par
En conséquence, considérant que l’Alsace est catholique, que son clergé catholique a de l’empire sur elle et que son clergé était antiallemand, nous avons mis tous nos soins à montrer à l’Alsace que nous étions les ennemis forcenés de la religion catholique et les ennemis enragés du clergé catholique. Nous avons mis tous nos soins à montrer à l’Alsace que nous étions, sinon ses ennemis, du moins, ce qui n’est pas si loin d’être la même chose, les ennemis de tout ce qu’elle aime.\par
Ce n’est peut-être pas très adroit. Cela rentre dans notre habitude, qui est de ne jamais nous occuper de ce qui se passe de l’autre côté de notre frontière. Cette habitude est française ; elle est particulièrement parlementaire et politicienne, les politiciens ne songeant qu’à être élus et réélus et les étrangers n’étant pas électeurs en France : « Que nous font les gens d’Allemagne ou d’Angleterre ? Ils ne votent pas. »\par
 Un député ou un aspirant député ne regarde jamais que dans sa circonscription. Aussitôt que l’Alsace a été matériellement allemande, il a été absolument impossible à un politicien français de s’en occuper.\par
Toujours est-il que de l’indisposer ou de la ménager nous n’avons pas eu la moindre cure, si ce n’est qu’on pourrait dire que plutôt nous avons apporté grande diligence à l’indisposer. Nous aurions voulu — et cela est peut-être la vérité et ce n’est pas ici le lieu de démontrer pourquoi, mais c’est peut-être la vérité et je pourrai le démontrer une autre fois — nous aurions voulu, de ferme propos, de dessein prémédité et de forte persévérance, nous aliéner l’Alsace et la pousser doucement du côté de l’Allemagne que nous n’aurions pas parlé, agi et légiféré autrement.\par
Et, bien entendu, pendant ce temps-là, car nous sommes peut-être les seuls en Europe qui depuis un demi-siècle agissions continuellement droit à contre-fil de nos intérêts, l’Allemagne qui, je le répète, n’est pas une puissance protestante, mais une puissance mi-partie protestante, mi-partie catholique, et qui n’est pas une puissance qui ait pris l’habitude d’agir contre ses intérêts, et qui n’est pas une puissance dénuée d’intelligence élémentaire, l’Allemagne comprenait très  bien qu’il fallait montrer la plus grande bienveillance à l’égard des catholiques alsaciens. L’empereur allemand multipliait — ces dernières années surtout, c’est-à-dire à mesure que l’anticléricalisme sévissait plus furieusement en France — les avances, les amabilités et les grâces à l’endroit des catholiques alsaciens.\par
L’Allemagne, il est vrai, parce qu’elle considérait cela comme une mesure nécessaire de rattachement et de centralisation, supprimait le séminaire de Strasbourg, pour que les jeunes clercs catholiques reçussent l’instruction de professeurs allemands de l’Université, ce qui, après tout, peut se défendre ; mais {\itshape dès 1872} elle élevait très fortement les traitements des ecclésiastiques catholiques ; elle les élevait une seconde fois en 1884. Les chanoines, qui en 1871 recevaient 1280 marks, en reçurent 1920 en 1872 et 2000 en 1884. Les curés reçurent une augmentation, suivant leur âge et leur ancienneté, de 500 marks ou de 700 marks. L’Allemagne a voulu qu’on dit : « Il n’y a qu’un pays français où le clergé catholique soit bien traité ; ce pays est en Allemagne. »\par
Les catholiques alsaciens ont été ménagés et caressés par l’Allemagne en proportion juste des mauvais traitements que la France faisait subir  aux catholiques français, et c’est de la politique élémentaire.\par
Les résultats favorables à l’Allemagne, je ne dirai point « ne se sont pas fait attendre », selon la formule, car, Dieu merci, l’Alsacien est entêté ; mais ils arrivent, et c’est maintenant l’affaire d’une génération et peut-être de beaucoup moins.\par
Le temps n’est plus où, confondant « Rome et la France », selon l’esprit bismarckien, un inspecteur prussien disait aux instituteurs : « On fêtera l’anniversaire de Sedan en souvenir de ceux qui sont tombés pour l’unité allemande ; on le fêtera pour rappeler que l’empire a devant lui dans la Papauté un second ennemi héréditaire » (1882). Non, beaucoup plus intelligemment, les catholiques allemands, en 1905, choisissent Strasbourg pour siège de leur Congrès annuel et entretiennent leurs coreligionnaires alsaciens, de quoi ? De la France considérée, avec quelque raison, comme l’ennemi du catholicisme, de tout clergé catholique et de toute croyance religieuse et de tout sentiment religieux.\par
Et les Alsaciens, que répondent ils ? Quelque chose qui déjà est très grave. Jusqu’à présent, au Reichstag, les catholiques alsaciens formaient groupe intransigeant et intangible. Ils représentaient la protestation. Ils se rencontraient quelquefois  dans leurs votes avec le centre catholique allemand ; mais ils restaient absolument indépendants de lui. Maintenant ils manifestent l’intention de s’unir au centre catholique allemand, de faire cause commune avec lui et presque de se confondre avec lui.\par
C’est plus qu’une nuance. C’est une démarche d’une portée considérable, d’une suite immense. L’Alsace devient allemande. Elle devient allemande catholique, mais elle devient allemande. Elle y voit surtout un intérêt local ; c’est évident ; elle veut avoir sa part efficace et réelle dans les délibérations de la grande assemblée allemande ; mais elle n’est plus aussi arrêtée qu’elle l’était par les scrupules de sa conscience traditionnelle et, pour ainsi parler, de sa conscience historique. Elle se fond peu à peu dans la partie catholique de l’empire allemand, mais enfin et tout compte fait dans l’empire allemand lui-même.\par
Voilà le fruit de la politique relativement habile et intelligente de l’Allemagne en Alsace, ou du moins voilà à quoi a contribué cette politique ; voilà le fruit de la politique intérieure — et extérieure malgré elle et sans qu’elle y songeât — de la France ; ou du moins voilà à quoi n’a pas mal contribué cette politique, la plus étroite et la plus aveugle que j’aie jamais rencontrée.\par
 
\asterism

\noindent C’est cette politique qu’il ne faut pas continuer. C’est de cette politique qu’il faut prendre exactement le contrepied. Il en faut prendre le contrepied pour toutes les raisons que j’ai dites et pour celle-ci, par laquelle je terminerai.\par
En régime despotique ou en état d’esprit despotique, ce qui revient à très peu près au même, {\itshape tout ce qui semble le défaut d’une mesure quelconque en est le principe}. Tout ce qui paraît vicieux, mauvais, dangereux, absurde dans une démarche du gouvernement despotique ou du parti despotique, en est le principe intime, la raison secrète, l’inspiration même, profonde, quelque fois inconsciente, consciente souvent, mais enfin en est la source.\par
Lorsque Louis XIV persécutait les protestants, si l’on avait dit, non pas peut-être à lui, mais à ses ministres, que cette persécution pouvait avoir pour effet de jeter un million de protestants à l’étranger, ils auraient sans doute répondu : « C’est précisément pour cela que nous les persécutons. Que la plèbe protestante émigre à l’étranger, nous n’y tenons pas du tout et nous nous efforcerons, même par la force, de la retenir (c’est ce qu’ils ont  fait). Mais l’émigration des chefs protestants et des sommités intellectuelles du parti, c’est la décapitation de la faction protestante, et c’est ce que nous souhaitons ; car ce que nous voulons, c’est qu’il n’y ait pas de partis en France. Nous savons, par l’exemple de l’Angleterre, qu’un parti religieux est toujours un parti politique et que les partis politiques décapitent les rois quand ils le peuvent, et nous aimons mieux décapiter les partis quand nous le pouvons. Nous disons comme Bossuet, qui est conseiller d’État : « … la loi ne permettait pas aux hérétiques de s’assembler en public, et le clergé, qui veillait sur eux, les empêchait de le faire en particulier, de sorte que la plus grande partie se réunissait [à l’église orthodoxe] et que les opiniâtres mouraient sans laisser de postérité, parce qu’ils ne pouvaient ni communiquer entre eux ni enseigner librement leurs dogmes. » Nous disons comme Bossuet : « Nous avons vu… leurs faux pasteurs les abandonner sans même en attendre l’ordre et heureux d’avoir à leur alléguer {\itshape leur bannissement} pour excuse. » Nous disons comme Bossuet : « Quelque chose de plus violent se remue dans le fond des cœurs [des hérétiques], c’est {\itshape un dégoût secret de tout ce qui a de l’autorité} et une démangeaison d’innover sans fin, après qu’on en a vu le premier exemple. »  Nous disons comme Bossuet : « Il ne faut pas s’étonner s’ils perdirent le respect de la majesté et des lois ni s’ils devinrent {\itshape factieux}, rebelles et opiniâtres… On ne laisse plus rien à ménager aux peuples quand on leur permet de se rendre maîtres de leur religion, et c’est de là que nous est né ce prétendu règne du Christ qui devait {\itshape anéantir toute royauté et égaler tous les hommes}. » Nous nous disons tout cela et qu’il n’y a donc aucune différence sensible entre un protestant et un factieux, entre un protestant et un républicain, entre un protestant et un anarchiste, et si l’on nous reproche de rejeter au-delà des frontières les maîtres d’anarchie, nous répondons que c’est précisément là ce que nous voulons. »\par
Lorsque Louis XV, après Louis XIV, persécutait les jansénistes, si on lui avait dit que, tout compte fait, les jansénistes étaient ce qu’il y avait de plus pur, de plus élevé, de plus noble et de plus croyant dans la religion catholique et que le jansénisme, plus ou moins bien compris, était la façon de croire de la haute bourgeoisie française et du monde parlementaire, partie singulièrement recommandable de la nation, il aurait répondu, s’il avait pu prendre sur sa nonchalance de répondre : « C’est précisément pour cela que je réprime et combats les jansénistes et que je n’en  veux plus. Ils sont comme l’âme d’une partie de la nation qui est trop indépendante pour moi et trop puissante et trop vivante, qui n’est pas une simple poussière d’hommes, qui s’entend, qui se comprend, qui se sent vivre en commun, dont les différents éléments et les différents groupes ont des intelligences entre eux, qui par conséquent forme presque une association, et l’on sait bien qu’une association est un État dans l’État, ce qui est insupportable. Et c’est précisément parce que le jansénisme est la religion de la haute bourgeoisie indépendante et du monde parlementaire, toujours sur le point d’être rebelle, que je ne veux point de jansénisme et que je ne veux qu’un peuple docile, ayant simplement la religion du confesseur du roi et ne s’avisant pas de vouloir « être maître de sa religion ». Que les jansénistes aient pour eux la haute bourgeoisie et le monde parlementaire, c’est ce qui les condamne. »\par
Si l’on avait dit à Napoléon I\textsuperscript{er} : « Point de liberté religieuse ; les prêtres asservis aux évêques et les évêques asservis à vous ; voilà quelle est votre idée et voilà quel est votre système. C’est fort bien peut-être, à un certain point de vue ; seulement ce n’est pas une religion ; ce n’est pas du tout une religion. C’est une administration  générale de la morale publique et ce n’est rien que cela… » Il aurait répondu : « Eh bien ! si vous croyez que je veux une religion ! C’est précisément ce dont je ne veux pas du tout. Un pouvoir spirituel, n’est-ce pas ? L’empire partagé, le gouvernement partagé ! Quelqu’un commandant aux âmes et quelqu’un commandant aux corps, c’est-à-dire chacun de mes sujets coupé en deux ! Non pas, s’il vous plaît ! Moi seul gouvernant, je ne sors pas de là et je n’entends pas à autre chose. Alors détruire la religion chrétienne ? Point du tout, s’il vous plaît encore. Ceux qui ont rêvé cela étaient des sots. Ils ne savaient pas qu’on ne détruit pas les religions tout d’un coup, ni même vite. Elles ne meurent que de vieillesse, d’épuisement de leur principe vital. Tant qu’elles n’ont pas complètement perdu ce principe, en essayant de les détruire on le leur rend. Tant qu’elles ne sont pas mortes, en les tuant on les ressuscite. Il faut savoir cela. C’est du reste élémentaire. Il faut être un avocat pour l’ignorer ou le méconnaître. Non ! non ! Je ne veux pas détruire la religion. Seulement je veux qu’elle n’existe pas. Elle n’existera pas, sans que rien soit fait pour la détruire, si l’Église est organisée de telle sorte qu’elle ne puisse pas et qu’elle ne veuille pas enseigner la religion. J’y mettrai ordre. L’Église sera tellement attachée à  moi, rivée à moi, qu’elle enseignera la religion dans les limites où la religion ne me gênera pas et ne me contredira pas. Dès lors, ce qui sera enseigné sous couleur de religion, ce sera bons propos de morale courante et bonnes vieilles histoires attendrissantes relativement à Jésus et aux martyrs. Vous me dites que ce n’est guère une religion et que même ce n’en est pas une. Je l’espère bien, et c’est justement ainsi que je l’entends. Ce que vous me signalez comme le défaut de mon système en est le principe. »\par
Si l’on avait dit à Napoléon : « Ni liberté de parole, ni liberté de presse ni liberté d’enseignement, ni liberté d’association : c’est bien votre pensée. Elle peut être soutenue. Il y a pourtant à tout cela cet inconvénient qu’une nation vit de liberté, qu’elle ne tient à elle-même qu’en raison des libertés dont elle jouit et dans l’exercice desquelles elle se sent vivre, qu’elle prend conscience d’elle-même dans cet exercice et qu’à n’en plus avoir l’usage elle s’abandonne, s’endort, languit, n’est plus une nation, à moins que, tout entière à l’action extérieure, elle ne bataille et conquière sans cesse, ce qui ne peut point, sans doute, être régime éternel » ; il aurait répondu probablement : « Précisément ce que je ne veux pas, c’est que la France soit une « nation », une nation comme vous l’entendez,  vous, avec vos propos d’idéologue. « Une nation ! » J’entends bien : une nation comme l’Angleterre ou la Hollande, une nation distincte de son gouvernement et ayant réellement une vie propre en dehors de ceux qui la régissent. C’est cela que vous appelez une nation. C’est ce que je ne veux pas que soit la France. Je veux qu’elle vive en moi, de ma pensée qu’elle épousera et de ma volonté à laquelle elle s’associera. D’elle-même, de ses pensées à elle, multiples et diverses, jamais ! De ses volontés multiples et divergentes, jamais ! Vous me dites : « Dans ces conditions, elle s’affaissera sur elle-même, à moins qu’elle ne combatte et conquière toujours. » Tout juste ! Comme Rome. Un peuple de commerce, de science, de lettres et de beaux-arts peut avoir besoin de libertés ; un peuple se destinant à la conquête du monde et à l’administration du monde, non seulement n’en a pas besoin, mais en serait gêné dans son œuvre. Si jamais j’accepte les libertés, c’est que j’aurai renoncé à mon rôle de conquérant et de César travaillant par la guerre à la pacification future du monde, à la {\itshape pax romana per orbem}. Il est possible que ce moment de renoncement arrive. Alors ce ne seront point mes idées qui auront changé, ce sera mon dessein. Mais tant que mon dessein sera celui que vous me voyez, ne me parlez pas d’une « nation ». Parbleu ! ce que je veux  autour de moi, ce n’est pas une nation ; c’est une armée. Ce que vous me signalez comme le défaut de mon système en est le principe. »\par
Tout de même, de nos jours, où l’instinct despotique reparaît sous une nouvelle forme et sévit de toutes ses forces, ce qui paraît le défaut de chaque mesure, générale ou particulière, dont on se plaint, en est le principe inspirateur, que ceux, du reste, qui prennent cette mesure s’en aperçoivent ou qu’ils ne s’en aperçoivent point.\par
A quoi songerons-nous pour prendre des exemples ? Au système parlementaire ? Soit. Qu’est-ce qu’on lui reproche ? D’être, de la manière dont il est organisé, le contraire même du système représentatif ; d’arriver, par toute suppression de la représentation des minorités, à ce résultat que le pays est gouverné par une majorité toute factice et, au vrai, par une minorité. C’est ce qu’on a appelé, très justement, le mensonge du gouvernement parlementaire.\par
Car enfin s’il est prouvé, et il l’a été, qu’en tenant compte de toutes les voix exprimées par le pays, les groupes de gauche qui gouvernent depuis huit ans représentent la minorité du pays, il est démontré que la façon de compter est mauvaise et que la France est gouvernée depuis huit ans par ceux qui devraient obéir, ou tout au moins ne pas  commander. C’est le mot d’un étranger aristocrate : « En choses de science, et la politique est une science, il faut peser les suffrages et non les compter. Vous, vous comptez au lieu de peser ; — mais encore vous comptez mal. »\par
Sans doute ; mais remarquez que cette façon de compter est éminemment démocratique et radicale. Si vous tenez compte des voix des minorités, qu’est-ce que vous faites ? Vous tenez compte, dans chaque circonscription, des voix de ceux qui n’obéissent pas « aux grands courants », qui ne suivent pas la foule, qui ne sont pas, comme dit Nietzsche, « bêtes de troupeau ». Eh ! mais ! justement, ces voix, il faut les supprimer ! Ce sont des suffrages fortement suspects d’être aristocrates. Ils se désignent eux-mêmes, en quelque sorte, comme étant tels. Supprimons-les ; tenons-les comme n’existant pas.\par
Autre aspect de la même question : Si (par exemple) décidant que sera déclaré député tout homme qui, {\itshape dans tout le pays}, aura réuni tel nombre de suffrages suffisant pour être élu dans une circonscription moyenne, vous amenez ainsi à la Chambre des hommes qui n’auraient été élus dans aucune circonscription, mais qui sont connus, aimés et admirés un peu partout dans le pays tout entier, que faites-vous ? Vous tenez compte du suffrage des  minorités, évidemment ; mais vous amenez à la Chambre des hommes beaucoup trop connus, beaucoup trop admirés et beaucoup trop aimés, des espèces d’hommes supérieurs, des illustrations politiques, scientifiques, littéraires. Or, l’homme supérieur n’est pas chose démocratique. Il représente cette sélection intellectuelle dont la démocratie a horreur et terreur, non sans raison ; il est l’homme, qu’au contraire de l’élire ou de le considérer comme élu, la démocratie devrait éliminer par ostracisme.\par
Il ne faut donc point de représentation des minorités. De quelque façon qu’on la mette en pratique, elle servira, au moins un peu, les intérêts aristocratiques. Il ne s’agit pas de dire, sottement, que toutes les voix des minorités se portent sur des gens qui appartiennent à une élite ; non ; mais les gens qui appartiennent à une élite trouvent le plus grand nombre des suffrages qui vont à eux dans les bulletins des minorités. En rayant de compte tous les bulletins des minorités, on éliminera donc toujours un certain nombre de gens d’élite. Le suffrage universel n’est vraiment démocratique qu’à la condition d’être brutal. Conservons-lui, sous des apparences très légales et très légitimes, son caractère de brutalité.\par
Ainsi raisonnent les démocrates\footnote{Il y a des exceptions : M. Jaurès entre autres (peut-être parce qu’il est une supériorité intellectuelle) est partisan de la représentation des minorités.}, avec un très  grand sens, si l’on se place à leur point de vue. Ils sentent vaguement, peut-être avec précision, que le prétendu défaut du système en est le principe ou du moins est très conforme à l’esprit même du système.\par
Qu’est-ce qu’on reproche encore au système parlementaire français ? De réduire à rien la fameuse « division des pouvoirs » et de concentrer, au contraire, tous les pouvoirs dans le Parlement, de légiférer, de gouverner et d’administrer, le tout ensemble. Eh bien mais, ce n’est pas pour autre chose que les politiciens se font nommer députés. Ils ne se font pas nommer députés pour faire des lois — si ce n’est des lois de circonstance, qui sont précisément des actes de gouvernement et de gouvernement despotique — ils se font nommer pour gouverner, par l’intermédiaire de leurs ministres, d’une manière conforme à leurs intérêts ; et, d’autre part, pour peser, chacun chez eux, sur l’administration de leur département et pour y être de petits rois. Sans cela, ils ne tiendraient pas le moins du monde à être élus députés. Le défaut ou l’abus du système parlementaire que vous leur signalez est pour eux l’essence même du système parlementaire et sa principale, sinon sa seule raison d’être.\par
 Le système parlementaire, direz-vous encore, tel qu’il est pratiqué en France, a de singuliers résultats de temps en temps et même toujours : il met un avocat à la marine, un homme de lettres à la guerre, un financier à la justice, un commerçant à l’instruction publique et ainsi de suite. Quoi de plus naturel, puisqu’il s’agit, pour un ministère, non pas de bien administrer, mais de ne pas être renversé, et que dès lors ce qu’il doit chercher c’est à distribuer les portefeuilles, non de manière qu’ils soient bien tenus, mais de manière que les différents groupes de la chambre aient, chacun, à peu près satisfaction ? Montesquieu disait : « En tel cas… la République est une dépouille. » En style moins noble on peut dire : « En France le ministère est un gâteau. » Quand il s’agit de partager, il ne s’agit pas d’attribuer. On partage comme on peut. De là ces attributions singulières, amusantes et parfaitement désastreuses pour les intérêts du pays.\par
Mais quoi ? le principe pour les politiciens, c’est de partager le pouvoir et l’influence. Les politiciens français sont toujours les fils de ces conventionnels qui n’admettaient pas que la France cessât jamais d’être leur propriété et qui se prorogeaient eux-mêmes dans les assemblées qui devaient succéder à la leur et qui écartaient par la proscription ceux  qui y pénétraient à leur tour ; pour qui enfin la République n’était qu’un syndicat de propriétaires de la République. De là un système parlementaire qui, dans la pratique, est le contraire même de la définition du système parlementaire.\par
Voyez encore quelques mesures récentes ou quelques projets en voie de réalisation. La caisse de retraite pour l’invalidité et la vieillesse est un projet, certes, excellent en soi. Quelques-uns lui adressent pourtant cette critique : « Qui profitera de ces secours ou de ces retraites ?\par

\begin{itemize}[itemsep=0pt,]
\item  — Les invalides et les vieillards.
\item  — Sans doute ; mais lesquels ?
\item  — Ceux que nous estimerons en avoir le plus besoin. A ceux-là nous épargnerons tout délai dans la liquidation de leur retraite. Pour les autres on saura voir…
\item  — C’est-à-dire que vous faites la loi pour vos protégés et vos clients ou ceux dont vous voudrez faire vos clients et vos protégés. C’est donc un simple {\itshape instrumentum regni}, une sportule à distribuer à titre, soit de récompense, soit d’encouragement, un moyen de vous conquérir des électeurs. N’est-ce pas cela ? » — A certaines dispositions des projets en discussion, cela en a l’air. Pour que tout soupçon de ce genre fût écarté, il faudrait que les caisses de retraite pour l’invalidité ou  la vieillesse ne fussent pas « d’État », fussent aux mains de mutualités parfaitement indépendantes de l’État, auxquelles l’État, cette fois incontestablement par humanité pure et sans pouvoir être soupçonné d’intérêt politique, accorderait toutes les subventions qu’il voudrait. Mais faites adopter ce système à nos hommes politiques. Je doute que vous y réussissiez. C’est très vraisemblablement pour se créer un instrument de règne de plus qu’ils organisent cette administration de secours. Ce qui est pour vous, ce qui est en soi le défaut de la mesure en est pour eux la raison d’être.
\end{itemize}

\noindent L’impôt sur le revenu n’a rien, assurément, que de très acceptable ; mais on fait remarquer qu’à moins de se contenter d’une déclaration, qui, hélas, étant donnée la nature humaine, serait toujours fausse, l’impôt sur le revenu ne pourrait s’exercer qu’avec une inquisition continuelle sur les sources de revenus de chacun, ou par une fixation tout arbitraire et faite pour ainsi dire au hasard.\par
Ou se contenter de cette affirmation du contribuable : « Mon revenu ? Il est de tant. »\par
Ou fouiller, et sans cesse, dans tous les papiers d’affaires et même dans la correspondance du contribuable.\par
Ou s’en rapporter aux signes visibles et extérieurs de la fortune.\par
 Ou taxer au hasard le contribuable.\par
Je ne crois pas qu’on puisse sortir de ces quatre partis.\par
Or le premier est purement vain ; il ne rendrait rien du tout ou quasi rien. — Le second, sans compter qu’il est épouvantablement vexatoire, est impraticable. Il exigerait une armée de commis, douaniers à l’intérieur et douaniers domestiques, plus nombreux que celle des contributions indirectes. — Le troisième est trompeur : les signes extérieurs de la fortune ne signifient rien, l’avare ne manifestant sa fortune par aucun signe extérieur et l’homme placé dans une certaine situation qui exige de la représentation, du prestige et de la poudre aux yeux, montrant des signes extérieurs de fortune, alors qu’il n’a pas de fortune du tout.\par
Reste donc la fixation arbitraire : « Nous supposons que Monsieur un tel est millionnaire. Nous le croyons. — Sur quoi le croyez-vous ? — Nous le croyons sur ce que nous le croyons. »\par
Or, ceci est du pur despotisme. Eh bien, il est à croire que c’est précisément parce qu’il n’y a, en impôt sur le revenu, que la taxation arbitraire qui soit pratique, que certain parti tient tellement à l’impôt sur le revenu. L’impôt sur le revenu sera un moyen de frapper qui déplaît et d’épargner qui plaît. C’est justement ce qui en fait le mérite aux  yeux d’un certain parti. Cela pourra avoir d’admirables conséquences électorales. Ici encore, ce qui est le défaut de la mesure en est le principe pour ceux qui la proposent.\par
Les idées du parti radical sur l’armée et l’organisation de l’armée sont exactement les mêmes en leur fond et dérivent exactement de la même pensée secrète. Réduire successivement le service militaire de sept ans à cinq ans, de cinq ans à trois ans, de trois ans à deux ans, tout le monde le sait, aussi bien ceux qui sont favorables à ce mouvement que ceux qui lui sont opposés, c’est abolir l’esprit militaire, c’est-à-dire la cohésion, l’entente cordiale, la communion d’esprit entre l’officier, élément permanent de l’armée, et le soldat, qui ne fait plus qu’y passer. Il est évident que le soldat qui ne passe que deux ans dans l’armée n’a que deux sentiments successifs : la première année, le regret d’avoir quitté sa famille et son village ; la seconde année, l’impatience d’y rentrer. En deux années, qui, dans la pratique, seront réduites à vingt mois, un troisième sentiment n’a pas le temps de se former. Dans un pays ardemment patriote, comme l’Allemagne ou l’Angleterre, l’inconvénient est ou serait moindre. Il est clair que si le sentiment patriotique et l’esprit militaire sont développés dès l’école primaire et dès le gymnase, il n’est  pas besoin d’un long temps passé sous les armes pour le former. Il existe à l’avance, il persiste pendant le temps du service militaire, il reste ensuite. Mais dans un pays où l’école primaire et le lycée sont hostiles au sentiment patriotique ou tout au moins ne s’appliquent aucunement à l’entretenir, où l’école primaire et le lycée sont hostiles à l’esprit militaire ou tout au moins songent à tout autre chose qu’à le faire naître, il n’est que très vraisemblable que le service militaire court, non seulement ne créera pas l’esprit militaire, mais l’empêchera d’éclore là où il aurait pu se produire, ne retenant le jeune homme sous les drapeaux que juste le temps de lui faire d’abord regretter, puis désirer la vie civile. Une armée sans esprit militaire, c’est ce que va créer notre nouvelle loi sur l’armée.\par
Mais c’est que précisément l’esprit militaire est ce que déteste et redoute le plus le parti démocratique.\par
Il est très embarrassé : il n’est pas précisément antipatriote ; car, après tout, la disparition de la France comme nation ne lui profiterait guère, puisque ce serait la France exploitée par d’autres et non plus par lui ; et l’on n’envisage jamais une pareille perspective de gaîté de cœur ; mais, d’autre part, il est antimilitariste fatalement et comme  forcément ; car il redoute toujours le despotisme militaire, le tyran militaire, celui qui détruit ou annihile le régime parlementaire et exploite le pays pour lui-même, pour ses favoris, pour ses généraux, pour ses officiers et non plus au profit des orateurs de village et des politiciens de sous-préfecture.\par
Ainsi partagé, le parti démocratique est donc dans un certain embarras ; mais envisageant la disparition de la France comme une chose lointaine, par suite de cette tendance qu’on a toujours à considérer un grand changement européen comme une chose lointaine, tendance instinctive et du reste absurde ; et envisageant le despotisme militaire comme une chose qui peut se produire demain, même en pleine paix (et l’aventure du général Boulanger l’a confirmé dans cette idée), il a pour l’esprit militaire une aversion sans aucun mélange et ne tient à rien tant qu’à le détruire partout où il est et à l’empêcher de naître partout où il pourrait germer.\par
C’est {\itshape le plus pressé} ; et pour le reste, selon la formule de tous les esprits bornés, pour le reste, {\itshape on verra plus tard}.\par
N’exprimez donc pas cette crainte qu’avec les nouvelles lois militaires l’esprit militaire ne s’affaiblisse et ne tende à disparaître : c’est précisément  pour diminuer l’esprit militaire que les nouvelles lois militaires ont été faites, que les nouvelles lois militaires se font et que se feront de nouvelles lois militaires, jusqu’à ce qu’il n’y ait plus en France que des milices ou une garde nationale. Il est vrai que, selon toute apparence, la France aura disparu auparavant ; mais je ne voulais démontrer que ceci qu’en cette question encore, ce qui paraît le défaut de la mesure prise en est justement l’esprit directeur et la cause efficiente et aussi la cause finale.\par
Or, pour y revenir, dans cette grande affaire du cléricalisme et de l’anticléricalisme il en va tout de même. Ce qui rend l’anticléricalisme incurable, c’est que toutes les sottises dont il est composé sont les raisons mêmes pour quoi il est chéri et caressé par ses partisans ; c’est que tous les périls qu’il renferme sont considérés par ses partisans comme des chances qu’il ne serait pas mauvais de courir ; c’est que toutes ses désastreuses conséquences sont, plus ou moins consciemment, considérées par ses partisans comme des progrès, ou tout au moins comme des choses qui ne seraient pas si mauvaises qu’on affecte de le croire.\par
Qu’ai-je dit, que disons-nous pour persuader à la France de ne pas s’hypnotiser dans la passion antireligieuse ? Nous lui disons que d’opprimer,  de molester, de persécuter une partie, et importante, de la population française pour ses opinions religieuses et pour sa manière de faire élever ses enfants, cela supprime des Français, cela diminue le nombre des Français, cela aliène des Français qui ne demanderaient qu’à aimer la France, cela fait des étrangers à l’intérieur.\par

\begin{itemize}[itemsep=0pt,]
\item  — C’est précisément ce qu’il faut, nous répondent nos adversaires, ou nous répondraient-ils s’ils allaient net jusqu’au bout de leur pensée ou s’ils se rendaient nettement compte du principe même de leur pensée. C’est précisément ce qu’il faut. L’unité morale le veut. Nous ne considérons comme Français que ceux qui pensent comme nous, que ceux qui datent de 1793, que ceux qui ne croient qu’à la démocratie et à la libre pensée, que ceux qui ont rejeté toute superstition et qui sont délivrés de cette maladie mentale qu’on appelle le sentiment religieux. Voilà ceux qui constituent l’unité morale, c’est-à-dire la France. Les autres sont des antiunitaires, c’est-à-dire des antifrançais.
\end{itemize}

\noindent Il ne faut donc pas nous dire que nous supprimons des Français ; nous frappons des gens qui ne sont pas des Français, qui {\itshape déjà} ne sont pas des Français. Nous n’{\itshape aliénons} pas des compatriotes, nous défendons la France contre des gens qui ne  sont pas des compatriotes. Nous ne créons pas des étrangers à l’intérieur ; nous trouvons des étrangers à l’intérieur, et nous leur défendons de l’être, et nous les prions ou de cesser de l’être ou de déguerpir. On ne peut pas être plus patriotes que nous le sommes. Le véritable patriotisme consiste à ne compter pour Français que ceux qui le sont. Ceux qui prétendent à la fois être français et romains, leur prétention à l’égard du titre de citoyen français est monstrueuse. Qu’ils choisissent. Ils n’aimeront pas la France si nous leur défendons de faire élever leurs enfants par des prêtres ? Oh ! Tant mieux ! Qu’ils ne l’aiment point et qu’ils la quittent. Il n’y a que profit pour elle, ou à ce qu’ils y restent en ne l’aimant point, car ils ne demanderont point ses faveurs, ses places, ses postes et ne les obtiendront pas et laisseront les vrais Français se partager tout cela ; ou à ce qu’ils la quittent, car ils supprimeront ainsi le scandale de l’étranger campé et organisé en France contre la France, contre son esprit, contre ses principes et contre son unité. Nous préférons une France une et serrée en faisceau à une France de population plus nombreuse, mais divisée, déchirée et incohérente.\par
Voilà en son fond le véritable esprit du Français « unitaire ». Ce que nous signalons comme  le point faible de sa mentalité en est le fond même.\par
Comprendre que l’unité morale, aux temps modernes, ne peut être que dans la liberté, dans le sentiment, répandu chez tous les citoyens, que, quoi qu’ils pensent, quoi qu’ils disent et quoi qu’ils fassent, excepté contre la patrie, ils trouveront dans la patrie une égale bienveillance à leur endroit ; comprendre que c’est là, désormais, le vrai lien, le vrai faisceau et la vraie unité ; comprendre que si l’unité américaine existe, c’est que les citoyens américains sentent et éprouvent que, quelles que soient leurs idées et leurs tendances particulières, la République s’en désintéresse absolument et ne leur demande que d’être des Américains ; comprendre que si l’unité anglaise existe, c’est que le citoyen anglais sent dans sa patrie une protectrice de tous ses droits et de toutes ses façons de penser, si différentes qu’elles soient de celles du voisin ; comprendre que si l’unité allemande existe, c’est que catholiques allemands sont aussi libres d’être catholiques que les protestants sont libres d’être protestants et protestants allemands aussi libres d’être protestants que les catholiques sont libres d’être catholiques et que, par conséquent, les uns et les autres sont avant tout allemands : « l’Allemagne au-dessus  de tout ! » ; — comprendre tout cela, voilà ce qui est absolument impossible à « l’unitaire » français, avec son âme du \textsc{xvi}\textsuperscript{e} siècle.\par
Il est remarquable comme le goût de l’unité ne donne que la passion de la guerre civile. Pourtant il en est ainsi.\par
On voit que ce qui est le défaut essentiel du raisonnement de nos unitaires en est la majeure.\par
Que disons-nous encore ? Que l’anticléricalisme mène peu à peu à une conception et à une organisation de l’enseignement qui rendraient les Français idiots. Oui, nous disons cela, à peu près, et je crois bien que nous le pensons tout à fait. L’unité effrénée d’enseignement, si l’on me permet de parler ainsi, ne peut avoir en effet pour résultat qu’une profonde débilitation de l’intelligence nationale. Un peuple à qui l’on n’enseigne qu’une manière de voir finit bientôt par n’avoir aucune manière de voir. Assez plaisamment, quoique avec un esprit un peu gros, Victor Hugo disait en 1850 aux « Jésuites » : « Vous demandez la liberté d’enseignement ? Ce que vous voulez, c’est la liberté de ne pas instruire. » Tous les arguments à l’adresse de la « France noire » se retournant mathématiquement contre la « France rouge », je dirai, un peu lourdement aussi, aux partisans du  monopole de l’enseignement : « Vous prétendez instruire seuls ; c’est vouloir ne pas enseigner ; c’est vouloir, à force de n’enseigner qu’une chose et fermer les esprits à toutes les autres, les fermer à toutes. Car on n’a une idée, on ne la possède vraiment, on ne la voit avec clarté, que quand on a fait le tour de toutes les idées et quand on en a choisi une. Intelligence, c’est comparaison, puis préférence. »\par
Les partisans intelligents du monopole sentent si bien la force de cette objection et ce qu’elle contient de vérité, qu’ils ne manquent pas d’assurer de tout leur courage que c’est dans l’enseignement monopolisé que l’on trouvera l’exposition de toutes les idées, les plus différentes, et pour l’élève la liberté la plus large de comparaison, de choix et de préférence. J’ai discuté plus haut, avec le plus grand sérieux, cette plaisanterie, et je n’y reviens que pour rappeler que personne n’en peut être dupe, non pas même ceux qui en sont les auteurs, s’il n’y a pas à dire plutôt que surtout ceux qui en sont les auteurs n’en peuvent croire un mot.\par
Donc, cette objection écartée, nous prétendons que les partisans du monopole de l’enseignement veulent surtout ne pas instruire, j’entends ne pas mettre les jeunes esprits à même de choisir  entre les idées. Notez que cela se démontre déjà par une foule de signes. Dans les programmes universitaires il y a tendance très visible à n’enseigner l’histoire que depuis 1789 et à laisser le jeune homme dans l’ignorance la plus profonde sur tout ce qui s’est passé auparavant. Voilà ce que j’appelle ne pas enseigner. L’homme qui ne connaît l’histoire que depuis 1789 est un homme si limité qu’il en est bouché. Il ne comprend rien du tout et non pas même 1789. Il est absolument inintelligent en humanité. Il a une complète inintelligence de l’histoire et une ignorance encyclopédique du genre humain, y compris celui où il vit. Cet homme-là, c’est l’homme que veulent les partisans du monopole de l’enseignement ; c’est pour eux l’homme de l’avenir.\par
De même au congrès de Liége, en septembre 1905, M. Salomon Reinach insistait pour que l’on éliminât décidément de l’enseignement littéraire les auteurs du \textsc{xvii}\textsuperscript{e} siècle, si profondément arriérés et qui ne peuvent rien apprendre, du moins de bon, aux générations du \textsc{XX}\textsuperscript{e} siècle. Voilà qui va fort bien, et je crois vous entendre. Mais si l’on croit que le jeune « studieux » comprendra un mot à l’\emph{Essai sur les mœurs} s’il n’a pas lu le \emph{Discours sur l’histoire universelle}, et à Montesquieu s’il n’a pas lu la \emph{Politique tirée de l’Ecriture sainte},  et, pour sortir un instant de la politique, à Marivaux s’il n’a pas lu Racine, et à Jean-Jacques Rousseau s’il n’a pas lu Fénelon, etc., etc., il faut être un peu inexpérimenté soi-même en matière d’enseignement et en matière intellectuelle.\par
Ce n’est donc pas conjecture de notre part, ni induction, du reste légitime, ni procès de tendances, du reste fondé, quand nous disons aux partisans du monopole de l’enseignement : « Vous ne voulez pas plus de liberté {\itshape dans} l’enseignement que de la liberté {\itshape de} l’enseignement. Vous ne voulez de liberté de choix, et c’est-à-dire de liberté d’intelligence, nulle part. Un peu partout, et particulièrement dans un pays comme le nôtre, la liberté {\itshape dans} l’enseignement ne peut être obtenue que par la liberté {\itshape de} l’enseignement. Si les clercs seuls enseignaient, ils enseigneraient l’histoire jusqu’en 1789 et jetteraient un voile sur tout ce qui s’est passé depuis ; ils enseigneraient la littérature avec Bossuet, Nicole, Bourdaloue et peut-être Corneille ; ne feraient lire aucun auteur du \textsc{xviii}\textsuperscript{e} siècle, et du \textsc{xix}\textsuperscript{e} siècle ne mettraient sous les yeux de leurs élèves que de Maistre et Veuillot (je n’invente rien : lisez le livre de M. l’abbé Delfour, \emph{Catholicisme et Romantisme}), et ils feraient des élèves stupides.\par
Inversement, si vous, antireligieux, vous enseigniez  seuls, vous n’enseigneriez que 1789, 1793, 1848 et 1870, vous laisseriez ignorer le \textsc{xvii}\textsuperscript{e} siècle, et vous enseigneriez la littérature avec Voltaire, Diderot, Helvétius, d’Holbach, Stendhal et Victor Hugo dernière manière ; et vous feriez des élèves stupides. Et, donc, nous avons le droit de dire : « En réclamant le monopole de l’enseignement, vous réclamez, non le privilège d’instruire seuls, mais le droit de ne pas instruire du tout. »\par
Voilà ce que nous disons, et ce qui ne sera jamais tout à fait vrai, mais ce qui se rapprochera de plus en plus de la vérité : c’est asymptotique. Oui, que le monopole de l’enseignement ait pour effet de mener un peuple à un état très voisin de l’ignorance et très proche de l’incompréhension universelle, cela me paraît incontestable ; mais c’est précisément pour cela, sans peut-être s’en rendre bien compte, que les républicains despotistes n’ont aucune répugnance pour le monopole de l’enseignement. Aucun despotisme n’aime le savoir, aucun despotisme n’aime l’intelligence. Montesquieu l’a répété à satiété pour le despotisme ancien style, c’est-à-dire pour le despotisme d’un seul. Il n’y a aucune raison pour que ce ne soit pas tout aussi vrai du despotisme collectif. « La République n’a pas besoin de savants » est peut-être un mot légendaire et, pour mon compte,  j’incline à croire qu’il a été inventé par un réactionnaire spirituel, par quelque Rivarol ; mais il exprime bien la pensée de la démocratie, de la plus basse, à la vérité ; mais le malheur est que la basse démocratie force la haute à tomber sans cesse un peu au-dessous du niveau de la basse.\par
Tant y a que le démocrate {\itshape moyen}, si vous voulez, ne pense point précisément du mal du savoir et de l’intelligence, mais se dit : « {\itshape Après tout}, ce n’est pas de grand savoir et de fine intelligence que nous avons {\itshape le plus} besoin ; {\itshape avant tout}, ce qu’il nous faut, ce sont des hommes dévoués à la démocratie et le plus antireligieux possible, parce que, quand on n’est antireligieux qu’à demi, si l’on n’est pas précisément avec les prêtres, du moins on les ménage. Or il est constaté que le vaste savoir et l’intelligence très exercée mènent souvent à cet état d’esprit sinon favorable, du moins indulgent, aux hommes de religion. C’est très fréquent. Il y a donc lieu de ne pas tenir essentiellement à ce qu’il y ait dans ce pays beaucoup d’hommes de haute culture. »\par
En somme, vous ne voulez que des ignorants, parce que vous ne voulez que des fanatiques. C’est l’arrière-pensée de tous les sectaires.\par

\begin{itemize}[itemsep=0pt,]
\item  — Vous exagérez !
\item  — Eh ! non ! Je précise.
\end{itemize}

\noindent  Il y a bien, sans qu’on en convienne et, je le reconnais, sans qu’on s’en rende compte, quelque chose comme cela. Ici encore, la plus grande erreur que l’on puisse trouver dans le système n’est pas autre chose que le principe secret du système.\par
Et enfin, pour abréger, car on pourrait poursuivre cette analyse en beaucoup de sens, nous assurons que l’anticléricalisme poussé à fond, comme les radicaux veulent l’y pousser en effet, mène tout droit au despotisme. Il y mène de toutes les façons. Il habitue les esprits à considérer qu’un homme n’a pas les droits de l’homme quand il pense d’autre manière que le gouvernement. Il habitue les esprits à considérer qu’un homme peut être proscrit parce qu’il vit d’une manière honorable, mais différente de la façon commune. Il habitue les esprits à mépriser la liberté et aussi l’égalité.\par
La liberté, puisque je n’ai pas celle de faire des vœux de morale sévère et de m’associer à ceux qui font les mêmes vœux ; la liberté, puisque je n’ai pas le droit d’enseigner ce que je crois vrai et qui, du reste, n’est pas contraire à la constitution de ce pays ; la liberté, puisque je n’ai pas le droit de faire instruire mon fils par qui me plaît, du reste honnête homme.\par
 Mais l’anticléricalisme habitue aussi les esprits à mépriser l’égalité ; parce qu’il crée des classes. Il en crée au moins deux : première classe, qui a toutes sortes de droits, à l’exclusion de l’autre, c’est-à-dire qui a toutes sortes de privilèges, droit d’enseigner, droit de prêcher, droit de faire des processions et de haranguer les foules dans la rue autour de la statue d’un martyr et d’entraver la circulation ; — seconde classe, qui est privée du droit d’enseigner, du droit de s’associer, du droit de vivre en commun, du droit de faire des processions dans la rue et des meetings sur la place publique, dernier droit du reste que je n’accorderais à personne, mais qu’enfin il est constant que vous attribuez aux uns et que vous refusez aux autres. — L’anticléricalisme crée donc des classes : les unes privilégiées, les autres dénuées ; les unes avantagées, les autres frustrées ; les unes accaparantes, les autres spoliées ; les unes oppressives, les autres opprimées. Il fait des parias. C’est le signe même du despotisme ou plutôt c’en est bien le fait ; c’en est la réalisation nette, pure et simple.\par

\begin{itemize}[itemsep=0pt,]
\item  — Mais le genre de despotisme que l’anticléricalisme institue s’arrête là !
\item  — Ce serait déjà suffisant, à mon avis, et « il y a oppression du corps social, dit la très réactionnaire  \emph{Déclaration des droits de l’homme}, dès qu’un seul de ses membres est opprimé » ; mais le genre de despotisme que l’anticléricalisme institue va plus loin encore dans ses conséquences et dans ses conséquences prochaines. Le mot de M. Jaurès est très profond : « Le collectivisme intellectuel mène tout droit au collectivisme économique. » Ce qui veut dire : « L’État omniscient, cela mène à l’État omnipossesseur ; le monopole de l’enseignement, c’est le collectivisme intellectuel, c’est la collectivité se substituant, pour instruire, aux individus ou aux associations ; — du même principe sortira ceci : la collectivité se substituant, pour posséder, aux individus ou aux associations ; il n’y a pas plus de raison pour que vous possédiez individuellement ou par sociétés particulières que pour que vous enseigniez individuellement ou par sociétés particulières ; l’État aujourd’hui enseigne seul ; l’État demain, du même droit, possédera seul ; il n’y a aujourd’hui d’autre professeur que l’État ; il n’y aura pas demain d’autre propriétaire que l’État ; collectivité partout ; le collectivisme intellectuel n’est qu’un essai, heureux du reste, du collectivisme des biens ; tout compte fait, collectivisme de l’enseignement et collectivisme des biens, c’est la même chose, c’est le collectivisme des droits. »
\end{itemize}

\noindent  C’est-à-dire la suppression des droits, c’est-à-dire le despotisme pur et simple. Il est très vrai que supprimer le droit du citoyen comme père de famille mène directement à supprimer son droit comme propriétaire. Il dit aujourd’hui : « Je suis père de famille » ; on lui répond : « Ça ne compte pas. » Il dira demain : « Je suis propriétaire » ; on lui répondra : « Ça ne signifie rien. » Le « droit éminent » d’enseignement appartenant à l’État et le « droit éminent » de propriété appartenant à l’État, c’est absolument la même théorie, la même doctrine et le même dogme.\par
Par toutes sortes de chemins, l’anticléricalisme mène donc au despotisme et de toutes sortes de façons il le renferme en lui.\par
Ai-je besoin de faire remarquer encore une fois que ce n’est son défaut que parce que c’est la source même d’où il dérive ? L’anticlérical est despotiste. Ce n’est pas où il arrive, c’est de quoi il part ; ce n’est pas la conséquence inattendue de ses démarches, c’est le premier pas de sa course ; ou plutôt c’est à la fois d’où il part et où il arrive ; car c’est de quoi il ne sort jamais. Il n’est autre chose, comme il a été dit mille fois, qu’un clérical retourné, et son \emph{Syllabus} est exactement aussi contraire à la \emph{Déclaration des droits de l’homme} que le \emph{Syllabus} romain. Un plaisant dirait : « Il n’y a  qu’une \emph{Déclaration des droits de l’homme}, mais il y a deux déclarations des droits de Rome. Que voulez-vous que celle-là fasse contre les deux autres. »\par
L’anticlérical n’est qu’un clérical retourné, comme cela se voit, même historiquement, par ce fait que les contrées de France qui contiennent le plus d’anticléricaux effrénés sont celles qui contenaient, il n’y a pas un siècle, le plus de cléricaux enfiévrés ; mais c’est un clérical d’autant plus violent et d’autant plus dangereux, ce me semble, qu’il ne compte pas, ou qu’il compte assez peu sur l’association privée, et qu’il ne compte que sur l’État, et que, par conséquent, il a tout intérêt à ce que l’État soit passionnément despotique pour qu’il le soit au service des passions antireligieuses.\par
Il est {\itshape possible} qu’un catholique soit libéral. Il place ou peut placer sa volonté de puissance ailleurs que dans l’État : « Je serai fort par corporation, par mon association, par mon Église. » Il est possible à un homme qui n’est ni clérical ni anticlérical d’être libéral : on peut être autoritaire tout en n’étant ni clérical ni anticlérical, mais au moins on a une raison de moins pour l’être.\par
Il est impossible à un anticlérical d’être libéral et il verse toujours dans le despotisme, si tant est qu’il n’y soit pas toujours.\par
 Entre catholiques et anticatholiques la lutte continuera donc très longtemps, puisque toutes les erreurs que l’on peut signaler aux anticatholiques dans leurs doctrines sont les principes mêmes de leurs doctrines et les sources profondes et secrètes qui les alimentent. Entre les catholiques — soutenus par les libéraux tant que les catholiques seront mis hors du droit — et les anticatholiques soutenus par tout ce qui, en ce pays, a des tendances vers le despotisme démocratique sous une forme ou sous une autre, la bataille, commencée depuis si longtemps, n’est pas près ni de finir, ni de se relâcher, ni de s’interrompre.
 \section[{Conclusions}]{Conclusions}\renewcommand{\leftmark}{Conclusions}

\noindent Pour la faire finir, cette bataille séculaire qui nous épuise, il faudrait qu’il se constituât un parti de modérés très énergiques, chose du reste qui ne s’est jamais vue.\par
Ce parti ne serait ni clérical ni anticlérical et ne permettrait pas que l’on fût ni l’un ni l’autre.\par
Il serait patriote et libéral, et libéral par patriotisme. Il serait convaincu de cette vérité que tout peuple a intérêt, non seulement à n’éliminer hors de la cité, à ne détruire, en l’éliminant ainsi, aucune des forces nationales, mais encore qu’il a intérêt à convertir en forces nationales tous les éléments d’énergie intellectuelle et morale qui se trouvent en lui.\par
Or, étant donnée l’infinie diversité de tempéraments, de caractères, de tendances, de croyances, d’opinions et d’idées qui existe dans le monde moderne, la patrie ne peut être aimée par tous, que si elle admet cette diversité, c’est-à-dire que si elle respecte la liberté et la favorise ; et la patrie  ne peut être aimée que de quelques-uns, ce qui est un danger épouvantable, si, dans cette diversité d’opinions, elle en prend une pour la faire sienne et pour l’imposer.\par
Il en résulte d’abord qu’il n’y a rien de plus rétrograde et de plus réactionnaire que de rebrousser, au \textsc{XX}\textsuperscript{e} siècle, vers l’État antique, vers l’État romain et vers l’État grec, qui a une doctrine religieuse à lui, qui l’impose, qui l’assène et qui fait boire la ciguë ou prendre le chemin de l’exil à ceux qui en ont une autre, démarche d’autant plus réactionnaire et rétrograde que jamais, même à Athènes et à Rome, on n’a proscrit la liberté d’enseigner, d’une façon permanente et continue, aussi rigoureusement qu’on le fait en France au temps où nous sommes.\par
Il en résulte ensuite que le patriotisme s’éteindrait bientôt si la patrie n’était plus, dans l’esprit des citoyens, qu’un être tracassier, impérieux et tyrannique, d’esprit étroit du reste et qui veut, comme un grand-père autoritaire et têtu, qu’on ne pense dans sa maison que comme il pense et qu’on ne parle qu’en répétant ce qu’il dit. C’est ainsi que les familles s’aigrissent et du reste se désunissent et se dissolvent.\par
Et l’on peut remarquer ceci, qui peut-être est fâcheux : c’est au moment où les hommes d’opinions  dites « avancées » renoncent au patriotisme ou l’attiédissent singulièrement dans leurs cœurs, qu’on s’arrange d’autre part de manière à refroidir le patriotisme chez les hommes d’opinions dites arriérées, de sorte qu’on peut se demander où, bientôt, le patriotisme se réfugiera.\par
Le parti libéral, et libéral par patriotisme, que je suppose, dirait donc, autant par patriotisme que par dévotion à la \emph{Déclaration des droits de l’homme}, d’une part à la « France noire » :\par
« Il vous est permis d’être noirs, à la condition que vous soyez patriotes.\par
« Il vous est permis de croire en Dieu et de dire que vous y croyez et pourquoi vous y croyez.\par
« Il vous est permis de vous associer en une communion de croyances, de pensées, d’espoirs et d’efforts, dite l’Église catholique, et de prêcher et d’enseigner ce que vous croyez ; parce que ce sont des droits de l’homme et qu’il ne suffit pas d’être catholique pour cesser d’être tenu pour un être humain.\par
« Il vous est permis de ne pas croire que la Révolution française est la vérité absolue et définitive et la seule religion que l’humanité doive embrasser ; et il vous est permis de donner toutes les raisons pourquoi vous ne croyez point cela.\par
« Il vous est permis de discuter toutes les lois que  font les représentants de la majorité du peuple français ; car ces lois n’ont rien de divin, de sacré, ni même d’irrévocable ; il vous est permis de les discuter, à la condition que, de fait, vous y obéissiez et que vous disiez que, de fait, il y faut obéir.\par
« Il vous est permis de discuter la Constitution, comme font les professeurs de droit constitutionnel dans leurs cours et les philosophes sociologues dans leurs leçons ; il vous est permis de la discuter tout entière, à la condition que, de fait, vous y obéissiez et que vous disiez que, de fait, il y faut obéir.\par
« Il vous est permis de faire à vos doctrines, idées et opinions, autant d’adhérents que vous en pourrez faire et d’intervenir dans les élections, sauf par corruption, tout autant que vous le voudrez ; et nous considérons que, non seulement c’est votre droit, mais que c’est même votre devoir, comme il est le nôtre et parce qu’il est le nôtre, car mon droit c’est le tien et ton droit c’est le mien, et user de son droit est un devoir.\par
« Il ne vous est défendu que de détacher les citoyens de la patrie. Vous pouvez aller jusqu’à dire du mal de la patrie, entendons par là jusqu’à lui dire des vérités dures, mais de telle façon qu’il soit visible que vous n’en dites du mal que pour  la rendre meilleure, et par conséquent par profonde affection pour elle, et à la condition que votre conclusion soit toujours qu’il faut aimer sa patrie, même, quand elle se trompe, plus diligemment, même, quand elle se trompe et ne jamais la renoncer, même dans le secret du cœur.\par
« Voilà tout ce qui vous est permis. Voilà la seule chose qui vous soit défendue. Allez, Messieurs. »\par

\astertri

\noindent Il dirait d’autre part à la « France rouge » :\par
« Il vous est permis d’être rouges, à la condition d’être patriotes.\par
« Il vous est permis d’être athées et de faire profession et propagande d’athéisme.\par
« Il vous est permis de considérer le sentiment religieux comme une maladie mentale que les gens bien portants doivent mettre tous leurs efforts à guérir chez ceux qui en sont atteints.\par
« Il vous est permis de dire à la nation la plus intelligente de l’univers que l’évêque de Rome l’opprime et qu’elle gémit encore sous le joug accablant du pontife romain.\par
« Il vous est permis de vous associer en corporations puissantes pour combattre les menées infâmes et dangereuses de la « congrégation » et pour répandre vos doctrines.\par
 « Cela vous est même recommandé ; car si vous réussissiez, peut-être, combattant par vous-mêmes et par les corporations dont vous seriez membres et que vous soutiendriez, perdriez-vous l’habitude de vous adresser toujours à l’État et de le persécuter pour qu’il se charge de vos querelles, de vos vengeances et de l’exercice de vos haines et pour qu’il consacre toutes ses forces à combattre une partie considérable de la nation française, ce qui donne à celle-ci le désir et presque le droit d’en faire autant quand elle a la majorité.\par
« Il vous est permis d’user de la parole, du livre, de la brochure, de la revue et du journal, de tout enfin, excepté de la rue (exception qui frappera aussi vos adversaires), pour faire connaître, faire juger et faire accepter les vérités politiques, religieuses, économiques et sociologiques dont vous êtes détenteurs.\par
« Il ne vous est défendu que de détacher les citoyens de la patrie. Et, comme à vos adversaires, il vous est même permis d’en dire du mal, à la condition que vous n’en concluiez jamais, quelque faute qu’elle puisse commettre, qu’il faut la quitter, la renoncer, déserter son drapeau et tirer des coups de fusil sur ceux qui le tiennent.\par
« Voilà tout ce qui vous est permis ; voilà la seule chose qui vous soit défendue. Allez, Messieurs. »\par
 Le parti libéral dirait enfin au gouvernement, quel qu’il fût : « Vous n’êtes où l’on vous a mis que pour défendre la patrie et l’administrer, c’est-à-dire pour pourvoir à l’armée, à la marine, aux finances, à la police et à la justice. Là se bornent votre rôle et votre fonction et, du reste, votre capacité. Quand vous vous mêlez d’autre chose, vous sortez de vos droits et, du reste, vous le faites mal.\par
« Vous devriez n’avoir qu’un budget de la guerre et de la marine, qu’un budget de l’administration intérieure, qu’un budget de la justice et de la police et qu’un budget des affaires étrangères.\par
« Dès que vous vous mêlez de religion, d’enseignement et même d’industrie et de beaux-arts, vous devenez un parti, et c’est au profit d’un parti, blanc, rouge ou noir, selon les saisons, que vous vous mêlez de tout cela ; de sorte que, créé pour assurer la concorde entre les citoyens, vous êtes toujours, plus ou moins vivement, un gouvernement de guerre civile ; et que, créé pour défendre, maintenir et agrandir la patrie, vous mettez au moins une partie de vos efforts à la diminuer.\par
« Restreignez-vous donc à votre rôle et à votre emploi, pour n’être du reste que plus fort dans votre emploi et dans votre rôle.\par
« Assurez à l’extérieur la défense, à l’intérieur  l’ordre et les bonnes finances. C’est tout. Vous n’avez pas autre chose à faire.\par
« Ne vous considérez pas comme le premier théologien de France, le premier moraliste de France, le premier professeur de France et le premier amateur d’art de France. Où avez-vous pris que vous fussiez tout cela ? Il n’en est rien, je vous assure, et à la fois vous avez trop de prétention et vous vous donnez trop de mal.\par
« Ne vous réduisez pas, nous le concédons par crainte des changements trop brusques, ne vous réduisez pas, du jour au lendemain, à ce simple rôle de soldat sur la frontière et de gendarme à l’intérieur que nous disons qui est le vôtre ; mais, en attendant une simplification plus complète, au moins sur les domaines qui ne sont pas les vôtres, où vous avez empiété et où vous vous trouvez installé, soyez neutre, véritablement neutre ; n’ayez pas d’opinion ; ne servez aucun parti.\par
« N’imposez aucune doctrine à vos professeurs, ne leur en interdisez aucune, sauf l’antipatriotisme ; car cela rentre précisément dans votre rôle d’interdire celle-là.\par
« N’exigez de vos fonctionnaires, quels qu’ils soient, aucune {\itshape qualité confessionnelle}, aucune doctrine et aucune pratique soit religieuse, soit antireligieuse.\par
 « N’exigez de vos fonctionnaires, quels qu’ils soient, aucune opinion politique, et soyez absolument indifférent à celles qu’ils peuvent avoir, pourvu qu’ils s’acquittent bien de leurs fonctions et qu’ils soient dévoués au pays.\par
« Laissez à la magistrature, quoique nommée par vous, une indépendance absolue et ne la mettez jamais, par promesses, encouragements, intimidation et autres influences, au service d’un parti, quelque bon que puisse être à vos yeux ce parti.\par
« Pour ce qui est de la masse des citoyens, qu’ils sentent bien, tous, que vous ne voulez pas même savoir à quel parti ils appartiennent et que ce serait en vain que, pour obtenir des faveurs et bénéficier d’injustices, ils se mettraient dans celui-ci, dans celui-là ou dans un troisième.\par
« Vous aimez à répandre des idées. N’en répandez qu’une, c’est que vous êtes la justice et l’impartialité absolues. A cause de l’expérience acquise, on ne vous croira pas tout de suite, mais on finira par s’apercevoir que vous dites vrai. »\par
A ce discours, il est infiniment probable que le gouvernement répondrait : « Je dépends des électeurs. Avec votre système, j’en aurais pour moi trois mille sur huit millions. Souffrez que je sois le gouvernement d’un parti, comme tous les  gouvernements depuis 1789 ; et du parti le plus fort.\par

\begin{itemize}[itemsep=0pt,]
\item  — Mais être le gouvernement d’un parti, c’est être gouverné par un parti. {\itshape Omnia serviliter pro dominatione.}
\item  — A qui le dites-vous ? C’est pourtant le seul moyen d’exister. »
\end{itemize}

\noindent Il faudrait voir. Ce n’est pas si certain qu’on s’obstine à le croire. Un gouvernement de parti est vite dévoré par son parti même, dont il ne réussit jamais à satisfaire tous les appétits. Un gouvernement de justice impose beaucoup. Il a, dans une République, quelque chose de la majesté du gouvernement royal, lequel est par définition et quelquefois en réalité, au-dessus de tous les partis ; et, naturellement, c’est dans une République qu’on est le plus sensible à ce qui est le bon aspect du gouvernement royal.\par
Mon scepticisme ne va peut-être pas assez loin, et c’est toujours ce que je suis tenté de lui reprocher ; mais j’ai tendance à croire qu’un gouvernement de justice se créerait assez vite un très grand parti et pourrait ainsi être, lui aussi, un gouvernement de parti, mais d’un parti juste.\par
En tout cas, pour continuer un instant le dialogue, le parti libéral dit au gouvernement : « Soyez libéral. » Le gouvernement répond aux  libéraux : « Soyez les plus nombreux, et je vous jure bien que je serai libéral. Je ne pourrai même pas faire autrement. »\par
C’est donc aux libéraux à être les plus nombreux. C’est pour en augmenter le nombre que j’ai écrit ces quelques pages, après d’autres, au cas que je pourrais avoir quelque force de persuasion. — Mais c’est sur ce point que mon scepticisme est radical.\par

\dateline{Août-Novembre 1905.}
 


% at least one empty page at end (for booklet couv)
\ifbooklet
  \pagestyle{empty}
  \clearpage
  % 2 empty pages maybe needed for 4e cover
  \ifnum\modulo{\value{page}}{4}=0 \hbox{}\newpage\hbox{}\newpage\fi
  \ifnum\modulo{\value{page}}{4}=1 \hbox{}\newpage\hbox{}\newpage\fi


  \hbox{}\newpage
  \ifodd\value{page}\hbox{}\newpage\fi
  {\centering\color{rubric}\bfseries\noindent\large
    Hurlus ? Qu’est-ce.\par
    \bigskip
  }
  \noindent Des bouquinistes électroniques, pour du texte libre à participation libre,
  téléchargeable gratuitement sur \href{https://hurlus.fr}{\dotuline{hurlus.fr}}.\par
  \bigskip
  \noindent Cette brochure a été produite par des éditeurs bénévoles.
  Elle n’est pas faîte pour être possédée, mais pour être lue, et puis donnée.
  Que circule le texte !
  En page de garde, on peut ajouter une date, un lieu, un nom ; pour suivre le voyage des idées.
  \par

  Ce texte a été choisi parce qu’une personne l’a aimé,
  ou haï, elle a en tous cas pensé qu’il partipait à la formation de notre présent ;
  sans le souci de plaire, vendre, ou militer pour une cause.
  \par

  L’édition électronique est soigneuse, tant sur la technique
  que sur l’établissement du texte ; mais sans aucune prétention scolaire, au contraire.
  Le but est de s’adresser à tous, sans distinction de science ou de diplôme.
  Au plus direct ! (possible)
  \par

  Cet exemplaire en papier a été tiré sur une imprimante personnelle
   ou une photocopieuse. Tout le monde peut le faire.
  Il suffit de
  télécharger un fichier sur \href{https://hurlus.fr}{\dotuline{hurlus.fr}},
  d’imprimer, et agrafer ; puis de lire et donner.\par

  \bigskip

  \noindent PS : Les hurlus furent aussi des rebelles protestants qui cassaient les statues dans les églises catholiques. En 1566 démarra la révolte des gueux dans le pays de Lille. L’insurrection enflamma la région jusqu’à Anvers où les gueux de mer bloquèrent les bateaux espagnols.
  Ce fut une rare guerre de libération dont naquit un pays toujours libre : les Pays-Bas.
  En plat pays francophone, par contre, restèrent des bandes de huguenots, les hurlus, progressivement réprimés par la très catholique Espagne.
  Cette mémoire d’une défaite est éteinte, rallumons-la. Sortons les livres du culte universitaire, cherchons les idoles de l’époque, pour les briser.
\fi

\ifdev % autotext in dev mode
\fontname\font — \textsc{Les règles du jeu}\par
(\hyperref[utopie]{\underline{Lien}})\par
\noindent \initialiv{A}{lors là}\blindtext\par
\noindent \initialiv{À}{ la bonheur des dames}\blindtext\par
\noindent \initialiv{É}{tonnez-le}\blindtext\par
\noindent \initialiv{Q}{ualitativement}\blindtext\par
\noindent \initialiv{V}{aloriser}\blindtext\par
\Blindtext
\phantomsection
\label{utopie}
\Blinddocument
\fi
\end{document}
