%%%%%%%%%%%%%%%%%%%%%%%%%%%%%%%%%
% LaTeX model https://hurlus.fr %
%%%%%%%%%%%%%%%%%%%%%%%%%%%%%%%%%

% Needed before document class
\RequirePackage{pdftexcmds} % needed for tests expressions
\RequirePackage{fix-cm} % correct units

% Define mode
\def\mode{a4}

\newif\ifaiv % a4
\newif\ifav % a5
\newif\ifbooklet % booklet
\newif\ifcover % cover for booklet

\ifnum \strcmp{\mode}{cover}=0
  \covertrue
\else\ifnum \strcmp{\mode}{booklet}=0
  \booklettrue
\else\ifnum \strcmp{\mode}{a5}=0
  \avtrue
\else
  \aivtrue
\fi\fi\fi

\ifbooklet % do not enclose with {}
  \documentclass[french,twoside]{book} % ,notitlepage
  \usepackage[%
    papersize={105mm, 297mm},
    inner=12mm,
    outer=12mm,
    top=20mm,
    bottom=15mm,
    marginparsep=3pt,
    marginpar=7mm,
  ]{geometry}
  \usepackage[fontsize=9.5pt]{scrextend} % for Roboto
\else\ifav % A5
  \documentclass[french,twoside]{book} % ,notitlepage
  \usepackage[%
    a5paper
  ]{geometry}
  \usepackage[fontsize=12pt]{scrextend}
\else% A4 2 cols
  \documentclass[twocolumn]{report}
  \usepackage[%
    a4paper,
    inner=15mm,
    outer=10mm,
    top=25mm,
    bottom=18mm,
    marginparsep=0pt,
  ]{geometry}
  \setlength{\columnsep}{20mm}
  \usepackage[fontsize=9.5pt]{scrextend}
\fi\fi

%%%%%%%%%%%%%%
% Alignments %
%%%%%%%%%%%%%%
% before teinte macros

\setlength{\arrayrulewidth}{0.2pt}
\setlength{\columnseprule}{\arrayrulewidth} % twocol
\setlength{\parskip}{0pt} % 1pt allow better vertical justification
\setlength{\parindent}{1.5em}

%%%%%%%%%%
% Colors %
%%%%%%%%%%
% before Teinte macros

\usepackage[dvipsnames]{xcolor}
\definecolor{rubric}{HTML}{0c71c3} % the tonic
\def\columnseprulecolor{\color{rubric}}
\colorlet{borderline}{rubric!30!} % definecolor need exact code
\definecolor{shadecolor}{gray}{0.95}
\definecolor{bghi}{gray}{0.5}

%%%%%%%%%%%%%%%%%
% Teinte macros %
%%%%%%%%%%%%%%%%%
%%%%%%%%%%%%%%%%%%%%%%%%%%%%%%%%%%%%%%%%%%%%%%%%%%%
% <TEI> generic (LaTeX names generated by Teinte) %
%%%%%%%%%%%%%%%%%%%%%%%%%%%%%%%%%%%%%%%%%%%%%%%%%%%
% This template is inserted in a specific design
% It is XeLaTeX and otf fonts

\makeatletter % <@@@

\usepackage{alphalph} % for alph couter z, aa, ab…
\usepackage{blindtext} % generate text for testing
\usepackage{booktabs} % for tables: \toprule, \midrule…
\usepackage[strict]{changepage} % for modulo 4
\usepackage{contour} % rounding words
\usepackage[nodayofweek]{datetime}
\usepackage{enumitem} % <list>
\usepackage{etoolbox} % patch commands
\usepackage{fancyvrb}
\usepackage{fancyhdr}
\usepackage{float}
\usepackage{fontspec} % XeLaTeX mandatory for fonts
\usepackage{footnote} % used to capture notes in minipage (ex: quote)
\usepackage{framed} % bordering correct with footnote hack
\usepackage{graphicx}
\usepackage{lettrine} % drop caps
\usepackage{lipsum} % generate text for testing
\usepackage{manyfoot} % for parallel footnote numerotation
\usepackage[framemethod=tikz,]{mdframed} % maybe used for frame with footnotes inside
\usepackage[defaultlines=2,all]{nowidow} % at least 2 lines by par (works well!)
\usepackage{pdftexcmds} % needed for tests expressions
\usepackage{poetry} % <l>, bad for theater
\usepackage{polyglossia} % bug Warning: "Failed to patch part"
\usepackage[%
  indentfirst=false,
  vskip=1em,
  noorphanfirst=true,
  noorphanafter=true,
  leftmargin=\parindent,
  rightmargin=0pt,
]{quoting}
\usepackage{ragged2e}
\usepackage{setspace} % \setstretch for <quote>
\usepackage{scrextend} % KOMA-common, used for addmargin
\usepackage{tabularx} % <table>
\usepackage[explicit]{titlesec} % wear titles, !NO implicit
\usepackage{tikz} % ornaments
\usepackage{tocloft} % styling tocs
\usepackage[fit]{truncate} % used im runing titles
\usepackage{unicode-math}
\usepackage[normalem]{ulem} % breakable \uline, normalem is absolutely necessary to keep \emph
\usepackage{xcolor} % named colors
\usepackage{xparse} % @ifundefined
\XeTeXdefaultencoding "iso-8859-1" % bad encoding of xstring
\usepackage{xstring} % string tests
\XeTeXdefaultencoding "utf-8"

\defaultfontfeatures{
  % Mapping=tex-text, % no effect seen
  Scale=MatchLowercase,
  Ligatures={TeX,Common},
}
\newfontfamily\zhfont{Noto Sans CJK SC}

% Metadata inserted by a program, from the TEI source, for title page and runing heads
\title{\textbf{ Sous le Théâtre la Mer }\par
\medskip
\textit{ Histoire courte }\par
}
\date{2003}
\author{Richard Pernollet}
\def\elbibl{Richard Pernollet. 2003. \emph{Sous le Théâtre la Mer}}
\def\elabstract{%
 
\labelblock{Chapeau de l’auteur}

 \noindent Au départ, \emph{La Femme\footnote{Ce texte a été écrit en 2003}}, mais le titre était déjà beaucoup pris, mais finalement ça ne change pas le problème : un homme, une femme, à leur fenêtre, de part et d’autre d’une cour. Fenêtre sur Cour, Hitchcock. Un double escalier tournant où ils ont des difficultés à se croiser dans ces longs couloirs d’un ancien théâtre parisien en décrépitude. Monsieur Paul, l’homme, revient du Laos, décalé, et essaye dans sa piaule d’écrire des contes laotiens et regarde bien sûr La Femme en face en espérant la rencontrer. En bas, le concierge vient d’Algérie. Vous mixez, lent, bref, rapide, court, c’est ma manière d’essayer d’avoir du swing ! Enfin j’espère ! C’est gratuit !\par
 
\labelblock{Avertissement de l’édition}

 \noindent L’auteur étant bien vivant, ce texte est protégé par le droit d’auteur. Il a exprimé l’intention que l’“œuvre“ de son esprit (ce sont les termes de la loi, pas du tout les siens) soit librement communicable et partageable, mais sans profit commercial, en gardant le nom de l’auteur, et sans modification. Le choix de cette \href{https://creativecommons.org/licenses/by-nc-nd/4.0/deed.fr}{\dotuline{license CC BY-NC-ND}}\footnote{\href{https://creativecommons.org/licenses/by-nc-nd/4.0/deed.fr}{\url{https://creativecommons.org/licenses/by-nc-nd/4.0/deed.fr}}} n’est pas seulement l’effet d’opinions politiques, elle est aussi conséquente avec son esthétique. Son texte a été longtemps mûri, puis jeté tout d’un trait, il témoigne d’un moment, signé, daté. Ce texte est un morceau de temps libre qui ne peut plus être modifié ; de son temps à lui et personne d’autre. Ce temps, il ne le vend pas, il le partage.
 \newpage

}
\def\elsource{ \href{https://labarresymbolique.files.wordpress.com/2016/08/la-femme-revue2-2016.pdf}{\dotuline{LaBarreSymbolique}}\footnote{\href{https://labarresymbolique.files.wordpress.com/2016/08/la-femme-revue2-2016.pdf}{\url{https://labarresymbolique.files.wordpress.com/2016/08/la-femme-revue2-2016.pdf}}} }
\def\eltitlepage{%
{\centering\parindent0pt
  {\LARGE\addfontfeature{LetterSpace=25}\bfseries Richard Pernollet\par}\bigskip
  {\Large 2003\par}\bigskip
  {\LARGE
\bigskip\textbf{Sous le Théâtre la Mer}\par
\bigskip\emph{Histoire courte}\par

  }
}

}

% Default metas
\newcommand{\colorprovide}[2]{\@ifundefinedcolor{#1}{\colorlet{#1}{#2}}{}}
\colorprovide{rubric}{red}
\colorprovide{silver}{lightgray}
\@ifundefined{syms}{\newfontfamily\syms{DejaVu Sans}}{}
\newif\ifdev
\@ifundefined{elbibl}{% No meta defined, maybe dev mode
  \newcommand{\elbibl}{Titre court ?}
  \newcommand{\elbook}{Titre du livre source ?}
  \newcommand{\elabstract}{Résumé\par}
  \newcommand{\elurl}{http://oeuvres.github.io/elbook/2}
  \author{Éric Lœchien}
  \title{Un titre de test assez long pour vérifier le comportement d’une maquette}
  \date{1566}
  \devtrue
}{}
\let\eltitle\@title
\let\elauthor\@author
\let\eldate\@date




% generic typo commands
\newcommand{\astermono}{\medskip\centerline{\color{rubric}\large\selectfont{\syms ✻}}\medskip\par}%
\newcommand{\astertri}{\medskip\par\centerline{\color{rubric}\large\selectfont{\syms ✻\,✻\,✻}}\medskip\par}%
\newcommand{\asterism}{\bigskip\par\noindent\parbox{\linewidth}{\centering\color{rubric}\large{\syms ✻}\\{\syms ✻}\hskip 0.75em{\syms ✻}}\bigskip\par}%

% lists
\newlength{\listmod}
\setlength{\listmod}{\parindent}
\setlist{
  itemindent=!,
  listparindent=\listmod,
  labelsep=0.2\listmod,
  parsep=0pt,
  % topsep=0.2em, % default topsep is best
}
\setlist[itemize]{
  label=—,
  leftmargin=0pt,
  labelindent=1.2em,
  labelwidth=0pt,
}
\setlist[enumerate]{
  label={\arabic*°},
  labelindent=0.8\listmod,
  leftmargin=\listmod,
  labelwidth=0pt,
}
% list for big items
\newlist{decbig}{enumerate}{1}
\setlist[decbig]{
  label={\bf\color{rubric}\arabic*.},
  labelindent=0.8\listmod,
  leftmargin=\listmod,
  labelwidth=0pt,
}
\newlist{listalpha}{enumerate}{1}
\setlist[listalpha]{
  label={\bf\color{rubric}\alph*.},
  leftmargin=0pt,
  labelindent=0.8\listmod,
  labelwidth=0pt,
}
\newcommand{\listhead}[1]{\hspace{-1\listmod}\emph{#1}}

\renewcommand{\hrulefill}{%
  \leavevmode\leaders\hrule height 0.2pt\hfill\kern\z@}

% General typo
\DeclareTextFontCommand{\textlarge}{\large}
\DeclareTextFontCommand{\textsmall}{\small}

% commands, inlines
\newcommand{\anchor}[1]{\Hy@raisedlink{\hypertarget{#1}{}}} % link to top of an anchor (not baseline)
\newcommand\abbr[1]{#1}
\newcommand{\autour}[1]{\tikz[baseline=(X.base)]\node [draw=rubric,thin,rectangle,inner sep=1.5pt, rounded corners=3pt] (X) {\color{rubric}#1};}
\newcommand\corr[1]{#1}
\newcommand{\ed}[1]{ {\color{silver}\sffamily\footnotesize (#1)} } % <milestone ed="1688"/>
\newcommand\expan[1]{#1}
\newcommand\foreign[1]{\emph{#1}}
\newcommand\gap[1]{#1}
\renewcommand{\LettrineFontHook}{\color{rubric}}
\newcommand{\initial}[2]{\lettrine[lines=2, loversize=0.3, lhang=0.3]{#1}{#2}}
\newcommand{\initialiv}[2]{%
  \let\oldLFH\LettrineFontHook
  % \renewcommand{\LettrineFontHook}{\color{rubric}\ttfamily}
  \IfSubStr{QJ’}{#1}{
    \lettrine[lines=4, lhang=0.2, loversize=-0.1, lraise=0.2]{\smash{#1}}{#2}
  }{\IfSubStr{É}{#1}{
    \lettrine[lines=4, lhang=0.2, loversize=-0, lraise=0]{\smash{#1}}{#2}
  }{\IfSubStr{ÀÂ}{#1}{
    \lettrine[lines=4, lhang=0.2, loversize=-0, lraise=0, slope=0.6em]{\smash{#1}}{#2}
  }{\IfSubStr{A}{#1}{
    \lettrine[lines=4, lhang=0.2, loversize=0.2, slope=0.6em]{\smash{#1}}{#2}
  }{\IfSubStr{V}{#1}{
    \lettrine[lines=4, lhang=0.2, loversize=0.2, slope=-0.5em]{\smash{#1}}{#2}
  }{
    \lettrine[lines=4, lhang=0.2, loversize=0.2]{\smash{#1}}{#2}
  }}}}}
  \let\LettrineFontHook\oldLFH
}
\newcommand{\labelchar}[1]{\textbf{\color{rubric} #1}}
\newcommand{\lnatt}[1]{\reversemarginpar\marginpar[\sffamily\scriptsize #1]{}}
\newcommand{\milestone}[1]{\autour{\footnotesize\color{rubric} #1}} % <milestone n="4"/>
\newcommand\name[1]{#1}
\newcommand\orig[1]{#1}
\newcommand\orgName[1]{#1}
\newcommand\persName[1]{#1}
\newcommand\placeName[1]{#1}
\newcommand{\pn}[1]{\IfSubStr{-—–¶}{#1}% <p n="3"/>
  {\noindent{\bfseries\color{rubric}   ¶  }}
  {{\footnotesize\autour{#1}}}}
\newcommand\reg{}
% \newcommand\ref{} % already defined
\newcommand\sic[1]{#1}
\newcommand\surname[1]{\textsc{#1}}
\newcommand\term[1]{\textbf{#1}}
\newcommand\zh[1]{{\zhfont #1}}


\def\mednobreak{\ifdim\lastskip<\medskipamount
  \removelastskip\nopagebreak\medskip\fi}
\def\bignobreak{\ifdim\lastskip<\bigskipamount
  \removelastskip\nopagebreak\bigskip\fi}

% commands, blocks

\newcommand{\byline}[1]{\bigskip{\RaggedLeft{#1}\par}\bigskip}
% \setlength{\RaggedLeftLeftskip}{2em plus \leftskip}
\newcommand{\bibl}[1]{{\smallskip\RaggedLeft\normalsize\normalfont #1\par\medskip}}
\newcommand{\biblitem}[1]{{\noindent\hangindent=\parindent   #1\par}}
\newcommand{\castItem}[1]{{\noindent\hangindent=\parindent #1\par}}
\newcommand{\dateline}[1]{\medskip{\RaggedLeft{#1}\par}\bigskip}
\newcommand{\docAuthor}[1]{{\large\bigskip #1 \par\bigskip}}
\newcommand{\docDate}[1]{#1 \ifvmode\par\fi }
\newcommand{\docImprint}[1]{\ifvmode\medskip\fi #1 \ifvmode\par\fi }
\newcommand{\labelblock}[1]{\medbreak{\noindent\color{rubric}\bfseries #1}\par\mednobreak}
\newcommand{\salute}[1]{\bigbreak{#1}\par\medbreak}
\newcommand{\signed}[1]{\medskip{\RaggedLeft #1\par}\bigbreak} % supposed bottom
\newcommand{\speaker}[1]{\medskip{\Centering\sffamily #1 \par\nopagebreak}} % supposed bottom
\newcommand{\stagescene}[1]{{\Centering\sffamily\textsf{#1}\par}\bigskip}
\newcommand{\stageblock}[1]{\begingroup\leftskip\parindent\noindent\it\sffamily\footnotesize #1\par\endgroup} % left margin, better than list envs
\newcommand{\spl}[1]{\noindent\hangindent=2\parindent  #1\par} % sp/l
\newcommand{\trailer}[1]{{\Centering\bigskip #1\par}} % sp/l

% environments for blocks (some may become commands)
\newenvironment{borderbox}{}{} % framing content
\newenvironment{citbibl}{\ifvmode\hfill\fi}{\ifvmode\par\fi }
\newenvironment{msHead}{\vskip6pt}{\par}
\newenvironment{msItem}{\vskip6pt}{\par}


% environments for block containers
\newenvironment{argument}{\itshape\parindent0pt}{\bigskip}
\newenvironment{biblfree}{}{\ifvmode\par\fi }
\newenvironment{bibitemlist}[1]{%
  \list{\@biblabel{\@arabic\c@enumiv}}%
  {%
    \settowidth\labelwidth{\@biblabel{#1}}%
    \leftmargin\labelwidth
    \advance\leftmargin\labelsep
    \@openbib@code
    \usecounter{enumiv}%
    \let\p@enumiv\@empty
    \renewcommand\theenumiv{\@arabic\c@enumiv}%
  }
  \sloppy
  \clubpenalty4000
  \@clubpenalty \clubpenalty
  \widowpenalty4000%
  \sfcode`\.\@m
}%
{\def\@noitemerr
  {\@latex@warning{Empty `bibitemlist' environment}}%
\endlist}
\newenvironment{docTitle}{\LARGE\bigskip\bfseries\onehalfspacing}{\bigskip}
% leftskip makes big bugs in Lexmark printing \sffamily
\newenvironment{epigraph}{\begin{addmargin}[2\parindent]{0em}\sffamily\large\setstretch{0.95}}{\end{addmargin}\bigskip}
\newenvironment{quoteblock}% may be used for ornaments
  {\begin{quoting}}
  {\end{quoting}}
\newenvironment{titlePage}
  {\Centering}
  {}






% table () is preceded and finished by custom command
\renewcommand\tabularxcolumn[1]{m{#1}}% for vertical centering text in X column
\newcommand{\tableopen}[1]{%
  \ifnum\strcmp{#1}{wide}=0{%
    \begin{center}
  }
  \else\ifnum\strcmp{#1}{long}=0{%
    \begin{center}
  }
  \else{%
    \begin{center}
  }
  \fi\fi
}
\newcommand{\tableclose}[1]{%
  \ifnum\strcmp{#1}{wide}=0{%
    \end{center}
  }
  \else\ifnum\strcmp{#1}{long}=0{%
    \end{center}
  }
  \else{%
    \end{center}
  }
  \fi\fi
}


% text structure
\newcommand\chapteropen{} % before chapter title
\newcommand\chaptercont{} % after title, argument, epigraph…
\newcommand\chapterclose{} % maybe useful for multicol settings
\setcounter{secnumdepth}{-2} % no counters for hierarchy titles
\setcounter{tocdepth}{5} % deep toc
\renewcommand\tableofcontents{\@starttoc{toc}}
% toclof format
% \renewcommand{\@tocrmarg}{0.1em} % Useless command?
% \renewcommand{\@pnumwidth}{0.5em} % {1.75em}
\renewcommand{\@cftmaketoctitle}{}
\setlength{\cftbeforesecskip}{\z@ \@plus.2\p@}
\renewcommand{\cftchapfont}{}
\renewcommand{\cftchapdotsep}{\cftdotsep}
\renewcommand{\cftchapleader}{\normalfont\cftdotfill{\cftchapdotsep}}
\renewcommand{\cftchappagefont}{\bfseries}
\setlength{\cftbeforechapskip}{0em \@plus\p@}
% \renewcommand{\cftsecfont}{\small\relax}
\renewcommand{\cftsecpagefont}{\normalfont}
% \renewcommand{\cftsubsecfont}{\small\relax}
\renewcommand{\cftsecdotsep}{\cftdotsep}
\renewcommand{\cftsecpagefont}{\normalfont}
\renewcommand{\cftsecleader}{\normalfont\cftdotfill{\cftsecdotsep}}
\setlength{\cftsecindent}{1em}
\setlength{\cftsubsecindent}{2em}
\setlength{\cftsubsubsecindent}{3em}
\setlength{\cftchapnumwidth}{1em}
\setlength{\cftsecnumwidth}{1em}
\setlength{\cftsubsecnumwidth}{1em}
\setlength{\cftsubsubsecnumwidth}{1em}

% footnotes
\newif\ifheading
\newcommand*{\fnmarkscale}{\ifheading 0.70 \else 1 \fi}
\renewcommand\footnoterule{\vspace*{0.3cm}\hrule height \arrayrulewidth width 3cm \vspace*{0.3cm}}
\setlength\footnotesep{1.5\footnotesep} % footnote separator
\renewcommand\@makefntext[1]{\parindent 1.5em \noindent \hb@xt@1.8em{\hss{\normalfont\@thefnmark . }}#1} % no superscipt in foot
\patchcmd{\@footnotetext}{\footnotesize}{\footnotesize\sffamily}{}{} % before scrextend, hyperref
\DeclareNewFootnote{A}[alph] % for editor notes
\renewcommand*{\thefootnoteA}{\alphalph{\value{footnoteA}}} % z, aa, ab…

% poem
\setlength{\poembotskip}{0pt}
\setlength{\poemtopskip}{0pt}
\setlength{\poemindent}{0pt}
\poemlinenumsfalse

%   see https://tex.stackexchange.com/a/34449/5049
\def\truncdiv#1#2{((#1-(#2-1)/2)/#2)}
\def\moduloop#1#2{(#1-\truncdiv{#1}{#2}*#2)}
\def\modulo#1#2{\number\numexpr\moduloop{#1}{#2}\relax}

% orphans and widows, nowidow package in test
% from memoir package
\clubpenalty=9996
\widowpenalty=9999
\brokenpenalty=4991
\predisplaypenalty=10000
\postdisplaypenalty=1549
\displaywidowpenalty=1602
\hyphenpenalty=400
% report h or v overfull ?
\hbadness=4000
\vbadness=4000
% good to avoid lines too wide
\emergencystretch 3em
\pretolerance=750
\tolerance=2000
\def\Gin@extensions{.pdf,.png,.jpg,.mps,.tif}

\PassOptionsToPackage{hyphens}{url} % before hyperref and biblatex, which load url package
\usepackage{hyperref} % supposed to be the last one, :o) except for the ones to follow
\hypersetup{
  % pdftex, % no effect
  pdftitle={\elbibl},
  % pdfauthor={Your name here},
  % pdfsubject={Your subject here},
  % pdfkeywords={keyword1, keyword2},
  bookmarksnumbered=true,
  bookmarksopen=true,
  bookmarksopenlevel=1,
  pdfstartview=Fit,
  breaklinks=true, % avoid long links, overrided by url package
  pdfpagemode=UseOutlines,    % pdf toc
  hyperfootnotes=true,
  colorlinks=false,
  pdfborder=0 0 0,
  % pdfpagelayout=TwoPageRight,
  % linktocpage=true, % NO, toc, link only on page no
}
\urlstyle{same} % after hyperref



\makeatother % /@@@>
%%%%%%%%%%%%%%
% </TEI> end %
%%%%%%%%%%%%%%


%%%%%%%%%%%%%
% footnotes %
%%%%%%%%%%%%%
\renewcommand{\thefootnote}{\bfseries\textcolor{rubric}{\arabic{footnote}}} % color for footnote marks

%%%%%%%%%
% Fonts %
%%%%%%%%%
% \linespread{0.90} % too compact, keep font natural
\ifav % A5
  \usepackage{DejaVuSans} % correct
  \setsansfont{DejaVuSans} % seen, if not set, problem with printer
\else\ifbooklet
  \usepackage[]{roboto} % SmallCaps, Regular is a bit bold
  \setmainfont[
    ItalicFont={Roboto Light Italic},
  ]{Roboto}
  \setsansfont{Roboto Light} % seen, if not set, problem with printer
  \newfontfamily\fontrun[]{Roboto Condensed Light} % condensed runing heads
\else
  \usepackage[]{roboto} % SmallCaps, Regular is a bit bold
  \setmainfont[
    ItalicFont={Roboto Italic},
  ]{Roboto Light}
  \setsansfont{Roboto Light} % seen, if not set, problem with printer
  \newfontfamily\fontrun[]{Roboto Condensed Light} % condensed runing heads
\fi\fi
\renewcommand{\LettrineFontHook}{\bfseries\color{rubric}}
% \renewenvironment{labelblock}{\begin{center}\bfseries\color{rubric}}{\end{center}}

%%%%%%%%
% MISC %
%%%%%%%%

\setdefaultlanguage[frenchpart=false]{french} % bug on part


\newenvironment{quotebar}{%
    \def\FrameCommand{{\color{rubric!10!}\vrule width 0.5em} \hspace{0.9em}}%
    \def\OuterFrameSep{0pt} % séparateur vertical
    \MakeFramed {\advance\hsize-\width \FrameRestore}
  }%
  {%
    \endMakeFramed
  }
\renewenvironment{quoteblock}% may be used for ornaments
  {%
    \savenotes
    \setstretch{0.9}
    \begin{quotebar}
    \smallskip
  }
  {%
    \smallskip
    \end{quotebar}
    \spewnotes
  }


\renewcommand{\headrulewidth}{\arrayrulewidth}
\renewcommand{\headrule}{{\color{rubric}\hrule}}
\renewcommand{\lnatt}[1]{\marginpar{\sffamily\scriptsize #1}}

% delicate tuning, image has produce line-height problems in title on 2 lines
\titleformat{name=\chapter} % command
  [display] % shape
  {\vspace{1.5em}\centering} % format
  {} % label
  {0pt} % separator between n
  {}
[{\color{rubric}\huge\textbf{#1}}\bigskip] % after code
% \titlespacing{command}{left spacing}{before spacing}{after spacing}[right]
\titlespacing*{\chapter}{0pt}{-2em}{0pt}[0pt]

\titleformat{name=\section}
  [display]{}{}{}{}
  [\vbox{\color{rubric}\large\centering\textbf{#1}}]
\titlespacing{\section}{0pt}{0pt plus 4pt minus 2pt}{\baselineskip}

\titleformat{name=\subsection}
  [block]
  {}
  {} % \thesection
  {} % separator \arrayrulewidth
  {}
[\vbox{\large\textbf{#1}}]
% \titlespacing{\subsection}{0pt}{0pt plus 4pt minus 2pt}{\baselineskip}

\ifaiv
  \fancypagestyle{main}{%
    \fancyhf{}
    \setlength{\headheight}{1.5em}
    \fancyhead{} % reset head
    \fancyfoot{} % reset foot
    \fancyhead[L]{\truncate{0.45\headwidth}{\fontrun\elbibl}} % book ref
    \fancyhead[R]{\truncate{0.45\headwidth}{ \fontrun\nouppercase\leftmark}} % Chapter title
    \fancyhead[C]{\thepage}
  }
  \fancypagestyle{plain}{% apply to chapter
    \fancyhf{}% clear all header and footer fields
    \setlength{\headheight}{1.5em}
    \fancyhead[L]{\truncate{0.9\headwidth}{\fontrun\elbibl}}
    \fancyhead[R]{\thepage}
  }
\else
  \fancypagestyle{main}{%
    \fancyhf{}
    \setlength{\headheight}{1.5em}
    \fancyhead{} % reset head
    \fancyfoot{} % reset foot
    \fancyhead[RE]{\truncate{0.9\headwidth}{\fontrun\elbibl}} % book ref
    \fancyhead[LO]{\truncate{0.9\headwidth}{\fontrun\nouppercase\leftmark}} % Chapter title, \nouppercase needed
    \fancyhead[RO,LE]{\thepage}
  }
  \fancypagestyle{plain}{% apply to chapter
    \fancyhf{}% clear all header and footer fields
    \setlength{\headheight}{1.5em}
    \fancyhead[L]{\truncate{0.9\headwidth}{\fontrun\elbibl}}
    \fancyhead[R]{\thepage}
  }
\fi

\ifav % a5 only
  \titleclass{\section}{top}
\fi

\newcommand\chapo{{%
  \vspace*{-3em}
  \centering\parindent0pt % no vskip ()
  \eltitlepage
  \bigskip
  {\color{rubric}\hline}
  \bigskip
  {\Large TEXTE LIBRE À PARTICIPATIONS LIBRES\par}
  \centerline{\small\color{rubric} {\href{https://hurlus.fr}{\dotuline{hurlus.fr}}}, tiré le \today}\par
  \bigskip
}}

\newcommand\cover{{%
  \thispagestyle{empty}
  \centering\parindent0pt
  \eltitlepage
  \vfill\null
  {\color{rubric}\setlength{\arrayrulewidth}{2pt}\hline}
  \vfill\null
  {\Large TEXTE LIBRE À PARTICIPATIONS LIBRES\par}
  \centerline{\href{https://hurlus.fr}{\dotuline{hurlus.fr}}, tiré le \today}\par
}}

\begin{document}
\pagestyle{empty}
\ifbooklet{
  \cover\newpage
  \thispagestyle{empty}\hbox{}\newpage
  \cover\newpage\noindent Les voyages de la brochure\par
  \bigskip
  \begin{tabularx}{\textwidth}{l|X|X}
    \textbf{Date} & \textbf{Lieu}& \textbf{Nom/pseudo} \\ \hline
    \rule{0pt}{25cm} &  &   \\
  \end{tabularx}
  \newpage
  \addtocounter{page}{-4}
}\fi

\thispagestyle{empty}
\ifaiv
  \twocolumn[\chapo]
\else
  \chapo
\fi
{\it\elabstract}
\bigskip
\makeatletter\@starttoc{toc}\makeatother % toc without new page
\bigskip

\pagestyle{main} % after style
\setcounter{footnote}{0}
\setcounter{footnoteA}{0}
  
\section[{— 1 —}]{— 1 —}
\renewcommand{\leftmark}{— 1 —}

\noindent Quand on a la chance comme moi \emph{– certains diront que ce n’en est pas une mais pour moi c’en est une et c’est d’ailleurs un point de vue qui se discute –} d’habiter un studio qui donne sur une cour, on se refait inévitablement le scénario d’Hitchcock de \emph{Fenêtre sur Cour}, on regarde ses voisins.\par
Tous les soirs, ainsi, je peux voir La Femme qui habite en face de chez moi. Elle rentre à 20h, elle allume la télé et ferme le rideau. Je l’appelle comme ça La Femme parce que je n’en sais pas plus. Elle rentre le soir vers 20 heures, allume, traverse la pièce, allume la télé, puis vient tirer le rideau. Voilà, c’est tout.\par
Un jour, je me suis dit, je lui dirais bonjour.\par
― BONJOUR !\par
― BONJOUR !\par
Monsieur Mostefari, notre concierge, me le rappelle tous les jours, c’est important de dire bonjour, c’est la preuve qu’on s’accroche à la vie, qu’on veut tenir debout.\par
― BONJOUR MONSIEUR PAUL !\par
\bigbreak
\noindent Voilà, c’est Monsieur Mostefari, en bras de chemise comme toujours devant sa porte, la voix forte et l’accent méditerranéen. 17 ans de loge derrière lui. Abderamane, de son prénom, fils de Harki, d’origine kabyle, ce qui est normalement contradictoire mais c’était la faute à son père, dit-il, qui avait fait le mauvais choix.\par
Lui, il avait grandi à Lodève.\par
Il m’avait raconté un peu sa vie.\par
Grandir parqués à Lodève ! Oui ! Enfin presque. 1962, l’Algérie perdue, la grande valse des Pieds-Noirs, il fallait choisir, les collabos avaient intérêt à suivre, il avait cinq ans.\par
Et puis voilà, ce fut d’abord Saint-Laurent du Var, première étape, des baraquements en tôle et des fils de fer autour pour remercier les soldats \emph{indigènes}, et ensuite Lodève, à la porte des Cévennes, mais déjà sinistrée, la mère employée pour ses petites mains à la Manufacture des Gobelins, les tapisseries. Le père qui râle ou soupire en espérant une pension. \emph{« Et nous, les gosses, rien ! Pas de fric ! Pas bouger ! Pas possible ! On passe des journées à glander ! »}

\section[{— 2 —}]{— 2 —}
\renewcommand{\leftmark}{— 2 —}

― BONJOUR MONSIEUR PAUL !\par
C’est vrai que parfois je suis distrait, mais Monsieur Mostefari tient tellement à ce qu’on lui dise bonjour que si jamais on l’oublie c’est lui qui nous le rappelle. Quand on passe devant sa loge – HÉ ! – il appelle. Je crois qu’il m’a réappris à dire bonjour.\par
— BONJOUR ! BONJOUR MONSIEUR MOSTEFARI !\par
Maintenant c’est devenu un exercice, un exercice de sourire, d’appel à la vie. Bien sûr, je suis au chômedu, mais bon dieu que la vie est belle, j’écris des contes laotiens !\par
Ah, le Laos ! – La première fois que j’en ai parlé à Monsieur Mostefari il m’a répondu MÉKONG. Et ça m’a fait du bien. Coup de gong.\par
Échanges.\par
Communauté d’esprit.\par
Bavardages.\par
Je lui en sus gré de me dispenser d’un cours de géographie. – MÉKONG !\par
— Bonjour, Monsieur Paul !\par
Du coup, je souris à tout le monde, je dis bonjour, \emph{salut, salut} !, je fais bonne figure. Présenter le meilleur aspect de soi-même. Ça fait trois semaines que suis là et je me dis déjà que bientôt, à ce régime, je connaîtrai tout le monde.\par
Bientôt il ne me restera plus que La Femme à qui je n’aurai pas dit bonjour. Et elle, elle rentre tous les soirs à 20 heures. A 23 heures elle sort. C’est un oiseau de nuit.

\section[{— 3 —}]{— 3 —}
\renewcommand{\leftmark}{— 3 —}

\noindent Et voilà : il est 20 heures, l’heure des infos, La Femme rentre, allume la lumière, branche la télé, vient ce soir ouvrir la fenêtre et reste accoudée pour fumer sa cigarette dehors.\par
Elle est brune et assez élancée. Regarde la cour.\par
Ici, c’est un puits.\par
Surtout la nuit.\par
Un ancien théâtre.\par
Toutes les loges ont été transformées en studios ou appartements. De son côté, elle a le soleil la journée, quand il y en a. Et, lorsqu’il est là, elle n’est souvent pas là, je peux voir son chat.\par
Il vient assez souvent se chauffer sur le bord de la fenêtre, reste un moment derrière les vitres puis disparaît dans les profondeurs de l’appartement. C’est un chat noir.\par
Ce soir, il est à côté d’elle, se frotte contre elle et regarde la cour lui aussi.\par
Moi, de mon côté, je peux me lever de bonne heure pour avoir le soleil, chez moi je suis toujours à l’ombre, c’est carrément le trou noir. Enfin je l’ai voulu. Le hasard en a décidé ainsi. J’ai La Femme en face de chez moi et je les regarde tous les deux, ensemble ils s’aèrent, j’aurais presque envie d’ouvrir la fenêtre moi aussi. De leur crier BONJOUR !\par
De taper la causette.\par
— \emph{Ça va} ?\par
\bigbreak
\noindent Pourquoi je ne le fais pas ? Début d’année tranquille, janvier sera mou, janvier est déjà doux, rien ne bouge, le Nouvel An est passé, c’est déjà la galette des Rois. Voulez-vous être ma Reine ? Je suis en panne moi aussi.\par
Elle finit sa cigarette.\par
La jette dans la cour.\par
Se recule.\par
Le chat descend.\par
Elle ferme la fenêtre.\par
Tire le rideau.

\section[{— 4 —}]{— 4 —}
\renewcommand{\leftmark}{— 4 —}

\noindent Depuis trois semaines donc que je suis là, ma vie se cadence à ce rythme d’enfer – 20 heures – l’heure des infos – La Femme – le tiré de rideau.\par
J’appelle ça mon petit retour au réel puisque le reste de la journée je ne fous rien, je reste suspendu devant mes contes laotiens.\par
Il est 20 heures, La Femme rentre.\par
A 23 heures, elle sort.\par
J’entends ses pas dans le couloir, je me concentre, j’imagine, dans quelle tenue est-elle ce soir ? Talons moyens certainement, vu le bruit sur le plancher. Nous partageons le même couloir. Le sien part à droite après l’escalier. Le mien à gauche. Tous les deux font un coude. Puis c’est la lignée des loges. Toutes les fenêtres donnent sur cour. Sauf devant l’escalier où elles donnent à la fois sur cour et sur l’escalier. C’est très beau certains jours, surtout quand il y a du soleil. C’est un très vieux théâtre du XVIII\textsuperscript{ème}, un peu en vrac aujourd’hui mais quand même.\par
Voilà, elle arrive à la rampe. C’est le moment du jeu de hanches. Elle balance. Enfin j’imagine ! Moi aussi j’attrape la rampe. Moi aussi je descends. Sa robe vole. Ses petits pas descendent. En général elle est toujours très classe, en robe ou en jupe, pantalons de temps en temps, de temps en temps un chapeau à plume…\par
Ses pas décroissent dans l’escalier.\par
Toc-toc.\par
Réapparaissent dans la cour.\par
Re-toc-toc.\par
Puis s’en vont.\par
J’en étais où dans mes contes laotiens ?\par
Le Mékong est un fleuve tranquille qui prend sa source en Chine.

\section[{— 5 —}]{— 5 —}
\renewcommand{\leftmark}{— 5 —}

\noindent \emph{Grandir à Lodève ! –} Si vous n’êtes jamais passé par Lodève, vous prenez l’A 75 à la sortie de Montpellier ou bien, dans l’autre sens, en descendant du Larzac. D’un côté, le Caylar et le Causse de la Celle. De l’autre, Lodève, la barre des Cévennes. C’est très beau. Il y a la montagne, l’air pur, le soleil, mais que c’est triste Lodève ! Même sous le soleil on sent la fin de quelque chose. Des maisons riches. Un évêché. Des maisons grises. Des usines fermées. Anciennes filatures perdues depuis la fin de l’industrie du ver à soie. Aujourd’hui qu’est-ce qu’on fait à Lodève ? Rien. On ne fait rien. J’ai passé ma jeunesse à glander, dit Monsieur Mostefari, attendre, glander. La vie s’écoule, on vieillit, et c’est toujours la même chose. Petits boulots. Intérim. Chômage. Travaux des champs. Gardiennage. Stages de reconversion. Et puis voilà, il y a dix-sept ans, la patrie reconnaissante, la France qui fait un geste, on offre au fils perdu un poste de concierge à Paris. Je fonce. Je monte à Paris. J’emmène ma famille. Les enfants grandissent ! Se marient ! Je suis déjà deux fois grand-père et je suis toujours là. Et vous, Monsieur Paul, vous avez des enfants ?\par
— Non !\par
— Vous avez tort, c’est la vie, les enfants !\par
Aujourd’hui, il se tape la cinquantaine, tout l’été ça avait encore pété dans les cités harkis du Sud, il avait suivi ça à la télé. Les fils de Harkis réclament le droit d’être Français, traités comme tout le monde, pas pris pour des chiens ou des moins que rien quand ils sortent dans la rue. – \emph{Comment voulez-vous, Monsieur Paul, qu’on ne soit pas aigri} ?\par
Ses frères étaient restés à Lodève.\par
Au début, c’est vrai, concierge, c’est pas mal, on se dit que ça ne va pas durer, et puis ça dure, et maintenant ça fait dix-sept ans ! Dix-sept ans ! Qu’est-ce que vous voulez que je vous dise ?\par
J’espère que ça n’arrivera pas à mes enfants !

\section[{— 6 —}]{— 6 —}
\renewcommand{\leftmark}{— 6 —}

\noindent J’ai tout fait pourtant pour qu’ils s’en sortent, mais qu’est-ce que vous voulez, Monsieur Paul, quand vous êtes né Harki, ça vous accompagne toute une vie, vous, vos enfants, petits-enfants, arrière-arrière-arrière-petits-enfants, tant qu’il restera encore quelqu’un pour vous rappeler qui vous êtes, d’où vous venez ! Harkis ! Fils de Harkis ! Vendus aux Français !\par
— Ici, pour vous, c’est simple, on est Beurs, Rebeus, Maghrébins, Arabes, mais pour nous ? Tous mes enfants sont nés en France. Quand on retourne là-bas et qu’on nous traite toujours de fils de pute ou d’enfants de salauds, comment voulez-vous que les gosses s’y retrouvent ?\par
L’autre jour, il me dit :\par
— Vous aimez les contes, Monsieur Paul ?\par
— Pourquoi vous me demandez ça ?\par
— Parce que j’essaye en ce moment de ne pas faire la même erreur que mon père avec mes enfants, oui, mon père ne disait rien, il avait honte, j’essaye de raconter à mes petits-enfants les histoires que me racontait ma grand-mère quand elle voulait me parler de la Kabylie. De temps en temps elle chantait. Chez nous, ce sont les vieux qui assurent la transmission. Des Oncles. Des Tantes. Le Père, en général, il est sévère. C’est pas avec lui qu’on parle le plus… Enfin, j’aimerais pas vous en dire trop mais je cherche quelqu’un. Pour m’aider. Je voudrais les écrire un peu. Vous voyez ce que je veux dire ? Quelqu’un prend note. Je raconte. J’aimerais laisser une trace.

\section[{— 7 —}]{— 7 —}
\renewcommand{\leftmark}{— 7 —}

On s’était pourtant presque engueulé la première fois.\par
— Le loyer se paye en début de mois, Monsieur Paul !\par
— Mais je vous ai déjà payé mardi !\par
— C’était le mois dernier, Monsieur Paul !\par
— Hé là, ça va vite chez vous les mois à Paris, ça fait juste deux jours que je suis ici.\par
— Et oui, avant-hier c’était l’année dernière, aujourd’hui c’est la nouvelle année, vous savez bien que tout se paye aujourd’hui, Monsieur Paul !\par
— Je sais. La tranquillité. Le silence.\par
— Hé oui ! a dit Monsieur Mostefari.\par
Il a pris un air triste. Désolé. Ce n’est pas lui qui décidait. N’est-ce pas ce que je cherchais ?\par
— Quoi ?\par
— Un appartement en sous-location, je dis bien même un appartement en sous-location, c’est tellement galère en ce moment de trouver à se loger dans Paris !\par
— Ah ?\par
Il a réfléchi, il m’a dit que peut-être cette fille qui était partie en Colombie. Oui. Là-haut ! \emph{« Ici, tous les gens disparaissent, partent, des vacances perpétuelles, je ne sais pas ; mais ils veulent tous garder un pied-à-terre à Paris ! »} J’ai dit que j’étais prêt à payer tout de suite \emph{– un mois ? –} que c’était urgent, et on s’est arrangé comme ça, en sous-main.\par
Mais quand deux jours après il est venu me relancer avec cette histoire de loyer, c’est moi qui ai pété les plombs. Oui.\par
— Mais dites, c’est pour qui tout ce black, pour vous ou pour la Mairie de Paris ?\par
— Monsieur Paul, voyons, si vous étiez passé par une agence… !\par
— Je sais !\par
— Mais ce n’est pas moi qui irai vous dénoncer !\par
— Ah !\par
En quelques mois, je me disais, les choses avaient bien changé. On part, on pense mettre un peu distance et, quand on revient, la Terre a vieilli de 250 ans. Tout s’accélère.\par
J’ai eu un sourire stupide.\par
— Et si je vous dis que je ne peux pas ! j’ai dit.\par
— Ah, ah la-la, Monsieur Paul… ça ne peut vous créer que des ennuis !\par
Je suis passé en silence. Je rageais. De quand dataient ces nouveaux usages ?

\section[{— 8 —}]{— 8 —}
\renewcommand{\leftmark}{— 8 —}

\noindent Je ne savais si je devais en rire ou en pleurer, j’avais envie de crier, le monde dans lequel je revenais ne me semblait plus rien avoir de réel. On ne parlait qu’argent. L’Argent ! L’Argent ! C’était encore pire qu’avant ! Et des concierges jouaient maintenant les gardiens de Théâtre en gage de nos tranquillités. Où on va ?\par
Enfin, je voyais les choses comme ça, un peu trop buté, déterminé sans doute, ce n’était peut-être pas vrai, c’était moi certainement qui me racontais des histoires, je revenais, j’avais du mal à atterrir, je me suis dit alors, mon Paulo, qu’est-ce qu’on fait, tu bosses ou tu refais tes valises ? J’ai pris la solution extrême, j’ai décidé d’écrire ces contes laotiens et ne veux plus sortir. Tant pis.\par
Est-ce que ça se vend des contes laotiens ? J’en sais rien. Le Mékong est un long fleuve tranquille qui prend sa source en Chine. \emph{Pascal Nouma passe à Petit, longue passe en avant, Trézeguet récupère, passe à Zidane… oh là là, tacle sur Zidane… TÂÂÂCLE SUR ZIDANE ET MONSIEUR L’ARBITRE NE DIT RIEN !… Ça c’est un peu fort de café} !\par
Et payer deux mois d’avance pour avoir le prix du silence ?\par
Il m’a rattrapé dans la cour.\par
— Tenez, votre voisine du dessus…\par
— Oui ?\par
— Et bien, elle est partie au Tibet !\par
— Et alors ?\par
— Et ça fait neuf mois !\par
— Oui et alors ?\par
— Et je n’ai toujours rien dit !\par
— Neuf mois !\par
— Oui !\par
Pendant deux jours on ne s’est rien dit.

\section[{— 9 —}]{— 9 —}
\renewcommand{\leftmark}{— 9 —}

\noindent La première fois que j’ai croisé La Femme, je revenais du marché, elle descendait les escaliers. Je suis resté en bas, à attendre très poli qu’elle passe, et elle est passée. J’ai vu d’abord ses jambes, vaguement son sourire, puis j’ai baissé les yeux et elle est passée.\par
— BONJOUR !\par
— Bonjour… et merci !\par
— Moi, c’est Paul, enfin Monsieur Paul !\par
— Ah, Monsieur Paul ! Enchantée ! Vous n’avez qu’à m’appeler la \emph{Sanseverina} !\par
Et elle est passée.\par
Quand j’ai entendu ses pas dans la cour, je me suis retourné. J’ai vu alors son cul. Elle balançait son petit sac.\par
Elle était bien foutue.\par
Elle portait un chapeau à plume.\par
— BONJOUR !\par
Je garde encore ce soir l’image de ses lèvres.\par
Elles étaient rouges !\par
À la tombée du jour le Mékong est un fleuve assez large de couleur plutôt rose orangé. Les enfants rient. Les gens rentrent dans l’eau. Les \emph{waterbuffalos} rentrent dans l’eau, les buffles, l’eau est lisse. Il y a des cocotiers. Le coucher du soleil est rose. Les robes des moines sont jaunes.\par
Je l’ai revue l’autre jour au café !\par
Elle rentrait pour acheter des cigarettes, j’étais au comptoir, c’est elle la première qui m’a dit bonjour et je lui ai dit bonjour.\par
— AH, BONJOUR MONSIEUR PAUL ! ÇA VA ?\par
Elle portait un costume sombre, façon garçon, le pantalon flottant, cheveux tirés en arrière presque danseuse de flamenco. Et pendant qu’elle cherchait dans son sac pour trouver son porte-monnaie, elle m’a dit :\par
{\itshape « Et vous, le boulot, ça va ? »\par}
J’ai dit ça va, ça va, deux fois, et elle a commandé deux Benson Light.\par
« Merci !… Allez, au revoir, Monsieur Paul, bonne journée ! Venez donc prendre le café dimanche ! J’aime bien connaître mes voisins ! »\par
Et elle est partie.

\section[{— 10 —}]{— 10 —}
\renewcommand{\leftmark}{— 10 —}

\noindent Je l’ai revue ce matin, dans le train. Les mornes plaines de Picardie défilent. Je devais faire un Paris Lille dans la journée. Longues terres labourées à perte de vue. On sent la main de l’homme partout, ses grosses machines. Et il y a si peu d’arbres que les champs sont gorgés d’eau – plus rien pour pomper – il y a des flaques partout. Le TGV ralentit.\par
Et elle, alors, elle apparaît !\par
Elle est là ! Devant la porte coulissante ! Ses deux bagages à roulettes ! Elle se retourne empêtrée et me voit. Elle me dit bonjour.\par
— AH ! BONJOUR MONSIEUR PAUL !\par
Elle est en femme – tailleur jaune – jupe échancrée – escarpins hauts – et me demande si elle peut s’asseoir en face de moi. \emph{Vous permettez} ?\par
— Mais bien sûr !\par
Un grand truc en fausses plumes fuchsia incandescent se soulève et flotte vaguement autour de son cou pendant qu’elle s’assied – cheveux raides noirs tombant nets – frange coupée devant façon Louise Brooks – et des yeux très noirs aussi, très très faits.\par
— Pouf, dit-elle en s’asseyant, j’en peux plus, ça fait au moins un quart d’heure que je remonte ce train !\par
Elle croise les jambes, laisse ses deux bagages dans le couloir, m’explique que, pendant un moment, elle avait cru s’être trompée de train. Elle avait cherché désespérément une place 64 dans le wagon 6 avant de se rendre compte qu’il n’y avait pas de wagon 6 dans ce train mais qu’il y avait deux trains et que son 6 était dans l’autre devant et qu’il était impossible à atteindre.\par
— Oui, celui-là s’arrête à Lille et l’autre devant file sur Bruxelles !\par
— Effectivement, dit-elle, c’est souvent qu’ils accrochent deux trains, mais ils pourraient prévenir !\par
J’ai regardé ses yeux et son visage blanchi à la poudre de geisha, et je me suis dit mon Dieu qu’elle est belle ! Ses cils et ses paupières noircis me faisaient penser à un Van Dongen. Elle avait les lèvres violettes et ses boucles d’oreilles étaient vertes.\par
— Vous allez à Bruxelles ?\par
— Oui, dit-elle en se penchant vers moi, \emph{JE SUIS AGENT SECRET} !\par
— Ah !\par
Elle avait un parfum de Chalimar et son visage s’approchait tellement de moi que je pouvais sentir l’effet sucré de sa poudre. Elle voulait continuer à parler à voix basse, chuchotant, articulant beaucoup, du coup je voyais ses lèvres prendre un mouvement de pelle mécanique.\par
Oui, elle allait à Bruxelles, elle allait au rapport, c’était le truc le plus emmerdant dans son métier, les rapports ; elle venait de terminer une grande enquête sur les mafias russes sur la Côte d’Azur, à Nice ; le paysage fuyait ; et il y eut tout à coup un bang au passage d’un train.\par
J’ai dit : Vous arrivez à faire l’aller-retour Paris-Nice-Bruxelles dans la journée ?\par
— Non, non, non, Monsieur Paul, quand même pas ! Ça fait trois semaines en fait que je médite à Paris. J’ai fini par appeler au secours. Ça vous arrive vous à devoir écrire ?\par
— Bof !\par
— Et bien pour moi c’est trop galère, même avec un clavier je suis coincée, alors vous imaginez Internet ! Mais là je crois que j’ai trouvé un truc !\par
— Ah ?\par
— Si ! A Bruxelles, ils vont me mettre dans une cabine, j’ai un code, je reste devant l’écran, j’appuie sur une touche et là, je déverse, je raconte. En fait, je raconte comme ça vient. Je suis mes notes quand même ! Mais avec des notes c’est facile, je suis plutôt une femme de parole, moi, la machine trie. C’est elle qui fait le boulot. C’est elle qui compare, regarde ce qui a été dit, fait la synthèse. À la sortie, elle me sort dix pages avec rien que l’essentiel et moi, à la sortie, moi j’ai mon beau rapport, neuf dix pages 21×29,7, couverture plastifiée, c’est pas mal, je vous raconterai. C’est toujours d’accord pour dimanche ?\par
— Oui, j’apporterai la galette des Rois, si vous voulez.\par
— D’accord, mais ne la prenez pas à la frangipane, s’il vous plaît, je déteste la frangipane. On dit 15 heures, 15 heures 30, ça ira ?\par
Et elle est partie.\par
Je l’ai revue plus tard de loin sur le quai, elle courait pour rejoindre son train. Elle traînait ses valises derrière elle et se donnait comme tout le monde un air pressé.\par
Les haut-parleurs de la gare hurlaient.\par
Pour les voyageurs de queue, c’était le terminus.

\section[{— 11 —}]{— 11 —}
\renewcommand{\leftmark}{— 11 —}

\noindent Le Mékong est un long fleuve tranquille qui prend sa source en Chine. Et lorsqu’il arrive à la frontière du Cambodge, il est si large (16 km) qu’il se partage en de multiples bras qu’on appelle là-bas SIPANDON – les Quatre Mille Îles.\par
Bon !\par
Après, il y a des chutes, des cascades.\par
Et il y a si peu d’eau au milieu du Mékong que même parmi les remous on peut voir des gens debout, de l’eau à mi-cuisse. Ils pêchent ! Ou ils relèvent leurs filets ! Leurs bateaux à côté d’eux.\par
Les bateaux avec lesquels on voyage sur le Mékong sont des bateaux à fond plat avec des moteurs à longue queue, c’est-à-dire une hélice fixée loin derrière qui sert en même temps de gouvernail. En général ils filent vite et donnent l’impression de pouvoir se manier comme une pirogue.\par
Bon !\par
Ce soir, décidément, je n’ai pas mon 20 heures, La Femme est à Bruxelles, je déprime.\par
J’en parlais l’autre jour à Monsieur Mostefari.\par
— Comment vous faites, vous, avec l’Algérie ?\par
— Ah, avec l’Algérie, c’est pas pareil, moi j’improvise ! C’est comme si je racontais un autre pays !\par
Et il a pris un air triste.

\section[{— 12 —}]{— 12 —}
\renewcommand{\leftmark}{— 12 —}

— HÉ, BONJOUR, MONSIEUR PAUL !\par
On s’était finalement vite réconciliés avec Monsieur Mostefari, oui, assez vite. Deux jours après nos deux jours de non-dit, je passais, c’est lui qui m’interpelle.\par
— HÉ, BONJOUR, MONSIEUR PAUL !\par
Il était sur le pas de sa porte, la chemise blanche dépoitraillée, et il m’a redit HÉ ! pour que je me retourne.\par
— Ça n’a pas l’air d’aller ?\par
— Non ! Si vous voulez le savoir, j’ai dit, j’ai un problème en ce moment avec des contes laotiens !\par
— Ah !\par
— Oui, le Laos, vous savez pas où c’est ? C’EST LOIN !\par
— LE MÉKONG ! a-t-il dit tout suite.\par
Et moi j’ai ri :\par
— Bravo, Monsieur Mostefari, vous pouvez aller à Gagner Des Millions !\par
— Mékong ! répéta-t-il en se tirant sur les yeux pour se donner un air chinois.\par
Du coup, je lui ai tiré la langue.\par
En réponse, il m’a dit alors aussi sec – VENEZ ! C’est pas avec ce qu’on s’est dit l’autre jour qu’on va se fâcher, j’ai mon dernier fils qui est là !\par
— Et alors ?\par
— Il revient d’Asie, du Mékong !\par
— Ah !\par
Et je l’ai suivi.\par
J’étais encore dans la mouvance où rien que le mot Asie suffisait à me faire partir. C’est dur de revenir. Faut se réadapter. Le dernier fils était un gros balaise aux cuisses d’acier, six mois sur les plates-formes pétrolières, les six mois restant en voyage, et il s’est relevé là dans son jean serré quand je suis rentré, un sweat-shirt Algérie, blanc rouge vert, avec le croissant et l’étoile, et il m’a dit si c’est vrai, il revenait d’Asie, du Mékong, il avait séjourné un mois au Laos et il m’a serré la main en me lançant un \emph{Sabaïdiii} avec un sourire large qui voulait bien dire qu’il revenait de là-bas.\par
— Ça fait pile deux heures que je suis là, si, j’arrive de Bangkok. Je suis venu dire bonjour à mon père.\par
Il était bronzé, les yeux encore remplis du voyage, et on s’est assis l’un en face de l’autre en disant presque ensemble : AH, LE MÉKONG !\par
Il avait eu la même idée que moi, ramener planqué dans sa chaussure un peu de cette herbe magique qui vous fait bien flasher devant les couchers de soleil. On s’est fumé un pétard en parlant du Mékong, on s’est échangé quelques souvenirs, c’était les mêmes ; à peu près le même itinéraire ; de toute façon au Laos, c’est pas large ; tout le monde suit le Mékong ; et nous étions encore en train de nous dire LE MÉKONG ! LE MÉKONG ! quand le père a zappé en baillant sur Planète et nous a laissé devant un documentaire sur la transhumance des élans dans le Grand Nord canadien.\par
— MÉKONG !

\section[{— 13 —}]{— 13 —}
\renewcommand{\leftmark}{— 13 —}

\noindent Les types sur les plates-formes pétrolières, c’est vrai, font tous à peu près ça. Il faut dire que les primes sont très élevées pour convaincre des gars d’aller passer six mois perdus paumés sur une plate-forme dans la Mer du Nord. C’est vrai qu’il faut se les fader ces six mois, c’est dur, mais les gars font en général à peu près tous ça : ils partent six mois. Loin ! De préférence dans les pays chauds. Et lui, il en connaissait pas mal qui, à quarante ans, avait déjà fait plusieurs fois le Tour du Monde, et lui, à vingt-cinq, il s’en donnait encore bien quinze avant d’avoir bouclé tous les endroits où il rêvait d’aller. Il avait fait trois semaines de fête à Bangkok. La tournée de boîtes de folie. Les filles à gogo. Un \emph{falang} (étranger) même avec une gueule d’Arabe est un mec qui a du fric, alors allons-y ! Si tu savais comme celle que j’ai quittée hier m’a fait l’amour, mon ami, tu en resterais sur le cul !\par
J’en suis resté sur le cul.\par
Le lendemain, le père est venu frapper à ma porte, il m’apportait du thé et voulait me proposer de venir voir la télé le soir avec lui ; en fait, lui, en ce moment, il était seul, sa femme était en Algérie – \emph{sa grand-mère} ! dit-il en prenant encore une fois un air triste. J’avais bien plu à son fils !

\section[{— 14 —}]{— 14 —}
\renewcommand{\leftmark}{— 14 —}

— Monsieur Mostefari, quand vous racontez vos histoires à vos petits-enfants, comment vous faites ?\par
— Je raconte, comment voulez-vous faire ?\par
Et voilà comment ça a commencé.\par
Petit à petit.\par
Depuis, on a fait un bout de chemin.\par
On parle beaucoup.\par
— Comme tout le monde, dit-il, je raconte, je laisse venir, je ferme les yeux et si je ne trouve pas le mot en français, je le dis en arabe, et si je ne le trouve pas en arabe je le dis en kabyle et voilà !\par
— Et bien moi, je n’y arrive pas.\par
— Vous n’avez qu’à vous enregistrer !\par
— Ça ne fonctionne pas !\par
— Comment ça, ça ne fonctionne pas ?\par
— Non, ça ne fonctionne pas !\par
Il avait fini par s’introduire entre nous une certaine complicité, mais je trouvais tout con de lui dire qu’on ne parle pas comme on écrit et que la langue parlée qu’on écrit est toujours une fausse langue parlée.\par
— Vous vous prenez trop la tête, Monsieur Paul !\par
On passait ainsi des soirées côte à côte, enfoncés dans nos fauteuils devant la télé, à nous enquiller des matches de foot ou Star Academy tout en continuant à parler lui de l’Algérie, moi de mon Asie. On s’envoyait comme ça des cartes postales. Et quand lui me parlait de ses cailloux, moi je voyais ses cailloux et j’avais envie de revoir des cailloux… du sable… LE SAHARA !\par
— Monsieur Mostefari, vous comprenez, si j’arrive à raconter le Mékong, peut-être que ça me permettra d’y retourner !\par
— Ça se vend des contes laotiens ?\par
— J’en sais rien.\par
— Vous voulez qu’on s’entraîne ?

\section[{— 15 —}]{— 15 —}
\renewcommand{\leftmark}{— 15 —}

\noindent Je ne sais pas pourquoi je vous raconte ça, ni quel intérêt peut y trouver Monsieur Mostefari, mais lui, il écoute, il me regarde, il hoche la tête, il approuve. De temps en temps il prend sa guitare, gratte l’hymne kabyle, me demande si je veux jouer avec lui, je n’ai qu’à prendre la darbouka, mais si je prends sa guitare, je ne m’en sors pas avec mes trois accords de blues. Elle n’est pas accordée pareille. L’Asie, évidemment, ce n’est pas l’Algérie. Il y a tant de façons de jouer !\par
— Monsieur Paul ! MONSIEUR PAUL !\par
Oh, pardon, excusez-moi, c’est lui qui m’appelle. Dans la cour ! Je ne suis pas le seul qu’il interpelle comme ça.\par
— MADEMOISELLE HEUNEBELLE ! MADEMOISELLE HEUNEBELLE ! VOUS AVEZ UN COLIS !\par
Pas la peine de descendre, il a signé pour vous, vous passerez dans sa loge, il en sera bienheureux. Je savais maintenant que c’était toujours pour nous garder un peu. Quand on descendrait, il était sûr qu’on lui ferait un brin de causette. Il vous inviterait à prendre le thé. Il demanderait de vos nouvelles. Il vous ferait parler de vous ou de vos amis. Il s’emmerdait. Mais quand il prenait sa grosse voix du sud pour nous interpeller, la cour prenait tout à coup un grand coup de soleil qui semblait vouloir rejoindre le ciel, ça faisait du bien.\par
J’ouvre la fenêtre.\par
— Oui, Monsieur Mostefari, qu’est-ce qu’il y a ?\par
— Dimanche, j’ai mes petits-enfants, Monsieur Paul !\par
— Ah !\par
— Si ! Venez leur raconter ! On fera un test ! Si c’est bon c’est que c’est bon ! Il y aura la galette des Rois !\par
— Mais vous savez très bien que le problème n’est pas là, Monsieur Mostefari !\par
On se parle ainsi de la fenêtre, comme à Naples, et nos voix montent dans cette cour qui semble à chaque fois faire effet d’amplificateur.\par
— Je ne suis pas libre, dimanche, je dis, j’ai déjà rendez-vous !\par
— Ah !\par
— Si !\par
Et j’ai montré la fenêtre en face.\par
— Ah !\par
Il a marqué un temps.\par
— Mais passez quand même, mon fils sera ravi !\par
Et il est retourné dans sa loge en me redisant à dimanche et j’ai suivi le son de sa voix en relevant la tête pour voir le bout de ciel qu’on peut voir d’ordinaire en se penchant un peu. Il était gris.\par
En face, le rideau de la Femme était tiré. Exactement comme elle l’avait laissé avant de partir à Bruxelles.\par
J’ai fermé la fenêtre.\par
\bigbreak
{\centering — 16 -\par}
\bigbreak
\noindent Il est 23 heures, les rives du Mékong se sont tues. La nuit est noire. Je n’ai toujours pas écrit une ligne. De toute façon, qu’est-ce qui presse ? Ce soir, comme avant-hier soir, comme depuis plusieurs soirs, je n’ai pas eu mon 20 heures, et je commence à m’en désaccoutumer. Pas de Femme. La Femme ne rentre plus, n’est pas rentrée, ne rentrera peut-être plus. La lumière de chez moi éclaire sa fenêtre. Parfois son chat pointe son nez, fait bouger le rideau, monte sur le rebord de la fenêtre et regarde la cour. Puis, d’un air indéfinissable de scrutateur, me fixe longuement de ses yeux jaunes tout drôles où j’ai l’impression de voir un serpent.\par
Quelle étonnante stabilité ! Quand un chat vous regarde et que vous regardez le chat, vous vous demandez sans cesse quand est-ce qu’il va baisser les yeux celui-là ? Mais, non, rien à faire, pas un sourcil ne bronche. Rien.\par
Bonsoir, le chat ! j’ai dit.\par
C’est vrai que ce soir ça ne va pas, je suis triste, mélancolique, cet après-midi, j’ai fait chou blanc ! Je m’étais pourtant bien préparé, j’avais laissé passer le quart d’heure après l’heure qui montre qu’on est élégant et qu’on ne veut rien précipiter, et je suis allé frapper à sa porte avec ma petite boîte de galette à la main. Personne n’a répondu.\par
J’ai refrappé deux fois et deux fois je n’ai pas eu de réponse.\par
J’ai fait le tour par le couloir pour voir la fenêtre.\par
Le rideau était tiré. Elle n’était pas là.\par
Mais comment peut-on fixer un rendez-vous un dimanche en sachant que ce dimanche-là on peut avoir un empêchement ? Elle devait le savoir quand même ou alors c’était qu’il s’était passé quelque chose ?\par
Je suis redescendu voir Monsieur Mostefari.\par
Mon gâteau à la main.\par
Il m’attendait.\par
Dans la loge il y avait toute la famille, cinq ou six gamins, garçons, filles, piaillant, applaudissant le Roi. Il y avait déjà un Roi. Et une Reine. Et le dernier fils était là. Il m’a salué de loin, souriant, levant la main. Il ne pouvait pas bouger, coincé entre la table et le fond de la loge où tous étaient serrés.\par
— Ah, salut le voisin !\par
J’ai salué vite fait. Devant moi, un petit brun aux yeux verts chouinait qu’on lui avait piqué sa fève et Monsieur Mostefari se penchait en lui tirant l’oreille gentiment.\par
Je l’ai pris à l’écart, tendant le cou en chuchotant.\par
— Vous savez pas où elle est passée la Femme qui habite en face de chez moi ?\par
— Ah, la Baronne, dit-il, c’est moi qui m’occupe de son chat !\par
— Ben, oui, son chat, pardon, ça fait quatre jours !… Je l’ai rencontrée l’autre jour dans le train. Elle m’a dit qu’elle serait là dimanche ?\par
Il m’a regardé en se marrant :\par
— Vous voulez dire que vous craignez que je m’occupe mal de son chat ! Ah, Monsieur Paul, je vois que ça vous inquiète, allez, venez avec moi, on va vérifier… D’ailleurs il fallait que je monte ! Vous venez ?\par
On est monté ensemble. L’appartement était sombre. J’ai vu le chat. Il est venu tout de suite se frotter contre la jambe de Monsieur Mostefari et l’a mené jusqu’à la cuisine, la queue en l’air, et lorsque Monsieur Mostefari a ouvert la boîte, il s’est étiré de tout son long sur le frigidaire puis s’est précipité sur sa pâtée et s’est mis à manger comme si nous n’étions plus là.\par
— Vous venez ?\par
\bigbreak
\noindent Il est 23 heures 10 ! Des talons passent dans la cour. La grosse voix de Monsieur Mostefari sous le porche d’entrée :\par
— MADEMOISELLE HEUNEBELLE ! MADEMOISELLE HEUNE-BELLE ! ATTENDEZ UNE MINUTE !\par
— OUI ?\par
— J’AI FAIT RÉPARER VOS CLÉS !\par
— AH, MERCI MONSIEUR MOSTEFARI !\par
Non, ce n’est pas Elle !

\section[{— 17 —}]{— 17 —}
\renewcommand{\leftmark}{— 17 —}

\noindent Il montait devant moi et moi je suivais. Nous discutions d’elle. Pourquoi ce subit intérêt ? Il se demandait pourquoi finalement elle louait cet appartement, elle n’était pratiquement jamais là, sauf ces dernières trois semaines, oui, elle était là.\par
— Oui, je sais, j’ai dit, elle rentre tous les soir à 20 heures, et à 23 heures elle sort.\par
— Oui, c’est un oiseau de nuit ! Enfin, je ne sais jamais comment l’appeler. De temps en temps, elle me dit La Baronne. Un autre jour, c’est la \emph{Sanseverina}. Ou bien c’est carrément La Baronne \emph{Sanseverina}. De toute façon, elle a de la classe, cette fille. Ça se voit.\par
Et nous montions.\par
Et je pensais à elle et n’imaginais rien.\par
LA BARONNE !\par
J’entends sa voix me dire bonjour, ça résonne, elle m’appelle, elle est loin, je suis Monsieur Mostefari, il a ouvert la porte. L’appartement est sombre.\par
Le chat vient l’accueillir.\par
Se frotte contre sa jambe.\par
— De temps en temps elle me dit de l’appeler\emph{ Sansé} pour faire plus simple, madame \emph{Sansé}, mais \emph{Sansé} comment ? Souvent, vous savez, ici, les gens prennent des faux noms. Moi, je m’arrange. L’autre jour, j’ai trouvé ses clefs sur mon comptoir, elle avait mis un mot, elle n’avait pas eu le temps de me prévenir, elle devait partir vite, elle me rembourserait pour les boîtes du chat. Enfin, c’est une fille sympathique !\par
Il referma la porte.\par
{\itshape « Vous voyez, comme c’est bien rangé chez elle ! »\par}
C’était un appartement de fille, peluches et décos cucul, très propre, le parquet brillait. Deux grands rayons de soleil le traversaient de biais en passant sous le rideau. Le lit était fait. J’ai regardé Monsieur Mostefari nourrir le chat, remettre de l’eau dans une soucoupe et, pendant qu’il replaçait la boîte dans le frigo, j’ai tourné et tourné le regard sur cet espace clos. Ici, le lit. Là, la fenêtre. D’où je la vois. Voilà. La télé. Moi là-bas… ! Et Moi et Elle ? Comment ça se fait ?\par
Monsieur Mostefari m’a dit venez et on est reparti. Et lui dit que demain il était bon pour changer la caisse du chat.\par
Ça avait l’air de le faire chier.\par
— Vous vous introduisez souvent dans les appartements des gens ?\par
— Seulement quand ils me le demandent, dit-il. C’est clair. C’est net. C’est comme ça. Mais je vais vous dire, Monsieur Paul. En fait tout le monde me le demande !\par
— Et vous, vous êtes déjà venu me voir dormir ?\par
— Non, pas encore, Monsieur Paul, dit-il en riant, mais ça se pourrait, hé hé, vous me donnez une idée, vous savez bien que j’ai toutes les clefs !\par
— Mais bien sûr, Monsieur Mostefari, je disais ça en blaguant, je me disais que l’histoire d’un type qui s’introduit dans tous les appartements pour venir voir simplement les gens dormir, il saurait tout sur eux, je ne sais pas, leur rêves, ça pourrait faire un très joli conte.\par
— Encore ! Et vos contes laotiens ?\par
— Ben, heu !\par
Quand nous sommes revenus à sa loge, la marmaille avait englouti la galette que j’avais laissée sur la table. Il y avait deux nouveaux rois. Un Roi et une Reine. Et le même petit brun aux yeux verts qui pleurait qu’on lui avait piqué sa fève.\par
Monsieur Mostefari lui a tiré l’oreille.\par
Gentiment.

\section[{— 18 —}]{— 18 —}
\renewcommand{\leftmark}{— 18 —}

\noindent J’ai fini par hériter du chat ! Oh ! ça n’a pas été très compliqué. C’est moi qui l’ai demandé. \emph{« Monsieur Mostefari, vous ne voulez pas que je m’occupe du chat ? »}\par
Et j’ai compris que ça l’arrangeait.\par
— Vous me dites seulement quand il faudra acheter des boîtes !\par
Je lui ai dit que je pouvais très bien m’en charger, mais il m’a répondu qu’il tenait que ce soit lui, c’était un arrangement avec La Baronne.\par
J’en avais marre de voir ce rideau tiré.\par
Ça me foutait le cafard.\par
C’était pire que quand elle était là. Je fantasmais sur elle. J’étais cinglé. Je ne sortais plus. Je voyais fondre mes économies. Et je me sentais avoir envie d’une femme. Oui, une femme ! Et c’est devenu par hasard La Femme que j’avais en face. C’est tout con. Un jour, on se dit, j’aimerais bien la voir de nouveau tirer le rideau, se pencher à la fenêtre pour fumer sa cigarette dehors. Mais qu’est-ce qui t’arrive, Paulo, de penser à elle comme ça !— \emph{BONJOUR MONSIEUR PAUL ! –} J’entends sa voix – le chat passe la tête derrière le rideau quand le soleil donne, j’ai pitié de lui, comment gère-t-il lui aussi cette absence de maîtresse, ce silence ?\par
En fait, j’étais bien content de pouvoir maintenant m’introduire chez elle, j’y allais trois fois par jour. Et, pendant que le chat mangeait, je m’asseyais sur le lit et me refaisais le film du tour de son appart, ici le lit, là la fenêtre, d’où elle me voit, et moi, là-bas, qui la voit, bzouing, la vie est une géométrie bizarre, je plains les gens qui disent que c’est simple.\par
Je restais de longs moments dans cet appartement sombre à essayer de m’imprégner d’elle, deviner comment elle était. L’accumulation des fanfreluches témoignait d’un caractère de fille qui aime se faire un nid. C’est son espace clos. Ailleurs, elle peut sourire, séduire. C’est une autre fille quand elle est dedans. Comment est-elle en vrai, en dehors de son côté loufoque JE SUIS UN AGENT SECRET ?\par
J’entendais sa voix.\par
Les choses étaient bien parties pour ne pas bouger. Le parquet brillait. Un jour, j’ai ouvert le rideau, j’ai pris le soleil en pleine gueule et j’ai dit au chat :\par
« Allez, viens, toi, tu vas venir chez moi ! »\par
Et il m’a suivi.

\section[{— 19 —}]{— 19 —}
\renewcommand{\leftmark}{— 19 —}

\noindent Voilà comment les choses se sont faites ou défaites, on ne sait comment dire et l’on ne sait qui danse ou qui décide. Cela faisait maintenant près d’un mois que nous étions dans cette configuration, elle, absente, et le chat qui avait fini par venir chez moi, le rideau en face continuait à rester ouvert, ça ne bougeait pas. Je ne bougeais pas. Je bougeais de moins en moins. Je voyais fondre mes économies. Dans un mois je n’aurais plus de quoi payer le loyer. Je m’en foutais. Je regardais le chat. Lui ne me regardait pas. Il avait pris ses aises sur le radiateur, roupillant toute la journée. Et les rares fois où il relevait le nez c’était soit quand il m’entendait ouvrir une boîte soit quand le soleil donnait en face et qu’il regardait son ancien chez lui.\par
Je manquais régulièrement mon 20 heures et ça ne me manquait pas. De temps en temps, j’essayais de lire au chat mes essais de contes, mais ma voix ne signifiant rien pour lui, il ne relevait même pas la tête quand je l’appelais le chat, tout simplement. Il avait l’air de s’en foutre. Il était là au chaud. Je méditais en le tenant sur moi, sur notre absence de fourrure qui nous rend nous les humains certainement encore plus frileux que des chats. Les lions baillent dans la savane pendant que leurs femelles chassent. Pourquoi on s’emmerde autant ici-bas ? Je laissais couler le Mékong. C’est détendu, l’Asie.\par
J’en parlais l’autre jour à Monsieur Mostefari.\par
Il m’a répondu :\par
« Chez moi aussi, autrefois, c’était détendu ! »

\section[{— 20 —}]{— 20 —}
\renewcommand{\leftmark}{— 20 —}

\noindent Mourir ou ne pas mourir, de toute façon on ne sait pas comment, et j’en étais là de mes pensées quand j’ai entendu qu’on frappait à ma porte. Ça me paraissait vraiment tôt. Il faisait noir. Il était quelle heure ? Il faisait froid. J’étais encore sous la couette, le chat était venu pendant la nuit se pelotonner contre moi. Je n’avais pas envie de me lever. Pas vraiment.\par
— MONSIEUR PAUL, MONSIEUR PAUL, JE VIENS CHERCHER MON CHAT !\par
Une voix de femme.\par
Et c’était Elle naturellement.\par
Elle ! Je ne rêvais pas.\par
Je suis sorti de mon lit assez vite, nu comme j’étais, attrapant une serviette au passage pour me couvrir le ventre avant d’ouvrir la porte. Elle est entrée en trombe, me bousculant presque, et moi je l’ai immédiatement vue de dos.\par
Elle était en jupe beige – pull gris clair – collant blanc – de grosses godasses de chantier pas lassées aux pieds. Elle cherchait son chat et moi je restais comme un con près de la porte avec ma serviette sur le zizi.\par
— Oh, Monsieur Paul, merci, vous n’auriez pas dû ! dit-elle en cherchant son chat. VIRGIN ! VIRGIN !\par
— Ce n’est pas grave, j’ai dit, j’étais bien con…\par
— Mais non…\par
— …tent…\par
— Où il est celui-là ? VIRGIN ! VIRGIN !\par
Elle a cherché sous le lit, inspecté le studio à quatre pattes, à hauteur de chat, appelé encore et tourné. Et le chat a fini par sortir la tête de la couette, l’air encore complètement endormi, complètement dans le potage.\par
Il l’a regardée vaguement, presque indifférent, et elle alors l’attrape là et vient s’asseoir sur le lit.\par
— Mais tu as grossi, Virgin !\par
De longues mèches rouges artificielles volent dans ses cheveux quand elle baisse la tête pour caresser son chat.\par
— Ah, Monsieur Paul, c’est vraiment trop gentil à vous, c’était vraiment sympa d’avoir voulu vous occuper de Virgin, ça ne lui faisait pas loin. Mais je crois quand même que vous lui avez donné trop à manger !\par
— Non !\par
— Si, il a grossi, regardez, il a pris du ventre !\par
Je me suis assis en face d’elle sur la deuxième chaise du bureau, gardant la distance entre elle et moi, avec ma serviette sur les cuisses. Elle a fait semblant de ne pas le voir. Elle caressait son chat. Elle ne s’occupait que de lui, lui faisant des mamours, et le chat, l’air de rire, lui tendait le museau pour lui renifler le nez.\par
— Vous savez d’où je viens ? dit-elle en me regardant soudain.\par
— Non ! j’ai dit.\par
— D’Afghanistan !\par
— Oh !\par
— Si ! Guido m’a envoyée là-bas !\par
— Qui c’est ça Guido ?\par
— C’est le Chef des Missions Secrètes, murmure-t-elle soudain, parlant à voix basse. L’Organisation !\par
— Café ? j’ai dit.\par
— Vous ne voulez pas plutôt prendre le café chez moi, j’ai ramené des croissants !\par

\begin{itemize}[itemsep=0pt,topsep=0pt,partopsep=0pt,parskip=0pt]
\item Heu, non non, non merci !
\end{itemize}

Elle continue dans un soupir.\par
— Je peux vous dire que c’est pas encore ça, là-bas, enfin ! Guido voulait savoir ce que ça donnait l’après-talibans, s’il y avait le moyen de s’implanter là-bas, l’Afghanistan, après tout, pourquoi pas ? Il y a tout à refaire. Je ne sais pas pourquoi je vous raconte ça. Vous ne savez pas ce que nous cherchons… Non, vraiment, ce café ?\par
— Non !\par
Je me suis relevé, traînant les pieds, serrant les genoux, ma serviette, je cherchais mon pantalon et tout à coup elle me dit :\par
— Mais pourquoi vous dites non comme ça, Monsieur Paul ? Il y a une raison forte dans mon invite. C’est qu’il y a longtemps que je n’ai pas vu un homme le cul nul avec un aussi beau cul, mm, mais en plus je trouve que c’est le bordel chez vous, j’ai pas envie de prendre un café ici.

\section[{— 21 —}]{— 21 —}
\renewcommand{\leftmark}{— 21 —}

C’était pas cool l’Afghanistan.\par
Elle plaisantait !\par
— J’aurais besoin que vous me rendiez un petit service ?\par
— Ah !\par
— Oui, que vous me disiez ce qui s’est passé pendant que j’étais là-bas, ce qu’on a dit, vous voyez !\par
— Pourquoi, ça ne marche plus Bruxelles ?\par
— Non, Guido veut qu’on revienne à l’écrit, il dit que la machine est trop lisse, elle trie trop. Il n’y a plus la magie de la phrase qu’on fabrique à la main…\par
— Des mots à double sens ?\par
— Oui, enfin ça, je ne sais pas !\par
Et elle est restée comme ça suspendue, ses lèvres en O, et moi, devant elle, debout avec ma serviette en V. Et pendant le court instant qu’a duré cet instant, j’ai vu de l’autre côté sa fenêtre ouverte et la voyant en même temps se lever, laisser tomber son chat, faire le tour de la table, venir vers moi et, d’un geste gracieux, me prendre par la main. Devant moi elle fait un pas de danse, une révérence, et m’emmène telle une sirène, moi avec ma serviette et son chat qui suivait.

\section[{— 22 —}]{— 22 —}
\renewcommand{\leftmark}{— 22 —}

« Ah, mon cher ange, vous partez déjà ?\par
— Hélas oui, Baronne, je vous cède la place ! »\par
Les deux femmes se croisent sur le haut des marches. Se la jouent. On se croirait sur le grand escalier à Cannes. Tapis rouge. L’une monte. L’autre descend. Les lettres du « S-MAGIK » suivent en bleu un cercle doré qui ressemble à une bouche. Il faut passer par la glotte. S’enfoncer. Après, c’est toboggan. Celle qui venait de se faire appeler Baronne, c’est bien sûr ma Femme. La même. Elle se promène en costume d’apparat. Robe-fourreau et gants noirs. Style Baronne, m’a-t-elle dit avant de sortir. C’est son costume de vamp, dit-elle. Nous avons fait l’amour tout l’après-midi. Ce soir, elle me sort. Elle veut me remercier de m’être occupé de son chat, et d’avoir aussi fait sourire sa chatte. Nous gardons la main sur la rampe en cuivre. De notre côté, ça descend. Puis dans un virage pour bobsleigh où il n’y a pas d’autre moyen que de se mettre sur le cul, nous nous mettons sur le cul et hop, on arrive directement sur la piste de danse. Elle devant. Moi derrière. Des danseurs s’écartent. On est un peu enchevêtrés. Il y a beaucoup de monde au bar. Et, quand elle se relève, la plupart des gens la saluent, viennent l’embrasser. Baronne ! Baronne ! Baronne par-ci, Baronne par là ! Elle est la reine ici ! C’est une Reine de la Nuit.\par
Elle me prend par la main :\par
— C’est un bar de gays gais ici, dit-elle, les homos sont toujours sympas, légers. Allez, viens, je vais te faire rencontrer Guido !\par
J’aime assez quand les femmes conduisent, je dois dire que quand je regarde ma vie, je me suis toujours fait mener. C’est vrai. Mais, que voulez-vous, moi, les femmes m’inspirent ! Je leur dois beaucoup. Ce voyage par exemple en Asie, je l’ai fait avec une femme mais finalement on n’est pas revenus ensemble, c’est tout.\par
— Je vous présente Monsieur Paul !\par
A une table de vieilles folles trône un gros bonhomme déguisé en romain, la toge plissée d’un côté sur l’épaule, le torse nu bronzé. À ses côtés, un petit jeune blond décoloré et un type plutôt morgue, la quarantaine bien sonnée, chauve, droit comme un I dans un col roulé noir.\par
— AH, BARONNE ! dit-il. VENEZ AVEC NOUS, NOUS CÉLÉBRONS LE BICENTENAIRE DE MADAME ALBERTA.\par
Une grosse femme en rouge se soulève, fait trois courtes révérences, sourit à tout le monde et se rassied.\par
— Alors donc c’est vous, Monsieur Paul ! dit-il. Et bien bonjour Monsieur Paul !\par
Et il me tend la main pour qu’on se dise bonjour.\par
— Bienvenu Monsieur Paul, la Baronne m’a parlé de vous. Vous écrivez des contes, paraît-il ?\par
— Ben heu !\par
La Baronne me touche le bras.\par
— J’ai téléphoné à Guido pendant que tu étais sous ta douche ! dit-elle et elle ajoute « MON CHÉRI ! » avant de finir dans un drôle de hi hi. Je voulais te faire une surprise ! »\par
Et, dans le quart de secondes où elle papillonne des yeux, repassent devant mes yeux les images à fond de son corps contre le mien, une peau, des caresses, des lèvres et l’ivresse inouïe de tomber dans les bras d’une Baronne et je ne sais plus quoi. LOVE LOVE LOVÉE.\par
J’en ai encore mal à la queue.\par
Elle avait un beau cul.\par
C’était une Baronne.\par
Elle était bien foutue.\par
Elle m’avait attrapé directement derrière la porte, « Allez, lâche ta serviette ! », et me bouffant la bouche elle m’avait fait reculer jusqu’au lit. En trois mouvements, elle est nue…\par
Elle est…\par
Dessus.\par
Plus belle que les encarts de pub d’Aubade dans la rue.\par
Son ventre se contracte lorsqu’elle passe les bras en arrière pour défaire son soutien-gorge vert-pomme.\par
Le string a des roses rouges.\par
Elle a des hanches satin.\par
Ses mains de femme sont folles.\par
Et les hommes ?\par
MONSIEUR PAUL !… MONSIEUR PAUL !\par
OUAIS !\par
MONSIEUR PAUL !

\section[{— 23 —}]{— 23 —}
\renewcommand{\leftmark}{— 23 —}

\noindent MONSIEUR PAUL ! MONSIEUR PAUL ! J’entendais des Monsieur Paul partout. La musique était forte. Le DJ jouait avec sa voix, la modifiant dans le synthé pour tortiller son Paul. – BIENVENU AU S-MAGIC MONSIEUR PAUL !\par
Paul… Paul… Les danseurs s’écartent. Elle me dit ils font toujours ça avec les nouveaux. Des restes de Pau-Paul tournent en écho sur le boum-boum techno ! Tout ça donnait bien sûr envie de danser. Furieux.\par
— Elle est pas belle la vie ? dit-elle dans un sourire éblouissant.\par
Je l’aimais.\par
J’avais l’impression de poursuivre un rêve.\par
En fait, tout ça ne pouvait durer.\par
Elle m’a donc dirigé louvoyant vers cette table de garçons et elle a dit à Guido :\par
« Ah, salut Guido, je t’amène Monsieur Paul ! »\par
Le pli de la toge glisse sur l’épaule du Guido qui bombe le torse pour faire admirer ses pectoraux.\par
Autour, les garçons ont fait : MONSIEUR PAUL ! MONSIEUR PAUL ! en prenant tous des grosses voix de baryton.\par
— Alors comme ça, m’a dit Guido, il paraît que vous êtes poète ?\par
— Ben euh !\par
— C’est de la poésie-poésie ou bien c’est des mirages ?\par
La Baronne s’est penchée à son oreille et, derrière sa main gantée, lui a soufflé quelque chose.\par
— Oh, pardon, Monsieur Paul, excusez-moi, la Baronne me dit que ce sont des contes laotiens que vous écrivez, je n’avais pas compris au téléphone ! Les filles, est-ce qu’on a déjà pensé au Laos ?\par
— NON, NON, GUIDO ! LE LAOS ! LE LAOS ! ont repris les garçons.\par
— Le Laos ! Le Laos ! Et bien ça tombe bien ! Monsieur Paul, nous cherchons des poètes. Vous allez peut-être avoir la chance de faire partie de notre bande. C’est pas mal. Ça vous dirait ? Allez, allez, venez vous asseoir près de moi, oui, oui, oui, plus près, là, VENEZ ! Pousse-toi un peu, Julien, laisse de la place à Monsieur Paul, je voudrais l’entendre !

\section[{— 24 —}]{— 24 —}
\renewcommand{\leftmark}{— 24 —}

\noindent Écoutez-moi bien, Monsieur Paul, il faut que je vous explique quand même l’évolution de notre société, c’est un tournant important que nous sommes en train de prendre, mais nous conservons nos acquis. Dans les années 80, nous avons commencé avec Julien, mon ami, on s’est connu au régiment… dans la planche à voile ! C’était l’époque, le goût des sports extrêmes, le culte du corps, et nous avons créé avec Julien plusieurs lignes de vêtements qui ont fait le tour du monde. Vous avez certainement entendu parler des vêtements GUIDO ? HÉ, GUIDO DIS DONC c’était d’ailleurs un T-shirt qui a pas mal marché. Julien va continuer de s’occuper de cette partie, mais moi j’ai envie de changer, aujourd’hui je veux vendre du vent ! Du silence et du vent ! Du soleil et du rien ! Et pourquoi le vent ? Parce qu’aujourd’hui les gens sont de plus en plus stressés, ils ont besoin de marquer le coup, de marquer un temps d’arrêt de temps en temps, besoin de silence ! C’est pourquoi l’OGC, l’Organisation Guido and Co, c’est l’anti-Club Med en tout cas, oui, c’est ça, je vous le promets. J’envoie en ce moment mes agents un peu partout, comme la Baronne, avec une mission très précise ! Repérer quelques endroits magiques qui restent encore sur cette planète et les acheter – pour les préserver – avec quelques aménagements simples, paillotes et douche collective, on les proposera à l’homme stressé qui veut se ressourcer. Plages du bout du monde ! Inconnues de tout le monde. Himalayas cachés, etc. Vous voyez le genre ? Mais je ne veux pas faire comme Richard Branson, le patron de Virgin, des îles pour milliardaires, non !… La nature stricte comme dans Robinson. Bien sûr ça se payera cher. Le silence c’est ce qui coûte le plus cher aujourd’hui. Aujourd’hui, il faut pouvoir se payer le prix du silence. Mais ça restera comme c’était. Quelques aménagements, c’est tout. Cabane de bambou, bungalow, le hamac, le strict minimum… et dès le lever du soleil, avoir là la plage blanche devant toi, la mer bleue. Le soleil se lève à droite et il se couche à gauche, on est dans un autre hémisphère, tu n’as pratiquement pas à tourner la tête, évidemment ça change pas mal de choses. Se poser ! Vous me direz, c’est du bouddhisme à l’eau de rose. Je suis bien d’accord. Mais je pense qu’une fois que l’homme stressé s’est posé, il peut aussi se passer quelque chose. J’en suis convaincu. Soit il revient, soit il ne revient pas. S’il revient, il va commencer à regarder son quotidien stressant d’une autre façon, et il va se dire, bon, qu’est-ce que je fais maintenant, je continue comme avant ? C’est une petite philosophie personnelle, elle se tient, n’est-ce pas ? Je suis sûr que vous êtes d’accord avec moi, Monsieur Paul ? Ça vous tente ? Je cherche pour vendre ça des gens qui sont capables de m’évoquer ça en dix ou vingt lignes, comme savent le faire les publicitaires d’aujourd’hui, mais moi je veux plus… Je veux des gens qui soient capables de me faire des fugues ! Ouais, de la danse et de la musique. M’évoquer la danse et la musique. Avec des mots… Et moi, avec ces mots, que j’ai envie de danser, que je sois joyeux, que j’ai envie de partir, et j’entends les images comme si j’étais déjà là-bas ! Vous voyez, Monsieur Paul. Si votre conte est bon, on me dira où c’est ? Et moi, je dirai : c’est là ! Suivez le guide ! Vous, la Baronne, bien d’autres ! Toutes mes folles ont été jusqu’à présent mes meilleures ambassadrices ! Je peux compter sur vous, Monsieur Paul ? Vous savez, Monsieur Paul. Avec moi, vous… vous aussi vous pouvez gagner des Millions !\par
\bigbreak
\noindent Et c’est comme ça que j’ai recommencé à voyager ! Guido me dit que 15 lignes ça fait une minute en radio, et en une minute on peut faire passer beaucoup de choses. Je dis toujours qu’on peut faire plus court, surtout si on veut faire passer un murmure. J’ai laissé tomber le Mékong, oublié La Femme, la cour, cherché d’autres bruits. Je décris une île. Et je raconte, comme me le demande Guido, je dis que le pêcheur chez qui je loge appelle son fils Lakitine, ouvre-bouteille en\emph{ tchalé}, le parler des Gitans des Mers qui vivent dans des îles entre la Thaïlande et la Malaisie, son fils étant le seul à savoir décapsuler une bière avec les dents. Dans la lueur de la lampe à pétrole, Wind raconte comment, gamin, il a eu peur la première fois qu’il a vu un Blanc, il est allé se cacher dans la forêt. Il tire à l’arc. Pêche. Peut sentir où se cache un singe rien qu’en sentant le vent. D’où son nom de Loum, en \emph{tchalé}, le vent, adapté en Wind en anglais, quand il parle aux touristes.\par
Autrefois il fallait deux jours pour rejoindre le continent.\par
Tout à la pagaie !\par
\bigbreak
{\centering\itshape SAM-PA-LO ! KAYO – KAYO !\par}
{\centering\itshape \noindent SAM-PA-LO ! KAYO – KAYO !\par}
{\centering\itshape \noindent SAM-PA-LO ! KAYO – KAYO !\par}
 


% at least one empty page at end (for booklet couv)
\ifbooklet
  \pagestyle{empty}
  \clearpage
  % 2 empty pages maybe needed for 4e cover
  \ifnum\modulo{\value{page}}{4}=0 \hbox{}\newpage\hbox{}\newpage\fi
  \ifnum\modulo{\value{page}}{4}=1 \hbox{}\newpage\hbox{}\newpage\fi


  \hbox{}\newpage
  \ifodd\value{page}\hbox{}\newpage\fi
  {\centering\color{rubric}\bfseries\noindent\large
    Hurlus ? Qu’est-ce.\par
    \bigskip
  }
  \noindent Des bouquinistes électroniques, pour du texte libre à participations libres,
  téléchargeable gratuitement sur \href{https://hurlus.fr}{\dotuline{hurlus.fr}}.\par
  \bigskip
  \noindent Cette brochure a été produite par des éditeurs bénévoles.
  Elle n’est pas faite pour être possédée, mais pour être lue, et puis donnée, ou déposée dans une boîte à livres.
  En page de garde, on peut ajouter une date, un lieu, un nom ;
  comme une fiche de bibliothèque en papier qui enregistre \emph{les voyages de la brochure}.
  \par

  Ce texte a été choisi parce qu’une personne l’a aimé,
  ou haï, elle a pensé qu’il partipait à la formation de notre présent ;
  sans le souci de plaire, vendre, ou militer pour une cause.
  \par

  L’édition électronique est soigneuse, tant sur la technique
  que sur l’établissement du texte ; mais sans aucune prétention scolaire, au contraire.
  Le but est de s’adresser à tous, sans distinction de science ou de diplôme.
  \par

  Cet exemplaire en papier a été tiré sur une imprimante personnelle
   ou une photocopieuse. Tout le monde peut le faire.
  Il suffit de
  télécharger un fichier sur \href{https://hurlus.fr}{\dotuline{hurlus.fr}},
  d’imprimer, et agrafer (puis lire et donner).\par

  \bigskip

  \noindent PS : Les hurlus furent aussi des rebelles protestants qui cassaient les statues dans les églises catholiques. En 1566 démarra la révolte des gueux dans le pays de Lille. L’insurrection enflamma la région jusqu’à Anvers où les gueux de mer bloquèrent les bateaux espagnols.
  Ce fut une rare guerre de libération dont naquit un pays toujours libre : les Pays-Bas.
  En plat pays francophone, par contre, restèrent des bandes de huguenots, les hurlus, progressivement réprimés par la très catholique Espagne.
  Cette mémoire d’une défaite est éteinte, rallumons-la. Sortons les livres du culte universitaire, débusquons les idoles de l’époque, pour les démonter.
\fi

\end{document}
