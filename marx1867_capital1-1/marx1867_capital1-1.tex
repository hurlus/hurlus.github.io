%%%%%%%%%%%%%%%%%%%%%%%%%%%%%%%%%
% LaTeX model https://hurlus.fr %
%%%%%%%%%%%%%%%%%%%%%%%%%%%%%%%%%

% Needed before document class
\RequirePackage{pdftexcmds} % needed for tests expressions
\RequirePackage{fix-cm} % correct units

% Define mode
\def\mode{a4}

\newif\ifaiv % a4
\newif\ifav % a5
\newif\ifbooklet % booklet
\newif\ifcover % cover for booklet

\ifnum \strcmp{\mode}{cover}=0
  \covertrue
\else\ifnum \strcmp{\mode}{booklet}=0
  \booklettrue
\else\ifnum \strcmp{\mode}{a5}=0
  \avtrue
\else
  \aivtrue
\fi\fi\fi

\ifbooklet % do not enclose with {}
  \documentclass[french,twoside]{book} % ,notitlepage
  \usepackage[%
    papersize={105mm, 297mm},
    inner=12mm,
    outer=12mm,
    top=20mm,
    bottom=15mm,
    marginparsep=0pt,
  ]{geometry}
  \usepackage[fontsize=9.5pt]{scrextend} % for Roboto
\else\ifav
  \documentclass[french,twoside]{book} % ,notitlepage
  \usepackage[%
    a5paper,
    inner=25mm,
    outer=15mm,
    top=15mm,
    bottom=15mm,
    marginparsep=0pt,
  ]{geometry}
  \usepackage[fontsize=12pt]{scrextend}
\else% A4 2 cols
  \documentclass[twocolumn]{report}
  \usepackage[%
    a4paper,
    inner=15mm,
    outer=10mm,
    top=25mm,
    bottom=18mm,
    marginparsep=0pt,
  ]{geometry}
  \setlength{\columnsep}{20mm}
  \usepackage[fontsize=9.5pt]{scrextend}
\fi\fi

%%%%%%%%%%%%%%
% Alignments %
%%%%%%%%%%%%%%
% before teinte macros

\setlength{\arrayrulewidth}{0.2pt}
\setlength{\columnseprule}{\arrayrulewidth} % twocol
\setlength{\parskip}{0pt} % classical para with no margin
\setlength{\parindent}{1.5em}

%%%%%%%%%%
% Colors %
%%%%%%%%%%
% before Teinte macros

\usepackage[dvipsnames]{xcolor}
\definecolor{rubric}{HTML}{800000} % the tonic 0c71c3
\def\columnseprulecolor{\color{rubric}}
\colorlet{borderline}{rubric!30!} % definecolor need exact code
\definecolor{shadecolor}{gray}{0.95}
\definecolor{bghi}{gray}{0.5}

%%%%%%%%%%%%%%%%%
% Teinte macros %
%%%%%%%%%%%%%%%%%
%%%%%%%%%%%%%%%%%%%%%%%%%%%%%%%%%%%%%%%%%%%%%%%%%%%
% <TEI> generic (LaTeX names generated by Teinte) %
%%%%%%%%%%%%%%%%%%%%%%%%%%%%%%%%%%%%%%%%%%%%%%%%%%%
% This template is inserted in a specific design
% It is XeLaTeX and otf fonts

\makeatletter % <@@@


\usepackage{blindtext} % generate text for testing
\usepackage[strict]{changepage} % for modulo 4
\usepackage{contour} % rounding words
\usepackage[nodayofweek]{datetime}
% \usepackage{DejaVuSans} % seems buggy for sffont font for symbols
\usepackage{enumitem} % <list>
\usepackage{etoolbox} % patch commands
\usepackage{fancyvrb}
\usepackage{fancyhdr}
\usepackage{float}
\usepackage{fontspec} % XeLaTeX mandatory for fonts
\usepackage{footnote} % used to capture notes in minipage (ex: quote)
\usepackage{framed} % bordering correct with footnote hack
\usepackage{graphicx}
\usepackage{lettrine} % drop caps
\usepackage{lipsum} % generate text for testing
\usepackage[framemethod=tikz,]{mdframed} % maybe used for frame with footnotes inside
\usepackage{pdftexcmds} % needed for tests expressions
\usepackage{polyglossia} % non-break space french punct, bug Warning: "Failed to patch part"
\usepackage[%
  indentfirst=false,
  vskip=1em,
  noorphanfirst=true,
  noorphanafter=true,
  leftmargin=\parindent,
  rightmargin=0pt,
]{quoting}
\usepackage{ragged2e}
\usepackage{setspace} % \setstretch for <quote>
\usepackage{tabularx} % <table>
\usepackage[explicit]{titlesec} % wear titles, !NO implicit
\usepackage{tikz} % ornaments
\usepackage{tocloft} % styling tocs
\usepackage[fit]{truncate} % used im runing titles
\usepackage{unicode-math}
\usepackage[normalem]{ulem} % breakable \uline, normalem is absolutely necessary to keep \emph
\usepackage{verse} % <l>
\usepackage{xcolor} % named colors
\usepackage{xparse} % @ifundefined
\XeTeXdefaultencoding "iso-8859-1" % bad encoding of xstring
\usepackage{xstring} % string tests
\XeTeXdefaultencoding "utf-8"
\PassOptionsToPackage{hyphens}{url} % before hyperref, which load url package

% TOTEST
% \usepackage{hypcap} % links in caption ?
% \usepackage{marginnote}
% TESTED
% \usepackage{background} % doesn’t work with xetek
% \usepackage{bookmark} % prefers the hyperref hack \phantomsection
% \usepackage[color, leftbars]{changebar} % 2 cols doc, impossible to keep bar left
% \usepackage[utf8x]{inputenc} % inputenc package ignored with utf8 based engines
% \usepackage[sfdefault,medium]{inter} % no small caps
% \usepackage{firamath} % choose firasans instead, firamath unavailable in Ubuntu 21-04
% \usepackage{flushend} % bad for last notes, supposed flush end of columns
% \usepackage[stable]{footmisc} % BAD for complex notes https://texfaq.org/FAQ-ftnsect
% \usepackage{helvet} % not for XeLaTeX
% \usepackage{multicol} % not compatible with too much packages (longtable, framed, memoir…)
% \usepackage[default,oldstyle,scale=0.95]{opensans} % no small caps
% \usepackage{sectsty} % \chapterfont OBSOLETE
% \usepackage{soul} % \ul for underline, OBSOLETE with XeTeX
% \usepackage[breakable]{tcolorbox} % text styling gone, footnote hack not kept with breakable


% Metadata inserted by a program, from the TEI source, for title page and runing heads
\title{\textbf{ Le Capital. Livre I, Section I : Marchandise et monnaie }}
\date{1867}
\author{Marx, Karl}
\def\elbibl{Marx, Karl. 1867. \emph{Le Capital. Livre I, Section I : Marchandise et monnaie}}
\def\elsource{\href{https://www.marxists.org/francais/marx/works/1867/Capital-I/}{\dotuline{https://www.marxists.org/francais/marx/works/1867/Capital-I/}}\footnote{\href{https://www.marxists.org/francais/marx/works/1867/Capital-I/}{\url{https://www.marxists.org/francais/marx/works/1867/Capital-I/}}}}

% Default metas
\newcommand{\colorprovide}[2]{\@ifundefinedcolor{#1}{\colorlet{#1}{#2}}{}}
\colorprovide{rubric}{red}
\colorprovide{silver}{lightgray}
\@ifundefined{syms}{\newfontfamily\syms{DejaVu Sans}}{}
\newif\ifdev
\@ifundefined{elbibl}{% No meta defined, maybe dev mode
  \newcommand{\elbibl}{Titre court ?}
  \newcommand{\elbook}{Titre du livre source ?}
  \newcommand{\elabstract}{Résumé\par}
  \newcommand{\elurl}{http://oeuvres.github.io/elbook/2}
  \author{Éric Lœchien}
  \title{Un titre de test assez long pour vérifier le comportement d’une maquette}
  \date{1566}
  \devtrue
}{}
\let\eltitle\@title
\let\elauthor\@author
\let\eldate\@date


\defaultfontfeatures{
  % Mapping=tex-text, % no effect seen
  Scale=MatchLowercase,
  Ligatures={TeX,Common},
}


% generic typo commands
\newcommand{\astermono}{\medskip\centerline{\color{rubric}\large\selectfont{\syms ✻}}\medskip\par}%
\newcommand{\astertri}{\medskip\par\centerline{\color{rubric}\large\selectfont{\syms ✻\,✻\,✻}}\medskip\par}%
\newcommand{\asterism}{\bigskip\par\noindent\parbox{\linewidth}{\centering\color{rubric}\large{\syms ✻}\\{\syms ✻}\hskip 0.75em{\syms ✻}}\bigskip\par}%

% lists
\newlength{\listmod}
\setlength{\listmod}{\parindent}
\setlist{
  itemindent=!,
  listparindent=\listmod,
  labelsep=0.2\listmod,
  parsep=0pt,
  % topsep=0.2em, % default topsep is best
}
\setlist[itemize]{
  label=—,
  leftmargin=0pt,
  labelindent=1.2em,
  labelwidth=0pt,
}
\setlist[enumerate]{
  label={\bf\color{rubric}\arabic*.},
  labelindent=0.8\listmod,
  leftmargin=\listmod,
  labelwidth=0pt,
}
\newlist{listalpha}{enumerate}{1}
\setlist[listalpha]{
  label={\bf\color{rubric}\alph*.},
  leftmargin=0pt,
  labelindent=0.8\listmod,
  labelwidth=0pt,
}
\newcommand{\listhead}[1]{\hspace{-1\listmod}\emph{#1}}

\renewcommand{\hrulefill}{%
  \leavevmode\leaders\hrule height 0.2pt\hfill\kern\z@}

% General typo
\DeclareTextFontCommand{\textlarge}{\large}
\DeclareTextFontCommand{\textsmall}{\small}

% commands, inlines
\newcommand{\anchor}[1]{\Hy@raisedlink{\hypertarget{#1}{}}} % link to top of an anchor (not baseline)
\newcommand\abbr[1]{#1}
\newcommand{\autour}[1]{\tikz[baseline=(X.base)]\node [draw=rubric,thin,rectangle,inner sep=1.5pt, rounded corners=3pt] (X) {\color{rubric}#1};}
\newcommand\corr[1]{#1}
\newcommand{\ed}[1]{ {\color{silver}\sffamily\footnotesize (#1)} } % <milestone ed="1688"/>
\newcommand\expan[1]{#1}
\newcommand\foreign[1]{\emph{#1}}
\newcommand\gap[1]{#1}
\renewcommand{\LettrineFontHook}{\color{rubric}}
\newcommand{\initial}[2]{\lettrine[lines=2, loversize=0.3, lhang=0.3]{#1}{#2}}
\newcommand{\initialiv}[2]{%
  \let\oldLFH\LettrineFontHook
  % \renewcommand{\LettrineFontHook}{\color{rubric}\ttfamily}
  \IfSubStr{QJ’}{#1}{
    \lettrine[lines=4, lhang=0.2, loversize=-0.1, lraise=0.2]{\smash{#1}}{#2}
  }{\IfSubStr{É}{#1}{
    \lettrine[lines=4, lhang=0.2, loversize=-0, lraise=0]{\smash{#1}}{#2}
  }{\IfSubStr{ÀÂ}{#1}{
    \lettrine[lines=4, lhang=0.2, loversize=-0, lraise=0, slope=0.6em]{\smash{#1}}{#2}
  }{\IfSubStr{A}{#1}{
    \lettrine[lines=4, lhang=0.2, loversize=0.2, slope=0.6em]{\smash{#1}}{#2}
  }{\IfSubStr{V}{#1}{
    \lettrine[lines=4, lhang=0.2, loversize=0.2, slope=-0.5em]{\smash{#1}}{#2}
  }{
    \lettrine[lines=4, lhang=0.2, loversize=0.2]{\smash{#1}}{#2}
  }}}}}
  \let\LettrineFontHook\oldLFH
}
\newcommand{\labelchar}[1]{\textbf{\color{rubric} #1}}
\newcommand{\milestone}[1]{\autour{\footnotesize\color{rubric} #1}} % <milestone n="4"/>
\newcommand\name[1]{#1}
\newcommand\orig[1]{#1}
\newcommand\orgName[1]{#1}
\newcommand\persName[1]{#1}
\newcommand\placeName[1]{#1}
\newcommand{\pn}[1]{\IfSubStr{-—–¶}{#1}% <p n="3"/>
  {\noindent{\bfseries\color{rubric}   ¶  }}
  {{\footnotesize\autour{ #1}  }}}
\newcommand\reg{}
% \newcommand\ref{} % already defined
\newcommand\sic[1]{#1}
\newcommand\surname[1]{\textsc{#1}}
\newcommand\term[1]{\textbf{#1}}

\def\mednobreak{\ifdim\lastskip<\medskipamount
  \removelastskip\nopagebreak\medskip\fi}
\def\bignobreak{\ifdim\lastskip<\bigskipamount
  \removelastskip\nopagebreak\bigskip\fi}

% commands, blocks
\newcommand{\byline}[1]{\bigskip{\RaggedLeft{#1}\par}\bigskip}
\newcommand{\bibl}[1]{{\RaggedLeft{#1}\par\bigskip}}
\newcommand{\biblitem}[1]{{\noindent\hangindent=\parindent   #1\par}}
\newcommand{\dateline}[1]{\medskip{\RaggedLeft{#1}\par}\bigskip}
\newcommand{\labelblock}[1]{\medbreak{\noindent\color{rubric}\bfseries #1}\par\mednobreak}
\newcommand{\salute}[1]{\bigbreak{#1}\par\medbreak}
\newcommand{\signed}[1]{\bigbreak\filbreak{\raggedleft #1\par}\medskip}

% environments for blocks (some may become commands)
\newenvironment{borderbox}{}{} % framing content
\newenvironment{citbibl}{\ifvmode\hfill\fi}{\ifvmode\par\fi }
\newenvironment{docAuthor}{\ifvmode\vskip4pt\fontsize{16pt}{18pt}\selectfont\fi\itshape}{\ifvmode\par\fi }
\newenvironment{docDate}{}{\ifvmode\par\fi }
\newenvironment{docImprint}{\vskip6pt}{\ifvmode\par\fi }
\newenvironment{docTitle}{\vskip6pt\bfseries\fontsize{18pt}{22pt}\selectfont}{\par }
\newenvironment{msHead}{\vskip6pt}{\par}
\newenvironment{msItem}{\vskip6pt}{\par}
\newenvironment{titlePart}{}{\par }


% environments for block containers
\newenvironment{argument}{\itshape\parindent0pt}{\vskip1.5em}
\newenvironment{biblfree}{}{\ifvmode\par\fi }
\newenvironment{bibitemlist}[1]{%
  \list{\@biblabel{\@arabic\c@enumiv}}%
  {%
    \settowidth\labelwidth{\@biblabel{#1}}%
    \leftmargin\labelwidth
    \advance\leftmargin\labelsep
    \@openbib@code
    \usecounter{enumiv}%
    \let\p@enumiv\@empty
    \renewcommand\theenumiv{\@arabic\c@enumiv}%
  }
  \sloppy
  \clubpenalty4000
  \@clubpenalty \clubpenalty
  \widowpenalty4000%
  \sfcode`\.\@m
}%
{\def\@noitemerr
  {\@latex@warning{Empty `bibitemlist' environment}}%
\endlist}
\newenvironment{quoteblock}% may be used for ornaments
  {\begin{quoting}}
  {\end{quoting}}

% table () is preceded and finished by custom command
\newcommand{\tableopen}[1]{%
  \ifnum\strcmp{#1}{wide}=0{%
    \begin{center}
  }
  \else\ifnum\strcmp{#1}{long}=0{%
    \begin{center}
  }
  \else{%
    \begin{center}
  }
  \fi\fi
}
\newcommand{\tableclose}[1]{%
  \ifnum\strcmp{#1}{wide}=0{%
    \end{center}
  }
  \else\ifnum\strcmp{#1}{long}=0{%
    \end{center}
  }
  \else{%
    \end{center}
  }
  \fi\fi
}


% text structure
\newcommand\chapteropen{} % before chapter title
\newcommand\chaptercont{} % after title, argument, epigraph…
\newcommand\chapterclose{} % maybe useful for multicol settings
\setcounter{secnumdepth}{-2} % no counters for hierarchy titles
\setcounter{tocdepth}{5} % deep toc
\markright{\@title} % ???
\markboth{\@title}{\@author} % ???
\renewcommand\tableofcontents{\@starttoc{toc}}
% toclof format
% \renewcommand{\@tocrmarg}{0.1em} % Useless command?
% \renewcommand{\@pnumwidth}{0.5em} % {1.75em}
\renewcommand{\@cftmaketoctitle}{}
\setlength{\cftbeforesecskip}{\z@ \@plus.2\p@}
\renewcommand{\cftchapfont}{}
\renewcommand{\cftchapdotsep}{\cftdotsep}
\renewcommand{\cftchapleader}{\normalfont\cftdotfill{\cftchapdotsep}}
\renewcommand{\cftchappagefont}{\bfseries}
\setlength{\cftbeforechapskip}{0em \@plus\p@}
% \renewcommand{\cftsecfont}{\small\relax}
\renewcommand{\cftsecpagefont}{\normalfont}
% \renewcommand{\cftsubsecfont}{\small\relax}
\renewcommand{\cftsecdotsep}{\cftdotsep}
\renewcommand{\cftsecpagefont}{\normalfont}
\renewcommand{\cftsecleader}{\normalfont\cftdotfill{\cftsecdotsep}}
\setlength{\cftsecindent}{1em}
\setlength{\cftsubsecindent}{2em}
\setlength{\cftsubsubsecindent}{3em}
\setlength{\cftchapnumwidth}{1em}
\setlength{\cftsecnumwidth}{1em}
\setlength{\cftsubsecnumwidth}{1em}
\setlength{\cftsubsubsecnumwidth}{1em}

% footnotes
\newif\ifheading
\newcommand*{\fnmarkscale}{\ifheading 0.70 \else 1 \fi}
\renewcommand\footnoterule{\vspace*{0.3cm}\hrule height \arrayrulewidth width 3cm \vspace*{0.3cm}}
\setlength\footnotesep{1.5\footnotesep} % footnote separator
\renewcommand\@makefntext[1]{\parindent 1.5em \noindent \hb@xt@1.8em{\hss{\normalfont\@thefnmark . }}#1} % no superscipt in foot
\patchcmd{\@footnotetext}{\footnotesize}{\footnotesize\sffamily}{}{} % before scrextend, hyperref


%   see https://tex.stackexchange.com/a/34449/5049
\def\truncdiv#1#2{((#1-(#2-1)/2)/#2)}
\def\moduloop#1#2{(#1-\truncdiv{#1}{#2}*#2)}
\def\modulo#1#2{\number\numexpr\moduloop{#1}{#2}\relax}

% orphans and widows
\clubpenalty=9996
\widowpenalty=9999
\brokenpenalty=4991
\predisplaypenalty=10000
\postdisplaypenalty=1549
\displaywidowpenalty=1602
\hyphenpenalty=400
% Copied from Rahtz but not understood
\def\@pnumwidth{1.55em}
\def\@tocrmarg {2.55em}
\def\@dotsep{4.5}
\emergencystretch 3em
\hbadness=4000
\pretolerance=750
\tolerance=2000
\vbadness=4000
\def\Gin@extensions{.pdf,.png,.jpg,.mps,.tif}
% \renewcommand{\@cite}[1]{#1} % biblio

\usepackage{hyperref} % supposed to be the last one, :o) except for the ones to follow
\urlstyle{same} % after hyperref
\hypersetup{
  % pdftex, % no effect
  pdftitle={\elbibl},
  % pdfauthor={Your name here},
  % pdfsubject={Your subject here},
  % pdfkeywords={keyword1, keyword2},
  bookmarksnumbered=true,
  bookmarksopen=true,
  bookmarksopenlevel=1,
  pdfstartview=Fit,
  breaklinks=true, % avoid long links
  pdfpagemode=UseOutlines,    % pdf toc
  hyperfootnotes=true,
  colorlinks=false,
  pdfborder=0 0 0,
  % pdfpagelayout=TwoPageRight,
  % linktocpage=true, % NO, toc, link only on page no
}

\makeatother % /@@@>
%%%%%%%%%%%%%%
% </TEI> end %
%%%%%%%%%%%%%%


%%%%%%%%%%%%%
% footnotes %
%%%%%%%%%%%%%
\renewcommand{\thefootnote}{\bfseries\textcolor{rubric}{\arabic{footnote}}} % color for footnote marks

%%%%%%%%%
% Fonts %
%%%%%%%%%
\usepackage[]{roboto} % SmallCaps, Regular is a bit bold
% \linespread{0.90} % too compact, keep font natural
\newfontfamily\fontrun[]{Roboto Condensed Light} % condensed runing heads
\ifav
  \setmainfont[
    ItalicFont={Roboto Light Italic},
  ]{Roboto}
\else\ifbooklet
  \setmainfont[
    ItalicFont={Roboto Light Italic},
  ]{Roboto}
\else
\setmainfont[
  ItalicFont={Roboto Italic},
]{Roboto Light}
\fi\fi
\renewcommand{\LettrineFontHook}{\bfseries\color{rubric}}
% \renewenvironment{labelblock}{\begin{center}\bfseries\color{rubric}}{\end{center}}

%%%%%%%%
% MISC %
%%%%%%%%

\setdefaultlanguage[frenchpart=false]{french} % bug on part


\newenvironment{quotebar}{%
    \def\FrameCommand{{\color{rubric!10!}\vrule width 0.5em} \hspace{0.9em}}%
    \def\OuterFrameSep{\itemsep} % séparateur vertical
    \MakeFramed {\advance\hsize-\width \FrameRestore}
  }%
  {%
    \endMakeFramed
  }
\renewenvironment{quoteblock}% may be used for ornaments
  {%
    \savenotes
    \setstretch{0.9}
    \normalfont
    \begin{quotebar}
  }
  {%
    \end{quotebar}
    \spewnotes
  }


\renewcommand{\headrulewidth}{\arrayrulewidth}
\renewcommand{\headrule}{{\color{rubric}\hrule}}

% delicate tuning, image has produce line-height problems in title on 2 lines
\titleformat{name=\chapter} % command
  [display] % shape
  {\vspace{1.5em}\centering} % format
  {} % label
  {0pt} % separator between n
  {}
[{\color{rubric}\huge\textbf{#1}}\bigskip] % after code
% \titlespacing{command}{left spacing}{before spacing}{after spacing}[right]
\titlespacing*{\chapter}{0pt}{-2em}{0pt}[0pt]

\titleformat{name=\section}
  [block]{}{}{}{}
  [\vbox{\color{rubric}\large\raggedleft\textbf{#1}}]
\titlespacing{\section}{0pt}{0pt plus 4pt minus 2pt}{\baselineskip}

\titleformat{name=\subsection}
  [block]
  {}
  {} % \thesection
  {} % separator \arrayrulewidth
  {}
[\vbox{\large\textbf{#1}}]
% \titlespacing{\subsection}{0pt}{0pt plus 4pt minus 2pt}{\baselineskip}

\ifaiv
  \fancypagestyle{main}{%
    \fancyhf{}
    \setlength{\headheight}{1.5em}
    \fancyhead{} % reset head
    \fancyfoot{} % reset foot
    \fancyhead[L]{\truncate{0.45\headwidth}{\fontrun\elbibl}} % book ref
    \fancyhead[R]{\truncate{0.45\headwidth}{ \fontrun\nouppercase\leftmark}} % Chapter title
    \fancyhead[C]{\thepage}
  }
  \fancypagestyle{plain}{% apply to chapter
    \fancyhf{}% clear all header and footer fields
    \setlength{\headheight}{1.5em}
    \fancyhead[L]{\truncate{0.9\headwidth}{\fontrun\elbibl}}
    \fancyhead[R]{\thepage}
  }
\else
  \fancypagestyle{main}{%
    \fancyhf{}
    \setlength{\headheight}{1.5em}
    \fancyhead{} % reset head
    \fancyfoot{} % reset foot
    \fancyhead[RE]{\truncate{0.9\headwidth}{\fontrun\elbibl}} % book ref
    \fancyhead[LO]{\truncate{0.9\headwidth}{\fontrun\nouppercase\leftmark}} % Chapter title, \nouppercase needed
    \fancyhead[RO,LE]{\thepage}
  }
  \fancypagestyle{plain}{% apply to chapter
    \fancyhf{}% clear all header and footer fields
    \setlength{\headheight}{1.5em}
    \fancyhead[L]{\truncate{0.9\headwidth}{\fontrun\elbibl}}
    \fancyhead[R]{\thepage}
  }
\fi

\ifav % a5 only
  \titleclass{\section}{top}
\fi

\newcommand\chapo{{%
  \vspace*{-3em}
  \centering % no vskip ()
  {\Large\addfontfeature{LetterSpace=25}\bfseries{\elauthor}}\par
  \smallskip
  {\large\eldate}\par
  \bigskip
  {\Large\selectfont{\eltitle}}\par
  \bigskip
  {\color{rubric}\hline\par}
  \bigskip
  {\Large TEXTE LIBRE À PARTICPATION LIBRE\par}
  \centerline{\small\color{rubric} {hurlus.fr, tiré le \today}}\par
  \bigskip
}}

\newcommand\cover{{%
  \thispagestyle{empty}
  \centering
  {\LARGE\bfseries{\elauthor}}\par
  \bigskip
  {\Large\eldate}\par
  \bigskip
  \bigskip
  {\LARGE\selectfont{\eltitle}}\par
  \vfill\null
  {\color{rubric}\setlength{\arrayrulewidth}{2pt}\hline\par}
  \vfill\null
  {\Large TEXTE LIBRE À PARTICPATION LIBRE\par}
  \centerline{{\href{https://hurlus.fr}{\dotuline{hurlus.fr}}, tiré le \today}}\par
}}

\begin{document}
\pagestyle{empty}
\ifbooklet{
  \cover\newpage
  \thispagestyle{empty}\hbox{}\newpage
  \cover\newpage\noindent Les voyages de la brochure\par
  \bigskip
  \begin{tabularx}{\textwidth}{l|X|X}
    \textbf{Date} & \textbf{Lieu}& \textbf{Nom/pseudo} \\ \hline
    \rule{0pt}{25cm} &  &   \\
  \end{tabularx}
  \newpage
  \addtocounter{page}{-4}
}\fi

\thispagestyle{empty}
\ifaiv
  \twocolumn[\chapo]
\else
  \chapo
\fi
{\it\elabstract}
\bigskip
\makeatletter\@starttoc{toc}\makeatother % toc without new page
\bigskip

\pagestyle{main} % after style

  \section[{1.1.1. La marchandise}]{1.1.1. La marchandise}\renewcommand{\leftmark}{1.1.1. La marchandise}

\subsection[{1.1.1.1. Les deux facteurs de la marchandise : valeur d’usage et valeur d’échange ou valeur proprement dite. (Substance de la valeur, Grandeur de la valeur.)}]{1.1.1.1. Les deux facteurs de la marchandise : valeur d’usage et valeur d’échange ou valeur proprement dite. (Substance de la valeur, Grandeur de la valeur.)}
\noindent La richesse des sociétés dans lesquelles règne le mode de production capitaliste s’annonce comme une « immense accumulation de marchandises\footnote{Karl MARX, \emph{Contribution à la critique de l’économie politique}, Berlin, 1859, p. 3.} ». L’analyse de la marchandise, forme élémentaire de cette richesse, sera par conséquent le point de départ de nos recherches.\par
La marchandise est d’abord un objet extérieur, une chose qui par ses propriétés satisfait des besoins humains de n’importe quelle espèce. Que ces besoins aient pour origine l’estomac ou la fantaisie, leur nature ne change rien à l’affaire\footnote{« Le désir implique le besoin ; c’est l’appétit de l’esprit, lequel lui est aussi naturel que la faim l’est au corps. C’est de là que la plupart des choses tirent leur valeur. » (Nicholas BARBON, \emph{A Discourse concerning coining the new money lighter, in answer to Mr Locke's Considerations}, etc., London, 1696, p. 2 et 3.)}. Il ne s’agit pas non plus ici de savoir comment ces besoins sont satisfaits, soit immédiatement, si l’objet est un moyen de subsistance, soit par une voie détournée, si c’est un moyen de production.\par
Chaque chose utile, comme le fer, le papier, etc., peut être considérée sous un double point de vue, celui de la qualité et celui de la quantité. Chacune est un ensemble de propriétés diverses et peut, par conséquent, être utile par différents côtés. Découvrir ces côtés divers et, en même temps, les divers usages des choses est une œuvre de l’histoire\footnote{« Les choses ont une vertu intrinsèque (\emph{virtue}, telle est chez Barbon la désignation spécifique pour valeur d’usage) qui en tout lieu ont la même qualité comme l’aimant, par exemple, attire le fer » (\emph{ibid.}, p. 6). La propriété qu’a l’aimant d’attirer le fer ne devint utile que lorsque, par son moyen, on eut découvert la polarité magnétique.}. Telle est la découverte de mesures sociales pour la quantité des choses utiles. La diversité de ces mesures des marchandises a pour origine en partie la nature variée des objets à mesurer, en partie la convention.\par
L’utilité d’une chose fait de cette chose une valeur d’usage\footnote{« Ce qui fait la valeur naturelle d’une chose, c’est la propriété qu’elle a de satisfaire les besoins ou les convenances de la vie humaine. » (John LOCKE, \emph{Some Considerations on the Consequences of the Lowering of Interest}, 1691 ; in \emph{Works}, Londres, 1777, t. II, p. 28.) Au XVIIe siècle on trouve encore souvent chez les écrivains anglais le mot \emph{Worth} pour valeur d’usage et le mot \emph{Value} pour valeur d’échange, suivant l’esprit d’une langue qui aime à exprimer la chose \emph{immédiate} en termes germaniques et la chose réfléchie en termes romans.}. Mais cette utilité n’a rien de vague et d’indécis. Déterminée par les propriétés du corps de la marchandise, elle n’existe point sans lui. Ce corps lui-même, tel que fer, froment, diamant, etc., est conséquemment une valeur d’usage, et ce n’est pas le plus ou moins de travail qu’il faut à l’homme pour s’approprier les qualités utiles qui lui donne ce caractère. Quand il est question de valeurs d’usage, on sous-entend toujours une quantité déterminée, comme une douzaine de montres, un mètre de toile, une tonne de fer, etc. Les valeurs d’usage des marchandises fournissent le fonds d’un savoir particulier, de la science et de la routine commerciales\footnote{Dans la société bourgeoise « nul n’est censé ignorer la loi ». – En vertu d’une \emph{fictio juris} [fiction juridique] économique, tout acheteur est censé posséder une connaissance encyclopédique des marchandises.}.\par
Les valeurs d’usage ne se réalisent que dans l’usage ou la consommation. Elles forment \emph{la matière de la richesse}, quelle que soit la forme sociale de cette richesse. Dans la société que nous avons à examiner, elles sont en même temps les soutiens matériels de la valeur d’échange.\par
La valeur d’échange apparaît d’abord comme le rapport \emph{quantitatif}, comme la proportion dans laquelle des valeurs d’usage d’espèce différente s’échangent l’une contre l’autre\footnote{« La valeur consiste dans le \emph{rapport d’échange} qui se trouve entre telle chose et telle autre, entre telle mesure d’une production et telle mesure des autres. » (LE TROSNE, \emph{De l’intérêt social}, in \emph{Physiocrates}, Ed. Daire, Paris, 1846, t. XII, p. 889.)}, rapport qui change constamment avec le temps et le lieu. La valeur d’échange semble donc quelque chose d’arbitraire et de purement relatif ; une valeur d’échange intrinsèque, immanente à la marchandise, paraît être, comme dit l’école, une \emph{contradictio in adjecto}\footnote{« Rien ne peut avoir une valeur intrinsèque. » (N. BARBON, \emph{op. cit.}, p. 6) ; ou, comme dit Butler : \emph{The value of a thing} \\
\emph{Is just as much as it will bring}.}. Considérons la chose de plus près.\par
Une marchandise particulière, un quarteron de froment, par exemple, s’échange dans les proportions les plus diverses avec d’autres articles. Cependant, sa valeur d’échange reste immuable, de quelque manière qu’on l’exprime, en \emph{x} cirage, \emph{y} soie, \emph{z} or, et ainsi de suite. Elle doit donc avoir un contenu distinct de ces expressions diverses.\par
Prenons encore deux marchandises, soit du froment et du fer. Quel que soit leur rapport d’échange, il peut toujours être représenté par une équation dans laquelle une quantité donnée de froment est réputée égale à une quantité quelconque de fer, par exemple : 1 quarteron de froment = \emph{a} kilogramme de fer. Que signifie cette équation ? C’est que dans deux objets différents, dans 1 quarteron de froment et dans \emph{a} kilogramme de fer, il existe quelque chose de commun. Les deux objets sont donc égaux à un \emph{troisième} qui, par lui-même, n’est ni l’un ni l’autre. Chacun des deux doit, en tant que valeur d’échange, être réductible au troisième, indépendamment de l’autre.\par
Un exemple emprunté à la géométrie élémentaire va nous mettre cela sous les yeux. Pour mesurer et comparer les surfaces de toutes les figures rectilignes, on les décompose en triangles. On ramène le triangle lui-même à une expression tout à fait différente de son aspect visible : au demi-produit de sa base par sa hauteur. De même, les valeurs d’échange des marchandises doivent être ramenées à quelque chose qui leur est commun et dont elles représentent un plus ou un moins.\par
Ce quelque chose de commun ne peut être une propriété naturelle quelconque, géométrique, physique, chimique, etc., des marchandises. Leurs qualités naturelles n’entrent en considération qu’autant qu’elles leur donnent une utilité qui en fait des valeurs d’usage. Mais, d’un autre côté, il est évident que l’on fait abstraction de la valeur d’usage des marchandises quand on les échange et que tout rapport d’échange est même caractérisé par cette abstraction. Dans l’échange, une valeur d’utilité vaut précisément autant que toute autre, pourvu qu’elle se trouve en proportion convenable. Ou bien, comme dit le vieux Barbon :\par
« Une espèce de marchandise est aussi bonne qu’une autre quand sa valeur d’échange est égale ; il n’y a aucune différence, aucune distinction dans les choses chez lesquelles cette valeur est la même\footnote{« \emph{One sort of wares are as good as another, if the value be equal… There is no difference or distinction in things of equal value}. » Barbon ajoute : « Cent livres sterling en plomb ou en fer ont autant de valeur que cent livres sterling en argent ou en or. » (N. BARBON, \emph{op. cit.}, p. 53 et 7.)}. »\par
Comme valeurs d’usage, les marchandises sont avant tout de qualité différente ; comme valeurs d’échange, elles ne peuvent être que de différente quantité.\par
La valeur d’usage des marchandises une fois mise de côté, il ne leur reste plus qu’une qualité, celle d’être des produits du travail. Mais déjà le produit du travail lui-même est métamorphosé à notre insu. Si nous faisons abstraction de sa valeur d’usage, tous les éléments matériels et formels qui lui donnaient cette valeur disparaissent à la fois. Ce n’est plus, par exemple, une table, ou une maison, ou du fil, ou un objet utile quelconque ; ce n’est pas non plus le produit du travail du tourneur, du maçon, de n’importe quel travail productif déterminé. Avec les caractères utiles particuliers des produits du travail disparaissent en même temps, et le caractère utile des travaux qui y sont contenus, et les formes concrètes diverses qui distinguent une espèce de travail d’une autre espèce. Il ne reste donc plus que le caractère commun de ces travaux ; ils sont tous ramenés au même travail humain, à une dépense de force humaine de travail sans égard à la forme particulière sous laquelle cette force a été dépensée.\par
Considérons maintenant le résidu des produits du travail. Chacun d’eux ressemble complètement à l’autre. Ils ont tous une même réalité fantomatique. Métamorphosés en \emph{sublimés} identiques, échantillons du même travail indistinct, tous ces objets ne manifestent plus qu’une chose, c’est que dans leur production une force de travail humaine a été dépensée, que du travail humain y est accumulé. En tant que cristaux de cette substance sociale commune, ils sont réputés valeurs.\par
Le quelque chose de commun qui se montre dans le rapport d’échange ou dans la valeur d’échange des marchandises est par conséquent leur valeur ; et une valeur d’usage, ou un article quelconque, n’a une valeur qu’autant que du travail humain est matérialisé en elle.\par
Comment mesurer maintenant la grandeur de sa valeur ? Par le \emph{quantum} de la substance « créatrice de valeur » contenue en lui, du travail. La quantité de travail elle-même a pour mesure sa durée dans le temps, et le temps de travail possède de nouveau sa mesure, dans des parties du temps telles que l’heure, le jour, etc.\par
On pourrait s’imaginer que si la valeur d’une marchandise est déterminée par le \emph{quantum} de travail dépensé pendant sa production plus un homme est paresseux ou inhabile, plus sa marchandise a de valeur, parce qu’il emploie plus de temps à sa fabrication. Mais le travail qui forme la substance de la valeur des marchandises est du travail égal et indistinct une dépense de la même force. La force de travail de la société tout entière, laquelle se manifeste dans l’ensemble des valeurs, ne compte par conséquent que comme force unique, bien qu’elle se compose de forces individuelles innombrables. Chaque force de travail individuelle est égale à toute autre, en tant qu’elle possède le caractère d’une force sociale moyenne et fonctionne comme telle, c’est-à-dire n’emploie dans la production d’une marchandise que le temps de travail nécessaire en moyenne ou le temps de travail nécessaire socialement.\par
Le temps socialement nécessaire à la production des marchandises est celui qu’exige tout travail, exécuté avec le degré moyen d’habileté et d’intensité et dans des conditions qui, par rapport au milieu social donné, sont normales. Après l’introduction en Angleterre du tissage à la vapeur, il fallut peut-être moitié moins de travail qu’auparavant pour transformer en tissu une certaine quantité de fil. Le tisserand anglais, lui, eut toujours besoin du même temps pour opérer cette transformation ; mais dès lors le produit de son heure de travail individuelle ne représenta plus que la moitié d’une heure sociale de travail et ne donna plus que la moitié de la valeur première.\par
C’est donc seulement le \emph{quantum} de travail, ou le temps de travail nécessaire, dans une société donnée, à la production d’un article qui en détermine la quantité de valeur\footnote{« Dans les échanges, la valeur des choses utiles est réglée par la quantité de travail nécessairement exigée et ordinairement employée pour leur production. » (\emph{Some Thoughts on the Interest of Money in general, and particulary in the Public Fonds}, etc., London, p. 36.) Ce remarquable écrit anonyme du siècle dernier ne porte aucune date. D’après son contenu, il est évident qu’il a paru sous George II, vers 1739 ou 1740. [Note à la deuxième édition]}. Chaque marchandise particulière compte en général comme un exemplaire moyen de son espèce\footnote{« Toutes les productions d’un même genre ne forment proprement qu’une masse, dont le prix se détermine en général et sans égard aux circonstances particulières. » (Le Trosne, \emph{op. cit.}, p. 893.)}. Les marchandises dans lesquelles sont contenues d’égales quantités de travail, ou qui peuvent être produites dans le même temps, ont, par conséquent, une valeur égale. La valeur d’une marchandise est à la valeur de toute autre marchandise, dans le même rapport que le temps de travail nécessaire à la production de l’une est au temps de travail nécessaire à la production de l’autre.\par
La quantité de valeur d’une marchandise resterait évidemment constante si le temps nécessaire à sa production restait aussi constant. Mais ce denier varie avec chaque modification de la force productive du travail, qui, de son côté, dépend de circonstances diverses, entre autres de l’habileté moyenne des travailleurs ; du développement de la science et du degré de son application technologique des combinaisons sociales de la production ; de l’étendue et de l’efficacité des moyens de produire et des conditions purement naturelles. La même quantité de travail est représentée, par exemple, par 8 boisseaux de froment si la saison est favorable, par 4 boisseaux seulement dans le cas contraire. La même quantité de travail fournit une plus forte masse de métal dans les mines riches que dans les mines pauvres, etc. Les diamants ne se présentent que rarement dans la couche supérieure de l’écorce terrestre ; aussi faut-il pour les trouver un temps considérable en moyenne, de sorte qu’ils représentent beaucoup de travail sous un petit volume. Il est douteux que l’or ait jamais payé complètement sa valeur. Cela est encore plus vrai du diamant. D’après \emph{Eschwege}, le produit entier de l’exploitation des mines de diamants du Brésil, pendant 80 ans, n’avait pas encore atteint en 1823 le prix du produit moyen d’une année et demie dans les plantations de sucre ou de café du même pays, bien qu’il représentât beaucoup plus de travail et, par conséquent plus de valeur. Avec des mines plus riches, la même quantité de travail se réaliserait dans une plus grande quantité de diamants dont la valeur baisserait. Si l’on réussissait à transformer avec peu de travail le charbon en diamant, la valeur de ce dernier tomberait peut-être au-dessous de celle des briques. En général, plus est grande la force productive du travail, plus est court le temps nécessaire à la production d’un article, et plus est petite la masse de travail cristallisée en lui, plus est petite sa valeur. Inversement, plus est petite la force productive du travail, plus est grand le temps nécessaire à la production d’un article, et plus est grande sa valeur. La quantité de valeur d’une marchandise varie donc en raison directe du \emph{quantum} et en raison inverse de la force productive du travail qui se réalise en elle.\par
Nous connaissons maintenant la substance de la valeur : c’est le travail. Nous connaissons la mesure de sa quantité : c’est la durée du travail.\par
Une chose peut être une valeur d’usage sans être une valeur. Il suffit pour cela qu’elle soit utile à l’homme sans qu’elle provienne de son travail. Tels sont l’air des prairies naturelles, un sol vierge, etc. Une chose peut être utile et produit du travail humain, sans être marchandise. Quiconque, par son produit, satisfait ses propres besoins ne crée qu’une valeur d’usage personnelle. Pour produire des marchandises, il doit non seulement produire des valeurs d’usage, mais des valeurs d’usage pour d’autres, des valeurs d’usage sociales\footnote{ \noindent (Et non simplement pour \emph{d’autres}. Le paysan au Moyen Age produisait la redevance en blé pour le seigneur féodal, la dîme en blé pour la prêtraille. Mais ni le blé de la redevance, ni le blé de la dîme ne devenaient marchandise, du fait d’être produits pour d’autres. Pour devenir marchandise, le produit doit être livré à \emph{l’autre}, auquel il sert de valeur d’usage, par voie d’échange.)\par
 J’intercale ici ce passage entre parenthèses, parce qu’en l’omettant, il est arrivé souvent que le lecteur se soit mépris en croyant que chaque produit, qui est consommé par un autre que le producteur, est considéré par Marx comme une marchandise. (F. E.) [Friedrich Engels pour la 4° édition allemande]
}. Enfin, aucun objet ne peut être une valeur s’il n’est une chose utile. S’il est inutile, le travail qu’il renferme est dépensé inutilement et conséquemment ne crée pas valeur.
\subsection[{1.1.1.2. Double caractère du travail présenté par la marchandise.}]{1.1.1.2. Double caractère du travail présenté par la marchandise.}
\noindent Au premier abord, la marchandise nous est apparue comme quelque chose à double face, valeur d’usage et valeur d’échange. Ensuite nous avons vu que tous les caractères qui distinguent le travail productif de valeurs d’usage disparaissent dès qu’il s’exprime dans la valeur proprement dite. J’ai, le premier, mis en relief ce double caractère du travail représenté dans la marchandise\footnote{K. MARX, \emph{Contribution}…, op. cit., p. 12, 13 et suivantes.}. Comme l’économie politique pivote autour de ce point, il nous faut ici entrer dans de plus amples détails. Prenons deux marchandises, un habit, par exemple, et 10 mètres de toile ; admettons que la première ait deux fois la valeur de la seconde, de sorte que si 10 mètres de toile = \emph{x}, l’habit = 2 \emph{x}. L’habit est une valeur d’usage qui satisfait un besoin particulier. Il provient genre particulier « activité productive, déterminée par son but, par son mode d’opération, son objet, ses moyens et son résultat. Le travail qui se manifeste dans l’utilité ou la valeur d’usage de son produit, nous le nommons tout simplement travail utile. A ce point de vue, il est toujours considéré par rapport à son rendement.\par
De même que l’habit et la toile sont deux choses utiles différentes, de même le travail du tailleur, qui fait l’habit, se distingue de celui du tisserand, qui fait de la toile. Si ces objets n’étaient pas des valeurs d’usage de qualité diverse et, par conséquent, des produits de travaux utiles de diverse qualité, ils ne pourraient se faire vis-à-vis comme marchandises. L’habit ne s’échange pas contre l’habit, une valeur d’usage contre la même valeur d’usage.\par
A l’ensemble des valeurs d’usage de toutes sortes correspond un ensemble de travaux utiles également variés, distincts de genre, d’espèce, de famille – une division sociale du travail. Sans elle pas de production de marchandises, bien que la production des marchandises ne soit point réciproquement indispensable à la division sociale du travail. Dans la vieille communauté indienne, le travail est socialement divisé sans que les produits deviennent pour cela marchandises. Ou, pour prendre un exemple plus familier, dans chaque fabrique le travail est soumis à une division systématique ; mais cette division ne provient pas de ce que les travailleurs échangent réciproquement leurs produits individuels. Il n’y a que les produits de travaux privés et indépendants les uns des autres qui se présentent comme marchandises réciproquement échangeables.\par
C’est donc entendu : la valeur d’usage de chaque marchandise recèle un travail utile spécial ou une activité productive qui répond à un but particulier. Des valeurs d’usage ne peuvent se faire face comme marchandises que si elles contiennent des travaux utiles de qualité différente. Dans une société dont les produits prennent en général la forme marchandise, c’est-à-dire dans une société où tout producteur doit être marchand, la différence entre les genres divers des travaux utiles qui s’exécutent indépendamment les uns des autres pour le compte privé de producteurs libres se développe en un système fortement ramifié, en une division sociale du travail.\par
Il est d’ailleurs fort indifférent à l’habit qu’il soit porté par le tailleur ou par ses pratiques. Dans les deux cas, il sert de valeur d’usage. De même le rapport entre l’habit et le travail qui le produit n’est pas le moins du monde changé parce que sa fabrication constitue une profession particulière, et qu’il devient un anneau de la division sociale du travail. Dès que le besoin de se vêtir l’y a forcé, pendant des milliers d’années, l’homme s’est taillé des vêtements sans qu’un seul homme devînt pour cela un tailleur. Mais toile ou habit, n’importe quel élément de la richesse matérielle non fourni par la nature, a toujours dû son existence à un travail productif spécial ayant pour but d’approprier des matières naturelles à des besoins humains. En tant qu’il produit des valeurs d’usage, qu’il est utile, le travail, indépendamment de toute forme de société, est la condition indispensable de l’existence de l’homme, une nécessité éternelle, le médiateur de la circulation matérielle entre la nature et l’homme.\par
Les valeurs d’usage, toile, habit, etc., c’est-à-dire les corps des marchandises, sont des combinaisons de deux éléments, matière et travail. Si l’on en soustrait la somme totale des divers travaux utiles qu’ils recèlent, il reste toujours un résidu matériel, un quelque chose fourni par la nature et qui ne doit rien à l’homme.\par
L’homme ne peut point procéder autrement que la nature elle-même, c’est-à-dire il ne fait que changer la forme des matières\footnote{« Tous les phénomènes de l’univers, qu’ils émanent de l’homme ou des lois générales de la nature, ne nous donnent pas l’idée de création réelle, mais seulement d’une modification de la matière. Réunir et séparer – voilà les seuls éléments que l’esprit humain saisisse en analysant l’idée de la reproduction. C’est aussi bien une reproduction de valeur (\emph{valeur d’usage}, bien qu’ici Verri, dans sa polémique contre les physiocrates, ne sache pas lui-même de quelle sorte de valeur il parle) et de richesse, que la terre, l’air et l’eau se transforment en grain, ou que la main de l’homme convertisse la glutine d’un insecte en soie, ou lorsque des pièces de métal s’organisent par un arrangement de leurs atomes. » (Pietro VERRI, \emph{Meditazioni sulla Economia politica}, imprimé pour la première fois en 1773, Edition des économistes italiens de Custodi, \emph{Parte moderna}, 1804, t. xv, p. 21-22.)}. Bien plus, dans cette œuvre de simple transformation, il est encore constamment soutenu par des forces naturelles. Le travail n’est donc pas l’unique source des valeurs d’usage qu’il produit, de la richesse matérielle. Il en est le père, et la terre, la mère, comme dit \emph{William Petty}.\par
Laissons maintenant la marchandise en tant qu’objet d’utilité et revenons à sa valeur.\par
D’après notre supposition, l’habit vaut deux fois la toile. Ce n’est là cependant qu’une différence \emph{quantitative} qui ne nous intéresse pas encore. Aussi observons-nous que si un habit est égal à deux fois 10 mètres de toile, 20 mètres de toile sont égaux à un habit. En tant que valeurs, l’habit et la toile sont des choses de même substance, des expressions objectives d’un travail identique. Mais la confection des habits et le tissage sont des travaux différents. Il y a cependant des états sociaux dans lesquels le même homme est tour à tour tailleur et tisserand, où par conséquent ces deux espèces de travaux sont de simples modifications du travail d’un même individu, au lieu d’être des fonctions fixes d’individus différents, de même que l’habit que notre tailleur fait aujourd’hui et le pantalon qu’il fera demain ne sont que des variations de son travail individuel. On voit encore au premier coup d’œil que dans notre société capitaliste, suivant la direction variable de la demande du travail, une portion donnée de travail humain doit s’offrir tantôt sous la forme de confection de vêtements, tantôt sous celle de tissage. Quel que soit le frottement causé par ces mutations de forme du travail, elles s’exécutent quand même.\par
En fin de compte, toute activité productive, abstraction faite de son caractère utile, est une dépense de force humaine. La confection des vêtements et le tissage, malgré leur différence, sont tous deux une dépense productive du cerveau, des muscles, des nerfs, de la main de l’homme, et en ce sens du travail humain au même titre. La force, humaine de travail, dont le mouvement ne fait que changer de forme dans les diverses activités productives, doit assurément être plus ou moins développée pour pouvoir être dépensée sous telle ou telle forme. Mais la valeur des marchandises représente purement et simplement le travail de l’homme, une dépense de force humaine en général. Or, de même que dans la société civile un général ou un banquier joue un grand rôle, tandis que l’homme pur et simple fait triste figure, de même en est-il du travail humain. C’est une dépense de la force simple que tout homme ordinaire, sans développement spécial, possède dans l’organisme de son corps. Le travail simple moyen change, il est vrai, de caractère dans différents pays et suivant les époques ; mais il est toujours déterminé dans une société donnée. Le travail complexe \emph{(skilled labour}, travail qualifié) n’est qu’une puissance du travail simple, ou plutôt n’est que le travail simple multiplié, de sorte qu’une quantité donnée de travail complexe correspond à une quantité plus grande de travail simple. L’expérience montre que cette réduction se fait constamment. Lors même qu’une marchandise est le produit du travail le plus complexe, sa valeur la ramène, dans une proportion quelconque, au produit d’un travail simple, dont elle ne représente par conséquent qu’une quantité déterminée\footnote{Le lecteur doit remarquer qu’il ne s’agit pas ici du salaire ou de la valeur que l’ouvrier reçoit pour une journée de travail, mais de la \emph{valeur} de la marchandise dans laquelle se réalise cette journée de travail. Aussi bien la catégorie du salaire n’existe pas encore au point où nous en sommes de notre exposition.}. Les proportions diverses, suivant lesquelles différentes espèces de travail sont réduites au travail simple comme à leur unité de mesure, s’établissent dans la société à l’insu des producteurs et leur paraissent des conventions traditionnelles. Il s’ensuit que, dans l’analyse de la valeur, on doit traiter chaque variété de force de travail comme une force de travail simple.\par
De même donc que dans les valeurs toile et habit la différence de leurs valeurs d’usage est éliminée, de même, disparaît dans le travail que ces valeurs représentent la différence de ses formes utiles taille de vêtements et tissage. De même que les valeurs d’usage toile et habit sont des combinaisons d’activités productives spéciales avec le fil et le drap, tandis que les valeurs de ces choses sont de pures cristallisations d’un travail identique, de même, les travaux fixés dans ces valeurs n’ont plus de rapport productif avec le fil et le drap, mais expriment simplement une dépense de la même force humaine. Le tissage et la taille forment la toile et l’habit, précisément parce qu’ils ont des qualités différentes ; mais ils n’en forment les valeurs que par leur qualité commune de travail humain.\par
L’habit et la toile ne sont pas seulement des valeurs en général mais des valeurs d’une grandeur déterminée ; et, d’après notre supposition, l’habit vaut deux fois autant que 10 mètres de toile. D’où vient cette différence ? De ce que la toile contient moitié moins de travail que l’habit, de sorte que pour la production de ce dernier la force de travail doit être dépensée pendant le double du temps qu’exige la production de la première.\par
Si donc, quant à la valeur d’usage, le travail contenu dans la marchandise ne vaut que qualitativement, par rapport à la grandeur de la valeur, à ne compte que quantitativement. Là, il s’agit de savoir comment le travail se fait et ce qu’il produit ; ici, combien de temps il dure. Comme la grandeur de valeur d’une marchandise ne représente que le \emph{quantum} de travail contenu en elle, il s’ensuit que toutes les marchandises, dans une certaine proportion, doivent être des valeurs égales.\par
La force productive de tous les travaux utiles qu’exige la confection d’un habit reste-t-elle constante, la quantité de la valeur des habits augmente avec leur nombre. Si un habit représente \emph{x} journées de travail, deux habits représentent \emph{2x}, et ainsi de suite. Mais, admettons que la durée du travail nécessaire à la production d’un habit augmente ou diminue de moitié ; dans le premier cas un habit a autant de valeur qu’en avaient deux auparavant, dans le second deux habits n’ont pas plus de valeur que n’en avait précédemment un seul, bien que, dans les deux cas, l’habit rende après comme avant les mêmes services et que le travail utile dont il provient soit toujours de même qualité. Mais le \emph{quantum} de travail dépensé dans sa production n’est pas resté le même.\par
Une quantité plus considérable de valeurs d’usage forme évidemment une plus grande \emph{richesse matérielle} ; avec deux habits on peut habiller deux hommes, avec un habit on n’en peut habiller qu’un, seul, et ainsi de suite. Cependant, à une masse croissante de la richesse matérielle peut correspondre un décroissement simultané de sa valeur. Ce mouvement contradictoire provient du double caractère du travail. L’efficacité, dans un temps donné, d’un travail utile dépend de sa force productive. Le travail utile devient donc une source plus ou moins abondante de produits en raison directe de l’accroissement ou de la diminution de sa force productive. Par contre, une variation de cette dernière force n’atteint jamais directement le travail représenté dans la valeur. Comme la force productive appartient au travail concret et utile, elle ne saurait plus toucher le travail dès qu’on fait abstraction de sa forme utile. Quelles que soient les variations de sa force productive, le même travail, fonctionnant durant le même temps, se fixe toujours dans la même valeur. Mais il fournit dans un temps déterminé plus de valeurs d’usage, si sa force productive augmente, moins, si elle diminue. Tout changement dans la force productive, qui augmente la fécondité du travail et par conséquent la masse des valeurs d’usage livrées par lui, diminue la valeur de cette masse ainsi augmentée, s’il raccourcit le temps total de travail nécessaire à sa production, et il en est de même inversement.\par
Il résulte de ce qui précède que s’il n’y a pas, à proprement parler, deux sortes de travail dans la marchandise, cependant le même travail y est opposé à lui-même, suivant qu’on le rapporte à la valeur d’usage de la marchandise comme à son produit, ou à la valeur de cette marchandise comme à sa pure expression objective. Tout travail est d’un côté dépense, dans le sens physiologique, de force humaine, et, à ce titre de travail humain égal, il forme la valeur des marchandises. De l’autre côté, tout travail est dépense de la force humaine sous telle ou telle forme productive, déterminée par un but particulier, et à ce titre de travail concret et utile, il produit des valeurs d’usage ou utilités. De même que la marchandise doit avant tout être une utilité pour être une valeur, de même, le travail doit être avant tout utile, pour être censé dépense de force humaine, travail humain, dans le sens abstrait du mot\footnote{ \noindent Pour démontrer que « le travail… est la seule mesure réelle et définitive qui puisse servir dans tous les temps et dans tous les lieux à apprécier et à comparer la valeur de toutes les marchandises », \emph{A. Smith} dit : « Des quantités égales de travail doivent nécessairement, dans tous les temps et dans tous les lieux, être d’une valeur égale pour celui qui travaille. Dans son état habituel de santé, de force et d’activité, et d’après le degré ordinaire d’habileté ou de dextérité qu’il peut avoir, il faut toujours qu’il donne la même portion de son repos, de sa liberté, de son bonheur. » (\emph{Wealth of nations}, l. 1, ch. v.) D’un côté, \emph{A. Smith} confond ici (ce qu’il ne fait pas toujours) la détermination de la valeur de la marchandise par le \emph{quantum de travail} dépensé dans sa production, avec la détermination de sa \emph{valeur} par la \emph{valeur du travail}, et cherche, par conséquent, a prouver que d’égales quantités de travail ont toujours la même valeur. D’un autre côté, il pressent, il est vrai, que tout travail n’est qu’une \emph{dépense de force humaine de travail}, en tant qu’il se représente dans la valeur de la marchandise ; mais il comprend cette dépense exclusivement comme abnégation, comme sacrifice de repos, de liberté et de bonheur, et non, en même temps, comme affirmation normale de la vie. Il est vrai aussi qu’il a en vue le travailleur salarié moderne. Un des prédécesseurs de \emph{A. Smith}, cité déjà par nous, dit avec beaucoup plus de justesse : « Un homme s’est occupé pendant une semaine à fournir une chose nécessaire à la vie… et celui qui lui en donne une autre en échange ne peut pas mieux estimer ce qui en est l’équivalent qu’en calculant ce que lui a coûté exactement le même travail et le même temps. Ce n’est en effet que l’échange du travail d’un homme dans une chose durant un certain temps contre le travail d’un autre homme dans une autre chose durant le même temps. » (\emph{Some Thoughts on the interest of money in general}, etc., p. 39.) [Note à la deuxième édition]\par
 La langue anglaise a l’avantage d’avoir deux mots différents pour ces différents aspects du travail. Le travail qui crée des valeurs d’usage et qui est déterminé qualitativement s’appelle \emph{work}, par opposition à \emph{labour} ; le travail qui crée de la valeur et qui n’est mesuré que quantitativement s’appelle \emph{labour}, par opposition à \emph{work}. Voyez la note de la traduction anglaise, p. 14. (F. E.) [Note d’Engels à la quatrième édition]
}.\par
La substance de la valeur et la grandeur de valeur sont maintenant déterminées. Reste à analyser la forme de la valeur.
\subsection[{1.1.1.3. Forme de la valeur.}]{1.1.1.3. Forme de la valeur.}
\noindent Les marchandises viennent au monde sous la forme de valeurs d’usage ou de matières marchandes, telles que fer, toile, laine, etc. C’est là tout bonnement leur forme naturelle. Cependant, elles ne sont marchandises que parce qu’elles sont deux choses à la fois, objets d’utilité et porte-valeur. Elles ne peuvent donc entrer dans la circulation qu’autant qu’elles se présentent sous une double forme : leur forme de nature et leur forme de valeur\footnote{Les économistes peu nombreux qui ont cherché, comme Bailey, à faire l’analyse de la forme de la valeur, ne pouvaient arriver à aucun résultat : premièrement, parce qu’ils confondent toujours la valeur avec sa forme ; secondement, parce que sous l’influence grossière de la pratique bourgeoise, ils se préoccupent dès l’abord exclusivement de la quantité. « \emph{The command of quantity… constitutes value} [Le pouvoir de disposer de la quantité… constitue la valeur]. » (S. BAYLEY, \emph{Money and its vicissitudes}, London, 1837, p. 11.)}.\par
La réalité que possède la valeur de la marchandise diffère en ceci de l’amie de Falstaff, la veuve l’Eveillé, qu’on ne sait où la prendre. Par un contraste des plus criants avec la grossièreté du corps de la marchandise, il n’est pas un atome de matière qui pénètre dans sa valeur. On peut donc tourner et ret ourner à volonté une marchandise prise à part ; en tant qu’objet de valeur, elle reste insaisissable. Si l’on se souvient cependant que les valeurs des marchandises n’ont qu’une réalité purement sociale, qu’elles ne l’acquièrent qu’en tant qu’elles sont des expressions de la même unité sociale, du travail humain, il devient évident que cette réalité sociale ne peut se manifester aussi que dans les transactions sociales, dans les rapports des marchandises les unes avec les autres. En fait, nous sommes partis de la valeur d’échange ou du rapport d’échange des marchandises pour trouver les traces de leur valeur qui y est cachée. Il nous faut revenir maintenant à cette forme sous laquelle la valeur nous est d’abord apparue.\par
Chacun sait, lors même qu’il ne sait rien autre chose, que les marchandises possèdent une forme valeur particulière qui contraste de la manière la plus éclatante avec leurs formes naturelles diverses : la forme monnaie. Il s’agit maintenant de faire ce que l’économie bourgeoise n’a jamais essayé ; il s’agit de fournir la \emph{genèse} de la forme monnaie, c’est-à-dire de développer l’expression de la valeur contenue dans le rapport de valeur des marchandises depuis son ébauche la plus simple et la moins apparente jusqu’à cette forme monnaie qui saute aux yeux de tout le monde. En même temps, sera résolue et disparaîtra l’énigme de la monnaie.\par
En général, les marchandises n’ont pas d’autre rapport entre elles qu’un rapport de valeur, et le rapport de valeur le plus simple est évidemment celui d’une marchandise avec une autre marchandise d’espèce différente, n’importe laquelle. Le rapport de valeur ou d’échange de deux marchandises fournit donc pour une marchandise l’expression de valeur la plus simple.\par
\subsubsection[{1.1.1.3.1 Forme simple ou accidentelle de la valeur.}]{1.1.1.3.1 Forme simple ou accidentelle de la valeur.}
\noindent x\emph{ marchandise} A\emph{ =} y\emph{ marchandise} B\emph{, ou} x\emph{ marchandise} A\emph{ vaut} y\emph{ marchandise} B.(20 mètres de toile = 1 habit, ou 20 mètres de toile ont la valeur d’un habit.)\par
\paragraph[{1.1.1.3.1.1. Les deux pôles de l’expression de la valeur : sa forme relative et sa forme équivalent.}]{1.1.1.3.1\emph{.1}. Les deux pôles de l’expression de la valeur : sa forme relative et sa forme équivalent.}
\noindent Le mystère de toute forme de valeur gît dans cette forme simple. Aussi c’est dans son analyse, que se trouve la difficulté.\par
Deux marchandises différentes A et B, et, dans l’exemple que nous avons choisi, la toile et l’habit, jouent ici évidemment deux rôles distincts. La toile exprime sa valeur dans l’habit et celui-ci sert de matière à cette expression. La première marchandise joue un rôle actif, la seconde un rôle passif. La valeur de la première est exposée comme valeur relative, la seconde marchandise fonctionne comme \emph{équivalent}.\par
La forme relative et la forme équivalent sont deux aspects corrélatifs, inséparables, mais, en même temps, des \emph{extrêmes opposés, exclusifs l’un de l’autre}, c’est-à-dire des pôles de la même expression de la valeur. Ils se distribuent toujours entre les diverses marchandises que cette expression met en rapport. Cette équation : 20 \emph{mètres de toile} = 20 mètres de toile, exprime seulement que 20 mètres de toile ne sont pas autre chose que 20 mètres de toile, c’est-à-dire ne sont qu’une certaine somme d’une valeur d’usage. La valeur de la toile ne peut donc être exprimée que dans une autre marchandise, c’est-à-dire relativement. Cela suppose que cette autre marchandise se trouve en face d’elle sous forme d’équivalent. Dun autre côté, la marchandise qui figure comme \emph{équivalent} ne peut se trouver à la fois sous forme de valeur relative. Elle n’exprime pas sa valeur, mais fournit seulement la matière pour l’expression de la valeur de la première marchandise.\par
L’expression : 20 \emph{mètres de toile} = \emph{un habit}, ou : 20 \emph{mètres de toile valent un habit}, renferme, il est vrai, la réciproque : 1 \emph{habit} = 20 \emph{mètres de toile}, ou : 1 \emph{habit vaut} 20 \emph{mètres de toile}. Mais il me faut alors renverser l’équation pour exprimer relativement la valeur de l’habit, et dès que je le fais, la toile devient \emph{équivalent} à sa place. Une même marchandise ne peut donc revêtir simultanément ces deux formes dans la même expression de la valeur. Ces deux formes s’excluent polariquement.
\paragraph[{1.1.1.3.1.2. La forme relative de la valeur.}]{1.1.1.3.1.\textbf{\emph{2}}. La forme relative de la valeur.}
\noindent \textbf{1}. \emph{Contenu de cette forme}. – Pour trouver comment l’expression simple de la valeur d’une marchandise est contenue dans le rapport de valeur de deux marchandises, il faut d’abord l’examiner, abstraction faite de son côté \emph{quantitatif}. C’est le contraire qu’on fait en général en envisageant dans le rapport de valeur exclusivement la proportion dans laquelle des quantités déterminées de deux sortes de marchandises sont dites égales entre elles. On oublie que des choses différentes ne peuvent être comparées \emph{quantitativement} qu’après avoir été ramenées à la même unité. Alors seulement elles ont le même dénominateur et deviennent commensurables.\par
Que 20 mètres de toile = 1 habit, ou = 20, ou x habits, c’est-à-dire qu’une quantité donnée de toile vaille plus ou moins d’habits, une proportion de ce genre implique toujours que l’habit et la toile, comme grandeurs de valeur, sont des expressions de la même unité. Toile = habit, voilà le fondement de l’équation.\par
Mais les deux marchandises dont la qualité égale, l’essence identique, est ainsi affirmée, n’y jouent pas le même rôle. Ce n’est que la valeur de la toile qui s’y trouve exprimée : Et comment ? En la comparant à une marchandise d’une espèce différente, l’habit comme son équivalent, c’est-à-dire une chose qui peut la remplacer ou est échangeable avec elle. Il est d’abord évident que l’habit entre dans ce rapport exclusivement comme forme d’existence de la valeur, car ce n’est qu’en exprimant de la valeur qu’il peut figurer comme valeur vis-à-vis d’une autre marchandise. De l’autre côté, le propre valoir de la toile se montre ici ou acquiert une expression distincte. En effet, la valeur habit pourrait-elle être mise en équation avec la toile ou lui servir d’équivalent, si celle-ci n’était pas elle-même valeur ?\par
Empruntons une analogie à la chimie. L’acide butyrique et le formiate de propyle sont deux corps qui diffèrent d’apparence aussi bien que de qualités physiques et chimiques. Néanmoins, ils contiennent les mêmes éléments : carbone, hydrogène et oxygène. En outre, ils les contiennent dans la même proportion de C\textsubscript{4}H\textsubscript{8}O\textsubscript{2}. Maintenant, si l’on mettait le formiate de propyle en équation avec l’acide butyrique ou si l’on en faisait l’équivalent, le formiate de propyle ne figurerait dans ce rapport que comme forme d’existence de C\textsubscript{4}H\textsubscript{8}O\textsubscript{2}, c’est-à-dire de la substance qui lui est commune avec l’acide. Une équation où le formiate de propyle jouerait le rôle d’équivalent de l’acide butyrique serait donc une manière un peu gauche d’exprimer la substance de l’acide comme quelque chose de tout à fait distinct de se forme corporelle.\par
Si nous disons : en tant que valeurs toutes les marchandises ne sont que du travail humain cristallisé, nous les ramenons par notre analyse à l’abstraction valeur, mais, avant comme après, elles ne possèdent qu’une seule forme, leur forme naturelle d’objets utiles. Il en est tout autrement dès qu’une marchandise est mise en rapport de valeur avec une autre marchandise. Dès ce moment, son caractère de valeur ressort et s’affirme comme sa propriété inhérente qui détermine sa relation avec l’autre marchandise.\par
L’habit étant posé l’équivalent de la toile, le travail contenu dans l’habit est affirmé être identique avec le travail contenu dans la toile. Il est vrai que la taille se distingue du tissage. Mais son équation avec le tissage la ramène par le fait à ce qu’elle a de réellement commun avec lui, à son caractère de travail humain. C’est une manière détournée d’exprimer que le tissage, en tant qu’il tisse de la valeur, ne se distingue en rien de la taille des vêtements, c’est-à-dire est du travail humain abstrait. Cette équation exprime donc le caractère spécifique du travail qui constitue la valeur de la toile.\par
Il ne suffit pas cependant d’exprimer le caractère spécifique du travail qui fait la valeur de la toile. La force de travail de l’homme à l’état fluide, ou le travail humain, forme bien de la valeur, mais n’est pas valeur. Il ne devient valeur qu’à l’état coagulé, sous la forme d’un objet. Ainsi, les conditions qu’il faut remplir pour exprimer la valeur de la toile paraissent se contredire elles-mêmes. D’un côté, il faut la représenter comme une pure condensation du travail humain abstrait, car en tant que valeur la marchandise n’a pas d’autre réalité. En même temps, cette condensation doit revêtir la forme d’un objet visiblement distinct de la toile, elle-même, et qui tout en lui appartenant, lui soit commune avec une autre marchandise. Ce problème est déjà résolu.\par
En effet, nous avons vu que, dès qu’il est posé comme équivalent, l’habit n’a plus besoin de passeport pour constater son caractère de valeur. Dans ce rôle, sa propre forme d’existence devient une forme d’existence de la valeur ; cependant l’habit, le corps de la marchandise habit, n’est qu’une simple valeur d’usage ; un habit exprime aussi peu de valeur que le premier morceau de toile venu. Cela prouve tout simplement que, dans le rapport de valeur de la toile, il signifie plus qu’en dehors de ce rapport ; de même que maint personnage important dans un costume galonné devient tout à fait insignifiant si les galons lui manquent.\par
Dans la production de l’habit, de la force humaine a été dépensée en fait sous une forme particulière. Du travail humain est donc accumulé en lui. A ce point de vue, l’habit est porte-valeur, bien qu’il ne laisse pas percer cette qualité à travers la transparence de ses fils, si râpé qu’il soit. Et, dans le rapport de valeur de la toile, il ne signifie pas autre chose. Malgré son extérieur si bien boutonné, la toile a reconnu en lui une âme sœur pleine de valeur. C’est le côté platonique de l’affaire. En réalité, l’habit ne peut point représenter dans ses relations extérieures la valeur, sans que la valeur, prenne en même temps l’aspect d’un habit. C’est ainsi que le particulier A ne saurait représenter pour l’individu B une majesté, sans que la majesté aux yeux de B revête immédiatement et la figure et le corps de A ; c’est pour cela probablement qu’elle change, avec chaque nouveau père du peuple, de visage, de cheveux, et de mainte autre chose.\par
Le rapport qui fait de l’habit l’équivalent de la toile métamorphose donc la forme habit en forme valeur de la toile ou exprime la valeur de la toile dans la valeur d’usage de l’habit. En tant que valeur d’usage, la toile est un objet sensiblement différent de l’habit ; en tant que valeur, elle est chose égale à l’habit et en a l’aspect ; comme cela est clairement prouvé par l’équivalence de l’habit avec elle. Sa propriété de valoir apparaît dans son égalité avec l’habit, comme la nature moutonnière du chrétien dans sa ressemblance avec l’agneau de Dieu.\par
Comme on le voit, tout ce que l’analyse de la valeur nous avait révélé auparavant, la toile elle-même le dit, dès qu’elle entre en société avec une autre marchandise, l’habit. Seulement, elle ne trahit ses pensées que dans le langage qui lui est familier ; le langage des marchandises. Pour exprimer que sa valeur vient du travail humain, dans sa propriété abstraite, elle dit que l’habit en tant qu’il vaut autant qu’elle, c’est-à-dire est valeur, se compose du même travail qu’elle même. Pour exprimer que sa réalité sublime comme valeur est distincte de son corps raide et filamenteux, elle dit que la valeur a l’aspect d’un habit, et que par conséquent elle-même, comme chose valable, ressemble à l’habit, comme un œuf à un autre. Remarquons en passant que la langue des marchandises possède, outre l’hébreu, beaucoup d’autres dialectes et patois plus ou moins corrects. Le mot allemand \emph{Werstein}, par exemple, exprime moins nettement que le verbe roman \emph{valere, valer}, et le français \emph{valoir}, que l’affirmation de l’équivalence de la marchandise B avec la marchandise A est l’expression propre de la valeur de cette dernière. \emph{Paris vaut bien une messe}.\par
En vertu du rapport de valeur, la forme naturelle de la marchandise B devient la forme de valeur de la marchandise A, ou bien le corps de B devient pour A le miroir de sa valeur\footnote{Sous un certain rapport, il en est de l’homme comme de la marchandise. Comme il ne vient point au monde avec un miroir, ni en philosophe à la Fichte dont le Moi n’a besoin de rien pour s’affirmer, il se mire et se reconnaît d’abord seulement dans un autre homme. Aussi cet autre, avec peau et poil, lui semble-t-il la forme phénoménale du genre homme.}. La valeur de la marchandise A ainsi exprimée dans la valeur d’usage de la marchandise B acquiert la forme de valeur relative.\par
\textbf{2.} \emph{Détermination quantitative de la valeur relative.} – Toute marchandise dont la valeur doit être exprimée est un certain \emph{quantum} d’un chose utile, par exemple : 15 boisseaux de froment, 100 livres de café, etc., qui contient un \emph{quantum} déterminé de travail. La forme de la valeur a donc à exprimer non seulement de la valeur en général, mais une valeur d’une certaine grandeur. Dans le rapport de valeur de la marchandise A avec la marchandise B, non seulement la marchandise B est déclarée égale à A au point de vue de la qualité, mais encore un certain \emph{quantum} de B équivaut au \emph{quantum} donné de A.\par
L’équation : 20 mètres de toile = 1 habit, ou 20 mètres de toile \emph{valent} un habit, suppose que les deux marchandises coûtent autant de travail l’une que l’autre, ou se produisent dans le même temps ; mais ce temps varie pour chacune d’elles avec chaque variation de la force productive du travail qui la crée. Examinons maintenant l’influence de ces variations sur l’expression relative de la grandeur de valeur.\par
\textbf{I.} Que la valeur de la toile change pendant que la \emph{valeur de l’habit} reste constante\footnote{L’expression \emph{valeur} est employée ici, comme plusieurs fois déjà de temps à autre, pour \emph{quantité de valeur}.}. – Le temps de travail nécessaire à sa production double-t-il, par suite, je suppose, d’un moindre rendement du sol qui fournit le lin, alors sa valeur double. Au lieu de 20 \emph{mètres de toile} = 1 \emph{habit}, nous aurions : 20 \emph{mètres de toile} = 2 \emph{habits}, parce que 1 \emph{habit} contient maintenant moitié moins de travail. Le temps nécessaire à la production de la toile diminue-t-il au contraire de moitié par suite d’un perfectionnement apporté aux métiers à tisser sa valeur diminue dans la même proportion. Dès lors, 20 \emph{mètres de toile} = 1/2 \emph{habit}. La valeur relative de la marchandise A, c’est-à-dire sa valeur exprimée dans la marchandise B, hausse ou baisse, par conséquent, en raison directe de la valeur de la marchandise A si celle de la marchandise B reste constante.\par
\textbf{II.} Que la valeur de la toile reste constante pendant que la valeur de 1 habit varie. – Le temps nécessaire à la production de l’habit double-t-il dans ces circonstances, par suite, je suppose, d’une tonte de laine peu favorable, au lieu de 20 \emph{mètres de toile} = 1 \emph{habit}, nous avons maintenant 20 \emph{mètres de toile} = 1/2 \emph{habit}. La valeur de l’habit tombe-t-elle au contraire de moitié, alors 20 \emph{mètres de toile} = 2 \emph{habits}. La valeur de la marchandise A demeurant constante, on voit que sa valeur relative exprimée dans la marchandise B hausse ou baisse en raison inverse du changement de valeur de B.\par
Si l’on compare les cas divers compris dans I et II, il est manifeste que le même changement de grandeur de la valeur relative peut résulter de causes tout opposées. Ainsi l’équation : 20 \emph{mètres de toile} = 1 \emph{habit} devient : 20 \emph{mètres de toile} = 2 \emph{habits}, soit parce que la valeur de la toile double ou que la valeur des habits diminue de moitié, et 20 \emph{mètres de toile} = 1/2 \emph{habit}, soit parce que la valeur de la toile diminue de moitié ou que la valeur de l’habit devient double.\par
\textbf{III.} Les quantités de travail nécessaires à la production de la toile et de l’habit changent-elles simultanément, dans le même sens et dans la même proportion ? Dans ce cas, 20 \emph{mètres de toile} = 1 \emph{habit} comme auparavant, quels que soient leurs changements de valeur. On découvre ces changements par comparaison avec une troisième marchandise dont la valeur reste, la même. Si les valeurs de toutes les marchandises augmentaient ou diminuaient simultanément et dans la même proportion, leurs valeurs-relatives n’éprouveraient aucune variation. Leur changement réel de valeur se reconnaîtrait à ce que, dans un même temps de travail, il serait maintenant livré en général une quantité de marchandises plus ou moins grande qu’auparavant.\par
\textbf{IV.} Les temps de travail nécessaires à la production et de la toile et de l’habit, ainsi que leurs valeurs, peuvent simultanément changer dans le même sens, mais à un degré différent, ou dans un sens opposé, etc. L’influence de toute combinaison possible de ce genre sur la valeur relative d’une marchandise se calcule facilement par l’emploi des cas I, II et III.\par
Les changements réels dans la grandeur de la valeur ne se reflètent point comme on le voit, ni clairement ni complètement dans leur expression relative. La valeur relative d’une marchandise peut changer, bien que sa valeur reste constante, elle peut rester constante, bien que sa valeur change, et, enfin, des changements dans la quantité de valeur et dans son expression relative peuvent être simultanés sans correspondre exactement\footnote{Dans un écrit dirigé principalement contre la théorie de la valeur de Ricardo, on lit ; « Vous n’avez qu’à admettre que le travail nécessaire à sa production restant toujours le même, A baisse parce que B, avec lequel il s’échange, hausse, et votre principe général au sujet de la valeur tombe… En admettant que B baisse relativement à A, quand la valeur de A hausse relativement à B, Ricardo détruit lui-même la base de son grand axiome que la valeur d’une marchandise est toujours déterminée par la quantité de travail incorporée en elle ; car si un changement dans les frais de A change non seulement sa valeur relativement à B, avec lequel il s’échange, mais aussi la valeur de B relativement à A, quoique aucun changement n’ait eu lieu dans la quantité de travail exigé pour la production de B : alors tombent non seulement la doctrine qui fait de la quantité de travail appliquée à un article la mesure de sa valeur, mais aussi la doctrine qui affirme que la valeur est réglée par les frais de production. » (J. BROADHURST, \emph{Political Economy}. London, 1842, p. 11, 14.) Maître Broadhurst pouvait aussi bien dire : Que l’on considère les fractions 10/20, 10/50, 10/100, le nombre 10 reste toujours le même, et cependant sa valeur proportionnelle décroît constamment, parce que la grandeur des dénominateurs augmente. Ainsi tombe le grand principe d’après lequel la grandeur des nombres entiers est déterminée par la quantité des unités qu’ils contiennent. [Note à la deuxième édition]}.
\paragraph[{1.1.1.3.1.3. La forme équivalent et ses particularités.}]{1.1.1.3.1.\textbf{\emph{3}}. La forme équivalent et ses particularités.}
\noindent On l’a déjà vu : en même temps qu’une marchandise A (la toile), exprime, sa valeur dans la valeur d’usage d’une marchandise différente B (l’habit), elle imprime à cette dernière une forme particulière de valeur, celle d’équivalent. La toile manifeste son propre caractère de valeur par un rapport dans lequel une autre marchandise, l’habit, tel qu’il est dans sa forme naturelle, lui fait équation. Elle exprime donc qu’elle-même vaut quelque chose, par ce fait qu’une autre marchandise, l’habit, est immédiatement échangeable avec elle.\par
En tant que valeurs, toutes les marchandises sont des expressions égales d’une même unité, le travail humain, remplaçables les unes par les autres. Une marchandise est, par conséquent, échangeable avec une autre marchandise, dès qu’elle possède une forme, qui la fait apparaître comme valeur.\par
Une marchandise est immédiatement échangeable avec toute autre dont elle est l’équivalent, c’est-à-dire : la place qu’elle occupe dans le rapport de valeur fait de sa forme naturelle la forme valeur de l’autre marchandise. Elle n’a pas besoin de revêtir une forme différente de sa forme naturelle pour se manifester comme valeur à l’autre marchandise, pour valoir comme telle et, par conséquent, pour être échangeable avec elle. La forme équivalent est donc pour une marchandise la forme sous laquelle elle est immédiatement échangeable avec une autre.\par
Quand une marchandise, comme des habits, par exemple, sert d’équivalent à une autre marchandise, telle que la toile, et acquiert ainsi la propriété caractéristique d’être immédiatement échangeable avec celle-ci, la proportion n’est pas le moins du monde donnée dans laquelle cet échange peut s’effectuer. Comme la quantité de valeur de la toile est donnée, cela dépendra de la quantité de valeur des habits. Que dans le rapport de valeur, l’habit figure comme équivalent et la toile comme valeur relative, ou que ce soit l’inverse, la proportion, dans laquelle se fait l’échange, reste la même. La quantité de valeur respective des deux marchandises, mesurée par la durée comparative du travail nécessaire à leur production, est, par conséquent, une détermination tout à fait indépendante de la forme de valeur.\par
La marchandise dont la valeur se trouve sous la forme relative est toujours exprimée comme quantité de valeur, tandis qu’au contraire il n’en est jamais ainsi de l’\emph{équivalent} qui figure toujours dans l’équation comme simple quantité d’une chose utile. 40 mètres de toile, par exemple, \emph{valent –} quoi ? 2 habits. La marchandise habit jouant ici le rôle d’équivalent, donnant ainsi un corps à la valeur de la toile, il suffit d’un certain \emph{quantum} d’habits pour exprimer le \emph{quantum} de valeur qui appartient à la toile. Donc, 2 habits peuvent exprimer la quantité de valeur de 40 mètres de toile, mais non la leur propre. L’observation superficielle de ce fait, que, dans l’équation de la valeur, l’équivalent ne figure jamais que comme simple \emph{quantum} d’un objet d’utilité, a induit en erreur S. Bailey ainsi que beaucoup d’économistes avant et après lui. Ils n’ont vu dans l’expression de la valeur qu’un rapport de quantité. Or, sous la forme équivalent une marchandise figure comme simple quantité d’une matière quelconque précisément parce que la quantité de sa valeur n’est pas exprimée.\par
Les contradictions que renferme la forme équivalent exigent maintenant un examen plus approfondi de ses particularités.\par
\emph{Première particularité de la forme équivalent} : la valeur d’usage devient la forme de manifestation de son contraire, la valeur.\par
La forme naturelle des marchandises devient leur forme de valeur. Mais, en fait, ce \emph{quid pro quo} n’a lieu pour une marchandise B (habit, froment, fer, etc.) que dans les limites du rapport de valeur, dans lequel une autre marchandise, A (toile, etc.) entre avec elle, et seulement dans ces limites. Considéré isolément, l’habit, par exemple, n’est qu’un objet d’utilité, une valeur d’usage, absolument comme la toile ; sa forme n’est que la forme naturelle d’un genre particulier de marchandise. Mais comme aucune marchandise ne peut se rapporter à elle-même comme équivalent, ni faire de sa forme naturelle la forme de sa propre valeur, elle doit nécessairement prendre pour équivalent une autre marchandise dont la valeur d’usage lui sert ainsi de forme valeur.\par
Une mesure appliquée aux marchandises en tant que matières, c’est-à-dire en tant que valeurs d’usage, va nous servir d’exemple pour mettre ce qui précède directement sous : les yeux du lecteur. Un pain de sucre, puisqu’il est un corps, est pesant et, par conséquent, a du poids ; mais il est impossible de voir ou de sentir ce poids rien qu’à l’apparence. Nous prenons maintenant divers morceaux de fer de poids connu. La forme matérielle du fer, considérée en elle-même, est aussi peu une forme de manifestation de la pesanteur que celle du pain de sucre. Cependant, pour exprimer que ce dernier est pesant, nous le plaçons en un rapport de poids avec le fer. Dans ce rapport, le fer est considéré comme un corps qui ne représente rien que de la pesanteur. Des quantités de fer employées pour mesurer le poids du sucre représentent donc vis-à-vis de la matière sucre une simple forme, la forme sous laquelle la pesanteur se manifeste. Le fer ne peut jouer ce rôle qu’autant que le sucre ou n’importe quel autre corps, dont le poids doit être trouvé, est mis en rapport avec lui à ce point de vue. Si les deux objets n’étaient pas pesants, aucun rapport de cette espèce ne serait possible entre eux, et l’un ne pourrait point servir d’expression à la pesanteur de l’autre. Jetons-les tous deux dans la balance et nous voyons en fait qu’ils sont la même chose comme pesanteur, et que, par conséquent, dans une certaine proportion ils sont aussi du même poids. De même que le corps fer, comme mesure de poids, vis-à-vis du pain de sucre ne représente que pesanteur, de même, dans notre expression de valeur, le corps habit vis-à-vis de la toile ne représente que valeur.\par
Ici cependant cesse l’analogie. Dans l’expression de poids du pain de sucre, le fer représente une qualité naturelle commune aux deux corps, leur pesanteur, tandis que dans l’expression de valeur de la toile, le corps habit représente une qualité surnaturelle des deux objets, leur valeur, un caractère d’empreinte purement sociale.\par
Du moment que la forme relative exprime la valeur d’une marchandise de la toile, par exemple, comme quelque chose de complètement différent de son corps lui-même et de ses propriétés, comme quelque chose qui ressemble, à un habit, par exemple, elle fait entendre que sous cette expression un rapport social est caché.\par
C’est l’inverse qui a lieu avec la forme équivalent. Elle consiste précisément en ce que le corps d’une marchandise, un habit, par exemple, en ce que cette chose, telle quelle, exprime de la valeur, et, par conséquent possède naturellement forme de valeur. Il est vrai que cela n’est juste qu’autant qu’une autre marchandise, comme la toile, se rapporte à elle comme équivalent\footnote{Dans un autre ordre d’idées il en est encore ainsi. Cet homme, par exemple, n’est roi que parce que d’autres hommes se considèrent comme ses sujets et agissent en conséquence. Ils croient au contraire être sujets parce qu’il est roi.}. Mais, de même que les propriétés matérielles d’une chose ne font que se confirmer dans ses rapports extérieurs avec d’autres choses au lieu d’en découler, de même, l’habit semble tirer de la nature et non du rapport de valeur de la toile sa forme équivalent, sa propriété d’être immédiatement échangeable, au même titre que sa propriété d’être pesant ou de tenir chaud. De là, le côté énigmatique de l’équivalent, côté qui ne frappe les yeux de l’économiste bourgeois que lorsque cette forme se montre à lui tout achevée, dans la monnaie. Pour dissiper ce caractère mystique de l’argent et de l’or, il cherche ensuite à les remplacer sournoisement par des marchandises moins éblouissantes ; il fait et refait avec un plaisir toujours nouveau le catalogue de tous les articles qui, dans leur temps, ont joué le rôle d’équivalent. Il ne pressent pas que l’expression la plus simple de la valeur, telle que 20 mètres de toile valent un habit, contient déjà l’énigme et que c’est sous cette forme simple qu’il doit chercher à la résoudre.\par
\emph{Deuxième particularité de la forme équivalent} : le travail concret devient la forme de manifestation de son contraire, le travail humain abstrait.\par
Dans l’expression de la valeur d’une marchandise, le corps de l’équivalent figure toujours comme matérialisation du travail humain abstrait, et est toujours le produit d’un travail particulier, concret et utile. Ce travail concret ne sert donc ici qu’à exprimer du travail abstrait. Un habit, par exemple, est-il une simple réalisation, l’activité du tailleur qui se réalise en lui n’est aussi qu’une simple forme de réalisation du travail abstrait. Quand on exprime la valeur de la toile dans l’habit, l’utilité du travail du tailleur ne consiste pas en ce qu’il fait des habits et, selon le proverbe allemand, des hommes, mais en ce qu’il produit un corps, transparent de valeur, échantillon d’un travail qui ne se distingue en rien du travail réalisé dans la valeur de la toile. Pour pouvoir s’incorporer dans un tel miroir de valeur, il faut que le travail du tailleur ne reflète lui-même rien que sa propriété de travail humain.\par
Les deux formes d’activité productive, tissage et confection de vêtements, exigent une dépense de force humaine. Toutes deux possèdent donc la propriété commune d’être du travail humain, et dans certains cas, comme par exemple, lorsqu’il s’agit de la production de valeur, on ne doit les considérer qu’à ce point de vue. Il n’y a là rien de mystérieux ; mais dans l’expression de valeur de la marchandise, la chose est prise au rebours. Pour exprimer, par exemple, que le tissage, non comme tel, mais, en sa qualité de travail, humain en général, forme la valeur de la toile, on lui oppose un autre travail, celui qui produit l’habit, l’équivalent de la toile, comme la forme expresse dans laquelle le travail humain se manifeste. Le travail du tailleur est ainsi métamorphosé en simple expression de sa propre qualité abstraite.\par
\emph{Troisième particularité de la forme équivalent} : le travail concret qui produit l’équivalent, dans notre exemple, celui du tailleur, en servant simplement d’expression au travail humain indistinct, possède la forme de l’égalité avec un autre travail, celui que recèle la toile, et devient ainsi, quoique travail privé, comme tout autre travail productif de marchandises, travail sous forme sociale immédiate. C est pourquoi il se réalise par un produit qui est immédiatement échangeable avec une autre marchandise.\par
Les deux particularités de la forme équivalent, examinées en dernier lieu, deviennent encore plus faciles à saisir, si nous remontons au grand penseur qui a analysé le premier la forme valeur, ainsi que tant d’autres formes, soit de la pensée, soit de la société, soit de la nature : nous avons nommé Aristote.\par
D’abord Aristote exprime clairement que la forme argent de la marchandise n’est que l’aspect développé de la forme valeur simple, c’est-à-dire de l’expression de la valeur d’une marchandise dans une autre marchandise quelconque, car il dit :\par
« 5 \emph{lits} = 1 \emph{maison} » («  ») « ne diffère pas » de :\par
« 5 \emph{lits} = \emph{tant et tant d’argent} » («  »).\par
Il voit de plus que le rapport de valeur qui confient cette expression de valeur suppose, de son côté, que la maison est déclarée égale au lit au point de vue de la qualité, et que ces objets, sensiblement différents, ne pourraient se comparer entre eux comme des grandeurs commensurables sans cette égalité d’essence. « L’échange, dit-il, ne peut avoir lieu sans l’égalité, ni l’égalité sans la commensurabilité » (“”). Mais ici il hésite et renonce à l’analyse de la forme valeur. « Il est, ajoute-t-il, impossible en vérité (“”) que des choses si dissemblables soient commensurables entre elles », c’est-à-dire de qualité égale. L’affirmation de leur égalité ne peut être que contraire à la nature des choses ; « on y a seulement recours pour le besoin pratique ».\par
Ainsi, Aristote nous dit lui-même où son analyse vient échouer, – contre l’insuffisance de son concept de valeur. Quel est le « je ne sais quoi » d’égal, c’est-à-dire la substance commune que représente la maison pour le lit dans l’expression de la valeur de ce dernier ? « Pareille chose, dit Aristote, ne peut en vérité exister. » Pourquoi ? La maison représente vis-à-vis du lit quelque chose d’égal, en tant qu’elle représente ce qu’il y a de réellement égal dans tous les deux. Quoi donc ? Le travail humain.\par
Ce qui empêchait Aristote de lire dans la forme valeur des marchandises, que tous les travaux sont exprimés ici comme travail humain indistinct et par conséquent égaux, c’est que là société grecque reposait sur le travail des esclaves et avait pour base naturelle l’inégalité des hommes et de leurs forces de travail. Le secret de l’expression de la valeur, l’égalité et l’équivalence de tous les travaux, parce que et en tant qu’ils sont du travail humain, ne peut être déchiffré que lorsque l’idée de l’égalité humaine a déjà acquis la ténacité d’un préjugé populaire. Mais cela n’a lieu que dans une société où la forme marchandise est devenue la forme générale des produits du travail, où, par conséquent, le rapport des hommes entre eux comme producteurs et échangistes de marchandises est le rapport social dominant. Ce qui montre le génie d’Aristote c’est qu’il a découvert dans l’expression de la valeur des marchandises un rapport d’égalité. L’état particulier de la société dans laquelle il vivait l’a seul empêché de trouver quel était le contenu réel de ce rapport.
\paragraph[{1.1.1.3.1.4. Ensemble de la forme valeur simple.}]{1.1.1.3.1\emph{.4}. Ensemble de la forme valeur simple.}
\noindent La forme simple de la valeur d’une marchandise est contenue dans son rapport valeur ou d’échange avec un seul autre genre de marchandise quel qu’il soit. La valeur de la marchandise A est exprimée qualitativement par la propriété de la marchandise B d’être immédiatement échangeable avec A. Elle est exprimée quantitativement par l’échange toujours possible d’un \emph{quantum} déterminé de B contre le \emph{quantum} donné de A. En d’autres termes, la valeur d’une marchandise est exprimée par cela seul qu’elle se pose comme valeur d’échange.\par
Si donc, au début de ce chapitre, pour suivre la manière de parler ordinaire, nous avons dit : la marchandise est valeur d’usage et valeur d’échange, pris à la lettre, c’était faux. La marchandise est valeur d’usage ou objet d’utilité, et valeur. Elle se présente pour ce qu’elle est, chose double, dès que sa valeur possède une forme phénoménale propre, distincte de sa forme naturelle, celle de valeur d’échange ; et elle ne possède jamais cette forme, si on la considère isolément. Dès qu’on sait cela, la vieille locution n’a plus de malice et sert pour l’abréviation.\par
Il ressort de notre analyse que c’est de la nature de la valeur des marchandises que provient sa forme, et que ce n’est pas au contraire de la manière de les exprimer par un rapport d’échange que découlent la valeur et sa grandeur. C’est là pourtant l’erreur des mercantilistes et de leurs modernes zélateurs, les Ferrier, les Ganilh, etc.\footnote{F. L. A. FERRIER (sous-inspecteur des douanes), \emph{Du gouvernement considéré dans ses rapports avec le commerce}, Paris, 1805 ; et Charles GANILH, \emph{Des systèmes d’économie politique}, 2° édit., Paris, 1821. [Note à la deuxième édition]}, aussi bien que de leurs antipodes, les commis voyageurs du libre-échange, tels que Bastiat et consorts. Les mercantilistes appuient surtout sur le côté qualitatif de l’expression de la valeur, conséquemment sur la forme équivalent de la marchandise, réalisée à l’œil, dans la forme argent ; les modernes champions du libre-échange, au contraire, qui veulent se débarrasser à tout prix de leur marchandise, font ressortir exclusivement le côté quantitatif de la forme relative de la valeur. Pour eux, il n’existe donc ni valeur ni grandeur de valeur en dehors de leur expression par le rapport d’échange, ce qui veut dire pratiquement en dehors de la cote quotidienne du prix courant. L’Ecossais Mac Leod, qui s’est donné pour fonction d’habiller et d’orner d’un si grand luxe d’érudition le fouillis des préjugés économiques de Lombardstreet, – la rue des grands banquiers de Londres, – forme la synthèse réussie des mercantilistes superstitieux et des esprits forts du libre-échange.\par
Un examen attentif de l’expression de la valeur de A en B a montré que dans ce rapport la forme naturelle de la marchandise A ne figure que comme forme de valeur d’usage, et la forme naturelle de la marchandise B que comme forme de valeur. L’opposition intime entre la valeur d’usage et la valeur d’une marchandise se montre ainsi par le rapport de deux marchandises, rapport dans lequel A, dont la valeur doit être exprimée, ne se pose immédiatement que comme valeur d’usage, tandis que B, au contraire, dans laquelle la valeur est exprimée, ne se pose immédiatement que comme valeur d’échange. La forme valeur simple d’une marchandise est donc la simple forme d’apparition des contrastes qu’elle recèle, c’est-à-dire de la valeur d’usage et de la valeur.\par
Le produit du travail est, dans n’importe quel état social, valeur d’usage ou objet d’utilité ; mais il n’y a qu’une époque déterminée dans le développement historique de la société, qui transforme généralement le produit du travail en marchandise, c’est celle où le travail dépensé dans la production des objets utiles revêt le caractère d’une qualité inhérente à ces choses, de leur valeur.\par
Le produit du travail acquiert la forme marchandise, dès que sa valeur acquiert la forme de la valeur d’échange, opposée à sa forme naturelle ; dès que, par conséquent, il est représenté comme l’unité dans laquelle se fondent ces contrastes. Il suit de là que la forme simple que revêt la valeur de la marchandise est aussi la forme primitive dans laquelle le produit du travail se présente comme marchandise et que le développement de la forme marchandise marche du même pas que celui de la forme valeur.\par
A première vue on s’aperçoit de l’insuffisance de la forme valeur simple, ce germe qui, doit subir une série de métamorphoses avant d’arriver à la forme prix.\par
En effet la forme simple ne fait que distinguer entre la valeur et la valeur d’usage d’une marchandise et la mettre en rapport d’échange avec une seule espèce de n’importe quelle autre marchandise, au lieu de représenter son égalité qualitative et sa proportionnalité quantitative avec toutes les marchandises. Dès que la valeur d’une marchandise est exprimée dans cette forme simple, une autre marchandise revêt de son côté la forme d’équivalent simple. Ainsi, par exemple, dans l’expression de la valeur relative de la toile l’habit ne possède la forme équivalent, forme qui indique qu’il est immédiatement échangeable, que par rapport à une seule marchandise, la toile.\par
Néanmoins, la forme valeur simple passe d’elle-même à une forme plus complète. Elle n’exprime, il est vrai, la valeur d’une marchandise A que, dans un seul autre genre de marchandise. Mais le genre de cette seconde marchandise peut être absolument tout ce qu’on voudra, habit, fer, froment, et ainsi de suite. Les expressions de la valeur d’une marchandise deviennent donc aussi variées que ses rapports de valeur avec d’autres marchandises\footnote{Par exemple chez Homère, la valeur d’une chose est exprimée en une série de choses différentes. [note à la 2° édition]}. L’expression isolée de sa valeur se métamorphose ainsi en une série d’expressions simples que l’on peut prolonger à volonté.
\subsubsection[{1.1.1.3.2. Forme valeur totale ou développée.}]{1.1.1.3.2. Forme valeur totale ou développée.}
\noindent z marchandise A = u marchandise B, ou = v marchandise C, ou = x marchandise E, ou =, etc.\par
20 mètres de toile = 1 habit, ou = 10 livres de thé, ou = 40 livres de café, ou = 2 onces d’or, ou = 1/2 tonne de fer, ou =, etc.\par
\paragraph[{1.1.1.3.2.1. La forme développée de la valeur relative.}]{1.1.1.3.2.1. La forme développée de la valeur relative.}
\noindent La valeur d’une marchandise, de la toile, par exemple, est maintenant représentée dans d’autres éléments innombrables. Elle se reflète dans tout autre corps de marchandise comme en un miroir\footnote{Voilà pourquoi l’on parle de la valeur habit de la toile quand on exprime sa valeur en habits, de sa valeur blé, quand on l’exprime en blé, etc. Chaque expression semblable donne à entendre que c’est sa propre valeur qui se manifeste dans ces diverses valeurs d’usage. \\
« La valeur d’une marchandise dénote son rapport d’échange [avec une autre marchandise quelconque] nous pouvons donc parler [de cette valeur comme] de sa valeur blé, sa valeur habit, par rapport à la marchandise à laquelle elle est comparée ; et alors il y a des milliers d’espèces de valeur, autant d’espèces de valeur qu’il y a de genres de marchandises, et toutes sont également réelles et également nominales. » (\emph{A Critical Dissertation on the Nature, Measure and Causes of Value : chiefly in reference to the writings of Mr. Ricardo and his followers. By the author of Essays on the Formation, etc., of Opinions}, London, 1825, p. 39.) S. Bailey, l’auteur de cet écrit anonyme qui fit dans son temps beaucoup de bruit en Angleterre, se figure avoir anéanti tout concept positif de valeur par cette énumération des expressions relatives variées de la valeur d’une même marchandise. Quelle que fût l’étroitesse de son esprit, il n’en a pas moins parfois mis à nu les défauts de la théorie de Ricardo. Ce qui le prouve, c’est l’animosité avec laquelle il a été attaqué par l’école ricardienne, par exemple dans la \emph{Westminster Review}.}.\par
Tout autre travail, quelle qu’en soit la forme naturelle, taille, ensemençage, extraction, de fer ou d’or, etc., est maintenant affirmé égal au travail fixé dans la valeur de la toile, qui manifeste ainsi son caractère de travail humain. La forme totale de la valeur relative met une marchandise en rapport social avec toutes. En même temps, la série interminable de ses expressions démontre que la valeur des marchandises revêt indifféremment toute forme particulière de valeur d’usage.\par
Dans la première forme : 20 \emph{mètres de toile} = 1 \emph{habit}, il peut sembler que ce soit par hasard que ces deux marchandises sont échangeables dans cette proportion déterminée.\par
Dans la seconde forme, au contraire, on aperçoit immédiatement ce que cache cette apparence. La valeur de la toile reste la même, qu’on l’exprime en vêtement en café, en fer, au moyen de marchandises sans nombre, appartenant à des échangistes les plus divers. Il devient évident que ce n’est pas l’échange qui règle la quantité de valeur d’une marchandise, mais, au contraire, la quantité de valeur de la marchandise qui règle ses rapports d’échange.
\paragraph[{1.1.1.3.2.2. La forme équivalent particulière.}]{1.1.1.3.2.\emph{2}. La forme équivalent particulière.}
\noindent Chaque marchandise, habit, froment, thé, fer, etc., sert d’équivalent dans l’expression de la valeur de la toile. La forme naturelle de chacune de ces marchandises est maintenant une forme équivalent particulière à côté de beaucoup d’autres. De même, les genres variés de travaux utiles, contenus dans les divers corps de marchandises, représentent autant de formes particulières de réalisation ou de manifestation du travail humain pur et simple.
\paragraph[{1.1.1.3.2.3. Défauts de la forme valeur totale, ou développée.}]{1.1.1.3.2.\emph{3}. Défauts de la forme valeur totale, ou développée.}
\noindent D’abord, l’expression relative de valeur est inachevée parce que la série de ses termes, n’est jamais close. La chaîne dont chaque comparaison de valeur forme un des anneaux peut s’allonger à volonté à mesure qu’une nouvelle espèce de marchandise fournit la matière d’une expression nouvelle. Si, de plus, comme cela doit se faire, on généralise cette forme en. l’appliquant à tout genre de marchandise, on obtiendra, au bout du compte, autant de séries diverses et interminables d’expressions de valeur qu’il y aura de marchandises. – Les défauts de la forme développée de la valeur relative se reflètent dans la forme équivalent qui lui correspond. Comme la forme naturelle de chaque espèce de marchandises fournit ici une forme équivalent particulière à côté d’autres en nombre infini, il n’existe en général que des formes équivalent fragmentaires dont chacune exclut l’autre. De même, le genre de travail utile, concret, contenu dans chaque équivalent, n’y présente qu’une forme particulière, c’est-à-dire une manifestation incomplète du travail humain. Ce travail possède bien, il est vrai, sa forme complète ou totale de manifestation dans l’ensemble de ses formes particulières. Mais l’unité de forme et d’expression fait défaut.\par
La forme totale ou développée de la valeur relative ne consiste cependant qu’en une somme d’expressions relatives simples ou d’équations de la première forme telles que :\par
20\emph{ mètres de toile =} 1\emph{ habit},\par
20\emph{ mètres de toile =} 10\emph{ livres de thé}, etc.,\par
dont chacune contient réciproquement l’équation identique : 1\emph{ habit =} 20\emph{ mètres de toile}, \\
10\emph{ livres de thé =} 20\emph{ mètres de toile}, etc.\par
En fait : le possesseur de la toile l’échange-t-il contre beaucoup d’autres marchandises et exprime-t-il conséquemment sa valeur dans une série d’autant de termes, les possesseurs des autres marchandises doivent les échanger contre la toile et exprimer les valeurs de leurs marchandises diverses dans un seul et même terme, la toile. – Si donc nous retournons la série : 20 mètres de toile = 1 habit, ou = 10 livres de thé, ou =, etc., c’est-à-dire si nous exprimons la réciproque qui y est déjà implicitement contenue, nous obtenons :
\subsubsection[{1.1.1.3.3. Forme valeur générale.}]{1.1.1.3.3. Forme valeur générale.}

\begin{itemize}[itemsep=0pt,]
\item 1 habit =
\item 10 livres de thé =
\item 40 livres de café =
\item 2 onces d’or =
\item ½ tonne de fer =
\item X marchandise A =
\item Etc. = 20 mètres de toile
\end{itemize}

\paragraph[{1.1.1.3.3.1. Changement de caractère de la forme valeur.}]{1.1.1.3.3.1. Changement de caractère de la forme valeur.}
\noindent Les marchandises expriment maintenant leurs valeurs : 1° d’une manière simple, parce qu’elles l’expriment dans une seule espèce de marchandise ; 2° avec ensemble, parce qu’elles l’expriment dans la même espèce de marchandises. Leur forme valeur est simple et commune, conséquemment générale.\par
Les formes I et II ne parvenaient à exprimer la valeur d’une marchandise que comme quelque chose de distinct de sa propre valeur d’usage ou de sa propre matière. La première forme fournit des équations telles que celle-ci : 1 \emph{habit} = 20 \emph{mètres de toile} ; 10 \emph{livres de thé} = 1/2 \emph{tonne de fer}, etc. La valeur de l’habit est exprimée comme quelque, chose d’égal à la toile, la valeur du thé comme quelque chose d’égal au fer, etc. ; mais ces expressions de la valeur de l’habit et du, thé sont aussi différentes l’une de l’autre que la toile et le fer. Cette forme ne se présente évidemment dans la pratique qu’aux époques primitives où les produits du travail n’étaient transformés en marchandises que par des échanges accidentels et isolés.\par
La seconde forme exprime plus complètement que la première la différence qui existe entre la valeur d’une marchandise, par exemple, d’un habit, et sa propre valeur d’usage. En effet, la valeur de l’habit y prend toutes les figures possibles vis-à-vis de sa forme naturelle ; elle ressemble à la toile, au thé, au fer, à tout, excepté à l’habit. D’un autre côté, cette forme rend impossible toute expression commune de la valeur des marchandises, car, dans l’expression de valeur d’une marchandise quelconque, toutes les autres figurent comme ses équivalents, et sont, par conséquent, incapables d’exprimer leur propre valeur. Cette forme valeur développée se présente dans la réalité dès qu’un produit du travail, le bétail, par exemple, est échangé contre d’autres marchandises différentes, non plus par exception, mais déjà par habitude.\par
Dans l’expression générale de la valeur relative, au contraire, chaque marchandise, telle qu’habit, café, fer, etc., possède une seule et même forme valeur, par exemple, la forme toile, différente de sa forme naturelle. En vertu de cette ressemblance avec la toile, la valeur de chaque marchandise est maintenant distincte non seulement de sa propre valeur d’usage, mais encore de toutes les autres valeurs d’usage, et, par cela même, représentée comme le caractère commun et indistinct de toutes les marchandises. Cette forme est la première qui mette les marchandises en rapport entre elles comme valeurs, en les faisant apparaître l’une vis-à-vis de l’autre comme valeurs d’échange.\par
Les deux premières formes expriment la valeur d’une marchandise quelconque, soit en une autre marchandise différente, soit en une série de beaucoup d’autres marchandises. Chaque fois c’est, pour ainsi dire, l’affaire particulière de chaque marchandise prise à part de se donner une forme valeur, et elle y parvient sans que les autres marchandises s’en mêlent. Celles-ci jouent vis-à-vis d’elle le rôle purement passif d’équivalent. La forme générale de la valeur relative ne se produit au contraire que comme l’œuvre commune des marchandises dans leur ensemble. Une marchandise n’acquiert son expression de valeur générale que parce que, en même temps, toutes les autres marchandises expriment leurs valeurs dans le même équivalent, et chaque espèce de marchandise nouvelle qui se présente doit faire de même. De plus, il devient évident que les marchandises qui, au point de vue de la valeur, sont des choses purement sociales, ne peuvent aussi exprimer cette existence sociale que par une série embrassant tous leurs rapports réciproques ; que leur forme valeur doit, par conséquent, être une forme socialement validée.\par
La forme naturelle de la marchandise qui devient l’équivalent commun, la toile, est maintenant la forme officielle des valeurs. C’est ainsi que les marchandises se montrent les unes aux autres non seulement leur égalité qualitative, mais encore leurs différences quantitatives de valeur. Les quantités de valeur projetées comme sur un même miroir, la toile, se reflètent réciproquement.\par
Exemple : 10 livres de thé = 20 mètres de toile, et 40 livres de café = 20 mètres de toile. Donc 10 livres de thé = 40 livres de café, ou bien il n’y a dans 1 livre de café que 1/4 du travail contenu dans 1 livre de thé.\par
La forme générale de la valeur relative embrassant le monde des marchandises imprime à la marchandise équivalent qui en est exclue le caractère d’équivalent général. La toile est maintenant immédiatement échangeable avec toutes les autres marchandises. Sa forme naturelle est donc en même temps sa forme sociale. Le tissage, le travail privé qui produit la toile, acquiert par cela même le caractère de travail social, la forme d’égalité avec tous les autres travaux. Les innombrables équations dont se compose la forme générale de la valeur identifient le travail réalisé dans la toile avec le travail contenu dans chaque marchandise qui lui est tour à tour comparée, et fait du tissage la forme générale dans laquelle se manifeste le travail humain. De cette manière, le travail réalisé dans la valeur des marchandises n’est pas seulement représenté négativement, c’est-à-dire comme une abstraction où s’évanouissent les formes concrètes et les propriétés utiles du travail réel ; sa nature positive s’affirme nettement. Elle est la réduction de tous les travaux réels à leur caractère commun de travail humain, de dépense de la même force humaine de travail.\par
La forme générale de la valeur montre, par sa structure même, qu’elle est l’expression sociale du monde des marchandises. Elle révèle, par conséquent, que dans ce monde le caractère humain ou général du travail forme son caractère social spécifique.
\paragraph[{1.1.1.3.3.2. Rapport de développement de la forme valeur relative et de la forme équivalent.}]{1.1.1.3.3.\emph{2}. Rapport de développement de la forme valeur relative et de la forme équivalent.}
\noindent La forme équivalent se développe simultanément et graduellement avec la forme relative ; mais, et c’est là ce qu’il faut bien remarquer, le développement de la première n’est que le résultat et l’expression du développement de la seconde. C’est de celle-ci que part l’initiative.\par
La forme valeur relative simple ou isolée d’une marchandise suppose une autre marchandise quelconque comme équivalent accidentel. La forme développée de la valeur relative, cette expression de la valeur d’une marchandise dans toutes les autres, leur imprime à toutes, la forme d’équivalents particuliers d’espèce différente. Enfin, une marchandise spécifique acquiert la forme d’équivalent général, parce que toutes les autres marchandises en font la matière de leur forme générale de valeur relative.\par
A mesure cependant que la forme valeur en général se développe, se développe aussi l’opposition entre ses deux pôles, valeur relative et équivalent. De même la première forme valeur, 20 \emph{mètres de toile} = 1\emph{ habit}, contient cette opposition, mais ne la fixe pas. Dans cette équation, l’un des termes, la toile, se trouve sous la forme valeur relative, et le terme opposé, l’habit, sous la forme équivalent. Si maintenant on lit à rebours cette équation, la toile et l’habit changent tout simplement de rôle, mais la forme de l’équation reste la même. Aussi est-il difficile de fixer ici l’opposition entre les deux termes.\par
Sous la forme II, une espèce de marchandise peut développer complètement sa valeur relative, revêt la forme totale de la valeur relative, parce que, et en tant que toutes les autres marchandises se trouvent vis-à-vis d’elle sous la forme équivalent.\par
Ici l’on ne peut déjà plus renverser les deux termes de l’équation sans changer complètement son caractère, et la faire passer de la forme valeur totale à la forme valeur générale.\par
Enfin, la dernière forme, la forme III, donne à l’ensemble des marchandises une expression de valeur relative générale et uniforme, parce que et en tant qu’elle exclut de la forme équivalent toutes les marchandises, à l’exception d’une seule. Une marchandise, la toile, se trouve conséquemment sous forme d’échangeabilité immédiate avec toutes les autres marchandises, parce que et en tant que celles-ci ne s’y trouvent pas\footnote{ \noindent La forme d’échangeabilité immédiate et universelle n’indique pas le moins du monde au premier coup d’œil qu’elle est une forme polarisée, renfermant en elle des oppositions, et tout aussi inséparable de la forme contraire sous laquelle l’échange immédiat n’est pas possible, que le rôle positif d’un des pôles d’un aimant l’est du rôle négatif de l’autre pôle. On peut donc s’imaginer qu’on a la faculté de rendre toutes les marchandises immédiatement échangeables, comme on peut se figurer que tous les catholiques peuvent être faits papes en même temps. Mais, en réalité, la forme valeur relative générale et la forme équivalent général sont les deux pôles opposés, se supposant et se repoussant réciproquement, du même rapport social des marchandises.\par
 Cette impossibilité d’échange immédiat entre les marchandises est un des principaux inconvénients attachés à la forme actuelle de la production dans laquelle cependant l’économiste bourgeois voit le \emph{nec plus ultra} de la liberté humaine et de l’indépendance individuelle. Bien des efforts inutiles, utopiques, ont été tentés pour vaincre cet obstacle. J’ai fait voir ailleurs que Proudhon avait été précédé dans cette tentative par Bray, Gray et d’autres encore.\par
 Cela n’empêche pas ce genre de sagesse de sévir aujourd’hui en France, sous le nom de « science ». Jamais une école n’avait plus abusé du mot « science » que l’école proudhonienne, car… là où manquent les idées, se présente à point un mot.
}.\par
Sous cette forme III, le monde des marchandises ne possède donc une forme valeur relative sociale et générale, que parce que toutes les marchandises qui en font partie sont exclues de la forme équivalent ou de la forme sous laquelle elles sont immédiatement échangeables. Par contre, la marchandise qui fonctionne comme équivalent général, la toile, par exemple, ne saurait prendre part à la forme générale de la valeur relative ; il faudrait pour cela qu’elle pût se servir à elle-même d’équivalent. Nous obtenons alors : 20 \emph{mètres de toile} = 20 \emph{mètres de toile}, tautologie qui n’exprime ni valeur ni quantité de valeur. Pour exprimer la valeur relative de l’équivalent général, il nous faut lire à rebours la forme III. Il ne possède aucune forme relative commune avec les autres marchandises, mais sa valeur s’exprime relativement dans la série interminable de toutes les autres marchandises. La forme développée de la valeur relative, ou forme II, nous apparaît ainsi maintenant comme la forme spécifique dans laquelle l’équivalent général exprime sa propre valeur.
\paragraph[{1.1.1.3.3.3. Transition de la forme valeur générale à la forme argent.}]{1.1.1.3.3.\emph{3}. Transition de la forme valeur générale à la forme argent.}
\noindent La forme équivalent général est une forme de la valeur en général. Elle peut donc appartenir à n’importe quelle marchandise. D’un autre côté, une marchandise ne peut se trouver sous cette forme (forme III) que parce qu’elle est exclue elle-même par toutes les autres marchandises comme équivalent. Ce n’est qu’à partir du moment où ce caractère exclusif vient s’attacher à un genre spécial de marchandise, que la forme valeur relative prend consistance, se fixe dans un objet unique et acquiert une authenticité sociale.\par
La marchandise spéciale avec la forme naturelle de laquelle la forme équivalent s’identifie peu à peu dans la société devient marchandise monnaie ou fonctionne comme monnaie. Sa fonction sociale spécifique, et conséquemment son monopole social, est de jouer le rôle de l’équivalent universel dans le monde des marchandises. Parmi les marchandises qui, dans la forme II, figurent comme équivalents particuliers de la toile et qui, sous la forme III, expriment, ensemble dans la toile leur valeur relative, c’est l’or qui a conquis historiquement ce privilège. Mettons donc dans la forme III la marchandise or à la place de la marchandise toile, et nous obtenons :
\subsubsection[{1.1.1.3.4. Forme monnaie ou argent.}]{1.1.1.3.4. Forme monnaie ou argent\protect\footnotemark .}
\footnotetext{La traduction exacte des mots allemands « \emph{Geld, Geldform} » présente une difficulté. L’expression : « forme argent » peut indistinctement s’appliquer à toutes les marchandises sauf les métaux précieux. On ne saurait pas dire, par exemple, sans amener une certaine confusion dans l’esprit des lecteurs : « forme argent de l’argent », ou bien « l’or devient argent. Maintenant l’expression « forme monnaie » présente un autre inconvénient, qui vient de ce qu’en français le mot « monnaie » est souvent employé dans le sens de pièces monnayées. Nous employons alternativement les mots « forme monnaie » et « forme argent » suivant les cas, mais toujours dans le même sens.}

\begin{itemize}[itemsep=0pt,]
\item 20 mètres de toile =
\item 1 habit =
\item 10 livres de thé =
\item 40 livres de café =
\item 2 onces d’or =
\item ½ tonne de fer =
\item X marchandise A =
\item Etc. = 2 onces d’or
\end{itemize}

\noindent Des changements essentiels ont lieu dans la transition de la forme I à la forme II, et de la forme II à la forme III. La forme IV, au contraire, ne diffère en rien de la forme III, si ce n’est que maintenant c’est l’or qui possède à la place de la toile la forme équivalent général. Le progrès consiste tout simplement en ce que la forme d’échangeabilité immédiate et universelle, ou la forme d’équivalent général, s’est incorporée définitivement dans la forme naturelle et spécifique de l’or.\par
L’or ne joue le rôle de monnaie vis-à-vis des autres marchandises que parce qu’il jouait déjà auparavant vis-à-vis d’elles le rôle de marchandise. De même qu’elles toutes, il fonctionnait aussi comme équivalent, soit accidentellement dans des échanges isolés, soit comme équivalent particulier à côte d’autres équivalents. Peu à peu il fonctionna dans des limites plus ou moins larges comme équivalent général. Dès qu’il a conquis le monopole de cette position dans l’expression de la valeur du monde marchand, il devient marchandise monnaie, et c’est seulement à partir du moment où il est déjà devenu marchandise monnaie que la forme IV se distingue de la forme III, ou que la forme générale de valeur se métamorphose en forme monnaie ou argent.\par
L’expression de valeur relative simple d’une marchandise, de la toile, par exemple, dans la marchandise qui fonctionne déjà comme monnaie, par exemple, l’or, est forme prix. La forme prix de la toile est donc :\par
20 \emph{mètres de toile} = 2 \emph{onces d’or},\par
ou, si 2 livres sterling sont le nom de monnaie de 2 onces d’or,\par
20 \emph{mètres de toile} = 2 \emph{livres sterling}.\par
La difficulté dans le concept de la forme argent, c’est tout simplement de bien saisir la forme équivalent général, c’est-à-dire la forme valeur générale, la forme III. Celle-ci se résout dans la forme valeur développée, la forme II, et l’élément constituant de cette dernière est la forme I :\par
20\emph{ mètres de toile =} 1\emph{ habit, ou} x\emph{ marchandise} A\emph{ =} y\emph{ marchandise} B.\par
La forme simple de la marchandise est par conséquent le germe de la forme argent\footnote{L’économie politique classique n’a jamais réussi à déduire de son analyse de la marchandise, et spécialement de la valeur de cette marchandise, la forme sous laquelle elle devient valeur d’échange, et c’est là un de ses vices principaux. Ce sont précisément ses meilleurs représentants, tels qu’Adam Smith et Ricardo, qui traitent la forme valeur comme quelque chose d’indifférent ou n’ayant aucun rapport intime avec la nature de la marchandise elle-même. Ce n’est pas seulement parce que la valeur comme quantité absorbe leur attention. La raison en est plus profonde. La forme valeur du produit du travail est la forme la plus abstraite et la plus générale du mode de production actuel, qui acquiert par cela même un caractère historique, celui d’un mode particulier de production sociale. Si on commet l’erreur de la prendre pour la forme naturelle, éternelle, de toute production dans toute société, on perd nécessairement de vue le côté spécifique de la forme valeur, puis de la forme marchandise, et à un degré plus développé, de la forme argent, forme capital, etc. C’est ce qui explique pourquoi on trouve chez des économistes complètement d’accord entre eux sur la mesure de la quantité de valeur par la durée de travail les idées les plus diverses et les plus contradictoires sur l’argent, c’est-à-dire sur la forme fixe de l’équivalent général. On remarque cela surtout dès qu’il s’agit de questions telles que celle des banques par exemple ; c’est alors à n’en plus finir avec les définitions de la monnaie et les lieux communs constamment débités à ce propos. – Je fais remarquer une fois pour toutes que j’entends par économie politique classique toute économie qui, à partir de William Petty, cherche à pénétrer l’ensemble réel et intime des rapports de production dans la société bourgeoise, par opposition à l’économie vulgaire qui se contente des apparences, rumine sans cesse pour son propre besoin et pour la vulgarisation des plus grossiers phénomènes les matériaux déjà élaborés par ses prédécesseurs, et se borne à ériger pédantesquement en système et à proclamer comme vérités éternelles les illusions dont le bourgeois aime à peupler son monde à lui, le meilleur des mondes possibles.}.
\subsection[{1.1.1.4. Le caractère fétiche de la marchandise et son secret.}]{1.1.1.4. Le caractère fétiche de la marchandise et son secret.}
\noindent Une marchandise paraît au premier coup d’œil quelque chose de trivial et qui se comprend de soi-même. Notre analyse a montré au contraire que c’est une chose très complexe, pleine de subtilités métaphysiques et d’arguties théologiques. En tant que valeur d’usage, il n’y a en elle rien de mystérieux, soit qu’elle satisfasse les besoins de l’homme par ses propriétés, soit que ses propriétés soient produites par le travail humain. Il est évident que l’activité de l’homme transforme les matières fournies par la nature de façon à les rendre utiles. La forme du bois, par exemple, est changée, si l’on en fait une table. Néanmoins, la table reste bois, une chose ordinaire et qui tombe sous les sens. Mais dès qu’elle se présente comme marchandise, c’est une tout autre, affaire. A la fois saisissable et insaisissable, il ne lui suffit pas de poser ses pieds sur le sol ; elle se dresse, pour ainsi dire, sur sa tête de bois en face des autres marchandises et se livre à des caprices plus bizarres que si elle se mettait à danser\footnote{On se souvient que la Chine et les tables commencèrent à danser, lorsque tout le reste du monde semblait ne pas bouger – \emph{pour encourager les autres}.}.\par
Le caractère mystique de la marchandise ne provient donc pas de sa valeur d’usage. Il ne provient pas davantage des caractères qui déterminent la valeur. D’abord, en effet, si variés que puissent être les travaux utiles ou les activités productives, c’est une vérité physiologique qu’ils sont avant tout des fonctions de l’organisme humain, et que toute fonction pareille, quels que soient son contenu et sa forme, est essentiellement une dépense du cerveau, des nerfs, des muscles, des organes, des sens, etc., de l’homme. En second lieu, pour ce qui sert à déterminer la quantité de la valeur, c’est-à-dire la durée de cette dépense ou la quantité de travail, on ne saurait nier que cette quantité de travail se distingue visiblement de sa qualité. Dans tous les états sociaux le temps qu’il faut pour produire les moyens de consommation a dû intéresser l’homme, quoique inégalement, suivant les divers degrés de la civilisation\footnote{Chez les anciens Germains la grandeur d’un arpent de terre était calculée d’après le travail d’un jour, et de là son nom \emph{Tagwerk, Mannwerk}, etc. (\emph{Jurnale} ou \emph{jurnalis, terra jurnalis} ou \emph{diurnalis}.) D’ailleurs l’expression de « journal » de terre subsiste encore dans certaines parties de la France (voir Georg Ludwig von MAURER, \emph{Einleitung zur Geschichte der Mark-, Hof-, Dorf – und Stadt-Verfassung…}, Munich, 1854, p. 129 et suiv.). [Deuxième édition]}. Enfin dès que les hommes travaillent d’une manière quelconque les uns pour les autres, leur travail acquiert aussi une forme sociale.\par
D’où provient donc le caractère énigmatique du produit du travail, dès qu’il revêt la forme d’une marchandise ? Évidemment de cette forme elle-même.\par
Le caractère d’égalité des travaux humains acquiert la forme de valeur des produits du travail ; la mesure des travaux individuels par leur durée acquiert la forme de la grandeur de valeur des produits du travail ; enfin les rapports des producteurs, dans lesquels s’affirment les caractères sociaux de leurs travaux, acquièrent la forme d’un rapport social des produits du travail. Voilà pourquoi ces produits se convertissent en marchandises, c’est-à-dire en choses qui tombent et ne tombent pas sous les sens, ou choses sociales. C’est ainsi que l’impression lumineuse d’un objet sur le nerf optique ne se présente pas comme une excitation subjective du nerf lui-même, mais comme la forme sensible de quelque chose qui existe en dehors de l’œil. Il faut ajouter que dans l’acte de la vision la lumière est réellement projetée d’un objet extérieur sur un autre objet, l’œil ; c’est un rapport physique entre des choses physiques. Mais la forme valeur et le rapport de valeur des produits du travail n’ont absolument rien à faire avec leur nature physique. C’est seulement un rapport social déterminé des hommes entre eux qui revêt ici pour eux la forme fantastique d’un rapport des choses entre elles. Pour trouver une analogie à ce phénomène, il faut la chercher dans la région nuageuse du monde religieux. Là les produits du cerveau humain ont l’aspect d’êtres indépendants, doués de corps particuliers, en communication avec les hommes et entre eux. Il en est de même des produits de la main de l’homme dans le monde marchand. C’est ce qu’on peut nommer le fétichisme attaché aux produits du travail, dès qu’ils se présentent comme des marchandises, fétichisme inséparable de ce mode de production.\par
En général, des objets d’utilité ne deviennent des marchandises que parce qu’ils sont les produits de travaux privés exécutés indépendamment les uns des autres. L’ensemble de ces travaux privés forme le travail social, Comme les producteurs n’entrent socialement en contact que par l’échange de leurs produits, ce n’est que dans les limites de cet échange que s’affirment d’abord les caractères sociaux de leurs travaux privés. Ou bien les travaux privés ne se manifestent en réalité comme divisions du travail social que par les rapports que l’échange établit entre les produits du travail et indirectement entre les producteurs. Il en résulte que pour ces derniers les rapports de leurs travaux privés apparaissent ce qu’ils sont, c’est-à-dire non des rapports sociaux immédiats des personnes dans leurs travaux mêmes, mais bien plutôt des rapports sociaux entre les choses.\par
C’est seulement dans leur échange que les produits du travail acquièrent comme valeurs une existence sociale identique et uniforme, distincte de leur existence matérielle et multiforme comme objets d’utilité. Cette scission du produit du travail en objet utile et en objet de valeur s’élargit dans la pratique dès que l’échange a acquis assez d’étendue et d’importance pour que des objets utiles soient produits en vue de l’échange, de sorte que le caractère de valeur de ces objets est déjà pris en considération dans leur production même. A partir de ce moment, les travaux privés des producteurs acquièrent en fait un double caractère social. D’un côté, ils doivent être travail utile, satisfaire des besoins sociaux, et, s’affirmer ainsi comme parties intégrantes du travail général, d’un système de division sociale du travail qui se forme spontanément ; de l’autre côté, ils ne satisfont les besoins divers des producteurs eux-mêmes, que parce que chaque espèce de travail privé utile est échangeable avec toutes les autres espèces de travail privé utile, c’est-à-dire est réputé leur égal. L’égalité de travaux qui diffèrent \emph{toto cœlo} [complètement] les uns des autres ne peut consister que dans une abstraction de leur inégalité réelle, que dans la réduction à leur caractère commun de dépense de force humaine, de travail humain en général, et c’est l’échange seul qui opère cette réduction en mettant en présence les uns des autres sur un pied d’égalité les produits des travaux les plus divers.\par
Le double caractère social des travaux privés ne se réfléchit dans le cerveau des producteurs que sous la forme que leur imprime le commerce pratique, l’échange des produits. Lorsque les producteurs mettent en présence et en rapport les produits de leur travail à titre de valeurs, ce n’est pas qu’ils voient en eux une simple enveloppe sous laquelle est caché un travail humain identique ; tout au contraire : en réputant égaux dans l’échange leurs produits différents, ils établissent par le fait que leurs différents travaux sont égaux. Ils le font sans le savoir\footnote{Quand donc Galiani dit : « La valeur est un rapport entre deux personnes » ! \emph{La Richezza è une ragione tra due persone}. (GALIANI, \emph{Della Moneta}, p. 221, t. III du recueil de Custodi : \emph{Scrittori classici italiani di Economia politica. – Parte moderna}, Milan, 1803), il aurait dû ajouter : un rapport caché sous l’enveloppe des choses. [Deuxième édition]}. La valeur ne porte donc pas écrit sur le front ce qu’elle est. Elle fait bien plutôt de chaque produit du travail un hiéroglyphe. Ce n’est qu’avec le temps que l’homme cherche à déchiffrer le sens de l’hiéroglyphe à pénétrer les secrets de l’œuvre sociale à laquelle il contribue, et la transformation des objets utiles en valeurs est un produit de la société, tout aussi bien que le langage.\par
La découverte scientifique faite plus tard que les produits du travail, en tant que valeurs, sont l’expression pure et simple du travail humain dépensé dans leur production, marque une époque dans l’histoire du développement de l’humanité mais ne dissipe point la fantasmagorie qui fait apparaître le caractère social du travail comme un caractère des choses, des produits eux-mêmes. Ce qui n’est vrai que pour cette forme de production particulière, la production marchande, à savoir : que le caractère social des travaux les plus divers consiste dans leur égalité comme travail humain, et que ce caractère social spécifique revêt ne forme objective, la forme valeur des produits du travail, ce fait, pour l’homme engrené dans les rouages et les rapports de la production des marchandises, parait, après. comme avant la découverte de la nature de la valeur, tout aussi invariable et d’un ordre tout aussi naturel que la forme gazeuse de l’air qui est restée la même après comme avant la découverte de ses éléments chimiques.\par
Ce qui intéresse tout d’abord pratiquement les échangistes, c’est de savoir combien ils obtiendront en échange de leurs produits, c’est-à-dire la proportion dans laquelle les produits s’échangent entre eux. Dès que cette proportion a acquis une certaine fixité habituelle, elle leur parait provenir de la nature même des produits du travail. Il semble qu’il réside dans ces choses une propriété de s’échanger en proportions déterminées comme les substances chimiques se combinent en proportions fixes.\par
Le caractère de valeur des produits du travail ne ressort en fait que lorsqu’ils se déterminent comme quantités de valeur. Ces dernières changent sans cesse, indépendamment de la volonté et des prévisions des producteurs, aux yeux desquels leur propre mouvement social prend ainsi la forme d’un mouvement des choses, mouvement qui les mène, bien loin qu’ils puissent le diriger. Il faut que la production marchande se soit complètement développée avant que de l’expérience même se dégage cette vérité scientifique : que les travaux privés, exécutés indépendamment les uns des autres, bien qu’ils s’entrelacent comme ramifications du système social et spontané de la division du travail, sont constamment ramenés à leur mesure sociale proportionnelle. Et comment ? Parce que dans les rapports d’échange accidentels et toujours variables de leurs produits, le temps de travail social nécessaire à leur production l’emporte de haute lutte comme loi naturelle régulatrice, de même que la loi de la pesanteur se fait sentir à n’importe qui lorsque sa maison s’écroule sur sa tête\footnote{« Que doit-on penser d’une loi qui ne peut s’exécuter que par des révolutions périodiques ? C’est tout simplement une loi naturelle fondée sur l’inconscience de ceux qui la subissent. » (Friedrich ENGELS « \emph{Umrisse, zu einer Kritik der Nationalökonomie} », p. 103, dans les \emph{Annales franco-allemandes}, éditées par Arnold Ruge et Karl Marx, Paris, 1844.)}. La détermination de la quantité de valeur par la durée de travail est donc un secret caché sous le mouvement apparent des valeurs des marchandises ; mais sa solution, tout en montrant que la quantité de valeur ne se détermine pas au hasard, comme il semblerait, ne fait pas pour cela disparaître la forme qui représente cette quantité comme un rapport de grandeur entre les choses, entre les produits eux-mêmes du travail.\par
La réflexion sur les formes de la vie sociale, et, par conséquent, leur analyse scientifique, suit une route complètement opposée au mouvement réel. Elle commence, après coup, avec des données déjà tout établies, avec les résultats du développement. Les formes qui impriment aux produits du travail le cachet de marchandises et qui, par conséquent, président déjà à leur circulation possèdent aussi déjà la fixité de formes naturelles de la vie sociale, avant que les hommes cherchent à se rendre compte, non du caractère historique de ces formes qui leur paraissent bien plutôt immuables, mais de leur sens intime. Ainsi c’est seulement l’analyse du prix des marchandises qui a conduit à la détermination de leur valeur quantitative, et c’est seulement l’expression commune des marchandises en argent qui a amené la fixation de leur caractère valeur. Or, cette forme acquise et fixe du monde des marchandises, leur forme argent, au lieu de révéler les caractères sociaux des travaux privés et les rapports sociaux des producteurs, ne fait que les voiler. Quand je dis que du froment, un habit, des bottes se rapportent à la toile comme à l’incarnation générale du travail humain abstrait, la fausseté et l’étrangeté de cette expression sautent immédiatement aux yeux. Mais quand les producteurs de ces marchandises les rapportent, à la toile, à l’or ou à l’argent, ce qui revient au même, comme à l’équivalent général, les rapports entre leurs travaux privés et l’ensemble du travail social leur apparaissent précisément sous cette forme bizarre.\par
Les catégories de l’économie bourgeoise sont des formes de l’intellect qui ont une vérité objective, en tant qu’elles reflètent des rapports sociaux réels, mais ces rapports n’appartiennent qu’à cette époque historique déterminée, où la production marchande est le mode de production social. Si donc nous envisageons d’autres formes de production, nous verrons disparaître aussitôt tout ce mysticisme qui obscurcit les produits du travail dans la période actuelle.\par
Puisque l’économie politique aime les Robinsonades\footnote{Ricardo lui-même a sa Robinsonade. Le chasseur et le pêcheur primitifs sont pour lui des marchands qui échangent le poisson et le gibier en raison de la durée du travail réalisé dans leurs valeurs. A cette occasion, il commet ce singulier anachronisme, que le chasseur et le pêcheur consultent, pour le calcul de leurs instruments de travail, les tableaux d’annuités en usage à la Bourse de Londres en 1817. Les « parallélogrammes de M. Owen » paraissent être la seule forme de société qu’il connaisse en dehors de la société bourgeoise (K. Marx, \emph{Contribution}…, \emph{op. cit.}, p. 38-39). [Deuxième édition]}, visitons d’abord Robinson dans son île.\par
Modeste, comme il l’est naturellement, il n’en a pas moins divers besoins à satisfaire, et il lui faut exécuter des \emph{travaux utiles de genre différent}, fabriquer des meubles, par exemple, se faire des outils, apprivoiser des animaux, pêcher, chasser, etc. De ses prières et autres bagatelles semblables nous n’avons rien à dire, puisque notre Robinson y trouve son plaisir et considère une activité de cette espèce comme une distraction fortifiante. Malgré la variété de ses fonctions productives, il sait qu’elles ne sont que les formes diverses par lesquelles s’affirme le même Robinson, c’est-à-dire tout simplement des modes divers de travail humain. La nécessité même le force à partager son temps entre ses occupations différentes. Que l’une prenne plus, l’autre moins de place dans l’ensemble de ses travaux, cela dépend de la plus ou moins grande difficulté qu’il a à vaincre pour obtenir l’effet utile qu’il a en vue. L’expérience lui apprend cela, et notre homme qui a sauvé du naufrage montre, grand livre, plume et encre, ne tarde pas, en bon Anglais qu’il est, à mettre en note tous ses actes quotidiens. Son inventaire contient le détail des objets utiles qu’il possède, des différents modes de travail exigés par leur production, et enfin du temps de travail que lui coûtent en moyenne des quantités déterminées de ces divers produits. Tous les rapports entre Robinson et les choses qui forment la richesse qu’il s’est créée lui-même sont tellement simples et transparents que M. Baudrillart pourrait les comprendre sans une trop grande tension d’esprit. Et cependant toutes les déterminations essentielles de la valeur y sont contenues.\par
Transportons-nous, maintenant de l’île lumineuse de Robinson dans le sombre moyen âge européen. Au lieu de l’homme indépendant, nous trouvons ici tout le monde dépendant, serfs et seigneurs, vassaux et suzerains, laïques et clercs. Cette dépendance personnelle, caractérise aussi bien les rapports sociaux de la production matérielle que toutes les autres sphères, de la vie auxquelles elle sert de fondement. Et c’est précisément parce que la société est basée sur la dépendance personnelle que tous, les rapports sociaux apparaissent comme des rapports entre les personnes. Les travaux divers et leurs produits n’ont en conséquence pas besoin de prendre une figure fantastique distincte de leur réalité. Ils se présentent comme services, prestations et livraisons en nature. La forme naturelle du travail, sa particularité – et non sa généralité, son caractère abstrait, comme dans la production marchande – en est aussi la forme sociale. La corvée est tout aussi bien mesurée par le temps que le travail qui produit des marchandises ; mais chaque corvéable sait fort bien, sans recourir à un Adam Smith, que c’est une quantité déterminée de sa force de travail personnelle qu’il dépense au service de son maître. La dîme à fournir au prêtre est plus claire que la bénédiction du prêtre. De quelque manière donc qu’on juge les masques que portent les hommes dans cette société, les rapports sociaux des personnes dans leurs travaux respectifs s’affirment nettement comme leurs propres rapports personnels, au lieu de se déguiser en rapports sociaux des choses, des produits du travail.\par
Pour rencontrer le travail commun, c’est-à-dire l’association immédiate, nous n’avons pas besoin de remonter à sa forme naturelle primitive, telle qu’elle nous apparaît au seuil de l’histoire de tous les peuples civilisés\footnote{C’est un préjugé ridicule, répandu ces derniers temps, de croire que la propriété collective \emph{primitive} est une forme de propriété spécifiquement slave, voire exclusivement russe. C’est la forme primitive dont on peut établir la présence chez les Romains, les Germains, les Celtes, mais dont on rencontre encore, aux Indes, tout un échantillonnage aux spécimens variés, bien qu’en partie à l’état de vestiges. Une étude rigoureuse des formes de la propriété collective en Asie, et spécialement aux Indes, montrerait qu’en se dissolvant les différentes formes de la propriété collective primitive ont donné naissance à différentes formes de propriété. C’est ainsi que l’on peut, par exemple, déduire les différents types originaux de propriété privée à Rome et chez les Germains de différentes formes de propriété collective aux Indes (K. Marx, \emph{Contribution}…, \emph{op. cit.}, p. 13).[Deuxième édition]}. Nous en avons un exemple tout près de nous dans l’industrie rustique et patriarcale d’une famille de paysans qui produit pour ses propres besoins bétail, blé, toile, lin, vêtements, etc. Ces divers objets se présentent à la famille comme les produits divers de son travail et non comme des marchandises qui s’échangent réciproquement. Les différents travaux d’où dérivent ces produits, agriculture, élève du bétail, tissage, confection de vêtements, etc., possèdent de prime abord la forme de fonctions sociales, parce qu’ils sont des fonctions de la famille qui a sa division de travail tout aussi bien que la production marchande. Les conditions naturelles variant avec le changement des saisons, ainsi que les différences d’âge et de sexe, règlent dans la famille la distribution du travail et sa durée pour chacun. La mesure de la dépense des forces individuelles par le temps de travail apparaît ici directement comme caractère social des travaux eux-mêmes, parce que les forces de travail individuelles ne fonctionnent que comme organes de la force commune de la famille.\par
Représentons-nous enfin une réunion d’hommes libres travaillant avec des moyens de production communs, et dépensant, d’après un plan concerté, leurs nombreuses forces individuelles comme une seule et même force de travail social. Tout ce que nous avons dit du travail de Robinson se reproduit ici, mais socialement et non individuellement. Tous les produits de Robinson étaient son produit personnel et exclusif, et, conséquemment, objets d’utilité immédiate pour lui. Le produit total des travailleurs unis est un produit social. Une partie sert de nouveau comme moyen de production et reste sociale ; mais l’autre partie est consommée et, par conséquent, doit se répartir entre tous. Le mode de répartition variera suivant l’organisme producteur de la société et le degré de développement historique des travailleurs. Supposons, pour mettre cet état de choses en parallèle avec la production marchande, que la part accordée à chaque travailleur soit en raison son temps de travail. Le temps de travail jouerait ainsi un double rôle. D’un côté, sa distribution dans la société règle le rapport exact des diverses fonctions aux divers besoins ; de l’autre, il mesure la part individuelle de chaque producteur dans le travail commun, et en même temps la portion qui lui revient dans la partie du produit commun réservée à la consommation. Les rapports sociaux des hommes dans leurs travaux et avec les objets utiles qui en proviennent restent ici simples et transparents dans la production aussi bien que dans la distribution.\par
Le monde religieux n’est que le reflet du monde réel. Une société où le produit du travail prend généralement la forme de marchandise et où, par conséquent, le rapport le plus général entre les producteurs consiste à comparer les valeurs de leurs produits et, sous cette enveloppe des choses, à comparer les uns aux autres leurs travaux privés à titre de travail humain égal, une telle société trouve dans le christianisme avec son culte de l’homme abstrait, et surtout dans ses types bourgeois, protestantisme, déisme, etc., le complément religieux le plus convenable. Dans les modes de production de la vieille Asie, de l’antiquité en général, la transformation du produit en marchandise ne joue qu’un rôle subalterne, qui cependant acquiert plus d’importance à mesure que les communautés approchent de leur dissolution. Des peuples marchands proprement dits n’existent que dans les intervalles du monde antique, à la façon des dieux d’Epicure, ou comme les Juifs dans les pores de la société polonaise. Ces vieux organismes sociaux sont, sous le rapport de la production, infiniment plus simples et plus transparents que la société bourgeoise ; mais ils ont pour base l’immaturité de l’homme individuel – dont l’histoire n’a pas encore coupé, pour ainsi dire, le cordon ombilical qui l’unit à la communauté naturelle d’une tribu primitive – ou des conditions de despotisme et d’esclavage. Le degré inférieur de développement des forces productives du travail qui les caractérise, et qui par suite imprègne, tout le cercle de la vie matérielle, l’étroitesse des rapports des hommes, soit entre eux, soit avec la nature, se reflète idéalement dans les vieilles religions nationales. En général, le reflet religieux du monde réel ne pourra disparaître que lorsque les conditions du travail et de la vie pratique présenteront à l’homme des rapports transparents et rationnels avec ses semblables et avec la nature. La vie sociale, dont la production matérielle et les rapports qu’elle implique forment la base, ne sera dégagée du nuage mystique qui en voile l’aspect, que le jour où s’y manifestera l’œuvre d’hommes librement associés, agissant consciemment et maîtres de leur propre mouvement social. Mais cela exige dans la société un ensemble de conditions d’existence matérielle qui ne peuvent être elles-mêmes le produit que d’un long et douloureux développement.\par
L’économie politique a bien, il est vrai, analysé la valeur et la grandeur de valeur\footnote{ \noindent Un des premiers économistes qui après \emph{William Petty} ait ramené la \emph{valeur} à son véritable contenu, le célèbre Franklin, peut nous fournir un exemple de la manière dont l’économie bourgeoise procède dans son analyse. Il dit : « Comme le commerce en général n’est pas autre chose qu’un échange de travail contre travail, c’est par le travail qu’on estime le plus exactement la valeur de toutes choses » (\emph{The Works of Benjamin Franklin}., etc., éditions Sparks, Boston, 1836, t. II. p. 267). Franklin trouve tout aussi naturel que les choses aient de la valeur, que le corps de la pesanteur. A son point de vue, il s’agit tout simplement de trouver comment cette \emph{valeur} sera estimée le plus exactement possible. Il ne remarque même pas qu’en déclarant que « c’est par le travail qu’on estime le plus exactement la valeur de toute chose », il fait abstraction de la différence des travaux échangés et les réduit à un travail humain égal. Autrement il aurait dû dire : puisque l’échange de bottes ou de souliers contre des tables n’est pas autre chose qu’un échange de cordonnerie contre menuiserie, c’est par le travail du menuisier qu’on estimera avec le plus d’exactitude la valeur des bottes ! En se servant du mot travail en général, il fait abstraction du caractère utile et de la forme concrète des divers travaux. \\
L’insuffisance de l’analyse que Ricardo a donnée de la grandeur de la valeur – et c’est la meilleure – sera démontrée dans les Livres III et IV de cet ouvrage. Pour ce qui est de la valeur en général, l’économie politique classique ne distingue jamais clairement ni expressément le travail représenté dans la valeur du même travail en tant qu’il se représente dans la valeur d’usage du produit. Elle fait bien en réalité cette distinction, puisqu’elle considère le travail tantôt au point de vue de la qualité, tantôt à celui de la quantité. Mais il ne lui vient pas à l’esprit qu’une différence simplement quantitative des travaux suppose leur unité ou leur égalité qualitative, c’est-à-dire leur réduction au travail humain abstrait. Ricardo, par exemple, se déclare d’accord avec Destutt de Tracy quand celui-ci dit : « Puisqu’il est certain que nos facultés physiques et morales sont notre seule richesse originaire, que l’emploi de ces facultés, le travail quelconque, est notre seul trésor primitif, et que c’est toujours de cet emploi que naissent toutes les choses que nous appelons des \emph{biens}… il est certain même que tous ces biens ne font que représenter le travail qui leur a donné naissance, et que, s’ils ont une valeur, ou même deux distinctes, ils ne peuvent tenir ces valeurs que de celle du travail dont ils émanent. » (DESTUTT DE TRACY, \emph{Eléments d’idéologie}, IVe et Ve parties, Paris, 1826, p. 35, 36.) (Comp. RICARDO, \emph{The Principles of Political Economy}, 3e éd., London, 1821, p. 334.) Ajoutons seulement que Ricardo prête aux paroles de Destutt un sens trop profond. Destutt dit bien d’un côté que les choses qui forment la richesse représentent le travail qui les a créées ; mais, de l’autre, il prétend qu’elles tirent leurs deux valeurs différentes (valeur d’usage et valeur d’échange) de la valeur du travail. Il tombe ainsi dans la platitude de l’économie vulgaire qui admet préalablement la valeur d’une marchandise (du travail, par exemple) pour déterminer la valeur des autres.\par
 Ricardo le comprend comme s’il disait que le travail (non sa valeur) se représente aussi bien dans la valeur d’usage que dans la valeur d’échange. Mais lui-même distingue si peu le caractère à double face du travail que dans tout son chapitre « Valeur et Richesse », il est obligé de discuter les unes après les autres les trivialités d’un J.-B. Say. Aussi est-il à la fin tout étonné de se trouver d’accord avec Destutt sur le travail comme source de valeur, tandis que celui-ci, d’un autre côté, se fait de la valeur la même idée que Say.
}, quoique d’une manière très imparfaite. Mais elle ne s’est jamais de mandé pourquoi le travail se représente dans la valeur, et la mesure du travail par sa durée dans la grandeur de valeur des produits. Des formes qui manifestent au premier coup d’œil qu’elles appartiennent à une période sociale dans laquelle la production et ses rapports régissent l’homme au lieu d’être régis par lui paraissent à sa conscience bourgeoise une nécessité tout aussi naturelle que le travail productif lui-même. Rien d’étonnant qu’elle traite les formes de production sociale qui ont précédé la production bourgeoise, comme les Pères de l’Eglise traitaient les religions qui avaient précédé le christianisme\footnote{« Les économistes ont une singulière manière de procéder. Il n’y a pour eux que deux sortes d’institutions, celles de l’art et celles de la nature. Les institutions de la féodalité sont des institutions artificielles, celles de la bourgeoisie sont des institutions naturelles. Ils ressemblent en cela aux théologiens, qui, eux aussi, établissent deux sortes de religions. Toute religion qui n’est pas la leur est une invention des hommes, tandis que leur propre religion est une émanation de Dieu… Ainsi il y a eu de l’histoire, mais il n’y en a plus. » (Karl MARX, \emph{Misère de la philosophie. Réponse à la Philosophie de la misère de M. Proudhon}, 1847, p. 113.) Le plus drôle est Bastiat, qui se figure que les Grecs et les Romains n’ont vécu que de rapine. Mais quand on vit de rapine pendant plusieurs siècles, il faut pourtant qu’il y ait toujours quelque chose à prendre ou que l’objet des rapines continuelles se renouvelle constamment. Il faut donc croire que les Grecs et les Romains avaient leur genre de production à eux, conséquemment une économie, qui formait la base matérielle de leur société, tout comme l’économie bourgeoise forme la base de la nôtre. Ou bien Bastiat penserait-il qu’un mode de production fondé sur le travail des esclaves est un système de vol ? Il se place alors sur un terrain dangereux. Quand un géant de la pensée, tel qu’Aristote, a pu se tromper dans son appréciation du travail esclave, pourquoi un nain comme Bastiat serait-il infaillible dans son appréciation du travail salarié ? – Je saisis cette occasion pour dire quelques mots d’une objection qui m’a été faite par un journal allemand-américain à propos de mon ouvrage : \emph{Contribution à la critique de l’économie politique}, paru en 1859. Suivant lui, mon opinion que le mode déterminé de production et les rapports sociaux qui en découlent, en un mot que la structure économique de la société est la base réelle sur laquelle s’élève ensuite l’édifice juridique et politique, de telle sorte que le mode de production de la vie matérielle domine en général le développement de la vie sociale, politique et intellectuelle – suivant lui, cette opinion est juste pour le monde moderne dominé par les intérêts matériels mais non pour le Moyen Age où régnait le catholicisme, ni pour Athènes et Rome où régnait la politique. Tout d’abord, il est étrange qu’il plaise à certaines gens de supposer que quelqu’un ignore ces manières de parler vieillies et usées sur le Moyen Age et l’Antiquité. Ce qui est clair, c’est que ni le premier ne pouvait vivre du catholicisme, ni la seconde de la politique. Les conditions économiques d’alors expliquent au contraire pourquoi là le catholicisme et ici la politique jouaient le rôle principal. La moindre connaissance de l’histoire de la République romaine, par exemple, fait voir que le secret de cette histoire, c’est l’histoire de la propriété foncière. D’un autre côté, personne n’ignore que déjà don Quichotte a eu à se repentir pour avoir cru que la chevalerie errante était compatible avec toutes les formes économiques de la société.}.\par
Ce qui fait voir, entre autres choses, l’illusion produite sur la plupart des économistes par le fétichisme inhérent au monde marchand ; ou par l’apparence matérielle des attributs sociaux du travail, c’est leur longue et insipide querelle à propos du rôle de la nature dans la création de la valeur d’échange. Cette valeur n’étant pas autre chose qu’une manière sociale particulière de compter le travail employé dans la production d’un objet ne peut pas plus contenir d’éléments matériels que le cours du change, par exemple.\par
Dans notre société, la forme économique la plus générale et la plus simple qui s’attache aux produits du travail, la forme marchandise, est si familière à tout le monde que personne n’y voit malice. Considérons d’autres formes économiques plus complexes. D’où proviennent, par exemple, les illusions du système mercantile ? Évidemment du caractère fétiche que la forme monnaie imprime aux métaux précieux. Et l’économie moderne, qui fait l’esprit fort et ne se fatigue pas de ressasser ses fades plaisanteries contre le fétichisme des mercantilistes, est-elle moins la dupe des apparences ? N’est-ce pas son premier dogme que des choses, des instruments de travail, par exemple, sont, par nature, capital, et, qu’en voulant les dépouiller de ce caractère purement social, on commet un crime de lèse-nature ? Enfin, les physiocrates, si supérieurs à tant d’égards, n’ont-ils pas imaginé que la rente foncière n’est pas un tribut arraché aux hommes, mais un présent fait par la nature même aux propriétaires ? Mais n’anticipons pas et contentons-nous encore d’un exemple à propos de la forme marchandise elle-même.\par
Les marchandises diraient, si elles pouvaient parler : Notre valeur d’usage peut bien intéresser l’homme ; pour nous, en tant qu’objets, nous nous en moquons bien. Ce qui nous regarde c’est notre valeur. Notre rapport entre nous comme choses de vente et d’achat le prouve. Nous ne nous envisageons les unes les autres que comme valeurs d’échange. Ne croirait-on pas que l’économiste emprunte ses paroles à l’âme même de la marchandise quand il dit : « La valeur (valeur d’échange) est une propriété des choses, la richesse (valeur d’usage) est une propriété de l’homme. La valeur dans ce sens suppose nécessairement l’échange, la richesse, non\footnote{« \emph{Value is a property of things}, riches of man. Value, in this sense, necessarily implies exchanges, riches do not. » (\emph{Observations on certain verbal Disputas in Political Economy, particularly relating to value and to demand and supply}, London, 1821, p. 16.)}. » « La richesse (valeur utile) est un attribut de l’homme ; la valeur, un attribut des marchandises. Un homme ou bien une communauté est riche, une perle ou un diamant possède de la valeur et la possède comme telle\footnote{« Riches are the attribute of men, value is the attribute of commodities. A man or a community is rich, a pearl or a diamond is valuable… A pearl or a diamond is \emph{valuable as a pearl or diamond}. » (S. Bailey, \emph{op. cit.}, p. 165.)}. » Jusqu’ici aucun chimiste n’a découvert de valeur d’échange dans une perle ou dans un diamant. Les économistes qui ont découvert ou inventé des substances chimiques de ce genre, et qui affichent une. certaine prétention à la profondeur, trouvent, eux, que la valeur utile des choses leur appartient indépendamment de leurs propriétés matérielles, tandis que leur valeur leur appartient en tant que choses. Ce qui les confirme dans cette opinion, c’est cette circonstance étrange que la valeur utile des choses se réalise pour l’homme sans échange, c’est-à-dire dans un rapport immédiat entre la chose et l’homme, tandis que leur valeur, au contraire, ne se réalise que dans l’échange, c’est-à-dire dans un rapport social. Qui ne se souvient ici du bon Dogberry, et de la leçon qu’il donne au veilleur de nuit, Seacoal :\par
« Être un homme bien fait est un don des circonstances, mais savoir lire et écrire, cela nous vient de la nature\footnote{L’auteur des \emph{Observations} et S. BAILEY accusent Ricardo d’avoir fait de la valeur d’échange, chose purement relative, quelque chose d’absolu. Tout au contraire, il a ramené la relativité apparente que ces objets, tels que perle et diamant, par exemple, possèdent comme valeur d’échange, au vrai rapport caché sous cette apparence, à leur relativité comme simples expressions de travail humain. Si les partisans de Ricardo n’ont su répondre à Bailey que d’une manière grossière et pas du tout concluante, c’est tout simplement parce qu’ils n’ont trouvé chez Ricardo lui-même rien qui les éclairât sur le rapport intime qui existe entre la valeur et sa forme, c’est-à-dire la valeur d’échange.}. » (\emph{To be a well-favoured man is the gift of fortune ; but to write and read comes by nature}.)
\section[{1.1.2. Des échanges}]{1.1.2. Des échanges}\renewcommand{\leftmark}{1.1.2. Des échanges}

\noindent Les marchandises ne peuvent point aller elles-mêmes au marché ni s’échanger elles-mêmes entre elles. Il nous faut donc tourner nos regards vers leurs gardiens et conducteurs, c’est-à-dire vers leurs possesseurs. Les marchandises sont des choses et, conséquemment, n’opposent à l’homme aucune résistance. Si elles manquent de bonne volonté, il peut employer la force, en d’autres termes s’en emparer\footnote{Dans le XII° siècle, si renommé pour sa piété, on trouve souvent parmi les marchandises des choses très délicates. Un poète français de cette époque signale, par exemple, parmi les marchandises qui se voyaient sur le marché du Landit, à côté des étoffes, des chaussures, des cuirs et des instruments d’agriculture, « des femmes folles de leurs corps ».}. Pour mettre ces choses en rapport les unes avec les autres à titre de marchandises, leurs gardiens doivent eux-mêmes se mettre en rapport entre eux à titre de personnes dont la volonté habite dans ces choses mêmes, de telle sorte que la volonté de l’un est aussi la volonté de l’autre et que chacun s’approprie la marchandise étrangère en abandonnant la sienne, au moyen d’un acte volontaire commun. Ils doivent donc se reconnaître réciproquement comme propriétaires privés. Ce rapport juridique, qui a pour forme le contrat, légalement développé ou non, n’est que le rapport des volontés dans lequel se reflète le rapport économique. Son contenu est donné par le rapport économique lui-même\footnote{Bien des gens puisent leur idéal de justice dans les rapports juridiques qui ont leur origine dans la société basée sur la production marchande, ce qui, soit dit en passant, leur fournit agréablement la preuve que ce genre de production durera aussi longtemps que la justice elle-même. Ensuite, dans cet idéal, tiré de la société actuelle, ils prennent lent point d’appui pour réformer cette société et son droit. Que penserait-on d’un chimiste qui, au lieu d’étudier les lois des combinaisons matérielles et de résoudre sur cette base des problèmes déterminés, voudrait transformer ces combinaisons d’après les « idées éternelles de l’affinité et de la naturalité ? » Sait-on quelque chose de plus sur « l’usure », par exemple, quand on dit qu’elle est en contradiction avec la « justice éternelle » et l’« équité éternelle », que n’en savaient les Pères de l’Église quand ils en disaient autant en proclamant sa contradiction avec la « grâce éternelle, la foi éternelle et la volonté éternelle de Dieu » ?}. Les personnes n’ont affaire ici les unes aux autres qu’autant qu’elles mettent certaines choses en rapport entre elles comme marchandises. Elles n’existent les unes pour les autres qu’à titre de représentants de la marchandise qu’elles possèdent. Nous verrons d’ailleurs dans le cours du développement que les masques divers dont elles s’affublent suivant les circonstances ne sont que les personnifications des rapports économiques qu’elles maintiennent les unes vis-à-vis des autres.\par
Ce qui distingue surtout l’échangiste de sa marchandise, c’est que pour celle-ci toute autre marchandise n’est qu’une forme d’apparition de sa propre valeur. Naturellement débauchée et cynique, elle est toujours sur le point d’échanger son âme et même son corps avec n’importe quelle autre marchandise, cette dernière fût-elle aussi dépourvue d’attraits que Maritorne. Ce sens qui lui manque pour apprécier le côté concret de ses sœurs, l’échangiste le compense et le développe par ses propres sens à lui, au nombre de cinq et plus. Pour lui, la marchandise n’a aucune valeur utile immédiate ; s’il en était autrement, il ne la mènerait pas au marché. La seule valeur utile qu’il lui trouve, c’est qu’elle est porte-valeur, utile à d’autres et, par conséquent, un instrument d’échange\footnote{« Car l’usage de chaque chose est de deux sortes : l’une est propre à la chose comme telle, l’autre non ; une sandale, par exemple, sert de chaussure et de moyen d’échange. Sous ces deux points de vue, la sandale est une valeur d’usage, car celui qui l’échange pour ce qui lui manque, la nourriture, je suppose, se sert aussi de la sandale comme sandale, mais non dans son genre d’usage naturel, car elle n’est pas là précisément pour l’échange. » (ARISTOTE, \emph{De Rep.}, l. I, ch. IX.)}. Il veut donc l’aliéner pour d’autres marchandises dont la valeur d’usage puisse le satisfaire. Toutes les marchandises sont des non-valeurs d’usage pour ceux qui les possèdent et des valeurs d’usage pour ceux qui ne les possèdent pas. Aussi faut-il qu’elles passent d’une main dans l’autre sur toute la ligne. Mais ce changement de mains constitue leur échange, et leur échange les rapporte les unes aux autres comme valeurs et les réalise comme valeurs. Il faut donc que les marchandises se manifestent comme valeurs, avant qu’elles puissent se réaliser comme valeurs d’usage.\par
D’un autre côté, il faut que leur valeur d’usage soit constatée avant qu’elles puissent se réaliser comme valeurs ; car le travail humain dépensé dans leur production ne compte qu’autant qu’il est dépensé sous une forme utile à d’autres. Or, leur échange seul peut démontrer si ce travail est utile à d’autres, c’est-à-dire si son produit peut satisfaire des besoins étrangers.\par
Chaque possesseur de marchandise ne veut l’aliéner que contre une autre dont la valeur utile satisfait son besoin. En ce sens, l’échange n’est pour lui qu’une affaire individuelle. En outre, il veut réaliser sa marchandise comme valeur dans n’importe quelle marchandise de même valeur qui lui plaise, sans s’inquiéter si sa propre marchandise a pour le possesseur de l’autre une valeur utile ou non. Dans ce sens, l’échange est pour lui un acte social général. Mais le même acte ne peut être simultanément pour tous les échangistes de marchandises simplement individuel et, en même temps, simplement social et général.\par
Considérons la chose de plus près : pour chaque possesseur de marchandises, toute marchandise étrangère est un équivalent particulier de la sienne ; sa marchandise est, par conséquent, l’équivalent général de toutes les autres. Mais comme tous les échangistes se trouvent dans le même cas, aucune marchandise n’est équivalent général, et la valeur relative des marchandises ne possède aucune forme générale sous laquelle elles puissent être comparées comme quantités de valeur. En un mot, elles ne jouent pas les unes vis-à-vis des autres le rôle de marchandises mais celui de simples produits ou de valeurs d’usage.\par
Dans leur embarras, nos échangistes pensent comme Faust : au commencement était l’action. Aussi ont-ils déjà agi avant d’avoir pensé, et leur instinct naturel ne fait que confirmer les lois provenant de la nature des marchandises. Ils ne peuvent comparer leurs articles comme valeurs et, par conséquent, comme marchandises qu’en les comparant à une autre marchandise quelconque qui se pose devant eux comme équivalent général. C’est ce que l’analyse précédente a déjà démontré. Mais cet équivalent général ne peut être le résultat que d’une action sociale. Une marchandise spéciale est donc mise à part par un acte commun des autres marchandises et sert à exposer leurs valeurs réciproques. La forme naturelle de cette marchandise devient ainsi la forme équivalent socialement valide. Le rôle d’équivalent général est désormais la fonction sociale spécifique de la marchandise exclue, et elle devient argent.\par

\begin{quoteblock}
 \noindent « Illi unum consilium habent et virtutem et potestatem suam bestiæ tradunt. Et ne quis possit emere aut vendere, nisi qui habet characterem aut nomen bestiæ, aut numerum nominis ejus » (Apocalypse)\footnote{« Ils ont tous un même dessein et ils donneront à la bête leur force et leur puissance. » (\emph{Apocalypse}, XVII, 13) « Et que personne ne puisse ni acheter, ni vendre, que celui qui aura le caractère ou le nom de la bête, ou le nombre de son nom. » (\emph{Apocalypse}, XIII, 17, trad. Lemaistre de Sacy.)}.
\end{quoteblock}

\noindent L’argent est un cristal qui se forme spontanément dans les échanges par lesquels les divers produits du travail sont en fait égalisés entre eux et, par cela même, transformés en marchandises. Le développement historique de l’échange imprime de plus en plus aux produits du travail le caractère de marchandises et développe en même temps l’opposition que recèle leur nature, celle de valeur d’usage et de valeur. Le besoin même du commerce force à donner un corps à cette antithèse, tend à faire naître une forme valeur palpable et ne laisse plus ni repos ni trêve jusqu’à ce que cette forme soit enfin atteinte par le dédoublement de la marchandise en marchandise et en argent. A mesure donc que s’accomplit la transformation générale des produits du travail en marchandises, s’accomplit aussi la transformation de la marchandise en argent\footnote{On peut d’après cela apprécier le socialisme bourgeois qui veut éterniser la production marchande et, en même temps, abolir « l’opposition de marchandise et argent », c’est-à-dire l’argent lui-même, car il n’existe que dans cette opposition. Voir sur ce sujet : \emph{Contribution}…, p. 61.}.\par
Dans l’échange immédiat des produits, l’expression de la valeur revêt d’un côté la forme relative simple et de l’autre ne la revêt pas encore. Cette forme était : \emph{x} marchandise A = \emph{y} marchandise B. La forme de l’échange immédiat est : \emph{x} objets d’utilité A = \emph{y} objets d’utilité B\footnote{Tant que deux objets utiles différents ne sont pas encore échangés, mais qu’une masse chaotique de choses est offerte comme équivalent pour une troisième, ainsi que nous le voyons chez les sauvages, l’échange immédiat des produits n’est lui-même qu’à son berceau.}. Les objets A et B ne sont point ici des marchandises avant l’échange, mais le deviennent seulement par l’échange même. Dès le moment qu’un objet utile dépasse par son abondance les besoins de son producteur, il cesse d’être valeur d’usage pour lui et, les circonstances données, sera utilisé comme valeur d’échange. Les choses sont par elles-mêmes extérieures à l’homme et, par conséquent, aliénables. Pour que l’aliénation soit réciproque, il faut tout simplement que des hommes se rapportent les uns aux autres, par une reconnaissance tacite, comme propriétaires privés de ces choses aliénables et, par là même, comme personnes indépendantes. Cependant, un tel rapport d’indépendance réciproque n’existe pas encore pour les membres d’une communauté primitive, quelle que soit sa forme, famille patriarcale, communauté indienne, État inca comme au Pérou, etc. L’échange des marchandises commence là où les communautés finissent, à leurs points de contact avec des communautés étrangères ou avec des membres de ces dernières communautés. Dès que les choses sont une fois devenues des marchandises dans la vie commune avec l’étranger, elles le deviennent également par contrecoup dans la vie commune intérieure. La proportion dans laquelle elles s’échangent est d’abord purement accidentelle, Elles deviennent échangeables par l’acte volontaire de leurs possesseurs qui se décident à les aliéner réciproquement. Peu à peu, le besoin d’objets utiles provenant de l’étranger se fait sentir davantage et se consolide. La répétition constante de l’échange en fait une affaire sociale régulière, et, avec le cours du temps, une partie au moins des objets utiles est produite intentionnellement en vue de l’échange. A partir de cet instant, s’opère d’une manière nette la séparation entre l’utilité des choses pour les besoins immédiats et leur utilité pour l’échange à effectuer entre elles, c’est à-dire entre leur valeur d’usage et leur valeur d’échange. D’un autre côté, la proportion dans laquelle elles s’échangent commence à se régler par leur production même. L’habitude les fixe comme quantités de valeur.\par
Dans l’échange immédiat des produits, chaque marchandise est moyen d’échange immédiat pour celui qui la possède, mais pour celui qui ne la possède pas, elle ne devient équivalent que dans le cas où elle est pour lui une valeur d’usage. L’article d’échange n’acquiert donc encore aucune forme valeur indépendante de sa propre valeur d’usage ou du besoin individuel des échangistes. La nécessité de cette forme se développe à mesure qu’augmentent le nombre et la variété des marchandises qui entrent peu à peu dans l’échange, et le problème éclôt simultanément avec les moyens de le résoudre. Des possesseurs de marchandises n’échangent et ne comparent jamais leurs propres articles avec d’autres articles différents, sans que diverses marchandises soient échangées et comparées comme valeurs par leurs maîtres divers avec une seule et même troisième espèce de marchandise. Une telle troisième marchandise, en devenant équivalent pour diverses autres, acquiert immédiatement, quoique dans d’étroites limites, la forme équivalent général ou social. Cette forme générale naît et disparaît avec le contact social passager qui l’a appelée à la vie, et s’attache rapidement et tour à tour tantôt à une marchandise, tantôt à l’autre. Dès que l’échange a atteint un certain développement, elle s’attache exclusivement à une espèce particulière de marchandise, ou se cristallise sous forme argent. Le hasard décide d’abord sur quel genre de marchandises elle reste fixée ; on peut dire cependant que cela dépend en général de deux circonstances décisives. La forme argent adhère ou bien aux articles d’importation les plus importants qui révèlent en fait les premiers la valeur d’échange des produits indigènes, ou bien aux objets ou plutôt à l’objet utile qui forme l’élément principal de la richesse indigène aliénable, comme le bétail, par exemple. Les peuples nomades développent les premiers la forme argent parce que tout leur bien et tout leur avoir se trouve sous forme mobilière, et par conséquent immédiatement aliénable. De plus, leur genre de vie les met constamment en contact avec des sociétés étrangères, et les sollicite par cela même à l’échange des produits. Les hommes ont souvent f ait de l’homme même, dans la figure de l’esclave, la matière primitive de leur argent ; il n’en a jamais été ainsi du sol. Une telle idée ne pouvait naître que dans une société bourgeoise déjà développée. Elle date du dernier tiers du XVII° siècle ; et sa réalisation n’a été essayée sur une grande échelle, par toute une nation, qu’un siècle plus tard, dans la révolution de 1789, en France.\par
A mesure que l’échange brise ses liens purement locaux, et que par suite la valeur des marchandises représente de plus en plus le travail humain en général, la forme argent passe à des marchandises que leur nature rend aptes à remplir la fonction sociale d’équivalent général, c’est-à-dire aux métaux précieux.\par
Que maintenant bien que, « par nature, l’or et l’argent ne soient pas monnaie, mais [que] la monnaie soit, par nature, or et argent\footnote{Karl MARX, \emph{Contribution}…, p. 121, – « Les métaux précieux… sont naturellement monnaie » (GALIANI, \emph{Della Monetta}, dans le recueil de Custodi, \emph{Parte moderna}, t. III, p. 137).} », c’est ce que montrent l’accord et l’analogie qui existent entre les propriétés naturelles de ces métaux et les fonctions de la monnaie\footnote{V. de plus amples détails à ce sujet dans mon ouvrage déjà cité, ch. « Les métaux précieux ».}. Mais jusqu’ici nous ne connaissons qu’une fonction de la monnaie, celle de servir comme forme de manifestation de la valeur des marchandises, ou comme matière dans laquelle les quantités de valeur des marchandises s’expriment socialement. Or, il n’y a qu’une seule matière qui puisse être une forme propre à manifester la valeur ou servir d’image concrète du travail humain abstrait et conséquemment égal, c’est celle dont tous les exemplaires possèdent la même qualité uniforme. D’un autre côté, comme des valeurs ne diffèrent que par leur quantité, la marchandise monnaie doit être susceptible de différences purement quantitatives ; elle doit être divisible à volonté et pouvoir être recomposée avec la somme de toutes ses parties. Chacun sait que l’or et l’argent possèdent naturellement toutes ces propriétés.\par
La valeur d’usage de la marchandise monnaie devient double. Outre sa valeur d’usage particulière comme marchandise – ainsi l’or, par exemple, sert de matière première pour articles de luxe, pour boucher les dents creuses, etc. – elle acquiert une valeur d’usage formelle qui a pour origine sa fonction sociale spécifique.\par
Comme toutes les marchandises ne sont que des équivalents particuliers de l’argent, et que ce dernier est leur équivalent général, il joue vis-à-vis d’elles le rôle de marchandise universelle, et elles ne représentent vis-à-vis de lui que des marchandises particulières\footnote{« L’argent est la marchandise universelle. » (VERRI, \emph{Meditazioni sulla Economia Politica}, p. 16.)}.\par
On a vu que la forme argent ou monnaie n’est que le reflet des rapports de valeur de toute sorte de marchandises dans une seule espèce de marchandise. Que l’argent lui-même soit marchandise, cela ne peut donc être une découverte que pour celui qui prend pour point de départ sa forme tout achevée pour en arriver à son analyse ensuite\footnote{« L’argent et l’or eux-mêmes, auxquels nous pouvons donner le nom général de lingots, sont… des marchandises… dont la valeur… hausse et baisse. Le lingot a une plus grande valeur là où, avec un moindre poids, on achète une plus grande quantité de produits ou de marchandises du pays. » (\emph{A Discourse on the general notions of Money, Trade and Exchange, as they stand in relations to each other, by a Merchant}, London, 1695, p. 7.) « L’argent et l’or, monnayés ou non, quoiqu’ils servent de mesure à toutes les autres choses, sont des marchandises tout aussi bien que le vin, l’huile, le tabac, le drap et les étoffes. » (\emph{A Discourse concerning Trade, and that in particular of the East Indies}, etc., London, 1689, p. 2.) « Les fonds et les richesses du royaume ne peuvent pas consister exclusivement en monnaie, et l’or et l’argent ne doivent pas être exclus du nombre des marchandises. » (\emph{The East India Trade, a most profitable Trade…}, London, 1677, p. 4.)}. Le mouvement des échanges donne à la marchandise qu’il transforme en argent non pas sa valeur, mais sa forme valeur spécifique. Confondant deux choses aussi disparates, on a été amené à considérer l’argent et l’or comme des valeurs purement imaginaires\footnote{« L’or et l’argent ont leur valeur comme métaux avant qu’ils deviennent monnaie. »(GALIANI, \emph{op. cit.}, p. 72.) Locke dit : « Le commun consentement des hommes assigna une valeur imaginaire à l’argent, à cause de ses qualités qui le rendaient propre à la monnaie. » Law, au contraire : « Je ne saurais concevoir comment différentes nations pourraient donner une valeur imaginaire à aucune chose… ou comment cette valeur imaginaire pourrait avoir été maintenue ? » Mais il n’entendait rien lui-même à cette question, car ailleurs il s’exprime ainsi : « L’argent s’échangeait sur le pied de ce qu’il était évalué pour les usages », c’est-à-dire d’après sa valeur réelle ; « il reçut une valeur additionnelle… de son usage comme monnaie ». (Jean LAW, \emph{Considérations sur le numéraire et le commerce}, Éd. Daire, « Économistes financiers du XVIIIe siècle », p. 469-470.)}. Le fait que l’argent dans certaines de ses fonctions peut être remplacé par de simples signes de lui-même a fait naître cette autre erreur qu’il n’est qu’un simple signe.\par
D’un autre côté, il est vrai, cette erreur faisait pressentir que, sous l’apparence d’un objet extérieur, la monnaie déguise en réalité un rapport social. Dans ce sens, toute marchandise serait un signe, parce qu’elle n’est valeur que comme enveloppe matérielle du travail humain dépensé dans sa production\footnote{ \noindent « L’argent en [des denrées] est le signe » (V. DE FORBONNAIS, \emph{Eléments du commerce}, nouv. éd. Leyde, 1766, t. II, p. 143). – « Comme signe il est attiré par les denrées » (\emph{op. cit.}, p. 155). – « L’argent est un signe d’une chose et la représente » (MONTESQUIEU, \emph{Esprit des lois} [Œuvres, Londres, 1766, t. II, p. 148]). L’argent « n’est pas simple signe, car il est lui-même richesse ; il ne représente pas les valeurs, il les équivaut » (LE TROSNE, \emph{op. cit.}, p. 910).\par
 « Si on considère le concept de valeur, la chose elle-même n’est prise que comme un signe, et elle ne représente pas ce qu’elle est elle-même, mais ce qu’elle vaut. » HEGEL, \emph{Philosophie du droit}. [Première édition]\par
 [Longtemps avant les économistes, les juristes avaient mis en vogue cette idée que l’argent n’est qu’un simple signe et que les métaux précieux n’ont qu’une valeur imaginaire. Valets et sycophantes du pouvoir royal, ils ont pendant tout le Moyen Age appuyé le droit des rois à la falsification des monnaies sur les traditions de l’Empire romain et sur le concept du rôle de l’argent tel qu’il se trouve dans les Pandectes. « Que aucun puisse ne doit faire doute, dit leur habile disciple Philippe de Valois dans un décret de 1346 (16 janvier), que à Nous et à Nostre Majesté royal, n’appartiengne seulement… le mestier, le fait, la provision et toute l’Ordenance de monoie et de faire monnoier teles monnoyes et donner tel cours, pour tel prix comme il Nous plaist et bon Nous semble » [\emph{Ordonnances des rois de France de la 3e race}…, Paris, 1729, t. II, p. 254]. C’était un dogme du droit romain que l’empereur décrétât la valeur de l’argent. Il était défendu expressément de le traiter comme une marchandise. \emph{Pecunias veto nulli emere fas erit, nam in usu publico constitutas oportet non esse mercem}. [Il ne peut être permis à personne d’acheter de l’argent, car, créé pour l’usage public, il ne peut être marchandise.] On trouve d’excellents commentaires là-dessus dans G.F. PAGNINI, \emph{Saggio sopra il giusto pregio delle cose}, 1751, dans Custodi, \emph{Parte moderna}, t. II. Dans la seconde partie de son écrit notamment, Pagnini dirige sa polémique contre les juristes.
}. Mais dès qu’on ne voit plus que de simples signes dans les caractères sociaux que revêtent les choses, ou dans les caractères matériels que revêtent les déterminations sociales du travail sur la base d’un mode particulier de production, on leur prête le sens de fictions conventionnelles, sanctionnées par le prétendu consentement universel des hommes.\par
C’était là le mode d’explication en vogue au XVIIIe siècle ; ne pouvant encore déchiffrer ni l’origine ni le développement des formes énigmatiques des rapports sociaux, on s’en débarrassait en déclarant qu’elles étaient d’invention humaine et non pas tombées du ciel.\par
Nous avons déjà fait la remarque que la forme équivalent d’une marchandise ne laisse rien savoir sur le montant de sa quantité de valeur. Si l’on sait que l’or est monnaie, c’est-à-dire échangeable contre toutes les marchandises, on ne sait point pour cela combien valent par exemple 10 livres d’or. Comme toute marchandise, l’argent ne peut exprimer sa propre quantité de valeur que, relativement, dans d’autres marchandises. Sa valeur propre est déterminée par le temps de travail nécessaire à sa production, et s’exprime dans le \emph{quantum} de toute autre marchandise qui a exigé un travail de même durée\footnote{« Si un homme peut livrer à Londres une once d’argent extraite des mines du Pérou, dans le même temps qu’il lui faudrait pour produire un boisseau de grain, alors l’un est le prix naturel de l’autre. Maintenant, si un homme, par l’exploitation de mines plus nouvelles et plus riches, peut se procurer aussi facilement deux onces d’argent qu’auparavant une seule, le grain sera aussi bon marché à 10 shillings le boisseau qu’il l’était auparavant à 5 shillings, \emph{caeteris paribus} [toutes choses égales d’ailleurs] (William PETTY, \emph{A Treatise of Taxes and Contributions}, London, 1667, p. 31).}. Cette fixation de sa quantité de valeur relative a lieu à la source même de sa production dans son premier échange. Dès qu’il entre dans la circulation comme monnaie, sa valeur est donnée. Déjà dans les dernières années du XVIIe siècle, on avait bien constaté que la monnaie est marchandise ; l’analyse n’en était cependant qu’à ses premiers pas. La difficulté ne consiste pas à comprendre que la monnaie est marchandise, mais à savoir comment et pourquoi une marchandise devient monnaie\footnote{Maître Roscher, le professeur, nous apprend d’abord : « Que les fausses définitions de l’argent peuvent se diviser en deux groupes principaux : il y a celles d’après lesquelles il est plus et celles d’après lesquelles il est moins qu’une marchandise. » Puis il nous fournit un catalogue des écrits les plus bigarrés sur la nature de l’argent, ce qui ne jette pas la moindre lueur sur l’histoire réelle de la théorie. A la fin, arrive la morale : « On ne peut nier, dit-il, que la plupart des derniers économistes ont accordé peu d’attention aux particularités qui distinguent l’argent des autres marchandises [il est donc plus ou moins qu’une marchandise ?]. En ce sens, la réaction mi-mercantiliste de Ganilh, etc., n’est pas tout à fait sans fondement. » (Wilhelm ROSCHER, \emph{Die Grundlagen der Nationalökonomie}, 3e édit., 1858, p. 207-210.) Plus – moins – trop peu – en ce sens – pas tout à fait – quelle clarté et quelle précision dans les idées et le langage ! Et c’est un tel fatras d’éclectisme professoral que maître Roscher baptise modestement du nom de « méthode anatomico-physiologique » de l’économie politique ! On lui doit cependant une découverte, à savoir que l’argent est « une marchandise agréable ».}.\par
Nous avons déjà vu que dans l’expression de valeur la plus simple \emph{x} marchandise A = \emph{y} marchandise B, l’objet dans lequel la quantité de valeur d’un autre objet est représentée semble posséder sa forme équivalent, indépendamment de ce rapport, comme une propriété sociale qu’il tire de la nature. Nous avons poursuivi cette fausse apparence jusqu’au moment de sa consolidation. Cette consolidation est accomplie dès que la forme équivalent général s’est attachée exclusivement à une marchandise particulière ou s’est cristallisée sous forme argent. Une marchandise ne paraît point devenir argent parce que les autres marchandises expriment en elle réciproquement leurs valeurs ; tout au contraire, ces dernières paraissent exprimer en elle leurs valeurs parce qu’elle est argent. Le mouvement qui a servi d’intermédiaire s’évanouit dans son propre résultat et ne laisse aucune trace. Les marchandises trouvent, sans paraître y avoir contribué en rien, leur propre valeur représentée et fixée dans le corps d’une marchandise qui existe à côté et en dehors d’elles. Ces simples choses, argent et or, telles qu’elles sortent des entrailles de la terre, figurent aussitôt comme incarnation immédiate de tout travail humain. De là la magie de l’argent.
\section[{1.1.3. La monnaie ou la circulation des marchandises}]{1.1.3. La monnaie ou la circulation des marchandises}\renewcommand{\leftmark}{1.1.3. La monnaie ou la circulation des marchandises}

\subsection[{1.1.3.1. Mesure des valeurs}]{1.1.3.1. Mesure des valeurs}
\noindent Dans un but de simplification, nous supposons que l’or est la marchandise qui remplit les fonctions de monnaie.\par
La première fonction de l’or consiste à fournir à l’ensemble des marchandises la matière dans laquelle elles expriment leurs valeurs comme grandeurs de la même dénomination, de qualité égale et comparables sous le rapport de la quantité. Il fonctionne donc comme mesure universelle des valeurs. C’est en vertu de cette fonction que l’or, la marchandise équivalent, devient monnaie.\par
Ce n’est pas la monnaie qui rend les marchandises commensurables : au contraire. C’est parce que les marchandises en tant que valeurs sont du travail matérialisé, et par suite commensurables entre elles, qu’elles peuvent mesurer toutes ensemble leurs valeurs dans une marchandise spéciale, et transformer cette dernière en monnaie, c’est‑à‑dire en faire leur mesure commune. Mais la mesure des valeurs par la monnaie est la forme que doit nécessairement revêtir leur mesure immanente, la durée de travail\footnote{Poser la question de savoir pourquoi la monnaie ne représente pas immédiatement le temps de travail lui‑même, de telle sorte, par exemple, qu’un billet représente un travail de x heures, revient tout simplement à ceci : pourquoi, étant donné la production marchande, les produits du travail doivent‑ils revêtir la forme de marchandises ? Ou à cette autre : pourquoi le travail privé ne peut‑il pas être traité immédiatement comme travail social, c’est‑à‑dire comme son contraire ? J’ai rendu compte ailleurs avec plus de détails de l’utopie d’une « monnaie ou bon de travail » dans le milieu actuel de production (I c., p. 61 et suiv.). Remarquons encore ici que le bon de travail d’Owen, par exemple, est aussi peu de l’argent qu’une contremarque de théâtre. Owen suppose d’abord un travail socialisé, ce qui est une forme de production diamétralement opposée à la production marchande. Chez lui le certificat de travail constate simplement la part individuelle du producteur au travail commun et son droit individuel à la fraction du produit commun destinée à la consommation. Il n’entre point dans l’esprit d’Owen de supposer d’un côte la production marchande et de vouloir de l’autre échapper à ses conditions inévitables par des bousillages d’argent.}.\par
L’expression de valeur d’une marchandise en or : x marchandise A = y marchandise monnaie, est sa forme monnaie ou son prix. Une équation isolée telle que : 1 tonne de fer = 2 onces d’or, suffit maintenant pour exposer la valeur du fer d’une manière socialement valide. Une équation de ce genre n’a plus besoin de figurer comme anneau dans la série des équations de toutes les autres marchandises, parce que la marchandise équivalent, l’or, possède déjà le caractère monnaie. La forme générale de la valeur relative des marchandises a donc maintenant regagné son aspect primitif, sa forme simple.\par
La marchandise monnaie de son côté n’a point de prix. Pour qu’elle pût prendre part à cette forme de la valeur relative, qui est commune à toutes les autres marchandises, il faudrait qu’elle pût se servir à elle-même d’équivalent. Au contraire la forme où la valeur d’une marchandise était exprimée dans une série interminable d’équations, devient pour l’argent la forme exclusive de sa valeur relative. Mais cette série est maintenant déjà donnée dans les prix des marchandises. Il suffit de lire à rebours la cote d’un prix courant pour trouver la quantité de valeur de l’argent dans toutes les marchandises possibles.\par
Le prix ou la forme monnaie des marchandises est comme la forme valeur en général distincte de leur corps ou de leur forme naturelle, quelque chose d’idéal. La valeur du fer, de la toile, du froment, etc., réside dans ces choses mêmes, quoique invisiblement. Elle est représentée par leur égalité avec l’or, par un rapport avec ce métal, qui n’existe, pour ainsi dire, que dans la tête des marchandises. L’échangiste est donc obligé soit de leur prêter sa propre langue soit de leur attacher des inscriptions sur du papier pour annoncer leur prix au monde extérieur\footnote{Le sauvage ou le demi‑sauvage se sert de sa langue autrement. Le capitaine Parry remarque, par exemple, des habitants de la côte ouest de la baie de Baffin : « Dans ce cas (l’échange des produits) ils passent la langue deux fois sur la chose présentée à eux, après quoi ils semblent croire que le traité est dûment conclu. » Les Esquimaux de l’est léchaient de même les articles qu’on leur vendait à mesure qu’ils les recevaient. Si la langue est employée dans le nord comme organe d’appropriation, rien d’étonnant que dans le sud le ventre passe pour l’organe de la propriété accumulée et que le Caffre juge de la richesse d’un homme d’après son embonpoint et sa bedaine. Ces Caffres sont des gaillards très clairvoyants, car tandis qu’un rapport officiel de 1864 sur la santé publique en Angleterre s’apitoyait sur le manque de substances adipogènes facile à constater dans la plus grande partie de la classe ouvrière, un docteur Harvey, qui pourtant n’a pas inventé la circulation du sang, faisait sa fortune dans la même année avec des recettes charlatanesques qui promettaient à la bourgeoisie et à l’aristocratie de les délivrer de leur superflu de graisse.}.\par
L’expression de la valeur des marchandises en or étant tout simplement idéale, il n’est besoin pour cette opération que d’un or idéal ou qui n’existe que dans l’imagination.\par
Il n’y a pas épicier qui ne sache fort bien qu’il est loin d’avoir fait de l’or avec ses marchandises quand il a donné à leur valeur la forme prix ou la forme or en imagination, et qu’il n’a pas besoin d’un grain d’or réel pour estimer en or des millions de valeurs en marchandises. Dans sa fonction de mesure des valeurs, la monnaie n’est employée que comme monnaie idéale. Cette circonstance a donné lieu aux théories les plus folles\footnote{V. Karl Marx : \emph{Critique de l’économie politique}, etc., la partie intitulée : Théories sur l’unité de mesure de l’argent.}. Mais quoique la monnaie en tant que mesure de valeur ne fonctionne qu’idéalement et que l’or employé dans ce but ne soit par conséquent que de l’or imaginé, le prix des marchandises n’en dépend pas moins complètement de la matière de la monnaie. La valeur, c’est‑à‑dire le \emph{quantum} de travail humain qui est contenu, par exemple, dans une tonne de fer, est exprimée en imagination par le \emph{quantum} de la marchandise monnaie qui coûte précisément autant de travail. Suivant que la mesure de valeur est empruntée à l’or, à l’argent, ou au cuivre, la valeur de la tonne de fer est exprimée en prix complètement différents les uns des autres, ou bien est représentée par des quantités différentes de cuivre, d’argent ou d’or. Si donc deux marchandises différentes, l’or et l’argent, par exemple, sont employées en même temps comme mesure de valeur, toutes les marchandises possèdent deux expressions différentes pour leur prix ; elles ont leur prix or et leur prix argent qui courent tranquillement l’un à côté de l’autre, tant que le rapport de valeur de l’argent à l’or reste immuable, tant qu’il se maintient, par exemple, dans la proportion de un à quinze. Toute altération de ce rapport de valeur altère par cela même la proportion qui existe entre les prix or et les prix argent des marchandises et démontre ainsi par le fait que la fonction de mesure des valeurs est incompatible avec sa duplication\footnote{ \noindent Partout où l’argent et l’or se maintiennent légalement l’un à côte de l’autre comme monnaie, c’est‑à‑dire comme mesure de valeurs, c’est toujours en vain qu’on a essayé de les traiter comme une seule et même matière. Supposer que la même quantité de travail se matérialise immuablement dans la même proportion d’or et d’argent, c’est supposer en fait que l’argent et l’or sont la même matière et qu’un \emph{quantum} donné d’argent, du métal qui a la moindre valeur, est une fraction immuable d’un \emph{quantum} donne d’or. Depuis le règne d’Edouard III jusqu’aux temps de George II, l’histoire de l’argent en Angleterre présente une série continue de perturbations provenant de la collision entre le rapport de valeur légale de l’argent et de l’or et les oscillations de leur valeur réelle. Tantôt c’était l’or qui était estimé trop haut, tantôt c’était l’argent. Le métal estimé au‑dessous de sa valeur était dérobé à la circulation, refondu et exporté. Le rapport de valeur des deux métaux était de nouveau légalement changé ; mais, comme l’ancienne, la nouvelle valeur nominale entrait bientôt en conflit avec le rapport réel de valeur.\par
 A notre époque même, une baisse faible et passagère de l’or par rapport à l’argent, provenant d’une demande d’argent dans l’Inde et dans la Chine, a produit en France le même phénomène sur la plus grande échelle, exportation de l’argent et son remplacement par l’or dans la circulation. Pendant les années 1855, 1856 et 1857, l’importation de l’or en France dépassa son exportation de quarante et un millions cinq cent quatre‑vingt mille livres sterling, tandis que l’exportation de l’argent dépassa son importation de quatorze millions sept cent quarante mille. En fait, dans les pays comme la France où les deux métaux sont des mesures de valeurs légales et ont tous deux un cours forcé, de telle sorte que chacun peut payer à volonté soit avec l’un, soit avec l’autre, le métal en hausse porte un agio et mesure son prix, comme toute autre marchandise, dans le métal surfait, tandis que ce dernier est employé seul comme mesure de valeur. L’expérience fournie par l’histoire à ce sujet se réduit tout simplement à ceci, que là où deux marchandises remplissent légalement la fonction de mesure de valeur, il n’y en a en fait qu’une seule qui se maintienne à ce poste. (Karl Marx, l. c., p. 52, 53.)
}.\par
Les marchandises dont le prix est déterminé, se présentent toutes sous la forme : \emph{a} marchandise A = \emph{x} or ; \emph{b} marchandise B = \emph{z} or ; \emph{c} marchandise C = \emph{y} or, etc., dans laquelle \emph{a, b, c}, sont des quantités déterminées des espèces de marchandises A, B, C ; \emph{x, z, y}, des quantités d’or déterminées également. En tant que grandeurs de la même dénomination, ou en tant que quantités différentes d’une même chose, l’or, elles se comparent et se mesurent entre elles, et ainsi se développe la nécessité technique de les rapporter à un \emph{quantum} d’or fixé et déterminé comme unité de mesure. Cette unité de mesure se développe ensuite elle-même et devient étalon par sa division en parties aliquotes. Avant de devenir monnaie, l’or, l’argent, le cuivre possèdent déjà dans leurs mesures de poids des étalons de ce genre, de telle sorte que la livre, par exemple, sert d’unité de mesure, unité qui se subdivise ensuite en onces, etc., et s’additionne en quintaux et ainsi de suite\footnote{Ce fait étrange que l’unité de mesure de la monnaie anglaise, l’once d’or, n’est pas subdivisée en parties aliquotes, s’explique de la manière suivante : « A l’origine notre monnaie était adaptée exclusivement à l’argent, et c’est pour cela qu’une once d’argent peut toujours être divisée dans un nombre de pièces aliquotes ; mais l’or n’ayant été introduit qu’à une période postérieure dans un système de monnayage exclusivement adapté à l’argent, une once d’or ne saurait pas être monnayée en un nombre de pièces aliquotes. » (Maclaren : \emph{History of the Currency}, etc., p. 16. London, 1858.)}. Dans toute circulation métallique, les noms préexistants de l’étalon de poids forment ainsi les noms d’origine de l’étalon monnaie.\par
Comme mesure des valeurs et comme étalon des prix, l’or remplit deux fonctions entièrement différentes. Il est mesure des valeurs en tant qu’équivalent général, étalon des prix en tant que poids de métal fixe. Comme mesure de valeur il sert à transformer les valeurs des marchandises en prix, en quantités d’or imaginées. Comme étalon des prix il mesure ces quantités d’or données contre un \emph{quantum} d’or fixe et subdivisé en parties aliquotes. Dans la mesure des valeurs, les marchandises expriment leur valeur propre : l’étalon des prix ne mesure au contraire que des \emph{quanta} d’or contre un \emph{quantum} d’or et non la valeur d’un \emph{quantum} d’or contre le poids d’un autre. Pour l’étalon des prix, il faut qu’un poids d’or déterminé soit fixé comme unité de mesure. Ici comme dans toutes les déterminations de mesure entre grandeurs de même nom, la fixité de l’unité de mesure est chose d’absolue nécessité. L’étalon des prix remplit donc sa fonction d’autant mieux que l’unité de mesure et ses subdivisions sont moins sujettes au changement. De l’autre côté, l’or ne peut servir de mesure de valeur, que parce qu’il est lui‑même un produit du travail, c’est‑à‑dire une valeur variable.\par
Il est d’abord évident qu’un changement dans la valeur de l’or n’altère en rien sa fonction comme étalon des prix. Quels que soient les changements de la valeur de l’or, différentes quantités d’or restent toujours dans le même rapport les unes avec les autres. Que cette valeur tombe de cent pour cent, douze onces d’or vaudront après comme avant douze fois plus qu’une once, et dans les prix il ne s’agit que du rapport de diverses quantités d’or entre elles. Dun autre côté, attendu qu’une once d’or ne change pas le moins du monde de poids par suite de la hausse ou de la baisse de sa valeur, le poids de ses parties aliquotes ne change pas davantage ; il en résulte que l’or comme étalon fixe des prix, rend toujours le même service de quelque façon que sa valeur change.\par
Le changement de valeur de l’or ne met pas non plus obstacle à sa fonction comme mesure de valeur. Ce changement atteint toutes les marchandises à la fois et laisse par conséquent, \emph{cœteris paribus}, leurs quantités relatives de valeur réciproquement dans le même état\footnote{« L’argent peut continuellement changer de valeur et néanmoins servir de mesure de valeur aussi bien que s’il restait parfaitement stationnaire. » (Bailey : \emph{Money and its vicissitudes}. London, 1837, p. 11.)}.\par
Dans l’estimation en or des marchandises, on suppose seulement que la production d’un quantum déterminé d’or coûte, à une époque donnée, un \emph{quantum} donné de travail. Quant aux fluctuations des prix des marchandises, elles sont réglées par les lois de la valeur relative simple développées plus haut.\par
Une hausse générale des prix des marchandises exprime une hausse de leurs valeurs, si la valeur de l’argent reste constante, et une baisse de la valeur de l’argent si les valeurs des marchandises ne varient pas. Inversement, une baisse générale des prix des marchandises exprime une baisse de leurs valeurs si la valeur de l’argent reste constante et une hausse de la valeur de l’argent si les valeurs des marchandises restent les mêmes. Il ne s’ensuit pas le moins du monde qu’une hausse de la valeur de l’argent entraîne une baisse proportionnelle des prix des marchandises et une baisse de la valeur de l’argent une hausse proportionnelle des prix des marchandises. Cela n’a lieu que pour des marchandises de valeur immuable. Les marchandises, par exemple, dont la valeur monte et baisse en même temps et dans la même mesure que la valeur de l’argent, conservent les mêmes prix. Si la hausse ou la baisse de leur valeur s’opère plus lentement ou plus rapidement que celles de la valeur de l’argent, le degré de hausse ou de baisse de leur prix dépend de la différence entre la fluctuation de leur propre valeur et celle de l’argent, etc.\par
Revenons à l’examen de la forme prix.\par
On va vu que l’étalon en usage pour les poids des métaux sert aussi avec son nom et ses subdivisions comme étalon des prix. Certaines circonstances historiques amènent pourtant des modifications ; ce sont notamment :\par

\begin{enumerate}[itemsep=0pt,]
\item l’introduction d’argent étranger chez des peuples moins développés, comme lorsque, par exemple, des monnaies d’or et d’argent circulaient dans l’ancienne Rome comme marchandises étrangères. Les noms de cette monnaie étrangère diffèrent des noms de poids indigènes ;
\item le développement de la richesse qui remplace dans sa fonction de mesure des valeurs le métal le moins précieux par celui qui l’est davantage, le cuivre par l’argent et ce dernier par l’or, bien que cette succession contredise la chronologie poétique. Le mot livre était, par exemple, le nom de monnaie employé pour une véritable livre d’argent. Dès que l’or. remplace l’argent comme mesure de valeur, le même nom s’attache peut‑être à un quinzième de livre d’or suivant la valeur proportionnelle de l’or et de l’argent. Livre comme nom de monnaie et livre comme nom ordinaire de poids d’or, sont maintenant distincts\footnote{Les monnaies qui sont aujourd’hui idéales, sont les plus anciennes de toute nation, et toutes étaient à une certaine période réelles (cette dernière assertion n’est pas juste dans une aussi large mesure), et parce qu’elles étaient réelles, elles servaient de monnaie de compte. » (Galiani, l. c., p. 153.)};
\item la falsification de l’argent par les rois et roitelets prolongée pendant des siècles, falsification qui du poids primitif des monnaies d’argent n’a en fait conservé que le nom\footnote{C’est ainsi que la livre anglaise ne désigne à peu près que ¼ de son poids primitif, la livre écossaise avant l’Union de 1701 1/36 seulement, la livre française 1/94, le maravédi espagnol moins de 1/100, le réis portugais une fraction encore bien plus petite. M. David Urquhart remarque dans ses « \emph{Familiar Words} », à propos de ce fait qui le terrifie, que la livre anglaise (l. st.) comme unité de mesure monétaire ne vaut plus que ¼ d’once d’or : « C’est falsifier une mesure et non pas établir un étalon. » Dans cette fausse dénomination de l’étalon monétaire il voit, comme partout, la main falsificatrice de la civilisation.}.
\end{enumerate}

\noindent La séparation entre le nom monétaire et le nom ordinaire des poids de métal est devenue une habitude populaire par suite de ces évolutions historiques. L’étalon de la monnaie étant d’un côté purement conventionnel et de l’autre ayant besoin de validité sociale, c’est la loi qui le règle en dernier lieu. Une partie de poids déterminée du métal précieux, une once d’or, par exemple, est divisée officiellement en parties aliquotes qui reçoivent des noms de baptême légaux tels que livre, écu, etc. Une partie aliquote de ce genre employée alors comme unité de mesure proprement dite, est à son tour subdivisée en d’autres parties ayant chacune leur nom légal. Shilling, Penny, etc\footnote{Dans différents pays, l’étalon légal des prix est naturellement différent. En Angleterre, par exemple, l’once comme poids de métal est divisée en Pennyweights, Grains et Karats Troy ; mais l’once comme unité de mesure monétaire est divisée en 37/8 sovereigns, le sovereign en 20 shillings, le shilling en 12 pence, de sorte que 100 livres d’or à 22 karats (1 200 onces) T 4 672 sovereigns et 10 shillings.}. Après comme avant ce sont des poids déterminés de métal qui restent étalons de la monnaie métallique. Il n’y a de changé que la subdivision et la nomenclature.\par
Les prix ou les \emph{quanta} d’or, en lesquels sont transformées idéalement les marchandises, sont maintenant exprimés par les noms monétaires de l’étalon d’or. Ainsi, au lieu de dire, le quart de froment est égal à une once d’or, on dirait en Angleterre : il est égal à trois livres sterling dix‑sept shillings dix pence et demi. Les marchandises se disent dans leurs noms d’argent ce qu’elles valent, et la monnaie sert comme monnaie de compte toutes les fois qu’il s’agit de fixer une chose comme valeur, et par conséquent sous forme monnaie\footnote{« Comme on demandait à Anacharsis, de quel usage était l’argent chez les Grecs, il répondit : ils s’en servent pour compter. » (Athenæus, Deipn., I, IV.)}.\par
Le nom d’une chose est complètement étranger à sa nature. Je ne sais rien d’un homme quand je sais qu’il s’appelle Jacques. De même, dans les noms d’argent : livre, thaler, franc, ducat, etc., disparaît toute trace du rapport de valeur. L’embarras et la confusion causés par le sens que l’on croit caché sous ces signes cabalistiques sont d’autant plus grands que les noms monétaires expriment en même temps la valeur des marchandises et des parties aliquotes d’un poids d’or\footnote{L’or possédant comme étalon des prix les mêmes noms que les prix des marchandises, et de plus étant monnayé suivant les parties aliquotes de l’unité de mesure, que ces noms désignent, de l’once, par exemple, de sorte qu’une once d’or peut être exprimée tout aussi bien que le prix d’une tonne de fer par 3 l. 17 s. 10 ½ d., on a donné à ces expressions le nom de prix de monnaie. C’est ce qui a fait naître l’idée merveilleuse que l’or pouvait être estimé en lui-même, sans comparaison avec aucune autre marchandise, et qu’à la différence de toutes les autres marchandises il recevait de l’Etat un prix fixe. On a confondu la fixation des noms de monnaie de compte pour des poids d’or déterminés avec la fixation de la valeur de ce poids. La littérature anglaise possède d’innombrables écrits dans lesquels ce quiproquo est délayé à l’infini. lis ont inoculé la même folie à quelques auteurs de l’autre côte du détroit.}. D’un autre côté, il est nécessaire que la valeur, pour se distinguer des corps variés des marchandises, revête cette forme bizarre, mais purement sociale\footnote{Comparez « \emph{Théories sur l’unité de mesure de l’argent} » dans l’ouvrage déjà cité (\emph{Critique de l’économie politique}, p. 53 et suiv.). ‑ Les fantaisies à propos de l’élévation ou de l’abaissement du « prix de monnaie » qui consistent de la part de l’Etat à donner les noms légaux déjà fixés pour des poids déterminés d’or ou d’argent à des poids supérieurs ou inférieurs, c’est‑à‑dire par exemple, à frapper 14 d’once d’or en 40 shillings au lieu de 20, de telles fantaisies, en tant qu’elles ne sont point de maladroites opérations financières contre les créanciers de l’Etat ou des particuliers, mais ont pour but d’opérer des « cures merveilleuses » économiques, ont été traitées d’une manière si complète par W. Petty, dans son ouvrage : « \emph{Quantulumcumque concerning money. To the Lord Marquis of Halifax »}, 1682, que ses successeurs immédiats, Sir Dudley North et John Locke, pour ne pas parler des plus récents, n’ont pu que délayer et affaiblir ses explications. « Si la richesse d’une nation pouvait être décuplée par de telles proclamations, il serait étrange que nos maîtres ne les eussent pas faites depuis longtemps », dit‑il entre autres, l. c., p. 36.}.\par
Le prix est le nom monétaire du travail réalisé dans la marchandise. L’équivalence de la marchandise et de la somme d’argent, exprimée dans son prix, est donc une tautologie\footnote{« Ou bien il faut consentir à dire qu’une valeur d’un million en argent vaut plus qu’une valeur égale en marchandises » (Le Trosne, l. c., p, 922), ainsi qu’une valeur vaut plus qu’une valeur égale.}, comme en général l’expression relative de valeur d’une marchandise est toujours l’expression de l’équivalence de deux marchandises. Mais si le prix comme exposant de la grandeur de valeur de la marchandise est l’exposant de son rapport d’échange avec la monnaie, il ne s’ensuit pas inversement que l’exposant de son rapport d’échange avec la monnaie soit nécessairement l’exposant de sa grandeur de valeur. Supposons qu’un quart de froment se produise dans le même temps de travail que deux onces d’or, et que deux livres sterling soient le nom de deux onces d’or. Deux livres sterling sont alors l’expression monnaie de la valeur du quart de froment, ou son prix. Si maintenant les circonstances permettent d’estimer le quart de froment à trois livres sterling, ou forcent de l’abaisser à une livre sterling, dès lors une livre sterling et trois livres sterling sont des expressions qui diminuent ou exagèrent la valeur du froment, mais elles restent néanmoins ses prix, car premièrement elles sont sa forme monnaie et secondement elles sont les exposants de son rapport d’échange avec la monnaie. Les conditions de production ou la force productive du travail demeurant constantes, la reproduction du quart de froment exige après comme avant la même dépense en travail. Cette circonstance ne dépend ni de la volonté du producteur de froment ni de celle des possesseurs des autres marchandises. La grandeur de valeur exprime donc un rapport de production, le lien intime qu’il y a entre un article quelconque et la portion du travail social qu’il faut pour lui donner naissance. Dès que la valeur se transforme en prix, ce rapport nécessaire apparaît comme rapport d’échange d’une marchandise usuelle avec la marchandise monnaie qui existe en dehors d’elle. Mais le rapport d’échange peut exprimer ou la valeur même de la marchandise, ou le plus ou le moins que son aliénation, dans des circonstances données, rapporte accidentellement. Il est donc possible qu’il y ait un écart, une différence quantitative entre le prix d’une marchandise et sa grandeur de valeur, et cette possibilité gît dans la forme prix elle‑même. C’est une ambiguïté, qui au lieu de constituer un défaut, est au contraire, une des beautés de cette forme, parce qu’elle l’adapte à un système de production où la règle ne fait loi que par le jeu aveugle des irrégularités qui, en moyenne, se compensent, se paralysent et se détruisent mutuellement.\par
La forme prix n’admet pas seulement la possibilité d’une divergence quantitative entre le prix et la grandeur de valeur, c’est‑à-dire entre cette dernière et sa propre expression monnaie, mais encore elle peut cacher une contradiction absolue, de sorte que le prix cesse tout à fait d’exprimer de la valeur, quoique l’argent ne soit que la forme valeur des marchandises. Des choses qui, par elles‑mêmes, ne sont point des marchandises, telles que, par exemple, l’honneur, la conscience, etc., peuvent devenir vénales et acquérir ainsi par le prix qu’on leur donne la forme marchandise. Une chose peut donc avoir un prix formellement sans avoir une valeur. Le prix devient ici une expression imaginaire comme certaines grandeurs en mathématiques. D’un autre côté, la forme prix imaginaire, comme par exemple le prix du sol non cultivé, qui n’a aucune valeur, parce qu’aucun travail humain n’est réalisé en lui, peut cependant cacher des rapports de valeur réels, quoique indirects.\par
De même que la forme valeur relative en général, le prix exprime la valeur d’une marchandise, par exemple, d’une tonne de fer, de cette façon qu’une certaine quantité de l’équivalent, une once d’or, si l’on veut, est immédiatement échangeable avec le fer, tandis que l’inverse n’a pas lieu ; le fer, de son côté, n’est pas immédiatement échangeable avec l’or.\par
Dans le prix, c’est‑à‑dire dans le nom monétaire des marchandises, leur équivalence avec l’or est anticipée, mais n’est pas encore un fait accompli. Pour avoir pratiquement l’effet d’une valeur d’échange, la marchandise doit se débarrasser de son corps naturel et se convertir d’or simplement imaginé en or réel, bien que cette transsubstantiation puisse lui coûter plus de peine qu’à « l’Idée » hégélienne son passage de la nécessité à la liberté, au crabe la rupture de son écaille, au Père de l’église Jérôme, le dépouillement du vieil Adam\footnote{Si dans sa jeunesse saint Jérôme avait beaucoup à lutter contre la chair matérielle, parce que des images de belles femmes obsédaient sans cesse son imagination, il luttait de même dans sa vieillesse contre la chair spirituelle. Je me figurai, dit‑il, par exemple, en présence du souverain juge. « Qui es‑tu ? » Je suis un chrétien. « Non, tu mens, répliqua le juge d’une voix de tonnerre\emph{, tu n’es qu’un Cicéronien}. »}. A côté de son apparence réelle, celle de fer, par exemple, la marchandise peut posséder dans son prix une apparence idéale ou une‑apparence d’or imaginé ; mais elle ne peut être en même temps fer réel et or réel. Pour lui donner un prix, il suffit de la déclarer égale à de l’or purement idéal ; mais il faut la remplacer par de l’or réel, pour qu’elle rende à celui qui la possède le service d’équivalent général. Si le possesseur du fer, s’adressant au possesseur d’un élégant article de Paris, lui faisait valoir le prix du fer sous prétexte qu’il est forme argent, il en recevrait la réponse que saint Pierre dans le paradis adresse à Dante qui venait de lui réciter les formules de la foi :\par

\begin{center}
\noindent \centerline{« \emph{Assai bene è trascorsa}}\par
\end{center}

\noindent Desta moneta già la lega e'l peso,\par

\begin{center}
\noindent \centerline{\emph{Ma dimmi se tu l’hai nella tua borsa}\footnote{« L’alliage et le poids de cette monnaie sont très bien examinés, mais, dis‑moi, l’as‑tu dans ta bourse ? »}. »}\par
\end{center}

\bigbreak
\noindent La forme prix renferme en elle‑même l’aliénabilité des marchandises contre la monnaie et la nécessité de cette aliénation. D’autre part, l’or ne fonctionne comme mesure de valeur idéale que parce qu’il se trouve déjà sur le marché à titre de marchandise monnaie. Sous son aspect tout idéal de mesure des valeurs se tient donc déjà aux aguets l’argent réel, les espèces sonnantes.
\subsection[{1.1.3.2. Moyen de circulation}]{1.1.3.2. Moyen de circulation}
\subsubsection[{1.1.3.2.1. La métamorphose des marchandises.}]{1.1.3.2.1. La métamorphose des marchandises.}
\noindent L’échange des marchandises ne peut, comme on l’a vu, s’effectuer qu’en remplissant des conditions contradictoires, exclusives les unes des autres. Son développement qui fait apparaître la marchandise comme chose à double face, valeur d’usage et valeur d’échange, ne fait pas disparaître ces contradictions, mais crée la forme dans laquelle elles peuvent se mouvoir. C’est d’ailleurs la seule méthode pour résoudre des contradictions réelles. C’est par exemple une contradiction qu’un corps tombe constamment sur un autre et cependant le fuie constamment. L’ellipse est une des formes de mouvement par lesquelles cette contradiction se réalise et se résout à la fois.\par
L’échange fait passer les marchandises des mains dans lesquelles elles sont des non‑valeurs d’usage aux mains dans lesquelles elles servent de valeurs d’usage. Le produit d’un travail utile remplace le produit d’un autre travail utile. C’est la circulation sociale des matières. Une fois arrivée au lieu où elle sert de valeur d’usage, la marchandise tombe de la sphère des échanges dans la sphère de consommation. Mais cette circulation matérielle ne s’accomplit que par une série de changements de forme ou une métamorphose de la marchandise que nous avons maintenant à étudier.\par
Ce côté morphologique du mouvement est un peu difficile à saisir, puisque tout changement de forme d’une marchandise s’effectue par l’échange de deux marchandises. Une marchandise dépouille, par exemple, sa forme usuelle pour revêtir sa forme monnaie. Comment cela arrive‑t‑il ? Par son échange avec l’or. Simple échange de deux marchandises, voilà le fait palpable ; mais il faut y regarder de plus près.\par
L’or occupe un pôle, tous les articles utiles le pôle opposé.\par
Des deux côtés, il y a marchandise, unité de valeur d’usage et de valeur d’échange. Mais cette unité de contraires se représente inversement aux deux extrêmes. La forme usuelle de la marchandise en est la forme réelle, tandis que sa valeur d’échange n’est exprimée qu’idéalement, en or imaginé, par son prix. La forme naturelle, métallique de l’or est au contraire sa forme d’échangeabilité générale, sa forme valeur, tandis que sa valeur d’usage n’est exprimée qu’idéalement dans la série des marchandises qui figurent comme ses équivalents. Or, quand une marchandise s’échange contre de l’or, elle change du même coup sa forme usuelle en forme valeur. Quand l’or s’échange contre une marchandise, il change de même sa forme valeur en forme usuelle.\par
Après ces remarques préliminaires, transportons‑nous maintenant sur le théâtre de l’action ‑ le marché. Nous y accompagnons un échangiste quelconque, notre vieille connaissance le tisserand, par exemple. Sa marchandise, vingt mètres de toile, a un prix déterminé, soit de deux livres sterling. Il l’échange contre deux livres sterling, et puis, en homme de vieille roche qu’il est, échange les deux livres sterling contre une bible d’un prix égal. La toile qui, pour lui, n’est que marchandise, porte‑valeur, est aliénée contre l’or, et cette figure de sa valeur est aliénée de nouveau contre une autre marchandise, la bible. Mais celle-ci entre dans la maisonnette du tisserand pour y servir de valeur d’usage et y porter réconfort à des âmes modestes.\par
L’échange ne s’accomplit donc pas sans donner lieu à deux métamorphoses opposées et qui se complètent l’une l’autre transformation de la marchandise en argent et sa retransformation d’argent en marchandise\footnote{« Eχ δε τού… προς άντάμείσόερθαι παιν δ Ήράχλειτοζ, Χαί πΰρ άπάντων, ώσπερ χρυζοΰ Χρήμάτα χαι χρήμάτων Χρυσόζ » F. Lassalle, \emph{La philosophie d’Héraclite l’obscur.} Berlin, 1858, t. I, p. 222. « Le feu, comme dit Héraclite, se convertit en tout, et tout se convertit en eu, de même que les marchandises en or et l’or en marchandises. »}. ‑ Ces deux métamorphoses de la marchandise présentent à la fois, au point de vue de son possesseur, deux actes ‑ vente, échange de la marchandise contre l’argent ; ‑ achat, échange de l’argent contre la marchandise ‑ et l’ensemble de ces deux actes : vendre pour acheter.\par
Ce qui résulte pour le tisserand de cette affaire, c’est qu’il possède maintenant une bible et non de la toile, à la place de sa première marchandise une autre d’une valeur égale, mais d’une utilité différente. Il se procure de la même manière ses autres moyens de subsistance et de production. De son point de vue, ce mouvement de vente et d’achat ne fait en dernier lieu que remplacer une marchandise par une autre ou qu’échanger des produits.\par
L’échange de la marchandise implique donc les changements de forme que voici :\par

\begin{center}
\noindent \centerline{ Marchandise – Argent – Marchandise \\
M – A – M }\par
\end{center}

\noindent Considéré sous son aspect purement matériel, le mouvement aboutit à M ‑ M, échange de marchandise contre marchandise, permutation de matières du travail social. Tel est le résultat dans lequel vient s’éteindre le phénomène.\par
Nous aurons maintenant à examiner à part chacune des deux métamorphoses successives que la marchandise doit traverser.\par
\textbf{M. ‑ A. Première métamorphose de la marchandise ou vente.} La valeur de la marchandise saute de son propre corps dans celui de l’or. C’est son saut périlleux. S’il manque, elle ne s’en portera pas plus mal, mais son possesseur sera frustré. Tout en multipliant ses besoins, la division sociale du travail a du même coup rétréci sa capacité productive. C’est précisément pourquoi son produit ne lui sert que de valeur d’échange ou d’équivalent général. Toutefois, il n’acquiert cette forme qu’en se convertissant en argent et l’argent se trouve dans la poche d’autrui. Pour le tirer de là, il faut avant tout que la marchandise soit valeur d’usage pour l’acheteur, que le travail dépensé en elle l’ait été sous une forme socialement utile ou qu’il soit légitimé comme branche de la division sociale du travail. Mais la division du travail crée un organisme de production spontané dont les fils ont été tissés et se tissent encore à l’insu des producteurs échangistes. Il se peut que la marchandise provienne d’un nouveau genre de travail destiné à satisfaire ou même à provoquer des besoins nouveaux. Entrelacé, hier encore, dans les nombreuses fonctions dont se compose un seul métier, un travail parcellaire peut aujourd’hui se détacher de cet ensemble, s’isoler et envoyer au marché son produit partiel à titre de marchandise complète sans que rien garantisse que les circonstances soient mûres pour ce fractionnement.\par
Un produit satisfait aujourd’hui un besoin social ; demain, il sera peut-être remplacé en tout ou en partie par un produit rival. Lors même que le travail, comme celui de notre tisserand, est un membre patenté de la division sociale du travail, la valeur d’usage de ses vingt mètres de toile n’est pas pour cela précisément garantie. Si le besoin de toile dans la société, et ce besoin a sa mesure comme toute autre chose, est déjà rassasié par des tisserands rivaux, le produit de notre ami devient superflu et conséquemment inutile. Supposons cependant que la valeur utile de son produit soit constatée et que l’argent soit attiré par la marchandise. Combien d’argent ? Telle est maintenant la question. Il est vrai que la réponse se trouve déjà par anticipation dans le prix de la marchandise, l’exposant de sa grandeur de valeur. Nous faisons abstraction du côté faible du vendeur, de fautes de calcul plus ou moins intentionnelles, lesquelles sont sans pitié corrigées sur le marché. Supposons qu’il n’ait dépensé que le temps socialement nécessaire pour faire son produit. Le prix de sa marchandise n’est donc que le nom monétaire du quantum de travail qu’exige en moyenne tout article de la même sorte. Mais à l’insu et sans la permission de notre tisserand, les vieux procédés employés pour le tissage ont été mis sens dessus-dessous ; le temps de travail socialement nécessaire hier pour la production d’un mètre de toile ne l’est plus aujourd’hui ; comme l’homme aux écus s’empresse de le lui démontrer par le tarif de ses concurrents. Pour son malheur, il y a beaucoup de tisserands au monde.\par
Supposons enfin que chaque morceau de toile qui se trouve sur le marché n’ait coûté que le temps de travail socialement nécessaire. Néanmoins, la somme totale de ces morceaux peut représenter du travail dépensé en pure perte. Si l’estomac du marché ne peut pas absorber toute la toile au prix normal de deux shillings par mètre, cela prouve qu’une trop grande partie du travail social a été dépensée sous forme de tissage. L’effet est le même que si chaque tisserand en particulier avait employé pour son produit individuel plus que le travail nécessaire socialement. C’est le cas de dire ici, selon le proverbe allemand : « Pris ensemble, ensemble pendus. » Toute la toile sur le marché ne constitue qu’un seul article de commerce dont chaque morceau n’est qu’une partie aliquote.\par
Comme on le voit, la marchandise aime l’argent, mais « the course of true love runs never smooth\footnote{« Le véritable amour est toujours cahoté dans sa course. » (Shakespeare.)} ». L’organisme social de production, dont les membres disjoints ‑ membra disjecta ‑ naissent de la division du travail, porte l’empreinte de la spontanéité et du hasard, que l’on considère ou les fonctions mêmes de ses membres ou leurs rapports de proportionnalité. Aussi nos échangistes découvrent‑ils que la même division du travail, qui fait d’eux des producteurs privés indépendants, rend la marche de la production sociale, et les rapports qu’elle crée, complètement indépendants de leurs volontés, de sorte que l’indépendance des personnes les unes vis‑à‑vis des autres trouve son complément obligé en un système de dépendance réciproque, imposée par les choses.\par
La division du travail transforme le produit du travail en marchandise, et nécessite par cela même sa transformation en argent. Elle rend en même temps la réussite de cette transsubstantiation accidentelle. Ici cependant nous avons à considérer le phénomène dans son intégrité, et nous devons donc supposer que sa marche est normale. Du reste, si la marchandise n’est pas absolument invendable, son changement de forme a toujours lieu quel que soit son prix de vente.\par
Ainsi, le phénomène qui, dans l’échange, saute aux yeux, c’est que marchandise et or, vingt mètres de toile par exemple, et deux livres sterling, changent de main ou de place. Mais avec quoi s’échange la marchandise ? Avec sa forme de valeur d’échange ou d’équivalent général. Et avec quoi l’or ? Avec une forme particulière de sa valeur d’usage. Pourquoi l’or se présente‑t‑il comme monnaie à la toile ? Parce que le nom monétaire de la toile, son prix de deux livres sterling, la rapporte déjà à l’or en tant que monnaie. La marchandise se dépouille de sa forme primitive en s’aliénant, c’est‑à‑dire au moment où sa valeur d’usage attire réellement l’or qui n’est que représenté dans son prix.\par
La \emph{réalisation du prix} ou de la forme valeur purement idéale de la marchandise est en même temps la réalisation inverse de la valeur d’usage purement idéale de la monnaie. La transformation de la marchandise en argent est la transformation simultanée de l’argent en marchandise. La même et unique transaction est bipolaire ; vue de l’un des pôles, celui du possesseur de marchandise, elle est vente ; vue du pôle opposé, celui du possesseur d’or, elle est achat. Ou bien \emph{vente est achat}, M.‑A. est en même temps A.‑M.\footnote{« Toute vente est achat. » (Dr Quesnay, \emph{Dialogues sur le commerce et les travaux des artisans. Physiocrates}, éd. Daire, I° partie, Paris, 1846, p. 170), ou, comme le dit le même auteur, dans ses \emph{Maximes générales} : Vendre est acheter.}.\par
Jusqu’ici nous ne connaissons d’autre rapport économique entre les hommes que celui d’échangistes, rapport dans lequel ils ne s’approprient le produit d’un travail étranger qu’en livrant. le leur. Si donc l’un des échangistes se présente à l’autre comme possesseur de monnaie, il faut de deux choses l’une : Ou le produit de son travail possède par nature la forme monnaie, c’est-à‑dire que son produit à lui est or, argent, etc., en un mot, matière de la monnaie ; ou sa marchandise a déjà changé de peau, elle a été vendue, et par cela même elle a dépouillé sa forme primitive. Pour fonctionner en qualité de monnaie, l’or doit naturellement se présenter sur le marché en un point quelconque. Il entre dans le marché à la source même de sa production, c’est‑à‑dire là où il se troque comme produit immédiat du travail contre un autre produit de même valeur.\par
Mais à partir de cet instant, il représente toujours un \emph{prix de marchandise réalisé}\footnote{« Le prix d’une marchandise ne pouvant être payé que par le prix d’une autre marchandise. » (Mercier de la Rivière : \emph{l’Ordre naturel et essentiel (les sociétés politiques. Physiocrates}, éd. Daire, II° partie, p. 554.)}. Indépendamment du troc de l’or contre des marchandises, à sa source de production, l’or est entre les mains de chaque producteur‑échangiste le produit d’une vente ou de la première métamorphose de sa marchandise, M.‑A.\footnote{« Pour avoir cet argent, il faut avoir vendu. » (L. c., p. 545.)}. L’or est devenu monnaie idéale ou mesure des valeurs, parce que les marchandises exprimaient leurs valeurs en lui et en faisaient ainsi leur figure valeur imaginée, opposée à leurs formes naturelles de produits utiles. Il devient monnaie réelle par l’aliénation universelle des marchandises. Ce mouvement les convertit toutes en or, et fait par cela même de l’or leur figure métamorphosée, non plus en imagination, mais en réalité. La dernière trace de leurs formes usuelles et des travaux concrets dont elles tirent leur origine ayant ainsi disparu, il ne reste plus que des échantillons uniformes et indistincts du même travail social. A voir une pièce de monnaie on ne saurait dire quel article a été converti en elle. La monnaie peut donc être de la boue, quoique la boue ne soit pas monnaie.\par
Supposons maintenant que les deux pièces d’or contre lesquelles notre tisserand a aliéné sa marchandise proviennent de la métamorphose d’un quart de froment. La vente de la toile, M.‑A. est en même temps son achat, A‑M En tant que la toile est vendue, cette marchandise commence un mouvement qui finit par son contraire, \emph{l’achat de la bible ;} en tant que la \emph{toile} est achetée, elle finit un mouvement qui a commencé par son contraire, la vente du froment. M.‑A. (toile‑argent), cette première phase de M.‑A.‑M. (toile‑argent‑bible), est en même temps A.‑M. (argent‑toile), la dernière phase d’un autre mouvement M.‑A.‑M. (froment‑argent‑toile). La \emph{première métamorphose d’une marchandise}, son passage de la forme marchandise à la forme argent est toujours \emph{seconde métamorphose} tout \emph{opposée d’une autre marchandise}, son retour de la forme argent à la forme marchandise\footnote{Ici, comme nous l’avons déjà fait remarquer, le producteur d’or ou d’argent fait exception : il vend son produit sans avoir préalablement acheté.}.\par
\textbf{\emph{A.‑M. Métamorphose deuxième et finale. ‑ Achat.}} L’argent est la marchandise qui a pour caractère l’aliénabilité absolue, parce qu’il est le produit de l’aliénation universelle de toutes les autres marchandises. Il lit tous les prix à rebours et se mire ainsi dans les corps de tous les produits, comme dans la matière qui se donne à lui pour qu’il devienne valeur d’usage lui‑même. En même temps, les prix, qui sont pour ainsi dire les œillades amoureuses que lui lancent les marchandises, indiquent la limite de sa faculté de conversion, c’est‑à‑dire sa propre quantité. La marchandise disparaissant dans l’acte de sa conversion en argent, l’argent dont dispose un particulier ne laisse entrevoir ni comment il est tombé sous sa main ni quelle chose a été transformée en lui. Impossible de sentir, \emph{non olet}, d’où il tire son origine. Si d’un côté, il représente des marchandises vendues, il représente de l’autre des marchandises à acheter\footnote{« Si l’argent représente, dans nos mains, les choses que nous pouvons désirer d’\emph{acheter}, il y représente aussi les choses que nous avons \emph{vendues} pour cet argent. » (Mercier de la Rivière, l. c., p. 586.)}.\par
A.‑M., l’achat, est en même temps vente, M.‑A., la dernière métamorphose d’une marchandise, la première d’une autre. Pour notre tisserand, la carrière de sa marchandise se termine à la bible, en laquelle il a converti ses deux livres sterling. Mais le vendeur de la bible dépense cette somme en eau‑de-vie.\par
A.‑M., la dernière phase de M.‑A.‑M. (toile‑argent‑bible) est en même temps M.‑A., la première phase de M.-A.-M. (bible‑argent‑eau‑de‑vie).\par
La division sociale du travail restreint chaque producteur-échangiste à la confection d’un article spécial qu’il vend souvent en gros. De l’autre côté, ses besoins divers et toujours renaissants le forcent d’employer l’argent ainsi obtenu à des achats plus ou moins nombreux. Une seule vente devient le point de départ d’achats divers. La métamorphose finale d’une marchandise forme ainsi une somme de métamorphoses premières d’autres marchandises.\par
Examinons maintenant la métamorphose complète, l’ensemble des deux mouvements M.‑A. et A.‑M. Ils s’accomplissent par deux transactions inverses de l’échangiste, la vente et l’achat, qui lui impriment le double caractère de vendeur et d’acheteur. De même que dans chaque changement de forme de la marchandise, ses deux formes, marchandise et argent, existent simultanément, quoique à des pôles opposés, de même dans chaque transaction de vente et d’achat les deux formes de l’échangiste, vendeur et acheteur, se font face. De même qu’une marchandise, la toile par exemple, subit alternativement deux transformations inverses, de marchandise devient argent et d’argent marchandise, de même son possesseur joue alternativement sur le marché les rôles de vendeur et d’acheteur. Ces caractères, au lieu d’être des attributs fixes, passent donc tour à tour d’un échangiste à l’autre.\par
La métamorphose complète d’une marchandise suppose dans sa forme la plus simple quatre termes. Marchandise et argent, possesseur de marchandise et possesseur d’argent, voilà les deux extrêmes qui se font face deux fois. Cependant un des échangistes intervient d’abord dans son rôle de vendeur, possesseur de marchandise, et ensuite dans son rôle d’acheteur, possesseur d’argent. Il n’y a donc que trois \emph{persona, dramatis}\footnote{« Il y a donc quatre termes et trois contractants, dont l’un intervient deux fois. » (Le Trosne, l. c., p. 908.)}. Comme terme final de la première métamorphose, l’argent est en même temps le point de départ de la seconde. De même, le vendeur du premier acte devient l’acheteur dans le second, où un troisième possesseur de marchandise se présente à lui comme vendeur.\par
Les deux mouvements inverses de la métamorphose d’une marchandise décrivent un cercle : forme marchandise, effacement de cette forme dans l’argent, retour à la forme marchandise.\par
Ce cercle commence et finit par la forme marchandise. Au point de départ, elle s’attache à un produit qui est non‑valeur d’usage pour son possesseur, au point de retour à un autre produit qui lui sert de valeur d’usage. Remarquons encore que l’argent aussi joue là un double rôle. Dans la première métamorphose, il se pose en face de la marchandise, comme la figure de sa valeur qui possède ailleurs, dans la poche d’autrui, une réalité dure et sonnante. Dès que la marchandise est changée en chrysalide d’argent, l’argent cesse d’être un cristal solide. Il n’est plus que la forme transitoire de la marchandise, sa forme équivalente qui doit s’évanouir et se convertir en valeur d’usage.\par
Les deux métamorphoses qui constituent le mouvement circulaire \emph{d’une marchandise} forment simultanément des métamorphoses partielles et inverses de deux autres marchandises.\par
La première métamorphose de la toile, par exemple (toile-argent), est la seconde et dernière métamorphose du froment (froment‑argent‑toile). La dernière métamorphose de la toile (argent‑bible) est la première métamorphose de, bible (bible-argent). Le cercle que forme la série des métamorphoses de chaque marchandise s’engrène ainsi dans les cercles que forment les autres. L’ensemble de tous ces cercles constitue la \emph{circulation des marchandises.}\par
La circulation des marchandises se distingue essentiellement de l’échange immédiat des produits. Pour s’en convaincre, il suffit de jeter un coup d’œil sur ce qui s’est passé. Le tisserand a bien échangé sa toile contre une bible, sa propre marchandise contre une autre ; mais ce phénomène n’est vrai que pour lui. Le vendeur de bibles, qui préfère le chaud au froid, ne pensait point échanger sa bible contre de la toile ; le tisserand n’a peut-être pas le moindre soupçon que c’était du froment qui s’est échangé contre sa toile, etc.\par
La marchandise de B est substituée à la marchandise de A ; mais A et B n’échangent point leurs marchandises réciproquement. Il se peut bien que A et B achètent l’un de l’autre ; mais c’est un cas particulier, et point du tout un rapport nécessairement donné par les conditions générales de la circulation. La circulation élargit au contraire la sphère de la permutation matérielle du travail social, en émancipant les producteurs des limites locales et individuelles, inséparables de l’échange immédiat de leurs produits. De l’autre côté, ce développement même donne lieu à un ensemble de rapports sociaux, indépendants des agents de la circulation, et qui échappent à leur contrôle. Par exemple, si le tisserand peut vendre sa toile, c’est que le paysan a vendu du froment ; si Pritchard vend sa bible, c’est que le tisserand a vendu sa toile ; le distillateur ne vend son eau brûlée que parce que l’autre a déjà vendu l’eau de la vie éternelle, et ainsi de suite.\par
La circulation ne s’éteint pas non plus, comme l’échange immédiat, dans le changement de place ou de main des produits. L’argent ne disparaît point, bien qu’il s’élimine à la fin de chaque série de métamorphoses d’une marchandise. Il se précipite toujours sur le point de la circulation qui a été évacué par la marchandise. Dans la métamorphose complète de la toile par exemple, \emph{toile‑argent‑bible, c’est} la toile qui sort la première de la circulation. L’argent la remplace. La bible sort après elle ; l’argent la remplace encore, et ainsi de suite. Or, quand la marchandise d’un échangiste remplace celle d’un autre, l’argent reste toujours aux doigts d’un troisième. La circulation sue l’argent par tous les pores.\par
Rien de plus niais que le dogme d’après lequel la circulation implique nécessairement l’équilibre des achats et des ventes, vu que toute vente est achat, et réciproquement. Si cela veut dire que le nombre des ventes réellement effectuées est égal au même nombre d’achats, ce n’est qu’une plate tautologie. Mais ce qu’on prétend prouver, c’est que le vendeur amène au marché son propre acheteur. Vente et achat sont un acte \emph{identique} comme rapport réciproque de \emph{deux personnes polariquement opposées}, du possesseur de la marchandise et du possesseur de l’argent. Ils forment \emph{deux actes polariquement opposés} comme actions \emph{de la même personne.} L’identité de vente et d’achat entraîne donc comme conséquence que la marchandise devient \emph{inutile}, si, une fois jetée dans la cornue alchimique de la circulation, elle n’en sort pas \emph{argent.} Si l’un n’achète pas, l’autre ne peut vendre. Cette identité suppose de plus que le succès de la transaction forme un point d’arrêt, un intermède dans la vie de la marchandise, intermède qui peut durer plus ou moins longtemps. La première métamorphose d’une marchandise étant à la fois vente et achat, est par cela même séparable de sa métamorphose complémentaire. L’acheteur a la marchandise, le vendeur a l’argent, c’est‑à‑dire une marchandise douée d’une forme qui la rend toujours la bienvenue au marché, à quelque moment qu’elle y réapparaisse. Personne ne peut vendre sans qu’un autre achète ; mais personne n’a besoin d’acheter immédiatement, parce qu’il a vendu.\par
La circulation fait sauter les barrières par lesquelles le temps, l’espace et les relations d’individu à individu rétrécissent le troc des produits. Mais comment ? Dans le commerce en troc, personne ne peut aliéner son produit sans que simultanément une autre personne aliène le sien. L’identité immédiate de ces deux actes, la circulation la scinde en y introduisant l’antithèse de la vente et de l’achat. Après avoir vendu, je ne suis forcé d’acheter ni au même lieu, ni au même temps, ni de la même personne à laquelle j’ai vendu. Il est vrai que l’achat est le complément obligé de la vente, mais il n’est pas moins vrai que leur unité est l’unité de contraires. Si la séparation des deux phases complémentaires l’une de l’autre de la métamorphose des marchandises se prolonge, si la scission entre la vente et l’achat s’accentue, leur liaison intime s’affirme par une crise. ‑ Les contradictions que recèle la marchandise, de valeur usuelle et valeur échangeable, de travail privé qui doit à la fois se représenter comme travail social, de travail concret qui ne vaut que comme travail abstrait ; ces contradictions immanentes à la nature de la marchandise acquièrent dans la circulation leurs formes de mouvement. Ces formes impliquent la possibilité, mais aussi seulement la possibilité des crises. Pour que cette possibilité devienne réalité, il faut tout un ensemble de circonstances qui, au point de vue de la circulation simple des marchandises, n’existent pas encore\footnote{V. mes remarques sur \emph{James Mill}, l. c., p. 74‑76. Deux points principaux caractérisent à ce sujet la méthode apologétique des économistes. D’abord ils identifient la circulation des marchandises et l’échange immédiat des produits, en faisant tout simplement abstraction de leurs différences. En second lieu, ils. essaient d’effacer les contradictions de la \emph{production capitaliste} en réduisant les rapports de ses agents aux rapports simples qui résultent de la circulation des marchandises. Or, circulation des marchandises et production des marchandises sont des phénomènes qui appartiennent aux modes de production les plus différents, quoique dans une mesure et une portée qui ne sont pas les mêmes, On ne sait donc encore rien de la différence spécifique des modes de production, et on ne peut les juger, si l’on ne connaît que les catégories abstraites de la circulation des marchandises qui leur sont communes. Il n’est pas de science où, avec des lieux communs élémentaires, l’on fasse autant l’important que dans l’économie politique. \emph{J. B. Say}, par exemple, se fait fort de juger les crises, parce qu’il sait que la marchandise est un \emph{produit}.}.
\subsubsection[{1.1.3.2.2. Cours de la monnaie.}]{1.1.3.2.2. Cours de la monnaie.}
\noindent Le mouvement M‑A‑M, ou la métamorphose complète d’une marchandise, est circulatoire en ce sens qu’une même valeur, après avoir subi des changements de forme, revient à sa forme première, celle de marchandise. Sa forme argent dis­paraît au contraire dès que le cours de sa circulation est achevé. Elle n’en a pas encore dépassé la première moitié, tant qu’elle est retenue sous cette forme d’équivalent par son vendeur. Dès qu’il complète la vente par l’achat, l’argent lui glisse aussi des mains. Le mouvement imprimé à l’argent par la circulation des marchandises n’est donc pas circulatoire. Elle l’éloigne de la main de son possesseur sans jamais l’y ramener. Il est vrai que si le tisserand, après avoir vendu vingt mètres de toile et puis acheté la bible, vend de nouveau de la toile, l’argent lui reviendra. Mais il ne proviendra point de la circulation des vingt premiers mètres de toile. Son retour exige le \emph{renouvelle­ment} ou la répétition du même mouvement circulatoire pour une marchandise nouvelle et se termine par le même résultat qu’auparavant. Le mouvement que la circulation des marchan­dises imprime à l’argent l’éloigne donc constamment de son point de départ, pour le faire passer sans relâche d’une main à l’autre : c’est ce que l’on a nommé le cours de la monnaie (\emph{currency}).\par
Le cours de la monnaie, c’est la répétition constante et monotone du même mouvement. La marchandise est toujours du côté du vendeur, l’argent toujours du côté de l’acheteur, comme \emph{moyen d’achat.} A ce titre sa fonction est de réaliser le prix des marchandises. En réalisant leurs prix, il les fait passer du vendeur à l’acheteur, tandis qu’il passe lui-même de ce dernier au premier, pour recommencer la même marche avec une autre marchandise.\par
A première vue ce mouvement unilatéral de la monnaie ne paraît pas provenir du mouvement bilatéral de la marchandise.\par
La circulation même engendre l’apparence contraire. Il est vrai que dans la première métamorphose, le mouvement de la marchandise est aussi apparent que celui de la monnaie avec laquelle elle change de place, mais sa deuxième métamorphose se fait sans qu’elle y apparaisse. Quand elle commence ce mouvement complémentaire de sa circulation, elle a déjà dépouillé son corps naturel et revêtu sa larve d’or. La continuité du mouvement échoit ainsi à la monnaie seule. C’est la monnaie qui paraît faire circuler des marchandises immobiles par elles-mêmes et les transférer de la main où elles sont des non‑valeurs d’usage à la main où elles sont des valeurs d’usage dans une direction toujours opposée à la sienne propre. Elle éloigne constamment les marchandises de la sphère de la circulation, en se mettant constamment à leur place et en abandonnant la sienne. Quoique le mouvement de la monnaie ne soit que l’expression de la circulation des marchandises, c’est au contraire la circulation des marchandises qui semble ne résulter que du mouvement de la monnaie\footnote{« Il (l’argent) n’a d’autre mouvement que celui qui lui est imprimé par les productions. » (Le Trosne, l. c., p. 885.)}.\par
D’un autre côté la monnaie ne fonctionne comme moyen de circulation que parce qu’elle est la forme valeur des marchandises réalisée. Son mouvement n’est donc en fait que leur propre mouvement de forme, lequel par conséquent doit se refléter et devenir palpable dans le cours de la monnaie. C’est aussi ce qui arrive. La toile, par exemple, change d’abord sa forme marchandise en sa forme monnaie. Le dernier terme de sa première métamorphose (M‑A), la forme monnaie, est le premier terme de sa dernière métamorphose, sa reconversion en marchandise usuelle, en bible (A‑M). Mais chacun de ces changements de forme s’accomplit par un échange entre marchandise et monnaie ou par leur déplacement réciproque. Les mêmes pièces d’or changent, dans le premier acte, de place avec la toile et dans le deuxième, avec la bible. Elles sont déplacées deux fois. La première métamorphose de la toile les fait entrer dans la poche du tisserand et la deuxième métamorphose les en fait sortir. Les deux changements de forme inverses, que la même marchandise subit, se reflètent donc dans le double changement de place, en direction opposée, des mêmes pièces de monnaie.\par
Si la marchandise ne passe que par une métamorphose partielle, par un seul mouvement qui est vente, considéré d’un pôle, et achat, considéré de l’autre, les mêmes pièces de monnaie ne changent aussi de place qu’une seule fois. Leur second changement de place exprime toujours la seconde métamorphose d’une marchandise, le retour qu’elle fait de sa forme monnaie à une forme usuelle. Dans la répétition fréquente du déplacement des mêmes pièces de monnaie ne se reflète plus seulement la série de métamorphoses d’une seule marchandise, mais encore l’engrenage de pareilles métamorphoses les unes dans les autres\footnote{Il faut bien remarquer que le développement donné dans le texte n’a trait qu’à la forme simple de la circulation, la seule que nous étudiions à présent.}.\par
Chaque marchandise, à son premier changement de forme, à son premier pas dans la circulation, en disparaît pour y être sans cesse remplacée par d’autres. L’argent, au contraire, en tant que moyen d’échange, habite toujours la sphère de la circulation et s’y promène sans cesse. Il s’agit maintenant de savoir quelle est la quantité de monnaie que cette sphère peut absorber.\par
Dans un pays il se fait chaque jour simultanément et à côté les unes des autres des ventes plus ou moins nombreuses ou des métamorphoses partielles de diverses marchandises. La valeur de ces marchandises est exprimée par leurs prix, c’est‑à‑dire en sommes d’or imaginé. La quantité de monnaie qu’exige la circulation de toutes les marchandises présentes au marché est donc déterminée par la somme totale de leurs prix. La monnaie ne fait que représenter réellement cette somme d’or déjà exprimée idéalement dans la somme des prix des marchandises. L’égalité de ces deux sommes se comprend donc d’elle-même. Nous savons cependant que si les valeurs des marchandises restent constantes, leurs prix varient avec la valeur de l’or, (de la matière monnaie), montant proportionnellement à sa baisse et descendant proportionnellement à sa hausse. De telles variations dans la somme des prix à réaliser entraînent nécessairement des changements proportionnels dans la quantité de la monnaie courante. Ces changements proviennent en dernier lieu de la monnaie elle-même, mais, bien entendu, non pas en tant qu’elle fonctionne comme instrument de circulation, mais en tant qu’elle fonctionne comme mesure de la valeur. Dans de pareils cas il y a d’abord des changements dans la valeur de la monnaie. Puis le prix des marchandises varie en raison inverse de la valeur de la monnaie, et enfin la masse de la monnaie courante varie en raison directe du prix des marchandises.\par
On a vu que la circulation a une porte par laquelle l’or (ou toute autre matière monnaie) entre comme marchandise. Avant de fonctionner comme mesure des valeurs, sa propre valeur est donc déterminée. Vient‑elle maintenant à changer, soit à baisser, on s’en apercevra d’abord à la source de la production du métal précieux, là où il se troque contre d’autres marchandises. Leurs prix monteront tandis que beaucoup d’autres marchandises continueront à être estimées dans la valeur passée et devenue illusoire du métal‑monnaie. Cet état de choses peut durer plus ou moins longtemps selon le degré de développement du marché universel. Peu à peu cependant une marchandise doit influer sur l’autre par son rapport de valeur avec elle ; les prix or ou argent des marchandises se mettent graduellement en équilibre avec leurs valeurs comparatives jusqu’à ce que les valeurs de toutes les marchandises soient enfin estimées d’après la valeur nouvelle du métal‑monnaie. Tout ce mouvement est accompagné d’une augmentation continue du métal précieux qui vient remplacer les marchandises troquées contre lui. A mesure donc que le tarif corrigé des prix des marchandises se généralise et qu’il y a par conséquent hausse générale des prix, le surcroît de métal qu’exige leur réalisation, se trouve aussi déjà disponible sur le marché. Une observation imparfaite des faits qui suivirent la découverte des nouvelles mines d’or et d’argent, conduisit au XVII° et notamment au XVIII° siècle, à cette conclusion erronée, que les prix des marchandises s’étaient élevés, parce qu’une plus grande quantité d’or et d’argent fonctionnait comme instrument de circulation. Dans les considérations qui suivent, la valeur de l’or est supposée \emph{donnée}, comme elle l’est en effet au moment de la fixation des prix.\par
Cela une fois admis, la masse de l’or circulant sera donc déterminée par le prix total des marchandises à réaliser. Si le prix de chaque espèce de marchandise est donné, la somme totale des prix dépendra évidemment de la masse des marchandises en circulation. On peut comprendre sans se creuser la tête que si un quart de froment coûte deux livres sterling, cent quarts coûteront deux cents livres sterling et ainsi de suite, et qu’avec la masse du froment doit croître la quantité d’or qui, dans la vente, change de place avec lui.\par
La masse des marchandises étant donnée, les fluctuations de leurs prix peuvent réagir sur la masse de la monnaie circulante. Elle va monter ou baisser selon que la somme totale des prix à réaliser augmente ou diminue. Il n’est pas nécessaire pour cela que les prix de toutes les marchandises montent ou baissent simultanément. La hausse ou la baisse d’un certain nombre d’articles principaux suffit pour influer sur la somme totale des prix à réaliser. Que le changement de prix des marchandises reflète des changements de valeur réels ou provienne de simples oscillations du marché, l’effet produit sur la quantité de la monnaie circulante reste le même.\par
Soit un certain nombre de ventes sans lien réciproque, simultanées et par cela même s’effectuant les unes à côté des autres, ou de métamorphoses partielles, par exemple, d’un quart de froment, vingt mètres de toile, une bible, quatre fûts d’eau‑de‑vie. Si chaque article coûte deux livres sterling, la somme de leurs prix est huit livres sterling et, pour les réaliser, il faut jeter huit livres sterling dans la circulation. Ces mêmes marchandises forment‑elles au contraire la série de métamorphoses connue : 1 quart de froment ‑ 2 l. st. ‑ 20 mètres de toile ‑ 2 l. st. ‑ 1 bible ‑ 2 l. st. ‑ 4 fûts d’eau‑de‑vie ‑ 2 l. st., alors \emph{les mêmes} deux livres sterling font circuler dans l’ordre indiqué ces marchandises diverses, en réalisant successivement leurs prix et s’arrêtent enfin dans la main du distillateur. Elles accomplissent ainsi quatre tours.\par
Le déplacement quatre fois répété des deux livres sterling résulte des métamorphoses complètes, entrelacées les unes dans les autres, du froment, de la toile et de la bible, qui finissent par la première métamorphose de l’eau‑de‑vie\footnote{« Ce sont les productions qui le mettent en mouvement (l’argent) et le font circuler… La célérité de son mouvement supplée à sa quantité. Lorsqu’il en est besoin, il ne fait que glisser d’une main dans l’autre sans s’arrëter un instant. » (Le Trosne, l. c., p. 915, 916.)}. Les mouvements opposés et complémentaires les uns des autres dont se forme une telle série, ont lieu successivement et non simultanément. Il leur faut plus ou moins de temps pour s’accomplir. La vitesse du cours de la monnaie se mesure donc par le nombre de tours des mêmes pièces de monnaie dans un temps donné. Supposons que la circulation des quatre marchandises dure un jour. La somme des prix à réaliser est de huit livres sterling, le nombre de tours de chaque pièce pendant le jour : quatre, la masse de la monnaie circulante : deux livres sterling et nous aurons donc :\par
\emph{Somme des prix des marchandises} divisée par le nombre des tours des pièces de la même dénomination dans un temps donné = Masse de la monnaie fonctionnant comme instrument de circulation.\par
Cette loi est générale. La circulation des marchandises dans un pays, pour un temps donné, renferme bien des ventes isolées (ou des achats), c’est‑à‑dire des métamorphoses partielles et simultanées où la monnaie ne change qu’une fois de place ou ne fait qu’un seul tour. D’un autre côté, il y a des séries de métamorphoses plus ou moins ramifiées, s’accomplissant côte à côte ou s’entrelaçant les unes dans les autres où les mêmes pièces de monnaie font des tours plus ou moins nombreux. Les pièces particulières dont se compose la somme totale de la monnaie en circulation fonctionnent donc à des degrés d’activité très divers, mais le total des pièces de chaque dénomination réalise, pendant une période donnée, une certaine somme de prix. Il s’établit donc une vitesse moyenne du cours de la monnaie.\par
La masse d’argent qui, par exemple, est jetée dans la circulation à un moment donné est naturellement déterminée par le prix total des marchandises vendues à côté les unes des autres. Mais dans le courant même de la circulation chaque pièce de monnaie est rendue, pour ainsi dire, responsable pour sa voisine. Si l’une active la rapidité de sa course, l’autre la ralentit, ou bien est rejetée complètement de la sphère de la circulation, attendu que celle‑ci ne peut absorber qu’une masse d’or qui, multipliée par le nombre moyen de ses tours, est égale à la somme des prix à réaliser. Si les tours de la monnaie augmentent, sa masse diminue ; si ses tours diminuent, sa masse augmente. La vitesse moyenne de la monnaie étant donnée, la masse qui peut fonctionner comme instrument de la circulation se trouve déterminée également. Il suffira donc, par exemple, de jeter dans la circulation un certain nombre de billets de banque d’une livre pour en faire sortir autant de livres sterling en or, ‑ truc bien connu par toutes les banques.\par
De même que le cours de la monnaie en général reçoit son impulsion et sa direction de la circulation des marchandises, de même la rapidité de son mouvement ne reflète que la rapidité de leurs changements de forme, la rentrée continuelle des séries de métamorphoses les unes dans les autres, la disparition subite des marchandises de la circulation et leur remplacement aussi subit par des marchandises nouvelles. Dans le cours accéléré de la monnaie apparaît ainsi \emph{l’unité fluide} des phases opposées et complémentaires, transformation de l’aspect usage des marchandises en leur aspect valeur et retransformation de leur aspect valeur en leur aspect usage, ou l’unité de la vente et de l’achat comme deux actes alternativement exécutés par les mêmes échangistes. Inversement, le ralentissement du cours de la monnaie fait apparaître la \emph{séparation} de ces phénomènes et leur \emph{tendance à s’isoler en opposition l’un de l’autre}, l’interruption des changements de forme et conséquemment des permutations de matières. La circulation naturellement ne laisse pas voir d’où provient cette interruption ; elle ne montre que le phénomène. Quant au vulgaire qui, à mesure que la circulation de la monnaie se ralentit, voit l’argent se montrer et disparaître moins fréquemment sur tous les points de la périphérie de la circulation, il est porté à chercher l’explication du phénomène dans l’insuffisante quantité du métal circulant\footnote{ \noindent « L’argent étant la mesure commune des ventes et des achats, quiconque a quelque chose à vendre et ne peut se procurer des acheteurs est enclin à penser que le manque d’argent dans le royaume est la cause qui fait que ses articles ne se vendent pas, et dès lors chacun de s’écrier que l’argent manque, ce qui est une grande méprise… Que veulent donc ces gens qui réclament de l’argent à grands cris ?… Le fermier se plaint, il pense que s’il y avait plus d’argent dans le pays il trouverait un prix pour ses denrées. Il semble donc que ce n’est pas l’argent, mais un prix qui fait défaut pour son blé et son bétail… et pourquoi ne trouve‑t‑il pas de prix ?… 1° Ou bien il y a trop de blé et de bétail dans le pays, de sorte que la plupart de ceux qui viennent au marché ont besoin de vendre comme lui et peu ont besoin d’acheter ; 2° ou bien le débouché ordinaire par exportation fait défaut… ou bien encore 3° la consommation diminue, comme lorsque bien des gens, pour raison de pauvreté, ne peuvent plus dépenser autant dans leur maison qu’ils le faisaient auparavant. Ce ne serait donc pas l’accroissement d’argent qui ferait vendre les articles du fermier, mais la disparition d’une de ces trois causes. C’est de la même façon que le marchand et le boutiquier manquent d’argent, c’est‑à‑dire qu’ils manquent d’un débouché pour les articles dont ils trafiquent, par la raison que le marché leur fait défaut… Une nation n’est jamais plus prospère que lorsque les richesses ne font qu’un bond d’une main à l’autre. » (Sir Dudley North : \emph{Discourses upon Trade}, London, 1691, p. 11‑15 \emph{passim.)}\par
 Toutes les élucubrations \emph{d’Herrenschwand} se résument en ceci, que les antagonismes qui résultent de la nature de la marchandise et qui se manifestent nécessairement dans la circulation pourraient être écartés en y jetant une masse plus grande de monnaie. Mais si c’est une illusion d’attribuer un ralentissement ou un arrêt dans la marche de la production et de la circulation au manque de monnaie, il ne s’ensuit pas le moins du monde qu’un manque réel de moyens de circulation provenant de limitations législatives ne puisse pas de son côté provoquer des stagnations.
}.\par
Le quantum total de l’argent qui fonctionne comme instrument de circulation dans une période donnée est donc déterminé d’un côté par la \emph{somme tics prix} de toutes les marchandises circulantes, de l’autre par la vitesse relative de leurs métamorphoses. Mais le prix total des marchandises dépend et de la \emph{masse} et des prix de chaque espèce de marchandise. Ces trois facteurs : mouvement \emph{des prix, niasse des marchandises circulantes} et enfin \emph{vitesse du cours tic la monnaie}, peuvent changer dans des proportions diverses et dans une direction différente ; la \emph{somme des prix à réaliser} et par conséquent la \emph{masse} des moyens de circulation qu’elle exige, peuvent donc également subir des combinaisons nombreuses dont nous ne mentionnerons ici que les plus importantes dans l’histoire des prix.\par
\emph{Les prix restant les mêmes}, la masse des moyens de circulation peut augmenter, soit que la masse des marchandises circulantes augmente, soit que la vitesse du cours de la monnaie diminue ou que ces deux circonstances agissent ensemble. Inversement la masse des moyens de circulation peut diminuer si la masse des marchandises diminue ou si la monnaie accélère son cours.\par
\emph{Les prix des marchandises subissant une hausse générale}, la masse des moyens de circulation peut rester la même, si la masse des marchandises circulantes diminue dans la même proportion que leur prix s’élève, ou si la vitesse du cours de la monnaie augmente aussi rapidement que la hausse des prix, tandis que la masse des marchandises en circulation reste la même' La masse des moyens de circulation peut décroître, soit que la masse des marchandises décroisse, soit que la vitesse du cours de l’argent croisse plus rapidement que leurs prix.\par
\emph{Les prix des marchandises subissant une baisse générale}, la masse des moyens de circulation peut rester la même, si la masse des marchandises croît dans la même proportion que leurs prix baissent ou si la vitesse du cours de l’argent diminue dans la même proportion que les prix. Elle peut augmenter si la masse des marchandises croît plus vite, ou si la rapidité de la circulation diminue plus promptement que les prix ne baissent.\par
Les variations des différents facteurs peuvent se compenser réciproquement, de telle sorte que malgré leurs oscillations perpétuelles la somme totale des prix à réaliser reste constante et par conséquent aussi la masse de la monnaie courante. En effet, si on considère des périodes d’une certaine durée, on trouve les déviations du niveau moyen bien moindres qu’on s’y attendrait à première vue, à part toutefois de fortes perturbations périodiques qui proviennent presque toujours de crises industrielles et commerciales, et exceptionnellement d’une variation dans la valeur même des métaux précieux.\par
Cette loi, que la quantité des moyens de circulation est déterminée par la somme des prix des marchandises circulantes et par la vitesse moyenne du cours de la monnaie\footnote{ \noindent « Il y a une certaine mesure et une certaine proportion de monnaie nécessaire pour faire marcher le commerce d’une nation, au‑dessus ou au‑dessous desquelles ce commerce éprouverait un préjudice. Il faut de même une certaine proportion de farthings (liards) dans un petit commerce de détail pour échanger la monnaie d’argent et surtout pour les comptes qui ne pourraient être réglés complètement avec les plus petites pièces d’argent… De même que la proportion du nombre de farthings exigée par le commerce doit être calculée d’après le nombre des marchands, la fréquence de leurs échanges, et surtout d’après la \emph{valeur} des plus petites pièces de monnaie d’argent ; de même la proportion de monnaie (argent ou or) requise par notre commerce doit être calculée sur le nombre des échanges et la grosseur des payements à effectuer. » (William Petty, \emph{A Treatise on Taxes and Contributions}, London, 1667, p. 17.)\par
 La théorie de Hume, d’après laquelle « les prix dépendent de l’abondance de l’argent », fut défendue contre Sir James Steuart et d’autres par A. Young, dans sa \emph{Political Arithmetic}, London, 1774, p. 112 et suiv. Dans mon livre : \emph{Zur Kritik}, etc., p. 183, j’ai dit qu’Adam Smith passa sous silence cette question de la quantité de la monnaie courante. Cela n’est vrai cependant qu’autant qu’il traite la question de l’argent \emph{ex professo.} A l’occasion, par exemple dans sa critique des systèmes antérieurs d’économie politique, il s’exprime correctement à ce sujet : « La quantité de monnaie dans chaque pays est réglée par la valeur des marchandises qu’il doit faire circuler… La valeur des articles achetés et vendus annuellement dans un pays requiert une certaine quantité de monnaie pour les faire circuler et les distribuer à leurs consommateurs et ne peut en employer davantage. Le canal de la circulation attire nécessairement une somme suffisante pour le remplir et n’admet jamais rien de plus. »\par
 Adam Smith commence de même son ouvrage, \emph{ex professo}, par une apothéose de la division du travail. Plus tard, dans le dernier livre sur les sources du revenu de l’Etat, il reproduit les observations de A. Ferguson, son maître, contre la division du travail. (\emph{Wealth of Nations}, l. IV, c. 1.)
}, revient à ceci : étant donné et la somme de valeur des marchandises et la vitesse moyenne de leurs métamorphoses, la quantité du métal précieux en circulation dépend de sa propre valeur. L’illusion d’après laquelle les prix des marchandises sont au contraire déterminés par la masse des moyens de circulation et cette masse par l’abondance des métaux précieux dans un pays\footnote{« Les prix des choses s’élèvent dans chaque pays à mesure que l’or et l’argent augmentent dans la population ; si donc l’or et l’argent diminuent dans un pays, les prix de toutes choses baisseront proportionnellement à cette diminution de monnaie. » (Jacob Vanderlint, \emph{Money answers all things}, London, 1734, p. 5.) ‑ Une comparaison plus attentive de l’écrit de Vanderlint et de l’essai de Hume ne me laisse pas le moindre doute que ce dernier connaissait l’œuvre de son prédécesseur et en tirait parti. On trouve aussi chez Barbon et beaucoup d’autres écrivains avant lui cette opinion que la masse des moyens de circulation détermine les prix. « Aucun inconvénient, dit‑il, ne peut provenir de la liberté absolue du commerce, mais au contraire un grand avantage… puisque si l’argent comptant d’une nation en éprouve une diminution, ce que les prohibitions sont chargées de prévenir, les autres nations qui acquièrent l’argent verront certainement les prix de toutes choses s’élever chez elles, à mesure que la monnaie y augmente… et nos manufactures parviendront à livrer à assez bas prix, pour faire incliner la balance du commerce en notre faveur et faire revenir ainsi la monnaie chez nous (l. c., p. 44).}, repose originellement sur l’hypothèse absurde que les marchandises et l’argent entrent dans la circulation, les unes sans prix, l’autre sans valeur, et qu’une partie aliquote du tas des marchandises s’y échange ensuite contre la même partie aliquote de la montagne de métal\footnote{Il est évident que chaque espèce de marchandise forme, \emph{par son prix}, un élément \emph{du prix total de toutes les marchandises en circulation.} Mais il est impossible de comprendre comment un tas de \emph{valeurs d’usage} incommensurables entre elles peut s’échanger contre la masse d’or ou d’argent qui se trouve dans un pays. Si l’on réduisait l’ensemble des marchandises à une \emph{marchandise générale unique}, dont chaque marchandise ne formerait qu’une partie aliquote, on obtiendrait cette équation absurde : Marchandise générale = x quintaux d’or, marchandise A = partie aliquote de la marchandise générale = même partie aliquote de x quintaux d’or. Ceci est très naïvement exprimé par Montesquieu : « Si l’on compare la masse de l’or et de l’argent qui est dans le monde, avec la somme des marchandises qui y sont, il est certain que chaque denrée ou marchandise, en particulier, pourra être comparée à une certaine portion de l’autre. \emph{Supposons qu’il n’y ait qu’une seule denrée ou marchandise dans le monde}, ou qu’il n’y en ait qu’une seule qui s’achète, et qu’elle \emph{se divise comme l’argent ;} une partie de cette marchandise répondra à une partie de la masse d’argent ; la moitié du total de l’une à la moitié du total de l’autre, etc. L’établissement du prix des choses dépend toujours fondamentalement de la raison du total des choses au total des signes. » (Montesquieu, l. c., t. III, p. 12, 13.) Pour les développements donnés à cette théorie par Ricardo, par son disciple James Mill, Lord Overstone, etc., V. mon écrit : Z\emph{ur Kritik}, etc., p. 140‑146 et p. 150 et suiv. M. J. St. Mill, avec la logique éclectique qu’il manie si bien, s’arrange de façon à être tout à la fois de l’opinion de son père James Mill et de l’opinion opposée. Si l’on compare le texte de son traité : \emph{Principes d’economie politique}, avec la préface de la première édition dans laquelle il se présente lui‑même comme l’Adam Smith de notre époque, on ne sait quoi le plus admirer, de la naïveté de l’homme ou de celle du public qui l’a pris, en effet, pour un Adam Smith, bien qu’il ressemble à ce dernier comme le général Williams de Kars au duc de Wellington. Les recherches originales, d’ailleurs peu étendues et peu profondes de M. J. Si. Mill dans le domaine de l’économie politique, se trouvent toutes rangées en bataille dans son petit écrit paru en 1844, sous le titre : \emph{Some unsettled questions of political economy. ‑} Quant à Locke, il exprime tout crûment la liaison entre sa théorie de la non‑valeur des métaux précieux et la détermination de leur valeur par leur seule quantité. « L’humanité ayant consenti à accorder à l’or et à l’argent une valeur imaginaire… la valeur intrinsèque considérée dans ces métaux n’est rien autre chose que quantité. » (Locke, \emph{« Some Considerations}, etc. », 1691. Ed. de 1777, vol. 11, p. 15.)}.
\subsubsection[{1.1.3.2.3. Le numéraire ou les espèces. ‑ Le signe de valeur.}]{1.1.3.2.\textbf{3}. Le numéraire ou les espèces. ‑ Le signe de valeur.}
\noindent Le numéraire tire son origine de la fonction que la monnaie remplit comme instrument de circulation. Les poids d’or, par exemple, exprimés selon l’étalon officiel dans les prix où les noms monétaires des marchandises, doivent leur faire face sur le marché comme espèces d’or de la même dénomination ou comme numéraire. De même que l’établissement de l’étalon des prix, le monnayage est une besogne qui incombe à l’Etat. Les divers uniformes nationaux que l’or et l’argent revêtent, en tant que numéraire, mais dont ils se dépouillent sur le marché du monde, marquent bien la séparation entre les sphères intérieures ou nationales et la sphère générale de la circulation des marchandises.\par
L’or monnayé et l’or en barre ne se distinguent de prime abord que par la figure, et l’or peut toujours passer d’une de ces formes à l’autre\footnote{Je n’ai pas à m’occuper ici du droit de seigneuriage et d’autres détails de ce genre. Je mentionnerai cependant à l’adresse du sycophante \emph{Adam Muller}, qui admire « la grandiose libéralité avec laquelle le gouvernement anglais monnaye gratuitement », le jugement suivant de Sir Dudley North : « L’or et l’argent, comme les autres marchandises, ont leur flux et leur reflux. En arrive-t‑il des quantités d’Espagne… on le porte à la Tour et il est aussitôt monnayé. Quelque temps après vient une demande de lingots pour l’exportation. S’il n’y en a pas et que tout soit en monnaie, que faire ? Eh bien ! qu’on refonde tout de nouveau ; il n’y a rien à y perdre, puisque cela ne coûte rien au possesseur. C’est ainsi qu’on se moque de la nation et qu’on lui fait payer le tressage de la paille à donner aux ânes. Si le marchand (North lui‑même était un des premiers négociants du temps de Charles II) avait à payer le prix du monnayage, il n’enverrait pas ainsi son argent à la Tour sans plus de réflexion, et la monnaie conserverait toujours une valeur supérieure à celle du métal non monnayé. » (North, l. c., p. 18.)}. Cependant en sortant de la Monnaie le numéraire se trouve déjà sur la voie du creuset. Les monnaies d’or ou d’argent s’usent dans leurs cours, les unes plus, les autres moins. A chaque pas qu’une guinée, par exemple, fait dans sa route, elle perd quelque chose de son poids tout en conservant sa dénomination. Le titre et la matière, la substance métallique et le nom monétaire commencent ainsi à se séparer. Des espèces de même nom deviennent de valeur inégale, n’étant plus de même poids. Le poids d’or indiqué par l’étalon des prix ne se trouve plus dans l’or qui circule, lequel cesse par cela même d’être l’équivalent réel des marchandises dont il doit réaliser les prix. L’histoire des monnaies au moyen âge et dans les temps modernes jusqu’au XVIII° siècle n’est guère que l’histoire de cet embrouillement. La tendance naturelle de la circulation à transformer les espèces d’or en un semblant d’or, ou le numéraire en symbole de son poids métallique officiel, est reconnue par les lois les plus récentes sur le degré de perte de métal qui met les espèces hors de cours ou les démonétise.\par
Le cours de la monnaie, en opérant une scission entre le contenu réel et le contenu nominal, entre l’existence métallique et l’existence fonctionnelle des espèces, implique déjà, sous forme latente, la possibilité de les remplacer dans leur fonction de numéraire par des jetons de billon, etc. Les difficultés techniques du monnayage de parties de poids d’or ou d’argent tout à fait diminutives, et cette circonstance que des métaux inférieurs servent de mesure de valeur et circulent comme monnaie jusqu’au moment où le métal précieux vient les détrôner, expliquent historiquement leur rôle de monnaie symbolique. Ils tiennent lieu de l’or monnayé dans les sphères de la circulation où le roulement du numéraire est le plus rapide, c’est‑à‑dire où les ventes et les achats se renouvellent incessamment sur la plus petite échelle. Pour empêcher ces satellites de s’établir à la place de l’or, les proportions dans lesquelles ils doivent être acceptés en payement sont déterminées par des lois. Les cercles particuliers que parcourent les diverses sortes de monnaie s’entrecroisent naturellement. Là monnaie d’appoint, par exemple, apparaît pour payer des fractions d’espèces d’or ; l’or entre constamment dans la circulation de détail, mais il en est constamment chassé par la monnaie d’appoint échangée contre lui\footnote{« Si l’argent ne dépassait jamais ce dont on a besoin pour les petits payements, il ne pourrait être ramassé en assez grande quantité pour les payements plus importants… L’usage de l’or dans les gros payements implique donc son usage dans le commerce de détail. Ceux qui ont de la monnaie d’or l’offrent pour de petits achats et reçoivent avec la marchandise achetée un solde d’argent en retour. Par ce moyen, le surplus d’argent qui sans cela encombrerait le commerce de détail est dispersé dans la circulation générale. Mais, s’il y a autant d’argent qu’en exigent les petits payements, indépendamment de l’or, le marchand en détail recevra alors de l’argent pour les petits achats et le verra nécessairement s’accumuler dans ses mains. » (David Buchanan, \emph{Inquiry into the Taxation and commercial Policy of Great Britain.} Edinburgh, 1844, p. 248, 249.)}.\par
La substance métallique des jetons d’argent ou de cuivre est déterminée arbitrairement par la loi. Dans leur cours ils s’usent encore plus rapidement que les pièces d’or. Leur fonction devient donc par le fait complètement indépendante de leur poids, c’est‑à-dire de toute valeur.\par
Néanmoins, et c’est le point important, ils continuent de fonctionner comme remplaçants des espèces d’or. La fonction numéraire de l’or entièrement détachée de sa valeur métallique est donc un phénomène produit par les frottements de sa circulation même. Il peut donc être remplacé dans cette fonction par des choses relativement sans valeur aucune, telles que des billets de papier. Si dans les jetons métalliques le caractère purement symbolique est dissimulé jusqu’à un certain point, il se manifeste sans équivoque dans le papier‑monnaie. Comme on le voit, ce n’est que le premier pas qui coûte.\par
Il ne s’agit ici que de \emph{papier‑monnaie d’Etat avec cours forcé.} Il naît spontanément de la circulation métallique. \emph{La monnaie de crédit}, au contraire, suppose un ensemble de conditions qui, du point de vue de la circulation simple des marchandises, nous sont encore inconnues. Remarquons en passant que si le papier‑monnaie proprement dit provient de la fonction de l’argent comme \emph{moyen de circulation, la monnaie de crédit} a sa racine naturelle dans la fonction de l’argent comme \emph{moyen de payement}\footnote{Le mandarin des finances \emph{Wan‑mao‑in} s’avisa un jour de présenter au fils du ciel un projet dont le but caché était de transformer les assignats de l’Empire chinois en billets de banque convertibles. Le comité des assignats d’avril 1854 se chargea de lui laver la tête, et proprement. Lui fit‑il administrer la volée de coups de bambous traditionnelle, c’est ce qu’on ne dit pas. « Le comité », telle est la conclusion du rapport, « a examiné ce projet avec attention et trouve que tout en lui a uniquement en vue l’intérêt des marchands, mais que rien n’y est avantageux pour la couronne. » (\emph{Arbeiten der Kaiserlich Russischen Gesandtschaft zu Peking fiber China. Aus dem Russischen von Dr. K. Abel und F. A. Mecklenburg. Erster Band.} Berlin, 1858, p. 47 et suiv.) ‑ Sur la \emph{perte métallique} éprouvée par les monnaies d’or dans leur circulation, voici ce que dit le gouverneur de la Banque d’Angleterre, appelé comme témoin devant la Chambre des lords (Bank-acts Committee). ‑ « Chaque année, une nouvelle classe de souverains (non politique ‑ le souverain est le nom d’une l. st.) est trouvée trop légère. Cette classe qui telle année possède le poids légal perd assez par le frottement pour faire pencher, l’année après, le plateau de Ia balance contre elle. »}.\par
L’Etat jette dans la circulation des billets de papier sur lesquels sont inscrits des dénominations de numéraire tels que une livre sterling, cinq livres sterling, etc. En tant que ces billets circulent réellement à la place du poids d’or de la même dénomination, leur mouvement ne fait que refléter les lois du cours de la monnaie réelle. Une loi spéciale de la circulation du papier ne peut résulter que de son rôle de représentant de l’or ou de l’argent, et cette loi est très simple ; elle consiste en ce que l’émission du papier‑monnaie doit être proportionnée à la quantité d’or (ou d’argent) dont il est le symbole et qui devrait réellement circuler. La quantité d’or que la circulation peut absorber oscille bien constamment au‑dessus ou au‑dessous d’un certain niveau moyen ; cependant elle ne tombe jamais au‑dessous d’un min\emph{imum} que l’expérience fait connaître en chaque pays. Que cette masse \emph{minima} renouvelle sans cesse ses parties intégrantes, c’est‑à‑dire qu’il y ait un va‑et‑vient des espèces particulières qui y entrent et en sortent, cela ne change naturellement rien ni à ses proportions ni à son roulement continu dans l’enceinte de la circulation. Rien n’empêche donc de la remplacer par des symboles de papier. Si au contraire les canaux de la circulation se remplissent de papier‑monnaie jusqu’à la limite de leur faculté d’absorption pour le métal précieux, alors la moindre oscillation dans le prix des marchandises pourra les faire déborder. Toute mesure est dès lors perdue.\par
Abstraction faite d’un discrédit général, supposons que le papier‑monnaie dépasse sa proportion légitime. Après comme avant, il ne représentera dans la circulation des marchandises que le \emph{quantum} d’or qu’elle exige selon ses lois immanentes et qui, par conséquent, est seul représentable. Si, par exemple, la masse totale du papier est le double de ce qu’elle devrait être, un billet d’une livre sterling, qui représentait un quart d’once d’or, n’en représentera plus que un huitième. L’effet est le même que si l’or, dans sa fonction d’étalon de prix, avait été altéré.\par
Le papier‑monnaie est signe d’or ou signe de monnaie. Le rapport qui existe entre lui et les marchandises consiste tout simplement en ceci, que les mêmes quantités d’or qui sont exprimées idéalement dans leurs prix sont représentées symboliquement par lui. Le papier‑monnaie n’est donc signe de valeur qu’autant qu’il représente des quantités d’or qui, comme toutes les autres quantités de marchandises, sont aussi des quantités de valeur\footnote{Le passage suivant, emprunté à Fullarton, montre quelle idée confuse se font même les meilleurs écrivains de la nature de l’argent et de ses fonctions diverses. « Un fait qui, selon moi, n’admet point de dénégation, c’est que pour tout ce qui concerne nos échanges à l’intérieur, les fonctions monétaires que remplissent ordinairement les monnaies d’or et d’argent peuvent être remplies avec autant d’efficacité par des billets inconvertibles, n’ayant pas d’autre valeur que cette valeur factice et conventionnelle qui leur vient de la loi. Une valeur de ce genre peut être réputée avoir tous les avantages d’une valeur intrinsèque et permettra même de se passer d’un étalon de valeur, à la seule condition qu’on en limitera, comme il convient, le nombre des émissions. » (John Fuilarton, \emph{Régulation of Currencies}, 2° éd., London, 1845, p. 21.) ‑ Ainsi donc, parce que la marchandise argent peut être remplacée dans la circulation par de simples signes de valeur, son rôle de mesure des valeurs et d’étalon des prix est déclaré superflu !}.\par
On demandera peut‑être pourquoi l’or peut être remplacé par des choses sans valeur, par de simples signes. Mais il n’est ainsi remplaçable qu’autant qu’il fonctionne exclusivement comme numéraire ou instrument de circulation. Le caractère exclusif de cette fonction ne se réalise pas, il est vrai, pour les monnaies d’or ou d’argent prises à part, quoiqu’il se manifeste dans le fait que des espèces usées continuent néanmoins à circuler. Chaque pièce d’or n’est simplement instrument de circulation qu’autant qu’elle circule. Il n’en est pas ainsi de la masse d’or minima qui peut être remplacée par le papier‑monnaie. Cette masse appartient toujours à la sphère de la circulation, fonctionne sans cesse comme son instrument et existe exclusivement comme soutien de cette fonction. Son roulement ne représente ainsi que l’alternation continuelle des mouvements inverses de la métamorphose M‑A‑M où la figure valeur des marchandises ne leur fait face que pour disparaître aussitôt après, où le remplacement d’une marchandise par l’autre fait glisser la monnaie sans cesse d’une main dans une autre. Son existence fonctionnelle absorbe, pour ainsi dire, son existence matérielle. Reflet fugitif des prix des marchandises, elle ne fonctionne plus que comme signe d’elle‑même et peut par conséquent être remplacée par des signes\footnote{De ce fait, que l’or et l’argent en tant que numéraire ou dans la fonction exclusive d’instrument de circulation arrivent à n’être que des simples signes d’eux-mêmes, Nicolas Barbon fait dériver le droit des gouvernements « \emph{to raise money »}, c’est‑à‑dire de donner à un quantum d’argent, qui s’appellerait franc, le nom d’un quantum plus grand, tel qu’un écu, et de ne donner ainsi à leurs créanciers qu’un franc, au lieu d’un écu. « La monnaie s’use et perd de son poids en passant par un grand nombre de mains… C’est sa dénomination et son cours que l’on regarde dans les marches et non sa qualité d’argent. Le métal n’est fait monnaie que par l’autorité publique. » (N. Barbon, l. c., p. 29, 30, 45.)}. Seulement il faut que le signe de la monnaie soit comme elle socialement valable, et il le devient par le cours forcé. Cette action coercitive de l’Etat ne peut s’exercer que dans l’enceinte nationale de la circulation, mais là seulement aussi peut s’isoler la fonction que la monnaie remplit comme numéraire.
\subsection[{1.1.3.3. La monnaie ou l’argent.}]{1.1.3.3. La monnaie ou l’argent.}
\noindent Jusqu’ici nous avons considéré le métal précieux sous le double aspect de mesure des valeurs et d’instrument de circulation. Il remplit la première fonction comme monnaie idéale, il peut être représenté dans la deuxième par des symboles. Mais il y a des fonctions où il doit se présenter dans son corps métallique comme équivalent réel des marchandises ou comme marchandise‑monnaie. Il y a une autre fonction encore qu’il peut remplir ou en personne ou par des suppléants, mais où il se dresse toujours en face des marchandises usuelles comme l’unique incarnation adéquate de leur valeur. Dans tous ces cas, nous dirons qu’il fonctionne comme monnaie ou argent proprement dit par opposition à ses fonctions de mesure des valeurs et de numéraire.\par
\subsubsection[{1.1.3.3.1. Thésaurisation.}]{1.1.3.3.1. Thésaurisation.}
\noindent Le mouvement circulatoire des deux métamorphoses inverses des marchandises ou l’alternation continue de vente et d’achat se manifeste par le cours infatigable de la monnaie ou dans sa fonction de \emph{perpetuum mobile}, de moteur perpétuel de la circulation. Il s’immobilise ou se transforme, comme dit \emph{Boisguillebert}, de \emph{meuble en immeuble}, de numéraire en \emph{monnaie} ou\emph{ argent}, dès que la série des métamorphoses est interrompue, dès qu’une vente n’est pas suivie d’un achat subséquent.\par
Dès que se développe la circulation des marchandises, se développent aussi la nécessité et le désir de fixer et de conserver le produit de la première métamorphose, la marchandise changée en chrysalide d’or ou d’argent\footnote{« Une richesse en argent n’est que… richesse en productions, converties en argent. » (Mercier de la Rivière, l. c., p. 557.) « Une valeur en productions n’a fait que changer de forme. » (\emph{Id}., p. 485.)}. On vend dès lors des marchandises non seulement pour en acheter d’autres, mais aussi pour remplacer la forme marchandise par la forme argent. La monnaie arrêtée à dessein dans sa circulation se pétrifie, pour ainsi dire, en devenant trésor, et le vendeur se change en thésauriseur.\par
C’est surtout dans l’enfance de la circulation qu’on n’échange que le superflu en valeurs d’usage contre la marchandise‑monnaie. L’or et l’argent deviennent ainsi d’eux‑mêmes l’expression sociale du superflu et de la richesse. Cette forme naïve de thésaurisation s’éternise chez les peuples dont le mode traditionnel de production satisfait directement un cercle étroit de besoins stationnaires. Il y a peu de circulation et beaucoup de trésors. C’est ce qui a lieu chez les Asiatiques, notamment chez les Indiens. Le vieux Vanderlint, qui s’imagine que le taux des prix dépend de l’abondance des métaux précieux dans un pays, se demande pourquoi les marchandises indiennes sont à si bon marché ? Parce que les Indiens, dit‑il, enfouissent l’argent. Il remarque que de 1602 à 1734 ils enfouirent ainsi cent cinquante millions de livres sterling en argent, qui étaient venues d’abord d’Amérique en Europe\footnote{« C’est grâce à cet usage qu’ils maintiennent leurs articles et leurs manufactures à des taux aussi bas. » (Vanderlint, l. c., p. 95, 96.)}. De 1856 à 1866, dans une période de dix ans, l’Angleterre exporta dans l’Inde et dans la Chine (et le métal importé en Chine tenue en grande partie dans l’Inde), cent vingt millions de livres sterling en argent qui avaient été auparavant échangées contre de l’or australien.\par
Dès que la production marchande a atteint un certain développement, chaque producteur doit faire provision d’argent. C’est alors le « gage social », le \emph{nervus rerum}, le nerf des choses\footnote{« Money is a pledge. » (John Bellers, \emph{Essay about the Poor, manufactures, trade, plantations and immorality}, London, 1699, p. 13.)}. En effet, les besoins du producteur se renouvellent sans cesse et lui imposent sans cesse l’achat de marchandises étrangères, tandis que la production et la vente des siennes exigent plus ou moins de temps et dépendent de mille hasards. Pour acheter sans vendre, il doit d’abord avoir vendu sans acheter. Il semble contradictoire que cette opération puisse s’accomplir d’une manière générale. Cependant les métaux précieux se troquent à leur source de production contre d’autres marchandises. Ici la vente a lieu (du côté du possesseur de marchandises) sans achat (du côté du possesseur d’or et d’argent)\footnote{\emph{Achat}, dans le sens catégorique, suppose en effet que l’or ou l’argent dans les mains de l’èchangiste proviennent, non pas directement de son industrie, mais de la vente de sa marchandise.}. Et des ventes postérieures qui ne sont pas complétées par des achats subséquents ne font que distribuer les métaux précieux entre tous les échangistes. Il se forme ainsi sur tous les points en relation d’affaires des réserves d’or et d’argent dans les proportions les plus diverses. La possibilité de retenir et de conserver la marchandise comme valeur d’échange ou la valeur d’échange comme marchandise éveille la passion de l’or. A mesure que s’étend la circulation des marchandises grandit aussi la puissance de la monnaie, forme absolue et toujours disponible de la richesse sociale. « L’or est une chose merveilleuse ! Qui le possède est maître de tout ce qu’il désire. Au moyen de l’or on peut même ouvrir aux âmes les portes du Paradis. » (Colomb, \emph{lettre de la Jamaïque, 1503.)}\par
L’aspect de la monnaie ne trahissant point ce qui a été transformé en elle, tout, marchandise ou non, se transforme en monnaie. Rien qui ne devienne vénal, qui ne se fasse vendre et acheter ! La circulation devient la grande cornue sociale où tout se précipite pour en sortir transformé en cristal monnaie. Rien ne résiste à cette alchimie, pas même les os des saints et encore moins des choses sacrosaintes, plus délicates, \emph{res sacrosanctoe, extra commercium hominum}\footnote{Henri III, roi très‑chrétien de France, dépouille les cloîtres, les monastères, etc., de leurs reliques pour en faire de l’argent. On sait quel rôle a joué dans l’histoire grecque le pillage des trésors du temple de Delphes par les Phocéens. Les temples, chez les anciens, servaient de demeure au dieu des marchandises. C’étaient des « banques sacrées ». Pour les Phéniciens, peuple marchand par excellence, l’argent était l’aspect transfiguré de toutes choses. Il était donc dans l’ordre que les jeunes filles qui se livraient aux étrangers pour de l’argent dans les fêtes d’Astarté offrissent à la déesse les pièces d’argent reçues comme emblème de leur virginité immolée sur son autel.}. De même que toute différence de qualité entre les marchandises s’efface dans l’argent, de même lui, niveleur radical, efface toutes les distinctions\footnote{ \noindent Gold, yellow, glittering precious Gold !\par
 Thus much of this will make black white ; foul, fair ;\par
 Wrong, right ; base, noble ; old, young ; coward, valiant\par
 … What this, you Gods ! why ibis\par
 Will lug your priests and servants front your sides ;\par
 This yellow slave\par
 Will knit and break religions ; bless the accursed ;\par
 Make the hoar leprosy adored ; place thieves\par
 And give them, title, knee and approbation,\par
 With senators of the bench ; this is it,\par
 That makes, the wappend widow wed again\par
 … Come damned earth,\par
 Thou common whore of mankind\par
 « Or précieux, or jaune et luisant’ en voici assez pour rendre le noir blanc, le laid beau, l’injuste juste, le vil noble, le vieux jeune, le lâche vaillant !… Qu’est‑ce, cela, ô dieux immortels ? Cela, c’est ce qui détourne de vos autels vos prêtres et leurs acolytes Cet esclave jaune bâtit et démolit vos religions, fait bénir les maudits, adorer la lèpre blanche ; place les voleurs au banc des sénateurs et leur donne titres, hommages et génuflexions. C’est lui qui fait une nouvelle mariée de la veuve vieille et usée. Allons, argile damnée, catin du genre humain… » (Shakespeare, \emph{Timon of Athens}.)
}. Mais l’argent est lui‑même marchandise, une chose qui peut tomber sous les mains de qui que ce soit. La puissance sociale devient ainsi puissance privée des particuliers. Aussi la société antique le dénonce‑t‑elle comme l’agent subversif, comme le dissolvant le plus actif de son organisation économique et de ses mœurs populaires\footnote{« Rien n’a, comme l’argent, suscité parmi les hommes de mauvaises lois et tic mauvaises mœurs ; c’est lui qui met la discussion dans les villes et chasse les habitants de leurs demeures ; c’est lui qui détourne les âmes les plus belles vers tout ce qu’il y a de honteux et de funeste à l’homme et leur apprend à e xtraire de chaque chose le mal et l’impiété. » (Sophocle, \emph{Antigone}.)}.\par
La société moderne qui, à peine née encore, tire déjà par les cheveux le dieu Plutus des entrailles de la terre, salue dans l’or, son saint Graal, l’incarnation éblouissante du principe même de sa vie.\par
La marchandise, en tant que valeur d’usage, satisfait un besoin particulier et forme un élément particulier de la richesse matérielle. Mais la \emph{valeur} de la marchandise mesure le degré de sa force d’attraction sur tous les éléments de cette richesse, et par conséquent la \emph{richesse sociale} de celui qui la possède. L’échangiste plus ou moins barbare, même le paysan de l’Europe occidentale, ne sait point séparer la valeur de sa forme. Pour lui, accroissement de sa réserve d’or et d’argent veut dire accroissement de valeur. Assurément la valeur du métal précieux change par suite des variations survenues soit dans sa propre valeur soit dans celle des marchandises. Mais cela n’empêche pas d’un côté, que deux cents onces d’or contiennent après comme avant plus de valeur que cent, trois cents plus que deux cents, etc., ni d’un autre côté, que la forme métallique de la monnaie reste la forme équivalente générale de toutes les marchandises, l’incarnation sociale de tout travail humain. Le penchant à thésauriser n’a, de sa nature, ni règle ni mesure. Considéré au point de vue de la qualité ou de la forme, comme représentant universel de la richesse matérielle, l’argent est sans limite parce qu’il est immédiatement transformable en toute sorte de marchandise. Mais chaque somme d’argent réelle a sa limite quantitative et n’a donc qu’une puissance d’achat restreinte. Cette contradiction entre la quantité toujours définie et la qualité de puissance infinie de l’argent ramène sans cesse le thésauriseur au travail de Sisyphe. Il en est de lui comme du conquérant que chaque conquête nouvelle ne mène qu’à une nouvelle frontière.\par
Pour retenir et conserver le métal précieux en qualité de monnaie, et par suite d’élément de la thésaurisation, il faut qu’on l’empêche de circuler ou de se résoudre comme \emph{moyen d’achat} en moyens de jouissance. Le thésauriseur sacrifie donc à ce fétiche tous les penchants de sa chair. Personne plus que lui ne prend au sérieux l’évangile du renoncement. D’un autre côté, il ne peut dérober en monnaie à la, circulation que ce qu’il lui donne en marchandises. Plus il produit, plus il peut vendre. Industrie, économie, avarice, telles sont ses vertus cardinales ; beaucoup vendre, peu acheter, telle est la somme de son économie politique\footnote{« Accroître autant que possible le nombre des vendeurs de toute marchandise, diminuer autant que possible le nombre des acheteurs, tel est le résumé des opérations de l’économie politique. » (Verri, l. c., p. 52.)}.\par
Le trésor n’a pas seulement une forme brute : il a aussi une forme esthétique. C’est l’accumulation d’ouvrages d’orfèvrerie qui se développe avec l’accroissement de la richesse sociale. « Soyons riches ou paraissons riches. » (Diderot.) Il se forme ainsi d’une part un marché toujours plus étendu pour les métaux précieux, de l’autre une source latente d’approvisionnement à laquelle on puise dans les périodes de crise sociale.\par
Dans l’économie de la circulation métallique, les trésors remplissent des fonctions diverses. La première tire son origine des conditions qui président au cours de la monnaie. On a vu comment la masse courante du numéraire s’élève ou s’abaisse avec les fluctuations constantes qu’éprouve la circulation des marchandises sous le rapport de l’étendue, des prix et de la vitesse. Il faut donc que cette masse soit capable de contraction et d’expansion.\par
Tantôt une partie de la monnaie doit sortir de la circulation, tantôt elle y doit rentrer. Pour que la masse d’argent courante corresponde toujours au degré où la sphère de la circulation se trouve saturée, ta quantité d’or ou d’argent qui réellement circule ne doit former qu’une partie du métal précieux existant dans un pays. C’est par la forme trésor de l’argent que cette condition se trouve remplie. Les réservoirs des trésors servent à la fois de canaux de décharge et d’irrigation, de façon que les canaux de circulation ne débordent jamais\footnote{« Pour faire marcher le commerce d’une nation, il faut une somme de monnaie déterminée, qui varie et se trouve tantôt plus grande, tantôt plus petite… Ce flux et reflux de la monnaie s’équilibre de lui‑même, sans le secours des politiques… Les pistons travaillent alternativement ; si la monnaie est rare, on monnaye les lingots ; si les lingots sont rares, on fond la monnaie. » (Sir D. North, l. c., p. 22.) John Stuart Mill, longtemps fonctionnaire de la Compagnie des Indes, confirme ce fait que les ornements et bijoux en argent sont encore employés dans l’Inde comme réserves. « On sort les ornements d’argent et on les monnaye quand le taux de l’intérêt est élevé, et ils retournent à leurs possesseurs quand le taux de l’intérêt baisse. » (J. St. Mill, Evidence, \emph{Reports on Bankacts}, 1857, n° 2084). D’après un document parlementaire de 1864 sur l’importation et l’exportation de l’or et de l’argent dans l’Inde, l’importation en 1863 dépassa l’exportation de dix‑neuf millions trois cent soixante‑sept mille sept cent soixante‑quatre livres sterling. Dans les huit années avant 1864, l’excédent de l’importation des métaux précieux sur leur exportation atteignit cent neuf millions six cent cinquante‑deux mille neuf cent dix‑sept livres sterling. Dans le cours de ce siècle, il a été monnayé dans l’Inde plus de deux cents millions de livres sterling.}.
\subsubsection[{1.1.3.3.2. Moyen de payement.}]{1.1.3.3.2. Moyen de payement.}
\noindent Dans la forme immédiate de la circulation des marchandises examinée jusqu’ici, la même valeur se présente toujours double, marchandise à un pôle, monnaie à l’autre. Les producteurs-échangistes entrent en rapport comme représentants d’équivalents qui se trouvent déjà en face les uns des autres. A mesure cependant que se développe la circulation, se développent aussi des circonstances tendant à séparer par un intervalle de temps l’aliénation de la marchandise et la réalisation de son prix. Les exemples les plus simples nous suffisent ici. Telle espèce de marchandise exige plus de temps pour sa production, telle autre en exige moins. Les saisons de production ne sont pas les mêmes pour des marchandises différentes. Si une marchandise prend naissance sur le lieu même de son marché, une autre doit voyager et se rendre à un marché lointain. Il se peut donc que l’un des échangistes soit prêt à vendre, tandis que l’autre n’est pas encore à même d’acheter. Quand les mêmes transactions se renouvellent constamment entre les mêmes personnes les conditions de la vente et de l’achat des marchandises se régleront peu à peu d’après les conditions de leur production. D’un autre côté, l’usage de certaines espèces de marchandise, d’une maison, par exemple, est aliéné pour une certaine période, et ce n’est qu’après l’expiration du terme que l’acheteur a réellement obtenu la valeur d’usage stipulée. Il achète donc avant de payer. L’un des échangistes vend une marchandise présente, l’autre achète comme représentant d’argent à venir. Le vendeur devient créancier, l’acheteur débiteur. Comme la métamorphose de la marchandise prend ici un nouvel aspect, l’argent lui aussi acquiert une nouvelle fonction. Il devient moyen de payement.\par
Les caractères de créancier et de débiteur proviennent ici de la circulation simple. Le changement de sa forme imprime au vendeur et à l’acheteur leurs cachets nouveaux. Tout d’abord, ces nouveaux rôles sont donc aussi passagers que les anciens et joués tour à tour par les mêmes acteurs, mais ils n’ont plus un aspect aussi débonnaire, et leur opposition devient plus susceptible de se solidifier\footnote{Voici quels étaient les rapports de créanciers à débiteurs en Angleterre au commencement du XVIII° siècle : « Il règne ici, en Angleterre, un tel esprit de cruauté parmi les gens de commerce qu’on ne pourrait rencontrer rien de semblable dans aucune autre société d’hommes, ni dans aucun autre pays du monde. » (\emph{An Essay on Credit and the Bankrupt Act}, London, 1707, p. 2).}. Les mêmes caractères peuvent aussi se présenter indépendamment de la circulation des marchandises. Dans le monde antique, le mouvement de la lutte des classes a surtout la forme d’un combat, toujours renouvelé entre créanciers et débiteurs, et se termine à Rome par la défaite et la ruine du débiteur plébéien qui est remplacé par l’esclave. Au moyen âge, la lutte se termine par la ruine du débiteur féodal. Celui‑là perd la puissance politique dès que croule la base économique qui en faisait le soutien. Cependant ce rapport monétaire de créancier à débiteur ne fait à ces deux époques que réfléchir à la surface des antagonismes plus profonds.\par
Revenons à la circulation des marchandises. L’apparition simultanée des équivalents marchandise et argent aux deux pôles de la vente a cessé. Maintenant l’argent fonctionne en premier lieu comme mesure de valeur dans la fixation du prix de la marchandise vendue. Ce prix établi par contrat, mesure l’obligation de l’acheteur, c’est‑à‑dire la somme d’argent dont il est redevable à terme fixe.\par
Puis il fonctionne comme moyen d’achat idéal. Bien qu’il n’existe que dans la promesse de l’acheteur, il opère cependant le déplacement de la marchandise. Ce n’est qu’à l’échéance du terme qu’il entre, comme moyen de payement, dans la circulation, c’est‑à‑dire qu’il passe de la main de l’acheteur dans celle du vendeur. Le moyen de circulation s’était transformé en trésor, parce que le mouvement de la circulation s’était arrêté à sa première moitié. Le moyen de payement entre dans la circulation, mais seulement après que la marchandise en est sortie. Le vendeur transformait la marchandise en argent pour satisfaire ses besoins, le thésauriseur pour la conserver sous forme d’équivalent général, l’acheteur‑débiteur enfin pour pouvoir payer. S’il ne paye pas, une vente forcée de son avoir a lieu. La conversion de la marchandise en sa figure valeur, en monnaie, devient ainsi une nécessité sociale qui s’impose au producteur‑échangiste indépendamment de ses besoins et de ses fantaisies personnelles.\par
Supposons que le paysan achète du tisserand vingt mètres de toile au prix de deux livres sterling, qui est aussi le prix d’un quart de froment, et qu’il les paye un mois après. Le paysan transforme son froment en toile avant de l’avoir transformé en monnaie. Il accomplit donc la dernière métamorphose de sa marchandise avant la première. Ensuite il vend du froment pour deux livres sterling, qu’il fait passer au tisserand au terme convenu. La monnaie réelle ne lui sert plus ici d’intermédiaire pour substituer la toile au froment. C’est déjà fait. Pour lui la monnaie est au contraire le dernier mot de la transaction en tant qu’elle est la forme absolue de la valeur qu’il doit fournir, la marchandise universelle. Quant au tisserand, sa marchandise a circulé et a réalisé son prix, mais seulement au moyen d’un titre qui ressortit du droit civil. Elle est entrée dans la consommation d’autrui avant d’être transformée en monnaie. La première métamorphose de sa toile reste donc suspendue et ne s’accomplit que plus tard, au terme d’échéance de la dette du paysan\footnote{La citation suivante empruntée à mon précédent ouvrage, \emph{Critique de l’économie politique}, 1859, montre pourquoi je n’ai pas parlé dans le texte d’une forme opposée. « Inversement, dans le procédé A ‑ M, l’argent peut être mis dehors comme moyen d’achat et le prix de la marchandise être ainsi réalisé avant que la valeur d’usage de l’argent soit réalisée ou la marchandise aliénée. C’est ce qui a lieu tous les jours, par exemple, sous forme de prénumération, et c’est ainsi que le gouvernement anglais achète dans l’Inde l’opium des Ryots. Dans ces cas cependant, l’argent agit toujours comme moyen d’achat et n’acquiert aucune nouvelle forme particulière… Naturellement, le capital est aussi avance sous forme argent ; mais il ne se montre pas encore à l’horizon de la circulation simple. » (L. c., p. 112‑120.)}.\par
Les obligations échues dans une période déterminée représentent le prix total des marchandises vendues. La quantité de monnaie exigée pour la réalisation de cette somme dépend d’abord de la vitesse du cours des moyens de payement. Deux circonstances la règlent :\par

\begin{enumerate}[itemsep=0pt,]
\item l’enchaînement des rapports de créancier à débiteur, comme lorsque A, par exemple, qui reçoit de l’argent de son débiteur B, le fait passer à son créancier C, et ainsi de suite ;
\item l’intervalle de temps qui sépare les divers termes auxquels les payements s’effectuent.
\end{enumerate}

\noindent La série des payements consécutifs ou des premières métamorphoses supplémentaires se distingue tout à fait de l’entrecroisement des séries de métamorphoses que nous avons d’abord analysé.\par
Non seulement la connexion entre vendeurs et acheteurs s’exprime dans le mouvement des moyens de circulation. Mais cette connexion naît dans le cours même de la monnaie. Le mouvement du moyen de payement au contraire exprime un ensemble de rapports sociaux préexistants.\par
La simultanéité et contiguïté des ventes (ou achats), qui fait que la quantité des moyens de circulation ne peut plus être compensée par la vitesse de leur cours, forme un nouveau levier dans l’économie des moyens de payement. Avec la concentration des payements sur une même place se développent spontanément des institutions et des méthodes pour les balancer les uns par les autres. Tels étaient, par exemple, à Lyon, au moyen âge, les virements. Les créances de A sur B, de B sur C, de C sur A, et ainsi de suite, n’ont besoin que d’être confrontées pour s’annuler réciproquement, dans une certaine mesure, comme quantités positives et négatives. Il ne reste plus ainsi qu’une balance de compte à solder. Plus est grande la concentration des payements, plus est relativement petite leur balance, et par cela même la masse des moyens de payement en circulation.\par
La fonction de la monnaie comme moyen de payement implique une contradiction sans moyen terme. Tant que les payements se balancent, elle fonctionne seulement d’une manière idéale, comme monnaie de compte et mesure des valeurs. Dès que les payements doivent s’effectuer réellement, elle ne se présente plus comme simple moyen de circulation, comme forme transitive servant d’intermédiaire au déplacement des produits, mais elle intervient comme incarnation individuelle du travail social, seule réalisation de la valeur d’échange, marchandise absolue. Cette contradiction éclate dans le moment des crises industrielles ou commerciales auquel on a donné le nom de crise monétaire\footnote{Il faut distinguer la crise monétaire dont nous parlons ici, et qui est une phase de n’importe quelle crise, de cette espèce de crise particulière, à laquelle on donne le même nom, mais qui peut former néanmoins un phénomène indépendant, de telle sorte que son action n’influe que par contrecoup sur l’industrie et le commerce. Les crises de ce genre ont pour pivot le capital‑argent et leur sphère immédiate est aussi celle de ce capital, ‑ la Banque, la Bourse et la Finance.}.\par
Elle ne se produit que là où l’enchaînement des payements et un système artificiel destiné à les compenser réciproquement se sont développés. Ce mécanisme vient‑il, par une cause quelconque, à être dérangé, aussitôt la monnaie, par un revirement brusque et sans transition, ne fonctionne plus sous sa forme purement idéale de monnaie de compte. Elle est réclamée comme argent comptant et ne peut plus être remplacée par des marchandises profanes. L’utilité de la marchandise ne compte pour rien et sa valeur disparaît devant ce qui n’en est que la forme. La veille encore, le bourgeois, avec la suffisance présomptueuse que lui donne la prospérité, déclarait que l’argent est une vaine illusion. La marchandise seule est argent, s’écriait‑il. L’argent seul est marchandise ! Tel est maintenant le cri qui retentit sur le marché du monde. Comme le cerf altéré brame après la source d’eau vive, ainsi son âme appelle à grands cris l’argent, la seule et unique richesse\footnote{« Le revirement subit du système de crédit en système monétaire ajoute l’effroi théorique à la panique pratique, et les agents de la circulation tremblent devant le mystère impénétrable de leurs propres rapports. » (Karl Marx, l. c., p. 126.) – « Le pauvre reste morne et étonne de ce que le riche n’a plus d’argent pour le faire travailler, et cependant le même soi et les mêmes mains qui fournissent la nourriture et les vêtements, sont toujours là ‑ et c’est là ce qui constitue la véritable richesse d’une nation, et non pas l’argent. » (John Bellers, \emph{Proposals for raising a College of Industry}, London, 1696, p. 33.)}. L’opposition qui existe entre la marchandise et sa forme valeur est, pendant la crise, poussée à l’outrance. Le genre particulier de la monnaie n’y fait rien. La disette monétaire reste la même, qu’il faille payer en or ou en monnaie de crédit, en billets de banque, par exemple\footnote{Voici de quelle façon ces moments‑là sont exploités : « Un jour (1839), un vieux banquier de la Cité causant avec un de ses amis dans son cabinet, souleva le couvercle du pupitre devant lequel il était assis et se mit à déployer des rouleaux de billets de banque. En voilà, dit‑il d’un air tout joyeux, pour cent mille livres sterling. Ils sont là en réserve pour tendre la situation monétaire (to make the money tight) et ils seront tous dehors à 3 heures, cet après‑midi. » \emph{(The Theory of the Exchanges, the Bank Charter Art of 1844}, London, 1864 p. 81.) L’organe semi‑officiel, l’\emph{Observer}, publiait à la date du 28 avril 1864 : « Il court certains bruits vraiment curieux sur les moyens auxquels on a eu recours pour créer une disette de billets de banque. Bien qu’il soit fort douteux, qu’on ait eu recours à quelque artifice de ce genre, la rumeur qui s’en est répandue a été si générale qu’elle mérite réellement d’être mentionnée. »}.\par
Si nous examinons maintenant la somme totale de la monnaie qui circule dans un temps déterminé, nous trouverons qu’étant donné la vitesse du cours des moyens de circulation et des moyens de payement, elle est égale à la somme des prix des marchandises à réaliser, plus la somme des payements échus, moins celle des payements qui se balancent, moins enfin l’emploi double ou plus fréquent des mêmes pièces pour la double fonction de moyen de circulation et de moyen de payement. Par exemple, le paysan a vendu son froment moyennant deux livres sterling qui opèrent comme moyen de circulation. Au terme d’échéance, il les fait passer au tisserand. Maintenant elles fonctionnent comme moyen de payement. Le tisserand achète avec elles une bible, et dans cet achat elles fonctionnent de nouveau comme moyen de circulation, et ainsi de suite.\par
Etant donné la vitesse du cours de la monnaie, l’économie des payements et les prix des marchandises, on voit que la masse des marchandises en circulation ne correspond plus à la masse de la monnaie courante dans une certaine période, un jour, par exemple. Il court de la monnaie qui représente des marchandises depuis longtemps dérobées à la circulation. Il court des marchandises dont l’équivalent en monnaie ne se présentera que bien plus tard. D’un autre côté, les dettes contractées et les dettes échues chaque jour sont des grandeurs tout à fait incommensurables\footnote{ \noindent « Le montant des ventes ou achats contractés dans le cours d’un jour quelconque n’affectera en rien la quantité de la monnaie en circulation ce jour‑là même, mais pour la plupart des cas, il se résoudra en une multitude de traites sur la quantité d’argent qui peut se trouver en circulation à des dates ultérieures plus ou moins éloignées. ‑ Il n’est pas nécessaire que les billets signés ou les crédits ouverts aujourd’hui aient un rapport quelconque relativement, soit à la quantité, au montant ou à la durée, avec ceux qui seront signés ou contractés demain ou après‑demain ; bien plus, beaucoup de billets et de crédits d’aujourd’hui se présentent à l’échéance avec une masse de payements, dont l’origine embrasse une suite de dates antérieures absolument indéfinies ; ainsi, souvent des billets à douze, six, trois et un mois, réunis ensemble, entrent dans la masse commune des payements à effectuer le même jour. » (\emph{The Currency question reviewed ; a letter to the Scotch people by a banker in England}, Edimburg, 1845, p. 29, 30, \emph{passim}.)
 }.\par
La monnaie de crédit a sa source immédiate dans la fonction de l’argent comme moyen de payement. Des certificats constatant les dettes contractées pour des marchandises vendues circulent eux‑mêmes à leur tour pour transférer à d’autres personnes les créances. A mesure que s’étend le système de crédit, se développe de plus en plus la fonction que la monnaie remplit comme moyen de payement. Comme tel, elle revêt des formes d’existence particulières dans lesquelles elle hante la sphère des grandes transactions commerciales, tandis que les espèces d’or et d’argent sont refoulées principalement dans la sphère du commerce de détail\footnote{ \noindent Pour montrer par un exemple dans quelle faible proportion l’argent comptant entre dans les opérations commerciales proprement dites, nous donnons ici le tableau des recettes et des dépenses annuelles d’une des plus grandes maisons de commerce de Londres. Ses transactions dans l’année 1856, lesquelles comprennent bien des millions de livres sterling, sont ici ramenées à l’échelle d’un million :\par
 (\emph{Report from the select Committee on the Bank‑acts}, juillet 1858, p. 71.)
}.\par
Plus la production marchande se développe et s’étend, moins la fonction de la monnaie comme moyen de payement est restreinte à la sphère de la circulation des produits. La monnaie devient la marchandise générale des contrats\footnote{. « Des que le train du commerce est ainsi changé, qu’on n’échange plus marchandise contre marchandise, mais qu’on \emph{vend et qu’on paie}, tous les \emph{marchés} s’établissant sur le \emph{pied d’un prix} en monnaie. » (\emph{An Essay upon Publick Credit}, 2° éd., London, 1710, p. 8.)}. Les rentes, les impôts, etc., payés jusqu’alors en nature, se payent désormais en argent. Un fait qui démontre, entre autres, combien ce changement dépend des conditions générales de la production, c’est que I’empire romain échoua par deux fois dans sa tentative de lever toutes les contributions en argent. La misère énorme de la population agricole en France sous Louis XIV, dénoncée avec tant d’éloquence par Boisguillebert, le maréchal Vauban, etc., ne provenait pas seulement de l’élévation de l’impôt, mais aussi de la substitution de sa forme monétaire à sa forme naturelle\footnote{« L’argent est devenu le bourreau de toutes choses. » ‑ « La finance est l’alambic qui a fait évaporer une quantité effroyable de biens et de denrées pour faire ce fatal précis. ‑ L’argent déclare la guerre à tout le genre humain. " (Boisguillebert, \emph{Dissertation sur la nature des richesses, de l’argent et des tributs}, édit. Daire ; \emph{Economistes financiers}, Paris, 1843, p. 413, 417, 419.)}. En Asie, la rente foncière constitue l’élément principal des impôts et se paye en nature. Cette forme de la rente, qui repose là sur des rapports de production stationnaires, entretient par contrecoup l’ancien mode de production. C’est un des secrets de la conservation de l’empire turc. Que le libre commerce, octroyé par l’Europe au Japon, amène dans ce pays la conversion de la rente‑nature en rente‑argent, et c’en est fait de son agriculture modèle, soumise à des conditions économiques trop étroites pour résister à une telle révolution.\par
Il s’établit dans chaque pays certains termes généraux où les payements se font sur une grande échelle. Si quelques‑uns de ces termes sont de pure convention, ils reposent en général sur les mouvements périodiques et circulatoires de la reproduction liés aux changements périodiques des saisons, etc. Ces termes généraux règlent également l’époque des payements qui ne résultent pas directement de la circulation des marchandises, tels que ceux de la rente, du loyer, des impôts, etc. La quantité de monnaie qu’exigent à certains jours de l’année ces payements disséminés sur toute la périphérie d’un pays occasionne des perturbations périodiques, mais tout à fait superficielles\footnote{« Le lundi de la Pentecôte 1824, raconte M. Kraig à la Commission d’enquête parlementaire de 1826, il y eut une demande si considérable de billets de banque à Edimbourg, qu’à 11 heures du matin nous n’en avions plus un seul dans notre portefeuille. Nous en envoyâmes chercher dans toutes les banques, les unes après les autres, sans pouvoir en obtenir, et beaucoup d’atfaires ne purent être conclues que sur des morceaux de papier. A 3 heures de l’après‑midi, cependant, tous les billets étaient de retour aux banques d’où ils étaient partis ; ils n’avaient fait que changer de mains. » Bien que la circulation effective moyenne des billets de banque en Écosse n’atteigne pas trois millions de livres sterling, il arrive cependant qu’à certains termes de payement dans l’année, tous les billets qui se trouvent entre les mains des banquiers, à peu près sept millions de livres sterling, sont appelés à l’activité. « Dans les circonstances de ce genre, les billets n’ont qu’une seule fonction à remplir, et dès qu’ils s’en sont acquittés, ils reviennent aux différentes banques qui les ont émis. » (John Fullarton, \emph{Regulation of Currencies}, 2° éd., London, 1845, p. 86, note.) Pour faire comprendre ce qui précède il est bon d’ajouter qu’au temps de Fullarton les banques d’Écosse donnaient contre les dépôts, non des chèques, mais des billets.}.\par
Il résulte de la loi sur la vitesse du cours des moyens de payement, que pour tous les payements périodiques, quelle qu’en soit la source, la masse des moyens de payement nécessaire est en raison inverse de la longueur des périodes\footnote{« Dans un cas où il faudrait quarante millions par an, les mêmes six millions (en or) pourraient‑ils suffire aux \emph{circulations} et aux \emph{évolutions} commerciales ? » « Oui répond Petty avec sa supériorité habituelle. Si les évolutions se font dans des cercles rapprochés, \emph{chaque semaine} par exemple, comme cela a lieu pour les pauvres ouvriers et artisans qui reçoivent et payent tous les samedis, alors 40/52 de un million en monnaie, permettront d’atteindre le but. Si les cercles d’évolution sont trimestriels, suivant notre coutume de payer la rente ou de percevoir l’impôt, dix millions seront nécessaires. Donc si nous supposons que les payements en général s’effectuent entre une semaine et trois, il faudra alors ajouter dix millions à 40/52, dont la moitié est cinq millions et demi de sorte que si nous avons cinq millions et demi, nous avons assez. » (William Petty, \emph{Political anatomy of Ireland}, 1672, édit., London, 1691, p. 13, 14.)}.\par
La fonction que l’argent remplit comme moyen de payement nécessite l’accumulation des sommes exigées pour les dates d’échéance. Tout en éliminant la thésaurisation comme forme propre d’enrichissement, le progrès de la société bourgeoise la développe sous la forme de réserve des moyens de payement.
\subsubsection[{1.1.3.3.3. La monnaie universelle.}]{1.1.3.3.3. La monnaie universelle.}
\noindent A sa sortie de la sphère intérieure de la circulation, l’argent dépouille les formes locales qu’il y avait revêtues, forme de numéraire, de monnaie d’appoint, d’étalon des prix, de signe de valeur, pour retourner à sa forme primitive de barre ou lingot. C’est dans le commerce entre nations que la valeur des marchandises se réalise universellement. C’est là aussi que leur figurevaleur leur fait vis‑à‑vis, sous l’aspect de monnaie universelle monnaie du monde (money of the world), comme l’appelle James Steuart, monnaie de la grande république commerçante, comme disait après lui Adam Smith. C’est sur le marché du monde et là seulement que la monnaie fonctionne dans toute la force du terme, comme la marchandise dont la forme naturelle est en même temps l’incarnation sociale du travail humain en général. Sa manière d’être y devient adéquate à son idée. Dans l’enceinte nationale de la circulation, ce n’est qu’une seule marchandise qui peut servir de mesure de valeur et par suite de monnaie. Sur le marché du monde règne une double mesure de valeur, l’or et l’argent\footnote{C’est ce qui démontre l’absurdité de toute législation qui prescrit aux banques nationales de ne tenir en réserve que le métal précieux qui fonctionne comme monnaie dans l’intérieur du pays. Les difficultés que s’est ainsi créées volontairement la banque d’Angleterre, par exemple, sont connues. Dans le \emph{Bank‑act} de 1844, Sir Robert Peel chercha à remédier aux inconvénients, en permettant à la banque d’émettre des billets sur des lingots d’argent, à la condition cependant que la réserve d’argent ne dépasserait jamais d’un quart la réserve d’or. Dans ces circonstances, la valeur de l’argent est estimée chaque fois d’après son prix en or sur le marché de Londres. ‑ Sur les grandes époques historiques du changement de la valeur relative de l’or et de l’argent, V. Karl Marx, l. c., p. 136 et suiv.}.\par
La monnaie universelle remplit les trois fonctions de moyen de payement, de moyen d’achat et de matière sociale de la richesse, en général (universal wealth). Quand il s’agit de solder les balances internationales, la première fonction prédomine. De là le mot d’ordre du système mercantile ‑ balance de commerce\footnote{Les adversaires du système mercantile, d’après lequel le but du commerce international n’est pas autre chose que le solde en or ou en argent de l’excédent d’une balance de commerce sur l’autre, méconnaissaient complètement de leur côté la fonction de la monnaie universelle. La fausse interprétation du mouvement international des métaux précieux, n’est que le reflet de la fausse interprétation des lois qui règlent la masse des moyens de la circulation intérieure, ainsi que je l’ai montré par l’exemple de Ricardo (l. c., p. 150). Son dogme erroné : « Une balance de commerce défavorable ne provient jamais que de la surabondance de la monnaie courante… » « l’exportation de la monnaie est causée par son bas prix, et n’est point l’effet, mais la cause d’une balance défavorable » se trouve dêiâ chez \emph{Barbon : « La balance du commerce}, s’il y en a une, \emph{n’est point la cause de l’exportation de la monnaie d’une nation ci l’étranger},, mais elle provient de \emph{la différence de valeur de l’or ou de l’argent en lingots dans chaque pays. »} (N. Barbon, l. c., p. 59, 60.) Mac Culloch, dans sa \emph{Literature of Political Economy, a classified catalogue}, London, 1845, loue Barbon pour cette anticipation, mais évite avec soin de dire un seul mot des formes naïves sous lesquelles se montrent encore chez ce dernier les suppositions absurdes du « currency principle ». L’absence de critique et même la déloyauté de ce catalogue éclatent surtout dans la partie qui traite de l’histoire de la théorie de l’argent. La raison en est que le sycophante Mac Culloch fait ici sa cour à Lord Overstone (l’ex‑banquier Loyd), qu’il désigne sous le nom de « facile princeps argentariorum ».}. L’or et l’argent servent essentiellement de moyen d’achat international toutes les fois que l’équilibre ordinaire dans l’échange des matières entre diverses nations se dérange. Enfin, ils fonctionnent comme forme absolue de la richesse, quand il ne s’agit plus ni d’achat ni de payement, mais d’un transfert de richesse d’un pays à un autre, et que ce transfert, sous forme de marchandise, est empêché, soit par les éventualités du marché, soit par le but même qu’on veut atteindre\footnote{Par exemple, la forme‑monnaie de la valeur peut être de rigueur dans les cas de subsides, d’emprunts contractés pour faire la guerre ou mettre une banque à même de reprendre le payement de ses billets, etc.}.\par
Chaque pays a besoin d’un fonds de réserve pour son commerce étranger, aussi bien que pour sa circulation intérieure. Les fonctions de ces réserves se rattachent donc en partie à la fonction de la monnaie comme moyen de circulation et de payement à l’intérieur, et en partie à sa fonction de monnaie universelle\footnote{« Il n’est pas, selon moi, de preuve plus convaincante de l’aptitude des fonds de réserve à mener à bon terme toutes les affaires internationales, sans aucun recours à la circulation générale, que la facilité avec laquelle la France, à peine revenue du choc d’une invasion étrangère, compléta dans l’espace de vingt‑sept mois le payement d’une contribution forcée de près de vingt millions de livres exigés par les Puissances alliées, et en fournit la plus grande partie en espèces, sans le moindre dérangement dans son commerce intérieur et même sans fluctuations alarmantes dans ses échanges. » (Fullarton, l. c., p. 141.)}. Dans cette dernière fonction, la monnaie matérielle, c’est‑à‑dire l’or et l’argent, est toujours exigée ; c’est pourquoi James Steuart, pour distinguer l’or et l’argent de leurs remplaçants purement locaux, les désigne expressément sous le nom de \emph{money of the world.}\par
Le fleuve aux vagues d’argent et d’or possède un double courant. D’un côté, il se répand à partir de sa source sur tout le marché du monde où les différentes enceintes nationales le détournent en proportions diverses, pour qu’il pénètre leurs canaux de circulation intérieure, remplace leurs monnaies usées, fournisse la matière des articles de luxe, et enfin se pétrifie sous forme de trésor\footnote{« L’argent se partage entre les nations relativement au besoin qu’elles en ont… étant toujours attiré par les productions. » (Le Trosne, l. c., p. 916.) « Les mines qui fournissent continuellement de l’argent et de l’or en fournissent assez pour subvenir aux besoins de tous les pays. » (Vanderlint, l. c., p. 80.)}. Cette première direction lui est imprimée par les pays dont les marchandises s’échangent directement avec l’or et l’argent aux sources de leur production. En même temps, les métaux précieux courent de côté et d’autre, sans fin ni trêve, entre les sphères de circulation des différents pays, et ce mouvement suit les oscillations incessantes du cours du changes\footnote{« Le change subit chaque semaine des alternations de hausse et de baisse ; il se tourne à certaines époques de l’année contre un pays et se tourne en sa faveur à d’autres époques. » (N. Barbon, l. c., p. 39).}.\par
Les pays dans lesquels la production a atteint un haut degré de développement restreignent au minimum exigé par leurs fonctions spécifiques les trésors entassés dans les réservoirs de banque\footnote{Ces diverses fonctions peuvent entrer en un conflit dangereux, dès qu’il s’y joint la fonction d’un fonds de conversion pour les billets de banque.}. À part certaines exceptions, le débordement de ces réservoirs par trop au‑dessus de leur niveau moyen est un signe de stagnation dans la circulation des marchandises ou d’une interruption dans le cours de leurs métamorphoses\footnote{« Tout ce qui, en fait de monnaie, dépasse le strict nécessaire pour un commerce intérieur, est un capital mort et ne porte aucun profit au pays dans lequel il est retenu. » (John Bellers, l. c., p 12.) ‑ « Si nous avons trop de monnaie, que faire ? Il faut fondre celle qui a le plus de poids et la transformer en vaisselle splendide, en vases ou ustensiles d’or et d’argent, ou l’exporter comme une marchandise là où on la désire, ou la placer à intérêt là où l’intérêt est élevé. » (W. Petty, \emph{Quantulumeumque}, p. 39.) ‑ « La monnaie n’est, pour ainsi dire, que la graisse du corps politique ; trop nuit à son agilité, trop peu le rend malade… de même que la graisse lubrifie les muscles et favorise leurs mouvements, entretient le corps quand la nourriture fait défaut, remplit les cavités et donne un aspect de beauté à tout l’ensemble ; de même la monnaie, dans un État accélère son action, le fait vivre du dehors dans un temps de disette au‑dedans, règle les comptes… et embellit le tout, mais plus spécialement, ajoute Petty avec ironie, les particuliers qui la possèdent en abondance. » (W. Petty, \emph{Political anatomy of Ireland}, p. 14.)}.
 


% at least one empty page at end (for booklet couv)
\ifbooklet
  \pagestyle{empty}
  \clearpage
  % 2 empty pages maybe needed for 4e cover
  \ifnum\modulo{\value{page}}{4}=0 \hbox{}\newpage\hbox{}\newpage\fi
  \ifnum\modulo{\value{page}}{4}=1 \hbox{}\newpage\hbox{}\newpage\fi


  \hbox{}\newpage
  \ifodd\value{page}\hbox{}\newpage\fi
  {\centering\color{rubric}\bfseries\noindent\large
    Hurlus ? Qu’est-ce.\par
    \bigskip
  }
  \noindent Des bouquinistes électroniques, pour du texte libre à participation libre,
  téléchargeable gratuitement sur \href{https://hurlus.fr}{\dotuline{hurlus.fr}}.\par
  \bigskip
  \noindent Cette brochure a été produite par des éditeurs bénévoles.
  Elle n’est pas faîte pour être possédée, mais pour être lue, et puis donnée.
  Que circule le texte !
  En page de garde, on peut ajouter une date, un lieu, un nom ; pour suivre le voyage des idées.
  \par

  Ce texte a été choisi parce qu’une personne l’a aimé,
  ou haï, elle a en tous cas pensé qu’il partipait à la formation de notre présent ;
  sans le souci de plaire, vendre, ou militer pour une cause.
  \par

  L’édition électronique est soigneuse, tant sur la technique
  que sur l’établissement du texte ; mais sans aucune prétention scolaire, au contraire.
  Le but est de s’adresser à tous, sans distinction de science ou de diplôme.
  Au plus direct ! (possible)
  \par

  Cet exemplaire en papier a été tiré sur une imprimante personnelle
   ou une photocopieuse. Tout le monde peut le faire.
  Il suffit de
  télécharger un fichier sur \href{https://hurlus.fr}{\dotuline{hurlus.fr}},
  d’imprimer, et agrafer ; puis de lire et donner.\par

  \bigskip

  \noindent PS : Les hurlus furent aussi des rebelles protestants qui cassaient les statues dans les églises catholiques. En 1566 démarra la révolte des gueux dans le pays de Lille. L’insurrection enflamma la région jusqu’à Anvers où les gueux de mer bloquèrent les bateaux espagnols.
  Ce fut une rare guerre de libération dont naquit un pays toujours libre : les Pays-Bas.
  En plat pays francophone, par contre, restèrent des bandes de huguenots, les hurlus, progressivement réprimés par la très catholique Espagne.
  Cette mémoire d’une défaite est éteinte, rallumons-la. Sortons les livres du culte universitaire, cherchons les idoles de l’époque, pour les briser.
\fi

\ifdev % autotext in dev mode
\fontname\font — \textsc{Les règles du jeu}\par
(\hyperref[utopie]{\underline{Lien}})\par
\noindent \initialiv{A}{lors là}\blindtext\par
\noindent \initialiv{À}{ la bonheur des dames}\blindtext\par
\noindent \initialiv{É}{tonnez-le}\blindtext\par
\noindent \initialiv{Q}{ualitativement}\blindtext\par
\noindent \initialiv{V}{aloriser}\blindtext\par
\Blindtext
\phantomsection
\label{utopie}
\Blinddocument
\fi
\end{document}
