%%%%%%%%%%%%%%%%%%%%%%%%%%%%%%%%%
% LaTeX model https://hurlus.fr %
%%%%%%%%%%%%%%%%%%%%%%%%%%%%%%%%%

% Needed before document class
\RequirePackage{pdftexcmds} % needed for tests expressions
\RequirePackage{fix-cm} % correct units

% Define mode
\def\mode{a4}

\newif\ifaiv % a4
\newif\ifav % a5
\newif\ifbooklet % booklet
\newif\ifcover % cover for booklet

\ifnum \strcmp{\mode}{cover}=0
  \covertrue
\else\ifnum \strcmp{\mode}{booklet}=0
  \booklettrue
\else\ifnum \strcmp{\mode}{a5}=0
  \avtrue
\else
  \aivtrue
\fi\fi\fi

\ifbooklet % do not enclose with {}
  \documentclass[french,twoside]{book} % ,notitlepage
  \usepackage[%
    papersize={105mm, 297mm},
    inner=12mm,
    outer=12mm,
    top=20mm,
    bottom=15mm,
    marginparsep=0pt,
  ]{geometry}
  \usepackage[fontsize=9.5pt]{scrextend} % for Roboto
\else\ifav
  \documentclass[french,twoside]{book} % ,notitlepage
  \usepackage[%
    a5paper,
    inner=25mm,
    outer=15mm,
    top=15mm,
    bottom=15mm,
    marginparsep=0pt,
  ]{geometry}
  \usepackage[fontsize=12pt]{scrextend}
\else% A4 2 cols
  \documentclass[twocolumn]{report}
  \usepackage[%
    a4paper,
    inner=15mm,
    outer=10mm,
    top=25mm,
    bottom=18mm,
    marginparsep=0pt,
  ]{geometry}
  \setlength{\columnsep}{20mm}
  \usepackage[fontsize=9.5pt]{scrextend}
\fi\fi

%%%%%%%%%%%%%%
% Alignments %
%%%%%%%%%%%%%%
% before teinte macros

\setlength{\arrayrulewidth}{0.2pt}
\setlength{\columnseprule}{\arrayrulewidth} % twocol
\setlength{\parskip}{0pt} % classical para with no margin
\setlength{\parindent}{1.5em}

%%%%%%%%%%
% Colors %
%%%%%%%%%%
% before Teinte macros

\usepackage[dvipsnames]{xcolor}
\definecolor{rubric}{HTML}{800000} % the tonic 0c71c3
\def\columnseprulecolor{\color{rubric}}
\colorlet{borderline}{rubric!30!} % definecolor need exact code
\definecolor{shadecolor}{gray}{0.95}
\definecolor{bghi}{gray}{0.5}

%%%%%%%%%%%%%%%%%
% Teinte macros %
%%%%%%%%%%%%%%%%%
%%%%%%%%%%%%%%%%%%%%%%%%%%%%%%%%%%%%%%%%%%%%%%%%%%%
% <TEI> generic (LaTeX names generated by Teinte) %
%%%%%%%%%%%%%%%%%%%%%%%%%%%%%%%%%%%%%%%%%%%%%%%%%%%
% This template is inserted in a specific design
% It is XeLaTeX and otf fonts

\makeatletter % <@@@


\usepackage{blindtext} % generate text for testing
\usepackage[strict]{changepage} % for modulo 4
\usepackage{contour} % rounding words
\usepackage[nodayofweek]{datetime}
% \usepackage{DejaVuSans} % seems buggy for sffont font for symbols
\usepackage{enumitem} % <list>
\usepackage{etoolbox} % patch commands
\usepackage{fancyvrb}
\usepackage{fancyhdr}
\usepackage{float}
\usepackage{fontspec} % XeLaTeX mandatory for fonts
\usepackage{footnote} % used to capture notes in minipage (ex: quote)
\usepackage{framed} % bordering correct with footnote hack
\usepackage{graphicx}
\usepackage{lettrine} % drop caps
\usepackage{lipsum} % generate text for testing
\usepackage[framemethod=tikz,]{mdframed} % maybe used for frame with footnotes inside
\usepackage{pdftexcmds} % needed for tests expressions
\usepackage{polyglossia} % non-break space french punct, bug Warning: "Failed to patch part"
\usepackage[%
  indentfirst=false,
  vskip=1em,
  noorphanfirst=true,
  noorphanafter=true,
  leftmargin=\parindent,
  rightmargin=0pt,
]{quoting}
\usepackage{ragged2e}
\usepackage{setspace} % \setstretch for <quote>
\usepackage{tabularx} % <table>
\usepackage[explicit]{titlesec} % wear titles, !NO implicit
\usepackage{tikz} % ornaments
\usepackage{tocloft} % styling tocs
\usepackage[fit]{truncate} % used im runing titles
\usepackage{unicode-math}
\usepackage[normalem]{ulem} % breakable \uline, normalem is absolutely necessary to keep \emph
\usepackage{verse} % <l>
\usepackage{xcolor} % named colors
\usepackage{xparse} % @ifundefined
\XeTeXdefaultencoding "iso-8859-1" % bad encoding of xstring
\usepackage{xstring} % string tests
\XeTeXdefaultencoding "utf-8"
\PassOptionsToPackage{hyphens}{url} % before hyperref, which load url package

% TOTEST
% \usepackage{hypcap} % links in caption ?
% \usepackage{marginnote}
% TESTED
% \usepackage{background} % doesn’t work with xetek
% \usepackage{bookmark} % prefers the hyperref hack \phantomsection
% \usepackage[color, leftbars]{changebar} % 2 cols doc, impossible to keep bar left
% \usepackage[utf8x]{inputenc} % inputenc package ignored with utf8 based engines
% \usepackage[sfdefault,medium]{inter} % no small caps
% \usepackage{firamath} % choose firasans instead, firamath unavailable in Ubuntu 21-04
% \usepackage{flushend} % bad for last notes, supposed flush end of columns
% \usepackage[stable]{footmisc} % BAD for complex notes https://texfaq.org/FAQ-ftnsect
% \usepackage{helvet} % not for XeLaTeX
% \usepackage{multicol} % not compatible with too much packages (longtable, framed, memoir…)
% \usepackage[default,oldstyle,scale=0.95]{opensans} % no small caps
% \usepackage{sectsty} % \chapterfont OBSOLETE
% \usepackage{soul} % \ul for underline, OBSOLETE with XeTeX
% \usepackage[breakable]{tcolorbox} % text styling gone, footnote hack not kept with breakable


% Metadata inserted by a program, from the TEI source, for title page and runing heads
\title{\textbf{ La guerre des Gaules }}
\date{1926}
\author{César, Jules}
\def\elbibl{César, Jules. 1926. \emph{La guerre des Gaules}}
\def\elsource{ \href{http://ugo.bratelli.free.fr/Cesar/CesarGuerreGaules.htm}{\dotuline{http://ugo.bratelli.free.fr/Cesar/CesarGuerreGaules.htm}}\footnote{\href{http://ugo.bratelli.free.fr/Cesar/CesarGuerreGaules.htm}{\url{http://ugo.bratelli.free.fr/Cesar/CesarGuerreGaules.htm}}} }

% Default metas
\newcommand{\colorprovide}[2]{\@ifundefinedcolor{#1}{\colorlet{#1}{#2}}{}}
\colorprovide{rubric}{red}
\colorprovide{silver}{lightgray}
\@ifundefined{syms}{\newfontfamily\syms{DejaVu Sans}}{}
\newif\ifdev
\@ifundefined{elbibl}{% No meta defined, maybe dev mode
  \newcommand{\elbibl}{Titre court ?}
  \newcommand{\elbook}{Titre du livre source ?}
  \newcommand{\elabstract}{Résumé\par}
  \newcommand{\elurl}{http://oeuvres.github.io/elbook/2}
  \author{Éric Lœchien}
  \title{Un titre de test assez long pour vérifier le comportement d’une maquette}
  \date{1566}
  \devtrue
}{}
\let\eltitle\@title
\let\elauthor\@author
\let\eldate\@date


\defaultfontfeatures{
  % Mapping=tex-text, % no effect seen
  Scale=MatchLowercase,
  Ligatures={TeX,Common},
}


% generic typo commands
\newcommand{\astermono}{\medskip\centerline{\color{rubric}\large\selectfont{\syms ✻}}\medskip\par}%
\newcommand{\astertri}{\medskip\par\centerline{\color{rubric}\large\selectfont{\syms ✻\,✻\,✻}}\medskip\par}%
\newcommand{\asterism}{\bigskip\par\noindent\parbox{\linewidth}{\centering\color{rubric}\large{\syms ✻}\\{\syms ✻}\hskip 0.75em{\syms ✻}}\bigskip\par}%

% lists
\newlength{\listmod}
\setlength{\listmod}{\parindent}
\setlist{
  itemindent=!,
  listparindent=\listmod,
  labelsep=0.2\listmod,
  parsep=0pt,
  % topsep=0.2em, % default topsep is best
}
\setlist[itemize]{
  label=—,
  leftmargin=0pt,
  labelindent=1.2em,
  labelwidth=0pt,
}
\setlist[enumerate]{
  label={\bf\color{rubric}\arabic*.},
  labelindent=0.8\listmod,
  leftmargin=\listmod,
  labelwidth=0pt,
}
\newlist{listalpha}{enumerate}{1}
\setlist[listalpha]{
  label={\bf\color{rubric}\alph*.},
  leftmargin=0pt,
  labelindent=0.8\listmod,
  labelwidth=0pt,
}
\newcommand{\listhead}[1]{\hspace{-1\listmod}\emph{#1}}

\renewcommand{\hrulefill}{%
  \leavevmode\leaders\hrule height 0.2pt\hfill\kern\z@}

% General typo
\DeclareTextFontCommand{\textlarge}{\large}
\DeclareTextFontCommand{\textsmall}{\small}

% commands, inlines
\newcommand{\anchor}[1]{\Hy@raisedlink{\hypertarget{#1}{}}} % link to top of an anchor (not baseline)
\newcommand\abbr[1]{#1}
\newcommand{\autour}[1]{\tikz[baseline=(X.base)]\node [draw=rubric,thin,rectangle,inner sep=1.5pt, rounded corners=3pt] (X) {\color{rubric}#1};}
\newcommand\corr[1]{#1}
\newcommand{\ed}[1]{ {\color{silver}\sffamily\footnotesize (#1)} } % <milestone ed="1688"/>
\newcommand\expan[1]{#1}
\newcommand\foreign[1]{\emph{#1}}
\newcommand\gap[1]{#1}
\renewcommand{\LettrineFontHook}{\color{rubric}}
\newcommand{\initial}[2]{\lettrine[lines=2, loversize=0.3, lhang=0.3]{#1}{#2}}
\newcommand{\initialiv}[2]{%
  \let\oldLFH\LettrineFontHook
  % \renewcommand{\LettrineFontHook}{\color{rubric}\ttfamily}
  \IfSubStr{QJ’}{#1}{
    \lettrine[lines=4, lhang=0.2, loversize=-0.1, lraise=0.2]{\smash{#1}}{#2}
  }{\IfSubStr{É}{#1}{
    \lettrine[lines=4, lhang=0.2, loversize=-0, lraise=0]{\smash{#1}}{#2}
  }{\IfSubStr{ÀÂ}{#1}{
    \lettrine[lines=4, lhang=0.2, loversize=-0, lraise=0, slope=0.6em]{\smash{#1}}{#2}
  }{\IfSubStr{A}{#1}{
    \lettrine[lines=4, lhang=0.2, loversize=0.2, slope=0.6em]{\smash{#1}}{#2}
  }{\IfSubStr{V}{#1}{
    \lettrine[lines=4, lhang=0.2, loversize=0.2, slope=-0.5em]{\smash{#1}}{#2}
  }{
    \lettrine[lines=4, lhang=0.2, loversize=0.2]{\smash{#1}}{#2}
  }}}}}
  \let\LettrineFontHook\oldLFH
}
\newcommand{\labelchar}[1]{\textbf{\color{rubric} #1}}
\newcommand{\milestone}[1]{\autour{\footnotesize\color{rubric} #1}} % <milestone n="4"/>
\newcommand\name[1]{#1}
\newcommand\orig[1]{#1}
\newcommand\orgName[1]{#1}
\newcommand\persName[1]{#1}
\newcommand\placeName[1]{#1}
\newcommand{\pn}[1]{\IfSubStr{-—–¶}{#1}% <p n="3"/>
  {\noindent{\bfseries\color{rubric}   ¶  }}
  {{\footnotesize\autour{ #1}  }}}
\newcommand\reg{}
% \newcommand\ref{} % already defined
\newcommand\sic[1]{#1}
\newcommand\surname[1]{\textsc{#1}}
\newcommand\term[1]{\textbf{#1}}

\def\mednobreak{\ifdim\lastskip<\medskipamount
  \removelastskip\nopagebreak\medskip\fi}
\def\bignobreak{\ifdim\lastskip<\bigskipamount
  \removelastskip\nopagebreak\bigskip\fi}

% commands, blocks
\newcommand{\byline}[1]{\bigskip{\RaggedLeft{#1}\par}\bigskip}
\newcommand{\bibl}[1]{{\RaggedLeft{#1}\par\bigskip}}
\newcommand{\biblitem}[1]{{\noindent\hangindent=\parindent   #1\par}}
\newcommand{\dateline}[1]{\medskip{\RaggedLeft{#1}\par}\bigskip}
\newcommand{\labelblock}[1]{\medbreak{\noindent\color{rubric}\bfseries #1}\par\mednobreak}
\newcommand{\salute}[1]{\bigbreak{#1}\par\medbreak}
\newcommand{\signed}[1]{\bigbreak\filbreak{\raggedleft #1\par}\medskip}

% environments for blocks (some may become commands)
\newenvironment{borderbox}{}{} % framing content
\newenvironment{citbibl}{\ifvmode\hfill\fi}{\ifvmode\par\fi }
\newenvironment{docAuthor}{\ifvmode\vskip4pt\fontsize{16pt}{18pt}\selectfont\fi\itshape}{\ifvmode\par\fi }
\newenvironment{docDate}{}{\ifvmode\par\fi }
\newenvironment{docImprint}{\vskip6pt}{\ifvmode\par\fi }
\newenvironment{docTitle}{\vskip6pt\bfseries\fontsize{18pt}{22pt}\selectfont}{\par }
\newenvironment{msHead}{\vskip6pt}{\par}
\newenvironment{msItem}{\vskip6pt}{\par}
\newenvironment{titlePart}{}{\par }


% environments for block containers
\newenvironment{argument}{\itshape\parindent0pt}{\vskip1.5em}
\newenvironment{biblfree}{}{\ifvmode\par\fi }
\newenvironment{bibitemlist}[1]{%
  \list{\@biblabel{\@arabic\c@enumiv}}%
  {%
    \settowidth\labelwidth{\@biblabel{#1}}%
    \leftmargin\labelwidth
    \advance\leftmargin\labelsep
    \@openbib@code
    \usecounter{enumiv}%
    \let\p@enumiv\@empty
    \renewcommand\theenumiv{\@arabic\c@enumiv}%
  }
  \sloppy
  \clubpenalty4000
  \@clubpenalty \clubpenalty
  \widowpenalty4000%
  \sfcode`\.\@m
}%
{\def\@noitemerr
  {\@latex@warning{Empty `bibitemlist' environment}}%
\endlist}
\newenvironment{quoteblock}% may be used for ornaments
  {\begin{quoting}}
  {\end{quoting}}

% table () is preceded and finished by custom command
\newcommand{\tableopen}[1]{%
  \ifnum\strcmp{#1}{wide}=0{%
    \begin{center}
  }
  \else\ifnum\strcmp{#1}{long}=0{%
    \begin{center}
  }
  \else{%
    \begin{center}
  }
  \fi\fi
}
\newcommand{\tableclose}[1]{%
  \ifnum\strcmp{#1}{wide}=0{%
    \end{center}
  }
  \else\ifnum\strcmp{#1}{long}=0{%
    \end{center}
  }
  \else{%
    \end{center}
  }
  \fi\fi
}


% text structure
\newcommand\chapteropen{} % before chapter title
\newcommand\chaptercont{} % after title, argument, epigraph…
\newcommand\chapterclose{} % maybe useful for multicol settings
\setcounter{secnumdepth}{-2} % no counters for hierarchy titles
\setcounter{tocdepth}{5} % deep toc
\markright{\@title} % ???
\markboth{\@title}{\@author} % ???
\renewcommand\tableofcontents{\@starttoc{toc}}
% toclof format
% \renewcommand{\@tocrmarg}{0.1em} % Useless command?
% \renewcommand{\@pnumwidth}{0.5em} % {1.75em}
\renewcommand{\@cftmaketoctitle}{}
\setlength{\cftbeforesecskip}{\z@ \@plus.2\p@}
\renewcommand{\cftchapfont}{}
\renewcommand{\cftchapdotsep}{\cftdotsep}
\renewcommand{\cftchapleader}{\normalfont\cftdotfill{\cftchapdotsep}}
\renewcommand{\cftchappagefont}{\bfseries}
\setlength{\cftbeforechapskip}{0em \@plus\p@}
% \renewcommand{\cftsecfont}{\small\relax}
\renewcommand{\cftsecpagefont}{\normalfont}
% \renewcommand{\cftsubsecfont}{\small\relax}
\renewcommand{\cftsecdotsep}{\cftdotsep}
\renewcommand{\cftsecpagefont}{\normalfont}
\renewcommand{\cftsecleader}{\normalfont\cftdotfill{\cftsecdotsep}}
\setlength{\cftsecindent}{1em}
\setlength{\cftsubsecindent}{2em}
\setlength{\cftsubsubsecindent}{3em}
\setlength{\cftchapnumwidth}{1em}
\setlength{\cftsecnumwidth}{1em}
\setlength{\cftsubsecnumwidth}{1em}
\setlength{\cftsubsubsecnumwidth}{1em}

% footnotes
\newif\ifheading
\newcommand*{\fnmarkscale}{\ifheading 0.70 \else 1 \fi}
\renewcommand\footnoterule{\vspace*{0.3cm}\hrule height \arrayrulewidth width 3cm \vspace*{0.3cm}}
\setlength\footnotesep{1.5\footnotesep} % footnote separator
\renewcommand\@makefntext[1]{\parindent 1.5em \noindent \hb@xt@1.8em{\hss{\normalfont\@thefnmark . }}#1} % no superscipt in foot
\patchcmd{\@footnotetext}{\footnotesize}{\footnotesize\sffamily}{}{} % before scrextend, hyperref


%   see https://tex.stackexchange.com/a/34449/5049
\def\truncdiv#1#2{((#1-(#2-1)/2)/#2)}
\def\moduloop#1#2{(#1-\truncdiv{#1}{#2}*#2)}
\def\modulo#1#2{\number\numexpr\moduloop{#1}{#2}\relax}

% orphans and widows
\clubpenalty=9996
\widowpenalty=9999
\brokenpenalty=4991
\predisplaypenalty=10000
\postdisplaypenalty=1549
\displaywidowpenalty=1602
\hyphenpenalty=400
% Copied from Rahtz but not understood
\def\@pnumwidth{1.55em}
\def\@tocrmarg {2.55em}
\def\@dotsep{4.5}
\emergencystretch 3em
\hbadness=4000
\pretolerance=750
\tolerance=2000
\vbadness=4000
\def\Gin@extensions{.pdf,.png,.jpg,.mps,.tif}
% \renewcommand{\@cite}[1]{#1} % biblio

\usepackage{hyperref} % supposed to be the last one, :o) except for the ones to follow
\urlstyle{same} % after hyperref
\hypersetup{
  % pdftex, % no effect
  pdftitle={\elbibl},
  % pdfauthor={Your name here},
  % pdfsubject={Your subject here},
  % pdfkeywords={keyword1, keyword2},
  bookmarksnumbered=true,
  bookmarksopen=true,
  bookmarksopenlevel=1,
  pdfstartview=Fit,
  breaklinks=true, % avoid long links
  pdfpagemode=UseOutlines,    % pdf toc
  hyperfootnotes=true,
  colorlinks=false,
  pdfborder=0 0 0,
  % pdfpagelayout=TwoPageRight,
  % linktocpage=true, % NO, toc, link only on page no
}

\makeatother % /@@@>
%%%%%%%%%%%%%%
% </TEI> end %
%%%%%%%%%%%%%%


%%%%%%%%%%%%%
% footnotes %
%%%%%%%%%%%%%
\renewcommand{\thefootnote}{\bfseries\textcolor{rubric}{\arabic{footnote}}} % color for footnote marks

%%%%%%%%%
% Fonts %
%%%%%%%%%
\usepackage[]{roboto} % SmallCaps, Regular is a bit bold
% \linespread{0.90} % too compact, keep font natural
\newfontfamily\fontrun[]{Roboto Condensed Light} % condensed runing heads
\ifav
  \setmainfont[
    ItalicFont={Roboto Light Italic},
  ]{Roboto}
\else\ifbooklet
  \setmainfont[
    ItalicFont={Roboto Light Italic},
  ]{Roboto}
\else
\setmainfont[
  ItalicFont={Roboto Italic},
]{Roboto Light}
\fi\fi
\renewcommand{\LettrineFontHook}{\bfseries\color{rubric}}
% \renewenvironment{labelblock}{\begin{center}\bfseries\color{rubric}}{\end{center}}

%%%%%%%%
% MISC %
%%%%%%%%

\setdefaultlanguage[frenchpart=false]{french} % bug on part


\newenvironment{quotebar}{%
    \def\FrameCommand{{\color{rubric!10!}\vrule width 0.5em} \hspace{0.9em}}%
    \def\OuterFrameSep{\itemsep} % séparateur vertical
    \MakeFramed {\advance\hsize-\width \FrameRestore}
  }%
  {%
    \endMakeFramed
  }
\renewenvironment{quoteblock}% may be used for ornaments
  {%
    \savenotes
    \setstretch{0.9}
    \normalfont
    \begin{quotebar}
  }
  {%
    \end{quotebar}
    \spewnotes
  }


\renewcommand{\headrulewidth}{\arrayrulewidth}
\renewcommand{\headrule}{{\color{rubric}\hrule}}

% delicate tuning, image has produce line-height problems in title on 2 lines
\titleformat{name=\chapter} % command
  [display] % shape
  {\vspace{1.5em}\centering} % format
  {} % label
  {0pt} % separator between n
  {}
[{\color{rubric}\huge\textbf{#1}}\bigskip] % after code
% \titlespacing{command}{left spacing}{before spacing}{after spacing}[right]
\titlespacing*{\chapter}{0pt}{-2em}{0pt}[0pt]

\titleformat{name=\section}
  [block]{}{}{}{}
  [\vbox{\color{rubric}\large\raggedleft\textbf{#1}}]
\titlespacing{\section}{0pt}{0pt plus 4pt minus 2pt}{\baselineskip}

\titleformat{name=\subsection}
  [block]
  {}
  {} % \thesection
  {} % separator \arrayrulewidth
  {}
[\vbox{\large\textbf{#1}}]
% \titlespacing{\subsection}{0pt}{0pt plus 4pt minus 2pt}{\baselineskip}

\ifaiv
  \fancypagestyle{main}{%
    \fancyhf{}
    \setlength{\headheight}{1.5em}
    \fancyhead{} % reset head
    \fancyfoot{} % reset foot
    \fancyhead[L]{\truncate{0.45\headwidth}{\fontrun\elbibl}} % book ref
    \fancyhead[R]{\truncate{0.45\headwidth}{ \fontrun\nouppercase\leftmark}} % Chapter title
    \fancyhead[C]{\thepage}
  }
  \fancypagestyle{plain}{% apply to chapter
    \fancyhf{}% clear all header and footer fields
    \setlength{\headheight}{1.5em}
    \fancyhead[L]{\truncate{0.9\headwidth}{\fontrun\elbibl}}
    \fancyhead[R]{\thepage}
  }
\else
  \fancypagestyle{main}{%
    \fancyhf{}
    \setlength{\headheight}{1.5em}
    \fancyhead{} % reset head
    \fancyfoot{} % reset foot
    \fancyhead[RE]{\truncate{0.9\headwidth}{\fontrun\elbibl}} % book ref
    \fancyhead[LO]{\truncate{0.9\headwidth}{\fontrun\nouppercase\leftmark}} % Chapter title, \nouppercase needed
    \fancyhead[RO,LE]{\thepage}
  }
  \fancypagestyle{plain}{% apply to chapter
    \fancyhf{}% clear all header and footer fields
    \setlength{\headheight}{1.5em}
    \fancyhead[L]{\truncate{0.9\headwidth}{\fontrun\elbibl}}
    \fancyhead[R]{\thepage}
  }
\fi

\ifav % a5 only
  \titleclass{\section}{top}
\fi

\newcommand\chapo{{%
  \vspace*{-3em}
  \centering % no vskip ()
  {\Large\addfontfeature{LetterSpace=25}\bfseries{\elauthor}}\par
  \smallskip
  {\large\eldate}\par
  \bigskip
  {\Large\selectfont{\eltitle}}\par
  \bigskip
  {\color{rubric}\hline\par}
  \bigskip
  {\Large TEXTE LIBRE À PARTICPATION LIBRE\par}
  \centerline{\small\color{rubric} {hurlus.fr, tiré le \today}}\par
  \bigskip
}}

\newcommand\cover{{%
  \thispagestyle{empty}
  \centering
  {\LARGE\bfseries{\elauthor}}\par
  \bigskip
  {\Large\eldate}\par
  \bigskip
  \bigskip
  {\LARGE\selectfont{\eltitle}}\par
  \vfill\null
  {\color{rubric}\setlength{\arrayrulewidth}{2pt}\hline\par}
  \vfill\null
  {\Large TEXTE LIBRE À PARTICPATION LIBRE\par}
  \centerline{{\href{https://hurlus.fr}{\dotuline{hurlus.fr}}, tiré le \today}}\par
}}

\begin{document}
\pagestyle{empty}
\ifbooklet{
  \cover\newpage
  \thispagestyle{empty}\hbox{}\newpage
  \cover\newpage\noindent Les voyages de la brochure\par
  \bigskip
  \begin{tabularx}{\textwidth}{l|X|X}
    \textbf{Date} & \textbf{Lieu}& \textbf{Nom/pseudo} \\ \hline
    \rule{0pt}{25cm} &  &   \\
  \end{tabularx}
  \newpage
  \addtocounter{page}{-4}
}\fi

\thispagestyle{empty}
\ifaiv
  \twocolumn[\chapo]
\else
  \chapo
\fi
{\it\elabstract}
\bigskip
\makeatletter\@starttoc{toc}\makeatother % toc without new page
\bigskip

\pagestyle{main} % after style

  \chapter[{Guerre des Gaules (trad. Constans 1926)}]{\emph{Guerre des Gaules} (trad. Constans 1926)}
\section[{Livre I}]{Livre I}\renewcommand{\leftmark}{Livre I}

\subsection[{§ 1.}]{ \textsc{§ 1.} }
\noindent L'ensemble de la Gaule est divisé en trois parties : l’une est habitée par les Belges, l’autre par les Aquitains, la troisième par le peuple qui, dans sa langue, se nomme Celte, et, dans la nôtre, Gaulois. Tous ces peuples diffèrent entre eux par le langage, les coutumes, les lois. Les Gaulois sont séparés des Aquitains par la Garonne, des Belges par la Marne et la Seine. Les plus braves de ces trois peuples sont les Belges, parce qu’ils sont les plus éloignés de la Province romaine et des raffinements de sa civilisation, parce que les marchands y vont très rarement, et, par conséquent, n’y introduisent pas ce qui est propre à amollir les cœurs, enfin parce qu’ils sont les plus voisins des Germains qui habitent sur l’autre rive du Rhin, et avec qui ils sont continuellement en guerre. C'est pour la même raison que les Helvètes aussi surpassent en valeur guerrière les autres Gaulois : des combats presque quotidiens les mettent aux prises avec les Germains, soit qu’ils leur interdisent l’accès de leur territoire, soit qu’ils les attaquent chez eux. La partie de la Gaule qu’occupent, comme nous l’avons dit, les Gaulois commence au Rhône, est bornée par la Garonne, l’Océan et la frontière de Belgique ; elle touche aussi au Rhin du côté des Séquanes et des Helvètes ; elle est orientée vers le nord. La Belgique commence où finit la Gaule ; elle va jusqu’au cours inférieur du Rhin ; elle regarde vers le nord et vers l’est. L'Aquitaine s’étend de la Garonne aux Pyrénées et à la partie de l’Océan qui baigne l’Espagne ; elle est tournée vers le nord-ouest.
\subsection[{§ 2.}]{ \textsc{§ 2.} }
\noindent Orgétorix était chez les Helvètes l’homme de beaucoup le plus noble et le plus riche. Sous le consulat de Marcus Messala et de Marcus Pison, séduit par le désir d’être roi, il forma une conspiration de la noblesse et persuada ses concitoyens de sortir de leur pays avec toutes leurs ressources : « Rien n’était plus facile, puisque leur valeur les mettait au-dessus de tous, que de devenir les maîtres de la Gaule entière ». Il eut d’autant moins de peine à les convaincre que les Helvètes, en raison des conditions géographiques, sont de toutes parts enfermés : d’un côté par le Rhin, dont le cours très large et très profond sépare l’Helvétie de la Germanie, d’un autre par le Jura, chaîne très haute qui se dresse entre les Helvètes et les Séquanes, et du troisième par le lac Léman et le Rhône, qui sépare notre province de leur territoire. Cela restreignait le champ de leurs courses vagabondes et les gênait pour porter la guerre chez leurs voisins : situation fort pénible pour des hommes qui avaient la passion de la guerre. Ils estimaient d’ailleurs que l’étendue de leur territoire, qui avait deux cent quarante milles de long et cent quatre-vingts de large, n’était pas en rapport avec leur nombre, ni avec leur gloire militaire et leur réputation de bravoure.
\subsection[{§ 3.}]{ \textsc{§ 3.} }
\noindent Sous l’influence de ces raisons, et entraînés par l’autorité d’Orgétorix, ils décidèrent de tout préparer pour leur départ : acheter bêtes de somme et chariots en aussi grand nombre que possible, ensemencer toutes les terres cultivables, afin de ne point manquer de blé pendant la route, assurer solidement des relations de paix et d’amitié avec les États voisins. A la réalisation de ce plan, deux ans, pensèrent-ils, suffiraient : une loi fixa le départ à la troisième année. Orgétorix fut choisi pour mener à bien l’entreprise : il se chargea personnellement des ambassades. Au cours de sa tournée, il persuade Casticos, fils de Catamantaloédis, Séquane, dont le père avait été longtemps roi dans son pays et avait reçu du Sénat romain le titre d’ami, de s’emparer du pouvoir qui avait auparavant appartenu à son père ; il persuade également l’Héduen Dumnorix, frère de Diviciacos, qui occupait alors le premier rang dans son pays et était particulièrement aimé du peuple, de tenter la même entreprise, et il lui donne sa fille en mariage. Il leur démontre qu’il est tout à fait aisé de mener ces entreprises à bonne fin, pour la raison qu’il est lui-même sur le point d’obtenir le pouvoir suprême dans son pays : on ne peut douter que de tous les peuples de la Gaule le peuple helvète ne soit le plus puissant ; il se fait fort de leur donner le pouvoir en mettant à leur service ses ressources et son armée. Ce langage les séduit ; les trois hommes se lient par un serment, et se flattent que, devenus rois, la puissance de leurs trois peuples, qui sont les plus grands et les plus forts, leur permettra de s’emparer de la Gaule entière.
\subsection[{§ 4.}]{ \textsc{§ 4.} }
\noindent Une dénonciation fit connaître aux Helvètes cette intrigue. Selon l’usage du pays, Orgétorix dut plaider sa cause chargé de chaînes. S'il était condamné, la peine qu’il devait subir était le supplice du feu. Au jour fixé pour son audition, Orgétorix amena devant le tribunal tous les siens, environ dix mille hommes, qu’il avait rassemblés de toutes parts, et il fit venir aussi tous ses clients et ses débiteurs, qui étaient en grand nombre : grâce à leur présence, il put se soustraire à l’obligation de parler. Cette conduite irrita ses concitoyens : ils voulurent obtenir satisfaction par la force, et les magistrats levèrent un grand nombre d’hommes dans la campagne ; sur ces entrefaites, Orgétorix mourut et l’on n’est pas sans soupçonner – c’est l’opinion des Helvètes – qu’il mit lui-même fin à ses jours.
\subsection[{§ 5.}]{ \textsc{§ 5.} }
\noindent Après sa mort, les Helvètes n’en persévèrent pas moins dans le dessein qu’ils avaient formé de quitter leur pays. Quand ils se croient prêts pour cette entreprise, ils mettent le feu à toutes leurs villes – il y en avait une douzaine, – à leurs villages – environ quatre cents – et aux maisons isolées ; tout le blé qu’ils ne devaient pas emporter, ils le livrent aux flammes : ainsi, en s’interdisant l’espoir du retour, ils seraient mieux préparés à braver tous les hasards qui les attendaient ; chacun devait emporter de la farine pour trois mois. Ils persuadent les Rauraques, les Tulinges et les Latobices, qui étaient leurs voisins, de suivre la même conduite, de brûler leurs villes et leurs villages et de partir avec eux ; enfin les Boïens, qui, d’abord établis au-delà du Rhin, venaient de passer dans le Norique et de mettre le siège devant Noréia, deviennent leurs alliés et se joignent à eux.
\subsection[{§ 6.}]{ \textsc{§ 6.} }
\noindent Il y avait en tout deux routes qui leur permettaient de quitter leur pays. L'une traversait le territoire des Séquanes : étroite et malaisée, elle était resserrée entre le Jura et le Rhône, et les chariots y passaient à peine un par un ; d’ailleurs, une très haute montagne la dominait, en sorte qu’une poignée d’hommes pouvait facilement l’interdire. L'autre route passait par notre province : elle était beaucoup plus praticable et plus aisée, parce que le territoire des Helvètes et celui des Allobroges, nouvellement soumis, sont séparés par le cours du Rhône, et que ce fleuve est guéable en plusieurs endroits. La dernière ville des Allobroges et la plus voisine de l’Helvétie est Genève. Un pont la joint à ce pays. Les Helvètes pensaient qu’ils obtiendraient des Allobroges le libre passage, parce que ce peuple ne leur paraissait pas encore bien disposé à l’égard de Rome ; en cas de refus, ils les contraindraient par la force. Une fois tous les préparatifs de départ achevés, on fixe le jour où ils doivent se rassembler tous sur les bords du Rhône. Ce jour était le 5 des calendes d’avril, sous le consulat de Lucius Pison et d’Aulus Gabinius.
\subsection[{§ 7.}]{ \textsc{§ 7.} }
\noindent César, à la nouvelle qu’ils prétendaient faire route à travers notre province, se hâte de quitter Rome, gagne à marches forcées la Gaule transalpine et arrive devant Genève. Il ordonne de lever dans toute la province le plus de soldats possible (il y avait en tout dans la Gaule transalpine une légion) et fait couper le pont de Genève. Quand ils savent son arrivée, les Helvètes lui envoient une ambassade composée des plus grands personnages de l’État, et qui avait à sa tête Namméios et Verucloétios ; ils devaient lui tenir ce langage : « L'intention des Helvètes est de passer, sans causer aucun dégât, à travers la province, parce qu’ils n’ont pas d’autre chemin ; ils lui demandent de vouloir bien autoriser ce passage. » César, se souvenant que les Helvètes avaient tué le consul L. Cassius, battu et fait passer sous le joug son armée, pensait qu’il ne devait pas y consentir : il estimait d’ailleurs que des hommes dont les dispositions d’esprit étaient hostiles, si on leur permettait de traverser la province, ne sauraient le faire sans violences ni dégâts. Néanmoins, voulant gagner du temps jusqu’à la concentration des troupes dont il avait ordonné la levée, il répondit aux envoyés qu’il se réservait quelque temps pour réfléchir : « S'ils avaient un désir à exprimer, qu’ils revinssent aux ides d’avril. »
\subsection[{§ 8.}]{ \textsc{§ 8.} }
\noindent En attendant, il employa la légion qu’il avait et les soldats qui étaient venus de la province à construire, sur une longueur de dix-neuf milles, depuis le lac Léman, qui déverse ses eaux dans le Rhône, jusqu’au Jura, qui forme la frontière entre les Séquanes et les Helvètes, un mur haut de seize pieds et précédé d’un fossé. Ayant achevé cet ouvrage, il distribue des postes, établit des redoutes, afin de pouvoir mieux leur interdire le passage s’ils veulent le tenter contre son gré. Quand on fut au jour convenu, et que les envoyés revinrent, il déclara que les traditions de la politique romaine et les précédents ne lui permettaient pas d’accorder à qui que ce fût le passage à travers la province ; s’ils voulaient passer de force, ils le voyaient prêt à s’y opposer. Les Helvètes, déchus de leur espérance, essayèrent, soit à l’aide de bateaux liés ensemble et de radeaux qu’ils construisirent en grand nombre, soit à gué, aux endroits où le Rhône avait le moins de profondeur, de forcer le passage du fleuve, quelquefois de jour, plus souvent de nuit ; mais ils se heurtèrent aux ouvrages de défense, furent repoussés par les attaques et les tirs de nos soldats, et finirent par renoncer à leur entreprise.
\subsection[{§ 9.}]{ \textsc{§ 9.} }
\noindent Il ne leur restait plus qu’une route, celle qui traversait le territoire des Séquanes ; ils ne pouvaient, à cause des défilés, s’y engager sans le consentement de ce peuple. Ne pouvant le persuader à eux seuls, ils envoient une ambassade à l’Héduen Dumnorix, afin que par son intercession il leur obtienne le passage. Dumnorix, qui était populaire et généreux, disposait de la plus forte influence auprès des Séquanes ; c’était en même temps un ami des Helvètes, parce qu’il s’était marié dans leur pays, ayant épousé la fille d’Orgétorix ; son désir de régner le poussait à favoriser les changements politiques, et il voulait s’attacher le plus de nations possible en leur rendant des services. Aussi prend-il l’affaire en main : il obtient des Séquanes qu’ils laissent passer les Helvètes sur leur territoire, et amène les deux peuples à échanger des otages, les Séquanes s’engageant à ne pas s’opposer au passage des Helvètes, ceux-ci garantissant que leur passage s’effectuera sans dommages ni violences.
\subsection[{§ 10.}]{ \textsc{§ 10.} }
\noindent On rapporte à César que les Helvètes se proposent de gagner, par les territoires des Séquanes et des Héduens, celui des Santons, qui n’est pas loin de la cité des Tolosates, laquelle fait partie de la province romaine. Il se rend compte que si les choses se passent ainsi, ce sera un grand danger pour la province que d’avoir, sur la frontière d’un pays sans défenses naturelles et très riche en blé, un peuple belliqueux, hostile aux Romains. Aussi, confiant à son légat Titus Labiénus le commandement de la ligne fortifiée qu’il avait établie, il gagne l’Italie par grandes étapes ; il y lève deux légions, en met en campagne trois autres qui prenaient leurs quartiers d’hiver autour d’Aquilée, et avec ses cinq légions il se dirige vers la Gaule ultérieure, en prenant au plus court, à travers les Alpes. Là, les Centrons, les Graiocèles, les Caturiges, qui avaient occupé les positions dominantes, essayent d’interdire le passage à son armée. Parti d’Océlum, qui est la dernière ville de la Gaule citérieure, il parvient en sept jours, après plusieurs combats victorieux, chez les Voconces, en Gaule ultérieure ; de là il conduit ses troupes chez les Allobroges, et des Allobroges chez les Ségusiaves. C'est le premier peuple qu’on rencontre hors de la province au-delà du Rhône.
\subsection[{§ 11.}]{ \textsc{§ 11.} }
\noindent Les Helvètes avaient déjà franchi les défilés et traversé le pays des Séquanes ; ils étaient parvenus chez les Héduens, et ravageaient leurs terres. Ceux-ci, ne pouvant se défendre ni protéger leurs biens, envoient une ambassade à César pour lui demander secours : « Ils s’étaient, de tout temps, assez bien conduits envers le peuple romain pour ne pas mériter que presque sous les yeux de notre armée leurs champs fussent dévastés, leurs enfants emmenés en esclavage, leurs villes prises d’assaut. En même temps les Ambarres, peuple ami des Héduens et de même souche, font savoir à César que leurs campagnes ont été ravagées, et qu’ils ont de la peine à défendre leurs villes des agressions de l’ennemi. Enfin des Allobroges qui avaient sur la rive droite du Rhône des villages et des propriétés cherchent un refuge auprès de César et lui exposent que, sauf le sol même, il ne leur reste plus rien. Ces faits décident César il n’attendra pas que les Helvètes soient arrivés en Saintonge après avoir consommé la ruine de nos alliés.
\subsection[{§ 12.}]{ \textsc{§ 12.} }
\noindent Il y a une rivière, la Saône, qui va se jeter dans le Rhône en passant par les territoires des Héduens et des Séquanes ; son cours est d’une incroyable lenteur, au point que l’œil ne peut juger du sens du courant. Les Helvètes étaient en train de la franchir à l’aide de radeaux et de barques assemblés. Quand César sut par ses éclaireurs que déjà les trois quarts de leurs troupes avaient franchi la rivière et qu’il ne restait plus sur la rive gauche que le quart environ de l’armée, il partit de son camp pendant la troisième veille avec trois légions et rejoignit ceux qui n’avaient pas encore passé. Ils étaient embarrassés de leurs bagages et ne s’attendaient pas à une attaque. César tailla en pièce la plus grande partie ; le reste chercha son salut dans la fuite et se cacha dans les forêts voisines. Ces hommes étaient ceux du canton des Tigurins : l’ensemble du peuple helvète se divise, en effet, en quatre cantons. Ces Tigurins, ayant quitté seuls leur pays au temps de nos pères, avaient tué le consul L. Cassius et fait passer son armée sous le joug. Ainsi, soit effet du hasard, soit dessein des dieux immortels, la partie de la nation helvète qui avait infligé aux Romains un grand désastre fut la première à être punie. En cette occasion, César ne vengea pas seulement son pays, mais aussi sa famille : L. Pison, aïeul de son beau-père L. Pison, et lieutenant de Cassius, avait été tué par les Tigurins dans le même combat où Cassius avait péri.
\subsection[{§ 13.}]{ \textsc{§ 13.} }
\noindent Après avoir livré cette bataille, César, afin de pouvoir poursuivre le reste de l’armée helvète, fait jeter un pont sur la Saône et par ce moyen porte son armée sur l’autre rive. Sa soudaine approche surprend les Helvètes, et ils s’effraient de voir qu’un jour lui a suffi pour franchir la rivière, quand ils ont eu beaucoup de peine à le faire en vingt. Ils lui envoient une ambassade : le chef en était Divico, qui avait commandé aux Helvètes dans la guerre contre Cassius. Il tint à César ce langage « Si le peuple Romain faisait la paix avec les Helvètes, ceux-ci iraient où César voudrait, et s’établiraient à l’endroit de son choix ; mais s’il persistait à les traiter en ennemis, il ne devait pas oublier que les Romains avaient éprouvé autrefois quelque désagrément, et qu’un long passé consacrait la vertu guerrière des Helvètes. Il s’était jeté à l’improviste sur les troupes d’un canton, alors que ceux qui avaient passé la rivière ne pouvaient porter secours à leurs frères ; il ne devait pas pour cela trop présumer de sa valeur ni mépriser ses adversaires. Ils avaient appris de leurs aïeux à préférer aux entreprises de ruse et de fourberie la lutte ouverte où le plus courageux triomphe. Qu'il prît donc garde les lieux où ils s’étaient arrêtés pourraient bien emprunter un nom nouveau à une défaite romaine et à la destruction de son armée, ou en transmettre le souvenir. »
\subsection[{§ 14.}]{ \textsc{§ 14.} }
\noindent César répondit en ces termes : « Il hésitait d’autant moins sur le parti à prendre que les faits rappelés par les ambassadeurs helvètes étaient présents à sa mémoire, et il avait d’autant plus de peine à en supporter l’idée que le peuple Romain était moins responsable de ce qui s’était passé. Si, en effet, il avait eu conscience d’avoir causé quelque tort, il ne lui eût pas été difficile de prendre ses précautions ; mais ce qui l’avait trompé, c’est qu’il ne voyait rien dans sa conduite qui lui donnât sujet de craindre, et qu’il ne pensait pas qu’il dût craindre sans motif. Et à supposer qu’il consentît à oublier l’ancien affront, leurs nouvelles insultes tentative pour passer de force à travers la province dont on leur refusait l’accès, violences contre les Héduens, les Ambarres, les Allobroges, pouvait-il les oublier ? Quant à l’insolent orgueil que leur inspirait leur victoire, et à leur étonnement d’être restés si longtemps impunis, la résolution de César s’en fortifiait. Car les dieux immortels, pour faire sentir plus durement les revers de la fortune aux hommes dont ils veulent punir les crimes, aiment à leur accorder des moments de chance et un certain délai d’impunité. Telle est la situation ; pourtant, s’ils lui donnent des otages qui lui soient une garantie de l’exécution de leurs promesses, et si les Héduens reçoivent satisfaction pour les torts qu’eux et leurs alliés ont subis, si les Allobroges obtiennent également réparation, il est prêt à faire la paix. » Divico répondit que « les Helvètes tenaient de leurs ancêtres un principe : ils recevaient des otages, ils n’en donnaient point ; le peuple Romain pouvait en porter témoignage. » Sur cette réponse, il partit.
\subsection[{§ 15.}]{ \textsc{§ 15.} }
\noindent Le lendemain, les Helvètes lèvent le camp. César fait de même, et il envoie en avant toute sa cavalerie, environ quatre mille hommes qu’il avait levés dans l’ensemble de la province et chez les Héduens et leurs alliés ; elle devait se rendre compte de la direction prise par l’ennemi. Ayant poursuivi avec trop d’ardeur l’arrière-garde des Helvètes, elle a un engagement avec leur cavalerie sur un terrain qu’elle n’a pas choisi, et perd quelques hommes. Ce combat exalta l’orgueil de nos adversaires, qui avaient avec cinq cents cavaliers repoussé une cavalerie si nombreuse : ils commencèrent à se montrer plus audacieux, faisant face quelquefois et nous harcelant de combats d’arrière-garde. César retenait ses soldats, et se contentait pour le moment d’empêcher l’ennemi de voler, d’enlever le fourrage et de détruire. On marcha ainsi près de quinze jours, sans qu’il y eût jamais entre l’arrière-garde ennemie et notre avant-garde plus de cinq ou six mille pas.
\subsection[{§ 16.}]{ \textsc{§ 16.} }
\noindent Cependant César réclamait chaque jour aux Héduens le blé qu’ils lui avaient officiellement promis. Car, à cause du froid – la Gaule, comme on l’a dit précédemment, est un pays septentrional –, non seulement les moissons n’étaient pas mûres, mais le fourrage aussi manquait ; quant au blé qu’il avait fait transporter par eau en remontant la Saône, il ne pouvait guère en user, parce que les Helvètes s’étaient écartés de la rivière et qu’il ne voulait pas les perdre de vue. Les Héduens différaient leur livraison de jour en jour : « On rassemblait les grains, disaient-ils, ils étaient en route, ils arrivaient. » Quand César vit qu’on l’amusait, et que le jour était proche où il faudrait distribuer aux soldats leur ration mensuelle, il convoque les chefs héduens qui étaient en grand nombre dans son camp ; parmi eux se trouvaient Diviciaros et Liscos ; ce dernier était le magistrat suprême, que les Héduens appellent vergobret ; il est nommé pour un an, et a droit de vie et de mort sur ses concitoyens ; César se plaint vivement que, dans l’impassibilité d’acheter du blé ou de s’en procurer dans la campagne, quand les circonstances sont si critiques, l’ennemi si proche, il ne trouve pas d’aide auprès d’eux, et cela, quand c’est en grande partie pour répondre à leurs prières qu’il a entrepris la guerre ; plus vivement encore il leur reproche d’avoir trahi sa confiance.
\subsection[{§ 17.}]{ \textsc{§ 17.} }
\noindent Ces paroles de César décident Liscos à dire enfin ce que jusqu’alors il avait tu : « Il y a un certain nombre de personnages qui ont une influence prépondérante sur le peuple, et qui, simples particuliers, sont plus puissants que les magistrats eux-mêmes. Ce sont ceux-là qui, par leurs excitations criminelles, détournent la masse des Héduens d’apporter le blé qu’ils doivent : ils leur disent qu’il vaut mieux, s’ils ne peuvent plus désormais prétendre au premier rang dans la Gaule, obéir à des Gaulois qu’aux Romains ; ils se déclarent certains que, si les Romains triomphent des Helvètes, ils raviront la liberté aux Héduens en même temps qu’au reste de la Gaule. Ce sont ces mêmes personnages qui instruisent l’ennemi de nos plans et de ce qui se passe dans l’armée ; il est impuissant à les contenir. Bien plus : s’il a attendu d’y être forcé pour révéler à César une situation aussi grave, c’est qu’il se rend compte du danger qu’il court ; voilà pourquoi, aussi longtemps qu’il l’a pu, il a gardé le silence. »
\subsection[{§ 18.}]{ \textsc{§ 18.} }
\noindent César sentait bien que ces paroles de Liscos visaient Dumnorix, frère de Diviciaros ; mais, ne voulant pas que l’affaire soit discutée en présence de plusieurs personnes, il congédie promptement l’assemblée, et ne retient que Liscos. Seul à seul, il l’interroge sur ce qu’il avait dit dans le conseil. Celui-ci parle avec plus de liberté et d’audace. César interroge en secret d’autres personnages ; il constate que Liscos a dit vrai. « C'était bien Dumnorix : l’homme était plein d’audace, sa libéralité l’avait mis en faveur auprès du peuple, et il voulait un bouleversement politique. Depuis de longues années il avait à vil prix la ferme des douanes et de tous les autres impôts des Héduens, parce que, lorsqu’il enchérissait, personne n’osait enchérir contre lui. Cela lui avait permis d’amasser, tout en enrichissant sa maison, de quoi pourvoir abondamment à ses largesses ; il entretenait régulièrement, à ses frais, une nombreuse cavalerie qui lui servait de garde du corps, et son influence ne se limitait pas à son pays, mais s’étendait largement sur les nations voisines. Il avait même, pour développer cette influence, marié sa mère, chez les Bituriges, à un personnage de haute noblesse et de grand pouvoir ; lui-même avait épousé une Helvète ; sa sœur du côté maternel et des parentes avaient été mariées par ses soins dans d’autres cités. Il aimait et favorisait les Helvètes à cause de cette union ; en outre, il nourrissait une haine personnelle contre César et les Romains, parce que leur arrivée avait diminué son pouvoir et rendu à son frère Diviciacos crédit et honneurs d’autrefois. Un malheur des Romains porterait au plus haut ses espérances de devenir roi grâce aux Helvètes ; la domination romaine lui ferait perdre l’espoir non seulement de régner, mais même de conserver son crédit. » L'enquête de César lui apprit encore que, dans le combat de cavalerie défavorable à nos armées qui avait eu lieu quelques jours auparavant, Dumnorix et ses cavaliers avaient été les premiers à tourner bride (la cavalerie auxiliaire que les Héduens avaient fournie à César était, en effet, commandée par Dumnorix) ; c’était leur fuite qui avait jeté la panique dans le reste de la troupe.
\subsection[{§ 19.}]{ \textsc{§ 19.} }
\noindent Aux soupçons que faisaient maître ces renseignements se joignaient d’absolues certitudes : il avait fait passer les Helvètes à travers le pays des Séquanes ; il s’était occupé de faire échanger des otages entre les deux peuples ; il avait agi en tout cela non seulement sans l’ordre de César ni de ses concitoyens, mais encore à leur insu ; il était dénoncé par le premier magistrat des Héduens. César pensait qu’il y avait là motif suffisant pour sévir lui-même ou inviter sa cité à le punir. A ces raisons, une seule s’opposait : il avait pu apprécier chez Diviciacos, frère du traître, un entier dévouement au peuple romain, un très grand attachement à sa personne, les plus remarquables qualités de fidélité, de droiture, de modération ; et il craignait de lui porter un coup cruel en envoyant son frère au supplice. Aussi, avant de rien tenter, il fait appeler Diviciacos, et, écartant ses interprètes ordinaires, il a recours, pour s’entretenir avec lui, à Caïus Valérius Troucillus, grand personnage de la Gaule romaine, qui était son ami et en qui iI avait la plus entière confiance. Il lui rappelle ce qu’on a dit de Dumnorix en sa présence, dans le conseil, et lui fait connaître les renseignements qu’il a obtenus dans des entretiens particuliers ; il le prie instamment de ne pas s’offenser s’il statue lui-même sur le coupable après information régulière ou s’il invite sa cité à le juger.
\subsection[{§ 20.}]{ \textsc{§ 20.} }
\noindent Diviciacos, tout en larmes, entoure César de ses bras et le conjure de ne pas prendre contre son frère des mesures trop rigoureuses. Il savait qu’on avait dit vrai, et personne n’en souffrait plus que lui : car alors qu’il jouissait dans son pays et dans le reste de la Gaule d’une très grande influence et que son frère, à cause de son jeune âge, n’en possédait aucune, il l’avait aidé à s’élever ; et la fortune et la puissance ainsi acquises, il s’en servait non seulement à affaiblir son crédit, mais même à préparer sa perte. Pourtant, c’était son frère, et d’autre part l’opinion publique ne pouvait le laisser indifférent. Si César le traitait avec rigueur quand lui, Diviciacos, occupait un si haut rang dans son amitié, personne ne penserait que c’eût été contre son gré : et dès lors tous les Gaulois lui deviendraient hostiles. Il parlait avec abondance et versait des larmes. César prend sa main, le rassure, lui demande de mettre fin à ses instances ; il lui déclare qu’il estime assez haut son amitié pour sacrifier à son désir et à ses prières le tort fait aux Romains et l’indignation qu’il éprouve. Il fait venir Dumnorix et, en présence de son frère, lui dit ce qu’il lui reproche ; il lui expose ce qu’il sait, et les griefs de ses compatriotes ; il l’avertit d’avoir à éviter, pour l’avenir, tout soupçon ; il lui pardonne le passé en faveur de son frère Diviciacos ; il lui donne des gardes, afin de savoir ce qu’il fait et avec qui il s’entretient.
\subsection[{§ 21.}]{ \textsc{§ 21.} }
\noindent Le même jour, ayant appris par ses éclaireurs que l’ennemi s’était arrêté au pied d’une montagne à huit milles de son camp, César envoya une reconnaissance pour savoir ce qu’était cette montagne et quel accès offrait son pourtour. On lui rapporta qu’elle était d’accès facile. Il ordonne à Titus Labiénus, légat propréteur, d’aller, au cours de la troisième veille, occuper la crête de la montagne avec deux légions, en se faisant guider par ceux qui avaient reconnu la route ; il lui fait connaître son plan. De son côté, pendant la quatrième veille, il marche à l’ennemi, par le même chemin que celui-ci avait pris, et détache en avant toute sa cavalerie. Elle était précédée par des éclaireurs sous les ordres de Publius Considius, qui passait pour un soldat très expérimenté et avait servi dans l’armée de Lucius Sulla, puis dans celle de Marcus Crassus.
\subsection[{§ 22.}]{ \textsc{§ 22.} }
\noindent Au point du jour, comme Labiénus occupait le sommet de la montagne, que lui-même n’était plus qu’à quinze cents pas du camp ennemi, et que – il le sut plus tard par des prisonniers – on ne s’était aperçu ni de son approche, ni de celle de Labiénus, Considius accourt vers lui à bride abattue : « La montagne, dit-il, que Labiénus avait ordre d’occuper, ce sont les ennemis qui la tiennent : il a reconnu les Gaulois à leurs armes et à leurs insignes. » César ramène ses troupes sur une colline voisine et les range en bataille. Il avait recommandé à Labiénus de n’engager le combat qu’après avoir vu ses troupes près du camp ennemi, car il voulait que l’attaque se produisît simultanément de tous côtés : aussi le légat, après avoir pris position sur la montagne, attendait-il les nôtres et s’abstenait-il d’attaquer. Ce ne fut que fort avant dans la journée que César apprit par ses éclaireurs la vérité : c’étaient les siens qui occupaient la montagne, les Helvètes avaient levé le camp, Considius, égaré par la peur, lui avait dit avoir vu ce qu’il n’avait pas vu. Ce jour même César suit les ennemis à la distance ordinaire et établit son camp à trois mille pas du leur.
\subsection[{§ 23.}]{ \textsc{§ 23.} }
\noindent Le lendemain, comme deux jours en tout et pour tout le séparaient du moment où il faudrait distribuer du blé aux troupes, et comme d’autre part Bibracte, de beaucoup la plus grande et la plus riche ville des Héduens, n’était pas à plus de dix-huit milles, il pensa qu’il fallait s’occuper de l’approvisionnement, et, laissant les Helvètes, il se dirigea vers Bibracte. Des esclaves de Lucius Emilius, décurion de la cavalerie gauloise, s’enfuient et apprennent la chose à l’ennemi. Les Helvètes crurent-ils que les Romains rompaient le contact sous le coup de la terreur, pensée d’autant plus naturelle que la veille, maîtres des hauteurs, nous n’avions pas attaqué ? ou bien se firent-ils forts de nous couper les vivres ? toujours est-il que, modifiant leurs plans et faisant demi-tour, ils se mirent à suivre et à harceler notre arrière-garde.
\subsection[{§ 24.}]{ \textsc{§ 24.} }
\noindent Quand il s’aperçut de cette manœuvre, César se mit en devoir de ramener ses troupes sur une colline voisine et détacha sa cavalerie pour soutenir le choc de l’ennemi. De son côté, il rangea en bataille sur trois rangs, à mi-hauteur, ses quatre légions de vétérans ; au-dessus de lui, sur la crête, il fit disposer les deux légions qu’il avait levées en dernier lieu dans la Gaule, et toutes les troupes auxiliaires ; la colline entière était ainsi couverte de soldats ; il ordonna qu’en même temps les sacs fussent réunis en un seul point et que les troupes qui occupaient la position la plus haute s’employassent à le fortifier. Les Helvètes, qui suivaient avec tout leurs chariots, les rassemblèrent sur un même point ; et les combattants, après avoir rejeté notre cavalerie en lui opposant un front très compact, formèrent la phalange et montèrent à l’attaque de notre première ligne.
\subsection[{§ 25.}]{ \textsc{§ 25.} }
\noindent César fit éloigner et mettre hors de vue son cheval d’abord, puis ceux de tous les officiers, afin que le péril fût égal pour tous et que personne ne pût espérer s’enfuir ; alors il harangua ses troupes et engagea le combat. Nos soldats, lançant le javelot de haut en bas, réussirent aisément à briser la phalange des ennemis. Quand elle fut disloquée, ils tirèrent l’épée et chargèrent. Les Gaulois éprouvaient un grave embarras du fait que souvent un seul coup de javelot avait percé et fixé l’un à l’autre plusieurs de leurs boucliers ; comme le fer s’était tordu, ils ne pouvaient l’arracher, et, n’ayant pas le bras gauche libre, ils étaient gênés pour se battre : aussi plusieurs, après avoir longtemps secoué le bras, préféraient-ils laisser tomber les boucliers et combattre à découvert. Enfin, épuisés par leurs blessures, ils commencèrent à reculer et à se replier vers une montagne qui était à environ un mille de là. Ils l’occupèrent, et les nôtres s’avançaient pour les en déloger quand les Boïens et les Tulinges, qui, au nombre d’environ quinze mille, fermaient la marche et protégeaient les derniers éléments de la colonne, soudain attaquèrent notre flanc droit et cherchèrent à nous envelopper ; ce que voyant, les Helvètes qui s’étaient réfugiés sur la hauteur redevinrent agressifs et engagèrent à nouveau le combat. Les Romains firent une conversion et attaquèrent sur deux fronts la première et la deuxième lignes résisteraient à ceux qui avaient été battus et forcés à la retraite, tandis que la troisième soutiendrait le choc des troupes fraîches.
\subsection[{§ 26.}]{ \textsc{§ 26.} }
\noindent Cette double bataille fut longue et acharnée. Quand il ne leur fut plus possible de supporter nos assauts, ils se replièrent, les uns sur la hauteur, comme ils l’avaient fait une première fois, les autres auprès de leurs bagages et de leurs chariots. Pendant toute cette action, qui dura de la septième heure du jour jusqu’au soir, personne ne put voir un ennemi tourner le dos. On se battit encore autour des bagages fort avant dans la nuit les Barbares avaient en effet formé une barricade de chariots et, dominant les nôtres, ils les accablaient de traits à mesure qu’ils approchaient ; plusieurs aussi lançaient par-dessous, entre les chariots et entre les roues, des piques et des javelots qui blessaient nos soldats. Après un long combat, nous nous rendîmes maîtres des bagages et du camp. La fille d’Orgétorix et un de ses fils furent faits prisonniers. Cent trente mille hommes environ s’échappèrent, et durant cette nuit-là ils marchèrent sans arrêt ; le quatrième jour, sans jamais avoir fait halte un moment la nuit, ils arrivèrent chez les Lingons ; nos troupes n’avaient pu les suivre, ayant été retenues trois jours par les soins à donner aux blessés et par l’ensevelissement des morts. César envoya aux Lingons une lettre et des messagers pour les inviter à ne fournir aux Helvètes ni ravitaillement, ni aide d’aucune sorte ; sinon, il les traiterait comme eux. Et lui-même, au bout de trois jours, se mit à les suivre avec toute son armée.
\subsection[{§ 27.}]{ \textsc{§ 27.} }
\noindent Les Helvètes, privés de tout, furent réduits à lui envoyer des députés pour traiter de leur reddition. Ceux-ci le rencontrèrent tandis qu’il était en marche ; ils se jetèrent à ses pieds et, suppliant, versant des larmes, lui demandèrent la paix ; il ordonna que les Helvètes attendissent sans bouger de place son arrivée : ils obéirent. Quand César les eut rejoints, il exigea la remise d’otages, la livraison des armes et celle des esclaves qui s’étaient enfuis auprès d’eux. Dès le lendemain, on recherche, on rassemble ce qui doit être livré ; cependant, six mille hommes du pagus Verbigénus, soit qu’ils craignissent d’être envoyés au supplice une fois leurs armes livrées, soit qu’ils eussent l’espoir que leur fuite, tandis qu’un si grand nombre d’hommes faisaient leur soumission, passerait sur le moment inaperçue, ou même resterait toujours ignorée, sortirent du camp des Helvètes aux premières heures de la nuit et partirent vers le Rhin et la Germanie.
\subsection[{§ 28.}]{ \textsc{§ 28.} }
\noindent Quand César apprit la chose, il enjoignit aux peuples dont ils avaient traversé les territoires de les rechercher et de les lui ramener, s’ils voulaient être justifiés à ses yeux ; on les ramena et il les traita comme des ennemis ; tous les autres, une fois qu’ils eurent livré otages, armes et déserteurs, virent leur soumission acceptée. Helvètes, Tulinges et Latobices reçurent l’ordre de regagner le pays d’où ils étaient partis ; comme ils avaient détruit toutes leurs récoltes, et qu’il ne leur restait rien pour se nourrir, César donna ordre aux Allobroges de leur fournir du blé ; à eux, il enjoignit de reconstruire les villes et les villages qu’ils avaient incendiés. Ce qui surtout lui dicta ces mesures, ce fut le désir de ne pas laisser désert le pays que les Helvètes avaient abandonné, car la bonne qualité des terres lui faisait craindre que les Germains qui habitent sur l’autre rive du Rhin ne quittassent leur pays pour s’établir dans celui des Helvètes, et ne devinssent ainsi voisins de la province et des Allobroges. Quant aux Boïens, les Héduens demandèrent, parce qu’ils étaient connus comme un peuple d’une particulière bravoure, à les installer chez eux ; César y consentit ; ils leur donnèrent des terres, et par la suites les admirent à jouir des droits et des libertés dont ils jouissaient eux-mêmes.
\subsection[{§ 29.}]{ \textsc{§ 29.} }
\noindent On trouva dans le camp des Helvètes des tablettes écrites en caractères grecs ; elles furent apportées à César. Elles contenaient la liste nominative des émigrants en état de porter les armes, et aussi une liste particulière des enfants, des vieillards et des femmes. Le total général était de 263000 Helvètes, 36000 Tulinges, 14000 Latobices, 23000 Rauraques, 32000 Boïens ; ceux qui parmi eux pouvaient porter les armes étaient environ 92§ 000.\par
En tout, c’était une population de 368000 âmes. Ceux qui retournèrent chez eux furent recensés, suivant un ordre de César on trouva le chiffre de 110§ 000.
\subsection[{§ 30.}]{ \textsc{§ 30.} }
\noindent Une fois achevée la guerre contre les Helvètes, des députés de presque toute la Gaule, qui étaient les chefs dans leur cité, vinrent féliciter César. Ils comprenaient, dirent-ils, que si par cette guerre, il avait vengé d’anciens outrages des Helvètes au peuple romain, toutefois les événements qui venaient de se produire n’étaient pas moins avantageux pour le pays gaulois que pour Rome car les Helvètes, en pleine prospérité, n’avaient abandonné leurs demeures que dans l’intention de faire la guerre à la Gaule entière, d’en devenir les maîtres, de choisir pour s’y fixer, parmi tant de régions, celle qu’ils jugeraient la plus favorable et la plus fertile, et de faire payer tribut aux autres nations. Ils exprimèrent leur désir de fixer un jour pour une assemblée générale de la Gaule et d’avoir pour cela la permission de César : ils avaient certaines choses à lui demander après s’être mis d’accord entre eux. César donna son assentiment ; ils fixèrent le jour de la réunion, et chacun s’engagea par serment à ne révéler à personne ce qui s’y dirait, sauf mandat formel de l’assemblée.
\subsection[{§ 31.}]{ \textsc{§ 31.} }
\noindent Quand celle-ci se fut séparée, les mêmes chefs de nations qui avaient une première fois parlé à César revinrent le trouver et sollicitèrent la faveur de l’entretenir sans témoins et dans un lieu secret d’une question qui intéressait leur salut et celui du pays tout entier. César y consentit ; alors ils se jetèrent tous à ses pieds en pleurant : « Leur désir, dirent-ils, de ne pas voir ébruiter leurs déclarations était aussi vif et aussi anxieux que celui d’obtenir ce qu’ils voulaient ; car, si leurs paroles étaient connues, ils se savaient voués aux pires supplices. » L'Héduen Diviciacos parla en leur nom : « L'ensemble de la Gaule était divisé en deux factions : l’une avait à sa tête les Héduens, l’autre les Arvernes. Depuis de longues années, ils luttaient âprement pour l’hégémonie, et il s’était produit ceci, que les Arvernes et les Séquanes avaient pris des Germains à leur solde. Un premier groupe d’environ quinze mille hommes avait d’abord passé le Rhin ; puis, ces rudes barbares prenant goût au pays, aux douceurs de sa civilisation, à sa richesse, il en vint un plus grand nombre ; ils étaient à présent aux environs de cent vingt mille. Les Héduens et leurs clients s’étaient plus d’une fois mesurés avec eux ; ils avaient été battus, subissant un grand désastre, où ils avaient perdu toute leur noblesse, tout leur sénat, toute leur cavalerie. Épuisés par ces combats, abattus par le malheur, eux qui auparavant avaient été, grâce à leur courage et aux liens d’hospitalité et d’amitié qui les unissaient aux Romains, si puissants en Gaule, ils avaient été réduits à donner comme otages aux Séquanes leurs premiers citoyens, et à jurer, au nom de la cité, qu’ils ne les redemanderaient pas, qu’ils n’imploreraient pas le secours de Rome, qu’ils ne chercheraient jamais à se soustraire à l’absolue domination des Séquanes. Il était le seul de toute la nation héduenne qui ne se fût pas plié à prêter serment et à livrer ses enfants comme otages. Il avait dû, pour cette raison, s’enfuir de son pays, et il était allé à Rome demander du secours au Sénat, étant le seul qui ne fût lié ni par un serment, ni par des otages. Mais les Séquanes avaient eu plus de malheur dans leur victoire que les Héduens dans leur défaite, car Arioviste, roi des Germains, s’était établi dans leur pays et s’était emparé d’un tiers de leurs terres, qui sont les meilleures de toute la Gaule ; et à présent il leur intimait l’ordre d’en évacuer un autre tiers, pour la raison que peu de mois auparavant vingt-quatre mille Harudes étaient venus le trouver, et qu’il fallait leur faire une place et les établir. Sous peu d’années, tous les Gaulois seraient chassés de Gaule et tous les Germains passeraient le Rhin car le sol de la Gaule et celui de la Germanie n’étaient pas à comparer, non plus que la façon dont on vivait dans l’un et l’autre pays. Et Arioviste, depuis qu’il a remporté une victoire sur les armées gauloises, – la victoire d’Admagétobrige – se conduit en tyran orgueilleux et cruel, exige comme otages les enfants des plus grandes familles et les livre, pour faire des exemples, aux pires tortures, si on n’obéit pas au premier signe ou si seulement son désir est contrarié. C'est un homme grossier, irascible, capricieux ; il est impossible de souffrir plus longtemps sa tyrannie. A moins qu’ils ne trouvent une aide auprès de César et du peuple romain, tous les Gaulois seront dans la nécessité de faire ce qu’ont fait les Helvètes, d’émigrer, de chercher d’autres toits, d’autres terres, loin des Germains, de tenter enfin la fortune, quelle qu’elle puisse être. Si ces propos sont rapportés à Arioviste, point de doute il fera subir le plus cruel supplice à tous les otages qui sont entre ses mains. Mais César, par son prestige personnel et celui de son armée, grâce à sa récente victoire, grâce au respect qu’inspire le nom romain, peut empêcher qu’un plus grand nombre de Germains ne franchisse le Rhin, et protéger toute la Gaule contre les violences d’Arioviste. »
\subsection[{§ 32.}]{ \textsc{§ 32.} }
\noindent Quand Diviciacos eut achevé ce discours, tous les assistants se mirent, avec force larmes, à implorer le secours de César. Celui-ci observa que seuls entre tous, les Séquanes ne faisaient rien de ce que faisaient les autres, mais gardaient tristement la tête baissée et les regards fixés au sol. Étonné de cette attitude, il leur en demanda la raison. Aucune réponse : les Séquanes restaient muets et toujours accablés. Il insista à plusieurs reprises, et ne put obtenir d’eux le moindre mot ; ce fut l’Héduen Diviciacos qui, reprenant la parole, lui répondit. « Le sort des Séquanes avait ceci de particulièrement pitoyable et cruel, que seuls entre tous ils n’osaient pas, même en cachette, se plaindre ni demander du secours, et, en l’absence d’Arioviste, redoutaient sa cruauté comme s’il était là, les autres peuples, en effet, avaient malgré tout la ressource de fuir, tandis qu’eux, qui avaient admis Arioviste sur leur territoire et dont toutes les villes étaient en sa possession, ils étaient voués à toutes les atrocités. »
\subsection[{§ 33.}]{ \textsc{§ 33.} }
\noindent Quand il eut connaissance de ces faits, César rassura les Gaulois et leur promit qu’il donnerait ses soins à cette affaire « Il avait, leur dit-il, grand espoir que par le souvenir de ses bienfaits et par son autorité il amènerait Arioviste à cesser ses violences. » Leur ayant tenu ce discours, il renvoya l’assemblée. Outre ce qu’il venait d’entendre, plusieurs motifs l’invitaient à penser qu’il devait se préoccuper de cette situation et intervenir ; le principal était qu’il voyait les Héduens, à qui le Sénat avait si souvent donné le nom de frères, soumis aux Germains, devenus leurs sujets, et qu’il savait que des otages héduens étaient au pouvoir d’Arioviste et des Séquanes cela lui paraissait, quand on songeait à la toute-puissance de Rome, une grande honte et pour la République et pour lui-même. Il se rendait compte d’ailleurs qu’il était dangereux pour le peuple Romain que les Germains prissent peu à peu l’habitude de passer le Rhin et de venir par grandes masses dans la Gaule ; il estimait que ces hommes violents et incultes ne sauraient se retenir, après avoir occupé toute la Gaule, de passer dans la province romaine et, de là, marcher sur l’Italie, comme avaient fait avant eux les Cimbres et les Teutons : entreprise d’autant plus aisée que les Séquanes n’étaient séparés de notre province que par le Rhône ; à de pareilles éventualités il fallait, pensait-il, parer au plus tôt. Arioviste enfin était devenu si orgueilleux, si insolent, qu’il le jugeait intolérable.
\subsection[{§ 34.}]{ \textsc{§ 34.} }
\noindent Il décida donc de lui envoyer une ambassade qui lui demanderait de choisir un endroit pour une entrevue à mi-chemin des deux armées : « Il voulait traiter avec lui d’affaires d’État et qui les intéressaient au plus haut point l’un et l’autre. » Arioviste répondit que « s’il avait eu quelque chose à demander à César, il serait allé le trouver ; si César voulait quelque chose de lui, c’était à César à le venir voir. » Il ajouta qu’il n’osait pas se rendre sans armée dans la partie de la Gaule qui était au pouvoir de César, que, d’autre part, le rassemblement d’une armée exigeait de grands approvisionnements et coûtait beaucoup de peine. Au reste, il se demandait ce qu’avaient à faire César, et d’une façon générale les Romains, dans une Gaule qui lui appartenait, qu’il avait conquise.
\subsection[{§ 35.}]{ \textsc{§ 35.} }
\noindent Quand on lui rapporta cette réponse du chef germain, César lui envoya une deuxième ambassade chargée du message suivant : « Il avait reçu de lui et du peuple Romain un grand bienfait, s’étant vu décerner par le Sénat, sous le consulat de César, les titres de roi et d’ami ; puisque sa façon de témoigner à César et à Rome sa reconnaissance, c’était, quand César l’invitait à une entrevue, de mal recevoir cette invitation, et de se refuser à un échange de vues sur les affaires qui leur étaient communes, il lui signifiait les exigences suivantes : en premier lieu, qu’il s’abstînt désormais de faire franchir le Rhin à de nouvelles bandes pour les établir en Gaule ; deuxièmement, qu’il rendît les otages que les Héduens lui avaient donnés, et laissât les Séquanes rendre, avec son consentement exprès, ceux qu’ils détenaient ; il devait enfin cesser de poursuivre de ses violences les Héduens, et ne faire la guerre ni à eux ni à leurs alliés. Si telle était sa conduite, César et le peuple Romain continueraient de lui donner leur faveur et leur amitié ; mais si ses demandes n’étaient pas reçues, César, fort de la décision du Sénat qui sous le consulat de Marcus Messala et de Marcus Pison, avait décrété que tout gouverneur de la province de Gaule devrait, autant que le permettrait le bien de l’État, protéger les Héduens et les autres amis de Rome, César ne laisserait pas impunis les torts qu’on leur ferait. »
\subsection[{§ 36.}]{ \textsc{§ 36.} }
\noindent Arioviste répliqua que les lois de la guerre voulaient que les vainqueurs imposassent leur autorité aux vaincus comme bon leur semblait. C'est ainsi qu’il était dans les traditions de Rome de dicter la loi aux vaincus non point d’après les ordres d’un tiers, mais selon son gré. Puisque, de son côté, il s’abstenait de prescrire aux Romains l’usage qu’ils devaient faire de leur droit, il ne convenait pas qu’il fût gêné par eux dans l’exercice du sien. Si les Héduens étaient ses tributaires, c’était parce qu’ils avaient tenté la fortune des armes, parce qu’ils avaient livré bataille et avaient eu le dessous. César lui faisait un tort grave en provoquant, par son arrivée, une diminution de ses revenus. Il ne rendrait pas les otages aux Héduens ; il ne leur ferait pas, à eux ni à leurs alliés, de guerre injuste, mais il fallait qu’ils observassent les conventions et payassent chaque année le tribut ; sinon, le titre de frères du peuple Romain ne leur servirait guère. Quant à l’avis que lui donnait César, qu’il ne laisserait pas impunis les torts qu’on ferait aux Héduens, personne ne s’était encore mesuré avec lui que pour son malheur. Il pouvait, quand il voudrait, venir l’attaquer il apprendrait ce que des Germains qui n’avaient jamais été vaincus, qui étaient très entraînés à la guerre, qui, dans l’espace de quatorze ans, n’avaient pas couché sous un toit, étaient capables de faire. »
\subsection[{§ 37.}]{ \textsc{§ 37.} }
\noindent En même temps qu’on rapportait à César cette réponse, arrivaient deux ambassades, l’une des Héduens, l’autre des Trévires ; les premiers venaient se plaindre que les Harudes, qui étaient récemment passés en Gaule, ravageaient leur territoire : « Ils avaient eu beau donner des otages, cela n’avait pu leur valoir la paix de la part d’Arioviste » ; quant aux Trévires, ils faisaient savoir que cent clans de Suèves s’étaient établis sur les bords du Rhin, et cherchaient à franchir le fleuves ; ils étaient commandés par Nasua et Cimbérios, deux frères. César, vivement ému de ces nouvelles, estima qu’il devait faire diligence, pour éviter que, la nouvelle troupe de Suèves ayant fait sa jonction avec les anciennes forces d’Arioviste, la résistance ne lui fût rendue plus difficile. Aussi, ayant réuni des vivres en toute hâte, il marcha contre Arioviste à grandes étapes.
\subsection[{§ 38.}]{ \textsc{§ 38.} }
\noindent Après trois jours de marche, on lui apprit qu’Arioviste, avec toutes ses forces, se dirigeait vers Besançon, la ville la plus importante des Séquanes, pour s’en emparer, et qu’il était déjà à trois jours des frontières de son royaume. César pensa qu’il fallait tout faire pour éviter que la place ne fût prise. En effet, elle possédait en très grande abondance tout ce qui est nécessaire pour faire la guerre ; de plus, sa position naturelle la rendait si forte qu’elle offrait de grandes facilités pour faire durer les hostilités : le Doubs entoure presque la ville entière d’un cercle qu’on dirait tracé au compas ; l’espace que la rivière laisse libre ne mesure pas plus de seize cents pieds, et une montagne élevée le ferme si complètement que la rivière en baigne la base des deux côtés. Un mur qui fait le tour de cette montagne la transforme en citadelle et la joint à la ville. César se dirige vers cette place à marches forcées de jour et de nuit ; il s’en empare et y met garnison.
\subsection[{§ 39.}]{ \textsc{§ 39.} }
\noindent Tandis qu’il faisait halte quelques jours près de Besançon pour se ravitailler en blé et autres vivres, les soldats questionnaient, indigènes et marchands bavardaient : ils parlaient de la taille immense des Germains, de leur incroyable valeur militaire, de leur merveilleux entraînement : « Bien des fois, disaient les Gaulois, nous nous sommes mesurés avec eux, et le seul aspect de leur visage, le seul éclat de leurs regards nous furent insoutenables. » De tels propos provoquèrent dans toute l’armée une panique soudaine, et si forte qu’un trouble considérable s’empara des esprits et des cœurs. Cela commença par les tribuns militaires, les préfets, et ceux qui, ayant quitté Rome avec César pour cultiver son amitié, n’avaient pas grande expérience de la guerres ; sous des prétextes variés dont ils faisaient autant de motifs impérieux de départ, ils demandaient la permission de quitter l’armée ; un certain nombre pourtant, retenus par le sentiment de l’honneur et voulant éviter le soupçon de lâcheté, restaient au camp : mais ils ne pouvaient composer leur visage, ni s’empêcher, par moments, de pleurer ; ils se cachaient dans leurs tentes pour gémir chacun sur leur sort ou pour déplorer, en compagnie de leurs intimes, le danger qui les menaçait tous. Dans tout le camp on ne faisait que sceller des testaments. Les propos, la frayeur de ces gens peu à peu ébranlaient ceux-là même qui avaient une grande expérience militaire, soldats, centurions, officiers de cavalerie. Ceux qui parmi eux voulaient passer pour plus braves disaient qu’ils ne craignaient point l’ennemi, mais les défilés qu’il fallait franchir et les forêts immenses qui les séparaient d’Arioviste, ou bien ils prétendaient redouter que le ravitaillement ne pût se faire dans d’assez bonnes conditions. Quelques-uns étaient allés jusqu’à faire savoir à César que, quand il aurait donné l’ordre de lever le camp et de se porter en avant, les soldats n’obéiraient pas et, sous l’empire de la peur, refuseraient de marcher.
\subsection[{§ 40.}]{ \textsc{§ 40.} }
\noindent Voyant cela, César réunit le conseil, et il y convoqua les centurions de toutes les cohortes ; il commença par leur reprocher avec véhémence leur prétention de savoir où on les menait, ce qu’on se proposait, et de raisonner là-dessus. « Arioviste avait, sous son consulat, recherché avec le plus grand empressement l’amitié des Romains ; quelle raison de penser qu’il manquerait avec tant de légèreté à son devoir ? Pour sa part, il était convaincu que lorsque le Germain connaîtrait ce que César demande et verrait combien ses propositions sont équitables, il ne refuserait pas de vivre en bonne intelligence avec lui et avec le peuple Romain. Et si, obéissant à l’impulsion d’une fureur démente, il déclarait la guerre, qu’avaient-ils donc à craindre ? Quelles raisons de désespérer de leur propre valeur ou du zèle attentif de leur chef ? On avait déjà connu cet adversaire du temps de nos pères, quand Marius remporta sur les Cimbres et les Teutons une victoire qui ne fut pas moins glorieuse pour ses soldats que pour lui-même ; on l’avait connu aussi, plus récemment, en Italie, lors de la révolte des esclaves, et encore ceux-ci trouvaient-ils un accroissement de force dans leur expérience militaire et leur discipline, qualités qu’ils nous devaient. Leur exemple permettait de juger ce qu’on pouvait attendre de la fermeté d’âme, puisque des hommes qu’on avait un moment redoutés sans motif quand ils étaient dépourvus d’armes, avaient été battus ensuite alors qu’ils étaient bien armés et avaient des victoires à leur actif. Enfin ces Germains sont les mêmes hommes avec qui, à maintes reprises, les Helvètes se sont mesurés, et dont ils ont presque toujours triomphé non seulement sur leur propre territoire, mais en Germanie même et pourtant les Helvètes n’ont pu tenir devant nos troupes. Si certains esprits s’alarmaient de l’échec et de la déroute des Gaulois, il leur suffisait de réfléchir pour en découvrir les causes ; à un moment où les Gaulois étaient fatigués de la longueur de la guerre, Arioviste, qui, pendant de longs mois s’était confiné dans son camp, au milieu des marécages, les avait attaqués soudainement, quand ils désespéraient de pouvoir jamais combattre et s’étaient disséminés ; sa victoire était due moins à la valeur des Germains qu’à l’habile tactique de leur chef. Mais une tactique qui avait été bonne pour combattre des hommes barbares et sans expérience, Arioviste lui-même n’espérait pas que nos armées s’y pussent laisser prendre.\par
\par
Ceux qui déguisaient leur lâcheté en prétextant qu’ils étaient inquiets de la question des vivres et des difficultés de la route, ceux-là étaient des insolents, car ils avaient l’air ou de n’avoir aucune confiance en leur général, ou de lui dicter des ordres. Il s’occupait de ces questions du blé, les Séquanes, les Leuques, les Lingons en fournissaient, et les moissons étaient déjà mûres dans les champs ; la route, ils en jugeraient sous peu par eux-mêmes. Quant à ce que l’on disait, qu’il ne serait pas obéi et que les troupes refuseraient de marcher, cela ne le troublait nullement : il savait bien en effet, que tous les chefs aux ordres de qui leur armée n’avait point obéi ou bien avaient essuyé des échecs et s’étaient vus abandonnés de la Fortune, ou bien avaient commis quelque mauvaise action dont la découverte les avait convaincus de malhonnêteté. Mais lui, sa vie entière témoignait de son désintéressement, et la guerre des Helvètes avait bien montré quelle était sa chance. Aussi, ce qu’il avait eu d’abord l’intention de ne faire que dans quelque temps, il l’exécuterait sur-le-champ, et il lèverait le camp cette nuit, au cours de la quatrième veille, car il voulait savoir au plus tôt s’ils obéissaient à la voix de l’honneur et du devoir, ou aux conseils de la peur. Si maintenant personne ne le suit, il n’en marchera pas moins, suivi seulement de la dixième légion, dont il était sûr, et qui lui servirait de cohorte prétoriennes. » Cette légion était celle à qui César avait témoigné le plus d’affection, et dont la valeur lui inspirait le plus de confiance.
\subsection[{§ 41.}]{ \textsc{§ 41.} }
\noindent Ce discours produisit un changement merveilleux dans les esprits ; il y fit naître un grand enthousiasme et la plus vive impatience de combattre ; on vit d’abord la dixième légion, par l’entremise de ses tribuns, remercier César de l’excellente opinion qu’il avait d’elle et lui confirmer qu’elle était toute prête à combattre. Puis les autres légions négocièrent avec leurs tribuns et les centurions de leur première cohorte pour qu’ils les fissent excuser par César : « Ils n’avaient jamais pensé qu’ils eussent à juger de la conduite des opérations ; c’était l’affaire de leur général. » César accepta leurs explications ; Diviciacos, chargé d’étudier l’itinéraire parce qu’il était celui des Gaulois en qui César avait le plus de confiance, conseilla de faire un détour de plus de cinquante milles, qui permettrait de marcher en terrain découvert ; César partit au cours de la quatrième veille, comme il l’avait dit. Après sept jours de marche continue, ses éclaireurs lui firent savoir que les troupes d’Arioviste étaient à vingt-quatre milles des nôtres.
\subsection[{§ 42.}]{ \textsc{§ 42.} }
\noindent Quand il apprend l’approche de César, Arioviste lui envoie une ambassade : « Il ne s’opposait pas, quant à lui, à ce qu’eût lieu l’entrevue précédemment demandée, puisque César s’était rapproché ; il estimait qu’il pouvait s’y rendre sans danger. » César ne refusa pas ; il croyait que le Germain revenait à la raison, puisqu’il proposait de lui-même ce qu’il avait précédemment refusé quand on le lui demandait ; et il espérait beaucoup que, se souvenant des bienfaits qu’il avait reçus de lui et du peuple Romain, quand il aurait examiné ses conditions, il cesserait d’être intraitable. L'entrevue fut fixée au cinquième jour suivant. Comme, en attendant, des envoyés allaient et venaient souvent de l’un à l’autre, Arioviste demanda que César n’amenât pas à l’entrevue de troupes à pied : « Il craignait, disait-il, que César ne l’attirât dans une embuscade ; que chacun vînt avec des cavaliers ; il ne viendrait qu’à cette condition. » César, ne voulant pas qu’un prétexte suffît à supprimer la rencontre, et n’osant pas, d’autre part, s’en remettre à la cavalerie gauloise du soin de veiller sur sa vie, jugea que le plus pratique était de mettre à pied tous les cavaliers gaulois et de donner leurs montures aux légionnaires de la dixième légion, en qui il avait la plus grande confiance, afin d’avoir, en cas de besoin, une garde aussi dévouée que possible. Ainsi fit-on ; et un soldat de la dixième légion remarqua assez plaisamment que « César faisait plus qu’il n’avait promis : il avait promis qu’il les emploierait comme gardes du corps, et il faisait d’eux des chevaliers. »
\subsection[{§ 43.}]{ \textsc{§ 43.} }
\noindent Dans une grande plaine s’élevait un tertre assez haute : il était à peu près à égale distance du camp d’Arioviste et de celui de César. C'est là que, suivant leur convention, les deux chefs vinrent pour se rencontrer. César fit arrêter sa légion montée à deux cents pas du tertre ; les cavaliers d’Arioviste s’arrêtèrent à la même distance. Le Germain demanda que l’on s’entretînt à cheval, et que chacun amenât avec lui dix hommes. Quand ils furent au lieu de la rencontre, César, pour commencer, rappela ses bienfaits et ceux du Sénat, le titre de roi que cette assemblée lui avait donné, celui d’ami, et les riches présents qu’on lui avait prodigués ; puis il lui expliqua que peu de princes avaient obtenu ces distinctions, et qu’on ne les accordait d’habitude que pour des services éminents ; lui, qui n’avait pas de titres pour y prétendre ni de justes motifs pour les solliciter, il ne les avait dues qu’à la bienveillance et à la libéralité de César et du Sénat. Il lui apprit encore combien étaient anciennes et légitimes les raisons de l’amitié qui unissait les Héduens aux Romains, quels sénatus-consultes avaient été rendus en leur faveur, à mainte reprise et dans les termes les plus honorables ; comment, de tout temps, l’hégémonie de la Gaule entière avait appartenu aux Héduens, avant même qu’ils n’eussent recherché leur amitié. C'était une tradition des Romains de vouloir que leurs alliés et leurs amis, non seulement ne subissent aucune diminution, mais encore vissent s’accroître leur crédit, leur considération, leur dignité vraiment, ce qu’ils avaient apporté avec eux en devenant amis de Rome, qui pourrait souffrir qu’on le leur arrachât ? Il formula ensuite les mêmes demandes dont il avait chargé ses envoyés : ne faire la guerre ni aux Héduens, ni à leurs alliés ; rendre les otages ; s’il ne pouvait renvoyer chez eux aucun de ses Germains, au moins ne pas permettre que d’autres franchissent le Rhin.
\subsection[{§ 44.}]{ \textsc{§ 44.} }
\noindent Arioviste ne répondit que peu de chose aux demandes de César, mais s’étendit longuement sur ses propres mérites. « S'il avait passé le Rhin, ce n’était point spontanément, mais sur la prière instante des Gaulois ; il avait fallu de grandes espérances, la perspective de riches compensations, pour qu’il abandonnât son foyer et ses proches ; les terres qu’il occupait en Gaule, il les tenait des Gaulois ; les otages lui avaient été donnés par eux librement ; le tribut, il le percevait en vertu des lois de la guerre, c’était celui que les vainqueurs ont coutume d’imposer aux vaincus. Il n’avait pas été l’agresseur, mais c’étaient les Gaulois qui l’avaient attaqué ; tous les peuples de la Gaule étaient venus l’assaillir et avaient opposé leurs armées à la sienne ; il avait culbuté et vaincu toutes ces troupes en un seul combat. S'ils voulaient tenter une deuxième expérience, il était prêt à une nouvelle bataille ; s’ils voulaient la paix, il était injuste de refuser un tribut que jusqu’à présent ils avaient payé volontairement. L'amitié du peuple Romain devait lui être honorable et utile, et non point désavantageuse ; c’était dans cet espoir qu’il l’avait demandée. Si, grâce au peuple Romain, ses tributaires sont dispensés de payer et ses sujets soustraits à ses lois, il renoncera à son amitié aussi volontiers qu’il l’a recherchée. Il fait passer en Gaule un grand nombre de Germains ? Ce n’est point pour attaquer ce pays, mais pour garantir sa propre sécurité : la preuve, c’est qu’il n’est venu que parce qu’on l’en avait prié, et qu’il n’a pas fait une guerre offensive, mais défensive. Il était venu en Gaule avant les Romains. Jamais jusqu’à présent une armée romaine n’avait franchi les frontières de la Province. Que lui voulait César, pour venir ainsi sur ses terres ? Cette partie de la Gaule était sa province comme l’autre était la nôtre. De même qu’il ne faudrait pas le laisser faire s’il envahissait notre territoire, de même nous commettions une injustice en venant le troubler dans l’exercice de ses droits. Les Héduens, disait César, avaient reçu le nom de frères : mais il n’était ni assez barbare ni assez peu au courant pour ne pas savoir que les Héduens n’avaient pas porté secours aux Romains dans la dernière guerre contre les Allobroges, et que Rome, à son tour, ne les avait point aidés dans le conflit qu’ils venaient d’avoir avec lui-même et avec les Séquanes. Il était obligé de soupçonner que, sous le prétexte de cette amitié, César n’avait une armée en Gaule que pour la jeter contre lui. Si César ne quitte point ce pays, s’il n’en retire pas ses troupes, il le considérera, non comme un ami, mais comme un ennemi. Et s’il le tue, il fera quelque chose d’agréable à bien des nobles et chefs politiques de Rome : eux-mêmes l’en avaient assuré par leurs agents ; la bienveillance et l’amitié de tous ces personnages, il pouvait l’acquérir à ce prix. Mais si César s’en allait et lui laissait la libre disposition de la Gaule, il lui témoignerait magnifiquement sa reconnaissance, et toutes les guerres qu’il voudrait, il prendrait sur lui de les faire, sans que César en connût les fatigues ni les dangers.
\subsection[{§ 45.}]{ \textsc{§ 45.} }
\noindent César lui expliqua longuement pour quelles raisons il ne pouvait se désintéresser de la question : « Il n’était ni dans ses habitudes, ni dans celles du peuple Romain de consentir à abandonner des alliés parfaitement dévoués, et d’ailleurs il ne pensait pas que la Gaule appartînt plus à Arioviste qu’aux Romains. Les Arvernes et les Rutènes avaient été vaincus par Q. Fabius Maximus ; le peuple Romain leur avait pardonné, sans réduire leur pays en province, sans même leur imposer de tribut. S'il fallait avoir égard à l’antériorité de date, le pouvoir des Romains en Gaule était le plus légitime ; s’il fallait observer la décision du Sénat, la Gaule devait être libre, puisqu’il avait voulu que, vaincue par Rome, elle conservât ses lois. »
\subsection[{§ 46.}]{ \textsc{§ 46.} }
\noindent Tandis qu’avaient lieu ces pourparlers, on vint dire à César que les cavaliers d’Arioviste s’approchaient du tertre, poussaient leurs chevaux vers notre troupe, lui jetaient des pierres et des traits. César rompit l’entretien, rejoignit les siens et leur donna l’ordre de ne pas répondre aux Germains, fût-ce par un seul trait. En effet, quoiqu’il ne risquât rien à engager une légion d’élite contre des cavaliers, il ne voulait cependant pas s’exposer à ce qu’on pût dire, une fois les ennemis défaits, qu’il les avait surpris pendant une entrevue en abusant de la parole donnée. Quand on sut dans les rangs de l’armée quelle arrogance avait montrée Arioviste au cours de l’entretien, prétendant interdire aux Romains toute la Gaule, comment ses cavaliers avaient attaqué les nôtres et comment cet incident avait rompu les pourparlers, l’impatience de nos soldats en fut accrue et ils éprouvèrent un plus vif désir de combattre.
\subsection[{§ 47.}]{ \textsc{§ 47.} }
\noindent Le lendemain, Arioviste envoie à César une ambassade : « Il désirait reprendre l’entretien qu’ils avaient entamé et qui avait été interrompu ; que César fixât le jour d’une nouvelle entrevue, ou, si cela ne lui plaisait point, qu’il lui envoyât un de ses légats. » César ne pensa pas qu’il eût motif d’aller s’entretenir avec lui, d’autant plus que la veille on n’avait pu empêcher les Germains de lancer des traits à nos soldats. Envoyer quelqu’un des siens, le jeter entre les mains de ces hommes barbares, c’était courir grand risque. Il pensa que le mieux c’était d’envoyer Caïus Valérius Procillus, fils de Caïus Valérius Caburus, jeune homme plein de courage et fort cultivé, dont le père avait reçu de Caïus Valérius Flaccus la cité romaine : il était loyal, il parlait le gaulois, qu’une pratique déjà longue avait rendu familier à Arioviste, enfin les Germains n’avaient pas de raison d’attenter à sa personne ; il lui adjoignit Marcus Métius que l’hospitalité liait à Arioviste. Ils avaient pour instructions d’écouter ce qu’il dirait et de le rapporter. Quand Arioviste les aperçut devant lui, dans son camp, il éclata, devant toute l’armée : « Pourquoi venaient-ils ? Pour espionner, sans doute ? » Ils voulaient parler, il les en empêcha et les fit charger de chaînes.
\subsection[{§ 48.}]{ \textsc{§ 48.} }
\noindent Le même jour, il se porta en avant et vint s’établir à six milles du camp de César, au pied d’une montagne. Le lendemain, il passa devant le camp de César et alla camper à deux milles au-delà, dans la pensée d’arrêter les convois de blé et autres vivres que lui enverraient les Séquanes et les Héduens. Alors, pendant cinq jours de suite, César fit sortir ses troupes en avant du camp et les tint rangées en bataille, de façon que, si Arioviste désirait combattre, l’occasion ne lui fît pas défaut. Mais Arioviste, pendant tous ces jours-là, garda son infanterie au camp, livrant, par contre, des combats de cavalerie quotidiens. Le genre de combat auquel les Germains étaient entraînés était le suivant. Ils étaient six mille cavaliers, et autant de fantassins, les plus agiles et les plus braves de tous chaque cavalier en avait choisi un sur l’ensemble des troupes, avec la préoccupation de sa sûreté personnelle : car ces fantassins étaient leurs compagnons de combat. C'était sur eux qu’ils se repliaient ; ils entraient en ligne si la situation devenait critique ; ils entouraient et protégeaient celui qui, grièvement blessé, était tombé de cheval ; s’il fallait avancer à quelque distance ou faire une retraite rapide, ils avaient, grâce à leur entraînement, une telle agilité, qu’en se tenant aux crinières des chevaux ils les suivaient à la course.
\subsection[{§ 49.}]{ \textsc{§ 49.} }
\noindent Lorsque César vit que son adversaire se tenait enfermé dans son camp, ne voulant pas être plus longtemps privé de ravitaillement, il choisit, au-delà de la position qu’avaient occupée les Germains, à environ six cents pas de ceux-ci, un endroit propre à l’établissement d’un camp et il y conduisit son armée, marchant en ordre de bataille sur trois rangs. Les deux premières lignes reçurent l’ordre de rester sous les armes, tandis que la troisième fortifierait le camp. Cette position était, comme on l’a dit, à environ six cents pas de l’ennemi. Arioviste y envoya environ seize mille hommes équipés à la légère et toute sa cavalerie, avec mission d’effrayer les nôtres et d’empêcher leurs travaux. César n’en maintint pas moins les dispositions qu’il avait prises : les deux premières lignes devaient tenir l’ennemi en respect, et la troisième achever son ouvrage. Une fois le camp fortifié, il y laissa deux légions et une partie des troupes auxiliaires, et ramena dans le grand camp les quatre autres légions.
\subsection[{§ 50.}]{ \textsc{§ 50.} }
\noindent Le lendemain, suivant sa tactique habituelle, César fit sortir ses troupes des deux camps et rangea son armée en bataille à une certaine distance en avant du grand, offrant le combat à l’ennemi. Quand il vit que même ainsi les Germains ne s’avançaient pas, vers midi il ramena ses troupes à leurs campements. Arioviste alors se décida à envoyer une partie de ses forces donner l’assaut au petit camp. On se battit avec acharnement de part et d’autre jusqu’au soir. Au coucher du soleil, Arioviste ramena ses troupes dans son camp ; les pertes avaient été sévères des deux côtés. César demanda aux prisonniers pourquoi Arioviste ne livrait pas une bataille générale ; il apprit que, suivant la coutume des Germains, leurs femmes devaient, en consultant le sort et en rendant des oracles, dire s’il convenait ou non de livrer bataille ; or, elles disaient que les destins ne permettaient pas la victoire des Germains s’ils engageaient le combat avant la nouvelle lune.
\subsection[{§ 51.}]{ \textsc{§ 51.} }
\noindent Le lendemain, César, laissant pour garder chacun des camps les forces qui lui parurent suffisantes, disposa toutes ses troupes auxiliaires à la vue de l’ennemi devant le petit camp ; comme ses légionnaires étaient numériquement inférieurs aux troupes d’Arioviste, il voulait faire illusion sur leur nombre en employant ainsi les auxiliaires. Lui-même, ayant dispersé ses légions en ordre de bataille sur trois rangs, il s’avança jusque devant le camp ennemi. Alors les Germains, contraints et forcés, se décidèrent à faire sortir leurs troupes : ils les établirent, rangées par peuplades, à des intervalles égaux, Harudes, Marcomans, Triboques, Vangions, Némètes, Sédusiens, Suèves ; et, pour s’interdire tout espoir de fuite, ils formèrent une barrière continue sur tout l’arrière du front avec les chariots et les voitures. Ils y firent monter leurs femmes, qui, tendant leurs mains ouvertes et versant des larmes, suppliaient ceux qui partaient au combat de ne pas faire d’elles des esclaves des Romains.
\subsection[{§ 52.}]{ \textsc{§ 52.} }
\noindent César confia le commandement particulier de chaque légion à chacun de ses légats et à son questeur, afin que les soldats eussent en eux des témoins de leur valeur individuelle ; lui-même engagea le combat par l’aile droite, parce qu’il avait observé que la ligne ennemie était moins solide de ce côté-là. Nos soldats, au signal donné, se ruèrent à l’ennemi avec une telle vigueur, l’ennemi, de son côté, s’élança si soudainement et d’une course si rapide à leur rencontre, qu’ils n’eurent pas devant eux l’espace nécessaire au lancement du javelot. Abandonnant cette arme, ils engagèrent un corps à corps avec l’épée. Mais les Germains, selon leur tactique habituelle, formèrent rapidement la phalange et reçurent ainsi le choc des épées. Il s’en trouva plus d’un parmi les nôtres pour se jeter sur le mur de boucliers que formait chaque phalange, les arracher et frapper l’ennemi de haut en bas. Tandis que l’aile gauche des Germains avait été complètement enfoncée, à droite ils nous accablaient sous le nombre. Le jeune Publius Crassus, qui commandait la cavalerie, se rendant compte du danger – il était mieux à même de suivre l’action que ceux qui se trouvaient dans la mêlée – envoya les troupes de troisième ligne au secours de celles qui étaient en péril.
\subsection[{§ 53.}]{ \textsc{§ 53.} }
\noindent Cette mesure rétablit la situation ; tous les ennemis prirent la fuite, et ne s’arrêtèrent qu’au Rhin, à environ cinq milles du lieu de la bataille. Là, un très petit nombre, ou bien, se fiant à leur vigueur, tâchèrent de passer le fleuve à la nage, ou bien découvrirent des barques auxquelles ils durent leur salut. Ce fut le cas d’Arioviste, qui trouva une embarcation attachée au rivage et put s’enfuir sur elle ; tous les autres furent rejoints par notre cavalerie et massacrés. Arioviste avait deux épouses : l’une Suève, qu’il avait emmenée de Germanie avec lui, l’autre du Norique, la sœur du roi Voccion, que celui-ci lui avait envoyée et qu’il avait épousée en Gaule ; toutes deux périrent dans la déroute. Il avait deux filles : l’une fut tuée, l’autre fut faite prisonnière. Laïus Valérus Procillus, que ses gardiens emmenaient avec eux dans leur fuite chargé de triples chaînes, tomba entre les mains de César lui-même qui poursuivait l’ennemi avec ses cavaliers ; cet incident ne lui causa pas moins de plaisir que la victoire même, car celui qu’il arrachait aux mains des ennemis et retrouvait ainsi était l’homme le plus estimable de toute la province de Gaule, son ami et son hôte, et la Fortune, en l’épargnant, avait voulu que rien ne fût ôté à la joie d’un pareil triomphe. Valérius raconta qu’à trois reprises, sous ses yeux, on avait consulté les sorts pour décider s’il devait être sur-le-champ livré aux flammes ou réservé pour un autre temps ; c’était aux sorts qu’il devait la vie. Marcus Métius fut également retrouvé et ramené à César.
\subsection[{§ 54.}]{ \textsc{§ 54.} }
\noindent Quand la nouvelle de cette bataille fut connue de l’autre côté du Rhin, les Suèves, qui étaient venus sur les bords du fleuve, reprirent le chemin de leur pays ; mais les peuples qui habitent près du Rhin, voyant leur panique, se mirent à leur poursuite et en tuèrent un grand nombre. César avait en un seul été achevé deux grandes guerres il mena ses troupes prendre leurs quartiers d’hiver chez les Séquanes un peu avant que la saison l’exigeât ; il en confia le commandement à Labiénus, et partit pour la Gaule citérieure afin d’y tenir ses assises.
 \section[{Livre II}]{Livre II}\renewcommand{\leftmark}{Livre II}

\subsection[{§ 1.}]{ \textsc{§ 1.} }
\noindent César était dans la Gaule citérieure et les légions avaient pris leurs quartiers d’hiver, ainsi que nous l’avons dit plus haut, quand le bruit lui parvint à maintes reprises, confirmé par une lettre de Labiénus, que tous les peuples de la Belgique, qui forme, comme on l’a vu, un tiers de la Gaule, conspiraient contre Rome et échangeaient des otages. Les motifs du complot étaient les suivants : d’abord, ils craignaient qu’une fois tout le reste de la Gaule pacifié nous ne menions contre eux nos troupes ; puis, un assez grand nombre de Gaulois les sollicitaient : les uns, de même qu’ils n’avaient pas voulu que les Germains s’attardassent en Gaule, supportaient mal de voir une armée romaine hiverner dans leur pays et s’y implanter ; les autres, en raison de la mobilité et de la légèreté de leur esprit, rêvaient de changer de maîtres ; ils recevaient aussi des avances de plusieurs personnages qui – le pouvoir se trouvant généralement en Gaule aux mains des puissants et des riches qui pouvaient acheter des hommes – arrivaient moins facilement à leurs fins sous notre dominations.
\subsection[{§ 2.}]{ \textsc{§ 2.} }
\noindent Ces rapports et cette lettre émurent César. Il leva deux légions nouvelles dans la Gaule citérieure et, au début de l’été, il envoya son légat Quintus Pédius les conduire dans la Gaule ultérieure. Lui-même rejoint l’armée dès qu’on commence à pouvoir faire du fourrage. Il charge les Sénons et les autres peuples gaulois qui étaient voisins des Belges de s’informer de ce qu’on fait chez eux et de l’en avertir. Ils furent tous unanimes à lui rapporter qu’on levait des troupes, qu’on opérait la concentration d’une armée. Alors il pensa qu’il ne fallait pas hésiter à prendre l’offensive. Après avoir fait des provisions de blé, il lève le camp et en quinze jours environ arrive aux frontières de la Belgique.
\subsection[{§ 3.}]{ \textsc{§ 3.} }
\noindent On ne s’y attendait pas, et personne n’avait prévu une marche aussi rapide ; aussi les Rèmes, qui sont le peuple de Belgique le plus proche de la Gaule, députèrent-ils à César Iccios et Andocumborios, les plus grands personnages de leur nation, afin de lui dire qu’ils se plaçaient, eux et tout leurs biens, sous la protection de Rome et sous son autorité : ils n’ont pas partagé le sentiment des autres Belges, ils n’ont pas conspiré contre Rome ; ils sont prêts à donner des otages, à exécuter les ordres qu’ils recevront, à ouvrir leurs places fortes, à fournir du blé et autres prestations ; ils ajoutent que le reste de la Belgique est en armes, que les Germains établis sur la rive gauche du Rhin se sont alliés aux Belges, qu’enfin il y a chez ceux-ci un tel déchaînement de passion, et si général, que les Suessions même, qui sont leurs frères de race, qui vivent sous les mêmes lois, qui ont même chef de guerre, même magistrat, ils n’ont pu les empêcher de prendre part au mouvements.
\subsection[{§ 4.}]{ \textsc{§ 4.} }
\noindent César leur demanda quelles étaient les cités qui avaient pris les armes, quelle était leur importance, leur puissance militaire ; il obtint les renseignements : suivants la plupart des Belges étaient d’origine germanique ; ils avaient, jadis, passé le Rhin, et s’étant arrêtés dans cette région à cause de sa fertilité, ils en avaient chassé les Gaulois qui l’occupaient ; c’était le seul peuple qui, du temps de nos pères, alors que les Cimbres et les Teutons ravageaient toute la Gaule, avait su leur interdire l’accès de son territoire ; il en était résulté que, pleins du souvenir de cet exploit, ils s’attribuaient beaucoup d’importance et avaient de grandes prétentions pour les choses de la guerre. Quant à leur nombre, les Rèmes se disaient en possession des renseignements les plus complets, car, étant liés avec eux par des parentés et des alliances, ils savaient le chiffre d’hommes que chaque cité avait promis pour cette guerre, dans l’assemblée générale des peuples belges. Les plus puissants d’entre eux par le courage, l’influence, le nombre, étaient les Bellovaques : ils pouvaient mettre sur pied cent mille hommes ; ils en avaient promis soixante mille d’élite, et réclamaient la direction générale de la guerre. Les Suessions étaient les voisins des Rèmes ; ils possédaient un très vaste territoire, et très fertile. Ils avaient eu pour roi, de notre temps encore, Diviciacos, le plus puissant chef de la Gaule entière, qui, outre une grande partie de ces régions, avait aussi dominé la Bretagne ; le roi actuel état Galba. C'est à lui, parce qu’il était juste et avisé, qu’on remettait, d’un commun accord, la direction suprême de la guerre. Il possédait douze villes, il s’engageait à fournir cinquante mille hommes. Les Nerviens en promettaient autant : ils passent pour les plus farouches des Belges et sont les plus éloignés ; les Atrébates amèneraient quinze mille hommes, les Ambiens dix mille, les Morins vingt-cinq mille, les Ménapes sept mille, les Calètes dix mille, les Véliocasses et les Viromandues autant, les Atuatuques dix-neuf mille ; les Condruses, les Eburons, les Caerœsi, les Pémanes, qu’on réunit sous le nom de Germains, pensaient pouvoir fournir environ quarante mille hommes.
\subsection[{§ 5.}]{ \textsc{§ 5.} }
\noindent César encouragea les Rèmes et leur parla avec bienveillance ; il les invita à lui envoyer tout leurs sénateurs et à lui remettre comme otages les enfants de leurs chefs. Ces conditions furent toutes remplies ponctuellement au jour dit. Il s’adresse, d’autre part, en termes pressants, à Diviciacos l’Héduen, lui faisant connaître quel intérêt essentiel il y a, pour Rome et pour le salut commun, à empêcher la jonction des contingents ennemis, afin de n’avoir pas à combattre en une fois une si nombreuse armée. On pouvait l’empêcher, si les Héduens faisaient pénétrer leurs troupes sur le territoire des Bellovaques et se mettaient à dévaster leurs champs. Chargé de cette mission, il le congédie. Quand César vit que les Belges avaient fait leur concentration et marchaient contre lui, quand il sut par ses éclaireurs et par les Rèmes qu’ils n’étaient plus bien loin, il fit rapidement passer son armée au nord de l’Aisne, qui est aux confins du pays rémois, et établit là son camp. Grâce à cette disposition, César fortifiait un des côtés de son camp en l’appuyant à la rivière, il mettait à l’abri de l’ennemi ce qu’il laissait derrière lui, il assurait enfin la sécurité des convois que lui enverraient les Rèmes et les autres cités. Un pont franchissait cette rivière. Il y place un poste, et laisse sur la rive gauche son légat Quintus Titurius Sabinus avec six cohortes ; il fait protéger le camp par un retranchement de douze pieds de haut et par un fossé de dix-huit pieds.
\subsection[{§ 6.}]{ \textsc{§ 6.} }
\noindent A huit milles de ce camp était une ville des Rèmes nommée Bibrax. Les Belges lui livrèrent au passage un violent assaut. On n’y résista ce jour-là qu’à grand-peine. Gaulois et Belges ont la même manière de donner l’assaut. Ils commencent par se répandre en foule tout autour des murs et à jeter des pierres de toutes parts ; puis, quand le rempart est dégarni de ses défenseurs, ils forment la tortue, mettent le feu aux postes et sapent la muraille. Cette tactique était en l’occurrence facile à suivre ; car les assaillants étaient si nombreux à lancer pierres et traits que personne ne pouvait rester au rempart. La nuit vint interrompre l’assaut ; le Rème Iccios, homme de haute naissance et en grand crédit auprès des siens, qui commandait alors la place, envoie à César un de ceux qui lui avaient été députés pour demander la paix, avec mission d’annoncer que si on ne vient pas à son aide, il ne pourra tenir plus longtemps.
\subsection[{§ 7.}]{ \textsc{§ 7.} }
\noindent En pleine nuit, César, utilisant comme guides ceux mêmes qui avaient porté le message d’Iccios, envoie au secours des assiégés des Numides, des archers Crétois et des frondeurs Baléares ; l’arrivée de ces troupes, rendant l’espoir aux Rèmes, leur communique une nouvelle ardeur défensive, cependant qu’elle ôtait aux ennemis l’espoir de prendre la place. Aussi, après un court arrêt devant la ville, ayant ravagé les terres des Rèmes et brûlé tous les villages et tous les édifices qu’ils purent atteindre, ils se dirigèrent avec toutes leurs forces vers le camp de César, et s’établirent à moins de deux mille pas ; leur campement, à en juger par la fumée et les feux, s’étendait sur plus de huit milles.
\subsection[{§ 8.}]{ \textsc{§ 8.} }
\noindent César, tenant compte du nombre des ennemis et de leur très grande réputation de bravoure, décida, pour commencer, de surseoir à la bataille ; il n’en livrait pas moins chaque jour des combats de cavalerie, pour éprouver la valeur de l’ennemi et l’audace des nôtres. Il vit bientôt que nos troupes n’étaient pas inférieures à celles de l’adversaire. L'espace qui s’étendait devant le camp était naturellement propre au déploiement d’une ligne de bataille, parce que la colline où était placé le camp, dominant de peu la plaine, avait, face à l’ennemi, juste autant de largeur qu’en occupaient nos troupes une fois mises en ligne, et se terminait à chaque extrémité par des pentes abruptes, tandis qu’en avant elle formait une crête peu accentuée pour s’abaisser ensuite insensiblement vers la plaine. César fit creuser à chaque bout un fossé d’environ quatre cents pas de long perpendiculairement à la ligne de bataille ; aux extrémités de ces fossés il établit des redoutes et disposa des machines, pour éviter que les ennemis, une fois nos troupes déployées, ne pussent, étant si nombreux, nous prendre de flanc tandis que nous serions occupés à combattre. Ces dispositions prises, il laissa dans le camp les deux légions de formation récente, pour qu’elles pussent, au besoin, être amenées en renfort, et il rangea les six autres en bataille en avant de son camp. L'ennemi, de même, avait fait sortir et déployé ses troupes.
\subsection[{§ 9.}]{ \textsc{§ 9.} }
\noindent Il y avait entre les deux armées un marais de peu d’étendue. L'ennemi attendait, espérant que les nôtres entreprendraient de le franchir ; de leur côté les nôtres se tenaient prêts à profiter des embarras de l’ennemi, s’il tentait le premier le passage, pour fondre sur lui. Pendant ce temps, un combat de cavalerie se livrait entre les deux lignes. Aucun des adversaires ne se hasardant le premier à travers le marais, César, après que l’engagement de cavalerie se fut terminé en notre faveur, ramena ses troupes dans le camp. Les ennemis, aussitôt, se portèrent sans désemparer vers l’Aisne qui, on l’a dit, coulait derrière nôtre camp. Là, ayant trouvé des gués, ils essayèrent de faire passer la rivière à une partie de leurs forces, dans le dessein d’enlever, s’ils le pouvaient, le poste commandé par le légat Quintus Titurius, et de couper le pont ; s’ils ne réussissaient pas, ils dévasteraient le territoire des Rèmes, d’où nous tirions de grandes ressources pour cette campagne et nous empêcheraient de nous ravitailler.
\subsection[{§ 10.}]{ \textsc{§ 10.} }
\noindent César, informé par Titurius, fait franchir le pont à sa cavalerie, à l’infanterie légère des Numides, aux frondeurs et aux archers, et marche contre les ennemis. Il y eut un violent combat. On les attaqua dans l’eau, qui gênait leurs mouvements, et l’on en tua un grand nombre ; les autres, pleins d’audace, essayaient de passer par-dessus les cadavres : une grêle de traits les repoussa ; ceux qui avaient déjà passé, la cavalerie les enveloppa et ils furent massacrés. Quand les Belges comprirent qu’ils devaient renoncer et à prendre Bibrax et à franchir la rivière, quand ils virent que nous nous refusions à avancer, pour livrer bataille, sur un terrain défavorable, comme enfin ils commençaient, eux aussi, à manquer de vivres, ils tinrent conseil et décidèrent que le mieux était de retourner chacun chez soi, sauf à se rassembler de toutes parts pour défendre ceux dont le territoire aurait été d’abord envahi par l’armée romaine ; de la sorte ils auraient l’avantage de combattre chez eux et non chez autrui, et ils pourraient user des ressources de ravitaillement que leur pays leur offrait. Ce qui les détermina, ce fut, outre les autres motifs, la raison suivante : ils avaient appris que Diviciacos et les Héduens approchaient du pays des Bellovaques, et on ne pouvait convaincre ces derniers de tarder plus longtemps à secourir les leurs.
\subsection[{§ 11.}]{ \textsc{§ 11.} }
\noindent La chose résolue, ils sortirent du camp pendant la deuxième veille en grand désordre et tumulte, sans méthode ni discipline, chacun voulant être le premier sur le chemin du retour et ayant hâte d’arriver chez lui ; si bien que leur départ avait tout l’air d’une fuite. César, aussitôt informé par ses observateurs de ce qui se passait, craignit un piège, parce qu’il ne savait pas encore la raison de leur retraite, et il retint au camp ses troupes, y compris la cavalerie. Au petit jour, apprenant par ses éclaireurs qu’il s’agissait bien d’une retraite, il envoya en avant toute sa cavalerie pour retarder l’arrière-garde ; il lui donna pour chefs les légats Quintus Pédius et Lucius Aurunculéius Cotta. Le légat Titus Labiénus reçut l’ordre de suivre avec trois légions. Ces troupes attaquèrent les derniers corps et, les poursuivant sur plusieurs milles, tuèrent un grand nombre de fuyards : l’arrière-garde, qu’on atteignit d’abord, fit face et soutint vaillamment le choc de nos soldats ; mais ceux qui étaient en avant pensaient être hors de danger et n’étaient retenus ni par la nécessité, ni par l’autorité des chefs : quand ils entendirent les clameurs de la bataille, le désordre se mit dans leurs rangs, et tous ne pensèrent plus à d’autre moyen de salut que la fuite. C'est ainsi que, sans courir de danger, nos soldats en massacrèrent autant que la durée du jour le leur permit ; au coucher du soleil, ils abandonnèrent la poursuite et revinrent au camp comme ils en avaient reçu l’ordre.
\subsection[{§ 12.}]{ \textsc{§ 12.} }
\noindent Le lendemain César, sans laisser à l’ennemi le temps de se ressaisir après cette panique, conduisit son armée dans le pays des Suessions, qui étaient voisins des Rèmes, et à marche forcée parvint à Noviodunum, leur capitales. Il voulut enlever la place d’emblée, parce qu’on lui disait qu’elle était sans défenseurs ; mais, bien que ceux-ci fussent effectivement peu nombreux, la largeur du fossé et la hauteur des murs firent échouer son assaut. Ayant établi un camp fortifié, il fit avancer des mantelets et commença les préparatifs ordinaires d’un siège. Cependant toute la multitude des Suessions en déroute se jeta la nuit suivante dans la place. On avait vivement poussé les mantelets, élevé le terrassement, construit les tours frappés par la grandeur de ces ouvrages, chose qu’ils n’avaient jamais vue, dont ils n’avaient même jamais ouï parler, et par la rapidité de l’exécution, les Gaulois envoient à César des députés pour se rendre ; à la prière des Rèmes, il leur fait grâce.
\subsection[{§ 13.}]{ \textsc{§ 13.} }
\noindent César reçut la soumission des Suessions, qui donnèrent comme otages les premiers personnages de la cité et deux fils du roi Galba lui-même, et livrèrent toutes les armes que leur ville renfermait puis il marcha sur les Bellovaques. Ceux-ci s’étaient rassemblés, emportant avec eux tout ce qu’ils possédaient, dans la ville de Bratuspantium ; César et son armée n’étaient plus qu’à cinq mille pas environ de cette place, quand tous les anciens sortirent de la ville et, tendant les mains vers César, puis usant de la parole, firent connaître qu’ils se remettaient à sa discrétion et n’entreprenaient pas de lutter contre Rome. César avança sous les murs de la ville et y campa et cette fois les enfants et les femmes, du haut des murs, bras écartés et mains ouvertes suivant leur geste habituel de supplication, demandèrent la paix aux Romains.
\subsection[{§ 14.}]{ \textsc{§ 14.} }
\noindent Diviciacos intervint en leur faveur (après la dissolution de l’armée belge, il avait renvoyé les troupes héduennes et était revenu auprès de César) : « Les Bellovaques, dit-il, ont été de tout temps les alliés et les amis des Héduens ; c’est sous l’impulsion de leurs chefs, qui leur représentaient les Héduens comme réduits par César en esclavage et supportant de sa part toutes sortes de traitements indignes et d’humiliations, qu’ils se sont détachés des Héduens et ont déclaré la guerre à Rome. Ceux qui étaient responsables de cette décision, comprenant l’étendue du mal qu’ils avaient fait à leur patrie, se sont réfugiés en Bretagne. Aux prières des Bellovaques, les Héduens joignent les leurs : « Qu'il les traite avec la clémence et la bonté qui sont dans sa nature. S'il agit ainsi, il augmentera le crédit des Héduens auprès de tous les peuples belges, dont les troupes et l’argent leur donnent régulièrement, en cas de guerre, le moyen d’y faire face. »
\subsection[{§ 15.}]{ \textsc{§ 15.} }
\noindent César répondit que, en considération de Diviciacos et des Héduens, il accepterait la soumission des Bellovaques et les épargnerait ; comme leur cité jouissait d’une grande influence parmi les cités belges et était la plus peuplée, il demanda six cents otages. Quand on les lui eut livrés, et qu’on lui eut remis toutes les armes de la place, il marcha vers le pays des Ambiens ceux-ci, à son arrivée, se hâtèrent de faire soumission complète. Ils avaient pour voisins les Nerviens. L'enquête que fit César sur le caractère et les mœurs de ce peuple lui fournit les renseignements suivants : les marchands n’avaient aucun accès auprès d’eux ; ils ne souffraient pas qu’on introduisît chez eux du vin ou quelque autre produit de luxe, estimant que cela amollissait leurs âmes et détendait les ressorts de leur courage ; c’étaient des hommes rudes et d’une grande valeur guerrière ; ils accablaient les autres Belges de sanglants reproches pour s’être soumis à Rome et avoir fait litière de la vertu de leurs ancêtres ; ils assuraient que, quant à eux, ils n’enverraient pas de députés et n’accepteraient aucune proposition de paix.
\subsection[{§ 16.}]{ \textsc{§ 16.} }
\noindent César, après trois jours de marche à travers leur pays, apprit en interrogeant les prisonniers que la Sambre n’était pas à plus de dix milles de son camp ; « tous les Nerviens avaient pris position de l’autre côté de cette rivière et ils y attendaient l’arrivée des Romains avec les Atrébates et les Viromandues, leurs voisins, car ils avaient persuadé ces deux peuples de tenter avec eux la chance de la guerre ; ils comptaient aussi sur l’armée des Atuatuques, et, en effet, elle était en route ; les femmes et ceux qui, en raison de leur âge, ne pouvaient être d’aucune utilité pour la bataille, on les avait entassés en un lieu que des marais rendaient inaccessible à une armée. »
\subsection[{§ 17.}]{ \textsc{§ 17.} }
\noindent Pourvu de ces renseignements, César envoie en avant des éclaireurs et des centurions chargés de choisir un terrain propre à l’établissement d’un camp. Un grand nombre de Belges soumis et d’autres Gaulois avaient suivi César et faisaient route avec lui ; certains d’entre eux, comme on le sut plus tard par les prisonniers, ayant étudié la façon dont avait été réglée pendant ces jours-là la marche de notre armée, allèrent de nuit trouver les Nerviens et leur expliquèrent que les légions étaient séparées l’une de l’autre par des convois très importants, et que c’était chose bien facile, quand la première légion serait arrivée sur l’emplacement du camp et que les autres seraient encore loin derrière elle, de l’attaquer avant que les soldats eussent mis sac à terre ; une fois cette légion mise en fuite, et le convoi pillé, les autres n’oseraient pas leur tenir tête. Une considération appuyait encore le conseil de leurs informateurs : les Nerviens, n’ayant qu’une cavalerie sans valeur (jusqu’à présent, en effet, ils ne s’y intéressent pas, mais toute leur force, ils la doivent à l’infanterie), avaient depuis longtemps recours, afin de mieux faire obstacle à la cavalerie de leurs voisins, dans le cas où ils viendraient faire des razzias chez eux, au procédé suivant : ils taillaient et courbaient de jeunes arbres ; ceux-ci poussaient en largeur de nombreuses branches ; des ronces et des buissons épineux croissaient dans les intervalles si bien que ces haies, semblables à des murs, leur offraient une protection que le regard même ne pouvait violer. Notre armée étant embarrassée dans sa marche par ces obstacles, les Nerviens pensèrent qu’ils ne devaient pas négliger le conseil qu’on leur donnait.
\subsection[{§ 18.}]{ \textsc{§ 18.} }
\noindent La configuration du terrain que les nôtres avaient choisi pour le camp était la suivante. Une colline toute en pente douce descendait vers la Sambre, cours d’eau mentionné plus haut ; en face, de l’autre côté de la rivière, naissait une pente semblable, dont le bas, sur deux cents pas environ, était découvert, tandis que la partie supérieure de la colline était garnie de bois assez épais pour que le regard y pût difficilement pénétrer. C'est dans ces bois que l’ennemi se tenait caché ; sur le terrain découvert, le long de la rivière, on ne voyait que quelques postes de cavaliers. La profondeur de l’eau était d’environ trois pieds.
\subsection[{§ 19.}]{ \textsc{§ 19.} }
\noindent César, précédé de sa cavalerie, la suivait à peu de distance avec toutes ses troupes. Mais il avait réglé sa marche autrement que les Belges ne l’avaient dit aux Nerviens car, à l’approche de l’ennemi, il avait pris les dispositions qui lui étaient habituelles : six légions avançaient sans bagages, puis venaient les convois de toute l’armée, enfin deux légions, celles qui avaient été levées le plus récemment, fermaient la marche et protégeaient les convois. Notre cavalerie passa la rivière, en même temps que les frondeurs et les archers, et engagea le combat avec les cavaliers ennemis. Ceux-ci, tour à tour, se retiraient dans la forêt auprès des leurs et, tour à tour, reparaissant, chargeaient les nôtres ; et les nôtres n’osaient pas les poursuivre au-delà de la limite où finissait le terrain découvert. Pendant ce temps, les six légions qui étaient arrivées les premières, ayant tracé le camp, entreprirent de le fortifier. Dès que la tête de nos convois fut aperçue par ceux qui se tenaient cachés dans la forêt – c’était le moment dont ils étaient convenus pour engager le combat –, comme ils avaient formé leur front et disposé leurs unités à l’intérieur de la forêt, augmentant ainsi leur assurance par la solidité de leur formation, ils s’élancèrent soudain tous ensemble et se précipitèrent sur nos cavaliers. Ils n’eurent pas de peine à les défaire et à les disperser ; puis, avec une rapidité incroyable, ils descendirent au pas de course vers la rivière, si bien que presque en même temps ils semblaient se trouver devant la forêt, dans la rivière, et déjà aux prises avec nous. Avec la même rapidité, ils gravirent la colline opposée, marchant sur notre camp et sur ceux qui étaient en train d’y travailler.
\subsection[{§ 20.}]{ \textsc{§ 20.} }
\noindent César avait tout à faire à la fois : il fallait faire arborer l’étendard, qui était le signal de l’alerte, faire sonner la trompette, rappeler les soldats du travail, envoyer chercher ceux qui s’étaient avancés à une certaine distance pour chercher de quoi construire le remblai, ranger les troupes en bataille, les haranguer, donner le signal de l’attaque. Le peu de temps, et l’ennemi qui approchait, rendaient impossible une grande partie de ces mesures. Dans cette situation critique, deux choses aidaient César : d’une part l’instruction et l’entraînement des soldats, qui, exercés par les combats précédents, pouvaient aussi bien se dicter à eux-mêmes la conduite à suivre que l’apprendre d’autrui ; d’autre part, l’ordre qu’il avait donné aux légats de ne pas quitter le travail et de rester chacun avec sa légion, tant que le camp ne serait pas achevé. En raison de la proximité de l’ennemi et de la rapidité de son mouvement, ils n’attendaient pas, cette fois, les ordres de César mais prenaient d’eux-mêmes les dispositions qu’ils jugeaient bonnes.
\subsection[{§ 21.}]{ \textsc{§ 21.} }
\noindent César se borna à donner les ordres indispensables et courut haranguer les troupes du côté que le hasard lui offrit il tomba sur la dixième légion. Il fut bref, recommandant seulement aux soldats de se souvenir de leur antique valeur, de ne pas se laisser troubler et de tenir ferme devant l’assaut ; puis, l’ennemi étant à portée de javelot, il donna le signal du combat. Il partit alors vers l’autre aile pour y exhorter aussi les soldats ; il les trouva déjà combattant. On fut tellement pris de court, et l’ardeur offensive des ennemis fut telle, que le temps manqua non seulement pour arborer les insignes, mais même pour mettre les casques et pour enlever les housses des boucliers. Chacun, au hasard de la place où il se trouvait en quittant les travaux du camp, rejoignit les premières enseignes qu’il aperçut, afin de ne pas perdre à la recherche de son unité le temps qu’il devait au combat.
\subsection[{§ 22.}]{ \textsc{§ 22.} }
\noindent Comme les troupes s’étaient rangées selon la nature du terrain et la pente de la colline, en obéissant aux circonstances plutôt qu’aux règles de la tactique et des formations usuelles, comme les légions, sans liaison entre elles, luttaient chacune séparément et que des haies très épaisses, ainsi qu’on l’a dit plus haut, barraient la vue, on n’avait pas de données précises pour l’emploi des réserves, on ne pouvait pourvoir aux besoins de chaque partie du front, et l’unité de commandement était impossible. Aussi bien, les chances étaient-elles trop inégales pour que la fortune des armes ne fût pas aussi très diverse.
\subsection[{§ 23.}]{ \textsc{§ 23.} }
\noindent La 9\textsuperscript{e} et la 10\textsuperscript{e} légion, qui se trouvaient à l’aile gaucher, lancèrent le javelot ; harassés par la course et tout hors d’haleine, et, pour finir, blessés par nos traits, les Atrébates (car c’étaient eux qui occupaient ce côté de la ligne ennemie), furent rapidement refoulés de la hauteur vers la rivière, et tandis qu’ils tâchaient de la franchir, les nôtres, les poursuivant à l’épée, en tuèrent un grand nombre. Puis ils n’hésitèrent pas à passer eux-mêmes la rivière, et, progressant sur un terrain qui ne leur était pas favorable, brisant la résistance des ennemis qui s’étaient reformés, ils les mirent en déroute après un nouveau combat. Sur une autre partie du front, deux légions, la 11\textsuperscript{e} et la 8\textsuperscript{e} agissant séparément, avaient défait les Viromandues, qui leur étaient opposés, leur avaient fait dévaler la pente et se battaient sur les bords mêmes de la rivière. Mais le camp presque entier, sur la gauche et au centre, se trouvant ainsi découvert, – à l’aile droite avaient pris position la 12\textsuperscript{e} légion et, non loin d’elle, la 7\textsuperscript{e} – tous les Nerviens, en rangs très serrés, sous la conduite de Boduognatos, leur chef suprême, marchèrent sur ce point ; et tandis que les uns entreprenaient de tourner les légions par leur droite, les autres se portaient vers le sommet du camp.
\subsection[{§ 24.}]{ \textsc{§ 24.} }
\noindent Dans le même moment, nos cavaliers et les soldats d’infanterie légère qui les avaient accompagnés, mis en déroute, comme je l’ai dit, au début de l’attaque ennemie, rentraient au camp pour s’y réfugier et se trouvaient face à face avec les Nerviens : ils se remirent à fuir dans une autre direction ; et les valets qui, de la porte décumane, sur le sommet de la colline, avaient vu les nôtres passer, victorieux, la rivière, et étaient sortis pour faire du butin, quand ils virent, en se retournant, que les ennemis étaient dans le camp romain, se mirent à fuir tête baissée. En même temps s’élevaient des clameurs et un grand bruit confus : c’étaient ceux qui arrivaient avec les bagages, et qui, pris de panique, se portaient au hasard dans toutes les directions. Tout cela émut fortement les cavaliers trévires, qui ont parmi les peuples de la Gaule une particulière réputation de courage, et que leur cité avait envoyés à César comme auxiliaires : voyant qu’une foule d’ennemis emplissait le camp, que les légions étaient serrées de près et presque enveloppées, que valets, cavaliers, frondeurs, Numides fuyaient de toutes parts à la débandade, ils crurent notre situation sans espoir et prirent le chemin de leur pays ; ils y apportèrent la nouvelle que les Romains avaient été défaits et vaincus, que l’ennemi s’était emparé de leur camp et de leurs bagages.
\subsection[{§ 25.}]{ \textsc{§ 25.} }
\noindent César, après avoir harangué la 10\textsuperscript{e} légion, était parti vers l’aile droite : les nôtres y étaient vivement pressés ; les soldats de la 12\textsuperscript{e} légion, ayant rassemblé leurs enseignes en un même point, étaient serrés les uns entre les autres et se gênaient mutuellement pour combattre ; la 4\textsuperscript{e} cohorte avait eu tous ses centurions et un porte-enseigne tués, elle avait perdu une enseigne ; dans les autres cohortes, presque tous les centurions étaient blessés ou tués, et parmi eux le primipile Publius Sextius Baculus, centurion particulièrement courageux qui, épuisé par de nombreuses et graves blessures, ne pouvait plus se tenir debout ; le reste faiblissait, et aux derniers rangs un certain nombre, se sentant abandonnés, quittaient le combat et cherchaient à se soustraire aux coups ; les ennemis montaient en face de nous sans relâche, tandis que leur pression augmentait sur les deux flancs ; la situation était critique. Ce que voyant, et comme il ne disposait d’aucun renfort, César prit à un soldat des derniers rangs son bouclier – car il ne s’était pas muni du sien – et s’avança en première ligne : là, il parla aux centurions en appelant chacun d’eux par son nom et harangua le reste de la troupe ; il donna l’ordre de porter les enseignes en avant et de desserrer les rangs afin de pouvoir plus aisément se servir de l’épée. Son arrivée ayant donné de l’espoir aux troupes et leur ayant rendu courage, car chacun, en présence du général, désirait, même si le péril était extrême, faire de son mieux, on réussit à ralentir un peu l’élan de l’ennemi.
\subsection[{§ 26.}]{ \textsc{§ 26.} }
\noindent César, voyant que la 7\textsuperscript{e} légion, qui était à côté de la 12\textsuperscript{e}, était également pressée par l’ennemi, fit savoir aux tribuns militaires que les deux légions devaient peu à peu se souder et faire face aux ennemis en s’épaulant l’une l’autre. Par cette manœuvre, les soldats se prêtaient un mutuel secours et ne craignaient plus d’être pris à revers ; la résistance en fut encouragée et devint plus vive. Cependant, les soldats des deux légions qui, à la queue de la colonne, formaient la garde des convois, ayant su qu’on se battait, avaient pris le pas de course et apparaissaient au sommet de la colline ; d’autre part, Titus Labiénus, qui s’était emparé du camp ennemi et avait vu, de cette hauteur, ce qui se passait dans le nôtre, envoya la 10\textsuperscript{e} légion à notre secours. La fuite des cavaliers et des valets ayant appris à ces soldats quelle était la situation, et quel danger couraient le camp, les légions, le général, ils ne négligèrent rien pour aller vite.
\subsection[{§ 27.}]{ \textsc{§ 27.} }
\noindent L'arrivée des trois légions produisit un tel changement dans la situation que ceux mêmes qui, épuisés par leurs blessures, gisaient sur le sol, recommencèrent à se battre en s’appuyant sur leurs boucliers, que les valets, voyant l’ennemi terrifié, se jetèrent sur lui, même sans armes, que les cavaliers enfin, pour effacer le souvenir de leur fuite honteuse, cherchèrent sur tous les points du champ de bataille à surpasser les légionnaires. Mais l’ennemi, même alors qu’il ne lui restait guère d’espoir, montra un tel courage que, quand les premiers étaient tombés, ceux qui les suivaient montaient sur leurs corps pour se battre, et quand ils tombaient à leur tour et que s’entassaient les cadavres, les survivants, comme du haut d’un tertre, lançaient des traits sur nos soldats et renvoyaient les javelots qui manquaient leur but : ainsi, ce n’était pas une folle entreprise, pour ces hommes d’un pareil courage, il fallait le reconnaître, que d’avoir osé franchir une rivière très large, escalader une berge fort élevée, monter à l’assaut d’une position très forte cette tâche, leur héroïsme l’avait rendue faciles.
\subsection[{§ 28.}]{ \textsc{§ 28.} }
\noindent Cette bataille avait presque réduit à néant la nation et le nom des Nerviens ; aussi, quand ils en apprirent la nouvelle, les vieillards qui, nous l’avons dit, avaient été rassemblés avec les enfants et les femmes dans une région de lagunes et d’étangs, jugeant que rien ne pouvait arrêter les vainqueurs ni rien protéger les vaincus, envoyèrent, avec le consentement unanime des survivants, des députés à César : ils firent soumission complète, et, soulignant l’infortune de leur peuple, déclarèrent que de six cents sénateurs ils étaient réduits à trois, de soixante mille hommes en état de porter les armes, à cinq cents à peine. César, soucieux de montrer qu’il était pitoyable aux malheureux et aux suppliants, prit grand soin de les ménager : il leur laissa la jouissance de leurs terres et de leurs villes, et ordonna à leurs voisins de s’interdire et d’interdire à leurs clients toute injustice et tout dommage à leur égard.
\subsection[{§ 29.}]{ \textsc{§ 29.} }
\noindent Les Atuatuques, dont il a été question plus haut, arrivaient au secours des Nerviens avec toutes leurs forces : à la nouvelle du combat, ils firent demi-tour et rentrèrent chez eux ; abandonnant toutes leurs villes et tout leurs villages fortifiés, ils réunirent tout leurs biens dans une seule place que sa situation rendait très forte. De toutes parts autour d’elle c’étaient de très hautes falaises d’où la vue plongeait, sauf sur un point qui laissait un passage en pente douce ne dépassant pas deux cents pieds de large : un double mur fort élevé défendait cette entrée, et ils le couronnèrent alors de pierres d’un grand poids et de poutres taillées en pointu. Ce peuple descendait des Cimbres et des Teutons, qui, tandis qu’ils marchaient vers notre province et vers l’Italie, avaient laissé sur la rive gauche du Rhin les bêtes et les bagages qu’ils ne pouvaient emmener, avec six mille hommes des leurs pour les garder. Ceux-ci, après la destruction de leur peuple, avaient été en lutte constante avec leurs voisins, tantôt les attaquant, tantôt repoussant leurs attaques ; enfin on avait fait la paix, et, avec le consentement de tous, ils avaient choisi cette région pour s’y installer.
\subsection[{§ 30.}]{ \textsc{§ 30.} }
\noindent Dans les premiers temps qui suivirent notre arrivée, ils faisaient de fréquentes sorties et engageaient avec nous de petits combats ; puis, quand nous les eûmes cernés d’un retranchement qui avait quinze mille pieds de tour et que complétaient de nombreuses redoutes, ils restèrent dans la place. Lorsqu’ils virent qu’après avoir poussé les mantelets et élevé un terrassement nous construisions au loin une tour, ils commencèrent par railler du haut de leur rempart et par nous couvrir de sarcasmes : « Un si grand appareil à une telle distance ! Quels bras, quels muscles avaient-ils donc, surtout avec leur taille infime (car aux yeux de tous les Gaulois, en général, notre petite taille à côté de leur haute stature est un objet de mépris) pour prétendre placer sur le mur une tour de ce poids ? »
\subsection[{§ 31.}]{ \textsc{§ 31.} }
\noindent Mais quand ils virent qu’elle se mouvait et approchait des murs, vivement frappés de ce spectacle nouveau et étrange pour eux, ils envoyèrent à César des députés, qui lui tinrent à peu près ce langage : « Ils ne pouvaient pas croire que les Romains ne fussent pas aidés par les dieux dans la conduite de la guerre, puisqu’ils étaient capables de faire avancer si vite des machines d’une telle hauteur » ; et ils déclarèrent qu’ils leur livraient leurs personnes et tout leurs biens. « Ils ne formulaient qu’une demande, une prière si César, dont ils entendaient vanter la clémence et la bonté, décidait de ne pas anéantir les Atuatuques, qu’il ne les privât pas de leurs armes. Presque tout leurs voisins les détestaient, étaient jaloux de leur valeur ; s’ils livraient leurs armes, ils seraient sans défense devant eux. Mieux valait, s’ils en étaient réduits là, voir les Romains leur infliger n’importe quel sort, que périr dans les tourments de la main de ces hommes, parmi lesquels ils avaient toujours régné en maîtres. »
\subsection[{§ 32.}]{ \textsc{§ 32.} }
\noindent César répondit que « ses habitudes de clémence, plutôt que leur conduite, l’engageaient à conserver leur nation, s’ils se rendaient avant que le béliers eût touché leur mur, mais il n’y avait de capitulation possible que si les armes étaient livrées. Il agirait comme il avait fait pour les Nerviens, il interdirait à leurs voisins de faire le moindre tort à un peuple soumis à Rome ». Les députés rapportèrent à leur peuple ces conditions, et vinrent dire qu’ils s’y soumettaient. Une grande quantité d’armes fut jetée du haut du mur dans le fossé qui était devant la place : elles s’élevaient en monceaux presque jusqu’au sommet du rempart et de notre terrassement ; et cependant, comme on le vit par la suite, les assiégés en avaient dissimulé environ un tiers, qu’ils avaient gardé dans la place. Ils ouvrirent leurs portes, et ce jour-là se passa dans le calme.
\subsection[{§ 33.}]{ \textsc{§ 33.} }
\noindent Quand vint le soir, César ordonna que les portes fussent fermées et que les soldats sortissent de la ville, pour éviter que pendant la nuit ils ne commissent contre les habitants quelque violence. Ceux-ci, qui – on le vit bien – s’étaient concertés au préalable, parce qu’ils avaient cru qu’une fois leur soumission faite, nous retirerions nos postes ou tout au moins relâcherions notre surveillance, se servant, d’une part, des armes qu’ils avaient retenues et cachées, d’autre part de boucliers qu’ils avaient fabriqués avec de l’écorce ou en tressant de l’osier et qu’ils avaient sur-le-champ, vu l’urgence, revêtus de peaux, firent à la troisième veille, du côté où la montée vers nos retranchements était le moins rude, une sortie soudaine et en masse. Promptement, selon les ordres que César avait donnés d’avance, des feux furent allumés comme signal et on accourut des postes voisins sur le point menacé ; les ennemis se battirent avec l’acharnement que devaient montrer des guerriers valeureux qui jouaient leur dernière chance de salut et qui avaient le désavantage de la position contre un adversaire lançant ses traits du haut d’un retranchement et de tours, dans des conditions enfin où ils ne pouvaient rien attendre que de leur courage. Après qu’on en eut tué environ quatre mille, ce qui restait fut rejeté dans la place. Le lendemain nous enfonçâmes les portes que ne défendait plus personne ; nos soldats pénétrèrent dans la ville, et César fit tout vendre à l’encan en un seul lot. Il sut par les acheteurs que le nombre des têtes était de 53 § 000.
\subsection[{§ 34.}]{ \textsc{§ 34.} }
\noindent A la même époque, Publius Crassus, que César avait envoyé avec une légion chez les Vénètes, les Unelles, les Osismes, les Coriosolites, les Esuvii, les Aulerques, les Redons, peuples marins riverains de l’Océan, lui fit savoir que tous ces peuples avaient été soumis à Rome.
\subsection[{§ 35.}]{ \textsc{§ 35.} }
\noindent Ces campagnes ayant procuré la pacification de toute la Gaule, la renommée qui en parvint aux Barbares fut telle que César reçut des nations habitant au-delà du Rhin des députés qui venaient promettre la livraison d’otages et l’obéissance. Comme il était pressé de partir pour l’Italie et l’Illyricum, César leur dit de revenir au début de l’été suivant. Il amena ses légions prendre leurs quartiers d’hiver chez les Carnutes, les Andes, les Turons et les peuples voisins des régions où il avait fait la guerre, et partit pour l’Italie. En raison de ces événements on décréta, à la suite du rapport de César, quinze jours de supplication, ce qui n’était encore arrivé à personne.
 \section[{Livre III}]{Livre III}\renewcommand{\leftmark}{Livre III}

\subsection[{§ 1.}]{ \textsc{§ 1.} }
\noindent En partant pour l’Italie, César envoya Servius Galba avec la 12\textsuperscript{e} légion et une partie de la cavalerie chez les Nantuates, les Véragres et les Sédunes, dont le territoire s’étend depuis les frontières des Allobroges, le lac Léman et le Rhône jusqu’aux grandes Alpes. Ce qui l’y détermina, ce fut le désir d’ouvrir au commerce la route des Alpes, où les marchands ne circulaient jusque-là qu’au prix de grands dangers et en payant de forts péages. Il autorisa Galba, s’il le jugeait nécessaire, à installer la légion dans ces parages pour y passer l’hiver. Celui-ci, après avoir livré divers combats heureux et pris un grand nombre de forteresses, reçut de toutes parts des députations, des otages, fit la paix, et résolut d’installer deux cohortes chez les Nantuates et de s’établir lui-même pour l’hiver, avec les autres cohortes de sa légion, dans un bourg des Véragres qui s’appelle Octoduros ; ce bourg, situé au fond d’une vallée étroite, est enfermé de tous côtés par de très hautes montagnes. Comme la rivière le coupait en deux, Galba autorisa les indigènes à s’installer pour l’hiver dans une moitié du bourg, tandis que l’autre, qu’il avait fait évacuer, était donnée à ses cohortes. Il la fortifia d’un retranchement et d’un fossé.
\subsection[{§ 2.}]{ \textsc{§ 2.} }
\noindent Il y avait fort longtemps qu’il hivernait là, et il venait de donner l’ordre qu’on y fît des provisions de blé, quand soudain ses éclaireurs lui apprirent que la partie du bourg laissée aux Gaulois avait été complètement abandonnée pendant la nuit et qu’une immense multitude de Sédunes et de Véragres occupait les montagnes environnantes. Plusieurs raisons avaient provoqué cette décision soudaine des Gaulois de recommencer la guerre et de tomber à l’improviste sur notre légion : d’abord cette légion, et qui n’était pas au complet, car on en avait distrait deux cohortes et un très grand nombre d’isolés qu’on avait envoyés chercher des vivres, leur semblait une poignée d’hommes méprisable ; puis l’avantage de leur position leur faisait croire que, quand ils dévaleraient les pentes de leurs montagnes et lanceraient une grêle de traits, cette attaque serait, dès le premier choc, irrésistible. A ces calculs s’ajoutait le ressentiment de s’être vu arracher leurs enfants à titre d’otages, et la conviction que les Romains cherchaient à occuper les sommets des Alpes, non seulement pour être maîtres des routes, mais pour s’y établir définitivement et annexer ces régions à leur province, qu’elles bordent.
\subsection[{§ 3.}]{ \textsc{§ 3.} }
\noindent A ces nouvelles, Galba, qui n’avait pas entièrement achevé le camp d’hiver et ses défenses, et n’avait pas fait encore une réserve suffisante de blé et autres approvisionnements, parce qu’il avait cru, les Gaulois s’étant soumis et lui ayant donné des otages, qu’aucun acte d’hostilité n’était à craindre, s’empressa d’assembler un conseil et recueillit les avis. Dans ce conseil, en face d’un si grand péril, et si inattendu, voyant presque toutes les hauteurs garnies d’une foule d’hommes en armes, ne pouvant espérer de secours ni de ravitaillement, puisque les chemins étaient coupés, désespérant presque déjà de leur salut, plusieurs formulaient l’avis d’abandonner les bagages et de chercher à échapper à la mort en faisant une sortie par les mêmes chemins qui les avaient conduits là. Cependant, le sentiment de la majorité fut qu’il fallait réserver ce parti comme un parti extrême et, en attendant, voir quelle tournure prendraient les choses et défendre le camp.
\subsection[{§ 4.}]{ \textsc{§ 4.} }
\noindent Peu après – on avait à peine eu le temps de mettre à exécution les mesures décidées –, les ennemis, de toutes parts, à un signal donné, descendent à la course et jettent contre le retranchement des pierres et des javelots. Les nôtres, au début, ayant toute leur force, résistèrent avec courage, et, comme ils dominaient l’assaillant, tout leurs traits portaient ; chaque fois qu’un point du camp, dégarni de défenseurs, paraissait menacé, on accourait à la rescousse ; mais ce qui faisait leur infériorité, c’est que, la lutte se prolongeant, les ennemis, s’ils étaient fatigués, quittaient le combat et étaient remplacés par des troupes fraîches ; les nôtres, en raison de leur petit nombre, ne pouvaient rien faire de semblable ; il était impossible, non seulement que le combattant épuisé se retirât de l’action, mais que le blessé même quittât son poste pour se ressaisir.
\subsection[{§ 5.}]{ \textsc{§ 5.} }
\noindent Il y avait déjà plus de six heures que l’on combattait sans relâche ; les nôtres étaient à bout de forces, et les munitions aussi leur manquaient ; l’ennemi redoublait ses coups et, notre résistance faiblissant, il entamait la palissade et comblait les fossés ; la situation était extrêmement grave. C'est alors que Publius Sextius Baculus, centurion primipile, qui avait été, comme on l’a vu, couvert de blessures lors du combat contre les Nerviens, et avec lui Caïus Volusénus, tribun militaire, homme plein de sens et de courage, viennent en courant trouver Galba et lui représentent qu’il n’y a qu’un espoir de salut faire une sortie, tenter cette chance suprême. Il convoque donc les centurions et par eux fait rapidement savoir aux soldats qu’ils aient à suspendre quelques instants le combat, en se contentant de se protéger des projectiles qu’on leur enverrait, et à refaire leurs forces ; puis, au signal donné, ils feront irruption hors du camp, et n’attendront plus leur salut que de leur valeur.
\subsection[{§ 6.}]{ \textsc{§ 6.} }
\noindent Ils exécutent les ordres reçus, et, sortant soudain par toutes les portes, ils surprennent l’ennemi qui ne peut ni se rendre compte de ce qui se passe ni se reformer. Ainsi le combat change de face, et ceux qui déjà se flattaient de prendre le camp sont enveloppés et massacrés sur plus de trente mille hommes qu’on savait s’être portés à l’attaque, plus du tiers est tué, les autres, effrayés, sont mis en fuite, et on ne les laisse même pas s’arrêter sur les hauteurs. Ayant ainsi mis en déroute et désarmés les forces ennemies, nos soldats rentrent dans leur camp, à l’abri de leurs retranchements. Après ce combat, ne voulant pas tenter de nouveau la fortune, considérant d’ailleurs que ce n’était pas pour cela qu’il était venu prendre ses quartiers d’hiver et qu’il se trouvait en face de circonstances imprévues, mais surtout fort inquiet à la pensée de manquer de vivres, Galba fit incendier dès le lendemain toutes les maisons du bourg et reprit la route de la Province ; sans qu’aucun ennemi arrêtât ou retardât sa marche, il conduisit sa légion sans pertes chez les Nantuates, et de là chez les Allobroges, où il hiverna.
\subsection[{§ 7.}]{ \textsc{§ 7.} }
\noindent Après ces événements, César avait tout lieu de penser que la Gaule était pacifiée : les Belges avaient été battus, les Germains chassés, les Sédunes vaincus dans les Alpes ; il était, dans ces conditions, parti après le commencement de l’hiver pour l’Illyricum, dont il voulait aussi visiter les peuples et connaître le territoire, soudain, la guerre éclata en Gaule. La cause en fut la suivante. Le jeune Publius Crassus, avec la 7\textsuperscript{e} légion, avait établi ses quartiers d’hiver chez les Andes : c’était lui qui était le plus près de l’Océan. Le blé manquant dans cette région, il envoya un bon nombre de préfets et de tribuns militaires chez les peuples voisins peur y chercher du blé entre autres, Titus Terrasidius fut envoyé chez les Esuvii, Marcus Trébius Galius chez les Coriosolites, Quintus Vélanius avec Titus Sillius chez les Vénètes.
\subsection[{§ 8.}]{ \textsc{§ 8.} }
\noindent Ce peuple est de beaucoup le plus puissant de toute cette côte maritime : c’est lui qui possède le plus grand nombre de navires, flotte qui fait le trafic avec la Bretagne ; il est supérieur aux autres par sa science et son expérience de la navigation ; enfin, comme la mer est violente et bat librement une côte où il n’y a que quelques ports, dont ils sont les maîtres, presque tous ceux qui naviguent habituellement dans ces eaux sont leurs tributaires. Les premiers, ils retiennent Sillius et Vélanius, pensant se servir d’eux pour recouvrer les otages qu’ils avaient donnés à Crassus.\par
\par
Leur exemple entraîne les peuples voisins – car les décisions des Gaulois sont soudaines et impulsives et, obéissant au même mobile, ils retiennent Trébius et Terrasidius ; on envoie promptement des ambassades, les chefs se concertent, on jure de ne rien faire que d’un commun accord et de courir tous la même chance ; ils pressent les autres cités de garder l’indépendance que les ancêtres leur ont transmise plutôt que de subir le joug des Romains. Toute la côte est promptement gagnée à leur avis, et une ambassade commune est envoyée à Publius Crassus pour l’inviter à rendre les otages s’il veut qu’on lui rende les officiers.
\subsection[{§ 9.}]{ \textsc{§ 9.} }
\noindent César, mis au courant par Crassus, ordonne qu’en l’attendant – car il était loin – on construise des navires de guerre sur la Loire, fleuve qui se jette dans l’Océan, qu’on lève des rameurs dans la province et qu’on se procure des matelots et des pilotes. On y pourvoit avec promptitude, et lui-même, dès que la saison le lui permit, se rend à l’armée. Les Vénètes, ainsi que les autres peuples, quand ils apprennent l’arrivée de César, comme d’ailleurs ils se rendaient compte de la gravité de leur crime, – n’avaient-ils pas retenu et chargé de fers des ambassadeurs, titre que toutes les nations ont toujours regardé comme sacré et inviolable ? – font des préparatifs de guerre proportionnés à un si grand péril, et pourvoient principalement à l’équipement de leurs navires ; leurs espoirs étaient d’autant plus forts que la nature du pays leur inspirait beaucoup de confiance. Ils savaient que les chemins de terre étaient coupés à marée haute par des baies, que l’ignorance des lieux et le petit nombre des ports nous rendaient la navigation difficile, et ils pensaient que nos armées, à cause du manque de blé, ne pourraient pas demeurer longtemps chez eux ; à supposer d’ailleurs que tout trompât leur attente, ils n’ignoraient pas la supériorité de leur marine, ils se rendaient compte que les Romains manquaient de vaisseaux, que dans le pays où ils devaient faire la guerre rades, ports, îles leur étaient inconnus, enfin que c’était tout autre chose de naviguer sur une mer fermée ou sur l’Océan immense et sans limites. Leurs résolutions prises, ils fortifient les villes, y entassent les moissons, assemblent en Vénétie, où chacun pensait que César ouvrirait les hostilités, une flotte aussi nombreuse que possible. Ils s’assurent pour cette guerre l’alliance des Osismes, des Lexovii, des Namnètes, des Ambiliates, des Morins, des Diablintes, des Ménapes ; ils demandent du secours à la Bretagne, qui est située en face de ces contrées.
\subsection[{§ 10.}]{ \textsc{§ 10.} }
\noindent On vient de voir quelles étaient les difficultés de cette guerre ; et cependant plusieurs raisons poussaient César à l’entreprendre : des chevaliers romains retenus au mépris du droit, une révolte après soumission, la trahison quand on avait livré des otages, tant de cités coalisées, et surtout la crainte que s’il négligeait de punir ces peuples les autres ne se crussent autorisés à agir comme eux. Aussi, sachant que les Gaulois en général aiment le changement et sont prompts à partir en guerre, que d’ailleurs tous les hommes ont naturellement au cœur l’amour de la liberté et la haine de la servitude, il pensa qu’il lui fallait, avant que la coalition se fît plus nombreuse, diviser son armée et la répartir sur une plus vaste étendue.
\subsection[{§ 11.}]{ \textsc{§ 11.} }
\noindent En conséquence, il envoie son légat Titus Labiénus avec de la cavalerie chez les Trévires, peuple voisin du Rhin. Il lui donne mission d’entrer en contact avec les Rèmes et les autres Belges et de les maintenir dans le devoir, de barrer la route aux Germains, que, disait-on, les Gaulois avaient appelés à leur aide, s’ils essaient de forcer avec leurs bateaux le passage du fleuve. Publius Crassus reçoit l’ordre de partir pour l’Aquitaine avec douze cohortes légionnaires et une importante cavalerie, afin d’empêcher que les peuples de ce pays n’envoient des secours aux Gaulois et que deux si grandes nations ne s’unissent. Le légat Quintus Titurius Sabinus est envoyé avec trois légions chez les Unelles, les Coriosolites et les Lexovii, avec charge de tenir leurs troupes en respect. Il donne au jeune Décimus Brutus le commandement de la flotte et des vaisseaux gaulois qu’il avait fait fournir par les Pictons et les Santons et par les autres régions pacifiées, avec l’ordre de partir le plus tôt possible chez les Vénètes. Lui-même se dirige de ce côté avec l’infanterie.
\subsection[{§ 12.}]{ \textsc{§ 12.} }
\noindent Les places de la région étaient en général situées à l’extrémité de langues de terre et de promontoires, en sorte qu’on n’y pouvait accéder à pied, quand la mer était haute – ce qui se produit régulièrement toutes les douze heures – et qu’elles n’étaient pas plus accessibles aux navires, car, à marée basse, ils se seraient échoués sur les bas-fonds. C'était là un double obstacle aux sièges. Et si jamais, grâce à d’énormes travaux, en contenant la mer par des terrassements et des digues et en élevant ces ouvrages à la hauteur des remparts, on amenait les assiégés à se croire perdus, ils poussaient au rivage une nombreuse flotte – ils avaient des navires en abondance –, y transportaient tout leurs biens et se retiraient dans les villes voisines là, ils retrouvaient les mêmes moyens naturels de défense. Cette manœuvre se renouvela une grande partie de l’été, d’autant plus aisément que nos vaisseaux étaient retenus par le mauvais temps et que sur cette mer vaste et ouverte, sujette à de hautes marées, où il y avait peu ou point de ports, la navigation était extrêmement difficile.
\subsection[{§ 13.}]{ \textsc{§ 13.} }
\noindent Les ennemis, eux, avaient des vaisseaux qui étaient construits et armés de la manière suivante. Leur carène était notablement plus plate que celle des nôtres, afin qu’ils eussent moins à craindre les bas-fonds et le reflux ; leurs proues étaient très relevées, et les poupes de même, appropriées à la hauteur des vagues et à la violence des tempêtes ; le navire entier était en bois de chêne, pour résister à tous les chocs et à tous les heurts ; les traverses avaient un pied d’épaisseur, et étaient assujetties par des chevilles de fer de la grosseur d’un pouce ; les ancres étaient retenues non par des cordes, mais par des chaînes de fer ; en guise de voiles, des peaux, des cuirs minces et souples, soit parce que le lin faisait défaut et qu’on n’en connaissait pas l’usage, soit, ce qui est plus vraisemblable, parce qu’on pensait que des voiles résisteraient mal aux tempêtes si violentes de l’Océan et à ses vents si impétueux, et seraient peu capables de faire naviguer des bateaux si lourds. Quand notre flotte se rencontrait avec de pareils vaisseaux, elle n’avait d’autre avantage que sa rapidité et l’élan des rames ; tout le reste était en faveur des navires ennemis, mieux adaptés à la nature de cette mer et à ses tempêtes. En effet, nos éperons ne pouvaient rien contre eux, tant ils étaient solides ; la hauteur de leur bord faisait que les projectiles n’y atteignaient pas aisément, et qu’il était difficile de les harponner. Ajoutez à cela qu’en filant sous le vent, lorsque celui-ci devenait violent, il leur était plus facile de supporter les tempêtes, qu’ils pouvaient mouiller sur des fonds bas sans craindre autant d’être mis à sec, enfin que, si le reflux les laissait, ils n’avaient rien à craindre des rochers et des écueils ; toutes choses qui constituaient pour nos vaisseaux un redoutable dangers.
\subsection[{§ 14.}]{ \textsc{§ 14.} }
\noindent Après s’être emparé de plusieurs places, César, voyant qu’il se donnait une peine inutile, que de prendre à l’ennemi ses villes, cela ne l’empêchait point de se dérober, et qu’il restait invulnérable, décida d’attendre sa flotte. Quand elle arriva, à peine l’ennemi l’eut-il aperçue qu’environ deux cent vingt navires tout prêts et équipés de façon parfaite sortirent d’un port et vinrent se ranger en face des nôtres. Ni Brutus, qui commandait la flotte, ni les tribuns militaires et les centurions, qui avaient chacun un vaisseau, n’étaient au clair sur la conduite à tenir, sur la méthode de combat à adopter. Ils se rendaient compte, en effet, que l’éperon était inefficace ; et si l’on élevait des tours, les vaisseaux ennemis les dominaient encore grâce à la hauteur de leurs poupes, en sorte que nos projectiles, tirés d’en bas, portaient mal, tandis que ceux des Gaulois tombaient au contraire avec plus de force. Un seul engin préparé par nous fut très utile : des faux très tranchantes emmanchées de longues perches, assez semblables aux faux de siège. Une fois qu’à l’aide de ces engins on avait accroché et tiré à soi les cordes qui attachaient les vergues au mât, on les coupait en faisant force de rames. Alors les vergues tombaient forcément, et les vaisseaux gaulois, qui ne pouvaient compter que sur les voiles et les agrès, s’en trouvant privés, étaient du même coup réduits à l’impuissance. Le reste du combat n’était plus qu’affaire de courage, et en cela nos soldats avaient aisément le dessus, d’autant plus que la bataille se déroulait sous les yeux de César et de l’armée tout entière, si bien qu’aucune action de quelque valeur ne pouvait rester inconnue : l’armée occupait, en effet, toutes les collines et toutes les hauteurs d’où l’on voyait de près la mer.
\subsection[{§ 15.}]{ \textsc{§ 15.} }
\noindent Une fois ses vergues abattues de la manière que nous avons dite, chaque navire était entouré de deux et parfois trois des nôtres, et nos soldats montaient de vive force à l’abordage. Quand les barbares virent ce qui se passait, comme déjà un grand nombre de leurs vaisseaux avaient été pris, et qu’ils ne trouvaient rien à opposer à cette tactique, ils cherchèrent leur salut dans la fuite. Déjà leurs navires prenaient le vent, quand soudain il tomba, et ce fut une telle bonace, un tel calme, que les vaisseaux ne pouvaient bouger. Cette circonstance nous fut des plus favorables pour compléter notre victoire car nous attaquâmes et prîmes les navires l’un après l’autre, et le nombre fut infime de ceux qui purent, grâce à la nuit, gagner le rivage, après un combat qui avait duré depuis la quatrième heure du jour environ jusqu’au coucher du soleil.
\subsection[{§ 16.}]{ \textsc{§ 16.} }
\noindent Cette bataille mit fin à la guerre des Vénètes et de tous les peuples de cette côte. Car, outre que tous les hommes jeunes étaient venus là, et même tous ceux qui, déjà âgés, étaient de bon conseil ou occupaient un certain rang, ils avaient rassemblé sur ce seul point tout ce qu’ils avaient de vaisseaux ; ces vaisseaux perdus, les survivants ne savaient où se réfugier ni comment défendre leurs villes. Aussi se rendirent-ils à César corps et biens. Celui-ci résolut de les châtier sévèrement pour qu’à l’avenir les barbares fussent plus attentifs à respecter le droit des ambassadeurs, En conséquence, il fit mettre à mort tous les sénateurs et vendit le reste à l’encan.
\subsection[{§ 17.}]{ \textsc{§ 17.} }
\noindent Tandis que ces événements se déroulaient chez les Vénètes, Quintus Titurius Sabinus arriva, avec les troupes que César lui avait confiées, chez les Unelles. Ceux-ci avaient à leur tête Viridovix ; il commandait aussi à toutes les cités révoltées, d’où il avait tiré une armée, et fort nombreuse ; peu de jours après l’arrivée de Sabinus, les Aulerques Eburovices et les Lexovii, ayant massacré leur sénat, qui était opposé à la guerre, fermèrent leurs portes et se joignirent à Viridovix ; en outre, une multitude considérable était venue de tous les coins de la Gaule, gens sans aveu et malfaiteurs que l’espoir du butin et l’amour de la guerre enlevaient à l’agriculture et aux travaux journaliers. Sabinus, établi dans un camp à tous égards bien situé, s’y cantonnait, tandis que Viridovix s’était posté en face de lui à deux milles de distance et chaque jour, faisant avancer ses troupes, offrait le combat : déjà l’ennemi commençait à mépriser Sabinus, et les propos de nos soldats mêmes ne l’épargnaient pas ; il donna si fort à croire qu’il avait peur, que l’ennemi poussait l’audace jusqu’à venir à notre parapet. Son attitude lui était dictée par la pensée qu’un légat ne devait pas, surtout en l’absence du général en chef, livrer bataille à une telle multitude, à moins d’avoir pour soi l’avantage du terrain ou quelque occasion favorable.
\subsection[{§ 18.}]{ \textsc{§ 18.} }
\noindent Une fois bien établie l’opinion qu’il avait peur, il choisit un homme capable et adroit, un Gaulois, qui faisait partie de ses auxiliaires. Il obtient de lui, par grands présents et promesses, qu’il passe à l’ennemi, et il lui explique ce qu’il désire. Celui-ci arrive en se donnant comme déserteur, dépeint la frayeur des Romains, dit dans quelle grave situation les Vénètes mettent César lui-même : pas plus tard que la nuit suivante, Sabinus lèvera le camp en secret pour aller le secourir. A cette nouvelle, tous s’écrient qu’on ne doit pas laisser perdre une si belle occasion il faut marcher sur le camp. Plusieurs motifs poussaient les Gaulois à cette détermination : l’hésitation de Sabinus pendant les jours précédents, les affirmations du déserteur, le manque de vivres, dont ils n’avaient pas assez pris soin de se munir, les espoirs qu’éveillait en eux la guerre : des Vénètes, et enfin la tendance qu’ont généralement les hommes à croire ce qu’ils désirent. Sous l’empire de ces idées, ils ne laissent pas Viridovix et les autres chefs quitter l’assemblée qu’ils n’aient obtenu l’ordre de prendre les armes et d’attaquer le camp. Joyeux de ce consentement, comme s’ils tenaient déjà la victoire, ils amassent des fascines et des branchages pour en combler les fossés des Romains, et ils marchent sur le camp.
\subsection[{§ 19.}]{ \textsc{§ 19.} }
\noindent Celui-ci était sur une hauteur où l’on accédait par une pente douce de mille pas environ. Ils s’y portèrent en courant très vite, afin que les Romains eussent le moins de temps possible pour se ressaisir et prendre les armes, et ils arrivèrent hors d’haleine. Sabinus, ayant harangué ses troupes, donne le signal qu’elles attendaient impatiemment. L'ennemi était embarrassé par les fardeaux dont il était chargé : Sabinus ordonne une sortie brusque par deux portes. L'avantage du terrain, l’inexpérience et la fatigue de l’ennemi, le courage de nos soldats et l’entraînement qu’ils avaient acquis dans les batailles précédentes, tout cela fit que dès le premier choc les ennemis cédèrent et prirent la fuite. Gênés dans leurs mouvements, poursuivis par les nôtres dont les forces étaient intactes, ils perdirent beaucoup de monde ; ceux qui restaient, furent harcelés par la cavalerie, qui n’en laissa échapper qu’un petit nombre. Sabinus apprit la bataille navale en même temps que César était informé de sa victoire, et toutes les cités s’empressèrent de lui faire leur soumission. Car autant les Gaulois sont, pour prendre les armes, enthousiastes et prompts, autant ils manquent, pour supporter les revers, de fermeté et de ressort.
\subsection[{§ 20.}]{ \textsc{§ 20.} }
\noindent Vers le même temps, Publius Crassus était arrivé en Aquitaine ; cette région, comme on l’a dit plus haut, peut être estimée, pour son étendue et sa population, au tiers de la Gaule. Voyant qu’il devait faire la guerre dans des contrées où peu d’années auparavant Lucius Valérius Préconinus, légat, avait été vaincu et tué, et d’où Lucius Manlius, proconsul, avait dû s’enfuir en abandonnant ses bagages, il se rendait compte qu’il lui faudrait être particulièrement attentif. Il fit donc ses provisions de blé, rassembla des auxiliaires et de la cavalerie, convoqua en outre individuellement, de Toulouse et de Narbonne, cités de la province de Gaule qui sont voisines de l’Aquitaine, un grand nombre de soldats éprouvés ; puis il pénétra sur le territoire des Sotiates. A la nouvelle de son approche, ceux-ci rassemblèrent des troupes nombreuses et de la cavalerie, qui était leur principale force, et attaquèrent notre armée pendant sa marche : ils livrèrent d’abord un combat de cavalerie, puis, comme leurs cavaliers avaient été refoulés et que les nôtres les poursuivaient, soudain ils découvrirent leur infanterie, qu’ils avaient placée en embuscade dans un vallon. Elle fonça sur nos soldats dispersés, et un nouveau combat s’engagea.
\subsection[{§ 21.}]{ \textsc{§ 21.} }
\noindent Il fut long et acharné : les Sotiates, forts de leurs précédentes victoires, pensaient que le salut de toute l’Aquitaine dépendait de leur valeur ; les nôtres voulaient montrer ce qu’ils pouvaient faire en l’absence du général en chef, sans les autres légions et sous le commandement d’un tout jeune homme. Enfin les ennemis, couverts de blessures, prirent la fuite. Crassus en fit un grand massacre et, sans désemparer, essaya d’attaquer la citadelle des Sotiates. Devant leur vigoureuse résistance, il fit avancer mantelets et tours. Eux, tantôt faisaient des sorties, tantôt creusaient des mines vers le terrassement et les mantelets (c’est une pratique où les Aquitains sont tout particulièrement habiles, car il y a chez eux, en maint endroit, des mines de cuivre et des carrières) ; mais, ayant compris que la vigilance de nos soldats les empêchait d’obtenir aucun résultat par ces moyens, ils envoient des députés à Crassus et demandent qu’il accepte leur soumission. Il consent, et, sur son ordre, ils livrent leurs armes.
\subsection[{§ 22.}]{ \textsc{§ 22.} }
\noindent Tandis que cette reddition retenait l’attention de toute l’armée, d’un autre côté de la place, Adiatuanos, qui détenait le pouvoir suprême, parut avec six cents hommes à sa dévotion, de ceux qu’ils nomment des {\itshape soldures} ; la condition de ces personnages est la suivante : celui à qui ils ont voué leur amitié doit partager avec eux tous les biens de la vie ; mais s’il périt de mort violente, ils doivent ou subir en même temps qu’eux le même sort ou se tuer eux-mêmes ; et de mémoire d’homme il ne s’est encore vu personne qui refusât de mourir quand avait péri l’ami auquel il s’était dévoués. C'est avec cette escorte qu’Adiatuanos tentait une sortie ; une clameur s’éleva de ce côté du retranchement, et nos soldats coururent aux armes : après un violent combat, Adiatuanos fut refoulé dans la place ; il n’en obtint pas moins de Crassus les mêmes conditions que les autres.
\subsection[{§ 23.}]{ \textsc{§ 23.} }
\noindent Ayant reçu armes et otages, Crassus partit pour le pays des Vocates et des Tarusates. Alors les Barbares, vivement émus d’apprendre qu’une place fortifiée par la nature et par l’art était tombée dans les quelques jours qui avaient suivi notre arrivée, envoient de toutes parts des députés, échangent des serments, des otages, et mobilisent leurs forces. On envoie aussi des ambassadeurs aux peuples qui appartiennent à l’Espagne citérieure, voisine de l’Aquitaine on en obtient des troupes de secours et des chefs. Leur arrivée permet d’entrer en guerre avec une excellente direction et de nombreux effectifs. On choisit pour chefs des hommes qui avaient été constamment les compagnons de Sertorius et passaient pour être très experts dans l’art militaire. Ils font la guerre à la romaine, occupant les positions favorables, fortifiant leurs camps, nous coupant les vivres. Lorsque Crassus s’aperçut que ses troupes, trop peu nombreuses, ne pouvaient guère être divisées, que les ennemis, eux, pouvaient circuler en tous sens, bloquer les routes, et cependant laisser au camp une garde suffisante, que pour cette raison il ne se ravitaillait qu’avec peine, que chaque jour les ennemis étaient plus nombreux, il jugea qu’il ne devait pas tarder davantage à livrer bataille. Il porta la question devant le conseil, et quand il vit que tous étaient du même avis, il fixa la bataille au lendemain.
\subsection[{§ 24.}]{ \textsc{§ 24.} }
\noindent Au point du jour, il déploya en avant du camp toutes ses troupes, sur deux lignes, les auxiliaires au centre, et il attendit la décision des ennemis. Mais eux, bien que leur nombre, leurs glorieuses traditions guerrières, la faiblesse de nos effectifs les rassurassent pleinement sur l’issue d’un combat, ils trouvaient cependant plus sûr encore, étant maîtres des routes et, ainsi, nous coupant les vivres, d’obtenir la victoire sans coup férir si la disette déterminait les Romains à battre en retraite, ils se proposaient de les attaquer en pleine marche, embarrassés de leurs convois et chargés de leurs bagages, dans des conditions où leur courage serait déprimé. Les chefs ayant approuvé ce dessein, ils laissaient les Romains déployer leurs troupes et restaient au camp. Lorsque Crassus vit cela, comme, par ses hésitations et en ayant l’air d’avoir peur, l’ennemi avait excité l’ardeur de nos troupes, et qu’il n’y avait qu’une voix pour dire qu’on ne devait pas tarder plus longtemps à attaquer, il les harangua et, répandant au vœu de tous, marcha sur le camp ennemi.
\subsection[{§ 25.}]{ \textsc{§ 25.} }
\noindent Là, tandis que les uns comblaient les fossés, les autres, lançant sur les défenseurs une grêle de traits, les forçaient à abandonner le parapet et les retranchements ; et les auxiliaires, en qui Crassus n’avait guère confiance comme combattants, passaient des pierres et des munitions, apportaient des mottes de gazon pour élever une terrasse, et ainsi donnaient à croire qu’effectivement ils combattaient ; l’ennemi, de son côté, opposait une résistance tenace et valeureuse, et ses projectiles lancés de haut, ne manquaient pas d’efficacité. Cependant des cavaliers, ayant fait le tour du camp ennemi, vinrent dire à Crassus que du côté de la porte décumane le camp était moins soigneusement fortifié, et offrait un accès faciles.
\subsection[{§ 26.}]{ \textsc{§ 26.} }
\noindent Crassus invita les préfets de la cavalerie à exciter le zèle de leurs hommes en leur promettant des récompenses, et leur expliqua ses intentions. Ceux-ci, selon l’ordre reçu, firent sortir les cohortes qui avaient été laissées à la garde du camp et qui étaient toutes fraîches, et, par un chemin détourné, afin qu’on ne pût les apercevoir du camp ennemi, elles atteignirent rapidement, tandis que le combat accaparait l’attention de tous, la partie du retranchement que nous avons dite ; elles le forcèrent, et se reformèrent dans le camp de l’ennemi avant que celui-ci ait pu les bien voir ni se rendre compte de ce qui se passait. Alors les nôtres, entendant la clameur qui s’élevait de ce côté, se sentirent des forces nouvelles, comme il arrive généralement quand on a l’espoir de vaincre, et ils redoublèrent d’ardeur. Les ennemis, se voyant enveloppés de toutes parts et perdant toute espérance, ne pensèrent plus qu’à sauter à bas du retranchement pour chercher leur salut dans la fuite. Nos cavaliers les poursuivirent en rase campagne, et sur les cinquante mille Aquitains et Cantabres qui formaient cette armée, un quart à peine échappa à leurs coups ; la nuit était fort avancée quand ils rentrèrent au camp.
\subsection[{§ 27.}]{ \textsc{§ 27.} }
\noindent A la nouvelle de ce combat, la plus grande partie de l’Aquitaine se soumit à Crassus et envoya spontanément des otages : parmi ces peuples étaient les Tarbelles, les Bigerrions, les Ptianii, les Vocates, les Tarusates, les Elusates, les Gates, les Ausques, les Garunni, les Sibuzates, les Cocsates ; seuls quelques-uns, qui étaient placés aux confins, se fiant à la saison avancée, car on était aux approches de l’hiver, ne suivirent pas cet exemple.
\subsection[{§ 28.}]{ \textsc{§ 28.} }
\noindent Vers le même temps, bien que l’été fût presque à son terme, César estima cependant, comme il n’y avait plus dans la Gaule toute entière pacifiée que les Morins et les Ménapes qui fussent en armes et ne lui eussent jamais envoyé demander la paix, que c’était là une guerre qui pouvait être achevée promptement, et il conduisit son armée dans ces régions. Il eut affaire à une tactique toute différente de celle des autres Gaulois. Voyant, en effet, que les plus grands peuples qui avaient livré bataille à César avaient été complètement battus, et possédant une région que couvraient sans interruption forêts et marécages, ils s’y transportèrent avec tout leurs biens. César était parvenu à la lisière de ces forêts, il avait commencé de construire un camp et les ennemis ne s’étaient pas encore montrés, quand soudain, au moment où nos soldats étaient au travail et dispersés, ils bondirent de toutes parts hors de la forêt et chargèrent les nôtres. Ceux-ci prirent rapidement les armes et les refoulèrent dans leurs bois ; après en avoir tué un très grand nombre, ils les poursuivirent trop loin sur un terrain trop difficile, et perdirent quelques hommes.
\subsection[{§ 29.}]{ \textsc{§ 29.} }
\noindent Les jours suivants, César décida de les employer sans relâche à abattre la forêt, et, pour que nos soldats ne pussent être surpris, sans armes, par une attaque de flanc, il disposait face à l’ennemi tous ces arbres coupés et les amoncelait sur chaque flanc en manière de rempart. On avait fait en quelques jours, avec une rapidité incroyable, une vaste clairière, et déjà nous nous étions emparés du bétail et des derniers bagages de l’ennemi, qui s’enfonçait au cœur des forêts, lorsque le temps se gâta si fort qu’il fallut interrompre le travail et que, la pluie ne cessant pas, il devint impossible de garder plus longtemps les hommes sous la tente. En conséquence, après avoir ravagé toute la campagne, brûlé les bourgs et les fermes, César ramena son armée et lui fit prendre ses quartiers d’hiver chez les Aulerques et les Lexovii, ainsi que chez les autres peuples qui venaient de nous faire la guerre.
 \section[{Livre IV}]{Livre IV}\renewcommand{\leftmark}{Livre IV}

\subsection[{§ 1.}]{ \textsc{§ 1.} }
\noindent L'hiver qui suivit – c’était l’année du consulat de Cnéus Pompée et de Marcus Crassus –, les Usipètes, peuple de Germanie, et aussi les Tencthères, passèrent le Rhin en masse, non loin de la mer où il se jette. La raison de ce passage fut que depuis plusieurs années les Suèves leur faisaient une guerre continuelle et très dure, et qu’ils ne pouvaient plus cultiver leurs champs.\par
\par
Les Suèves sont le peuple de beaucoup le plus grand et le plus belliqueux de toute la Germanie. On dit qu’ils forment cent clans, lesquels fournissent chacun mille hommes par an, qu’on emmène faire des guerres extérieures. Les autres, ceux qui sont restés au pays, pourvoient à leur nourriture et à celle de l’armée ; l’année suivante, ceux-ci prennent à leur tour les armes, tandis que ceux-là restent au pays. De la sorte, la culture des champs, l’instruction et l’entraînement militaires sont également assurés sans interruption. D'ailleurs, la propriété privée n’existe pas chez eux, et on ne peut séjourner plus d’un an sur le même sol pour le cultiver. Le blé compte peu dans leur alimentation, ils vivent principalement du lait et de la chair des troupeaux, et ils sont grands chasseurs ; ce genre de vie – leur alimentation, l’exercice quotidien, la vie libre, car, dès l’enfance, n’étant pliés à aucun devoir, à aucune discipline, ils ne font rien que ce qui leur plaît –, tout cela les fortifie et fait d’eux des hommes d’une taille extraordinaire. Ajoutez qu’ils se sont entraînés, bien qu’habitant des régions très froides, à n’avoir pour tout vêtement que des peaux, dont l’exiguïté laisse à découvert une grande partie de leur corps, et à se baigner dans les fleuves.
\subsection[{§ 2.}]{ \textsc{§ 2.} }
\noindent Ils donnent accès chez eux aux marchands, plus pour avoir à qui vendre leur butin de guerre que par besoin d’importations. Les Germains n’importent même pas de chevaux, qui sont la grande passion des Gaulois et qu’ils acquièrent à n’importe quel prix ; ils se contentent des chevaux indigènes, qui sont petits et laids, mais qu’ils arrivent à rendre extrêmement résistants grâce à un entraînement quotidien. Dans les combats de cavalerie, on les voit souvent sauter à bas de leur monture et combattre à pied ; les chevaux ont été dressés à rester sur place, et ils ont vite fait de les rejoindre en cas de besoin ; il n’y a pas à leurs yeux de plus honteuse mollesse que de faire usage de selles. Aussi n’hésitent-ils pas à attaquer, si peu nombreux soient-ils, n’importe quel corps de cavalerie dont les chevaux sont sellés. Ils prohibent absolument l’importation du vin, parce qu’ils estiment que cette boisson diminue chez l’homme l’endurance et le courage.
\subsection[{§ 3.}]{ \textsc{§ 3.} }
\noindent Ils pensent que la plus grande gloire d’une nation, c’est d’avoir au-delà de ses frontières un désert aussi vaste que possible, car cela signifie qu’un grand nombre de cités n’ont pu soutenir la puissance de ses armes. Aussi dit-on que sur un côté de la frontière des Suèves il y a une solitude de six cent mille pas. De l’autre côté, ils ont pour voisins les Ubiens, qui formèrent un État considérable et florissant, autant qu’un État germain peut l’être ; ils sont un peu plus civilisés que les autres peuples de même race, parce qu’ils touchent au Rhin et que les marchands viennent beaucoup chez eux, parce qu’aussi, étant voisins des Gaulois, ils se sont façonnés à leurs mœurs. Les Suèves se mesurèrent avec eux à mainte reprise, mais ne purent, en raison de l’importance et de la force de cette nation, les chasser de leur territoire ; ils les assujettirent cependant à un tribut, et les abaissèrent et affaiblirent très sensiblement.
\subsection[{§ 4.}]{ \textsc{§ 4.} }
\noindent Ce fut aussi le sort des Usipètes et des Tencthères, dont il a été question plus haut ; pendant de longues années ils résistèrent aux attaques des Suèves, mais ils furent finalement chassés de leur territoire, et après avoir erré trois ans dans maintes régions de la Germanie, ils atteignirent le Rhin ; c’était le pays des Ménapes, qui avaient des champs, des maisons, des villages sur les deux rives du fleuve ; mais, épouvantés par l’arrivée d’une telle multitude, ils abandonnèrent les maisons qu’ils avaient jusque-là possédées au-delà du fleuve et disposèrent de ce côté-ci du Rhin des postes qui barraient la route aux envahisseurs. Ceux-ci, après toutes sortes de tentatives, ne pouvant passer de vive force faute de navires, ni clandestinement à cause des postes des Ménapes, feignirent de rentrer chez eux et firent trois journées de marche sur le chemin du retour ; puis, refaisant tout ce trajet en une nuit, leur cavalerie tomba à l’improviste sur les Ménapes qui, ayant appris par leurs éclaireurs le départ des Germains, avaient sans crainte repassé le Rhin et regagné leurs villages. Ils les massacrèrent et, s’emparant de leurs navires, franchirent le fleuve avant que les Ménapes de l’autre rive fussent informés de riens ; ils occupèrent toutes leurs demeures et vécurent de leurs provisions pendant le reste de l’hiver.
\subsection[{§ 5.}]{ \textsc{§ 5.} }
\noindent César, instruit de ces événements, et redoutant la pusillanimité des Gaulois, car ils changent facilement d’avis et sont presque toujours séduits par ce qui est nouveau, estima qu’il ne devait se reposer sur eux de rien. Il est, en effet, dans les habitudes des Gaulois d’arrêter les voyageurs, même contre leur gré, et de les interroger sur tout ce que chacun d’eux peut savoir ou avoir entendu dire ; dans les villes, la foule entoure les marchands et les oblige à dire de quel pays ils viennent et ce qu’ils y ont appris. Sous le coup de l’émotion que provoquent ces nouvelles ou ces bavardages, il leur arrive souvent de prendre sur les affaires les plus importantes des décisions dont il leur faut incontinent se repentir, car ils accueillent en aveugles des bruits mal fondés et la plupart de leurs informateurs inventent des réponses conformes à ce qu’ils désirent.
\subsection[{§ 6.}]{ \textsc{§ 6.} }
\noindent César, connaissant ces habitudes, et ne voulant pas se trouver en face d’une guerre particulièrement redoutable, part pour l’armée plus tôt qu’il ne faisait d’ordinaire. Quand il y arriva, il apprit que ce qu’il avait prévu s’était produit : un grand nombre de cités avaient envoyé des ambassades aux Germains et les avaient invités à ne pas se cantonner au Rhin ; elles s’engageaient à fournir à toutes leurs demandes. Séduits par ces promesses, les Germains poussaient plus loin, et ils étaient arrivés sur le territoire des Eburons et des Condruses, qui sont les clients des Trévires. César, ayant convoqué les chefs gaulois, jugea préférable de dissimuler ce qu’il savait après les avoir tranquillisés et rassurés, il leur ordonna de lui fournir de la cavalerie et se déclara résolu à la guerre.
\subsection[{§ 7.}]{ \textsc{§ 7.} }
\noindent Après qu’il eut fait ses provisions de blé et recruté sa cavalerie, il se mit en route pour la région où l’on disait qu’étaient les Germains : Il n’en était plus qu’à peu de journées, quand il reçut d’eux des députés qui lui tinrent ce langage : « Les Germains ne prennent pas l’initiative de faire la guerre au peuple romain, mais, si on les attaque, ils ne refusent pas la lutte ; car la tradition des Germains c’est, quel que soit l’agresseur, de se défendre et de ne pas implorer la paix. Voici cependant ce qu’ils déclarent : ils ne sont venus que contre leur gré, parce qu’on les chassait de chez eux ; si les Romains acceptent leur amitié, ils peuvent leur être d’utiles amis : qu’ils leur assignent des terres, ou qu’ils les laissent conserver celles qu’ils ont conquises. Ils ne le cèdent qu’aux Suèves, auxquels les dieux mêmes ne sauraient être comparés : sauf eux, il n’est personne sur la terre qu’ils ne soient capables de vaincre.
\subsection[{§ 8.}]{ \textsc{§ 8.} }
\noindent César fit à ce discours la réponse qu’il jugea convenable ; mais pour sa conclusion, elle fut qu’il n’y avait pas d’amitié possible d’eux à lui, s’ils restaient en Gaule : « D'abord il n’est pas juste qu’un peuple qui n’a pas su défendre son territoire s’empare de celui d’autrui ; d’autre part, il n’y a pas en Gaule de terres vacantes qu’on puisse donner, surtout à une telle multitude, sans nuire à personne ; mais ils peuvent, s’ils le veulent, s’établir sur le territoire des Ubiens, dont il a auprès de lui des députés qui se plaignent des violences des Suèves et lui demandent du secours ; il leur donnera l’ordre de les accueillir.
\subsection[{§ 9.}]{ \textsc{§ 9.} }
\noindent Les ambassadeurs germains dirent qu’ils allaient rapporter cette réponse, et qu’ils reviendraient dans trois jours, une fois qu’on en aurait délibéré ; ils demandèrent qu’en attendant César n’avançât point davantage. Celui-ci se déclara dans l’impossibilité de faire pareille concession. Il savait, en effet, qu’une grande partie de leur cavalerie avait été envoyée par eux, quelques jours auparavant, chez les Ambivarites d’au-delà la Meuse pour y faire du butin et y prendre du blé ; il pensait qu’on attendait ces cavaliers et que c’était pour cela qu’on demandait un délai.
\subsection[{§ 10.}]{ \textsc{§ 10.} }
\noindent La Meuse prend sa source dans les Vosges, qui sont sur le territoire des Lingons et, après avoir reçu un bras du Rhin, qu’on appelle le Waal, et formé avec lui l’île des Bataves, elle se jette dans l’Océan et à quatre-vingt mille pas environ de l’Océan, elle se jette dans le Rhin. Quant à ce fleuve, il prend sa source chez les Lépontes, habitant des Alpes, parcourt d’une allure rapide un long espace à travers les pays des Nantuates, des Helvètes, des Séquanes, des Médiomatrices, des Triboques, des Trévires ; à l’approche de l’Océan, il se divise en plusieurs bras en formant des îles nombreuses et immenses, dont la plupart sont habitées par des nations farouches et barbares, au nombre desquelles sont ces hommes qu’on dit se nourrir de poissons et d’œufs d’oiseaux ; il se jette dans l’Océan par plusieurs embouchures.
\subsection[{§ 11.}]{ \textsc{§ 11.} }
\noindent César n’était pas à plus de douze milles de l’ennemi quand les députés, observant le délai fixé, revinrent. Ils le rencontrèrent en marche, et se mirent à le supplier de ne pas aller glus avant ; leurs prières restant vaines, ils essayèrent d’obtenir qu’il fît porter aux cavaliers qui étaient en avant-garde l’ordre de ne pas engager le combat, et qu’il les laissât envoyer aux Ubiens des députés ; si les chefs de ce peuple et son sénat s’engageaient sous serment, ils déclaraient accepter la proposition que faisait César ; ils demandaient qu’il leur accordât trois jours pour ces négociations. César pensait que tout cela visait toujours au même but : gagner trois jours pour permettre à leur cavalerie, qui était absente, de revenir ; néanmoins, il dit qu’il n’avancerait ce jour-là que de quatre milles, pour se procurer de l’eau ; qu’ils vinssent le trouver le lendemain à cet endroit en aussi grand nombre que possible, afin qu’il pût se prononcer en connaissance de cause sur leurs demandes. En attendant, il fait dire à ses préfets, qui le précédaient avec toute la cavalerie, de ne pas attaquer l’ennemi, et, si on les attaque, de se borner à la défensive, jusqu’à ce qu’il soit là avec l’armée.
\subsection[{§ 12.}]{ \textsc{§ 12.} }
\noindent Mais les ennemis, dès qu’ils aperçurent nos cavaliers, qui étaient au nombre d’environ cinq mille, tandis qu’eux-mêmes n’en avaient pas plus de huit cents – ceux qui étaient allés chercher du blé au-delà de la Meuse n’étant pas encore revenus –, chargèrent les nôtres, qui ne se méfiaient de rien, parce que les députés ennemis venaient de quitter César et avaient demandé une trêve pour cette journée même ; ils eurent vite fait de mettre le désordre dans nos rangs ; puis, comme nos cavaliers se reformaient, ils mirent pied à terre, selon leur coutume, et, frappant les chevaux par-dessous, jetant à bas un très grand nombre de nos hommes, ils mirent les autres en fuite : la panique fut telle, et la poursuite si vive, qu’ils ne s’arrêtèrent qu’une fois en vue de nos colonnes. Dans ce combat, soixante-quatorze de nos cavaliers trouvèrent la mort, et parmi eux un homme très valeureux, l’Aquitain Pison, personnage de haute naissance dont l’aïeul avait été roi dans sa cité et avait reçu de notre sénat le titre d’ami. Comme il portait secours à son frère, que les ennemis enveloppaient, il réussit à l’arracher au danger, mais il eut lui-même son cheval blessé et fut jeté à terre ; aussi longtemps qu’il put, il tint tête avec un grand courage ; mais, entouré de toutes parts, couvert de blessures, il tomba, et son frère, qui déjà était hors de la mêlée, voyant de loin le drame, se jeta au galop sur l’ennemi et fut tué.
\subsection[{§ 13.}]{ \textsc{§ 13.} }
\noindent Après ce combat, César estimait qu’il ne devait plus donner audience aux députés ni accueillir les propositions de gens qui avaient commencé les hostilités par traîtrise, à la faveur d’une demande de paix ; quant à attendre, en laissant les forces des ennemis s’accroître par le retour de leur cavalerie, il jugeait que c’eût été folie pure ; connaissant d’ailleurs la pusillanimité des Gaulois, il comprenait tout ce que déjà par ce seul combat l’ennemi avait gagné de prestige à leurs yeux : il ne fallait pas leur laisser le temps de se décider. Sa pensée était bien arrêtée sur tout cela, et il avait communiqué à ses légats et à son questeur sa résolution de ne pas différer d’un jour la bataille, quand une circonstance très favorable se présenta le lendemain au matin, agissant toujours avec la même traîtrise et la même hypocrisie, les Germains vinrent en grand nombre, avec tous les chefs et tous les anciens trouver César dans son camp ; ils voulaient – c’était le prétexte – s’excuser de ce qu’ils avaient la veille engagé le combat contrairement aux conventions et à leurs propres demandes ; mais en même temps ils se proposaient d’obtenir, s’ils le pouvaient, en nous trompant, quelque trêves. César, heureux qu’ils vinssent ainsi s’offrir, ordonna de les garder ; puis il fit sortir du camp toutes ses troupes ; la cavalerie, démoralisée, pensait-il, par le dernier combat, fut placée à l’arrière-garde.
\subsection[{§ 14.}]{ \textsc{§ 14.} }
\noindent Ayant disposé son armée en ordre de bataille sur trois rangs, et ayant parcouru rapidement huit milles, il arriva au camp des ennemis avant qu’ils pussent s’apercevoir de ce qui se passait. Tout concourait à frapper les Germains d’une peur subite la promptitude de notre approche, l’absence de leurs chefs, et de n’avoir le temps ni de tenir conseil, ni de prendre leurs armes ; ils s’affolent, ne sachant s’il vaut mieux aller au-devant de l’ennemi, ou défendre le camp, ou chercher son salut dans la fuite. Comme la rumeur et le rassemblement confus des hommes manifestaient leur frayeur, nos soldats, stimulés par la perfidie de la veille, firent irruption dans le camp. Là, ceux qui purent s’armer promptement résistèrent un moment aux nôtres, engageant le combat parmi les chariots et les bagages ; mais il restait une foule d’enfants et de femmes (car ils étaient partis de chez eux et avaient passé le Rhin avec tous les leurs) qui se mit à fuir de tous côtés. César envoya sa cavalerie à leur poursuite.
\subsection[{§ 15.}]{ \textsc{§ 15.} }
\noindent Les Germains, entendant une clameur derrière eux, et voyant qu’on massacrait les leurs, jetèrent leurs armes, abandonnèrent leurs enseignes et se précipitèrent hors du camp ; arrivés au confluent de la Meuse et du Rhin, désespérant de pouvoir continuer leur fuite et voyant qu’un grand nombre d’entre eux avaient été tués, ceux qui restaient se jetèrent dans le fleuve et là, vaincus par la peur, par la fatigue, par la force du courant, ils périrent. Les nôtres, sans avoir perdu un seul homme et n’ayant qu’un tout petit nombre de blessés, après avoir redouté une lutte terrible, car ils avaient eu affaire à quatre cent trente mille ennemis, se retirèrent dans leur camp. César autorisa ceux qu’il avait retenus à s’en aller ; mais eux, craignant que les Gaulois, dont ils avaient ravagé les champs, ne leur fissent subir de cruels supplices, déclarèrent qu’ils désiraient rester auprès de lui. César leur accorda la liberté.
\subsection[{§ 16.}]{ \textsc{§ 16.} }
\noindent La guerre germanique achevée, César, pour maintes raisons, décida de franchir le Rhin ; la meilleure était que, voyant avec quelle facilité les Germains se déterminaient à venir en Gaule, il voulut qu’eux aussi eussent à craindre pour leurs biens, quand ils comprendraient qu’une armée romaine pouvait et osait traverser le Rhin. Un autre motif était que ceux des cavaliers Usipètes et Tencthères dont j’ai dit plus haut qu’ils avaient passé la Meuse pour faire du butin et prendre du blé, et qu’ils n’avaient pas participé au combat, s’étaient, après la défaite des leurs, réfugiés au-delà du Rhin chez les Sugambres, et avaient fait alliance avec eux. César ayant fait demander aux Sugambres de lui livrer ces hommes qui avaient porté les armes contre lui et contre les Gaulois, ils répondirent que « la souveraineté du peuple Romain expirait au Rhin ; s’il ne trouvait pas juste que les Germains passassent en Gaule malgré lui, pourquoi prétendrait-il à quelque souveraineté ou autorité au-delà du Rhin ? » D'autre part, les Ubiens, qui seuls parmi les Transrhénans avaient envoyé des députés à César, avaient lié amitié avec lui, lui avaient donné des otages, le priaient très instamment de leur porter secours, parce que les Suèves menaçaient leur existence. « Si les affaires de la république le retenaient, qu’il fît seulement passer le Rhin à son armée ; cela suffirait pour écarter le danger de l’heure présente et pour garantir leur sécurité future le renom et la réputation de cette armée étaient tels, depuis la défaite d’Arioviste et après ce dernier combat, même chez les plus lointaines peuplades de la Germanie, que si on les savait amis de Rome, on les respecterait. » Ils promettaient une grande quantité d’embarcations pour le transport de l’armée.
\subsection[{§ 17.}]{ \textsc{§ 17.} }
\noindent César, pour les raisons que j’ai dites, avait décidé de franchir le Rhin ; mais les bateaux lui semblaient un moyen trop peu sûr, et qui convenait mal à sa dignité et à celle du peuple romain. Aussi, en dépit de l’extrême difficulté que présentait la construction d’un pont, à cause de la largeur, de la rapidité et de la profondeur du fleuve, il estimait qu’il devait tenter l’entreprise ou renoncer à faire passer ses troupes autrement. Voici le nouveau procédé de construction qu’il employa. Il accouplait, à deux pieds l’une de l’autre, deux poutres d’un pied et demi d’épaisseur, légèrement taillées en pointe par le bas et dont la longueur était proportionnée à la profondeur du fleuve. Il les descendait dans le fleuve au moyen de machines et les enfonçait à coups de mouton, non point verticalement, comme des pilotis ordinaires, mais obliquement, inclinées dans la direction du courant ; en face de ces poutres, il en plaçait deux autres, jointes de même façon, à une distance de quarante pieds en aval et penchées en sens inverse du courant. Sur ces deux paires on posait des poutres larges de deux pieds, qui s’enclavaient exactement entre les pieux accouplés, et on plaçait de part et d’autre deux crampons qui empêchaient les couples de se rapprocher par le haut ; ceux-ci étant ainsi écartés et retenus chacun en sens contraire, l’ouvrage avait tant de solidité, et cela en vertu des lois de la physique, que plus la violence du courant était grande, plus le système était fortement lié. On posait sur les traverses des poutrelles longitudinales et, par dessus, des lattes et des claies. En outre, on enfonçait en aval des pieux obliques qui, faisant contrefort, appuyant l’ensemble de l’ouvrage, résistaient au courant ; d’autres étaient plantés à une petite distance en avant du pont c’était une défense qui devait, au cas où les Barbares lanceraient des troncs d’arbres ou des navires destinés à le jeter bas, atténuer la violence du choc et préserver l’ouvrage.
\subsection[{§ 18.}]{ \textsc{§ 18.} }
\noindent Dix jours après qu’on avait commencé à apporter les matériaux, toute la construction est achevée et l’armée passe le fleuve. César laisse aux deux têtes du pont une forte garde et se dirige vers le pays des Sugambres. Sur ces entrefaites, il reçoit des députations d’un grand nombre de cités ; à leur demande de paix et d’amitié, il répond avec bienveillance et ordonne qu’on lui amène des otages. Mais les Sugambres, qui avaient, dès l’instant où l’on commença de construire le pont, préparé leur retraite, sur le conseil des Tencthères et des Usipètes qui étaient auprès d’eux, avaient quitté leur pays en emportant tout leurs biens et étaient allés se cacher dans des contrées inhabitées et couvertes de forêts.
\subsection[{§ 19.}]{ \textsc{§ 19.} }
\noindent César, après être resté quelques jours sur leur territoire, incendia tous les villages et tous les bâtiments, coupa le blé, et se retira chez les Ubiens ; il leur promit de les secourir si les Suèves les attaquaient, et reçut d’eux les informations suivantes : les Suèves, ayant appris par leurs éclaireurs qu’on jetait un pont sur le Rhin, avaient, à la suite d’un conseil tenu selon leur usage, envoyé de tous côtés l’avis qu’on abandonnât les villes, qu’on déposât dans les forêts enfants, femmes et tout ce qu’on possédait, et que tous les hommes capables de porter les armes se concentrassent sur un même point. Le lieu choisi était à peu près au centre de la contrée habitée par les Suèves c’est là qu’ils avaient décidé d’attendre l’arrivée des Romains et là qu’ils devaient leur livrer la bataille décisive. Quand César connut ce plan, comme il avait atteint tous les objectifs qu’il s’était proposés en franchissant le Rhin – faire peur aux Germains, punir les Sugambres, délivrer les Ubiens de la pression qu’ils subissaient –, après dix-huit jours complets passés au-delà du Rhin, estimant avoir atteint un résultat suffisamment glorieux et suffisamment utile, il revint en Gaule et coupa le pont derrière lui.
\subsection[{§ 20.}]{ \textsc{§ 20.} }
\noindent César n’avait plus devant lui qu’une petite partie de l’été ; bien que dans ces régions – car toute la Gaule est tournée vers le nord – les hivers soient précoces, il voulut néanmoins partir pour la Bretagne, parce qu’il se rendait compte que dans presque toutes les guerres que nous avions faites contre les Gaulois, ceux-ci avaient reçu des secours de la Bretagne ; il pensait d’ailleurs que si la saison trop avancée ne lui laissait pas le temps de faire campagne, il lui serait néanmoins fort utile d’avoir seulement abordé dans l’île, et d’avoir vu ce qu’étaient ses habitants, reconnu les lieux, les ports, les points de débarquement : toutes choses qui étaient à peu près ignorées des Gaulois. En effet, à part les marchands, il est rare que personne se risque là-bas, et les marchands eux-mêmes ne connaissent rien en dehors de la côte et des régions qui font face à la Gaule. Aussi eut-il beau faire venir de partout des marchands, il lui était impossible de rien apprendre ni sur l’étendue de l’île, ni sur le caractère et l’importance des peuples qui l’habitent, ni sur leur manière de faire la guerre ou de vivre, ni sur les ports qui étaient capables de recevoir un grand nombre de gros navires.
\subsection[{§ 21.}]{ \textsc{§ 21.} }
\noindent Pour se renseigner là-dessus, avant de tenter l’entreprise, César détache, avec un navire de guerre, Casus Volusénus, qu’il jugeait propre à cette mission. Il lui donne comme instructions de faire une reconnaissance générale et de revenir au plus vite. De son côté, il part avec toutes ses troupes pour le pays des Morins, car c’est de là que le passage en Bretagne est le plus court. Il y rassemble des navires tirés de toutes les contrées voisines et la flotte qu’il avait construite l’été précédent pour la guerre des Vénètes. Cependant son projet s’ébruite et les marchands en portent la nouvelle aux Bretons : maints peuples de l’île lui envoient des députés pour offrir de livrer des otages et de faire soumission à Rome. Il leur donne audience, leur fait des promesses généreuses, les engage à persévérer dans ces sentiments, et les renvoie chez eux accompagnés de Commios, qu’il avait fait roi des Atrébates après sa victoire sur ce peuple ; il appréciait son courage et son intelligence, il le jugeait fidèle, et son autorité était grande dans le pays. Il lui ordonne de visiter le plus de peuples possible, de les engager à se placer sous le protectorat de Rome, et d’annoncer son arrivée prochaine. Volusénus, après avoir reconnu les lieux autant qu’il put le faire sans oser débarquer et courir les risques d’un contact avec les Barbares, rentre au bout de quatre jours et rapporte à César ce qu’il a observé.
\subsection[{§ 22.}]{ \textsc{§ 22.} }
\noindent Pendant que César s’attardait chez les Morins pour armer sa flotte, beaucoup de leurs tribus envoyèrent des députés lui présenter des excuses au sujet de leur conduite passées ils avaient fait la guerre au peuple romain en hommes frustes et ignorants de notre caractère ; ils se déclaraient prêts à exécuter les ordres de César. Celui-ci, trouvant la conjoncture fort heureuse – car il ne voulait pas laisser un ennemi derrière lui, la saison était trop avancée pour leur faire la guerre, enfin, il estimait que l’expédition de Bretagne passait avant d’aussi minces soucis –, fixe un chiffre élevé d’otages à livrer. Ils les lui amènent, et il reçoit leur soumission. Ayant rassemblé et fait ponter environ quatre-vingts navires de transport, nombre qu’il jugeait suffisant pour transporter deux légions, il distribua ce qu’il avait en outre de vaisseaux de guerre à son questeur, à ses légats et à ses préfets. A ces unités s’ajoutaient dix-huit transports qui étaient à huit milles de là, empêchés par les vents contraires de rallier le même port : il les assigna à la cavalerie. Le reste de l’armée fut confié aux légats Quintus Titurius Sabinus et Lucius Aurunculéius Cotta, avec mission de la conduire chez les Ménapes et dans les cantons morins qui n’avaient pas envoyé de députés. Le légat Publius Sulpicius Rufus, avec la garnison qui fut jugée convenable, fut préposé à la garde du port.
\subsection[{§ 23.}]{ \textsc{§ 23.} }
\noindent Quand il eut pris ces mesures, profitant d’un temps favorable, il leva l’ancre aux environs de la troisième veille ; les cavaliers devaient gagner l’autre port, s’y embarquer et le suivre. Tandis que ceux-ci procédaient avec un peu trop de lenteur, César, vers la quatrième heure du jour, atteignit la Bretagne avec ses premiers navires, et là il vit, rangées sur toutes les collines, les troupes de l’ennemi en armes. La configuration des lieux était telle, la mer était si étroitement resserrée entre les hauteurs, que de celles-ci on pouvait lancer des projectiles sur le rivage. Jugeant un pareil lieu tout à fait impropre à un débarquement, César attendit à l’ancre jusqu’à la neuvième heure que le reste de sa flotte fût arrivé. Cependant, ayant convoqué les légats et les tribuns, il leur expliqua ce qu’il avait appris de Volusénus et quels étaient ses desseins ; il leur recommanda que, conformément aux exigences de la guerre, et surtout de la guerre navale où les choses vont vite et changent sans cesse, toutes les manœuvres fussent exécutées au commandement et dans l’instant voulu. Quand il les eut renvoyés, il se trouva avoir en même temps un bon vent et une marée propice ; il donna le signal, on leva l’ancre et, après avoir parcouru environ sept milles, il arriva à une plage découverte où il put ranger ses vaisseaux.
\subsection[{§ 24.}]{ \textsc{§ 24.} }
\noindent Mais les Barbares, quand ils s’étaient rendus compte de nos intentions, avaient envoyé en avant leur cavalerie et leurs chars – moyen de combat qui leur est familiers –, le reste de leurs troupes avait suivi de près, et ils s’opposaient à notre débarquement. Ce qui rendait notre entreprise très difficile, c’est que nos vaisseaux, en raison de leurs dimensions, étaient forcés de s’arrêter en pleine eau, et que nos soldats, ignorant la nature des lieux, ayant les mains embarrassées, ployant sous le poids considérable de leurs armes, devaient en même temps sauter à bas des navires, lutter pour n’être pas renversés par les vagues, et se battre avec les ennemis, tandis que ceux-ci, restant à pied sec ou n’avançant que peu dans l’eau, ayant la liberté de leurs membres, connaissant les lieux à merveille, lançaient leurs traits avec assurance et poussaient contre nous leurs chevaux, qui avaient l’habitude de la mer. Tout cela troublait nos hommes, qui, d’ailleurs, n’avaient aucune expérience de ce genre de combat : aussi n’avaient-ils pas le même mordant et le même entrain qu’habituellement, lorsqu’ils combattaient sur terre.
\subsection[{§ 25.}]{ \textsc{§ 25.} }
\noindent Quand César vit cela, il ordonna que les vaisseaux longs, dont l’aspect était plus nouveau pour les Barbares et qui manœuvraient avec plus de souplesse, s’éloignassent un peu des transports et, faisant force de rames, allassent se mettre en ligne sur le flanc droit de l’ennemi ; de là, mettant en action frondes, arcs, balistes, ils devaient refouler l’ennemi. Cette manœuvre nous fut d’une grande utilité. En effet, troublés par la forme de nos navires, par le mouvement des rames, par ce que nos machines leur offraient de singulier, les Barbares s’arrêtèrent, puis reculèrent légèrement. Mais nos soldats hésitaient surtout à cause de la profondeur de l’eau ; alors celui qui portait l’aigle de la dixième légion, après avoir demandé aux dieux que son initiative fût favorable à la légion : « Camarades, s’écria-t-il d’une voix forte, sautez à la mer, si vous ne voulez pas livrer votre aigle à l’ennemi moi, du moins, j’aurai fait mon devoir envers Rome et envers notre général. » A ces mots, il s’élança du navire et se dirigea vers l’ennemi, l’aigle en mains. Alors les nôtres, s’exhortant mutuellement à ne pas souffrir un tel déshonneur, sautèrent ensemble hors du vaisseau. Et quand ceux des navires voisins les aperçurent, ils les suivirent et s’avancèrent vers l’ennemi.
\subsection[{§ 26.}]{ \textsc{§ 26.} }
\noindent On combattit avec acharnement des deux côtés. Cependant, comme les nôtres ne pouvaient ni garder leurs rangs, ni prendre pied solidement, ni suivre leurs enseignes, et que chacun au sortir de son navire se rangeait sous les enseignes qu’il rencontrait, il en résultait un grand désordre ; les ennemis, eux, qui connaissaient tous les bas-fonds, dès qu’ils apercevaient quelques isolés sortant d’un navire, profitant de leur embarras, poussaient leurs chevaux sur eux et les attaquaient ; ils entouraient en force les petits groupes, tandis que d’autres, sur notre droite, prenaient de flanc l’ensemble sous une grêle de traits. Voyant cela, César fit emplir de soldats les chaloupes des vaisseaux longs et les bateaux de reconnaissance, et il envoyait des renforts à ceux qu’il voyait en danger. Dès que nos soldats purent se reformer sur le rivage, et comme tous avaient rejoint, ils chargèrent l’ennemi et le mirent en déroute ; mais ils ne purent le poursuivre bien loin, parce que la cavalerie n’avait pu rester dans la bonne direction et atteindre l’île. Ce fut tout ce qui manqua à la fortune accoutumée de César.
\subsection[{§ 27.}]{ \textsc{§ 27.} }
\noindent Les ennemis, après leur défaite, dès qu’ils eurent cessé de fuir, s’empressèrent d’envoyer une ambassade à César pour lui demander la paix : ils promirent de donner des otages et d’exécuter ce qu’il commanderait. En même temps qu’elle, vint Commios l’Atrébate, dont j’ai dit plus haut que César l’avait envoyé avant lui en Bretagne. Comme il venait de débarquer et faisait connaître aux Bretons, en porte-parole de César, son message, ils s’étaient emparés de lui et l’avaient chargé de chaînes ; après le combat, ils le renvoyèrent, et en demandant la paix ils rejetèrent sur la foule la responsabilité de cet attentat, en le priant de pardonner une faute due à l’ignorance. César, après leur avoir reproché de lui avoir fait la guerre sans motif, alors qu’ils lui avaient spontanément envoyé des députés sur le continent pour solliciter la paix, déclara qu’il pardonnait à leur ignorance et demanda des otages ; ils en fournirent une partie sur-le-champ ; les autres, qu’ils devaient faire venir d’assez loin, seraient livrés dans peu de jours. En attendant, ils renvoyèrent leurs soldats aux champs et les chefs commencèrent de venir de toutes parts pour recommander à César leurs intérêts et ceux de leurs cités.
\subsection[{§ 28.}]{ \textsc{§ 28.} }
\noindent La paix étant ainsi assurée, quatre jours après que nous étions arrivés en Bretagne, les dix-huit navires dont il a été question plus haut, qui avaient embarqué la cavalerie, quittèrent le port du nord par vent léger. Ils approchaient de l’île et on les voyait de notre camp, lorsque soudain s’éleva une tempête d’une telle violence qu’aucun d’eux ne put plus tenir sa route, et que les uns furent ramenés à leur point de départ, tandis que les autres étaient fort dangereusement entraînés vers l’extrémité sud-ouest de l’île ; ils jetèrent l’ancre malgré la tempête, mais menacés d’être submergés par les vagues, ils durent piquer vers le large et s’enfoncer dans la nuit ; ils finirent par atteindre le continent.
\subsection[{§ 29.}]{ \textsc{§ 29.} }
\noindent Le sort voulut que cette même nuit ce fût pleine lune, moment où les marées de l’océan sont les plus hautes ; et les nôtres ignoraient la chose. Aussi les vaisseaux longs, dont César s’était servi pour transporter son infanterie et qu’il avait tirés au sec, se trouvèrent-ils remplis d’eau, cependant que les vaisseaux de transport, qu’on avait mis à l’ancre, étaient maltraités par la tempête sans qu’on eût aucun moyen d’y faire la manœuvre ou de leur porter secours. Un très grand nombre de navires furent brisés ; les autres, ayant perdu câbles, ancres et autres agrès, étaient hors d’usage : cette situation, comme il était inévitable, émut fort toute l’armée. Il n’y avait pas, en effet, d’autres navires qui pussent nous ramener, on n’avait rien de ce qu’il fallait pour réparer la flotte, enfin, chacun pensant qu’on devait hiverner en Gaule, on n’avait pas fait de provisions de blé pour passer l’hiver dans cette île.
\subsection[{§ 30.}]{ \textsc{§ 30.} }
\noindent Quand ils surent notre embarras, les chefs bretons qui étaient venus trouver César après la bataille se concertèrent : voyant que les Romains n’avaient ni cavalerie, ni bateaux, ni blé, se rendant compte du petit nombre de nos effectifs d’après les dimensions de notre camp, qui était d’autant plus restreint que César avait emmené ses légions sans bagages, il leur parut que le meilleur parti à prendre était de se révolter, de nous empêcher de nous procurer du blé et des vivres, et de traîner les choses jusqu’à l’hiver : quand ils nous auraient vaincus, ou qu’ils nous auraient interdit le retour, personne, pensaient-ils, n’oserait plus passer en Bretagne pour y porter la guerre. Ayant donc renoué leur coalition, ils se mirent à quitter peu à peu le camp et à rappeler en secret les hommes qu’ils avaient renvoyés aux champs.
\subsection[{§ 31.}]{ \textsc{§ 31.} }
\noindent César n’était pas encore au courant de leurs projets ; mais, après ce qui était arrivé à sa flotte, et en voyant les Bretons interrompre leurs livraisons d’otages, il se doutait de ce qui allait se produire. Aussi prenait-il des précautions pour parer à tout événement. Chaque jour il faisait apporter du blé de la campagne dans le camp ; le bois et le bronze des vaisseaux qui avaient le plus souffert étaient employés à réparer les autres, et il faisait venir du continent ce qu’il fallait pour ces travaux. De la sorte, les soldats s’y employant avec la plus grande ardeur, César arriva, avec une perte de douze navires, à ce que les autres fussent en état de bien naviguer.
\subsection[{§ 32.}]{ \textsc{§ 32.} }
\noindent Sur ces entrefaites, comme, selon l’habitude, une légion – c’était la septième – avait été envoyée au blé, et sans que rien jusque-là se fût produit qui pût faire craindre des hostilités, une partie des Bretons restant aux champs, d’autres même fréquentant notre camp, les gardes qui étaient en avant des portes annoncèrent à César qu’un nuage de poussière d’une grosseur insolite se voyait du côté où était partie la légion. César – et il ne se trompait point – soupçonna quelque surprise des Barbares il prit avec lui, pour aller de ce côté, les cohortes qui étaient aux postes de garde, et ordonna que deux de celles qui restaient en fissent la relève, tandis que les autres s’armeraient et le suivraient sans retard. S'étant avancé à quelque distance du camp, il vit que les siens étaient pressés par l’ennemi et se défendaient péniblement la légion formait une masse compacte sur laquelle les traits pleuvaient de toutes parts. Comme, en effet, le blé avait été coupé partout, sauf en un endroit, l’ennemi, soupçonnant que nous y viendrions, s’était caché la nuit dans des bois ; puis, tandis que nos hommes étaient dispersés, sans armes, et occupés à moissonner, ils les avaient assaillis soudainement, en avaient tué quelques-uns, et avaient jeté le trouble chez les autres qui n’arrivaient pas à se former régulièrement ; en même temps, la cavalerie et les chars les avaient enveloppés.
\subsection[{§ 33.}]{ \textsc{§ 33.} }
\noindent Voici comment ils combattent de ces chars. Ils commencent par courir de tous côtés en tirant la peur qu’inspirent leurs chevaux et le fracas des roues suffisent en général à jeter le désordre dans les rangs ; puis, ayant pénétré entre les escadrons, ils sautent à bas de leurs chars et combattent à pied. Cependant les conducteurs sortent peu à peu de la mêlée et placent leurs chars de telle manière que, si les combattants sont pressés par le nombre, ils puissent aisément se replier sur eux. Ils réunissent ainsi dans les combats la mobilité du cavalier à la solidité du fantassin ; leur entraînement et leurs exercices quotidiens leur permettent, quand leurs chevaux sont lancés au galop sur une pente très raide, de les retenir, de pouvoir rapidement les prendre en mains et les faire tourner ; ils ont aussi l’habitude de courir sur le timon, de se tenir ferme sur le joug, et de là, de rentrer dans leurs chars en un instant.
\subsection[{§ 34.}]{ \textsc{§ 34.} }
\noindent Cette tactique inattendue troublait nos soldats, et César vint fort à propos les secourir, car à son arrivée les ennemis s’arrêtèrent, et les nôtres se ressaisirent. Ayant obtenu ce résultat, César jugea l’occasion peu favorable pour attaquer et livrer bataille il resta sur place, et, après une brève attente, ramena ses légions au camp. Pendant que ces événements se déroulaient, accaparant l’attention de toutes nos troupes, les Bretons qui étaient restés dans la campagne se retirèrent. Ce fut ensuite pendant plusieurs jours une série ininterrompue de mauvais temps, qui nous retint au camp et empêcha l’ennemi d’attaquer. Dans cet intervalle, les Barbares envoyèrent de tous côtés des messagers, faisant savoir combien nous étions peu nombreux, expliquant quelle occasion s’offrait de faire du butin et de conquérir pour toujours l’indépendance, si les Romains étaient chassés de leur camp. Cela amena la concentration rapide de grandes forces d’infanterie et de cavalerie, qui se dirigèrent vers notre camp.
\subsection[{§ 35.}]{ \textsc{§ 35.} }
\noindent César prévoyait qu’il arriverait ce qui était arrivé précédemment : si les ennemis étaient repoussés, l’avantage de la rapidité leur permettrait de nous échapper ; néanmoins, disposant d’environ trente cavaliers, que Commios l’Atrébate, dont on a parlé plus haut, avait emmenés avec lui, il rangea ses légions en bataille devant le camp. Le combat s’engagea, et presque aussitôt les ennemis cédèrent devant notre attaque et prirent la fuite. Nos soldats les poursuivirent aussi loin qu’ils purent courir et que leurs forces le leur permirent, en tuèrent un grand nombre, puis rentrèrent au camp après avoir incendié toutes les maisons sur une vaste étendue.
\subsection[{§ 36.}]{ \textsc{§ 36.} }
\noindent Le même jour, des députés vinrent trouver César de la part des ennemis pour lui demander la paix. César doubla le nombre d’otages qu’il avait exigés et prescrivit qu’on les lui amenât sur le continent, car il ne voulait pas, l’équinoxe étant proche, s’exposer aux dangers de l’hiver avec des vaisseaux en mauvais état. Profitant d’un vent favorable, il leva l’ancre peu après minuit ; sa flotte atteignit intacte le continent ; mais deux navires de transport ne purent toucher aux mêmes ports que les autres, et furent poussés un peu plus bas.
\subsection[{§ 37.}]{ \textsc{§ 37.} }
\noindent Ces navires débarquèrent environ trois cents soldats, qui se dirigèrent vers le camp romain ; mais les Morins, que César, en partant pour la Bretagne, avait laissés pacifiés, cédant à l’appât du butin, les entourèrent avec un nombre d’hommes d’abord peu considérable, et les invitèrent à déposer les armes, s’ils ne voulaient pas être massacrés. Comme ceux-ci, ayant formé le cercle, se défendaient, ils ne tardèrent pas à avoir autour d’eux quelque six mille hommes, accourus aux cris. Quand il apprit la chose, César envoya au secours des siens toute la cavalerie qui était au camp. Pendant ce temps, les nôtres tinrent tête à l’attaque : plus de quatre heures durant, ils combattirent avec un grand courage et tuèrent beaucoup d’adversaires tout en n’ayant que peu de blessés. Quand notre cavalerie apparut, les ennemis jetèrent leurs armes et prirent la fuite : on en fit un grand massacre.
\subsection[{§ 38.}]{ \textsc{§ 38.} }
\noindent César, le lendemain, envoya son légat Titus Labiénus, avec les légions qu’il avait ramenées de Bretagne, chez les Morins qui s’étaient révoltés. Ceux-ci, les marais étant à sec, ne pouvaient s’y réfugier comme ils l’avaient fait l’année précédente ; ils tombèrent presque tous entre les mains de Labiénus. Par contre, les légats Quintus Titurius et Lucius Cotta, qui avaient conduit les légions sur le territoire des Ménages, après avoir ravagé tout leurs champs, coupé leur blé, incendié leurs maisons, durent revenir auprès de César, parce que les Ménapes s’étaient tous cachés dans de très épaisses forêts. César fit hiverner toutes ses légions chez les Belges. Il n’y eut que deux cités de Bretagne qui lui envoyèrent là leurs otages ; les autres négligèrent leurs promesses. Ces campagnes terminées, le Sénat, à la suite du rapport de César, décréta vingt jours d’actions de grâces.
 \section[{Livre V}]{Livre V}\renewcommand{\leftmark}{Livre V}

\subsection[{§ 1.}]{ \textsc{§ 1.} }
\noindent Sous le consulat de Lucius Domitius et d’Appius Claudius, César, quittant ses quartiers d’hiver pour aller en Italie, comme il avait accoutumé de faire chaque année, ordonne à ses légats, qu’il avait mis à la tête des légions, de faire construire pendant l’hiver le plus grand nombre de vaisseaux possible et de faire réparer les anciens. Il indique quelles doivent en être les dimensions et la forme. Pour la rapidité du chargement et la facilité de la mise à sec, il les fait un peu plus bas que ceux dont nous avons coutume d’user sur notre mer, d’autant qu’il avait observé que les vagues, par suite du flux et du reflux, étaient moins hautes ; à cause des charges et du grand nombre de chevaux et bêtes de somme qu’ils devaient transporter, il leur donne une largeur un peu supérieure à celle des bâtiments dont nous nous servons sur les autres mers. Il ordonne qu’ils soient tous du type léger, à voiles et à rames, disposition que facilite beaucoup leur faible hauteur. Ce qui est nécessaire à leur armement, il le fait venir d’Espagne. Puis, ayant achevé de tenir ses assises dans la Gaule citérieure, il part pour l’Illyricum, sur la nouvelle que les Pirustes désolaient par leurs incursions les confins de la province. Dès son arrivée, il ordonne aux cités de lever des troupes et leur fixe un point de rassemblement. Quand ils apprennent la chose, les Pirustes lui envoient des députés pour lui dire que la nation n’est pour rien dans ce qui s’est passé, et se déclarent prêts à fournir toutes les satisfactions qu’il exigera. Après les avoir entendus, César leur ordonne de lui livrer des otages et fixe le jour de la remise : en cas de manquement, ce sera la guerre. On les amène au jour dit, selon ses ordres ; il nomme des arbitres pour estimer les dommages subis par chaque cité et en fixer la réparation.
\subsection[{§ 2.}]{ \textsc{§ 2.} }
\noindent Ayant réglé cette affaire et tenu ses assises, il retourne dans la Gaule citérieure, et de là, part pour l’armée. Dès son arrivée, il visite tous les quartiers d’hiver et trouve tout équipés, grâce à l’activité singulière des troupes, alors qu’on manquait de tout, environ six cents navires du type que nous avons décrit plus haut, et vingt-huit vaisseaux longs : il ne manquait pas grand-chose pour qu’on pût les mettre à la mer sous peu de jours. Il félicite les soldats et ceux qui ont dirigé l’entreprise, explique ses intentions, et ordonne que tous se concentrent à Portus Itius, d’où il savait que la traversée était la plus aisée, et d’où il n’y a que trente milles environ du continent en Bretagne ; il laissa les effectifs qu’il jugea nécessaires pour cette opération. Quant à lui, prenant quatre légions sans bagages et huit cents cavaliers, il se rend chez les Trévires, parce qu’ils s’abstenaient de venir aux assemblées, ne reconnaissaient pas son autorité et essayaient, disait-on, d’attirer les Germains trans-rhénans.
\subsection[{§ 3.}]{ \textsc{§ 3.} }
\noindent Ce peuple a la plus forte cavalerie de toute la Gaule, une infanterie nombreuse et il touche, comme nous l’avons dit, au Rhin. Deux hommes s’y disputaient le pouvoir Indutiomaros et Cingétorix. Celui-ci, dès qu’on sut l’approche de César et de ses légions, vint le trouver, donna l’assurance que lui et les siens resteraient dans le devoir et ne trahiraient pas l’amitié du peuple romain, et l’instruisit de ce qui se passait chez les Trévires. Indutiomaros, au contraire, se mit à lever de la cavalerie et de l’infanterie et à préparer la guerre, cachant dans la forêt des Ardennes, qui s’étend sur une immense étendue, au milieu du territoire des Trévires, depuis le Rhin jusqu’aux frontières des Rèmes, ceux à qui leur âge ne permettait pas de porter les armes. Puis quand il voit qu’un assez grand nombre de chefs trévires, cédant à leur amitié pour Cingétorix et à la frayeur que leur causait l’arrivée de nos troupes, se rendaient auprès de César et, ne pouvant rien pour la nation, le sollicitaient pour eux-mêmes, il craint d’être abandonné de tous et envoie des députés à César : « S'il n’avait pas voulu quitter les siens et venir le trouver, c’était pour pouvoir mieux maintenir la cité dans le devoir, car on pouvait craindre que, si tous les nobles s’en allaient, le peuple, dans son ignorance, ne se laissât entraîner ; la cité lui obéissait donc, et si César y consentait, il viendrait dans son camp placer sous sa protection sa personne et la cité. »
\subsection[{§ 4.}]{ \textsc{§ 4.} }
\noindent César n’ignorait pas ce qui lui dictait ces paroles et ce qui le détournait de ses premiers desseins ; pourtant, ne voulant pas être contraint de passer tout l’été chez les Trévires quand tout était prêt pour la guerre de Bretagne, il ordonna à Indutiomaros de venir avec deux cents otages. Quand celui-ci les eut amenés, et parmi eux son fils et tous ses proches que César avait réclamés nommément, il le rassura et l’exhorta à rester dans le devoir ; mais il n’en fit pas moins comparaître les chefs trévires et les rallia un à un à Cingétorix : ce n’était pas seulement une juste récompense de ses services ; César voyait aussi un grand intérêt à fortifier autant que possible le crédit d’un homme en qui il avait trouvé un exceptionnel dévouement. Ce fut pour Indutiomaros un coup sensible, que de se voir mis en moindre faveur auprès des siens ; et lui qui déjà nous était hostile, il en conçut un ressentiment qui exaspéra sa haine.
\subsection[{§ 5.}]{ \textsc{§ 5.} }
\noindent Ces affaires une fois réglées, César se rend à Portus Itius avec ses légions. Là, il apprend que soixante navires, qui avaient été construits chez les Meldes, ont été rejetés par la tempête, et, incapables de tenir leur route, ont dû revenir à leur point de départ ; quant aux autres, il les trouve prêts à naviguer et pourvus de tout le nécessaire. La cavalerie de toute la Gaule se rassemble là, forte de quatre mille chevaux, avec les chefs de toutes les nations ; César avait résolu de n’en laisser en Gaule qu’un tout petit nombre, ceux dont il était sûr, et d’emmener les autres comme otages, parce qu’il craignait un soulèvement de la Gaule en son absence.
\subsection[{§ 6.}]{ \textsc{§ 6.} }
\noindent Au nombre de ces chefs était l’Héduen Dumnorix, dont nous avons déjà parlé. Il était des premiers que César eût pensé à garder avec lui, car il savait son goût de l’aventure, sa soif de domination, sa hardiesse et l’autorité dont il jouissait parmi les Gaulois. De plus, Dumnorix avait dit dans une assemblée des Héduens que César lui offrait d’être roi de ce peuple, propos qui les inquiétait fort, sans qu’ils osassent députer à César pour dire qu’ils n’acceptaient pas son projet ou prier qu’il y renonçât. César avait connu le trait par ses hôtes. Dumnorix commença par user de toutes sortes de prières pour obtenir qu’on le laissât en Gaule : « Il n’avait pas l’habitude de naviguer et redoutait la mer ; il était retenu par des devoirs religieux. » Quand il vit qu’il se heurtait à un refus catégorique, n’ayant plus aucun espoir de succès, il se mit à intriguer auprès des chefs gaulois, leur faisant peur, les prenant chacun à part et les exhortant à rester sur le continent : « Ce n’était pas sans raison, disait-il, qu’on enlevait à la Gaule toute sa noblesse : le projet de César, qui n’osait pas la massacrer sous les yeux des Gaulois, était de la transporter en Bretagne pour l’y faire périr. » Aux autres, Dumnorix jurait et faisait jurer qu’ils exécuteraient d’un commun accord ce qu’ils croiraient utile aux intérêts de la Gaule. Bien des gens dénonçaient ces menées à César.
\subsection[{§ 7.}]{ \textsc{§ 7.} }
\noindent Lorsqu’il connut cette situation, sa pensée fut la suivante : en raison du rang où il plaçait la nation héduenne, tout tenter pour retenir Dumnorix et le détourner de ses desseins ; mais comme, d’autre part, l’égarement du personnage ne faisait, visiblement, que croître, prendre ses précautions pour qu’il ne pût être un danger ni pour lui, ni pour l’État. En conséquence, ayant été retenu au port environ vingt-cinq jours par le chorus, vent qui souffle le plus souvent, en toute saison, sur ces côtes, il s’appliqua à garder Dumnorix dans le devoir, sans pour cela négliger de se tenir au courant de tous les plans qu’il formait ; enfin, profitant d’un vent favorable, il donne aux fantassins et aux cavaliers l’ordre d’embarquer. Mais, tandis que cette opération occupait l’attention de tous, Dumnorix quitta le camp, à l’insu de César, avec la cavalerie héduenne, et prit le chemin de son pays. Quand il apprend la chose, César suspend le départ et, toute affaire cessante, envoie une grande partie de la cavalerie à sa poursuite, avec ordre de le ramener ; s’il résiste, s’il refuse d’obéir, il commande qu’on le tue, car il n’attendait rien de sensé, loin de sa présence, d’un homme qui lui avait désobéi en face. Dumnorix, sommé de revenir, résiste, met l’épée à la main, supplie les siens de faire leur devoir, répétant à grands cris qu’il est libre et appartient à un peuple libre. Conformément aux ordres, on l’entoure et on le tue ; quant aux cavaliers héduens, tous reviennent auprès de César.
\subsection[{§ 8.}]{ \textsc{§ 8.} }
\noindent Cette affaire terminée, César laissa Labiénus sur le continent avec trois légions et deux mille cavaliers, pour garder les ports et pourvoir au blé, pour surveiller les événements de Gaule et prendre les décisions que comporteraient les circonstances ; lui-même, avec cinq légions et autant de cavaliers qu’il en avait laissés sur le continent, il leva l’ancre au coucher du soleil. Il fut d’abord poussé par un léger vent du sud-ouest ; mais vers minuit le vent tomba, il ne put tenir sa route, et, emporté assez loin par le courant de marée, quand le jour parut, il aperçut sur sa gauche la Bretagne qu’il avait manquée. Alors il suivit le courant qui portait maintenant en sens contraire et fit force de rames pour aborder à cet endroit de l’île que l’été précédent il avait reconnu pour très favorable à un débarquement. En cette occasion nos soldats furent au-dessus de tout éloge avec des navires de transport, et lourdement chargés, ils purent, en ramant sans relâche, aller aussi vite que les vaisseaux longs. On atteignit la Bretagne, avec toute la flotte, vers midi, sans voir l’ennemi sur ce point ; comme César le sut plus tard par des prisonniers, des groupes nombreux s’y étaient assemblés et, effrayés à la vue de tant de vaisseaux – avec ceux de l’année précédente, et ceux que des particuliers avaient construits pour leur usage, c’était plus de huit cents navires qui avaient paru à la fois, – ils avaient quitté le rivage pour aller se cacher sur les hauteurs.
\subsection[{§ 9.}]{ \textsc{§ 9.} }
\noindent César fit débarquer ses troupes et choisit un emplacement convenable pour son camp ; lorsqu’il sut par des prisonniers où s’était arrêté l’ennemi, laissant près de la mer dix cohortes et trois cents cavaliers pour la garde des navires, avant la fin de la troisième veille, il marcha à l’ennemi ; il craignait d’autant moins pour sa flotte qu’il la laissait à l’ancre sur une plage douce et tout unie ; il donna le commandement du détachement et de la flotte à Quintus Atrius. Pour lui, une marche de nuit d’environ douze milles l’amena en vue de l’ennemi. Celui-ci s’avança vers le fleuve avec sa cavalerie et ses chars et, d’une position dominante, essaya de nous interdire le passage et engagea la bataille. Repoussés par nos cavaliers, les Barbares se cachèrent dans les bois : ils trouvaient là une position remarquablement fortifiée par la nature et par l’art, qu’ils avaient aménagée antérieurement, sans doute pour quelque guerre entre eux : car on avait abattu un grand nombre d’arbres, et on s’en était servi pour obstruer tous les accès. Disséminés en tirailleurs, ils lançaient des traits de l’intérieur de la forêt et nous interdisaient l’entrée de leur forteresse. Mais les soldats de la septième légion, ayant formé la tortue et poussé une terrasse d’approche jusqu’au retranchement ennemi, prirent pied dans la place et les chassèrent de la forêt sans éprouver de pertes sensibles. César défendit qu’on les poursuivît plus loin, parce qu’il ne connaissait pas le pays et parce que, la journée étant déjà fort avancée, il voulait en consacrer la fin à la fortification du camp.
\subsection[{§ 10.}]{ \textsc{§ 10.} }
\noindent Le lendemain matin, il envoya fantassins et cavaliers en trois corps, à la poursuite de l’ennemi en fuite. Ils avaient fait un assez long chemin, et déjà on apercevait les derniers fuyards, quand des cavaliers envoyés par Quintus Atrius vinrent annoncer à César que, la nuit précédente, une très violente tempête s’était élevée, et que presque tous les vaisseaux avaient été désemparés et jetés à la côte, câbles et ancres ayant cédé et les matelots et les pilotes n’ayant pu soutenir la violence de l’ouragan les navires, heurtés les uns contre les autres, avaient beaucoup souffert.
\subsection[{§ 11.}]{ \textsc{§ 11.} }
\noindent A cette nouvelle, César ordonne qu’on rappelle légionnaires et cavaliers, qu’ils s’arrêtent et fassent demi-tour ; lui-même retourne aux navires ; ce que messagers et lettre lui avaient appris se confirme, en somme, à ses yeux : quarante navires étaient perdus, mais les autres paraissaient pouvoir être réparés, au prix d’un grand travail. Il choisit des ouvriers dans les légions et en fait venir d’autres du continent ; il écrit à Labiénus d’avoir à construire, avec les légions dont il dispose, le plus de navires possible. De son côté, bien que ce fût un grand travail, et qui dût coûter beaucoup de peine, il prit le parti, qui lui parut le meilleur, de tirer à sec toute la flotte et de l’enfermer avec le camp dans une fortification commune. Cette opération demanda environ dix jours d’un labeur que la nuit même n’interrompait pas. Une fois les navires mis à sec et le camp parfaitement fortifié, laissant pour garder la flotte les mêmes troupes que précédemment, il retourne à l’endroit qu’il avait quitté. Il y trouva des forces bretonnes déjà nombreuses qui s’étaient rassemblées là de toutes parts, sous les ordres de Cassivellaunos à qui, d’un commun accord, on avait confié tous pouvoirs pour la conduite de la guerre c’est un prince dont le territoire est séparé des États maritimes par un fleuve qu’on nomme la Tamise, à environ quatre-vingt milles de la mer. Il n’avait cessé jusque-là d’être en guerre avec les autres peuples ; mais l’effroi causé par notre arrivée avait déterminé les Bretons à lui donner le commandement suprême.
\subsection[{§ 12.}]{ \textsc{§ 12.} }
\noindent L'intérieur de la Bretagne est peuplé d’habitants qui se disent, en vertu d’une tradition orale, autochtones ; sur la côte vivent des peuplades qui étaient venues de Belgique pour piller et faire la guerre (presque toutes portent les noms des cités d’où elles sont issues) ; ces hommes, après la guerre, restèrent dans le pays et y devinrent colons. La population de l’île est extrêmement dense, les maisons s’y pressent, presque entièrement semblables à celles des Gaulois, le bétail abonde. Pour monnaie on se sert de cuivre, de pièces d’or ou de lingots de fer d’un poids déterminée. Le centre de l’île produit de l’étain, la région côtière du fer, mais en petite quantité ; le cuivre vient du dehors. Il y a des arbres de toute espèce, comme en Gaule, sauf le hêtre et le sapin. Le lièvre, la poule et l’oie sont à leurs yeux nourriture interdite ; ils en élèvent cependant, pour le plaisir. Le climat est plus tempéré que celui de la Gaule, les froids y étant moins rigoureux.
\subsection[{§ 13.}]{ \textsc{§ 13.} }
\noindent L'île a la forme d’un triangle, dont un côté fait face à la Gaule. Des deux angles de ce côté, l’un, vers le Cantium, où abordent à peu près tous les navires venant de Gaule, regarde l’orient ; l’autre, plus bas, est au midi. Ce côté se développe sur environ cinq cents milles. Le deuxième regarde l’Espagne et le couchant dans ces parages est l’Hibernie, qu’on estime deux fois plus petite que la Bretagne ; elle est à la même distance de la Bretagne que celle-ci de la Gaule. A mi-chemin est l’île qu’on appelle Mona ; il y a aussi, dit-on, plusieurs autres îles plus petites, voisines de la Bretagne, à propos desquelles certains auteurs affirment que la nuit y règne pendant trente jours de suite, au moment du solstice d’hiver. Pour nous, nos enquêtes ne nous ont rien révélé de semblable ; nous constations toutefois, par nos clepsydres, que les nuits étaient plus courtes que sur le continent. La longueur de ce côté du triangle, d’après l’opinion desdits auteurs, est de sept cents milles. Le troisième fait face au nord ; il n’y a aucune terre devant lui, sauf, à son extrémité, la Germanie. La longueur de cette côte est évaluée à huit cents milles. Ainsi l’ensemble de l’île a deux mille milles de tour.
\subsection[{§ 14.}]{ \textsc{§ 14.} }
\noindent De tous les habitants de la Bretagne, les plus civilisés, de beaucoup, sont ceux qui peuplent le Cantium, région tout entière maritime ; leurs mœurs ne diffèrent guère de celles des Gaulois. Ceux de l’intérieur, en général, ne sèment pas de blé ; ils vivent de lait et de viande, et sont vêtus de peaux. Mais c’est un usage commun à tous les Bretons de se teindre le corps au pastel, qui donne une couleur bleue, et cela rend leur aspect particulièrement terrible dans les combats. Ils portent de longues chevelures, et se rasent toutes les parties du corps à l’exception de la tête et de la lèvre supérieure. Leurs femmes sont en commun entre dix ou douze, particulièrement entre frères et entre pères et fils ; mais les enfants qui naissent de cette promiscuité sont réputés appartenir à celui qui a été le premier époux.
\subsection[{§ 15.}]{ \textsc{§ 15.} }
\noindent La cavalerie et les chars ennemis eurent un vif engagement avec notre cavalerie pendant que nous étions en marche ; néanmoins nous eûmes partout le dessus et les Bretons furent repoussés vers des forêts et des collines ; nous en tuâmes beaucoup, mais une poursuite trop ardente nous causa quelques pertes. Les ennemis attendirent un certain temps ; puis, tandis que nos soldats étaient sans méfiance et tout occupés à fortifier le camp, ils firent soudain irruption hors des forêts et, tombant sur ceux qui étaient de garde en avant du camp, livrèrent un violent combat ; César envoya en soutien deux cohortes, et il choisit les premières de deux légions ; elles prirent position en ne laissant entre elles qu’un très petit intervalle ; mais l’ennemi, profitant du trouble que causait chez les nôtres ce nouveau genre de combat, eut l’audace de se précipiter entre les deux cohortes et se dégagea sans pertes. Ce jour-là Quintus Labérius Durus, tribun militaire, est tué. L'envoi de nouvelles cohortes permet de repousser l’ennemi.
\subsection[{§ 16.}]{ \textsc{§ 16.} }
\noindent L'affaire, avec tous ses incidents, fut instructive : comme elle se déroulait sous les yeux de tous et devant le camp, on put se rendre compte que nos soldats, trop pesamment armés, ne pouvant poursuivre l’ennemi s’il se retirait et n’osant pas s’écarter de leurs enseignes, étaient mal préparés à combattre un tel adversaire ; que, d’autre part, notre cavalerie ne pouvait livrer bataille sans grave danger, parce que les ennemis cédaient le plus souvent par feinte, et, quand ils avaient attiré les nôtres à quelque distance des légions, sautaient à bas de leurs chars et livraient, à pied, un combat inégal. Et tant que le combat restait un combat de cavalerie, il se livrait dans de telles conditions que le danger était exactement le même pour le poursuivant et le poursuivi. Ajoutez à cela qu’ils ne combattaient jamais par masses, mais en ordre dispersé et à de très grands intervalles, et qu’ils avaient des postes de réserve échelonnés de distance en distance, ce qui leur permettait de s’offrir mutuellement, à tour de rôle, une ligne de retraite et de remplacer les combattants fatigués par d’autres dont les forces étaient intactes.
\subsection[{§ 17.}]{ \textsc{§ 17.} }
\noindent Le lendemain, les ennemis prirent position loin du camp, sur les collines : ils ne se montraient que par petits groupes, et attaquaient nos cavaliers avec moins de vigueur que la veille. Mais à midi, comme César avait envoyé au fourrage trois légions et toute la cavalerie sous le commandement du légat Caïus Trébonius, soudain, de toutes parts, ils se précipitèrent sur nos fourrageurs, et leur élan les porta jusqu’aux enseignes et aux légions. Les nôtres, contre-attaquant avec vigueur, les repoussèrent, et ils les suivirent sans relâche ; nos cavaliers, rassurés par ce soutien, puisqu’ils voyaient les légions derrière eux, les chargèrent impétueusement et, en faisant un grand massacre, ne leur laissèrent pas le moyen de se reformer ni de faire front ou de sauter à bas des chars. Cette déroute entraîna sur-le-champ la dispersion des auxiliaires qui étaient venus de tous côtés, et jamais plus dans la suite les ennemis ne nous livrèrent bataille avec l’ensemble de leurs forces.
\subsection[{§ 18.}]{ \textsc{§ 18.} }
\noindent César, mis au courant de leur plan, conduisit son armée vers la Tamise, pour la faire pénétrer dans le pays de Cassivellaunos ; ce fleuve n’est guéable qu’en un seul endroit, et non sans peine. Quand il y fut arrivé, il s’aperçut que sur l’autre rive d’importantes forces ennemies étaient rangées : En outre, la berge était défendue par des pieux pointus qui la bordaient, et d’autres pieux du même genre, que l’eau recouvrait, étaient enfoncés dans le lit du fleuve. Ayant su cela par des prisonniers et des déserteurs, César envoya en avant la cavalerie, et ordonna aux légions de marcher sans retard à sa suite. La rapidité et l’élan de nos troupes furent tels, bien que les hommes eussent la tête seule hors de l’eau, que l’ennemi ne put soutenir le choc des légions et de la cavalerie, et, abandonnant les bords du fleuve, prit la fuite.
\subsection[{§ 19.}]{ \textsc{§ 19.} }
\noindent Cassivellaunos, ainsi que nous l’avons dit plus haut, avait, désespérant de nous vaincre en bataille rangée, renvoyé le gros de ses troupes ; il n’avait gardé que quatre mille essédaires environ, avec lesquels il surveillait nos marches il se tenait à quelque distance de la route et se dissimulait dans des terrains peu praticables et couverts de bois : là où il savait que nous allions passer, il faisait évacuer les campagnes, poussant bêtes et hommes dans les forêts ; s’il arrivait que notre cavalerie se fût répandue un peu loin pour piller et dévaster, il lançait ses essédaires hors des bois par toutes les issues, routes ou sentiers, et livrait à nos cavaliers un combat si redoutable qu’il leur ôtait l’envie de s’aventurer à quelque distance. Il ne restait à César d’autre parti que d’interdire qu’on s’éloignât de la colonne d’infanterie, et de nuire à l’ennemi, en dévastant ses campagnes et en allumant des incendies, dans la mesure restreinte où la fatigue de la marche le permettait aux légionnaires.
\subsection[{§ 20.}]{ \textsc{§ 20.} }
\noindent Cependant les Trinovantes, qui étaient, ou peu s’en fallait, le peuple le plus puissant de ces contrées – Mandubracios, jeune homme de cette cité, s’était attaché à César et était venu le trouver sur le continent, son père avait été roi des Trinovantes, il avait été tué par Cassivellaunos, et le fils n’avait évité la mort qu’en s’enfuyant –, ce peuple donc envoie des députés à César, promettant de se soumettre et d’obéir à ses ordres ; ils lui demandent de protéger Mandubracios contre les violences de Cassivellaunos, et de l’envoyer dans sa cité pour qu’il y exerce le pouvoir souverain. César exige d’eux quarante otages et du blé pour l’année, et il leur envoie Mandubracios. Ils obéirent sans retard, envoyèrent le nombre d’otages demandé et du blé.
\subsection[{§ 21.}]{ \textsc{§ 21.} }
\noindent Voyant les Trinovantes protégés contre Cassivellaunos et mis à l’abri de toute violence de la part des troupes, les Cénimagnes, les Ségontiaques, les Ancalites, les Bibroques et les Casses députent à César et se soumettent. Par eux, il apprend qu’il n’est pas loin de la place forte de Cassivellaunos, qui est défendue par des forêts et des marécages et où se trouve un rassemblement assez considérable d’hommes et de bétail. Ce que les Bretons appellent place forte, c’est une forêt d’accès difficile, et qui leur sert de refuge habituel pour éviter les incursions de leurs ennemis. César y mène ses légions : il trouve un endroit singulièrement bien fortifié par la nature et par l’art ; pourtant, il l’attaque vivement de deux côtés. L'ennemi, après une courte résistance, céda devant l’impétuosité de notre assaut et s’enfuit par un autre côté de la place. On trouva là beaucoup de bétail, et bon nombre de fuyards furent pris ou tués.
\subsection[{§ 22.}]{ \textsc{§ 22.} }
\noindent Tandis que ces événements se déroulent à l’intérieur, Cassivellaunos envoie dans le Cantium, qui est, comme nous l’avons dit plus haut, une région maritime, et qui obéissait à quatre rois, Cingétorix, Carvilios, Taximagulos et Ségovax, des messagers portant à ces rois l’ordre d’attaquer à l’improviste, toutes forces réunies, le camp des vaisseaux. Quand ils s’y présentèrent, les nôtres firent une sortie et leur tuèrent beaucoup de monde, faisant même prisonnier un chef de haute naissance, Lugotorix ; ils rentrèrent ensuite au camp sans pertes. A la nouvelle de ce combat, Cassivellaunos, découragé par tant d’échecs, ému par la dévastation de son territoire, et surtout alarmé de la défection des cités, envoie des députés à César, par l’intermédiaire de l’Atrébate Commios, pour traiter de sa soumission. César, qui avait résolu de passer l’hiver sur le continent, à cause des mouvements soudains qui pouvaient se produire en Gaule, qui, d’autre part, voyait l’été déjà avancé et comprenait qu’il serait facile à l’ennemi de temporiser jusqu’à son terme, ordonne la livraison d’otages et fixe le tribut que la Bretagne devra payer chaque année au peuple Romains ; il interdit formellement à Cassivellaunos d’inquiéter ni Mandubracios ni les Trinovantes.
\subsection[{§ 23.}]{ \textsc{§ 23.} }
\noindent Ayant reçu les otages, il ramène son armée au bord de la mer, et trouve les navires réparés. Après les avoir mis à l’eau, comme il avait beaucoup de prisonniers et qu’un certain nombre de navires avaient péri dans la tempête, il décide de ramener son armée en deux convois. Et il se trouva que sur un grand nombre de navires, malgré tant de traversées, il n’y en eut pas un seul parmi ceux qui portaient des troupes, ni cette année, ni la précédente, qui n’accomplit normalement le voyage ; par contre, de ceux qui lui étaient renvoyés à vide du continent, qu’il s’agît des navires du premier convoi qui avaient débarqué leurs troupes ou des soixante navires que Labiénus avait fait construire après le départ de l’expédition, très peu touchèrent au but, et les autres furent presque tous rejetés à la côte. Après les avoir vainement attendus un certain temps, César voulant éviter que la saison lui interdît la mer, car on approchait de l’équinoxe, se vit forcé d’embarquer ses troupes plus à l’étroit ; survint un grand calme, et, levant l’ancre au début de la deuxième veille, il atteignit la terre au lever du jour, avec tous ses vaisseaux intacts.
\subsection[{§ 24.}]{ \textsc{§ 24.} }
\noindent Il fit mettre les navires à sec et tint l’assemblée des Gaulois à Samarobriva ; comme cette année la récolte de blé, en raison de la sécheresse, était maigre en Gaule, il fut contraint d’organiser l’hivernage de ses troupes autrement que les années précédentes, en distribuant les légions dans un plus grand nombre de cités. Il en envoya une chez les Morins, sous le commandement du légat Laïus Fabius ; une autre chez les Nerviens avec Quintus Cicéron, une troisième chez les Esuvii avec Lucius Roscius ; une quatrième reçut l’ordre d’hiverner chez les Rèmes, à la frontière des Trévires, avec Titus Labiénus ; il en plaça trois chez les Belges, sous les ordres du questeur Marcus Crassus, des légats Lucius Munatius Plancus et Laïus Trébonius. Il envoya une légion, levée en dernier lieu, dans la Transpadane, et cinq cohortes chez les Eburons, dont la plus grande partie habite entre la Meuse et le Rhin, et qui étaient gouvernés par Ambiorix et Catuvolcos. Ces troupes furent placées sous les ordres des légats Quintus Titurius Sabinus et Lucius Aurunculéius Cotta. Semblable distribution des légions devait, pensait-il, lui permettre de remédier très aisément à la pénurie de blé. Et, néanmoins, les quartiers de toutes ces légions, sauf celle que Lucius Roscius avait été chargé de conduire dans une région tout à fait pacifiée et très tranquille, n’étaient pas à plus de cent mille pas les uns des autres. César résolut d’ailleurs de rester en Gaule jusqu’à ce qu’il sût les légions en place et les camps d’hiver fortifiés.
\subsection[{§ 25.}]{ \textsc{§ 25.} }
\noindent Il y avait chez les Carnutes un homme de haute naissance, Tasgétios, dont les ancêtres avaient été rois dans leur cité. César, pour récompenser sa valeur et son dévouement, car dans toutes les guerres il avait trouvé chez lui un concours singulièrement actif, avait rendu à cet homme le rang de ses aïeux. Il était, cette année-là, dans la troisième année de son règne, quand ses ennemis secrètement l’assassinèrent ; plusieurs de leurs concitoyens les avaient d’ailleurs encouragés publiquement. On apprend la chose à César. Craignant, en raison du nombre des coupables, que leur influence n’amenât la défection de la cité, il fait partir en hâte Lucius Plancus, avec sa légion, de Belgique chez les Carnutes, avec ordre d’hiverner là, d’arrêter ceux qu’il savait responsables du meurtre de Tasgétios et de les lui envoyer. Sur ces entrefaites, tous ceux à qui il avait confié les légions lui firent savoir qu’on était arrivé dans les quartiers d’hiver et que les fortifications étaient faites.
\subsection[{§ 26.}]{ \textsc{§ 26.} }
\noindent Il y avait environ quinze jours que les troupes hivernaient, quand éclata une révolte soudaine, excitée par Ambiorix et Catuvolcos ; ces rois étaient venus à la frontière de leur pays se mettre à la disposition de Sabinus et de Cotta et avaient fait porter du blé à leur quartier d’hiver, quand des messages du Trévire Indutiomaros les déterminèrent à appeler leurs sujets aux armes ; aussitôt ils attaquèrent nos corvées de bois et vinrent en grandes forces assiéger le camp. Mais les nôtres s’armèrent sans retard et montèrent au retranchement, cependant que les cavaliers espagnols, sortant par une des portes, livraient un combat de cavalerie où ils eurent l’avantage ; les ennemis, voyant l’entreprise manquée, retirèrent leurs troupes ; puis, à grands cris, selon leur coutume, ils demandèrent que quelqu’un des nôtres s’avançât pour des pourparlers ; ils avaient à nous faire certaines communications qui n’avaient pas moins d’intérêt pour nous que pour eux et qui étaient de nature, pensaient-ils, à apaiser le conflit.
\subsection[{§ 27.}]{ \textsc{§ 27.} }
\noindent On leur envoie pour cette entrevue Caïus Arpinéius, chevalier romain, ami de Quintus Titurius, et un certain Quintus Junius, Espagnol, qui déjà avait eu plusieurs missions de César auprès d’Ambiorix. Celui-ci leur parla à peu près en ces termes : « Il reconnaissait qu’il avait envers César de grandes obligations à cause des bienfaits qu’il avait reçus de lui : c’était grâce à lui qu’il avait été délivré du tribut qu’il payait régulièrement aux Atuatuques, ses voisins, et César lui avait rendu son fils et son neveu, qui, étant au nombre des otages envoyés aux Atuatuques, avaient été traités par eux en esclaves et chargés de chaînes. En ce qui concerne l’attaque du camp, il a agi contre son avis et contre sa volonté, il a été contraint par son peuple, car la nature de son pouvoir ne le soumet pas moins à la multitude qu’elle ne la soumet à lui. Et si la cité a pris les armes, c’est qu’elle n’a pu opposer de résistance à la soudaine conjuration des Gaulois. Sa faiblesse est une preuve aisée de ce qu’il avance car il n’est pas assez novice pour croire qu’il puisse vaincre avec ses seules forces le peuple romain. Mais il s’agit d’un dessein commun à toute la Gaule tous les quartiers d’hiver de César doivent être attaqués ce jour même, afin qu’une légion ne puisse porter secours à l’autre. Des Gaulois n’auraient pu facilement dire non à d’autres Gaulois, surtout quand le but qu’on les voyait se proposer était la reconquête de la liberté commune. Puisqu’il avait répondu à leur appel, payant ainsi sa dette à sa patrie, il songeait maintenant au devoir de reconnaissance auquel l’obligeaient les bienfaits de César, et il avertissait Titurius, il le suppliait, au nom des liens d’hospitalité qui l’unissaient à lui, de pourvoir à son salut et à celui de ses soldats. Une troupe nombreuse de mercenaires germains avait passé le Rhin : elle serait là dans deux jours. A eux de voir s’ils veulent, avant que les peuples voisins s’en aperçoivent, faire sortir leurs troupes du camp et les conduire, soit auprès de Cicéron, soit auprès de Labiénus, qui sont l’un à environ cinquante milles, l’autre un peu plus loin. Pour lui, il promet, et sous serment, qu’il leur donnera libre passage sur son territoire. En agissant ainsi, il sert son pays, puisqu’il le débarrasse du cantonnement des troupes, et il reconnaît les bienfaits de César. » Après ce discours, Ambiorix se retire.
\subsection[{§ 28.}]{ \textsc{§ 28.} }
\noindent Arpinéius et Junius rapportent aux légats ce qu’ils viennent d’entendre. La nouvelle les surprend, les trouble ; bien que ce fussent propos d’un ennemi, ils ne pensaient pas devoir les négliger ; ce qui les frappait le plus, c’est qu’il n’était guère croyable qu’une cité obscure et peu puissante comme celle des Eburons eût osé de son propre chef faire la guerre au peuple romain. Ils portent donc l’affaire devant le conseil une vive discussion s’élève. Lucius Aurunculéius, un grand nombre de tribuns et les centurions de la première cohorte étaient d’avis qu’il ne fallait rien aventurer, ni quitter les quartiers d’hiver sans un ordre de César ; ils montraient qu’« on pouvait résister aux Germains, quels que fussent leurs effectifs, du moment qu’on était dans un camp retranché la preuve en est qu’ils ont fort bien résisté à un premier assaut, et en infligeant à l’ennemi des pertes sévères ; le blé ne manque pas ; avant qu’il vienne à manquer, des secours arriveront et des camps voisins et de César ; et puis enfin, y a-t-il conduite plus légère et plus honteuse que de se déterminer, sur une question d’extrême importance, d’après les suggestions d’un ennemi ? »
\subsection[{§ 29.}]{ \textsc{§ 29.} }
\noindent Mais Titurius se récriait : « Il serait trop tard, une fois que les ennemis, renforcés des Germains, se seraient assemblés en plus grand nombre, ou qu’il serait arrivé quelque malheur dans les quartiers voisins. On n’avait que cet instant pour se décider. César, selon lui, était parti pour l’Italie autrement, les Carnutes n’auraient pas résolu l’assassinat de Tasgétios, et les Eburons, s’il était en Gaule, ne seraient pas venus nous attaquer en faisant si bon marché de nos forces. Que l’avis vînt des ennemis, peu lui importait : il regardait les faits : le Rhin était tout proche ; les Germains éprouvaient un vif ressentiment de la mort d’Arioviste et de nos précédentes victoires ; la Gaule brûlait de se venger, n’acceptant pas d’avoir été si souvent humiliée et finalement soumise à Rome, ni de voir ternie sa gloire militaire d’autrefois. Enfin, qui pourrait croire qu’Ambiorix se fût résolu à une telle démarche sans motif sérieux ? Son avis, dans un cas comme dans l’autre, était sûr : si le péril était imaginaire, on rejoindrait sans courir aucun risque la plus proche légion ; si la Gaule entière était d’accord avec les Germains, il n’y avait de salut que dans la promptitude. Cotta et ceux qui pensaient comme lui, où allait leur avis ? S'il n’exposait pas les troupes à un danger immédiat, du moins c’était la certitude d’un long siège, avec la menace de la famine. »
\subsection[{§ 30.}]{ \textsc{§ 30.} }
\noindent Après qu’on eut ainsi soutenu les deux thèses, comme Cotta et les centurions de la première cohorte résistaient énergiquement : « Eh bien ! soit, dit Sabinus, puisque vous le voulez ! » – et il élevait la voix, pour qu’une grande partie des soldats l’entendissent – « ce n’est pas moi qui parmi vous ai le plus peur de la mort ; ceux-là jugeront sainement des choses : s’il arrive un malheur, c’est à toi qu’ils demanderont des comptes ; si tu voulais, ils auraient après-demain rejoint les quartiers voisins et ils soutiendraient en commun, avec les autres, les chances de la guerre, au lieu de rester abandonnés, exilés, loin de leurs camarades, pour être massacrés ou mourir de faim. »
\subsection[{§ 31.}]{ \textsc{§ 31.} }
\noindent On se lève ; on entoure les deux légats, on les presse de ne pas s’obstiner dans un conflit qui rend la situation extrêmement périlleuse : « Il est aisé d’en sortir, que l’on reste ou que l’on s’en aille, à la condition que tout le monde soit d’accord ; mais si l’on se querelle, toute chance de salut disparaît. » On continue de discuter jusqu’au milieu de la nuit. Enfin Cotta, très ému, se rend : l’avis de Sabinus l’emporte. On annonce qu’on partira au lever du jour. Le reste de la nuit se passe à veiller, chaque soldat cherchant dans ce qui lui appartient ce qu’il peut emporter, ce qu’il est forcé d’abandonner de son installation d’hiver. On fait tout ce qui est imaginable pour qu’on ne puisse partir au matin sans péril et que le danger soit encore augmenté par la fatigue des soldats privés de sommeil. Au petit jour, ils quittent le camp comme des gens bien persuadés que le conseil d’Ambiorix vient non pas d’un ennemi, mais du meilleur de leurs amis : ils formaient une très longue colonne encombrée de nombreux bagages.
\subsection[{§ 32.}]{ \textsc{§ 32.} }
\noindent Les ennemis, quand l’agitation nocturne et les veilles de nos soldats leur eurent fait comprendre que ceux-ci allaient partir, dressèrent une double embuscade dans les bois, sur un terrain favorable et couvert, à deux mille pas environ du camp, et ils y attendirent les Romains ; la plus grande partie de la colonne venait de s’engager dans un grand vallon, quand soudain ils se montrèrent aux deux bouts de cette vallée, et tombant sur l’arrière-garde, interdisant à la tête de colonne de progresser vers les hauteurs, forcèrent nos troupes à combattre dans une position fort désavantageuses.
\subsection[{§ 33.}]{ \textsc{§ 33.} }
\noindent Titurius, en homme qui n’avait rien su prévoir, maintenant s’agite et court de tous côtés, plaçant les cohortes ; mais cela même il le fait sans assurance, et d’une manière qui laisse voir qu’il a perdu tous ses moyens, ce qui arrive généralement à ceux qui sont forcés de se décider en pleine action. Cotta, au contraire, en homme qui avait pensé que pareille surprise était possible et pour cette raison n’avait pas approuvé le départ, ne négligeait rien pour le salut commun il adressait la parole aux troupes et les exhortait comme l’eût fait le général en chef, et il combattait dans le rang comme un soldat. La longueur de la colonne ne permettant guère aux légats de tout diriger personnellement et de prendre les mesures qui s’imposaient en chaque endroit, ils firent donner l’ordre d’abandonner les bagages et de former le cercle. Cette décision, bien que dans un cas de ce genre elle ne soit pas condamnable, eut cependant de fâcheuses conséquences : elle diminua la confiance des soldats et donna aux ennemis un surcroît d’ardeur, car il semblait que la crainte et le désespoir avaient seuls pu l’inspirer. Il se produisit, en outre, ceci, qui était inévitable : nombre de soldats quittaient les rangs et couraient aux bagages pour chercher et emporter les objets auxquels chacun tenait le plus ; ce n’étaient partout que cris et gémissements.
\subsection[{§ 34.}]{ \textsc{§ 34.} }
\noindent Les Barbares, au contraire, furent fort bien inspirés. Leurs chefs firent transmettre sur toute la ligne de bataille l’ordre de ne pas quitter sa place ; tout ce que les Romains laisseraient, c’était leur butin, c’était pour eux : par conséquent, ils ne devaient penser qu’à la victoire, dont tout dépendait… Les nôtres, bien qu’abandonnés de leur général et de la Fortune, ne pensaient pas à d’autres moyens de salut que leur courage, et chaque fois qu’une cohorte chargeait, c’était de ce côté un grand massacre d’ennemis. Voyant cela, Ambiorix fait donner l’ordre à ses hommes de lancer leurs traits de loin, en évitant d’approcher, et de céder partout où les Romains attaqueront ; grâce à la légèreté de leurs armes et à leur entraînement quotidien, il ne pourra leur être fait aucun mal ; quand l’ennemi se repliera sur ses enseignes, qu’on le poursuive.
\subsection[{§ 35.}]{ \textsc{§ 35.} }
\noindent Ce mot d’ordre fut soigneusement observé chaque fois que quelque cohorte sortait du cercle et attaquait, les ennemis s’enfuyaient à toute allure. Cependant la place laissée vide était forcément découverte, et le côté droit, non protégé, recevait des traits. Puis, quand la cohorte avait fait demi-tour pour revenir à son point de départ, elle était enveloppée par ceux qui lui avaient cédé le terrain et par ceux qui étaient restés sur les côtés. Voulaient-ils, au contraire, ne pas quitter le cercle, le courage alors état sans emploi et, pressés les uns contre les autres, ils ne pouvaient éviter les traits que faisait pleuvoir toute cette multitude. Pourtant, accablés par tant de difficultés, malgré des pertes sensibles, ils tenaient ; une grande partie de la journée s’était écoulée – on se battait depuis le lever du jour et on était à la huitième heure – et ils ne faisaient rien qui fût indigne d’eux. A ce moment, Titus Balventius, qui l’année précédente avait été nommé primipile, vaillant combattant, et très écouté, a les deux cuisses traversées d’une tragule ; Quintus Lucanius, officier du même grade, est tué en combattant vaillamment pour secourir son fils que l’ennemi entoure ; le légat Lucius Cotta, tandis qu’il exhorte toutes les unités, cohortes et centuries même, est blessé d’une balle de fronde en plein visage.
\subsection[{§ 36.}]{ \textsc{§ 36.} }
\noindent Sous le coup de ces événements, Quintus Titurius, ayant aperçu au loin Ambiorix qui haranguait ses troupes, lui envoie son interprète Cnéus Pompée pour le prier de l’épargner, lui et ses soldats. Aux premières paroles du messager, Ambiorix répondit : « S'il veut conférer avec lui, il y consent ; il espère pouvoir obtenir de ses troupes que la vie soit laissée aux soldats ; quant au général, il ne lui sera fait aucun mal, et de cela il se porte garant. » Titurius fait proposer à Cotta, qui était blessé, de quitter avec lui, s’il le veut bien, le combat peur aller conférer ensemble avec Ambiorix : « Il espère qu’on pourra obtenir de lui la vie sauve pour eux et pour les soldats. » Cotta déclare qu’il ne se rendra pas auprès d’un ennemi en armes, et il persiste dans ce refus.
\subsection[{§ 37.}]{ \textsc{§ 37.} }
\noindent Sabinus ordonne aux tribuns qu’il avait en ce moment autour de lui et aux centurions de la première cohorte de le suivre, et il s’avance vers Ambiorix ; sommé de mettre bas les armes, il obéit, et enjoint aux siens de faire de même. Tandis qu’ils discutent les conditions, et qu’Ambiorix prolonge à dessein l’entretien, on l’entoure peu à peu et on le tue. Alors ce sont des cris de triomphe, les hurlements accoutumés ; ils se précipitent sur nos troupes et mettent le désordre dans leurs rangs. C'est là que Lucius Cotta trouve la mort, les armes à la main, avec la plupart des soldats. Les survivants se retirent dans le camp d’où ils étaient partis. L'un d’eux, le porte-aigle Lucius Pétrosidius, se voyant pressé par une foule d’ennemis, jette l’aigle à l’intérieur du retranchement et se fait tuer en brave en avant du camp. Jusqu’à la fin du jour ils soutiennent péniblement l’assaut ; à la nuit, n’ayant plus aucun espoir, tous jusqu’au dernier se donnent la mort. Une poignée d’hommes, échappés du combat, sans connaître le chemin, parviennent à travers les bois aux quartiers d’hiver du légat Titus Labiénus, et l’informent de ce qui s’est passés.
\subsection[{§ 38.}]{ \textsc{§ 38.} }
\noindent Transporté d’orgueil par cette victoire, Ambiorix part sur-le-champ avec sa cavalerie chez les Atuatuques, qui confinaient à son royaume, et nuit et jour marche sans arrêt ; l’infanterie a ordre de le suivre de près. Il raconte ce qui s’est passé, soulève les Atuatuques, arrive le lendemain chez les Nerviens et les exhorte à ne pas laisser échapper cette occasion de s’affranchir pour toujours et de faire expier aux Romains le mal qu’ils leur ont fait : « Deux légats, explique-t-il, ont été tués, une grande partie de l’armée romaine est anéantie ; c’est chose bien facile que d’attaquer à l’improviste la légion qui prend ses quartiers d’hiver avec Cicéron et de la massacrer ». Il promet son concours pour ce coup de main. Les Nerviens se laissent aisément persuader par ce discours.
\subsection[{§ 39.}]{ \textsc{§ 39.} }
\noindent Ils s’empressent donc d’envoyer des messagers aux Centrons, aux Grudii, aux Lévaques, aux Pleumoxii, aux Geidumnes, toutes tribus qui sont sous leur dépendance ; ils réunissent le plus de troupes qu’ils peuvent et à l’improviste se jettent sur le camp de Cicéron, avant que la nouvelle de la mort de Titurius lui soit parvenue. Lui aussi, il lui arriva – ce qui était inévitable – qu’un certain nombre de soldats, qui s’étaient éloignés pour aller dans les forêts chercher du bois de chauffage et du bois de charpente pour la fortification, furent surpris par l’arrivée soudaine de la cavalerie. On les enveloppe, et en masse Eburons, Nerviens, Atuatuques, ainsi que les alliés et clients de tous ces peuples, commencent l’attaque de la légion. Les nôtres vivement courent aux armes, montent au retranchement. Ce fut une rude journée : les ennemis plaçaient tout leur espoir dans une action prompte et, ayant été une fois vainqueurs, ils croyaient qu’ils devaient l’être toujours.
\subsection[{§ 40.}]{ \textsc{§ 40.} }
\noindent Cicéron écrit aussitôt à César en promettant aux courriers de grandes récompenses s’ils réussissent à faire parvenir sa lettre ; mais l’ennemi tient toutes les routes, ils sont interceptés. Pendant la nuit, avec le bois qu’on avait apporté pour la fortification, on n’élève pas moins de cent vingt tours, par un prodige de rapidité ; ce que les ouvrages de défense présentaient d’incomplet, on l’achève. Le jour suivant, l’ennemi, dont les forces s’étaient considérablement accrues, donne l’assaut, comble le fossé. Les nôtres résistent dans les mêmes conditions que la veille. Même chose les jours suivants. Pendant la nuit, on travaille sans relâche, pour les malades, pour les blessés, pas de repos. Tout ce qu’il faut pour soutenir l’assaut du lendemain, on le prépare la nuit : on aiguise et on durcit au feu un grand nombre d’épieux, on fabrique beaucoup de javelots de siège ; on garnit les tours de plates-formes, on munit le rempart de créneaux et d’un parapet en clayonnage. Cicéron lui-même, bien qu’il fût de santé très délicate, ne s’accordait même pas le repos de la nuit ce fut au point qu’on vit les soldats se presser autour de lui et le forcer par leurs instances à se ménager.
\subsection[{§ 41.}]{ \textsc{§ 41.} }
\noindent Alors des chefs et des nobles Nerviens qui avaient quelque accès auprès de Cicéron, ayant prétexte à se dire ses amis, font savoir qu’ils désirent un entretien. On le leur accorde, et ils font les mêmes déclarations qu’Ambiorix avait faites à Titurius : « Toute la Gaule est en armes, les Germains ont passé le Rhin ; les quartiers d’hiver de César et ceux de ses lieutenants sont assiégés. » En outre, ils narrent la mort de Sabinus et, pour qu’on les croie, ils font parade de la présence d’Ambiorix. « C'est se faire illusion, disent-ils, que d’attendre le moindre secours de troupes qui ont des inquiétudes pour elles-mêmes ; eux, cependant, ils ne sont nullement hostiles à Cicéron et au peuple romain ; tout ce qu’ils demandent, c’est d’être débarrassés des quartiers d’hiver et de ne pas voir s’en implanter l’habitude : ils n’inquièteront pas la légion dans sa retraite, et elle pourra sans crainte s’en aller du côté qui lui plaira. » Cicéron borna sa réponse à ces mots : « Il n’était pas dans les usages de Rome d’accepter les conditions d’un ennemi en armes ; s’ils veulent désarmer, son appui leur est assuré pour l’envoi d’une ambassade à César : il espère que, dans sa justice, il leur donnera satisfaction. »
\subsection[{§ 42.}]{ \textsc{§ 42.} }
\noindent Déçus dans cet espoir, les Nerviens entourent le camp d’un rempart haut de dix pieds et d’un fossé large de quinze. Ils avaient acquis à notre contact, dans les années précédentes, l’expérience de ces travaux ; et d’ailleurs, ayant quelques prisonniers de notre armée, ils profitaient de leurs leçons. Mais comme ils manquaient des outils nécessaires, ils devaient couper les mottes de gazon avec leurs épées, enlever la terre avec leurs mains et la porter dans leurs sayons. On put voir là quel était leur nombre en moins de trois heures, ils achevèrent une ligne fortifiée qui avait quinze mille pieds de tour. Les jours suivants, ils entreprirent de construire des tours proportionnées à la hauteur du rempart, de fabriquer des faux et des tortues, toujours d’après les indications des prisonniers.
\subsection[{§ 43.}]{ \textsc{§ 43.} }
\noindent Le septième jour du siège, un vent violent s’étant élevé, ils se mirent à lancer sur les maisons, qui, selon l’usage gaulois, étaient couvertes de chaume, des balles de fronde brûlantes faites d’une argile qui pouvait rougir au feu, et des traits enflammés. Le feu prit rapidement, et la violence du vent le dispersa sur tous les points du camp. Les ennemis, poussant une immense clameur, comme si déjà ils tenaient la victoire, firent avancer leurs tours et leurs tortues et, à l’aide d’échelles, entreprirent d’escalader le rempart : Mais tels furent le courage et le sang-froid de nos soldats que, malgré la cuisante chaleur du feu qui les entourait, malgré la grêle de traits dont ils étaient accablés, bien qu’ils se rendissent compte que tout leurs bagages, tout ce qu’ils possédaient était la proie des flammes, personne ne quitta le rempart pour aller ailleurs, ni même, peut-on presque dire, ne détourna seulement la tête : tous au contraire combattirent alors avec une vigueur et une vaillance sans égales. Cette journée fut de beaucoup la plus dure pour nos troupes, mais elle eut aussi ce résultat que les ennemis eurent plus de blessés et de tués que jamais, car ils s’étaient entassés au pied même du rempart et les derniers venus barraient la retraite à ceux qui étaient devant. Comme l’incendie s’était un peu apaisé et qu’en un certain point une tour avait été poussée tout contre le rempart, les centurions de la troisième cohorte quittèrent la place qu’ils occupaient et reculèrent avec tout leurs hommes, puis, faisant des signes aux ennemis et les appelant, ils les invitaient à entrer mais pas un n’osa avancer. Alors une grêle de pierres, pleuvant de toutes parts, les mit en fuite, et la tour fut incendiée.
\subsection[{§ 44.}]{ \textsc{§ 44.} }
\noindent Il y avait dans cette légion deux centurions d’une grande bravoure, qui approchaient des premiers grades, Titus Pullo et Lucius Vorénus. C'était entre eux une perpétuelle rivalité à qui passerait avant l’autre, et chaque année la question de l’avancement les mettait en violent conflit. Pullo, au moment où l’on se battait avec le plus d’acharnement au rempart, s’écria : « Pourquoi hésiter, Vorénus ? quelle autre occasion attends-tu de prouver ta valeur ? c’est ce jour qui décidera entre nous. » A ces mots, il s’avance hors du retranchement, et choisissant l’endroit le plus dense de la ligne ennemie, il fonce. Vorénus ne reste pas davantage derrière le rempart, mais craignant l’opinion des troupes, il suit de près son rival. Quand il n’est plus qu’à peu de distance de l’ennemi, Pullo jette son javelot et atteint un Gaulois qui s’était détaché du gros de l’ennemi pour courir en avant ; transpercé, mourant, ses compagnons le couvrent de leurs boucliers, cependant que tous à la fois ils lancent leurs traits contre le Romain et l’empêchent d’avancer. Il a son bouclier traversé d’un javelot qui se plante dans le baudrier de l’épée : ce coup déplace le fourreau, et retarde le mouvement de sa main qui cherche à dégainer ; tandis qu’il tâtonne, l’ennemi l’enveloppe. Son rival, Vorénus, accourt à son aide. Aussitôt, toute la multitude des ennemis se tourne contre lui et laisse là Pullo, croyant que le javelot l’a percé de part en part. Vorénus, l’épée au poing, lutte corps à corps, en tue un, écarte un peu les autres ; mais, emporté par son ardeur, il se jette dans un creux, et tombe. C'est à son tour d’être enveloppé ; mais Pullo lui porte secours, et ils rentrent tous deux au camp, sains et saufs, ayant tué beaucoup d’ennemis et s’étant couverts de gloire. La Fortune traita de telle sorte ces rivaux, qu’en dépit de leur inimitié ils se secoururent l’un l’autre et se sauvèrent mutuellement la vie, et qu’il fut impossible de décider à qui revenait le prix de la bravoure.
\subsection[{§ 45.}]{ \textsc{§ 45.} }
\noindent Le siège devenait chaque jour plus angoissant et plus difficile à soutenir ; d’autant plus que, beaucoup de soldats étant épuisés par leurs blessures, on en était réduit à une poignée de défenseurs ; Cicéron écrivait toujours plus de lettres à César, lui dépêchait courriers sur courriers ; plusieurs de ceux-ci, pris sur-le-champ, étaient suppliciés sous les yeux de nos soldats. Il y avait dans le camp un Nervien, du nom de Vertico, homme de bonne naissance, qui dès le début du siège avait passé à Cicéron et lui avait juré fidélité. Il décide un Gaulois, son esclave, en lui promettant la liberté et de grandes récompenses, à porter une lettre à César. L'homme l’emporte fixée à son javelot, passe au milieu de ses compatriotes sans éveiller aucun soupçon et parvient auprès de César. Par lui on apprend quels dangers courent Cicéron et sa légion.
\subsection[{§ 46.}]{ \textsc{§ 46.} }
\noindent César, ayant reçu la lettre vers la onzième heure du jour, envoie sur-le-champ un courrier chez les Bellovaques, auprès du questeur Marcus Crassus, dont les quartiers d’hiver étaient éloignés de vingt-cinq milles : la légion doit partir au milieu de la nuit et venir en hâte le rejoindre. Crassus sort de son camp avec le messager. Un autre est envoyé au légat Caïus Fabius : il doit conduire sa légion dans le pays des Atrébates, par où César savait qu’il lui fallait passer. Il écrit à Titus Labiénus de venir avec sa légion à la frontière des Nerviens, s’il peut le faire sans rien compromettre. Le reste de l’armée étant un peu plus éloigné, il ne croit pas devoir l’attendre ; comme cavalerie, il réunit environ quatre cents hommes qu’il tire des quartiers les plus voisins.
\subsection[{§ 47.}]{ \textsc{§ 47.} }
\noindent Ayant appris vers la troisième heure par les éclaireurs que Crassus arrivait, il avance ce jour-là de vingt milles. Il donne à Crassus le commandement de Samarobriva, et lui attribue Ia légion qu’il amenait, car César laissait là les bagages de l’armée, les otages fournis par les cités, les archives, et tout le blé qu’il y avait fait rassembler comme provision d’hiver. Fabius, suivant l’ordre reçu, le rejoint sur la route avec sa légion, sans grand retard. Labiénus connaissait la mort de Sabinus et le massacre des cohortes ; les Trévires avaient porté contre lui toutes leurs forces ; il craignit les conséquences d’un départ qui ressemblerait à une fuite : il ne pourrait soutenir l’assaut des ennemis, étant donné surtout que la récente victoire les avait, il ne l’ignorait pas, transportés d’orgueil. Il répond donc à César par une lettre où il lui représente tout le danger qu’il courait à faire sortir sa légion ; il lui raconte en détail ce qui s’est passé chez les Eburons ; il lui fait connaître que toutes les forces des Trévires, cavalerie et infanterie, ont pris position à trois milles de son camp.
\subsection[{§ 48.}]{ \textsc{§ 48.} }
\noindent César approuva ses vues, et bien que réduit à deux légions après avoir compté sur trois, il n’en continuait pas moins de penser qu’une action rapide était le seul moyen de sauver l’armée. Il gagne donc à marches forcées le pays des Nerviens. Là, il apprend par des prisonniers ce qui se passe au camp de Cicéron et combien la situation est critique. Il décide alors un cavalier gaulois, en lui promettant de grandes récompenses, à porter une lettre à Cicéron. Il l’écrit en grec pour que, si elle est interceptée l’ennemi ne connaisse pas nos plans. Dans le cas où il ne pourrait arriver jusqu’à Cicéron, il devra attacher la lettre à la courroie de sa tragule et la lancer à l’intérieur des fortifications. Dans sa lettre, il annonce qu’il s’est mis en route avec des légions et sera bientôt là ; il presse le légat de ne pas laisser fléchir son courage. Le Gaulois, n’osant pas approcher, lance son javelot, selon les instructions qu’il avait reçues. Le hasard fit que le trait allât se planter dans une tour, où il reste deux jours sans que les nôtres le remarquent : le troisième jour, un soldat l’aperçoit, l’arrache et le porte à Cicéron. Celui-ci, après avoir pris connaissance du message, en donne lecture devant les troupes, chez qui il excite la joie la plus vive. A ce moment, on apercevait au loin des fumées d’incendie : cela ne permit plus de douter de l’approche des légions.
\subsection[{§ 49.}]{ \textsc{§ 49.} }
\noindent Les Gaulois, mis au courant par leurs éclaireurs, lèvent le siège et marchent au-devant de César avec toutes leurs forces. Elles étaient d’environ soixante mille hommes. Cicéron, grâce à ce même Vertico dont il a été question plus haut, trouve un Gaulois qui se charge de porter une lettre à César ; il lui recommande d’aller avec précaution et diligence. Dans sa lettre, il explique que l’ennemi l’a quitté et a tourné toutes ses forces contre César. Le message est remis vers minuit : César en fait part à son armée et l’exhorte au combat. Le lendemain, au point du jour, il lève le camp, et il avait parcouru environ quatre milles quand il aperçoit les masses ennemies de l’autre côté d’une vallée où coulait un cours d’eau. C'était s’exposer à de grands périls que d’engager le combat sur un terrain défavorable avec une telle infériorité numérique ; de plus, puisqu’il savait Cicéron délivré du siège, il pouvait sans inquiétude ralentir son action : il fit donc halte ; il établit un camp fortifié en choisissant la meilleure position possible et, bien que ce camp fût déjà par lui-même de petites dimensions, puisqu’il était pour une troupe de sept mille hommes à peine, et, qui plus est, dépourvue de bagages, néanmoins il le resserre tant qu’il peut, en diminuant la largeur des rues, afin d’inspirer à l’ennemi le plus parfait mépris. En même temps, il envoie de tous côtés des éclaireurs rechercher par quel chemin il pourra franchir la vallée le plus commodément.
\subsection[{§ 50.}]{ \textsc{§ 50.} }
\noindent Ce jour-là il y eut de petits engagements de cavalerie près de l’eau, mais les deux armées restèrent sur leurs positions : les Gaulois attendaient des forces plus nombreuses, qui n’avaient pas encore rejoint, et César voulait livrer bataille en deçà du vallon, devant son camp, s’il réussissait, en simulant la peur, à attirer l’ennemi sur son terrain ; au cas où il n’y parviendrait pas, il désirait bien connaître les chemins pour pouvoir traverser le vallon et passer la rivière avec moins de danger. Au lever du jour, la cavalerie ennemie approche de notre position et engage le combat avec nos cavaliers. César ordonne à ceux-ci de céder par tactique et de rentrer dans le camp : en même temps, on exhaussera partout le rempart, on bouchera les portes, et on agira en tout cela avec une extrême précipitation, comme si l’on avait peur.
\subsection[{§ 51.}]{ \textsc{§ 51.} }
\noindent Attirés par toutes ces feintes, les ennemis traversent la vallée et se mettent en ligne avec le désavantage de la position ; mais nous allons jusqu’à évacuer le rempart ; alors ils approchent encore, lancent de toutes parts des traits à l’intérieur du retranchement, et font publier tout autour du camp par des hérauts que tout Gaulois ou Romain qui voudra passer de leur côté avant la troisième heure pourra le faire sans crainte ; après, il ne sera plus temps. Et tel fut le mépris que nous leur inspirâmes que, croyant ne pas pouvoir enfoncer nos portes que nous avions barricadées, pour donner le change, d’un simple rang de mottes de gazon, les uns entreprenaient de faire brèche à la main dans la palissade, et d’autres de combler les fossés. A ce moment, César fait une sortie par toutes les portes et lance sa cavalerie : les ennemis sont rapidement mis en déroute, et dans de telles conditions que pas un d’eux ne tint tête ; beaucoup sont tués, aucun ne garde ses armes.
\subsection[{§ 52.}]{ \textsc{§ 52.} }
\noindent César, jugeant dangereux de s’engager plus avant à leur poursuite, à cause des bois et des marais, et voyant d’ailleurs qu’il n’était plus possible de leur faire le moindre mal, rejoint Cicéron le jour même, sans avoir subi aucune perte. Les tours, les tortues, les retranchements construits par l’ennemi provoquent son étonnement ; une revue de la légion lui permet de constater qu’il n’y a pas un soldat sur dix qui soit sans blessure ; tout cela lui montre quels dangers on a courus et quelle valeur on a déployée. Il donne à Cicéron, aux soldats, les éloges qu’ils méritent ; il félicite individuellement les centurions et les tribuns qui, au témoignage de Cicéron, s’étaient particulièrement distingués. Des prisonniers lui donnent des détails sur ce qui est arrivé à Sabinus et à Cotta. Le lendemain, il assemble les troupes, leur explique le drame, les réconforte et les rassure : « Ce malheur, qui est dû aux fautes et à la légèreté d’un légat, doit d’autant moins les troubler que, par la protection des dieux immortels et grâce à leur propre vaillance, l’affront est vengé, la joie de l’ennemi a été courte, et leur tristesse ne doit pas durer plus longtemps. »
\subsection[{§ 53.}]{ \textsc{§ 53.} }
\noindent Cependant la nouvelle de la victoire de César parvient à Labiénus, par les Rèmes, avec une rapidité incroyable : le camp de Cicéron se trouvant à environ soixante milles, et César, étant arrivé après la neuvième heure du jour, avant minuit une clameur s’élevait aux portes du camp : c’étaient les Rèmes qui annonçaient la victoire à Labiénus et le congratulait. La même nouvelle parvient aux Trévires, et Indutiomaros, qui avait résolu d’attaquer le camp de Labiénus le lendemain, s’enfuit pendant la nuit et ramène toutes ses troupes chez les Trévires. César renvoie Fabius dans ses quartiers d’hiver avec sa légion ; quant à lui, il décide d’hiverner autour de Samarobriva avec trois légions en trois camps, et la gravité des troubles qui avaient éclaté en Gaule le détermina à rester lui-même à l’armée pendant tout l’hiver. En effet, depuis que s’était répandu le bruit de cet échec où Sabinus avait trouvé la mort, presque toutes les cités de Gaule parlaient de guerre, elles envoyaient de tous côtés des courriers et des ambassades, s’informant de ce que méditaient les autres et d’où partirait le soulèvement ; des réunions se tenaient la nuit dans des lieux déserts. De tout l’hiver, César n’eut pour ainsi dire pas un moment de répit : sans cesse il recevait quelque avis sur les projets des Gaulois, sur la révolte qu’ils préparaient. Il apprit notamment de Lucius Roscius, qu’il avait mis à la tête de la treizième légion, que des forces gauloises importantes, appartenant aux cités qu’on nomme Armoricaines, s’étaient réunies pour l’attaquer et étaient venues jusqu’à huit milles de son camp, mais qu’à l’annonce de la victoire de César elles s’étaient retirées avec tant de hâte que leur retraite ressemblait à une fuite.
\subsection[{§ 54.}]{ \textsc{§ 54.} }
\noindent César appela auprès de lui les chefs de chaque cité et tantôt par la crainte, en leur signifiant qu’il savait tout, tantôt par la persuasion, il réussit à maintenir dans le devoir une grande partie de la Gaule. Cependant les Sénons, un des peuples gaulois les plus puissants et qui jouit parmi les autres d’une grande autorité, voulurent mettre à mort, par décision de leur assemblée, Cavarinos, que César leur avait donné pour roi, dont le frère Moritasgos régnait quand César arriva en Gaule, et dont les ancêtres avaient été rois ; comme il s’était douté de leurs intentions et avait pris la fuite, ils le poursuivirent jusqu’à la frontière, le détrônèrent et le bannirent ; puis ils envoyèrent des députés à César pour justifier leur conduite, et comme celui-ci avait ordonné que tout le sénat vînt le trouver, ils n’obéirent point. L'impression fut si forte sur ces esprits barbares, quand on sut qu’il s’était trouvé quelques audacieux pour nous déclarer la guerre, il en résulta un tel changement dans les dispositions de tous les peuples, que sauf les Héduens et les Rèmes, à qui César témoigna toujours une particulière estime, les uns à cause de leur vieille et fidèle amitié pour Rome, les autres en raison de leurs services récents dans la guerre gauloise, il n’y eut guère de cité qui ne nous donnât lieu de la soupçonner. Est-ce bien étonnant ? je ne sais ; car outre maint autre motif, une nation qu’on plaçait, pour sa valeur guerrière, plus haut que toutes, ne se voyait pas sans un vif chagrin déchue de cette réputation au point d’être soumise à la souveraineté de Rome.
\subsection[{§ 55.}]{ \textsc{§ 55.} }
\noindent Les Trévires, avec Indutiomaros, firent plus de tout l’hiver, ils ne cessèrent d’intriguer au-delà du Rhin, envoyant des ambassades, essayant de gagner les cités, promettant de l’argent, racontant que la plus grande partie de notre armée avait été détruite, qu’il en restait bien moins de la moitié. Et pourtant, aucun peuple germain ne se laissa persuader de passer le Rhin : « Ils en avaient fait deux fois l’expérience, avec la guerre d’Arioviste et avec l’émigration des Tencthères : ils n’étaient pas disposés à tenter encore la fortune. » Déchu de cet espoir, Indutiomaros ne s’en mit pas moins à rassembler des troupes, à les exercer, à se fournir de chevaux chez les voisins, à attirer par de grandes promesses les exilés et les condamnés de la Gaule entière. Et tel était le crédit que ces initiatives lui avaient déjà acquis en Gaule, que de toutes parts accouraient à lui des ambassades sollicitant, à titre public ou privé, la faveur de son amitié.
\subsection[{§ 56.}]{ \textsc{§ 56.} }
\noindent Lorsqu’il vit qu’on venait à lui avec cet empressement, et que d’un côté, les Sénons et les Carnutes étaient poussés à la révolte par le souvenir de leurs crimes, que de l’autre les Nerviens et les Atuatuques se préparaient à la guerre, qu’enfin les volontaires ne manqueraient pas de venir en foule quand il aurait commencé d’avancer hors de son pays, il convoque l’assemblée armée. C'est là, selon l’usage des Gaulois, l’acte initial de la guerre une loi, la même chez tous, veut que tous ceux qui ont l’âge d’homme y viennent en armes ; celui qui arrive le dernier est livré, en présence de la multitude, aux plus cruels supplices. Dans cette assemblée, il déclare Cingétorix ennemi public et confisque ses biens : c’était le chef du parti adverse, et son gendre ; nous avons dit plus haut qu’il s’était donné à César et lui était resté fidèles. Après cela, Indutiomaros fait connaître à l’assemblée qu’il est appelé par les Sénons et les Carnutes et par beaucoup d’autres cités de la Gaule il se propose d’y aller en traversant le pays des Rèmes, dont il dévastera les terres, et, auparavant, il attaquera le camp de Labiénus. Il donne des ordres en conséquence.
\subsection[{§ 57.}]{ \textsc{§ 57.} }
\noindent Labiénus, qui occupait un camp très bien situé et non moins bien fortifiée, ne craignait rien pour lui et sa légion ; mais il veillait à ne pas laisser échapper l’occasion d’une action heureuse. Aussi, ayant appris par Cingétorix et par ses proches ce qu’Indutiomaros avait dit dans l’assemblée, il envoie des messagers aux cités voisines et appelle de toutes parts des cavaliers, qu’il convoque à jour fixe. Cependant, presque journellement, Indutiomaros avec toute sa cavalerie venait rôder aux abords du camp, tantôt pour reconnaître la position, tantôt pour entrer en pourparlers ou pour nous effrayer ; la plupart du temps, ils jetaient tous des traits à l’intérieur de nos lignes. Labiénus retenait ses troupes derrière le retranchement et par tous les moyens possibles tâchait de fortifier chez l’ennemi l’idée que nous avions peur.
\subsection[{§ 58.}]{ \textsc{§ 58.} }
\noindent Tandis qu’Indutiomaros montrait à s’approcher de notre camp une audace chaque jour plus méprisante, Labiénus y introduisit, en une, nuit, les cavaliers des cités voisines qu’il avait fait appeler et il sut si bien faire interdire toute sortie par les postes de garde qu’il n’y eut pas moyen que la chose fût ébruitée et connue des Trévires. Cependant Indutiomaros, comme il faisait chaque jour, vient aux abords du camp et y passe la plus grande partie de la journée ; ses cavaliers lancent des traits et provoquent nos hommes au combat en termes fort outrageants. N'ayant reçu aucune réponse, quand ils en ont assez, à l’approche du soir, ils s’en vont, dans le plus complet désordre. Tout à coup Labiénus fait sortir par deux portes toute sa cavalerie ; il prescrit qu’une fois l’ennemi surpris et mis en déroute – ce qu’il prévoyait, et qui arriva – chacun ne pense qu’à joindre le seul Indutiomaros, et s’abstienne de frapper personne avant de l’avoir vu mort : il ne voulait pas qu’en s’attardant à poursuivre les autres on lui laissât le temps d’échapper ; il promet de grandes récompenses à ceux qui l’auront tué ; il envoie les cohortes en soutien de la cavalerie. La Fortune vient justifier ses prévisions : tous s’attachant à la poursuite d’un seul, Indutiomaros est pris au moment même où il passait à gué une rivière, on le tue et sa tête est rapportée au camp ; en revenant, les cavaliers pourchassent et massacrent qui ils peuvent. A la nouvelle de l’événement, toutes les forces des Eburons et des Nerviens qui s’étaient concentrées se dispersent, et César put voir, après cela, la Gaule relativement tranquille.
 \section[{Livre VI}]{Livre VI}\renewcommand{\leftmark}{Livre VI}

\subsection[{§ 1.}]{ \textsc{§ 1.} }
\noindent César, qui avait maintes raisons de s’attendre à un plus sérieux soulèvement de la Gaule, charge ses légats Marcus Silanus, Caïus Antistius Réginus et Titus Sextius de lever des troupes ; en même temps, il demande à Cnéus Pompée, proconsul, puisque dans l’intérêt de l’État, il restait revêtu de l’imperium, devant Rome, de mobiliser et de lui envoyer les recrues de Gaule Cisalpine auxquelles il avait fait prêter serment pendant son consulats ; il jugeait en effet très important, et même pour l’avenir, au point de vue de l’opinion gauloise, de montrer que les ressources de l’Italie lui permettaient, en cas de revers, non seulement d’y remédier promptement, mais encore d’être mieux pourvu de troupes qu’auparavant. Pompée, par patriotisme et par amitié, fit droit à sa demande, et ses légats ayant procédé avec rapidité aux opérations de recrutement, avant que l’hiver fût achevé trois légions avaient été mises sur pied et amenées en Gaule, ce qui lui donnait deux fois plus de cohortes qu’il r'en avait péri avec Quintus Titurius par un accroissement aussi prompt et aussi considérable de ses forces, il fit voir ce que pouvaient l’organisation et les ressources du peuple Romain.
\subsection[{§ 2.}]{ \textsc{§ 2.} }
\noindent Indutiomaros ayant été tué, comme nous l’avons dit, les Trévires donnent le pouvoir à des membres de sa famille. Ceux-ci continuent de solliciter les Germains du voisinage et de leur promettre de l’argent. Ne pouvant décider les peuples les plus proches, ils s’adressent à de plus éloignés. Un certain nombre consentent : on se lie par serment, les subsides sont garantis au moyen d’otages ; on fait entrer Ambiorix dans la ligue. Informé de ces intrigues, et comme il ne voyait de tous côtés que préparatifs de guerre – les Nerviens, les Atuatuques, les Ménapes en armes avec tous les Germains cisrhénans, les Sénons s’abstenant de répondre à sa convocation et se concertant avec les Carnutes et les cités voisines, les Trévires ne cessant de députer aux Germains pour tâcher de les gagner –, César pensa qu’il devait entrer en campagne plus tôt qu’à l’ordinaire.
\subsection[{§ 3.}]{ \textsc{§ 3.} }
\noindent Donc, avant que l’hiver fût achevé, il rassembla les quatre légions les plus proches et, à l’improviste, marcha sur le pays des Nerviens ; sans leur laisser le temps de se rassembler ou de fuir, enlevant beaucoup de bétail, faisant un grand nombre de prisonniers – butin qu’il abandonna aux soldats, – dévastant leurs campagnes, il les força à se soumettre et à lui fournir des otages. L'affaire fut vivement terminée ; après quoi, il fit demi-tour, et ramena les légions dans leurs quartiers d’hiver. Aux premiers jours du printemps, il convoqua, selon la règle qu’il avait établie, l’assemblée de la Gaule ; tous y vinrent sauf les Sénons, les Carnutes et les Trévires ; il interpréta cette abstention comme le début de la révolte ouverte, et, pour faire voir qu’il subordonnait tout à sa répression, il transporte l’assemblée à Lutèce, ville des Parisii. Ce peuple était limitrophe des Sénons, et jadis il s’était uni à eux en un seul État ; mais il paraissait être resté étranger au complot. César annonce sa résolution du haut de son tribunal et le même jour il part avec ses légions pour le pays des Sénons, qu’il gagne à marches forcées.
\subsection[{§ 4.}]{ \textsc{§ 4.} }
\noindent A la nouvelle de son approche, Acco, qui était l’instigateur de la révolte, ordonne que les populations se rassemblent dans les places fortes. La mesure était en cours d’exécution quand on annonce que les Romains sont là. Les Sénons ne peuvent faire autrement que de renoncer à leur projet et d’envoyer des députés à César pour tâcher de le fléchir ; les Héduens, qui étaient depuis longtemps leurs protecteurs, les introduisent. Volontiers César, à la prière des Héduens, leur pardonne et accepte leurs excuses, car il estimait que la saison d’été n’était pas faite pour mener des enquêtes, mais devait être réservée à la guerre qui était tout près d’éclater. Il exige cent otages, et en confie la garde aux Héduens. Les Carnutes lui envoient aussi chez les Sénons députés et otages ; ils font plaider leur cause par les Rèmes, dont ils étaient les clients, et obtiennent semblable réponse. César va achever la session de l’assemblée ; il commande aux cités de lui fournir des cavaliers.
\subsection[{§ 5.}]{ \textsc{§ 5.} }
\noindent Ayant pacifié cette partie de la Gaule, il se donne tout entier à la guerre des Trévires et d’Ambiorix. Il invite Cavarinos à l’accompagner avec la cavalerie des Sénons, de crainte que son caractère violent ou la haine qu’il s’était attirée ne fissent naître des troubles. Ces affaires réglées, comme il tenait pour assuré qu’Ambiorix ne livrerait pas bataille, il cherchait à deviner quel autre parti il pourrait prendre. Près du pays des Eburons, derrière une ligne continue de marécages et de forêts, vivaient les Ménapes, le seul peuple de la Gaule qui n’eût jamais envoyé d’ambassade à César pour traiter de la paix. Il savait qu’Ambiorix était uni à eux par des liens d’hospitalité ; il savait également que par l’entremise des Trévires il avait fait alliance avec les Germains. César pensait qu’avant de l’attaquer il fallait lui enlever ces appuis ; sinon il était à craindre que, se voyant perdu, il n’allât se cacher chez les Ménapes ou se joindre aux Transrhénans. Il adopte donc ce plan ; il envoie les bagages de toute l’armée à Labiénus, chez les Trévires, et fait partir pour son camp deux légions ; quant à lui, avec cinq légions sans bagages, il se dirige vers le territoire des Ménapes. Ceux-ci, sans rassembler de troupes, confiants dans la protection que leur offrait le pays, se réfugient dans les forêts et les marécages, et y transportent leurs biens.
\subsection[{§ 6.}]{ \textsc{§ 6.} }
\noindent César partage ses troupes avec son légat Caïus Fabius et son questeur Marcus Crassus, fait jeter rapidement des ponts et pénètre dans le pays en trois endroits : il incendie fermes et villages, prend beaucoup de bétail et fait de nombreux prisonniers. Les Ménapes se voient contraints de lui envoyer des députés pour demander la paix. Il reçoit leurs otages et déclare qu’il les tiendra pour ennemis s’ils reçoivent sur leur territoire Ambiorix ou ses représentants. Ayant ainsi réglé l’affaire, il laisse chez les Ménapes, pour les surveiller, Commios l’Atrébate avec de la cavalerie, et il marche contre les Trévires.
\subsection[{§ 7.}]{ \textsc{§ 7.} }
\noindent Pendant cette campagne de César, les Trévires ayant rassemblé d’importantes forces d’infanterie et de cavalerie, s’apprêtaient à attaquer Labiénus qui, avec une seule légion, avait passé l’hiver dans leur pays ; déjà ils n’étaient plus qu’à deux journées de son camp, lorsqu’ils apprennent qu’il a reçu deux autres légions envoyées par César. Ils s’établissent alors à quinze milles de distance et décident d’attendre là le renfort des Germains. Labiénus, instruit de leurs intentions, pensa que leur imprudence lui fournirait quelque heureuse occasion de livrer bataille laissant cinq cohortes à la garde des bagages, il marche à la rencontre des ennemis avec vingt-cinq cohortes et une nombreuse cavalerie, et se retranche à mille pas de leur camp. Il y avait entre eux et Labiénus une rivière difficile à franchir, bordée de rives abruptes. Il n’avait pas, quant à lui, l’intention de la traverser, et il ne pensait pas que l’ennemi voulût le faire. Celui-ci espérait chaque jour davantage voir arriver les Germains. Labiénus parle dans le conseil de façon à être entendu des soldats : « Puisqu’on dit que les Germains approchent, il ne veut pas hasarder le sort de l’armée et le sien, et le lendemain, au lever du jour, il s’en ira. » Ces propos ne tardent pas à être rapportés à l’ennemi, car sur tant de cavaliers gaulois plus d’un était naturellement porté à favoriser la cause gauloise. Labiénus convoque pendant la nuit les tribuns et les centurions des premières cohortes il leur expose son dessein et, pour mieux faire croire à l’ennemi qu’il a peur, il ordonne de lever le camp plus bruyamment et plus confusément que ne font à leur ordinaire les armées de Rome. Par ce moyen, il donne à son départ l’allure d’une fuite. L'ennemi en est également informé avant le jour, vu la proximité des deux camps, il est au courant par ses éclaireurs.
\subsection[{§ 8.}]{ \textsc{§ 8.} }
\noindent A peine l’arrière-garde avait-elle dépassé les retranchements que, s’excitant les uns les autres à ne pas laisser échapper de leurs mains une proie désirée – « Il était trop long, disaient-ils, du moment que les Romains avaient peur, d’attendre l’appui des Germains ; leur honneur ne souffrait point qu’avec de telles forces ils n’eussent pas l’audace d’attaquer une troupe si peu nombreuse et, qui plus est, en fuite, embarrassée de ses bagages » –, les Gaulois n’hésitent pas à passer la rivière et à engager le combat dans une position défavorable. Labiénus avait prévu la chose et, pour les attirer tous en deçà du cours d’eau, il continuait sa feinte et avançait lentement. Puis, après avoir envoyé les bagages un peu en avant et les avoir fait placer sur un tertre, il adresse aux troupes ces paroles : « Voici, soldats, l’occasion souhaitée : vous tenez l’ennemi sur un terrain où ses mouvements ne sont pas libres et où nous le dominons ; montrez sous nos ordres la même bravoure que le général en chef vous a vu si souvent déployer, et faites comme s’il était là, s’il voyait ce qui se passe. » Aussitôt il fait tourner les enseignes contre l’ennemi et former le front de bataille ; il envoie quelques escadrons garder les bagages et place le reste de la cavalerie aux ailes. Promptement les nôtres poussent la clameur de l’attaque et lancent le javelot. Quand les ennemis, étonnés, virent marcher contre eux ceux qu’ils croyaient en fuite, ils ne purent soutenir le choc et, mis en déroute à la première attaque, ils gagnèrent les forêts voisines. Labiénus lança la cavalerie à leur poursuite, en tua un grand nombre, fit une multitude de prisonniers et, peu de jours après, reçut la soumission de la cité. Quant aux Germains, qui arrivaient en renfort, lorsqu’ils apprirent la déroute des Trévires, ils rentrèrent dans leur pays. Les parents d’Indutiomaros, auteurs de la sédition, s’exilèrent et partirent avec eux. Cingétorix, qui, nous l’avons dit, était resté depuis le début dans le devoir, fut investi de l’autorité civile et militaire.
\subsection[{§ 9.}]{ \textsc{§ 9.} }
\noindent César, quand il fut venu du pays des Ménapes dans celui des Trévires, résolut, pour deux motifs, de passer le Rhin : d’abord parce que les Germains avaient envoyé des secours aux Trévires contre lui, et en second lieu pour qu’Ambiorix ne pût trouver chez eux un refuge. Ayant décidé cette expédition, il entreprend de construire un pont un peu en amont de l’endroit où il avait fait précédemment passer son armée. Le système de construction était connu, on l’avait déjà pratiqué ; les soldats travaillent avec ardeur, et en peu de jours l’ouvrage est achevé. Laissant une forte garde au pont, chez les Trévires, pour éviter qu’une révolte n’éclate soudain de ce côté, il passe le fleuve avec le reste des légions et la cavalerie. Les Ubiens, qui avaient précédemment donné des otages et fait leur soumission, lui envoient des députés pour se justifier ils déclarent que les secours envoyés aux Trévires ne venaient pas de leur cité, que ce n’est point par eux que la foi jurée a été violée ; ils supplient César de les épargner, de ne pas confondre, dans son ressentiment contre les Germains en général, les innocents avec les coupables ; s’il veut plus d’otages, on lui en donnera. César fait une enquête et découvre que ce sont les Suèves qui ont envoyé les renforts ; il accepte les explications des Ubiens, et s’enquiert soigneusement des voies d’accès chez les Suèves.
\subsection[{§ 10.}]{ \textsc{§ 10.} }
\noindent Sur ces entrefaites, peu de jours après, il apprend par les Ubiens que les Suèves concentrent toutes leurs forces et font tenir aux peuples qui sont sous leur dépendance l’ordre d’envoyer des renforts d’infanterie et de cavalerie. A cette nouvelle, il fait des provisions de blé, choisit une bonne position pour y établir son camp, ordonne aux Ubiens de quitter la campagne et de s’enfermer dans les villes avec le bétail et tout ce qu’ils possèdent, il espérait que ces hommes barbares et inexpérimentés, quand ils se verraient près de manquer de vivres, pourraient être amenés à livrer bataille dans des conditions désavantageuses ; il donne mission aux Ubiens d’envoyer de nombreux éclaireurs dans le pays des Suèves et de s’enquérir de ce qui s’y passe. L'ordre est exécuté, et au bout de peu de jours il reçoit le rapport suivant : « Quand les Suèves ont eu des informations sûres au sujet de l’armée romaine, tous, avec toutes leurs troupes et celles de leurs alliés, qu’ils avaient rassemblées, ils se sont retirés très loin, vers l’extrémité de leur territoire ; il y a là une forêt immense, qu’on appelle Bacenis ; elle s’étend profondément vers l’intérieur et forme entre les Suèves et les Chérusques comme un mur naturel qui s’oppose à leurs incursions et à leurs ravages réciproques : c’est à l’entrée de cette forêt que les Suèves ont résolu d’attendre les Romains.
\subsection[{§ 11.}]{ \textsc{§ 11.} }
\noindent Parvenus à cet endroit du récit, il ne nous semble pas hors de propos de décrire les mœurs des Gaulois et des Germains et d’exposer les différences qui distinguent ces deux nations. En Gaule, non seulement toutes les cités, tous les cantons et fractions de cantons, mais même, peut-on dire, toutes les familles sont divisées en partis rivaux ; à la tête de ces partis sont les hommes à qui l’on accorde le plus de crédit ; c’est à ceux-là qu’il appartient de juger en dernier ressort pour toutes les affaires à régler, pour toutes les décisions à prendre. Il y a là une institution très ancienne qui semble avoir pour but d’assurer à tout homme du peuple une protection contre plus puissant que lui : car le chef de faction défend ses gens contre les entreprises de violence ou de ruse, et s’il lui arrive d’agir autrement, il perd tout crédit. Le même système régit la Gaule considérée dans son ensemble tous les peuples y sont groupés en deux grands partis.
\subsection[{§ 12.}]{ \textsc{§ 12.} }
\noindent Quand César arriva en Gaule, un de ces partis avait à sa tête les Héduens, et l’autre les Séquanes. Ces derniers qui, réduits à leurs seules forces, étaient les plus faibles, car les Héduens jouissaient depuis longtemps d’une très grande influence et leur clientèle était considérable, s’étaient adjoint Arioviste et ses Germains, et se les étaient attachés au prix de grands sacrifices et de grandes promesses. Après plusieurs combats heureux, et où toute la noblesse héduenne avait péri, leur prépondérance était devenue telle qu’une grande partie des clients des Héduens passèrent de leur côté, qu’ils se firent donner comme otages les fils des chefs héduens, exigèrent de cette cité l’engagement solennel de ne rien entreprendre contre eux et s’attribuèrent une partie de son territoire contiguë au leur, qu’ils avaient conquise ; qu’enfin ils eurent la suprématie sur la Gaule entière. Réduit à cette extrémité, Diviciacos était allé à Rome demander secours au Sénat, et était revenu sans avoir réussi. L'arrivée de César avait changé la face des choses, les Héduens s’étaient vu restituer leurs otages, avaient recouvré leurs anciens clients, en avaient acquis de nouveaux grâce à César, car ceux qui étaient entrés dans leur amitié constataient qu’ils étaient plus heureux et plus équitablement gouvernés ; enfin ils avaient de toute façon grandi en puissance et en dignité, et les Séquanes avaient perdu leur hégémonie. Les Rèmes avaient pris leur place ; et comme on croyait que ceux-ci étaient également en faveur auprès de César, les peuples à qui de vieilles inimitiés rendaient absolument impossible l’union avec les Héduens se rangeaient dans la clientèle des Rèmes. Ceux-ci les protégeaient avec zèle, et ainsi réussissaient à conserver une autorité qui était pour eux chose nouvelle et qui leur était venue d’un coup. La situation à cette époque était la suivante : les Héduens avaient de loin le premier rang, les Rèmes occupaient le second.
\subsection[{§ 13.}]{ \textsc{§ 13.} }
\noindent Partout en Gaule il y a deux classes d’hommes qui comptent et sont considérés. Quant aux gens du peuple, ils ne sont guère traités autrement que des esclaves, ne pouvant se permettre aucune initiative, n’étant consultés sur rien. La plupart, quand ils se voient accablés de dettes, ou écrasés par l’impôt, ou en butte aux vexations de plus puissants qu’eux, se donnent à des nobles ; ceux-ci ont sur eux tous les droits qu’ont les maîtres sur leurs esclaves. Pour en revenir aux deux classes dont nous parlions, l’une est celle des druides, l’autre celle des chevaliers. Les premiers s’occupent des choses de la religion, ils président aux sacrifices publics et privés, règlent les pratiques religieuses ; les jeunes gens viennent en foule s’instruire auprès d’eux, et on les honore grandement. Ce sont les druides, en effet, qui tranchent presque tous les conflits entre États ou entre particuliers et, si quelque crime a été commis, s’il y a eu meurtre, si un différend s’est élevé à propos d’héritage ou de délimitation, ce sont eux qui jugent, qui fixent les satisfactions à recevoir et à donner ; un particulier ou un peuple ne s’est-il pas conformé à leur décision, ils lui interdisent les sacrifices. C'est chez les Gaulois la peine la plus grave. Ceux qui ont été frappés de cette interdiction, on les met au nombre des impies et des criminels, on s’écarte d’eux, on fuit leur abord et leur entretien, craignant de leur contact impur quelque effet funeste ; ils ne sont pas admis à demander justice, ni à prendre leur part d’aucun honneur. Tous ces druides obéissent à un chef unique, qui jouit parmi eux d’une très grande autorité. A sa mort, si l’un d’entre eux se distingue par un mérite hors ligne, il lui succède si plusieurs ont des titres égaux, le suffrage des druides, quelquefois même les armes en décident. Chaque année, à date fixe, ils tiennent leurs assises en un lieu consacré, dans le pays des Carnutes, qui passe pour occuper le centre de la Gaule. Là, de toutes parts affluent tous ceux qui ont des différends, et ils se soumettent à leurs décisions et à leurs arrêts. On croit que leur doctrine est née en Bretagne, et a été apportée de cette île dans la Gaule ; de nos jours encore ceux qui veulent en faire une étude approfondie vont le plus souvent s’instruire là-bas.
\subsection[{§ 14.}]{ \textsc{§ 14.} }
\noindent Il est d’usage que les druides n’aillent point à la guerre et ne paient pas d’impôt comme les autres ils sont dispensés du service militaire et exempts de toute charge. Attirés par de si grands avantages, beaucoup viennent spontanément suivre leurs leçons, beaucoup leur sont envoyés par les familles. On dit qu’auprès d’eux ils apprennent par cœur un nombre considérable de vers. Aussi plus d’un reste-t-il vingt ans à l’école. Ils estiment que la religion ne permet pas de confier à l’écriture la matière de leur enseignement, alors que pour tout le reste en général, pour les comptes publics et privés, ils se servent de l’alphabet grec. Ils me paraissent avoir établi cet usage pour deux raisons : parce qu’ils ne veulent pas que leur doctrine soit divulguée, ni que, d’autre part, leurs élèves, se fiant à l’écriture, négligent leur mémoire ; car c’est une chose courante quand on est aidé par des textes écrits, on s’applique moins à retenir par cœur et on laisse se rouiller sa mémoire. Le point essentiel de leur enseignement, c’est que les âmes ne périssent pas, mais qu’après la mort elles passent d’un corps dans un autre ; ils pensent que cette croyance est le meilleur stimulant du courage, parce qu’on n’a plus peur de la mort. En outre, ils se livrent à de nombreuses spéculations sur les astres et leurs mouvements, sur les dimensions du monde et celles de la terre, sur la nature des choses, sur la puissance des dieux et leurs attributions, et ils transmettent ces doctrines à la jeunesse.
\subsection[{§ 15.}]{ \textsc{§ 15.} }
\noindent L'autre classe est celle des chevaliers. Ceux-ci, quand il le faut, quand quelque guerre éclate (et avant l’arrivée de César cela arrivait à peu près chaque année, soit qu’ils prissent l’offensive, soit qu’ils eussent à se défendre), prennent tous part à la guerre, et chacun, selon sa naissance et sa fortune, a autour de soi un plus ou moins grand nombre d’ambacts et de clients. Ils ne connaissent pas d’autre signe du crédit et de la puissance.
\subsection[{§ 16.}]{ \textsc{§ 16.} }
\noindent Tout le peuple gaulois est très religieux ; aussi voit-on ceux qui sont atteints de maladies graves, ceux qui risquent leur vie dans les combats ou autrement, immoler ou faire vœu d’immoler des victimes humaines, et se servir pour ces sacrifices du ministère des druides ; ils pensent, en effet, qu’on ne saurait apaiser les dieux immortels qu’en rachetant la vie d’un homme par la vie d’un autre homme, et il y a des sacrifices de ce genre qui sont d’institution publique. Certaines peuplades ont des mannequins de proportions colossales, faits d’osier tressé, qu’on remplit d’hommes vivants : on y met le feu, et les hommes sont la proie des flammes. Le supplice de ceux qui ont été arrêtés en flagrant délit de vol ou de brigandage ou à la suite de quelque crime passe pour plaire davantage aux dieux ; mais lorsqu’on n’a pas assez de victimes de ce genre, on va jusqu’à sacrifier des innocents.
\subsection[{§ 16.}]{ \textsc{§ 16.} }
\noindent Le dieu qu’ils honorent le plus est Mercure : ses statues sont les plus nombreuses, ils le considèrent comme l’inventeur de tous les arts, il est pour eux le dieu qui indique la route à suivre, qui guide le voyageur, il est celui qui est le plus capable de faire gagner de l’argent et de protéger le commerce. Après lui ils adorent Apollon, Mars, Jupiter et Minerve. Ils se font de ces dieux à peu près la même idée que les autres peuples : Apollon guérit les maladies, Minerve enseigne les principes des travaux manuels, Jupiter est le maître des dieux, Mars préside aux guerres. Quand ils ont résolu de livrer bataille, ils promettent généralement à ce dieu le butin qu’ils feront ; vainqueurs, ils lui offrent en sacrifice le butin vivant et entassent le reste en un seul endroit. On peut voir dans bien des cités, en des lieux consacrés, des tertres élevés avec ces dépouilles ; et il n’est pas arrivé souvent qu’un homme osât, au mépris de la loi religieuse, dissimuler chez lui son butin ou toucher aux offrandes : semblable crime est puni d’une mort terrible dans les tourments.
\subsection[{§ 18.}]{ \textsc{§ 18.} }
\noindent Tous les Gaulois se prétendent issus de {\itshape Dis Pater} : c’est, disent-ils, une tradition des druides. En raison de cette croyance, ils mesurent la durée, non pas d’après le nombre des jours, mais d’après celui des nuits ; les anniversaires de naissance, les débuts de mois et d’années, sont comptés en faisant commencer la journée avec la nuit. Dans les autres usages de la vie, la principale différence qui les sépare des autres peuples, c’est que leurs enfants, avant qu’ils ne soient en âge de porter les armes, n’ont pas le droit de se présenter devant eux en public, et c’est pour eux chose déshonorante qu’un fils encore enfant prenne place dans un lieu public sous les yeux de son père.
\subsection[{§ 19.}]{ \textsc{§ 19.} }
\noindent Les hommes, en se mariant, mettent en communauté une part de leurs biens égale, d’après estimation, à la valeur de la dot apportée par les femmes. On fait de ce capital un compte unique, et les revenus en sont mis de côté ; le conjoint survivant reçoit l’une et l’autre part, avec les revenus accumulés. Les maris ont droit de vie et de mort sur leurs femmes comme sur leurs enfants ; toutes les fois que meurt un chef de famille de haute lignée, les parents s’assemblent, et, si la mort est suspecte, on met à la question les épouses comme on fait des esclaves ; les reconnaît-on coupables, elles sont livrées au feu et aux plus cruels tourments. Les funérailles sont, relativement au degré de civilisation des Gaulois, magnifiques et somptueuses ; tout ce qu’on pense que le mort chérissait est porté au bûcher, même des êtres vivants, et, il n’y a pas longtemps encore, la règle d’une cérémonie funèbre complète voulait que les esclaves et les clients qui lui avaient été chers fussent brûlés avec lui.
\subsection[{§ 20.}]{ \textsc{§ 20.} }
\noindent Les cités qui passent pour être particulièrement bien organisées ont des lois qui prescrivent que quiconque a reçu d’un pays voisin quelque nouvelle intéressant l’État doit la faire connaître au magistrat sans en parler à nul autre, parce que l’expérience leur a montré que des hommes qui sont impulsifs et ignorants, souvent, sur de faux bruits, s’effraient, se portent à des excès, prennent les résolutions les plus graves. Les magistrats gardent secret ce qu’ils pensent devoir cacher, livrent à la masse ce qu’ils croient utile de divulguer. On n’a le droit de parler des affaires publiques qu’en prenant la parole dans le conseil.
\subsection[{§ 21.}]{ \textsc{§ 21.} }
\noindent Les mœurs des Germains sont très différentes. En effet, ils n’ont pas de druides qui président au culte des dieux et ils font peu de sacrifices. Ils ne comptent pour dieux que ceux qu’ils voient et dont ils éprouvent manifestement les bienfaits, le Soleil, Vulcain, la Lune ; les autres, ils n’en ont même pas entendu parler. Toute leur vie se passe à la chasse et aux exercices militaires ; dès leur enfance, ils s’entraînent à une existence fatigante et dure. Plus on a gardé longtemps sa virginité, plus on est estimé par son entourage : les uns pensent qu’on devient ainsi plus grand, les autres plus fort et plus nerveux. De fait, connaître la femme avant l’âge de vingt ans est à leurs yeux une honte des plus grandes ; on ne fait pourtant point mystère de ces choses-là, car hommes et femmes se baignent ensemble dans les rivières, et d’ailleurs, ils n’ont d’autres vêtements que des peaux ou de courts rénons qui laissent la plus grande partie du corps à nu.
\subsection[{§ 22.}]{ \textsc{§ 22.} }
\noindent L'agriculture les occupe peu, et leur alimentation consiste surtout en lait, fromage et viande. Personne ne possède en propre une étendue fixe de terrain, un domaine ; mais les magistrats et les chefs de cantons attribuent pour une année aux clans et aux groupes de parents vivant ensemble une terre dont ils fixent à leur gré l’étendue et l’emplacement ; l’année suivante, ils les forcent d’aller ailleurs. Ils donnent plusieurs raisons de cet usage : crainte qu’ils ne prennent goût à la vie sédentaire, et ne négligent la guerre pour l’agriculture ; qu’ils ne veuillent étendre leurs possessions, et qu’on ne voie les plus forts chasser de leurs champs les plus faibles ; qu’ils ne se préoccupent trop de se protéger du froid et de la chaleur en bâtissant des demeures confortables ; que ne naisse l’amour de l’argent, source des divisions et des querelles ; désir enfin de contenir le peuple en le gardant de l’envie, chacun se voyant, pour la fortune, l’égal des plus puissants.
\subsection[{§ 23.}]{ \textsc{§ 23.} }
\noindent Il n’est pas de plus grand honneur pour les peuples germains que d’avoir fait le vide autour de soi et d’être entourés d’espaces désertiques aussi vastes que possible. C'est à leurs yeux la marque même de la vertu guerrière, que leurs voisins, chassés de leurs champs, émigrent, et que personne n’ose demeurer près d’eux ; ils voient là en même temps une garantie de sécurité, puisqu’ils n’ont plus à craindre d’invasion subite. Quand un État a à se défendre ou en attaque un autre, on choisit des magistrats qui conduiront cette guerre et auront le droit de vie et de mort. En temps de paix, il n’y a pas de magistrat commandant à tous, mais les chefs de régions et de cantons rendent la justice et apaisent les querelles chacun parmi les siens. Le vol n’a rien de déshonorant, lorsqu’il est commis hors des frontières de l’État : ils professent que c’est un moyen d’exercer les jeunes gens et de combattre chez eux la paresse. Lorsqu’un chef, dans une assemblée, propose de diriger une entreprise et invite les volontaires à se déclarer, ceux à qui plaisent et la proposition et l’homme promettent leur concours, et ils reçoivent les félicitations de toute l’assistance ; ceux qui par la suite se dérobent, on les tient pour déserteurs et traîtres, et toute confiance leur est désormais refusée. Ne pas respecter un hôte, c’est à leurs yeux commettre un sacrilège : ceux qui, pour une raison quelconque, viennent chez eux, ils les protègent, leur personne leur est sacrée ; toutes les maisons leur sont ouvertes et ils ont place à toutes les tables.
\subsection[{§ 24.}]{ \textsc{§ 24.} }
\noindent Il fut un temps où les Gaulois surpassaient les Germains en bravoure, portaient la guerre chez eux, envoyaient des colonies au-delà du Rhin parce qu’ils étaient trop nombreux et n’avaient pas assez de terres. C'est ainsi que les contrées les plus fertiles de la Germanie, au voisinage de la forêt Hercynienne, forêt dont Eratosthène et certains autres auteurs grecs avaient, à ce que je vois, entendu parler, – ils l’appellent Orcynienne – furent occupées par les Volques Tectosages, qui s’y fixèrent ; ce peuple habite toujours le pays, et il a la plus grande réputation de justice et de valeur militaire. Mais aujourd’hui, tandis que les Germains continuent de mener une vie de pauvreté et de privations patiemment supportées, qu’ils n’ont rien changé à leur alimentation ni à leur vêtement, les Gaulois, au contraire, grâce au voisinage de nos provinces et au commerce maritime, ont appris à connaître la vie large et à en jouir peu à peu, ils se sont accoutumés à être les plus faibles et, maintes fois vaincus, ils renoncent eux-mêmes à se comparer aux Germains pour la valeur militaire.
\subsection[{§ 25.}]{ \textsc{§ 25.} }
\noindent Cette forêt Hercynienne, dont il été question plus haut, a une largeur équivalant à huit journées de marche d’un voyageur légèrement équipé : c’est le seul moyen d’en déterminer les dimensions, les Germains ne connaissant pas les mesures itinéraires. Elle commence aux frontières des Helvètes, des Némètes et des Rauraques, et, en suivant la ligne du Danube, va jusqu’aux pays des Daces et des Anartes ; à partir de là, elle tourne à gauche en s’écartant du fleuve, et, en raison de son étendue, touche au territoire de bien des peuples ; il n’est personne, dans cette partie de la Germanie, qui puisse dire qu’il en a atteint l’extrémité, après soixante jours de marche, ou qu’il sait en quel lieu elle se termine ; il s’y trouve, assure-t-on, beaucoup d’espèces de bêtes sauvages qu’on ne voit pas ailleurs ; celles qui diffèrent le plus des autres et paraissent le plus dignes d’être notées sont les suivantes.
\subsection[{§ 26.}]{ \textsc{§ 26.} }
\noindent Il y a un bœuf ressemblant au cerf, qui porte au milieu du front, entre les oreilles, une corne unique, plus haute et plus droite que les cornes de nous connues ; à son sommet elle s’épanouit en empaumures et rameaux. Mâle et femelle sont de même type, leurs cornes ont même forme et même grandeur.
\subsection[{§ 27.}]{ \textsc{§ 27.} }
\noindent Il y a aussi les animaux qu’on appelle élans. Ils ressemblent aux chèvres et ont même variété de pelage ; leur taille est un peu supérieure, leurs cornes sont tronquées et ils ont des jambes sans articulations : ils ne se couchent pas pour dormir, et, si quelque accident les fait tomber, ils ne peuvent se mettre debout ni même se soulever. Les arbres leur servent de lits : ils s’y appuient et c’est ainsi, simplement un peu penchés, qu’ils dorment. Quand, en suivant leurs traces, les chasseurs ont découvert leur retraite habituelle, ils déracinent ou coupent au ras du sol tous les arbres du lieu, en prenant soin toutefois qu’ils se tiennent encore debout et gardent leur aspect ordinaire. Lorsque les élans viennent s’y accoter comme à leur habitude, les arbres s’abattent sous leur poids, et ils tombent avec eux.
\subsection[{§ 28.}]{ \textsc{§ 28.} }
\noindent Une troisième espèce est celle des urus. Ce sont des animaux dont la taille est un peu au-dessous de celle de l’éléphant, et qui ont l’aspect général, la couleur et la forme du taureau. Ils sont très vigoureux, très agiles, et n’épargnent ni l’homme ni l’animal qu’ils ont aperçu. On s’applique à les prendre à l’aide de pièges à fosse, et on les tue ; cette chasse fatigante est pour les jeunes gens un moyen de s’endurcir, et ils s’y entraînent : ceux qui ont tué le plus grand nombre de ces animaux en rapportent les cornes pour les produire publiquement à titre de preuve, et cela leur vaut de grands éloges. Quant à habituer l’urus à l’homme et à l’apprivoiser, on n’y peut parvenir, même en le prenant tout petit. Ses cornes, par leur ampleur, leur forme, leur aspect, sont très différentes de celles de nos bœufs. Elles sont fort recherchées : on en garnit les bords d’un cercle d’argent, et on s’en sert comme de coupes dans les grands festins.
\subsection[{§ 29.}]{ \textsc{§ 29.} }
\noindent Lorsque César apprit par les éclaireurs ubiens que les Suèves s’étaient retirés dans les forêts, craignant de manquer de blé, car, ainsi que nous l’avons dit, l’agriculture est fort négligée de tous les Germains, il résolut de ne pas aller plus avant ; toutefois, pour ne pas ôter aux Barbares tout sujet de craindre son retour et pour retarder les auxiliaires qu’ils pourraient envoyer en Gaule, une fois ses troupes ramenées il fait couper sur une longueur de deux cents pieds la partie du pont qui touchait à la rive ubienne, et à son extrémité il construit une tour de quatre étages, installe pour assurer la défense du pont une garnison de douze cohortes et fortifie ce lieu de grands travaux. Il donne le commandement de la place au jeune Caïus Volcacius Tullus. Quant à lui, il part, comme les blés commençaient à mûrir, pour aller combattre Ambiorix ; à travers la forêt des Ardennes – c’est la plus grande forêt de toute la Gaule, elle s’étend depuis les bords du Rhin, en pays trévire, jusqu’aux Nerviens, sur plus de cinq cents milles – il envoie en avant Lucius Minucius Basilus et toute la cavalerie, avec ordre de profiter de la rapidité de sa marche et de toute occasion favorable ; il lui recommande d’interdire les feux au campement, pour ne pas signaler de loin son approche ; il l’assure qu’il le suit de près.
\subsection[{§ 30.}]{ \textsc{§ 30.} }
\noindent Basilus se conforme aux ordres reçus. Arrivant après une marche rapide, et qui surprend tout le monde, il s’empare de nombreux ennemis qui travaillaient aux champs sans méfiance ; sur leurs indications, il va droit à Ambiorix, là où, disait-on, il se trouvait avec quelques cavaliers. Le pouvoir de la Fortune est grand en toutes choses, et spécialement dans les événements militaires. Ce fut un grand hasard, en effet, qui permit à Basilus de tomber sur Ambiorix à l’improviste, sans même qu’il fût en garde, et de paraître aux yeux de l’ennemi avant que la rumeur publique ou des messagers l’eussent averti de son approche ; mais ce fut pour Ambiorix une grande chance que de pouvoir, tout en perdant la totalité de son attirail militaire, ses chars et ses chevaux, échapper à la mort. Voici comment cela se fit : sa maison étant entourée de bois selon l’usage général des Gaulois qui, pour éviter la chaleur, recherchent le plus souvent le voisinage des forêts et des rivières, ses compagnons et ses amis purent soutenir quelques instants, dans un passage étroit, le choc de nos cavaliers. Pendant qu’on se battait, un des siens le mit à cheval : les bois protégèrent sa fuite. C'est ainsi qu’il fit successivement mis en péril et sauvé par la toute-puissance de la Fortune.
\subsection[{§ 31.}]{ \textsc{§ 31.} }
\noindent Ambiorix ne rassembla pas ses troupes : le fit-il de propos délibéré, parce qu’il estimait qu’il ne fallait point livrer bataille, ou bien faute de temps et empêché par la soudaine arrivée de notre cavalerie, qu’il croyait suivie du reste de l’armée ? On ne sait ; toujours est-il qu’il envoya de tous côtés dans les campagnes dire que chacun eût à pourvoir à sa sûreté. Une partie se réfugia dans la forêt des Ardennes, une autre dans une région que couvraient sans interruption des marécages ; ceux qui habitaient près de l’océan se cachèrent dans des îles que forment les marées ; beaucoup quittèrent leur pays pour aller se confier, eux et tout ce qu’ils possédaient, à des peuples qu’ils ne connaissaient aucunement. Catuvolcos, roi de la moitié des Eburons, qui s’était associé au dessein d’Ambiorix, affaibli par l’âge et ne pouvant supporter les fatigues de la guerre ou de la fuite, après avoir chargé d’imprécations Ambiorix, auteur de l’entreprise, s’empoisonna avec de l’if, arbre très commun en Gaule et en Germanie.
\subsection[{§ 32.}]{ \textsc{§ 32.} }
\noindent Les Sègnes et les Condruses, peuples de race germanique et comptés parmi les Germains, qui habitent entre les Eburons et les Trévires, envoyèrent des députés à César pour le prier de ne pas les mettre au nombre de ses ennemis et de ne pas considérer tous les Germains d’en deçà du Rhin comme faisant cause commune : « Ils n’avaient pas songé à la guerre, ils n’avaient envoyé aucun secours à Ambiorix. » César, après s’être assuré du fait en interrogeant des prisonniers, leur ordonna de lui amener les Eburons qui pouvaient s’être réfugiés chez eux : « s’ils obéissaient, il respecterait leur territoire. » Après quoi il divisa ses troupes en trois corps et rassembla les bagages de toutes les légions à Atuatuca. C'est le nom d’une forteresse. Elle est située à peu près au centre du pays des Eburons ; c’est là que Titurius et Aurunculéius avaient eu leurs quartiers d’hiver. Ce lieu lui avait paru convenable pour plusieurs raisons, mais particulièrement parce que les fortifications de l’année précédente restaient intactes, ce qui épargnait la peine des soldats. Il laissa pour garder les bagages la quatorzième légion, l’une des trois qui avaient été récemment levées en Italie et emmenées en Gaule. Il confie le commandement de cette légion et du camp à Quintus Tullius Cicéron, et lui donne deux cents cavaliers.
\subsection[{§ 33.}]{ \textsc{§ 33.} }
\noindent Il avait partagé son armée : Titus Labiénus, avec trois légions, reçoit l’ordre de partir vers l’océan, dans la partie du pays qui touche aux Ménapes ; il envoie Caïus Trébonius, avec le même nombre de légions, ravager la contrée qui est contiguë aux Atuatuques ; quant à lui, prenant les trois légions restantes, il décide de marcher vers l’Escaut, qui se jette dans la Meuse, et vers l’extrémité des Ardennes, où on lui disait qu’Ambiorix s’était retiré avec quelques cavaliers. En partant, il assure qu’il sera de retour dans sept jours : il savait que c’était le moment où la légion qu’on laissait dans la forteresse devait recevoir sa ration de blé. Labiénus et Trébonius sont invités à revenir pour la même date, s’ils peuvent le faire sans inconvénient, afin qu’ayant tenu conseil et examiné les intentions de l’ennemi d’après de nouvelles données, on puisse recommencer la guerre sur d’autres plans.
\subsection[{§ 34.}]{ \textsc{§ 34.} }
\noindent Il n’y avait dans le pays, comme nous l’avons dit plus haut, aucune troupe régulière, pas de place forte, pas de garnison prête à se défendre, mais une population qui s’était disséminée de tous côtés. Partout où une vallée secrète, un lieu boisé, un marécage d’accès difficile offrait quelque espoir de protection ou de salut, on y avait cherché asile. Ces retraites, les indigènes qui habitaient dans leur voisinage les connaissaient bien, et il fallait observer une grande prudence, non point pour la sûreté des troupes dans leur ensemble (car, réunies, elles ne pouvaient courir aucun danger de la part d’une population terrifiée et dispersée), mais pour la sûreté individuelle des hommes, ce qui, dans une certaine mesure, importait au salut de l’armée. En effet, beaucoup étaient attirés à de longues distances par l’appât du butin, et comme les chemins, dans les bois, étaient incertains et peu visibles, ils ne pouvaient marcher en troupe. Voulait-on en finir et exterminer cette race de brigands, il fallait fractionner l’armée en un grand nombre de détachements et disperser les troupes ; voulait-on garder les manipules groupés autour de leurs enseignes, selon la règle ordinairement suivie par les armées romaines, la nature même des lieux où se tenaient les Barbares leur était une protection, et ils ne manquaient pas d’audace pour dresser de petites embuscades et envelopper les isolés. On agissait avec toute la prudence dont il était possible d’user dans des conjonctures si délicates, préférant sacrifier quelque occasion de nuire à l’ennemi, malgré le désir de vengeance dont brûlait chacun, plutôt que de lui nuire en sacrifiant un certain nombre de soldats. César envoie des messagers aux peuples voisins il excite chez eux l’espoir du butin et appelle tout le monde au pillage des Eburons : il aimait mieux exposer aux dangers de cette guerre de forêts des Gaulois plutôt que des légionnaires, et il voulait en même temps qu’en punition d’un tel forfait cette grande invasion anéantît la race des Eburons et leur nom mêmes. Des forces nombreuses accoururent bientôt de toutes parts.
\subsection[{§ 35.}]{ \textsc{§ 35.} }
\noindent Tandis que toutes les parties du territoire éburon étaient ainsi livrées au pillage, on approchait du septième jour, date à laquelle César avait décidé qu’il rejoindrait les bagages et la légion. On vit alors quel est à la guerre le pouvoir de la Fortune, et quels graves incidents elle produit. L'ennemi étant dispersé et terrifié, comme nous l’avons dit, il n’y avait devant nous aucune troupe qui pût nous donner le moindre sujet de crainte. Mais au-delà du Rhin parvient aux Germains la nouvelle que l’on pillait les Eburons, et, de plus, que tout le monde y était convié. Les Sugambres, qui sont voisins du fleuve, rassemblent deux mille cavaliers : c’est ce peuple dont nous avons rapporté plus haut qu’il avait recueilli les Tencthères et les Usipètes fugitifs. Ils passent le Rhin à l’aide de barques et de radeaux, à trente milles en aval du lieu où César avait construit un pont et laissé une garde ; ils franchissent la frontière des Eburons, ramassent beaucoup de fuyards qui s’étaient dispersés là, s’emparent d’un nombreux bétail, proie très recherchée des Barbares. Alléchés par le butin, ils poussent plus avant. Les marais, les bois ne sont pas un obstacle pour ces hommes qui sont nés dans la guerre et le brigandage. Ils demandent à leurs prisonniers où est César : ceux-ci répondent qu’il est parti, que toute l’armée s’en est allée. Et l’un d’eux : « Pourquoi, leur dit-il, courir après une proie misérable et chétive, quand une occasion magnifique s’offre à vous ? En trois heures, vous pouvez être à Atuatucal : l’armée romaine a entassé là toutes ses richesses ; pour les garder, une troupe si faible qu’elle ne pourrait même pas garnir la muraille et que personne n’oserait sortir des retranchements. » Devant l’espoir qui leur était offert, les Germains cachent le butin qu’ils avaient fait et se dirigent sur Atuatuca, guidés par le même homme dont ils tenaient cet avis.
\subsection[{§ 36.}]{ \textsc{§ 36.} }
\noindent Cicéron avait, tous les jours précédents, suivant les recommandations de César, très soigneusement retenu les soldats au camp sans même laisser sortir un valet hors du retranchement ; mais le septième jour, n’espérant plus que César observât le délai qu’il avait fixé, car il entendait dire qu’il était allé loin et aucun bruit ne lui parvenait touchant son retour, ému en même temps par les propos de ceux qui disaient que sa prétendue patience les mettait presque en posture d’assiégés, puisqu’on ne pouvait pas sortir du camp, comme enfin il ne pensait pas, quand l’ennemi avait en face de lui neuf légions appuyées par une cavalerie fort importante, et que ses forces étaient dispersées et presque détruites, avoir quelque chose à craindre dans un rayon de trois milles, il envoie cinq cohortes chercher du blé dans les champs les plus proches, qui n’étaient séparés du camp que par une colline. Les légions avaient lassé beaucoup de malades ; ceux qui avaient guéri au cours de la semaine – ils étaient environ trois cents – forment un détachement qui part avec les cohortes ; en outre, un grand nombre de valets, avec beaucoup de bêtes de somme, qui étaient restés au camp, sont autorisés à les suivre.
\subsection[{§ 37.}]{ \textsc{§ 37.} }
\noindent Le hasard voulut que juste à ce moment survînt la cavalerie germaine incontinent, sans changer d’allure, elle essaie de pénétrer dans le camp par la porte décumane, et, comme des bois masquaient la vue de ce côté, on ne la vit pas avant qu’elle ne fût tout près, tant et si bien que les marchands qui avaient dressé leur tente au pied du rempart ne purent se mettre en sûreté. La surprise trouble les nôtres, et c’est à peine si la cohorte de garde soutient le premier choc. L'ennemi se répand tout autour du camp, cherchant un point d’accès. Nos soldats défendent, non sans mal, les portes ; le reste n’a d’autre protection que celle du terrain et du retranchement. L'alarme est partout dans le camp, et on s’interroge à l’envi sur la cause du tumulte : on ne songe pas à prescrire où il faut porter les enseignes, de quel côté chacun doit se diriger. L'un annonce que le camp est pris, l’autre prétend que les Barbares sont venus après une victoire, qu’ils ont détruit l’armée et tué le général ; la plupart sont effrayés par une idée superstitieuse que les lieux à ce moment leur suggèrent : ils se représentent la catastrophe de Cotta et de Titurius, qui sont morts dans ce même poste. Tandis que ces terreurs paralysent tout le monde, les Barbares se persuadent que le prisonnier avait dit vrai, que l’intérieur du camp est vide. Ils s’efforcent d’y faire irruption et s’exhortent mutuellement à ne pas laisser échapper une occasion si belle.
\subsection[{§ 38.}]{ \textsc{§ 38.} }
\noindent Parmi les malades laissés dans la place était Publius Sextius Baculus, qui avait été primipile sous les ordres de César, et dont nous avons parlé à propos de précédents combats : il y avait cinq jours qu’il n’avait pris de nourriture. Inquiet sur son sort et sur celui de tous, il s’avance sans armes hors de sa tente : il voit que l’ennemi est sur nous, que la situation est des plus critiques : il emprunte des armes à ceux qui sont le plus près de lui et va se placer dans la porte. Les centurions de la cohorte de garde se joignent à lui : ensemble, ils soutiennent quelques instants le combat. Sextius, grièvement blessé, perd connaissance ; non sans peine, en le passant de main en main, on le sauve. Ce délai avait permis aux autres de recouvrer assez de sang-froid pour oser prendre position au retranchement et pour fournir l’apparence d’une défense.
\subsection[{§ 39.}]{ \textsc{§ 39.} }
\noindent Sur ces entrefaites, nos moissonneurs, qui avaient achevé leur tâche, entendent des cris : les cavaliers partent en avant, se rendent compte de la gravité du danger. Mais ici, point de retranchement où des soldats effrayés puissent trouver un abri nos hommes, recrues récentes et sans expérience militaire, tournent leurs regards vers le tribun et les centurions ; ils attendent leurs ordres. Le plus brave est troublé par une situation si inattendue. Les Barbares, apercevant au loin les enseignes, cessent l’attaque ; ils croient d’abord au retour des légions dont leurs prisonniers leur avaient dit qu’elles s’étaient fort éloignées ; mais bientôt, pleins de mépris pour une si faible troupe, ils fondent sur elle de tous côtés.
\subsection[{§ 40.}]{ \textsc{§ 40.} }
\noindent Les valets courent au tertre le plus proche. Ils en sont promptement chassés et se jettent au milieu des enseignes et des manipules, ce qui augmente la frayeur de soldats faciles à troubler. Les uns sont d’avis de se former en coin et d’ouvrir vivement un passage, puisque le camp est si près en admettant que quelques-uns soient enveloppés et périssent, du moins pourra-t-on, pensent-ils, sauver le reste ; les autres veulent qu’on s’arrête sur la colline et que tous partagent le même sort. Ce parti n’est point approuvé des vieux soldats qui formaient le détachement dont nous avons parlé. Après de mutuelles exhortations, conduits par Caïus Trébonius, chevalier romain, qui les commandait, ils percent la ligne ennemie et arrivent au camp sans avoir perdu un seul homme. Les valets et la cavalerie, qui s’étaient jetés à leur suite, passent dans la même charge et la vaillance des légionnaires les sauve. Mais ceux qui avaient fait halte sur la colline, n’ayant encore aucune expérience des choses militaires, ne surent ni persévérer dans le dessein qu’ils avaient adopté de se défendre sur la hauteur, ni imiter la vigueur et la rapidité qu’ils avaient vu si bien réussir à leurs camarades : ils essayèrent de rentrer au camp et s’engagèrent sur un terrain bas et désavantageux. Les centurions, dont un certain nombre avaient été promus pour leur valeur des dernières cohortes des autres légions aux premières de celle-ci, ne voulant pas perdre la réputation qu’ils s’étaient acquise, se firent tuer en braves. Quant aux soldats, la vaillance de leurs officiers ayant un peu écarté l’ennemi, une partie d’entre eux put, contre tout espoir, atteindre le camp sans dommage ; les autres furent entourés et massacrés.
\subsection[{§ 41.}]{ \textsc{§ 41.} }
\noindent Les Germains, désespérant d’enlever le camp, parce qu’ils voyaient que les nôtres avaient pris maintenant position au retranchement, se retirèrent au-delà du Rhin en emportant le butin qu’ils avaient déposé dans les bois. Mais même après le départ de l’ennemi, la terreur fut telle que Laïus Volusénus, qui avait été envoyé avec la cavalerie et arriva au camp cette nuit-là, ne pouvait faire croire que César allait être là avec son armée intacte. La frayeur s’était si bien emparée de tous qu’ils en perdaient presque la raison, disant que toutes les troupes avaient été détruites, que la cavalerie avait réussi à échapper, et prétendant que, si l’armée avait été intacte, les Germains n’auraient pas attaqué le camp. L'arrivée de César mit fin à cette panique.
\subsection[{§ 42.}]{ \textsc{§ 42.} }
\noindent Une fois de retour, César, qui n’ignorait pas les hasards de la guerre, se plaignit seulement d’une chose, qu’on eût fait quitter leur poste aux cohortes pour les envoyer hors du camp : il n’aurait pas fallu laisser la moindre place à l’imprévu ; par ailleurs il estima que le rôle de la Fortune avait été grand dans la soudaine arrivée des ennemis, et qu’elle était intervenue plus puissamment encore en écartant les Barbares du retranchement et des portes quand ils en étaient presque maîtres. Le plus étonnant de toute l’affaire, c’était que les Germains, dont le but, en franchissant le Rhin, était de ravager le territoire d’Ambiorix, avaient apporté à celui-ci, parce que les circonstances les avaient conduits au camp romain, le concours le plus précieux qu’il eût pu souhaiter.
\subsection[{§ 43.}]{ \textsc{§ 43.} }
\noindent César, reprenant sa campagne de dévastation, disperse de tous côtés un fort contingent de cavalerie qu’il avait tiré des cités voisines. On incendiait les villages, tous les bâtiments isolés qu’on apercevait, on massacrait le bétail ; partout on faisait du butin ; toute cette multitude de bêtes et d’hommes consommait les céréales, sans compter que la saison avancée et les pluies les avaient couchées en sorte que, même si quelques-uns avaient pu pour le moment échapper en se cachant, on voyait bien qu’ils devraient, une fois l’armée partie, succomber à la disette. Souvent, avec une cavalerie battant le pays dans tous les sens en si nombreux détachements, il arriva qu’on fît des prisonniers qui venaient de voir passer Ambiorix en fuite, et le cherchaient des yeux, assurant qu’il n’était pas encore tout à fait hors de vue : on espérait alors l’atteindre et l’on faisait des efforts infinis ; soutenu par l’idée d’entrer dans les bonnes grâces de César, on dépassait presque la limite des forces humaines, et toujours il s’en fallait d’un rien qu’on n’atteignît le but tant désiré : lui, cependant, trouvait des cachettes ou des bois épais qui le dérobaient, et à la faveur de la nuit il gagnait d’autres contrées, dans une direction nouvelle, sans autre escorte que quatre cavaliers, à qui seuls il osait confier sa vie.
\subsection[{§ 44.}]{ \textsc{§ 44.} }
\noindent Après avoir ainsi dévasté le pays, César ramena son armée, moins les deux cohortes perdues, à Durocortorum des Rèmes ; ayant convoqué dans cette ville l’assemblée de la Gaule, il entreprit de juger l’affaire de la conjuration des Sénons et des Carnutes : Acco, qui en avait été l’instigateur, fut condamné à mort et supplicié selon la vieille coutume romaine. Un certain nombre, craignant d’être également jugés, prirent la fuite. César leur interdit l’eau et le feu ; puis il répartit ses légions en quartiers d’hiver, deux sur la frontière des Trévires, deux chez les Lingons, les six autres dans le pays sénon, à Agédincum, et, après les avoir approvisionnées de blé, il partit pour l’Italie, comme il faisait d’habitude, pour y tenir ses assises.
 \section[{Livre VII}]{Livre VII}\renewcommand{\leftmark}{Livre VII}

\subsection[{§ 1.}]{ \textsc{§ 1.} }
\noindent Voyant la Gaule tranquille, César, ainsi qu’il l’avait décidé, part pour l’Italie afin d’y tenir ses assises. Là, il apprend le meurtre de Publius Clodius et, ayant eu connaissance du sénatus-consulte qui ordonnait l’enrôlement en masse de la jeunesse d’Italie, il entreprend une levée dans toute sa province. La nouvelle de ces événements parvient vite en Transalpine. Les Gaulois y ajoutent de leur propre chef, inventent et répandent une nouvelle qui leur paraissait être le complément naturel de la première : César était retenu par les troubles de Rome, et il ne lui était pas possible de se rendre à l’armée quand la lutte des partis était si vive. L'occasion excite ces hommes qui déjà ne supportaient qu’avec impatience d’être soumis au peuple Romain : ils commencent à faire des projets de guerre avec plus de liberté et de hardiesse. Les chefs gaulois s’entendent pour tenir des conciliabules dans des lieux écartés, au milieu des bois là, ils se plaignent de la mort d’Acco ; ils montrent que ce sort peut devenir le leur ; ils déplorent le malheur commun des Gaulois ; en promettant toutes sortes de récompenses, ils demandent instamment qu’on entre en guerre et qu’on joue sa vie pour rendre à la Gaule sa liberté. « La première chose, disent-ils, à laquelle on doit aviser, c’est de couper César de son armée avant que leurs projets clandestins ne soient divulgués. C'est chose facile, car les légions n’osent pas, en l’absence du chef, sortir de leurs quartiers d’hiver et, de son côté, le chef, sans escorte, ne peut rejoindre ses légions ; et puis mieux vaut mourir en combattant que de ne pas recouvrer l’antique honneur militaire et la liberté que les aïeux ont légués. »
\subsection[{§ 2.}]{ \textsc{§ 2.} }
\noindent Après mainte discussion sur ces projets, les Carnutes déclarent que pour le salut de la patrie il n’est pas de danger qu’ils n’acceptent, et ils promettent d’être au premier rang des révoltés. « Puisque pour le moment on ne peut se garantir mutuellement par un échange d’otages, car cela risquerait d’ébruiter leur projet, que du moins, disent-ils, on s’engage par des serments solennels, autour des étendards réunis en faisceau – cérémonie qui noue, chez eux, le plus sacré des liens – à ne pas les abandonner une fois les hostilités commencées. » On félicite à l’envi les Carnutes ; le serment est prêté par toute l’assistance, et on se sépare après avoir fixé la date du soulèvement.
\subsection[{§ 3.}]{ \textsc{§ 3.} }
\noindent Quand arrive le jour convenu, les Carnutes, entraînés par Cotuatos et Conconnétodumnos, hommes dont on ne pouvait rien attendre que des folies, se jettent, à un signal donné, dans Cénabum, massacrent les citoyens romains qui s’y étaient établis pour faire du commerce, mettent leurs biens au pillage ; parmi eux était Caïus Fufius Cita, honorable chevalier romain, que César avait chargé de l’intendance des vivres. La nouvelle parvient vite à toutes les cités de la Gaule. En effet, quand il arrive quelque chose d’important, quand un grand événement se produit, les Gaulois en clament la nouvelle à travers la campagne dans les différentes directions ; de proche en proche, on la recueille et on la transmet. Ainsi firent-ils alors ; et ce qui s’était passé à Cénabum au lever du jour fut connu avant la fin de la première veille chez les Arvernes, à une distance d’environ cent soixante milles.
\subsection[{§ 4.}]{ \textsc{§ 4.} }
\noindent L'exemple y fut suivi : Vercingétorix, fils de Celtillos, Arverne, jeune homme qui était parmi les plus puissants du pays, dont le père avait eu l’empire de la Gaule et avait été tué par ses compatriotes parce qu’il aspirait à la royautés, convoqua ses clients et n’eut pas de peine à les enflammer. Quand on connaît son dessein, on court aux armes. Gobannitio, son oncle, et les autres chefs, qui n’étaient pas d’avis de tenter la chance de cette entreprise, l’empêchent d’agir ; on le chasse de la ville forte de Gergovie. Pourtant, il ne renonce point, et il enrôle dans la campagne des miséreux et des gens sans aveu. Après avoir réuni cette troupe, il convertit à sa cause tous ceux de ses compatriotes qu’il rencontre ; il les exhorte à prendre les armes pour la liberté de la Gaule ; il rassemble de grandes forces et chasse ses adversaires qui, peu de jours auparavant, l’avaient chassé lui-même. Ses partisans le proclament roi. Il envoie des ambassades à tous les peuples : il les supplie de rester fidèles à la parole jurée. Il ne lui faut pas longtemps pour avoir à ses côtés les Sénons, les Parisii, les Pictons, les Cadurques, les Turons, les Aulerques, les Lémovices, les Andes et tous les autres peuples qui touchent à l’océan. A l’unanimité, on lui confère le commandement suprême. Investi de ces pouvoirs, il exige de tous ces peuples des otages, il ordonne qu’un nombre déterminé de soldats lui soit amené sans délai, il fixe quelle quantité d’armes chaque cité doit fabriquer, et avant quelle date ; il donne un soin particulier à la cavalerie. A la plus grande activité il joint une sévérité extrême dans l’exercice du commandement ; la rigueur des châtiments rallie ceux qui hésitent. Pour une faute grave, c’est la mort par le feu et par toutes sortes de supplices ; pour une faute légère, il fait couper les oreilles au coupable ou lui crever un œil, et il le renvoie chez lui, afin qu’il serve d’exemple et que la sévérité du châtiment subi frappe les autres de terreur.
\subsection[{§ 5.}]{ \textsc{§ 5.} }
\noindent Ayant, par de telles cruautés, rassemblé en peu de temps une armée, il envoie chez les Rutènes, avec une partie des troupes, le Cadurque Luctériosi, homme d’une rare intrépidité, et part lui-même chez les Bituriges. Ceux-ci, à son arrivée, envoient une ambassade aux Héduens, dont ils étaient les clients, pour leur demander de les aider à soutenir l’attaque des ennemis. Les Héduens, sur l’avis des légats que César avait laissés à l’armée, envoient au secours des Bituriges des cavaliers et des fantassins. Quand ceux-ci eurent atteint la Loire, qui sépare les deux peuples, ils s’arrêtèrent, et, au bout de peu de jours, ils s’en retournent sans avoir osé franchir le fleuve ; ils rapportent à nos légats que s’ils ont fait demi-tour, c’est qu’ils craignaient la perfidie des Bituriges, car ils ont appris que leur intention était de les envelopper, eux d’un côté, les Arvernes de l’autre, au cas où ils auraient passé le fleuve. Agirent-ils ainsi pour le motif qu’ils déclarèrent aux légats, ou obéissaient-ils à des pensées de trahison ? N'ayant là-dessus aucune certitude, nous ne croyons pas devoir rien affirmer. Les voyant s’en aller, les Bituriges s’empressent de se joindre aux Arvernes.
\subsection[{§ 6.}]{ \textsc{§ 6.} }
\noindent Quand la nouvelle de ces événements parvint en Italie à César, celui-ci, voyant que désormais la situation intérieure, grâce à la fermeté de Pompée, s’était améliorée, partit pour la Gaule transalpines. Une fois arrivé, il se trouva dans un grand embarras comment parviendrait-il à rejoindre son armée ? Si, en effet, il appelait les légions dans la Province, il voyait qu’elles devraient en chemin livrer bataille sans lui ; s’il allait vers elles, il se rendait compte que, dans les circonstances présentes, il ne pouvait sans imprudence confier sa vie à ceux-là même qui paraissaient tranquilles.
\subsection[{§ 7.}]{ \textsc{§ 7.} }
\noindent Cependant Luctérios le Cadurque, qui avait été envoyé chez les Rutènes, les gagne aux Arvernes. Il pousse chez les Nitiobroges et chez les Gabales, reçoit de chaque peuple des otages, et, ayant réuni une forte troupe, entreprend d’envahir la Province, en direction de Narbonne. A cette nouvelle, César pensa qu’il devait, de préférence à tout autre plan, partir pour Narbonne. Il arrive, il rassure les courages ébranlés, place des détachements chez les Rutènes de la province, chez les Volques Arécomiques, chez les Tolosates et autour de Narbonne, toutes régions qui confinaient au territoire ennemi ; il ordonne qu’une partie des troupes de la province et les renforts qu’il a amenés d’Italie se concentrent chez les Helviens, qui touchent aux Arvernes.
\subsection[{§ 8.}]{ \textsc{§ 8.} }
\noindent Après avoir pris ces dispositions, comme déjà Luctérios arrêtait son mouvement et même reculait, parce qu’il trouvait dangereux de s’aventurer au milieu de nos détachements, César part chez les Helviens. Les Cévennes, qui forment barrière entre les Helviens et les Arvernes, étaient en cette saison, à l’époque la plus rude de l’année, couvertes d’une neige très haute qui interdisait le passage, néanmoins, les soldats fendent et écartent la neige sur une profondeur de six pieds, et, le chemin ainsi frayé au prix des plus grandes fatigues pour les hommes, on débouche dans le pays des Arvernes. Cette arrivée inattendue les frappe de stupeur, car ils se croyaient protégés par les Cévennes comme par un rempart et jamais, à cette époque de l’année, on n’avait vu personne, fût-ce un voyageur isolé, pouvoir en pratiquer les sentiers ; alors César ordonne à ses cavaliers de rayonner le plus loin possible en terrorisant l’ennemi le plus qu’ils peuvent. Rapidement, par la rumeur publique, par des messagers, Vercingétorix apprend ce qui se passe ; tous les Arvernes, au comble de l’émotion, l’entourent, le pressent qu’il pense à défendre leurs biens, qu’il ne laisse pas l’ennemi les piller entièrement, surtout quand – il le voyait bien – tout le poids de la guerre était pour eux. Cédant à leurs prières, il lève le camp et quitte le pays des Bituriges pour se rendre chez les Arvernes.
\subsection[{§ 9.}]{ \textsc{§ 9.} }
\noindent Mais César ne resta que deux jours sur place : il avait prévu que Vercingétorix agirait effectivement de la sorte ; sous prétexte d’aller chercher du renfort et de la cavalerie, il quitte l’armée, laissant le commandement des troupes au jeune Brutus : il lui recommande de faire des incursions de cavalerie de tous côtés, et de les pousser le plus loin possible ; quant à lui, il tâchera de n’être pas absent plus de trois jours. Les choses ainsi réglées, il se dirige à marches forcées vers Vienne, au grand étonnement de son escorte. Il y trouve de la cavalerie fraîche, qu’il y avait envoyée un certain temps auparavant, et, ne cessant de marcher ni jour ni nuit, se dirige, à travers le pays des Héduens, vers celui des Lingons, où deux légions hivernaient : il voulait, au cas où les Héduens iraient jusqu’à tramer quelque plan contre sa vie, en prévenir, par sa rapidité, l’exécution. Une fois arrivé, il envoie des ordres aux autres légions et les concentre toutes sur un seul point avant que les Arvernes aient pu apprendre qu’il était là. Quand il connaît la situation, Vercingétorix, à nouveau, ramène son armée chez les Bituriges, puis quitte leur territoire et se dispose à assiéger Gorgobina, ville des Boïens : César les y avait établis après les avoir vaincus dans la bataille contre les Helvètes, et il les avait placés sous l’autorité des Héduens.
\subsection[{§ 10.}]{ \textsc{§ 10.} }
\noindent Cette manœuvre mettait César dans un grand embarras : s’il gardait ses légions dans leurs quartiers pendant le reste de l’hiver, il devait craindre que, ayant laissé écraser un peuple qui était tributaire des Héduens, la Gaule entière n’entrât en dissidence, puisqu’on verrait que ses amis ne trouvaient en lui aucune protection ; s’il les faisait sortir prématurément, il devait craindre d’avoir à souffrir du côté du ravitaillement, par suite de la difficulté des transports. Il crut qu’il valait mieux néanmoins tout supporter, plutôt que de s’aliéner, en acceptant un tel affront, l’unanimité de ses partisans. Il invite donc les Héduens à lui fournir des vivres, et se fait précéder chez les Boïens d’une ambassade qui annoncera sa venue et les exhortera à rester fidèles, à supporter vaillamment le choc de l’ennemi. Laissant à Agédincum deux légions et les bagages de toute l’armée, il se met en route pour le pays des Boïens.
\subsection[{§ 11.}]{ \textsc{§ 11.} }
\noindent Le second jour, il arriva devant Vellaunodunum, ville des Sénons voulant ne pas laisser d’ennemi derrière lui pour n’être pas gêné dans son ravitaillement, il entreprit d’en faire le siège, et en deux jours, il l’eut entourée d’un retranchement ; le troisième jour, la place envoya des parlementaires pour traiter de la reddition : il ordonne qu’on livre les armes, qu’on amène les chevaux, qu’on fournisse six cents otages. Il laisse Caïus Trébonius, son légat, pour terminer le règlement de cette affaire, et part – car il désirait achever sa route au plus vite – se dirigeant vers Cénabum, ville des Carnutes. Ceux-ci, qui venaient à peine d’apprendre que Vellaunodunum était assiégé, pensant que l’affaire traînerait quelque temps, s’occupaient de rassembler des troupes pour la défense de Cénabum, et se disposaient à les y envoyer. Mais en deux jours César y fut. Il campe devant la ville, et, l’heure avancée lui interdisant de commencer l’attaque, il la remet au lendemain ; il ordonne à ses troupes de faire les préparatifs ordinaires en pareil cas, et, comme il y avait sous les murs de la place un pont qui franchissait la Loire, craignant que les habitants ne prissent la fuite à la faveur de la nuit, il fait veiller deux légions sous les armes. Les gens de Cénabum, peu avant minuit, sortirent en silence de la ville et commencèrent de passer le fleuve. César, averti par ses éclaireurs, introduit, après avoir fait incendier les portes, les deux légions qu’il tenait prêtes, et se rend maître de la place : il s’en fallut d’un bien petit nombre que tous les ennemis ne fussent faits prisonniers, car l’étroitesse du pont et des chemins qui y conduisaient avait bloqué cette multitude en fuite. César pille et brûle la ville, fait don du butin aux soldats, passe la Loire et arrive dans le pays des Bituriges.
\subsection[{§ 12.}]{ \textsc{§ 12.} }
\noindent Dès que Vercingétorix est informé de l’approche de César, il lève le siège de Gorgobina et se porte à sa rencontre. Celui-ci avait entrepris d’assiéger une ville des Bituriges, Noviodunum, qui se trouvait sur sa route. La place lui ayant envoyé des députés pour le supplier de pardonner et d’épargner la vie des habitants, César, soucieux d’achever sa tâche en allant vite, méthode qui lui avait valu la plupart de ses précédents succès, ordonne qu’on livre les armes, qu’on amène les chevaux, qu’on fournisse des otages. Déjà une partie des otages avait été livrée et on procédait à l’exécution des autres clauses – des centurions et quelques soldats avaient été introduits dans la place pour rassembler les armes et les chevaux – quand on aperçut au loin la cavalerie ennemie, qui précédait l’armée de Vercingétorix. A peine les assiégés l’eurent-ils vue et eurent-ils conçu l’espoir d’être secourus qu’une clameur s’éleva et qu’on se mit à courir aux armes, à fermer les portes, à garnir les murailles. Les centurions qui étaient dans la ville, comprenant à l’attitude des Gaulois qu’il y avait quelque chose de changé dans leurs dispositions, mirent l’épée à la main, s’emparèrent des portes et ramenèrent leurs soldats au complet et sans blessures.
\subsection[{§ 13.}]{ \textsc{§ 13.} }
\noindent César fait sortir du camp sa cavalerie et engage la bataille ; puis, les siens étant en difficulté, il envoie à leur secours environ quatre cents Germains qu’il avait coutume, depuis le début de la guerre, d’avoir avec lui. Les Gaulois ne purent supporter leur charge : ils furent mis en déroute et se replièrent sur le gros, non sans avoir subi de lourdes pertes. Ce revers ramena les assiégés à leurs premiers sentiments : pris de peur, ils arrêtèrent ceux qu’ils considéraient comme responsables du mouvement populaire, les amenèrent à César et firent leur soumission. Ayant terminé cette affaire, César partit pour Avaricum, qui était la ville la plus grande et la plus forte du pays des Bituriges, et située dans une région très fertile : il pensait que la prise de cette place lui soumettrait toute la nation des Bituriges.
\subsection[{§ 14.}]{ \textsc{§ 14.} }
\noindent Vercingétorix, après cette suite ininterrompue de revers essuyés à Vellaunodunum, à Cénabum, à Noviodunum, convoque un conseil de guerre. Il démontre qu’il faut conduire les opérations tout autrement qu’on ne l’a fait jusqu’ici : « Par tous les moyens on devra viser à ce but interdire aux Romains le fourrage et les approvisionnements. C'est chose facile, car la cavalerie des Gaulois est très nombreuse, et la saison est leur auxiliaire. Il n’y a pas d’herbe à couper : les ennemis devront donc se disperser pour chercher du foin dans les granges ; chaque jour, les cavaliers peuvent anéantir tous ces fourrageurs. Il y a plus quand on joue son existence, les biens de fortune deviennent chose négligeable ; il faut incendier les villages et les fermes dans toute la zone que les Romains, autour de la route qu’ils suivent, paraissent pouvoir parcourir pour fourrager. Pour eux, ils ont tout en abondance, car les peuples sur le territoire desquels se fait la guerre les ravitaillent ; les Romains, au contraire, ou bien devront céder à la disette, ou bien s’exposeront à de graves dangers en s’avançant à une certaine distance de leur camp ; que d’ailleurs on les tue ou qu’on leur enlève leurs bagages, cela reviendra au même, car sans ses bagages une armée ne peut faire campagne. Ce n’est pas tout : il faut encore incendier les villes que leurs murailles et leur position ne mettent pas à l’abri de tout danger, afin qu’elles ne servent pas d’asile aux déserteurs et qu’elles n’offrent pas aux Romains l’occasion de se procurer des quantités de vivres et de faire du butin. Trouvent-ils ces mesures dures, cruelles ? Ils doivent trouver bien plus dur encore que leurs enfants et leurs femmes soient emmenés en esclavage ; et qu’eux-mêmes soient égorgés car c’est là le sort qui attend fatalement les vaincus. »
\subsection[{§ 15.}]{ \textsc{§ 15.} }
\noindent D'un accord unanime, on approuve cet avis : en un seul jour, plus de vingt villes des Bituriges sont incendiées. On fait de même chez les autres peuples d’alentour ; de tous côtés, on aperçoit des incendies. C'était pour tous une grande douleur ; mais ils se consolaient par cette pensée que, la victoire étant presque une chose assurée, ils recouvreraient avant longtemps ce qu’ils avaient perdu. On délibère en conseil de guerre sur Avaricum : veut-on brûler la ville ou la défendre ? Les Bituriges se jettent aux pieds des chefs des diverses nations, suppliant qu’on ne les force point de mettre le feu de leurs mains à une ville qui est, ou peu s’en faut, la plus belle de toute la Gaule, qui est la force et l’ornement de leur pays ; il leur sera facile, vu sa position, de la défendre, car presque de tous côtés elle est entourée par l’eau courante et le marais, et n’offre qu’un accès, qui est d’une extrême étroitesse. On cède à leurs prières Vercingétorix s’y était d’abord opposé, puis s’était laissé fléchir, ému par les supplications des chefs bituriges, et par la commisération générale. On choisit pour la défense de la place les hommes qu’il faut.
\subsection[{§ 16.}]{ \textsc{§ 16.} }
\noindent Vercingétorix suit César à petites étapes et choisit pour son camp une position couverte par des marécages et des bois, à seize mille pas d’Avaricum. Là, un service régulier de liaison lui permettait de connaître heure par heure les péripéties du siège et de transmettre ses ordres. Il guettait nos détachements qui allaient chercher du fourrage et du blé, et si, poussés par la nécessité, ils s’avançaient un peu trop loin, il les attaquait et leur causait des pertes sérieuses, bien qu’il prissent toutes les précautions possibles, ne sortant pas à intervalles réguliers ni par les mêmes chemins.
\subsection[{§ 17.}]{ \textsc{§ 17.} }
\noindent César campa devant la ville du côté où les cours d’eau et les marais laissaient, comme nous l’avons dit, un étroit passage, et il entreprit de construire une terrasse, de faire avancer des mantelets, d’élever deux tours ; car la nature du terrain interdisait la circonvallation. Pour le blé, il harcèle de demandes les Boïens et les Héduens ; ceux ci, manquant de zèle, n’apportaient qu’une aide médiocre ; ceux-là manquaient de moyens, car ils ne formaient qu’un petit État de faibles ressources et ils eurent tôt fait d’épuiser ce qu’ils possédaient. L'armée souffrait d’une grande disette, à cause de la pauvreté des Boïens, de la mauvaise volonté des Héduens, et parce qu’on avait mis le feu aux granges : ce fut au point que pendant de longs jours les soldats manquèrent de pain, et n’échappèrent aux horreurs de la famine que grâce à quelque bétail qu’on amena de lointains villages ; pourtant, dans cette situation, on ne leur entendit pas proférer une parole qui fût indigne de la majesté du peuple Romain et de leurs précédentes victoires. Bien plus, comme César, visitant les travaux, adressait la parole à chaque légion et disait que si les privations leur étaient trop pénibles, il renoncerait au siège, ce fut un cri unanime pour le prier de n’en rien faire : « Ils avaient pendant de longues années servi sous ses ordres sans subir aucun affront, sans jamais s’en aller en laissant inachevé ce qu’ils avaient entrepris : ils considéreraient comme un déshonneur d’abandonner le siège commencé ; ils aimaient mieux tout souffrir plutôt que de ne pas venger les citoyens romains qui, à Cénabum, avaient été victimes de la perfidie des Gaulois. » Ils exprimaient aux centurions et aux tribuns les mêmes sentiments, afin que César en fût informé par eux.
\subsection[{§ 18.}]{ \textsc{§ 18.} }
\noindent Déjà les tours étaient proches du rempart, quand César apprit par des prisonniers que Vercingétorix, n’ayant plus de fourrage, avait rapproché son camp d’Avaricum, qu’il avait pris en personne, le commandement de la cavalerie et de l’infanterie légère exercée à combattre parmi les cavaliers, et était parti pour dresser une embuscade à l’endroit où il pensait que les nôtres viendraient fourrager le lendemain. A cette nouvelle, César partit au milieu de la nuit en silence et parvint le matin au camp des ennemis. Mais leurs éclaireurs les avaient rapidement avertis de son approche : ils cachèrent leurs chariots et leurs bagages dans l’épaisseur des forêts, et rangèrent toutes leurs troupes sur un lieu élevé et découverte. Quand César l’apprit, il fit promptement rassembler les bagages et prendre la tenue de combat.
\subsection[{§ 19.}]{ \textsc{§ 19.} }
\noindent La position de l’ennemi était une colline qui s’élevait en pente douce. Elle était entourée presque de toutes parts d’un marais difficile à traverser et plein d’obstacles, dont la largeur n’excédait pas cinquante pieds. Les Gaulois avaient coupé les passages et, confiants dans la force de leur position, ne bougeaient pas de leur colline ; rangés par cités, ils occupaient solidement tous les gués et tous les fourrés de ce marais, prêts, au cas où les Romains essaieraient de le franchir, à profiter de leur embarras pour fondre sur eux du haut de leur colline : qui ne voyait que la proximité des deux armées croyait les Gaulois disposés à engager le combat à armes à peu près égales ; mais pour qui se rendait compte de l’inégalité des positions, leur contenance apparaissait comme une vaine parade. Les soldats s’indignaient que l’ennemi pût, à une si courte distance, soutenir leur vue, et ils réclamaient le signal du combat ; mais César leur explique ce que coûtera, nécessairement, la victoire, combien de braves il y faudra sacrifier ; devant tant de résolution, quand ils acceptent tous les dangers pour sa gloire, il mériterait d’être taxé de monstrueux égoïsme, si leur vie ne lui était plus précieuse que la sienne propre. Ayant calmé les soldats par ces paroles, il les ramène au camp le jour même, et prend les dernières mesures pour l’assaut de la place.
\subsection[{§ 20.}]{ \textsc{§ 20.} }
\noindent Vercingétorix, de retour auprès des siens, se vit accuser de trahison : « Il avait porté son camp plus près des Romains, il était parti avec toute la cavalerie, il avait laissé des forces si importantes sans leur donner de commandant en chef, enfin les Romains, après son départ, étaient arrivés bien a propos et bien vite ; tout cela n’avait pu se produire par l’effet du hasard et sans être voulu ; il aimait mieux régner sur la Gaule par concession de César que de leur en être redevable. » A de telles accusations, il répondit en ces termes : « Il avait déplacé le camp : c’était parce que le fourrage manquait, et eux-mêmes y avaient poussé. Il s’était rapproché des Romains : il y avait été déterminé par les avantages de la position, qui se défendait d’elle-même, sans qu’on eût à la fortifier. La cavalerie, il n’y avait pas eu lieu, sur un terrain marécageux, d’en regretter les services, et elle avait été utile là où il l’avait menée. Le commandement en chef, c’était à dessein qu’il ne l’avait confié à personne, par crainte que le chef désigné, cédant aux désirs de la multitude, ne livrât bataille, ce qui, il le voyait bien, était le vœu de tous, à cause de leur manque d’énergie, parce qu’ils étaient incapables d’endurer la fatigue un peu longtemps. Si l’intervention des Romains est due au hasard, il faut en remercier la Fortune ; si elle est due à quelque indicateur, c’est à lui qu’il faut rendre grâce car ils ont pu, de leur position dominante, juger du petit nombre et de la pitoyable valeur de ces soldats qui n’ont pas osé combattre et, honteusement, ont regagné leur camp. Il n’a pas besoin de recevoir de César, en trahissant, une autorité que peut lui donner la victoire, désormais assurée pour lui et pour tous les Gaulois ; et d’ailleurs, ce pouvoir, il le remet entre leurs mains, s’ils croient lui faire plus d’honneur qu’il ne leur apporte de chances de salut. Pour vous rendre compte, ajoute-t-il, que je dis vrai, écoutez ce que vont vous dire des soldats romains. » Il fait comparaître des esclaves qu’il avait pris peu de jours avant tandis qu’ils faisaient du fourrage et qu’il avait soumis à la torture de la faim et des chaînes. On leur avait, au préalable, fait la leçon, ils savaient ce qu’ils devaient dire quand ils seraient interrogés : ils déclarent qu’ils sont des soldats légionnaires, que la faim, la détresse les ont poussés à quitter le camp en secret, pour tâcher de trouver dans les champs un peu de blé ou de bétail : « Toute l’armée est dans la même détresse, chacun est à bout de forces, on ne peut plus supporter la fatigue des travaux ; aussi le général a-t-il décidé de lever le siège dans trois jours, si l’on n’a pas obtenu de résultat ». – « Voilà, dit alors Vercingétorix, ce que vous devez à l’homme que vous accusez de trahison : grâce à moi, sans qu’il vous en ait coûté une goutte de sang, vous voyez une grande armée victorieuse anéantie par la famine ; et le jour où, honteusement, elle fuira et cherchera un asile, j’ai pris mes dispositions pour qu’aucun peuple ne l’accueille sur son territoire. »
\subsection[{§ 21.}]{ \textsc{§ 21.} }
\noindent La foule entière pousse des clameurs et agite bruyamment ses armes, ce qui est leur façon de faire quand ils approuvent un orateur : « Vercingétorix est un grand chef, sa loyauté est au-dessus de tout soupçon, il est impossible de conduire la guerre avec plus d’habileté. » On décide d’envoyer dans la place dix mille hommes choisis dans toute l’armée, estimant qu’il ne faut pas laisser aux seuls Bituriges le soin du salut commun on se rendait compte, en effet, que, s’ils conservaient la ville, ce serait à eux qu’appartiendrait la victoire décisive.
\subsection[{§ 22.}]{ \textsc{§ 22.} }
\noindent A l’exceptionnelle valeur de nos soldats les Gaulois opposaient toutes sortes de moyens : c’est une race d’une extrême ingéniosité et ils ont de singulières aptitudes à imiter et à exécuter ce qu’ils voient faire par d’autres. A l’aide de lacets, ils détournaient les coups de nos faux, et quand ils les avaient bien serrées dans leurs nœuds, ils les tiraient avec des machines à l’intérieur des remparts ; ils faisaient écrouler notre terrassement en creusant des sapes, d’autant plus savants dans cet art qu’il y a chez eux de grandes mines de fer et qu’ils connaissent et emploient tous les genres de galeries souterraines. Ils avaient garni toute l’étendue de leurs murailles de tours reliées par un plancher et protégées par des peaux. De plus, faisant souvent, de jour et de nuit, des sorties, ou bien ils mettaient le feu à notre terrasse, ou bien ils attaquaient nos soldats en train de travailler ; à mesure que l’avance quotidienne de nos travaux augmentait la hauteur de nos tours, ils haussaient les leurs à proportion en reliant entre eux les poteaux verticaux qui en constituaient l’ossature ; ils entravaient l’achèvement de nos galeries en lançant dans les parties encore découvertes des pièces de bois taillées en pointe et durcies au feu, de la poix bouillante, des pierres énormes, et nous interdisaient ainsi de les prolonger jusqu’au pied des murs.
\subsection[{§ 23.}]{ \textsc{§ 23.} }
\noindent Tous les murs gaulois sont faits, en général, de la manière suivante. On pose sur le sol, sans interruption sur toute la longueur du mur, des poutres perpendiculaires à sa direction et séparées par des intervalles égaux de deux pieds. On les relie les unes aux autres dans la fondation, et on les recouvre d’une grande quantité de terre ; le parement est formé de grosses pierres encastrées dans les intervalles dont nous venons de parler. Ce premier rang solidement établi, on élève par dessus un deuxième rang semblable, en conservant le même intervalle de deux pieds entre les poutres, sans que cependant pour cela elles touchent celles du rang inférieur ; mais elles en sont séparées par un espace de deux pieds aussi, et chaque poutre est ainsi isolée de ses voisines par une pierre, ce qui la fixe solidement. On continue toujours de même jusqu’à ce que le mur ait atteint la hauteur voulue. Ce genre d’ouvrage offre un aspect varié qui n’est pas désagréable à l’œil, avec son alternance de poutres et de pierres, celles-ci n’en formant pas moins des lignes continues qui se coupent à angle droit ; il est, de plus, très pratique et parfaitement adapté à la défense des villes, car la pierre le défend du feu et le bois des ravages du bélier, celui-ci ne pouvant ni briser, ni disjoindre une charpente où les pièces qui forment liaison à l’intérieur ont en général quarante pieds d’un seul tenant.
\subsection[{§ 24.}]{ \textsc{§ 24.} }
\noindent Tout cela mettait obstacle au siège ; les soldats étaient, en outre, retardés dans leurs travaux par un froid opiniâtre et des pluies continuelles ; ils surent néanmoins, en travaillant sans relâche, venir à bout de toutes ces difficultés, et en vingt-cinq jours ils construisirent une terrasse qui avait trois cents trente pieds de large et quatre-vingts pieds de haut. Elle touchait presque le rempart ennemi, et César, qui selon son habitude passait la nuit sur le chantier, exhortait ses soldats à ne pas perdre un instant, quand peu avant la troisième veille on remarqua qu’une fumée s’élevait de la terrasse ; l’ennemi y avait mis le feu par une mine. Au même moment, tout le long du rempart une clameur s’élevait, et les ennemis faisaient une sortie par deux portes, de chaque côté des tours. D'autres jetaient du haut du mur sur la terrasse des torches et du bois sec, ils versaient de la poix et tout ce qui était de nature à activer l’incendie il était difficile, dans ces conditions, de régler la défense, de décider où il fallait d’abord se porter et à quel danger il fallait parer. Pourtant, comme, par ordre de César, deux légions veillaient toujours devant le camp, et que des forces plus considérables travaillaient au chantier en se relayant, la défense s’organisa vite les uns tenaient tête aux ennemis qui débouchaient des portes, les autres ramenaient les tours en arrière et faisaient une tranchée dans le terrassement, tandis que tout ce qui était au camp accourait pour éteindre le feu.
\subsection[{§ 25.}]{ \textsc{§ 25.} }
\noindent Le reste de la nuit s’était écoulé et on combattait encore sur tous les points ; l’espoir de vaincre se ranimait sans cesse chez l’ennemi, d’autant plus qu’il voyait les mantelets des tours consumés par le feu, qu’il se rendait compte de la difficulté qu’éprouvaient les nôtres pour venir, à découvert, au secours de leurs camarades, et que sans cesse, de leur côté, des troupes fraîches remplaçaient les troupes fatiguées ; tout le sort de la Gaule leur paraissait dépendre de cet instant. Il se produisit alors à nos regards quelque chose qui nous parut digne de mémoire, et que nous n’avons pas cru devoir passer sous silence. Il y avait devant une porte un Gaulois qui jetait vers la tour en feu des boules de suif et de poix qu’on lui passait de main en main ; un trait parti d’un scorpion, lui perça le côté droit et il tomba sans connaissance. Un de ses voisins, enjambant son corps, le remplaça dans sa besogne ; il tomba de même, frappé à son tour par le scorpion ; un troisième lui succéda, et au troisième un quatrième ; et le poste ne cessa d’être occupé par des combattants jusqu’au moment où, l’incendie ayant été éteint et les ennemis repoussés sur tout le front de bataille, le combat prit fin.
\subsection[{§ 26.}]{ \textsc{§ 26.} }
\noindent Ayant tout essayé, et toujours sans succès, les Gaulois, le lendemain, décidèrent d’abandonner la ville : Vercingétorix les y exhortait, le leur ordonnait. En tâchant d’effectuer cette opération dans le silence de la nuit, ils espéraient y réussir sans trop de pertes, parce que le camp de Vercingétorix n’était pas loin de la place, et que le marécage qui formait entre celle-ci et les Romains une ligne continue retarderait la poursuite. Ils faisaient déjà leurs préparatifs, la nuit venue, quand soudain les mères de famille accoururent sur les places et se jetant, en larmes, à leurs pieds, les supplièrent de mille façons de ne pas les livrer à la cruauté de l’ennemi, elles et leur commune progéniture, à qui la faiblesse du sexe ou de l’âge ne permettait pas la fuite. Quand elles les virent inflexibles – en général, dans les cas de péril extrême, l’âme en proie à la peur reste inaccessible à la pitié – elles se mirent à crier toutes ensemble et à signaler aux Romains le projet de fuite. Alors les Gaulois, craignant que la cavalerie romaine ne leur coupât la route, renoncèrent à leur dessein.
\subsection[{§ 27.}]{ \textsc{§ 27.} }
\noindent Le lendemain César fit avancer une tour et redresser les terrassements qu’il avait entrepris ; là-dessus il se mit à pleuvoir abondamment, et ce temps lui parut favorable pour décider l’attaque, car il apercevait quelque relâchement dans la garde du rempart ; il dit à ses soldats de ralentir leur travail, et leur fit connaître ce qu’il attendait d’eux. Il réunit secrètement les légions, en tenue de combat, en deçà des baraques, et les exhorta à cueillir enfin après tant de fatigues le fruit de la victoire ; il promit des récompenses pour ceux qui auraient les premiers escaladé le rempart, et donna le signal de l’assaut. Ils bondirent soudain de toutes parts et eurent vite fait de garnir la muraille.
\subsection[{§ 28.}]{ \textsc{§ 28.} }
\noindent Les ennemis, effrayés par ce coup inattendu, furent chassés du mur et des tours ; ils se reformèrent sur le forum et sur les places, résolus à faire front du côté où viendrait l’attaque et à livrer une bataille rangée. Mais quand ils virent qu’au lieu de descendre lutter de plain-pied nos soldats les enveloppaient en occupant toute la muraille, ils craignirent de se voir ôter toute chance de retraite et, jetant leurs armes, ils gagnèrent d’un seul élan l’extrémité de la ville ; là, comme ils se pressaient devant l’étroite issue des portes, nos fantassins les massacrèrent, tandis que ceux qui étaient déjà sortis tombaient sous les coups de nos cavaliers. Personne ne pensa au butin ; excités par le souvenir du carnage de Cénabum et par les fatigues du siège, ils n’épargnèrent ni les vieillards, ni les femmes, ni les enfants. Bref, d’un ensemble d’environ quarante mille hommes, à peine huit cents, qui s’enfuirent hors de la ville aux premiers cris, arrivèrent sains et saufs auprès de Vercingétorix. Celui-ci, craignant que leur arrivée tumultueuse et l’émotion que leur vue provoquerait dans une foule impressionnable ne fussent cause d’une émeute, les reçut en pleine nuit et silencieusement, ayant pris soin de disposer sur la route, à bonne distance du camp, ses compagnons d’armes et les chefs des cités, qui avaient mission de les trier et de conduire chaque groupe vers les divers quartiers assignés au début de la campagne à chaque peuple.
\subsection[{§ 29.}]{ \textsc{§ 29.} }
\noindent Le lendemain ayant convoqué le conseil, il apporta aux siens consolations et encouragements, les invitant à ne pas se laisser abattre ni bouleverser pour un revers : « Ce n’est point par leur valeur et en bataille rangée que les Romains ont triomphé, mais grâce à une technique, à un art des sièges qui ont surpris l’ignorance des Gaulois. On se trompe, si l’on s’attend, dans la guerre, à n’avoir que des succès. Pour lui, il n’a jamais été d’avis de défendre Avaricum, eux-mêmes en sont témoins ; le malheur est dû au manque de sagesse des Bituriges et à l’excessive complaisance des autres. N'importe, il aura vite fait de le réparer par de plus importants succès. Les peuples gaulois qui se tiennent encore à l’écart entreront, par ses soins, dans l’alliance, et il fera de toute la Gaule un faisceau de volontés communes auquel le monde entier même sera incapable de résister ; ce résultat, il l’a déjà presque atteint. En attendant, il est juste qu’ils veuillent bien, pour le salut de tous, se mettre à fortifier le camp, afin d’être mieux à même de résister aux attaques soudaines de l’ennemi. »
\subsection[{§ 30.}]{ \textsc{§ 30.} }
\noindent Ce discours ne déplut pas aux Gaulois : on lui savait gré surtout de n’avoir pas perdu courage après un coup si rude, de ne s’être point caché ni dérobé aux regards : on lui reconnaissait des dons supérieurs de discernement et de prévision, parce qu’il avait été d’avis, alors que la situation était entière, d’abord d’incendier Avaricum, puis de l’abandonner. Aussi, tandis que les autres chefs voient les revers diminuer leur autorité, lui, au contraire, après un échec, grandissait de jour en jours. En même temps, ses assurances faisaient naître l’espoir que les autres cités entreraient dans l’alliance ; les Gaulois se mirent alors, pour la première fois, à fortifier leur camp le choc avait été si rude que ces hommes qui n’étaient pas habitués au travail pensaient devoir se soumettre à tout ce qu’on leur commandait.
\subsection[{§ 31.}]{ \textsc{§ 31.} }
\noindent Cependant Vercingétorix, comme il l’avait promis, faisait tous ses efforts pour adjoindre à la coalition les autres cités, et cherchait à en gagner les chefs par des présents et des promesses. Il choisissait pour atteindre ce but les auxiliaires les plus qualifiés, ceux à qui l’habitude de leur éloquence ou leurs relations d’amitié donnaient le plus de moyens de séduction. Il s’occupe, d’autre part, d’équiper et d’habiller les soldats qui avaient pu s’échapper lors de la prise d’Avaricum ; pour réparer les pertes de ses effectifs, il demande aux différents peuples de lui fournir un certain nombre de soldats, fixant le chiffre et la date avant laquelle il veut les voir amener dans son camp ; en outre, il ordonne qu’on recrute et qu’on lui envoie tous les archers, qui étaient très nombreux en Gaule. De semblables mesures lui permettent de combler rapidement les pertes d’Avaricum. C'est sur ces entrefaites que Teutomatos, fils d’Ollovico et roi des Nitiobroges, dont le père avait reçu du Sénat le titre d’ami, vint le rejoindre avec une forte troupe de cavaliers de sa nation et des mercenaires qu’il avait recrutés en Aquitaine.
\subsection[{§ 32.}]{ \textsc{§ 32.} }
\noindent César demeura plusieurs jours à Avaricum, et y trouva une grande abondance de blé et d’autres vivres ; il permit ainsi à son armée de se remettre de ses fatigues et de ses privations. On était déjà presque à la fin de l’hiver ; la saison invitait à se mettre en campagne, et d’ailleurs César avait résolu de marcher à l’ennemi, pour le faire sortir de ses marécages et de ses forêts, ou bien l’y assiéger, quand une députation de nobles héduens vient le trouver pour implorer son aide dans des circonstances particulièrement critiques : « La situation est des plus graves : alors que l’antique usage veut qu’on ne nomme qu’un magistrat suprême, qui détient pendant un an le pouvoir royal, deux hommes exercent cette magistrature et chacun d’eux se prétend légalement nommé. L'un est Convictolitavis, jeune homme riche et de naissance illustre ; l’autre est Cotos, issu d’une très vieille famille, jouissant d’ailleurs d’une grande influence personnelle et ayant de nombreux parents ; son frère Valétiacos a rempli l’année précédente la même charge. Tout le pays est en armes ; le sénat est divisé, le peuple est divisé, les clients des deux rivaux forment deux partis ennemis. Si le conflit dure, on verra les deux moitiés de la nation en venir aux mains. Il dépend de César d’empêcher ce malheur par une enquête attentive et par le poids de son intervention. »
\subsection[{§ 33.}]{ \textsc{§ 33.} }
\noindent César pensait qu’il y avait des inconvénients à interrompre les opérations et à abandonner l’ennemi ; mais il savait aussi quels maux engendrent les discordes et il ne voulait pas qu’une si grande nation, et si étroitement unie à Rome, que personnellement il avait toujours favorisée et comblée d’honneurs, en vînt à la guerre civile, et qu’alors le parti qui se croirait le moins fort demandât du secours à Vercingétorix : il jugea donc qu’il fallait d’abord parer à cela, et comme les lois des Héduens interdisaient à ceux qui géraient la magistrature suprême de franchir les frontières, voulant éviter de paraître porter atteinte à la constitution du pays, il décida de s’y rendre lui-même, et il convoqua tout le sénat et les deux compétiteurs à Decize. Presque toute la cité y vint ; il apprit que Cotos était l’élu d’une poignée d’hommes réunis en secret ailleurs et à un autre moment qu’il ne convenait, que le frère avait proclamé l’élection du frère, alors que les lois interdisaient que deux membres d’une même famille fussent l’un du vivant de l’autre, non seulement nommés magistrats, mais même admis au sénat. Il obligea Cotos à déposer le pouvoir, et invita Convictolitavis, qui avait été nommé, conformément aux usages, sous la présidence des prêtres et alors que la magistrature était vacante, à prendre le pouvoir.
\subsection[{§ 34.}]{ \textsc{§ 34.} }
\noindent Cette décision étant intervenue, il exhorta les Héduens à oublier discussions et querelles, à tout laisser pour se consacrer à la présente guerre ; il leur promit qu’ils recevraient de lui, une fois la Gaule vaincue, les récompenses qu’ils auraient méritées ; il les invita à lui envoyer sans retard toute leur cavalerie, et dix mille fantassins qu’il répartirait dans divers postes pour la protection des convois de vivres. Il fit ensuite deux parts de son armée quatre légions furent confiées à Labiénus pour marcher contre les Sénons et les Parisii, et il mena lui-même les six autres chez les Arvernes, vers la ville de Gergovie, en suivant l’Allier ; il donna une partie de la cavalerie à Labiénus et garda l’autre part. Quand Vercingétorix apprit ces nouvelles, il coupa tous les ponts de l’Allier et se mit à remonter le fleuve sur la rive opposée.
\subsection[{§ 35.}]{ \textsc{§ 35.} }
\noindent Les deux armées se voyaient l’une l’autre et campaient généralement face à face ; et comme Vercingétorix disposait des éclaireurs pour empêcher les Romains de faire un pont et de franchir le fleuve, César se trouvait dans une situation fort difficile : il risquait d’être arrêté par l’Allier la plus grande partie de l’été, car ce n’est guère avant l’automne que, d’habitude, l’Allier est guéable. Pour éviter qu’il en fût ainsi, César alla camper dans une région boisée en face de l’un des ponts que Vercingétorix avait fait détruire, et le lendemain il y demeura secrètement avec deux légions, tandis qu’il faisait partir comme à l’habitude le reste de ses troupes avec tous les bagages, ayant eu soin de fractionner un certain nombre de cohortes pour faire croire que le nombre des légions n’avait pas changé. Il leur donna l’ordre de se porter aussi loin que possible en avant, et quand l’heure lui fit supposer qu’elles étaient arrivées au campement, il se mit à rétablir le pont sur les anciens pilotis, dont la partie inférieure restait entière. L'ouvrage fut rapidement terminé ; il fit passer les légions et, ayant choisi un emplacement favorable pour son camp, rappela à lui les autres corps. Quand Vercingétorix apprit la chose, craignant d’être obligé à livrer bataille malgré lui, il força les étapes pour prendre de l’avance.
\subsection[{§ 36.}]{ \textsc{§ 36.} }
\noindent César parvint à Gergovie en quatre étapes ; ayant livré le jour de son arrivée un petit combat de cavalerie, et ayant reconnu la place, qui était sur une montagne fort haute et d’accès partout difficile, il désespéra de l’enlever de force ; quant à un siège, il décida de n’y songer qu’après avoir pourvu aux subsistances. De son côté, Vercingétorix avait campé près de la ville, sur la hauteur, et il avait disposé autour de lui les forces de chaque cité, en ne les séparant que par un léger intervalle tous les sommets de cette chaîne que la vue découvrait étaient occupés par ses troupes, en sorte qu’elles offraient un spectacle terrifiant. Ceux des chefs de cités qu’il avait choisis pour former son conseil étaient convoqués par lui chaque jour à la première heure pour les décisions à prendre ou les mesures à exécuter ; et il ne se passait presque point de jour qu’il n’éprouvât, par des engagements de cavalerie auxquels se mêlaient les archers, l’ardeur et la valeur de chacun. Il y avait en face de la ville, au pied même de la montagne, une colline très bien fortifiée par la nature, et isolée de toutes parts : si nous l’occupions, nous priverions l’ennemi d’une grande partie de son eau et il ne fourragerait plus librement. Mais cette position était tenue par une garnison qui n’était pas méprisable. Pourtant César, étant sorti de son camp au milieu du silence de la nuit, bouscula les défenseurs avant que l’on eût pu les secourir de la place et, maître de la position, y installa deux légions ; il relia le petit camp au grand camp par un double fossé de douze pieds de large, afin que même des hommes isolés pussent aller de l’un à l’autre à l’abri des surprises de l’ennemi.
\subsection[{§ 37.}]{ \textsc{§ 37.} }
\noindent Tandis que ces événements se déroulent devant Gergovie, Convictolitavis, cet Héduen à qui, comme on l’a vu, César avait donné la magistrature suprême, cédant aux séductions de l’or arverne, entre en rapports avec certains jeunes gens, à la tête desquels étaient Litaviccos et ses frères, issus d’une très grande famille. Il partage avec eux le prix de sa trahison, et les exhorte à se souvenir qu’ils sont des hommes libres et nés pour commander. « Il n’y a qu’un seul obstacle à la victoire des Gaulois, qui est certaine : c’est l’attitude des Héduens ; l’autorité de leur exemple retient les autres cités qu’ils abandonnent les Romains, et ceux-ci ne pourront plus tenir en Gaule. Sans doute, il n’est pas sans avoir à César quelque obligation, quoique celui-ci n’ait fait, après tout, que reconnaître la justice de sa cause ; mais le désir de l’indépendance nationale est le plus fort. Car enfin, pourquoi les Héduens recourraient-ils à l’arbitrage de César quand il s’agit de leur constitution et de leurs lois, plutôt que Rome à celui des Héduens ? » Le discours du magistrat et l’argent ont vite fait d’entraîner ces jeunes hommes : ils se déclarent même prêts à prendre la tête du mouvement, et nos conjurés cherchent un plan d’action, car ils ne se flattaient pas d’amener les Héduens à la guerre si facilement. On décida que Litaviccos recevrait le commandement des dix mille hommes qu’on devait envoyer à César, et il se chargerait de les conduire, tandis que ses frères le devanceraient auprès de César. Les autres parties du plan sont également réglées.
\subsection[{§ 38.}]{ \textsc{§ 38.} }
\noindent On remit l’armée à Litaviccos. Quand il fut à environ trente milles de Gergovie, il réunit soudain ses troupes et, tout en larmes, leur dit : « Où allons-nous, soldats ? Toute notre cavalerie, toute notre noblesse ont péri ; des citoyens du plus haut rang, Eporédorix et Viridomaros, accusés de trahison par les Romains, ont été mis à mort sans qu’on leur eût permis de se défendre. Apprenez le détail du drame de la bouche de ceux qui ont échappé au massacre, car pour moi, qui ai perdu mes frères et tous mes proches, la douleur m’empêche d’en faire le récit. » On fait avancer des hommes à qui il avait fait la leçon, et ils racontent à la multitude ce que Litaviccos venait d’annoncer « Les cavaliers héduens ont été massacrés sous prétexte qu’ils étaient entrés en pourparlers avec les Arvernes ; quant à eux, ils ont pu se cacher au milieu de la foule des soldats et échapper ainsi au carnage. » Une clameur s’élève, on supplie Litaviccos d’indiquer le parti à prendre. Mais lui : « S'agit-il de délibérer ? ne sommes-nous pas dans l’obligation d’aller à Gergovie et de nous joindre aux Arvernes ? A moins que nous ne doutions que les Romains, après un tel crime, n’accourent pas déjà pour nous égorger ? Ainsi donc, si nous avons du cœur, vengeons la mort des victimes qu’ils ont indignement massacrées, et exterminons ces bandits. » Ce disant, il désigne des citoyens romains qui s’étaient joints à lui, confiants dans sa protection ; il livre au pillage le blé et les approvisionnements dont il convoyait une grande quantité, et fait périr ces malheureux dans de cruelles tortures. Il envoie des messagers dans tout le pays des Héduens, y provoque une profonde émotion par la même nouvelle mensongère d’un massacre des cavaliers et des notables ; il exhorte ses concitoyens à venger leurs injures de la même manière qu’il a fait lui-même.
\subsection[{§ 39.}]{ \textsc{§ 39.} }
\noindent L'Héduen Eporédorix, jeune homme de très grande famille et très puissant dans son pays, et avec lui Viridomaros, de même âge et de même crédit, mais de moindre naissance, que César, sur la recommandation de Diviciacos, avait élevé d’une condition obscure aux plus grands honneurs, s’étaient joints à la cavalerie héduenne sur convocation spéciale de sa part. Ils se disputaient le premier rang, et dans ce conflit des deux magistrats suprêmes qu’on a raconté plus, haut, ils avaient lutté de toutes leurs forces : l’un pour Convictolitavis, l’autre pour Lotos. Eporédorix, instruit des projets de Litaviccos, vient, vers le milieu de la nuit, mettre César au courant ; il le supplie de ne pas souffrir que les desseins pervers de quelques jeunes gens fassent abandonner à son pays l’amitié de Rome ; ce qui se produira, si tant de milliers d’hommes se joignent à l’ennemi, car leurs proches ne pourront se désintéresser de leur sort, ni la nation ne point y attacher d’importance.
\subsection[{§ 40.}]{ \textsc{§ 40.} }
\noindent Cette nouvelle affecta vivement César, car il avait toujours eu pour les Héduens des bontés particulières ; sans hésiter, il fait sortir du camp quatre légions sans bagages et toute la cavalerie ; et on n’eut pas le temps, dans des conjonctures si pressantes, de resserrer le camp, car le succès dépendait de Ia rapidité ; il laisse son légat Laïus Fabius avec deux légions pour la garde du camp. Ayant ordonné qu’on se saisît des frères de Litaviccos, il apprend qu’ils viennent de s’enfuir chez l’ennemi. Il exhorte ses soldats à ne pas se rebuter d’une marche pénible que la nécessité impose ; tous le suivent avec ardeur, et après avoir parcouru vingt-cinq milles, il aperçoit les Héduens ; il lance sa cavalerie, les arrête, les empêche d’avancer, mais fait défense générale de tuer personne. Il ordonne à Eporédorix et à Viridomaros, que les Héduens croyaient morts, de se mêler aux cavaliers et d’appeler leurs compatriotes. On les reconnaît, on découvre l’imposture de Litaviccos ; alors les Héduens tendent les mains, font signe qu’ils se rendent et, jetant leurs armes, demandent grâce. Litaviccos se réfugie à Gergovie, accompagné de ses clients, car, selon la coutume des Gaulois, il est impie, même si la situation est sans issue, d’abandonner son patron.
\subsection[{§ 41.}]{ \textsc{§ 41.} }
\noindent César envoya des messagers chez les Héduens pour leur faire savoir que sa bonté avait laissé la vie à des hommes que le droit de la guerre lui eût permis de faire périr ; puis, ayant fait reposer son armée pendant trois heures de nuit, il se mit en route pour Gergovie. Il était à peu près à mi-chemin quand des cavaliers dépêchés par Fabius lui font connaître quel danger le camp a couru. « Des forces considérables ont donné l’assaut ; une relève fréquente remplaçait les troupes fatiguées par des troupes fraîches, tandis que les nôtres étaient obligés à un effort ininterrompu et épuisant car, en raison de l’étendue du camp, les mêmes devaient demeurer sans cesse au retranchement. Une grêle de flèches et de traits de toutes sortes en avait blessé un grand nombre ; pour résister à cette attaque, notre artillerie avait été d’un grand secours. Fabius profitait de leur départ pour boucher les portes du camp, sauf deux, garnir la palissade de mantelets, et se préparer à pareil assaut pour le lendemain. » A cette nouvelle, César hâta sa marche, et grâce à l’ardeur extrême des soldats, parvint au camp avant le lever du soleil.
\subsection[{§ 42.}]{ \textsc{§ 42.} }
\noindent Tandis que ces événements se déroulent devant Gergovie, les Héduens, aux premières nouvelles qu’ils reçoivent de Litaviccos, ne se donnent pas le temps de s’informer. La cupidité excite les uns, les autres obéissent à leur emportement naturel et à la légèreté qui est le trait dominant de la race, et qui leur fait prendre un bruit sans consistance pour un fait certain. Ils pillent les biens des citoyens romains, ils tuent, ils emmènent en esclavage. Convictolitavis encourage le mouvement qui se déclenche : il excite le peuple, il le rend furieux, pour qu’une fois souillé d’un crime la honte l’empêche de revenir à la raison. Marcus Aristius, tribun militaire, était en route pour rejoindre sa légion ; on le force à quitter Cavillonum en lui promettant sur l’honneur qu’il ne sera pas inquiété ; on expulse aussi les Romains qui s’étaient établis dans la ville pour y faire du commerce. A peine ceux-ci s’étaient-ils mis en route, qu’on les attaque et qu’on leur enlève tout leurs bagages ; comme ils résistent, ils subissent un assaut d’un jour et d’une nuit ; les pertes étant sérieuses des deux côtés, les assaillants appellent aux armes des bandes plus nombreuses.
\subsection[{§ 43.}]{ \textsc{§ 43.} }
\noindent Sur ces entrefaites arrive la nouvelle que tous les soldats héduens sont au pouvoir de César : alors on se précipite vers Aristius, on explique que le gouvernement n’est pour rien dans ce qui s’est passé ; on ordonne une enquête sur les pillages, on confisque les biens de Litaviccos et de ses frères, on députe à César pour se disculper. Cette conduite leur est dictée par le désir de recouvrer leurs troupes ; mais ils avaient sur eux la souillure d’un crime, ils étaient retenus par ce que leur avait rapporté le pillage – car beaucoup y avaient participé, – enfin ils avaient peur du châtiment : aussi se mettent-ils à se concerter en secret au sujet de la guerre, et ils envoient des ambassades aux autres cités pour essayer de les gagner. César se rendait compte de ces manœuvres ; néanmoins, il parle aux députés avec toute la douceur possible, leur déclarant que, tenant compte de l’aveuglement et de la légèreté de la populace, il ne prend aucune mesure sévère contre la nation des Héduens et ne retire rien de sa bienveillance à leur égards. Cependant, comme il s’attendait à un grand soulèvement de la Gaule, voulant éviter d’être enveloppé par tous les peuples gaulois, il songea aux moyens de quitter Gergovie et de rassembler à nouveau toute son armée, afin qu’un départ qui n’était dû qu’à la crainte de la défection ne pût avoir l’air d’une fuites.
\subsection[{§ 44.}]{ \textsc{§ 44.} }
\noindent Au milieu de ces pensées, il lui sembla qu’une occasion s’offrait de vaincre. Étant venu au petit camp pour inspecter les ouvrages, il remarqua qu’une colline qui était dans les lignes de l’ennemi était dégarnie de troupes, alors que les jours précédents elles y étaient si denses que le sol s’en voyait à peine. Étonné, il s’enquiert auprès des déserteurs, dont il venait un grand nombre chaque jour. Tous font la même déclaration : comme César l’avait déjà appris par ses éclaireurs, le revers de cette colline était presque plat, mais boisé et étroit dans la partie par où l’on accédait à l’autre côté de la ville ; l’ennemi craignait beaucoup pour cet endroit, et il sentait bien que, les Romains occupant déjà une colline, s’il perdait l’autre, il serait presque enveloppé et ne pourrait ni sortir, ni fourrager. Vercingétorix avait appelé toutes ses troupes pour la fortifier.
\subsection[{§ 45.}]{ \textsc{§ 45.} }
\noindent Ainsi renseigné, César envoie vers la position, au milieu de la nuit, de nombreux escadrons ; il leur ordonne de se répandre de tous côtés en faisant du bruit. A l’aube, il fait sortir du camp un grand nombre de mulets chargés de bagages, les fait débâter et ordonne que les muletiers, coiffés de casques, prenant l’air et l’allure de cavaliers, fassent le tour par les collines. Il leur adjoint quelques cavaliers qui doivent, pour donner le change, rayonner largement. Par un long détour, ils se concentreront tous au même point. Les gens de la ville apercevaient au loin ces mouvements, car de Gergovie la vue plongeait sur le camp, sans toutefois qu’il fût possible, à une telle distance, de se rendre un compte exact des choses. César envoie par la même ligne de hauteurs une légion, et après qu’elle s’est un peu avancée, il l’établit dans un fond où des bois la cachent aux regards. L'inquiétude des Gaulois augmente et toutes leurs troupes sont acheminées sur ce point pour travailler aux retranchements. Quand il voit que le camp ennemi est vide, César fait passer ses soldats du grand camp dans le petit par petits groupes et en ayant soin que les ornements des casques soient recouverts et les enseignes cachées, afin de ne pas attirer l’attention des défenseurs de la ville ; il révèle ses intentions aux légats qu’il avait mis à la tête de chaque légion ; il leur recommande avant tout de contenir leurs troupes, de veiller à ce que l’ardeur au combat ou l’espoir du pillage ne les emporte pas trop loin ; il leur explique les difficultés qui viennent de l’inégalité des positions : seule une action prompte peut y remédier ; il s’agit d’une surprise, non d’une bataille en règle. Après quoi, il donne le signal de l’assaut et lance en même temps, sur la droite, par une autre montée, les Héduens.
\subsection[{§ 46.}]{ \textsc{§ 46.} }
\noindent La distance entre le mur de la ville et la plaine, depuis l’endroit où commençait la montée, était, en ligne droite sans aucun détour, de douze cents pas ; mais tous les lacets qu’on avait faits pour faciliter l’ascension augmentaient la longueur du chemin. Environ à mi-hauteur, les Gaulois avaient construit un mur de grandes pierres, haut de six pieds, qui suivait le flanc de la colline aussi régulièrement que le permettait la nature du terrain, et était destiné à ralentir notre assaut ; toute la zone inférieure avait été laissée vide, tandis que la partie de la colline comprise entre ce mur et le rempart de la ville était remplie de campements très serrés. Nos soldats, au signal donné, arrivent promptement à ce premier mur ; ils le franchissent, et s’emparent de trois camps ; et ils le firent si promptement que Teutomatos, roi des Nitiobroges, surpris dans sa tente, où il faisait la sieste, n’échappa qu’à grand-peine des mains des soldats qui y entraient pour faire du butin il s’enfuit à demi nu, et son cheval fut blessé.
\subsection[{§ 47.}]{ \textsc{§ 47.} }
\noindent Comme il avait atteint le but qu’il s’était proposé, César ordonna de sonner la retraite, et ayant harangué la dixième légion, avec laquelle il était, il lui fit faire halte. Les autres légions n’entendirent pas la trompette, parce qu’elles étaient au-delà d’un ravin assez large ; pourtant les tribuns et les légats, suivant les instructions de César, s’efforçaient de les retenir. Mais les soldats, exaltés par l’espoir d’une prompte victoire, par le spectacle de l’ennemi en fuite, par le souvenir de leurs précédents succès, pensaient qu’il n’y avait pas d’entreprise si ardue que leur valeur ne pût mener à bien, et ils ne cessèrent la poursuite qu’une fois arrivés près des murs et des portes de la cité. A ce moment, une clameur s’éleva de tous les points de la ville ; ceux qui étaient loin, effrayés de ce soudain tumulte, crurent que l’ennemi avait franchi les portes et sortirent de la place précipitamment. Les mères de famille jetaient du haut des murs des étoffes et de l’argent et, le sein découvert, penchées sur la muraille et tendant leurs mains ouvertes, elles suppliaient les Romains de les épargner, de ne pas massacrer, comme ils avaient fait à Avaricum, les femmes même et les enfants ; plusieurs, se suspendant aux mains de leurs compagnes et se laissant glisser, venaient se rendre aux soldats. Lucius Fabius, centurion de la huitième légion, avait – c’était connu – déclaré ce jour-là au milieu de ses hommes que les récompenses de la journée d’Avaricum le remplissaient d’ardeur et qu’il ne souffrirait pas que personne escaladât le mur avant lui ; il prit avec lui trois de ses soldats et, hissé par eux, il monta sur le rempart ; puis, à son tour, les tirant à lui, il les fit monter l’un après l’autre.
\subsection[{§ 48.}]{ \textsc{§ 48.} }
\noindent Cependant, ceux des Gaulois qui s’étaient assemblés de l’autre côté de la ville, ainsi que nous l’avons expliqué plus haut, pour y faire des travaux de défense, entendant d’abord des cris, puis recevant à plusieurs reprises la nouvelle que les Romains étaient maîtres de la ville, se portèrent au pas de course vers le lieu de l’action, précédés de la cavalerie. A mesure qu’ils arrivaient, ils prenaient position au pied de la muraille et grossissaient les rangs de nos adversaires. Lorsqu’ils furent en grand nombre, on vit les mères de famille, qui, quelques instants auparavant, nous tendaient les mains du haut des murs, adresser leurs prières aux Gaulois et, selon la coutume de ce peuple, leur montrer leurs cheveux épars et tendre vers eux leurs enfants. Les Romains ne luttaient pas à armes égales : la position, le nombre étaient contre eux ; sans compter que, fatigués par la course et par la durée du combat, il ne leur était pas facile de soutenir le choc de troupes toutes fraîches.
\subsection[{§ 49.}]{ \textsc{§ 49.} }
\noindent César, voyant que l’ennemi avait l’avantage de la position et, de plus en plus, celui du nombre, conçut des craintes pour la suite du combat : il envoya à son légat Titus Sextius, à qui il avait confié la garde du petit camp, l’ordre d’en faire sortir promptement ses cohortes et de les disposer au pied de la colline, sur la droite de l’ennemi, afin que, s’il voyait les nôtres lâcher pied, il pût intimider l’ennemi et gêner sa poursuite. De son côté, César, s’étant porté avec sa légion un peu en avant du point où il avait fait halte, attendait l’issue du combat.
\subsection[{§ 50.}]{ \textsc{§ 50.} }
\noindent Le corps à corps était acharné, l’ennemi se fiant aux avantages que lui donnaient le terrain et le nombre, et nos soldats à leur valeur, quand soudain on vit paraître sur notre flanc droit les Héduens, que César avait envoyés par une autre montée, à droite, pour faire diversion. Trompés par la ressemblance de leurs armes avec celles des ennemis, les Romains furent vivement émus, et bien qu’ils eussent l’épaule droite découverte, ce qui était le signe conventionnel en usage, nos soldats crurent que c’était là un stratagème employé par l’ennemi pour les abuser. Au même moment, le centurion Lucius Fabius et ceux qui avaient escaladé la muraille avec lui étaient enveloppés, massacrés et jetés à bas du rempart. Marcus Pétronius, centurion de la même légion, après avoir essayé de briser les portes, écrasé par le nombre et voyant sa mort certaine – il était couvert de blessures, – s’adressa en ces termes à ses hommes qui l’avaient suivi « Puisque je ne peux me sauver avec vous, je veux du moins préserver votre vie, que ma passion de la gloire a mise en péril. Songez à votre salut, je vais vous en donner le moyen. » Ce disant, il se précipita au milieu des ennemis, en tua deux et réussit à dégager un peu la porte. Ses hommes essayaient de l’aider ; mais lui : « En vain, dit-il, vous tentez de me sauver ; j’ai perdu trop de sang et mes forces me trahissent. Partez donc, pendant que vous le pouvez encore, et repliez-vous sur la légion. » C'est ainsi que peu après il tomba, les armes à la main, en assurant le salut des siens.
\subsection[{§ 51.}]{ \textsc{§ 51.} }
\noindent Les nôtres, pressés de toutes parts, ayant perdu quarante-six centurions, furent bousculés. La poursuite furieuse des Gaulois fut ralentie par la dixième légion qui s’était établie en soutien sur un point où la pente était un peu moins forte. Cette légion fut à son tour appuyée par les cohortes de la treizième, que le légat Titus Sextius avait fait sortir du petit camp et qui avaient pris position au-dessus de la plaine. Dès que l’ensemble de nos légions atteignit cette plaine, elles s’arrêtèrent et se reformèrent face à l’ennemi. Vercingétorix ramena ses troupes du pied de la colline à l’intérieur du retranchement. Nous perdîmes ce jour-là un peu moins de sept cents hommes.
\subsection[{§ 52.}]{ \textsc{§ 52.} }
\noindent Le lendemain, César, ayant assemblé ses troupes, leur reprocha leur manque de réflexion et de sang-froid : « Ils avaient décidé d’eux-mêmes jusqu’où ils devaient aller et ce qu’ils devaient faire, ils ne s’étaient pas arrêtés quand on avait sonné la retraite, et les tribuns, les légats même n’avaient pu les retenir. Il leur expliqua de quelle importance était le désavantage de la position, et quelle avait été sa pensée à Avaricum, lorsque, ayant surpris l’ennemi sans chef et sans cavalerie, sûr de la victoire, il y avait pourtant renoncé, parce qu’il ne voulait pas éprouver dans cette rencontre les pertes, fussent-elles légères, que lui aurait values le désavantage de sa position. Autant il admirait l’héroïsme d’hommes que n’avaient arrêtés ni les fortifications du camp ennemi, ni la hauteur de la montagne, ni le mur de la ville, autant il réprouvait leur l’indiscipline et leur présomption, qui leur avaient fait croire qu’ils étaient plus capables que leur général d’avoir une opinion sur les conditions de la victoire et sur l’issue d’une action. Et il ne demandait au soldat pas moins de discipline et de domination de soi-même que de courage et de force d’âme. »
\subsection[{§ 53.}]{ \textsc{§ 53.} }
\noindent Ses derniers mots furent des mots de réconfort : « Il n’y avait pas lieu de se décourager, et ils ne devaient pas attribuer aux qualités guerrières de l’ennemi un échec que leur avait valu le désavantage de leur position. » Après cette harangue, étant toujours du même avis sur l’opportunité du départ, il fit sortir ses légions du camp et les rangea en bataille sur un terrain favorable. Comme Vercingétorix n’en restait pas moins derrière ses retranchements et ne descendait pas dans la plaine, après un petit engagement de cavalerie, et où il eut l’avantage, il ramena ses troupes dans le camp. Il recommença le lendemain, et jugeant dès lors qu’il en avait assez fait pour rabattre la jactance gauloise et pour relever le courage des siens, il se mit en route pour le pays des Héduens. L'ennemi n’osa pas davantage nous poursuivre ; le troisième jour, César atteint l’Allier, y reconstruit les ponts et fait passer ses troupes sur l’autre rive.
\subsection[{§ 54.}]{ \textsc{§ 54.} }
\noindent Là, les Héduens Viridomaros et Eporédorix ayant demandé à lui parler, il apprend d’eux que Litaviccos est parti avec toute la cavalerie pour tâcher de soulever les Héduens ; il faut, disent-ils, qu’ils aillent en avant pour maintenir la cité dans le devoir. Bien qu’il eût déjà maintes preuves de la perfidie des Héduens, et qu’il lui parût que leur départ ne ferait que hâter la défection de ce peuple, il ne crut point pourtant devoir les retenir, ne voulant pas les offenser ni laisser supposer qu’il fût inquiet. Au moment de leur départ, il leur exposa, en quelques mots, ses titres à la reconnaissance des Héduens : ce qu’ils étaient, et dans quel abaissement, quand il les accueillit : refoulés dans les places fortes, dépouillés de leurs terres, privés de toutes leurs troupes, soumis à un tribut, obligés, par les contraintes les plus humiliantes, à livrer des otages ; ce qu’il avait fait d’eux, et comment il les avait portés si haut que non seulement on les voyait rendus à leur premier état, mais plus honorés et plus puissants qu’ils n’avaient jamais été. Sur ces paroles, qu’ils avaient charge de répéter, il les congédia.
\subsection[{§ 55.}]{ \textsc{§ 55.} }
\noindent Noviodunum était une ville des Héduens située sur les bords de la Loire, dans une position avantageuse. César y avait rassemblé tous les otages de la Gaule, du blé, de l’argent des caisses publiques, une grande partie de ses bagages et de ceux de l’armée, il y avait envoyé un grand nombre de chevaux achetés en Italie et en Espagne en vue de la présente guerre. Eporédorix et Viridomaros, en arrivant dans cette ville, apprirent quelle était la situation chez les Héduens : ceux-ci avaient accueilli Litaviccos à Bibracte, ville qui jouit chez eux d’une très grosse influence ; Convictolitavis, magistrat suprême de la nation, et une grande partie du sénat étaient venus l’y trouver ; on avait envoyé officiellement des ambassadeurs à Vercingétorix pour conclure avec lui un traité de paix et d’alliance aussi pensèrent-ils qu’ils ne devaient pas laisser échapper une occasion aussi avantageuse. Ayant donc massacré le détachement de garde à Noviodunum et les marchands qui s’y trouvaient, ils se partagèrent l’argent et les chevaux ; ils firent conduire les otages des divers peuples à Bibracte, auprès du magistrat suprême ; quant à la ville, jugeant impossible de la tenir, ils l’incendièrent, pour qu’elle ne pût servir aux Romains ; ils emportèrent dans des bateaux tout le blé qu’ils purent charger sur l’heure, et le reste, ils le jetèrent dans le fleuve ou le brûlèrent. Ils s’employèrent personnellement à lever des troupes dans les régions voisines, à disposer des détachements et des petits postes sur les bords de la Loire, à faire partout des raids terroristes de cavalerie, espérant ainsi couper les Romains de leur ravitaillement ou les déterminer, par la disette, à s’en aller dans la Province. Ce qui les encourageait beaucoup dans cet espoir, c’est que la fonte des neiges avait provoqué une crue du fleuve, en sorte que le franchir à gué apparaissait comme une chose absolument impossible.
\subsection[{§ 56.}]{ \textsc{§ 56.} }
\noindent Quand il apprit cela, César pensa qu’il devait faire diligence : s’il lui fallait, en construisant des ponts, courir le danger d’une attaque, il importait qu’il pût livrer bataille avant qu’on n’eût réuni sur ce point de trop grandes forces. Quant à changer ses plans et à se diriger vers la Province, mesure que personne à ce moment-là ne jugeait indispensable, maintes raisons s’y opposaient les Gaulois nous mépriseraient, la chose était déshonorante, les Cévennes barraient la route, les chemins étaient malaisés, mais surtout, il craignait fort pour Labiénus, qui était séparé de lui, et pour les légions qu’il avait détachées sous ses ordres. Aussi, surprenant tout le monde, il atteignit la Loire à très fortes étapes de jour et de nuit, puis, ses cavaliers ayant découvert un gué convenable, du moins dans la circonstance, car c’était tout juste si les bras et les épaules pouvaient rester hors de l’eau pour soutenir les armes, il disposa sa cavalerie de façon à briser le courant, et comme l’ennemi s’était d’abord troublé à notre vue, il passa sans pertes. Il trouva dans la campagne du blé et beaucoup de bétail, se réapprovisionna, et se mit en route pour le pays des Sénons.
\subsection[{§ 57.}]{ \textsc{§ 57.} }
\noindent Tandis que ces événements se déroulent du côté de César, Labiénus, laissant à Agédincum, pour garder les bagages, les troupes de renfort qu’il venait de recevoir d’Italie part vers Lutèce avec quatre légions. C'est la ville des Parisii, située dans une île de la Seine. Quand l’ennemi sut qu’il approchait, d’importants contingents venus des cités voisines se rassemblèrent. On donne le commandement en chef à l’Aulerque Camulogène il était presque épuisé par l’âge, mais sa particulière connaissance de l’art militaire lui valut cet honneur. Ayant observé l’existence d’un marais continu qui déversait ses eaux dans la Seine et rendait l’accès de toute la région fort difficile, il s’y établit et entreprit de nous interdire le passage.
\subsection[{§ 58.}]{ \textsc{§ 58.} }
\noindent Labiénus commença par essayer de faire avancer des mantelets, de combler le marais avec des fascines et des matériaux de remblayage, enfin de construire une chaussée. Voyant que l’entreprise offrait trop de difficultés, il sortit sans bruit de son camp à la troisième veille et, reprenant le chemin qu’il avait suivi pour venir, arriva à Metlosédum. C'est une ville des Sénons située dans une île de la Seine comme nous venons de dire qu’était Lutèce. Labiénus s’empare d’environ cinquante embarcations, les unit rapidement les unes aux autres et y jette des soldats. Grâce à la surprise et à la terreur des gens de la ville, dont un grand nombre d’habitants étaient partis pour la guerre, il se rend sans combat maître de la place. Il rétablit le pont que l’ennemi avait coupé les jours précédents, y fait passer son armée et fait route vers Lutèce en suivant le cours du fleuve. Les ennemis, informés par ceux qui s’étaient enfuis de Metlosédum, font incendier Lutèce et couper les ponts de cette ville ; de leur côté, ils quittent le marais et s’établissent sur la rive de la Seine, devant Lutèce et face au camp de Labiénus.
\subsection[{§ 59.}]{ \textsc{§ 59.} }
\noindent Déjà on entendait dire que César avait quitté Gergovie, déjà des bruits couraient concernant la défection des Héduens et le succès du soulèvement général, et les Gaulois, dans leurs entretiens, affirmaient que César avait été coupé, n’avait pu franchir la Loire, et, contraint par la disette, avait pris le chemin de la Province. Quand la trahison des Héduens fut connue des Bellovaques qui, déjà auparavant, s’étaient d’eux-mêmes montrés peu sûrs, ils se mirent à mobiliser et à préparer ouvertement les hostilités. Alors Labiénus, comprenant, en présence d’un tel renversement de la situation, qu’il devait complètement changer ses plans, songea non plus à faire des conquêtes et à livrer bataille à l’ennemi, mais à ramener son armée saine et sauve à Agédincum. Et en effet, d’un côté, c’était la menace des Bellovaques, peuple qui est réputé parmi les peuples gaulois pour le plus valeureux ; de l’autre, Camulogène avec une armée prête au combat et bien équipée ; de plus, les légions étaient séparées de leurs réserves et de leurs bagages par un grand fleuve. Devant de telles difficultés soudainement surgies, il voyait bien qu’il fallait chercher le salut dans une résolution courageuse.
\subsection[{§ 60.}]{ \textsc{§ 60.} }
\noindent Donc, ayant réuni à la tombée du jour un conseil de guerre et ayant exhorté ses officiers à exécuter soigneusement et rigoureusement ses ordres, il confie chacune des embarcations qu’il avait amenées de Metlosédum à un chevalier romain et ordonne qu’après la première veille on descende en silence le fleuve jusqu’à quatre milles de distance, et que là on attende son arrivée. Il laisse pour la garde du camp cinq cohortes, celles qu’il jugeait les moins solides ; il ordonne aux cinq autres cohortes de la même légion de partir au milieu de la nuit avec tous les bagages en remontant le fleuve, et de faire grand bruit. Il réquisitionne aussi des barques, et les dirige du même côté à grand fracas de rames. Lui-même, peu après, sort en silence avec trois légions et gagne l’endroit où la flotte avait ordre d’aborder.
\subsection[{§ 61.}]{ \textsc{§ 61.} }
\noindent Là, les éclaireurs ennemis – on en avait disposé tout le long du fleuve – sont surpris par notre arrivée, car un orage avait éclaté soudain, et ils périssent sous nos coups ; l’infanterie et la cavalerie, sous la direction des chevaliers romains à qui Labiénus avait confié cette tâche, sont transportées rapidement sur l’autre rive. A l’aube, l’ennemi apprend presque simultanément qu’une agitation inaccoutumée règne dans le camp romain, qu’une importante colonne remonte le fleuve, que du même côté on entend le bruit des rames, et qu’un peu en aval il y a des navires qui transportent des soldats d’une rive à l’autre. A cette nouvelle, pensant que les légions franchissaient le fleuve en trois endroits et qu’effrayés par la défection des Héduens les Romains préparaient une fuite générale, ils divisèrent, eux aussi, leurs troupes en trois corps. Laissant un poste en face du camp et envoyant un petit détachement dans la direction de Metlosédum, avec mission de n’avancer qu’autant que l’auraient fait les embarcations, ils menèrent le reste de leurs forces à la rencontre de Labiénus.
\subsection[{§ 62.}]{ \textsc{§ 62.} }
\noindent Au lever du jour, tous les nôtres avaient franchi le fleuve, et on voyait en face la ligne ennemie. Labiénus, adressant la parole à ses soldats, les exhorte à se souvenir de leur valeur, si souvent éprouvée et de tant de glorieuses victoires, enfin à se conduire comme si César en personne, lui qui maintes fois les avait menés à la victoire, assistait à la bataille ; puis il donne le signal du combat. Au premier choc, à l’aile droite, où avait pris position la septième légion, l’ennemi est enfoncé et mis en déroute ; à gauche, où était la douzième, les premiers rangs ennemis avaient été abattus par les javelots ; mais le reste opposait une résistance farouche, et pas un n’eût pu être soupçonné de songer à fuir. Le chef ennemi, Camulogène, était là auprès des siens, et les encourageait. Mais, tandis que la victoire était encore incertaine, les tribuns de la septième légion, ayant appris ce qui se passait à l’aile gauche, firent paraître leur légion sur les derrières de l’ennemi et la portèrent à l’attaque. Même alors, personne ne lâcha pied, mais ils furent tous enveloppés et massacrés. Camulogène partagea le sort commun. Quant à ceux qui avaient été laissés en face du camp de Labiénus, ayant appris que l’on se battait, ils allèrent au secours des leurs et s’emparèrent d’une colline ; mais ils ne purent soutenir le choc de nos soldats victorieux. Ils se mêlèrent donc aux autres Gaulois qui fuyaient, et ceux que les bois et les collines ne dérobèrent pas à notre poursuite furent tués par nos cavaliers. Cette action terminée, Labiénus retourne à Agédincum, où avaient été laissés les bagages de toute l’armée ; puis, avec toutes ses troupes, il rejoint César.
\subsection[{§ 63.}]{ \textsc{§ 63.} }
\noindent Quand on connaît la trahison des Héduens, la guerre prend une extension nouvelle. Ils envoient partout des ambassades ; par tout ce qu’ils ont d’influence, d’autorité, d’argent, ils s’efforcent de gagner les cités ; comme ils détiennent les otages que César avait laissés chez eux, leur supplice sert à terrifier ceux qui hésitent. Ils demandent à Vercingétorix de venir les trouver et de se concerter avec eux sur la conduite de la guerre. Celui-ci ayant consenti, ils prétendent se faire remettre le commandement suprême, et comme l’affaire dégénère en conflit, une assemblée générale de la Gaule est convoquée à Bibracte. On s’y rend en foule de toutes parts. La décision est laissée au suffrage populaire ; celui-ci, à l’unanimité, confirme Vercingétorix dans le commandement suprême. Les Rèmes, les Lingons, les Trévires ne prirent point part à cette assemblée ; les premiers parce qu’ils restaient les amis de Rome, les Trévires parce qu’ils étaient trop loin et étaient menacés par les Germains, ce qui fut cause qu’ils se tinrent constamment en dehors de la guerre et n’envoyèrent de secours à aucun des deux partis. Les Héduens éprouvent un vif ressentiment à se voir déchus du premier rang, ils déplorent le changement de leur fortune et regrettent les bontés de César, sans oser toutefois, les hostilités étant commencées, se tenir à part du plan commun. Eporédorix et Viridomaros, qui nourrissaient les plus hautes ambitions, ne se subordonnent qu’à contre cœur à l’autorité de Vercingétorix.
\subsection[{§ 64.}]{ \textsc{§ 64.} }
\noindent Celui-ci commande aux autres cités de lui fournir des otages, et fixe un jour pour leur remise. Il donne l’ordre que tous les cavaliers, au nombre de quinze mille, se concentrent rapidement : « Pour l’infanterie, il se contentera de ce qu’il avait jusque-là, il ne veut pas tenter la fortune ni livrer de bataille rangée ; mais, puisqu’il dispose d’une cavalerie très nombreuse, rien n’est plus facile que d’empêcher les Romains de se procurer du blé et de faire du fourrage ; seulement, ils ne devront pas hésiter à rendre de leurs propres mains leurs blés inutilisables et à incendier leurs granges, tactique de destruction de leurs biens qui, ils le savent, leur assure pour toujours la souveraineté et la liberté. » Ces mesures prises, il ordonne aux Héduens et aux Ségusiaves, qui sont à la frontière de la Province, de mettre sur pied dix mille fantassins ; il y joint huit cents cavaliers. Il confie cette troupe au frère d’Eporédorix et lui commande d’attaquer les Allobroges. De l’autre côté, il lance les Gabales et les tribus arvernes de la frontière contre les Helviens, et envoie les Rutènes et les Cadurques ravager le pays des Volques Arécomiques. Cela ne l’empêche point de solliciter en secret les Allobroges par des courriers privés et des ambassades, car il espérait que les souvenirs de la dernière guerre n’étaient pas encore éteints dans leur esprit. Aux chefs il promet des sommes d’argent, et à la nation que toute la Province lui appartiendra.
\subsection[{§ 65.}]{ \textsc{§ 65.} }
\noindent Pour faire face à tous ces dangers, on avait préparé une force défensive de vingt-deux cohortes, levée dans la Province même par le légat Lucius César et qui, de tous les côtés, s’opposait aux envahisseurs. Les Helviens livrent spontanément bataille à leurs voisins et sont battus ; ayant perdu le chef de la cité, Caïus Valérius Domnotaurus, fils de Caburus, et un très grand nombre d’autres, ils sont contraints de se réfugier dans leurs villes, à l’abri de leurs remparts. Les Allobroges organisent avec soin et diligence la défense de leurs frontières, en disposant le long du Rhône une ligne serrée de postes. César, qui savait la supériorité de l’ennemi en cavalerie, et qui, toutes les routes étant coupées, ne pouvait recevoir aucun secours de la Province ni de l’Italie, envoie des messagers au-delà du Rhin en Germanie, chez les peuples qu’il avait soumis au cours des années précédentes, et se fait fournir par eux des cavaliers avec les soldats d’infanterie légère qui sont habitués à combattre dans leurs rangs. A leur arrivée, comme ils avaient des chevaux médiocres, il prend ceux des tribuns militaires, des autres chevaliers romains, des évocatsi, et les leur donne.
\subsection[{§ 66.}]{ \textsc{§ 66.} }
\noindent Sur ces entrefaites, les forces ennemies qui venaient de chez les Arvernes et les cavaliers que devait fournir toute la Gaule se réunissent. Vercingétorix forme de ceux-ci un corps nombreux et, comme César faisait route vers le pays des Séquanes en traversant l’extrémité du territoire des Lingons, afin de pouvoir plus aisément secourir la Province, il s’établit, dans trois camps, à environ dix mille pas des Romains ; il réunit les chefs de ses cavaliers et leur déclare que l’heure de la victoire est venue : « Les Romains sont en fuite vers la Province, ils quittent la Gaule ; cela suffit à assurer la liberté dans le temps présent ; mais c’est trop peu pour la sécurité du lendemain ; car ils reviendront avec des forces plus considérables, ils ne cesseront pas les hostilités. Il faut donc les attaquer tandis qu’ils sont en ordre de marche et embarrassés de leurs bagages. Si les fantassins essaient de secourir ceux qu’on attaque, et s’y attardent, ils ne peuvent plus avancer ; si, ce qu’il croit plus probable, ils abandonnent les bagages pour ne plus penser qu’à leur vie, ils perdront en même temps leurs moyens d’existence et l’honneur. Quant aux cavaliers ennemis, il ne faut pas douter qu’il ne s’en trouve pas un parmi eux pour oser seulement quitter la colonne. Afin qu’ils aient plus de cœur à cette attaque, il tiendra toutes ses forces en avant du camp et intimidera l’ennemi. » Les cavaliers l’acclament, crient qu’il leur faut se lier par le plus sacré des serments, pas d’asile sous un toit, pas d’accès auprès de ses enfants, de ses parents, de sa femme, pour celui qui n’aura pas deux fois traversé à cheval les rangs ennemis.
\subsection[{§ 67.}]{ \textsc{§ 67.} }
\noindent La proposition est approuvée : on fait prêter à tous le serment. Le lendemain, les cavaliers sont répartis en trois corps et deux apparaissent soudain sur nos flancs tandis que le troisième, en tête de la colonne, s’apprête à lui barrer la route. Quand César apprend la chose, il ordonne que sa cavalerie, également partagée en trois, coure à l’ennemi. On se bat partout à la fois. La colonne fait halte ; on rassemble les bagages au milieu des légions. S'il voyait nos cavaliers en difficulté ou en dangereuse posture sur quelque point, César faisait faire front et attaquer de ce côté-là ; cette intervention retardait la poursuite des ennemis et rendait courage aux nôtres, qui se sentaient soutenus. Enfin les Germains, sur la droite, avisant une hauteur qui dominait le pays, bousculent les ennemis qui s’y trouvaient ; ils les poursuivent jusqu’à la rivière, où Vercingétorix avait pris position avec son infanterie, et en font un grand carnage. Voyant cela, les autres craignent d’être enveloppés et se mettent à fuir. Partout on les massacre. Trois Héduens de la plus haute naissance sont faits prisonniers et conduits à César Cotos, chef de la cavalerie, qui avait été en conflit avec Convictolitavis lors des dernières élections ; Cavarillos, qui avait été placé à la tête de l’infanterie héduenne après la défection de Litaviccos, et Eporédorix, qui avant l’arrivée de César avait dirigé la guerre des Héduens contre les Séquanes.
\subsection[{§ 68.}]{ \textsc{§ 68.} }
\noindent Après cette déroute de toute sa cavalerie, Vercingétorix qui avait disposé ses troupes en avant de son camp, les mit en retraite incontinent, et prit la route d’Alésia, ville des Mandubiens, en ordonnant qu’on se hâtât de faire sortir du camp les bagages et de les acheminer à sa suite. César, ayant fait conduire ses bagages sur la colline la plus proche et ayant laissé deux légions pour les garder, poursuivit l’ennemi aussi longtemps que le jour le lui permit, et lui tua environ trois mille hommes à l’arrière-garde ; le lendemain, il campa devant Alésia. S'étant rendu compte de la force de la position, et voyant, d’autre part, que l’ennemi était terrifié, parce que sa cavalerie, qui était l’arme sur laquelle il comptait le plus, avait été battue, il exhorta ses soldats au travail et entreprit l’investissement de la place.
\subsection[{§ 69.}]{ \textsc{§ 69.} }
\noindent La ville proprement dite était au sommet d’une colline, à une grande altitude, en sorte qu’on voyait bien qu’il était impossible de la prendre autrement que par un siège en règle. Le pied de la colline était de deux côtés baigné par des cours d’eau. En avant de la ville une plaine s’étendait sur une longueur d’environ trois milles ; de tous les autres côtés la colline était entourée à peu de distance de hauteurs dont l’altitude égalait la sienne. Au pied du rempart, tout le flanc oriental de la colline était occupé par les troupes gauloises, et en avant elles avaient creusé un fossé et construit un mur grossier de six pieds. Les travaux qu’entreprenaient les Romains se développaient sur une longueur de dix milles. Les camps avaient été placés aux endroits convenables, et on avait construit, également en bonne place, vingt-trois postes fortifiés ; dans ces postes, on détachait pendant le jour des corps de garde, pour empêcher qu’une attaque soudaine se produisît sur quelque point ; pendant la nuit, il y avait dans ces mêmes postes des veilleurs, et de fortes garnisons les occupaient.
\subsection[{§ 70.}]{ \textsc{§ 70.} }
\noindent Les travaux étaient en cours d’exécution quand a lieu un combat de cavalerie dans la plaine qui, comme nous l’avons expliqué tout à l’heure, s’étendait entre les collines sur une longueur de trois mille pas. L'acharnement est extrême de part et d’autre. César envoie les Germains au secours des nôtres qui fléchissent, et il range ses légions en avant du camp, pour prévenir une attaque soudaine de l’infanterie ennemie. L'appui des légions donne du cœur à nos combattants ; les ennemis sont mis en déroute ; leur nombre les gêne, et comme on a laissé des portes trop étroites, ils s’y écrasent. Les Germains les poursuivent vivement jusqu’aux fortifications. Ils en tuent beaucoup ; un assez grand nombre abandonnent leurs chevaux pour tenter de franchir le fossé et d’escalader la murailles. César fait avancer un peu les légions qu’il avait établies en avant du retranchement. Un trouble égal à celui des fuyards s’empare des Gaulois qui étaient derrière la muraille : ils s’imaginent qu’on marche sur eux de ce pas, et ils crient aux armes ; un certain nombre, pris de panique, se précipitent dans la ville. Vercingétorix fait fermer les portes, pour éviter que le camp ne se vide. Après avoir tué beaucoup d’ennemis et pris un très grand nombre de chevaux, les Germains se replient.
\subsection[{§ 71.}]{ \textsc{§ 71.} }
\noindent Vercingétorix décide de faire partir nuitamment tous ses cavaliers avant que les Romains n’achèvent leurs travaux d’investissement. En se séparant d’eux, il leur donne mission d’aller chacun dans leur pays et d’y réunir pour la guerre tous les hommes en âge de porter les armes. Il leur expose ce qu’ils lui doivent, et les conjure de songer à son salut, de ne pas le livrer aux tortures de l’ennemi, lui qui a tant fait pour la liberté de la patrie. Il leur montre que s’ils ne sont pas assez actifs, quatre-vingt mille hommes d’élite périront avec lui. D'après ses calculs, il a tout juste trente jours de blé, mais il est possible, avec un strict rationnement, de subsister un peu plus longtemps encore. Après leur avoir confié ce message, il fait partir ses cavaliers en silence, pendant la deuxième veille, par le passage qui s’ouvrait encore dans nos lignes. Il réquisitionne tout le blé ; il décrète la peine de mort contre ceux qui n’obéiront pas ; il donne à chaque homme sa part du bétail, dont les Mandubiens avaient amené une grande quantité ; le blé, il le distribue parcimonieusement et peu à peu ; il fait rentrer dans la ville toutes les troupes qu’il avait établies sous ses murs. C'est par ces mesures qu’il s’apprête à attendre le moment où la Gaule le secourra, et qu’il règle la conduite de la guerre.
\subsection[{§ 72.}]{ \textsc{§ 72.} }
\noindent Mis au courant par des déserteurs et des prisonniers, César entreprit les travaux que voici. Il creusa un fossé de vingt pieds de large, à côtés verticaux, en sorte que la largeur du fond était égale à la distance entre les deux bords ; il mit entre ce fossé et toutes les autres fortifications une distance de quatre cents pieds ; il voulait ainsi éviter des surprises, car ayant été obligé d’embrasser un si vaste espace et pouvant difficilement garnir de soldats toute la ligne, il devait craindre soit que pendant la nuit l’ennemi ne se lançât en masse contre les retranchements, soit que de jour il ne lançât des traits contre nos troupes, qui avaient à travailler aux fortifications. Ayant donc laissé semblable intervalle entre cette ligne et la suivante, il creusa deux fossés larges de quinze pieds et chacun de profondeur égale ; il remplit le fossé intérieur, dans les parties qui étaient en plaine et basses, d’eau qu’il dériva de la rivière. Derrière ces fossés, il construisit un terrassement surmonté d’une palissade, dont la hauteur était de douze pieds ; il compléta celle-ci par un parapet et des créneaux, et disposa à la jonction de la terrasse et de la paroi de protection de grandes pièces de bois fourchues qui, pointées vers l’ennemi, devaient lui rendre l’escalade plus malaisée ; il éleva sur toute la périphérie de l’ouvrage des tours distantes les unes des autres de quatre-vingts pieds.
\subsection[{§ 73.}]{ \textsc{§ 73.} }
\noindent Il fallait en même temps aller chercher des matériaux, se procurer du blé, et faire des fortifications aussi considérables, alors que nos effectifs étaient réduits par l’absence des troupes qui poussaient leur recherche assez loin du camp ; en outre, à plus d’une reprise on vit les Gaulois s’attaquer à nos travaux et tenter des sorties très violentes par plusieurs portes à la fois. Aussi César pensa-t-il qu’il devait encore ajouter à ces ouvrages, afin de pouvoir défendre la fortification avec de moindres effectifs. On coupa donc des troncs d’arbres ayant des branches très fortes et l’extrémité de celles-ci fut dépouillée de son écorce et taillée en pointe ; d’autre part, on creusait des fossés continus profonds de cinq pieds. On y enfonçait ces pieux, on les reliait entre eux par le bas, pour empêcher qu’on les pût arracher, et on ne laissait dépasser que le branchage. Il y en avait cinq rangées, reliées ensemble et entrelacées, ceux qui s’engageaient dans cette zone s’empalaient à la pointe acérée des pieux. On les avait surnommés les cippes. Devant eux, on creusait, en rangées obliques et formant quinconce, des trous profonds de trois pieds, qui allaient en se rétrécissant peu à peu vers le bas. On y enfonçait des pieux lisses de la grosseur de la cuisse, dont l’extrémité supérieure avait été taillée en pointe et durcie au feu ; on ne les laissait dépasser le sol que de quatre doigts ; en outre, pour en assurer la solidité et la fixité, on comblait le fond des trous, sur une hauteur d’un pied, de terre qu’on foulait ; le reste était recouvert de branchages et de broussailles afin de cacher le piège. On en fit huit rangs, distants les uns des autres, de trois pieds. On les appelait lis, à cause de leur ressemblance avec cette fleur. En avant de ces trous, deux pieux longs d’un pied, dans lesquels s’enfonçait, un crochet de fer, étaient entièrement enfouis dans le sol ; on en semait partout et à intervalles rapprochés ; on leur donnait le nom d’aiguillons.
\subsection[{§ 74.}]{ \textsc{§ 74.} }
\noindent Ces travaux achevés, César, en suivant autant que le lui permit le terrain la ligne la plus favorable, fit, sur quatorze milles de tour, une fortification pareille à celle-là, mais inversement orientée, contre les attaques du dehors, afin que même des forces très supérieures ne pussent, s’il lui arrivait d’avoir à s’éloigner, envelopper les postes de défense ou ne le contraignissent à s’exposer dangereusement hors de son camp ; il ordonna que chacun se procure du fourrage et du blé pour trente jours.
\subsection[{§ 75.}]{ \textsc{§ 75.} }
\noindent Tandis que devant Alésia s’accomplissent ces travaux, les Gaulois, ayant tenu une assemblée des chefs, décident qu’il convient non pas d’appeler, comme le voulait Vercingétorix, tous les hommes en état de porter les armes, mais de demander à chaque cité un contingent déterminé, afin d’éviter que dans la confusion d’une telle multitude il devienne impossible de maintenir la discipline, de distinguer les troupes des divers peuples, de pourvoir au ravitaillement. On demande aux Héduens et à leurs clients, Ségusiaves, Ambivarètes, Aulerques, Brannovices, Blannovii, trente-cinq mille hommes ; un chiffre égal aux Arvernes, auxquels on joint les Eleutètes, les Cadurques, les Gabales, les Vellavii, qui sont, par longue tradition, leurs vassaux ; aux Séquanes, aux Sénons, aux Bituriges, aux Santons, aux Rutènes, aux Carnutes, douze mille hommes par cité ; aux Bellovaques dix mille ; huit mille aux Pictons, aux Turons, aux Parisii, aux Helvètes ; aux Ambiens, aux Médiomatrices, aux Petrocorii, aux Nerviens, aux Morins, aux Nitiobroges, cinq mille ; autant aux Aulerques Cénomans ; quatre mille aux Atrébates ; trois mille aux Véliocasses, aux Lexovii, aux Aulerques Eburovices ; mille aux Rauraques, aux Boïens ; vingt mille à l’ensemble des peuples qui bordent l’Océan et qui se donnent le nom d’Armoricains : Coriosolites, Redons, Ambibarii, Calètes, Osismes, Lémovices, Unelles. Les Bellovaques ne fournirent pas leur contingent, parce qu’ils prétendaient faire la guerre aux Romains à leur compte et à leur guise, et n’obéir aux ordres de personne ; pourtant, à la prière de Commios, ils envoyèrent deux mille hommes en faveur des liens d’hospitalité qui les unissaient à lui.
\subsection[{§ 76.}]{ \textsc{§ 76.} }
\noindent Ce Commios, comme nous l’avons exposé plus haut, avait fidèlement et utilement servi César, dans les années précédentes, en Bretagne ; en récompense, celui-ci avait ordonné que sa cité fût exempte d’impôts, lui avait restitué ses lois et ses institutions, et avait donné à Commios la suzeraineté sur les Morins. Pourtant, telle fut l’unanimité de la Gaule entière à vouloir reconquérir son indépendance et recouvrer son antique gloire militaire, que la reconnaissance et les souvenirs de l’amitié restèrent sans force, et qu’ils furent unanimes à se jeter dans la guerre de tout leur cœur et avec toutes leurs ressources. On réunit huit mille cavaliers et environ deux cent quarante mille fantassins et on procéda sur le territoire des Héduens au recensement et au dénombrement de ces forces, à la nomination d’officiers. Le commandement supérieur est confié à Commios l’Atrébate, aux Héduens Viridomaros et Eporédorix, à l’Arverne Vercassivellaunos, cousin de Vercingétorix. On leur adjoint des délégués des cités, qui formeront un conseil chargé de la conduite de la guerre. Tous partent pour Alésia pleins d’enthousiasme et de confiance, car aucun d’entre eux ne pensait qu’il fût possible de tenir devant le seul aspect d’une telle multitude, surtout quand il y aurait à livrer deux combats à la fois, les assiégés faisant une sortie tandis qu’à l’extérieur paraîtraient des forces si imposantes de cavalerie et d’infanterie.
\subsection[{§ 77.}]{ \textsc{§ 77.} }
\noindent Cependant les assiégés, une fois passé le jour pour lequel ils attendaient l’arrivée des secours, n’ayant plus de blé, ne sachant pas ce qu’on faisait chez les Héduens, avaient convoqué une assemblée et délibéraient sur la façon dont devait s’achever leur destin. Plusieurs avis furent exprimés, les uns voulant qu’on se rendît, les autres qu’on fît une sortie tandis qu’on en avait encore la force ; mais je ne crois pas devoir passer sous silence le discours de Critognatos, à cause de sa cruauté singulière et sacrilège. Ce personnage, issu d’une grande famille arverne et jouissant d’un grand prestige, parla en ces termes : « Je ne dirai rien de l’opinion de ceux qui parlent de reddition, mot dont ils voilent le plus honteux esclavage ; j’estime que ceux-là ne doivent pas être considérés comme des citoyens et ne méritent pas de faire partie du conseil. Je ne veux avoir affaire qu’à ceux qui sont pour la sortie, dessein dans lequel il vous semble à tous reconnaître le souvenir de l’antique vertu gauloise. Mais non, c’est lâcheté, et non pas vertu, que de ne pouvoir supporter quelque temps la disette. Aller au-devant de la mort, c’est d’un courage plus commun que de supporter la souffrance patiemment. Et pourtant, je me rangerais à cet avis, – tant je respecte l’autorité de ceux qui la préconisent – s’il ne s’agissait d’aventurer que nos existences ; mais en prenant une décision, nous devons tourner nos regards vers la Gaule entière, que nous avons appelée à notre secours. De quel cœur pensez-vous qu’ils combattront, quand en un même lieu auront péri quatre-vingt mille hommes de leurs familles, de leur sang, et qu’ils seront forcés de livrer bataille presque sur leurs cadavres ? Ne frustrez pas de votre appui ces hommes qui ont fait le sacrifice de leur vie pour vous sauver, et n’allez pas, par manque de sens et de réflexion, ou par défaut de courage, courber la Gaule entière sous le joug d’une servitude éternelle. Est-ce que vous doutez de leur loyauté et de leur fidélité, parce qu’ils ne sont pas arrivés au jour dit ? Eh quoi ? pensez-vous donc que ce soit pour leur plaisir que les Romains s’exercent chaque jour là-bas, dans les retranchements de la zone extérieure ? Si vous ne pouvez, tout accès vers nous leur étant fermé, apprendre par leurs messagers que l’arrivée des nôtres est proche, ayez-en pour témoins les Romains eux mêmes : car c’est la terreur de cet événement qui les fait travailler nuit et jour à leurs fortifications. Qu'est-ce donc que je conseille ? Faire ce que nos ancêtres ont fait dans la guerre qui n’était nullement comparable à celle-ci, une guerre des Cimbres et des Teutons obligés de s’enfermer dans leurs villes et pressés comme nous par la disette, ils ont fait servir à la prolongation de leurs existences ceux qui, trop âgés, étaient des bouches inutiles, et ils ne se sont point rendus. N'y eût-il pas ce précédent, je trouverais beau néanmoins que pour la liberté nous prenions l’initiative d’une telle conduite et en léguions l’exemple à nos descendants. Car en quoi cette guerre-là ressemblait-elle à celle d’aujourd’hui ? Les Cimbres ont ravagé la Gaule et y ont déchaîné un grand fléau : du moins un moment est venu où ils ont quitté notre sol pour aller dans d’autres contrées ; ils nous ont laissé notre droit, nos lois, nos champs, notre indépendance. Mais les Romains, que cherchent-ils ? Que veulent-ils ? C'est l’envie qui les inspire lorsqu’ils savent qu’une nation est glorieuse et ses armes puissantes, ils rêvent de s’installer dans ses campagnes et au cœur de ses cités, de lui imposer pour toujours le joug de l’esclavage. Jamais ils n’ont fait la guerre autrement. Si vous ignorez ce qui se passe pour les nations lointaines, regardez, tout près de vous, cette partie de la Gaule qui, réduite en province, ayant reçu des lois, des institutions nouvelles, soumise aux haches des licteurs, ploie sous une servitude éternelle.
\subsection[{§ 78.}]{ \textsc{§ 78.} }
\noindent Après discussion, on décide que ceux qui, malades ou trop âgés, ne peuvent rendre de services, sortiront de la ville, et qu’on tentera tout avant d’en venir au parti extrême de Critognatos ; mais on y viendra, s’il le faut, si les secours tardent, plutôt que de capituler ou de subir les conditions de paix du vainqueur. Les Mandubiens, qui pourtant les avaient accueillis dans leur ville, sont contraints d’en sortir avec leurs enfants et leurs femmes. Arrivés aux retranchements romains, ils demandaient, avec des larmes et toutes sortes de supplications, qu’on voulût bien les accepter comme esclaves et leur donner quelque nourriture. Mais César disposa sur le rempart des postes de garde et interdisait de les recevoir.
\subsection[{§ 79.}]{ \textsc{§ 79.} }
\noindent Sur ces entrefaites, Commios et les autres chefs à qui on avait donné le haut commandement arrivent devant Alésia avec toutes leurs troupes et, ayant occupé une colline située en retrait, s’établissent à mille pas à peine de nos lignes. Le lendemain, ils font sortir leur cavalerie et couvrent toute la plaine dont nous avons dit qu’elle avait trois milles de long ; leur infanterie, ils la ramènent un peu en arrière et l’établissent sur les pentes en la dérobant à la vue des Romains. D'Alésia, la vue s’étendait sur cet espace. Quand on aperçoit l’armée de secours, on s’assemble, on se congratule, tous les cœurs bondissent d’allégresse. Les assiégés font avancer leurs troupes et les établissent en avant de la ville ; ils jettent des passerelles sur le fossé le plus proche ou le comblent de terre, ils s’apprêtent à faire une sortie et à braver tous les hasards.
\subsection[{§ 80.}]{ \textsc{§ 80.} }
\noindent César dispose toute son infanterie sur ses deux lignes de retranchement afin que, en cas de besoin, chacun soit à son poste et le connaisse ; puis il ordonne que la cavalerie sorte du camp et engage le combat. De tous les camps, qui de toutes parts occupaient les crêtes, la vue plongeait, et tous les soldats, le regard attaché sur les combattants, attendaient l’issue de la lutte. Les Gaulois avaient disséminé dans les rangs de leur cavalerie des archers et des fantassins armés à la légère, qui devaient se porter au secours des leurs s’ils faiblissaient et briser les charges des nôtres. Blessés par eux à l’improviste, beaucoup de nos hommes abandonnaient le combat. Persuadés de la supériorité de leurs troupes, et voyant les nôtres accablés par le nombre, les Gaulois, de toutes parts, ceux qui étaient enfermés dans l’enceinte de nos lignes et ceux qui étaient venus à leur secours, encourageaient leurs frères d’armes par des clameurs et des hurlements. Comme l’action se déroulait sous les yeux de tous, et qu’il n’était pas possible qu’un exploit ou une lâcheté restassent ignorés, des deux côtés l’amour de la gloire et la crainte du déshonneur excitaient les hommes à se montrer braves. Le combat durait depuis midi, on était presque au coucher du soleil, et la victoire restait indécise, quand les Germains, massés sur un seul point, chargèrent l’ennemi en rangs serrés et le refoulèrent ; les cavaliers mis en fuite, les archers furent enveloppés et massacrés. De leur côté nos cavaliers, s’élançant des autres points du champ de bataille, poursuivirent les fuyards jusqu’à leur camp et ne leur permirent pas de se ressaisir. Ceux qui d’Alésia s’étaient portés en avant, accablés, désespérant presque de la victoire, rentrèrent dans la ville.
\subsection[{§ 81.}]{ \textsc{§ 81.} }
\noindent Les Gaulois ne laissent passer qu’un jour, et pendant ce temps fabriquent une grande quantité de passerelles, d’échelles et de harpons ; puis, au milieu de la nuit, en silence, ils sortent de leur camp et s’avancent vers nos fortifications de la plaine. Ils poussent une clameur soudaine, pour avertir les assiégés de leur approche, et ils se mettent en mesure de jeter leurs passerelles, de bousculer, en se servant de la fronde, de l’arc, en lançant des pierres, les défenseurs du retranchement, enfin de déployer tout l’appareil d’un assaut en règle. Au même moment, entendant la clameur, Vercingétorix fait sonner la trompette pour alerter ses troupes et les conduit hors de la ville. Les nôtres rejoignent au retranchement le poste qui, dans les jours précédents, avait été attribué individuellement à chacun : avec des frondes, des casse-têtes, des épieux qu’ils avaient disposés sur le retranchement, ils effraient les Gaulois et les repoussent. L'obscurité empêche qu’on voie devant soi, et les pertes sont lourdes des deux côtés. L'artillerie lance une grêle de projectiles. Cependant les légats Marcus Antonius et Caïus Trébonius, à qui incombait la défense de ce secteur, envoyaient sur les points où ils comprenaient que nous faiblissions, des renforts qu’ils empruntaient aux fortins situés en arrières.
\subsection[{§ 82.}]{ \textsc{§ 82.} }
\noindent Tant que les Gaulois étaient à une certaine distance du retranchement, la multitude de traits qu’ils lançaient leur assurait un avantage ; mais lorsqu’ils furent plus près, les aiguillons les transperçaient soudain, ou bien ils tombaient dans des trous et s’y empalaient, ou bien du haut du retranchement et des tours les javelots de siège les frappaient mortellement. Ayant sur tous les points subi des pertes sévères sans avoir pu percer nulle part, à l’approche du jour, craignant d’être tournés par leur flanc droit si on faisait une sortie du camp qui dominait la plaine, ils se retirèrent sur leurs positions. Quant aux assiégés, occupés à faire avancer les engins que Vercingétorix avait préparés en vue de la sortie, à combler les premiers fossés, ils s’attardèrent plus qu’il n’eût fallu à ces manœuvres, et ils apprirent la retraite des troupes de secours avant d’être parvenus au retranchement. Ayant ainsi échoué dans leur tentative, ils regagnèrent la ville.
\subsection[{§ 83.}]{ \textsc{§ 83.} }
\noindent Repoussés par deux fois avec de grandes pertes, les Gaulois délibèrent sur la conduite à tenir : ils consultent des hommes à qui les lieux sont familiers : ceux-ci les renseignent sur les emplacements des camps dominant la plaine et sur l’organisation de leur défense. Il y avait au nord une montagne qu’en raison de sa vaste superficie nous n’avions pu comprendre dans nos lignes, et on avait été forcé de construire le camp sur un terrain peu favorable et légèrement en pente. Il était occupé par les légats Laïus Antistius Réginus et Laïus Caninius Rébilus, à la tête de deux légions. Après avoir fait reconnaître les lieux par leurs éclaireurs, les chefs ennemis choisissent soixante mille hommes sur l’effectif total des cités qui avaient la plus grande réputation guerrière ; ils déterminent secrètement entre eux l’objet et le plan de leur action ; ils fixent l’heure de l’attaque au moment où l’on verra qu’il est midi. Ils donnent le commandement de ces troupes à l’Arverne Vercassivellaunos, l’un des quatre chefs, parent de Vercingétorix. Il sortit du camp à la première veille ; ayant à peu près terminé son mouvement au lever du jour, il se dissimula derrière la montagne et fit reposer ses soldats des fatigues de la nuit. Quand il vit qu’il allait être midi, il se dirigea vers le camp dont il a été question ; en même temps, la cavalerie s’approchait des fortifications de la plaine et le reste des troupes se déployait en avant du camp gaulois.
\subsection[{§ 84.}]{ \textsc{§ 84.} }
\noindent Vercingétorix, apercevant les siens du haut de la citadelle d’Alésia, sort de la place ; il fait porter en avant les fascines, les perches, les toits de protection, les faux, et tout ce qu’il avait préparé en vue d’une sortie. On se bat partout à la fois, on s’attaque à tous les ouvrages ; un point paraît-il faible, on s’y porte en masse. Les Romains, en raison de l’étendue des lignes, sont partout occupés, et il ne leur est pas facile de faire face à plusieurs attaques simultanées. Ce qui contribue beaucoup à effrayer nos soldats, ce sont les cris qui s’élèvent derrière eux, parce qu’ils voient que leur sort dépend du salut d’autrui le danger qu’on n’a pas devant les yeux est, en général, celui qui trouble le plus.
\subsection[{§ 85.}]{ \textsc{§ 85.} }
\noindent César, qui a choisi un bon observatoire suit l’action dans toutes ses parties ; il envoie du renfort sur les points menacés. Des deux côtés règne l’idée que cette heure est unique, que c’est celle de l’effort suprême : les Gaulois se sentent perdus s’ils n’arrivent pas à percer ; les Romains pensent que s’ils l’emportent, c’est la fin de toutes leurs misères. Le danger est surtout grand aux fortifications de la montagne où nous avons dit qu’on avait envoyé Vercassivellaunos. La pente défavorable du terrain joue un grand rôle. Les uns jettent des traits, les autres s’approchent en formant la tortue ; des troupes fraîches remplacent sans cesse les troupes fatiguées. La terre que tous les assaillants jettent dans nos ouvrages leur permet l’escalade et recouvre les obstacles que nous avions dissimulés dans le sol ; déjà les nôtres n’ont plus d’armes, et leurs forces les abandonnent.
\subsection[{§ 86.}]{ \textsc{§ 86.} }
\noindent Quand il apprend cela, César envoie Labiénus avec six cohortes au secours de ceux qui sont en péril ; il lui donne l’ordre, s’il ne peut tenir, de ramener des cohortes et de faire une contre-attaque, mais seulement à la dernière extrémité. Il se rend lui-même auprès des autres combattants, les exhorte à ne pas céder à la fatigue ; il leur montre que de ce jour, de cette heure dépend le fruit de tous les combats précédents. Les assiégés, désespérant de venir à bout des fortifications de la plaine, car elles étaient formidables, tentent l’escalade des hauteurs ; ils y portent toutes les machines qu’ils avaient préparées. Ils chassent les défenseurs des tours sous une grêle de traits, comblent les fossés avec de la terre et des fascines, font à l’aide de faux une brèche dans la palissade et le parapet.
\subsection[{§ 87.}]{ \textsc{§ 87.} }
\noindent César envoie d’abord le jeune Brutus avec des cohortes, puis son légat Caïus Fabius avec d’autres ; à la fin, la lutte devenant plus vive, il amène lui-même des troupes fraîches. Ayant rétabli le combat et refoulé l’ennemi, il se dirige vers l’endroit où il avait envoyé Labiénus ; il prend quatre cohortes au fort le plus voisin, et ordonne qu’une partie de la cavalerie le suive, que l’autre contourne les retranchements extérieurs et attaque l’ennemi à revers. Labiénus, voyant que ni terrassements ni fossés ne pouvaient arrêter l’élan de l’ennemi, rassemble trente-neuf cohortes, qu’il eut la chance de pouvoir tirer des postes voisins, et informe César de ce qu’il croit devoir faire.
\subsection[{§ 88.}]{ \textsc{§ 88.} }
\noindent César se hâte pour prendre part au combat. Reconnaissant son approche à la couleur de son vêtement – le manteau de général qu’il avait l’habitude de porter dans l’action – et apercevant les escadrons et les cohortes dont il s’était fait suivre – car des hauteurs que les Gaulois occupaient on voyait les pentes que descendait César –, les ennemis engagent le combat. Une clameur s’élève des deux côtés, et aussitôt y répond de la palissade et de tous les retranchements une clameur. Les nôtres, renonçant au javelot, combattent avec l’épée. Soudain les ennemis aperçoivent la cavalerie derrière eux. De nouvelles cohortes approchaient ils prirent la fuite. Nos cavaliers leur coupent la retraite. Le carnage est grand. Sédullus, chef militaire des Lémovices et leur premier citoyen, est tué ; l’Arverne Vercassivellaunos est pris vivant tandis qu’il s’enfuit ; on apporte à César soixante-quatorze enseignes ; bien peu, d’une armée si nombreuse, rentrent au camp sans blessure. Apercevant de la ville le massacre et la fuite de leurs compatriotes, les assiégés, désespérant d’être délivrés, ramènent leurs troupes du retranchement qu’elles attaquaient. A peine entendent-elles le signal de la retraite, les troupes de secours sortent de leur camp et s’enfuient. Si nos soldats n’avaient été harassés pour être maintes fois intervenus en renfort et avoir été à la peine toute la journée, on aurait pu détruire entièrement l’armée ennemie. La cavalerie, lancée à sa poursuite, atteint l’arrière-garde peu de temps après minuit ; beaucoup sont pris ou massacrés ; les autres, ayant réussi à s’échapper, se dispersent dans leurs cités.
\subsection[{§ 89.}]{ \textsc{§ 89.} }
\noindent Le lendemain, Vercingétorix convoque l’assemblée il déclare que cette guerre n’a pas été entreprise par lui à des fins personnelles, mais pour conquérir la liberté de tous ; puisqu’il faut céder à la fortune, il s’offre à eux, ils peuvent, à leur choix, apaiser les Romains par sa mort ou le livrer vivant. On envoie à ce sujet une députation à César. Il ordonne qu’on lui remette les armes, qu’on lui amène les chefs des cités. Il installa son siège au retranchement, devant son camp c’est là qu’on lui amène les chefs ; on lui livre Vercingétorix, on jette les armes à ses pieds. Il met à part les prisonniers héduens et arvernes, pensant essayer de se servir d’eux pour regagner ces peuples, et il distribue les autres à l’armée entière, à titre de butin, à raison d’un par tête.
\subsection[{§ 90.}]{ \textsc{§ 90.} }
\noindent Tout cela réglé, il part chez les Héduens : la cité fait sa soumission. Des ambassadeurs arvernes viennent l’y trouver, se déclarant prêts à exécuter ses ordres. Il exige un grand nombre d’otages. Il envoie les légions prendre leurs quartiers d’hiver. Il rend aux Héduens et aux Arvernes environ vingt mille prisonniers. Titus Labiénus, avec deux légions et la cavalerie, reçoit l’ordre de partir chez les Séquanes ; il lui adjoint Marcus Sempronius Rutilus. Il place Laïus Fabius et Lucius Minucius Basilus avec deux légions chez les Rèmes, pour que ceux-ci n’aient rien à souffrir de leurs voisins les Bellovaques. Laïus Antistius Réginus est envoyé chez les Ambivarètes, Titus Sextius chez les Bituriges, Laïus Caninius Rébilus chez les Rutènes, chacun avec une légion. Quintus Tullius Cicéron et Publius Sulpicius sont cantonnés à Chalon et à Mâcon, chez les Héduens, sur la Saône, pour veiller au ravitaillement. Quant à lui, il décide de prendre ses quartiers d’hiver à Bibracte. Lorsque ces événements sont connus à Rome par une lettre de César, on célèbre des supplications de vingt jours.
 \section[{Livre VIII}]{Livre VIII}\renewcommand{\leftmark}{Livre VIII}

\noindent J'ai dû céder à tes instances, Balbus, puisque mes refus quotidiens, au lieu d’être excusés sur la difficulté de la tâche, étaient interprétés comme la dérobade d’un paresseux, et je me suis engagé dans une entreprise pleine de périls : j’ai ajouté aux commentaires de la guerre des Gaules de notre cher César ce qui y manquait, et les ai reliés aux écrits suivants du même auteur ; de plus, j’ai terminé le dernier de ceux-ci, laissé inachevé depuis la guerre d’Alexandrie jusqu’à la fin non point de la guerre civile, dont nous ne voyons nullement le terme, mais de la vie de César. Puissent les lecteurs de ces commentaires savoir quelle violence je me suis faite pour les écrire ; j’espère échapper ainsi plus aisément au reproche de sotte présomption que j’encours en plaçant ma prose au milieu des œuvres de César. Car c’est un fait reconnu de tous : il n’est pas d’ouvrage, quelque soin qu’on y ait mis, qui ne le cède à la pureté de ces commentaires. Ils ont été publiés pour fournir des documents aux historiens sur des événements si considérables ; or ; telle est la valeur que chacun leur attribue qu’ils semblent, au lieu d’avoir facilité la tâche des historiens, la leur avoir rendue impossible. Et cependant notre admiration passe encore celle des autres : car s’ils savent quelle est la perfection souveraine de l’ouvrage, nous savons, en autre, avec quelle facilité et quelle promptitude il l’a écrit. César n’avait pas seulement au plus haut degré le don du style et la pureté naturelle de l’expression, mais il avait aussi le talent d’expliquer ses desseins avec une clarté et une exactitude absolues. Pour moi, il ne m’a même pas été donné de prendre part à la guerre d’Alexandrie ni à la guerre d’Afrique ; sans doute, ces guerres nous sont, en partie, connues par les propos de César, mais c’est autre chose d’entendre un récit dont la nouveauté nous captive ou qui nous transporte d’admiration, autre chose de l’écouter pour en faire un rapport qui aura valeur de témoignage. Mais que fais-je ? tandis que je rassemble toutes les excuses possibles pour n’être pas comparé à César, je m’expose au reproche même de présomption que je veux éviter, en paraissant croire que semblable comparaison puisse venir à l’esprit de quelqu’un. Adieu.\par
\subsection[{§ 1.}]{ \textsc{§ 1.} }
\noindent La Gaule entière était vaincue ; depuis l’été précédent, César n’avait pas cessé de se battre, et il désirait donner aux soldats, après tant de fatigues, le repos réparateur des quartiers d’hiver : mais alors on apprit qu’un grand nombre de cités à la fois recommençaient à faire des plans de guerre et complotaient. On expliquait cette attitude par des motifs vraisemblables : tous les Gaulois s’étaient rendu compte qu’avec les troupes les plus nombreuses, si elles étaient concentrées en un seul lieu, on ne pouvait résister aux Romains, mais que si plusieurs peuples les attaquaient en même temps sur divers points, l’armée romaine n’aurait pas assez de ressources, ni de temps, ni d’effectifs, pour faire face à tout ; dût quelque cité en souffrir, il lui fallait accepter l’épreuve, si en retenant ainsi l’ennemi elle permettait aux autres de reconquérir leur indépendance.
\subsection[{§ 2.}]{ \textsc{§ 2.} }
\noindent César ne voulut pas laisser les Gaulois se fortifier dans cette idée : confiant à son questeur Marcus Antonius le commandement de ses quartiers d’hiver, il quitte Bibracte, la veille des calendes de janvier, avec une escorte de cavaliers, pour rejoindre la treizième légion, qu’il avait placée à proximité de la frontière héduenne, dans le pays des Bituriges ; il lui adjoint la onzième, qui était la plus voisine. Laissant deux cohortes de chacune à la garde des bagages, il emmène le reste des troupes dans les plus fertiles campagnes des Bituriges : ce peuple avait un vaste territoire, où les villes étaient nombreuses, et l’hivernage d’une seule légion n’avait pu suffire à l’empêcher de préparer la guerre et de former des complots.
\subsection[{§ 3.}]{ \textsc{§ 3.} }
\noindent L'arrivée soudaine de César produisit l’effet qu’elle devait nécessairement produire sur des gens surpris et dispersés tandis que, très tranquilles, ils cultivaient leurs champs, la cavalerie tomba sur eux avant qu’ils pussent se réfugier dans les villes. Car même l’indice qui signale communément une incursion ennemie, un ordre de César l’avait supprimé : il avait interdit qu’on mît le feu aux constructions, pour ne pas manquer de fourrage et de blé, au cas où il voudrait avancer plus loin, et pour éviter que la vue des incendies ne donnât l’alarme. On avait fait plusieurs milliers de prisonniers, et la terreur s’était répandue chez les Bituriges : ceux qui avaient pu échapper à la première approche des Romains s’étaient réfugiés chez les voisins, se fiant à des liens d’hospitalité privée ou à l’alliance qui unissait les peuples. Vainement car César, par des marches forcées, se montre partout, et ne donne à aucune cité le temps de penser au salut d’autrui plutôt qu’au sien ; par cette promptitude, il retenait dans le devoir les peuples amis, et ceux qui hésitaient, il les amenait par la terreur à accepter la paix. Devant une telle situation, voyant que la clémence de César leur rendait possible de redevenir ses amis et que les cités voisines, sans être aucunement punies, avaient été admises à donner des otages et à se soumettre, les Bituriges suivirent leur exemple.
\subsection[{§ 4.}]{ \textsc{§ 4.} }
\noindent Pour récompenser ses soldats d’avoir supporté avec tant de patience une campagne si dure, d’avoir montré la plus parfaite persévérance dans la saison des jours courts, dans des étapes très difficiles, par des froids intolérables, César leur promet, comme gratification tenant lieu de butin, deux cents sesterces par tête, mille pour les centurions ; puis il renvoie les légions dans leurs quartiers et regagne Bibracte après une absence de quarante jours. Comme il y rendait la justice, les Bituriges lui envoient une ambassade pour demander secours contre les Carnutes, qui, se plaignaient-ils, leur avaient déclaré la guerre. A cette nouvelle, bien qu’il n’eût séjourné que dix-huit jours à Bibracte, il tire de leurs quartiers d’hiver, sur la Saône, les quatorzième et sixième légions, qui avaient été placées là, comme on l’a vu au livre précédent, pour assurer le ravitaillement, et il part ainsi avec deux légions pour aller châtier les Carnutes.
\subsection[{§ 5.}]{ \textsc{§ 5.} }
\noindent Quand ceux-ci entendent parler de l’approche d’une armée, ils se souviennent des malheurs des autres et, abandonnant leurs villages et leurs villes, où ils habitaient dans d’étroites constructions de fortune qu’ils avaient bâties rapidement pour pouvoir passer l’hiver (car leur récente défaite leur avait coûté un grand nombre de villes), ils s’enfuient dans toutes les directions. César, ne voulant pas exposer les soldats aux rigueurs de la mauvaise saison qui était alors dans son plein, campe dans la capitale des Carnutes, Cénabum, où il entassa ses troupes partie dans les maisons des Gaulois, partie dans les abris qu’on avait formés en jetant rapidement du chaume sur les tentes. Toutefois, il envoie la cavalerie et l’infanterie auxiliaire partout où l’on disait que l’ennemi s’était retiré ; et non sans succès, car les nôtres rentrent, le plus souvent, chargés de butin. Les difficultés de l’hiver, la crainte du danger accablaient les Carnutes ; chassés de leurs demeures, ils n’osaient faire nulle part d’arrêt prolongé, et leurs forêts ne les protégeaient pas entre l’extrême violence des intempéries : ils finissent par se disperser chez les peuples du voisinage, non sans avoir perdu une grande partie des leurs.
\subsection[{§ 6.}]{ \textsc{§ 6.} }
\noindent César, jugeant qu’il suffisait, au plus fort de la mauvaise saison, de disperser les groupes qui se formaient, afin de prévenir par ce moyen la naissance d’une guerre, ayant d’autre part la conviction, autant qu’on pouvait raisonnablement prévoir, qu’aucune grande guerre ne saurait éclater pendant qu’on était encore en quartiers d’hiver, confia ses deux légions à Caïus Trébonius, avec ordre d’hiverner à Cénabum ; quant à lui, comme de fréquentes ambassades des Rèmes l’avertissaient que les Bellovaques, dont la gloire militaire surpassait celle de tous les Gaulois et des Belges, unis aux peuples voisins sous la conduite du Bellovaque Corréos et de l’Atrébate Commios, mobilisaient et concentraient leurs forces, dans le dessein de prononcer une attaque en masse contre les Suessions, qu’il avait placés sous l’autorité des Rèmes, estimant, d’autre part, que son intérêt autant que son honneur exigeaient qu’il ne fût fait aucun mal à des alliés dont Rome avait tout lieu de se louer, il rappelle la onzième légion, écrit par ailleurs à Caïus Fabius d’amener chez les Suessions les deux légions qu’il avait, et demande à Labiénus l’une des deux siennes. C'est ainsi que, dans la mesure où le permettaient la répartition des quartiers et les nécessités militaires, il ne faisait supporter qu’à tour de rôle aux légions, sans jamais se reposer lui-même, les fatigues des expéditions.
\subsection[{§ 7.}]{ \textsc{§ 7.} }
\noindent Quand il a réuni ces troupes, il marche contre les Bellovaques, campe sur leur territoire et envoie dans toutes les directions des détachements de cavalerie pour faire quelques prisonniers qui pourront lui apprendre les desseins de l’ennemi. Les cavaliers, s’étant acquittés de leur mission, rapportent qu’ils n’ont trouvé que peu d’hommes dans les maisons, – et qui n’étaient pas restés pour cultiver leurs champs (car on avait procédé avec soin à une évacuation totale), mais qu’on avait renvoyés pour faire de l’espionnage. En demandant à ces hommes où se trouvait le gros de la population et quelles étaient les intentions des Bellovaques, César obtint les renseignements suivants : tous les Bellovaques en état de porter les armes s’étaient rassemblés en un même lieu, et avec eux les Ambiens, les Aulerques, les Calètes, les Véliocasses, les Atrébates ; ils avaient choisi pour leur camp une position dominante, au milieu d’un bois qu’entourait un marais, et ils avaient réuni tout leurs bagages dans des forêts situées en arrière. Nombreux étaient les chefs qui avaient poussé à la guerre, mais c’était surtout à Corréos que la masse obéissait, parce qu’on le savait animé d’une haine particulièrement violente contre Rome. Peu de jours auparavant, l’Atrébate Commios avait quitté le camp pour aller chercher des renforts chez les Germains, qui étaient à proximité et en nombre infini. Le plan des Bellovaques, arrêté de l’avis unanime des chefs et approuvé avec enthousiasme par le peuple, était le suivant si, comme on le disait, César venait avec trois légions, ils offriraient le combat, pour ne pas être forcés plus tard de lutter avec l’armée entière dans des conditions beaucoup plus dures ; s’il amenait de plus gros effectifs, ils ne quitteraient pas la position qu’ils avaient choisie, mais ils empêcheraient les Romains, en dressant des embuscades, de faire du fourrage, qui, vu la saison, était rare et dispersé, et de se procurer du blé et autres vivres.
\subsection[{§ 8.}]{ \textsc{§ 8.} }
\noindent César, en possession de ces renseignements que confirmait l’accord de nombreux témoignages, jugeant que le plan qu’on lui exposait était fort sage et très éloigné de l’ordinaire témérité des Barbares, décida qu’il devait tout faire pour que l’ennemi, méprisant la faiblesse de ses effectifs, livrât bataille au plus tôt. Il avait, en effet, avec lui ses légions les plus anciennes, d’une valeur hors ligne, la septième, la huitième et la neuvième, plus une autre, la onzième, dont on pouvait attendre beaucoup, qui était composée d’excellents éléments, mais qui pourtant, après huit ans de campagnes, n’avait pas, comparée aux autres, la même réputation de solidité éprouvée. Il convoque donc un conseil, expose tout ce qu’il a appris, affermit le courage des troupes. Pour tâcher d’attirer l’ennemi au combat en ne lui montrant que trois légions, il règle ainsi l’ordre de marche : les septième, huitième et neuvième légions iraient en avant ; ensuite viendraient les bagages, qui, bien que tous groupés ensemble, ne formaient qu’une assez mince colonne, comme c’est l’usage dans les expéditions ; la onzième légion fermerait la marche : ainsi on éviterait de montrer à l’ennemi des effectifs supérieurs à ce qu’il souhaitait. Tout en observant cette disposition, on forme à peu près le carré, et l’armée ainsi rangée arrive à la vue de l’ennemi plus tôt qu’il ne s’y attendait.
\subsection[{§ 9.}]{ \textsc{§ 9.} }
\noindent Lorsque soudain les Gaulois voient les légions s’avancer d’un pas ferme et rangées comme à la bataille, eux dont on avait rapporté à César les résolutions pleines d’assurance, soit qu’alors l’idée du danger les intimide, ou que la soudaineté de notre approche les surprenne, ou qu’ils veuillent attendre nos décisions, ils se contentent de ranger leurs troupes en avant du camp sans quitter la hauteur. César avait souhaité la bataille mais, surpris à la vue d’une telle multitude, dont le séparait une vallée plus profonde que large, il établit son camp en face du camp ennemi. Il fait construire un rempart de douze pieds, avec un parapet proportionné à cette hauteur, creuser deux fossés de quinze pieds de large à parois verticales, élever de nombreuses tours à trois étages, jeter entre elles des ponts que protégeaient du côté extérieur des parapets d’osier : de la sorte le camp était défendu par un double fossé et par un double rang de défenseurs, l’un qui, des passerelles, moins exposé en raison de la hauteur de sa position, pouvait lancer ses traits avec plus d’assurance et à plus longue portée, l’autre qui était placé plus près de l’assaillant, sur le rempart même, et que la passerelle abritait de la chute des projectiles. Il garnit les portes de battants et les flanqua de tours plus hautes.
\subsection[{§ 10.}]{ \textsc{§ 10.} }
\noindent Le but de cette fortification était double. L'importance des ouvrages devait, en faisant croire que César avait peur, encourager les Barbares ; d’autre part, comme il fallait aller loin pour faire du fourrage et se procurer du blé, de faibles effectifs pouvaient assurer la défense du camp, que protégeaient déjà ses fortifications. Il arrivait fréquemment que, de part et d’autre, de petits groupes s’avançaient en courant et escarmouchaient entre les deux camps, sans franchir le marais ; parfais cependant il était traversé soit par nos auxiliaires gaulois ou germains qui poursuivaient alors vivement l’ennemi, soit par l’ennemi lui-même qui, à son tour, nous repoussait assez loin ; il arrivait aussi, comme on allait chaque jour au fourrage – et l’inconvénient était inévitable, car les granges où l’on devait aller prendre le foin étaient rares et dispersées –, qu’en des endroits d’accès difficile des fourrageurs isolés fussent enveloppés ; ces incidents ne nous causaient que des pertes assez légères de bêtes et de valets, mais ils inspiraient aux Barbares des espoirs insensés, et cela d’autant plus que Commios qui, je l’ai dit, était allé chercher des auxiliaires germains, venait d’arriver avec des cavaliers : ils n’étaient pas plus de cinq cents, mais que les Germains fussent là, c’était assez pour exalter les Barbares.
\subsection[{§ 11.}]{ \textsc{§ 11.} }
\noindent César, voyant que les jours passaient et que l’ennemi restait dans son camp sous la protection d’un marais et avec l’avantage d’une position naturelle très forte, qu’on ne pouvait en faire l’assaut sans une lutte meurtrière et que pour l’investir il fallait une armée plus nombreuse, écrit à Caïus Trébonius d’appeler au plus vite la treizième légion, qui hivernait avec le légat Titus Sextius chez les Bituriges, et, ayant ainsi trois légions, de venir le trouver à grandes étapes ; en attendant, il emprunte à tour de rôle à la cavalerie des Rèmes, des Lingons et des autres peuples, dont il avait mobilisé un fort contingent, des détachements qu’il charge d’assurer la protection des fourrageurs en soutenant les brusques attaques de l’ennemi.
\subsection[{§ 12.}]{ \textsc{§ 12.} }
\noindent Chaque jour on procédait de la sorte, et déjà l’habitude amenait la négligence, conséquence ordinaire de la routine ; les Bellovaques, qui savaient où se pistaient chaque jour nos cavaliers, font dresser par des fantassins d’élite une embuscade dans un endroit boisé, et y envoient le lendemain des cavaliers, qui devront d’abord attirer les nôtres, pour qu’ensuite les gens de l’embuscade les enveloppent et les attaquent. La mauvaise chance tomba sur les Rèmes, dont c’était le jour de service. Apercevant soudain des cavaliers ennemis, comme ils étaient les plus nombreux et n’éprouvaient que du mépris pour cette poignée d’hommes, ils les poursuivirent avec trop d’ardeur, et furent entourés de tous côtés par les fantassins. Surpris par cette attaque, ils se retirèrent à plus vive allure que ne le veut la règle ordinaire d’un combat de cavalerie, et perdirent le premier magistrat de leur cité, Vertiscos, qui commandait la cavalerie : il pouvait à peine, en raison de son grand âge, se tenir à cheval, mais, selon l’usage des Gaulois, il n’avait pas voulu que cette raison le dispensât du commandement, ni que l’on combattît sans lui. Ce succès – et la mort du chef civil et militaire des Rèmes – enorgueillit et excite l’ennemi ; les nôtres apprennent à leurs dépens à reconnaître les lieux avec plus de soin avant d’établir leurs postes, et à poursuivre avec plus de prudence quand l’ennemi cède le terrain.
\subsection[{§ 13.}]{ \textsc{§ 13.} }
\noindent Cependant il ne se passe pas de jour qu’on ne se batte à la vue des deux camps, aux endroits guéables du marais. Au cours d’un de ces engagements, les Germains que César avait fait venir d’au-delà du Rhin pour les faire combattre mêlés aux cavaliers, franchissent résolument tous ensemble le marécage, tuent les quelques ennemis qui résistent et poursuivent avec vigueur la masse des autres ; la peur saisit l’ennemi non seulement ceux qui étaient serrés de près ou que les projectiles atteignaient de loin, mais même les troupes qui étaient, selon l’habitude, placées en soutien à bonne distance, prirent honteusement la fuite et, délogés à plusieurs reprises de positions dominantes, ils ne s’arrêtèrent qu’une fois à l’abri de leur camp : quelques-uns même, confus de leur conduite, se sauvèrent au-delà. Cette aventure démoralisa si fort toute l’armée ennemie qu’on n’aurait pu dire qui l’emportait de leur insolence au moindre succès ou de leur frayeur au moindre revers.
\subsection[{§ 14.}]{ \textsc{§ 14.} }
\noindent Plusieurs jours se passèrent sans qu’ils bougent de ce camp ; lorsqu’ils apprennent que les légions et le légat Caïus Trébonius sont à peu de distance, les chef des Bellovaques, craignant un blocus comme celui d’Alésia, renvoient pendant la nuit ceux qui sont trop âgés, ou trop faibles, ou sans armes, et avec eux tous les bagages. Ils étaient occupés à mettre de l’ordre dans la colonne où régnaient l’agitation et la confusion (les Gaulois ont l’habitude, même pour les expéditions les plus braves, de se faire suivre d’une foule de chariots), lorsque le jour les surprend : ils rangent devant le camp des troupes en armes, pour empêcher les Romains de se mettre à leur poursuite avant que la colonne des bagages ne soit déjà à une certaine distance. César, s’il ne pensait pas devoir attaquer des forces prêtes à la résistance quand il fallait gravir une colline si escarpée, n’hésitait pas en revanche à faire avancer ses légions assez loin pour que les Barbares, sous la menace de nos troupes, ne pussent quitter les lieux sans danger. Voyant donc que les deux camps étaient séparés par le marais qui formait un obstacle sérieux et capable d’empêcher une poursuite rapide, observant d’autre part que la hauteur qui, de l’autre côté du marais, touchait presque au camp ennemi, en était séparée par un petit vallon, il jette des passerelles sur le marais, le fait franchir par ses légions, et atteint promptement le plateau qui couronnait la colline et qu’une pente rapide protégeait sur les deux flancs. Là, il reforme ses légions, puis, ayant gagné l’extrémité du plateau, les range en bataille sur un emplacement d’où les projectiles d’artillerie pouvaient atteindre les formations ennemies.
\subsection[{§ 15.}]{ \textsc{§ 15.} }
\noindent Les Barbares, confiants dans leur position, ne refusaient pas de combattre si jamais les Romains essayaient de monter à l’assaut de la colline ; quant à renvoyer leurs troupes peu à peu par petits paquets, ils ne pouvaient le faire sans avoir à craindre que la dispersion ne les démoralisât : ils restèrent donc en ligne. Quand il les voit bien décidés, César, laissant vingt cohortes sous les armes, trace un camp à cet endroit et ordonne qu’on le fortifie. Les travaux achevés, il range les légions devant le retranchement, place les cavaliers en grand-garde avec leurs chevaux tout bridés. Les Bellovaques, voyant que les Romains étaient prêts à la poursuite, et ne pouvant, d’autre part, ni veiller toute la nuit, ni demeurer sans risque plus longtemps sur place, eurent recours pour se retirer au stratagème suivant. Se faisant passer de main en main les bottes de paille et les fascines qui leur avaient servi de sièges – on a vu dans les précédents commentaires de César que les Gaulois ont coutume de s’asseoir sur une fascine – et dont il y avait dans le camp une grande quantité, ils les placèrent devant leur ligne et, à la chute du jour, à un signal donné, ils les enflammèrent toutes ensemble. De la sorte, un rideau de feu déroba brusquement toutes leurs troupes à la vue des Romains. Les Barbares profitèrent de ce moment-là pour s’enfuir à toutes jambes.
\subsection[{§ 16.}]{ \textsc{§ 16.} }
\noindent La barrière des incendies masquait à César la retraite des ennemis ; mais, se doutant qu’ils les avaient allumés pour faciliter leur fuite, il porte les légions en avant et lance la cavalerie à leur poursuite ; toutefois, craignant un piège, au cas où l’intention de l’ennemi serait de se maintenir sur sa position et de nous attirer sur un terrain désavantageux, il n’avance lui-même qu’avec lenteur. Les cavaliers hésitaient à entrer dans la fumée et les flammes qui étaient fort épaisses ; ceux qui, plus audacieux, y pénétraient, voyaient à peine la tête de leurs chevaux : ils craignirent une embuscade, et laissèrent les Bellovaques se retirer librement. Ainsi cette fuite où se mêlaient la peur et l’habileté leur permit de gagner sans être aucunement inquiétés, à une distance de dix milles au plus, une position très forte où ils établirent leur camp. De là, plaçant souvent des fantassins et des cavaliers en embuscade, ils faisaient beaucoup de mal aux Romains quand ceux-ci allaient au fourrage.
\subsection[{§ 17.}]{ \textsc{§ 17.} }
\noindent Ces incidents se multipliaient, lorsque César apprit par un prisonnier que Corréos, chef des Bellovaques, ayant formé une troupe de six mille fantassins particulièrement valeureux et de mille cavaliers choisis entre tous, les avait placés en embuscade à un endroit où il soupçonnait que l’abondance du blé et du fourrage attirerait les Romains. Informé de ce plan, il fait sortir plus de légions qu’à l’habitude et envoie en avant la cavalerie, qui escortait toujours les fourrageurs ; il y mêle des auxiliaires légèrement armés ; lui-même, à la tête des légions, approche le plus près possible.
\subsection[{§ 18.}]{ \textsc{§ 18.} }
\noindent Les ennemis placés en embuscade avaient choisi pour l’action qu’ils méditaient une plaine qui n’avait pas plus de mille pas d’étendue en tous sens, et que défendaient de tous côtés des bois ou une rivière très difficile à franchir ; ils s’étaient embusqués alentour, l’enveloppant comme d’un filet. Les nôtres, qui s’étaient rendu compte des intentions de l’ennemi, et qui étaient équipés pour le combat et le désiraient, car, se sentant soutenus par les légions qui suivaient, il n’était pas de lutte qu’ils n’acceptassent, entrèrent dans la plaine escadron par escadron. Les voyant arriver, Corréos pensa que l’occasion d’agir lui était offerte : il commença par se montrer avec un petit nombre d’hommes et chargea les premières unités. Les nôtres soutiennent fermement le choc, en évitant de se réunir en un groupe compact, formation qui généralement, dans les combats de cavalerie, quand elle est l’effet de quelque panique, rend redoutable pour la troupe son nombre même.
\subsection[{§ 19.}]{ \textsc{§ 19.} }
\noindent Les escadrons avaient pris chacun position et n’engageaient que de petits groupes qui se relayaient en évitant de laisser prendre de flanc les combattants : alors, tandis que Corréos luttait, les autres sortent des bois. De vifs combats s’engagent dans deux directions. L'action se prolongeant sans décision, le gros des fantassins, en ordre de bataille, sort peu à peu des bois : il força nos cavaliers à la retraite. Mais ceux-ci sont promptement secourus par l’infanterie légère qui, je l’ai dit, avait été envoyée en avant des légions, et, mêlée à nos escadrons, elle combat de pied ferme. Pendant un certain temps, on lutte à armes égales ; puis, comme le voulait la loi naturelle des batailles, ceux qui avaient été les premiers attaqués ont le dessus par cela même que l’embuscade ne leur avait causé aucun effet de surprise. Sur ces entrefaites, les légions approchent, et simultanément les nôtres et l’ennemi apprennent par de nombreux agents de liaison que le général en chef est là avec des forces toutes prêtes. A cette nouvelle, nos cavaliers, que rassure l’appui des cohortes, déploient une vigueur extrême, ne voulant pas avoir à partager avec les légions, s’ils ne mènent pas l’action assez vivement, l’honneur de la victoire ; les ennemis, eux, perdent courage et cherchent de tous côtés par quels chemins fuir. Vainement : le terrain dont ils avaient voulu faire un piège pour les Romains devenait un piège pour eux. Battus, bousculés, ayant perdu la plus grande partie des leurs, ils réussissent néanmoins à s’enfuir en désordre, les uns gagnant les bois, les autres la rivière ; mais, tandis qu’ils fuient, les nôtres, au cours d’une vigoureuse poursuite, les achèvent. Cependant Corréos, que nul malheur n’abat, ne se résout point à abandonner la lutte et à gagner les bois, et il ne cède pas davantage aux sommations des nôtres qui l’invitent à se rendre ; mais, combattant avec un grand courage et nous blessant beaucoup de monde, il finit par obliger les vainqueurs, emportés par la colère, à l’accabler de leurs traits.
\subsection[{§ 20.}]{ \textsc{§ 20.} }
\noindent Ainsi venait de se terminer l’affaire quand César arriva sur le champ de bataille ; il pensa qu’après un tel désastre l’ennemi, lorsque la nouvelle lui en parviendrait, ne resterait plus dans son camp, dont la distance au lieu du carnage n’était, disait-on, que d’environ huit milles : aussi, bien que la rivière lui opposât un obstacle sérieux, il la fait passer par son armée et marche en avant. Les Bellovaques et les autres peuples voient soudain arriver, en petit nombre et blessés, les quelques fuyards que les bois avaient préservés du massacre : devant un malheur aussi complet, apprenant la défaite, la mort de Corréos, la perte de leur cavalerie et de leurs meilleurs fantassins, ne doutant pas que les Romains n’approchent, ils convoquent sur-le-champ l’assemblée au son des trompettes et proclament qu’il faut envoyer à César des députés et des otages.
\subsection[{§ 21.}]{ \textsc{§ 21.} }
\noindent Tous approuvent la mesure ; mais Commios l’Atrébate s’enfuit auprès des Germains à qui il avait emprunté des auxiliaires pour cette guerre. Les autres envoient immédiatement des députés à César ; ils lui demandent de se contenter d’un châtiment que sans aucun doute, étant donné sa clémence et sa bonté, s’il était en son pouvoir de l’infliger sans combat à des ennemis dont les forces seraient intactes, il ne leur ferait jamais subir. « Les forces de cavalerie des Bellovaques ont été anéanties ; plusieurs milliers de fantassins d’élite ont péri, à peine si ont pu s’échapper ceux qui ont annoncé le désastre. Toutefois ce combat a procuré aux Bellovaques un grand bien, pour autant que pareil malheur en peut comporter : Corréos, auteur responsable de la guerre, agitateur du peuple, a été tué. Jamais, en effet, tant, qu’il a vécu, le pouvoir du sénat ne fut aussi fort que celui de la plèbe ignorante. »
\subsection[{§ 22.}]{ \textsc{§ 22.} }
\noindent A ces prières des députés, César répond en leur rappelant que l’année précédente les Bellovaques sont entrés en guerre en même temps que les autres peuples de la Gaule, et que seuls entre tous ils ont persévéré avec opiniâtreté, sans que la reddition des autres les ramenât à la raison. Il sait fort bien que la responsabilité des fautes se met très volontiers au compte des morts. Mais, en vérité, personne n’est assez puissant pour pouvoir faire naître la guerre et la conduire contre le gré des chefs, malgré l’opposition du sénat et la résistance de tous les gens de bien, avec le seul concours d’une plèbe sans autorité. Néanmoins, il se contentera du châtiment qu’ils se sont eux-mêmes attirés. »
\subsection[{§ 23.}]{ \textsc{§ 23.} }
\noindent La nuit suivante, les députés rapportent aux leurs la réponse obtenue, ils rassemblent les otages nécessaires. Les députés des autres peuples, qui guettaient le résultat de l’ambassade des Bellovaques, se précipitent. Ils donnent des otages, exécutent les conditions imposées ; seul Commios s’abstient, car il avait trop peur pour confier à qui que ce fût son existence. C'est qu’en effet l’année précédente Titus Labiénus, en l’absence de César qui rendait la justice dans la Gaule citérieure, ayant appris que Commios intriguait auprès des cités et formait une coalition contre César, crut qu’il était possible d’étouffer sa trahison sans manquer aucunement à la loyauté. Comme il ne pensait pas qu’il vînt au camp, si on l’y invitait, il ne voulut pas éveiller sa défiance en essayant, et envoya Caïus Volusénus Quadratus avec mission de le tuer sous le prétexte d’une entrevue. Il lui adjoignit des centurions spécialement choisis pour cette besogne. L'entrevue avait lieu, et Volusénus – c’était le signal convenu – venait de saisir la main de Commios : mais le centurion, soit qu’il fût troublé par ce rôle nouveau pour lui, soit que les familiers de Commios l’eussent promptement arrêté, ne put achever sa victime : le premier coup d’épée qu’il lui donna lui fit néanmoins une blessure grave à la tête. De part et d’autre on avait dégainé, mais chacun songea moins à combattre qu’à se frayer un passage pour fuir : les nôtres, en effet, croyaient que Commios avait reçu une blessure mortelle, et les Gaulois, comprenant qu’il y avait un piège tendu, craignaient que le danger fût au-delà de ce qu’ils voyaient. A la suite de cette affaire Commios, disait-on, avait résolu de ne jamais se trouver en présence d’aucun Romain.
\subsection[{§ 24.}]{ \textsc{§ 24.} }
\noindent Vainqueur des nations les plus belliqueuses, César, voyant qu’il n’y avait plus aucune cité qui préparât une guerre de résistance, mais qu’en revanche nombreux étaient les habitants qui abandonnaient les villes, désertaient les campagnes pour éviter d’obéir aux Romains, décide de répartir son armée dans plusieurs régions. Il s’adjoint le questeur Marcus Antonius avec la douzième légion. Il envoie le légat Laïus Fabius avec vingt-cinq cohortes à l’autre extrémité de la Gaule, parce qu’il entendait dire que là-bas certains peuples étaient en armes, et que les deux légions du légat Laïus Caninius Rébilus, qui était dans ces contrées, ne lui paraissaient pas assez solides. Il appelle Titus Labiénus auprès de lui ; la quinzième légion, qui avait passé l’hiver avec ce dernier, il l’envoie dans la Gaule qui jouit du droit de cité pour assurer la protection des colonies de citoyens romains, voulant ainsi éviter qu’une descente de Barbares ne leur infligeât un malheur semblable à celui qu’avaient subi, l’été précédent, les Tergestins, qui avaient été brusquement attaqués et pillés par eux. De son côté, il part pour ravager et saccager le pays d’Ambiorix ; ayant renoncé à l’espoir de réduire ce personnage, bien qu’il l’eût contraint de trembler et de fuir, il jugeait que son honneur exigeait au moins cette satisfaction : faire de son pays un désert, y tout détruire, hommes, maisons, bétail, si bien qu’Ambiorix, abhorré des siens, – si le sort permettait qu’il en restât -n’eût plus aucun moyen, en raison de tels désastres, de rentrer dans sa cité.
\subsection[{§ 25.}]{ \textsc{§ 25.} }
\noindent Il dirigea sur toutes les parties du territoire d’Ambiorix, soit des légions, soit des auxiliaires, et massacrant, incendiant, pillant, porta partout la désolation ; un grand nombre d’hommes furent tués ou faits prisonniers. Il envoie ensuite Labiénus avec deux légions chez les Trévires, ce peuple, à cause du voisinage de la Germanie, était entraîné à la guerre, qu’il faisait quotidiennement ; sa civilisation primitive et ses mœurs barbares le faisaient assez semblable aux Germains, et il n’obéissait jamais que sous la pression d’une armée.
\subsection[{§ 26.}]{ \textsc{§ 26.} }
\noindent Sur ces entrefaites, le légat Laïus Caninius, informé qu’une grande multitude d’ennemis s’était rassemblée dans le pays des Pictons par une lettre et des messagers de Duratios, qui était resté constamment fidèle à l’amitié des Romains alors qu’une partie assez importante de sa cité avait fait défection, se dirigea vers la ville de Lémonum. En approchant, il eut par des prisonniers des informations plus précises : plusieurs milliers d’hommes, conduits par Dumnacos, chef des Andes, assiégeaient Duratios dans Lémonum ; n’osant pas risquer dans une rencontre des légions peu solides. Il campa sur une forte position. Dumnacos, ayant appris l’arrivée de Caninius, tourne toutes ses forces contre les légions et entreprend d’attaquer le camp romain. Après y avoir vainement employé plusieurs jours sans arriver, malgré de gros sacrifices, à enlever aucune partie des retranchements, il revient assiéger Lémonum.
\subsection[{§ 27.}]{ \textsc{§ 27.} }
\noindent Dans le même temps, le légat Caïus Fabius, tandis qu’il reçoit la soumission d’un grand nombre de cités et la sanctionne en se faisant remettre des otages, apprend par une lettre de Caninius ce qui se passe chez les Pictons. A cette nouvelle, il se porte au secours de Duratios. Mais Dumnacos, en apprenant l’approche de Fabius, pensa qu’il était perdu s’il devait à la fois subir l’attaque des Romains de Caninius et celle d’un ennemi du dehors, tout en ayant à surveiller et à redouter les gens de Lémonum : il se retire donc sur-le-champ, et juge qu’il ne sera en sûreté que lorsqu’il aura fait passer ses troupes de l’autre côté de la Loire, fleuve qu’on ne pouvait franchir, en raison de sa largeur, que sur un pont. Fabius n’était pas encore arrivé en vue de l’ennemi et n’avait pas encore fait sa jonction avec Caninius ; cependant, renseigné par ceux qui connaissaient le pays, il s’arrêta de préférence à l’idée que l’ennemi, poussé par la peur, gagnerait la région qu’effectivement il gagnait. En conséquence, il se dirige avec ses troupes vers le même pont et ordonne aux cavaliers de se porter en avant des légions, mais en conservant la possibilité de revenir au camp commun sans avoir à fatiguer leur monture. Ils se lancent à la poursuite de Dumnacos, conformément aux ordres reçus, surprenant son armée en marche et se jetant sur ces hommes en fuite, démoralisés, chargés de leurs bagages, ils en tuent un grand nombre et font un important butin. Après cette heureuse opération, ils rentrent au camp.
\subsection[{§ 28.}]{ \textsc{§ 28.} }
\noindent La nuit suivante, Fabius envoie en avant sa cavalerie avec mission d’accrocher l’ennemi et de retarder la marche de l’armée entière, en attendant son arrivée. Pour assurer l’exécution de ses ordres, Quintus Atius Varus, préfet de la cavalerie, homme que son courage et son intelligence mettaient hors de pair, exhorte ses troupes et, ayant rejoint la colonne ennemie, place une partie de ses escadrons sur des positions propices, tandis qu’avec les autres il engage un combat de cavalerie. Les cavaliers ennemis luttent avec une particulière audace, car ils se sentent appuyés par les fantassins : ceux-ci, en effet, d’un bout à l’autre de la colonne, font halte et se portent contre nos cavaliers, au secours des leurs. La lutte est chaude. Nos hommes, qui méprisaient un ennemi vaincu la veille et qui savaient que les légions suivaient à peu de distance, pensant qu’ils se déshonoreraient s’ils cédaient et voulant que tout le combat fût leur œuvre, luttent avec le plus grand courage contre l’infanterie ; quant à l’ennemi, fort de l’expérience de la veille, il s’imaginait qu’il ne viendrait pas d’autres troupes, et il croyait avoir trouvé une occasion d’anéantir notre cavalerie.
\subsection[{§ 29.}]{ \textsc{§ 29.} }
\noindent Comme on luttait depuis un certain temps avec un acharnement extrême, Dumnacos met ses troupes en ordre de batailler, de telle sorte qu’elles puissent protéger les cavaliers en se relayant régulièrement : soudain apparaissent, marchant en rangs serrés, les légions. A cette vue, le trouble s’empare des escadrons ennemis, la ligne des fantassins est frappée de terreur, et, tandis que la colonne des bagages est en pleine confusion, ils s’enfuient de tous côtés, en poussant de grands cris, dans une course éperdue. Nos cavaliers, qui tout à l’heure, quand l’ennemi tenait bon, s’étaient battus en braves, maintenant, dans l’ivresse de la victoire, font entendre de toutes parts une immense clameur et enveloppent l’ennemi qui se dérobe ; tant que leurs chevaux ont la force de poursuivre et leurs bras celle de frapper, ils tuent sans cesse. Plus de douze mille hommes, qu’ils eussent les armes à la main ou les eussent jetées dans la panique, sont massacrés, et l’on capture tout le convoi des bagages.
\subsection[{§ 30.}]{ \textsc{§ 30.} }
\noindent Comme on savait qu’après cette déroute le Sénon Drappès, qui, dès le début du soulèvement de la Gaule avait rassemblé de toute part des gens sans aveu, appelé les esclaves à la liberté, fait venir à lui les bannis de toutes les cités, accueilli les voleurs, et intercepté les convois de bagages et de ravitaillement des Romains, comme on savait que ce Drappès avait formé avec les restes de l’armée en fuite une troupe atteignant au plus deux mille hommes et marchait sur la Province, qu’il avait pour complice le Cadurque Luctérios qui, au début de la révolte gauloise, s’était proposé, comme on l’a vu dans le commentaire précédent, d’envahir la Province, le légat Caninius se lança à leur poursuite avec deux légions, ne voulant pas que la Province eût à souffrir ou que la peur s’emparât d’elle, et qu’ainsi nous fussions déshonorés par les brigandages d’une bande criminelle.
\subsection[{§ 31.}]{ \textsc{§ 31.} }
\noindent Caïus Fabius, avec le reste de l’armée, part chez les Carnutes et les autres peuples dont il savait que les forces avaient été très éprouvées dans le combat qu’il avait livré à Dumnacos. Il ne doutait pas, en effet, que la défaite qui venait de leur être infligée ne dût les rendre moins fiers, mais non plus que, s’il leur en laissait le temps, ils ne pussent, excités par ce même Dumnacos, relever la tête. En cette occurrence, Fabius eut la chance de pouvoir procéder, dans la soumission des cités, avec la plus heureuse promptitude. Les Carnutes, qui, bien que souvent éprouvés, n’avaient jamais parlé de paix, donnent des otages et se soumettent ; les autres cités, situées aux confins de la Gaule, touchant à l’océan, et qu’on appelle armoricaines, entraînées par l’exemple des Carnutes, remplissent sans délai, à l’approche de Fabius et de ses légions, les conditions imposées. Dumnacos, chassé de son pays, dut, errant et se cachant, aller chercher un refuge dans la partie la plus retirée de la Gaule.
\subsection[{§ 32.}]{ \textsc{§ 32.} }
\noindent Mais Drappès et avec lui Luctérios, sachant que Caninius et ses légions étaient tout proches et se pensant certainement perdus s’ils pénétraient sur le territoire de la Province avec une armée à leurs trousses, n’ayant d’ailleurs plus la possibilité de battre librement la campagne en commettant des brigandages, s’arrêtent dans le pays des Cadurques. Luctérios y avait joui autrefois, avant la défaite, d’une grande influence sur ses concitoyens, et maintenant même ses excitations à la révolte rencontraient auprès de ces Barbares un grand crédit : il occupe avec ses troupes et celles de Drappès la ville d’Uxellodunum, qui avait été dans sa clientèle ; c’était une place remarquablement défendue par la nature ; il en gagne à sa cause les habitants.
\subsection[{§ 33.}]{ \textsc{§ 33.} }
\noindent Caïus Caninius y vint tout aussitôt ; se rendant compte que de tous côtés la place était défendue par des rochers à pic, dont l’escalade, même en l’absence de tout défenseur, était difficile pour des hommes portant leurs armes, voyant, d’autre part, qu’il y avait dans la ville une grande quantité de bagages et que, si l’on essayait de fuir secrètement en les emportant, il n’était pas possible d’échapper non seulement à la cavalerie, mais aux légionnaires même, il divisa ses cohortes en trois corps et les établit dans trois camps placés sur des points très élevés ; en partant de là, il entreprit de construire peu à peu, selon ce que permettaient ses effectifs, un retranchement qui faisait le tour de la ville.
\subsection[{§ 34.}]{ \textsc{§ 34.} }
\noindent A cette vue, ceux qui étaient dans la ville, tourmentés par le tragique souvenir d’Alésia, se mirent à craindre un siège du même genre ; Luctérios, qui avait vécu ces heures-là, était le premier à rappeler qu’il fallait se préoccuper d’avoir du blé ; les chefs décident donc, à l’unanimité, de laisser là une partie des troupes et de partir eux-mêmes, avec des soldats sans bagages, pour aller chercher du blé. Le plan est approuvé, et la nuit suivante, laissant deux mille soldats dans la place, Drappès et Luctérios emmènent les autres. Ils ne restent que quelques jours absents, et prennent une grande quantité de blé sur le territoire des Cadurques, dont une partie désirait les aider en les ravitaillant, et l’autre ne pouvait les empêcher de se pourvoir ; ils font aussi, plus d’une fois, des expéditions nocturnes contre nos postes. Pour ce motif, Caninius ne se presse point d’entourer toute la place d’une ligne fortifiée il craignait qu’une fois achevée il ne lui fût impossible d’en assurer la défense, ou que, s’il établissait un grand nombre de postes, ils n’eussent que de trop faibles effectifs.
\subsection[{§ 35.}]{ \textsc{§ 35.} }
\noindent Après avoir fait une ample provision de blé, Drappès et Luctérios s’établissent à un endroit qui n’était pas à plus de dix milles de la place, et d’où ils se proposaient d’y faire passer le blé peu à peu. Ils se répartissent la tâche : Drappès reste au camp, pour en assurer la garde, avec une partie des troupes, Luctérios conduit le convoi vers la ville. Arrivé aux abords de la place, il dispose des postes de protection et, vers la dixième heure de la nuit, entreprend d’introduire le blé en prenant à travers bois par d’étroits chemins. Mais les veilleurs du camp entendent le bruit de cette troupe en marche, on envoie des éclaireurs qui rapportent ce qui se passe, et Caninius, promptement, avec les cohortes qui étaient sous les armes dans les postes voisins, charge les pourvoyeurs aux premières lueurs du jour. Ceux-ci, surpris, prennent peur et s’enfuient de tous côtés vers les troupes de protection dès que les nôtres aperçoivent ces dernières, la vue d’hommes en armes accroît encore leur ardeur, et ils ne font pas un seul prisonnier. Luctérios réussit à s’enfuir avec une poignée d’hommes, mais il ne rentre pas au camp.
\subsection[{§ 36.}]{ \textsc{§ 36.} }
\noindent Après cette heureuse opération, Caninius apprend par des prisonniers qu’une partie des troupes est restée avec Drappès dans un camp qui n’est pas à plus de douze milles. S'étant assuré du fait par un grand nombre de témoignages, il voyait bien que, puisque l’un des deux chefs avait été mis en fuite, il serait facile de surprendre et d’écraser ceux qui restaient ; mais il n’ignorait pas non plus que ce serait une grande chance si aucun survivant n’était rentré au camp et n’avait apporté à Drappès la nouvelle du désastre ; néanmoins, comme il ne voyait aucun risque à tenter la chance, il envoie en avant vers le camp ennemi toute la cavalerie et les fantassins Germains, qui étaient d’une agilité extrême ; lui-même, après avoir réparti une légion dans les trois camps, emmène l’autre en tenue de combat. Arrivé à peu de distance des ennemis, les éclaireurs dont il s’était fait précéder lui apprennent que, selon l’usage ordinaire des Barbares, ils ont laissé les hauteurs pour établir leur camp sur les bords de la rivière ; les Germains et les cavaliers n’en sont pas moins tombés sur eux à l’improviste et ont engagé le combat. Fort de ces renseignements, il y mène sa légion en armes et rangée pour la bataille. Les troupes, à un signal donné, surgissant de toutes parts, occupent les hauteurs. Là-dessus, les Germains et les cavaliers, à la vue des enseignes de la légion, redoublent d’ardeur. Sans désemparer, les cohortes, de tous côtés, se précipitent : tous les ennemis sont tués ou pris, et l’on fait un grand butin. Drappès même est fait prisonnier au cours de l’action.
\subsection[{§ 37.}]{ \textsc{§ 37.} }
\noindent Caninius, après cette affaire si heureusement menée, sans qu’il eût presque aucun blessé, retourne assiéger les gens d’Uxellodunum et, débarrassé maintenant de l’ennemi extérieur, dont la crainte l’avait jusque-là empêché de disperser ses forces dans des postes et d’investir complètement la place, il ordonne qu’on travaille partout à la fortification. Laïus Fabius arrive le lendemain avec ses troupes, et se charge d’un secteur d’investissement.
\subsection[{§ 38.}]{ \textsc{§ 38.} }
\noindent Cependant César laisse son questeur Marcus Antonius avec quinze cohortes chez les Bellovaques, pour que les Belges ne puissent pas une fois encore former des projets de révolte. Il va lui-même chez les autres peuples, se fait livrer de nouveaux otages, ramène des idées saines dans les esprits qui tous étaient en proie à la peur. Arrivé chez les Carnutes, dont César a raconté dans le précédent commentaire comment la guerre avait pris naissance dans leur citée, voyant que leurs alarmes étaient particulièrement vives, parce qu’ils avaient conscience de la gravité de leur faute, afin d’en libérer plus vite l’ensemble de la population, il demande qu’on lui livre, pour le châtier, Gutuater, principal coupable et auteur responsable de la guerre. Bien que le personnage ne se fiât plus même à ses propres concitoyens, néanmoins, chacun s’appliquant à le rechercher, on l’amène promptement au camp. César, malgré sa naturelle clémence, est contraint de le livrer au supplice par les soldats accourus en foule : ils mettaient à son compte tous les dangers courus, tous les maux soufferts au cours de la guerre, et il fallut qu’il fût d’abord frappé de verges jusqu’à perdre connaissance, avant que la hache l’achevât.
\subsection[{§ 39.}]{ \textsc{§ 39.} }
\noindent César était chez les Carnutes quand il reçoit coup sur coup plusieurs lettres de Caninius l’informant de ce qui avait été fait concernant Drappès et Luctérios, et de la résistance à laquelle s’obstinaient les habitants d’Uxellodunum. Bien que leur petit nombre lui parût méprisable, il estimait cependant qu’il fallait châtier sévèrement leur opiniâtreté, afin que l’ensemble des Gaulois n’en vînt pas à s’imaginer que ce qui leur avait manqué pour tenir tête aux Romains, ce n’était pas la force, mais la constance, et pour éviter que, se réglant sur cet exemple, les autres cités ne cherchassent à se rendre libres en profitant de positions avantageuses : car toute la Gaule, il ne l’ignorait pas, savait qu’il ne lui restait plus qu’un été à passer dans sa Province, et s’ils pouvaient tenir pendant ce temps-là, ils n’auraient ensuite plus rien à craindre. Il laissa donc son légat Quintus Calénus, à la tête de deux légions, avec ordre de le suivre à étapes normales ; quant à lui, avec toute la cavalerie, il va rejoindre Caninius à marches forcées.
\subsection[{§ 40.}]{ \textsc{§ 40.} }
\noindent Son arrivée à Uxellodunum surprit tout le monde ; quand il vit que les travaux de fortification entouraient complètement la place, il jugea qu’à aucun prix on ne pouvait lever le sièges ; et comme des déserteurs lui avaient appris que les assiégés avaient d’abondantes provisions de blé, il voulut essayer de les priver d’eau. Une rivière coulait au milieu d’une vallée profonde qui entourait presque complètement la montagne sur laquelle était juché Uxellodunum. Détourner la rivière, le terrain ne s’y prêtait pas : elle coulait, en effet, au pied de la montagne dans la partie la plus basse, si bien qu’en aucun endroit on ne pouvait creuser des fossés de dérivation. Mais les assiégés n’y avaient accès que par une descente difficile et abrupte : pour peu que les nôtres en défendissent l’abord, ils ne pouvaient ni approcher de la rivière, ni remonter, pour rentrer, la pente raide, sans s’exposer aux coups et risquer la mort. S'étant rendu compte de ces difficultés que rencontrait l’ennemi, César posta des archers et des frondeurs, plaça même de l’artillerie sur certains points en face des pentes les plus aisées, et ainsi il empêchait les assiégés d’aller puiser l’eau de la rivière.
\subsection[{§ 41.}]{ \textsc{§ 41.} }
\noindent Alors ils se mirent à venir tous chercher de l’eau en un seul endroit, au pied même du mur de la ville, où jaillissait une source abondante, du côté que laissait libre, sur une longueur d’environ trois cents pieds, le circuit de la rivière. Chacun souhaitait qu’il fût possible d’interdire aux assiégés l’accès de cette source, mais César seul en voyait le moyen il entreprit de faire, face à la source, pousser des mantelets le long de la pente et construire un terrassement au prix d’un dur travail et de continuelles escarmouches. Les assiégés, en effet, descendant au pas de course de leur position qui dominait la nôtre, combattent de loin sans avoir rien à craindre et blessent un grand nombre de nos hommes qui s’obstinent à avancer ; pourtant, cela n’empêche pas nos soldats de faire progresser les mantelets et, à force de fatigue et de travaux, de vaincre les difficultés du terrain. En même temps, ils creusent des conduits souterrains dans la direction des filets d’eau et de la source où ceux-ci aboutissaient ; ce genre de travail pouvait être accompli sans aucun danger et sans que l’ennemi le soupçonnât. On construit un terrassement de soixante pieds de haut, on y installe une tour de dix étages, qui sans doute n’atteignait pas la hauteur des murs (il n’était pas d’ouvrage qui permît d’obtenir ce résultat), mais qui, du moins, dominait l’endroit où naissait la source. Du haut de cette tour, de l’artillerie lançait des projectiles sur le point par où on l’abordait, et les assiégés ne pouvaient venir chercher de l’eau sans risquer leur vie si bien que non seulement le bétail et les bêtes de somme, mais encore la nombreuse population de la ville souffraient de la soifs.
\subsection[{§ 42.}]{ \textsc{§ 42.} }
\noindent Une aussi grave menace alarme les assiégés, qui, remplissant des tonneaux avec du suif de la poix et de minces lattes de bois, les font rouler en flammes sur nos ouvrages. Dans le même temps, ils engagent un combat des plus vifs, afin que les Romains, occupés à une lutte dangereuse, ne puissent songer à éteindre le feu. Un violent incendie éclate brusquement au milieu de nos ouvrages. En effet, tout ce qui avait été lancé sur la pente, étant arrêté par les mantelets et par la terrasse, mettait le feu à ces obstacles mêmes. Cependant nos soldats, malgré les difficultés que leur créaient un genre de combat si périlleux et le désavantage de la position, faisaient face à tout avec le plus grand courage. L'action, en effet, se déroulait sur une hauteur, à la vue de notre armée, et des deux côtés on poussait de grands cris. Aussi chacun s’exposait-il aux traits des ennemis et aux flammes avec d’autant plus d’audace qu’il avait plus de réputation, voyant là un moyen que sa valeur fût mieux connue et mieux attestée.
\subsection[{§ 43.}]{ \textsc{§ 43.} }
\noindent César, voyant qu’un grand nombre de ses hommes étaient blessés, ordonne aux cohortes de monter de tous les côtés à l’assaut de la montagne et de pousser partout des clameurs pour faire croire qu’elles sont en train d’occuper les remparts. Ainsi fait-on, et les assiégés, fort alarmés, car ils ne savaient que supposer sur ce qui se passait ailleurs, rappellent les soldats qui assaillaient nos ouvrages et les dispersent sur la muraille. Ainsi le combat prend fin et nos hommes ont vite fait ou d’éteindre l’incendie ou de faire la part du feu. La résistance des assiégés se prolongeait, opiniâtre, et bien qu’un grand nombre d’entre eux fussent morts de soif, ils ne cédaient pas à la fin, les ruisselets qui alimentaient la source furent coupés par nos canaux souterrains et détournés de leur cours. Alors la source, qui ne tarissait jamais, fut brusquement à sec, et les assiégés se sentirent du coup si irrémédiablement perdus qu’ils virent là l’effet non de l’industrie humaine, mais de la volonté divine. Aussi, cédant à la nécessité, ils se rendirent.
\subsection[{§ 44.}]{ \textsc{§ 44.} }
\noindent César savait que sa bonté était connue de tous et il n’avait pas à craindre qu’on n’expliquât par la cruauté de son caractère un acte de rigueur ; comme, d’autre part, il ne voyait pas l’achèvement de ses desseins, si d’autres, sur divers points de la Gaule, se lançaient dans de semblables entreprises, il estima qu’il fallait les en détourner par un châtiment exemplaire. En conséquence, il fit couper les mains à tous ceux qui avaient porté les armes et leur accorda la vie sauve, pour qu’on sût mieux comment il punissait les rebelles. Drappès, qui, je l’ai dit, avait été fait prisonnier par Caninius, soit qu’il ne pût supporter l’humiliation d’être dans les fers, soit qu’il redoutât les tourments d’un cruel supplice, s’abstint pendant quelques jours de nourriture et mourut de faim. Dans le même temps Luctérios, dont j’ai rapporté qu’il avait pu s’enfuir de la bataille, était venu se mettre entre les mains de l’Arverne Epasnactos : il changeait, en effet, souvent de résidence, et ne se confiait pas longtemps au même hôte, car, sachant combien César devait le haïr, il estimait dangereux tout séjour de quelque durée : l’Arverne Epasnactos, qui était un grand ami du peuple Romain, sans aucune hésitation le fit charger de chaînes et l’amena à César.
\subsection[{§ 45.}]{ \textsc{§ 45.} }
\noindent Cependant Labiénus, chez les Trévires, livre un combat de cavalerie heureux : il leur tue beaucoup de monde, ainsi qu’aux Germains, qui ne refusaient à aucun peuple de secours contre les Romains, prend vivants leurs chefs, et parmi eux l’Héduen Suros, homme dont le courage était réputé et la naissance illustre, et qui, seul parmi les Héduens, n’avait pas encore déposé les armes.
\subsection[{§ 46.}]{ \textsc{§ 46.} }
\noindent A cette nouvelle, César, qui voyait que partout en Gaule la situation lui était favorable et jugeait que la Gaule proprement dite avait été, par les campagnes des années précédentes, complètement vaincue et soumise, qui, d’autre part, n’était jamais allé lui-même en Aquitaine, mais y avait seulement remporté, grâce à Publius Crassus, une victoire partielle, se mit en route, à la tête de deux légions, pour cette partie de la Gaule, avec l’intention d’y employer la fin de la saison. Cette expédition, comme les autres, fut menée rapidement et avec bonheur ; toutes les cités d’Aquitaine lui envoyèrent des députés et lui donnèrent des otages. Après cela, il partit pour Narbonne avec une escorte de cavaliers, laissant à ses légats le soin de mettre l’armée en quartiers d’hiver : il établit quatre légions chez les Belges, sous les ordres des légats Marcus Antonius, Caïus Trébonius et Publius Vatinius ; deux furent conduites chez les Héduens, qu’il savait posséder l’influence la plus considérable sur toute la Gaule ; deux autres, chez les Turons, à la frontière des Carnutes, devaient maintenir dans l’obéissance toute cette région jusqu’à l’océan ; les deux dernières furent placées chez les Lémovices, non loin des Arvernes, afin qu’aucune partie de la Gaule ne fût vide de troupes. Il ne resta que quelques jours dans la Province : il parcourut rapidement tous les centres d’audience, jugea les conflits politiques, récompensa les services rendus, il lui était, en effet, très facile de se rendre compte des sentiments de chacun envers Rome pendant le soulèvement général de la Gaule, auquel la fidélité et les secours de ladite Province lui avaient permis de tenir tête. Quand il eut achevé, il revint auprès de ses légions en Belgique et hiverna à Némétocenna.
\subsection[{§ 47.}]{ \textsc{§ 47.} }
\noindent Là, il apprend que Commios l’Atrébate a livré bataille à sa cavalerie. Antoine était arrivé dans ses quartiers d’hiver, et les Atrébates étaient tranquilles ; mais Commios, depuis la blessure dont j’ai parlé plus haut, était sans cesse à la disposition de ses concitoyens pour toute espèce de troubles, prêt à fournir à ceux qui voulaient la guerre un agitateur et un chef tandis que sa cité obéissait aux Romains, il se livrait, avec sa cavalerie, à des actes de brigandage dont il vivait, lui et sa bande, infestant les routes et interceptant nombre de convois destinés aux quartiers d’hiver des Romains.
\subsection[{§ 48.}]{ \textsc{§ 48.} }
\noindent Antoine avait sous ses ordres comme préfet de la cavalerie Caïus Volusénus Quadratus qui devait passer l’hiver avec lui. Il l’envoie à la poursuite des cavaliers ennemis. Volusénus, outre qu’il était un homme d’un rare courage, détestait Commios : aussi obéit-il avec joie. Ayant organisé des embuscades, il attaquait fréquemment ses cavaliers, et toujours avec succès. A la fin, au cours d’un engagement plus vif que les autres, Volusénus, emporté par le désir de s’emparer de la personne de Commios, s’était acharné à le poursuivre avec un petit groupe, et lui, fuyant à toute bride, avait entraîné Volusénus à bonne distance, quand soudain Commios, qui le haïssait, fait appel à l’honneur de ses compagnons, leur demande de le secourir, de ne pas laisser sans vengeance les blessures qu’il doit à la fourberie de cet homme, et, tournant bride, il se sépare des autres, audacieusement, pour se précipiter sur le préfet. Tous ses cavaliers l’imitent, font faire demi-tour aux nôtres, qui n’étaient pas en force, et les poursuivent. Commios éperonne furieusement son cheval, le pousse contre celui de Quadratus, et, se jetant sur son ennemi, la lance en avant, avec une grande violence, il lui transperce la cuisse. Quand ils voient leur préfet touché, les nôtres n’hésitent pas : ils s’arrêtent de fuir et, tournant leurs chevaux contre l’ennemi, le repoussent. Alors un grand nombre d’ennemis, bousculés par la violence de notre charge, sont blessés, et les uns sont foulés aux pieds des chevaux dans la poursuite, tandis que les autres sont faits prisonniers ; leur chef, grâce à la rapidité de sa monture, évita ce malheur ; ainsi, ce fut une victoire mais le préfet, grièvement atteint par Commios et paraissant en danger de mort, fut ramené au camp. Cependant Commios, soit parce qu’il avait satisfait sa rancune, soit parce qu’il avait perdu la plupart des siens, envoie des députés à Antoine et promet, sous caution d’otages, d’avoir tel séjour qu’il prescrira, d’exécuter ce qu’il commandera il ne demande qu’une chose, c’est qu’on ménage sa frayeur en lui évitant de paraître devant un Romain. Antoine, jugeant que sa demande était inspirée par une crainte légitime, y fit droit et reçut ses otages.\par
\par
Je sais que César a composé un commentaire pour chaque année ; je n’ai pas cru devoir faire de même, parce que l’année suivante, celle du consulat de Lucius Paulus et de Caïus Marcellus, n’offre aucune opération importante en Gaule. Toutefois, pour ne pas laisser ignorer où furent pendant ce temps César et son armée, j’ai résolu d’écrire quelques pages que je joindrai à ce commentaire.
\subsection[{§ 49.}]{ \textsc{§ 49.} }
\noindent César, en hivernant en Belgique n’avait d’autre but que de maintenir les cités dans notre alliance, d’éviter de donner à aucune d’elles espoir ou prétexte de guerre. Rien, en effet, ne lui paraissait moins souhaitable que de se voir contraint à une guerre, au moment de sa sortie de charge, et de laisser derrière lui, lorsqu’il devrait emmener son armée, une guerre où toute la Gaule, n’ayant rien à craindre pour l’instant, se jetterait volontiers. Aussi, en traitant les cités avec honneur, en récompensant très largement les principaux citoyens, en évitant d’imposer aucune charge nouvelle, il maintint aisément la paix dans la Gaule que tant de défaites avaient épuisée et à qui il rendait l’obéissance plus douce.
\subsection[{§ 50.}]{ \textsc{§ 50.} }
\noindent Il partit contre son habitude, l’hiver fini, et en forçant les étapes, pour l’Italie, afin de parler aux municipes et aux colonies à qui il avait recommandé son questeur Marcus Antonius, candidat au sacerdoce. Il l’appuyait, en effet, de tout son crédit, parce qu’il était heureux de servir un ami intime qu’il venait d’autoriser à partir en avant pour faire acte de candidat, mais aussi parce qu’il désirait vivement combattre les intrigues d’une minorité puissante qui voulait, en faisant échouer Antoine, ruiner le crédit de César à sa sortie de charge. Bien qu’il eût appris en chemin, avant d’atteindre l’Italie, qu’Antoine avait été nommé augure, il estima cependant qu’il n’avait pas moins de raison de visiter les municipes et les colonies, afin de les remercier de leurs votes nombreux et empressés pour Antoine, et aussi pour recommander sa propre candidature aux élections de l’année suivante : ses adversaires, en effet, triomphaient insolemment du succès de Lucius Lentulus et de Caïus Marcellus qui, nommés consuls, se proposaient de dépouiller César de toute charge, de toute dignité, et de l’échec de Servius Galba qui, bien qu’il fût beaucoup plus populaire et eût obtenu beaucoup plus de voix, avait été frustré du consulat parce qu’il était l’ami de César et avait été ses légats.
\subsection[{§ 51.}]{ \textsc{§ 51.} }
\noindent L'arrivée de César fut accueillie par tous les municipes et colonies avec des témoignages incroyables de respect et d’affection. C'était en effet, la première fois qu’il y venait depuis le grand soulèvement général de la Gaule. On ne négligeait rien de tout ce qui pouvait être imaginé pour décorer les portes, les chemins, tous les endroits par où César devait passer. La population entière, avec les enfants, se portait à sa rencontre, on immolait partout des victimes, les places et les temples, où l’on avait dressé des tables, étaient pris d’assaut : on pouvait goûter à l’avance les joies d’un triomphe impatiemment attendus. Telle était la magnificence déployée par les riches, et l’enthousiasme que montraient les pauvres.
\subsection[{§ 52.}]{ \textsc{§ 52.} }
\noindent Après avoir parcouru toutes les parties de la Gaule cisalpine, César revint avec la plus grande promptitude auprès de ses troupes à Némétocenna : ayant envoyé aux légions, dans tous les quartiers d’hiver, l’ordre de faire mouvement vers le territoire des Trévires, il y alla lui-même et y passa son armée en revue. Il donna à Titus Labiénus le commandement de la Cisalpine, afin que sa candidature au consulat fût bien soutenue dans ce pays. Quant à lui, il ne se déplaçait qu’autant qu’il jugeait utile, pour l’hygiène des troupes, de changer de cantonnement. Des bruits nombreux lui parvenaient touchant les intrigues de ses ennemis auprès de Labiénus, et il était informé que, sous l’inspiration de quelques-uns, on cherchait à provoquer une intervention du Sénat pour le dépouiller d’une partie de ses troupes ; néanmoins, on ne put rien lui faire croire sur Labiénus ni rien lui faire entreprendre contre l’autorité du Sénat. Il pensait, en effet, que si les sénateurs votaient librement il obtiendrait aisément justice. Laïus Curion, tribun de la plèbe, qui s’était fait le défenseur de César et de sa dignité menacée, avait plusieurs fois pris devant le Sénat l’engagement suivant si la puissance militaire de César inquiétait quelqu’un, et puisque, d’autre part, le pouvoir absolu et les armements de Pompée éveillaient chez les citoyens des craintes qui n’étaient pas médiocres, il proposait que l’un et l’autre désarmât et licenciât ses troupes du coup, la république recouvrerait la liberté et l’indépendance. Il ne se borna point à cet engagement, mais il prit même l’initiative de provoquer un vote du Sénat ; les consuls et les amis de Pompée s’y opposèrent, et, sur cette manœuvre dilatoire, l’assemblée se sépara.
\subsection[{§ 53.}]{ \textsc{§ 53.} }
\noindent On avait là un important témoignage des sentiments du Sénat tout entier, et qui corroborait la leçon d’un incident antérieur. Marcus Marcellus, l’année précédente, cherchant à abattre César, avait, en violation d’une loi de Pompée et de Crassus, porté à l’ordre du jour du Sénat, avant le temps, la question des provinces du proconsul ; comme, après discussion, il mettait sa proposition aux voix, Marcellus, qui attendait de ses attaques contre César la satisfaction de toutes ses ambitions politiques, avait vu le Sénat se ranger en masse à l’avis contraire. Mais ces échecs ne décourageaient pas les ennemis de César : ils les avertissaient seulement d’avoir à trouver des moyens de pression plus énergiques, grâce auxquels ils pourraient forcer le Sénat d’approuver ce qu’ils étaient seuls à vouloir.
\subsection[{§ 54.}]{ \textsc{§ 54.} }
\noindent Ensuite un sénatus-consulte décide que Cnéus Pompée et Caïus César devront envoyer chacun une légion pour la guerre des Parthes ; mais il est bien clair qu’on en prend deux au même. En effet, Cnéus Pompée donna, comme provenant de son contingent, la première légion, qu’il avait envoyée à César après l’avoir levée dans la province de César lui-même. Celui-ci pourtant, bien que les intentions de ses adversaires ne fissent aucun doute, renvoya la légion à Pompée et donna pour son compte, en exécution du sénatus-consulte, la quinzième, qui était dans la Gaule citérieures. A sa place, il envoie en Italie la treizième, pour tenir garnison dans les postes que celle-là évacuait. Il assigne, d’autre part, des quartiers d’hiver à son armée : Laïus Trébonius est placé en Belgique avec quatre légions ; Laïus Fabius est envoyé avec les mêmes effectifs chez les Héduens. Il estimait, en effet, que le meilleur moyen d’assurer la tranquillité de la Gaule, c’était de contenir par la présence des troupes les Belges, qui étaient les plus braves, et les Héduens, qui avaient le plus d’influence. Il partit ensuite pour l’Italie.
\subsection[{§ 55.}]{ \textsc{§ 55.} }
\noindent À son arrivée, il apprend que les deux légions qu’il avait renvoyées et qui, d’après le sénatus-consulte, étaient destinées à la guerre des Parthes, le consul Caïus Marcellus les a remises à Pompée, et qu’on les a gardées en Italie. Après cela, personne ne pouvait plus douter de ce qui se tramait contre César ; celui-ci pourtant résolut de tout souffrir, tant qu’il lui resterait quelque espoir d’obtenir une solution légale du conflit au lieu d’avoir recours aux armes. Il s’efforça…
 


% at least one empty page at end (for booklet couv)
\ifbooklet
  \pagestyle{empty}
  \clearpage
  % 2 empty pages maybe needed for 4e cover
  \ifnum\modulo{\value{page}}{4}=0 \hbox{}\newpage\hbox{}\newpage\fi
  \ifnum\modulo{\value{page}}{4}=1 \hbox{}\newpage\hbox{}\newpage\fi


  \hbox{}\newpage
  \ifodd\value{page}\hbox{}\newpage\fi
  {\centering\color{rubric}\bfseries\noindent\large
    Hurlus ? Qu’est-ce.\par
    \bigskip
  }
  \noindent Des bouquinistes électroniques, pour du texte libre à participation libre,
  téléchargeable gratuitement sur \href{https://hurlus.fr}{\dotuline{hurlus.fr}}.\par
  \bigskip
  \noindent Cette brochure a été produite par des éditeurs bénévoles.
  Elle n’est pas faîte pour être possédée, mais pour être lue, et puis donnée.
  Que circule le texte !
  En page de garde, on peut ajouter une date, un lieu, un nom ; pour suivre le voyage des idées.
  \par

  Ce texte a été choisi parce qu’une personne l’a aimé,
  ou haï, elle a en tous cas pensé qu’il partipait à la formation de notre présent ;
  sans le souci de plaire, vendre, ou militer pour une cause.
  \par

  L’édition électronique est soigneuse, tant sur la technique
  que sur l’établissement du texte ; mais sans aucune prétention scolaire, au contraire.
  Le but est de s’adresser à tous, sans distinction de science ou de diplôme.
  Au plus direct ! (possible)
  \par

  Cet exemplaire en papier a été tiré sur une imprimante personnelle
   ou une photocopieuse. Tout le monde peut le faire.
  Il suffit de
  télécharger un fichier sur \href{https://hurlus.fr}{\dotuline{hurlus.fr}},
  d’imprimer, et agrafer ; puis de lire et donner.\par

  \bigskip

  \noindent PS : Les hurlus furent aussi des rebelles protestants qui cassaient les statues dans les églises catholiques. En 1566 démarra la révolte des gueux dans le pays de Lille. L’insurrection enflamma la région jusqu’à Anvers où les gueux de mer bloquèrent les bateaux espagnols.
  Ce fut une rare guerre de libération dont naquit un pays toujours libre : les Pays-Bas.
  En plat pays francophone, par contre, restèrent des bandes de huguenots, les hurlus, progressivement réprimés par la très catholique Espagne.
  Cette mémoire d’une défaite est éteinte, rallumons-la. Sortons les livres du culte universitaire, cherchons les idoles de l’époque, pour les briser.
\fi

\ifdev % autotext in dev mode
\fontname\font — \textsc{Les règles du jeu}\par
(\hyperref[utopie]{\underline{Lien}})\par
\noindent \initialiv{A}{lors là}\blindtext\par
\noindent \initialiv{À}{ la bonheur des dames}\blindtext\par
\noindent \initialiv{É}{tonnez-le}\blindtext\par
\noindent \initialiv{Q}{ualitativement}\blindtext\par
\noindent \initialiv{V}{aloriser}\blindtext\par
\Blindtext
\phantomsection
\label{utopie}
\Blinddocument
\fi
\end{document}
