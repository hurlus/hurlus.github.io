%%%%%%%%%%%%%%%%%%%%%%%%%%%%%%%%%
% LaTeX model https://hurlus.fr %
%%%%%%%%%%%%%%%%%%%%%%%%%%%%%%%%%

% Needed before document class
\RequirePackage{pdftexcmds} % needed for tests expressions
\RequirePackage{fix-cm} % correct units

% Define mode
\def\mode{a4}

\newif\ifaiv % a4
\newif\ifav % a5
\newif\ifbooklet % booklet
\newif\ifcover % cover for booklet

\ifnum \strcmp{\mode}{cover}=0
  \covertrue
\else\ifnum \strcmp{\mode}{booklet}=0
  \booklettrue
\else\ifnum \strcmp{\mode}{a5}=0
  \avtrue
\else
  \aivtrue
\fi\fi\fi

\ifbooklet % do not enclose with {}
  \documentclass[french,twoside]{book} % ,notitlepage
  \usepackage[%
    papersize={105mm, 297mm},
    inner=12mm,
    outer=12mm,
    top=20mm,
    bottom=15mm,
    marginparsep=0pt,
  ]{geometry}
  \usepackage[fontsize=9.5pt]{scrextend} % for Roboto
\else\ifav
  \documentclass[french,twoside]{book} % ,notitlepage
  \usepackage[%
    a5paper,
    inner=25mm,
    outer=15mm,
    top=15mm,
    bottom=15mm,
    marginparsep=0pt,
  ]{geometry}
  \usepackage[fontsize=12pt]{scrextend}
\else% A4 2 cols
  \documentclass[twocolumn]{report}
  \usepackage[%
    a4paper,
    inner=15mm,
    outer=10mm,
    top=25mm,
    bottom=18mm,
    marginparsep=0pt,
  ]{geometry}
  \setlength{\columnsep}{20mm}
  \usepackage[fontsize=9.5pt]{scrextend}
\fi\fi

%%%%%%%%%%%%%%
% Alignments %
%%%%%%%%%%%%%%
% before teinte macros

\setlength{\arrayrulewidth}{0.2pt}
\setlength{\columnseprule}{\arrayrulewidth} % twocol
\setlength{\parskip}{0pt} % classical para with no margin
\setlength{\parindent}{1.5em}

%%%%%%%%%%
% Colors %
%%%%%%%%%%
% before Teinte macros

\usepackage[dvipsnames]{xcolor}
\definecolor{rubric}{HTML}{800000} % the tonic 0c71c3
\def\columnseprulecolor{\color{rubric}}
\colorlet{borderline}{rubric!30!} % definecolor need exact code
\definecolor{shadecolor}{gray}{0.95}
\definecolor{bghi}{gray}{0.5}

%%%%%%%%%%%%%%%%%
% Teinte macros %
%%%%%%%%%%%%%%%%%
%%%%%%%%%%%%%%%%%%%%%%%%%%%%%%%%%%%%%%%%%%%%%%%%%%%
% <TEI> generic (LaTeX names generated by Teinte) %
%%%%%%%%%%%%%%%%%%%%%%%%%%%%%%%%%%%%%%%%%%%%%%%%%%%
% This template is inserted in a specific design
% It is XeLaTeX and otf fonts

\makeatletter % <@@@


\usepackage{blindtext} % generate text for testing
\usepackage[strict]{changepage} % for modulo 4
\usepackage{contour} % rounding words
\usepackage[nodayofweek]{datetime}
% \usepackage{DejaVuSans} % seems buggy for sffont font for symbols
\usepackage{enumitem} % <list>
\usepackage{etoolbox} % patch commands
\usepackage{fancyvrb}
\usepackage{fancyhdr}
\usepackage{float}
\usepackage{fontspec} % XeLaTeX mandatory for fonts
\usepackage{footnote} % used to capture notes in minipage (ex: quote)
\usepackage{framed} % bordering correct with footnote hack
\usepackage{graphicx}
\usepackage{lettrine} % drop caps
\usepackage{lipsum} % generate text for testing
\usepackage[framemethod=tikz,]{mdframed} % maybe used for frame with footnotes inside
\usepackage{pdftexcmds} % needed for tests expressions
\usepackage{polyglossia} % non-break space french punct, bug Warning: "Failed to patch part"
\usepackage[%
  indentfirst=false,
  vskip=1em,
  noorphanfirst=true,
  noorphanafter=true,
  leftmargin=\parindent,
  rightmargin=0pt,
]{quoting}
\usepackage{ragged2e}
\usepackage{setspace} % \setstretch for <quote>
\usepackage{tabularx} % <table>
\usepackage[explicit]{titlesec} % wear titles, !NO implicit
\usepackage{tikz} % ornaments
\usepackage{tocloft} % styling tocs
\usepackage[fit]{truncate} % used im runing titles
\usepackage{unicode-math}
\usepackage[normalem]{ulem} % breakable \uline, normalem is absolutely necessary to keep \emph
\usepackage{verse} % <l>
\usepackage{xcolor} % named colors
\usepackage{xparse} % @ifundefined
\XeTeXdefaultencoding "iso-8859-1" % bad encoding of xstring
\usepackage{xstring} % string tests
\XeTeXdefaultencoding "utf-8"
\PassOptionsToPackage{hyphens}{url} % before hyperref, which load url package

% TOTEST
% \usepackage{hypcap} % links in caption ?
% \usepackage{marginnote}
% TESTED
% \usepackage{background} % doesn’t work with xetek
% \usepackage{bookmark} % prefers the hyperref hack \phantomsection
% \usepackage[color, leftbars]{changebar} % 2 cols doc, impossible to keep bar left
% \usepackage[utf8x]{inputenc} % inputenc package ignored with utf8 based engines
% \usepackage[sfdefault,medium]{inter} % no small caps
% \usepackage{firamath} % choose firasans instead, firamath unavailable in Ubuntu 21-04
% \usepackage{flushend} % bad for last notes, supposed flush end of columns
% \usepackage[stable]{footmisc} % BAD for complex notes https://texfaq.org/FAQ-ftnsect
% \usepackage{helvet} % not for XeLaTeX
% \usepackage{multicol} % not compatible with too much packages (longtable, framed, memoir…)
% \usepackage[default,oldstyle,scale=0.95]{opensans} % no small caps
% \usepackage{sectsty} % \chapterfont OBSOLETE
% \usepackage{soul} % \ul for underline, OBSOLETE with XeTeX
% \usepackage[breakable]{tcolorbox} % text styling gone, footnote hack not kept with breakable


% Metadata inserted by a program, from the TEI source, for title page and runing heads
\title{\textbf{ Essai sur les privilèges }}
\date{1788}
\author{Siéyès, Emmanuel Joseph (1748-1836)}
\def\elbibl{Siéyès, Emmanuel Joseph (1748-1836). 1788. \emph{Essai sur les privilèges}}
\def\elsource{Abbé Emmanuel Joseph Siéyès, {\itshape Essai sur les privilèges} [1788], in {\itshape Qu’est-ce que le Tiers-État, précédé de l’Essai sur les privilèges}, nouvelle édition augmentée de 23 notes par l’abbé Morellet, Paris, Alexandre Corréard, 1822, 248 p. PDF : \href{http://classiques.uqac.ca/classiques/sieyes_emmanuel_joseph/qu_est_ce_que_tiers_etat/que_est_de_que_le_tiers_etat.pdf}{\dotuline{Classiques des sciences sociales}}\footnote{\href{http://classiques.uqac.ca/classiques/sieyes_emmanuel_joseph/qu_est_ce_que_tiers_etat/que_est_de_que_le_tiers_etat.pdf}{\url{http://classiques.uqac.ca/classiques/sieyes_emmanuel_joseph/qu_est_ce_que_tiers_etat/que_est_de_que_le_tiers_etat.pdf}}}.X\phantomsection
\label{essai\_privileges}}

% Default metas
\newcommand{\colorprovide}[2]{\@ifundefinedcolor{#1}{\colorlet{#1}{#2}}{}}
\colorprovide{rubric}{red}
\colorprovide{silver}{lightgray}
\@ifundefined{syms}{\newfontfamily\syms{DejaVu Sans}}{}
\newif\ifdev
\@ifundefined{elbibl}{% No meta defined, maybe dev mode
  \newcommand{\elbibl}{Titre court ?}
  \newcommand{\elbook}{Titre du livre source ?}
  \newcommand{\elabstract}{Résumé\par}
  \newcommand{\elurl}{http://oeuvres.github.io/elbook/2}
  \author{Éric Lœchien}
  \title{Un titre de test assez long pour vérifier le comportement d’une maquette}
  \date{1566}
  \devtrue
}{}
\let\eltitle\@title
\let\elauthor\@author
\let\eldate\@date


\defaultfontfeatures{
  % Mapping=tex-text, % no effect seen
  Scale=MatchLowercase,
  Ligatures={TeX,Common},
}


% generic typo commands
\newcommand{\astermono}{\medskip\centerline{\color{rubric}\large\selectfont{\syms ✻}}\medskip\par}%
\newcommand{\astertri}{\medskip\par\centerline{\color{rubric}\large\selectfont{\syms ✻\,✻\,✻}}\medskip\par}%
\newcommand{\asterism}{\bigskip\par\noindent\parbox{\linewidth}{\centering\color{rubric}\large{\syms ✻}\\{\syms ✻}\hskip 0.75em{\syms ✻}}\bigskip\par}%

% lists
\newlength{\listmod}
\setlength{\listmod}{\parindent}
\setlist{
  itemindent=!,
  listparindent=\listmod,
  labelsep=0.2\listmod,
  parsep=0pt,
  % topsep=0.2em, % default topsep is best
}
\setlist[itemize]{
  label=—,
  leftmargin=0pt,
  labelindent=1.2em,
  labelwidth=0pt,
}
\setlist[enumerate]{
  label={\bf\color{rubric}\arabic*.},
  labelindent=0.8\listmod,
  leftmargin=\listmod,
  labelwidth=0pt,
}
\newlist{listalpha}{enumerate}{1}
\setlist[listalpha]{
  label={\bf\color{rubric}\alph*.},
  leftmargin=0pt,
  labelindent=0.8\listmod,
  labelwidth=0pt,
}
\newcommand{\listhead}[1]{\hspace{-1\listmod}\emph{#1}}

\renewcommand{\hrulefill}{%
  \leavevmode\leaders\hrule height 0.2pt\hfill\kern\z@}

% General typo
\DeclareTextFontCommand{\textlarge}{\large}
\DeclareTextFontCommand{\textsmall}{\small}

% commands, inlines
\newcommand{\anchor}[1]{\Hy@raisedlink{\hypertarget{#1}{}}} % link to top of an anchor (not baseline)
\newcommand\abbr[1]{#1}
\newcommand{\autour}[1]{\tikz[baseline=(X.base)]\node [draw=rubric,thin,rectangle,inner sep=1.5pt, rounded corners=3pt] (X) {\color{rubric}#1};}
\newcommand\corr[1]{#1}
\newcommand{\ed}[1]{ {\color{silver}\sffamily\footnotesize (#1)} } % <milestone ed="1688"/>
\newcommand\expan[1]{#1}
\newcommand\foreign[1]{\emph{#1}}
\newcommand\gap[1]{#1}
\renewcommand{\LettrineFontHook}{\color{rubric}}
\newcommand{\initial}[2]{\lettrine[lines=2, loversize=0.3, lhang=0.3]{#1}{#2}}
\newcommand{\initialiv}[2]{%
  \let\oldLFH\LettrineFontHook
  % \renewcommand{\LettrineFontHook}{\color{rubric}\ttfamily}
  \IfSubStr{QJ’}{#1}{
    \lettrine[lines=4, lhang=0.2, loversize=-0.1, lraise=0.2]{\smash{#1}}{#2}
  }{\IfSubStr{É}{#1}{
    \lettrine[lines=4, lhang=0.2, loversize=-0, lraise=0]{\smash{#1}}{#2}
  }{\IfSubStr{ÀÂ}{#1}{
    \lettrine[lines=4, lhang=0.2, loversize=-0, lraise=0, slope=0.6em]{\smash{#1}}{#2}
  }{\IfSubStr{A}{#1}{
    \lettrine[lines=4, lhang=0.2, loversize=0.2, slope=0.6em]{\smash{#1}}{#2}
  }{\IfSubStr{V}{#1}{
    \lettrine[lines=4, lhang=0.2, loversize=0.2, slope=-0.5em]{\smash{#1}}{#2}
  }{
    \lettrine[lines=4, lhang=0.2, loversize=0.2]{\smash{#1}}{#2}
  }}}}}
  \let\LettrineFontHook\oldLFH
}
\newcommand{\labelchar}[1]{\textbf{\color{rubric} #1}}
\newcommand{\milestone}[1]{\autour{\footnotesize\color{rubric} #1}} % <milestone n="4"/>
\newcommand\name[1]{#1}
\newcommand\orig[1]{#1}
\newcommand\orgName[1]{#1}
\newcommand\persName[1]{#1}
\newcommand\placeName[1]{#1}
\newcommand{\pn}[1]{\IfSubStr{-—–¶}{#1}% <p n="3"/>
  {\noindent{\bfseries\color{rubric}   ¶  }}
  {{\footnotesize\autour{ #1}  }}}
\newcommand\reg{}
% \newcommand\ref{} % already defined
\newcommand\sic[1]{#1}
\newcommand\surname[1]{\textsc{#1}}
\newcommand\term[1]{\textbf{#1}}

\def\mednobreak{\ifdim\lastskip<\medskipamount
  \removelastskip\nopagebreak\medskip\fi}
\def\bignobreak{\ifdim\lastskip<\bigskipamount
  \removelastskip\nopagebreak\bigskip\fi}

% commands, blocks
\newcommand{\byline}[1]{\bigskip{\RaggedLeft{#1}\par}\bigskip}
\newcommand{\bibl}[1]{{\RaggedLeft{#1}\par\bigskip}}
\newcommand{\biblitem}[1]{{\noindent\hangindent=\parindent   #1\par}}
\newcommand{\dateline}[1]{\medskip{\RaggedLeft{#1}\par}\bigskip}
\newcommand{\labelblock}[1]{\medbreak{\noindent\color{rubric}\bfseries #1}\par\mednobreak}
\newcommand{\salute}[1]{\bigbreak{#1}\par\medbreak}
\newcommand{\signed}[1]{\bigbreak\filbreak{\raggedleft #1\par}\medskip}

% environments for blocks (some may become commands)
\newenvironment{borderbox}{}{} % framing content
\newenvironment{citbibl}{\ifvmode\hfill\fi}{\ifvmode\par\fi }
\newenvironment{docAuthor}{\ifvmode\vskip4pt\fontsize{16pt}{18pt}\selectfont\fi\itshape}{\ifvmode\par\fi }
\newenvironment{docDate}{}{\ifvmode\par\fi }
\newenvironment{docImprint}{\vskip6pt}{\ifvmode\par\fi }
\newenvironment{docTitle}{\vskip6pt\bfseries\fontsize{18pt}{22pt}\selectfont}{\par }
\newenvironment{msHead}{\vskip6pt}{\par}
\newenvironment{msItem}{\vskip6pt}{\par}
\newenvironment{titlePart}{}{\par }


% environments for block containers
\newenvironment{argument}{\itshape\parindent0pt}{\vskip1.5em}
\newenvironment{biblfree}{}{\ifvmode\par\fi }
\newenvironment{bibitemlist}[1]{%
  \list{\@biblabel{\@arabic\c@enumiv}}%
  {%
    \settowidth\labelwidth{\@biblabel{#1}}%
    \leftmargin\labelwidth
    \advance\leftmargin\labelsep
    \@openbib@code
    \usecounter{enumiv}%
    \let\p@enumiv\@empty
    \renewcommand\theenumiv{\@arabic\c@enumiv}%
  }
  \sloppy
  \clubpenalty4000
  \@clubpenalty \clubpenalty
  \widowpenalty4000%
  \sfcode`\.\@m
}%
{\def\@noitemerr
  {\@latex@warning{Empty `bibitemlist' environment}}%
\endlist}
\newenvironment{quoteblock}% may be used for ornaments
  {\begin{quoting}}
  {\end{quoting}}

% table () is preceded and finished by custom command
\newcommand{\tableopen}[1]{%
  \ifnum\strcmp{#1}{wide}=0{%
    \begin{center}
  }
  \else\ifnum\strcmp{#1}{long}=0{%
    \begin{center}
  }
  \else{%
    \begin{center}
  }
  \fi\fi
}
\newcommand{\tableclose}[1]{%
  \ifnum\strcmp{#1}{wide}=0{%
    \end{center}
  }
  \else\ifnum\strcmp{#1}{long}=0{%
    \end{center}
  }
  \else{%
    \end{center}
  }
  \fi\fi
}


% text structure
\newcommand\chapteropen{} % before chapter title
\newcommand\chaptercont{} % after title, argument, epigraph…
\newcommand\chapterclose{} % maybe useful for multicol settings
\setcounter{secnumdepth}{-2} % no counters for hierarchy titles
\setcounter{tocdepth}{5} % deep toc
\markright{\@title} % ???
\markboth{\@title}{\@author} % ???
\renewcommand\tableofcontents{\@starttoc{toc}}
% toclof format
% \renewcommand{\@tocrmarg}{0.1em} % Useless command?
% \renewcommand{\@pnumwidth}{0.5em} % {1.75em}
\renewcommand{\@cftmaketoctitle}{}
\setlength{\cftbeforesecskip}{\z@ \@plus.2\p@}
\renewcommand{\cftchapfont}{}
\renewcommand{\cftchapdotsep}{\cftdotsep}
\renewcommand{\cftchapleader}{\normalfont\cftdotfill{\cftchapdotsep}}
\renewcommand{\cftchappagefont}{\bfseries}
\setlength{\cftbeforechapskip}{0em \@plus\p@}
% \renewcommand{\cftsecfont}{\small\relax}
\renewcommand{\cftsecpagefont}{\normalfont}
% \renewcommand{\cftsubsecfont}{\small\relax}
\renewcommand{\cftsecdotsep}{\cftdotsep}
\renewcommand{\cftsecpagefont}{\normalfont}
\renewcommand{\cftsecleader}{\normalfont\cftdotfill{\cftsecdotsep}}
\setlength{\cftsecindent}{1em}
\setlength{\cftsubsecindent}{2em}
\setlength{\cftsubsubsecindent}{3em}
\setlength{\cftchapnumwidth}{1em}
\setlength{\cftsecnumwidth}{1em}
\setlength{\cftsubsecnumwidth}{1em}
\setlength{\cftsubsubsecnumwidth}{1em}

% footnotes
\newif\ifheading
\newcommand*{\fnmarkscale}{\ifheading 0.70 \else 1 \fi}
\renewcommand\footnoterule{\vspace*{0.3cm}\hrule height \arrayrulewidth width 3cm \vspace*{0.3cm}}
\setlength\footnotesep{1.5\footnotesep} % footnote separator
\renewcommand\@makefntext[1]{\parindent 1.5em \noindent \hb@xt@1.8em{\hss{\normalfont\@thefnmark . }}#1} % no superscipt in foot
\patchcmd{\@footnotetext}{\footnotesize}{\footnotesize\sffamily}{}{} % before scrextend, hyperref


%   see https://tex.stackexchange.com/a/34449/5049
\def\truncdiv#1#2{((#1-(#2-1)/2)/#2)}
\def\moduloop#1#2{(#1-\truncdiv{#1}{#2}*#2)}
\def\modulo#1#2{\number\numexpr\moduloop{#1}{#2}\relax}

% orphans and widows
\clubpenalty=9996
\widowpenalty=9999
\brokenpenalty=4991
\predisplaypenalty=10000
\postdisplaypenalty=1549
\displaywidowpenalty=1602
\hyphenpenalty=400
% Copied from Rahtz but not understood
\def\@pnumwidth{1.55em}
\def\@tocrmarg {2.55em}
\def\@dotsep{4.5}
\emergencystretch 3em
\hbadness=4000
\pretolerance=750
\tolerance=2000
\vbadness=4000
\def\Gin@extensions{.pdf,.png,.jpg,.mps,.tif}
% \renewcommand{\@cite}[1]{#1} % biblio

\usepackage{hyperref} % supposed to be the last one, :o) except for the ones to follow
\urlstyle{same} % after hyperref
\hypersetup{
  % pdftex, % no effect
  pdftitle={\elbibl},
  % pdfauthor={Your name here},
  % pdfsubject={Your subject here},
  % pdfkeywords={keyword1, keyword2},
  bookmarksnumbered=true,
  bookmarksopen=true,
  bookmarksopenlevel=1,
  pdfstartview=Fit,
  breaklinks=true, % avoid long links
  pdfpagemode=UseOutlines,    % pdf toc
  hyperfootnotes=true,
  colorlinks=false,
  pdfborder=0 0 0,
  % pdfpagelayout=TwoPageRight,
  % linktocpage=true, % NO, toc, link only on page no
}

\makeatother % /@@@>
%%%%%%%%%%%%%%
% </TEI> end %
%%%%%%%%%%%%%%


%%%%%%%%%%%%%
% footnotes %
%%%%%%%%%%%%%
\renewcommand{\thefootnote}{\bfseries\textcolor{rubric}{\arabic{footnote}}} % color for footnote marks

%%%%%%%%%
% Fonts %
%%%%%%%%%
\usepackage[]{roboto} % SmallCaps, Regular is a bit bold
% \linespread{0.90} % too compact, keep font natural
\newfontfamily\fontrun[]{Roboto Condensed Light} % condensed runing heads
\ifav
  \setmainfont[
    ItalicFont={Roboto Light Italic},
  ]{Roboto}
\else\ifbooklet
  \setmainfont[
    ItalicFont={Roboto Light Italic},
  ]{Roboto}
\else
\setmainfont[
  ItalicFont={Roboto Italic},
]{Roboto Light}
\fi\fi
\renewcommand{\LettrineFontHook}{\bfseries\color{rubric}}
% \renewenvironment{labelblock}{\begin{center}\bfseries\color{rubric}}{\end{center}}

%%%%%%%%
% MISC %
%%%%%%%%

\setdefaultlanguage[frenchpart=false]{french} % bug on part


\newenvironment{quotebar}{%
    \def\FrameCommand{{\color{rubric!10!}\vrule width 0.5em} \hspace{0.9em}}%
    \def\OuterFrameSep{\itemsep} % séparateur vertical
    \MakeFramed {\advance\hsize-\width \FrameRestore}
  }%
  {%
    \endMakeFramed
  }
\renewenvironment{quoteblock}% may be used for ornaments
  {%
    \savenotes
    \setstretch{0.9}
    \normalfont
    \begin{quotebar}
  }
  {%
    \end{quotebar}
    \spewnotes
  }


\renewcommand{\headrulewidth}{\arrayrulewidth}
\renewcommand{\headrule}{{\color{rubric}\hrule}}

% delicate tuning, image has produce line-height problems in title on 2 lines
\titleformat{name=\chapter} % command
  [display] % shape
  {\vspace{1.5em}\centering} % format
  {} % label
  {0pt} % separator between n
  {}
[{\color{rubric}\huge\textbf{#1}}\bigskip] % after code
% \titlespacing{command}{left spacing}{before spacing}{after spacing}[right]
\titlespacing*{\chapter}{0pt}{-2em}{0pt}[0pt]

\titleformat{name=\section}
  [block]{}{}{}{}
  [\vbox{\color{rubric}\large\raggedleft\textbf{#1}}]
\titlespacing{\section}{0pt}{0pt plus 4pt minus 2pt}{\baselineskip}

\titleformat{name=\subsection}
  [block]
  {}
  {} % \thesection
  {} % separator \arrayrulewidth
  {}
[\vbox{\large\textbf{#1}}]
% \titlespacing{\subsection}{0pt}{0pt plus 4pt minus 2pt}{\baselineskip}

\ifaiv
  \fancypagestyle{main}{%
    \fancyhf{}
    \setlength{\headheight}{1.5em}
    \fancyhead{} % reset head
    \fancyfoot{} % reset foot
    \fancyhead[L]{\truncate{0.45\headwidth}{\fontrun\elbibl}} % book ref
    \fancyhead[R]{\truncate{0.45\headwidth}{ \fontrun\nouppercase\leftmark}} % Chapter title
    \fancyhead[C]{\thepage}
  }
  \fancypagestyle{plain}{% apply to chapter
    \fancyhf{}% clear all header and footer fields
    \setlength{\headheight}{1.5em}
    \fancyhead[L]{\truncate{0.9\headwidth}{\fontrun\elbibl}}
    \fancyhead[R]{\thepage}
  }
\else
  \fancypagestyle{main}{%
    \fancyhf{}
    \setlength{\headheight}{1.5em}
    \fancyhead{} % reset head
    \fancyfoot{} % reset foot
    \fancyhead[RE]{\truncate{0.9\headwidth}{\fontrun\elbibl}} % book ref
    \fancyhead[LO]{\truncate{0.9\headwidth}{\fontrun\nouppercase\leftmark}} % Chapter title, \nouppercase needed
    \fancyhead[RO,LE]{\thepage}
  }
  \fancypagestyle{plain}{% apply to chapter
    \fancyhf{}% clear all header and footer fields
    \setlength{\headheight}{1.5em}
    \fancyhead[L]{\truncate{0.9\headwidth}{\fontrun\elbibl}}
    \fancyhead[R]{\thepage}
  }
\fi

\ifav % a5 only
  \titleclass{\section}{top}
\fi

\newcommand\chapo{{%
  \vspace*{-3em}
  \centering % no vskip ()
  {\Large\addfontfeature{LetterSpace=25}\bfseries{\elauthor}}\par
  \smallskip
  {\large\eldate}\par
  \bigskip
  {\Large\selectfont{\eltitle}}\par
  \bigskip
  {\color{rubric}\hline\par}
  \bigskip
  {\Large TEXTE LIBRE À PARTICPATION LIBRE\par}
  \centerline{\small\color{rubric} {hurlus.fr, tiré le \today}}\par
  \bigskip
}}

\newcommand\cover{{%
  \thispagestyle{empty}
  \centering
  {\LARGE\bfseries{\elauthor}}\par
  \bigskip
  {\Large\eldate}\par
  \bigskip
  \bigskip
  {\LARGE\selectfont{\eltitle}}\par
  \vfill\null
  {\color{rubric}\setlength{\arrayrulewidth}{2pt}\hline\par}
  \vfill\null
  {\Large TEXTE LIBRE À PARTICPATION LIBRE\par}
  \centerline{{\href{https://hurlus.fr}{\dotuline{hurlus.fr}}, tiré le \today}}\par
}}

\begin{document}
\pagestyle{empty}
\ifbooklet{
  \cover\newpage
  \thispagestyle{empty}\hbox{}\newpage
  \cover\newpage\noindent Les voyages de la brochure\par
  \bigskip
  \begin{tabularx}{\textwidth}{l|X|X}
    \textbf{Date} & \textbf{Lieu}& \textbf{Nom/pseudo} \\ \hline
    \rule{0pt}{25cm} &  &   \\
  \end{tabularx}
  \newpage
  \addtocounter{page}{-4}
}\fi

\thispagestyle{empty}
\ifaiv
  \twocolumn[\chapo]
\else
  \chapo
\fi
{\it\elabstract}
\bigskip
\makeatletter\@starttoc{toc}\makeatother % toc without new page
\bigskip

\pagestyle{main} % after style

  \section[{Essai sur les privilèges}]{Essai sur les privilèges}\renewcommand{\leftmark}{Essai sur les privilèges}

\noindent On a dit que le privilège est {\itshape dispense pour celui qui l’obtient, et découragement pour les autres}. S’il en est ainsi, convenez que c’est une pauvre invention que celle des privilèges. Imaginons une société la mieux constituée et la plus heureuse possible : n’est-il pas clair que, pour la bouleverser, il ne faudra que dispenser les uns et décourager les autres.\par
J’aurais voulu examiner les privilèges dans leur origine, dans leur nature et dans leurs effets. Mais cette division, toute méthodique qu’elle est, m’eût forcé de revenir trop souvent sur les mêmes idées. D’ailleurs, quant à l’origine, elle m’eût jeté dans une fastidieuse et interminable discussion de faits, car que ne trouve-t-on pas dans les faits, en cherchant comme l’on cherche ? J’aime encore mieux supposer, si l’on m’y force, aux privilèges l’origine la plus pure. Leurs partisans, c’est-à-dire à peu près tous ceux qui en profitent, ne peuvent demander davantage.\par
Tous les privilèges, sans distinction, ont certainement pour objet ou de {\itshape dispenser} de la loi, ou de donner un {\itshape droit exclusif} à quelque chose qui n’est pas défendu par la loi. L’essence du privilège est d’être hors du droit commun, et l’on ne peut en sortir que de l’une ou de l’autre de ces deux manières. En saisissant donc notre sujet sous ce double point de vue, on doit convenir que tous les privilèges à la fois seront à juste titre enveloppés dans le jugement qui pourra résulter de cet examen.\par
Demandons-nous d’abord quelle est l’objet de la loi : c’est sans doute d’empêcher qu’il ne soit porté atteinte à la liberté ou à la propriété de quelqu’un. On ne fait pas des lois pour le plaisir d’en faire. Celles qui n’auraient pour effet que de gêner mal à propos la liberté des citoyens, seraient contraires à la fin de toute association ; il faudrait se hâter de les abolir.\par
Il est une {\itshape loi mère} d’où toutes les autres doivent découler : {\itshape Ne fais point de tort à autrui}. C’est cette grande loi naturelle que le législateur distribue en quelque sorte en détail par les diverses applications qu’il en fait pour le bon ordre de la société ; de là sortent toutes les lois positives. Celles qui peuvent empêcher qu’on ne fasse du tort à autrui, sont bonnes ; celles qui ne serviraient à ce but ni médiatement, ni immédiatement, quand même elles manifesteraient point une intention malfaisante, sont pourtant mauvaise ; car, d’abord, elles gênent la liberté, et puis, ou elles tiennent la place des véritablement bonnes lois, ou au moins elles les repoussent de toutes leurs forces.\par
Hors de la loi, tout est libre : hors de ce qui est garanti à quelqu’un par la loi, chaque chose appartient à tous.\par
Cependant, tel est le déplorable effet du long asservissement des esprits, que les peuples, loin de connaître leur vraie position sociale, loin de sentir qu’ils ont le droit même de faire révoquer les mauvaises lois, en sont venus jusqu’à croire que rien n’est à eux que ce que la loi, bonne ou mauvaise, veut bien leur accorder. Ils semblent ignorer que la liberté, que la propriété sont antérieures à tout ; que les hommes, en s’associant, n’ont pu avoir pour objet que de mettre leurs droits à couvert des entreprises des méchants, et de se livrer en même temps à l’abri de cette sécurité, à un développement de leurs facultés morales et physiques, plus étendue, plus énergique, et plus fécond en jouissances ; qu’ainsi, leur propriété, accrue de tout ce qu’une nouvelle industrie a pu y ajouter dans l’état social, est bien à eux, et ne saurait jamais être considérée comme le don d’un pouvoir étranger ; que l’autorité tutélaire est établie par eux ; qu’elle l’est, non pour accorder ce qui leur appartient, mais pour le protéger ; et qu’enfin, chaque citoyen, indistinctement, a un droit inattaquable, non à ce que la loi permet, puisque la loi n’a rien à permettre, mais à tout ce qu’elle ne défend pas.\par
À l’aide de ces principes élémentaires, nous pouvons déjà juger les privilèges. Ceux qui auraient pour objet de dispenser de la loi, ne peuvent soutenir ; toute loi, avons-nous observé, dit ou directement ou indirectement :\par
{\itshape Ne fais pas tort à autrui} ; ce serait donc dire aux privilégiés : {\itshape Permis à vous de faire tort à autrui}. Il n’est pas de pouvoir à qui il soit donné de faire une telle concession. Si la loi est bonne, elle doit obliger tout le monde ; si elle est mauvaise, il faut l’anéantir : elle est un attentat contre la liberté.\par
Pareillement, on ne peut donner à personne un droit exclusif à ce qui n’est pas défendu par la loi ; ce serait ravir aux citoyens une portion de leur liberté. Tout ce qui n’est pas défendu par la loi, avons-nous observé aussi, est du domaine de la liberté civile, et appartient à tout le monde. Accorder un privilège exclusif à quelqu’un sur ce qui appartient à tout le monde, e serait faire tort à tout le monde pour quelqu’un : ce qui présente à la fois l’idée de l’injustice et de la plus absurde déraison.\par
Tous les privilèges sont donc, par la nature des choses, injustes, odieux et contradictoires à la fin suprême de toute société politique.\par
Les privilèges honorifiques ne peuvent être sauvés de la prospection générale, puisqu’ils ont un des caractères que nous venons de citer, celui de donner un droit exclusif à ce qui n’est pas défendu par la loi ; sans compter que, sous le titre hypocrite de privilèges honorifiques, il n’est presque point de profit pécuniaire qu’ils ne tendent à envahir. Mais comme, même parmi les bons esprits, on en trouve plusieurs qui se déclarent pour ce genre de privilèges, ou du moins qui demandent grâce pour eux, il est bon d’examiner avec attention si réellement ils sont plus excusables que les autres.\par
Pour moi, je le dirai franchement, je leur trouve un vice de plus, et ce vice me paraît énorme : c’est qu’ils tendent à avilir le grand corps des citoyens ; et, certes, ce n’est pas un petit mal fait aux hommes que de les avilir. Concevra-t-on jamais qu’on ait pu consentir à vouloir ainsi humilier vingt-cinq millions huit cent mille individus, pour en honorer ridiculement deux cent mille ? Le sophiste le plus adroit vaudrait-il bien nous montrer dans une combinaison aussi antisociale, ce qu’il peut y voir de conforme à l’intérêt général ?\par
Le titre le plus favorable à la concession d’un privilège honorifique, serait d’avoir rendu un grand service à la patrie, c’est-à-dire à la nation, qui ne peut être que la généralité des citoyens. Eh bien ! récompensez le membre qui a bien mérité du corps ; mais n’ayez pas l’absurde folie de rabaisser le corps vis-à-vis du membre. L’ensemble des citoyens est toujours la chose principale, la chose qui est servie. Doit-elle, en aucun sens, être sacrifiée au serviteur à qui il n’est dû un pris que pour l’avoir servie ?\par
Une conservation aussi choquante aurait dû se faire généralement sentir ; et pourtant notre résultat paraîtra peut-être nouveau, ou du moins fort étrange. À cet égard il existe, parmi nous, une superstition invétérée qui repousse la raison, et s’offense même du doute. Quelques peuples sauvages se plaisent à de ridicules difformités, et leur rendent l’hommage dû aux charmes naturels. Chez les nations hyperboréennes, c’est à des excroissances politiques, bien plus difformes, et surtout bien autrement nuisibles, puisqu’elles rongent et ruinent le corps social, que l’on prodigue de stupides hommages. Mais la superstition passe, et le corps qu’elle dégradait reparaît dans toute sa force et sa beauté naturelle.\par
Quoi ! dira-t-on, est-ce que vous ne voulez pas reconnaître les services rendus à l’État ? Pardonnez-moi ; mais je ne fais consister les récompenses de l’État en aucune chose qui soit injuste ou avilissante ; il ne faut pas récompenser quelqu’un aux dépens d’un autre, et surtout aux dépens de presque tous les autres. Ne confondons point ici deux choses aussi différentes que le sont les privilèges et les récompenses.\par
Parlez-vous de services ordinaires ? Il existe pour les acquitter, des salaires ordinaires, ou des gratifications de même nature. S’agit-il d’un service important, ou d’une action d’éclat ? Offrez un avancement rapide de grade, ou un emploi distingué, en proportion des talents de celui que vous avez à récompenser. Enfin, s’il le faut, ajoutez la ressource d’une pension, mais dans un très petit nombre de cas, et seulement lorsqu’à raison des circonstances, telles que vieillesse, blessures, etc., aucun autre moyen ne peut tenir lieu de récompense suffisante.\par
Ce n’est pas assez, dites-vous ; il nous faut encore des distinctions apparentes ; nous voulons nous assurer les égards et la considération publique…\par
À mon tour, je dois vous répondre que la véritable distinction est dans le service que vous avez rendu à la patrie, à l’humanité, et que les égards et la considération publique ne peuvent manquer d’aller où ce genre de mérite les appelle.\par
Laissez, laissez le public dispenser librement les témoignages de son estime. Lorsque dans vos vues philosophiques vous la regardez, cette estime, comme une monnaie morale, puissante par ses effets, vous avez raison ; mais si vous voulez que le prince s’en arroge la distribution, vous vous égarez dans vos idées : la nature, plus philosophe que vous, a placé la vraie source de la considération dans les sentiments du peuple. C’est que chez le peuple sont les vrais besoins ; là réside la patrie, à laquelle les hommes supérieurs sont appelés à consacrer leurs talents ; là, par conséquent, devait être déposé le trésor des récompenses qu’ils peuvent ambitionner.\par
Les événements aveugles, les mauvaises lois, plus aveugles encore, ont conspiré contre la multitude. Elle a été déshéritée, privée de tout. Il ne lui reste que le pouvoir d’honorer de son estime ceux qui la servent ; elle n’a plus que ce moyen d’exciter encore des hommes dignes de la servir : voulez-vous la dépouiller de son dernier bien, de sa dernière réserve, et rendre ainsi sa propriété même la plus intime, inutile à son bonheur ?\par
Les administrateurs ordinaires, après avoir ruiné, avili le grand corps des citoyens, s’accoutument aisément à le négliger. Ils dédaignent, ils méprisent presque de bonne foi un peuple qui ne peut jamais être devenu méprisable que par leur crime. S’ils s’en occupent encore, ce n’est que pour en punir les fautes. Leur colère veille sur le peuple, leur tendresse n’appartient qu’aux privilégiés. Mais alors même la vertu et le génie s’efforcent encore de remplir la destination de la nature. Une voix secrète parle sans cesse au fond des âmes énergiques en faveur des faibles. Oui, les besoins sacrés du peuple seront éternellement l’objet adoré des méditations du philosophe indépendant, le but secret ou public des soins et des sacrifices du citoyen vertueux. Le pauvre, à la vérité, ne répond à ses bienfaiteurs que par des bénédictions ; mais que cette récompense est supérieure à toutes les faveurs du pouvoir ! Ah ! laissez le prix de la considération publique couler librement du sein de la nation pour acquitter sa dette envers le génie et la vertu ! Gardons-nous de violer les sublimes rapports d’humanité que la nature a été attentive à graver dans le fond de nos cœurs. Applaudissons à cet admirable commerce de bienfaits et d’hommages qui s’établit, pour la consolation de la terre, entre les besoins des peuples reconnaissants, et les grands hommes surabondamment payés de tous leurs services par un simple tribut de reconnaissance. Tout est pur dans cet échange : il est fécond en vertu, puissant en bonheur, tant qu’il n’est point troublé dans sa marche naturelle et libre.\par
Mais, si la cour s’en empare, je ne vois plus dans l’estime publique qu’une monnaie altérée par les combinaisons d’un indigne monopole. Bientôt, de l’abus qu’on en fait, doit sortir et se déborder sur toutes les classes de citoyens l’immoralité la plus audacieuse. Les signaux convenus pour appeler la considération sont mal placés, ils en égarent le sentiment. Chez la plupart des hommes, ce sentiment finit par se corrompre par l’alliance même à laquelle on le force ; comment échapperait-il au poison des vices auxquels il prend l’habitude de s’attacher ? Chez le petit nombre de gens éclairés, l’estime se retire au fond du cœur, indignée du rôle honteux auquel on prétendait la soumettre ; il n’y a donc plus d’estime réelle : et pourtant son langage, son maintien subsistent dans la société, pour prostituer de faux honneurs publics aux intrigants, aux favoris, souvent aux hommes les plus coupables.\par
Dans un tel désordre de mœurs, le génie est persécuté, la vertu est ridiculisée ; et, à côté, une foule de signes et de décorations diversement bigarrées commandent impérieusement le respect et les égards envers la médiocrité, la bassesse et le crime. Comment les honneurs ne parviendraient-ils pas à étouffer l’honneur, à corrompre tout à fait l’opinion, et à dégrader toutes les âmes ?\par
En vais prétendriez-vous que, vertueux vous-même, vous ne confondrez jamais le charlatan habile ou le vil courtisan, avec le bon serviteur qui présente de justes titres aux récompenses publiques : à cet égard, l’expérience atteste vos nombreuses erreurs. Et après tout, ne devez-vous pas convenir au moins, que ceux à qui vous avez livré vos étranges brevets d’honneur, peuvent ensuite dégénérer dans leurs sentiments, dans leurs actions ? Ils continueront pourtant à exiger, à attirer les hommages de la multitude. Ce sera donc pour des citoyens indignes, pour des hommes notés peut-être par nos justes mépris, que vous aurez aliéné sans retour une portion de la considération publique.\par
Il n’en est pas ainsi de l’estime qui émane des peuples. Nécessairement libre, elle se retire lorsqu’elle cesse d’être méritée. Plus pure dans son principe, plus naturelle dans ses mouvements, elle est aussi plus certaine dans sa marche, plus utile dans ses effets. Elle est le seul prix toujours proportionné à l’âme du citoyen vertueux ; le seul propre à inspirer de bonnes actions, et non à irriter la soif de la vanité et de l’orgueil ; le seul qu’on puisse rechercher, et obtenir sans manœuvres et sans bassesse.\par
Encore une fois, laissez les citoyens faire les honneurs de leurs sentiments, et se livrer d’eux-mêmes à cette expression si flatteuse, si encourageante, qu’ils savent leur donner comme par inspiration, et vous connaîtrez alors au libre concours de toutes les âmes qui ont de l’énergie, aux efforts multipliés de tous les genres de bien, ce qui doit produire, pour l’avancement social, le grand ressort de l’estime publique\footnote{Je parle, au surplus, d’une nation libre ou qui va le devenir. Il est bien certain que la dispensation des honneurs publics ne peut point appartenir à un peuple esclave. Chez un peuple esclave, la monnaie morale est toujours fausse, quelle que soit la main qui la distribue.}.\par
Mais votre paresse et votre orgueil s’accommodent mieux des privilèges. Je le vois, vous demandez moins à être distingué {\itshape par} vos concitoyens, que vous ne cherchez à être distingué {\itshape de} vos concitoyens\footnote{ \noindent Quand on devrait accuser cette note d’être un peu {\itshape métaphysique}, sans connaître la valeur de ce mot devenu si effrayant pour les esprits inattentifs, je dirai que la direction {\itshape de} n’est rien que {\itshape différence} ; elle appartient aux deux termes à la fois ; car si {\itshape A} est distingué {\itshape de B}, il est clair que, par la même raison, {\itshape B} sera distingué {\itshape de A}. Ainsi {\itshape A} et {\itshape B} sont entre eux, comme l’on dit, à deux de jeu. Il faut bien que tous les individus, tous les êtres soient différents l’un de l’autre. Il n’y a pas là de quoi s’enorgueillir, ou tous y auraient le même droit. Dans la nature, la supériorité ou l’infériorité ne sont pas des choses de droit, mais des choses de fait : celui-là devient supérieur qui l’emporte sur l’autre. Cet avantage de fait suppose, à la vérité, plus de force d’un côté que d’autre ; mais, si l’on veut en venir à ce premier titre, de quel côté sera la supériorité ? à qui croyez-vous qu’elle appartienne ? au corps des citoyens ou aux privilégiés ?La distinction {\itshape par} est, au contraire, le principe social le plus fécond en bonnes actions, en bonnes mœurs, etc. Mais si son siège est dans l’âme de ceux qui {\itshape distinguent}, et non dans la main de celui qui prétend disputer les distinctions ; si c’est un sentiment de leur part, et ne peut pas être autre chose sans cesser d’être une vérité, il faut dire aussi que ce sentiment est essentiellement libre, et qu’il y a une extrême folie, à qui que ce soit, de vouloir disposer malgré moi de mon estime et de mes hommages.\par
 
 }. Le voilà donc manifesté, ce sentiment secret, ce désir inhumain, plein d’orgueil, et pourtant si honteux, que vous vous efforciez de le cacher sous l’apparence de l’intérêt public. Ce n’est pas à l’estime ou à l’amour de vos semblables que vous aspirez ; vous n’obéissez, au contraire, qu’aux irritations d’une vanité hostile contre des hommes dont légalité vous blesse. Vous faites, au fond de votre cœur, un reproche à la nature de n’avoir pas rangé vos concitoyens dans des espèces inférieures destinées uniquement à vous servir. Pourquoi tout le monde ne partage-t-il pas l’indignation qui m’anime ? Certes, vous étiez loin d’avoir un intérêt personnel à la question qui nous occupe. Il s’agissait des récompenses à décerner au mérite, et non des châtiments qu’il faudrait, dans un état policé, infliger aux plus perfides ennemis de la félicité sociale.\par
De ces considérations générales sur les privilèges honorifiques, descendons maintenant dans leurs {\itshape effets}, soit relativement à l’intérêt public, soit relativement à l’intérêt des privilégiés eux-mêmes.\par
Au moment où les ministres impriment le caractère de privilégié à un moyen, ils ouvrent son âme à un intérêt particulier, et la ferment plus ou moins aux inspirations de l’intérêt commun. L’idée de patrie se resserre pour le privilégié ; elle se renferme dans la caste où il est adopté. Tous ses efforts, auparavant employés avec fruit au service de la chose nationale, vont se tourner contre elle. On voulait l’encourager à mieux faire ; on n’a réussi qu’à le dépraver.\par
Alors naît dans son cœur le besoin de primer, un désir insatiable de domination. Ce désir, malheureusement trop analogue à la constitution humaine, est une vraie maladie antisociale ; et si par son essence il doit toujours être nuisible, qu’on juge de ses ravages lorsque l’opinion et la loi viennent lui prêter leur puissant appui.\par
Pénétrez un moment dans les nouveaux sentiments d’un privilégié. Il se considère, avec ses collègues, comme faisant un ordre à part, une nation choisie dans la nation. Il pense qu’il se doit d’abord à ceux de sa caste, et s’il continue à s’occuper des autres, ce ne sont plus en effet que les {\itshape autres}, ce ne sont plus les siens. Ce n’est plus ce corps dont il était membre ; ce n’est que le {\itshape peuple}, le peuple qui, bientôt dans son langage, ainsi que dans son cœur, n’est qu’un assemblage de {\itshape gens de rien}, une classe d’hommes créée tout exprès pour servir ; au lieu qu’il est fait, lui, pour commander et pour jouir.\par
Oui, les privilèges en viennent réellement à se regarder comme une espèce d’hommes\footnote{Comme je ne veux pas qu’on m’accuse d’exagérer, lisez à la fin une pièce authentique que je tire du procès-verbal de l’ordre de la noblesse aux États de 1614.}. Cette opinion, en apparence si exagérée, et qui ne paraît pas renfermée dans la notion du privilège, en devient insensiblement comme la conséquence, naturelle, et finit par s’établir dans tous les esprits. Je le demande à tout privilégié franc et loyal, comme sans doute il s’en trouve : lorsqu’il voit auprès de lui un homme du peuple, qui n’est pas venu là pour se faire protéger, n’éprouve-t-il pas, le plus souvent, un mouvement involontaire de répulsion, prêt à s’échapper sur le plus léger prétexte, par quelque parole dure, ou quelque geste offensant ?\par
Le faux sentiment d’une supériorité personnelle est tellement cher aux privilégiés, qu’ils veulent l’étendre à tous leurs rapports avec le reste des citoyens. Ils ne {\itshape sont point faits} pour être {\itshape confondus}, pour être {\itshape à côté}, pour concourir, ou se trouver {\itshape ensemble}, etc., etc. C’est se {\itshape manquer} essentiellement que de disputer, que de paraître avoir tort, quand on a tort ; c’est se {\itshape compromettre} même que d’avoir raison avec, etc., etc.…\par
Mais rien n’est plus curieux, à cet égard, que le spectacle qui s’offre dans des campagnes éloignées de la capitale. C’est là que le noble sentiment de sa supériorité se nourrit et s’enfle à l’abri de la raison et des passions des villes. Dans les vieux châteaux, le privilégié se respecte mieux, il peut se tenir plus longtemps en extase devant les portraits de ses ancêtres et s’enivrer plus à loisir de l’honneur de descendre d’hommes qui vivaient dans les treizième et quatorzième siècles ; car il ne soupçonne pas qu’un tel avantage puisse être commun à toutes les familles. Dans son opinion, c’est un caractère particulier à certaines races.\par
Souvent il présente, avec toute la modestie possible, au respect des étrangers, cette suite d’aïeux, dont la vue a si souvent excité en lui les rêves les plus doux. Mais il s’arrête peu sur le père ou le grand-père (ces mots ont même je ne sais quoi d’offensant pour la dignité d’une langue privilégiée). Ses ancêtres les plus reculés sont les meilleurs, ils sont plus près de son amour, comme de sa vanité.\par
J’ai vu de ces longues galeries d’images paternelles, elles ne sont pas précieuses par l’art du peintre, ni même, il faut l’avouer, par le sentiment de la parenté\footnote{Qui n’a pas entendu, dans ces moments, le démonstrateur faire des réflexions aimables sur {\itshape celui-ci, qui, en douze cent et tant, était un rude chrétien : ses vassaux n’avaient pas beau jeu, etc}… ; {\itshape celui-là} (bien entendu qu’on en prononce le nom ancien), {\itshape qui, s’étant maladroitement engagé dans une trahison, paya de sa tête, etc}… mais toujours en {\itshape douze cent}… Je veux raconter à ce sujet le propos assez récent d’une dame qui, dans un cercle nombreux et {\itshape bien composé}, blâmait à outrance la conduite criminelle, en effet, de quelqu’un d’une des plus grandes maisons du royaume. Tout à coup elle s’interrompt pour dire, d’un air difficile à peindre : « Mais je ne sais pas pourquoi j’en dis tant de mal, car j’ai {\itshape l’honneur} de lui appartenir. »} ; mais qu’elles sont sublimes par les souvenirs des temps et des mœurs de la {\itshape bonne féodalité} !\par
C’est dans les châteaux qu’on sent avec enthousiasme, ainsi qu’il faut sentir les beaux-arts, tout l’effet d’un arbre généalogique, à rameaux touffus et à tige élancée. C’est là qu’on connaît, à n’en rien oublier, même dans les plus petites occasions, tout ce que {\itshape vaut} un homme comme il {\itshape faut}\footnote{Je renonce à saisir toutes les nuances, toutes les finesses du langage habituel des privilégiés ; nous aurions besoin pour cette langue d’un dictionnaire particulier qui serait neuf par plus d’un endroit ; car, au lieu d’y présenter le sens propre ou métaphorique des mots, il s’agirait, au contraire, de détacher des morts leur véritable sens, pour ne rien laisser dessous qu’un vide pour la raison, mais d’admirables profondeurs pour le préjugé : nous y lirions ce que c’est qu’être privilégié d’un privilège qui n’a pas {\itshape commencé}. Ceux qui en ont de cette nature sont {\itshape des bons} ; ils sont, par la {\itshape grâce} de Dieu, bien différents de cette foule de nouveaux privilégiés qui sont par la {\itshape grâce} du prince. On ne compte pas des citoyens qui, n’aspirant pas à être par grâce, sont réduits à ne se montrer que par leurs qualités personnelles : c’est fort peu de chose ; c’est la nation. Nous apprendrions, dans ce nouveau dictionnaire, qu’il n’y a de la naissance que pour ceux qui n’ont point d’origine. Les privilégiés du prince eux-mêmes n’osent pas penser avoir plus d’une {\itshape demi-naissance}, et la nation n’en a point. Il serait superflu de remarquer que la naissance dont il s’agit ici n’est pas celle qui vient d’un père et d’une mère, mais celle que le prince donne avec un brevet et sa signature, ou mieux encore, celle qui vient de je ne sais où : c’est la plus estimée. Si vous avez cru, par exemple, que tout homme a nécessairement son père, son grand-père, ses aïeux, etc., vous vous êtes trompé. À cet égard, la certitude physique ne suffit pas : il n’y a de valable que l’attestation de M. Chérin. Pour être {\itshape ancien}, il faut être {\itshape des bons}, nous l’avons dit. Les nouveaux privilégiés sont {\itshape des hommes d’hier}, et les citoyens non privilégiés, je ne sais que vous dire, si ce n’est qu’apparemment ils ne sont pas encore nés. Je suis émerveillé, je l’avoue, du talent avec lequel les privilégiés prolongent à perte de vue, sans jamais se perdre, ces sublimes, quoique incessables conversations. Les plus curieux à entendre, à mon avis, sont ceux qui, constamment à genoux devant leur propre {\itshape honneur}, leurs propres prétentions, rient pourtant de si bon cœur des mêmes prétentions chez les autres. Je soutiens que les opinions des privilégiés sont à la hauteur de leurs sentiments ; et, pour en donner une nouvelle preuve, je vais exposer, d’après leur manière de voir, le vrai tableau d’une société politique. Ils la composent de six à sept classes subordonnées les unes aux autres. Dans la première, sont les {\itshape grands seigneurs}, c’est-à-dire cette partie des gens de la cour en qui sont réunies la naissance, une grande place et l’opulence. La seconde classe comprend les {\itshape présentés} connus, ceux qui {\itshape paraissent} : ce sont les gens de {\itshape qualité}. En troisième ligne viennent les {\itshape présentés} inconnus, qui n’en voulaient qu’aux honneurs de la Gazette : ce sont des gens de {\itshape quelque chose}. On confond dans la classe des {\itshape non présentés}, qui peuvent cependant être {\itshape bons}, tous les {\itshape gentillâtres} de province : c’est l’expression dont ils se servent. Dans la cinquième classe, il faut mettre les {\itshape anoblis} un peu anciens, ou gens de {\itshape néant}. Dans la sixième, se présentent ou plutôt sont relégués les nouveaux anoblis ou gens de moins que rien. Enfin, et pour ne rien oublier, on veut bien laisser dans une septième division le reste des citoyens qu’il n’est pas possible de caractériser autrement que par des injures. Tel est l’ordre social pour le préjugé régnant, et je ne dis rien de nouveau que pour ceux qui ne sont pas de ce monde.}, et le rang dans lequel il faut placer tout le monde.\par
Auprès de ces hautes contemplations, combien paraissent petites et méprisables les occupations des {\itshape gens} de la ville ! S’il était permis d’en prononcer le véritable nom, on pourrait se demander : Qu’est-ce qu’un bourgeois près d’un bon privilégié ? Celui-ci a sans cesse les yeux sur le noble {\itshape temps} passé. Il y voit tous ses titres, toute sa force, il vit de ses ancêtres. Le bourgeois, au contraire, les yeux toujours fixés sur l’ignoble présent, sur l’indifférent {\itshape avenir}, prépare l’un, et soutient l’autre par les ressources de son industrie. Il est, au lieu d’avoir été ; il essuie la peine, et qui pis est, la honte d’employer toute son intelligence, toute sa force à notre service actuel, et de vivre de son travail nécessaire à tous. Ah ! pourquoi le privilégié ne peut-il aller dans le {\itshape passé} jouir de ses titres, de ses grandeurs, et laisser à une stupide nation le {\itshape présent} avec toute son ignobilité !\par
Un bon privilégié se complaît en lui-même, autant qu’il méprise les autres. Il caresse, il idolâtre sérieusement sa dignité personnelle ; et quoique tout l’effort d’une telle superstition ne puisse prêter à d’aussi ridicules erreurs le moindre degré de réalité, elles n’en remplissent pas moins toute la capacité de son âme ; le privilégié s’y abandonne avec autant de conviction, avec autant d’amour que le fou du Pyrée croyait à sa chimère.\par
La vanité, qui pour l’ordinaire est individuelle et se plaît à s’isoler, se transforme ici promptement en un esprit de corps indomptable. Un privilégié vient-il à éprouver la moindre difficulté de la part de la classe qu’il méprise ? d’abord il s’irrite ; il se sent blessé dans sa prérogative ; il croit l’être dans son bien, dans sa propriété ; bientôt il excite, il enflamme tous ses coprivilégiés, et il vient à bout de former une confédération terrible, prête à tout sacrifier pour le maintien, puis pour l’accroissement de son odieuse prérogative. C’est ainsi que l’ordre politique se renverse, et ne laisse plus voir qu’un détestable aristocratisme.\par
Cependant, dira-t-on, on est poli dans la société avec les non-privilégiés, comme avec les autres. Ce n’est pas moi qui ai remarqué, le premier, le caractère de la politesse française. Le privilégié français n’est pas poli parce qu’il croit le {\itshape devoir} aux autres, mais parce qu’il croit {\itshape se} le {\itshape devoir} à lui-même. Ce n’est pas les droits d’autrui qu’il respecte, c’est soi, c’est sa dignité. Il ne veut point être confondu, par des manières vulgaires, avec ce qu’il nomme {\itshape mauvaise compagnie}. Que dirai-je ! il craindrait que l’objet de sa politesse ne le prît pour un {\itshape non-privilégié comme lui}.\par
Ah ! gardez-vous de vous laisser séduire par des apparences grimacières et trompeuses ; ayez le bon esprit de ne voir en elles que ce qui y est, un orgueilleux attribut de ces mêmes privilèges que nous détestons.\par
Pour expliquer la soif ardente d’acquérir des privilèges, on pensera peut-être que, du moins, au prix du bonheur public, il s’est composé, en faveur des privilégiés, un genre de félicité particulière, dans le charme enivrant de cette supériorité dont le petit nombre jouit, auquel un grand nombre aspire, et dont les autres sont réduits à se venger par les ressources de l’envie ou de la haine.\par
Mais oublierait-on que la nature n’imposa jamais des lois impuissantes ou vaines ; qu’elle a arrêté de ne départir le bonheur aux hommes que dans l’égalité ; et que c’est un échange perfide que celui qui est offert par la vanité contre cette multitude de sentiments naturels dont la félicité réelle se compose ?\par
Écoutons là-dessus notre propre expérience\footnote{La société est, pour tous ceux que le sort n’a pas condamnés à un travail sans relâche, une source pure et féconde de jouissances agréables : on le sait, et le peuple qui se croit le plus civilisé se vante aussi d’avoir la meilleure société. Où doit être la meilleure société ? là sans doute où les hommes qui se conviendraient le mieux pourraient se rapprocher librement, et ceux qui ne se conviendraient pas, se séparer sans obstacle ; là où, dans un nombre donné d’hommes, il y en aurait davantage qui posséderaient les talents et l’esprit de société, et où le choix, parmi eux, ne serait embarrassé d’aucune considération étrangère au but qu’on se propose en se réunissant. Qu’on dise si les préjugés d’état ne s’opposent point de toutes manières à cet arrangement si simple ? Combien de maîtresses de maisons sont forcées d’éloigner les hommes qui les intéresseraient le plus, par égard pour les hauts privilégiés qui les ennuient ! vous avez beau, dans vos sociétés si vantées et si insipides, {\itshape singer} cette égalité dont vous ne pouvez pas dans des instants passagers que les hommes peuvent se modifier intérieurement au point de devenir les uns pour les autres ce qu’ils seraient sans doute si l’égalité était la réalité de toute la vie, plutôt que le jeu de quelques moments. Cette matière serait inépuisable : je ne puis qu’indiquer quelques vues.} ; ouvrons les yeux sur celle de tous les grands privilégiés, de tous les grands mandataires que leur état expose à jouir, dans les provinces, des prétendus charmes de la supériorité. Elle fait tout pour eux, cette supériorité ; cependant ils se trouvent seuls : l’ennui fatigue leur âme, et venge les droits de la nature. Voyez à l’ardeur impatiente avec laquelle ils reviennent chercher des égaux dans la capitale, combien il est insensé de semer continuellement sur le terrain de la vanité, pour n’y recueillir que les ronces de l’orgueil ou les pavots de l’ennui.\par
Nous ne confondons point avec la supériorité absurde et chimérique, qui est l’ouvrage des privilégiés, cette supériorité légale qui suppose seulement des gouvernants et des gouvernés. Celle-ci est réelle, elle est nécessaire ; elle n’enorgueillit pas les uns, elle n’humilie pas les autres : c’est une supériorité de fonctions, et non de personnes. Or, puisque cette supériorité même ne peut dédommager des douceurs de l’égalité, que doit-on penser de la chimère dont se repaissent les simples privilégiés ?\par
Ah ! si les hommes voulaient connaître leurs intérêts ; s’ils savaient faire quelque chose pour leur bonheur ! s’ils consentaient à ouvrir enfin les yeux sur la cruelle imprudence qui leur a fait dédaigner si longtemps les droits de citoyens libres, pour les vains privilèges de la servitude, comme ils se hâteraient d’abjurer les nombreuses vanités auxquelles ils ont été dressés dès l’enfance ! comme ils se méfieraient d’un ordre de choses qui s’allie si bien avec le despotisme ! Les droits de citoyen embrassent tout ; les privilèges gâtent tout et ne dédommagent de rien que chez les esclaves.\par
Jusqu’à présent j’ai confondu tous les privilèges, ceux qui sont héréditaires avec ceux que l’on obtient soi-même ; ce n’est pas qu’ils soient tous également nuisibles, également dangereux dans l’état social. S’il y a des places dans l’ordre des maux et de l’absurdité, sans doute les privilèges héréditaires y doivent occuper la première, et je n’abaisserai pas ma raison jusqu’à prouver une vérité si palpable. Faire d’un privilège une propriété transmissible, c’est vouloir s’ôter jusqu’aux faibles prétextes par lesquels on cherche à justifier la concession des privilèges ; c’est renverser tout principe, toute raison.\par
D’autres observations jetteront un nouveau jour sur les funestes effets des privilèges. Remarquons auparavant une vérité générale, c’est qu’une fausse idée n’a besoin que d’être fécondée par l’intérêt personnel, et soutenue de l’exemple de quelques siècles, pour corrompre à la fin tout l’entendement. Insensiblement, et de préjugés en préjugés, on tombe dans un corps de doctrine qui présente l’extrême de la déraison, et ce qu’il y a de plus révoltant, sans que la longue et superstitieuse crédulité des peuples en soit plus ébranlée.\par
Ainsi, voyons-nous s’élever sous nos yeux, et sans que la nation songe à réclamer, de nombreux essaims de privilégiés dans une forte et presque religieuse persuasion qu’ils ont un droit acquis aux honneurs, par leur naissance, et à une portion du tribut des peuples, par cela seul qu’ils continuent de vivre : c’est pour eux un titre suffisant.\par
Ce n’était pas assez, en effet, que les privilégiés se regardassent comme une autre espèce d’homme ; ils devaient se considérer modestement, et presque de bonne foi, eux et leurs descendants, comme un {\itshape besoin} des peuples, non comme fonctionnaires de la chose publique : à ce titre, ils ressembleraient à l’universalité des mandataires publics, de quelque classe qu’on les tire. C’est comme formant un corps privilégié qu’ils s’imaginent être nécessaires à toute société qui vit sous un régime monarchique. S’ils parlent aux chefs du gouvernement ou au monarque lui-même, ils se représentent comme l’appui du trône, et ses défenseurs naturels contre le peuple ; si, au contraire, ils parlent à la nation, ils deviennent alors les vrais défenseurs d’un peuple qui, sans eux, serait bientôt écrasé par la royauté.\par
Avec un peu plus de lumières, le gouvernement verrait qu’il ne faut dans une société que des citoyens vivant et agissant sous la protection de la loi, et une autorité tutélaire chargée de veiller et de protéger. La seule hiérarchie nécessaire, nous l’avons dit, s’établit entre les agents de la souveraineté ; c’est là qu’on a besoin d’une gradation de pouvoirs ; c’est là que se trouvent les vrais rapports d’inférieur à supérieur, parce que la machine publique ne peut se mouvoir qu’au moyen de cette correspondance.\par
Hors de là, il n’y a que des citoyens égaux devant la loi, tous dépendants, non les uns des autres, ce serait une servitude inutile, mais de l’autorité qui les protège, qui les juge, qui les défend, etc. Celui qui jouit des plus grandes possessions n’est pas {\itshape plus} que celui qui jouit de son salaire journalier. Si le riche paye plus de contributions, il offre plus de propriétés à protéger. Mais le denier du pauvre serait-il moins précieux, son droit moins respectable ? Et sa personne ne doit-elle pas reposer sous une protection au moins égale ?\par
C’est en confondant ces notions simples que les privilégiés parlent sans cesse de la nécessité d’une subordination étrangère à celle qui nous soumet au gouvernement et à la loi. L’esprit militaire veut juger des rapports civils, et ne voir une nation que comme une grande caserne. Dans une brochure nouvelle n’a-t-on pas osé établir une comparaison entre le soldat et les officiers d’un côté, et de l’autre les privilégiés et les non-privilégiés ! Si vous consultiez l’esprit monacal, qui a tant de rapport avec l’esprit militaire, il prononcerait aussi qu’il n’y aura de l’ordre dans une nation que quand on l’aura soumise à cette foule de règlements de détail avec lesquels il maîtrise ses nombreuses victimes. L’esprit monacal conserve parmi nous, sous un nom moins avili, plus de faveur qu’on ne pense.\par
Disons-le tout à fait, des vues aussi mesquines, aussi misérables, ne peuvent appartenir qu’à des gens qui ne connaissent rien aux vrais rapports qui lient les hommes dans l’état social. Un citoyen, quel qu’il soit, qui n’est point mandataire de l’autorité, est entièrement le maître de ne s’occuper qu’à améliorer son sort et à jouir de ses droits, sans blesser les droits d’autrui, c’est-à-dire sans manquer à la loi. Tous les rapports de citoyen à citoyen sont des rapports libres ; l’un donne son temps ou sa marchandise, l’autre rend en échange son argent : il n’y a point là de subordination, mais échange continuel…\footnote{Je crois important, pour la facilité de la conversation, de distinguer les deux hiérarchies dont nous venons de parler, par les noms de {\itshape vraie} et de {\itshape fausse} hiérarchie. La gradation entre les gouvernants et l’obéissance des gouvernés envers les différents pouvoirs légaux, forment la véritable hiérarchie nécessaire dans toutes les sociétés. Celle des gouvernés entre eux n’est qu’une fausse hiérarchie, inutile, odieuse, reste informe de coutumes féodales. Pour concevoir une subordination possible entre les gouvernés, il faudrait supposer une troupe armée, s’emparant d’un pays, se rendant propriétaire, et conservant, pour la défense commune, les rapports habitués de la discipline militaire. C’est que là, le gouvernement est fondu dans l’état civil : ce n’est pas un peuple, c’est une armée. Chez nous, au contraire, les différentes branches du pouvoir public existent à part, et sont organisées, y compris une armée immense, de manière à n’exiger des simples citoyens qu’une contribution pour acquitter les charges publiques. Qu’on ne s’y trompe point, au milieu de tous ces noms de {\itshape subordination}, de {\itshape dépendance, etc}., que les privilégiés invoquent avec tant de clameur, ce n’est pas l’intérêt de la véritable subordination qui les conduit, ils ne dont cas que de la {\itshape fausse} hiérarchie ; c’est celle-ci qu’ils voudraient rétablir sur les débris de la véritable. Écoutez-les lorsqu’ils parlent des agents ordinaires du gouvernement ; voyez avec quel dédain un bon privilégié croit devoir les traiter. Que voient-ils dans un lieutenant de police ? un homme de peu ou de rien, établi pour faire peur au peuple, et non pour se mêler de tout ce qui peut regarder les gens {\itshape comme il faut}. L’exemple que je cite est à la portée de tout le monde ; qu’on dise de bonne foi s’il est un privilégié qui se croie subordonné au lieutenant de police ? Comment regardent-ils les autres mandataires des différentes branches du pouvoir exécutif, excepté les seuls chefs militaires ? Est-il si rare de les entendre dire : « Je ne suis pas {\itshape fait pour} me soumettre au ministre ; si le Roi me fait l’honneur de me donner des ordres, etc. » J’abandonne ce sujet à l’imagination ou plutôt à l’expérience du lecteur. Mais il était bion de faire remarquer que les véritables ennemis de la subordination et de la vraie hiérarchie, ce sont ces hommes-là même qui prêchent avec tant d’ardeur la soumission à la {\itshape fausse} hiérarchie.}. Si, dans votre étroite politique, vous distinguez un corps de citoyens pour le mettre entre le gouvernement et les peuples, ou ce gouvernement, et alors ce ne sera pas la classe privilégiée dont nous parlons, ou bien il n’appartiendra pas aux fonctions essentielles du pouvoir public, et alors qu’on m’explique ce que peut être un corps intermédiaire, si ce n’est une masse étrangère, nuisible, soit en interceptant les rapports directs entre les gouvernants et les gouvernés, soit en pressant sur les ressorts de la machine publique, soit enfin en devenant, par tout ce qui la distingue du grand corps des citoyens, un fardeau de plus pour la communauté.\par
Toutes les classes de citoyens ont leurs fonctions, leur genre de travail particulier, dont l’ensemble forme le mouvement général de la société. S’il en est une qui prétende se soustraire à cette loi générale, on voit bien qu’elle ne se contente pas d’être inutile, et qu’il faut nécessairement qu’elle soit à charge des autres.\par
Quels sont les deux grands mobiles de la société ? {\itshape l’argent} et {\itshape l’honneur}. C’est par le besoin que l’on a de l’un et de l’autre qu’elle se soutient, et ce n’est pas l’un sans l’autre que ces deux besoins doivent se faire sentir dans une nation où l’on connaît le prix des bonnes mœurs. Le désir de mériter l’estime publique, et il en est une pour chaque profession, est un frein nécessaire à la passion des richesses. Il faut avoir comment ces deux sentiments se modifient dans la classe privilégiée.\par
D’abord, {\itshape l’honneur} lui est assuré : c’est sinon apanage certain. Que pour les autres citoyens l’honneur soit le prix de la conduite, à la bonne heure ; mais aux privilégiés, il a suffi de naître. Ce n’est pas à eux à sentir le besoin de l’acquérir, et ils peuvent renoncer d’avance à tout ce qui tend à le mériter\footnote{On doit d’apercevoir que nous ne confondons pas ici l’honneur avec le {\itshape point d’honneur}, par lequel on a cru le remplacer.}.\par
Quant à {\itshape l’argent}, les privilégiés, il est vrai, doivent en sentir vivement le besoin. Ils sont même plus exposés à se livrer aux inspirations de cette passion ardente, parce que le préjugé de leur supériorité les excite sans cesse à forcer leur dépense, et parce qu’en s’y livrant ils n’ont pas à craindre, comme les autres, de perdre tout honneur, toute considération.\par
Mais, par une contradiction bizarre, en même temps que le préjugé d’état pousse continuellement le privilégié à déranger sa fortune, il lui interdit impérieusement presque toutes les voies honnêtes par où il pourrait parvenir à la réparer.\par
Quel moyen restera-t-il donc aux privilégiés pour satisfaire ces amours de l’argent qui doit les dominer plus que les autres ? {\itshape l’intrigue} et {\itshape la mendicité}. L’intrigue et la mendicité deviendront {\itshape l’industrie} particulière de cette classe de citoyens : ils sembleront, en quelque sorte, par deux professions, reprendre une place dans l’ensemble des travaux de la société. S’y attachant exclusivement, ils s’y excelleront ; ainsi, partout où ce double talent pourra s’exercer avec fruit, soyez sûr qu’ils s’établiront de manière à écarter toute concurrence de la part des non-privilégiés.\par
Ils remplissent la cour, ils assiégeront les ministres, ils accapareront toutes les grâces, toutes les pensions, tous les bénéfices. {\itshape L’intrigue} jette à la fois un regard usurpateur sur l’église, la robe et l’épée. Elle découvre un revenu considérable ou un pouvoir qui y mène, attaché à une multitude innombrable de places, et bientôt elle parvient à faire considérer ces places comme des postes à argent, établis, non pour remplir des fonctions qui exigent des talents, mais pour assurer un état {\itshape convenable} à des familles privilégiées.\par
Ces hommes habiles ne se rassureront pas sur leur supériorité dans l’art de l’intrigue ; comme s’ils craignaient que l’amour du bien public ne vînt, dans des moments de distraction, à séduire le ministère, ils profiteront à propos de l’ineptie ou de la trahison de quelques administrateurs ; ils feront enfin consacrer leur monopole par de bonnes ordonnances, ou par un régime d’administration équivalent à une loi exclusive.\par
C’est ainsi qu’on dévoue l’État aux principes les plus destructeurs de toute économie publique. Elle a beau prescrire de préférer en toutes choses les serviteurs les plus habiles et les moins chers, le monopole commande de choisir les plus coûteux, et nécessairement les moins habiles, puisque le monopole a pour effet connu d’arrêter l’essor de ceux qui auraient pu montrer des talents dans une concurrence libre.\par
{\itshape La mendicité} privilégiée a moins d’inconvénients pour la chose publique : c’est une branche gourmande qui attire le plus de sève qu’elle peut ; mais au moins elle ne prétend pas à remplacer les rameaux utiles ; elle consiste, comme toute autre mendicité, à tendre la main en s’efforçant d’exciter la compassion et à recevoir gratuitement ; seulement la posture est moins humiliante ; elle semble, quand il le faut, dicter un devoir, plutôt qu’implorer un secours.\par
Au reste, il a suffi pour l’opinion que l’intrigue et la mendicité dont il s’agit ici fussent spécialement affectées à la classe privilégiée, pour qu’elles devinssent honorables et honorées, chacun est bien venu à se vanter hautement de ses succès en ce genre ; ils inspirent l’envie, l’émulation, jamais le mépris.\par
Ce genre de mendicité s’exerce principalement à la cour, où les hommes les plus puissants et les plus opulents en tirent le premier et le plus grand parti.\par
De là cet exemple fécond va ranimer jusque dans le fond le plus reculé des provinces la prétention honorable de vivre dans l’oisiveté, et aux dépens du public.\par
Ce n’est pas que l’ordre privilégié ne soit déjà, et sans aucune espèce de comparaison, le plus riche du royaume, que presque toutes les terres et les grandes fortunes n’appartiennent aux membres de cette classe ; mais le goût de la dépense et le plaisir de se ruiner sont supérieurs à toute richesse ; et il faut enfin qu’il y ait de pauvres privilégiés.\par
Mais à peine on entend le mot de {\itshape pauvre} s’unir à celui de {\itshape privilégié}, qu’il s’élève partout comme un cri d’indignation. Un privilégié hors d’état de soutenir d’indignation. Un privilégié hors d’état de soutenir son nom, son rang, est certes une honte pour la nation ; il faut se hâter de remédier à ce désordre public ; et, quoiqu’on ne demande pas expressément pour cela un excédent de contribution, il est bien clair que tout emploi des deniers publics ne peut avoir d’autre origine.\par
Ce n’est pas vainement que l’administration est composée de privilégiés ; elle veille avec une tendresse paternelle à tous leurs intérêts. Ici, ce sont des établissements pompeux, vantés, comme l’on croit, de toute l’Europe, pour donner l’éducation aux pauvres privilégiés de l’un et l’autre sexe. Inutilement le hasard se montrait plus sage que vos institutions, et voulait ramener ceux qui ont besoin à la loi commune de travailler pour vivre. Vous ne voyez dans ce retour au bon ordre qu’un crime de la fortune, et vous vous gardez bien de donner à vos élèves les habitudes d’une profession laborieuse, capable de faire vivre celui qui l’exerce.\par
Dans vos admirables desseins, vous allez jusqu’à leur inspirer une sorte d’orgueil d’avoir été de si bonne heure à charge au public ; comme si, dans aucun cas, il pouvait être plus glorieux de recevoir la charité que de n’en avoir pas besoin.\par
Vous les récompensez encore par des secours d’argent, par des pensions, par des cordons, d’avoir été exposés à goûter ce premier gage de votre tendresse.\par
À peine sortis de l’enfance, les jeunes privilégiés ont déjà un état et des appointements ; et on ose les plaindre de leur modicité ! Voyez cependant parmi les non-privilégiés du même âge ; qui se destinent aux professions pour lesquelles il faut des talents et de l’étude, voyez s’il en est un seul qui, bien qu’attaché à des occupations vraiment pénibles, ne coûte longtemps encore à ses parents de grandes avances, avant qu’il soit admis à la chance incertaine de retirer de ses longs travaux le nécessaire de la vie.\par
Toutes les portes sont ouvertes à la sollicitation des privilégiés ; il leur suffit de se montrer, et tout le monde se fait honneur de s’intéresser à leur avancement. On s’occupe avec chaleur de leurs affaires, de leur fortune. L’État lui-même, oui, la chose publique mille fois a concouru secrètement à leurs arrangements de famille.\par
On l’a mêlée dans des négociations particulières de mariage ; l’administration s’est prêtée à des créations de places, à des échanges ruineux, ou même à des acquisitions dont le trésor public a été forcé de fournir les fonds ; etc., etc.\par
Les privilégiés qui ne peuvent atteindre à ces hautes faveurs trouvent ailleurs d’abondantes ressources. Une foule de chapitres pour l’un et l’autre sexe, des ordres militaires sans leur offrent des prébendes, des commanderies, des pensions et toujours des décorations. Et comme si ce n’était pas assez des fautes de nos pères, on s’occupe avec un renouvellement d’ardeur, depuis quelques années, d’augmenter le nombre de ces brillantes soldes de l’inutilité\footnote{Il se manifeste une étrange contradiction dans la conduite du gouvernement. Il aide, d’un côté, à déclamer sans mesure contre les biens consacrés au culte, et qui dispensent au moins le trésor national de payer cette partie des fonctions publiques ; et il cherche en même temps à dévouer le plus qu’il peut de ces biens, et d’autres, à la classe des privilégiés sans fonctions. Il est curieux de lire la liste des chapitres nouvellement créés, ou divertis à l’usage des privilégiés de l’un et l’autre sexe ; plus curieux encore de connaître les motifs secrets qui ont porté à manquer ainsi sans pudeur au véritable esprit des fondations ecclésiastiques qui, si elles doivent être modifiées, ne doivent l’être au moins que pour un intérêt vraiment national, et par la nation seule.}.\par
Ce serait une erreur de croire que la mendicité privilégiée dédaigne les petites occasions ou les petits secours. Les fonds destinés aux aumônes du roi sont en grande partie absorbés par elle ; et pour se dire pauvre dans l’ordre des privilégiés, on n’attend pas que la nature pâtisse, il suffit que la vanité souffre. Ainsi, la véritable indigence de toutes les classes de citoyens est sacrifiée à des besoins de vanité.\par
En remontant un peu avant dans l’histoire, on voit les privilégiés dans l’usage de ravir et de s’attribuer tout ce qui peut leur convenir. La violence et la rapine, sûres de l’impunité ; pouvaient sans doute se passer de mendier ; ainsi, la mendicité privilégiée n’a dû commencer qu’avec les premiers rayons de l’ordre public, ce qui prouve sa grande différence d’avec la mendicité du peuple. Celle-ci se manifeste à mesure que le gouvernement se gâte, l’autre à mesure qu’il s’améliore. Il est vrai qu’avec quelques progrès de plus, il fera cesser à la fois ces deux maladies sociales ; mais certes, ce ne sera pas en les alimentant, ni surtout en faisant honorer celle de deux qui est la plus inexcusable.\par
On ne peut disconvenir qu’il n’y ait une prodigieuse habileté à dérober à la compassion ce qu’on ne peut plus arracher à la faiblesse ; à mettre ainsi à profit tantôt l’audace de l’oppresseur, tantôt la sensibilité de l’opprimé. La classe privilégiée, à cet égard, a su se distinguer de l’une et de l’autre manière. Du moment qu’elle n’a plus réussi à prendre de force, elle s’est hâtée, en toute occasion, de se recommander à la libéralité du roi et de la nation.\par
Les cahiers des anciens états généraux, ceux des anciennes assemblées de notables, sont pleins de demandes en faveur de la {\itshape pauvre classe privilégiée}\footnote{Aujourd’hui que les principes de justice générale sont plus répandus, et que les assemblées de bailliages auront de si grands objets à traiter, on peut espérer sans doute qu’elles ne saliront pas leurs cahiers de ce qu’on pourrait appeler autrefois {\itshape le couplet du mendiant}.}. Les Pays d’États s’occupent depuis longtemps, et toujours avec un zèle nouveau, de tout ce qui peut accroître le nombre des pensions qu’ils ont su attribuer {\itshape à la pauvre classe privilégiée}. Les administrations provinciales suivent déjà de si nobles traces, et les trois ordres en commun, parce qu’ils ne sont encore composés que de privilégiés, écoutent avec une respectueuse approbation tous les avis qui peuvent tendre à soulager {\itshape la pauvre classe privilégiée}. Les intendants se sont procuré des fonds particuliers pour cet objet ; un moyen de succès pour eux est de prendre un vif intérêt au triste sort de la pauvre classe privilégiée ; enfin, dans les livres, dans les chaires, dans les discours académiques, dans les conversations, et partout, voulez-vous intéresser à l’instant tous vos auditeurs ? il n’y a qu’à parler de {\itshape la pauvre classe privilégiée}. Avoir cette pente générale des esprits, et les innombrables moyens que la superstition, à qui rien n’est impossible, s’est déjà ménagés pour secourir les pauvres privilégiés, en vérité, je ne puis m’explique pourquoi on n’a pas encore ajouté à la porte des églises, s’il n’existe déjà, un tronc pour {\itshape la pauvre classe privilégiée}\footnote{Je m’attends bien que l’on trouvera cet endroit de mauvais ton. Cela doit être : le pouvoir de proscrire, sur ce prétexte, des expressions exactes, souvent même énergiques, est encore un droit des privilégiés.}.\par
Il faut encore citer ici un genre de trafic inépuisable en richesses pour les privilégiés. Il est fondé, d’une part sur la superstition des noms ; de l’autre, sur une cupidité plus puissante encore que la vanité. Je parle de ce qu’on ose appeler des {\itshape mésalliances}\footnote{On devrait bien, ne fût-ce que pour la clarté du langage, se servir d’un autre mot pour désigner l’action de tendre la main aux riches offrandes de la sottise : il faudrait un mot qui marquât clairement aussi de quel côté est la {\itshape mésalliance}.}, sans que ce terme ait pu décourager les stupides citoyens qui payent si cher pour se faire insulter.\par
Dès qu’à force de travail et d’industrie, quelqu’un de l’ordre commun a élevé une fortune digne d’envie ; dès que les agents du fisc, par des moyens plus faciles, sont parvenus à entasser des trésors, toutes ces richesses sont aspirées par les privilégiés. Il semble que notre malheureuse nation soit condamnée à travailler et à s’appauvrir sans esse pour la classe privilégiée.\par
Inutilement l’agriculture, les fabriques, le commerce, et tous les arts réclament-ils, pour se soutenir, pour s’agrandir, et pour la prospérité publique, une partie des capitaux immenses qu’ils ont servi à former ; les privilégiés engloutissent et les capitaux et les personnes ; tout est voué sans retour à la stérilité privilégiée\footnote{Si l’{\itshape honneur} est, comme l’on dit, le {\itshape principe} de la monarchie, il faut convenir au moins que la France fait depuis longtemps de terribles sacrifices pour se fortifier en {\itshape principe}.}.\par
La matière des privilèges est inépuisable comme les préjugés qui conspirent à les soutenir. Mais laissons ce sujet, et épargnons-nous les réflexions qu’il inspire. Un temps viendra où nos neveux indignés resteront stupéfaits à la lecture de notre histoire, et donneront à la plus inconvenable démence les noms qu’elle mérite. Nous avions vu, dans notre jeunesse, des hommes de lettres se signaler par leur courage à attaquer des opinions aussi puissantes que pernicieuses à l’humanité. Aujourd’hui, leurs propos et dans leurs préjugés qui n’existent plus. Le préjugé qui soutient les privilèges est le plus funeste qui ait affligé la terre ; il s’est plus intimement lié avec l’organisation sociale ; il la corrompt plus profondément ; plus d’intérêts s’occupent à le défendre. Que de motifs pour exciter le zèle des vrais patriotes, et pour refroidir celui des gens de lettres nos contemporains !
\section[{Note relative à la note 3}]{Note relative à la note 3}\phantomsection
\label{essai\_privileges\_note\_3}\renewcommand{\leftmark}{Note relative à la note 3}

\noindent Extrait du procès-verbal de la Noblesse aux États de 1614, page 113.\par
{\itshape Du mardi 25 Novembre : « Et ayant eu audience, M. de Senecey}\footnote{M. le baron de Senecey était président de la noblesse.}{\itshape  parla au Roi en cette sorte} : \par
\par
Sire,\par
« La bonté de nos a concédé de tout temps cette liberté à leur noblesse, que de recourir à eux en toutes sortes d’occasions, l’éminence de leur qualité les ayant approchés auprès de leurs personnes, qu’ils ont toujours été les principaux exécuteurs de leurs royales actions.\par
« Je n’aurais jamais fait de rapporteur à V. M out ce que l’antiquité nous apprend que la naissance a donné de prééminences à cet ordre, et avec telle différence de ce qui est de tout le reste du peuple, qu’elle n’en a jamais pu souffrir aucune sorte de comparaison. Je pourrais, Sire, m’étendre en ce discours ; mais une vérité si claire n’a pas besoin de témoignage plus certain que ce qui est connu de tout le monde… ; et puis je parle devant le roi ; lequel, nous espérons trouver aussi jaloux de nous conserver en ce que nous participions de son lustre, que nous saurions l’être de l’en requérir et supplier, bien marris qu’une nouveauté extraordinaire nous ouvre la bouche plutôt aux plaintes qu’aux très humbles supplications pour lesquelles nous sommes assemblés.\par
« Sire, votre majesté a eu pour agréable de convoquer les états généraux des trois ordres de votre royaume, ordres destinés et séparés entre eux de fonctions et de qualités. L’église, vouée au service de Dieu et au régime des âmes, y tient le premier rang ; nous honorons les prélats et ministres come nos pères, et comme médiateurs de notre réconciliation avec Dieu.\par
« La noblesse, Sire, y tient le second rang. Elle est le bras droit de votre justice, le soutien de votre couronne, et les forces invicibles de l’État.\par
« Sous les heureux auspices et valeureuse conduite des rois, au prix de leur sang, et par l’emploi de leurs armes victorieuses, la tranquillité publique a été établie, et par leurs peines et travaux, le Tiers-État va jouissant des commodités que la paix leur apporte.\par
« Cet ordre, Sire, qui tie\phantomsection
\label{\_GoBack}nt le dernier rang en cette assemblée, ordre composé du peuple, des villes et des champs, ces derniers sont quasi tous hommagers et justiciables des deux premiers ordres ; ceux des villes, bourgeois, marchands, artisans, et quelques officiers. Ce sont ceux-ci qui méconnaissent leur condition, et oubliant toute sortes de devoirs, sans aveu de ceux qu’ils représentent, se veulent comparer à nous.\par
« J’ai honte, Sire, de vous dire les termes qui de nouveau nous ont offensés. Ils comparent votre État à une famille composée de trois frères. Ils disent l’ordre ecclésiastique être l’aîné, le nôtre le puîné, {\itshape et eux les cadets}\footnote{Telle est l’injure dont la noblesse demande vengeance. La veille, le lieutenant civil, à la tête d’une députation du Tiers-État, avait osé dire : « Traitez-nous comme vos frères cadet, et nous vous honorerons et aimerons. » Toute cette tracasserie doit être lue dans le procès-verbal même, à commencer par le discours du président Savaron, qui en fut le prétexte. On trouvera dans la réponse du baron de Senecey à la députation du Tiers, du 24 novembre, des expressions plus outrageantes encore que celles qui remplissent le discours du Roi.}.\par
« En quelle misérable condition sommes-nous tombés, si cette parole est véritable ! En quoi tant de service rendus d’un temps immémorial, tant d’honneurs et de dignités transmis héréditairement à la noblesse, et mérités par leurs labeurs et fidélité, l’auraient-elle bien, au lieu de l’élever, tellement rabaissée, qu’elle fût avec le vulgaire en la plus étroite sorte de société qui soit parmi les hommes, qui est la fraternité ! Et non contents de se dire frères, ils s’attribuent la restauration de l’État à quoi, comme la France sait assez qu’ils n’ont aucunement participé, aussi chacun connaît qu’ils ne peuvent en aucune façon se comparer à nous, et serait insupportable une entreprise si mal fondée.\par
« Rendez, Sire, le jugement, et par une déclaration pleine de justice, faites-les mettre en leurs devoirs, et reconnaître ce que nous sommes, et la différence qu’il y a. Nous en supplions très humblement votre majesté au nom de toute la noblesse de France, puisque c’est d’elle que nous sommes ici députés, afin que, conservée en ses prééminences, elle porte comme elle a toujours fait, son honneur et sa vie au service de Votre Majesté. »\par
\par
 {\itshape Ecquid sentitis in quanto contemptu vivatis ? Lucis vobis hujus partem, si liceat, adimant. Quod spiratis, quod vocem mittitis, quod formas hominum habetis indignatur.} \par
{\itshape Liv. lib.} 4, {\itshape c.} 56.
 


% at least one empty page at end (for booklet couv)
\ifbooklet
  \pagestyle{empty}
  \clearpage
  % 2 empty pages maybe needed for 4e cover
  \ifnum\modulo{\value{page}}{4}=0 \hbox{}\newpage\hbox{}\newpage\fi
  \ifnum\modulo{\value{page}}{4}=1 \hbox{}\newpage\hbox{}\newpage\fi


  \hbox{}\newpage
  \ifodd\value{page}\hbox{}\newpage\fi
  {\centering\color{rubric}\bfseries\noindent\large
    Hurlus ? Qu’est-ce.\par
    \bigskip
  }
  \noindent Des bouquinistes électroniques, pour du texte libre à participation libre,
  téléchargeable gratuitement sur \href{https://hurlus.fr}{\dotuline{hurlus.fr}}.\par
  \bigskip
  \noindent Cette brochure a été produite par des éditeurs bénévoles.
  Elle n’est pas faîte pour être possédée, mais pour être lue, et puis donnée.
  Que circule le texte !
  En page de garde, on peut ajouter une date, un lieu, un nom ; pour suivre le voyage des idées.
  \par

  Ce texte a été choisi parce qu’une personne l’a aimé,
  ou haï, elle a en tous cas pensé qu’il partipait à la formation de notre présent ;
  sans le souci de plaire, vendre, ou militer pour une cause.
  \par

  L’édition électronique est soigneuse, tant sur la technique
  que sur l’établissement du texte ; mais sans aucune prétention scolaire, au contraire.
  Le but est de s’adresser à tous, sans distinction de science ou de diplôme.
  Au plus direct ! (possible)
  \par

  Cet exemplaire en papier a été tiré sur une imprimante personnelle
   ou une photocopieuse. Tout le monde peut le faire.
  Il suffit de
  télécharger un fichier sur \href{https://hurlus.fr}{\dotuline{hurlus.fr}},
  d’imprimer, et agrafer ; puis de lire et donner.\par

  \bigskip

  \noindent PS : Les hurlus furent aussi des rebelles protestants qui cassaient les statues dans les églises catholiques. En 1566 démarra la révolte des gueux dans le pays de Lille. L’insurrection enflamma la région jusqu’à Anvers où les gueux de mer bloquèrent les bateaux espagnols.
  Ce fut une rare guerre de libération dont naquit un pays toujours libre : les Pays-Bas.
  En plat pays francophone, par contre, restèrent des bandes de huguenots, les hurlus, progressivement réprimés par la très catholique Espagne.
  Cette mémoire d’une défaite est éteinte, rallumons-la. Sortons les livres du culte universitaire, cherchons les idoles de l’époque, pour les briser.
\fi

\ifdev % autotext in dev mode
\fontname\font — \textsc{Les règles du jeu}\par
(\hyperref[utopie]{\underline{Lien}})\par
\noindent \initialiv{A}{lors là}\blindtext\par
\noindent \initialiv{À}{ la bonheur des dames}\blindtext\par
\noindent \initialiv{É}{tonnez-le}\blindtext\par
\noindent \initialiv{Q}{ualitativement}\blindtext\par
\noindent \initialiv{V}{aloriser}\blindtext\par
\Blindtext
\phantomsection
\label{utopie}
\Blinddocument
\fi
\end{document}
