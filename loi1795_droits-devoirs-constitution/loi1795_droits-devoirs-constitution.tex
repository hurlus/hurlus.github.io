%%%%%%%%%%%%%%%%%%%%%%%%%%%%%%%%%
% LaTeX model https://hurlus.fr %
%%%%%%%%%%%%%%%%%%%%%%%%%%%%%%%%%

% Needed before document class
\RequirePackage{pdftexcmds} % needed for tests expressions
\RequirePackage{fix-cm} % correct units

% Define mode
\def\mode{a4}

\newif\ifaiv % a4
\newif\ifav % a5
\newif\ifbooklet % booklet
\newif\ifcover % cover for booklet

\ifnum \strcmp{\mode}{cover}=0
  \covertrue
\else\ifnum \strcmp{\mode}{booklet}=0
  \booklettrue
\else\ifnum \strcmp{\mode}{a5}=0
  \avtrue
\else
  \aivtrue
\fi\fi\fi

\ifbooklet % do not enclose with {}
  \documentclass[french,twoside]{book} % ,notitlepage
  \usepackage[%
    papersize={105mm, 297mm},
    inner=12mm,
    outer=12mm,
    top=20mm,
    bottom=15mm,
    marginparsep=0pt,
  ]{geometry}
  \usepackage[fontsize=9.5pt]{scrextend} % for Roboto
\else\ifav
  \documentclass[french,twoside]{book} % ,notitlepage
  \usepackage[%
    a5paper,
    inner=25mm,
    outer=15mm,
    top=15mm,
    bottom=15mm,
    marginparsep=0pt,
  ]{geometry}
  \usepackage[fontsize=12pt]{scrextend}
\else% A4 2 cols
  \documentclass[twocolumn]{report}
  \usepackage[%
    a4paper,
    inner=15mm,
    outer=10mm,
    top=25mm,
    bottom=18mm,
    marginparsep=0pt,
  ]{geometry}
  \setlength{\columnsep}{20mm}
  \usepackage[fontsize=9.5pt]{scrextend}
\fi\fi

%%%%%%%%%%%%%%
% Alignments %
%%%%%%%%%%%%%%
% before teinte macros

\setlength{\arrayrulewidth}{0.2pt}
\setlength{\columnseprule}{\arrayrulewidth} % twocol
\setlength{\parskip}{0pt} % 1pt allow better vertical justification
\setlength{\parindent}{1.5em}

%%%%%%%%%%
% Colors %
%%%%%%%%%%
% before Teinte macros

\usepackage[dvipsnames]{xcolor}
\definecolor{rubric}{HTML}{800000} % the tonic 0c71c3
\def\columnseprulecolor{\color{rubric}}
\colorlet{borderline}{rubric!30!} % definecolor need exact code
\definecolor{shadecolor}{gray}{0.95}
\definecolor{bghi}{gray}{0.5}

%%%%%%%%%%%%%%%%%
% Teinte macros %
%%%%%%%%%%%%%%%%%
%%%%%%%%%%%%%%%%%%%%%%%%%%%%%%%%%%%%%%%%%%%%%%%%%%%
% <TEI> generic (LaTeX names generated by Teinte) %
%%%%%%%%%%%%%%%%%%%%%%%%%%%%%%%%%%%%%%%%%%%%%%%%%%%
% This template is inserted in a specific design
% It is XeLaTeX and otf fonts

\makeatletter % <@@@


\usepackage{blindtext} % generate text for testing
\usepackage[strict]{changepage} % for modulo 4
\usepackage{contour} % rounding words
\usepackage[nodayofweek]{datetime}
\usepackage{enumitem} % <list>
\usepackage{epigraph} % <epigraph>
\usepackage{etoolbox} % patch commands
\usepackage{fancyvrb}
\usepackage{fancyhdr}
\usepackage{float}
\usepackage{fontspec} % XeLaTeX mandatory for fonts
\usepackage{footnote} % used to capture notes in minipage (ex: quote)
\usepackage{framed} % bordering correct with footnote hack
\usepackage{graphicx}
\usepackage{lettrine} % drop caps
\usepackage{lipsum} % generate text for testing
\usepackage[framemethod=tikz,]{mdframed} % maybe used for frame with footnotes inside
\usepackage{pdftexcmds} % needed for tests expressions
\usepackage{polyglossia} % bug Warning: "Failed to patch part"
\usepackage[%
  indentfirst=false,
  vskip=1em,
  noorphanfirst=true,
  noorphanafter=true,
  leftmargin=\parindent,
  rightmargin=0pt,
]{quoting}
\usepackage{ragged2e}
\usepackage{setspace} % \setstretch for <quote>
\usepackage{tabularx} % <table>
\usepackage[explicit]{titlesec} % wear titles, !NO implicit
\usepackage{tikz} % ornaments
\usepackage{tocloft} % styling tocs
\usepackage[fit]{truncate} % used im runing titles
\usepackage{unicode-math}
\usepackage[normalem]{ulem} % breakable \uline, normalem is absolutely necessary to keep \emph
\usepackage{verse} % <l>
\usepackage{xcolor} % named colors
\usepackage{xparse} % @ifundefined
\XeTeXdefaultencoding "iso-8859-1" % bad encoding of xstring
\usepackage{xstring} % string tests
\XeTeXdefaultencoding "utf-8"
\PassOptionsToPackage{hyphens}{url} % before hyperref, which load url package

% TOTEST
% \usepackage{hypcap} % links in caption ?
% \usepackage{marginnote}
% TESTED
% \usepackage{background} % doesn’t work with xetek
% \usepackage{bookmark} % prefers the hyperref hack \phantomsection
% \usepackage[color, leftbars]{changebar} % 2 cols doc, impossible to keep bar left
% \usepackage{DejaVuSans} % override too much, was for for symbols
% \usepackage[utf8x]{inputenc} % inputenc package ignored with utf8 based engines
% \usepackage[sfdefault,medium]{inter} % no small caps
% \usepackage{firamath} % choose firasans instead, firamath unavailable in Ubuntu 21-04
% \usepackage{flushend} % bad for last notes, supposed flush end of columns
% \usepackage[stable]{footmisc} % BAD for complex notes https://texfaq.org/FAQ-ftnsect
% \usepackage{helvet} % not for XeLaTeX
% \usepackage{multicol} % not compatible with too much packages (longtable, framed, memoir…)
% \usepackage[default,oldstyle,scale=0.95]{opensans} % no small caps
% \usepackage{sectsty} % \chapterfont OBSOLETE
% \usepackage{soul} % \ul for underline, OBSOLETE with XeTeX
% \usepackage[breakable]{tcolorbox} % text styling gone, footnote hack not kept with breakable

\defaultfontfeatures{
  % Mapping=tex-text, % no effect seen
  Scale=MatchLowercase,
  Ligatures={TeX,Common},
}
\newfontfamily\zhfont{Noto Sans CJK SC}

% Metadata inserted by a program, from the TEI source, for title page and runing heads
\title{\textbf{ Déclaration des droits et des devoirs de l’homme et du citoyen }\par
\medskip
\textit{ Constitution du 5 fructidor an III }\par
}
\date{1795}
\author{Première République}
\def\elbibl{Première République. 1795. \emph{Déclaration des droits et des devoirs de l’homme et du citoyen}}
\def\elsource{\{source\}}

% Default metas
\newcommand{\colorprovide}[2]{\@ifundefinedcolor{#1}{\colorlet{#1}{#2}}{}}
\colorprovide{rubric}{red}
\colorprovide{silver}{lightgray}
\@ifundefined{syms}{\newfontfamily\syms{DejaVu Sans}}{}
\newif\ifdev
\@ifundefined{elbibl}{% No meta defined, maybe dev mode
  \newcommand{\elbibl}{Titre court ?}
  \newcommand{\elbook}{Titre du livre source ?}
  \newcommand{\elabstract}{Résumé\par}
  \newcommand{\elurl}{http://oeuvres.github.io/elbook/2}
  \author{Éric Lœchien}
  \title{Un titre de test assez long pour vérifier le comportement d’une maquette}
  \date{1566}
  \devtrue
}{}
\let\eltitle\@title
\let\elauthor\@author
\let\eldate\@date




% generic typo commands
\newcommand{\astermono}{\medskip\centerline{\color{rubric}\large\selectfont{\syms ✻}}\medskip\par}%
\newcommand{\astertri}{\medskip\par\centerline{\color{rubric}\large\selectfont{\syms ✻\,✻\,✻}}\medskip\par}%
\newcommand{\asterism}{\bigskip\par\noindent\parbox{\linewidth}{\centering\color{rubric}\large{\syms ✻}\\{\syms ✻}\hskip 0.75em{\syms ✻}}\bigskip\par}%

% lists
\newlength{\listmod}
\setlength{\listmod}{\parindent}
\setlist{
  itemindent=!,
  listparindent=\listmod,
  labelsep=0.2\listmod,
  parsep=0pt,
  % topsep=0.2em, % default topsep is best
}
\setlist[itemize]{
  label=—,
  leftmargin=0pt,
  labelindent=1.2em,
  labelwidth=0pt,
}
\setlist[enumerate]{
  label={\arabic*°},
  labelindent=0.8\listmod,
  leftmargin=\listmod,
  labelwidth=0pt,
}
% list for big items
\newlist{decbig}{enumerate}{1}
\setlist[decbig]{
  label={\bf\color{rubric}\arabic*.},
  labelindent=0.8\listmod,
  leftmargin=\listmod,
  labelwidth=0pt,
}
\newlist{listalpha}{enumerate}{1}
\setlist[listalpha]{
  label={\bf\color{rubric}\alph*.},
  leftmargin=0pt,
  labelindent=0.8\listmod,
  labelwidth=0pt,
}
\newcommand{\listhead}[1]{\hspace{-1\listmod}\emph{#1}}

\renewcommand{\hrulefill}{%
  \leavevmode\leaders\hrule height 0.2pt\hfill\kern\z@}

% General typo
\DeclareTextFontCommand{\textlarge}{\large}
\DeclareTextFontCommand{\textsmall}{\small}

% commands, inlines
\newcommand{\anchor}[1]{\Hy@raisedlink{\hypertarget{#1}{}}} % link to top of an anchor (not baseline)
\newcommand\abbr[1]{#1}
\newcommand{\autour}[1]{\tikz[baseline=(X.base)]\node [draw=rubric,thin,rectangle,inner sep=1.5pt, rounded corners=3pt] (X) {\color{rubric}#1};}
\newcommand\corr[1]{#1}
\newcommand{\ed}[1]{ {\color{silver}\sffamily\footnotesize (#1)} } % <milestone ed="1688"/>
\newcommand\expan[1]{#1}
\newcommand\foreign[1]{\emph{#1}}
\newcommand\gap[1]{#1}
\renewcommand{\LettrineFontHook}{\color{rubric}}
\newcommand{\initial}[2]{\lettrine[lines=2, loversize=0.3, lhang=0.3]{#1}{#2}}
\newcommand{\initialiv}[2]{%
  \let\oldLFH\LettrineFontHook
  % \renewcommand{\LettrineFontHook}{\color{rubric}\ttfamily}
  \IfSubStr{QJ’}{#1}{
    \lettrine[lines=4, lhang=0.2, loversize=-0.1, lraise=0.2]{\smash{#1}}{#2}
  }{\IfSubStr{É}{#1}{
    \lettrine[lines=4, lhang=0.2, loversize=-0, lraise=0]{\smash{#1}}{#2}
  }{\IfSubStr{ÀÂ}{#1}{
    \lettrine[lines=4, lhang=0.2, loversize=-0, lraise=0, slope=0.6em]{\smash{#1}}{#2}
  }{\IfSubStr{A}{#1}{
    \lettrine[lines=4, lhang=0.2, loversize=0.2, slope=0.6em]{\smash{#1}}{#2}
  }{\IfSubStr{V}{#1}{
    \lettrine[lines=4, lhang=0.2, loversize=0.2, slope=-0.5em]{\smash{#1}}{#2}
  }{
    \lettrine[lines=4, lhang=0.2, loversize=0.2]{\smash{#1}}{#2}
  }}}}}
  \let\LettrineFontHook\oldLFH
}
\newcommand{\labelchar}[1]{\textbf{\color{rubric} #1}}
\newcommand{\milestone}[1]{\autour{\footnotesize\color{rubric} #1}} % <milestone n="4"/>
\newcommand\name[1]{#1}
\newcommand\orig[1]{#1}
\newcommand\orgName[1]{#1}
\newcommand\persName[1]{#1}
\newcommand\placeName[1]{#1}
\newcommand{\pn}[1]{\IfSubStr{-—–¶}{#1}% <p n="3"/>
  {\noindent{\bfseries\color{rubric}   ¶  }}
  {{\footnotesize\autour{ #1}  }}}
\newcommand\reg{}
% \newcommand\ref{} % already defined
\newcommand\sic[1]{#1}
\newcommand\surname[1]{\textsc{#1}}
\newcommand\term[1]{\textbf{#1}}
\newcommand\zh[1]{{\zhfont #1}}


\def\mednobreak{\ifdim\lastskip<\medskipamount
  \removelastskip\nopagebreak\medskip\fi}
\def\bignobreak{\ifdim\lastskip<\bigskipamount
  \removelastskip\nopagebreak\bigskip\fi}

% commands, blocks
\newcommand{\byline}[1]{\bigskip{\RaggedLeft{#1}\par}\bigskip}
\newcommand{\bibl}[1]{{\RaggedLeft{#1}\par\bigskip}}
\newcommand{\biblitem}[1]{{\noindent\hangindent=\parindent   #1\par}}
\newcommand{\dateline}[1]{\medskip{\RaggedLeft{#1}\par}\bigskip}
\newcommand{\labelblock}[1]{\medbreak{\noindent\color{rubric}\bfseries #1}\par\mednobreak}
\newcommand{\salute}[1]{\bigbreak{#1}\par\medbreak}
\newcommand{\signed}[1]{\medskip{\raggedleft #1\par}\bigbreak} % supposed bottom


% environments for blocks (some may become commands)
\newenvironment{borderbox}{}{} % framing content
\newenvironment{citbibl}{\ifvmode\hfill\fi}{\ifvmode\par\fi }
\newenvironment{docAuthor}{\ifvmode\vskip4pt\fontsize{16pt}{18pt}\selectfont\fi\itshape}{\ifvmode\par\fi }
\newenvironment{docDate}{}{\ifvmode\par\fi }
\newenvironment{docImprint}{\vskip6pt}{\ifvmode\par\fi }
\newenvironment{docTitle}{\vskip6pt\bfseries\fontsize{18pt}{22pt}\selectfont}{\par }
\newenvironment{msHead}{\vskip6pt}{\par}
\newenvironment{msItem}{\vskip6pt}{\par}
\newenvironment{titlePart}{}{\par }


% environments for block containers
\newenvironment{argument}{\itshape\parindent0pt}{\bigskip}
\newenvironment{biblfree}{}{\ifvmode\par\fi }
\newenvironment{bibitemlist}[1]{%
  \list{\@biblabel{\@arabic\c@enumiv}}%
  {%
    \settowidth\labelwidth{\@biblabel{#1}}%
    \leftmargin\labelwidth
    \advance\leftmargin\labelsep
    \@openbib@code
    \usecounter{enumiv}%
    \let\p@enumiv\@empty
    \renewcommand\theenumiv{\@arabic\c@enumiv}%
  }
  \sloppy
  \clubpenalty4000
  \@clubpenalty \clubpenalty
  \widowpenalty4000%
  \sfcode`\.\@m
}%
{\def\@noitemerr
  {\@latex@warning{Empty `bibitemlist' environment}}%
\endlist}
\newenvironment{quoteblock}% may be used for ornaments
  {\begin{quoting}}
  {\end{quoting}}
% \newenvironment{epigraph}{\parindent0pt\raggedleft\it}{\bigskip}
 % epigraph pack
\setlength{\epigraphrule}{0pt}
\setlength{\epigraphwidth}{0.8\textwidth}
% \renewcommand{\epigraphflush}{center} ? dont work

% table () is preceded and finished by custom command
\newcommand{\tableopen}[1]{%
  \ifnum\strcmp{#1}{wide}=0{%
    \begin{center}
  }
  \else\ifnum\strcmp{#1}{long}=0{%
    \begin{center}
  }
  \else{%
    \begin{center}
  }
  \fi\fi
}
\newcommand{\tableclose}[1]{%
  \ifnum\strcmp{#1}{wide}=0{%
    \end{center}
  }
  \else\ifnum\strcmp{#1}{long}=0{%
    \end{center}
  }
  \else{%
    \end{center}
  }
  \fi\fi
}


% text structure
\newcommand\chapteropen{} % before chapter title
\newcommand\chaptercont{} % after title, argument, epigraph…
\newcommand\chapterclose{} % maybe useful for multicol settings
\setcounter{secnumdepth}{-2} % no counters for hierarchy titles
\setcounter{tocdepth}{5} % deep toc
\renewcommand\tableofcontents{\@starttoc{toc}}
% toclof format
% \renewcommand{\@tocrmarg}{0.1em} % Useless command?
% \renewcommand{\@pnumwidth}{0.5em} % {1.75em}
\renewcommand{\@cftmaketoctitle}{}
\setlength{\cftbeforesecskip}{\z@ \@plus.2\p@}
\renewcommand{\cftchapfont}{}
\renewcommand{\cftchapdotsep}{\cftdotsep}
\renewcommand{\cftchapleader}{\normalfont\cftdotfill{\cftchapdotsep}}
\renewcommand{\cftchappagefont}{\bfseries}
\setlength{\cftbeforechapskip}{0em \@plus\p@}
% \renewcommand{\cftsecfont}{\small\relax}
\renewcommand{\cftsecpagefont}{\normalfont}
% \renewcommand{\cftsubsecfont}{\small\relax}
\renewcommand{\cftsecdotsep}{\cftdotsep}
\renewcommand{\cftsecpagefont}{\normalfont}
\renewcommand{\cftsecleader}{\normalfont\cftdotfill{\cftsecdotsep}}
\setlength{\cftsecindent}{1em}
\setlength{\cftsubsecindent}{2em}
\setlength{\cftsubsubsecindent}{3em}
\setlength{\cftchapnumwidth}{1em}
\setlength{\cftsecnumwidth}{1em}
\setlength{\cftsubsecnumwidth}{1em}
\setlength{\cftsubsubsecnumwidth}{1em}

% footnotes
\newif\ifheading
\newcommand*{\fnmarkscale}{\ifheading 0.70 \else 1 \fi}
\renewcommand\footnoterule{\vspace*{0.3cm}\hrule height \arrayrulewidth width 3cm \vspace*{0.3cm}}
\setlength\footnotesep{1.5\footnotesep} % footnote separator
\renewcommand\@makefntext[1]{\parindent 1.5em \noindent \hb@xt@1.8em{\hss{\normalfont\@thefnmark . }}#1} % no superscipt in foot
\patchcmd{\@footnotetext}{\footnotesize}{\footnotesize\sffamily}{}{} % before scrextend, hyperref


%   see https://tex.stackexchange.com/a/34449/5049
\def\truncdiv#1#2{((#1-(#2-1)/2)/#2)}
\def\moduloop#1#2{(#1-\truncdiv{#1}{#2}*#2)}
\def\modulo#1#2{\number\numexpr\moduloop{#1}{#2}\relax}

% orphans and widows
\clubpenalty=9996
\widowpenalty=9999
\brokenpenalty=4991
\predisplaypenalty=10000
\postdisplaypenalty=1549
\displaywidowpenalty=1602
\hyphenpenalty=400
% Copied from Rahtz but not understood
\def\@pnumwidth{1.55em}
\def\@tocrmarg {2.55em}
\def\@dotsep{4.5}
\emergencystretch 3em
\hbadness=4000
\pretolerance=750
\tolerance=2000
\vbadness=4000
\def\Gin@extensions{.pdf,.png,.jpg,.mps,.tif}
% \renewcommand{\@cite}[1]{#1} % biblio

\usepackage{hyperref} % supposed to be the last one, :o) except for the ones to follow
\urlstyle{same} % after hyperref
\hypersetup{
  % pdftex, % no effect
  pdftitle={\elbibl},
  % pdfauthor={Your name here},
  % pdfsubject={Your subject here},
  % pdfkeywords={keyword1, keyword2},
  bookmarksnumbered=true,
  bookmarksopen=true,
  bookmarksopenlevel=1,
  pdfstartview=Fit,
  breaklinks=true, % avoid long links
  pdfpagemode=UseOutlines,    % pdf toc
  hyperfootnotes=true,
  colorlinks=false,
  pdfborder=0 0 0,
  % pdfpagelayout=TwoPageRight,
  % linktocpage=true, % NO, toc, link only on page no
}

\makeatother % /@@@>
%%%%%%%%%%%%%%
% </TEI> end %
%%%%%%%%%%%%%%


%%%%%%%%%%%%%
% footnotes %
%%%%%%%%%%%%%
\renewcommand{\thefootnote}{\bfseries\textcolor{rubric}{\arabic{footnote}}} % color for footnote marks

%%%%%%%%%
% Fonts %
%%%%%%%%%
\usepackage[]{roboto} % SmallCaps, Regular is a bit bold
% \linespread{0.90} % too compact, keep font natural
\newfontfamily\fontrun[]{Roboto Condensed Light} % condensed runing heads
\ifav
  \setmainfont[
    ItalicFont={Roboto Light Italic},
  ]{Roboto}
\else\ifbooklet
  \setmainfont[
    ItalicFont={Roboto Light Italic},
  ]{Roboto}
\else
\setmainfont[
  ItalicFont={Roboto Italic},
]{Roboto Light}
\fi\fi
\renewcommand{\LettrineFontHook}{\bfseries\color{rubric}}
% \renewenvironment{labelblock}{\begin{center}\bfseries\color{rubric}}{\end{center}}

%%%%%%%%
% MISC %
%%%%%%%%

\setdefaultlanguage[frenchpart=false]{french} % bug on part


\newenvironment{quotebar}{%
    \def\FrameCommand{{\color{rubric!10!}\vrule width 0.5em} \hspace{0.9em}}%
    \def\OuterFrameSep{2pt} % séparateur vertical
    \MakeFramed {\advance\hsize-\width \FrameRestore}
  }%
  {%
    \endMakeFramed
  }
\renewenvironment{quoteblock}% may be used for ornaments
  {%
    \savenotes
    \setstretch{0.9}
    \normalfont
    \begin{quotebar}
  }
  {%
    \end{quotebar}
    \spewnotes
  }


\renewcommand{\headrulewidth}{\arrayrulewidth}
\renewcommand{\headrule}{{\color{rubric}\hrule}}

% delicate tuning, image has produce line-height problems in title on 2 lines
\titleformat{name=\chapter} % command
  [display] % shape
  {\vspace{1.5em}\centering} % format
  {} % label
  {0pt} % separator between n
  {}
[{\color{rubric}\huge\textbf{#1}}\bigskip] % after code
% \titlespacing{command}{left spacing}{before spacing}{after spacing}[right]
\titlespacing*{\chapter}{0pt}{-2em}{0pt}[0pt]

\titleformat{name=\section}
  [display]{}{}{}{}
  [\vbox{\color{rubric}\large\raggedleft\textbf{#1}}]
\titlespacing{\section}{0pt}{0pt plus 4pt minus 2pt}{\baselineskip}

\titleformat{name=\subsection}
  [block]
  {}
  {} % \thesection
  {} % separator \arrayrulewidth
  {}
[\vbox{\large\textbf{#1}}]
% \titlespacing{\subsection}{0pt}{0pt plus 4pt minus 2pt}{\baselineskip}

\ifaiv
  \fancypagestyle{main}{%
    \fancyhf{}
    \setlength{\headheight}{1.5em}
    \fancyhead{} % reset head
    \fancyfoot{} % reset foot
    \fancyhead[L]{\truncate{0.45\headwidth}{\fontrun\elbibl}} % book ref
    \fancyhead[R]{\truncate{0.45\headwidth}{ \fontrun\nouppercase\leftmark}} % Chapter title
    \fancyhead[C]{\thepage}
  }
  \fancypagestyle{plain}{% apply to chapter
    \fancyhf{}% clear all header and footer fields
    \setlength{\headheight}{1.5em}
    \fancyhead[L]{\truncate{0.9\headwidth}{\fontrun\elbibl}}
    \fancyhead[R]{\thepage}
  }
\else
  \fancypagestyle{main}{%
    \fancyhf{}
    \setlength{\headheight}{1.5em}
    \fancyhead{} % reset head
    \fancyfoot{} % reset foot
    \fancyhead[RE]{\truncate{0.9\headwidth}{\fontrun\elbibl}} % book ref
    \fancyhead[LO]{\truncate{0.9\headwidth}{\fontrun\nouppercase\leftmark}} % Chapter title, \nouppercase needed
    \fancyhead[RO,LE]{\thepage}
  }
  \fancypagestyle{plain}{% apply to chapter
    \fancyhf{}% clear all header and footer fields
    \setlength{\headheight}{1.5em}
    \fancyhead[L]{\truncate{0.9\headwidth}{\fontrun\elbibl}}
    \fancyhead[R]{\thepage}
  }
\fi

\ifav % a5 only
  \titleclass{\section}{top}
\fi

\newcommand\chapo{{%
  \vspace*{-3em}
  \centering\parindent0pt % no vskip ()
  {\Large\addfontfeature{LetterSpace=25}\bfseries{\elauthor}\par}
  \smallskip
  {\large\eldate\par}
  \bigskip
  {\Large\eltitle}
  \bigskip
  {\color{rubric}\hline\par}
  \bigskip
  {\Large TEXTE LIBRE À PARTICIPATIONS LIBRES\par}
  \centerline{\small\color{rubric} {\href{https://hurlus.fr}{\dotuline{hurlus.fr}}}, tiré le \today}\par
  \bigskip
}}

\newcommand\cover{{%
  \thispagestyle{empty}
  \centering
  {\LARGE\addfontfeature{LetterSpace=25}\bfseries{\elauthor}\par}
  \bigskip
  {\Large\eldate\par}
  \bigskip
  \bigskip
  {\LARGE \eltitle}
  \vfill\null
  {\color{rubric}\setlength{\arrayrulewidth}{2pt}\hline\par}
  \vfill\null
  {\Large TEXTE LIBRE À PARTICIPATIONS LIBRES\par}
  \centerline{\href{https://hurlus.fr}{\dotuline{hurlus.fr}}, tiré le \today}\par
}}

\begin{document}
\pagestyle{empty}
\ifbooklet{
  \cover\newpage
  \thispagestyle{empty}\hbox{}\newpage
  \cover\newpage\noindent Les voyages de la brochure\par
  \bigskip
  \begin{tabularx}{\textwidth}{l|X|X}
    \textbf{Date} & \textbf{Lieu}& \textbf{Nom/pseudo} \\ \hline
    \rule{0pt}{25cm} &  &   \\
  \end{tabularx}
  \newpage
  \addtocounter{page}{-4}
}\fi

\thispagestyle{empty}
\ifaiv
  \twocolumn[\chapo]
\else
  \chapo
\fi
{\it\elabstract}
\bigskip
\makeatletter\@starttoc{toc}\makeatother % toc without new page
\bigskip

\pagestyle{main} % after style

  
\chapteropen

\chapter[{Déclaration des droits et des devoirs de l’homme et du citoyen}]{{\scshape Déclaration} des droits et des devoirs de l’homme et du citoyen}
\renewcommand{\leftmark}{{\scshape Déclaration} des droits et des devoirs de l’homme et du citoyen}


\chaptercont
\noindent Le peuple français proclame, en présence de l'Être suprême, la déclaration suivante des droits et des devoirs de l'homme et du citoyen.\par

\section[{Droits}]{Droits}

\labelchar{Art. 1.} Les droits de l’homme en société sont la liberté, l’égalité, la sûreté, la propriété.\par
\labelchar{Art. 2.} La liberté consiste à pouvoir faire ce qui ne nuit pas aux droits d’autrui.\par
\labelchar{Art. 3.} L’égalité consiste en ce que la loi est la même pour tous, soit qu’elle protège, soit qu’elle punisse. L’égalité n’admet aucune distinction de naissance, aucune hérédité de pouvoirs.\par
\labelchar{Art. 4.} La sûreté résulte du concours de tous pour assurer les droits de chacun.\par
\labelchar{Art. 5.} La propriété est le droit de jouir et de disposer de ses biens, de ses revenus, du fruit de son travail et de son industrie.\par
\labelchar{Art. 6.} La loi est la volonté générale, exprimée par la majorité ou des citoyens ou de leurs représentants.\par
\labelchar{Art. 7.} Ce qui n’est pas défendu par la loi ne peut être empêché.\par
Nul ne peut être contraint à faire ce qu’elle n’ordonne pas.\par
\labelchar{Art. 8.} Nul ne peut être appelé en justice, accusé, arrêté ni détenu, que dans les cas déterminés par la loi, et selon les formes qu’elle a prescrites.\par
\labelchar{Art. 9.} Ceux qui sollicitent, expédient, signent, exécutent ou font exécuter des actes arbitraires sont coupables et doivent être punis.\par
\labelchar{Art. 10.} Toute rigueur qui ne serait pas nécessaire pour s’assurer de la personne d’un prévenu doit être sévèrement réprimée par la loi.\par
\labelchar{Art. 11.} Nul ne peut être jugé qu’après avoir été entendu ou légalement appelé.\par
\labelchar{Art. 12.} La loi ne doit décerner que des peines strictement nécessaires et proportionnées au délit.\par
\labelchar{Art. 13.} Tout traitement qui aggrave la peine déterminée par la loi, est un crime.\par
\labelchar{Art. 14.} Aucune loi, ni criminelle ni civile, ne peut avoir d’effet rétroactif.\par
\labelchar{Art. 15.} Tout homme peut engager son temps et ses services ; mais il ne peut se vendre ni être vendu ; sa personne n’est pas une propriété aliénable.\par
\labelchar{Art. 16.} Toute contribution est établie pour l’utilité générale ; elle doit être répartie entre les contribuables, en raison de leurs facultés.\par
\labelchar{Art. 17.} La souveraineté réside essentiellement dans l’universalité des citoyens.\par
\labelchar{Art. 18.} Nul individu, nulle réunion partielle de citoyens ne peut s’attribuer la souveraineté.\par
\labelchar{Art. 19.} Nul ne peut, sans une délégation légale, exercer aucune autorité, ni remplir aucune fonction publique.\par
\labelchar{Art. 20.} Chaque citoyen a un droit égal de concourir, immédiatement ou médiatement, à la formation de la loi, à la nomination des représentants du peuple et des fonctionnaires publics.\par
\labelchar{Art. 21.} Les fonctions publiques ne peuvent devenir la propriété de ceux qui les exercent.\par
\labelchar{Art. 22.} La garantie sociale ne peut exister si la division des pouvoirs n’est pas établie, si leurs limites ne sont pas fixées, et si la responsabilité des fonctionnaires publics n’est pas assurée.

\section[{Devoirs}]{Devoirs}

\labelchar{Art. 1.} La Déclaration des droits contient les obligations des législateurs : le maintien de la société demande que ceux qui la composent connaissent et remplissent également leurs devoirs.\par
\labelchar{Art. 2.} Tous les devoirs de l’homme et du citoyen dérivent de ces deux principes, gravés par la nature dans tous les cœurs :\par

\begin{itemize}[itemsep=0pt,topsep=0pt,partopsep=0pt,parskip=0pt]
\item Ne faites pas à autrui ce que vous ne voudriez pas qu’on vous fît.
\item Faites constamment aux autres le bien que vous voudriez en recevoir.
\end{itemize}

\labelchar{Art. 3.} Les obligations de chacun envers la société consistent à la défendre, à la servir, à vivre soumis aux lois, et à respecter ceux qui en sont les organes.\par
\labelchar{Art. 4.} Nul n’est bon citoyen, s’il n’est bon fils, bon père, bon frère, bon ami, bon époux.\par
\labelchar{Art. 5.} Nul n’est homme de bien, s’il n’est franchement et religieusement observateur des lois.\par
\labelchar{Art. 6.} Celui qui viole ouvertement les lois se déclare en état de guerre avec la société.\par
\labelchar{Art. 7.} Celui qui, sans enfreindre ouvertement les lois, les élude par ruse ou par adresse, blesse les intérêts de tous : il se rend indigne de leur bienveillance et de leur estime.\par
\labelchar{Art. 8.} C’est sur le maintien des propriétés que reposent la culture des terres, toutes les productions, tout moyen de travail, et tout l’ordre social.\par
\labelchar{Art. 9.} Tout citoyen doit ses services à la patrie et au maintien de la liberté, de l’égalité et de la propriété, toutes les fois que la loi l’appelle à les défendre.
\chapterclose


\chapteropen

\chapter[{Constitution de la République française}]{{\scshape Constitution} de la République française}
\renewcommand{\leftmark}{{\scshape Constitution} de la République française}


\chaptercont
\labelchar{Art. 1.} La République Française est une et indivisible.\par
\labelchar{Art. 2.} L’universalité des citoyens français est le souverain.\par

\section[{TITRE PREMIER. Division du territoire}]{TITRE PREMIER \\
Division du territoire}

\labelchar{Art. 3.} La France est divisée en … départements.\par
Ces départemens sont : l’Ain, l’Aisne, l’Allier, les Basses-Alpes, les Hautes-Alpes, les Alpes Maritimes, l’Ardèche, les Ardennes, l’Arriège, l’Aube, l’Aude, l’Aveyron ; les Bouches-du-Rhône ; le Calvados, le Cantal, la Charente, la Charente-Inférieure, le Cher, la Corrèze, la Côte-d’Or, les Côtes-du-Nord, la Creuse ; la Dordogne, le Doubs, la Drôme, la Dyle ; l’Eure, Eure-et-Loir, le Finistère, les Forêts ; le Gard, la Haute-Garonne, le Gers, la Gironde, le Golo ; l’Hérault ; Ille-et-Vilaine, l’Indre, Indre-et-Loire, l’Isère Jemmapes, le Jura ; les Landes, le Liamone, Loir-et-Cher, la Loire, la Haute-Loire, la Loire-Inférieure, le loiret, le Lot, Lot-et-Garonne, la Losère, la Lys ; Maine-et-Loire, la Manche, la Marne, la Haute-Marne, la Mayenne, la Meurthe, la Meuse, le Mont-Blanc, le Mont-Terrible, le Morbihan, la Moselle ; les Deux-Nèthes, la Nièvre, le Nord ; l’Oise, l’Orne, l’Ourthe ; le Pas-de-Calais, le Puy-de-Dôme, les Basses-Pyrénées, les Hautes-Pyrénées, les Pyrénées-Orientales ; le Bas-Rhin, le Haut-Rhin ; lz Sambre-et-Meuse, la Haute-Saône, Saône-et-Loire, la Sarthe, la Seine, la Seine-Inférieure, Seine-et-Marne, Seine-et-Oise, les Deux-Sèvres, la Somme ; le Tarn ; le Var, Vaucluse, la Vendée, la Vienne, la Haute-Vienne, les Vosges ; l’Yonne…\par
\labelchar{Art. 4.} Les limites des départements peuvent être changées ou rectifiées par le Corps législatif ; mais, en ce cas, la surface d’un département ne peut excéder cent myriamètres carrés (quatre cents lieues carrées moyennes\footnote{lieue moyenne linéaire = 2 566 toises})\par
\labelchar{Art. 5.} Chaque département est distribué en cantons, chaque canton en communes.\par
Les cantons conservent leurs circonscriptions actuelles.\par
Leurs limites pourront néanmoins être changées ou rectifiées par le Corps législatif ; mais, en ce cas, il ne pourra y avoir plus d’un myriamètre (deux lieues moyennes de deux mille cinq cent soixante-six toises chacune) de la commune la plus éloignée au chef-lieu du canton.\par
\labelchar{Art. 6.} Les colonies françaises sont parties intégrantes de la République, et sont soumises à la même loi constitutionnelle.\par
\labelchar{Art. 7.} Elles sont divisées en départements, ainsi qu’il suit ;\par

\begin{itemize}[itemsep=0pt,topsep=0pt,partopsep=0pt,parskip=0pt]
\item L’île de Saint-Domingue, dont le Corps législatif déterminera la division en quatre départements au moins, et en six au plus ;
\item La Guadeloupe, Marie-Galande, la Désirade, les Saintes, et la partie française de Saint-Martin ;
\item La Martinique ;
\item La Guyane française et Cayenne ;
\item Sainte-Lucie et Tabago ;
\item L’île de France, les Séchelles, Rodrigue, et les établissements de Madagascar ;
\item L’île de la Réunion ;
\item Les Indes-Orientales, Pondichéri, Chandernagor, Mahé, Karical et autres établissements.
\end{itemize}


\section[{TITRE II. État politique des citoyens}]{TITRE II \\
État politique des citoyens}

\labelchar{Art. 8.} Tout homme né et résidant en France, qui, âgé de vingt et un ans accomplis, s’est fait inscrire sur le registre civique de son canton, qui a demeuré depuis pendant une année sur le territoire de la République, et qui paie une contribution directe, foncière ou personnelle, est citoyen français.\par
\labelchar{Art. 9.} Sont citoyens, sans aucune condition de contribution, les Français qui auront fait une ou plusieurs campagnes pour l’établissement de la République.\par
\labelchar{Art. 10.} L’étranger devient citoyen français, lorsque après avoir atteint l’âge de vingt et un ans accomplis, et avoir déclaré l’intention de se fixer en France, il y a résidé pendant sept années consécutives, pourvu qu’il y paie une contribution directe, et qu’en outre il y possède une propriété foncière, ou un établissement d’agriculture ou de commerce, ou qu’il y ait épousé une femme française.\par
\labelchar{Art. 11.} Les citoyens français peuvent seuls voter dans les Assemblées primaires, et être appelés aux fonctions établies par la Constitution.\par
\labelchar{Art. 12.} L’exercice des Droits de citoyen se perd :\par

\begin{enumerate}[itemsep=0pt,topsep=0pt,partopsep=0pt,parskip=0pt]
\item Par la naturalisation en pays étrangers ;
\item Par l’affiliation à toute corporation étrangère qui supposerait des distinctions de naissance, ou qui exigerait des vœux de religion ;
\item Par l’acceptation de fonctions ou de pensions offertes par un gouvernement étranger ;
\item Par la condamnation à des peines afflictives ou infamantes, jusqu’à réhabilitation.
\end{enumerate}

\labelchar{Art. 13.} L’exercice des Droits de citoyen est suspendu :\par

\begin{enumerate}[itemsep=0pt,topsep=0pt,partopsep=0pt,parskip=0pt]
\item Par l’interdiction judiciaire pour cause de fureur, de démence ou d’imbécillité ;
\item Par l’état de débiteur failli, ou d’héritier immédiat ; détenteur à titre gratuit, de tout ou partie de la succession d’un failli ;
\item Par l’état de domestique à gage, attaché au service de la personne ou du ménage ;
\item Par l’état d’accusation ;
\item Par un jugement de contumace, tant que le jugement n’est pas anéanti.
\end{enumerate}

\labelchar{Art. 14.} L’exercice des Droits de citoyen n’est perdu ni suspendu que dans les cas exprimés dans les deux articles précédents.\par
\labelchar{Art. 15.} Tout citoyen qui aura résidé sept années consécutives hors du territoire de la République, sans mission ou autorisation donnée au nom de la nation, est réputé étranger ; il ne redevient citoyen français qu’après avoir satisfait aux conditions prescrites par l’article dixième.\par
\labelchar{Art. 16.} Les jeunes gens ne peuvent être inscrits sur le registre civique, s’ils ne prouvent qu’ils savent lire et écrire, et exercer une profession mécanique. Les opérations manuelles de l’agriculture appartiennent aux professions mécaniques.\par
Cet article n’aura d’exécution qu’à compter de l’an XII de la République.

\section[{TITRE III. Assemblées primaires}]{TITRE III \\
Assemblées primaires}

\labelchar{Art. 17.} Les Assemblées primaires se composent des citoyens domiciliés dans le même canton.\par
Le domicile requis pour voter dans ces Assemblées, s’acquiert par la seule résidence pendant une année, et il ne se perd que par un an d’absence.\par
\labelchar{Art. 18.} Nul ne peut se faire remplacer dans les Assemblées primaires, ni voter pour le même objet dans plus d’une de ces Assemblées.\par
\labelchar{Art. 19.} Il y a au moins une Assemblée primaire par canton.\par
Lorsqu’il y en a plusieurs, chacune est composée de quatre cent cinquante citoyens au moins, de neuf cents au plus.\par
Ces nombres s’entendent des citoyens présents ou absents, ayant droit d’y voter.\par
\labelchar{Art. 20.} Les Assemblées primaires se constituent provisoirement sous la présidence du plus ancien d’âge ; le plus jeune remplit provisoirement les fonctions de secrétaire.\par
\labelchar{Art. 21.} Elles sont définitivement constituées par la nomination, au scrutin, d’un président, d’un secrétaire et de trois scrutateurs.\par
\labelchar{Art. 22.} S’il s’élève des difficultés sur les qualités requises pour voter, l’Assemblée statue provisoirement, sauf le recours au tribunal civil du département.\par
\labelchar{Art. 23.} En tout autre cas, le Corps législatif prononce seul sur la validité des opérations des Assemblées primaires.\par
\labelchar{Art. 24.} Nul ne peut paraître en armes dans les Assemblées primaires.\par
\labelchar{Art. 25.} Leur police leur appartient.\par
\labelchar{Art. 26.} Les Assemblées primaires se réunissent :\par

\begin{enumerate}[itemsep=0pt,topsep=0pt,partopsep=0pt,parskip=0pt]
\item Pour accepter ou rejeter les changements à l’acte constitutionnel, proposés par les Assemblées de révision ;
\item Pour faire les élections qui leur appartiennent suivant l’acte constitutionnel.
\end{enumerate}

\labelchar{Art. 27.} Elles s’assemblent de plein droit le premier germinal de chaque année, et procèdent, selon qu’il y a lieu, à la nomination :\par

\begin{enumerate}[itemsep=0pt,topsep=0pt,partopsep=0pt,parskip=0pt]
\item Des membres de l’Assemblée électorale ;
\item Du juge de paix et de ses assesseurs ;
\item Du président de l’administration du canton, ou des officiers municipaux dans les communes au-dessus de cinq mille habitants.
\end{enumerate}

\labelchar{Art. 28.} Immédiatement après ces élections, il se tient, dans les communes au-dessous de cinq mille habitants, des Assemblées communales qui élisent les agents de chaque commune et leurs adjoints.\par
\labelchar{Art. 29.} Ce qui se fait dans une Assemblée primaire ou communale au-delà de l’objet de sa convocation, et contre les formes déterminées par la Constitution, est nul.\par
\labelchar{Art. 30.} Les Assemblées, soit primaires, soit communales, ne font aucune autre élection que celles qui leur sont attribuées par l’acte constitutionnel.\par
\labelchar{Art. 31.} Toutes les élections se font au scrutin secret.\par
\labelchar{Art. 32.} Tout citoyen qui est légalement convaincu d’avoir vendu ou acheté un suffrage, est exclu des Assemblées primaires et communales, et de toute fonction publique, pendant vingt ans ; en cas de récidive, il l’est pour toujours.

\section[{TITRE IV. Assemblées électorales}]{TITRE IV \\
Assemblées électorales}

\labelchar{Art. 33.} Chaque Assemblée primaire nomme un électeur à raison de deux cents citoyens, présents ou absents, ayant droit de voter dans ladite Assemblée. Jusqu’au nombre de trois cents citoyens inclusivement, il n’est nommé qu’un électeur.\par

\begin{itemize}[itemsep=0pt,topsep=0pt,partopsep=0pt,parskip=0pt]
\item Il en est nommé deux depuis trois cent un jusqu’à cinq cents ;
\item Trois depuis cinq cent un jusqu’à sept cents ;
\item Quatre depuis sept cent un jusqu’à neuf cents.
\end{itemize}

\labelchar{Art. 34.} Les membres des Assemblées électorales sont nommés chaque année, et ne peuvent être réélus qu’après un intervalle de deux ans.\par
\labelchar{Art. 35.} Nul ne pourra être nommé électeur, s’il n’a vingt-cinq ans accomplis, et s’il ne réunit aux qualités nécessaires pour exercer les droits de citoyen français, l’une des conditions suivantes, savoir :\par

\begin{itemize}[itemsep=0pt,topsep=0pt,partopsep=0pt,parskip=0pt]
\item Dans les communes au-dessus de six mille habitants, celle d’être propriétaire ou usufruitier d’un bien évalué à un revenu égal à la valeur locale de deux cents journées de travail, ou d’être locataire, soit d’une habitation évaluée à un revenu égal à la valeur de cent cinquante journées de travail, soit d’un bien rural évalué à deux cents journées de travail ; Dans les communes au-dessous de six mille habitants, celle d’être propriétaire ou usufruitier d’un bien évalué à un revenu égal à la valeur locale de cent cinquante journées de travail, ou d’être locataire, soit d’une habitation évaluée à un revenu égal à la valeur de cent journées de travail, soit d’un bien rural évalué à cent journées de travail ;
\item Et dans les campagnes, celle d’être propriétaire ou usufruitier d’un bien évalué à un revenu égal à la valeur locale de cent cinquante journées de travail, ou d’être fermier ou métayer de biens évalués à la valeur de deux cents journées de travail.
\item A l’égard de ceux qui seront en même temps propriétaires ou usufruitiers d’une part, et locataires, fermiers ou métayers de l’autre, leurs facultés à ces divers titres seront cumulées jusqu’au taux nécessaire pour établir leur éligibilité.
\end{itemize}

\labelchar{Art. 36.} L’Assemblée électorale de chaque département se réunit le 20 germinal de chaque année, et termine, en une seule session de dix jours au plus, et sans pouvoir s’ajourner, toutes les élections qui se trouvent à faire ; après quoi, elle est dissoute, de plein droit.\par
\labelchar{Art. 37.} Les Assemblées électorales ne peuvent s’occuper d’aucun objet étranger aux élections dont elles sont chargées ; elles ne peuvent envoyer ni recevoir aucune adresse, aucune pétition, aucune députation.\par
\labelchar{Art. 38.} Les Assemblées électorales ne peuvent correspondre entre elles.\par
\labelchar{Art. 39.} Aucun citoyen, ayant été membre d’une Assemblée électorale, ne peut prendre le titre d’électeur, ni se réunir, en cette qualité, à ceux qui ont été avec lui membres de cette même Assemblée.\par
La contravention au présent article est un attentat à la sûreté générale.\par
\labelchar{Art. 40.} Les articles 18, 20, 21, 23, 24, 25, 29, 30, 31 et 32 du titre précédent, sur les Assemblées primaires, sont communs aux Assemblées électorales.\par
\labelchar{Art. 41.} Les Assemblées électorales élisent, selon qu’il y a lieu :\par

\begin{enumerate}[itemsep=0pt,topsep=0pt,partopsep=0pt,parskip=0pt]
\item Les membres du Corps législatif, savoir : les membres du Conseil des Anciens, ensuite les membres du Conseil des Cinq-Cents ;
\item Les membres du Tribunal de cassation ;
\item Les hauts-jurés ;
\item Les administrateurs de département ;
\item Les président, accusateur public et greffier du tribunal criminel ;
\item Les juges des tribunaux civils.
\end{enumerate}

\labelchar{Art. 42.} Lorsqu’un citoyen est élu par les Assemblées électorales pour remplacer un fonctionnaire mort, démissionnaire ou destitué, ce citoyen n’est élu que pour le temps qui restait au fonctionnaire remplacé.\par
\labelchar{Art. 43.} Le commissaire du Directoire exécutif près l’administration de chaque département est tenu, sous, peine de destitution, d’informer le Directoire de l’ouverture et de la clôture des Assemblées électorales : ce commissaire n’en peut arrêter ni suspendre les opérations, ni entrer dans le lieu des séances ; mais il a le droit de demander communication du procès-verbal de chaque séance dans les vingt-quatre heures qui la suivent ; et il est tenu de dénoncer au Directoire les infractions qui seraient faites à l’acte constitutionnel.\par
Dans tous les cas, le Corps législatif prononce seul sur la validité des opérations des Assemblées électorales.\par
\bigbreak

\section[{TITRE V. Pouvoir législatif}]{TITRE V \\
Pouvoir législatif}


\subsection[{Dispositions générales}]{Dispositions générales}

\labelchar{Art. 44.} Le Corps législatif est composé d’un Conseil des Anciens et d’un Conseil des Cinq-Cents.\par
\labelchar{Art. 45.} En aucun cas, le Corps législatif ne peut déléguer à un ou plusieurs de ses membres, ni à qui que ce soit, aucune des fonctions qui lui sont attribuées par la présente Constitution.\par
\labelchar{Art. 46.} Il ne peut exercer par lui-même, ni par des délégués, le Pouvoir exécutif, ni le Pouvoir judiciaire.\par
\labelchar{Art. 47.} Il y a incompatibilité entre la qualité de membre du Corps législatif et l’exercice d’une autre fonction publique, excepté celle d’archiviste de la République.\par
\labelchar{Art. 48.} La loi détermine le mode du remplacement définitif ou temporaire des fonctionnaires publics qui viennent à être élus membres du Corps législatif.\par
\labelchar{Art. 49.} Chaque département concourt, à raison de sa population seulement, à la nomination des membres du Conseil des Anciens et des membres du Conseil des Cinq-Cents.\par
\labelchar{Art. 50.} Tous les dix ans, le Corps législatif, d’après les états de population qui lui sont envoyés, détermine le nombre des membres de l’un et de l’autre Conseil que chaque département doit fournir.\par
\labelchar{Art. 51.} Aucun changement ne peut être fait dans cette répartition, durant cet intervalle.\par
\labelchar{Art. 52.} Les membres du Corps législatif ne sont pas représentants du département qui les a nommés, mais de la Nation entière, et il ne peut leur être donné aucun mandat.\par
\labelchar{Art. 53.} L’un et l’autre Conseil est renouvelé tous les ans par tiers.\par
\labelchar{Art. 54.} Les membres sortant après trois années peuvent être immédiatement réélus pour les trois années suivantes, après quoi il faudra un intervalle de deux ans pour qu’ils puissent être élus de nouveau.\par
\labelchar{Art. 55.} Nul, en aucun cas, ne peut être membre du Corps législatif durant plus de six années consécutives.\par
\labelchar{Art. 56.} Si, par des circonstances extraordinaires, l’un des deux Conseils se trouve réduit à moins des deux tiers de ses membres, il en donne avis au Directoire exécutif, lequel est tenu de convoquer, sans délai, les Assemblées primaires des départements qui ont des membres du Corps législatif à remplacer par l’effet de ces circonstances ; les Assemblées primaires nomment sur-le-champ les électeurs, qui procèdent aux remplacements nécessaires.\par
\labelchar{Art. 57.} Les membres nouvellement élus pour l’un et pour l’autre Conseil, se réunissent, le premier prairial de chaque année, dans la commune qui a été indiquée par le Corps législatif précédent, ou dans la commune même où il a tenu ses dernières séances, s’il n’en a pas désigné une autre.\par
\labelchar{Art. 58.} Les deux Conseils résident toujours dans la même commune.\par
\labelchar{Art. 59.} Le Corps législatif est permanent ; il peut, néanmoins, s’ajourner à des termes qu’il désigne.\par
\labelchar{Art. 60.} En aucun cas, les deux Conseils ne peuvent se réunir dans une même salle.\par
\labelchar{Art. 61.} Les fonctions de président et de secrétaire ne peuvent excéder la durée d’un mois, ni dans le Conseil des Anciens, ni dans celui des Cinq-Cents.\par
\labelchar{Art. 62.} Les deux Conseils ont respectivement les droit de police dans le lieu de leurs séances, et dans l’enceinte extérieure qu’ils ont déterminée.\par
\labelchar{Art. 63.} Ils ont respectivement le droit de police sur leurs membres ; mais ils ne peuvent prononcer de peine plus forte que la censure, les arrêts pour huit jours, et la prison pour trois.\par
\labelchar{Art. 64.} Les séances de l’un et de l’autre Conseil sont publiques ; les assistants ne peuvent excéder en nombre la moitié des membres respectifs de chaque Conseil.\par
Les procès-verbaux des séances sont imprimés.\par
\labelchar{Art. 65.} Toute délibération se prend par assis et levé : en cas de doute, il se fait un appel nominal ; mais alors les votes sont secrets.\par
\labelchar{Art. 66.} Sur la demande de cent de ses membres, chaque Conseil peut se former en comité général et secret, mais seulement pour discuter, et non pour délibérer.\par
\labelchar{Art. 67.} Ni l’un ni l’autre de ces Conseils ne peut créer dans son sein aucun comité permanent.\par
Seulement chaque Conseil a la faculté, lorsqu’une matière lui paraît susceptible d’un examen préparatoire, de nommer parmi ses membres une commission spéciale, qui se renferme uniquement dans l’objet de sa formation.\par
Cette commission est dissoute aussitôt que le Conseil a statué sur l’objet dont elle était chargée.\par
\labelchar{Art. 68.} Les membres du Corps législatif reçoivent une indemnité annuelle : elle est, dans l’un et l’autre Conseil, fixée à la valeur de trois mille myriagrammes de froment (six cent treize quintaux trente-deux livres).\par
\labelchar{Art. 69.} Le Directoire exécutif ne peut faire passer ou séjourner aucun corps de troupes dans la distance de six myriamètres (douze lieues moyennes) de la commune où le Corps législatif tient ses séances, si ce n’est sur sa réquisition ou avec son autorisation.\par
\labelchar{Art. 70.} Il y a près du Corps législatif une garde de citoyens pris dans la Garde nationale sédentaire de tous les départements, et choisis par leurs frères d’armes. Cette garde ne peut être au-dessous de quinze cents hommes en activité de service.\par
\labelchar{Art. 71.} Le Corps législatif détermine le mode de ce service et sa durée.\par
\labelchar{Art. 72.} Le Corps législatif n’assiste à aucune cérémonie publique, et n’y envoie point de députations.

\subsection[{Conseil des Cinq-Cents}]{Conseil des Cinq-Cents}

\labelchar{Art. 73.} Le Conseil des Cinq-Cents est invariablement fixé à ce nombre.\par
\labelchar{Art. 74.} Pour être élu membre du Conseil des Cinq-Cents, il faut être âgé de trente ans accomplis, et avoir été domicilié sur le territoire de la République pendant les dix années qui auront immédiatement précédé l’élection.\par
La condition de l’âge de trente ans ne sera point exigible avant l’an septième de la République ; jusqu’à cette époque, l’âge de vingt-cinq ans accomplis sera suffisant.\par
\labelchar{Art. 75.} Le Conseil des Cinq-Cents ne peut délibérer, si la séance n’est composée de deux cents membres au moins.\par
\labelchar{Art. 76.} La proposition des lois appartient exclusivement au Conseil des Cinq-Cents.\par
\labelchar{Art. 77.} Aucune proposition ne peut être délibérée ni résolue dans le Conseil des Cinq-Cents, qu’en observant les formes suivantes.\par

\begin{itemize}[itemsep=0pt,topsep=0pt,partopsep=0pt,parskip=0pt]
\item Il se fait trois lectures de la proposition ; l’intervalle entre deux de ces lectures ne peut être moindre de dix jours.
\item La discussion est ouverte après chaque lecture ; et, néanmoins, après la première ou la seconde, le Conseil des Cinq-Cents peut déclarer qu’il y a lieu à l’ajournement, ou qu’il n’y a pas lieu à délibérer.
\item Toute proposition doit être imprimée et distribuée deux jours avant la seconde lecture.
\item Après la troisième lecture, le Conseil des Cinq-Cents décide s’il y a lieu ou non à l’ajournement.
\end{itemize}

\labelchar{Art. 78.} Toute proposition qui, soumise à la discussion, a été définitivement rejetée après la troisième lecture, ne peut être reproduite qu’après une année révolue.\par
\labelchar{Art. 79.} Les propositions adoptées par le Conseil des Cinq-Cents s’appellent résolutions.\par
\labelchar{Art. 80.} Le préambule de toute résolution énonce :\par

\begin{enumerate}[itemsep=0pt,topsep=0pt,partopsep=0pt,parskip=0pt]
\item Les dates des séances auxquelles les trois lectures de la proposition auront été faites ;
\item L’acte par lequel il a été déclaré, après la troisième lecture, qu’il n’y a pas lieu à l’ajournement.
\end{enumerate}

\labelchar{Art. 81.} Sont exemptes des formes prescrites par l’article 77, les propositions reconnues urgentes par une déclaration préalable du Conseil des Cinq-Cents.\par
Cette déclaration énonce les motifs de l’urgence, et il en est fait mention dans le préambule de la résolution.

\subsection[{Conseil des Anciens}]{Conseil des Anciens}

\labelchar{Art. 82.} Le Conseil des Anciens est composé de deux cent cinquante membres.\par
\labelchar{Art. 83.} Nul ne peut être élu membre du Conseil des Anciens : S’il n’est âgé de quarante ans accomplis ; Si, de plus, il n’est marié ou veuf ;\par
Et s’il n’a pas été domicilié sur le territoire de la République pendant les quinze années qui auront immédiatement précédé l’élection.\par
\labelchar{Art. 84.} La condition de domicile exigée par le présent article, et celle prescrite par l’article 74, ne concernent point les citoyens qui sont sortis du territoire de la République avec mission du gouvernement.\par
\labelchar{Art. 85.} Le Conseil des Anciens ne peut délibérer si la séance n’est composée de cent vingt-six membres au moins.\par
\labelchar{Art. 86.} Il appartient exclusivement au Conseil des Anciens d’approuver ou de rejeter les résolutions du Conseil des Cinq-Cents.\par
\labelchar{Art. 87.} Aussitôt qu’une résolution du Conseil des Cinq-Cents est parvenue au Conseil des Anciens, le président donne lecture du préambule.\par
\labelchar{Art. 88.} Le Conseil des Anciens refuse d’approuver les résolutions du Conseil des Cinq-Cents qui n’ont point été prises dans les formes prescrites par la Constitution.\par
\labelchar{Art. 89.} Si la proposition a été déclarée urgente par le Conseil des Cinq-Cents, le Conseil des Anciens délibère pour approuver ou rejeter l’acte d’urgence.\par
\labelchar{Art. 90.} Si le Conseil des Anciens rejette l’acte d’urgence, il ne délibère point sur le fond de la résolution.\par
\labelchar{Art. 91.} Si la résolution n’est pas précédée d’un acte d’urgence, il en est fait trois lectures : l’intervalle entre deux de ces lectures ne peut être moindre de cinq jours.\par
La discussion est ouverte après chaque lecture.\par
Toute résolution est imprimée et distribuée deux jours au moins avant la seconde lecture.\par
\labelchar{Art. 92.} Les résolutions du Conseil des Cinq-Cents, adoptées par le Conseil des Anciens, s’appellent lois.\par
\labelchar{Art. 93.} Le préambule des lois énonce les dates des séances du Conseil des Anciens auxquelles les trois lectures ont été faites.\par
\labelchar{Art. 94.} Le décret par lequel le Conseil des Anciens reconnaît l’urgence d’une loi, est motivé et mentionné dans le préambule de cette loi.\par
\labelchar{Art. 95.} La proposition de la loi, faite par le Conseil des Cinq-Cents, s’entend de tous les articles d’un même projet ; le Conseil des Anciens doit les rejeter tous, ou les approuver dans leur ensemble.\par
\labelchar{Art. 96.} L’approbation du Conseil des Anciens est exprimée sur chaque proposition de loi par cette formule, signée du président et des secrétaires : Le Conseil des Anciens approuve…\par
\labelchar{Art. 97.} Le refus d’adopter pour cause d’omission des formes indiquées dans l’article 77, est exprimé par cette formule, signée du président et des secrétaires : La Constitution annule…\par
\labelchar{Art. 98.} Le refus d’approuver le fond de la loi proposée, est exprimé par cette formule, signée du président et des secrétaires : Le Conseil des Anciens ne peut adopter…\par
\labelchar{Art. 99.} Dans le cas du précédent article, le projet de loi rejeté ne peut plus être présenté par le Conseil des Cinq-Cents qu’après une année révolue.\par
\labelchar{Art. 100.} Le Conseil des Cinq-Cents peut néanmoins présenter, à quelque époque que ce soit, un projet de loi qui contienne des articles faisant partie d’un projet qui a été rejeté.\par
\labelchar{Art. 101.} Le Conseil des Anciens envoie dans le jour les lois qu’il a adoptées, tant au Conseil des Cinq-Cents qu’au Directoire exécutif.\par
\labelchar{Art. 102.} Le Conseil des Anciens peut changer la résidence du Corps législatif ; il indique, en ce cas, un nouveau lieu et l’époque à laquelle les deux Conseils sont tenus de s’y rendre.\par
Le décret du Conseil des Anciens sur cet objet est irrévocable.\par
\labelchar{Art. 103.} Le jour même de ce décret, ni l’un ni l’autre des Conseils ne peuvent plus délibérer dans la commune où ils ont résidé jusqu’alors.\par
Les membres qui y continueraient leurs fonctions, se rendraient coupables d’attentat contre la sûreté de la République.\par
\labelchar{Art. 104.} Les membres du Directoire exécutif qui retarderaient ou refuseraient de sceller, promulguer et envoyer le décret de translation du Corps législatif, seraient coupables du même délit.\par
\labelchar{Art. 105.} Si, dans les vingt jours après celui fixé par le Conseil des Anciens, la majorité de chacun des deux Conseils n’a pas fait connaître à la République son arrivée au nouveau lieu indiqué, ou sa réunion dans un autre lieu quelconque, les administrateurs de département, ou, à leur défaut, les tribunaux civils de département convoquent les Assemblées primaires pour nommer des électeurs qui procèdent aussitôt à la formation d’un nouveau Corps législatif, par l’élection de deux cent cinquante députés pour le Conseil des Anciens, et de cinq cents pour l’autre Conseil.\par
\labelchar{Art. 106.} Les administrateurs de département qui, dans le cas de l’article précédent, seraient en retard de convoquer les Assemblées primaires, se rendraient coupables de haute trahison et d’attentat contre la sûreté de la République.\par
\labelchar{Art. 107.} Sont déclarés coupables du même délit tous citoyens qui mettraient obstacle à la convocation des Assemblées primaires et électorales, dans le cas de l’article 106.\par
\labelchar{Art. 108.} Les membres du nouveau Corps législatif se rassemblent dans le lieu où le Conseil des Anciens avait transféré ses séances.\par
S’ils ne peuvent se réunir dans ce lieu, dans quelque endroit qu’ils se trouvent en majorité, là est le Corps législatif.\par
\labelchar{Art. 109.} Excepté dans le cas de l’article 102, aucune proposition de loi ne peut prendre naissance dans le Conseil des Anciens.

\subsection[{De la garantie des membres du Corps législatif}]{De la garantie des membres du Corps législatif}

\labelchar{Art. 110.} Les citoyens qui sont, ou ont été, membres du Corps législatif, ne peuvent être recherchés, accusés ni jugés en aucun temps, pour ce qu’ils ont dit ou écrit dans l’exercice de leurs fonctions.\par
\labelchar{Art. 111.} Les membres du Corps législatif, depuis le moment de leur nomination jusqu’au trentième jour après l’expiration de leurs fonctions, ne peuvent être mis en jugement que dans les formes prescrites par les articles qui suivent.\par
\labelchar{Art. 112.} Ils peuvent, pour faits criminels, être saisis en flagrant délit ; mais il en est donné avis, sans délai, au Corps législatif, et la poursuite ne pourra être continuée qu’après que le Conseil des Cinq-Cents aura proposé la mise en jugement que le Conseil des Anciens l’aura décrétée.\par
\labelchar{Art. 113.} Hors le cas du flagrant-délit, les membres du Corps législatif ne peuplent être amenés devant les officiers de police, ni mis en état d’arrestation, avant que le Conseil des Cinq-Cents ait proposé la mise en jugement, et que le Conseil des Anciens l’ait décrétée.\par
\labelchar{Art. 114.} Dans les cas des deux articles précédents, un membre du Corps législatif ne peut être traduit devant aucun autre tribunal que la Haute Cour de justice.\par
\labelchar{Art. 115.} Ils sont traduits devant la même Cour pour les faits de trahison, de dilapidation, de manœuvres pour renverser la Constitution, et d’attentat contre la sûreté intérieure de la République.\par
\labelchar{Art. 116.} Aucune dénonciation contre un membre du Corps législatif ne peut donner lieu à poursuite, si elle n’est rédigée par écrit, signée et adressée au Conseil des Cinq-Cents.\par
\labelchar{Art. 117.} Si, après y avoir délibéré en la forme prescrite par l’article 77, le Conseil des Cinq-Cents admet la dénonciation, il le déclare en ces termes :\par
La dénonciation contre… pour le fait de… datée… signée de… est admise.\par
\labelchar{Art. 118.} L’inculpé est alors appelé : il a, pour comparaître, un délai de trois jours francs, et lorsqu’il comparaît, il est entendu dans l’intérieur du lieu des séances du Conseil des Cinq-Cents.\par
\labelchar{Art. 119.} Soit que l’inculpé se soit présenté ou non, le Conseil des Cinq-Cents déclare, après ce délai, s’il y a lieu, ou non à l’examen de sa conduite.\par
\labelchar{Art. 120.} S’il est déclaré par le Conseil des Cinq-Cents qu’il y a lieu à examen, le prévenu est appelé par le Conseil des Anciens ; il a pour comparaître, un délai de deux jours francs ; et s’il comparaît, il est entendu dans l’intérieur du lieu des séances du Conseil des Anciens.\par
\labelchar{Art. 121.} Soit que le prévenu se soit présenté, ou non, le Conseil des Anciens, après ce délai, et après avoir délibéré dans les formes prescrites par l’article 91, prononce l’accusation, s’il y a lieu, et renvoie l’accusé devant la Haute Cour de justice, laquelle est tenue d’instruire le procès sans aucun délai.\par
\labelchar{Art. 122.} Toute discussion, dans l’un et dans l’autre Conseil, relative à la prévention ou à l’accusation d’un membre du Corps législatif, se fait en Conseil général. Toute délibération sur les mêmes objets est prise à l’appel nominal et au scrutin secret.\par
\labelchar{Art. 123.} L’accusation prononcée contre un membre du Corps législatif entraîne suspension.\par
S’il est acquitté par le jugement de la Haute Cour de justice, il reprend ses fonctions.

\subsection[{Relations des deux Conseils entre eux}]{Relations des deux Conseils entre eux}

\labelchar{Art. 124.} Lorsque les deux Conseils sont définitivement constitués, ils s’en avertissent mutuellement par un messager d’Etat.\par
\labelchar{Art. 125.} Chaque Conseil nomme quatre messagers d’Etat pour son service.\par
\labelchar{Art. 126.} Ils portent à chacun des Conseils et au Directoire exécutif les lois et les actes du Corps législatif ; ils ont entrée à cet effet dans le lieu des séances du Directoire exécutif.\par
Ils marchent précédés de deux huissiers.\par
\labelchar{Art. 127.} L’un des Conseils ne peut s’ajourner au-delà de cinq jours sans le consentement de l’autre.

\subsection[{Promulgation des lois}]{Promulgation des lois}

\labelchar{Art. 128.} Le Directoire exécutif fait sceller et publier les lois et les autres actes du Corps législatif, dans les deux jours après leur réception.\par
\labelchar{Art. 129.} Il fait sceller, promulguer dans le jour, les lois et actes du Corps législatif qui sont précédés d’un décret d’urgence.\par
\labelchar{Art. 130.} La publication de la loi et des actes du Corps législatif est ordonnée en la forme suivante :\par
Au nom de la République française (loi) ou (acte du Corps législatif)… Le Directoire ordonne que la loi ou l’acte législatif ci-dessus sera publié, exécuté, et qu’il sera muni du sceau de la République.\par
\labelchar{Art. 131.} Les lois dont le préambule n’atteste pas l’observation des formes prescrites par les articles 77 et 91, ne peuvent être promulguées par le Directoire exécutif, et sa responsabilité à cet égard dure six années.\par
Sont exceptées les lois pour lesquelles l’acte d’urgence a été approuvé par le Conseil des Anciens.

\section[{TITRE VI. Pouvoir exécutif}]{TITRE VI \\
Pouvoir exécutif}

\labelchar{Art. 132.} Le Pouvoir exécutif est délégué à un Directoire de cinq membres, nommé par le Corps législatif, faisant alors les fonctions d’Assemblée électorale, au nom de la Nation.\par
\labelchar{Art. 133.} Le Conseil des Cinq-Cents forme, au scrutin secret, une liste décuple du nombre des membres du Directoire qui sont à nommer, et la présente au Conseil des Anciens, qui choisit aussi au scrutin secret, dans cette liste.\par
\labelchar{Art. 134.} Les membres du Directoire doivent être âgés de quarante ans au moins.\par
\labelchar{Art. 135.} Ils ne peuvent être pris que parmi les citoyens qui ont été membres du Corps législatif, ou ministres.\par
La disposition du présent article ne sera observée qu’à commencer de l’an neuvième de la République.\par
\labelchar{Art. 136.} A compter du premier jour de l’an V de la République, les membres du Corps législatif ne pourront être élus membres du Directoire ni ministres, soit pendant la durée de leurs fonctions législatives, soit pendant la première année après l’expiration de ces mêmes fonctions.\par
\labelchar{Art. 137.} Le Directoire est partiellement renouvelé par l’élection d’un nouveau membre, chaque année.\par
Le sort décidera, pendant les quatre premières années, de la sortie successive de ceux qui auront été nommés la première fois.\par
\labelchar{Art. 138.} Aucun des membres sortants ne peut être réélu qu’après un intervalle de cinq ans.\par
\labelchar{Art. 139.} L’ascendant et le descendant en ligne directe, les frères, l’oncle et le neveu, les cousins au premier degré, et les alliés à ces divers degrés, ne peuvent être en même temps membres du Directoire, ni s’y succéder, qu’après un intervalle de cinq ans.\par
\labelchar{Art. 140.} En cas de vacance par mort, démission ou autrement, d’un des membres du Directoire, son successeur est élu par le Corps législatif dans dix jours pour tout délai.\par
Le Conseil des Cinq-Cents est tenu de proposer les candidats dans les cinq premiers jours, et le Conseil des Anciens doit consommer l’élection dans les cinq derniers.\par
Le nouveau membre n’est élu que pour le temps d’exercice qui restait à celui qu’il remplace.\par
Si, néanmoins, ce temps n’excède pas six mois, celui qui est élu demeure en fonctions jusqu’à la fin de la cinquième année suivante.\par
\labelchar{Art. 141.} Chaque membre du Directoire le préside à son tour durant trois mois seulement.\par
Le président a la signature et la garde du sceau.\par
Les lois et les actes du Corps législatif sont adressés au Directoire, en la personne de son président.\par
\labelchar{Art. 142.} Le Directoire exécutif ne peut délibérer, s’il n’y a trois membres présents au moins.\par
\labelchar{Art. 143.} Il se choisit, hors de son sein, un secrétaire qui contresigne les expéditions, et rédige les délibérations sur un registre où chaque membre a le droit de faire inscrire son avis motivé.\par
Le Directoire peut, quand il le juge à propos, délibérer sans l’assistance de son secrétaire ; en ce cas, les délibérations sont rédigées, un registre particulier, par un des membres du Directoire.\par
\labelchar{Art. 144.} Le Directoire pourvoit, d’après les lois, à la sûreté extérieure ou intérieure de la République. Il peut faire des proclamations conformes aux lois et pour leur exécution.\par
Il dispose de la force armée, sans qu’en aucun cas, le Directoire collectivement, ni aucun de ses membres, puisse la commander, ni pendant le temps de ses fonctions, ni pendant les deux années qui suivent immédiatement l’expiration de ces mêmes fonctions.\par
\labelchar{Art. 145.} Si le Directoire est informé qu’il se trouve quelque conspiration contre la sûreté extérieure ou intérieure de l’Etat, il peut décerner des mandats d’amener et des mandats d’arrêt contre ceux qui en sont présumés les auteurs out les complices ; il peut les interroger ; mais il est obligé, sous les peines portées contre le crime de détention arbitraire, de les renvoyer par-devant l’officier de police, dans le délai de deux jours, pour procéder suivant les lois.\par
\labelchar{Art. 146.} Le Directoire nomme les généraux en chef ; il ne peut les choisir parmi les parents ou alliés de ses membres, dans les degrés exprimés par l’article 139.\par
\labelchar{Art. 147.} Il surveille et assure l’exécution des lois dans les administrations et tribunaux, par des commissaires à sa nomination.\par
\labelchar{Art. 148.} Il nomme hors de son sein les ministres, et les révoque lorsqu’il le juge convenable. Il ne peut les choisir au-dessous de l’âge de trente ans, ni parmi les parents ou alliés de ses membres, aux degrés énoncés dans l’article 139.\par
\labelchar{Art. 149.} Les ministres correspondent immédiatement avec les autorités qui leur sont subordonnées.\par
\labelchar{Art. 150.} Le Corps législatif détermine les attributions et le nombre des ministres.\par
Ce nombre est de six au moins et de huit au plus.\par
\labelchar{Art. 151.} Les ministres ne forment point un Conseil.\par
\labelchar{Art. 152.} Les ministres sont respectivement responsables, tant de l’inexécution des lois, que de l’inexécution des arrêtés du Directoire.\par
\labelchar{Art. 153.} Le Directoire nomme le receveur des impositions directes de chaque département.\par
\labelchar{Art. 154.} Il nomme les préposés en chef aux régies des contributions indirectes et à l’administration des domaines nationaux.\par
\labelchar{Art. 155.} Tous les fonctionnaires publics dans les colonies françaises, excepté les départements des îles de France et de la Réunion, seront nommés par le Directoire jusqu’à la paix.\par
\labelchar{Art. 156.} Le Corps législatif peut autoriser le Directoire à envoyer dans toutes les colonies françaises, suivant l’exigence des cas, un ou plusieurs agents particuliers nommés par lui pour un temps limité.\par
Les agents particuliers exerceront les mêmes fonctions que le Directoire, et lui seront subordonnés.\par
\labelchar{Art. 157.} Aucun membre du Directoire ne peut sortir du territoire de la République, que deux ans après la cessation de ses fonctions.\par
\labelchar{Art. 158.} Il est tenu, pendant cet intervalle, de justifier au Corps législatif de sa résidence.\par
L’article 112 et les suivants, jusqu’à l’article 123 inclusivement, relatifs à la garantie du Corps législatif, sont communs aux membres du Directoire.\par
\labelchar{Art. 159.} Dans le cas où plus de deux membres du Directoire seraient mis en jugement, le Corps législatif pourvoiera dans les formes ordinaires, à leur remplacement provisoire durant le jugement.\par
\labelchar{Art. 160.} Hors les cas des articles 119 et 120, le Directoire, ni aucun de ses membres, ne peut être appelé, ni par le Conseil des Cinq-Cents, ni par le Conseil des Anciens.\par
\labelchar{Art. 161.} Les comptes et les éclaircissements demandés par l’un ou par l’autre Conseil au Directoire, sont fournis par écrit.\par
\labelchar{Art. 162.} Le Directoire est tenu, chaque année, de présenter, par écrit, à l’un et à l’autre Conseil, l’aperçu des dépenses, la situation des finances, l’état des pensions existantes, ainsi que le projet de celles qu’il croit convenable d’établir.\par
Il doit indiquer les abus qui sont à sa connaissance.\par
\labelchar{Art. 163.} Le Directoire peut, en tout cas, inviter, par écrit, le Conseil des Cinq-Cents à prendre un objet en considération ; il peut lui proposer des mesures, mais non des projets rédigés en forme de loi.\par
\labelchar{Art. 164.} Aucun membre du Directoire ne peut s’absenter plus de cinq jours, ni s’éloigner au-delà de quatre myriamètres (huit lieues moyennes), du lieu de la résidence du Directoire, sans l’autorisation du Corps législatif.\par
\labelchar{Art. 165.} Les membres du Directoire ne peuvent paraître, dans l’exercice de leurs fonctions, soit au-dehors, soit dans l’intérieur de leurs maisons, que revêtus du costume qui leur est propre.\par
\labelchar{Art. 166.} Le Directoire a sa garde habituelle, et soldée aux frais de la République, composée de cent vingt hommes à pied, et de cent vingt hommes à cheval.\par
\labelchar{Art. 167.} Le Directoire est accompagné de sa garde dans les cérémonies et marches publiques où il a toujours le premier rang.\par
\labelchar{Art. 168.} Chaque membre du Directoire se fait accompagner au-dehors de deux gardes.\par
\labelchar{Art. 169.} Tout poste de force armée doit au Directoire et à chacun de ses membres les honneurs militaires supérieurs.\par
\labelchar{Art. 170.} Le Directoire a quatre messagers d’Etat, qu’il nomme et qu’il peut destituer.\par
Ils portent aux deux Conseils législatifs les lettres et les mémoires du Directoire ; ils ont entrée à cet effet dans le lieu des séances des Conseils législatifs.\par
Ils marchent précédés de deux huissiers.\par
\labelchar{Art. 171.} Le Directoire réside dans la même commune que le Corps législatif.\par
\labelchar{Art. 172.} Les membres du Directoire sont logés aux frais de la République, et dans un même édifice.\par
\labelchar{Art. 173.} Le traitement de chacun d’eux est fixé, pour chaque année, à la valeur de cinquante mille myriagrammes de froment (dix mille deux cent vingt-deux quintaux).

\section[{TITRE VII. Corps administratifs et municipaux}]{TITRE VII \\
Corps administratifs et municipaux}

\labelchar{Art. 174.} Il y a dans chaque département une administration centrale, et dans chaque canton une administration municipale au moins.\par
\labelchar{Art. 175.} Tout membre d’une administration départementale ou municipale doit être âgé de vingt-cinq ans au moins.\par
\labelchar{Art. 176.} L’ascendant et le descendant en ligne directe, les frères, l’oncle et le neveu, et les alliés aux mêmes degrés, ne peuvent simultanément être membres de la même administration, ni s’y succéder qu’après un intervalle de deux ans.\par
\labelchar{Art. 177.} Chaque administration de département est composée de cinq membres ; elle est renouvelée par cinquième tous les ans.\par
\labelchar{Art. 178.} Toute commune dont la population s’élève depuis cinq mille habitants jusqu’à cent mille, a pour elle seule une administration municipale.\par
\labelchar{Art. 179.} Il y a dans chaque commune, dont la population est inférieure à cinq mille habitants, un agent municipal et un adjoint.\par
\labelchar{Art. 180.} La réunion des agent municipaux de chaque commune forme la municipalité de canton.\par
\labelchar{Art. 181.} Il y a de plus un président de l’administration municipale, choisi dans tout le canton.\par
\labelchar{Art. 182.} Dans les communes, dont la population s’élève de cinq à dix mille habitants, il y a cinq officiers municipaux ; Sept, depuis dix mille jusqu’à cinquante mille ; Neuf, depuis cinquante mille jusqu’à cent mille.\par
\labelchar{Art. 183.} Dans les communes, dont la population excède cent mille habitants, il y a au moins trois administrations municipales.\par
Dans ces communes, la division des municipalités se fait de manière que la population de l’arrondissement de chacune n’excède pas cinquante mille individus, et ne soit pas moindre de trente mille. La municipalité de chaque arrondissement est composée de sept membres.\par
\labelchar{Art. 184.} Il y a, dans les communes divisées en municipalités, un bureau central pour les objets jugés indivisibles par le Corps législatif.\par
Ce bureau est composé de trois membres nommés par l’administration de département, et confirmé par le Pouvoir exécutif.\par
\labelchar{Art. 185.} Les membres de toute administration municipale sont nommés pour deux ans, et renouvelés chaque année par moitié ou par partie la plus approximative de la moitié, et alternativement par la fraction la plus forte et par la fraction la plus faible.\par
\labelchar{Art. 186.} Les administrateurs de département et les membres des administrations municipales peuvent être réélus une fois sans intervalle.\par
\labelchar{Art. 187.} Tout citoyen qui a été deux fois de suite élu administrateur de département ou membre d’une administration municipale, et qui en a rempli les fonctions en vertu de l’une et l’autre élection, ne peut être élu de nouveau qu’après un intervalle de deux années.\par
\labelchar{Art. 188.} Dans le cas où une Administration départementale ou municipale perdrait un ou plusieurs de ses membres par mort, démission ou autrement, les administrateurs restants peuvent s’adjoindre en remplacement des administrateurs temporaires, et qui exercent en cette qualité jusqu’aux élections suivantes.\par
\labelchar{Art. 189.} Les administrations départementales et municipales ne peuvent modifier les actes du Corps législatif, ni ceux du Directoire exécutif, ni en suspendre l’exécution.\par
Elles ne peuvent s’immiscer dans les objets dépendant de l’ordre judiciaire.\par
\labelchar{Art. 190.} Les administrateurs sont essentiellement chargés de la répartition des contributions directes et de la surveillance des deniers provenant des revenus publics dans leur territoire.\par
Le Corps législatif détermine les règles et le mode de leurs fonctions, tant sur ces objets, que sur les autres parties de l’Administration intérieure.\par
\labelchar{Art. 191.} Le Directoire exécutif nomme, auprès de chaque administration départementale et municipale, un commissaire qu’il révoque lorsqu’il le juge convenable.\par
Ce commissaire surveille et requiert l’exécution des lois.\par
\labelchar{Art. 192.} Le commissaire près de chaque administration locale, doit être pris parmi les citoyens domiciliés depuis un an dans le département où cette administration est établie.\par
Il doit être âgé de vingt-cinq ans au moins.\par
\labelchar{Art. 193.} Les administrations municipales sont subordonnées aux administrations de département, et celles-ci aux ministres.\par
En conséquence, les ministres peuvent annuler, chacun dans sa partie, les actes des administrations de département ; et celles-ci, les actes des administrations municipales, lorsque ces actes sont contraires aux lois ou aux ordres des autorités supérieures.\par
\labelchar{Art. 194.} Les ministres peuvent aussi suspendre les administrations de département qui ont contrevenu aux lois ou aux ordres des autorités supérieures ; et les administrations de département ont le même droit à l’égard des membres des administrations municipales.\par
\labelchar{Art. 195.} Aucune suspension ni annulation ne devient définitive sans la confirmation formelle du Directoire exécutif.\par
\labelchar{Art. 196.} Le Directoire peut aussi annuler immédiatement les actes des administrations départementales ou municipales.\par
Il peut suspendre ou destituer immédiatement, lorsqu’il le croit nécessaire, les administrateurs soit de département, soit de canton, et les envoyer devant les tribunaux de département lorsqu’il y a lieu.\par
\labelchar{Art. 197.} Tout arrêté portant cassation d’actes, suspension ou destitution d’administrateur, doit être motivé.\par
\labelchar{Art. 198.} Lorsque les cinq membres d’une administration départementale sont destitués, le Directoire exécutif pourvoit à leur remplacement jusqu’à l’élection suivante ; mais il ne peut choisir leurs suppléants provisoires, que parmi les anciens administrateurs du même département.\par
\labelchar{Art. 199.} Les administrations, soit de département, soit de canton, ne peuvent correspondre entre elles que sur les affaires qui leur sont attribuées par la loi, et non sur les intérêts généraux de la République.\par
\labelchar{Art. 200.} Toute administration doit annuellement le compte de sa gestion.\par
Les comptes rendus par les administrations départementales sont imprimés.\par
\labelchar{Art. 201.} Tous les actes des corps administratifs sont rendus publics par le dépôt du registre où ils sont consignés, et qui est ouvert à tous les administrés.\par
Ce registre est clos tous les six mois, et n’est déposé que du jour qu’il a été clos.\par
Le Corps législatif peut proroger, selon les circonstances, le délai fixé pour ce dépôt.

\section[{TITRE VIII. Pouvoir judiciaire}]{TITRE VIII \\
Pouvoir judiciaire}


\subsection[{Dispositions générales}]{Dispositions générales}

\labelchar{Art. 202.} Les fonctions judiciaires ne peuvent être exercées, ni par le Corps législatif, ni par le Pouvoir exécutif.\par
\labelchar{Art. 203.} Les juges ne peuvent s’immiscer dans l’exercice du Pouvoir législatif, ni faire aucun règlement.\par
Ils ne peuvent arrêter ou suspendre l’exécution d’aucune loi, ni citer devant eux les administrateurs pour raison de leurs fonctions.\par
\labelchar{Art. 204.} Nul ne peut être distrait des juges que la loi lui assigne, par aucune commission, ni par d’autres attributions que celles qui sont déterminées par une loi antérieure.\par
\labelchar{Art. 205.} La justice est rendue gratuitement.\par
\labelchar{Art. 206.} Les juges ne peuvent être destitués que pour forfaiture légalement jugée, ni suspendus que par une accusation admise.\par
\labelchar{Art. 207.} L’ascendant et le descendant en ligne directe, les frères, l’oncle et le neveu, les cousins au premier degré, et les alliés à ces divers degrés, ne peuvent être simultanément membres du même tribunal.\par
\labelchar{Art. 208.} Les séances des tribunaux sont publiques ; les juges délibèrent en secret ; les jugements sont prononcés à haute voix ; ils sont motivés, et on y énonce les termes de la loi appliquée.\par
\labelchar{Art. 209.} Nul citoyen, s’il n’a l’âge de trente ans accomplis, ne peut être élu juge d’un tribunal de département, ni juge de paix, ni assesseur de juge de paix, ni juge d’un tribunal de commerce, ni membre du Tribunal de cassation, ni juré, ni commissaire du Directoire exécutif près les tribunaux.

\subsection[{De la Justice civile}]{De la Justice civile}

\labelchar{Art. 210.} Il ne petit être porté atteinte au droit de faire prononcer sur les différends par des arbitres du choix des parties.\par
\labelchar{Art. 211.} La décision de ces arbitres est sans appel, et sans recours en cassation, si les parties ne l’ont expressément réservé.\par
\labelchar{Art. 212.} Il y a, dans chaque arrondissement déterminé par la loi un juge de paix et ses assesseurs.\par
Ils sont tous élus pour deux ans, et peuvent être immédiatement et indéfiniment réélus.\par
\labelchar{Art. 213.} La loi détermine les objets dont les juges de paix et leurs assesseurs connaissent en dernier ressort.\par
Elle leur en attribue d’autres qu’ils jugent à la charge de l’appel.\par
\labelchar{Art. 214.} Il y a des tribunaux particuliers pour le commerce de terre et de mer ; la loi détermine les lieux où il est utile de les établir.\par
Leur pouvoir de juger en dernier ressort ne peut être étendu au-delà de la valeur de cinq cents myriagrammes de froment (cent deux quintaux, vingt-deux livres).\par
\labelchar{Art. 215.} Les affaires dont le jugement n’appartient ni aux juges de paix ni aux tribunaux de commerce, soit en dernier ressort, soit à la charge d’appel, sont portées immédiatement devant le juge de paix et ses assesseurs, pour être conciliées.\par
Si le juge de paix ne peut les concilier, il les renvoie devant le tribunal civil.\par
\labelchar{Art. 216.} Il y a un tribunal civil par département.\par
Chaque tribunal civil est composé de vingt juges au moins, d’un commissaire et d’un substitut nommés et destituables par le Directoire exécutif, et d’un greffier.\par
Tous les cinq ans on procède à l’élection de tous les membres du tribunal.\par
Les juges peuvent être réélus.\par
\labelchar{Art. 217.} Lors de l’élection des juges, il est nommé cinq suppléants, dont trois sont pris parmi les citoyens résidant dans la commune où siège le tribunal.\par
\labelchar{Art. 218.} Le tribunal civil prononce en dernier ressort, dans les cas déterminés par la loi, sur les appels des jugements soit des juges de paix, soit des arbitres, soit des tribunaux de commerce.\par
\labelchar{Art. 219.} L’appel des jugements prononcés par le tribunal civil se porte au tribunal civil de l’un des trois départements les plus voisins, ainsi qu’il est déterminé par la loi.\par
\labelchar{Art. 220.} Le tribunal civil se divise en sections.\par
Une section ne peut juger au-dessous du nombre de cinq juges.\par
\labelchar{Art. 221.} Les juges réunis dans chaque tribunal nomment, entre eux, au scrutin secret le président de chaque section.

\subsection[{De la Justice correctionnelle et criminelle}]{De la Justice correctionnelle et criminelle}

\labelchar{Art. 222.} Nul ne peut être saisi que pour être conduit devant l’officier de police ; et nul ne peut être mis en arrestation ou détenu qu’en vertu, d’un mandat d’arrêt des officiers de police, ou du Directoire exécutif, dans le cas de l’article 145, ou d’une ordonnance de prise de corps, soit d’un tribunal, soit du directeur du jury d’accusation, ou d’un décret d’accusation du Corps législatif, dans le cas où il lui appartient de la prononcer, ou d’un jugement de condamnation à la prison ou détention correctionnelle.\par
\labelchar{Art. 223.} Pour que l’acte qui ordonne l’arrestation puisse être exécuté, il faut :\par

\begin{enumerate}[itemsep=0pt,topsep=0pt,partopsep=0pt,parskip=0pt]
\item Qu’il exprime formellement le motif de l’arrestation, et la loi en conformité de laquelle elle est ordonnée ;
\item Qu’il ait été notifié à celui qui en est l’objet, et qu’il lui en ait été laissé copie.
\end{enumerate}

\labelchar{Art. 224.} Toute personne saisie et conduite devant l’officier de police sera examinée sur-le-champ, ou dans le jour au plus tard.\par
\labelchar{Art. 225.} S’il résulte de l’examen qu’il n’y a aucun sujet d’inculpation contre elle, elle sera remise aussitôt en liberté ; ou, s’il y a lieu de l’envoyer à la maison d’arrêt, elle y sera conduite dans le plus bref délai, qui, en aucun cas, ne pourra excéder trois jours.\par
\labelchar{Art. 226.} Nulle personne arrêtée ne peut être retenue, si elle donne caution suffisante, dans tous les cas où la loi permet de rester libre sous le cautionnement.\par
\labelchar{Art. 227.} Nulle personne, dans le cas où sa détention est autorisée par la loi, ne peut être conduite on détenue que dans les lieux légalement et publiquement désignés pour servir de maison d’arrêt, de maison de justice ou de maison de détention.\par
\labelchar{Art. 228.} Nul gardien ou geôlier ne peut recevoir ni retenir aucune personne qu’en vertu d’un mandat d’arrêt, selon les formes prescrites par les articles 222 et 223, d’une ordonnance de prise de corps, d’un décret d’accusation ou d’un jugement de condamnation à prison ou détention correctionnelle, et sans que la transcription en ait été faite sur son registre.\par
\labelchar{Art. 229.} Tout gardien ou geôlier est tenu, sans qu’aucun ordre puisse l’en dispenser, de présenter la personne détenue à l’officier civil ayant la police de la maison de détention, toutes les fois qu’il en sera requis par cet officier.\par
\labelchar{Art. 230.} La représentation de la personne détenue ne pourra être refusée à ses parents et amis porteurs de l’ordre de l’officier civil, lequel sera toujours tenu de l’accorder, à moins que le gardien ou geôlier ne représente une ordonnance du juge, transcrite sur son registre, pour tenir la personne arrêtée au secret.\par
\labelchar{Art. 231.} Tout homme, quelle que soit sa place ou son emploi, autre que ceux à qui la loi donne le droit d’arrestation, qui donnera, signera, exécutera ou fera exécuter l’ordre d’arrêter un individu, recevra ou retiendra un individu dans un lieu de détention non publiquement et légalement désigné et tous les gardiens ou geôliers qui contreviendront aux dispositions des trois articles précédents, seront coupables du crime de détention arbitraire.\par
\labelchar{Art. 232.} Toutes rigueurs employées dans les arrestations, détentions ou exécutions, autres que celles prescrites par la loi, sont des crimes.\par
\labelchar{Art. 233.} Il y a dans chaque département, pour le jugement des délits dont la peine n’est ni afflictive ni infamante, trois tribunaux correctionnels au moins, et six au plus.\par
Ces tribunaux ne pourront prononcer de peines plus graves que l’emprisonnement pour deux années.\par
La connaissance des délits dont la peine n’excède pas, soit la valeur de trois journées de travail, soit un emprisonnement de trois jours, est déléguée au juge de paix, qui prononce en dernier ressort.\par
\labelchar{Art. 234.} Chaque tribunal correctionnel est composé d’un président, de deux juges de paix ou assesseurs de juges de paix de la commune où il est établi, d’un commissaire du Pouvoir exécutif, nommé et destituable par le Directoire exécutif et d’un greffier.\par
\labelchar{Art. 235.} Le président de chaque tribunal correctionnel est pris tous les six mois, et par tour, parmi les membres des sections du tribunal civil du département, les présidents exceptés.\par
\labelchar{Art. 236.} Il y a appel des jugements du tribunal correctionnel par-devant le tribunal criminel du département.\par
\labelchar{Art. 237.} En matière de délits emportant peine afflictive ou infamante, nulle personne ne peut être jugée que sur une accusation admise par les jurés ou décrétée par le Corps législatif, dans le cas où il lui appartient de décréter l’accusation.\par
\labelchar{Art. 238.} Un premier jury déclare si l’accusation doit être admise, ou rejetée : le fait est reconnu par un second jury, et la peine déterminée par la loi est appliquée par des tribunaux criminels.\par
\labelchar{Art. 239.} Les jurés ne votent que par scrutin secret.\par
\labelchar{Art. 240.} Il y a dans chaque département autant de jurys d’accusation que de tribunaux correctionnels.\par
Les présidents des tribunaux correctionnels en sont les directeurs, chacun dans son arrondissement.\par
Dans les communes au-dessus de cinquante mille âmes, il pourra être établi par la loi, outre le président du tribunal correctionnel, autant de directeurs de jurys d’accusation que l’expédition des affaires l’exigera.\par
\labelchar{Art. 241.} Les fonctions de commissaire du Pouvoir exécutif et de greffier près le directeur du jury d’accusation, sont remplies par le commissaire et par le greffier du tribunal correctionnel.\par
\labelchar{Art. 242.} Chaque directeur du jury d’accusation a la surveillance immédiate de tous les officiers de police de son arrondissement.\par
\labelchar{Art. 243.} Le directeur du jury poursuit immédiatement, comme officier de police, sur les dénonciations que lui fait l’accusateur public, soit d’office, soit d’après les ordres du Directoire exécutif :\par

\begin{enumerate}[itemsep=0pt,topsep=0pt,partopsep=0pt,parskip=0pt]
\item Les attentats contre la liberté ou la sûreté individuelle des citoyens ;
\item Ceux commis contre le droit des gens ;
\item La rébellion à l’exécution, soit des jugements, soit de tous les actes exécutoires émanés des autorités constituées ;
\item Les troubles occasionnés et les voies de fait commises pour entraver la perception des contributions, la libre circulation des subsistances et des autres objets de commerce.
\end{enumerate}

\labelchar{Art. 244.} Il y a un tribunal criminel pour chaque département.\par
\labelchar{Art. 245.} Le tribunal criminel est composé d’un président, d’un accusateur public, de quatre juges pris dans le tribunal civil, du commissaire du Pouvoir exécutif près le même tribunal, ou de son substitut et d’un greffier.\par
Il y a dans le tribunal criminel du département de la Seine, un vice-président et un substitut de l’accusateur public : ce tribunal est divisé en deux sections ; huit membres du tribunal civil y exercent les fonctions de juges.\par
\labelchar{Art. 246.} Les présidents des sections du tribunal civil ne peuvent remplir les fonctions de juges au tribunal criminel.\par
\labelchar{Art. 247.} Les autres juges y font le service, chacun à son tour, pendant six mois, dans l’ordre de leur nomination, et ils ne peuvent pendant ce temps exercer aucune fonction au tribunal civil.\par
\labelchar{Art. 248.} L’accusateur public est chargé :\par

\begin{enumerate}[itemsep=0pt,topsep=0pt,partopsep=0pt,parskip=0pt]
\item De poursuivre les délits sur les actes d’accusation admis par les premiers jurés ;
\item De transmettre aux officiers de police les dénonciations qui lui sont adressées directement ;
\item De surveiller les officiers de police du département, et d’agir contre eux suivant la loi, en cas de négligence ou de faits plus graves.
\end{enumerate}

\labelchar{Art. 249.} Le commissaire du Pouvoir exécutif est chargé :\par

\begin{enumerate}[itemsep=0pt,topsep=0pt,partopsep=0pt,parskip=0pt]
\item De requérir, dans le cours de l’instruction, pour la régularité des formes, et avant le jugement, pour l’application de la loi ;
\item De poursuivre l’exécution des jugements rendus par le tribunal criminel.
\end{enumerate}

\labelchar{Art. 250.} Les juges ne peuvent proposer aux jurés aucune question complexe.\par
\labelchar{Art. 251.} Le jury de jugement est de douze jurés au moins : l’accusé a la faculté d’en récuser, sans donner de motifs, un nombre que la loi détermine.\par
\labelchar{Art. 252.} L’instruction devant le jury de jugement est publique, et l’on ne peut refuser aux accusés le secours d’un conseil qu’ils ont la faculté de choisir, ou qui leur est nommé d’office.\par
\labelchar{Art. 253.} Toute personne acquittée par un jury légal ne peut être reprise ni accusée pour le même fait.

\subsection[{Tribunal de cassation}]{Tribunal de cassation}

\labelchar{Art. 254.} Il y a pour toute la République un Tribunal de cassation.\par
Il prononce :\par

\begin{enumerate}[itemsep=0pt,topsep=0pt,partopsep=0pt,parskip=0pt]
\item Sur les demandes en cassation contre les jugements en dernier ressort rendus par les tribunaux ;
\item Sur les demandes en renvoi d’un tribunal à un autre, pour cause de suspicion légitime ou de sûreté publique ;
\item Sur les règlements de juges et les prises à partie contre un tribunal entier.
\end{enumerate}

\labelchar{Art. 255.} Le Tribunal de cassation ne peut jamais connaître du fond des affaires ; mais il casse les jugements rendus sur des procédures dans lesquelles les formes ont été violées, ou qui contiennent quelque contravention expresse à la loi, et il renvoie le fond du procès au tribunal qui doit en connaître.\par
\labelchar{Art. 256.} Lorsque, après une cassation, le second jugement sur le fond est attaqué par les mêmes moyens que le premier, la question ne peut plus être agitée au Tribunal de cassation, sans avoir été soumise au Corps législatif, qui porte une loi à laquelle le Tribunal de cassation est tenu de se conformer.\par
\labelchar{Art. 257.} Chaque année, le Tribunal de cassation est tenu d’envoyer à chacune des sections du Corps législatif une députation qui lui présente l’état des jugements rendus, avec la notice en marge, et le texte de la loi qui a déterminé le jugement.\par
\labelchar{Art. 258.} Le nombre des juges du Tribunal de cassation ne peut excéder les trois quarts du nombre des départements.\par
\labelchar{Art. 259.} Ce Tribunal est renouvelé par cinquième tous les ans.\par
Les Assemblées électorales des départements nomment successivement et alternativement les juges qui doivent remplacer ceux qui sortent du Tribunal de cassation.\par
Les juges de ce Tribunal peuvent toujours être réélus.\par
\labelchar{Art. 260.} Chaque juge du Tribunal de cassation a un suppléant élu par la même Assemblée électorale.\par
\labelchar{Art. 261.} Il y a près du Tribunal de cassation un commissaire et des substituts nommés et destituables par le Directoire exécutif.\par
\labelchar{Art. 262.} Le Directoire exécutif dénonce au Tribunal de cassation, par la voie de son commissaire, et sans préjudice du droit des parties intéressées, les actes par lesquels les juges ont excédé leurs pouvoirs.\par
\labelchar{Art. 263.} Le Tribunal annule ces actes ; et s’ils donnent lieu à la forfaiture, le fait est dénoncé au Corps législatif, qui rend le décret d’accusation, après avoir entendu ou appelé les prévenus.\par
\labelchar{Art. 264.} Le Corps législatif ne peut annuler les jugements du Tribunal de cassation, sauf à poursuivre personnellement les juges qui auraient encouru la forfaiture.

\subsection[{Haute Cour de justice}]{Haute Cour de justice}

\labelchar{Art. 265.} Il y a une Haute Cour de justice pour juger les accusations admises par le Corps législatif, soit contre ses propres membres, soit contre ceux du Directoire exécutif.\par
\labelchar{Art. 266.} La Haute Cour de justice est composée de cinq juges et de deux accusateurs nationaux tirés du Tribunal de cassation, et de hauts jurés nommés par les assemblées électorales des départements.\par
\labelchar{Art. 267.} La Haute Cour de justice ne se forme qu’en vertu d’une proclamation du Corps législatif, rédigée et publiée par le Conseil des Cinq-Cents.\par
\labelchar{Art. 268.} Elle se forme et tient ses séances dans le lieu désigné par la proclamation du Conseil des Cinq-Cents.\par
Ce lieu ne peut être plus près qu’à douze myriamètres de celui où réside le Corps législatif.\par
\labelchar{Art. 269.} Lorsque le Corps législatif a proclamé la formation de la Haute Cour de justice, le Tribunal de cassation tire au sort quinze de ses membres dans une séance publique ; il nomme de suite, dans la même séance, par la voie du scrutin secret, cinq de ces quinze : les cinq juges ainsi nommés sont les juges de la Haute Cour de justice ; ils choisissent entre eux un président.\par
\labelchar{Art. 270.} Le Tribunal de cassation nomme, dans la même séance, par scrutin, à la majorité absolue, deux de ses membres pour remplir à la Haute Cour de justice les fonctions d’accusateurs nationaux.\par
\labelchar{Art. 271.} Les actes d’accusation sont dressés et rédigés par le Conseil des Cinq-Cents.\par
\labelchar{Art. 272.} Les Assemblées électorales de chaque département nomment, tous les ans, un jury pour la Haute Cour de justice.\par
\labelchar{Art. 273.} Le Directoire exécutif fait imprimer et publier, un mois après l’époque des élections, la liste des jurés nommés par la Haute Cour de justice.

\section[{TITRE IX. De la force armée}]{TITRE IX \\
De la force armée}

\labelchar{Art. 274.} La force armée est instituée pour défendre l’Etat contre les ennemis du dehors, et pour assurer au-dedans le maintien de l’ordre et l’exécution des lois.\par
\labelchar{Art. 275.} La force publique est essentiellement obéissante : nul corps armé ne peut délibérer.\par
\labelchar{Art. 276.} Elle se distingue en garde nationale sédentaire et garde nationale en activité.\par

\subsection[{De la garde nationale sédentaire}]{De la garde nationale sédentaire}

\labelchar{Art. 277.} La garde nationale sédentaire est composée de tous les citoyens et fils de citoyens en état de porter les armes.\par
\labelchar{Art. 278.} Son organisation et sa discipline sont les mêmes pour toute la République ; elles sont déterminées par la loi.\par
\labelchar{Art. 279.} Aucun Français ne peut exercer les droits de citoyen, s’il n’est inscrit au rôle de la garde nationale sédentaire.\par
\labelchar{Art. 280.} Les distinctions de garde et la subordination n’y subsistent que relativement au service et pendant sa durée.\par
\labelchar{Art. 281.} Les officiers de la garde nationale sédentaire sont élus à temps par les citoyens qui la composent et ne peuvent être réélus qu’après un intervalle.\par
\labelchar{Art. 282.} Le commandement de la garde nationale d’un département entier ne peut être confié habituellement à un seul citoyen.\par
\labelchar{Art. 283.} S’il est jugé nécessaire de rassembler toute la garde nationale d’un département, le Directoire exécutif peut nommer un commandement temporaire.\par
\labelchar{Art. 284.} Le commandement de la garde nationale sédentaire, dans une ville de cent mille habitants et au-dessus, ne peut être habituellement confié à un seul homme.

\subsection[{De la garde nationale en activité}]{De la garde nationale en activité}

\labelchar{Art. 285.} La République entretient à sa solde, même en temps de paix, sous le nom de gardes nationales en activité, une armée de terre et de mer.\par
\labelchar{Art. 286.} L’armée se forme par enrôlements, volontaires, et, en cas de besoin, par le mode que la loi détermine.\par
\labelchar{Art. 287.} Aucun étranger qui n’a point acquis les droits de citoyen français, ne peut être admis dans les armées françaises, à moins qu’il n’ait fait une ou plusieurs campagnes pour l’établissement de la République.\par
\labelchar{Art. 288.} Les commandants en chef de terre et de mer ne sont nommés qu’en cas de guerre ; ils reçoivent du Directoire exécutif des commissions révocables à volonté. La durée de ces commissions se borne à une campagne ; mais elles peuvent être continuées.\par
\labelchar{Art. 289.} Le commandement général des armées de la République ne peut être confié à un seul homme.\par
\labelchar{Art. 290.} L’armée de terre et de mer est soumise à des lois particulières, pour la discipline, la forme des jugements et la nature des peines.\par
\labelchar{Art. 291.} Aucune partie de la garde nationale sédentaire, ni de la garde nationale en activité, ne peut agir, pour le service intérieur de la République, que sur la réquisition par écrit de l’autorité civile, dans les formes prescrites par la loi.\par
\labelchar{Art. 292.} La force publique ne peut être requise par les autorités civiles que dans l’étendue de leur territoire ; elle ne peut se transporter d’un canton dans un autre, sans y être autorisée par l’administration du département, ni d’un département dans un autre, sans les ordres du Directoire exécutif.\par
\labelchar{Art. 293.} Néanmoins le Corps législatif détermine les moyens d’assurer par la force publique l’exécution des jugements et la poursuite des accusés sur le territoire français.\par
\labelchar{Art. 294.} En cas de danger imminent, l’administration municipale d’un canton peut requérir la garde nationale des cantons voisins ; en ce cas, l’administration qui a requis et les chefs des gardes nationales qui ont été requises, sont également tenus d’en rendre compte au même instant à l’administration départementale.\par
\labelchar{Art. 295.} Aucune troupe étrangère ne peut être introduite sur le territoire français, sans le consentement préalable du Corps législatif.

\section[{TITRE X. Instruction publique}]{TITRE X \\
Instruction publique}

\labelchar{Art. 296.} Il y a dans la République des écoles primaires où les élèves apprennent à lire, à écrire, les éléments du calcul et ceux de la morale. La République pourvoit aux frais de logement des instituteurs préposés à ces écoles.\par
\labelchar{Art. 297.} Il y a, dans les diverses parties de la République, des écoles supérieures aux écoles primaires, et dont le nombre sera tel, qu’il y en ait au moins une pour deux départements.\par
\labelchar{Art. 298.} Il y a, pour toute la République, un institut national chargé de recueillir les découvertes, de perfectionner les arts et les sciences.\par
\labelchar{Art. 299.} Les divers établissements d’instruction publique n’ont entre eux aucun rapport de subordination, ni de correspondance administrative.\par
\labelchar{Art. 300.} Les citoyens ont le droit de former des établissements particuliers d’éducation et d’instruction, que des sociétés libres pour concourir aux progrès des sciences, des lettres et des arts.\par
\labelchar{Art. 301.} Il sera établi des fêtes nationales, pour entretenir la fraternité entre les citoyens et les attacher à la Constitution, à la patrie et aux lois.

\section[{TITRE XI. Finances}]{TITRE XI \\
Finances}


\subsection[{Contributions}]{Contributions}

\labelchar{Art. 302.} Les contributions publiques sont délibérées et fixées chaque année par le Corps législatif. A lui seul apparient d’en établir. Elles ne peuvent subsister au-delà d’un an, si elles ne sont expressément renouvelées.\par
\labelchar{Art. 303.} Le Corps législatif peut créer tel genre de contribution qu’il croira nécessaire ; mais il doit établir chaque année une imposition foncière et une imposition personnelle.\par
\labelchar{Art. 304.} Tout individu qui, n’étant pas dans le cas des articles 12 et 13 de la Constitution, n’a pas été compris au rôle des contributions directes, a le droit de se présenter à l’administration municipale de sa commune, et de s’y inscrire pour une contribution personnelle égale à la valeur locale de trois journées de travail agricole.\par
\labelchar{Art. 305.} L’inscription mentionnés dans l’article précédent ne peut se faire que durant le mois de messidor de chaque année.\par
\labelchar{Art. 306.} Les contributions de toute nature sont réparties entre tous les contribuables à raison de leurs facultés.\par
\labelchar{Art. 307.} Le Directoire exécutif dirige et surveille la perception et le versement des contributions, et donne à cet effet tous les ordres nécessaires.\par
\labelchar{Art. 308.} Les comptes détaillés de la dépense des ministres, signés et certifiés par eux, sont rendus publics au commencement de chaque année.\par
Il en sera de même des états de recette des diverses contributions, et de tous les revenus publics.\par
\labelchar{Art. 309.} Les états de ces dépenses et recettes sont distingués suivant leur nature ; ils expriment les sommes touchées et dépensées, année par année, dans chaque partie d’administration générale.\par
\labelchar{Art. 310.} Sont également publiés les comptes des dépenses particulières aux départements, et relatives aux tribunaux, aux administrations, au progrès des sciences, à tous les travaux et établissements publics.\par
\labelchar{Art. 311.} Les administrations de département et les municipalités ne peuvent faire aucune répartition au-delà des sommes fixées par le Corps législatif, ni délibérer ou permettre, sans être autorisées par lui, aucun emprunt local à la charge des citoyens du département, de la commune et du canton.\par
\labelchar{Art. 312.} Au Corps législatif seul appartient le droit de régler la fabrication et l’émission de toute espèce de monnaies, d’en fixer la valeur et le poids, et d’en déterminer le type.\par
\labelchar{Art. 313.} Le Directoire surveille la fabrication des monnaies, et nomme les officiers chargés d’exercer immédiatement cette inspection.\par
\labelchar{Art. 314.} Le Corps législatif détermine les contributions des colonies et leurs rapports commerciaux avec la métropole.

\subsection[{Trésorerie nationale et comptabilité}]{Trésorerie nationale et comptabilité}

\labelchar{Art. 315.} Il y a cinq commissaires de la Trésorerie nationale, élus par le Conseil des Anciens, sur une liste triple présentée par celui des Cinq-Cents.\par
\labelchar{Art. 316.} La durée de leurs fonctions est de cinq années : l’un d’eux est renouvelé tous les ans, et peut être réélu sans intervalle et indéfiniment.\par
\labelchar{Art. 317.} Les commissaires de la Trésorerie sont chargés de surveiller la recette de tous les deniers nationaux ;\par
D’ordonner les mouvements de fonda et le paiement de toutes les dépenses publiques consenties par le Corps législatif ;\par
De tenir un compte ouvert de dépense et de recette avec le receveur des contributions directes de chaque département, avec les différentes régies nationales, et avec les payeurs qui seraient établis dans les départements ;\par
D’entretenir avec lesdits receveurs et payeurs, avec les régies et administrations, la correspondance nécessaire pour assurer la rentrée exacte et régulière des fonds.\par
\labelchar{Art. 318.} Ils ne peuvent rien faire payer, sous peine de forfaiture, qu’en vertu :\par

\begin{enumerate}[itemsep=0pt,topsep=0pt,partopsep=0pt,parskip=0pt]
\item D’un décret du Corps législatif, et jusqu’à concurrence des fonds décrétés par lui sur chaque objet ;
\item D’une décision du Directoire ;
\item De la signature du ministre qui ordonne la dépense.
\end{enumerate}

\labelchar{Art. 319.} Ils ne peuvent, aussi sous peine de forfaiture, approuver aucun paiement, si le mandat, signé par le ministre que ce genre de dépense concerne, n’énonce pas la date, tant de la décision du Directoire exécutif, que des décrets du Corps législatif, qui autorisent le paiement.\par
\labelchar{Art. 320.} Les receveurs des contributions directes de chaque département, les différentes régies nationales, et les payeurs dans les départements, remettent à la Trésorerie nationale leurs comptes respectifs : la Trésorerie les vérifie et les arrête.\par
\labelchar{Art. 321.} Il y a cinq commissaires de la comptabilité nationale, élus par le Corps législatif, aux mêmes époques et selon les mêmes formes et conditions que les commissaires de la Trésorerie.\par
\labelchar{Art. 322.} Le compte général des recettes et des dépenses de la République, appuyé des comptes particuliers et des pièces justificatives, est présenté par les commissaires de la Trésorerie aux commissaires de la comptabilité, qui le vérifient et l’arrêtent.\par
\labelchar{Art. 323.} Les commissaires de la Comptabilité donnent connaissance au Corps législatif des abus, malversations, et de tous les cas de responsabilité qu’ils découvrent dans le cours de leurs opérations ; ils proposent dans leur partie les mesures convenables aux intérêts de la République.\par
\labelchar{Art. 324.} Le résultat des comptes arrêtés par les commissaires de la Comptabilité est imprimé et rendu public.\par
\labelchar{Art. 325.} Les commissaires, tant de la Trésorerie nationale que de la comptabilité, ne peuvent être suspendus ni destitués que par le Corps législatif. Mais, durant l’ajournement du Corps législatif, le Directoire exécutif peut suspendre et remplacer provisoirement les commissaires de la Trésorerie nationale au nombre de deux au plus, à charge d’en référer à l’un et l’autre Conseil du Corps législatif, aussitôt qu’ils ont repris leurs séances.\par
\bigbreak

\section[{TITRE XII. Relations extérieures}]{TITRE XII \\
Relations extérieures}

\labelchar{Art. 326.} La guerre ne peut être décidée que par un décret du Corps législatif, sur la proposition formelle et nécessaire du Directoire exécutif.\par
\labelchar{Art. 327.} Les deux Conseils législatifs concourent, dans les formes ordinaires, au décret par lequel la guerre est décidée.\par
\labelchar{Art. 328.} En cas d’hostilités imminentes ou commencées, de menaces ou de préparatifs de guerre contre la République française, le Directoire exécutif est tenu d’employer, pour la défense de l’Etat, les moyens mis à sa disposition, à la charge d’en prévenir sans délai le Corps législatif.\par
Il peut même indiquer, en ce cas, les augmentations de force et les nouvelles dispositions législatives que les circonstances pourraient exiger.\par
\labelchar{Art. 329.} Le Directoire seul peut entretenir des relations politiques au-dehors, conduire les négociations, distribuer les forces de terre et de mer, ainsi qu’il le juge convenable, et en régler la direction en cas de guerre.\par
\labelchar{Art. 330.} Il est autorisé à faire les stipulations préliminaires, telles que des armistices, des neutralisations ; il peut arrêter aussi des conventions secrètes.\par
\labelchar{Art. 331.} Le Directoire exécutif arrête, signe ou fait signer avec les puissances étrangères, tous les traités de paix, d’alliance, de trêve, de neutralité, de commerce, et autres conventions qu’il juge nécessaires au bien de l’Etat.\par
Ces traités et conventions sont négociés au nom de la République française, par des agents diplomatiques nommés par le Directoire exécutif, et chargés de ses instructions.\par
\labelchar{Art. 332.} Dans le cas où un traité renferme des articles secrets, les dispositions de ces articles ne peuvent être destructives des articles patents, ni contenir aucune aliénation du territoire de la République.\par
\labelchar{Art. 333.} Les traités ne sont valables qu’après avoir été examinés et ratifiés par le Corps législatif ; néanmoins les conditions secrètes peuvent recevoir provisoirement leur exécution dès l’instant même où elles sont arrêtées par le Directoire.\par
\labelchar{Art. 334.} L’un et l’autre Conseils législatifs ne délibèrent sur la guerre ni sur la paix, qu’en comité général.\par
\labelchar{Art. 335.} Les étrangers établis ou non en France, succèdent à leurs parents étrangers ou français ; ils peuvent contracter, acquérir et recevoir des biens situés en France, et en disposer, de même que les citoyens français, par tous les moyens autorisés par les lois.

\section[{TITRE XIII. Révision de la Constitution}]{TITRE XIII \\
Révision de la Constitution}

\labelchar{Art. 336.} Si l’expérience faisait sentir les inconvénients de quelques articles de la Constitution, le Conseil des Anciens en proposerait la révision.\par
\labelchar{Art. 337.} La proposition du Conseil des Anciens est, en ce cas, soumise à la ratification du Conseil des Cinq-Cents.\par
\labelchar{Art. 338.} Lorsque, dans un espace de neuf années, la proposition du Conseil des Anciens, ratifiée par le Conseil des Cinq-Cents, a été faite à trois époques éloignées l’une de l’autre de trois années au moins, une Assemblée de révision est convoquée.\par
\labelchar{Art. 339.} Cette Assemblée est formée de deux membres par département, tous élus de la même manière que les membres du Corps législatif, et réunissant les mêmes conditions que celles exigées par le Conseil des Anciens.\par
\labelchar{Art. 340.} Le Conseil des Anciens désigne, pour la réunion de l’Assemblée de révision, un lieu distant de 20 myriamètres au moins de celui où siège le Corps législatif.\par
\labelchar{Art. 341.} L’Assemblée de révision a le droit de changer le lieu de sa résidence, en observant la distance prescrite par l’article précédent.\par
\labelchar{Art. 342.} L’Assemblée de révision n’exerce aucune fonction législative ni de gouvernement ; elle se borne à la révision des seuls articles constitutionnels qui lui ont été désignée par le Corps législatif.\par
\labelchar{Art. 343.} Tous les articles de la Constitution, sans exception, continuent d’être en vigueur tant que les changements proposés par l’Assemblée de révision n’ont pas été acceptés par le peuple.\par
\labelchar{Art. 344.} Les membres de l’Assemblée de révision délibèrent en commun.\par
\labelchar{Art. 345.} Les citoyens qui sont membres du Corps législatif au moment où une Assemblée de révision est convoquée, ne peuvent être élus membres de cette Assemblée.\par
\labelchar{Art. 346.} L’Assemblée de révision adresse immédiatement aux Assemblées primaires le projet de réforme qu’elle a arrêté.\par
Elle est dissoute dès que ce projet leur a été adressé.\par
\labelchar{Art. 347.} En aucun cas, la durée de l’Assemblée de révision ne peut excéder trois mois.\par
\labelchar{Art. 348.} Les membres de l’Assemblée de révision ne peuvent être recherchés, accusés ni jugés, en aucun temps, pour ce qu’ils ont dit ou écrit dans l’exercice de leurs fonctions.\par
Pendant la durée de ces fonctions, ils ne peuvent être mis en jugement, si ce n’est par une décision des membres mêmes de l’Assemblée de révision.\par
\labelchar{Art. 349.} L’Assemblée de révision n’assiste à aucune cérémonie publique ; ses membres reçoivent la même indemnité que celle des membres du Corps législatif.\par
\labelchar{Art. 350.} L’Assemblée de révision a le droit d’exercer ou faire exercer la police dans la commune où elle réside.

\section[{TITRE XIV. Dispositions générales}]{TITRE XIV \\
Dispositions générales}

\labelchar{Art. 351.} Il n’existe entre les citoyens d’autre supériorité que celle des fonctionnaires publics, et relativement à l’exercice de leurs fonctions.\par
\labelchar{Art. 352.} La loi ne reconnaît ni vœux religieux, ni aucun engagement contraire aux droits naturels de l’homme.\par
\labelchar{Art. 353.} Nul ne peut être empêché de dire, écrire, imprimer et publier sa pensée.\par
Les écrits ne peuvent être soumis à aucune censure avant leur publication.\par
Nul ne peut être responsable de ce qu’il a écrit ou publié, que dans les cas prévus par la loi.\par
\labelchar{Art. 354.} Nul ne peut être empêché d’exercer, en se conformant aux lois, le culte qu’il a choisi.\par
Nul ne peut être forcé de contribuer aux dépenses d’un culte. La République n’en salarie aucun.\par
\labelchar{Art. 355.} Il n’y a ni privilège, ni maîtrise, ni jurande, ni limitation à la liberté de la presse, du commerce, et à l’exercice de l’industrie et des arts de toute espèce.\par
Toute loi prohibitive en ce genre, quand les circonstances la rendent nécessaire, est essentiellement provisoire, et n’a d’effet que pendant un an au plus, à moins qu’elle ne soit formellement renouvelée.\par
\labelchar{Art. 356.} La loi surveille particulièrement les professions qui intéressent les mœurs publiques, la sûreté et la santé des citoyens ; mais on ne peut faire dépendre l’admission à l’exercice de ces professions, d’aucune prestation pécuniaire.\par
\labelchar{Art. 357.} La loi doit pourvoir à la récompense des inventeurs ou au maintien de la propriété exclusive de leurs découvertes ou de leurs productions.\par
\labelchar{Art. 358.} La Constitution garantit l’inviolabilité de toutes les propriétés, ou la juste indemnité de celles dont la nécessité publique, légalement constatée, exigerait le sacrifice.\par
\labelchar{Art. 359.} La maison de chaque citoyen est un asile inviolable : pendant la nuit, nul n’a le droit d’y entrer que dans le cas d’incendie, d’inondation, ou de réclamation venant de l’intérieur de la maison.\par
Pendant le jour, on peut y exécuter les ordres des autorités constituées.\par
Aucune visite domiciliaire ne peut avoir lieu qu’en vertu d’une loi, et pour la personne ou l’objet expressément désigné dans l’acte qui ordonne la visite.\par
\labelchar{Art. 360.} Il ne peut être formé de corporations ni d’associations contraires, à l’ordre public.\par
\labelchar{Art. 361.} Aucune assemblée de citoyens ne peut se qualifier de société populaire.\par
\labelchar{Art. 362.} Aucune société particulière, s’occupant de questions politiques, ne peut correspondre avec une autre, ni s’affilier à elle, ni tenir des séances publiques, composées de sociétaires et d’assistants distingués les uns des autres, ni imposer des conditions d’admission et d’éligibilité, ni s’arroger des droits d’exclusion, ni faire porter à ses membres aucun signe extérieur de leur association.\par
\labelchar{Art. 363.} Les citoyens ne peuvent exercer leurs droits politiques que dans les Assemblées primaires ou communales\par
\labelchar{Art. 364.} Tous les citoyens sont libres d’adresser aux autorités publiques des pétitions, mais elles doivent être individuelles ; nulle association ne peut en présenter de collectives, si ce n’est les autorités constituées, et pour des objets propres à leur attribution.\par
Les pétitionnaires ne doivent jamais oublier le respect dû aux autorités constituées.\par
\labelchar{Art. 365.} Tout attroupement armé est un attentat à la Constitution ; il doit être dissipé sur-le-champ par la force.\par
\labelchar{Art. 366.} Tout attroupement non armé doit être également dissipé, d’abord par voie de commandement verbal, et, s’il est nécessaire, par le développement de la force armée.\par
\labelchar{Art. 367.} Plusieurs autorités constituées ne peuvent jamais se réunir pour délibérer ensemble ; aucun acte émané d’une telle réunion ne peut être exécuté.\par
\labelchar{Art. 368.} Nul ne peut porter des marques distinctives qui rappellent des fonctions antérieurement exercées, ni des services rendus.\par
\labelchar{Art. 369.} Les membres du Corps législatif, et tous les Fonctionnaires publics, portent, dans l’exercice de leurs l’onctions, le costume ou le signe de l’autorité dont ils sont revêtus : la loi en détermine la forme.\par
\labelchar{Art. 370.} Nul citoyen ne peut renoncer, ni en tout ni en partie, à l’indemnité ou au traitement qui lui est attribué par la loi, à raison de fonctions publiques.\par
\labelchar{Art. 371.} Il y a dans la République uniformité de poids et de mesures.\par
\labelchar{Art. 372.} L’ère française commence au 22 septembre 1792, jour de la fondation de la République.\par
\labelchar{Art. 373.} La Nation française déclare qu’en aucun cas elle ne souffrira le retour des Français qui, ayant abandonné leur patrie depuis le 15 juillet 1789, ne sont pas compris dans les exceptions portées aux lois rendues contre les émigrés ; et elle interdit au Corps législatif de créer de nouvelles exceptions sur ce point.\par
Les biens des émigrés sont irrévocablement acquis au profit de la République.\par
\labelchar{Art. 374.} La Nation française proclame pareillement, comme garantie de la foi publique, qu’après une adjudication légalement consommée de biens nationaux, quelle qu’en soit l’origine, l’acquéreur légitime ne peut en être dépossédé, sauf aux tiers réclamants à être, s’il y a lieu, indemnisés par le Trésor national.\par
\labelchar{Art. 375.} Aucun des pouvoirs institués par la Constitution, n’a le droit de la changer dans son ensemble ni dans aucune de ses parties, sauf les réformes qui pourront y être faites par la voie de la révision, conformément aux dispositions du titre XIII.\par
\labelchar{Art. 376.} Les citoyens se rappelleront sans cesse que c’est de la sagesse des choix dans les Assemblées primaires et électorales, que dépendent principalement la durée, la conservation et la prospérité de la République.\par
\labelchar{Art. 377.} Le peuple français remet le dépôt de la présente Constitution à la fidélité du Corps législatif, du Directoire exécutif, des administrateurs et des juges ; à la vigilance des pères de famille, aux épouses et aux mères, à l’affection des jeunes citoyens, au courage de tous les Français.
\chapterclose

 


% at least one empty page at end (for booklet couv)
\ifbooklet
  \pagestyle{empty}
  \clearpage
  % 2 empty pages maybe needed for 4e cover
  \ifnum\modulo{\value{page}}{4}=0 \hbox{}\newpage\hbox{}\newpage\fi
  \ifnum\modulo{\value{page}}{4}=1 \hbox{}\newpage\hbox{}\newpage\fi


  \hbox{}\newpage
  \ifodd\value{page}\hbox{}\newpage\fi
  {\centering\color{rubric}\bfseries\noindent\large
    Hurlus ? Qu’est-ce.\par
    \bigskip
  }
  \noindent Des bouquinistes électroniques, pour du texte libre à participations libres,
  téléchargeable gratuitement sur \href{https://hurlus.fr}{\dotuline{hurlus.fr}}.\par
  \bigskip
  \noindent Cette brochure a été produite par des éditeurs bénévoles.
  Elle n’est pas faite pour être possédée, mais pour être lue, et puis donnée.
  Que circule le texte !
  En page de garde, on peut ajouter une date, un lieu, un nom ;
  comme une fiche de bibliothèque en papier,
  pour suivre le voyage du texte. Qui sait, un jour, vous la retrouverez ?
  \par

  Ce texte a été choisi parce qu’une personne l’a aimé,
  ou haï, elle a pensé qu’il partipait à la formation de notre présent ;
  sans le souci de plaire, vendre, ou militer pour une cause.
  \par

  L’édition électronique est soigneuse, tant sur la technique
  que sur l’établissement du texte ; mais sans aucune prétention scolaire, au contraire.
  Le but est de s’adresser à tous, sans distinction de science ou de diplôme.
  Au plus direct ! (possible)
  \par

  Cet exemplaire en papier a été tiré sur une imprimante personnelle
   ou une photocopieuse. Tout le monde peut le faire.
  Il suffit de
  télécharger un fichier sur \href{https://hurlus.fr}{\dotuline{hurlus.fr}},
  d’imprimer, et agrafer ; puis de lire et donner.\par

  \bigskip

  \noindent PS : Les hurlus furent aussi des rebelles protestants qui cassaient les statues dans les églises catholiques. En 1566 démarra la révolte des gueux dans le pays de Lille. L’insurrection enflamma la région jusqu’à Anvers où les gueux de mer bloquèrent les bateaux espagnols.
  Ce fut une rare guerre de libération dont naquit un pays toujours libre : les Pays-Bas.
  En plat pays francophone, par contre, restèrent des bandes de huguenots, les hurlus, progressivement réprimés par la très catholique Espagne.
  Cette mémoire d’une défaite est éteinte, rallumons-la. Sortons les livres du culte universitaire, débusquons les idoles de l’époque, pour les démonter.
\fi

\end{document}
