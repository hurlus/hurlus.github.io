%%%%%%%%%%%%%%%%%%%%%%%%%%%%%%%%%
% LaTeX model https://hurlus.fr %
%%%%%%%%%%%%%%%%%%%%%%%%%%%%%%%%%

% Needed before document class
\RequirePackage{pdftexcmds} % needed for tests expressions
\RequirePackage{fix-cm} % correct units

% Define mode
\def\mode{a4}

\newif\ifaiv % a4
\newif\ifav % a5
\newif\ifbooklet % booklet
\newif\ifcover % cover for booklet

\ifnum \strcmp{\mode}{cover}=0
  \covertrue
\else\ifnum \strcmp{\mode}{booklet}=0
  \booklettrue
\else\ifnum \strcmp{\mode}{a5}=0
  \avtrue
\else
  \aivtrue
\fi\fi\fi

\ifbooklet % do not enclose with {}
  \documentclass[french,twoside]{book} % ,notitlepage
  \usepackage[%
    papersize={105mm, 297mm},
    inner=12mm,
    outer=12mm,
    top=20mm,
    bottom=15mm,
    marginparsep=0pt,
  ]{geometry}
  \usepackage[fontsize=9.5pt]{scrextend} % for Roboto
\else\ifav
  \documentclass[french,twoside]{book} % ,notitlepage
  \usepackage[%
    a5paper,
    inner=25mm,
    outer=15mm,
    top=15mm,
    bottom=15mm,
    marginparsep=0pt,
  ]{geometry}
  \usepackage[fontsize=12pt]{scrextend}
\else% A4 2 cols
  \documentclass[twocolumn]{report}
  \usepackage[%
    a4paper,
    inner=15mm,
    outer=10mm,
    top=25mm,
    bottom=18mm,
    marginparsep=0pt,
  ]{geometry}
  \setlength{\columnsep}{20mm}
  \usepackage[fontsize=9.5pt]{scrextend}
\fi\fi

%%%%%%%%%%%%%%
% Alignments %
%%%%%%%%%%%%%%
% before teinte macros

\setlength{\arrayrulewidth}{0.2pt}
\setlength{\columnseprule}{\arrayrulewidth} % twocol
\setlength{\parskip}{0pt} % classical para with no margin
\setlength{\parindent}{1.5em}

%%%%%%%%%%
% Colors %
%%%%%%%%%%
% before Teinte macros

\usepackage[dvipsnames]{xcolor}
\definecolor{rubric}{HTML}{800000} % the tonic 0c71c3
\def\columnseprulecolor{\color{rubric}}
\colorlet{borderline}{rubric!30!} % definecolor need exact code
\definecolor{shadecolor}{gray}{0.95}
\definecolor{bghi}{gray}{0.5}

%%%%%%%%%%%%%%%%%
% Teinte macros %
%%%%%%%%%%%%%%%%%
%%%%%%%%%%%%%%%%%%%%%%%%%%%%%%%%%%%%%%%%%%%%%%%%%%%
% <TEI> generic (LaTeX names generated by Teinte) %
%%%%%%%%%%%%%%%%%%%%%%%%%%%%%%%%%%%%%%%%%%%%%%%%%%%
% This template is inserted in a specific design
% It is XeLaTeX and otf fonts

\makeatletter % <@@@


\usepackage{blindtext} % generate text for testing
\usepackage[strict]{changepage} % for modulo 4
\usepackage{contour} % rounding words
\usepackage[nodayofweek]{datetime}
% \usepackage{DejaVuSans} % seems buggy for sffont font for symbols
\usepackage{enumitem} % <list>
\usepackage{etoolbox} % patch commands
\usepackage{fancyvrb}
\usepackage{fancyhdr}
\usepackage{float}
\usepackage{fontspec} % XeLaTeX mandatory for fonts
\usepackage{footnote} % used to capture notes in minipage (ex: quote)
\usepackage{framed} % bordering correct with footnote hack
\usepackage{graphicx}
\usepackage{lettrine} % drop caps
\usepackage{lipsum} % generate text for testing
\usepackage[framemethod=tikz,]{mdframed} % maybe used for frame with footnotes inside
\usepackage{pdftexcmds} % needed for tests expressions
\usepackage{polyglossia} % non-break space french punct, bug Warning: "Failed to patch part"
\usepackage[%
  indentfirst=false,
  vskip=1em,
  noorphanfirst=true,
  noorphanafter=true,
  leftmargin=\parindent,
  rightmargin=0pt,
]{quoting}
\usepackage{ragged2e}
\usepackage{setspace} % \setstretch for <quote>
\usepackage{tabularx} % <table>
\usepackage[explicit]{titlesec} % wear titles, !NO implicit
\usepackage{tikz} % ornaments
\usepackage{tocloft} % styling tocs
\usepackage[fit]{truncate} % used im runing titles
\usepackage{unicode-math}
\usepackage[normalem]{ulem} % breakable \uline, normalem is absolutely necessary to keep \emph
\usepackage{verse} % <l>
\usepackage{xcolor} % named colors
\usepackage{xparse} % @ifundefined
\XeTeXdefaultencoding "iso-8859-1" % bad encoding of xstring
\usepackage{xstring} % string tests
\XeTeXdefaultencoding "utf-8"
\PassOptionsToPackage{hyphens}{url} % before hyperref, which load url package

% TOTEST
% \usepackage{hypcap} % links in caption ?
% \usepackage{marginnote}
% TESTED
% \usepackage{background} % doesn’t work with xetek
% \usepackage{bookmark} % prefers the hyperref hack \phantomsection
% \usepackage[color, leftbars]{changebar} % 2 cols doc, impossible to keep bar left
% \usepackage[utf8x]{inputenc} % inputenc package ignored with utf8 based engines
% \usepackage[sfdefault,medium]{inter} % no small caps
% \usepackage{firamath} % choose firasans instead, firamath unavailable in Ubuntu 21-04
% \usepackage{flushend} % bad for last notes, supposed flush end of columns
% \usepackage[stable]{footmisc} % BAD for complex notes https://texfaq.org/FAQ-ftnsect
% \usepackage{helvet} % not for XeLaTeX
% \usepackage{multicol} % not compatible with too much packages (longtable, framed, memoir…)
% \usepackage[default,oldstyle,scale=0.95]{opensans} % no small caps
% \usepackage{sectsty} % \chapterfont OBSOLETE
% \usepackage{soul} % \ul for underline, OBSOLETE with XeTeX
% \usepackage[breakable]{tcolorbox} % text styling gone, footnote hack not kept with breakable


% Metadata inserted by a program, from the TEI source, for title page and runing heads
\title{\textbf{ Physiologie du correcteur d’imprimerie }}
\date{1843}
\author{A.-T. Breton}
\def\elbibl{A.-T. Breton. 1843. \emph{Physiologie du correcteur d’imprimerie}}
\def\elsource{ \href{http://gallica.bnf.fr/ark:/12148/btv1b85303376}{\dotuline{http://gallica.bnf.fr/ark:/12148/btv1b85303376}}\footnote{\href{http://gallica.bnf.fr/ark:/12148/btv1b85303376}{\url{http://gallica.bnf.fr/ark:/12148/btv1b85303376}}} \href{http://efele.net/ebooks/livres/000331}{\dotuline{http://efele.net/ebooks/livres/000331}}\footnote{\href{http://efele.net/ebooks/livres/000331}{\url{http://efele.net/ebooks/livres/000331}}} }

% Default metas
\newcommand{\colorprovide}[2]{\@ifundefinedcolor{#1}{\colorlet{#1}{#2}}{}}
\colorprovide{rubric}{red}
\colorprovide{silver}{lightgray}
\@ifundefined{syms}{\newfontfamily\syms{DejaVu Sans}}{}
\newif\ifdev
\@ifundefined{elbibl}{% No meta defined, maybe dev mode
  \newcommand{\elbibl}{Titre court ?}
  \newcommand{\elbook}{Titre du livre source ?}
  \newcommand{\elabstract}{Résumé\par}
  \newcommand{\elurl}{http://oeuvres.github.io/elbook/2}
  \author{Éric Lœchien}
  \title{Un titre de test assez long pour vérifier le comportement d’une maquette}
  \date{1566}
  \devtrue
}{}
\let\eltitle\@title
\let\elauthor\@author
\let\eldate\@date


\defaultfontfeatures{
  % Mapping=tex-text, % no effect seen
  Scale=MatchLowercase,
  Ligatures={TeX,Common},
}


% generic typo commands
\newcommand{\astermono}{\medskip\centerline{\color{rubric}\large\selectfont{\syms ✻}}\medskip\par}%
\newcommand{\astertri}{\medskip\par\centerline{\color{rubric}\large\selectfont{\syms ✻\,✻\,✻}}\medskip\par}%
\newcommand{\asterism}{\bigskip\par\noindent\parbox{\linewidth}{\centering\color{rubric}\large{\syms ✻}\\{\syms ✻}\hskip 0.75em{\syms ✻}}\bigskip\par}%

% lists
\newlength{\listmod}
\setlength{\listmod}{\parindent}
\setlist{
  itemindent=!,
  listparindent=\listmod,
  labelsep=0.2\listmod,
  parsep=0pt,
  % topsep=0.2em, % default topsep is best
}
\setlist[itemize]{
  label=—,
  leftmargin=0pt,
  labelindent=1.2em,
  labelwidth=0pt,
}
\setlist[enumerate]{
  label={\bf\color{rubric}\arabic*.},
  labelindent=0.8\listmod,
  leftmargin=\listmod,
  labelwidth=0pt,
}
\newlist{listalpha}{enumerate}{1}
\setlist[listalpha]{
  label={\bf\color{rubric}\alph*.},
  leftmargin=0pt,
  labelindent=0.8\listmod,
  labelwidth=0pt,
}
\newcommand{\listhead}[1]{\hspace{-1\listmod}\emph{#1}}

\renewcommand{\hrulefill}{%
  \leavevmode\leaders\hrule height 0.2pt\hfill\kern\z@}

% General typo
\DeclareTextFontCommand{\textlarge}{\large}
\DeclareTextFontCommand{\textsmall}{\small}

% commands, inlines
\newcommand{\anchor}[1]{\Hy@raisedlink{\hypertarget{#1}{}}} % link to top of an anchor (not baseline)
\newcommand\abbr[1]{#1}
\newcommand{\autour}[1]{\tikz[baseline=(X.base)]\node [draw=rubric,thin,rectangle,inner sep=1.5pt, rounded corners=3pt] (X) {\color{rubric}#1};}
\newcommand\corr[1]{#1}
\newcommand{\ed}[1]{ {\color{silver}\sffamily\footnotesize (#1)} } % <milestone ed="1688"/>
\newcommand\expan[1]{#1}
\newcommand\foreign[1]{\emph{#1}}
\newcommand\gap[1]{#1}
\renewcommand{\LettrineFontHook}{\color{rubric}}
\newcommand{\initial}[2]{\lettrine[lines=2, loversize=0.3, lhang=0.3]{#1}{#2}}
\newcommand{\initialiv}[2]{%
  \let\oldLFH\LettrineFontHook
  % \renewcommand{\LettrineFontHook}{\color{rubric}\ttfamily}
  \IfSubStr{QJ’}{#1}{
    \lettrine[lines=4, lhang=0.2, loversize=-0.1, lraise=0.2]{\smash{#1}}{#2}
  }{\IfSubStr{É}{#1}{
    \lettrine[lines=4, lhang=0.2, loversize=-0, lraise=0]{\smash{#1}}{#2}
  }{\IfSubStr{ÀÂ}{#1}{
    \lettrine[lines=4, lhang=0.2, loversize=-0, lraise=0, slope=0.6em]{\smash{#1}}{#2}
  }{\IfSubStr{A}{#1}{
    \lettrine[lines=4, lhang=0.2, loversize=0.2, slope=0.6em]{\smash{#1}}{#2}
  }{\IfSubStr{V}{#1}{
    \lettrine[lines=4, lhang=0.2, loversize=0.2, slope=-0.5em]{\smash{#1}}{#2}
  }{
    \lettrine[lines=4, lhang=0.2, loversize=0.2]{\smash{#1}}{#2}
  }}}}}
  \let\LettrineFontHook\oldLFH
}
\newcommand{\labelchar}[1]{\textbf{\color{rubric} #1}}
\newcommand{\milestone}[1]{\autour{\footnotesize\color{rubric} #1}} % <milestone n="4"/>
\newcommand\name[1]{#1}
\newcommand\orig[1]{#1}
\newcommand\orgName[1]{#1}
\newcommand\persName[1]{#1}
\newcommand\placeName[1]{#1}
\newcommand{\pn}[1]{\IfSubStr{-—–¶}{#1}% <p n="3"/>
  {\noindent{\bfseries\color{rubric}   ¶  }}
  {{\footnotesize\autour{ #1}  }}}
\newcommand\reg{}
% \newcommand\ref{} % already defined
\newcommand\sic[1]{#1}
\newcommand\surname[1]{\textsc{#1}}
\newcommand\term[1]{\textbf{#1}}

\def\mednobreak{\ifdim\lastskip<\medskipamount
  \removelastskip\nopagebreak\medskip\fi}
\def\bignobreak{\ifdim\lastskip<\bigskipamount
  \removelastskip\nopagebreak\bigskip\fi}

% commands, blocks
\newcommand{\byline}[1]{\bigskip{\RaggedLeft{#1}\par}\bigskip}
\newcommand{\bibl}[1]{{\RaggedLeft{#1}\par\bigskip}}
\newcommand{\biblitem}[1]{{\noindent\hangindent=\parindent   #1\par}}
\newcommand{\dateline}[1]{\medskip{\RaggedLeft{#1}\par}\bigskip}
\newcommand{\labelblock}[1]{\medbreak{\noindent\color{rubric}\bfseries #1}\par\mednobreak}
\newcommand{\salute}[1]{\bigbreak{#1}\par\medbreak}
\newcommand{\signed}[1]{\bigbreak\filbreak{\raggedleft #1\par}\medskip}

% environments for blocks (some may become commands)
\newenvironment{borderbox}{}{} % framing content
\newenvironment{citbibl}{\ifvmode\hfill\fi}{\ifvmode\par\fi }
\newenvironment{docAuthor}{\ifvmode\vskip4pt\fontsize{16pt}{18pt}\selectfont\fi\itshape}{\ifvmode\par\fi }
\newenvironment{docDate}{}{\ifvmode\par\fi }
\newenvironment{docImprint}{\vskip6pt}{\ifvmode\par\fi }
\newenvironment{docTitle}{\vskip6pt\bfseries\fontsize{18pt}{22pt}\selectfont}{\par }
\newenvironment{msHead}{\vskip6pt}{\par}
\newenvironment{msItem}{\vskip6pt}{\par}
\newenvironment{titlePart}{}{\par }


% environments for block containers
\newenvironment{argument}{\itshape\parindent0pt}{\vskip1.5em}
\newenvironment{biblfree}{}{\ifvmode\par\fi }
\newenvironment{bibitemlist}[1]{%
  \list{\@biblabel{\@arabic\c@enumiv}}%
  {%
    \settowidth\labelwidth{\@biblabel{#1}}%
    \leftmargin\labelwidth
    \advance\leftmargin\labelsep
    \@openbib@code
    \usecounter{enumiv}%
    \let\p@enumiv\@empty
    \renewcommand\theenumiv{\@arabic\c@enumiv}%
  }
  \sloppy
  \clubpenalty4000
  \@clubpenalty \clubpenalty
  \widowpenalty4000%
  \sfcode`\.\@m
}%
{\def\@noitemerr
  {\@latex@warning{Empty `bibitemlist' environment}}%
\endlist}
\newenvironment{quoteblock}% may be used for ornaments
  {\begin{quoting}}
  {\end{quoting}}

% table () is preceded and finished by custom command
\newcommand{\tableopen}[1]{%
  \ifnum\strcmp{#1}{wide}=0{%
    \begin{center}
  }
  \else\ifnum\strcmp{#1}{long}=0{%
    \begin{center}
  }
  \else{%
    \begin{center}
  }
  \fi\fi
}
\newcommand{\tableclose}[1]{%
  \ifnum\strcmp{#1}{wide}=0{%
    \end{center}
  }
  \else\ifnum\strcmp{#1}{long}=0{%
    \end{center}
  }
  \else{%
    \end{center}
  }
  \fi\fi
}


% text structure
\newcommand\chapteropen{} % before chapter title
\newcommand\chaptercont{} % after title, argument, epigraph…
\newcommand\chapterclose{} % maybe useful for multicol settings
\setcounter{secnumdepth}{-2} % no counters for hierarchy titles
\setcounter{tocdepth}{5} % deep toc
\markright{\@title} % ???
\markboth{\@title}{\@author} % ???
\renewcommand\tableofcontents{\@starttoc{toc}}
% toclof format
% \renewcommand{\@tocrmarg}{0.1em} % Useless command?
% \renewcommand{\@pnumwidth}{0.5em} % {1.75em}
\renewcommand{\@cftmaketoctitle}{}
\setlength{\cftbeforesecskip}{\z@ \@plus.2\p@}
\renewcommand{\cftchapfont}{}
\renewcommand{\cftchapdotsep}{\cftdotsep}
\renewcommand{\cftchapleader}{\normalfont\cftdotfill{\cftchapdotsep}}
\renewcommand{\cftchappagefont}{\bfseries}
\setlength{\cftbeforechapskip}{0em \@plus\p@}
% \renewcommand{\cftsecfont}{\small\relax}
\renewcommand{\cftsecpagefont}{\normalfont}
% \renewcommand{\cftsubsecfont}{\small\relax}
\renewcommand{\cftsecdotsep}{\cftdotsep}
\renewcommand{\cftsecpagefont}{\normalfont}
\renewcommand{\cftsecleader}{\normalfont\cftdotfill{\cftsecdotsep}}
\setlength{\cftsecindent}{1em}
\setlength{\cftsubsecindent}{2em}
\setlength{\cftsubsubsecindent}{3em}
\setlength{\cftchapnumwidth}{1em}
\setlength{\cftsecnumwidth}{1em}
\setlength{\cftsubsecnumwidth}{1em}
\setlength{\cftsubsubsecnumwidth}{1em}

% footnotes
\newif\ifheading
\newcommand*{\fnmarkscale}{\ifheading 0.70 \else 1 \fi}
\renewcommand\footnoterule{\vspace*{0.3cm}\hrule height \arrayrulewidth width 3cm \vspace*{0.3cm}}
\setlength\footnotesep{1.5\footnotesep} % footnote separator
\renewcommand\@makefntext[1]{\parindent 1.5em \noindent \hb@xt@1.8em{\hss{\normalfont\@thefnmark . }}#1} % no superscipt in foot
\patchcmd{\@footnotetext}{\footnotesize}{\footnotesize\sffamily}{}{} % before scrextend, hyperref


%   see https://tex.stackexchange.com/a/34449/5049
\def\truncdiv#1#2{((#1-(#2-1)/2)/#2)}
\def\moduloop#1#2{(#1-\truncdiv{#1}{#2}*#2)}
\def\modulo#1#2{\number\numexpr\moduloop{#1}{#2}\relax}

% orphans and widows
\clubpenalty=9996
\widowpenalty=9999
\brokenpenalty=4991
\predisplaypenalty=10000
\postdisplaypenalty=1549
\displaywidowpenalty=1602
\hyphenpenalty=400
% Copied from Rahtz but not understood
\def\@pnumwidth{1.55em}
\def\@tocrmarg {2.55em}
\def\@dotsep{4.5}
\emergencystretch 3em
\hbadness=4000
\pretolerance=750
\tolerance=2000
\vbadness=4000
\def\Gin@extensions{.pdf,.png,.jpg,.mps,.tif}
% \renewcommand{\@cite}[1]{#1} % biblio

\usepackage{hyperref} % supposed to be the last one, :o) except for the ones to follow
\urlstyle{same} % after hyperref
\hypersetup{
  % pdftex, % no effect
  pdftitle={\elbibl},
  % pdfauthor={Your name here},
  % pdfsubject={Your subject here},
  % pdfkeywords={keyword1, keyword2},
  bookmarksnumbered=true,
  bookmarksopen=true,
  bookmarksopenlevel=1,
  pdfstartview=Fit,
  breaklinks=true, % avoid long links
  pdfpagemode=UseOutlines,    % pdf toc
  hyperfootnotes=true,
  colorlinks=false,
  pdfborder=0 0 0,
  % pdfpagelayout=TwoPageRight,
  % linktocpage=true, % NO, toc, link only on page no
}

\makeatother % /@@@>
%%%%%%%%%%%%%%
% </TEI> end %
%%%%%%%%%%%%%%


%%%%%%%%%%%%%
% footnotes %
%%%%%%%%%%%%%
\renewcommand{\thefootnote}{\bfseries\textcolor{rubric}{\arabic{footnote}}} % color for footnote marks

%%%%%%%%%
% Fonts %
%%%%%%%%%
\usepackage[]{roboto} % SmallCaps, Regular is a bit bold
% \linespread{0.90} % too compact, keep font natural
\newfontfamily\fontrun[]{Roboto Condensed Light} % condensed runing heads
\ifav
  \setmainfont[
    ItalicFont={Roboto Light Italic},
  ]{Roboto}
\else\ifbooklet
  \setmainfont[
    ItalicFont={Roboto Light Italic},
  ]{Roboto}
\else
\setmainfont[
  ItalicFont={Roboto Italic},
]{Roboto Light}
\fi\fi
\renewcommand{\LettrineFontHook}{\bfseries\color{rubric}}
% \renewenvironment{labelblock}{\begin{center}\bfseries\color{rubric}}{\end{center}}

%%%%%%%%
% MISC %
%%%%%%%%

\setdefaultlanguage[frenchpart=false]{french} % bug on part


\newenvironment{quotebar}{%
    \def\FrameCommand{{\color{rubric!10!}\vrule width 0.5em} \hspace{0.9em}}%
    \def\OuterFrameSep{\itemsep} % séparateur vertical
    \MakeFramed {\advance\hsize-\width \FrameRestore}
  }%
  {%
    \endMakeFramed
  }
\renewenvironment{quoteblock}% may be used for ornaments
  {%
    \savenotes
    \setstretch{0.9}
    \normalfont
    \begin{quotebar}
  }
  {%
    \end{quotebar}
    \spewnotes
  }


\renewcommand{\headrulewidth}{\arrayrulewidth}
\renewcommand{\headrule}{{\color{rubric}\hrule}}

% delicate tuning, image has produce line-height problems in title on 2 lines
\titleformat{name=\chapter} % command
  [display] % shape
  {\vspace{1.5em}\centering} % format
  {} % label
  {0pt} % separator between n
  {}
[{\color{rubric}\huge\textbf{#1}}\bigskip] % after code
% \titlespacing{command}{left spacing}{before spacing}{after spacing}[right]
\titlespacing*{\chapter}{0pt}{-2em}{0pt}[0pt]

\titleformat{name=\section}
  [block]{}{}{}{}
  [\vbox{\color{rubric}\large\raggedleft\textbf{#1}}]
\titlespacing{\section}{0pt}{0pt plus 4pt minus 2pt}{\baselineskip}

\titleformat{name=\subsection}
  [block]
  {}
  {} % \thesection
  {} % separator \arrayrulewidth
  {}
[\vbox{\large\textbf{#1}}]
% \titlespacing{\subsection}{0pt}{0pt plus 4pt minus 2pt}{\baselineskip}

\ifaiv
  \fancypagestyle{main}{%
    \fancyhf{}
    \setlength{\headheight}{1.5em}
    \fancyhead{} % reset head
    \fancyfoot{} % reset foot
    \fancyhead[L]{\truncate{0.45\headwidth}{\fontrun\elbibl}} % book ref
    \fancyhead[R]{\truncate{0.45\headwidth}{ \fontrun\nouppercase\leftmark}} % Chapter title
    \fancyhead[C]{\thepage}
  }
  \fancypagestyle{plain}{% apply to chapter
    \fancyhf{}% clear all header and footer fields
    \setlength{\headheight}{1.5em}
    \fancyhead[L]{\truncate{0.9\headwidth}{\fontrun\elbibl}}
    \fancyhead[R]{\thepage}
  }
\else
  \fancypagestyle{main}{%
    \fancyhf{}
    \setlength{\headheight}{1.5em}
    \fancyhead{} % reset head
    \fancyfoot{} % reset foot
    \fancyhead[RE]{\truncate{0.9\headwidth}{\fontrun\elbibl}} % book ref
    \fancyhead[LO]{\truncate{0.9\headwidth}{\fontrun\nouppercase\leftmark}} % Chapter title, \nouppercase needed
    \fancyhead[RO,LE]{\thepage}
  }
  \fancypagestyle{plain}{% apply to chapter
    \fancyhf{}% clear all header and footer fields
    \setlength{\headheight}{1.5em}
    \fancyhead[L]{\truncate{0.9\headwidth}{\fontrun\elbibl}}
    \fancyhead[R]{\thepage}
  }
\fi

\ifav % a5 only
  \titleclass{\section}{top}
\fi

\newcommand\chapo{{%
  \vspace*{-3em}
  \centering % no vskip ()
  {\Large\addfontfeature{LetterSpace=25}\bfseries{\elauthor}}\par
  \smallskip
  {\large\eldate}\par
  \bigskip
  {\Large\selectfont{\eltitle}}\par
  \bigskip
  {\color{rubric}\hline\par}
  \bigskip
  {\Large TEXTE LIBRE À PARTICPATION LIBRE\par}
  \centerline{\small\color{rubric} {hurlus.fr, tiré le \today}}\par
  \bigskip
}}

\newcommand\cover{{%
  \thispagestyle{empty}
  \centering
  {\LARGE\bfseries{\elauthor}}\par
  \bigskip
  {\Large\eldate}\par
  \bigskip
  \bigskip
  {\LARGE\selectfont{\eltitle}}\par
  \vfill\null
  {\color{rubric}\setlength{\arrayrulewidth}{2pt}\hline\par}
  \vfill\null
  {\Large TEXTE LIBRE À PARTICPATION LIBRE\par}
  \centerline{{\href{https://hurlus.fr}{\dotuline{hurlus.fr}}, tiré le \today}}\par
}}

\begin{document}
\pagestyle{empty}
\ifbooklet{
  \cover\newpage
  \thispagestyle{empty}\hbox{}\newpage
  \cover\newpage\noindent Les voyages de la brochure\par
  \bigskip
  \begin{tabularx}{\textwidth}{l|X|X}
    \textbf{Date} & \textbf{Lieu}& \textbf{Nom/pseudo} \\ \hline
    \rule{0pt}{25cm} &  &   \\
  \end{tabularx}
  \newpage
  \addtocounter{page}{-4}
}\fi

\thispagestyle{empty}
\ifaiv
  \twocolumn[\chapo]
\else
  \chapo
\fi
{\it\elabstract}
\bigskip
\makeatletter\@starttoc{toc}\makeatother % toc without new page
\bigskip

\pagestyle{main} % after style

   \chapter[{Physiologie du correcteur d’imprimerie}]{Physiologie du correcteur d’imprimerie}
\noindent Dictu quam re facilius est.\par

\bibl{Livius.}
\noindent Parmi la foule des illustrations que le monde scientifique, artistique, littéraire et industriel étale pompeusement sous nos yeux depuis quelques années, au milieu de tous ces types \emph{peints par eux-mêmes} ou par une main tantôt hostile, tantôt amie, il est encore, à Paris même, ce centre incontesté des  sciences, des arts, de la littérature et de l’industrie, un type complètement inconnu, et cependant bien digne des regards de l’observateur : ce type oublie ou négligé, c’est le Correcteur d’imprimerie.\par
Depuis Alexandre de Russie, qui, en l’an de grâce 1814, visita la typographie de M. Firmin Didot, jusqu’à notre époque, où la lumière a, dit-on, pénétré partout, il n’est pas, que je sache, un seul homme de quelque mérite en Europe qui ait daigné jeter un coup d’œil attentif sur ce digne sujet d’une monographie que nous avons pris à cœur et que nous tenons à honneur d’esquisser aujourd’hui.\par
Une expérience de vingt ans, une longue suite d’observations attentives, le vide que nous avons remarqué au milieu des mille portraits de fantaisie ou d’après nature dont la librairie meuble actuellement ses rayons, tout cela nous a fourni les motifs de l’entreprise :  le lecteur dira s’il nous était permis de la tenter.\par
Et d’abord, disons un mot de la position du Correcteur dans une imprimerie, de son importance et de ses fonctions, que beaucoup de personnes, beaucoup d’écrivains mêmes, confondent avec celles du prote, erreur propagée sans doute par le \emph{Dictionnaire de l’Académie}, qui, parmi tant d’autres erreurs qu’il renferme, prétend que ce nom s’applique indifféremment aux correcteurs : si messieurs de l’Institut sont réellement les auteurs de ce lexique incomplet, il faut supposer qu’ils avaient, en écrivant l’article \emph{prote}, oublié son étymologie ; prote vient de \emph{πρωτος, premier ;} or, dans un État, dans une réunion d’hommes quelconque, dans un atelier, il ne peut y avoir qu’un \emph{premier}, tandis que le nombre des correcteurs dans une imprimerie peut être de deux, de trois, de tel nombre enfin qu’il convient relativement à  l’importance du personnel des compositeurs et au soin que l’on veut apporter aux ouvrages imprimés dans la maison qui les occupe.\par
Mais ce n’est pas ici le lieu de disserter sur la valeur de ce mot, et nous ne nous y sommes arrêté que pour éviter le ridicule d’un contre-sens à ceux qui, en entrant dans une imprimerie, demandent où est le bureau \emph{des protes}, et encore aux hommes de lettres, qui, affectant une modestie assez rare, il faut le dire, chez la plupart de ces messieurs, veulent bien quelquefois déclarer qu’ils se reposent sur \emph{les protes} du soin de relever les fautes qui ont pu leur échapper : ceux-là évidemment ont voulu désigner le Correcteur.\par
Le Correcteur est donc placé sous les ordres immédiats du prote ; et, sauf quelques exceptions que nous signalerons en leur lieu, leurs fonctions sont tellement distinctes, que nous ne comprenons pas qu’on ait pu leur attribuer une dénomination commune.\par
 Mais la priorité de l’un n’entraîne pas nécessairement l’infériorité de l’autre, et lorsque le Correcteur, que son érudition, d’ailleurs, place généralement au premier rang, s’acquitte avec zèle et discernement de la partie si importante d’une bonne impression, celle de la lecture des épreuves, on se repose entièrement sur lui de la pureté des textes et de la précision grammaticale. Il est alors l’ami et le conseiller du prote plutôt que son subordonné ; un prote que l’importance de la maison qu’il dirige empêche de se livrer à la correction est un corps sans âme s’il n’a pas au moins un bon correcteur pour le seconder : ce qui fait que dans chaque imprimerie, du moins dans plusieurs, on voit souvent un correcteur qui, à l’exclusion des autres, jouit de privilèges dont il use quelquefois assez largement pour se placer sur la ligne de celui qui les lui accorde. Si cette faveur était toujours la récompense du vrai  mérite, de l’homme du métier que son instruction et son érudition appellent de droit à l’exercice d’une influence morale, d’une autorité tacite, dans un établissement, nous ne verrions en elle que l’effet d’une considération justement acquise, et il est un point de vue sous lequel nous pourrions l’envisager favorablement ; mais donnée souvent à la sottise et à l’ignorance, une telle prédominance est préjudiciable dans ses conséquences aux correcteurs en général, elle est funeste aux compositeurs et aux imprimeurs en particulier, par les lenteurs qu’elle apporte dans le travail, par l’énorme impôt de temps qu’elle lève sur ceux-ci et par l’impôt dix fois plus considérable dont elle grève ceux-là. L’ignorance des matières, la présomption, le caprice, amènent trop souvent des bévues, et par-conséquent des corrections qu’il faut faire, refaire et supporter, soit dans une première typographique, soit dans une tierce ; ce point  de départ vicieux cause d’ailleurs un si grand déficit au bout de l’année dans la caisse du maître imprimeur qui est affligé de cette calamité, que, dût cette esquisse souffrir de la prolixité de mes détails aux yeux de quelques personnes étrangères à la typographie, j’insisterai sur ce point, dont la gravité fait à l’homme de l’art un devoir de signaler tous les abus. Je blesserai bien des susceptibilités, sans doute, mais si les vérités sont dures aux oreilles de quelques hommes, elles profitent au plus grand nombre, et cette considération me suffit. C’est un tableau que je fais ici, et, en peinture, rien n’est plus beau que l’histoire.\par
On appelle \emph{première typographique}\footnote{ \noindent En général, la lecture en première est confiée à des gens trop inhabiles. On rétribue moins un correcteur \emph{en première} qu’un correcteur \emph{en seconde}, et pourtant il est bien démontré que la seconde ne saurait être parfaite si la première a été négligée.
 }  l’épreuve lue pour la première fois par le Correcteur aussitôt que la composition est terminée. Les fautes relevées dans cette lecture sont à la charge du compositeur. Qu’arrive-t-il si le correcteur qui a lu cette épreuve n’est pas conséquent avec lui-même dans le système de correction ou de ponctuation qu’il a suivi la veille ? Il en résulte que le compositeur qui s’est attaché à suivre ce système se trouve alors en opposition directe avec lui : de là un nombre infini de corrections dont la plupart sont vides et onéreuses, et que l’ouvrier, par un usage bizarre dont la réforme serait assez difficile, est obligé d’exécuter à ses frais, ce qui ouvre à son budget quotidien une brèche d’un franc au moins, c’est-à-dire du cinquième, maximum, et du quart, terme moyen ; je ne parle pas des jours où il ne gagne que 2 fr. ou 2 fr. 50 c., car je ne veux pas en aller chercher la raison où elle est : je laisse à la philanthropie de  M. Charles Dupin le soin de fermer toutes les plaies ouvertes de ce côté....\par
On nomme \emph{bon à tirer} la dernière épreuve vue par l’auteur, et sur laquelle il écrit ces mots\footnote{ \noindent Entre la première typographique et le bon à tirer, il arrive souvent que l’auteur revoie une ou plusieurs épreuves : elles prennent alors, selon leur ordre, le nom de \emph{première}, de \emph{deuxième} d’auteur, etc.
 } pour autoriser le tirage. Malheur à l’ouvrier imprimeur si elle tombe entre les mains du malencontreux favori ! il y va de sa journée. Et qu’on ne croie pas que j’exagère. J’ai vu maintes fois une confiance aveugle ou une tolérance coupable livrer ainsi à l’arbitraire d’un correcteur, dont les capacités étaient loin d’égaler les prétentions, la journée de deux ouvriers, qui s’écoule alors dans une attente ennuyeuse et stérile. En dehors de cette considération importante, ne pourrions-nous pas mettre en ligne de compte l’amour-propre des autres correcteurs,  qui se trouve froissé à plaisir, et surtout leur emploi, qui peut être compromis sans profit pour la maison qui les occupe ?\par
On donne le nom de \emph{tierce} à la dernière épreuve, à celle que l’imprimeur dépose au bureau des correcteurs après avoir mis sous presse, afin qu’on y vérifie les dernières corrections de l’auteur et celles du correcteur qui a lu le \emph{bon à tirer.}\par
L’audace, dit-on, tient souvent lieu de mérite, et, en chargeant les marges de cette épreuve de corrections qu’il serait bien en peine de justifier, le favori fascine l’œil du maître, entre les mains duquel elle peut passer, en même temps qu’il écrase ses collègues d’une supériorité dont on pourrait trouver le secret dans l’élasticité de la ponctuation, qui offre toujours une ressource à celui qui lit le dernier. Mais, à cet égard, je puis me montrer moins exclusif sans m’écarter de la vérité. Ce défaut, qui ne tend  à rien moins qu’a altérer la pureté des textes en les couvrant d’irrégularités, à dénaturer même la pensée de l’auteur, ce défaut, dis-je, est malheureusement commun à trop de correcteurs (soit dit en passant et pour faire de suite la part de la critique que l’on peut tirer de ce sujet). En effet, ce qui a eu lieu à la tierce s’est souvent présenté à la seconde, et s’il fallait récapituler au bout d’une année tous les frais occasionnés par le seul fait de ces incertitudes, de ces petites rivalités d’amour-propre entre les correcteurs de certaines imprimeries, il en résulterait certainement un total dont la vue ne manquerait pas d’attirer l’attention du maître ou de celui qui le représente, et qui l’engagerait à intervenir d’une manière efficace dans les débats qui s’élèvent journellement au sein de cette petite république d’hommes plus ou moins lettrés.\par
Comme nous venons de le voir, les fonctions du Correcteur se bornent à la lecture  des épreuves. Les personnes qui n’ont aucune notion de l’imprimerie s’étonneront à bon droit que les auteurs, qui sont les correcteurs naturels de leurs ouvrages, aient réellement besoin de cet auxiliaire ; elles comprendront difficilement que la science d’un homme de lettres soit insuffisante pour obtenir, sinon la perfection, du moins une exécution satisfaisante. Un homme du métier s’offenserait d’abord d’une semblable erreur, mais en réfléchissant, il la pardonnera sans peine, s’il n’a pas oublié l’exemple de notre célèbre Labruyère : ce profond observateur, ce moraliste éclairé, que son esprit, ses études de mœurs, et surtout ses relations fréquentes avec les imprimeurs, auraient dû mettre en garde contre un tel démenti à la vérité, Labruyère n’a pas craint de présenter l’état de correcteur comme le pis-aller de toutes les capacités avortées, [{\corr de}] toutes les espérances déçues. Si Labruyère  a pu s’égarer sous ce point de vue, il faut bien que nous donnions l’absolution aux autres ; mais pardonnerons-nous aussi bénignement aux Zoïles contemporains qui ravalent à plaisir une profession illustrée par les noms des cardinaux Morone, Mula et Trani, de l’évêque Antonius Campanus, d’Alde l’ancien, d’Alde Manuce, du docteur de Sorbonne Flavigny, de Robert Etienne, et de tant d’autres hommes de génie et de science qui ont salué l’aurore de la typographie ? A l’autorité irréfléchie de Labruyère, que d’ailleurs la simple appréciation des faits peut détruire, j’opposerai celle du fameux correcteur Corneille Kilian, que cinquante ans d’exercice en typographie, et ses connaissances profondes dans les lettres, ont rendu justement célèbre, à une époque pourtant où un savant, dont le nom m’échappe à regret, disait que l’on pouvait admettre l’imprimerie pour \emph{huitième dans les sciences}, et où le  roi de France lui-même la proclamait comme une œuvre plutôt divine qu’humaine. L’autorité de Kilian sera d’un grand poids sans doute pour asseoir notre opinion sur la valeur réelle du Correcteur d’imprimerie. Écoutons ce qu’il disait des \emph{écrivains} du {\scshape xvi}\textsuperscript{e} siècle :\par
« Notre fonction est de corriger les fautes des livres, et de relever les passages défectueux. Mais un méchant brouillon qui accumule les fautes et les tournures barbares, emporté par la rage d’écrire, fait des compilations sans discernement, couvre les feuillets de ratures et souille le papier. Il ne passe pas neuf années à ce travail, il ne prend pas la peine de le polir ; mais il se hâte de faire imprimer ses rêveries par des presses diligentes ; et lorsque des savants proclament qu’il écrit en dépit des Muses et d’Apollon, notre brouillon enrage ; il se défend de toutes  ses forces, et s’en prend au correcteur. Eh ! cesse donc, lourdaud, d’attribuer au correcteur un tort qu’il n’a pas. Ce qu’il y a de bien dans ton livre, l’a-t-il gâté ? N’entends-tu pas ?.. Désormais, brouillon, débarbouille toi-même tes petits. Corriger les fautes d’autrui, c’est s’exposer à le mécontenter, sans en retirer aucune gloire. »\par
Le pape Léon X lui-même accorda la plus haute considération au Correcteur, et dès 1515, lorsqu’il donna à l’imprimerie les honneurs du Capitule, il voulut que les plus savants de ses États s’adjoignissent à Alde l’ancien, son imprimeur, pour la lecture des épreuves.\par
En dépit de cette sollicitude des grands et des savants pour l’art de la correction, le Correcteur ne fut pas toujours à l’abri des inductions malveillantes, et l’histoire nous fournit plusieurs exemples d’un aveuglement dont heureusement les correcteurs de nos  jours ne peuvent plus être victimes. D’ailleurs, quels que soient le nombre et l’ignorance de ceux qui se croient en droit de les juger, l’exemple du docteur de Sorbonne Flavigny, professeur du collège de France, que nous avons déjà cité, proteste hautement contre les jugements hasardés des censeurs ignorants, car il fut lui-même victime d’une de ces fautes si communes et si souvent inexplicables dont on ne manque jamais d’attribuer la cause au Correcteur. « Flavigny fut accusé d’impiété, injurié, soupçonné de mauvaises mœurs pour une faute bizarre occasionnée par la disparition d’une seule lettre. Dans ses observations critiques contre la Bible polyglotte de Lejay, il avait cité ces deux versets de saint Matthieu : \emph{Quid ! vides festucam in oculo fratris tui, et trabem in oculo tuo non vides ? — Ejice primum trabem de oculo tuo, et tum videbis ejicere festucam de oculo fratris tui.}  La première lettre du mot \emph{oculo} (2\textsuperscript{e} partie du premier verset) ayant été enlevée à l’impression, il resta \emph{culo}, avec ce sens mal figuré : « Et tu ne vois pas une poutre dans ton c.. ! » Par une maladresse typographique, le second mot du second verset se trouvait ainsi divisé : \emph{o-culo ;} il n’en fallut pas davantage pour faire supposer chez l’auteur les plus coupables intentions. Il avait entre les mains la dernière épreuve de la feuille, où la faute n’existait pas. Il ne comprenait rien à ce changement diabolique. Il protesta publiquement, en jurant par les saintes Écritures qu’il était innocent d’une faute aussi impie, aussi odieuse, aussi sacrilège : mais il ne se remit jamais de cette mésaventure typographique, et trente ans après, au lit de mort, il était encore courroucé contre son malencontreux \emph{imprimeur.} »\par
Flavigny ne put pas accuser le correcteur, il avait lui-même corrigé les épreuves, et il  avait une bonne feuille en mains ; mais il s’en prit aussi injustement à son imprimeur, qui ne pouvait ni prévoir ni empêcher cet accident, dont la cause était toute manuelle. Il fut aussi inexorable, lui qui connaissait tous les écueils de la pratique, que les juges fanatiques d’Etienne Dolet, imprimeur à Lyon, qui fut pendu et brûlé comme \emph{athée } et \emph{relaps}, pour avoir laissé subsister les mots \emph{du tout}, ajoutés à la fin de cette phrase, qu’il avait traduite de Platon : « Après la mort tu ne seras plus rien. » Certes, on ne peut porter plus loin l’ignorance et la barbarie : il eût été de bonne équité que ces juges fussent écorchés vifs à leur tour, et que sur leur peau, réduite en parchemin, on imprimât la justification du malheureux Etienne Dolet. Mais, sans que l’on puisse soupçonner la malveillance du Correcteur, ou l’accuser d’incapacité, il est des fautes qui échappent si communément à l’œil le  plus exercé, que l’on s’étonne à bon droit en voyant des hommes éclairés en tirer d’aussi graves conséquences : le Correcteur, dont la vue embrasse plusieurs mots à la fois, lira souvent \emph{comme il doit y avoir}, et non \emph{comme il y a.} Il arrive encore, comme nous l’a fait voir l’anecdote de Flavigny, que des lettres disparaissent même pendant le tirage : de là ces mots tronqués, ces transpositions de lettres, desquels il résulte une mutilation plus ou moins complète de l’idée qu’on a voulu émettre, et qui, suivant les circonstances diverses où se trouvent placés les auteurs et suivant la nature du sujet qu’ils traitent, peuvent avoir les conséquences les plus déplorables : témoin l’abbé Sieyès, qui trouva dans la première épreuve d’un discours justificatif de sa conduite politique, les mots : \emph{J’ai abjuré la République, } au lieu de \emph{fai adjuré}.... « Comment fait-on de pareilles fautes ? » dit-il à celui qui lui remettait  cette épreuve : « l’imprimeur veut donc me faire guillotiner ? » Témoin encore le \emph{Moniteur de l’Empire}, qui, à l’époque où Napoléon fondait les plus grandes espérances sur son projet d’alliance avec l’empereur de Russie, publia en ce sens un article où il était dit : « Ces deux souverains, dont l’\emph{union} ne peut être qu’invincible... » Les trois dernières lettres du mot \emph{union} ayant été enlevées pendant l’impression, l’indignation d’Alexandre fut au comble lorsqu’il lut cette phrase ainsi dénaturée : « Ces deux souverains, dont l’{\scshape un} ne peut être qu’invincible... » Et toutes les réclames des numéros suivants ne purent suffire à détruire l’idée qu’il en avait conçue, que l’on avait voulu le mystifier. Que doit-on conclure de tout cela, si ce n’est à l’imperfection de l’art ; imperfection à laquelle les écrivains les plus méticuleux, les critiques les plus sévères, les ennemis les plus acharnés du Correcteur  sont les plus impuissants à remédier ? J’invoquerai à l’appui de cette assertion le témoignage du savant philologue Ange Rocca (\emph{Bibliotheca apostolica Vaticana commentario illustrata ; } Rome, 1591), qui pose en principe, d’après sa propre expérience, qu’il est impossible d’imprimer un ouvrage sans fautes. Il en accuse avec raison les auteurs, dont les copies sont trop souvent incorrectes, et. surtout ceux qui, par suite de leurs incertitudes, font faire une foule de corrections sur le plomb. Préposé par Sixte-Quint à la surveillance de l’imprimerie du Vatican, le docteur Rocca avait pu voir de près les difficultés sans nombre attachées à la lecture des épreuves, et il n’avait pas eu besoin d’avoir un échantillon de celles qui sortent des mains de M. de Balzac pour porter son jugement à cet égard. Une chose étrange, en effet, bizarre, inexplicable, c’est que l’attention la plus soutenue, les soins les plus scrupuleux,  ne puissent pas conduire à l’épuration complète d’une épreuve ; on pourrait même admettre qu’une trop grande tension d’esprit n’est pas sans inconvénient dans ce genre de travail, en ce qu’elle jette la perturbation dans les centres nerveux, provoque l’afflux du sang vers les régions supérieures, cause de l’engourdissement dans toute la périphérie du crâne, et par suite le trouble de la vue ; ces accidents morbides se rencontrent souvent chez les correcteurs, surtout aujourd’hui qu’ils sont astreints à passer dix heures consécutives, et quelquefois davantage, dans une espèce d’échoppe que l’on décore du nom de bureau. Là le Correcteur, atteint déjà moralement par la nature de son travail, souffre encore physiquement de la posture qu’il est obligé de tenir : la barre d’arrêt d’un pupitre trop haut, le bord anguleux d’une table trop basse, lui meurtrissent le thorax, et ses heures de travail sont des  heures de torture que chaque jour aggrave.\par
Voyez le \emph{Psalmum Codex de} 1457, imprimé à Mayence par Jean Fust et Pierre Schœffer : vous trouverez, à la première page de la souscription, \emph{prœsens spalmorum } pour \emph{prœsens psalmorum.} Ces deux inventeurs des caractères mobiles avaient dû pourtant apporter une grande attention à la correction de cette pièce, premier résultat et prospectus de leur précieuse découverte.\par
Le silence est la première condition d’ordre à observer dans le lieu où travaillent les correcteurs, et, sans exiger d’eux tout le stoïcisme qu’en réclame le savant austère Jérôme Hornschuch, docteur en médecine et correcteur dans l’imprimerie de Beyer, à Meinungen, qui leur recommande d’éviter avec le plus grand soin de s’abandonner à la colère, à l’amour, à la tristesse, enfin à toutes les émotions vives, il serait peut-être bon de leur recommander de faire leur profit  de cette inscription, qu’Alde Manuce avait fait placer à la porte de son bureau :\par
« Qui que vous soyez, Alde vous prie très instamment, si vous avez quelque chose à lui demander, d’être bref et de vous retirer de suite ; à moins que vous ne vouliez partager avec lui ses rudes travaux, car il y en aura toujours pour vous et pour tous ceux qui viendront le trouver. »\par
Les correcteurs ont en général assez de sujets de distraction dans les relations obligées du travail, et il est aussi indigne de leur caractère que préjudiciable à la bonne lecture des ouvrages, de se livrer près d’eux à des conversations frivoles ou tout au moins intempestives. On ne saurait trop blâmer le correcteur qui, peu soucieux de son devoir et du préjudice qu’il cause au maître imprimeur, se plaît à provoquer ou à alimenter des entretiens sans motif. La correction n’est  pas plus un travail mathématique qu’un travail manuel, et, s’il repose sur quelques règles générales, comme la connaissance des langues et l’expérience que réclame une bonne exécution typographique, il est le plus souvent soumis à l’arbitraire, et ne cède par conséquent que fort peu à l’habitude. Il ne suffit pas, en effet, de posséder à fond la connaissance des lettres, d’être bachelier ou docteur pour s’acquitter au mieux de l’emploi de correcteur : on peut le devenir sans cela, mais on ne peut pas l’être avec cela seul ; il faut encore avoir acquis une connaissance parfaite de la typographie, c’est-à-dire être bon compositeur et savoir apprécier le travail des imprimeurs. On peut, je le répète, être bon correcteur sans être bachelier ou docteur ès-lettres, parce que les titres ne sont pas toujours la preuve certaine du savoir, et qu’à moins d’y être contraint par la profession à laquelle on se  destine, on ne prend guère ces degrés que par pure vanité : aussi le plus souvent ceux qui les recherchent oublient-ils en les obtenant ce qui les leur a fait avoir. Sur la foi seule d’un diplôme on ne fait pas un correcteur ; il faut avant tout des connaissances typographiques, il faut qu’une longue expérience de l’imprimerie et de l’impression ait formé l’œil et le jugement du Correcteur. Il est impossible de se faire une idée des mille difficultés qui se dressent devant celui qui corrige une épreuve pour la première fois. Il est facile d’écrire, la plume vole, la ponctuation se sème au hasard, on orthographie selon Boiste, Noel, Napoléon Landais, l’Académie même ; on n’est point arrêté par l’emploi raisonné des majuscules, des minuscules, de l’italique, des points d’interrogation, d’exclamation ; par l’accord des mots entre eux ; par l’emploi des guillemets, des parenthèses, des traits d’union ; on  n’est point astreint surtout et rigoureusement à l’observation des règles de tel ou tel dictionnaire, de celui de l’Académie, par exemple, vrai dédale, vrai labyrinthe dans lequel viennent souvent se perdre les réputations les mieux établies, qui écrit la \emph{Bohême } avec un accent circonflexe, le \emph{Bohème} avec un accent grave, le \emph{Bohémien} avec un accent aigu ; \emph{séve} avec un accent aigu, \emph{fève} avec un accent grave, des \emph{pot-au-feu}, quand tous les autres écrivent des \emph{pots-au-feu}, \emph{Grand-Seigneur} avec deux capitales et une division, \emph{grand-duc} avec une division sans capitales, \emph{grand maître} sans division ni capitale, \emph{Sa Majesté} avec deux capitales, \emph{sa seigneurie} sans capitales, et mille autres mots entre lesquels l’Académie établit des distinctions bizarres, absurdes, sans compter les nombreuses exceptions créées par le caprice du maître, qui n’est pas toujours conséquent avec lui-même, el qui n’en exige pas moins que l’on  se conforme à sa volonté ! Le Correcteur, au contraire, ne voit autour de lui que difficultés ou écueils ; il se doit tout entier à l’observation religieuse dont les écrivains s’affranchissent sans scrupule, et son esprit, tendu dès la première page d’un ouvrage, est condamné à ne pas en perdre de vue un seul instant la marche, les détails et l’ensemble. Tantôt un auteur lui imposera des principes généraux d’orthographe ; tantôt un autre l’enfermera clans un labyrinthe grammatical qui lui est propre ; les uns voudront le \emph{t} au pluriel, d’autres le proscriront ; ceux-ci exigeront encore l’\emph{o} à l’imparfait, ceux-là écriront \emph{temps} sans \emph{p ;} et si, dans une maison, trois ouvrages se rencontrent, soumis chacun à une orthographe particulière, le Correcteur s’épuisera en efforts de mémoire pour satisfaire aux exigences de chacun, heureux encore d’en être quitte pour un travail aussi fastidieux qu’ingrat, et  si les reproches ne s’ajoutent pas parfois à la somme de ses tribulations quotidiennes.\par
Et voyez avec quelle facilité les auteurs élaborent leurs ouvrages, avec quel laisser-aller ils procèdent. Voici un échantillon de l’orthographe de Voltaire. Dans une lettre autographe de cet écrivain célèbre, signée V{\scshape oltaire}, \emph{chamberlan du roy de Prusse}, on lit les mots suivants écrits de la sorte : \emph{nouvau, touttes, nourit, souhaitté, baucoup, ramaux}, le \emph{fonds} de mon cœur, \emph{Adidote, crétien}, etc., etc., et tous les verbes sans distinction du présent ou du subjonctif ; \emph{à} préposition, comme \emph{a} verbe.\par
Eh bien ! en présence de tant de faits concluants, de tant de considérations dont la moindre suffirait certainement à défendre le Correcteur contre les attaques de la calomnie ou de l’ignorance, verrons-nous encore des censeurs impitoyables le charger du fardeau des erreurs typographiques ? Un  esprit rigide, une âme sainte, trop accessible au scandale, pardonnera-t-elle enfin au malheureux qui, prenant pour un \emph{a} un \emph{u} mal venu sans doute à l’impression (ce qu’on appelle \emph{laisser filer la coquille}), négligea de relever la singulière méprise que contenait la phrase suivante imprimée dans un missel : « Ici le prêtre ôte sa c\emph{u}lotte \emph{(c\emph{a}lotte)} et baise l’autel. » « Il faut que les correcteurs apportent à l’orthographe une attention d’autant plus minutieuse, dit quelque part Hornschuch, d’après Zeltner, que les savants ne sont pas forts sur cet article, qu’ils traitent souvent de bagatelle. »\par
En effet, si l’auteur a le génie, la propriété du style, au Correcteur appartient la netteté du style, l’exactitude et la régularité orthographique ; l’auteur est le sol qui crée et nourrit ; le Correcteur est le soleil qui fait éclore, la rosée qui alimente la création. Un livre dans lequel les fautes fourmillent  n’est pas seulement un mauvais livre, une œuvre informe, un salmigondis littéraire : c’est un livre dangereux. En effet, quoique l’imprimerie ait beaucoup perdu de son ancienne splendeur, il est encore une foule de gens qui ont une telle foi en toute chose imprimée, que quand ils liraient \emph{..tac} pour \emph{intact}, ils soutiendraient opiniâtrément et quand même que le mot est logique.\par
A-t-il donc dépendu du Correcteur de s’opposer à cette rétrogradation de son art, de lui conserver la considération dont l’entouraient les personnes les plus recommandables, et l’espèce de culte qu’avaient les petits esprits pour tout ce qui émanait de la presse ? a-t-il dépendu de lui que tant de myrmidons intellectuels devinssent ses censeurs les plus acharnés ? Non, sans doute, et tous ces Aristarques trempés en paquets, tous ces critiques dont l’ignorance n’a d’égale que leur audace, seraient bien en peine  de donner une raison à leurs attaques. Rien n’est plus difficile que de se juger soi-même ; aussi, comment leur faire avouer la large part qui leur revient dans les causes qui ont amené cette décadence ? Écrivains à la ligne, ils ont des correcteurs au mille. Sans suspecter la consciencieuse attention d’un correcteur qui lit une feuille \emph{aux pièces}, c’est-à-dire \emph{à la tâche}, n’est-il pas à craindre qu’entraîné sur le terrain glissant de l’intérêt personnel, il ne puisse se défendre d’une rapidité nuisible aux recherches minutieuses qu’une bonne correction exige de ses yeux et de son esprit ? Un libraire de Paris a payé, dit-on, jusqu’à 48 fr. la lecture de chaque feuille d’une collection in-52 de classiques latins, imprimée en 1824 et en 1825. Il est vrai de dire que les caractères employés à cette collection étaient infiniment petits ; mais si, dans des proportions équitables, le maître imprimeur conforme son tarif à cet exemple,  il aura le droit d’exiger de ses correcteurs beaucoup de temps, de soins et de savoir, et les abus intérieurs disparaîtront de l’imprimerie, en même temps que les \emph{errata} cesseront de déprécier les livres aux yeux du public. Mais les choses iraient ainsi naturellement, et cela n’aurait pas le sens commun.\par
Il faut innover, innover quand même, et sans considération aucune pour la dignité de l’art et de l’homme qui le professe, sans respect pour des précédents consacrés par la raison et l’expérience de plus de trois siècles !\par
Eh ! pour Dieu, messieurs, à quelque degré d’illustration typographique ou littéraire que vous apparteniez, cessez donc de vilipender sans pudeur le mérite d’un homme qui souvent, après sept ou huit années d’une jeunesse studieuse, en a consacré encore quatre ou cinq à l’apprentissage aussi long que pénible d’un métier sans profit. Les quelques centimes que vous lui jetez pour rendre  lisible un texte de mille lettres assemblées sans ordre, souvent même sans intelligence ; cette aumône, qui flétrit la main qui donne, sans avilir pourtant celle qui reçoit, est-ce donc là une rétribution équitable des jours qu’il a usés dans un travail ingrat, des nuits qu’il a consumées dans l’étude ? Avez-vous oublié l’exemple de Robert Etienne, le célèbre imprimeur du {\scshape xvi}\textsuperscript{e} siècle, le savant, l’homme juste, nommé par François I\textsuperscript{er} imprimeur royal pour le latin, qui sollicitait instamment les écoliers eux-mêmes à lui signaler les fautes que pouvaient contenir les ouvrages imprimés par ses soins. Ce n’était pas, comme l’a dit certain écrivain moderne, un pyrrhonisme outré, un vain système de doute, une défiance froide et méprisante envers les hommes laborieux dont il s’entourait, qui le poussaient à rechercher partout des censeurs ; lorsque Robert Etienne, stimulant l’attention de tous, offrait une récompense pécuniaire  par chaque faute trouvée dans les épreuves qu’il exposait à sa porte\footnote{ \noindent La maison de Robert Etienne était voisine du collège de Beauvais et des Écoles de droit, situées rue Saint-Jean-de-Beauvais. Il faisait placarder à sa porte les épreuves des pages composées, longtemps avant leur impression, et une annonce permanente appelait la critique de chacun sur ces épreuves.\par
 De nos jours, l’honorable Pierre Didot usa du même moyen pour obtenir des livres purgés de toute erreur. On assure qu’il donnait un petit écu (3 fr.) pour chaque faute qui lui était signalée : ce qui n’empêcha pas Pierre Didot d’acquérir une belle fortune. Abstenons-nous de tout rapprochement.
 }, le sentiment qui dominait sa pensée, c’était une modeste abnégation de son propre mérite, une noble confiance dans les lumières d’autrui ; son but, l’objet de toute son ambition, c’était la gloire et la perfection de son art. Il n’en estimait pas moins les talents et l’habileté d’un correcteur, seulement il croyait à la perfectibilité infinie. On raconte qu’un jour François I\textsuperscript{er}, voulant lui faire une visite, le trouva  occupé de la lecture d’une épreuve ; le roi ne souffrit pas que l’illustre artisan se dérangeât pour lui, et il attendit patiemment que sa tâche fut remplie. Un pareil trait les honore tous deux.\par
S’il m’était permis de m’étendre, je ne manquerais pas de textes et d’arguments qui, réunis à ceux que j’ai déjà produits, établiraient victorieusement à sa juste place le sujet de cette étude rapide ; mais, réduit par l’espace au \emph{paucis multùm}, je vais courir à la conclusion.\par
La vie de relation du correcteur est extrêmement bornée, et, à l’exclusion de quelques confrères, qu’il aime sans les connaître et qu’il quitte sans les regretter, son existence est toute patriarcale, et se réduit à l’intimité de la famille. Bon époux, bon père, bon garde national même quand ses moyens le lui permettent, il n’a d’ordinaire d’autre passion que celle de tous les typographes en  général; qu’ils soient imprimeurs, compositeurs ou correcteurs, et qui consiste dans un amour-propre exagéré ; encore est-il juste de dire qu’elle est née des deux premiers ordres, et que le sien n’en est que l’héritier. Rien n’est si plaisant, en effet, que d’entendre les discussions parfois très sérieuses qui s’élèvent entre des typographes sur les choses les plus insignifiantes. En vain on m’objectera que tout est grave en typographie : je n’en dirai pas moins qu’il est aussi absurde que puéril d’attacher une si grande importance à des faits qui en ont quelquefois si peu, et qui ne sont, le plus souvent, que l’effet d’une inadvertance bien pardonnable. Jamais ouvrier typographe, quelque habile qu’il fût, n’a eu raison contre son maître. Les choses les plus simples, omises avec l’intention d’y revenir, ou négligées avec discernement, sont pour lui des crimes de \emph{lèse art :} ce qui fait qu’une réputation  laborieusement acquise pourrait être perdue ou gravement compromise pour un bout de ligne laissé en tête d’une page, pour une feuille mal retournée sur la presse, ou une correction mal indiquée, si l’on s’en rapportait au jugement de ces oracles de l’imprimerie... S’agit-il de la composition d’un titre, le compositeur le dispose mal, le prote le corrige mal : le maître seul excelle dans l’art de faire, et le titre n’est présentable que quand il y a mis la main. De nos jours pourtant bien des ouvriers sont devenus maîtres : est-ce quand ils cessent de travailler qu’ils acquièrent l’essence du bon goût ? J’en doute ; j’affirme même qu’en cette matière Phèdre n’est que trop rarement vrai quand il dit : \emph{Dominum videre plurimùm in suis rebus.}\par
Il en est cependant qui sont plus excentriques, et ce ne sont pas les moins sages. Le culte que l’homme doit à l’art qui le fait  vivre n’est pas sans limites ; et, après avoir passé six jours dans la contemplation des \emph{premiers-Paris}, des harangues princières, de la pharmacopée, des épopées divines, du Digeste, du Codex ou des allocutions archiépiscopales ; après avoir péniblement \emph{filé} six jours à travers cette atmosphère d’idées politiques, aristocratiques, pharmaceutiques, drôlatico-poétiques, judiciaires, médicales, pastorales ou pontificales, le correcteur qu’on peut appeler le prototype de l’espèce salue avec transport l’aurore du jour d’heureuse tradition qui lui permet enfin d’étaler au soleil ses membres engourdis par des journées quotidiennes de dix à douze heures. Dans sa sainte abnégation de toutes les choses de ce monde, et surtout des épreuves, n’étaient les énormes besicles qui d’ordinaire enfourchent son nez, dont la dimension et les nombreuses aspérités ne nous offrent pas toujours la forme gracieuse  d’un nez fait à l’image du créateur, je le donnerais entre mille comme le modèle le plus parfait, comme un \emph{polytypage}\footnote{ \noindent Ce mot n’est d’usage en ce sens que dans l’imprimerie.
 } enfin du premier homme, tant il est beau d’abandon et de simplicité dans la solennité du septième jour, consacré au repos par le divin protecteur de l’humanité, tant il apporte d’empressement et de zèle dans l’observation du dimanche. Arrangeant à son usage une collection de petits aphorismes à la manière des proverbes de Basile, il arrive à cette déduction théologique, que Dieu, qui a voulu qu’à son exemple on travaillât six jours, a entendu aussi qu’on se reposât le septième, et il se repose : ceci n’a pas besoin de commentaire.\par
Voyons maintenant comment il emploie cette journée tant désirée. Le dimanche a sur les autres jours cet avantage, pour l’observateur,  que, par le mouvement qu’il provoque, la liberté d’action qu’il autorise, les plaisirs de toutes sortes qu’il suggère ou qu’il ranime, il met les mœurs à découvert, les expose à l’œil nu, comme disent ceux qui souvent ne voient rien à l’aide du microscope. Le sujet de cette variété, la seule qui offre quelque originalité, frise communément la soixantaine. Nous ne nous étendrons pas sur toutes les phases de sa vie : abstraction faite de son état, le Correcteur, dans un âge moins avancé, n’a rien de bien saillant sur les autres hommes ; nous pourrions même dire qu’il est d’autant moins original qu’il est plus jeune ; car sa fortune ne répondant pas toujours à son éducation, il est obligé de vivre dans un état d’isolement complet. Il tient le milieu entre la trinité de l’étudiant, du clerc et du commis, et l’unité de l’ouvrier ; il n’a point d’équivalent dans cette tourbe où paraissent se  confondre tant d’hommes de conditions et de rangs si divers ; car, on voudrait en vain se le dissimuler, malgré toute la vérité de la charte, il est encore bien des lignes que certaines bourses ne peuvent pas franchir.\par
Si vous avez quelquefois dépassé les limites de la banlieue, et poussé vos explorations jusqu’aux prés Saint-Gervais, à Romainville, au bois de Boulogne, à Gentilly, au bois de Vincennes, vous n’avez pas été sans rencontrer quelque personnage de vingt-cinq à trente ans, à la démarche grave, mais quelque peu étudiée, aux cheveux longs et bouclés avec soin, habit noir d’une coupe antérieure d’un an à la mode du jour, pantalon idem, bottes selon le temps, le tout d’une propreté irréprochable, cravate mise avec goût, et tenant à la main un livre dont il paraît dévorer la substance : c’est le correcteur cosmopolite aux limites de son univers, le Correcteur au début de sa carrière,  distillant avec feu sa sève juvénile sur le traité de la ponctuation de Lequien, ou le jardin des racines grecques de Claude Lancelot. Tel est ce néophyte de l’art : esclave de l’étude ou de l’ambition qu’elle lui suggère, trop pour être si peu, trop peu pour être quelque chose, sa vie n’est que labeur et déception. Obligé de travailler pour vivre, il ne peut faire que pour cela ; le peu de temps qui lui reste il le passe en méditations vagabondes. Il ne s’arrête à rien, son imagination bouillante le pousse malgré lui et l’empêche de rien achever. Enfin, la quarantaine sonne, son ardeur ambitieuse se calme peu à peu, ses yeux se dessillent. Il récapitule : un drame, trois vaudevilles, un traité typographique à l’état d’embryon sont entassés pêle-mêle au milieu d’un monceau de paperasses, telles que nouvelles, poésies, toutes plus fugitives les unes que les autres, quelques vers latins, des chansons,  des acrostiches, c’est à peu près ce que peut compter tout correcteur de son âge ; mais, désormais fixé sur le prix de toutes ces productions, il s’en remet, avec J.-B. Rousseau, à la lumière divine, du soin\par

D’\emph{illuminer} ses actions.\\

\noindent Puis, comme il faut se consoler de tout, « \emph{Félix qui potuit rerum cognoscere causas !} dit-il ; c’est encore être riche que de savoir que l’on est pauvre. »\par
Dès-lors tout change de face autour de lui ; ce n’est plus ce jeune insensé qu’une passion désordonnée jette sur tous les chemins à la poursuite d’une idée insaisissable, affectant de s’occuper d’une étude sérieuse pour faire diversion à une préoccupation incessante des conceptions les plus extravagantes, et, par une instabilité d’esprit, par une anomalie bizarre de goût et de pensée, se jetant parfois dans l’excès contraire ; il a su profiter des quelques traits de lumière qui ont  jailli du foyer même de ses aberrations littéraires ; il a meublé son esprit d’une foule de connaissances acquises par les nombreuses recherches qu’il a faites dans ses rêves de gloire et de postérité, et, rentré dans les limites de la saine raison, il en corrobore son éducation première pour en faire une utile application à son état, dont il sent, aujourd’hui tout le mérite, et qu’il veut rehausser des talents réels qu’il possède dans la matière et qui ont su résister aux débordements de son imagination autrefois exaltée.\par
Vingt années se sont encore écoulées ; il a atteint l’âge que nous lui connaissions avant de nous livrer à cette digression. Il s’est marié — parmi les correcteurs, les célibataires forment l’exception — et, ce qui prouve d’ailleurs, malgré sa vie studieuse, que le mariage et son objet sont inhérents à sa constitution, c’est sa rare fécondité, qui ne le cède qu’a celle d’un  dévot ou d’un maître d’école de campagne ; il a donné son nom à de nombreux enfants qui font ses délices, mais auxquels il inspire de bonne heure un profond dégoût pour la typographie, malgré sa vocation déterminée pour cet art, dont il exalte souvent lui-même les beautés et les difficultés, disant avec raison qu’il est encore aujourd’hui très honorable, mais en définitive beaucoup plus onéreux qu’honoré.\par
Le correcteur marié n’a que deux choses en vue : sa famille et le travail qui la fait vivre. Cependant, et malgré l’exiguité de sa fortune, il est rare qu’il n’augmente pas sa maison des deux commensaux obligés de tous les ménages parisiens, le chien et le chat, auxquels il donne toujours des noms tirés du théâtre, de l’histoire ou de la mythologie. Il est vrai qu’il ne voit tout cela que tous les huit jours, ses occupations l’obligeant à quitter son chez lui le matin  pour n’y rentrer que le soir ; mais il aime, lorsque vient le dimanche, à s’entourer de ce qu’il appelle son intérieur. Il ne sort jamais seul, à moins qu’une affaire toute personnelle ne l’y oblige ; l’hiver, ses goûts casaniers se développent au profit de ses chers enfants plus ou moins \emph{terribles}. Il s’occupe alors de leur éducation, et s’il les envoie à l’école dans la semaine, c’est uniquement pour les préserver du contact des gamins oisifs du quartier ; quelquefois seulement, et par mesure hygiénique, il conduit tout son personnel au jardin du Luxembourg, dont il est presque toujours voisin. Il est surtout curieux de le voir, dans ces moments de délicieuse abnégation paternelle, courant après son chien qui s’enfuit avec la casquette que le vent vient d’enlever à son aîné, traînant à grand’peine celui-ci, qui le tient par la main, et un autre qui s’est cramponné au pan de sa redingote,  laissant loin derrière lui sa femme avec trois autres marmots pleins de force et de santé, mais dont les jambes sont encore trop courtes, et qui tiraillent obstinément le bas de la robe de cette pauvre mère, embarrassée souvent d’un dernier nourrisson endormi dans ses bras. La balle, le cerceau, le bouchon, qui font les délices de ses aînés, mettent sa patience à une terrible épreuve ; il faut qu’il prenne une part active à tous leurs jeux, et, si son ardeur vient à se ralentir, un éternel \emph{encore} le presse et le poursuit sans pitié.\par
Il y a quelques jours que, longeant l’allée de l’Observatoire, mon attention fut vivement excitée par les cris d’un bambin et l’anxiété d’un de ces estimables pères de famille ; il marchait à grands pas, la chapeau à la main, le regard inquiet, la sueur au visage. Ce qu’il cherchait d’un œil avide, je ne l’aurais pas deviné, car  nous étions loin des pains d’épices et des verres de coco ; mais en voyant courir après lui un petit diable qui tenait à deux mains sa culotte, ma pénétration pressentit la difficulté de la conjoncture, et j’en conclus sans peine que le jardin de MM. les pairs n’a point été disposé à l’usage des enfants.\par
Mais, lorsque revient le printemps, lorsque l’aurore matinale et dorée succède aux jours sombres et paresseux de la saison brumeuse et glacée, ne laissant pour gardien de son foyer qu’un vieux chat dormeur et ingrat, il envahit, dès le dimanche matin, l’omnibus à sa première course, et fait bientôt irruption dans la plaine.\par
Nous ne suivrons pas cette colonne vagabonde chez le marchand devins traiteur du Petit-Montrouge, où l’appelle, à une heure quelconque de la journée, l’\emph{appétibilité} des solides comestibles dont la sollicitude maternelle et conjugale a gonflé un énorme cabas.  Il s’agit de vider le cabas au profit de l’estomac : on entre dans le premier \emph{bouchon} venu ; on demande un litre, une carafe d’eau pour en faire double dose, et l’officieux traiteur, joyeux ou non d’une si maigre aubaine, mais toujours empressé, couvre une table de verres, d’assiettes, etc., etc., et la petite troupe se restaure à loisir. Tout le monde connaît ce tableau de famille : nous conseillerons seulement aux philosophes pratiques, aux partisans du principe exclusif de la perception méditée, que le besoin d’une alimentation plus positive peut bien pousser quelquefois sur les traces de nos voyageurs, de prier Dieu qu’il les conduise dans la maison voisine.\par
Parmi ceux-ci, nous trouvons le correcteur célibataire. Le temps n’a point affaibli en lui les velléités littéraires qui ont présidé à son jeune âge ; il y a joint, au contraire, un esprit d’observation qui le pousse  toujours davantage vers la solitude. Mais il faut bien se garder d’en inférer que la réprobation dont l’ancienne Rome frappait ceux d’entre ses citoyens qui se montraient ainsi réfractaires aux liens et aux devoirs de la famille, doive nécessairement atteindre ce nouveau sujet d’observation au {\scshape xix}\textsuperscript{e} siècle, et surtout en France, où l’instruction commence à prendre la direction des mœurs, et où, par conséquent, la doctrine et la tradition ont beaucoup moins d’influence qu’en tout autre temps et en tout autre pays sur la moralisation des masses ; sans remonter aux causes qui ont suscité chez les Romains la fameuse loi contre le célibat, on peut, avec quelque raison, en induire que le besoin de soldats de la nation guerrière n’était pas l’unique objet qu’eussent en vue ses législateurs : parmi tant d’autres motifs dont les hauts dignitaires des empires se réservent toujours le secret quand il est question de  s’adresser au peuple, et quelquefois même de mettre un frein aux envahissements des grands, on peut supposer qu’une cause inconnue agit aussi puissamment que cette première raison d’État sur l’esprit du sénat, qui, cette fois, eut raison de déguiser le motif de sa détermination sous celui de l’intérêt national. Quel rapport y a-t-il en effet entre les mœurs de cette antique société et les mœurs de la société nouvelle ? Aucun évidemment ; et eût-on en vue de frapper l’égoïsme, cette plaie gangrenée de notre époque, on aurait fort mauvaise grâce de s’attaquer à la mansarde du célibataire. Ce vice odieux, élément unique du malaise qui afflige aujourd’hui les peuples, prend son exemple ailleurs, et ceux de nos soi-disant moralistes qui le poursuivent jusque là devraient s’apercevoir, à mesure qu’ils avancent dans cette voie, qu’ils perdront bientôt de vue jusqu’aux rayons vacillants  d’un astre épuisé par la longueur de sa projection, et qu’ils en approchent d’autant moins qu’ils s’éloignent davantage de leur point de départ. Mais, cette classe d’hommes méritât-elle la large part que lui font ces messieurs dans la répartition des nombreuses imperfections de la pauvre humanité, je revendiquerais l’exception en faveur du Correcteur que ses inclinations ont converti à cette caste, pour me servir de l’expression autrefois consacrée.\par
Ce n’est pas par égoïsme que le Correcteur se voue au célibat ; l’expérience du malheur des autres lui a appris à se défier de ses propres ressources, et s’il n’a pas voulu se créer une famille, c’est bien moins dans la crainte de manquer lui-même de quelque chose que dans l’appréhension, pour ceux qui l’eussent entouré, des privations sans nombre auxquelles sont exposés tous ceux qui doivent vivre du produit d’un emploi aussi  médiocre que le sien. Les quarante sous par jour, la veste de bure, les sabots et le front humilié dont le plus injustement célèbre de nos économistes veut que l’homme du peuple se contente et s’honore, tous les efforts oratoires et littéraires que ce savant a faits pour \emph{enseigner la manière de s’en servir}, n’ont pu ni le tenter, ni le dissuader, ni le convaincre, non plus pour les autres que pour lui ; ce qui ne prouve pas le moins du monde qu’il manque de philanthropie, mais ce qui établit indubitablement qu’il sait à quoi s’en tenir sur la logique de M. Dupin, et sur les prétentions aristocratiques de la noblesse agricole et industrielle éperonnée dont il se fait officieusement l’organe. S’il ignore les douceurs de la vie de famille, il sait en apprécier les peines ; et si quelque parent, quelque ami, peu soucieux d’imiter sa prudente réserve, vient à se trouver aux prises avec la misère qu’il a su éviter,  ce n’est pas par de vaines paroles qu’il accueille ses plaintes, par des reproches aussi inhumains qu’inopportuns sur son manque de prévision : philanthrope vraiment éclairé, il sait que nul remède n’a plus d’efficacité en pareille circonstance que celui que tous ces prôneurs humanitaires s’attachent avec le plus de soins à rendre odieux aux autres, afin sans doute d’en accaparer la substance, dans l’unique but d’en repaître leurs yeux avides, et sans que jamais il vienne à leur pensée d’en détourner la moindre parcelle au profit de tant de gens dont les souffrances s’accroissent et se multiplient dans la même proportion que leurs trésors ; et, heureux de saisir cette occasion d’être utile à son prochain, c’est par le généreux sacrifice d’une petite bourse remplie avec beaucoup de soins et de peines, mais qu’il délie de la meilleure grâce du monde et qu’il met à la disposition de celui  qu’il oblige, avec une délicatesse dont il faut aller chercher l’exemple dans la charité des premiers chrétiens, et qui épargne à celui-ci jusqu’à la honte d’un remercîment.\par
Il en est un, que je connais particulièrement, qui passe communément pour le misanthrope le plus farouche. Interrogez les compagnons de sa longue vie sur les bonnes œuvres qui l’honorent : il n’y aura qu’une voix pour les proclamer au dessus de tout éloge ; ils vous diront que cet homme, déjà plus que sexagénaire, s’applique encore avec un zèle infatigable à l’exercice de la plus noble philanthropie ; que, ne pouvant plus aider de sa bourse, il sacrifie tous ses instants à ramasser des aumônes sous toutes les formes imaginables pour donner des vêtements à celui-ci, du pain à celui-là, ou enfin pour couvrir d’un linceul et conduire dignement à la dernière demeure tel autre de ses confrères que la piété religieuse de  notre siècle jetterait sans scrupule au coin de la borne si la police voulait le souffrir, parce que sa vieillesse ou ses infirmités ne lui ont pas permis d’économiser les quelques francs que coûterait son cercueil, ou parce qu’une mort prématurée l’a privé subitement de sa réconciliation avec l’Église.\par
Ici comme dans bien d’autres classes, de tels caractères forment sans doute l’exception, et cette propension au mutualisme ne trouve malheureusement pas beaucoup d’imitateurs consciencieux. J’en sais, par exemple, dans la même catégorie, qui nous offriraient un déplorable contraste à l’examen ; mais la vie privée doit être murée, quand elle est indigne d’être offerte comme modèle, et, bornant notre rôle d’observateur à ce qu’il a de consolant, d’historique ou de technologique, nous nous renfermerons dans l’appréciation du type général.\par
Aussi simple dans ses goûts que modéré  dans ses opinions, le Correcteur, après sa chute au Parnasse, n’a jamais vu l’ambition obscurcir ses jours ; initié par état à toutes les manœuvres politiques, diplomatiques et financières ; correcteur aujourd’hui aux \emph{Débats}, demain au \emph{Moniteur} ou au \emph{Siècle}, il sait à quoi s’en tenir sur la fixité de principes de l’un, sur l’exactitude de l’autre, et enfin sur l’esprit d’intérêt général qui préside à la polémique du troisième ; il n’est pas jusqu’aux nouvelles télégraphiques insérées dans les colonnes du \emph{Commerce} dont il ne connaisse la source, et sur la valeur desquelles il ne soit fixé bien avant que ce puissant véhicule n’aille mettre en émoi tous les agioteurs de la Bourse et de Tortoni ; il assiste à la rédaction des lettres particulières du Levant ; il connaît l’estaminet d’où émanent tous les secrets d’ambassade et de cabinet ; il est à \emph{tu} et à \emph{toi} avec le fabricant de \emph{faits-Paris ;} le feuilletoniste ne  dédaigne pas lui-même de faire quelquefois la conversation avec lui, et l’un de ces confidents de toutes les pensées abandonnées à la presse a dû savoir pourquoi tel publiciste a placé dernièrement, dans un feuilleton du \emph{Courrier français}, la petite ville de Vic sur les bords de la Meurthe. Édifié de la sorte, le Correcteur voit d’un œil impassible toutes les marionnettes politiques ou littéraires ; il jaugerait, à un millième près, l’éminence d’un homme d’État et la profondeur d’un écrivain accrédité. Peut-être il avouerait dans l’intimité que l’un a les pieds dans le sable, et que la tête de l’autre est perdue dans les nuages. Aussi les livres bons ou utiles sont-ils l’objet de son culte invariable ; il n’est pas un prolétaire qui ait une bibliothèque aussi comfortable que le Correcteur, et, à moins que ses ressources ne s’harmonient pas avec ses désirs, il ne paraît pas un ouvrage de quelque mérite, par  livraisons ou autrement, qu’il n’en enrichisse sa collection. Un de ces \emph{bibliothécomanes} (que je pourrais nommer) gagnait par an dix-huit cents francs tout au plus, et ne dépensait pas moins de six cents francs en achats de livres. Indépendamment des productions nouvelles, il faut qu’il sache tout ce que l’étalage des bouquinistes recèle de plus précieux ; il est à la piste de toutes les ventes, et s’il se trouve dans un lot un ouvrage qui lui plaise et qu’il ne puisse en faire extraire, il en reconnaît l’adjudicataire et le poursuit jusqu’il ce qu’il l’ait déterminé à le lui céder. Partant, les ponts et les quais sont ses galeries les plus habituelles, et si plus tard son penchant pour la littérature vient à s’affaiblir, il abaisse ses regards sur la Seine, et c’est à la pêche à la ligne qu’il voue le reste de sa vie active.\par
On parlera longtemps dans l’imprimerie d’un correcteur très estimé chez lequel cette  passion s’était développée à ce point, qu’ayant obtenu de l’imprimeur chez lequel il travaillait de lire des épreuves à ses pièces et chez lui, il partait le matin, ses épreuves, son écritoire et son filet dans sa poche, sa ligne et sa boîte à la main, et allait s’établir au bas du pont Marie, où il cumulait l’industrie des premiers disciples du Christ avec l’art de Robert Etienne.\par
Malheureusement, il n’est rien dans la typographie parisienne qui ne se sache, et le ridicule, par un tact essentiellement propre à ce monde tout-à-fait exceptionnel, y est saisi et poursuivi aussitôt qu’il s’y décèle ; la discrétion du pêcheur se mesure aussi bien, d’ailleurs, sur le succès du jour que celle du chasseur, et notre honorable confrère, s’étant sans doute laissé dominer quelquefois par l’enthousiasme immodéré auquel s’abandonnent si facilement ces rois des deux plaines, ne pouvait échapper longtemps  aux sarcasmes et aux malicieuses entreprises des plaisants plus ou moins spirituels du métier.\par
Un jour que, se livrant, à sa place de prédilection, \emph{au double attrait de la pêche et de la lecture des épreuves}, il s’était profondément endormi, un de ses camarades, qui avait à dessein suivi tous ses mouvements, résolut de mettre cet accident à profit. L’exécution suivit de près la pensée, et, en un instant la ligne fut amenée, replacée dans la main insensible qui l’avait complaisamment abandonnée, et l’hameçon, garni d’un énorme hareng saur, discrètement restitué aux eaux paisibles du fleuve. Tout-à-coup, et comme par un mouvement électrique, la main du bonhomme se contracte pour retenir ce qui semble vouloir lui échapper... Il s’éveille, se lève d’un bond, et, déjà tout radieux d’une aussi bonne fortune, pique adroitement d’un petit coup moelleux,  amène à fleur d’eau... Il allait, hélas ! rendre hommage au dieu Glaucus, lorsque, donnant enfin à sa ligne une impulsion décisive, il aperçut se balançant au soleil le corps inflexible du gymnopome enfumé !\par
Je crois que le plaisant rit seul, mais la plaisanterie guérit de sa manie aquatique le Correcteur, qui, ne voulant plus s’exposer à pareille déception, laissa désormais les poissons tranquilles.\par
De quelque manière qu’ils vivent, quels que soient leurs travaux ou leurs plaisirs, les correcteurs ont ordinairement une destinée commune ; et, à l’exclusion du correcteur de province, multinôme indispensable aux imprimeurs en sous ordre, pour lesquels une seule fraction de ses connaissances en moins en ferait un être incomplet, et qui par cela même arrive si rarement à la perfection, qu’il est souvent oblige de renoncer prématurément à sa profession, après avoir  exploité à satiété par ces rameaux des imprimeurs rayonnants de la capitale, à l’exclusion de celui-là, disons-nous, tous convergent à peu près vers le même but. Enfant banni des régions enchantées de ce monde, le Correcteur, contraint de bonne heure à réprimer les élans d’une imagination trop vive, et dont l’éducation obligée de son état tend toujours à élargir le champ, trace lui-même le cercle dans lequel il doit vivre et mourir ; et, sans que jamais personne ait daigné jeter un regard d’intérêt sur les efforts constants de ce modèle de philosophie naturelle, il va s’éteindre misérablement à Bicêtre, quand toutefois des infirmités anticipées ne le contraignent pas à implorer la faveur d’aller mourir plus misérablement encore au dépôt de Saint-Denis.\par


\begin{raggedleft}FIN\end{raggedleft}
 


% at least one empty page at end (for booklet couv)
\ifbooklet
  \pagestyle{empty}
  \clearpage
  % 2 empty pages maybe needed for 4e cover
  \ifnum\modulo{\value{page}}{4}=0 \hbox{}\newpage\hbox{}\newpage\fi
  \ifnum\modulo{\value{page}}{4}=1 \hbox{}\newpage\hbox{}\newpage\fi


  \hbox{}\newpage
  \ifodd\value{page}\hbox{}\newpage\fi
  {\centering\color{rubric}\bfseries\noindent\large
    Hurlus ? Qu’est-ce.\par
    \bigskip
  }
  \noindent Des bouquinistes électroniques, pour du texte libre à participation libre,
  téléchargeable gratuitement sur \href{https://hurlus.fr}{\dotuline{hurlus.fr}}.\par
  \bigskip
  \noindent Cette brochure a été produite par des éditeurs bénévoles.
  Elle n’est pas faîte pour être possédée, mais pour être lue, et puis donnée.
  Que circule le texte !
  En page de garde, on peut ajouter une date, un lieu, un nom ; pour suivre le voyage des idées.
  \par

  Ce texte a été choisi parce qu’une personne l’a aimé,
  ou haï, elle a en tous cas pensé qu’il partipait à la formation de notre présent ;
  sans le souci de plaire, vendre, ou militer pour une cause.
  \par

  L’édition électronique est soigneuse, tant sur la technique
  que sur l’établissement du texte ; mais sans aucune prétention scolaire, au contraire.
  Le but est de s’adresser à tous, sans distinction de science ou de diplôme.
  Au plus direct ! (possible)
  \par

  Cet exemplaire en papier a été tiré sur une imprimante personnelle
   ou une photocopieuse. Tout le monde peut le faire.
  Il suffit de
  télécharger un fichier sur \href{https://hurlus.fr}{\dotuline{hurlus.fr}},
  d’imprimer, et agrafer ; puis de lire et donner.\par

  \bigskip

  \noindent PS : Les hurlus furent aussi des rebelles protestants qui cassaient les statues dans les églises catholiques. En 1566 démarra la révolte des gueux dans le pays de Lille. L’insurrection enflamma la région jusqu’à Anvers où les gueux de mer bloquèrent les bateaux espagnols.
  Ce fut une rare guerre de libération dont naquit un pays toujours libre : les Pays-Bas.
  En plat pays francophone, par contre, restèrent des bandes de huguenots, les hurlus, progressivement réprimés par la très catholique Espagne.
  Cette mémoire d’une défaite est éteinte, rallumons-la. Sortons les livres du culte universitaire, cherchons les idoles de l’époque, pour les briser.
\fi

\ifdev % autotext in dev mode
\fontname\font — \textsc{Les règles du jeu}\par
(\hyperref[utopie]{\underline{Lien}})\par
\noindent \initialiv{A}{lors là}\blindtext\par
\noindent \initialiv{À}{ la bonheur des dames}\blindtext\par
\noindent \initialiv{É}{tonnez-le}\blindtext\par
\noindent \initialiv{Q}{ualitativement}\blindtext\par
\noindent \initialiv{V}{aloriser}\blindtext\par
\Blindtext
\phantomsection
\label{utopie}
\Blinddocument
\fi
\end{document}
