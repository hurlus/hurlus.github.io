%%%%%%%%%%%%%%%%%%%%%%%%%%%%%%%%%
% LaTeX model https://hurlus.fr %
%%%%%%%%%%%%%%%%%%%%%%%%%%%%%%%%%

% Needed before document class
\RequirePackage{pdftexcmds} % needed for tests expressions
\RequirePackage{fix-cm} % correct units

% Define mode
\def\mode{a4}

\newif\ifaiv % a4
\newif\ifav % a5
\newif\ifbooklet % booklet
\newif\ifcover % cover for booklet

\ifnum \strcmp{\mode}{cover}=0
  \covertrue
\else\ifnum \strcmp{\mode}{booklet}=0
  \booklettrue
\else\ifnum \strcmp{\mode}{a5}=0
  \avtrue
\else
  \aivtrue
\fi\fi\fi

\ifbooklet % do not enclose with {}
  \documentclass[french,twoside]{book} % ,notitlepage
  \usepackage[%
    papersize={105mm, 297mm},
    inner=12mm,
    outer=12mm,
    top=20mm,
    bottom=15mm,
    marginparsep=3pt,
    marginpar=7mm,
  ]{geometry}
  \usepackage[fontsize=9.5pt]{scrextend} % for Roboto
\else\ifav % A5
  \documentclass[french,twoside]{book} % ,notitlepage
  \usepackage[%
    a5paper
  ]{geometry}
  \usepackage[fontsize=12pt]{scrextend}
\else% A4 2 cols
  \documentclass[twocolumn]{report}
  \usepackage[%
    a4paper,
    inner=15mm,
    outer=10mm,
    top=25mm,
    bottom=18mm,
    marginparsep=0pt,
  ]{geometry}
  \setlength{\columnsep}{20mm}
  \usepackage[fontsize=9.5pt]{scrextend}
\fi\fi

%%%%%%%%%%%%%%
% Alignments %
%%%%%%%%%%%%%%
% before teinte macros

\setlength{\arrayrulewidth}{0.2pt}
\setlength{\columnseprule}{\arrayrulewidth} % twocol
\setlength{\parskip}{0pt} % 1pt allow better vertical justification
\setlength{\parindent}{1.5em}

%%%%%%%%%%
% Colors %
%%%%%%%%%%
% before Teinte macros

\usepackage[dvipsnames]{xcolor}
\definecolor{rubric}{HTML}{800000} % the tonic 0c71c3
\def\columnseprulecolor{\color{rubric}}
\colorlet{borderline}{rubric!30!} % definecolor need exact code
\definecolor{shadecolor}{gray}{0.95}
\definecolor{bghi}{gray}{0.5}

%%%%%%%%%%%%%%%%%
% Teinte macros %
%%%%%%%%%%%%%%%%%
%%%%%%%%%%%%%%%%%%%%%%%%%%%%%%%%%%%%%%%%%%%%%%%%%%%
% <TEI> generic (LaTeX names generated by Teinte) %
%%%%%%%%%%%%%%%%%%%%%%%%%%%%%%%%%%%%%%%%%%%%%%%%%%%
% This template is inserted in a specific design
% It is XeLaTeX and otf fonts

\makeatletter % <@@@

\usepackage{alphalph} % for alph couter z, aa, ab…
\usepackage{blindtext} % generate text for testing
\usepackage{booktabs} % for tables: \toprule, \midrule…
\usepackage[strict]{changepage} % for modulo 4
\usepackage{contour} % rounding words
\usepackage[nodayofweek]{datetime}
\usepackage{enumitem} % <list>
\usepackage{etoolbox} % patch commands
\usepackage{fancyvrb}
\usepackage{fancyhdr}
\usepackage{float}
\usepackage{fontspec} % XeLaTeX mandatory for fonts
\usepackage{footnote} % used to capture notes in minipage (ex: quote)
\usepackage{framed} % bordering correct with footnote hack
\usepackage{graphicx}
\usepackage{lettrine} % drop caps
\usepackage{lipsum} % generate text for testing
\usepackage{manyfoot} % for parallel footnote numerotation
\usepackage[framemethod=tikz,]{mdframed} % maybe used for frame with footnotes inside
\usepackage[defaultlines=2,all]{nowidow} % at least 2 lines by par (works well!)
\usepackage{pdftexcmds} % needed for tests expressions
\usepackage{poetry} % <l>, bad for theater
\usepackage{polyglossia} % bug Warning: "Failed to patch part"
\usepackage[%
  indentfirst=false,
  vskip=1em,
  noorphanfirst=true,
  noorphanafter=true,
  leftmargin=\parindent,
  rightmargin=0pt,
]{quoting}
\usepackage{ragged2e}
\usepackage{setspace} % \setstretch for <quote>
\usepackage{scrextend} % KOMA-common, used for addmargin
\usepackage{tabularx} % <table>
\usepackage[explicit]{titlesec} % wear titles, !NO implicit
\usepackage{tikz} % ornaments
\usepackage{tocloft} % styling tocs
\usepackage[fit]{truncate} % used im runing titles
\usepackage{unicode-math}
\usepackage[normalem]{ulem} % breakable \uline, normalem is absolutely necessary to keep \emph
\usepackage{xcolor} % named colors
\usepackage{xparse} % @ifundefined
\XeTeXdefaultencoding "iso-8859-1" % bad encoding of xstring
\usepackage{xstring} % string tests
\XeTeXdefaultencoding "utf-8"

\defaultfontfeatures{
  % Mapping=tex-text, % no effect seen
  Scale=MatchLowercase,
  Ligatures={TeX,Common},
}
\newfontfamily\zhfont{Noto Sans CJK SC}

% Metadata inserted by a program, from the TEI source, for title page and runing heads
\title{\textbf{ Réflexions sur la puissance motrice du feu }\par
\medskip
\textbf{ et sur les machines propres à développer cette puissance. }\par
}
\date{1824}
\author{Sadi Carnot}
\def\elbibl{Sadi Carnot. 1824. \emph{Réflexions sur la puissance motrice du feu}}
\def\elsource{ \href{https://fr.wikisource.org/wiki/R\%C3\%A9flexions\_sur\_la\_puissance\_motrice\_du\_feu}{\dotuline{Wikisource}}\footnote{\href{https://fr.wikisource.org/wiki/R\%C3\%A9flexions\_sur\_la\_puissance\_motrice\_du\_feu}{\url{https://fr.wikisource.org/wiki/R\%C3\%A9flexions\_sur\_la\_puissance\_motrice\_du\_feu}}} }
\def\eltitlepage{%
{\centering\parindent0pt
  {\LARGE\addfontfeature{LetterSpace=25}\bfseries Sadi Carnot\par}\bigskip
  {\Large 1824\par}\bigskip
  {\LARGE
\bigskip\textbf{Réflexions sur la puissance motrice du feu}\par
\bigskip\textbf{et sur les machines propres à développer cette puissance.}\par

  }
}

}

% Default metas
\newcommand{\colorprovide}[2]{\@ifundefinedcolor{#1}{\colorlet{#1}{#2}}{}}
\colorprovide{rubric}{red}
\colorprovide{silver}{lightgray}
\@ifundefined{syms}{\newfontfamily\syms{DejaVu Sans}}{}
\newif\ifdev
\@ifundefined{elbibl}{% No meta defined, maybe dev mode
  \newcommand{\elbibl}{Titre court ?}
  \newcommand{\elbook}{Titre du livre source ?}
  \newcommand{\elabstract}{Résumé\par}
  \newcommand{\elurl}{http://oeuvres.github.io/elbook/2}
  \author{Éric Lœchien}
  \title{Un titre de test assez long pour vérifier le comportement d’une maquette}
  \date{1566}
  \devtrue
}{}
\let\eltitle\@title
\let\elauthor\@author
\let\eldate\@date




% generic typo commands
\newcommand{\astermono}{\medskip\centerline{\color{rubric}\large\selectfont{\syms ✻}}\medskip\par}%
\newcommand{\astertri}{\medskip\par\centerline{\color{rubric}\large\selectfont{\syms ✻\,✻\,✻}}\medskip\par}%
\newcommand{\asterism}{\bigskip\par\noindent\parbox{\linewidth}{\centering\color{rubric}\large{\syms ✻}\\{\syms ✻}\hskip 0.75em{\syms ✻}}\bigskip\par}%

% lists
\newlength{\listmod}
\setlength{\listmod}{\parindent}
\setlist{
  itemindent=!,
  listparindent=\listmod,
  labelsep=0.2\listmod,
  parsep=0pt,
  % topsep=0.2em, % default topsep is best
}
\setlist[itemize]{
  label=—,
  leftmargin=0pt,
  labelindent=1.2em,
  labelwidth=0pt,
}
\setlist[enumerate]{
  label={\arabic*°},
  labelindent=0.8\listmod,
  leftmargin=\listmod,
  labelwidth=0pt,
}
% list for big items
\newlist{decbig}{enumerate}{1}
\setlist[decbig]{
  label={\bf\color{rubric}\arabic*.},
  labelindent=0.8\listmod,
  leftmargin=\listmod,
  labelwidth=0pt,
}
\newlist{listalpha}{enumerate}{1}
\setlist[listalpha]{
  label={\bf\color{rubric}\alph*.},
  leftmargin=0pt,
  labelindent=0.8\listmod,
  labelwidth=0pt,
}
\newcommand{\listhead}[1]{\hspace{-1\listmod}\emph{#1}}

\renewcommand{\hrulefill}{%
  \leavevmode\leaders\hrule height 0.2pt\hfill\kern\z@}

% General typo
\DeclareTextFontCommand{\textlarge}{\large}
\DeclareTextFontCommand{\textsmall}{\small}

% commands, inlines
\newcommand{\anchor}[1]{\Hy@raisedlink{\hypertarget{#1}{}}} % link to top of an anchor (not baseline)
\newcommand\abbr[1]{#1}
\newcommand{\autour}[1]{\tikz[baseline=(X.base)]\node [draw=rubric,thin,rectangle,inner sep=1.5pt, rounded corners=3pt] (X) {\color{rubric}#1};}
\newcommand\corr[1]{#1}
\newcommand{\ed}[1]{ {\color{silver}\sffamily\footnotesize (#1)} } % <milestone ed="1688"/>
\newcommand\expan[1]{#1}
\newcommand\foreign[1]{\emph{#1}}
\newcommand\gap[1]{#1}
\renewcommand{\LettrineFontHook}{\color{rubric}}
\newcommand{\initial}[2]{\lettrine[lines=2, loversize=0.3, lhang=0.3]{#1}{#2}}
\newcommand{\initialiv}[2]{%
  \let\oldLFH\LettrineFontHook
  % \renewcommand{\LettrineFontHook}{\color{rubric}\ttfamily}
  \IfSubStr{QJ’}{#1}{
    \lettrine[lines=4, lhang=0.2, loversize=-0.1, lraise=0.2]{\smash{#1}}{#2}
  }{\IfSubStr{É}{#1}{
    \lettrine[lines=4, lhang=0.2, loversize=-0, lraise=0]{\smash{#1}}{#2}
  }{\IfSubStr{ÀÂ}{#1}{
    \lettrine[lines=4, lhang=0.2, loversize=-0, lraise=0, slope=0.6em]{\smash{#1}}{#2}
  }{\IfSubStr{A}{#1}{
    \lettrine[lines=4, lhang=0.2, loversize=0.2, slope=0.6em]{\smash{#1}}{#2}
  }{\IfSubStr{V}{#1}{
    \lettrine[lines=4, lhang=0.2, loversize=0.2, slope=-0.5em]{\smash{#1}}{#2}
  }{
    \lettrine[lines=4, lhang=0.2, loversize=0.2]{\smash{#1}}{#2}
  }}}}}
  \let\LettrineFontHook\oldLFH
}
\newcommand{\labelchar}[1]{\textbf{\color{rubric} #1}}
\newcommand{\lnatt}[1]{\reversemarginpar\marginpar[\sffamily\scriptsize #1]{}}
\newcommand{\milestone}[1]{\autour{\footnotesize\color{rubric} #1}} % <milestone n="4"/>
\newcommand\name[1]{#1}
\newcommand\orig[1]{#1}
\newcommand\orgName[1]{#1}
\newcommand\persName[1]{#1}
\newcommand\placeName[1]{#1}
\newcommand{\pn}[1]{\IfSubStr{-—–¶}{#1}% <p n="3"/>
  {\noindent{\bfseries\color{rubric}   ¶  }}
  {{\footnotesize\autour{#1}}}}
\newcommand\reg{}
% \newcommand\ref{} % already defined
\newcommand\sic[1]{#1}
\newcommand\surname[1]{\textsc{#1}}
\newcommand\term[1]{\textbf{#1}}
\newcommand\zh[1]{{\zhfont #1}}


\def\mednobreak{\ifdim\lastskip<\medskipamount
  \removelastskip\nopagebreak\medskip\fi}
\def\bignobreak{\ifdim\lastskip<\bigskipamount
  \removelastskip\nopagebreak\bigskip\fi}

% commands, blocks

\newcommand{\byline}[1]{\bigskip{\RaggedLeft{#1}\par}\bigskip}
% \setlength{\RaggedLeftLeftskip}{2em plus \leftskip}
\newcommand{\bibl}[1]{{\smallskip\RaggedLeft\normalsize\normalfont #1\par\medskip}}
\newcommand{\biblitem}[1]{{\noindent\hangindent=\parindent   #1\par}}
\newcommand{\castItem}[1]{{\noindent\hangindent=\parindent #1\par}}
\newcommand{\dateline}[1]{\medskip{\RaggedLeft{#1}\par}\bigskip}
\newcommand{\docAuthor}[1]{{\large\bigskip #1 \par\bigskip}}
\newcommand{\docDate}[1]{#1 \ifvmode\par\fi }
\newcommand{\docImprint}[1]{\ifvmode\medskip\fi #1 \ifvmode\par\fi }
\newcommand{\labelblock}[1]{\medbreak{\noindent\color{rubric}\bfseries #1}\par\mednobreak}
\newcommand{\salute}[1]{\bigbreak{#1}\par\medbreak}
\newcommand{\signed}[1]{\medskip{\RaggedLeft #1\par}\bigbreak} % supposed bottom
\newcommand{\speaker}[1]{\medskip{\Centering\sffamily #1\par\nopagebreak}} % supposed bottom
\newcommand{\stagescene}[1]{{\Centering\sffamily #1\par}\bigskip}
\newcommand{\stagesp}[1]{\begingroup\leftskip\parindent\noindent\it\sffamily #1\par\endgroup} % left margin, better than list envs
\newcommand{\spl}[1]{\noindent\hangindent=2\parindent  #1\par} % sp/l
\newcommand{\trailer}[1]{{\Centering\bigskip #1\par}} % sp/l

% environments for blocks (some may become commands)
\newenvironment{borderbox}{}{} % framing content
\newenvironment{citbibl}{\ifvmode\hfill\fi}{\ifvmode\par\fi }
\newenvironment{msHead}{\vskip6pt}{\par}
\newenvironment{msItem}{\vskip6pt}{\par}


% environments for block containers
\newenvironment{argument}{\itshape\parindent0pt}{\bigskip}
\newenvironment{biblfree}{}{\ifvmode\par\fi }
\newenvironment{bibitemlist}[1]{%
  \list{\@biblabel{\@arabic\c@enumiv}}%
  {%
    \settowidth\labelwidth{\@biblabel{#1}}%
    \leftmargin\labelwidth
    \advance\leftmargin\labelsep
    \@openbib@code
    \usecounter{enumiv}%
    \let\p@enumiv\@empty
    \renewcommand\theenumiv{\@arabic\c@enumiv}%
  }
  \sloppy
  \clubpenalty4000
  \@clubpenalty \clubpenalty
  \widowpenalty4000%
  \sfcode`\.\@m
}%
{\def\@noitemerr
  {\@latex@warning{Empty `bibitemlist' environment}}%
\endlist}
\newenvironment{docTitle}{\LARGE\bigskip\bfseries\onehalfspacing}{\bigskip}
% leftskip makes big bugs in Lexmark printing \sffamily
\newenvironment{epigraph}{\begin{addmargin}[2\parindent]{0em}\sffamily\large\setstretch{0.95}}{\end{addmargin}\bigskip}
\newenvironment{quoteblock}% may be used for ornaments
  {\begin{quoting}}
  {\end{quoting}}
\newenvironment{titlePage}
  {\Centering}
  {}






% table () is preceded and finished by custom command
\renewcommand\tabularxcolumn[1]{m{#1}}% for vertical centering text in X column
\newcommand{\tableopen}[1]{%
  \ifnum\strcmp{#1}{wide}=0{%
    \begin{center}
  }
  \else\ifnum\strcmp{#1}{long}=0{%
    \begin{center}
  }
  \else{%
    \begin{center}
  }
  \fi\fi
}
\newcommand{\tableclose}[1]{%
  \ifnum\strcmp{#1}{wide}=0{%
    \end{center}
  }
  \else\ifnum\strcmp{#1}{long}=0{%
    \end{center}
  }
  \else{%
    \end{center}
  }
  \fi\fi
}


% text structure
\newcommand\chapteropen{} % before chapter title
\newcommand\chaptercont{} % after title, argument, epigraph…
\newcommand\chapterclose{} % maybe useful for multicol settings
\setcounter{secnumdepth}{-2} % no counters for hierarchy titles
\setcounter{tocdepth}{5} % deep toc
\renewcommand\tableofcontents{\@starttoc{toc}}
% toclof format
% \renewcommand{\@tocrmarg}{0.1em} % Useless command?
% \renewcommand{\@pnumwidth}{0.5em} % {1.75em}
\renewcommand{\@cftmaketoctitle}{}
\setlength{\cftbeforesecskip}{\z@ \@plus.2\p@}
\renewcommand{\cftchapfont}{}
\renewcommand{\cftchapdotsep}{\cftdotsep}
\renewcommand{\cftchapleader}{\normalfont\cftdotfill{\cftchapdotsep}}
\renewcommand{\cftchappagefont}{\bfseries}
\setlength{\cftbeforechapskip}{0em \@plus\p@}
% \renewcommand{\cftsecfont}{\small\relax}
\renewcommand{\cftsecpagefont}{\normalfont}
% \renewcommand{\cftsubsecfont}{\small\relax}
\renewcommand{\cftsecdotsep}{\cftdotsep}
\renewcommand{\cftsecpagefont}{\normalfont}
\renewcommand{\cftsecleader}{\normalfont\cftdotfill{\cftsecdotsep}}
\setlength{\cftsecindent}{1em}
\setlength{\cftsubsecindent}{2em}
\setlength{\cftsubsubsecindent}{3em}
\setlength{\cftchapnumwidth}{1em}
\setlength{\cftsecnumwidth}{1em}
\setlength{\cftsubsecnumwidth}{1em}
\setlength{\cftsubsubsecnumwidth}{1em}

% footnotes
\newif\ifheading
\newcommand*{\fnmarkscale}{\ifheading 0.70 \else 1 \fi}
\renewcommand\footnoterule{\vspace*{0.3cm}\hrule height \arrayrulewidth width 3cm \vspace*{0.3cm}}
\setlength\footnotesep{1.5\footnotesep} % footnote separator
\renewcommand\@makefntext[1]{\parindent 1.5em \noindent \hb@xt@1.8em{\hss{\normalfont\@thefnmark . }}#1} % no superscipt in foot
\patchcmd{\@footnotetext}{\footnotesize}{\footnotesize\sffamily}{}{} % before scrextend, hyperref
\DeclareNewFootnote{A}[alph] % for editor notes
\renewcommand*{\thefootnoteA}{\alphalph{\value{footnoteA}}} % z, aa, ab…

% poem
\setlength{\poembotskip}{0pt}
\setlength{\poemtopskip}{0pt}
\setlength{\poemindent}{0pt}
\poemlinenumsfalse

%   see https://tex.stackexchange.com/a/34449/5049
\def\truncdiv#1#2{((#1-(#2-1)/2)/#2)}
\def\moduloop#1#2{(#1-\truncdiv{#1}{#2}*#2)}
\def\modulo#1#2{\number\numexpr\moduloop{#1}{#2}\relax}

% orphans and widows, nowidow package in test
% from memoir package
\clubpenalty=9996
\widowpenalty=9999
\brokenpenalty=4991
\predisplaypenalty=10000
\postdisplaypenalty=1549
\displaywidowpenalty=1602
\hyphenpenalty=400
% report h or v overfull ?
\hbadness=4000
\vbadness=4000
% good to avoid lines too wide
\emergencystretch 3em
\pretolerance=750
\tolerance=2000
\def\Gin@extensions{.pdf,.png,.jpg,.mps,.tif}

\PassOptionsToPackage{hyphens}{url} % before hyperref and biblatex, which load url package
\usepackage{hyperref} % supposed to be the last one, :o) except for the ones to follow
\hypersetup{
  % pdftex, % no effect
  pdftitle={\elbibl},
  % pdfauthor={Your name here},
  % pdfsubject={Your subject here},
  % pdfkeywords={keyword1, keyword2},
  bookmarksnumbered=true,
  bookmarksopen=true,
  bookmarksopenlevel=1,
  pdfstartview=Fit,
  breaklinks=true, % avoid long links, overrided by url package
  pdfpagemode=UseOutlines,    % pdf toc
  hyperfootnotes=true,
  colorlinks=false,
  pdfborder=0 0 0,
  % pdfpagelayout=TwoPageRight,
  % linktocpage=true, % NO, toc, link only on page no
}
\urlstyle{same} % after hyperref



\makeatother % /@@@>
%%%%%%%%%%%%%%
% </TEI> end %
%%%%%%%%%%%%%%


%%%%%%%%%%%%%
% footnotes %
%%%%%%%%%%%%%
\renewcommand{\thefootnote}{\bfseries\textcolor{rubric}{\arabic{footnote}}} % color for footnote marks

%%%%%%%%%
% Fonts %
%%%%%%%%%
% \linespread{0.90} % too compact, keep font natural
\ifav % A5

\else\ifbooklet
  \usepackage[]{roboto} % SmallCaps, Regular is a bit bold
  \setmainfont[
    ItalicFont={Roboto Light Italic},
  ]{Roboto}
  \setsansfont{Roboto Light} % seen, if not set, problem with printer
  \newfontfamily\fontrun[]{Roboto Condensed Light} % condensed runing heads
\else
  \usepackage[]{roboto} % SmallCaps, Regular is a bit bold
  \setmainfont[
    ItalicFont={Roboto Italic},
  ]{Roboto Light}
  \setsansfont{Roboto Light} % seen, if not set, problem with printer
  \newfontfamily\fontrun[]{Roboto Condensed Light} % condensed runing heads
\fi\fi
\renewcommand{\LettrineFontHook}{\bfseries\color{rubric}}
% \renewenvironment{labelblock}{\begin{center}\bfseries\color{rubric}}{\end{center}}

%%%%%%%%
% MISC %
%%%%%%%%

\setdefaultlanguage[frenchpart=false]{french} % bug on part


\newenvironment{quotebar}{%
    \def\FrameCommand{{\color{rubric!10!}\vrule width 0.5em} \hspace{0.9em}}%
    \def\OuterFrameSep{0pt} % séparateur vertical
    \MakeFramed {\advance\hsize-\width \FrameRestore}
  }%
  {%
    \endMakeFramed
  }
\renewenvironment{quoteblock}% may be used for ornaments
  {%
    \savenotes
    \setstretch{0.9}
    \begin{quotebar}
    \smallskip
  }
  {%
    \smallskip
    \end{quotebar}
    \spewnotes
  }


\renewcommand{\headrulewidth}{\arrayrulewidth}
\renewcommand{\headrule}{{\color{rubric}\hrule}}
\renewcommand{\lnatt}[1]{\marginpar{\sffamily\scriptsize #1}}

% delicate tuning, image has produce line-height problems in title on 2 lines
\titleformat{name=\chapter} % command
  [display] % shape
  {\vspace{1.5em}\centering} % format
  {} % label
  {0pt} % separator between n
  {}
[{\color{rubric}\huge\textbf{#1}}\bigskip] % after code
% \titlespacing{command}{left spacing}{before spacing}{after spacing}[right]
\titlespacing*{\chapter}{0pt}{-2em}{0pt}[0pt]

\titleformat{name=\section}
  [display]{}{}{}{}
  [\vbox{\color{rubric}\large\centering\textbf{#1}}]
\titlespacing{\section}{0pt}{0pt plus 4pt minus 2pt}{\baselineskip}

\titleformat{name=\subsection}
  [block]
  {}
  {} % \thesection
  {} % separator \arrayrulewidth
  {}
[\vbox{\large\textbf{#1}}]
% \titlespacing{\subsection}{0pt}{0pt plus 4pt minus 2pt}{\baselineskip}

\ifaiv
  \fancypagestyle{main}{%
    \fancyhf{}
    \setlength{\headheight}{1.5em}
    \fancyhead{} % reset head
    \fancyfoot{} % reset foot
    \fancyhead[L]{\truncate{0.45\headwidth}{\fontrun\elbibl}} % book ref
    \fancyhead[R]{\truncate{0.45\headwidth}{ \fontrun\nouppercase\leftmark}} % Chapter title
    \fancyhead[C]{\thepage}
  }
  \fancypagestyle{plain}{% apply to chapter
    \fancyhf{}% clear all header and footer fields
    \setlength{\headheight}{1.5em}
    \fancyhead[L]{\truncate{0.9\headwidth}{\fontrun\elbibl}}
    \fancyhead[R]{\thepage}
  }
\else
  \fancypagestyle{main}{%
    \fancyhf{}
    \setlength{\headheight}{1.5em}
    \fancyhead{} % reset head
    \fancyfoot{} % reset foot
    \fancyhead[RE]{\truncate{0.9\headwidth}{\fontrun\elbibl}} % book ref
    \fancyhead[LO]{\truncate{0.9\headwidth}{\fontrun\nouppercase\leftmark}} % Chapter title, \nouppercase needed
    \fancyhead[RO,LE]{\thepage}
  }
  \fancypagestyle{plain}{% apply to chapter
    \fancyhf{}% clear all header and footer fields
    \setlength{\headheight}{1.5em}
    \fancyhead[L]{\truncate{0.9\headwidth}{\fontrun\elbibl}}
    \fancyhead[R]{\thepage}
  }
\fi

\ifav % a5 only
  \titleclass{\section}{top}
\fi

\newcommand\chapo{{%
  \vspace*{-3em}
  \centering\parindent0pt % no vskip ()
  \eltitlepage
  \bigskip
  {\color{rubric}\hline}
  \bigskip
  {\Large TEXTE LIBRE À PARTICIPATIONS LIBRES\par}
  \centerline{\small\color{rubric} {\href{https://hurlus.fr}{\dotuline{hurlus.fr}}}, tiré le \today}\par
  \bigskip
}}

\newcommand\cover{{%
  \thispagestyle{empty}
  \centering\parindent0pt
  \eltitlepage
  \vfill\null
  {\color{rubric}\setlength{\arrayrulewidth}{2pt}\hline}
  \vfill\null
  {\Large TEXTE LIBRE À PARTICIPATIONS LIBRES\par}
  \centerline{\href{https://hurlus.fr}{\dotuline{hurlus.fr}}, tiré le \today}\par
}}

\begin{document}
\pagestyle{empty}
\ifbooklet{
  \cover\newpage
  \thispagestyle{empty}\hbox{}\newpage
  \cover\newpage\noindent Les voyages de la brochure\par
  \bigskip
  \begin{tabularx}{\textwidth}{l|X|X}
    \textbf{Date} & \textbf{Lieu}& \textbf{Nom/pseudo} \\ \hline
    \rule{0pt}{25cm} &  &   \\
  \end{tabularx}
  \newpage
  \addtocounter{page}{-4}
}\fi

\thispagestyle{empty}
\ifaiv
  \twocolumn[\chapo]
\else
  \chapo
\fi
{\it\elabstract}
\bigskip
\makeatletter\@starttoc{toc}\makeatother % toc without new page
\bigskip

\pagestyle{main} % after style
\setcounter{footnote}{0}
\setcounter{footnoteA}{0}
  \noindent {\scshape \textbf{P}ersonne} n’ignore que la chaleur peut être la cause du mouvement, qu’elle possède même une grande puissance motrice : les machines à vapeur, aujourd’hui si répandues, en sont une preuve parlante à tous les yeux.\par
C’est à la chaleur que doivent être attribués les grands mouvements qui frappent nos regards sur la terre ; c’est à elle que sont dues les agitations de l’atmosphère, l’ascension des nuages, la chute des pluies et des autres météores, les courants d’eau qui sillonnent la surface du globe et dont l’homme est parvenu à employer pour son usage une faible partie ; enfin les tremblements de terre, les éruptions volcaniques, reconnaissent aussi pour cause la chaleur.\par
C’est dans cet immense réservoir que nous pouvons puiser la force mouvante nécessaire à nos besoins ; la nature, en nous offrant de toutes parts le combustible, nous a donné la faculté de faire naître en tous temps et en tous lieux la chaleur et la puissance motrice qui en est la suite. Développer cette puissance, l’approprier à notre usage, tel est l’objet des machines à feu.\par
L’étude de ces machines est du plus haut intérêt, leur importance est immense, leur emploi s’accroît tous les jours. Elles paraissent destinées à produire une grande révolution dans le monde civilisé. Déjà la machine à feu exploite nos mines, fait mouvoir nos navires, creuse nos ports et nos rivières, forge le fer, façonne les bois, écrase les grains, file et ourdit nos étoffes, transporte les plus pesants fardeaux, etc. Elle semble devoir un jour servir de moteur universel et obtenir la préférence sur la force des animaux, les chutes d’eau et les courants d’air. Elle a, sur le premier de ces moteurs, l’avantage de l’économie ; sur les deux autres, l’avantage inappréciable de pouvoir s’employer en tous temps et en tous lieux, et de ne jamais souffrir d’interruption dans son travail.\par
Si quelque jour les perfectionnements de la machine à feu s’étendent assez loin pour la rendre peu coûteuse en établissement et en combustible, elle réunira toutes les qualités désirables, et fera prendre aux arts industriels un essor dont il serait difficile de prévoir toute l’étendue.\par
Non seulement, en effet, un moteur puissant et commode, que l’on peut se procurer ou transporter partout, se substitue aux moteurs déjà en usage ; mais il fait prendre aux arts où on l’applique une extension rapide, il peut même créer des arts entièrement nouveaux.\par
Le service le plus signalé que la machine à feu ait rendu à l’Angleterre est sans contredit d’avoir ranimé l’exploitation de ses mines de houille, devenue languissante et qui menaçait de s’éteindre entièrement à cause de la difficulté toujours croissante des épuisements et de l’extraction du combustible\footnote{On peut affirmer que l’extraction de la houille a décuplé en Angleterre depuis l’invention des machines à feu. Il en est à peu prés de même de l’extraction du cuivre, de l’étain et du fer. L’effet produit il y a un demi-siècle par la machine à feu sur les mines d’Angleterre se répète aujourd’hui sur les mines d’or et d’argent du nouveau monde, mines dont l’exploitation déclinait de jour en jour, principalement à cause de l’insuffisance des moteurs employés aux épuisements et à l’extraction des minerais.}. On doit mettre sur le second rang les services rendus à la fabrication du fer, tant par la houille, offerte avec abondance et substituée aux bois au moment où ceux-ci commençaient à s’épuiser, que par les machines puissantes de toutes espèces dont l’emploi de la machine à feu a permis ou facilité l’usage.\par
Le fer et le feu sont, comme on sait, les aliments, les soutiens des arts mécaniques. Il n’existe peut-être pas en Angleterre un établissement d’industrie dont l’existence ne soit fondée sur l’usage de ces agents et qui ne les emploie avec profusion. Enlever aujourd’hui à l’Angleterre ses machines à vapeur, ce serait lui ôter à la fois la houille et le fer ; ce serait tarir toutes ses sources de richesses, ruiner tous ses moyens de prospérité ; ce serait anéantir cette puissance colossale. La destruction de sa marine, qu’elle regarde comme son plus ferme appui, lui serait peut-être moins funeste.\par
La navigation sûre et rapide des bâtiments à vapeur peut être regardée comme un art entièrement nouveau dû aux machines à feu. Déjà cet art a permis l’établissement de communications promptes et régulières sur les bras de mer, sur les grands fleuves de l’ancien et du nouveau continent. Il a permis de parcourir des régions encore sauvages, où naguère on pouvait à peine pénétrer ; il a permis de porter les fruits de la civilisation sur des points du globe où ils eussent été attendus encore bien des années. La navigation due aux machines à feu rapproche en quelque sorte les unes des autres les nations les plus lointaines. Elle tend à réunir entre eux les peuples de la terre comme s’ils habitaient tous une même contrée. Diminuer en effet le temps, les fatigues, les incertitudes et les dangers des voyages, n’est-ce pas abréger beaucoup les distances\footnote{Nous disons diminuer les dangers des voyages : en effet, quoique l’emploi de la machine à feu sur un navire offre quelques dangers que l’on s’est beaucoup exagérés, ils sont compensés et au delà par la faculté de se tenir toujours sur une route frayée et bien connue, de résister à l’effort des vents lorsqu’ils poussent le navire contre les côtes, contre les bas-fonds ou contre les écueils.} ?\par
La découverte des machines à feu a dû, comme la plupart des inventions humaines, sa naissance à des essais presque informes, essais qui ont été attribués à diverses personnes et dont on ne connaît pas bien le véritable auteur. C’est au reste moins dans ces premiers essais que consiste la principale découverte, que dans les perfectionnements successifs qui ont amené les machines à feu à l’état où nous les voyons aujourd’hui. Il y a à peu près autant de distance entre les premiers appareils où l’on a développé la force expansive de la vapeur et les machines actuelles, qu’entre le premier radeau que les hommes aient formé et le vaisseau de haut bord.\par
Si l’honneur d’une découverte appartient à la nation où elle a acquis tout son accroissement, tous ses développements, cet honneur ne peut être ici refusé à l’Angleterre : Savery, Newcomen, Smeathon, le célèbre Watt, Woolf, Trevetick et quelques autres ingénieurs anglais, sont les véritables créateurs de la machine à feu ; elle a acquis entre leurs mains tous ses degrés successifs de perfectionnement. Il est naturel, au reste, qu’une invention prenne naissance et surtout se développe, se perfectionne, là où le besoin s’en fait le plus impérieusement sentir.\par
Malgré les travaux de tous genres entrepris sur les machines à feu, malgré l’état satisfaisant où elles sont aujourd’hui parvenues, leur théorie est fort peu avancée, et les essais d’amélioration tentés sur elles sont encore dirigés presque au hasard.\par
L’on a souvent agité la question de savoir si la puissance motrice\footnote{Nous nous servons ici de l’expression puissance motrice pour désigner l’effet utile qu’un moteur est capable de produire. Cet effet peut toujours être assimilé à l’élévation d’un poids à une certaine hauteur ; il a, comme on sait, pour mesure le produit du poids multiplié par la hauteur dont il est censé élevé.} de la chaleur est limitée, ou si elle est sans bornes ; si les perfectionnements possibles des machines à feu ont un terme assignable, terme que la nature des choses empêche de dépasser par quelque moyen que ce soit, ou si au contraire ces perfectionnements sont susceptibles d’une extension indéfinie. L’on a aussi cherché long-temps, et l’on cherche encore aujourd’hui, s’il n’existerait pas des agents préférables à la vapeur d’eau pour développer la vapeur motrice du feu ; si l’air atmosphérique, par exemple, ne présenterait pas, à cet égard, de grands avantages. Nous nous proposons de soumettre ici ces questions à un examen réfléchi.\par
Le phénomène de la production du mouvement par la chaleur n’a pas été considéré sous un point de vue assez général. On l’a considéré seulement dans des machines dont la nature et le mode d’action ne lui permettaient pas de prendre toute l’étendue dont il est susceptible. Dans de pareilles machines le phénomène se trouve en quelque sorte tronqué, incomplet ; il devient difficile de reconnaître ses principes et d’étudier ses lois.\par
Pour envisager dans toute sa généralité le principe de la production du mouvement par la chaleur, il faut le concevoir indépendamment d’aucun mécanisme, d’aucun agent particulier ; il faut établir des raisonnements applicables, non seulement aux machines à vapeur\footnote{Nous distinguons ici la machine à vapeur de la machine à feu en général : celle-ci peut faire usage d’un agent quelconque, de la vapeur d’eau ou de tout autre, pour réaliser la puissance motrice de la chaleur.}, mais à toute machine à feu imaginable, quelle que soit la substance mise en œuvre et quelle que soit la manière dont on agisse sur elle.\par
Les machines qui ne reçoivent pas leur mouvement de la chaleur, celles qui ont pour moteur la force des hommes ou des animaux, une chute d’eau, un courant d’air, etc., peuvent être étudiées jusque dans leurs moindres détails par la théorie mécanique. Tous les cas sont prévus, tous les mouvements imaginables sont soumis à des principes généraux solidement établis et applicables en toute circonstance. C’est là le caractère d’une théorie complète. Une semblable théorie manque évidemment pour les machines à feu. On ne la possédera que lorsque les lois de la physique seront assez étendues, assez généralisées, pour faire connaître à l’avance tous les effets de la chaleur agissant d’une manière déterminée sur un corps quelconque.\par
Nous supposerons dans ce qui va suivre une connaissance au moins superficielle des diverses parties qui composent une machine à vapeur ordinaire. Ainsi nous jugeons inutile d’expliquer ce que c’est que foyer, chaudière, cylindre à vapeur, piston, condenseur, etc.\par
La production du mouvement dans les machines à vapeur est toujours accompagnée d’une circonstance sur laquelle nous devons fixer l’attention. Cette circonstance est le rétablissement d’équilibre dans le calorique, c’est-à-dire son passage d’un corps où la température est plus ou moins élevée à un autre où elle est plus basse. Qu’arrive-t-il en effet dans une machine à vapeur actuellement en activité ? Le calorique, développé dans le foyer par l’effet de la combustion, traverse les parois de la chaudière, vient donner naissance à de la vapeur, s’y incorpore en quelque sorte. Celle-ci, l’entraînant avec elle, la porte d’abord dans le cylindre, où elle remplit un office quelconque, et de là dans le condenseur, où elle se liquéfie par le contact de l’eau froide qui s’y rencontre. L’eau froide du condenseur s’empare donc en dernier résultat du calorique développé par la combustion. Elle s’échauffe par l’intermédiaire de la vapeur, comme si elle eût été placée directement sur le foyer. La vapeur n’est ici qu’un moyen de transporter le calorique ; elle remplit le même office que dans le chauffage des bains par la vapeur, à l’exception que dans le cas où nous sommes son mouvement est rendu utile.\par
L’on reconnaît facilement, dans les opérations que nous venons de décrire, le rétablissement d’équilibre dans le calorique, son passage d’un corps plus ou moins échauffé à un corps plus froid. Le premier de ces corps est ici l’air brûlé du foyer, le second est l’eau de condensation. Le rétablissement d’équilibre du calorique se fait entre eux, si ce n’est complètement, du moins en partie : car, d’une part, l’air brûlé, après avoir rempli son office, après avoir enveloppé la chaudière, s’échappe par la cheminée avec une température bien moindre que celle qu’il avait acquise par l’effet de la combustion ; et, d’autre part, l’eau du condenseur, après avoir liquéfié la vapeur, s’éloigne de la machine avec une température supérieure à celle qu’elle y avait apportée.\par
La production de la puissance motrice est donc due, dans les machines à vapeur, non à une consommation réelle du calorique, \emph{mais à son transport d’un corps chaud à un corps froid}, c’est-à-dire à son rétablissement d’équilibre, équilibre supposé rompu par quelque cause que ce soit, par une action chimique, telle que la combustion, ou par toute autre. Nous verrons bientôt que ce principe est applicable à toute machine mise en mouvement par la chaleur.\par
D’après ce principe, il ne suffit pas, pour donner naissance à la puissance motrice, de produire de la chaleur : il faut encore se procurer du froid ; sans lui la chaleur serait inutile. Et en effet, si l’on ne rencontrait autour de soi que des corps aussi chauds que nos foyers, comment parviendrait-on à condenser la vapeur ? où la placerait-on une fois qu’elle aurait pris naissance ? Il ne faudrait pas croire que l’on pût, ainsi que cela se pratique dans certaines machines\footnote{Certaines machines à haute pression rejettent la vapeur dans l’atmosphère, au lieu de la condenser : on les emploie particulièrement dans les lieux où il serait difficile de se procurer un courant d’eau froide suffisant pour opérer la condensation.}, la rejeter dans l’atmosphère : l’atmosphère ne la recevrait pas. Il ne la reçoit, dans l’état actuel des choses, que parce qu’il remplit pour elle l’office d’un vaste condenseur, parce qu’il se trouve à une température plus froide : autrement il en serait bientôt rempli, ou plutôt il en serait d’avance saturé\footnote{ \noindent L’existence de l’eau à l’état liquide, admise nécessairement ici, puisque sans elle les machines à vapeur ne pourraient pas s’alimenter, suppose l’existence d’une pression capable d’empêcher cette eau de se vaporiser, par conséquent d’une pression égale ou supérieure à la tension de la vapeur, eu égard à la température. Si une pareille pression n’était pas exercée par l’air atmosphérique, il s’élèverait à l’instant une quantité de vapeur d’eau suffisante pour l’exercer sur elle-même, et il faudrait toujours surmonter cette pression, pour rejeter la vapeur des machines dans la nouvelle atmosphère. Or cela équivaudrait évidemment à surmonter la tension qui reste à la vapeur après sa condensation effectuée par les moyens ordinaires.\par
 Si une température très-élevée régnait à la surface de notre globe, comme il ne paraît pas douteux qu’elle règne dans son intérieur, toutes les eaux de l’Océan existeraient en vapeur dans l’atmosphère, et il ne s’en rencontrerait aucune portion à l’état liquide.
}.\par
Partout où il existe une différence de température, partout où il peut y avoir rétablissement d’équilibre du calorique, il peut y avoir aussi production de puissance motrice. La vapeur d’eau est un moyen de réaliser cette puissance, mais elle n’est pas le seul : tous les corps de la nature peuvent être employés à cet usage ; tous sont susceptibles de changements de volume, de contractions et de dilatations successives par des alternatives de chaleur et de froid ; tous sont capables de vaincre, dans leurs changements de volume, certaines résistances et de développer ainsi la puissance motrice. Un corps solide, une barre métallique, par exemple, alternativement chauffée et refroidie, augmente et diminue de longueur, et peut mouvoir des corps fixés à ses extrémités. Un liquide alternativement chauffé et refroidi augmente et diminue de volume et peut vaincre des obstacles plus ou moins grands opposés à sa dilatation. Un fluide aériforme est susceptible de changements considérables de volume par les variations de température : s’il est renfermé dans une capacité extensible, telle qu’un cylindre muni d’un piston, il produira des mouvements d’une grande étendue. Les vapeurs de tous les corps susceptibles de passer à l’état gazeux, de l’alcool, du mercure, du soufre, etc., pourraient remplir le même office que la vapeur d’eau. Celle-ci, alternativement chauffée et refroidie, produirait de la puissance motrice à la manière des gaz permanents, c’est-à-dire sans jamais retourner à l’état liquide. La plupart de ces moyens ont été proposés, plusieurs même ont été essayés, quoique ce soit jusqu’ici sans succès remarquable.\par
Nous avons fait voir que, dans les machines à vapeur, la puissance motrice est due à un rétablissement d’équilibre dans le calorique : cela a lieu, non seulement pour les machines à vapeur, mais aussi pour toute machine à feu, c’est-à-dire pour toute machine dont le calorique est le moteur. La chaleur ne peut évidemment être une cause de mouvement qu’en vertu des changements de volume ou de forme qu’elle fait subir aux corps ; ces changements ne sont pas dus à une constance de température, mais bien à des alternatives de chaleur et de froid : or, pour échauffer une substance quelconque, il faut un corps plus chaud qu’elle ; pour la refroidir, il faut un corps plus froid. On prend nécessairement du calorique au premier de ces corps pour le transmettre au second par le moyen de la substance intermédiaire. C’est là rétablir, ou du moins travailler à rétablir, l’équilibre du calorique.\par
Il est naturel de se faire ici cette question à la fois curieuse et importante : La puissance motrice de la chaleur est-elle immuable en quantité, ou varie-t-elle avec l’agent dont on fait usage pour la réaliser avec la substance intermédiaire, choisie comme sujet d’action de la chaleur ?\par
Il est clair que cette question ne peut être faite que pour une quantité de calorique donnée\footnote{Nous jugeons inutile d’expliquer ici ce que c’est que quantité de calorique ou quantité de chaleur (car nous employons indifféremment les deux expressions), ni de décrire comment on mesure ces quantités par le calorimètre. Nous n’expliquerons pas non plus ce que c’est que chaleur latente, degré de température, chaleur spécifique, etc. : le lecteur doit être familiarisé avec ces expressions par l’étude des traités élémentaires de physique ou de chimie.}, la différence des températures étant également donnée. L’on dispose, par exemple, d’un corps A, maintenu à la température 100°, et d’un autre corps B, maintenu à la température 0°, et l’on demande quelle quantité de puissance motrice peut naître par le transport d’une portion donnée de calorique (par exemple celle qui est nécessaire pour fondre un kilogramme de glace) du premier de ces corps au second ; on demande si cette quantité de puissance motrice est nécessairement limitée, si elle varie avec la substance employée à la réaliser, si la vapeur d’eau offre à cet égard plus ou moins d’avantage que la vapeur d’alcool, de mercure, qu’un gaz permanent ou que toute autre substance.\par
Nous essaierons de résoudre ces questions en faisant usage des notions précédemment établies.\par
L’on a remarqué plus haut ce fait évident par lui-même, ou qui du moins devient sensible dès que l’on réfléchit aux changements de volume occasionnés par la chaleur : \emph{Partout où il existe une différence de température, il peut y avoir production de puissance motrice}. Réciproquement partout où l’on peut consommer de cette puissance, il est possible de faire naître une différence de température, il est possible d’occasionner une rupture d’équilibre dans le calorique. La percussion, le frottement des corps ne sont-ils pas en effet des moyens d’élever leur température, de la faire arriver spontanément à un degré plus haut que celui des corps environnants, et par conséquent de produire une rupture d’équilibre dans le calorique, là où existait auparavant cet équilibre ? C’est un fait d’expérience que la température des fluides gazeux s’élève par la compression et s’abaisse par la raréfaction. Voilà un moyen certain de changer la température des corps et de rompre l’équilibre du calorique autant de fois qu’on le voudra avec la même substance. La vapeur d’eau employée d’une manière inverse de celle où on l’emploie dans les machines à vapeur, peut aussi être regardée comme un moyen de rompre l’équilibre du calorique. Pour s’en convaincre, il suffit de réfléchir attentivement à la manière dont se développe la puissance motrice par l’action de sa chaleur sur la vapeur d’eau. Concevons deux corps A et B entretenus chacun à une température constante, celle de A étant plus élevée que celle de B : ces deux corps, auxquels on peut donner ou enlever de la chaleur sans faire varier leur température, feront les fonctions de deux réservoirs indéfinis de calorique. Nous nommerons le premier foyer et le second réfrigérant.\par
Si l’on veut donner naissance à de la puissance motrice par le transport d’une certaine quantité de chaleur du corps A au corps B, l’on pourra procéder de la manière suivante :\par

\begin{enumerate}[itemsep=0pt,topsep=0pt,partopsep=0pt,parskip=0pt]
\item Emprunter du calorique au corps A pour en former de la vapeur, c’est-à-dire faire remplir à ce corps les fonctions du foyer, ou plutôt du métal composant la chaudière, dans les machines ordinaires : nous supposerons ici que la vapeur prend naissance à la température même du corps A.
\item La vapeur ayant été reçue dans une capacité extensible, telle qu’un cylindre muni d’un piston, augmenter le volume de cette capacité et par conséquent aussi celui de la vapeur. Ainsi raréfiée, elle descendra spontanément de température, comme cela arrive pour tous les fluides élastiques : admettons que la raréfaction soit poussée jusqu’au point où la température devient précisément celle du corps B.
\item Condenser la vapeur en la mettant en contact avec le corps B, et en exerçant en même temps sur elle une pression constante, jusqu’à ce qu’elle soit entièrement liquéfiée. Le corps B remplit ici le rôle de l’eau d’injection dans les machines ordinaires, avec cette différence qu’il condense la vapeur sans se mêler avec elle et sans changer lui-même de température\footnote{ \noindent On s’étonnera peut-être ici que le corps B se trouvant à la même température que la vapeur puisse la condenser : sans doute cela n’est pas rigoureusement possible ; mais la plus petite différence de température déterminera la condensation, ce qui suffit pour établir la justesse de notre raisonnement. C’est ainsi que, dans le calcul différentiel, il suffit que l’on puisse concevoir les quantités négligées indéfiniment réductibles par rapport aux quantités conservées dans les équations, pour acquérir la certitude du résultat définitif.\par
 Le corps B condense la vapeur sans changer lui-même de température : cela résulte de notre supposition. Nous avons admis que ce corps était maintenu à une température constante. On lui enlève le calorique à mesure que la vapeur le lui fournit. C’est le cas où se trouve le métal du condenseur lorsque la liquéfaction de la vapeur s’exécute en appliquant l’eau froide extérieurement, chose pratiquée autrefois dans plusieurs machines. C’est ainsi que l’eau d’un réservoir pourrait être maintenue à un niveau constant, si le liquide s’écoulait d’un côté tandis qu’il arrive de l’autre.\par
 On pourrait même concevoir les corps A et B se maintenant d’eux-mêmes à une température constante, quoique pouvant perdre ou acquérir certaines quantités de chaleur. Si, par exemple, le corps A était une masse de vapeur prête à se liquéfier, et le corps B une masse de glace prête à se fondre, ces corps pourraient, comme on sait, fournir ou recevoir du calorique sans changer de degré thermométrique.
}.
\end{enumerate}

\noindent Les opérations que nous venons de décrire eussent pu être faites dans un sens et dans un ordre inverse. Rien n’empêchait de former de la vapeur avec le calorique du corps B, et à la température de ce corps, de la comprimer de manière à lui faire acquérir la température du corps A, enfin de la condenser par son contact avec ce dernier corps, et cela en continuant la compression jusqu’à une liquéfaction complète.\par
Par nos premières opérations, il y avait eu à la fois production de puissance motrice et transport du calorique du corps A au corps B ; par les opérations inverses, il y a à la fois dépense de puissance motrice et retour du calorique du corps B au corps A. Mais si l’on a agi de part et d’autre sur la même quantité de vapeur, s’il ne s’est fait aucune perte ni de puissance motrice ni de calorique, la quantité de puissance motrice produite dans le premier cas sera égale à celle qui aura été dépensée dans le second, et la quantité de calorique passée, dans le premier cas, du corps A au corps B sera égale à la quantité qui repasse, dans le second, du corps B au corps A, de sorte qu’on pourrait faire un nombre indéfini d’opérations alternatives de ce genre sans qu’il y eût en somme ni puissance motrice produite, ni calorique passé d’un corps à l’autre.\par
Or, s’il existait des moyens d’employer la chaleur préférables à ceux dont nous avons fait usage, c’est-à-dire s’il était possible, par quelque méthode que ce fût, de faire produire au calorique une quantité de puissance motrice plus grande que nous ne l’avons fait par notre première série d’opérations, il suffirait de distraire une portion de cette puissance pour faire remonter, par la méthode qui vient d’être indiquée, le calorique du corps B au corps A, du réfrigérant au foyer, pour rétablir les choses dans leur état primitif et se mettre par-là en mesure de recommencer une opération entièrement semblable à la première et ainsi de suite : ce serait là, non seulement le mouvement perpétuel, mais une création indéfinie de force motrice sans consommation ni de calorique ni de quelque autre agent que ce soit. Une semblable création est tout-à-fait contraire aux idées reçues jusqu’à présent, aux lois de la mécanique et de la saine physique ; elle est inadmissible\footnote{ \noindent On objectera peut-être ici que le mouvement perpétuel, démontré impossible par les seules actions mécaniques, ne l’est peut-être pas lorsqu’on emploie l’influence soit de la chaleur, soit de l’électricité ; mais peut-on concevoir les phénomènes de la chaleur et de l’électricité comme dus à autre chose qu’à des mouvements quelconques de corps, et comme tels ne doivent-ils pas être soumis aux lois générales de la mécanique ? Ne sait-on pas d’ailleurs \emph{a posteriori} que toutes les tentatives faites pour produire le mouvement perpétuel par quelque moyen que ce soit ont été infructueuses ? Que l’on n’est jamais parvenu à produire un mouvement véritablement perpétuel, c’est-à-dire un mouvement qui se continuât toujours sans altération dans les corps mis en œuvre pour le réaliser ?\par
 L’on a regardé quelquefois l’appareil électromoteur (la pile de Volta) comme capable de produire le mouvement perpétuel ; on a cherché à réaliser cette idée en construisant des piles sèches, prétendues inaltérables. Mais, quoi que l’on ait pu faire, l’appareil a toujours éprouvé des détériorations sensibles, lorsque son action a été soutenue pendant un certain temps avec quelque énergie.\par
 L’acception générale et philosophique des mots \emph{mouvement perpétuel} doit comprendre, non pas seulement un mouvement susceptible de se prolonger indéfiniment après une première impulsion reçue, mais l’action d’un appareil, d’un assemblage quelconque, capable de créer la puissance motrice en quantité illimitée, capable de tirer successivement du repos tous les corps de la nature, s’ils s’y trouvaient plongés, de détruire en eux le principe de l’inertie, capable enfin de puiser en lui-même les forces nécessaires pour mouvoir l’univers tout entier, pour prolonger, pour accélérer incessamment son mouvement. Telle serait une véritable création de puissance motrice. Si elle était possible, il serait inutile de chercher dans les courans d’eau et d’air, dans les combustibles, cette puissance motrice ; nous en aurions à notre disposition une source intarissable où nous pourrions puiser à volonté.
}. On doit donc conclure que \emph{le maximum de puissance motrice résultant de l’emploi de la vapeur est aussi le maximum de puissance motrice réalisable par quelque moyen que ce soit}. Nous donnerons, au reste, bientôt une seconde démonstration plus rigoureuse de ce théorème. Celle-ci ne doit être considérée que comme un aperçu. (V. pag. 29).\par
On est en droit de nous faire, au sujet de la proposition qui vient d’être énoncée, la question suivante : Quel est ici le sens du mot \emph{maximum} ? à quel signe reconnaîtra-t-on que ce maximum est atteint ? à quel signe reconnaîtra-t-on si la vapeur est employée le plus avantageusement possible à la production de la puissance motrice ?\par
Puisque tout rétablissement d’équilibre dans le calorique peut être la cause de la production de la puissance motrice, tout rétablissement d’équilibre qui se fera sans production de cette puissance devra être considéré comme une véritable perte : or, pour peu qu’on y réfléchisse, on s’apercevra que tout changement de température qui n’est pas dû à un changement de volume des corps ne peut être qu’un rétablissement inutile d’équilibre dans le calorique\footnote{Nous ne supposons ici aucune action chimique entre les corps mis en usage pour réaliser la puissance motrice de la chaleur. L’action chimique qui se passe dans le foyer est une action en quelque sorte préliminaire, une opération destinée, non à produire immédiatement de la puissance motrice, mais à rompre l’équilibre du calorique, à produire une différence de température qui doit ensuite donner naissance au mouvement.}. La condition nécessaire du maximum est donc \emph{qu’il ne se fasse dans les corps employés à réaliser la puissance motrice de la chaleur aucun changement de température qui ne soit dû à un changement de volume}. Réciproquement, toutes les fois que cette condition sera remplie, le maximum sera atteint.\par
Ce principe ne doit jamais être perdu de vue dans la construction des machines à feu ; il en est la base fondamentale. Si l’on ne peut pas l’observer rigoureusement, il faut du moins s’en écarter le moins possible.\par
Tout changement de température qui n’est pas dû à un changement de volume ou à une action chimique (action que provisoirement nous supposons ne pas se rencontrer ici) est nécessairement dû au passage direct du calorique d’un corps plus ou moins échauffé à un corps plus froid. Ce passage a lieu principalement au contact de corps de températures diverses : aussi un pareil contact doit-il être évité autant que possible. Il ne peut pas être évité complètement, sans doute ; mais il faut du moins faire en sorte que les corps mis en contact les uns avec les autres diffèrent peu entre eux de température.\par
Lorsque nous avons supposé tout à l’heure, dans notre démonstration, le calorique du corps A employé à former de la vapeur, cette vapeur était censée prendre naissance à la température même du corps A : ainsi le contact n’avait lieu qu’entre des corps de températures égales ; le changement de température arrivé ensuite dans la vapeur était dû à la dilatation, par conséquent à un changement de volume ; enfin la condensation s’opérait aussi sans contact de corps de températures diverses. Elle s’opérait en exerçant une pression constante sur la vapeur mise en contact avec le corps B de même température qu’elle. Les conditions du maximum se trouvaient donc remplies. À la vérité les choses ne peuvent pas se passer rigoureusement comme nous l’avons supposé. Pour déterminer le passage du calorique d’un corps à l’autre, il faut dans le premier un excès de température ; mais cet excès peut être supposé aussi petit qu’on le voudra ; on peut le regarder comme nul en théorie, sans que pour cela les raisonnements perdent rien de leur exactitude.\par
L’on peut faire à notre démonstration une objection plus réelle, que voici :\par
Lorsque l’on emprunte du calorique au corps A, pour donner naissance à de la vapeur, et que cette vapeur est ensuite condensée par son contact avec le corps B, l’eau employée à la former et que l’on supposait d’abord à la température du corps A se trouve, à la fin de l’opération, à la température du corps B ; elle s’est refroidie. Si l’on veut recommencer une opération semblable à la première, si l’on veut développer une nouvelle quantité de puissance motrice avec le même instrument, avec la même vapeur, il faut d’abord rétablir les choses dans leur état primitif, il faut rendre à l’eau le degré de température qu’elle avait d’abord. Cela peut se faire sans doute en la remettant immédiatement en contact avec le corps A ; mais il y a alors contact entre des corps de températures diverses et perte de puissance motrice\footnote{Ce genre de perte se rencontre dans toutes les machines à vapeur : en effet, l’eau destinée à alimenter la chaudière est toujours plus froide que l’eau qui y est déjà contenue ; il se fait entre elles un rétablissement inutile d’équilibre dans le calorique. On se convaincra aisément \emph{à posteriori} que ce rétablissement d’équilibre entraîne une perte de puissance motrice, si l’on réfléchit qu’il eût été possible d’échauffer préalablement l’eau d’alimentation en l’employant comme eau de condensation dans une petite machine accessoire, où l’on eût fait usage de la vapeur tirée de la grande chaudière et où la condensation se fût opérée à une température intermédiaire entre celle de la chaudière et celle du condenseur principal. La force produite par la petite machine n’eût coûté aucune dépense de chaleur, puisque toute celle qui eût été employée serait rentrée dans la chaudière avec l’eau de condensation.} : il deviendrait impossible d’exécuter l’opération inverse, c’est-à-dire de faire retourner au corps A le calorique employé à élever la température du liquide.\par
Cette difficulté peut être levée en supposant la différence de température entre le corps A et le corps B infiniment petite ; la quantité de chaleur nécessaire pour reporter le liquide à sa température première sera aussi infiniment petite et négligeable relativement à celle qui est nécessaire pour donner naissance à la vapeur, quantité toujours finie.\par
La proposition, se trouvant d’ailleurs démontrée pour le cas où la différence entre les températures des deux corps est infiniment petite, sera facilement étendue au cas général. En effet, s’il s’agissait de faire naître la puissance motrice par le transport du calorique du corps A au corps Z, la température de ce dernier corps étant fort différente de celle du premier, on imaginerait une suite de corps B, C, D, etc., de températures intermédiaires entre celles des corps A, Z, et choisies de manière à ce que les différences de A à B, de B à C, etc., soient toutes infiniment petites. Le calorique émané de A n’arriverait à Z qu’après avoir passé par les corps B, C, D, etc., et après avoir développé dans chacun de ses transports le maximum de puissance motrice. Les opérations inverses seraient ici toutes possibles, et le raisonnement de la pag. 20 deviendrait rigoureusement applicable.\par
D’après les notions établies jusqu’à présent, on peut comparer avec assez de justesse la puissance motrice de la chaleur à celle d’une chute d’eau : toutes deux ont un maximum que l’on ne peut pas dépasser, quelle que soit d’une part la machine employée à recevoir l’action de l’eau, et quelle que soit de l’autre la substance employée à recevoir l’action de la chaleur. La puissance motrice d’une chute d’eau dépend de sa hauteur et de la quantité du liquide ; la puissance motrice de la chaleur dépend aussi de la quantité de calorique employé, et de ce qu’on pourrait nommer, de ce que nous appellerons en effet la hauteur de sa chute\footnote{La matière ici traitée étant tout-à-fait nouvelle, nous sommes forcés d’employer des expressions encore inusitées et qui n’ont peut-être pas toute la clarté désirable.}, c’est-à-dire de la différence de température des corps entre lesquels se fait l’échange du calorique. Dans la chute d’eau, la puissance motrice est rigoureusement proportionnelle à la différence de niveau entre le réservoir supérieur et le réservoir inférieur. Dans la chute du calorique, la puissance motrice augmente sans doute avec la différence de température entre le corps chaud et le corps froid ; mais nous ignorons si elle est proportionnelle à cette différence. Nous ignorons, par exemple, si la chute du calorique de 100° à 50° fournit plus ou moins de puissance motrice que la chute de ce même calorique de 50° à 0°. C’est une question que nous nous proposons d’examiner plus tard.\par
Nous allons donner ici une seconde démonstration de la proposition fondamentale énoncée pag. 22, et présenter cette proposition sous une forme plus générale que nous ne l’avons fait ci-dessus.\par
Lorsqu’un fluide gazeux est rapidement comprimé, sa température s’élève ; elle s’abaisse au contraire lorsqu’il est rapidement dilaté. C’est là un des faits les mieux constatés par l’expérience : nous le prendrons pour base de notre démonstration\footnote{ \noindent Les faits d’expérience qui prouvent le mieux le changement de température des gaz par la compression ou la dilatation sont les suivants :\par
 
\begin{enumerate}[itemsep=0pt,topsep=0pt,partopsep=0pt,parskip=0pt]
\item L’abaissement du thermomètre placé sous le récipient d’une machine pneumatique où l’on fait le vide. Cet abaissement est très-sensible sur le thermomètre de Breguet : il peut aller au-delà de 40 à 50 degrés. Le nuage qui se forme dans cette occasion semble devoir être atribué à la condensation de la vapeur d’eau causée par le refroidissement de l’air.
\item L’inflammation de l’amadou dans les briquets dit pneumatiques, qui sont, comme on sait, de petits corps de pompe où l’on fait éprouver à l’air une compression rapide ;
\item L’abaissement du thermomètre placé dans une capacité où, après avoir comprimé de l’air, on le laisse échapper par l’ouverture d’un robinet ;
\item Les résultats d’expérience sur la vitesse du son. M. de Laplace a fait voir que, pour soumettre exactement ces résultats à la théorie et au calcul, il fallait admettre l’échauffement de l’air par une compression subite.
\end{enumerate}

 \noindent Le seul fait qui puisse être opposé à ceux-ci est une expérience de MM. Gay-Lussac et Welter, décrite dans les \emph{Annales de physique et de chimie}. Une petite ouverture ayant été faite à un vaste réservoir d’air comprimé, et la boule d’un thermomètre ayant été présentée au courant d’air qui sortait par cette ouverture, l’on n’a pas observé d’abaissement sensible dans le degré de température marqué par le thermomètre.\par
 L’on peut donner à ce fait deux explications : 1° le frottement de l’air contre les parois de l’ouverture par laquelle il s’échappe développe peut-être de la chaleur en quantité notable ; 2° l’air qui vient toucher immédiatement la boule du thermomètre reprend peut-être, par son choc contre cette boule, ou plutôt par l’effet du détour qu’il est forcé de prendre à sa rencontre, une densité égale à celle qu’il avait dans le récipient, à peu près comme l’eau d’un courant s’élève, contre un obstacle fixe, au-dessus de son niveau.\par
 Le changement de température occasioné dans les gaz par le changement de volume peut être regardé comme l’un des faits les plus importans de la physique, à cause des nombreuses conséquences qu’il entraîne, et en même temps comme l’un des plus difficiles à éclaircir et à mesurer par des expériences décisives. Il semble présenter dans plusieurs circonstances des anomalies singulières.\par
 N’est-ce pas au refroidissement de l’air par la dilatation qu’il faut attribuer le froid des régions supérieures de l’atmosphère ? Les raisons données jusqu’ici pour expliquer ce froid sont tout-à-fait insuffisantes : on a dit que l’air des régions élevées, recevant peu de chaleur réfléchie par la terre, et rayonnant lui-même vers les espaces célestes, devait perdre du calorique, et que c’était là la cause de son refroidissement ; mais cette explication se trouve détruite si l’on remarque qu’à égale hauteur le froid règne aussi bien et même avec plus d’intensité sur les plaines élevées que sur le sommet des montagnes, ou que dans les parties d’atmosphère éloignées du sol.
}.\par
Si, lorsqu’un gaz s’est élevé de température par l’effet de la compression, l’on veut le ramener à sa température primitive sans faire subir à son volume de nouveaux changements, il faut lui enlever du calorique. Ce calorique pourrait aussi être enlevé à mesure que la compression s’exécute, de manière à ce que la température du gaz restât constante. De même, si le gaz est raréfié, l’on peut éviter qu’il ne baisse de température en lui fournissant une certaine quantité de calorique. Nous appellerons le calorique employé dans ces occasions où il ne se fait aucun changement de température, calorique dû au changement de volume. Cette dénomination n’indique pas que le calorique appartienne au volume, il ne lui appartient pas plus qu’il n’appartient à la pression, et pourrait être tout aussi bien appelé calorique dû au changement de pression. Nous ignorons quelles lois il suit relativement aux variations de volume : il est possible que sa quantité change soit avec la nature du gaz, soit avec sa densité, soit avec sa température. L’expérience ne nous a rien appris sur ce sujet ; elle nous a appris seulement que ce calorique se développe en quantité plus ou moins grande par la compression des fluides élastiques.\par
Cette notion préliminaire étant posée, imaginons un fluide élastique, de l’air atmosphérique par exemple, renfermé dans un vaisseau cylindrique \emph{abcd}, fig. 1, muni d’un diaphragme mobile ou piston \emph{cd} ; soient en outre les deux corps A, B, entretenus chacun à une température constante, celle de A étant plus élevée que celle de B ; figurons-nous maintenant la suite des opérations qui vont être décrites :\par

\begin{enumerate}[itemsep=0pt,topsep=0pt,partopsep=0pt,parskip=0pt]
\item Contact du corps A avec l’air renfermé dans la capacité \emph{abcd}, ou avec la paroi de cette capacité, paroi que nous supposerons transmettre facilement le calorique. L’air se trouve par ce contact à la température même du corps A ; \emph{cd} est la position actuelle du piston.
\item Le piston s’élève graduellement, et vient prendre la position \emph{ef}. Le contact a toujours lieu entre le corps A et l’air, qui se trouve ainsi maintenu à une température constante pendant la raréfaction. Le corps A fournit le calorique nécessaire pour maintenir la constance de température.
\item Le corps A est éloigné, et l’air ne se trouve plus en contact avec aucun corps capable de lui fournir du calorique ; le piston continue cependant à se mouvoir, et passe de la position \emph{ef} à la position \emph{gh}. L’air se raréfie sans recevoir de calorique, et sa température s’abaisse. Imaginons qu’elle s’abaisse ainsi jusqu’à devenir égale à celle du corps B : à ce moment le piston s’arrête et occupe la position \emph{gh}.
\item L’air est mis en contact avec le corps B ; il est comprimé par le retour du piston, que l’on ramène de la position \emph{gh} à la position \emph{cd}. Cet air reste cependant à une température constante, à cause de son contact avec le corps B auquel il cède son calorique.
\item Le corps B est écarté, et l’on continue la compression de l’air, qui, se trouvant alors isolé, s’élève de température. La compression est continuée jusqu’à ce que l’air ait acquis la température du corps A. Le piston passe pendant ce temps de la position \emph{cd} à la position \emph{ik}.
\item L’air est remis en contact avec le corps A ; le piston retourne de la position \emph{ik} à la position \emph{ef} ; la température demeure invariable.
\item La période décrite sous le no 3 se renouvelle, puis successivement les périodes 4, 5, 6, 3, 4, 5, 6, 3, 4, 5, ainsi de suite.
\end{enumerate}

\noindent Dans ces diverses opérations, le piston éprouve un effort plus ou moins grand de la part de l’air renfermé dans le cylindre ; la force élastique de cet air varie, tant à cause des changements de volume que des changements de température ; mais l’on doit remarquer qu’à volume égal, c’est-à-dire pour des positions semblables du piston, la température se trouve plus élevée pendant les mouvements de dilatation que pendant les mouvements de compression. Pendant les premiers, la force élastique de l’air se trouve donc plus grande et par conséquent la quantité de puissance motrice produite par les mouvements de dilatation est plus considérable que celle qui est consommée pour produire les mouvements de compression. Ainsi, l’on obtiendra un excédant de puissance motrice, excédant dont on pourra disposer pour des usages quelconques. L’air nous a donc servi de machine à feu ; nous l’avons même employé de la manière la plus avantageuse possible, car il ne s’est fait aucun rétablissement inutile d’équilibre dans le calorique.\par
Toutes les opérations ci-dessus décrites peuvent être exécutées dans un sens et dans un ordre inverses. Imaginons qu’après la sixième période, c’est-à-dire le piston étant arrivé à la position \emph{ef}, on le fasse revenir à la position \emph{ik}, et qu’en même temps on maintienne l’air en contact avec le corps A : le calorique fourni par ce corps, pendant la sixième période, retournera à sa source, c’est-à-dire au corps A, et les choses se trouveront dans l’état où elles étaient à la fin de la période cinquième. Si maintenant on écarte le corps A, et que l’on fasse mouvoir le piston de \emph{ef} en \emph{cd}, la température de l’air décroîtra d’autant de degrés qu’elle s’est accrue pendant la période cinquième, et deviendra celle du corps B. L’on peut évidemment continuer une suite d’opérations inverses de celles que nous avons d’abord décrites : il suffit de se placer dans les mêmes circonstances et d’exécuter pour chaque période un mouvement de dilatation au lieu d’un mouvement de compression, et réciproquement.\par
Le résultat des premières opérations avait été la production d’une certaine quantité de puissance motrice et le transport du calorique du corps A au corps B ; le résultat des opérations inverses est la consommation de la puissance motrice produite, et le retour du calorique du corps B au corps A : de sorte que ces deux suites d’opérations s’annulent, se neutralisent en quelque sorte l’une l’autre.\par
L’impossibilité de faire produire au calorique une quantité de puissance motrice plus grande que celle que nous en avons obtenue par notre première suite d’opérations est maintenant facile à prouver. Elle se démontrera par un raisonnement entièrement semblable à celui dont nous avons fait usage pag. 20. Le raisonnement aura même ici un degré d’exactitude de plus : l’air dont nous nous servons pour développer la puissance motrice est ramené, à la fin de chaque cercle d’opérations, précisément à l’état où il se trouvait d’abord, tandis qu’il n’en était pas tout-à-fait de même pour la vapeur d’eau, ainsi que nous l’avons remarqué\footnote{Nous supposons implicitement dans notre démonstration que, lorsqu’un corps a éprouvé des changements quelconques et qu’après un certain nombre de transformations il est ramené identiquement à son état primitif, c’est-à-dire à cet état considéré relativement à la densité, à la température, au mode d’agrégation, nous supposerons, dis-je, que ce corps se trouve contenir la même quantité de chaleur qu’il contenait d’abord, ou autrement, que les quantités de chaleur absorbées ou dégagées dans ses diverses transformations sont exactement compensées. Ce fait n’a jamais été révoqué en doute ; il a été d’abord admis sans réflexion et vérifié ensuite dans beaucoup de cas par les expériences du calorimètre. Le nier, ce serait renverser toute la théorie de la chaleur, à laquelle il sert de base. Au reste, pour le dire en passant, les principaux fondements sur lesquels repose la théorie de la chaleur auraient besoin de l’examen le plus attentif. Plusieurs faits d’expérience paraissent à peu près inexplicables dans l’état actuel de cette théorie.}.\par
Nous avons choisi l’air atmosphérique comme l’instrument qui devait développer la puissance motrice de la chaleur ; mais il est évident que les raisonnements eussent été les mêmes pour toute autre substance gazeuse, et même pour tout autre corps susceptible de changer de température par des contractions et des dilatations successives, ce qui comprend tous les corps de la nature, ou du moins tous ceux qui sont propres à réaliser la puissance motrice de la chaleur. Ainsi nous sommes conduits à établir la proposition générale que voici :\par
\emph{La puissance motrice de la chaleur est indépendante des agents mis en œuvre pour la réaliser ; sa quantité est fixée uniquement par les températures des corps entre lesquels se fait en dernier résultat le transport du calorique}.\par
Il faut sous-entendre ici que chacune des méthodes de développer la puissance motrice atteint la perfection dont elle est susceptible. Cette condition se trouvera remplie si, comme nous l’avons remarqué plus haut, il ne se fait dans les corps aucun changement de température qui ne soit dû à un changement de volume, ou, ce qui est la même chose autrement exprimée, s’il n’y a jamais de contact entre des corps de températures sensiblement différentes.\par
Les diverses méthodes de réaliser la puissance motrice peuvent être prises d’ailleurs, soit dans l’emploi de substances diverses, soit dans l’emploi de la même substance à deux états différents, par exemple, d’un gaz à deux densités différentes.\par
Ceci nous conduit naturellement à des recherches intéressantes sur les fluides aériformes ; recherches qui nous mèneront d’ailleurs à de nouveaux résultats sur la puissance motrice de la chaleur, et nous donneront les moyens de vérifier dans quelques cas particuliers la proposition fondamentale ci-dessus énoncée\footnote{Nous supposerons, dans ce qui va suivre, le lecteur au courant des derniers progrès de la physique moderne, en ce qui concerne les substances gazeuses et la chaleur.}.\par
L’on remarquera facilement que notre démonstration eût été simplifiée en supposant les températures des corps A et B fort peu différentes entre elles. Alors les mouvements du piston se trouvant fort peu étendus pendant les périodes 3 et 5, ces périodes eussent pu être supprimées sans influence sensible sur la production de la puissance motrice. Un fort petit changement de volume doit suffire en effet pour produire un fort petit changement de température, et ce petit changement de volume est négligeable à côté de celui des périodes 4 et 6, dont l’étendue est illimitée.\par
Si l’on supprime les périodes 3 et 5 dans la série d’opérations ci-dessus décrite, elle se réduit aux suivantes :\par

\begin{enumerate}[itemsep=0pt,topsep=0pt,partopsep=0pt,parskip=0pt]
\item Contact du gaz renfermé en \emph{abcd} (fig. 2) avec le corps A, passage du piston de \emph{cd} en \emph{ef} ;
\item Éloignement du corps A, contact du gaz renfermé en \emph{abef} avec le corps B, retour du piston de \emph{ef} en \emph{cd} ;
\item Éloignement du corps B, contact du gaz avec le corps A, passage du piston de \emph{cd} en \emph{ef}, c’est-à-dire renouvellement de la période première, ainsi de suite.
\end{enumerate}

\noindent La puissance motrice résultante de l’ensemble des opérations 1, 2, sera évidemment la différence entre celle qui est produite par l’expansion du gaz tandis qu’il se trouve à la température du corps A, et celle qui est consommée pour comprimer ce gaz tandis qu’il se trouve à la température du corps B.\par
Supposons que les opérations 1, 2, soient exécutées sur deux gaz de natures chimiques différentes, mais pris sous la même pression, sous la pression atmosphérique, par exemple : ces deux gaz se comporteront absolument l’un comme l’autre dans les mêmes circonstances, c’est-à-dire que leurs forces expansives, primitivement égales entre elles, demeureront toujours égales, quelles que soient les variations de volume et de température, pourvu que ces variations soient les mêmes de part et d’autre. Cela résulte évidemment des lois de Mariotte et de MM. Gay-Lussac et Dalton, lois communes à tous les fluides élastiques et en vertu desquelles les mêmes rapports existent pour tous ces fluides entre le volume, la force expansive et la température.\par
Puisque deux gaz différents, pris à la même température et sous la même pression, doivent se comporter l’un comme l’autre dans les mêmes circonstances, si on leur fait subir à tous deux les opérations ci-dessus décrites, ils devront donner naissance à des quantités de puissance motrice égales. Or cela suppose, d’après la proposition fondamentale que nous avons établie, l’emploi des deux quantités égales de calorique, c’est-à-dire cela suppose que la quantité de calorique passée du corps A au corps B est la même, soit que l’on opère sur l’un des gaz, soit que l’on opère sur l’autre.\par
La quantité de calorique passée du corps A au corps B est évidemment celle qui est absorbée par le gaz dans son extension de volume, ou celle que ce gaz abandonne ensuite par la compression. Nous sommes donc conduits à établir la proposition suivante :\par
\emph{Lorsqu’un gaz passe, sans changer de température, d’un volume et d’une pression déterminés à un autre volume et à une autre pression également déterminés, la quantité de calorique absorbée ou abandonnée est toujours la même, quelle que soit la nature du gaz choisi comme sujet d’expérience}.\par
Soit, par exemple, un litre d’air à la température 100° et sous la pression une atmosphère : si l’on double le volume de cet air et qu’on veuille le maintenir à la température 100°, il faut lui fournir une certaine quantité de chaleur. Or cette quantité sera précisément la même si, au lieu d’opérer sur l’air, on opère sur le gaz acide carbonique, sur l’azote, sur l’hydrogène, sur la vapeur d’eau, d’alcool, c’est-à-dire si l’on double le volume d’un litre de ces gaz pris à la température 100° et sous la pression atmosphérique.\par
Il en serait de même, dans un sens inverse, si, au lieu de doubler le volume des gaz, on le réduisait à moitié par la compression.\par
La quantité de chaleur que les fluides élastiques dégagent ou absorbent dans leurs changements de volume n’a jamais été mesurée par aucune expérience directe, expérience qui offrirait sans doute de grandes difficultés ; mais il existe une donnée qui est à peu près l’équivalent pour nous : cette donnée a été fournie par la théorie du son ; elle mérite beaucoup de confiance à cause de la rigueur des considérations par lesquelles on est parvenu à l’établir. Voici en quoi elle consiste :\par
L’air atmosphérique doit s’élever de 1° centigrade, lorsqu’il éprouve par la compression subite une réduction de volume de 1/116\footnote{M. Poisson, à qui cette donnée est due, a fait voir qu’elle s’accorde assez bien avec le résultat d’une expérience de MM. Clément et Désormes sur la rentrée de l’air dans le vide, ou plutôt dans l’air un peu raréfié. Elle s’accorde aussi à quelque chose près avec certain résultat trouvé par MM. Gay-Lussac et Welter. (Voyez la note pag. 59.)}.\par
Les expériences sur la vitesse du son ayant été faites dans l’air sous la pression 760 millimètres de mercure et à la température 6°, c’est seulement à ces deux circonstances que doit se rapporter notre donnée. Nous la rapporterons cependant pour plus de facilité à la température 0°, qui en diffère peu.\par
L’air comprimé de 1/116, et élevé par-là de 1°, ne diffère de l’air échauffé directement de t° que par sa densité. Le volume primitif étant supposé V, la compression de 1/116 le réduit à V – 1/116 V.\par
L’échauffement direct sous pression constante doit, d’après la règle de M. Gay-Lussac, augmenter le volume de l’air de 1/267 de ce qu’il serait à 0° : ainsi l’air est d’une part réduit au volume V – 1/116  V ; de l’autre il est porté à V + 1/267 V.\par
La différence entre les quantités de chaleur que possède l’air dans l’un et l’autre cas est évidemment la quantité employée à l’élever directement de 1° : ainsi donc la quantité de chaleur que l’air absorberait en passant du volume V – 1/166 V au volume V + 1/267 V est égale à celle qui est nécessaire pour l’élever de 1°.\par
Imaginons maintenant qu’au lieu d’échauffer de 1° l’air soumis à une pression constante et pouvant se dilater librement, on le renferme dans une capacité inextensible, et qu’en cet état, on lui fasse acquérir 1° de température. L’air ainsi échauffé de 1° ne différera de l’air comprimé de 1/116 que par son volume plus grand de 1/116. Ainsi donc la quantité de chaleur que l’air abandonnerait par une réduction de volume de 1/116 est égale à celle qu’il exigerait pour s’élever de 1° centigrade sous volume constant. Comme les différences entre les volumes V – 1/116 V, V, et V + 1/267 V, sont petites relativement aux volumes eux-mêmes, on peut regarder les quantités de chaleur absorbées par l’air, en passant du premier de ces volumes au second, et du premier au troisième, comme sensiblement proportionnelles aux changements de volume : l’on se trouve donc conduit à établir la relation suivante :\par
La quantité de chaleur nécessaire pour élever de 1° l’air sous pression constante est à la quantité de chaleur nécessaire pour élever de 1° le même air sous volume constant, dans le rapport des nombres\par

\begin{center}
1/116 + 1/267 à 1/116\par
\end{center}

\noindent ou bien en multipliant de part et d’autre par 116.267, dans le rapport des nombres 267 + 116 à 267.\par
C’est donc là le rapport qui existe entre la capacité de l’air pour la chaleur sous pression constante, et sa capacité sous volume constant. Si la première de ces deux capacités est exprimée par l’unité, l’autre sera exprimée par le chiffre 267/(267+116) ou à peu près 0,700 ; leur différence 1 – 0,700 ou 0,300 exprimera évidemment la quantité de chaleur destinée à produire l’augmentation de volume de l’air lorsqu’il est échauffé de 1° sous pression constante.\par
D’après la loi de MM. Gay-Lussac et Dalton, cette augmentation de volume serait la même pour tous les autres gaz ; d’après le théorème démontré pag. 41, la chaleur absorbée par des augmentations égales de volume est la même pour tous les fluides élastiques : nous sommes donc conduits à établir la proposition suivante :\par
\emph{La différence entre la chaleur spécifique sous pression constante et la chaleur spécifique sous volume constant est la même pour tous les gaz}.\par
Il faut remarquer ici que tous les gaz sont supposés pris sous la même pression, la pression atmosphérique, par exemple, et qu’en outre les chaleurs spécifiques sont mesurées par rapport aux volumes.\par
Rien ne nous est plus aisé maintenant que de dresser une table des chaleurs spécifiques des gaz sous volume constant, d’après la connaissance de leurs chaleurs spécifiques sous pression constante. Nous présentons ici cette table, dont la première colonne est le résultat des expériences directes de MM. Delaroche et Bérard, sur la chaleur spécifique des gaz soumis à la pression atmosphérique, et dont la seconde colonne est composée des nombres de la première diminués de 0,300.\par

\begin{center}
\noindent \emph{Table de la chaleur spécifique des gaz}.\par
\end{center}


\tableopen{}
\begin{tabularx}{\linewidth}
{
  | X
  | X
  | X |
}
\toprule
\raggedright\arraybackslash {\footnotesize NOMS DES GAZ.}
   & \raggedleft\arraybackslash {\footnotesize Chal. spéc.  sous pression constante.}
   & \raggedleft\arraybackslash {\footnotesize Chal. spéc. sous volume constant.} \\
\midrule
\raggedright\arraybackslash {\footnotesize Air atmosphérique}
   & \raggedleft\arraybackslash 1,000
   & \raggedleft\arraybackslash 0,700 \\
\hline
\raggedright\arraybackslash {\footnotesize Gaz hydrogène}
   & \raggedleft\arraybackslash 0,903
   & \raggedleft\arraybackslash 0,603 \\
\hline
\raggedright\arraybackslash {\footnotesize Acide carbonique}
   & \raggedleft\arraybackslash 1,258
   & \raggedleft\arraybackslash 0,958 \\
\hline
\raggedright\arraybackslash {\footnotesize Oxygène}
   & \raggedleft\arraybackslash 0,976
   & \raggedleft\arraybackslash 0,676 \\
\hline
\raggedright\arraybackslash {\footnotesize Azote}
   & \raggedleft\arraybackslash 1,000
   & \raggedleft\arraybackslash 0,700 \\
\hline
\raggedright\arraybackslash {\footnotesize Protoxyde d’azote}
   & \raggedleft\arraybackslash 1,350
   & \raggedleft\arraybackslash 1,050 \\
\hline
\raggedright\arraybackslash {\footnotesize Gaz oléfiant}
   & \raggedleft\arraybackslash 1,553
   & \raggedleft\arraybackslash 1,253 \\
\hline
\raggedright\arraybackslash {\footnotesize Oxyde de carbone}
   & \raggedleft\arraybackslash 1,034
   & \raggedleft\arraybackslash 0,734 \\
\bottomrule
\end{tabularx}
\tableclose{}

\noindent Les nombres de la première colonne et ceux de la seconde sont ici rapportés à la même unité, à la chaleur spécifique de l’air atmosphérique sous pression constante.\par
La différence entre chaque nombre de la première colonne et le nombre correspondant de la seconde étant constante, le rapport entre ces nombres doit être variable : ainsi, le rapport entre la chaleur spécifique des gaz sous pression constante et la chaleur spécifique sous volume constant varie lorsqu’on passe d’un gaz à un autre.\par
Nous avons vu que l’air, lorsqu’il éprouve une compression subite de 1/116 de son volume, s’élève de 1°. Les autres gaz par une compression semblable doivent s’élever aussi de température : ils doivent s’élever, mais non pas également en raison inverse de leur chaleur spécifique sous volume constant. En effet, la réduction de volume étant par hypothèse toujours la même, la quantité de chaleur due à cette réduction doit être aussi toujours la même, et par conséquent doit produire une élévation de température dépendante seulement de la chaleur spécifique acquise par le gaz après sa compression, et évidemment en raison inverse de cette chaleur spécifique. Il nous est donc facile de former la table des élévations de température des différents gaz, pour une compression de 1/116.\par
La voici :\par

\begin{center}
\noindent \emph{Tableau de l’élévation de température des gaz par l’effet de la compression}.\par
\end{center}


\tableopen{}
\begin{tabularx}{\linewidth}
{
  | X
  | X |
}
\toprule
\raggedright\arraybackslash {\footnotesize NOMS DES GAZ.}
   & \raggedleft\arraybackslash Élévation de temp.  pour une réduct.  de volume de 1/116. \\
\midrule
\raggedright\arraybackslash {\footnotesize Air atmosphérique}
   & \raggedleft\arraybackslash 1,000 \\
\hline
\raggedright\arraybackslash {\footnotesize Gaz hydrogène}
   & \raggedleft\arraybackslash 1,160 \\
\hline
\raggedright\arraybackslash {\footnotesize Acide carbonique}
   & \raggedleft\arraybackslash 0,730 \\
\hline
\raggedright\arraybackslash {\footnotesize Oxigène}
   & \raggedleft\arraybackslash 1,035 \\
\hline
\raggedright\arraybackslash {\footnotesize Azote}
   & \raggedleft\arraybackslash 1,000 \\
\hline
\raggedright\arraybackslash {\footnotesize Protoxide d’azote}
   & \raggedleft\arraybackslash 0,667 \\
\hline
\raggedright\arraybackslash {\footnotesize Gaz oléfiant}
   & \raggedleft\arraybackslash 0,558 \\
\hline
\raggedright\arraybackslash {\footnotesize Oxide de carbone}
   & \raggedleft\arraybackslash 0,955 \\
\bottomrule
\end{tabularx}
\tableclose{}

\noindent Une nouvelle compression de 1/116 (du volume varié) élèverait encore, comme on le verra bientôt, la température de ces gaz d’une quantité à peu près égale à la première ; mais il n’en serait pas de même d’une troisième, d’une quatrième, d’une centième compression pareille. La capacité des gaz pour la chaleur change avec leur volume ; il est très-possible qu’elle change aussi avec la température.\par
Nous allons maintenant déduire de la proposition générale énoncée pag. 38, un second théorème qui servira de complément à celui qui vient d’être démontré.\par
Imaginons que le gaz renfermé dans la capacité cylindrique \emph{abcd} (fig. 2) soit transporté dans la capacité \emph{a’b’c’d’} (fig. 3), d’égale hauteur, mais de base différente et plus étendue : ce gaz augmentera de volume, diminuera de densité et de force élastique dans le rapport inverse des deux volumes \emph{abcd, a’b’c’d’}. Quant à la pression totale exercée sur chaque piston \emph{cd, c’d’}, elle sera la même de part et d’autre, car la surface de ces pistons est en raison directe des volumes.\par
Supposons que l’on exécute sur le gaz renfermé en \emph{a’b’c’d’} les opérations décrites pag. 39, et qui étaient censées faites sur le gaz renfermé en \emph{abcd}, c’est-à-dire supposons que l’on donne au piston \emph{c’d’} des mouvements égaux en amplitude à ceux du piston \emph{cd}, qu’on lui fasse occuper successivement les positions \emph{c’d’} correspondantes à \emph{cd}, et \emph{e’f’} correspondantes à \emph{ef}, et qu’en même temps on fasse subir au gaz, par le moyen des deux corps A, B, les mêmes variations de températures que lorsqu’il était renfermé en \emph{abcd} : l’effort total exercé sur le piston se trouvera être, dans les deux cas, toujours le même aux instans correspondans. Cela résulte uniquement de la loi de Mariotte\footnote{ \noindent La loi de Mariotte, sur laquelle nous nous fondons ici pour établir notre démonstration, est une des lois physiques les mieux constatées. Elle a servi de base à plusieurs théories vérifiées par l’expérience, et qui vérifient à leur tour les lois sur lesquelles elles sont assises. On peut citer encore comme vérification précieuse de la loi de Mariotte et aussi de celle de MM. Gay-Lussac et Dalton, pour un grand intervalle de température, les expériences de MM. Dulong et Petit. (Voy. \emph{Annales de physique et de chimie}, février 1818, tome 7, page 122.) On peut citer encore des expériences plus récentes de MM. Davy et Faraday.\par
 Les théorèmes que nous déduisons ici ne seraient peut-être pas exacts si on les appliquait hors de certaines limites, soit de densité, soit de température : ils ne doivent être regardés comme vrais que dans les limites où les lois de Mariotte et de MM. Gay-Lussac et Dalton sont elles-mêmes constatées.
} : en effet les densités des deux gaz conservant toujours entre elles les mêmes rapports pour des positions semblables des pistons, et les températures étant toujours égales de part et d’autre, les pressions totales exercées sur les pistons conserveront toujours le même rapport entre elles. Si ce rapport est, à un instant quelconque, celui d’égalité, les pressions seront toujours égales.\par
Comme d’ailleurs les mouvements des deux pistons ont des amplitudes égales, la puissance motrice produite de part et d’autre sera évidemment la même : d’où l’on doit conclure, d’après la proposition de la pag. 38, que les quantités de chaleur employées de part et d’autre sont les mêmes, c’est-à-dire qu’il passe du corps A au corps B la même quantité de chaleur dans les deux cas.\par
La chaleur empruntée au corps A et rendue au corps B n’est autre chose que la chaleur absorbée par la raréfaction du gaz et dégagée ensuite par sa compression. Nous sommes donc conduits à établir le théorème suivant :\par
\emph{Lorsqu’un fluide élastique passe sans changer de température du volume U au volume V, et qu’une pareille quantité pondérable du même gaz passe sous la même température du volume U’ au volume V’, si le rapport de U’ à V’ se trouve le même que le rapport de U à V, les quantités de chaleur absorbées ou dégagées dans l’un et l’autre cas seront égales entre elles}.\par
Ce théorème peut être énoncé d’une autre manière que voici :\par
\emph{Lorsqu’un gaz varie de volume sans changer de température, les quantités de chaleur absorbées ou dégagées par ce gaz sont en progression arithmétique, si les accroissements ou les réductions de volume se trouvent être en progression géométrique}.\par
Lorsque l’on comprime un litre d’air maintenu à la température 10°, et qu’on le réduit à 1/2 litre, il se dégage une certaine quantité de chaleur. Cette quantité se trouvera toujours la même si l’on réduit de nouveau le volume de 1/2 litre à 1/4 de litre, de 1/4 de litre à 1/8, ainsi de suite.\par
Si au lieu de comprimer l’air on le porte successivement à 2 litres, 4 litres, 8 litr., etc., il faudra lui fournir des quantités de chaleur toujours égales pour maintenir la température au même degré.\par
Ceci rend facilement raison de la température élevée à laquelle parvient l’air par une compression rapide. On sait que cette température suffit pour enflammer l’amadou, et même pour que l’air devienne lumineux. Si l’on suppose pour un instant la chaleur spécifique de l’air constante, malgré les changements de volume et de température, la température croîtra en progression arithmétique, pour des réductions de volume en progression géométrique. En partant de cette donnée, et admettant qu’un degré d’élévation dans la température correspond à une compression de 1/116, on arrivera facilement à conclure que l’air réduit à 1/14 de son volume primitif doit s’élever d’environ 300°, degré suffisant pour enflammer l’amadou\footnote{ \noindent Lorsque le volume est réduit de 1/116, c’est-à-dire lorsqu’il devient 115/116 de ce qu’il était d’abord, la température s’élève de 1 degré.\par
 Une nouvelle réduction de 1/116 porte le volume à (115/116)² et la température doit s’élever d’un nouveau degré.\par
 Après \emph{x} réductions pareilles, le volume devient (115/116)\textsuperscript{x}, et la température doit s’être élevée de \emph{x} degrés.\par
 Si l’on pose (115/116)\textsuperscript{x}=1/14, et que l’on prenne les logarithmes de part et d’autre, on trouve\par
 
\begin{center}
\noindent \emph{x} = 300° environ.\par
\end{center}

 \noindent Si l’on pose (115/116)\textsuperscript{x}=1/2, on trouve\par
 
\begin{center}
\noindent \emph{x} = 80°.\par
\end{center}

 \noindent Ce qui montre que l’air comprimé de moitié s’élève de 80°.\par
 Tout ceci est subordonné à l’hypothèse que la chaleur spécifique de l’air ne change pas quoique le volume diminue ; mais si, d’après les raisons exposées ci-après pages 58 et 61, on regarde la chaleur spécifique de l’air comprimé de moitié comme réduite dans le rapport de 700 à 616, il faut multiplier le nombre 80° par 700/616, ce qui le porte à 90°.
}.\par
L’élévation de température doit être évidemment encore plus considérable si la capacité de l’air pour la chaleur devient moindre à mesure que son volume diminue : or c’est ce qui est présumable, et c’est même ce qui semble résulter des expériences de MM. Delaroche et Bérard sur le calorique spécifique de l’air pris à diverses densités. (Voyez le Mémoire imprimé dans les \emph{Annales de chimie}, tom. 85, pag. 72, 224.)\par
Les deux théorèmes énoncés pag. 41 et pag. 52 suffisent pour comparer entre elles les quantités de chaleur absorbées ou dégagées dans les changements de volume des fluides élastiques, quelles que soient d’ailleurs la densité et la nature chimique de ces fluides, pourvu toutefois qu’ils soient tous pris et maintenus à une certaine température invariable, mais ces théorèmes ne fournissent aucun moyen de comparer entre elles les quantités de chaleur dégagées ou absorbées par des fluides élastiques qui changent de volume à des températures différentes. Ainsi nous ignorons quel rapport existe entre la chaleur dégagée par un litre d’air réduit à moitié, la température étant maintenue à 0°, et la chaleur dégagée par le même litre d’air réduit à moitié, la température étant maintenue à 100°. La connaissance de ce rapport est liée à celle de la chaleur spécifique des gaz à divers degrés de température, et à quelques autres données que la physique actuelle refuse de nous fournir.\par
Le second de nos théorèmes nous offre un moyen de connaître suivant quelle loi varie la chaleur spécifique des gaz avec leur densité.\par
Admettons que les opérations décrites p. 39, au lieu de s’exécuter avec deux corps A, B, dont les températures diffèrent entre elles d’une quantité infiniment petite, s’exécutent avec deux corps dont les températures diffèrent entre elles d’une quantité finie, de 1° par exemple. Dans un cercle complet d’opérations le corps A fournit au fluide élastique une certaine quantité de chaleur, qui peut être divisée en deux portions : 1° celle qui est nécessaire pour maintenir la température du fluide à un degré constant pendant la dilatation ; 2° celle qui est nécessaire pour faire revenir le fluide de la température du corps B à la température du corps A, lorsque après avoir ramené ce fluide à son volume primitif, on le remet en contact avec le corps A. Nommons \emph{a} la première de ces quantités, et \emph{b} la seconde : le calorique total fourni par le corps A sera exprimé par \emph{a} + \emph{b}.\par
Le calorique transmis par le fluide au corps B peut aussi se diviser en deux parties : l’une, \emph{b’}, due au refroidissement du gaz par le corps B, l’autre, \emph{a’}, que le gaz abandonne par l’effet de sa réduction de volume. La somme de ces deux quantités est \emph{a’} + \emph{b’} ; elle doit être égale à \emph{a} + \emph{b}, car après un cercle complet d’opérations, le gaz est ramené identiquement à son état primitif. Il a dû céder tout le calorique qui lui avait d’abord été fourni. Nous avons donc\par

\begin{center}
\noindent \emph{a} + \emph{b} = \emph{a’} + \emph{b’},\par
\end{center}

\noindent ou bien\par

\begin{center}
\noindent \emph{a} — \emph{a’} = \emph{b’} — \emph{b}.\par
\end{center}

\noindent Or, d’après le théorème énoncé pag. 52, les quantités \emph{a} et \emph{a’} sont indépendantes de la densité du gaz, pourvu toutefois que la quantité pondérable reste la même, et que les variations de volume soient proportionnelles au volume primitif. La différence \emph{a} — \emph{a’} doit remplir les mêmes conditions, et par conséquent aussi la différence \emph{b’} — \emph{b}, qui lui est égale. Mais \emph{b’} est le calorique nécessaire pour élever d’un degré le gaz renfermé en \emph{abcd} (fig. 2) ; \emph{b’} est le calorique abandonné par le gaz, lorsque, renfermé en \emph{abef}, il se refroidit de 1 degré ; ces quantités peuvent servir de mesure aux chaleurs spécifiques. Nous sommes donc conduits à établir la proposition suivante :\par
\emph{Le changement opéré dans la chaleur spécifique d’un gaz par suite d’un changement de volume dépend uniquement du rapport entre le volume primitif et le volume varié}. C’est-à-dire que la différence des chaleurs spécifiques ne dépend pas de la grandeur absolue des volumes, mais seulement de leur rapport.\par
Cette proposition peut être énoncée d’une autre manière que voici :\par
\emph{Lorsqu’un gaz augmente de volume en progression géométrique, sa chaleur spécifique s’accroît en progression arithmétique}.\par
Ainsi \emph{a} étant la chaleur spécifique de l’air pris à une densité donnée, et \emph{a} + \emph{h} la chaleur spécifique pour une densité moitié moindre, elle sera, pour une densité égale au quart, \emph{a} + 2\emph{h} ; pour une densité égale au huitième \emph{a} + 3\emph{h} ; ainsi de suite.\par
Les chaleurs spécifiques sont ici rapportées aux poids. Elles sont supposées prises sous volume invariable ; mais, ainsi qu’on va le voir, elles suivraient la même loi si elles étaient prises sous pression constante.\par
À quelle cause est due en effet la différence entre les chaleurs spécifiques prises sous volume constant et sous pression constante ? Au calorique nécessaire pour produire dans le second cas l’augmentation de volume. Or, d’après la loi de Mariotte, l’augmentation de volume d’un gaz doit être, pour un changement donné de température, une fraction déterminée du volume primitif, une fraction indépendante de la pression. D’après le théorème énoncé pag. 52, si le rapport entre le volume primitif et le volume varié est donné, la chaleur nécessaire pour produire l’augmentation de volume est par-là déterminée. Elle dépend uniquement de ce rapport et de la quantité pondérable du gaz. Il faut donc conclure que :\par
\emph{La différence entre la chaleur spécifique sous pression constante et la chaleur spécifique sous volume constant est toujours la même, quelle que soit la densité du gaz, pourvu que la quantité pondérable reste la même}.\par
Ces chaleurs spécifiques augmentent toutes deux à mesure que la densité du gaz diminue, mais leur différence ne varie pas\footnote{ \noindent MM. Gay-Lussac et Welter ont trouvé, par des expériences directes citées dans la \emph{Mécanique céleste} et dans les \emph{Annales de physique et de chimie}, juillet 1822, p. 267, que le rapport entre la chaleur spécifique sous pression constante et la chaleur spécifique sous volume constant varie très-peu avec la densité du gaz. D’après ce que l’on vient de voir, c’est la différence qui doit rester constante, et non le rapport. Comme d’ailleurs la chaleur spécifique des gaz, pour un poids donné, varie très-peu avec la densité, il est tout simple que le rapport n’éprouve lui-même que de petits changements.\par
 Le rapport entre la chaleur spécifique de l’air atmosphérique sous pression constante et sous volume constant est, d’après MM. Gay-Lussac et Welter, 1,3748, nombre à peu près constant pour toutes les pressions et même pour toutes les températures. Nous sommes parvenus, par d’autres considérations, au nombre (267+116)/267= 1,44, qui en diffère 1/20 de , et nous nous sommes servis de ce nombre pour dresser une table des chaleurs spécifiques des gaz sous volume constant : ainsi il ne faut pas considérer cette table comme bien exacte, non plus que la table donnée pag. 61. Ces tables sont destinées principalement à mettre en évidence les lois que suivent les chaleurs spécifiques des fluides aériformes.
}.\par
Puisque la différence entre les deux capacités pour la chaleur est constante, si l’une s’accroît en progression arithmétique, l’autre doit suivre une progression semblable : ainsi notre loi est applicable aux chaleurs spécifiques prises sous pression constante.\par
Nous avons supposé tacitement l’augmentation de la chaleur spécifique avec celle du volume. Cette augmentation résulte des expériences de MM. Delaroche et Bérard : en effet, ces physicients ont trouvé 0,967 pour la chaleur spécifique de l’air sous la pression 1 mètre de mercure (voyez le Mémoire déjà cité), en prenant pour unité la chaleur spécifique du même poids d’air sous la pression 0,760 mètres.\par
D’après la loi que suivent les chaleurs spécifiques par rapport aux pressions, il suffit de les avoir observées dans deux cas particuliers, pour les conclure dans tous les cas possibles : c’est ainsi qu’en faisant usage du résultat d’expérience de MM. Delaroche et Bérard, qui vient d’être rapporté, nous avons dressé la table suivante des chaleurs spécifiques de l’air sous diverses pressions.\par

\tableopen{}
\begin{tabularx}{\linewidth}
{
  | X
  | X
  | X
  | X |
}
\toprule
\raggedright\arraybackslash {\footnotesize {\scshape Pression} en atmosphères.}
   & \raggedright\arraybackslash Chal. spéc. celle de l’air sous pression atmos. étant 1.
   & \raggedright\arraybackslash {\footnotesize {\scshape Pression} en atmosphères.}
   & \raggedright\arraybackslash Chal. spéc. celle de l’air sous pression atmos. étant 1. \\
\midrule
\raggedleft\arraybackslash 1/1024
   & \raggedleft\arraybackslash 1,840
   & \raggedleft\arraybackslash 1
   & \raggedleft\arraybackslash 1,000 \\
\hline
\raggedleft\arraybackslash 1/512
   & \raggedleft\arraybackslash 1,756
   & \raggedleft\arraybackslash 2
   & \raggedleft\arraybackslash 0,916 \\
\hline
\raggedleft\arraybackslash 1/256
   & \raggedleft\arraybackslash 1,672
   & \raggedleft\arraybackslash 4
   & \raggedleft\arraybackslash 0,832 \\
\hline
\raggedleft\arraybackslash 1/128
   & \raggedleft\arraybackslash 1,588
   & \raggedleft\arraybackslash 8
   & \raggedleft\arraybackslash 0,748 \\
\hline
\raggedleft\arraybackslash 1/64
   & \raggedleft\arraybackslash 1,504
   & \raggedleft\arraybackslash 16
   & \raggedleft\arraybackslash 0,664 \\
\hline
\raggedleft\arraybackslash 1/32
   & \raggedleft\arraybackslash 1,420
   & \raggedleft\arraybackslash 32
   & \raggedleft\arraybackslash 0,580 \\
\hline
\raggedleft\arraybackslash 1/16
   & \raggedleft\arraybackslash 1,336
   & \raggedleft\arraybackslash 64
   & \raggedleft\arraybackslash 0,496 \\
\hline
\raggedleft\arraybackslash 1/8
   & \raggedleft\arraybackslash 1,252
   & \raggedleft\arraybackslash 128
   & \raggedleft\arraybackslash 0,412 \\
\hline
\raggedleft\arraybackslash 1/4
   & \raggedleft\arraybackslash 1,165
   & \raggedleft\arraybackslash 256
   & \raggedleft\arraybackslash 0,328 \\
\hline
\raggedleft\arraybackslash 1/2
   & \raggedleft\arraybackslash 1,084
   & \raggedleft\arraybackslash 512
   & \raggedleft\arraybackslash 0,244 \\
\hline
\raggedleft\arraybackslash 1
   & \raggedleft\arraybackslash 1,000
   & \raggedleft\arraybackslash 1024
   & \raggedleft\arraybackslash 0,160 \\
\bottomrule
\end{tabularx}
\tableclose{}

\noindent La première colonne est, comme on voit, une progression géométrique, et la seconde une progression arithmétique.\par
Nous avons étendu la table jusqu’à des compressions et des raréfactions extrêmes. Il est à croire que l’air, avant d’acquérir une densité 1024 fois sa densité ordinaire, c’est-à-dire avant de devenir plus dense que l’eau, se serait liquéfié. Les chaleurs spécifiques s’annuleraient et même deviendraient négatives en prolongeant la table au-delà du dernier terme. Nous pensons au reste que les chiffres de la seconde colonne décroissent ici en progression trop rapide. Les expériences qui servent de base à notre calcul ont été faites dans des limites trop resserrées pour que l’on puisse s’attendre à une grande justesse dans les nombres que nous avons obtenus, surtout dans les nombres extrêmes.\par
Puisque nous connaissons d’une part la loi suivant laquelle la chaleur se dégage par la compression des gaz, et de l’autre la loi suivant laquelle varie la chaleur spécifique avec le volume, il nous sera facile de calculer les accroissements de température d’un gaz que l’on comprime sans lui laisser perdre de calorique. En effet la compression peut être censée décomposée en deux opérations successives : 1° compression à température constante, 2° restitution du calorique émis. La température s’élèvera par cette seconde opération en raison inverse de la chaleur spécifique acquise par le gaz, après sa réduction de volume, chaleur spécifique que nous savons calculer au moyen de la loi démontrée ci-dessus. La chaleur dégagée par la compression doit, d’après le théorème de la pag. 52, être représentée par une expression de la forme \emph{s} = A + B \emph{log v, s} étant cette chaleur, \emph{v} le volume du gaz après la compression, A et B des constantes arbitraires dépendantes du volume primitif du gaz, de sa pression et des unités dont on fait choix.\par
La chaleur spécifique, variant avec le volume suivant la loi démontrée tout à l’heure, doit être représentée par une expression de la forme \emph{z} = A’ + B’ \emph{log v}, A’ et B’ étant des constantes arbitraires différentes de A et B.\par
L’accroissement de température acquis par le gaz par l’effet de la compression est proportionnel au rapport \emph{8/Z}, ou au rapport ((\emph{A+B}) \emph{log v}) / ((\emph{A’+B’}) \emph{log v}). Il peut être représenté par ce rapport lui-même : ainsi, en le nommant \emph{t}, nous aurons t = ((\emph{A+B}) \emph{log v}) / ((\emph{A’+B’}) \emph{log v}). Si volume primitif du gaz est 1 et la température primitive 0° l’on aura à la fois \emph{t} = 0, \emph{log v} = 0, d’où A = 0. \emph{t} exprimera alors non seulement l’accroissement de température, mais la température elle-même au-dessus du zéro thermométrique.\par
Il ne faudrait pas considérer la formule que nous venons de donner comme applicable à de très-grands changements de volume des gaz. Nous avons regardé l’élévation de température comme étant en raison inverse de la chaleur spécifique ; ce qui suppose implicitement la chaleur spécifique constante à toutes les températures. De grands changements de volume entraînent dans le gaz de grands changements de température, et rien ne nous prouve la constance de la chaleur spécifique à divers degrés, surtout à des degrés fort éloignés les uns des autres. Cette constance n’est qu’une hypotèse, admise pour les gaz par analogie, vérifiée passablement pour les corps solides et liquides dans une certaine étendue de l’échelle thermométrique, mais dont les expériences de MM. Dulong et Petit ont fait voir l’inexactitude lorsqu’on veut l’étendre à des températures fort au-dessus de 100°\footnote{ \noindent L’on ne voit pas de raison pour admettre \emph{a priori} la constance de la chaleur spécifique des corps à diverses températures, c’est-à-dire pour admettre que des quantités égales de chaleur produiront des accroissements égaux dans le degré thermométrique d’un corps, quand même ce corps ne changerait ni d’état ni de densité ; quand ce serait, par exemple, un fluide élastique renfermé dans une capacité inextensible. Des expériences directes sur des corps solides et liquides avaient prouvé qu’entre 0° et 100°, des accroissements égaux dans les quantités de chaleur produisaient des accroissements à peu près égaux dans les degrés de température ; mais les expériences plus récentes de MM. Dulong et Petit (Voy. \emph{Annales de chimie et de physique}, février, mars et avril 1818) ont fait voir que cette correspondance ne se soutenait plus à des températures fort au-dessus de 100°, soit que ces températures fussent mesurées sur le thermomètre à mercure, soit qu’elles fussent mesurées sur le thermomètre à air.\par
 Non seulement les chaleurs spécifiques ne restent pas les mêmes aux diverses températures, mais en outre elles ne conservent pas entre elles les mêmes rapports ; de sorte qu’aucune échelle thermométrique ne pourrait établir la constance de toutes les chaleurs spécifiques à la fois. Il eût été intéressant de vérifier si les mêmes irrégularités subsistent pour les substances gazeuses ; mais les expériences présentaient ici des difficultés presque insurmontables.\par
 Les irrégularités des chaleurs spécifiques des corps solides pourraient être attribuées, ce nous semble, à de la chaleur latente employée à produire un commencement de fusion, un ramollissement qui se fait sentir dans la plupart de ces corps, long-temps avant la fusion complète. On peut appuyer cette opinion de la remarque suivante : d’après les expériences mêmes de MM. Dulong et Petit, l’accroissement de chaleur spécifique avec la température est plus rapide dans les solides que dans les liquides, quoique ceux-ci jouissent d’une dilatabilité plus considérable. La cause d’irrégularité que nous venons de signaler, si elle est réelle, disparaîtrait entièrement dans les gaz.
}.\par
D’après une loi due à MM. Clément et Désormes, loi établie par la voie de l’expérience directe, la vapeur d’eau, sous quelque pression qu’elle soit formée, contient toujours, à poids égaux, la même quantité de chaleur, ce qui revient à dire que la vapeur comprimée ou dilatée mécaniquement sans perte de chaleur sera toujours constituée à saturation de l’espace, si elle est primitivement ainsi constituée. La vapeur d’eau ainsi constituée peut donc être regardée comme un gaz permanent ; elle doit en observer toutes les lois. Par conséquent la formule\par

\begin{center}
\noindent \emph{t} = ((\emph{A+B}) \emph{log v}) / ((\emph{A’+B’}) \emph{log v})\par
\end{center}

\noindent doit lui être applicable, et se trouver en concordance avec la table des tensions résultante des expériences directes de M. Dalton.\par
On peut s’assurer en effet que notre formule, par une détermination convenable des constantes arbitraires, représente d’une manière fort approchée les résultats de l’expérience. Les petites anomalies que l’on peut y rencontrer n’excèdent pas celles qui doivent être attribuées raisonnablement aux erreurs d’observation\footnote{ \noindent Pour déterminer les constantes arbitraires A, B, A’, B’, d’après des résultats choisis dans la table de M. Dalton, il faut commencer par calculer le volume de la vapeur d’après sa pression et sa température, ce qui est chose facile au moyen des lois de Mariotte et de M. Gay-Lussac, la quantité pondérable de la vapeur étant d’ailleurs fixée.\par
 Le volume sera donné par l’équation\par
 
\begin{center}
\noindent \emph{v} = \emph{c} (267+\emph{t})/\emph{p},\par
\end{center}

 \noindent dans laquelle \emph{v} est ce volume, \emph{t} la température, \emph{p} la pression, et \emph{c} une quantité constante dépendante du poids de la vapeur et des unités dont on a fait choix.\par
 Voici la table des volumes occupés par un gramme de vapeur formée à diverses températures, et par conséquent sous diverses pressions :\par
 
\begin{itemize}[itemsep=0pt,topsep=0pt,partopsep=0pt,parskip=0pt]
\item \emph{t} : degrés centigrades. 
\item \emph{p} : tension de la vapeur exprimée en millimètres de mercure.
\item \emph{v} : volume d’un gramme de vapeur, exprimé en litres.
\end{itemize}

 
\tableopen{}
\begin{tabularx}{\linewidth}
{
  | X
  | X
  | X |
}
\toprule
\raggedleft\arraybackslash t
   & \raggedleft\arraybackslash p
   & \raggedleft\arraybackslash v \\
\midrule
\raggedleft\arraybackslash 0°
   & \raggedleft\arraybackslash 5,060
   & \raggedleft\arraybackslash 185,00 \\
\hline
\raggedleft\arraybackslash 20°
   & \raggedleft\arraybackslash 17,320
   & \raggedleft\arraybackslash 58,20 \\
\hline
\raggedleft\arraybackslash 40°
   & \raggedleft\arraybackslash 53,000
   & \raggedleft\arraybackslash 20,40 \\
\hline
\raggedleft\arraybackslash 60°
   & \raggedleft\arraybackslash 144,600
   & \raggedleft\arraybackslash 7,96 \\
\hline
\raggedleft\arraybackslash 80°
   & \raggedleft\arraybackslash 352,100
   & \raggedleft\arraybackslash 3,47 \\
\hline
\raggedleft\arraybackslash 100°
   & \raggedleft\arraybackslash 760,000
   & \raggedleft\arraybackslash 1,70 \\
\bottomrule
\end{tabularx}
\tableclose{}

 \noindent Les deux premières colonnes de cette table sont tirées du \emph{Traité de Physique} de M. Biot (1er vol. pag. 272 et 531). La troisième est calculée au moyen de la formule ci-dessus et d’après ce résultat d’expérience que l’eau vaporisée sous la pression atmosphérique occupe un espace 1700 fois aussi grand qu’à l’état liquide.\par
 En faisant usage de trois nombres de la première colonne et des trois nombres correspondans de la troisième colonne, on déterminera facilement les constantes de notre équation\par
 
\begin{center}
\noindent \emph{t} = (\emph{A+B log v})/(\emph{A’+B’ log v}).\par
\end{center}

 \noindent Nous n’entrerons pas dans les détails du calcul nécessaire pour déterminer ces quantités : il nous suffira de dire que les valeurs suivantes :\par
 
\begin{center}
\noindent A = 2268, A’ = 19,64, \\
B = - 1000, B’ = 3,30,\par
\end{center}

 \noindent satisfont passablement bien aux conditions prescrites, de sorte que l’équation\par
 
\begin{center}
\noindent \emph{t} = (2268–1000 \emph{log v}) / (19,64+3,30 \emph{log v})\par
\end{center}

 \noindent exprime d’une manière très-approchée la relation qui existe entre le volume de la vapeur et sa température.\par
 On remarquera ici que la quantité B’ est positive et fort petite, ce qui tend à confirmer cette proposition, que la chaleur spécifique d’un fluide élastique croît avec le volume, mais suivant une progression peu rapide.
}.\par
Nous reviendrons ici à notre sujet principal, dont nous nous sommes déjà trop écartés, à la puissance motrice de la chaleur.\par
Nous ayons fait voir que la quantité de puissance motrice développée par le transport du calorique d’un corps à un autre dépendait essentiellement des températures des deux corps, mais nous n’avons pas fait connaître de relation entre ces températures et les quantités de puissance motrice produites. Il semblerait d’abord assez naturel de supposer que pour des différences égales de température les quantités de puissance motrice produites sont égales entre elles, c’est-à-dire que, par exemple, le passage d’une quantité donnée de calorique d’un corps A maintenu à 100° à un corps B maintenu à 50° doit donner naissance à une quantité de puissance motrice égale à celle qui serait développée par le transport du même calorique, d’un corps B maintenu à 50° à un corps C maintenu à 0°. Une pareille loi serait sans doute fort remarquable, mais l’on ne voit pas de motifs suffisans pour l’admettre \emph{à priori}. Nous allons discuter sa réalité par des raisonnements rigoureux. Imaginons que les opérations décrites pag. 40 soient exécutées successivement sur deux quantités d’air atmosphérique égales en poids et en volume, mais prises à des températures différentes ; supposons en outre les différences de degré entre ces corps A et B égales de part et d’autre : ainsi ces corps auront, par exemple, dans l’un des cas, les températures 100° et 100° − \emph{h}° (\emph{h} étant infiniment petit), et dans l’autre, 1° et 1° − \emph{h}. La quantité de puissance motrice produite est dans chaque cas la différence entre celle que fournit le gaz par sa dilation et celle dont il nécessite l’emploi pour revenir à son volume primitif. Or cette différence est ici, comme on peut s’en assurer par un raisonnement simple que nous ne croyons pas nécessaire de détailler, la même dans l’un et l’autre cas : ainsi la puissance motrice produite est la même.\par
Comparons maintenant entre elles les quantités de chaleur employées dans les deux cas. Dans le premier, la quantité de chaleur employée est celle que le corps A fournit à l’air pour le maintenir à la température 100° pendant son expansion ; dans le second, c’est la quantité de chaleur que ce même corps doit lui fournir pour maintenir sa température à 1° pendant un changement de volume absolument semblable. Si ces deux quantités de chaleur étaient égales entre elles, il en résulterait évidemment la loi que nous avons d’abord supposée. Mais rien ne prouve qu’il en soit ainsi ; on va même voir que ces quantités de chaleur sont inégales.\par
L’air, que nous supposerons d’abord occuper l’espace \emph{abcd} (fig. 2) et se trouver à la température 1°, peut être amenée à occuper l’espace \emph{abef} et à acquérir la température 100° par deux moyens différents :\par

\begin{enumerate}[itemsep=0pt,topsep=0pt,partopsep=0pt,parskip=0pt]
\item On peut l’échauffer d’abord sans faire varier son volume, puis le dilater en maintenant sa température à un degré constant ;
\item On peut commencer par le dilater, en maintenant la constance de la température, puis l’échauffer lorsqu’il a acquis son nouveau volume.
\end{enumerate}

\noindent Soient \emph{a} et \emph{b} les quantités de chaleur employées successivement dans la première des deux opérations, et soient \emph{b’} et \emph{a’} les quantités de chaleur employées successivement dans la seconde ; comme le résultat final de ces deux opérations est le même, les quantités de chaleur employées de part et d’autre doivent être égales : l’on a donc\par

\begin{center}
\noindent a + b = a’ + b’, \\
d’où a’ – a = b – b’.\par
\end{center}

\noindent \emph{a’} est la quantité de chaleur nécessaire pour faire passer le gaz de 1° à 100°, lorsqu’il occupe l’espace \emph{abef}.\par
\emph{a} est la quantité de chaleur nécessaire pour faire passer le gaz de 1° à 100°, lorsqu’il occupe l’espace \emph{abcd}.\par
La densité de l’air est moindre dans le premier cas que dans le second, et d’après les expériences de MM. Delaroche et Bérard déjà citées page 60, sa capacité pour la chaleur doit être un peu plus grande.\par
La quantité \emph{a’} se trouvant être plus grande que la quantité \emph{a, b} doit être plus grand que \emph{b’}. Par conséquent, en généralisant la proposition, nous dirons :\par
\emph{La quantité de chaleur due au changement de volume d’un gaz est d’autant plus considérable que la température est plus élevée}.\par
Ainsi, par exemple, il faut plus de calorique pour maintenir à 100° la température d’une certaine quantité d’air dont on double le volume, que pour maintenir à 1° la température de ce même air pendant une dilatation absolument pareille.\par
Ces quantités inégales de chaleur produiraient cependant, comme nous l’avons vu, des quantités égales de puissance motrice pour des chutes égales du calorique, prises à différentes hauteurs sur l’échelle thermométrique ; d’où l’on peut tirer la conclusion suivante :\par
\emph{La chute du calorique produit plus de puissance motrice dans les degrés inférieurs que dans les degrés supérieurs}.\par
Ainsi, une quantité donnée de chaleur développera plus de puissance motrice en passant d’un corps maintenu à 1°, à un autre maintenu à 0°, que si ces deux corps eussent possédé les degrés 101° et 100°.\par
Du reste, la différence doit être fort petite ; elle serait nulle si la capacité de l’air pour la chaleur demeurait constante, malgré les changements de densité. D’après les expériences de MM. Delaroche et Bérard, cette capacité varie peu, si peu même, que les différences remarquées pourraient à la rigueur être attribuées à des erreurs d’observation, ou à quelques circonstances dont on aurait négligé de tenir compte.\par
Nous sommes hors d’état de déterminer rigoureusement, avec les seules données expérimentales que nous possédons, la loi suivant laquelle varie la puissance motrice de la chaleur dans les différents degrés de l’échelle thermométrique. Cette loi est liée à celle des variations de la chaleur spécifique des gaz à diverses températures, loi que l’expérience n’a pas encore fait connaître avec une suffisante exactitude.\footnote{ \noindent Si l’on admettait la constance de la chaleur spécifique d’un gaz lorsque son volume ne change pas, mais que sa température varie, l’analyse pourrait conduire à une relation entre la puissance motrice et le degré thermométrique. Nous allons faire voir de quelle manière, cela nous donnera d’ailleurs occasion de montrer comment quelques unes des propositions établies ci-dessus doivent être énoncés en langage algébrique.\par
 Soit \emph{r} la quantité de puissance motrice produite par l’expansion d’une quantité donnée d’air passant du volume un litre au volume \emph{v} litres, sous température constante : si \emph{v} augmente de la quantité infiniment petite \emph{dv, r} augmentera de la quantité \emph{dr}, qui, d’après la nature de la puissance motrice, sera égale à l’accroissement \emph{dv} de volume multiplié par la force expansive que possède alors le fluide élastique : soit \emph{p} cette force expansive, on aura l’équation :\par
 
\begin{center}
\noindent \emph{ dr} = \emph{pdv} . . . (1).\par
\end{center}

 \noindent Supposons la température constante sous laquelle la dilatation a lieu égale à \emph{t} degrés centigrades : si l’on nomme \emph{q} la force élastique de l’air occupant le volume un litre à la même température \emph{t}, on aura, d’après la loi de Mariotte,\par
 
\begin{center}
\noindent \emph{v} ∶ \emph{t} ∷ \emph{q} ∷ \emph{p}, d’où \emph{p} = \emph{q/v}.\par
\end{center}

 \noindent Si maintenant P est la force élastique de ce même air occupant toujours le volume 1, mais à la température 0°, on aura, d’après la règle de M. Gay-Lussac,\par
 
\begin{center}
\noindent \emph{q} = P + P \emph{t}/267 = \emph{P}/267 (267+\emph{t}), \\
d’où \emph{q/v} = \emph{p} = (\emph{P}/267)((267+\emph{t})/\emph{v}).\par
\end{center}

 \noindent Si, pour abréger, l’on nomme N la quantité \emph{P}/267, l’équation deviendra :\par
 
\begin{center}
\noindent \emph{p} = N((\emph{t}+267)\emph{/v}),\par
\end{center}

 \noindent d’où l’on tire, d’après l’équation (1),\par
 
\begin{center}
\noindent \emph{dr} = N ((\emph{t}+267)\emph{/v}) \emph{dv}.\par
\end{center}

 \noindent Regardons \emph{t} comme constant, et prenons l’intégrale des deux membres, nous aurons\par
 
\begin{center}
\noindent \emph{r} = N (\emph{t} + 267) \emph{log v} + C.\par
\end{center}

 \noindent Si l’on suppose \emph{r} = 0 lorsque \emph{v} = 1, on aura C = 0 ;\par
 
\begin{center}
\noindent d’où \emph{r} = N (\emph{t} + 267) \emph{log v} . . . (2).\par
\end{center}

 \noindent C’est là la puissance motrice produite par l’expansion de l’air, qui, sous la température \emph{t}, a passé du volume 1 au volume \emph{v}.\par
 Si, au lieu d’opérer à la température \emph{t}, on opère d’une manière absolument semblable à la température \emph{t} + \emph{dt}, la puissance développée sera\par
 
\begin{center}
\noindent \emph{r} + \emph{δr} = N (\emph{t} + \emph{dt} + 267) \emph{log v}.\par
\end{center}

 \noindent Retranchant l’équation (2), il vient\par
 
\begin{center}
\noindent \emph{δr} = N \emph{log v.dt} . . . (3)\par
\end{center}

 \noindent Soit \emph{e} la quantité de chaleur employée à maintenir la température du gaz à un degré constant pendant sa dilatation : d’après le raisonnement de la page 40, \emph{δr} sera la puissance développée par la chute de la quantité \emph{e} de chaleur du degré \emph{t} + \emph{dt} au degré \emph{t}. Si nous nommons \emph{u} la puissance motrice développée par la chute d’une unité de chaleur du degré \emph{t} au degré 0°, comme, d’après le principe général établi pag. 38, cette quantité \emph{u} doit dépendre uniquement de \emph{t}, elle pourra être représentée par la fonction F\emph{t}, d’où u = F\emph{t}.\par
 Lorsque \emph{t} s’accroît et devient \emph{t + dt , u} devient \emph{u + du}, d’où :\par
 
\begin{center}
\noindent \emph{u + du} = F(\emph{t + dt})\par
\end{center}

 \noindent Retranchant l’équation précédente, il vient\par
 
\begin{center}
\noindent \emph{du} = F(\emph{t + dt}) — F\emph{t} = F’ t.dt.\par
\end{center}

 \noindent C’est évidemment là la quantité de puissance motrice produite par la chute d’une unité de chaleur du degré \emph{t + dt} au degré \emph{t}.\par
 Si la quantité de chaleur, au lieu d’être une unité, eût été \emph{e}, sa puissance motrice produite aurait eu pour valeur :\par
 
\begin{center}
\noindent \emph{edu} = \emph{e} F’ \emph{t.dt} . . . (4).\par
\end{center}

 \noindent Mais \emph{edu} est la même chose que \emph{δr} ; toutes deux sont la puissance développée par la chute de la quantité \emph{e} de chaleur du degré \emph{t + dt} au degré \emph{t} : par conséquent,\par
 
\begin{center}
\noindent {\itshape edu = δr ;}\par
\end{center}

 \noindent et, à cause des équations 3, 4,\par
 
\begin{center}
\noindent \emph{e}F’ \emph{t.dt} = N \emph{log v.dt} ;\par
\end{center}

 \noindent ou, divisant par F’\emph{t.dt},\par
 
\begin{center}
\noindent e = (\emph{N/F’t}) \emph{log v} = T \emph{log v}\par
\end{center}

 \noindent en nommant T la fraction (\emph{N/F’\textsuperscript{t}}) qui est une fonction de \emph{t} seul.\par
 L’équation :\par
 
\begin{center}
\noindent e = T \emph{log v}\par
\end{center}

 \noindent est l’expression analytique de la loi énoncée pag. 52 ; elle est commune à tous les gaz, puisque les lois dont nous avons fait usage sont communes à tous.\par
 Si l’on nomme \emph{s} la quantité de chaleur nécessaire pour amener l’air sur lequel nous avons opéré, du volume \emph{1} et de la température 0° au volume \emph{v} et à la température \emph{t}, la différence entre \emph{s} et \emph{e} sera la quantité de chaleur nécessaire pour amener l’air sous le volume \emph{1} du degré 0 sera une fonction quelconque de \emph{t} : on aura\par
 
\begin{center}
\noindent \emph{s} = \emph{e} + U = T \emph{log v} + U.\par
\end{center}

 \noindent Si l’on différencie cette équation par rapport à \emph{t} seul, et que l’on représente par \emph{T’} et \emph{U’} les coefficients différentiels de \emph{T} et \emph{U}, il viendra\par
 
\begin{center}
\noindent \emph{ds/dt} = T’ \emph{log v} + U’ . . . (5).\par
\end{center}

 \noindent \emph{ds/dt} n’est autre chose que la chaleur spécifique du gaz sous volume constant, et notre équation (1) est l’expression analytique de la loi énoncée pag. 58.\par
 Si l’on suppose la chaleur spécifique constante à toutes les températures (hypothèse discutée ci-dessus pag. 64), la quantité \emph{ds/dt} sera indépendante de \emph{t} ; et, afin de satisfaire à l’équation (5) pour deux valeurs particulières de \emph{r}. il sera nécessaire que \emph{T’} et \emph{U’} soient indépendants de \emph{t}, nous aurons donc T’ = C , quantité constante. Multipliant T’ et \emph{C} par \emph{dt}, et prenant l’intégrale de part et d’autre, on trouve :\par
 
\begin{center}
\noindent T = C\emph{t} + C ;\par
\end{center}

 \noindent mais comme T = \emph{N/F’t} on a\par
 
\begin{center}
\noindent F’ \emph{t} = \emph{N/T} = \emph{N}/(\emph{Ct}+\emph{C})\par
\end{center}

 \noindent Multipliant de part et d’autre par \emph{dt}, et intégrant, il vient\par
 
\begin{center}
\noindent F\emph{t} = (\emph{N/C}) \emph{log} (C\emph{t} + C1) + C2;\par
\end{center}

 \noindent ou, en changeant de constantes arbitraires, et remarquant d’ailleurs que F\emph{t} est nul lorsque \emph{t} = 0°,\par
 
\begin{center}
\noindent F\emph{t} = A log (1 + (\emph{t/B})) . . .(6).\par
\end{center}

 \noindent La nature de la fonction F\emph{t} se trouverait ainsi déterminée, et l’on se verrait par-là en état d’évaluer la puissance motrice développée par une chute quelconque de la chaleur. Mais cette dernière conclusion est fondée sur l’hypothèse de la constance de la chaleur spécifique d’un gaz qui ne change pas de volume, hypothèse dont l’expérience n’a pas encore assez bien vérifié l’exactitude. Jusqu’à nouvelle preuve, notre équation (6) me peut être admise que dans une étendue médiocre de l’échelle thermométrique.\par
 Dans l’équation (5), le premier membre représente, comme nous l’avons remarqué, la chaleur spécifique de l’air occupant le volume \emph{v}. L’expérience ayant appris que cette chaleur varie peu malgré des changements assez considérables de volume, il faut que le coefficient T’ de \emph{log v} soit une quantité fort petite. Si on la suppose nulle, et qu’après avoir multiplié par \emph{dt} l’équation\par
 
\begin{center}
\noindent T’ = 0 ,\par
\end{center}

 \noindent on en prenne l’intégrale, on trouve\par
 
\begin{center}
\noindent T = C quantité constante ; \\
mais T =  \emph{N/F’t}; \\
d’où F’\emph{t} = \emph{N/T} = \emph{N/C} = A ;\par
\end{center}

 \noindent d’où l’on tire enfin, par une seconde intégration,\par
 
\begin{center}
\noindent F\emph{t} = At + B.\par
\end{center}

 \noindent Comme F\emph{t} = 0 , lorsque \emph{t} = 0 , B est nul : ainsi\par
 
\begin{center}
\noindent F\emph{t} = A\emph{t} ,\par
\end{center}

 \noindent c’est-à-dire que la puissance motrice produite se trouverait être exactement proportionnelle à la chute du calorique. Ceci est la traduction analytique de ce que nous avons dit page 70.
 }\par
Nous chercherons ici à évaluer d’une manière absolue la puissance motrice de la chaleur ; et afin de vérifier notre proposition fondamentale, afin de vérifier si l’agent mis en œuvre pour réaliser la puissance motrice est réellement indifférent, relativement à la quantité de cette puissance, nous en choisirons successivement plusieurs, l’air atmosphérique, la vapeur d’eau, la vapeur d’alcool.\par
Supposons que l’on emploie d’abord l’air atmosphérique, l’opération se conduira d’après la méthode indiquée pag. 39. Nous ferons les hypothèses suivantes :\par
L’air est pris sous la pression atmosphérique ; la température du corps A est un millième de degré au-dessus de 0°, celle du corps B est 0°. La différence est comme on voit fort petite, circonstance nécessaire ici.\par
L’accroissement de volume donné à l’air dans notre opération sera 1/116+1/267 du volume primitif : c’est un accroissement fort petit, absolument parlant, mais grand relativement à la différence des températures entre les corps A et B.\par
La puissance motrice développée par l’ensemble des deux opérations décrites pag. 39. sera, à très-peu près, proportionnelle à l’accroissement de volume et à la différence entre les deux pressions exercées par l’air, lorsqu’il se trouve aux températures 0°,001 et 0°. Cette différence est, d’après la règle de M. Gay-Lussac, 1/267 millième de la force élastique du gaz, ou à très-peu près 1/267 millième de la pression atmosphérique.\par
La pression atmosphérique fait équilibre à\par
10 mètres 4/100 de hauteur d’eau ; 1/267 millième de cette pression équivaut à 1/267000 . 10,40 mètres de hauteur d’eau.\par
Quant à l’accroissement de volume, il est, par supposition, 1/116+1/267 du volume primitif, c’est-à-dire du volume occupé par un kilogramme d’air à 0°, volume égal à 0,77 mètres cubes, eu égard à la pesanteur spécifique de l’air : ainsi donc le produit\par

\begin{center}
\noindent (1/116+1/267)  0,77  1/267000  10,40\par
\end{center}

\noindent exprimera la puissance motrice développée. Cette puissance est estimée ici en mètres cubes d’eau élevés de 1 mètre de hauteur.\par
Si l’on exécute les multiplications indiquées, on trouve pour valeur du produit 0,000000372.\par
Cherchons maintenant à évaluer la quantité de chaleur employée à donner ce résultat, c’est-à-dire la quantité de chaleur passée du corps A au corps B.\par
Le corps A fournit\par

\begin{enumerate}[itemsep=0pt,topsep=0pt,partopsep=0pt,parskip=0pt]
\item 1° la chaleur nécessaire pour porter la température de 1 kilogramme d’air de 0° à 0°,001 ;
\item 2° La quantité nécessaire pour maintenir à ce degré 0°,001 la température de l’air lorsqu’il éprouve une dilatation de 1/116+1/267
\end{enumerate}

\noindent La première de ces quantités de chaleur étant fort petite par rapport à la seconde, nous la négligerons. La seconde est, d’après le raisonnement de la pag. 44, égale à celle qui serait nécessaire pour accroître de 1° la température de 1 kilog. d’air soumis à la pression atmosphérique.\par
D’après les expériences de MM. Delaroche et Bérard sur la chaleur spécifique des gaz, celle de l’air est, à poids égaux, 0,267 de celle de l’eau. Si donc nous prenons pour unité de chaleur la quantité nécessaire pour élever de 1° un kilogramme d’eau, celle qui sera nécessaire pour élever de 1° un kilogramme d’air aura pour valeur 0,267. Ainsi la quantité de chaleur fournie par le corps A est\par

\begin{center}
0,267 unités.\par
\end{center}

\noindent C’est là la chaleur capable de produire 0,000000372 unités de puissance motrice par sa chute de 0°001 à 0°\par
Pour une chute mille fois aussi grande, pour une chute de 1°, la puissance motrice produite sera à très-peu près 1000 fois la première ou\par

\begin{center}
0,000372.\par
\end{center}

\noindent Si maintenant, au lieu de 0,267 unités de chaleur, nous employons 1000 unités, la puissance motrice produite sera donnée par la proportion\par

\begin{center}
0,267 : 0,000372 : : 1000 : \emph{x}, \\
d’où x = 372/267 = 1,395 unités.\par
\end{center}

\noindent Ainsi 1000 unités de chaleur, passant d’un corps maintenu à la température 1° à un autre corps maintenu à la température 0°, produiront, en agissant sur l’air,\par

\begin{center}
1,395 unités de puissance motrice.\par
\end{center}

\noindent Nous allons comparer ce résultat à celui que fournit l’action de la chaleur sur la vapeur d’eau.\par
Supposons 1 kilogramme d’eau liquide renfermée dans la capacité cylindrique \emph{abcd}, fig. 4, entre le fond \emph{ab} et le piston \emph{cd} ; supposons aussi l’existence des deux corps A, B, maintenus chacun à une température constante, celle de A étant élevée au-dessus de celle de B d’une quanti té fort petite. Figurons-nous maintenant les opérations suivantes :\par

\begin{enumerate}[itemsep=0pt,topsep=0pt,partopsep=0pt,parskip=0pt]
Contact de l’eau avec le corps A, passage du piston de la position \emph{cd} à la position \emph{ef} formation de la vapeur à la température du corps A pour remplir le vide auquel donne lieu l’extension de la capacité : nous supposerons la capacité \emph{abef} assez grande pour que toute l’eau y soit contenue à l’état de vapeur ;\item Éloignement du corps A, contact de la vapeur avec le corps B, précipitation d’une partie de cette vapeur, décroissement de sa force élastique, retour du piston de \emph{ef} en \emph{ab}, liquéfaction du reste de la vapeur par l’effet de la pression combinée avec le contact du corps B ;
\item Éloignement du corps B, nouveau contact de l’eau avec le corps A, retour de l’eau à la température de ce corps, renouvellement de la première période, ainsi de suite.
\end{enumerate}

\noindent La quantité de puissance motrice développée dans un cercle complet d’opérations est mesurée par le produit du volume de la vapeur multiplié par la différence entre les tensions qu’elle possède à la température du corps A et à celle du corps B.\par
Quant à la chaleur employée, c’est-à-dire transportée du corps A au corps B, c’est évidemment celle qui a été nécessaire pour transformer l’eau en vapeur, en négligeant toutefois la petite quantité nécessaire pour reporter l’eau liquide de la température du corps B à celle du corps A.\par
Supposons la température du corps A 100°, et celle du corps B 99° : la différence des tensions sera, d’après la table de M. Dalton, 26 millimètres de mercure ou 0,36 mètres de hauteur d’eau.\par
Le volume occupé par la vapeur est 1700 fois celui de l’eau. Si nous opérons sur un kilogramme, ce sera 1700 litres ou 1,700 mètres cubes.\par
Ainsi la puissance motrice développée a pour valeur le produit\par

\begin{center}
1,700×0,36 = 0,611 unités\par
\end{center}

\noindent de l’espèce dont nous avons fait usage précédemment.\par
La quantité de chaleur employée est la quantité nécessaire pour transformer en vapeur l’eau amenée déjà à 100°. Cette quantité est donnée par l’expérience : on l’a trouvée égale à 550°, ou, pour parler plus exactement, à 550 de nos unités de chaleur.\par
Ainsi 0,611 unités de puissance motrice résultent de l’emploi de 550 unités de chaleur.\par
La quantité de puissance motrice résultante de 1000 unités de chaleur sera donnée par la proportion\par

\begin{center}
550 : 0,611 : : 100 : \emph{x}, d’où x = 611/550 = 1,112.\par
\end{center}

\noindent Ainsi 1000 unités de chaleur transportées d’un corps maintenu à 100° à un autre corps maintenu à 99° produiront, en agissant sur la vapeur d’eau, 1,112 unités de puissance motrice.\par
Le nombre 1,112 diffère de 1/4 environ du nombre 1,395, trouvé précédemment pour la valeur de la puissance motrice développée par 1000 unités de chaleur agissant sur l’air, mais il faut observer que dans ce cas les températures des corps A et B étaient 1° et 0°, tandis qu’ici elles sont 100° et 99°. La différence est bien la même ; mais elle ne se trouve pas à la même hauteur dans l’échelle thermométrique. Il aurait fallu, pour faire une comparaison exacte, évaluer la puissance motrice développée par la vapeur formée à 1° et condensée à 0° ; il aurait fallu en outre pouvoir connoître la quantité de chaleur contenue dans la vapeur formée à 1°.\par
La loi due à MM. Clément et Désormes, et rapportée ci-dessus, pag. 66, nous fournit cette donnée. La chaleur constituante de la vapeur d’eau étant toujours la même, à quelque température que la vaporisation ait lieu, s’il faut 550 degrés de chaleur pour vaporiser l’eau déjà amenée à 100°, il en faudra 550 + 100 où 650 pour vaporiser le même poids d’eau prise a 0°.\par
En faisant usage de cette donnée et raisonnant d’ailleurs absolument comme nous l’avons fait pour l’eau à 100°, l’on trouve, ainsi qu’il est facile de s’en assurer,\par

\begin{center}
1,290\par
\end{center}

\noindent pour la puissance motrice développée par 1000 unités de chaleur agissant sur la vapeur d’eau entre 1° et 0°.\par
Ce nombre se rapproche plus que le premier de\par

\begin{center}
1,395\par
\end{center}

\noindent Il n’en diffère plus que de 1/13, erreur qui n’est pas hors des limites présumables, eu égard au grand nombre de données de diverses espèces dont nous avons été forcés de faire usage pour arriver à ce rapprochement. Ainsi se trouve vérifiée, dans un cas particulier, notre loi fondamentale\footnote{On trouve (\emph{Annales de chimie et de physique}, juillet 1818, pag. 294) dans un Mémoire de M. Petit une évaluation de la puissance motrice de la chaleur appliquée à l’air et à la vapeur d’eau. Cette évaluation conduit à attribuer a l’air atmosphérique un grand avantage ; mais elle est due a une méthode tout-à-fait incomplète de considérer l’action de la chaleur.}.\par
Nous examinerons un autre cas, celui où l’on fait agir la chaleur sur la vapeur d’alcool.\par
Les raisonnements sont ici absolument les mêmes que pour la vapeur d’eau ; les données seules changent.\par
L’alcool pur bout sous la pression ordinaire à 78°,7 centigrades. Un kilogramme absorbe, d’après MM. Delaroche et Bérard, 207 unités de chaleur pour se transformer en vapeur à cette même température 78°,7.\par
La tension de la vapeur d’alcool à 1° au-dessous du point d’ébullition se trouve diminuée de 1/25 ; elle est de 1/25 moindre que la pression atmosphérique (c’est du moins ce qui résulte des expériences de M. Bétancour, rapportées dans la seconde partie de l’\emph{Architecture hydraulique} de M. Prony, pag. 180, 195).\footnote{M. Dalton avait cru apercevoir que les vapeurs de divers liquides, à des distances thermométriques égales du point d’ébullition, jouissaient de tensions égales ; mais cette loi n’est pas rigoureusement exacte, elle n’est qu’approximative. Il en est de même de la loi de proportionnalité de la chaleur latente des vapeurs avec leurs densités. (V. Extraits d’un Mémoire de M. C. Desprets, \emph{Annales de physique et de chimie}, tome 16, p. 105, et tome 24, p. 323.) Les questions de ce genre se lient de près avec celles de la puissance motrice du feu. Tout récemment MM. H Davy et Faraday, après avoir fait de belles expériences sur la liquéfaction des gaz, par l’effet d’une pression considérable, ont cherché à reconnaître les changements de tension de ces gaz liquéfies pour de légers changements de température. Ils avaient en vue l’application des nouveaux liquides à la production de la puissance motrice. ( V. \emph{Annales de chimie et de physique}, janvier 1824, p. 80. ) D’après la théorie ci-dessus exposée, l’on peut prévoir que l’emploi de ces liquides ne présenterait pas d’avantages relativement à l’économie de la chaleur. Les avantages ne pourraient se rencontrer que dans la basse température, à laquelle il serait possible d’agir, et dans les sources où, par cette raison, il deviendrait possible de puiser le calorique : [\emph{sic}]}\par
Si l’on fait usage de ces données, l’on trouve qu’en agissant sur un kilogramme d’alcool aux températures 78°,7 et 77°,7, la puissance motrice développée serait 0,251 unités.\par
Elle résulte de l’emploi de 207 unités de chaleur. Pour 1000 unités il faut poser la proportion\par

\begin{center}
207 : 0,254 :: 1000 : \emph{x}, d’où \emph{x} = 1,230.\par
\end{center}

\noindent Ce nombre est un peu plus fort que 1,112 résultant de l’emploi de la vapeur d’eau aux températures 100° et 99°. Mais si l’on suppose la vapeur d’eau employée aux températures 78° et 77°, on trouve, en faisant usage de la loi de MM. Clément et Désormes, 1,212 pour la puissance motrice due à 1000 unités de chaleur. Ce dernier nombre se rapproche, comme on voit, beaucoup de 1,230 ; il n’en diffère que de 1/50.\par
Nous aurions désiré pouvoir faire d’autres rapprochements de ce genre, pouvoir calculer, par exemple, la puissance motrice développée par l’action de la chaleur sur des solides et des liquides, par la congélation de l’eau, etc. ; mais la physique actuelle nous refuse les données nécessaires\footnote{Celles qui nous manquent sont la force expansive qu’acquièrent les solides et les liquides par un accroissement donné de température, et la quantité de chaleur absorbée ou abandonnée dans les changements de volume de ces corps.}. La loi fondamentale que nous avions en vue de confirmer nous semblerait exiger cependant, pour être mise hors de doute, des vérifications nouvelles ; elle est assise sur la théorie de la chaleur telle qu’on la conçoit aujourd’hui, et, il faut l’avouer, cette base ne nous paraît pas d’une solidité inébranlable. Des expériences nouvelles pourraient seules décider la question ; en attendant, nous nous occuperons d’appliquer les idées théoriques ci-dessus exposées, en les regardant comme exactes, à l’examen des divers moyens proposés jusqu’à présent pour réaliser la puissance motrice de la chaleur.\par
L’on a quelquefois proposé de développer de la puissance motrice par l’action de la chaleur sur les corps solides. La manière de procéder qui se présente le plus naturellement à l’esprit est de fixer invariablement un corps solide, une barre métallique, par exemple, par l’une de ses extrémités ; d’attacher l’autre extrémité à une partie mobile de machine ; puis, par des réchauffements et des refroidissements successifs, de faire varier la longueur de la barre et de produire ainsi des mouvements quelconques. Essayons de juger si cette manière de développer la puissance motrice peut être avantageuse. Nous avons fait voir que le caractère du meilleur emploi de la chaleur à la production du mouvement était que tous les changements de température survenus dans les corps fussent dus à des changements de volume. Plus on approchera de remplir cette condition, et mieux la chaleur sera utilisée. Or, en opérant de la manière qui vient d’être décrite, on est bien loin de remplir la condition dont il s’agit ; aucun changement de température n’est dû ici à un changement de volume : tous sont dus aux contacts de corps diversement échauffés, au contact de la barre métallique, soit avec le corps chargé de lui fournir la chaleur, soit avec le corps chargé de la lui enlever.\par
Le seul moyen de remplir la condition prescrite serait d’agir sur le corps solide absolument, comme nous l’avons fait sur l’air dans les opérations décrites pag. 33. Mais il faudrait pour cela pouvoir produire par le seul changement de volume du corps solide des changements considérables de température, si du moins l’on voulait utiliser des chutes considérables du calorique : or c’est ce qui paraît impraticable. Plusieurs considérations conduisent en effet à penser que les changements opérés dans la température des corps solides ou liquides par l’effet de la compression et de la raréfaction seraient assez faibles.\par

\begin{enumerate}[itemsep=0pt,topsep=0pt,partopsep=0pt,parskip=0pt]
\item L’on observe souvent dans les machines (dans les machines à feu particulièrement) des pièces solides qui supportent des efforts très-considérables, tantôt dans un sens, tantôt dans l’autre, et quoique ces efforts soient quelquefois aussi grands que le permette la nature des substances mises en œuvre, les variations de température sont peu sensibles.
\item Dans l’action de frapper les médailles, dans celle du laminoir, de la filière, les métaux subissent la plus grande compression que nos moyens nous permettent de leur faire éprouver en employant les outils les plus durs et les plus résistans. Cependant l’élévation de température n’est pas considérable : si elle l’était, les pièces d’acier dont on fait usage dans ces opérations seraient bientôt détrempées.
\item On sait qu’il faudrait exercer sur les solides et les liquides un très-grand effort pour produire en eux une réduction de volume comparable à celle que leur fait éprouver le refroidissement (un refroidissement de 100° à 0°, par exemple). Or le refroidissement exige une suppression de calorique plus grande que ne l’exigerait la simple réduction de volume. Si cette réduction était produite par un moyen mécanique, la chaleur dégagée ne pourrait donc pas faire varier la température du corps d’autant de degrés que le fait le refroidissement. Elle nécessiterait cependant l’emploi d’une force à coup sûr très-considérable.
\end{enumerate}

\noindent Puisque les corps solides sont susceptibles de peu de changement de température par les changements de volume, puisque d’ailleurs la condition du meilleur emploi de la chaleur au développement de la puissance motrice est précisément que tout changement de température soit dû à un changement de volume, les corps solides paraissent peu propres à réaliser cette puissance.\par
Les corps liquides sont absolument dans le même cas ; les mêmes raisons peuvent être données pour rejeter leur emploi\footnote{Des expériences récentes de M. Oersted sur la compressibilité de l’eau ont fait voir que, pour une pression de 5 atmosphères, la température de ce liquide n’éprouvait pas de changement appréciable. (Voy. \emph{Annales de physique et de chimie}, février 1823, p. 192.)}.\par
Nous ne parlons pas ici des difficultés pratiques : elles seraient sans nombre. Les mouvements produits par la dilatation et la compression des corps solides ou liquides ne pourraient être que fort petits ; on se verrait forcé, pour leur donner de l’extension, de faire usage de mécanismes compliqués ; il faudrait employer des matériaux de la plus grande force pour transmettre des pressions énormes ; enfin les opérations successives s’exécuteraient avec beaucoup de lenteur, comparées à celles de la machine à feu ordinaire, de sorte que des appareils de grandes dimensions et d’un prix considérable ne produiraient en somme que de médiocres effets.\par
Les fluides élastiques, gaz ou vapeurs, sont les véritables instruments appropriés au développement de la puissance motrice de chaleur. Ils réunissent toutes les conditions nécessaires pour bien remplir cet emploi. Ils sont faciles à comprimer, ils jouissent de la faculté de se distendre presque infiniment ; les variations de volume occasionent chez eux de grands changements de température ; enfin ils sont très-mobiles, faciles à échauffer et à refroidir promptement, faciles à transporter d’un lieu à un autre, ce qui donne la faculté de leur faire produire rapidement les effets que l’on attend d’eux.\par
Ou peut aisément concevoir une foule de machines propres à développer la puissance motrice de la chaleur par l’emploi des fluides élastiques ; mais de quelque manière que l’on s’y prenne, il ne faut pas perdre de vue les principes suivans :\par

\begin{enumerate}[itemsep=0pt,topsep=0pt,partopsep=0pt,parskip=0pt]
\item La température du fluide doit être portée d’abord au degré le plus élevé possible, afin d’obtenir une grande chute de calorique, et par suite une grande production de puissance motrice.
\item Par la même raison le refroidissement doit être porté aussi loin que possible.
\item Il faut faire en sorte que le passage du fluide élastique de la température la plus élevée à la température la plus basse soit dû à l’extension de volume, c’est-à-dire il faut faire en sorte que le refroidissement du gaz arrive spontanément par l’effet de la raréfaction.
\end{enumerate}

\noindent Les bornes de la température à laquelle il est possible de faire arriver d’abord le fluide ne sont que les bornes de la température produite par la combustion ; elles sont très-éloignées.\par
Les bornes du refroidissement se rencontrent dans la température des corps les plus froids dont on puisse disposer facilement et en grande abondance : ces corps sont ordinairement les eaux du lieu où l’on se trouve.\par
Quant à la troisième condition, elle apporte des difficultés à la réalisation de la puissance motrice de la chaleur lorsqu’il s’agit de mettre à profit de grandes différences de température, d’utiliser de grandes chutes du calorique. En effet, il faut alors que le gaz, par l’effet de sa raréfaction, passe d’une température très-élevée à une température très-basse, ce qui exige un grand changement de volume et de densité, ce qui exige que le gaz soit pris d’abord sous une pression très-forte, ou qu’il acquière, par l’effet de sa dilatation, un volume énorme, conditions l’une et l’autre difficiles à remplir. La première nécessite l’emploi de vaisseaux très-solides pour contenir le gaz à la fois sous une forte pression et à une haute température ; la seconde nécessite l’emploi de vaisseaux d’une dimension très-considérable.\par
Ce sont là en effet les principaux obstacles qui s’opposent à ce que l’on mette à profit, dans les machines à vapeur, une grande portion de la puissance motrice de la chaleur. On est forcé de se borner à utiliser une faible chute du calorique, tandis que la combustion du charbon fournit les moyens de s’en procurer une très-grande.\par
Il est rare que dans les machines à vapeur on donne naissance au fluide élastique sous une pression supérieure à 6 atmosphères, pression correspondante à environ 160 degrés centigrades, et il est rare que la condensation s’opère à une température fort au-dessous de 40° ; la chute de calorique de 160° à 40° est 120°, tandis qu’on peut se procurer, par la combustion, une chute de 1000 à 2000 degrés.\par
Pour mieux faire concevoir ceci, nous rappellerons ce que nous avons désigné par chute de calorique : c’est le passage de la chaleur d’un corps A, où la température est élevée, à un autre B, où elle est plus basse. Nous disons que la chute du calorique est de 100 degrés ou de 1000 degrés lorsque la différence de température entre les corps A et B est 100° ou 1000°.\par
Dans une machine à vapeur qui travaille sous la pression 6 atmosphères, la température de la chaudière est 160 degrés : c’est là le corps A ; il est entretenu, par le contact du foyer, à la température constante 160° et fournit continuellement la chaleur nécessaire à la formation de la vapeur.\par
Le condenseur est le corps B ; il est entretenu, au moyen d’un courant d’eau froide, à la température à peu près constante de 40 degrés ; il absorbe continuellement le calorique qui lui est apporté du corps A par la vapeur.\par
La différence de température entre ces deux corps est 160° — 40° ou 120° : c’est pourquoi nous disons que la chute du calorique est ici 120°.\par
Le charbon étant capable de produire par sa combustion une température supérieure à 1000°, et l’eau froide dont on dispose le plus ordinairement dans nos climats étant à 10° environ, l’on peut se procurer facilement une chute de calorique de 1000°, chute dont 120° seulement sont utilisés par les machines à vapeur. Encore ces 120° ne sont-ils pas mis entièrement à profit. Il se fait toujours des pertes considérables dues à des rétablissements inutiles d’équilibre dans le calorique.\par
Il est aisé d’apercevoir maintenant quelles sont les causes de l’avantage des machines dites à haute pression sur les machines à pression plus basse : \emph{cet avantage réside essentiellement dans la faculté de rendre utile une plus grande chute du calorique}. La vapeur prenant naissance sous une pression plus forte se trouve aussi à une température plus élevée, et comme d’ailleurs la température de la condensation reste toujours à peu près la même ; la chute du calorique est évidemment plus considérable.\par
Mais pour tirer des machines à haute pression des résultats vraiment avantageux, il faut que la chute du calorique y soit mise à profit le mieux possible. Il ne suffit pas que la vapeur prenne naissance à une température élevée : il faut encore que par l’extension de son volume elle arrive à une température assez basse. Le caractère d’une bonne machine à vapeur doit donc être non seulement d’employer la vapeur sous une forte pression, mais de \emph{l’employer sous des pressions successives très-variables, très-différentes les unes des autres, et progressivement décroissantes}\footnote{ \noindent Ce principe, véritable fondement de la théorie des machines à vapeur, a été développé avec beaucoup de clarté, par M. Clément, dans un mémoire présenté à l’Académie des sciences, il y a quelques années. Ce mémoire n’a jamais été imprimé, mais j’en ai dû la connaissance à la complaisance de l’auteur. Non seulement le principe y est établi, mais il y est appliqué aux divers systèmes de machines à vapeur actuellement en usage ; la puissance motrice de chacune y est évaluée par le secours de la loi citée (pag. 66), et comparée aux résultats de l’expérience.\par
 Le principe dont il est ici question est tellement mal connu ou mal apprécié, que récemment M. Perkins, célèbre mécanicien de Londres, a construit une machine où la vapeur formée sous la pression 55 atmosphères, pression jusque alors inusitée, ne reçoit presque aucune extension de volume, comme on peut s’en convaincre par la plus légère connaissance de cette machine. Elle est composée d’un seul cylindre, de dimensions fort petites, qui, à chaque pulsation, se remplit entièrement de vapeur formée sous la pression de 35 atmosphères. La vapeur ne produit aucun effet par l’extension de son volume, car on ne lui présente aucune capacité où cette extension puisse avoir lieu ; on la condense aussitôt après sa sortie du petit cylindre. Elle travaille donc seulement sous une pression de 55 atmosphères, et non, comme l’exigerait son bon emploi, sous des pressions progressivement décroissantes. Aussi la machine de M. Perkins ne paraît-elle pas réaliser les espérances qu’elle avait d’abord fait concevoir. On avait prétendu que l’économie de charbon produite par cette machine était des 9/10 sur les bonnes machines de Watt, et que l’on y rencontrait encore d’autres avantages. (V. \emph{Ann. de physique et de chimie}, avril 1823, pag. 429) Ces assertions ne se sont pas vérifiées. La machine de M. Perkins n’en est pas moins une invention précieuse en en qu’elle a montre la possibilité de faire usage de la vapeur sons des pressions beaucoup plus élevées qu’on ne l’avait fait jusque alors, et parce qu’elle peut conduire, étant habilement modifiée, à des résultats vraiment utiles.\par
 Watt, à qui l’on doit presque toutes les grandes améliorations des machines à vapeur, et qui a porté ces machines à un état de perfection aujourd’hui difficile à dépasser, Watt est aussi le premier qui ait employé la vapeur sous des pressions progressivement décroissantes. Dans beaucoup de cas, il suspendait l’introduction de la vapeur dans le cylindre, à moitié, au tiers, au quart, de la course du piston, qui s’achevait ainsi sous une pression de plus en plus faible. Les premières machines agissant sur ce principe datent de 1778. Watt en avait conçu l’idée dès 1769, et prit patente pour cet objet en 1782.\par
 Voici une table qui se trouvait annexée à la patente de Watt. Il supposait la vapeur introduite dans le cylindre pendant le premier quart de la course du piston ; puis, divisant cette course en vingt parties, il calculait ainsi la pression moyenne :\par
   \includegraphics[width=\linewidth,]{carnot1824_puissance-motrice-feu/portions-pression.jpg}
 \noindent Sur quoi il remarquait que la pression moyenne est plus de moitié de la pression première ; qu’ainsi, en employant une quantité de vapeur égale au quart, il produisait un effet plus que moitié.\par
 Watt supposait ici que la vapeur observe dans sa dilatation la loi de Mariotte : ce qu’il ne devait pas regarder comme exact, parce que, d’une part, le fluide élastique, en se dilatant, s’abaisse de température, et que, de l’autre, rien ne prouvait qu’il ne se condense pas une partie de ce fluide par l’effet de son expansion. Watt aurait dû aussi avoir égard à la force nécessaire pour expulser la vapeur qui reste après la condensation, et qui se trouve en quantité d’autant plus grande que l’extension du volume a été poussée plus loin. Le docteur Robinson avait ajouté au travail de Watt une formule simple pour calculer l’effet de l’expansion de la vapeur ; mais cette formule se trouve entachée des mêmes vices que nous venons de signaler. Elle a été néanmoins utile aux constructeurs en leur fournissant une donnée approximative à peu près suffisante pour la pratique. Nous avons jugé utile de rappeler ces faits parce qu’ils sont peu connus, surtout en France. On y construit des machines sur les modèles des inventeurs, mais on apprécie mai les motifs qui ont guidé ceux-ci dans l’origine. L’oubli de ces motifs a conduit souvent dans des fautes graves. Des machines originairement bien conçues se sont détériorées entre les mains de constructeurs inhabiles, qui, voulant y introduire des perfectionnements de peu d’importance, ont négligé les considérations capitales qu’ils ne savaient pas apprécier.
 }.\par
Pour faire sentir, en quelque sorte, \emph{a posteriori}, l’avantage des machines à haute pression, supposons de la vapeur formée sous la pression atmosphérique et introduite dans la capacité cylindrique \emph{abcd} (fig. 5 ) sous le piston \emph{cd}, qui joignait d’abord le fonds ab : la vapeur, après avoir fait mouvoir le piston de \emph{ab} en \emph{cd}, poursuivra ultérieurement ses effets d’une manière quelconque dont nous ne nous occupons pas.\par
Imaginons que l’on force le piston parvenu en \emph{cd} à s’abaisser en \emph{ef}, sans permettre à la vapeur de s’échapper, ni de perdre aucune portion de son calorique. Elle sera refoulée dans l’espace \emph{abef}, et augmentera à la fois de densité, de force élastique et de température.\par
Si la vapeur, au lieu de prendre naissance sous la pression atmosphérique, eût pris naissance précisément à l’état où elle se trouve étant refoulée en \emph{abef}, et qu’après avoir fait mouvoir par son introduction dans le cylindre le piston de \emph{ab} en \emph{ef}, elle l’eût poussé par le seul effet de son extension de volume de \emph{ef} en \emph{cd}, la puissance motrice produite eût été plus considérable que dans le premier cas. En effet, le mouvement du piston, égal en amplitude, aurait eu lieu sous l’effort d’une pression plus grande, quoique variable, quoique progressivement décroissante.\par
La vapeur n’eût cependant exigé pour sa formation qu’une quantité de calorique précisément égale : seulement ce calorique eût été pris à une température plus élevée.\par
C’est d’après des considérations de ce genre qu’ont été établies les machines à deux cylindres, machines inventées par M. Hornblower, perfectionnées par M. Woolf, et qui passent pour les plus avantageuses relativement à l’économie du combustible. Elles sont composées d’un petit cylindre qui, à chaque pulsation, se remplit plus ou moins de vapeur (souvent entièrement), et d’un second cylindre auquel on donne ordinairement une capacité quadruple de celle du premier, et qui ne reçoit d’autre vapeur que celle qui a déjà agi dans le premier cylindre. Ainsi la vapeur, au terme de son action, a au moins quadruplé de volume. Du second cylindre elle est portée directement dans le condenseur ; mais on conçoit qu’elle pourrait être portée dans un troisième cylindre quadruple du second et où son volume deviendrait 16 fois le volume primitif. Le principal obstacle qui s’oppose à l’emploi d’un troisième cylindre de ce genre est la capacité qu’il faudrait lui donner, et les grandes dimensions qu’il faudrait faire acquérir aux ouvertures destinées à livrer passage à la vapeur\footnote{ \noindent L’avantage de deux cylindres substitués à un seul est facile à apercevoir. Dans un seul cylindre, l’impulsion du piston serait excessivement variable du commencement à la fin de la course. Il faudrait que toutes les pièces destinées à transmettre le mouvement fussent d’une force suffisante pour résister à la première impulsion, et parfaitement assemblées entre elles pour éviter des mouvements brusques dont elles auraient beaucoup à souffrir, qui même les auraient bientôt détruites. Ce serait surtout sur le balancier, sur les supports, sur la bielle, sur la manivelle, sur les premières roues dentées, que l’inégalité d’impulsion se ferait sentir et produirait les effets les plus nuisibles. Il serait nécessaire en outre que le cylindre à vapeur fût à la fois d’une force suffisante pour supporter la pression la plus élevée, et d’une capacité assez considérable pour contenir la vapeur après son extension de volume, tandis qu’en faisant usage de deux cylindres successifs, il suffit de donner au premier la force avec une capacité médiocre, ce qui est chose facile, et au dernier les grandes dimensions avec une force médiocre.\par
 Les machines à deux cylindres, quoique conçues sur d’assez bons principes, se trouvent souvent loin de produire les résultats avantageux que l’on aurait droit d’attendre d’elles : cela tient surtout à ce que les dimensions des diverses parties de ces machines sont difficiles à bien régler, et qu’elles se trouvent rarement dans un juste rapport les unes avec les autres. On manque de bons modèles pour la construction des machines à deux cylindres, tandis que l’on en possède d’excellents pour la construction des machines du système de Watt. De là vient la diversité que l’on observe dans les effets des unes et la presque uniformité que l’on observe dans ceux des autres.
 }. Nous n’en dirons pas davantage sur ce sujet, notre objet n’étant pas d’entrer ici dans les détails de construction des machines à feu : ces détails réclameraient un ouvrage qui en traitât spécialement, et qui n’existe pas encore, du moins en France\footnote{On trouve dans l’ouvrage intitulé \emph{de la Richesse minérale}, par M. Héron de Villefosse, 3e vol., pag. 50 et suivantes, une bonne description des machines à vapeur actuellement en usage dans l’exploitation des mines. En Angleterre, on a traité des machines a vapeur d’une manière assez complète dans l’\emph{Encyclopédie britannique}. Quelques-unes des données dont nous nous servons ici sont tirées de ce dernier ouvrage.}.\par
Si la distension de la vapeur est bornée principalement par les dimensions des vaisseaux où elle doit se dilater, le degré de condensation auquel il est possible de l’employer d’abord n’est limité que par la résistance des vaisseaux où elle prend naissance, c’est-à-dire des chaudières. Sous ce rapport on est loin d’avoir atteint la limite du mieux possible, la disposition des chaudières généralement en usage est tout à fait vicieuse, quoique la tension de la vapeur y soit rarement portée au-delà de 4 à 6 atmosphères ; elles éclatent souvent et ont causé des accidents graves. Il serait sans doute très-possible d’éviter de pareils accidents et de porter cependant la vapeur à des tensions beaucoup plus fortes qu’on ne le fait généralement.\par
Outre les machines à haute pression à deux cylindres et dont nous avons parlé, il existe encore des machines à haute pression à un seul cylindre. La plupart de ces dernières ont été construites par deux habiles ingénieurs anglais, MM. Trevetick et Vivian. Elles emploient la vapeur sous une pression très-élevée, quelquefois 8 à 10 atmosphères, maïs elles sont sans condenseur. La vapeur, après avoir été introduite dans le cylindre, y reçoit une certaine extension de volume, mais conserve toujours une pression plus élevée que la pression atmosphérique. Lorsqu’elle a rempli son office, on la rejette dans l’atmosphère. Il est évident que cette façon d’agir équivaut tout-à-fait, sous le rapport de la puissance motrice produite, à condenser la vapeur à 100°, et que l’on perd une partie de l’effet utile ; mais les machines qui opèrent ainsi dispensent de condenseur et de pompe à air. Elles sont moins coûteuses que les autres, moins compliquées, elles occupent moins de place, et peuvent s’employer dans les lieux où l’on ne dispose pas d’un courant d’eau froide suffisant pour opérer la condensation. Elles sont là d’un avantage inappréciable, puisque l’on ne peut pas les remplacer par d’autres. Ces machines sont principalement employées, en Angleterre, à mouvoir des chariots destinés au transport de la houille sur des chemins de fer établis, soit dans l’intérieur des mines, soit à ciel ouvert.\par
Il nous reste à faire quelques réflexions sur l’emploi des gaz permanents et des vapeurs autres que celle de l’eau au développement de la puissance motrice du feu.\par
L’on a essayé à diverses reprises de faire agir la chaleur sur l’air atmosphérique pour donner naissance à la puissance motrice. Ce gaz présente, relativement à la vapeur d’eau, des avantages et des inconvénients que nous allons examiner.\par

\begin{enumerate}[itemsep=0pt,topsep=0pt,partopsep=0pt,parskip=0pt]
\item Il présente, relativement à la vapeur d’eau, un avantage notable en ce qu’ayant à volume égal une capacité pour la chaleur beaucoup moindre, il se refroidirait davantage par une extension semblable au volume. (Ce fait est prouvé par ce que nous avons dit précédemment). Or on a vu de quelle importance il était d’occasioner, par les changements de volume, les plus grands changements possibles dans la température.
\item La vapeur d’eau ne peut être formée que par l’intermédiaire d’une chaudière, tandis que l’air atmosphérique pourrait être échauffe immédiatement par une combustion exécutée dans son sein. On éviterait ainsi une perte considérable, non seulement dans la quantité de chaleur, mais encore dans son degré thermométrique. Cet avantage appartient exclusivement à l’air atmosphérique ; les autres gaz n’en jouissent pas ; ils seraient même plus difficiles à échauffer que la vapeur d’eau.
\item Afin de pouvoir donner à l’air une grande extension de volume, afin de produire par cette extension un grand changement de température, il serait nécessaire de le prendre d’abord sous une pression assez élevée : il faudrait donc le comprimer par une pompe pneumatique ou par tout autre moyen avant de l’échauffer. Cette opération exigerait un appareil particulier, appareil qui n’existe pas dans les machines à vapeur. Dans celles-ci, l’eau est à l’état liquide lorsqu’on la fait pénétrer dans la chaudière ; elle n’exige, pour y être introduite, qu’une pompe foulante de petites dimensions.
\item Le refroidissement de la vapeur par le contact du corps réfrigérant est bien plus prompt et bien plus facile que ne peut l’être celui de l’air. À la vérité, on aurait la ressource de rejeter celui-ci dans l’atmosphère, ce qui aurait en outre l’avantage d’éviter l’emploi d’un réfrigérant dont on ne dispose pas toujours, mais il faudrait pour cela que l’extension de volume de l’air ne l’eût pas fait arriver à une pression moindre que la pression atmosphérique.
\item Un des inconvénients les plus graves de la vapeur est de ne pouvoir pas être prise à de hautes températures sans nécessiter l’emploi de vaisseaux d’une force extraordinaire. Il n’en est pas de même de l’air, pour lequel il n’existe pas de rapport nécessaire entre la force élastique et la température. L’air semblerait donc plus propre que la vapeur à réaliser la puissance motrice des chutes du calorique dans les degrés élevés ; peut-être dans les degrés inférieurs la vapeur d’eau est-elle plus convenable. On concevrait même la possibilité de faire agir la même chaleur successivement sur l’air et sur la vapeur d’eau. Il suffirait de laisser à l’air, après son emploi, une température élevée, et, au lieu de le rejeter immédiatement dans l’atmosphère, de lui faire envelopper une chaudière à vapeur, comme s’il sortait immédiatement d’un foyer.
\end{enumerate}

\noindent L’emploi de l’air atmosphérique au développement de la puissance motrice de la chaleur présenterait, dans la pratique, des difficultés très-grandes, mais peut-être pas insurmontables ; si on parvenait à les vaincre, il offrirait sans doute une supériorité remarquable sur la vapeur d’eau\footnote{ \noindent Parmi les tentatives faites pour développer la puissance motrice du feu par l’intermédiaire de l’air atmosphérique, on doit distinguer celles de MM. Niepce, qui ont eu lieu en France il y a plusieurs années, au moyen d’un appareil nommé par les inventeurs \emph{pyréolophore}. Voici en quoi consistait à peu près cet appareil : c’était un cylindre, muni d’un piston, où l’air atmosphérique était introduit à la densité ordinaire. L’on y projetait une matière très-combustible, réduite à un grand état de ténuité, et qui restait un moment en suspension dans l’air, puis on y mettait le feu. L’inflammation produisait à peu près le même effet que si le fluide élastique eût été un mélange d’air et de gaz combustible, d’air et d’hydrogène carboné, par exemple ; il y avait une sorte d’explosion et une dilatation subite du fluide élastique, dilatation que l’on mettait à profit en la faisant agir tout entière contre le piston. Celui-ci prenait un mouvement d’une amplitude quelconque, et la puissance motrice se trouvait ainsi réalisée. Rien n’empêchait ensuite de renouveler l’air et de recommencer une opération semblable à la première.\par
 Cette machine, fort ingénieuse et intéressante surtout par la nouveauté de son principe, péchait par un point capital. La matière dont on faisait usage comme combustible (c’était la poussière de lycopode, employée à produire des flammes sur nos théâtres) était trop chère pour que tout avantage ne disparût pas par cette cause ; et malheureusement il était difficile d’employer un combustible de prix modéré, car il fallait un corps en poudre très-fine, dont l’inflammation fût prompte, facile à propager, et laissât peu ou point de cendres.\par
 Au lieu d’opérer comme le faisaient MM. Niepce, il nous eût semblé préférable de comprimer l’air par des pompes pneumatiques, de lui faire traverser un foyer parfaitement clos, et dans lequel on eût introduit le combustible en petites portions par un mécanisme facile à concevoir ; de lui faire développer son action dans un cylindre à piston ou dans toute autre capacité extensible ; de le rejeter enfin dans l’atmosphère, ou même de le faire passer sous une chaudière à vapeur, afin d’utiliser la température qui lui serait restée.\par
 Les principales difficultés que l’on eût rencontrées dans ce mode d’opération eussent été de renfermer le foyer dans une enveloppe d’une solidité suffisante, d’entretenir cependant la combustion à un état convenable, de maintenir les diverses parties de l’appareil à une température modérée, et d’empêcher les dégradations rapides du cylindre et du piston : nous ne croyons pas ces difficultés insurmontables.\par
 Il a été fait, dit-on, tout récemment en Angleterre des essais heureux sur le développement de la puissance motrice par l’action de la chaleur sur l’air atmosphérique. Nous ignorons entièrement en quoi ces essais ont consisté, si toutefois ils sont réels.
 }.\par
Quant aux autres gaz permanents, ils doivent être absolument rejetés : ils ont tous les inconvénients de l’air atmosphérique, sans présenter aucun de ses avantages.\par
On peut en dire autant des vapeurs autres que celle de l’eau comparées à cette dernière.\par
S’il se rencontrait un corps liquide abondant, qui se vaporisât à une température plus élevée que l’eau, dont la vapeur eût sous le même volume une chaleur spécifique moindre, qui n’attaquât pas les métaux employés à la construction des machines, il mériterait sans doute la préférence ; mais la nature ne nous offre pas un pareil corps.\par
On a proposé quelquefois l’emploi de la vapeur d’alcool, on a même construit des machines dont le but était de rendre cet emploi possible en évitant de mêler les vapeurs avec l’eau de condensation, c’est-à-dire en appliquant le corps froid extérieurement, au lieu de l’introduire dans la machine. On croyait apercevoir dans la vapeur d’alcool un avantage remarquable en ce qu’elle possède une tension plus forte que la vapeur d’eau à égale température. Nous ne pouvons voir là qu’un nouvel obstacle à surmonter. Le principal défaut de la vapeur d’eau est sa tension excessive à une température élevée : or ce défaut existe à plus forte raison dans la vapeur d’alcool. Quant à l’avantage relatif à une plus grande production de puissance motrice, avantage que l’on croyait devoir rencontrer, nous savons, par les principes exposés ci-dessus, qu’il est imaginaire.\par
C’est donc sur l’emploi de la vapeur d’eau et de l’air atmosphérique que doivent porter les tentatives ultérieures de perfectionnement des machines à feu ; c’est à utiliser, par le moyen de ces agents, les plus grandes chutes possibles du calorique, que doivent être dirigés tous les efforts.\par
Nous terminerons en faisant apercevoir combien on est loin d’avoir réalisé, par les moyens connus jusqu’à présent, toute la puissance motrice des combustibles.\par
Un kilogramme de charbon brûlé dans le calorimètre fournit une quantité de chaleur capable d’élever d’un degré centigrade 7000 kilogrammes d’eau environ, c’est-à-dire qu’il fournit 7000 unités de chaleur d’après la définition donnée (pag. 81 ) de ces unités.\par
La plus grande chute réalisable du calorique est mesurée par la différence entre la température produite par la combustion et celle des corps employés au refroidissement. Il est difficile d’apercevoir à la température de la combustion d’autres limites que celles où la combinaison entre l’oxigène et le combustible peut s’effectuer. Admettons cependant que 1000° soient cette limite, et nous nous tiendrons certainement au-dessous de la vérité. Quant à la température du réfrigérant, supposons-la 0°.\par
Nous avons évalué approximativement, p. 84, la quantité de puissance motrice que développent 1000 unités de chaleur du degré 100 au degré 99 : nous l’avons trouvée 1,12 unité de puissance égales chacune à 1 mètre d’eau élevé d’un mètre de hauteur.\par
Si la puissance motrice était proportionnelle à la chute du calorique, si elle était la même pour chaque degré thermométrique, rien ne serait plus facile que de l’estimer de 1000° à 0° : elle aurait pour valeur.\par

\begin{center}
1,12 × 1000 = 1120.\par
\end{center}

\noindent Mais comme cette loi n’est qu’approximative et qu’elle s’écarte peut-être beaucoup de la vérité dans les degrés élevés, nous ne pouvons faire qu’une évaluation tout-à-fait grossière : nous supposerons le nombre 1120 réduit à moitié, c’est-à-dire à 560.\par
Puisque un kilogramme de charbon produit 7000 unités de chaleur et que le nombre 560 est relatif à 1000 unités, il faut le multiplier par 7, ce qui donne\par

\begin{center}
7×560 = 3920.\par
\end{center}

\noindent Voilà la puissance motrice d’un kilogramme de charbon.\par
Pour comparer ce résultat théorique avec les résultats d’expérience, examinons combien un kilogramme de charbon développe réellement de puissance motrice dans les meilleures machines à feu connues.\par
Les machines qui ont présenté jusqu’ici les résultats les plus avantageux sont les grandes machines à deux cylindres employées à l’épuisement des mines d’étain et cuivre de Cornwall. Voici les meilleurs produits qu’elles aient jamais fourni.\par
56 millions de livres d’eau ont été élevées d’un pied anglais par boisseau de charbon brûlé (le boisseau pèse 88 livres ). Cet effet équivaut à élever, par kilogramme de charbon, 195 mètres cubes d’eau à un mètre de hauteur, à produire par conséquent 195 unités de puissance motrice par kilogramme de charbon brûlé\footnote{ \noindent Le résultat que nous rapportons ici a été fourni par une machine dont le grand cylindre a pour dimensions 45 pouces de diamètre et 7 pieds de course ; elle est employée à l’épuisement d’une des mines de Cornwall, nommée Wheal Abraham. Ce résultat doit être considéré en quelque sorte comme une exception, car il n’a été que momentané et ne s’est soutenu que pendant un seul mois. Le produit de 30 millions de livres élevées de un pied anglais par boisseau de charbon de 88 livres est regardé généralement comme un excellent résultat des machines à vapeur ; il est quelquefois atteint par les machines du système de Watt, mais bien rarement dépassé. Ce dernier produit revient, en mesures françaises, à 104000 kilogrammes élevés à un mètre de hauteur par kilogramme de charbon brûlé.\par
 D’après ce que l’on entend ordinairement par force d’un cheval, dans l’évaluation des effets des machines à vapeur, une machine de 10 chevaux doit élever par seconde 10 X 75 kilogr., ou 750 kilogr., à un mètre de hauteur, ou bien, par heure, 750 X 3600 = 2700000 kilogrammes à un mètre. Si nous supposons que chaque kilog. de charbon élève à cette hauteur 104000 kilog., il faudra, pour connaître le charbon brûlé en une heure par notre machine de 10 chevaux, diviser 2700000 par 104000, ce qui donne 2700/104 = 26 kilog. Or il est bien rare de voir une machine de 10 chevaux consommer moins de 26 kilog. de charbon par heure.
 }.\par
195 unités ne sont que le vingtième de 3920, maximum théorique : par conséquent 1/20 seulement de la puissance motrice du combustible a été utilisée.\par
Nous avons cependant choisi notre exemple parmi les meilleures machines à vapeur connues.\par
La plupart des autres leur sont bien inférieures. L’ancienne machine de Chaillot, par exemple, élève 20 mètres cubes d’eau à 33 mètres pour 30 kilogrammes de charbon brûlé, ce qui revient à 23 unités de puissance motrice par kilogramme, résultat neuf fois moindre que celui cité ci-dessus, et 180 fois moindre que le maximum théorique.\par
On ne doit pas se flatter de mettre jamais à profit, dans la pratique, toute la puissance motrice des combustibles. Les tentatives que l’on ferait pour approcher de ce résultat seraient même plus nuisibles qu’utiles, si elles faisaient négliger d’autres considérations importantes. L’économie du combustible n’est qu’une des conditions à remplir par les machines à feu ; dans beaucoup de circonstances, elle n’est que secondaire, elle doit souvent céder le pas à la sûreté, à la solidité, à la durée de la machine, au peu de place qu’il faut lui faire occuper, au peu de frais de son établissement, etc. Savoir apprécier, dans chaque cas, à leur juste valeur, les considérations de convenance et d’économie qui peuvent se présenter, savoir discerner les plus importantes de celles qui sont seulement accessoires, les balancer toutes convenablement entre elles, afin de parvenir par les moyens les plus faciles au meilleur résultat, tel doit être le principal talent de l’homme appelé à diriger, à coordonner entre eux les travaux de ses semblables, à les faire concourir vers un but utile de quelque genre qu’il soit.\par

\trailer{FIN.}
 


% at least one empty page at end (for booklet couv)
\ifbooklet
  \pagestyle{empty}
  \clearpage
  % 2 empty pages maybe needed for 4e cover
  \ifnum\modulo{\value{page}}{4}=0 \hbox{}\newpage\hbox{}\newpage\fi
  \ifnum\modulo{\value{page}}{4}=1 \hbox{}\newpage\hbox{}\newpage\fi


  \hbox{}\newpage
  \ifodd\value{page}\hbox{}\newpage\fi
  {\centering\color{rubric}\bfseries\noindent\large
    Hurlus ? Qu’est-ce.\par
    \bigskip
  }
  \noindent Des bouquinistes électroniques, pour du texte libre à participations libres,
  téléchargeable gratuitement sur \href{https://hurlus.fr}{\dotuline{hurlus.fr}}.\par
  \bigskip
  \noindent Cette brochure a été produite par des éditeurs bénévoles.
  Elle n’est pas faite pour être possédée, mais pour être lue, et puis donnée.
  En page de garde, on peut ajouter une date, un lieu, un nom ;
  comme une fiche de bibliothèque en papier,
  pour suivre le voyage du texte. Qui sait, un jour, il vous reviendra ?
  \par

  Ce texte a été choisi parce qu’une personne l’a aimé,
  ou haï, elle a pensé qu’il partipait à la formation de notre présent ;
  sans le souci de plaire, vendre, ou militer pour une cause.
  \par

  L’édition électronique est soigneuse, tant sur la technique
  que sur l’établissement du texte ; mais sans aucune prétention scolaire, au contraire.
  Le but est de s’adresser à tous, sans distinction de science ou de diplôme.
  \par

  Cet exemplaire en papier a été tiré sur une imprimante personnelle
   ou une photocopieuse. Tout le monde peut le faire.
  Il suffit de
  télécharger un fichier sur \href{https://hurlus.fr}{\dotuline{hurlus.fr}},
  d’imprimer, et agrafer (puis lire et donner).\par

  \bigskip

  \noindent PS : Les hurlus furent aussi des rebelles protestants qui cassaient les statues dans les églises catholiques. En 1566 démarra la révolte des gueux dans le pays de Lille. L’insurrection enflamma la région jusqu’à Anvers où les gueux de mer bloquèrent les bateaux espagnols.
  Ce fut une rare guerre de libération dont naquit un pays toujours libre : les Pays-Bas.
  En plat pays francophone, par contre, restèrent des bandes de huguenots, les hurlus, progressivement réprimés par la très catholique Espagne.
  Cette mémoire d’une défaite est éteinte, rallumons-la. Sortons les livres du culte universitaire, débusquons les idoles de l’époque, pour les démonter.
\fi

\end{document}
