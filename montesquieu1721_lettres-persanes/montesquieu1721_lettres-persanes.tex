%%%%%%%%%%%%%%%%%%%%%%%%%%%%%%%%%
% LaTeX model https://hurlus.fr %
%%%%%%%%%%%%%%%%%%%%%%%%%%%%%%%%%

% Needed before document class
\RequirePackage{pdftexcmds} % needed for tests expressions
\RequirePackage{fix-cm} % correct units

% Define mode
\def\mode{a4}

\newif\ifaiv % a4
\newif\ifav % a5
\newif\ifbooklet % booklet
\newif\ifcover % cover for booklet

\ifnum \strcmp{\mode}{cover}=0
  \covertrue
\else\ifnum \strcmp{\mode}{booklet}=0
  \booklettrue
\else\ifnum \strcmp{\mode}{a5}=0
  \avtrue
\else
  \aivtrue
\fi\fi\fi

\ifbooklet % do not enclose with {}
  \documentclass[french,twoside]{book} % ,notitlepage
  \usepackage[%
    papersize={105mm, 297mm},
    inner=12mm,
    outer=12mm,
    top=20mm,
    bottom=15mm,
    marginparsep=0pt,
  ]{geometry}
  \usepackage[fontsize=9.5pt]{scrextend} % for Roboto
\else\ifav
  \documentclass[french,twoside]{book} % ,notitlepage
  \usepackage[%
    a5paper,
    inner=25mm,
    outer=15mm,
    top=15mm,
    bottom=15mm,
    marginparsep=0pt,
  ]{geometry}
  \usepackage[fontsize=12pt]{scrextend}
\else% A4 2 cols
  \documentclass[twocolumn]{report}
  \usepackage[%
    a4paper,
    inner=15mm,
    outer=10mm,
    top=25mm,
    bottom=18mm,
    marginparsep=0pt,
  ]{geometry}
  \setlength{\columnsep}{20mm}
  \usepackage[fontsize=9.5pt]{scrextend}
\fi\fi

%%%%%%%%%%%%%%
% Alignments %
%%%%%%%%%%%%%%
% before teinte macros

\setlength{\arrayrulewidth}{0.2pt}
\setlength{\columnseprule}{\arrayrulewidth} % twocol
\setlength{\parskip}{0pt} % classical para with no margin
\setlength{\parindent}{1.5em}

%%%%%%%%%%
% Colors %
%%%%%%%%%%
% before Teinte macros

\usepackage[dvipsnames]{xcolor}
\definecolor{rubric}{HTML}{800000} % the tonic 0c71c3
\def\columnseprulecolor{\color{rubric}}
\colorlet{borderline}{rubric!30!} % definecolor need exact code
\definecolor{shadecolor}{gray}{0.95}
\definecolor{bghi}{gray}{0.5}

%%%%%%%%%%%%%%%%%
% Teinte macros %
%%%%%%%%%%%%%%%%%
%%%%%%%%%%%%%%%%%%%%%%%%%%%%%%%%%%%%%%%%%%%%%%%%%%%
% <TEI> generic (LaTeX names generated by Teinte) %
%%%%%%%%%%%%%%%%%%%%%%%%%%%%%%%%%%%%%%%%%%%%%%%%%%%
% This template is inserted in a specific design
% It is XeLaTeX and otf fonts

\makeatletter % <@@@


\usepackage{blindtext} % generate text for testing
\usepackage[strict]{changepage} % for modulo 4
\usepackage{contour} % rounding words
\usepackage[nodayofweek]{datetime}
% \usepackage{DejaVuSans} % seems buggy for sffont font for symbols
\usepackage{enumitem} % <list>
\usepackage{etoolbox} % patch commands
\usepackage{fancyvrb}
\usepackage{fancyhdr}
\usepackage{float}
\usepackage{fontspec} % XeLaTeX mandatory for fonts
\usepackage{footnote} % used to capture notes in minipage (ex: quote)
\usepackage{framed} % bordering correct with footnote hack
\usepackage{graphicx}
\usepackage{lettrine} % drop caps
\usepackage{lipsum} % generate text for testing
\usepackage[framemethod=tikz,]{mdframed} % maybe used for frame with footnotes inside
\usepackage{pdftexcmds} % needed for tests expressions
\usepackage{polyglossia} % non-break space french punct, bug Warning: "Failed to patch part"
\usepackage[%
  indentfirst=false,
  vskip=1em,
  noorphanfirst=true,
  noorphanafter=true,
  leftmargin=\parindent,
  rightmargin=0pt,
]{quoting}
\usepackage{ragged2e}
\usepackage{setspace} % \setstretch for <quote>
\usepackage{tabularx} % <table>
\usepackage[explicit]{titlesec} % wear titles, !NO implicit
\usepackage{tikz} % ornaments
\usepackage{tocloft} % styling tocs
\usepackage[fit]{truncate} % used im runing titles
\usepackage{unicode-math}
\usepackage[normalem]{ulem} % breakable \uline, normalem is absolutely necessary to keep \emph
\usepackage{verse} % <l>
\usepackage{xcolor} % named colors
\usepackage{xparse} % @ifundefined
\XeTeXdefaultencoding "iso-8859-1" % bad encoding of xstring
\usepackage{xstring} % string tests
\XeTeXdefaultencoding "utf-8"
\PassOptionsToPackage{hyphens}{url} % before hyperref, which load url package

% TOTEST
% \usepackage{hypcap} % links in caption ?
% \usepackage{marginnote}
% TESTED
% \usepackage{background} % doesn’t work with xetek
% \usepackage{bookmark} % prefers the hyperref hack \phantomsection
% \usepackage[color, leftbars]{changebar} % 2 cols doc, impossible to keep bar left
% \usepackage[utf8x]{inputenc} % inputenc package ignored with utf8 based engines
% \usepackage[sfdefault,medium]{inter} % no small caps
% \usepackage{firamath} % choose firasans instead, firamath unavailable in Ubuntu 21-04
% \usepackage{flushend} % bad for last notes, supposed flush end of columns
% \usepackage[stable]{footmisc} % BAD for complex notes https://texfaq.org/FAQ-ftnsect
% \usepackage{helvet} % not for XeLaTeX
% \usepackage{multicol} % not compatible with too much packages (longtable, framed, memoir…)
% \usepackage[default,oldstyle,scale=0.95]{opensans} % no small caps
% \usepackage{sectsty} % \chapterfont OBSOLETE
% \usepackage{soul} % \ul for underline, OBSOLETE with XeTeX
% \usepackage[breakable]{tcolorbox} % text styling gone, footnote hack not kept with breakable


% Metadata inserted by a program, from the TEI source, for title page and runing heads
\title{\textbf{ Lettres persanes }}
\date{1721}
\author{Montesquieu, Charles-Louis de Secondat (1689-1755 ; baron de La Brède et de) }
\def\elbibl{Montesquieu, Charles-Louis de Secondat (1689-1755 ; baron de La Brède et de) . 1721. \emph{Lettres persanes}}
\def\elsource{Montesquieu, {\itshape Lettres persanes} [1721, éd. de 1758], avec préface, notes et variantes… par André Lefèvre, Paris, A. Lemerre, 1873, 2 vol. HTML : \href{https://fr.wikisource.org/wiki/Lettres_persanes}{\dotuline{Wikisource}}\footnote{\href{https://fr.wikisource.org/wiki/Lettres_persanes}{\url{https://fr.wikisource.org/wiki/Lettres_persanes}}}.}

% Default metas
\newcommand{\colorprovide}[2]{\@ifundefinedcolor{#1}{\colorlet{#1}{#2}}{}}
\colorprovide{rubric}{red}
\colorprovide{silver}{lightgray}
\@ifundefined{syms}{\newfontfamily\syms{DejaVu Sans}}{}
\newif\ifdev
\@ifundefined{elbibl}{% No meta defined, maybe dev mode
  \newcommand{\elbibl}{Titre court ?}
  \newcommand{\elbook}{Titre du livre source ?}
  \newcommand{\elabstract}{Résumé\par}
  \newcommand{\elurl}{http://oeuvres.github.io/elbook/2}
  \author{Éric Lœchien}
  \title{Un titre de test assez long pour vérifier le comportement d’une maquette}
  \date{1566}
  \devtrue
}{}
\let\eltitle\@title
\let\elauthor\@author
\let\eldate\@date


\defaultfontfeatures{
  % Mapping=tex-text, % no effect seen
  Scale=MatchLowercase,
  Ligatures={TeX,Common},
}


% generic typo commands
\newcommand{\astermono}{\medskip\centerline{\color{rubric}\large\selectfont{\syms ✻}}\medskip\par}%
\newcommand{\astertri}{\medskip\par\centerline{\color{rubric}\large\selectfont{\syms ✻\,✻\,✻}}\medskip\par}%
\newcommand{\asterism}{\bigskip\par\noindent\parbox{\linewidth}{\centering\color{rubric}\large{\syms ✻}\\{\syms ✻}\hskip 0.75em{\syms ✻}}\bigskip\par}%

% lists
\newlength{\listmod}
\setlength{\listmod}{\parindent}
\setlist{
  itemindent=!,
  listparindent=\listmod,
  labelsep=0.2\listmod,
  parsep=0pt,
  % topsep=0.2em, % default topsep is best
}
\setlist[itemize]{
  label=—,
  leftmargin=0pt,
  labelindent=1.2em,
  labelwidth=0pt,
}
\setlist[enumerate]{
  label={\bf\color{rubric}\arabic*.},
  labelindent=0.8\listmod,
  leftmargin=\listmod,
  labelwidth=0pt,
}
\newlist{listalpha}{enumerate}{1}
\setlist[listalpha]{
  label={\bf\color{rubric}\alph*.},
  leftmargin=0pt,
  labelindent=0.8\listmod,
  labelwidth=0pt,
}
\newcommand{\listhead}[1]{\hspace{-1\listmod}\emph{#1}}

\renewcommand{\hrulefill}{%
  \leavevmode\leaders\hrule height 0.2pt\hfill\kern\z@}

% General typo
\DeclareTextFontCommand{\textlarge}{\large}
\DeclareTextFontCommand{\textsmall}{\small}

% commands, inlines
\newcommand{\anchor}[1]{\Hy@raisedlink{\hypertarget{#1}{}}} % link to top of an anchor (not baseline)
\newcommand\abbr[1]{#1}
\newcommand{\autour}[1]{\tikz[baseline=(X.base)]\node [draw=rubric,thin,rectangle,inner sep=1.5pt, rounded corners=3pt] (X) {\color{rubric}#1};}
\newcommand\corr[1]{#1}
\newcommand{\ed}[1]{ {\color{silver}\sffamily\footnotesize (#1)} } % <milestone ed="1688"/>
\newcommand\expan[1]{#1}
\newcommand\foreign[1]{\emph{#1}}
\newcommand\gap[1]{#1}
\renewcommand{\LettrineFontHook}{\color{rubric}}
\newcommand{\initial}[2]{\lettrine[lines=2, loversize=0.3, lhang=0.3]{#1}{#2}}
\newcommand{\initialiv}[2]{%
  \let\oldLFH\LettrineFontHook
  % \renewcommand{\LettrineFontHook}{\color{rubric}\ttfamily}
  \IfSubStr{QJ’}{#1}{
    \lettrine[lines=4, lhang=0.2, loversize=-0.1, lraise=0.2]{\smash{#1}}{#2}
  }{\IfSubStr{É}{#1}{
    \lettrine[lines=4, lhang=0.2, loversize=-0, lraise=0]{\smash{#1}}{#2}
  }{\IfSubStr{ÀÂ}{#1}{
    \lettrine[lines=4, lhang=0.2, loversize=-0, lraise=0, slope=0.6em]{\smash{#1}}{#2}
  }{\IfSubStr{A}{#1}{
    \lettrine[lines=4, lhang=0.2, loversize=0.2, slope=0.6em]{\smash{#1}}{#2}
  }{\IfSubStr{V}{#1}{
    \lettrine[lines=4, lhang=0.2, loversize=0.2, slope=-0.5em]{\smash{#1}}{#2}
  }{
    \lettrine[lines=4, lhang=0.2, loversize=0.2]{\smash{#1}}{#2}
  }}}}}
  \let\LettrineFontHook\oldLFH
}
\newcommand{\labelchar}[1]{\textbf{\color{rubric} #1}}
\newcommand{\milestone}[1]{\autour{\footnotesize\color{rubric} #1}} % <milestone n="4"/>
\newcommand\name[1]{#1}
\newcommand\orig[1]{#1}
\newcommand\orgName[1]{#1}
\newcommand\persName[1]{#1}
\newcommand\placeName[1]{#1}
\newcommand{\pn}[1]{\IfSubStr{-—–¶}{#1}% <p n="3"/>
  {\noindent{\bfseries\color{rubric}   ¶  }}
  {{\footnotesize\autour{ #1}  }}}
\newcommand\reg{}
% \newcommand\ref{} % already defined
\newcommand\sic[1]{#1}
\newcommand\surname[1]{\textsc{#1}}
\newcommand\term[1]{\textbf{#1}}

\def\mednobreak{\ifdim\lastskip<\medskipamount
  \removelastskip\nopagebreak\medskip\fi}
\def\bignobreak{\ifdim\lastskip<\bigskipamount
  \removelastskip\nopagebreak\bigskip\fi}

% commands, blocks
\newcommand{\byline}[1]{\bigskip{\RaggedLeft{#1}\par}\bigskip}
\newcommand{\bibl}[1]{{\RaggedLeft{#1}\par\bigskip}}
\newcommand{\biblitem}[1]{{\noindent\hangindent=\parindent   #1\par}}
\newcommand{\dateline}[1]{\medskip{\RaggedLeft{#1}\par}\bigskip}
\newcommand{\labelblock}[1]{\medbreak{\noindent\color{rubric}\bfseries #1}\par\mednobreak}
\newcommand{\salute}[1]{\bigbreak{#1}\par\medbreak}
\newcommand{\signed}[1]{\bigbreak\filbreak{\raggedleft #1\par}\medskip}

% environments for blocks (some may become commands)
\newenvironment{borderbox}{}{} % framing content
\newenvironment{citbibl}{\ifvmode\hfill\fi}{\ifvmode\par\fi }
\newenvironment{docAuthor}{\ifvmode\vskip4pt\fontsize{16pt}{18pt}\selectfont\fi\itshape}{\ifvmode\par\fi }
\newenvironment{docDate}{}{\ifvmode\par\fi }
\newenvironment{docImprint}{\vskip6pt}{\ifvmode\par\fi }
\newenvironment{docTitle}{\vskip6pt\bfseries\fontsize{18pt}{22pt}\selectfont}{\par }
\newenvironment{msHead}{\vskip6pt}{\par}
\newenvironment{msItem}{\vskip6pt}{\par}
\newenvironment{titlePart}{}{\par }


% environments for block containers
\newenvironment{argument}{\itshape\parindent0pt}{\vskip1.5em}
\newenvironment{biblfree}{}{\ifvmode\par\fi }
\newenvironment{bibitemlist}[1]{%
  \list{\@biblabel{\@arabic\c@enumiv}}%
  {%
    \settowidth\labelwidth{\@biblabel{#1}}%
    \leftmargin\labelwidth
    \advance\leftmargin\labelsep
    \@openbib@code
    \usecounter{enumiv}%
    \let\p@enumiv\@empty
    \renewcommand\theenumiv{\@arabic\c@enumiv}%
  }
  \sloppy
  \clubpenalty4000
  \@clubpenalty \clubpenalty
  \widowpenalty4000%
  \sfcode`\.\@m
}%
{\def\@noitemerr
  {\@latex@warning{Empty `bibitemlist' environment}}%
\endlist}
\newenvironment{quoteblock}% may be used for ornaments
  {\begin{quoting}}
  {\end{quoting}}

% table () is preceded and finished by custom command
\newcommand{\tableopen}[1]{%
  \ifnum\strcmp{#1}{wide}=0{%
    \begin{center}
  }
  \else\ifnum\strcmp{#1}{long}=0{%
    \begin{center}
  }
  \else{%
    \begin{center}
  }
  \fi\fi
}
\newcommand{\tableclose}[1]{%
  \ifnum\strcmp{#1}{wide}=0{%
    \end{center}
  }
  \else\ifnum\strcmp{#1}{long}=0{%
    \end{center}
  }
  \else{%
    \end{center}
  }
  \fi\fi
}


% text structure
\newcommand\chapteropen{} % before chapter title
\newcommand\chaptercont{} % after title, argument, epigraph…
\newcommand\chapterclose{} % maybe useful for multicol settings
\setcounter{secnumdepth}{-2} % no counters for hierarchy titles
\setcounter{tocdepth}{5} % deep toc
\markright{\@title} % ???
\markboth{\@title}{\@author} % ???
\renewcommand\tableofcontents{\@starttoc{toc}}
% toclof format
% \renewcommand{\@tocrmarg}{0.1em} % Useless command?
% \renewcommand{\@pnumwidth}{0.5em} % {1.75em}
\renewcommand{\@cftmaketoctitle}{}
\setlength{\cftbeforesecskip}{\z@ \@plus.2\p@}
\renewcommand{\cftchapfont}{}
\renewcommand{\cftchapdotsep}{\cftdotsep}
\renewcommand{\cftchapleader}{\normalfont\cftdotfill{\cftchapdotsep}}
\renewcommand{\cftchappagefont}{\bfseries}
\setlength{\cftbeforechapskip}{0em \@plus\p@}
% \renewcommand{\cftsecfont}{\small\relax}
\renewcommand{\cftsecpagefont}{\normalfont}
% \renewcommand{\cftsubsecfont}{\small\relax}
\renewcommand{\cftsecdotsep}{\cftdotsep}
\renewcommand{\cftsecpagefont}{\normalfont}
\renewcommand{\cftsecleader}{\normalfont\cftdotfill{\cftsecdotsep}}
\setlength{\cftsecindent}{1em}
\setlength{\cftsubsecindent}{2em}
\setlength{\cftsubsubsecindent}{3em}
\setlength{\cftchapnumwidth}{1em}
\setlength{\cftsecnumwidth}{1em}
\setlength{\cftsubsecnumwidth}{1em}
\setlength{\cftsubsubsecnumwidth}{1em}

% footnotes
\newif\ifheading
\newcommand*{\fnmarkscale}{\ifheading 0.70 \else 1 \fi}
\renewcommand\footnoterule{\vspace*{0.3cm}\hrule height \arrayrulewidth width 3cm \vspace*{0.3cm}}
\setlength\footnotesep{1.5\footnotesep} % footnote separator
\renewcommand\@makefntext[1]{\parindent 1.5em \noindent \hb@xt@1.8em{\hss{\normalfont\@thefnmark . }}#1} % no superscipt in foot
\patchcmd{\@footnotetext}{\footnotesize}{\footnotesize\sffamily}{}{} % before scrextend, hyperref


%   see https://tex.stackexchange.com/a/34449/5049
\def\truncdiv#1#2{((#1-(#2-1)/2)/#2)}
\def\moduloop#1#2{(#1-\truncdiv{#1}{#2}*#2)}
\def\modulo#1#2{\number\numexpr\moduloop{#1}{#2}\relax}

% orphans and widows
\clubpenalty=9996
\widowpenalty=9999
\brokenpenalty=4991
\predisplaypenalty=10000
\postdisplaypenalty=1549
\displaywidowpenalty=1602
\hyphenpenalty=400
% Copied from Rahtz but not understood
\def\@pnumwidth{1.55em}
\def\@tocrmarg {2.55em}
\def\@dotsep{4.5}
\emergencystretch 3em
\hbadness=4000
\pretolerance=750
\tolerance=2000
\vbadness=4000
\def\Gin@extensions{.pdf,.png,.jpg,.mps,.tif}
% \renewcommand{\@cite}[1]{#1} % biblio

\usepackage{hyperref} % supposed to be the last one, :o) except for the ones to follow
\urlstyle{same} % after hyperref
\hypersetup{
  % pdftex, % no effect
  pdftitle={\elbibl},
  % pdfauthor={Your name here},
  % pdfsubject={Your subject here},
  % pdfkeywords={keyword1, keyword2},
  bookmarksnumbered=true,
  bookmarksopen=true,
  bookmarksopenlevel=1,
  pdfstartview=Fit,
  breaklinks=true, % avoid long links
  pdfpagemode=UseOutlines,    % pdf toc
  hyperfootnotes=true,
  colorlinks=false,
  pdfborder=0 0 0,
  % pdfpagelayout=TwoPageRight,
  % linktocpage=true, % NO, toc, link only on page no
}

\makeatother % /@@@>
%%%%%%%%%%%%%%
% </TEI> end %
%%%%%%%%%%%%%%


%%%%%%%%%%%%%
% footnotes %
%%%%%%%%%%%%%
\renewcommand{\thefootnote}{\bfseries\textcolor{rubric}{\arabic{footnote}}} % color for footnote marks

%%%%%%%%%
% Fonts %
%%%%%%%%%
\usepackage[]{roboto} % SmallCaps, Regular is a bit bold
% \linespread{0.90} % too compact, keep font natural
\newfontfamily\fontrun[]{Roboto Condensed Light} % condensed runing heads
\ifav
  \setmainfont[
    ItalicFont={Roboto Light Italic},
  ]{Roboto}
\else\ifbooklet
  \setmainfont[
    ItalicFont={Roboto Light Italic},
  ]{Roboto}
\else
\setmainfont[
  ItalicFont={Roboto Italic},
]{Roboto Light}
\fi\fi
\renewcommand{\LettrineFontHook}{\bfseries\color{rubric}}
% \renewenvironment{labelblock}{\begin{center}\bfseries\color{rubric}}{\end{center}}

%%%%%%%%
% MISC %
%%%%%%%%

\setdefaultlanguage[frenchpart=false]{french} % bug on part


\newenvironment{quotebar}{%
    \def\FrameCommand{{\color{rubric!10!}\vrule width 0.5em} \hspace{0.9em}}%
    \def\OuterFrameSep{\itemsep} % séparateur vertical
    \MakeFramed {\advance\hsize-\width \FrameRestore}
  }%
  {%
    \endMakeFramed
  }
\renewenvironment{quoteblock}% may be used for ornaments
  {%
    \savenotes
    \setstretch{0.9}
    \normalfont
    \begin{quotebar}
  }
  {%
    \end{quotebar}
    \spewnotes
  }


\renewcommand{\headrulewidth}{\arrayrulewidth}
\renewcommand{\headrule}{{\color{rubric}\hrule}}

% delicate tuning, image has produce line-height problems in title on 2 lines
\titleformat{name=\chapter} % command
  [display] % shape
  {\vspace{1.5em}\centering} % format
  {} % label
  {0pt} % separator between n
  {}
[{\color{rubric}\huge\textbf{#1}}\bigskip] % after code
% \titlespacing{command}{left spacing}{before spacing}{after spacing}[right]
\titlespacing*{\chapter}{0pt}{-2em}{0pt}[0pt]

\titleformat{name=\section}
  [block]{}{}{}{}
  [\vbox{\color{rubric}\large\raggedleft\textbf{#1}}]
\titlespacing{\section}{0pt}{0pt plus 4pt minus 2pt}{\baselineskip}

\titleformat{name=\subsection}
  [block]
  {}
  {} % \thesection
  {} % separator \arrayrulewidth
  {}
[\vbox{\large\textbf{#1}}]
% \titlespacing{\subsection}{0pt}{0pt plus 4pt minus 2pt}{\baselineskip}

\ifaiv
  \fancypagestyle{main}{%
    \fancyhf{}
    \setlength{\headheight}{1.5em}
    \fancyhead{} % reset head
    \fancyfoot{} % reset foot
    \fancyhead[L]{\truncate{0.45\headwidth}{\fontrun\elbibl}} % book ref
    \fancyhead[R]{\truncate{0.45\headwidth}{ \fontrun\nouppercase\leftmark}} % Chapter title
    \fancyhead[C]{\thepage}
  }
  \fancypagestyle{plain}{% apply to chapter
    \fancyhf{}% clear all header and footer fields
    \setlength{\headheight}{1.5em}
    \fancyhead[L]{\truncate{0.9\headwidth}{\fontrun\elbibl}}
    \fancyhead[R]{\thepage}
  }
\else
  \fancypagestyle{main}{%
    \fancyhf{}
    \setlength{\headheight}{1.5em}
    \fancyhead{} % reset head
    \fancyfoot{} % reset foot
    \fancyhead[RE]{\truncate{0.9\headwidth}{\fontrun\elbibl}} % book ref
    \fancyhead[LO]{\truncate{0.9\headwidth}{\fontrun\nouppercase\leftmark}} % Chapter title, \nouppercase needed
    \fancyhead[RO,LE]{\thepage}
  }
  \fancypagestyle{plain}{% apply to chapter
    \fancyhf{}% clear all header and footer fields
    \setlength{\headheight}{1.5em}
    \fancyhead[L]{\truncate{0.9\headwidth}{\fontrun\elbibl}}
    \fancyhead[R]{\thepage}
  }
\fi

\ifav % a5 only
  \titleclass{\section}{top}
\fi

\newcommand\chapo{{%
  \vspace*{-3em}
  \centering % no vskip ()
  {\Large\addfontfeature{LetterSpace=25}\bfseries{\elauthor}}\par
  \smallskip
  {\large\eldate}\par
  \bigskip
  {\Large\selectfont{\eltitle}}\par
  \bigskip
  {\color{rubric}\hline\par}
  \bigskip
  {\Large TEXTE LIBRE À PARTICPATION LIBRE\par}
  \centerline{\small\color{rubric} {hurlus.fr, tiré le \today}}\par
  \bigskip
}}

\newcommand\cover{{%
  \thispagestyle{empty}
  \centering
  {\LARGE\bfseries{\elauthor}}\par
  \bigskip
  {\Large\eldate}\par
  \bigskip
  \bigskip
  {\LARGE\selectfont{\eltitle}}\par
  \vfill\null
  {\color{rubric}\setlength{\arrayrulewidth}{2pt}\hline\par}
  \vfill\null
  {\Large TEXTE LIBRE À PARTICPATION LIBRE\par}
  \centerline{{\href{https://hurlus.fr}{\dotuline{hurlus.fr}}, tiré le \today}}\par
}}

\begin{document}
\pagestyle{empty}
\ifbooklet{
  \cover\newpage
  \thispagestyle{empty}\hbox{}\newpage
  \cover\newpage\noindent Les voyages de la brochure\par
  \bigskip
  \begin{tabularx}{\textwidth}{l|X|X}
    \textbf{Date} & \textbf{Lieu}& \textbf{Nom/pseudo} \\ \hline
    \rule{0pt}{25cm} &  &   \\
  \end{tabularx}
  \newpage
  \addtocounter{page}{-4}
}\fi

\thispagestyle{empty}
\ifaiv
  \twocolumn[\chapo]
\else
  \chapo
\fi
{\it\elabstract}
\bigskip
\makeatletter\@starttoc{toc}\makeatother % toc without new page
\bigskip

\pagestyle{main} % after style

  \section[{Lettre première. Usbek à son ami Rustan. À Ispahan.}]{Lettre première. Usbek à son ami Rustan. À Ispahan.}\renewcommand{\leftmark}{Lettre première. Usbek à son ami Rustan. À Ispahan.}

\noindent Nous n’avons séjourné qu’un jour à Com. Lorsque nous eûmes fait nos dévotions sur le tombeau de la vierge qui a mis au monde douze prophètes, nous nous remîmes en chemin, et hier, vingt-cinquième jour de notre départ d’Ispahan, nous arrivâmes à Tauris.\par
Rica et moi sommes peut-être les premiers, parmi les Persans, que l’envie de savoir ait fait sortir de leur pays, et qui aient renoncé aux douceurs d’une vie tranquille, pour aller chercher laborieusement la sagesse.\par
Nous sommes nés dans un royaume florissant ; mais nous n’avons pas cru que ses bornes fussent celles de nos connaissances, et que la lumière orientale dût seule nous éclairer.\par
Mande-moi ce que l’on dit de notre voyage ; ne me flatte point : je ne compte pas sur un grand nombre d’approbateurs. Adresse ta lettre à Erzeron, où je séjournerai quelque temps. Adieu, mon cher Rustan. Sois assuré qu’en quelque lieu du monde où je sois, tu as un ami fidèle.\par

\dateline{De Tauris, le 15 de la lune de Saphar, 1711.}
\section[{Lettre II. Usbek au premier eunuque noir. À son sérail d’Ispahan.}]{Lettre II. Usbek au premier eunuque noir. À son sérail d’Ispahan.}\renewcommand{\leftmark}{Lettre II. Usbek au premier eunuque noir. À son sérail d’Ispahan.}

\noindent Tu es le gardien fidèle des plus belles femmes de Perse ; je t’ai confié ce que j’avais dans le monde de plus cher : tu tiens en tes mains les clefs de ces portes fatales, qui ne s’ouvrent que pour moi. Tandis que tu veilles sur ce dépôt précieux de mon cœur, il se repose, et jouit d’une sécurité entière. Tu fais la garde dans le silence de la nuit, comme dans le tumulte du jour. Tes soins infatigables soutiennent la vertu lorsqu’elle chancelle. Si les femmes que tu gardes voulaient sortir de leur devoir, tu leur en ferais perdre l’espérance. Tu es le fléau du vice et la colonne de la fidélité.\par
Tu leur commandes, et leur obéis. Tu exécutes aveuglément toutes leurs volontés, et leur fais exécuter de même les lois du sérail ; tu trouves de la gloire à leur rendre les services les plus vils ; tu te soumets avec respect et avec crainte à leurs ordres légitimes ; tu les sers comme l’esclave de leurs esclaves. Mais, par un retour d’empire, tu commandes en maître comme moi-même, quand tu crains le relâchement des lois de la pudeur et de la modestie.\par
Souviens-toi toujours du néant d’où je t’ai fait sortir, lorsque tu étais le dernier de mes esclaves, pour te mettre en cette place, et te confier les délices de mon cœur : tiens-toi dans un profond abaissement auprès de celles qui partagent mon amour ; mais fais-leur en même temps sentir leur extrême dépendance. Procure-leur tous les plaisirs qui peuvent être innocents ; trompe leurs inquiétudes ; amuse-les par la musique, les danses, les boissons délicieuses ; persuade-leur de s’assembler souvent. Si elles veulent aller à la campagne, tu peux les y mener ; mais fais faire main basse sur tous les hommes qui se présenteront devant elles. Exhorte-les à la propreté, qui est l’image de la netteté de l’âme ; parle-leur quelquefois de moi. Je voudrais les revoir dans ce lieu charmant qu’elles embellissent. Adieu.\par

\dateline{De Tauris, le 18 de la lune de Saphar, 1711}
\section[{Lettre III. Zachi à Usbek. À Tauris.}]{Lettre III. Zachi à Usbek. À Tauris.}\renewcommand{\leftmark}{Lettre III. Zachi à Usbek. À Tauris.}

\noindent Nous avons ordonné au chef des eunuques de nous mener à la campagne ; il te dira qu’aucun accident ne nous est arrivé. Quand il fallut traverser la rivière et quitter nos litières, nous nous mîmes, selon la coutume, dans des boîtes : deux esclaves nous portèrent sur leurs épaules, et nous échappâmes à tous les regards.\par
Comment aurais-je pu vivre, cher Usbek, dans ton sérail d’Ispahan ; dans ces lieux qui, me rappelant sans cesse mes plaisirs passés, irritaient tous les jours mes désirs avec une nouvelle violence ? J’errais d’appartements en appartements, te cherchant toujours et ne te trouvant jamais, mais rencontrant partout un cruel souvenir de ma félicité passée. Tantôt je me voyais en ce lieu où, pour la première fois de ma vie, je te reçus dans mes bras ; tantôt dans celui où tu décidas cette fameuse querelle entre tes femmes. Chacune de nous se prétendait supérieure aux autres en beauté. Nous nous présentâmes devant toi, après avoir épuisé tout ce que l’imagination peut fournir de parures et d’ornements : tu vis avec plaisir les miracles de notre art ; tu admiras jusqu’où nous avait emportées l’ardeur de te plaire. Mais tu fis bientôt céder ces charmes empruntés à des grâces plus naturelles, tu détruisis tout notre ouvrage : il fallut nous dépouiller de ces ornements qui t’étaient devenus incommodes ; il fallut paraître à ta vue dans la simplicité de la nature. Je comptai pour rien la pudeur, je ne pensai qu’à ma gloire. Heureux Usbek, que de charmes furent étalés à tes yeux ! Nous te vîmes longtemps errer d’enchantements en enchantements : ton âme incertaine demeura longtemps sans se fixer, chaque grâce nouvelle te demandait un tribut, nous fûmes en un moment toutes couvertes de tes baisers ; tu portas tes regards dans les lieux les plus secrets ; tu nous fis passer en un instant dans mille situations différentes ; toujours de nouveaux commandements, et une obéissance nouvelle. Je te l’avoue, Usbek, une passion encore plus vive que l’ambition me fit souhaiter de te plaire. Je me vis insensiblement devenir la maîtresse de ton cœur ; tu me pris, tu me quittas, tu revins à moi, et je sus te retenir : le triomphe fut tout pour moi, et le désespoir pour mes rivales. Il nous sembla que nous fussions seuls dans le monde : tout ce qui nous entourait ne fut plus digne de nous occuper. Plût au ciel que mes rivales eussent eu le courage de rester témoins de toutes les marques d’amour que je reçus de toi ! Si elles avaient bien vu mes transports, elles auraient senti la différence qu’il y a de mon amour au leur ; elles auraient vu que, si elles pouvaient disputer avec moi de mes charmes, elles ne pouvaient disputer de sensibilité…\par
Mais où suis-je ? Où m’emmène ce vain récit ? C’est un malheur de n’être point aimée ; mais c’est un affront de ne l’être plus. Tu nous quittes, Usbek, pour aller errer dans des climats barbares. Quoi ! Tu comptes pour rien l’avantage d’être aimé ? Hélas ! Tu ne sais pas même ce que tu perds ! Je pousse des soupirs qui ne sont point entendus ; mes larmes coulent, et tu n’en jouis pas ; il semble que l’amour respire dans le sérail, et ton insensibilité t’en éloigne sans cesse ! Ah ! mon cher Usbek, si tu savais être heureux !\par

\dateline{Du sérail de Fatmé, le 21 de la lune de Maharram, 1711.}
\section[{Lettre IV. Zephis à Usbek. À Erzeron.}]{Lettre IV. Zephis à Usbek. À Erzeron.}\renewcommand{\leftmark}{Lettre IV. Zephis à Usbek. À Erzeron.}

\noindent Enfin ce monstre noir a résolu de me désespérer. Il veut à toute force m’ôter mon esclave Zélide, Zélide qui me sert avec tant d’affection, et dont les adroites mains portent partout les ornements et les grâces ; il ne lui suffit pas que cette séparation soit douloureuse, il veut encore qu’elle soit déshonorante. Le traître veut regarder comme criminels les motifs de ma confiance ; et parce qu’il s’ennuie derrière la porte, où je le renvoie toujours, il ose supposer qu’il a entendu ou vu des choses, que je ne sais même pas imaginer. Je suis bien malheureuse ! Ma retraite, ni ma vertu, ne sauraient me mettre à l’abri de ses soupçons extravagants : un vil esclave vient m’attaquer jusque dans ton cœur, il faut que je m’y défende ! Non, j’ai trop de respect pour moi-même pour descendre jusqu’à des justifications : je ne veux d’autre garant de ma conduite que toi-même, que ton amour, que le mien, et, s’il faut te le dire, cher Usbek, que mes larmes.\par

\dateline{Du sérail de Fatmé, le 29 de la lune de Maharram, 1711.}
\section[{Lettre V. Rustan à Usbek. À Erzeron.}]{Lettre V. Rustan à Usbek. À Erzeron.}\renewcommand{\leftmark}{Lettre V. Rustan à Usbek. À Erzeron.}

\noindent Tu es le sujet de toutes les conversations d’Ispahan ; on ne parle que de ton départ : les uns l’attribuent à une légèreté d’esprit, les autres à quelque chagrin ; tes amis seuls te défendent, et ils ne persuadent personne. On ne peut comprendre que tu puisses quitter tes femmes, tes parents, tes amis, ta patrie, pour aller dans des climats inconnus aux Persans. La mère de Rica est inconsolable ; elle te demande son fils, que tu lui as, dit-elle, enlevé. Pour moi, mon cher Usbek, je me sens naturellement porté à approuver tout ce que tu fais : mais je ne saurais te pardonner ton absence ; et, quelques raisons que tu m’en puisses donner, mon cœur ne les goûtera jamais. Adieu. Aime-moi toujours.\par

\dateline{D’Ispahan, le 28 de la lune de Rebiab I, 1711.}
\section[{Lettre VI. Usbek à son ami Nessir. À Ispahan.}]{Lettre VI. Usbek à son ami Nessir. À Ispahan.}\renewcommand{\leftmark}{Lettre VI. Usbek à son ami Nessir. À Ispahan.}

\noindent À une journée d’Erivan nous quittâmes la Perse pour entrer dans les terres de l’obéissance des Turcs. Douze jours après, nous arrivâmes à Erzeron, où nous séjournerons trois ou quatre mois.\par
Il faut que je te l’avoue, Nessir ; j’ai senti une douleur secrète quand j’ai perdu la Perse de vue, et que je me suis trouvé au milieu des perfides Osmanlins. À mesure que j’entrais dans le pays de ces profanes, il me semblait que je devenais profane moi-même.\par
Ma patrie, ma famille, mes amis, se sont présentés à mon esprit ; ma tendresse s’est réveillée ; une certaine inquiétude a achevé de me troubler, et m’a fait connaître que, pour mon repos, j’avais trop entrepris.\par
Mais ce qui afflige le plus mon cœur, ce sont mes femmes. Je ne puis penser à elles que je ne sois dévoré de chagrins.\par
Ce n’est pas, Nessir, que je les aime : je me trouve à cet égard dans une insensibilité qui ne me laisse point de désirs. Dans le nombreux sérail où j’ai vécu, j’ai prévenu l’amour, et l’ai détruit par lui-même : mais, de ma froideur même, il sort une jalousie secrète qui me dévore. Je vois une troupe de femmes laissées presque à elles-mêmes ; je n’ai que des âmes lâches qui m’en répondent. J’aurais peine à être en sûreté, si mes esclaves étaient fidèles : que sera-ce, s’ils ne le sont pas ? Quelles tristes nouvelles peuvent m’en venir, dans les pays éloignés que je vais parcourir ! C’est un mal où mes amis ne peuvent porter de remède : c’est un lieu dont ils doivent ignorer les tristes secrets ; et qu’y pourraient-ils faire ? N’aimerais-je pas mille fois mieux une obscure impunité qu’une correction éclatante ? Je dépose en ton cœur tous mes chagrins, mon cher Nessir : c’est la seule consolation qui me reste dans l’état où je suis.\par

\dateline{D’Erzeron, le 10 de la lune de Rebiab 2, 1711.}
\section[{Lettre VII. Fatme à Usbek. À Erzeron.}]{Lettre VII. Fatme à Usbek. À Erzeron.}\renewcommand{\leftmark}{Lettre VII. Fatme à Usbek. À Erzeron.}

\noindent Il y a deux mois que tu es parti, mon cher Usbek ; et, dans l’abattement où je suis, je ne puis pas me le persuader encore. Je cours tout le sérail comme si tu y étais ; je ne suis point désabusée. Que veux-tu que devienne une femme qui t’aime ; qui était accoutumée à te tenir dans ses bras ; qui n’était occupée que du soin de te donner des preuves de sa tendresse ; libre par l’avantage de sa naissance, esclave par la violence de son amour ?\par
Quand je t’épousai, mes yeux n’avaient point encore vu le visage d’un homme : tu es le seul dont la vue m’ait été permise\footnote{Les femmes persanes sont beaucoup plus étroitement gardées que les femmes turques et les femmes indiennes.} ; car je ne compte point au rang des hommes ces eunuques affreux dont la moindre imperfection est de n’être point des hommes. Quand je compare la beauté de ton visage avec la difformité du leur, je ne puis m’empêcher de m’estimer heureuse : mon imagination ne me fournit point d’idée plus ravissante que les charmes enchanteurs de ta personne. Je te le jure, Usbek, quand il me serait permis de sortir de ce lieu où je suis enfermée par la nécessité de ma condition ; quand je pourrais me dérober à la garde qui m’environne ; quand il me serait permis de choisir parmi tous les hommes qui vivent dans cette capitale des nations ; Usbek, je te le jure, je ne choisirais que toi. Il ne peut y avoir que toi dans le monde qui mérite d’être aimé.\par
Ne pense pas que ton absence m’ait fait négliger une beauté qui t’est chère : quoique je ne doive être vue de personne, et que les ornements dont je me pare soient inutiles à ton bonheur, je cherche cependant à m’entretenir dans l’habitude de plaire ; je ne me couche point que je ne sois parfumée des essences les plus délicieuses. Je me rappelle ce temps heureux où tu venais dans mes bras ; un songe flatteur, qui me séduit, me montre ce cher objet de mon amour ; mon imagination se perd dans ses désirs, comme elle se flatte dans ses espérances : je pense quelquefois que, dégoûté d’un pénible voyage, tu vas revenir à nous : la nuit se passe dans des songes qui n’appartiennent ni à la veille ni au sommeil ; je te cherche à mes côtés, et il me semble que tu me fuis ; enfin le feu qui me dévore dissipe lui-même ces enchantements, et rappelle mes esprits. Je me trouve pour lors si animée… Tu ne le croirais pas, Usbek ; il est impossible de vivre dans cet état ; le feu coule dans mes veines : que ne puis-je t’exprimer ce que je sens si bien ? Et comment sens-je si bien ce que je ne puis t’exprimer ? Dans ces moments, Usbek, je donnerais l’empire du monde pour un seul de tes baisers. Qu’une femme est malheureuse d’avoir des désirs si violents, lorsqu’elle est privée de celui qui peut seul les satisfaire ; que, livrée à elle-même, n’ayant rien qui puisse la distraire, il faut qu’elle vive dans l’habitude des soupirs et dans la fureur d’une passion irritée ; que, bien loin d’être heureuse, elle n’a pas même l’avantage de servir à la félicité d’un autre : ornement inutile d’un sérail, gardée pour l’honneur et non pas pour le bonheur de son époux !\par
Vous êtes bien cruels, vous autres hommes ! Vous êtes charmés que nous ayons des désirs que nous ne puissions pas satisfaire : vous nous traitez comme si nous étions insensibles, et vous seriez bien fâchés que nous le fussions : vous croyez que nos désirs, si longtemps mortifiés seront irrités à votre vue. Il y a de la peine à se faire aimer ; il est plus court d’obtenir de notre tempérament ce que vous n’osez espérer de votre mérite.\par
Adieu, mon cher Usbek, adieu. Compte que je ne vis que pour t’adorer : mon âme est toute pleine de toi ; et ton absence, bien loin de te faire oublier, animerait mon amour s’il pouvait devenir plus violent\par

\dateline{Du sérail d’Ispahan, le 12 de la lune de Rebiab I, 1711.}
\section[{Lettre VIII. Usbek à son ami Rustan. À Ispahan.}]{Lettre VIII. Usbek à son ami Rustan. À Ispahan.}\renewcommand{\leftmark}{Lettre VIII. Usbek à son ami Rustan. À Ispahan.}

\noindent Ta lettre m’a été rendue à Erzeron, où je suis. Je m’étais bien douté que mon départ ferait du bruit : je ne m’en suis point mis en peine : que veux-tu que je suive, la prudence de mes ennemis, ou la mienne ?\par
Je parus à la cour dès ma plus tendre jeunesse ; je puis le dire, mon cœur ne s’y corrompit point : je formai même un grand dessein, j’osai y être vertueux. Dès que je connus le vice, je m’en éloignai ; mais je m’en approchai ensuite pour le démasquer. Je portai la vérité jusqu’au pied du trône : j’y parlai un langage jusqu’alors inconnu ; je déconcertai la flatterie, et j’étonnai en même temps les adorateurs et l’idole.\par
Mais quand je vis que ma sincérité m’avait fait des ennemis ; je m’étais attiré la jalousie des ministres sans avoir la faveur du prince ; que, dans une cour corrompue, je ne me soutenais plus que par une faible vertu, je résolus de la quitter. Je feignis un grand attachement pour les sciences ; et, à force de feindre, il me vint réellement. Je ne me mêlai plus d’aucunes affaires, et je me retirai dans une maison de campagne. Mais ce parti même avait ses inconvénients : je restais toujours exposé à la malice de mes ennemis, et je m’étais presque ôté les moyens de m’en garantir. Quelques avis secrets me firent penser à moi sérieusement : je résolus de m’exiler de ma patrie, et ma retraite même de la cour m’en fournit un motif plausible. J’allai au roi ; je lui marquai l’envie que j’avais de m’instruire dans les sciences de l’Occident ; je lui insinuai qu’il pourrait tirer de l’utilité de mes voyages : je trouvai grâce devant ses yeux ; je partis, et je dérobai une victime à mes ennemis.\par
Voilà, Rustan, le véritable motif de mon voyage. Laisse parler Ispahan ; ne me défends que devant ceux qui m’aiment. Laisse à mes ennemis leurs interprétations malignes : je suis trop heureux que ce soit le seul mal qu’ils me puissent faire.\par
On parle de moi à présent : peut-être ne serai-je que trop oublié, et que mes amis… Non, Rustan, je ne veux point me livrer à cette triste pensée : je leur serai toujours cher ; je compte sur leur fidélité, comme sur la tienne.\par

\dateline{D’Erzeron, le 20 de la lune de Gemmadi 2, 1711.}
\section[{Lettre IX. Le premier eunuque à Ibbi. À Erzeron.}]{Lettre IX. Le premier eunuque à Ibbi. À Erzeron.}\renewcommand{\leftmark}{Lettre IX. Le premier eunuque à Ibbi. À Erzeron.}

\noindent Tu suis ton ancien maître dans ses voyages ; tu parcours les provinces et les royaumes ; les chagrins ne sauraient faire d’impression sur toi ; chaque instant te montre des choses nouvelles ; tout ce que tu vois te récrée, et te fait passer le temps sans le sentir.\par
Il n’en est pas de même de moi, qui, enfermé dans une affreuse prison, suis toujours environné des mêmes objets et dévoré des mêmes chagrins. Je gémis accablé sous le poids des soins et des inquiétudes de cinquante années ; et, dans le cours d’une longue vie, je ne puis pas dire avoir eu un jour serein et un moment tranquille.\par
Lorsque mon premier maître eut formé le cruel projet de me confier ses femmes, et m’eut obligé, par des séductions soutenues de mille menaces, de me séparer pour jamais de moi-même ; las de servir dans les emplois les plus pénibles, je comptai sacrifier mes passions à mon repos et à ma fortune. Malheureux que j’étais ! Mon esprit préoccupé me faisait voir le dédommagement, et non pas la perte : j’espérais que je serais délivré des atteintes de l’amour par l’impuissance de le satisfaire. Hélas ! On éteignit en moi l’effet des passions, sans en éteindre la cause ; et, bien loin d’en être soulagé, je me trouvai environné d’objets qui les irritaient sans cesse. J’entrai dans le sérail, où tout m’inspirait le regret de ce que j’avais perdu : je me sentais animé à chaque instant ; mille grâces naturelles semblaient ne se découvrir à ma vue que pour me désoler ; pour comble de malheurs, j’avais toujours devant les yeux un homme heureux. Dans ce temps de trouble, je n’ai jamais conduit une femme dans le lit de mon maître, je ne l’ai jamais déshabillée, que je ne sois rentré chez moi la rage dans le cœur, et un affreux désespoir dans l’âme.\par
Voilà comment j’ai passé ma misérable jeunesse : je n’avais de confident que moi-même. Chargé d’ennuis et de chagrins, il me les fallait dévorer ; et ces mêmes femmes que j’étais tenté de regarder avec des yeux si tendres, je ne les envisageais qu’avec des regards sévères : j’étais perdu si elles m’avaient pénétré ; quel avantage n’en auraient-elles pas pris !\par
Je me souviens qu’un jour que je mettais une femme dans le bain, je me sentis si transporté que je perdis entièrement la raison, et que j’osai porter ma main dans un lieu redoutable. Je crus, à la première réflexion, que ce jour était le dernier de mes jours. Je fus pourtant assez heureux pour échapper à mille morts ; mais la beauté que j’avais faite confidente de ma faiblesse me vendit bien cher son silence ; je perdis entièrement mon autorité sur elle, et elle m’a obligé depuis à des condescendances qui m’ont exposé mille fois à perdre la vie.\par
Enfin les feux de la jeunesse ont passé ; je suis vieux, et je me trouve, à cet égard, dans un état tranquille ; je regarde les femmes avec indifférence, et je leur rends bien tous leurs mépris, et tous les tourments qu’elles m’ont fait souffrir. Je me souviens toujours que j’étais né pour les commander ; et il me semble que je redeviens homme dans les occasions où je leur commande encore. Je les hais depuis que je les envisage de sang-froid, et que ma raison me laisse voir toutes leurs faiblesses. Quoique je les garde pour un autre, le plaisir de me faire obéir me donne une joie secrète ; quand je les prive de tout, il me semble que c’est pour moi, et il m’en revient toujours une satisfaction indirecte : je me trouve dans le sérail comme dans un petit empire ; et mon ambition, la seule passion qui me reste, se satisfait un peu. Je vois avec plaisir que tout roule sur moi, et qu’à tous les instants je suis nécessaire ; je me charge volontiers de la haine de toutes ces femmes, qui m’affermit dans le poste où je suis. Aussi n’ont-elles pas affaire à un ingrat : elles me trouvent au-devant de tous leurs plaisirs les plus innocents, je me présente toujours à elles comme une barrière inébranlable ; elles forment des projets, et je les arrête soudain : je m’arme de refus, je me hérisse de scrupules ; je n’ai jamais dans la bouche que les mots de devoir, de vertu, de pudeur, de modestie. Je les désespère, en leur parlant sans cesse de la faiblesse de leur sexe, et de l’autorité du maître ; je me plains ensuite d’être obligé à tant de sévérité, et je semble vouloir leur faire entendre que je n’ai d’autre motif que leur propre intérêt, et un grand attachement pour elles.\par
Ce n’est pas qu’à mon tour je n’aie un nombre infini de désagréments, et que tous les jours ces femmes vindicatives ne cherchent à renchérir sur ceux que je leur donne : elles ont des revers terribles. Il y a entre nous comme un flux et un reflux d’empire et de soumission : elles font toujours tomber sur moi les emplois les plus humiliants ; elles affectent un mépris qui n’a point d’exemple ; et, sans égard pour ma vieillesse, elles me font lever, la nuit, dix fois pour la moindre bagatelle ; je suis accablé sans cesse d’ordres, de commandements, d’emplois, de caprices ; il semble qu’elles se relayent pour m’exercer, et que leurs fantaisies se succèdent. Souvent elles se plaisent à me faire redoubler de soins ; elles me font faire de fausses confidences : tantôt on vient me dire qu’il a paru un jeune homme autour de ces murs, une autre fois qu’on a entendu du bruit, ou bien qu’on doit rendre une lettre : tout ceci me trouble, et elles rient de ce trouble ; elles sont charmées de me voir ainsi me tourmenter moi-même. Une autre fois elles m’attachent derrière leur porte, et m’y enchaînent nuit et jour. Elles savent bien feindre des maladies, des défaillances, des frayeurs : elles ne manquent point de prétexte pour me mener au point où elles veulent. Il faut, dans ces occasions, une obéissance aveugle et une complaisance sans bornes : un refus dans la bouche d’un homme comme moi serait une chose inouïe ; et si je balançais à leur obéir, elles seraient en droit de me châtier. J’aimerais autant perdre la vie, mon cher Ibbi, que de descendre à cette humiliation.\par
Ce n’est pas tout : je ne suis jamais sûr d’être un instant dans la faveur de mon maître ; j’ai autant d’ennemies dans son cœur, qui ne songent qu’à me perdre : elles ont des quarts d’heure où je ne suis point écouté, des quarts d’heure où l’on ne refuse rien, des quarts d’heure où j’ai toujours tort. Je mène dans le lit de mon maître des femmes irritées : crois-tu que l’on y travaille pour moi, et que mon parti soit le plus fort ? J’ai tout à craindre de leurs larmes, de leurs soupirs, de leurs embrassements, et de leurs plaisirs mêmes : elles sont dans le lieu de leurs triomphes ; leurs charmes me deviennent terribles : les services présents effacent dans un moment tous mes services passés ; et rien ne peut me répondre d’un maître qui n’est plus à lui-même.\par
Combien de fois m’est-il arrivé de me coucher dans la faveur, et de me lever dans la disgrâce ! Le jour que je fus fouetté si indignement autour du sérail, qu’avais-je fait ? Je laisse une femme dans les bras de mon maître : dès qu’elle le vit enflammé, elle versa un torrent de larmes ; elle se plaignit, et ménagea si bien ses plaintes, qu’elles augmentaient à mesure de l’amour qu’elle faisait naître. Comment aurais-je pu me soutenir dans un moment si critique ? Je fus perdu lorsque je m’y attendais le moins ; je fus la victime d’une négociation amoureuse, et d’un traité que les soupirs avaient fait. Voilà, cher Ibbi, l’état cruel dans lequel j’ai toujours vécu. Que tu es heureux ! Tes soins te bornent uniquement à la personne d’Usbek. Il t’est facile de lui plaire et de te maintenir dans sa faveur jusqu’au dernier de tes jours.\par

\dateline{Du sérail d’Ispahan, le dernier de la lune de Saphar, 1711.}
\section[{Lettre X. Mirza à son ami Usbek. À Erzeron.}]{Lettre X. Mirza à son ami Usbek. À Erzeron.}\renewcommand{\leftmark}{Lettre X. Mirza à son ami Usbek. À Erzeron.}

\noindent Tu étais le seul qui pût me dédommager de l’absence de Rica ; et il n’y avait que Rica qui pût me consoler de la tienne. Tu nous manques, Usbek : tu étais l’âme de notre société. Qu’il faut de violence pour rompre les engagements que le cœur et l’esprit ont formés !\par
Nous disputons ici beaucoup ; nos disputes roulent ordinairement sur la morale. Hier on mit en question si les hommes étaient heureux par les plaisirs et les satisfactions des sens, ou par la pratique de la vertu. Je t’ai souvent ouï dire que les hommes étaient nés pour être vertueux, et que la justice est une qualité qui leur est aussi propre que l’existence. Explique-moi, je te prie, ce que tu veux dire.\par
J’ai parlé à des mollaks, qui me désespèrent avec leurs passages de l’Alcoran : car je ne leur parle pas comme vrai croyant, mais comme homme, comme citoyen, comme père de famille. Adieu.\par

\dateline{D’Ispahan, le dernier de la lune de Saphar, 1711}
\section[{Lettre XI. Usbek à Mirza. À Ispahan.}]{Lettre XI. Usbek à Mirza. À Ispahan.}\renewcommand{\leftmark}{Lettre XI. Usbek à Mirza. À Ispahan.}

\noindent Tu renonces à ta raison pour essayer la mienne ; tu descends jusqu’à me consulter ; tu me crois capable de t’instruire. Mon cher Mirza, il y a une chose qui me flatte encore plus que la bonne opinion que tu as conçue de moi : c’est ton amitié, qui me la procure.\par
Pour remplir ce que tu me prescris, je n’ai pas cru devoir employer des raisonnements fort abstraits. Il y a de certaines vérités qu’il ne suffit pas de persuader, mais qu’il faut encore faire sentir : telles sont les vérités de morales. Peut-être que ce morceau d’histoire te touchera plus qu’une philosophie subtile.\par
Il y avait en Arabie un petit peuple, appelé Troglodyte, qui descendait de ces anciens Troglodytes qui, si nous en croyons les historiens, ressemblaient plus à des bêtes qu’à des hommes. Ceux-ci n’étaient point si contrefaits, ils n’étaient point velus comme des ours, ils ne sifflaient point, ils avaient des yeux ; mais ils étaient si méchants et si féroces, qu’il n’y avait parmi eux aucun principe d’équité ni de justice.\par
Ils avaient un roi d’une origine étrangère, qui, voulant corriger la méchanceté de leur naturel, les traitait sévèrement ; mais ils conjurèrent contre lui, le tuèrent, et exterminèrent toute la famille royale.\par
Le coup étant fait, ils s’assemblèrent pour choisir un gouvernement ; et, après bien des dissensions, ils créèrent des magistrats. Mais à peine les eurent-ils élus, qu’ils leur devinrent insupportables ; et ils les massacrèrent encore.\par
Ce peuple, libre de ce nouveau joug, ne consulta plus que son naturel sauvage. Tous les particuliers convinrent qu’ils n’obéiraient plus à personne ; que chacun veillerait uniquement à ses intérêts, sans consulter ceux des autres.\par
Cette résolution unanime flattait extrêmement tous les particuliers. Ils disaient : Qu’ai-je affaire d’aller me tuer à travailler pour des gens dont je ne me soucie point ? Je penserai uniquement à moi. Je vivrai heureux : que m’importe que les autres le soient ? Je me procurerai tous mes besoins ; et, pourvu que je les aie, je ne me soucie point que tous les autres Troglodytes soient misérables.\par
On était dans le mois où l’on ensemence les terres ; chacun dit : je ne labourerai mon champ que pour qu’il me fournisse le blé qu’il me faut pour me nourrir ; une plus grande quantité me serait inutile : je ne prendrai point de la peine pour rien.\par
Les terres de ce petit royaume n’étaient pas de même nature : il y en avait d’arides et de montagneuses, et d’autres qui, dans un terrain bas, étaient arrosées de plusieurs ruisseaux. Cette année la sécheresse fut très grande ; de manière que les terres qui étaient dans les lieux élevés manquèrent absolument, tandis que celles qui purent être arrosées furent très fertiles : ainsi les peuples des montagnes périrent presque tous de faim par la dureté des autres, qui leur refusèrent de partager la récolte.\par
L’année d’ensuite fut très pluvieuse : les lieux élevés se trouvèrent d’une fertilité extraordinaire, et les terres basses furent submergées. La moitié du peuple cria une seconde fois famine ; mais ces misérables trouvèrent des gens aussi durs qu’ils l’avaient été eux-mêmes.\par
Un des principaux habitants avait une femme fort belle ; son voisin en devint amoureux, et l’enleva : il s’émut une grande querelle ; et, après bien des injures et des coups, ils convinrent de s’en remettre à la décision d’un Troglodyte qui, pendant que la république subsistait, avait eu quelque crédit. Ils allèrent à lui, et voulurent lui dire leurs raisons. Que m’importe, dit cet homme, que cette femme soit à vous, ou à vous ? J’ai mon champ à labourer ; je n’irai peut-être pas employer mon temps à terminer vos différends et à travailler à vos affaires, tandis que je négligerai les miennes ; je vous prie de me laisser en repos, et de ne m’importuner plus de vos querelles. Là-dessus il les quitta, et s’en alla travailler ses terres. Le ravisseur, qui était le plus fort, jura qu’il mourrait plutôt que de rendre cette femme ; et l’autre, pénétré de l’injustice de son voisin et de la dureté du juge, s’en retournait désespéré, lorsqu’il trouva dans son chemin une femme jeune et belle, qui revenait de la fontaine. Il n’avait plus de femme, celle-là lui plut ; et elle lui plut bien davantage lorsqu’il apprit que c’était la femme de celui qu’il avait voulu prendre pour juge, et qui avait été si peu sensible à son malheur : il l’enleva, et l’emmena dans sa maison.\par
Il y avait un homme qui possédait un champ assez fertile, qu’il cultivait avec grand soin : deux de ses voisins s’unirent ensemble, le chassèrent de sa maison, occupèrent son champ ; ils firent entre eux une union pour se défendre contre tous ceux qui voudraient l’usurper ; et effectivement ils se soutinrent par là pendant plusieurs mois ; mais un des deux, ennuyé de partager ce qu’il pouvait avoir tout seul, tua l’autre et devint seul maître du champ. Son empire ne fut pas long : deux autres Troglodytes vinrent l’attaquer ; il se trouva trop faible pour se défendre, et il fut massacré.\par
Un Troglodyte presque tout nu vit de la laine qui était à vendre : il en demanda le prix ; le marchand dit en lui-même : Naturellement je ne devrais espérer de ma laine qu’autant d’argent qu’il en faut pour acheter deux mesures de blé ; mais je la vais vendre quatre fois davantage, afin d’avoir huit mesures. Il fallut en passer par là, et payer le prix demandé. Je suis bien aise, dit le marchand ; j’aurai du blé à présent. Que dites-vous ? reprit l’étranger ; vous avez besoin de blé ? J’en ai à vendre : il n’y a que le prix qui vous étonnera peut-être ; car vous saurez que le blé est extrêmement cher, et que la famine règne presque partout : mais rendez-moi mon argent, et je vous donnerai une mesure de blé ; car je ne veux pas m’en défaire autrement, dussiez-vous crever de faim.\par
Cependant une maladie cruelle ravageait la contrée. Un médecin habile y arriva du pays voisin, et donna ses remèdes si à propos, qu’il guérit tous ceux qui se mirent dans ses mains. Quand la maladie eut cessé, il alla chez tous ceux qu’il avait traités demander son salaire ; mais il ne trouva que des refus : il retourna dans son pays, et il y arriva accablé des fatigues d’un si long voyage. Mais bientôt après il apprit que la même maladie se faisait sentir \\
de nouveau, et affligeait plus que jamais cette terre ingrate. Ils allèrent à lui cette fois, et n’attendirent pas qu’il vînt chez eux. Allez, leur dit-il hommes injustes, vous avez dans l’âme un poison plus mortel que celui dont vous voulez guérir ; vous ne méritez pas d’occuper une place sur la terre, parce que vous n’avez point d’humanité, et que les règles de l’équité vous sont inconnues : je croirais offenser les dieux, qui vous punissent, si je m’opposais à la justice de leur colère.\par

\dateline{À Erzeron, le 3 de la lune de Gemmadi 2, 1711.}
\section[{Lettre XII. Usbek au même. À Ispahan.}]{Lettre XII. Usbek au même. À Ispahan.}\renewcommand{\leftmark}{Lettre XII. Usbek au même. À Ispahan.}

\noindent Tu as vu, mon cher Mirza, comment les Troglodytes périrent par leur méchanceté même, et furent les victimes de leurs propres injustices. De tant de familles, il n’en resta que deux qui échappèrent aux malheurs de la nation. Il y avait dans ce pays deux hommes bien singuliers : ils avaient de l’humanité ; ils connaissaient la justice ; ils aimaient la vertu ; autant liés par la droiture de leur cœur que par la corruption de celui des autres, ils voyaient la désolation générale, et ne la ressentaient que par la pitié : c’était le motif d’une union nouvelle. Ils travaillaient avec une sollicitude commune pour l’intérêt commun ; ils n’avaient de différends que ceux qu’une douce et tendre amitié faisait naître ; et dans l’endroit du pays le plus écarté, séparés de leurs compatriotes indignes de leur présence, ils menaient une vie heureuse et tranquille : la terre semblait produire d’elle-même, cultivée par ces vertueuses mains.\par
Ils aimaient leurs femmes, et ils en étaient tendrement chéris. Toute leur attention était d’élever leurs enfants à la vertu. Ils leur représentaient sans cesse les malheurs de leurs compatriotes, et leur mettaient devant les yeux cet exemple si touchant ; ils leur faisaient surtout sentir que l’intérêt des particuliers se trouve toujours dans l’intérêt commun ; que vouloir s’en séparer, c’est vouloir se perdre ; que la vertu n’est point une chose qui doive nous coûter ; qu’il ne faut point la regarder comme un exercice pénible ; et que la justice pour autrui est une charité pour nous.\par
Ils eurent bientôt la consolation des pères vertueux, qui est d’avoir des enfants qui leur ressemblent. Le jeune peuple qui s’éleva sous leurs yeux s’accrut par d’heureux mariages : le nombre augmenta, l’union fut toujours la même ; et la vertu, bien loin de s’affaiblir dans la multitude, fut fortifiée, au contraire, par un plus grand nombre d’exemples.\par
Qui pourrait représenter ici le bonheur de ces Troglodytes ? Un peuple si juste devait être chéri des dieux. Dès qu’il ouvrit les yeux pour les connaître, il apprit à les craindre ; et la religion vint adoucir dans les mœurs ce que la nature y avait laissé de trop rude.\par
Ils instituèrent des fêtes en l’honneur des dieux. Les jeunes filles, ornées de fleurs, et les jeunes garçons, les célébraient par leurs danses, et par les accords d’une musique champêtre ; on faisait ensuite des festins, où la joie ne régnait pas moins que la frugalité. C’était dans ces assemblées que parlait la nature naïve, c’est là qu’on apprenait à donner le cœur et à le recevoir ; c’est là que la pudeur virginale faisait en rougissant un aveu surpris, mais bientôt confirmé par le consentement des pères ; et c’est là que les tendres mères se plaisaient à prévoir par avance une union douce et fidèle.\par
On allait au temple pour demander les faveurs des dieux : ce n’était pas les richesses et une onéreuse abondance ; de pareils souhaits étaient indignes des heureux Troglodytes ; ils ne savaient les désirer que pour leurs compatriotes. Ils n’étaient au pied des autels que pour demander la santé de leurs pères, l’union de leurs frères, la tendresse de leurs femmes, l’amour et l’obéissance de leurs enfants. Les filles y venaient apporter le tendre sacrifice de leur cœur, et ne leur demandaient d’autre grâce que celle de pouvoir rendre un Troglodyte heureux.\par
Le soir, lorsque les troupeaux quittaient les prairies, et que les bœufs fatigués avaient ramené la charrue, ils s’assemblaient ; et, dans un repas frugal, ils chantaient les injustices des premiers Troglodytes et leurs malheurs, la vertu renaissante avec un nouveau peuple, et sa félicité : ils chantaient ensuite les grandeurs des dieux, leurs faveurs toujours présentes aux hommes qui les implorent, et leur colère inévitable à ceux qui ne les craignent pas ; ils décrivaient ensuite les délices de la vie champêtre, et le bonheur d’une condition toujours parée de l’innocence. Bientôt ils s’abandonnaient à un sommeil que les soins et les chagrins n’interrompaient jamais.\par
La nature ne fournissait pas moins à leurs désirs qu’à leurs besoins. Dans ce pays heureux, la cupidité était étrangère : ils se faisaient des présents, où celui qui donnait croyait toujours avoir l’avantage. Le peuple troglodyte se regardait comme une seule famille ; les troupeaux étaient presque toujours confondus ; la seule peine qu’on s’épargnait ordinairement, c’était de les partager.\par

\dateline{D’Erzeron, le 6 de la lune de Gemmai 2, 1711.}
\section[{Lettre XIII. Usbek au même.}]{Lettre XIII. Usbek au même.}\renewcommand{\leftmark}{Lettre XIII. Usbek au même.}

\noindent Je ne saurais assez te parler de la vertu des Troglodytes. Un d’eux disait un jour : Mon père doit demain labourer son champ ; je me lèverai deux heures avant lui, et quand il ira à son champ, il le trouvera tout labouré.\par
Un autre disait en lui-même : Il me semble que ma sœur a du goût pour un jeune Troglodyte de nos parents ; il faut que je parle à mon père, et que je le détermine à faire ce mariage.\par
On vint dire à un autre que des voleurs avaient enlevé son troupeau : J’en suis bien fâché, dit-il, car il y avait une génisse toute blanche que je voulais offrir aux dieux.\par
On entendait dire à un autre : Il faut que j’aille au temple remercier les dieux ; car mon frère, que mon père aime tant et que je chéris si fort, a recouvré la santé.\par
Ou bien : Il y a un champ qui touche celui de mon père, et ceux qui le cultivent sont tous les jours exposés aux ardeurs du soleil ; il faut que j’aille y planter deux arbres, afin que ces pauvres gens puissent aller quelquefois se reposer sous leur ombre.\par
Un jour que plusieurs Troglodytes étaient assemblés, un vieillard parla d’un jeune homme qu’il soupçonnait d’avoir commis une mauvaise action, et lui en fit des reproches. Nous ne croyons pas qu’il ait commis ce crime, dirent les jeunes Troglodytes ; mais, s’il l’a fait, puisse-t-il mourir le dernier de sa famille !\par
On vint dire à un Troglodyte que des étrangers avaient pillé sa maison, et avaient tout emporté. S’ils n’étaient pas injustes, répondit-il, je souhaiterais que les dieux leur en donnassent un plus long usage qu’à moi.\par
Tant de prospérités ne furent pas regardées sans envie : les peuples voisins s’assemblèrent ; et, sous un vain prétexte, ils résolurent d’enlever leurs troupeaux. Dès que cette résolution fut connue, les Troglodytes envoyèrent au-devant d’eux des ambassadeurs, qui leur parlèrent ainsi :\par
"Que vous ont fait les Troglodytes ? Ont-ils enlevé vos femmes, dérobé vos bestiaux, ravagé vos campagnes ? Non : nous sommes justes, et nous craignons les dieux. Que voulez-vous donc de nous ? Voulez-vous de la laine pour vous faire des habits ? Voulez-vous du lait de nos troupeaux, ou des fruits de nos terres ? Posez bas les armes ; venez au milieu de nous, et nous vous donnerons de tout cela. Mais nous jurons, par ce qu’il y a de plus sacré, que, si vous entrez dans nos terres comme ennemis, nous vous regarderons comme un peuple injuste, et que nous vous traiterons comme des bêtes farouches."\par
Ces paroles furent renvoyées avec mépris ; ces peuples sauvages entrèrent armés dans la terre des Troglodytes, qu’ils ne croyaient défendus que par leur innocence.\par
Mais ils étaient bien disposés à la défense. Ils avaient mis leurs femmes et leurs enfants au milieu d’eux. Ils furent étonnés de l’injustice de leurs ennemis, et non pas de leur nombre. Une ardeur nouvelle s’était emparée de leur cœur : l’un voulait mourir pour son père, un autre pour sa femme et ses enfants, celui-ci pour ses frères, celui-là pour ses amis, tous pour le peuple troglodyte ; la place de celui qui expirait était d’abord prise par un autre, qui, outre la cause commune, avait encore une mort particulière à venger.\par
Tel fut le combat de l’injustice et de la vertu. Ces peuples lâches, qui ne cherchaient que le butin, n’eurent pas honte de fuir ; et ils cédèrent à la vertu des Troglodytes, même sans en être touchés.\par

\dateline{D’Erzeron, le 9 de la lune de Gemmadi 2, 1711.}
\section[{Lettre XIV. Usbek au même.}]{Lettre XIV. Usbek au même.}\renewcommand{\leftmark}{Lettre XIV. Usbek au même.}

\noindent Comme le peuple grossissait tous les jours, les Troglodytes crurent qu’il était à propos de se choisir un roi : ils convinrent qu’il fallait déférer la couronne à celui qui était le plus juste ; et ils jetèrent tous les yeux sur un vieillard vénérable par son âge et par une longue vertu. Il n’avait pas voulu se trouver à cette assemblée ; il s’était retiré dans sa maison, le cœur serré de tristesse.\par
Lorsqu’on lui envoya des députés pour lui apprendre le choix qu’on avait fait de lui : À Dieu ne plaise, dit-il, que je fasse ce tort aux Troglodytes, que l’on puisse croire qu’il n’y a personne parmi eux de plus juste que moi ! Vous me déférez la couronne, et, si vous le voulez absolument, il faudra bien que je la prenne ; mais comptez que je mourrai de douleur d’avoir vu en naissant les Troglodytes libres, et de les voir aujourd’hui assujettis. À ces mots, il se mit à répandre un torrent de larmes. Malheureux jour ! disait-il ; et pourquoi ai-je tant vécu ? Puis il s’écria d’une voix sévère : Je vois bien ce que c’est, ô Troglodytes ! Votre vertu commence à vous peser. Dans l’état où vous êtes, n’ayant point de chef, il faut que vous soyez vertueux malgré vous ; sans cela vous sauriez subsister, et vous tomberiez dans le malheur de vos premiers pères. Mais ce joug vous paraît trop dur : vous aimez mieux être soumis à un prince, et obéir à ses lois, moins rigides que vos mœurs. Vous savez que pour lors vous pourrez contenter votre ambition, acquérir des richesses, et languir dans une lâche volupté ; et que, pourvu que vous évitiez de tomber dans les grands crimes, vous n’aurez pas besoin de la vertu. Il s’arrêta un moment, et ses larmes coulèrent plus que jamais. Et que prétendez-vous que je fasse ? Comment se peut-il que je commande quelque chose à un Troglodyte ? Voulez-vous qu’il fasse une action vertueuse parce que je la lui commande, lui qui la ferait tout de même sans moi, et par le seul penchant de la nature ? Ô Troglodytes ! je suis à la fin de mes jours, mon sang est glacé dans mes veines, je vais bientôt revoir vos sacrés aïeux : pourquoi voulez-vous que je les afflige, et que je sois obligé de leur dire que je vous ai laissés sous un autre joug que celui de la vertu ?\par

\dateline{D’Erzeron, le 10 de la lune de Gemmadi 2, 1711.}
\section[{Lettre XV. Le premier eunuque à Jaron, eunuque noir. À Erzeron.}]{Lettre XV. Le premier eunuque à Jaron, eunuque noir. À Erzeron.}\renewcommand{\leftmark}{Lettre XV. Le premier eunuque à Jaron, eunuque noir. À Erzeron.}

\noindent Je prie le ciel qu’il te ramène dans ces lieux, et te dérobe à tous les dangers.\par
Quoique je n’aie guère jamais connu cet engagement qu’on appelle amitié, et que je me sois enveloppé tout entier dans moi-même, tu m’as cependant fait sentir que j’avais encore un cœur ; et, pendant que j’étais de bronze pour tous ces esclaves qui vivaient sous mes lois, je voyais croître ton enfance avec plaisir.\par
Le temps vint où mon maître jeta sur toi les yeux. Il s’en fallait bien que la nature eût encore parlé, lorsque le fer te sépara de la nature. Je ne te dirai point si je te plaignis, ou si je sentis du plaisir à te voir élevé jusqu’à moi. J’apaisai tes pleurs et tes cris. Je crus te voir prendre une seconde naissance, et sortir d’une servitude où tu devais toujours obéir, pour entrer dans une servitude où tu devais commander. Je pris soin de ton éducation. La sévérité, toujours inséparable des instructions, te fit longtemps ignorer que tu m’étais cher. Tu me l’étais pourtant ; et je te dirai que je t’aimais comme un père aime son fils, si ces noms de père et de fils pouvaient convenir à notre destinée.\par
Tu vas parcourir les pays habités par les chrétiens, qui n’ont jamais cru. Il est impossible que tu n’y contractes bien des souillures. Comment le prophète pourrait-il te regarder au milieu de tant de millions de ses ennemis ? Je voudrais que mon maître fît, à son retour, le pèlerinage de la Mecque : vous vous purifieriez tous dans la terre des anges.\par

\dateline{Du sérail d’Ispahan, le 10 de la lune de Gemmadi 2, 1711.}
\section[{Lettre XVI. Usbek au mollak Mehemet Ali, gardien des Trois Tombeaux. À Com.}]{Lettre XVI. Usbek au mollak Mehemet Ali, gardien des Trois Tombeaux. À Com.}\renewcommand{\leftmark}{Lettre XVI. Usbek au mollak Mehemet Ali, gardien des Trois Tombeaux. À Com.}

\noindent Pourquoi vis-tu dans les tombeaux, divin mollak ? Tu es bien plus fait pour le séjour des étoiles. Tu te caches sans doute de peur d’obscurcir le soleil : tu n’as point de taches comme cet astre ; mais, comme lui, tu te couvres de nuages.\par
Ta science est un abîme plus profond que l’Océan ; ton esprit est plus perçant que Zufagar, cette épée d’Ali, qui avait deux pointes ; tu sais ce qui se passe dans les neuf chœurs des puissances célestes ; tu lis l’Alcoran sur la poitrine de notre divin prophète ; et, lorsque tu trouves quelque passage obscur, un ange, par son ordre, déploie ses ailes rapides, et descend du trône pour t’en révéler le secret.\par
Je pourrais par ton moyen avoir avec les séraphins une intime correspondance : car enfin, treizième immaum, n’es-tu pas le centre où le ciel et la terre aboutissent, et le point de communication entre l’abîme et l’empyrée ?\par
Je suis au milieu d’un peuple profane : permets que je me purifie avec toi ; souffre que je tourne mon visage vers les lieux sacrés que tu habites ; distingue-moi des méchants, comme on distingue au lever de l’aurore, le filet blanc d’avec le filet noir ; aide-moi de tes conseils ; prends soin de mon âme ; enivre-la de l’esprit des prophètes ; nourris-la de la science du paradis, et permets que je mette ses plaies à tes pieds. Adresse tes lettres sacrées à Erzeron, où je resterai quelques mois.\par

\dateline{D’Erzeron, le 11 de la lune de Gemmadi 2, 1711.}
\section[{Lettre XVII. Usbek au même.}]{Lettre XVII. Usbek au même.}\renewcommand{\leftmark}{Lettre XVII. Usbek au même.}

\noindent Je ne puis, divin mollak, calmer mon impatience : je ne saurais attendre ta sublime réponse. J’ai des doutes, il faut les fixer : je sens que ma raison s’égare ; ramène-la dans le droit chemin ; viens m’éclairer, source de lumière ; foudroie avec ta plume divine les difficultés que je vais te proposer ; fais-moi pitié de moi-même, et rougir de la question que je vais faire.\par
D’où vient que notre législateur nous prive de la chair de pourceau, et de toutes les viandes qu’il appelle immondes ? D’où vient qu’il nous défend de toucher un corps mort, et que, pour purifier notre âme, il nous ordonne de nous laver sans cesse le corps ? Il me semble que les choses ne sont en elles-mêmes ni pures ni impures : je ne puis concevoir aucune qualité inhérente au sujet qui puisse les rendre telles. La boue ne nous paraît sale que parce qu’elle blesse notre vue, ou quelque autre de nos sens ; mais, en elle-même, elle ne l’est pas plus que l’or et les diamants. L’idée de souillure contractée par l’attouchement d’un cadavre ne nous est venue que d’une certaine répugnance naturelle que nous en avons. Si les corps de ceux qui ne se lavent point ne blessaient ni l’odorat ni la vue, comment aurait-on pu s’imaginer qu’ils fussent impurs ?\par
Les sens, divin mollak, doivent donc être les seuls juges de la pureté ou de l’impureté des choses ? Mais, comme les objets n’affectent point les hommes de la même manière ; que ce qui donne une sensation agréable aux uns en produit une dégoûtante chez les autres, il suit que le témoignage des sens ne peut servir ici de règle, à moins qu’on ne dise que chacun peut à sa fantaisie décider ce point, et distinguer, pour ce qui le concerne, les choses pures d’avec celles qui ne le sont pas.\par
Mais cela même, sacré mollak, ne renverserait-il pas les distinctions établies par notre divin prophète, et les points fondamentaux de la loi qui a été écrite de la main des anges ?\par

\dateline{D’Erzeron, le 20 de la lune de Gemmadi 2, 1711}
\section[{Lettre XVIII. Mehemet Ali, serviteur des prophètes, à Usbek.}]{Lettre XVIII. Mehemet Ali, serviteur des prophètes, à Usbek.}\renewcommand{\leftmark}{Lettre XVIII. Mehemet Ali, serviteur des prophètes, à Usbek.}

\noindent Vous nous faites toujours des questions qu’on a faites mille fois à notre saint prophète. Que ne lisez-vous les traditions des docteurs ? que n’allez-vous à cette source pure de toute intelligence ? vous trouveriez tous vos doutes résolus.\par
Malheureux, qui, toujours embarrassés des choses de la terre, n’avez jamais regardé d’un œil fixe celles du ciel, et qui révérez la condition des mollaks, sans oser ni l’embrasser ni la suivre !\par
Profanes, qui n’entrez jamais dans les secrets de l’Éternel, vos lumières ressemblent aux ténèbres de l’abîme, et les raisonnements de votre esprit sont comme la poussière que vos pieds font élever lorsque le soleil est dans son midi, dans le mois ardent de Chahban.\par
Aussi le zénith de votre esprit ne va pas au nadir de celui du moindre des immaums\footnote{Ce mot est plus en usage chez les Turcs que chez les Persans.}. Votre vaine philosophie est cet éclair qui annonce l’orage et l’obscurité : vous êtes au milieu de la tempête, et vous errez au gré des vents.\par
Il est bien facile de répondre à votre difficulté : il ne faut pour cela que vous raconter ce qui arriva un jour à notre saint prophète, lorsque, tenté par les chrétiens, éprouvé par les juifs, il confondit également et les uns et les autres.\par
Le juif Abdias Ibesalon lui demanda pourquoi Dieu avait défendu de manger de la chair de pourceau. Ce n’est pas sans raison, reprit le prophète : c’est un animal immonde ; et je vais vous en convaincre. Il fit sur sa main, avec de la boue, la figure d’un homme ; il le jeta à terre, et lui cria : Levez-vous ! Sur-le-champ un homme se leva, et dit : Je suis Japhet, fils de Noé. Avais-tu les cheveux aussi blancs quand tu es mort ? lui dit le saint prophète. Non, répondit-il : mais, quand tu m’as réveillé, j’ai cru que le jour du jugement était venu : et j’ai eu une si grande frayeur, que mes cheveux ont blanchi tout à coup.\par
Or çà, raconte-moi, dit l’envoyé de Dieu, toute l’histoire de l’arche de Noé. Japhet obéit, et détailla exactement tout ce qui s’était passé les premiers mois ; après quoi il parla ainsi :\par
Nous mîmes les ordures de tous les animaux dans un côté de l’arche ; ce qui la fit si fort pencher, que nous en eûmes une peur mortelle, surtout nos femmes, qui se lamentaient de la belle manière. Notre père Noé ayant été au conseil de Dieu, il lui commanda de prendre l’éléphant, de lui faire tourner la tête vers le côté qui penchait. Ce grand animal fit tant d’ordures, qu’il en naquit un cochon. Croyez-vous, Usbek, que depuis ce temps-là nous nous en soyons abstenus, et que nous l’ayons regardé comme un animal immonde ?\par
Mais comme le cochon remuait tous les jours ces ordures, il s’éleva une telle puanteur dans l’arche, qu’il ne put lui-même s’empêcher d’éternuer ; et il sortit de son nez un rat, qui allait rongeant tout ce qui se trouvait devant lui : ce qui devint si insupportable à Noé, qu’il crut qu’il était à propos de consulter Dieu encore. Il lui ordonna de donner au lion un grand coup sur le front, qui éternua aussi, et fit sortir de son nez un chat. Croyez-vous que ces animaux soient encore immondes ? Que vous en semble ?\par
Quand donc vous n’apercevez pas la raison de l’impureté de certaines choses, c’est que vous en ignorez beaucoup d’autres, et que vous n’avez pas la connaissance de ce qui s’est passé entre Dieu, les anges et les hommes. Vous ne savez pas l’histoire de l’éternité ; vous n’avez point lu les livres qui sont écrits au ciel ; ce qui vous en a été révélé n’est qu’une petite partie de la bibliothèque divine ; et ceux qui, comme nous, en approchent de plus près, tandis qu’ils sont en cette vie, sont encore dans l’obscurité et les ténèbres. Adieu. Mahomet soit dans votre cœur.\par

\dateline{À Com, le dernier de la lune de Chahban, 1711.}
\section[{Lettre XIX. Usbek à son ami Rustan. À Ispahan.}]{Lettre XIX. Usbek à son ami Rustan. À Ispahan.}\renewcommand{\leftmark}{Lettre XIX. Usbek à son ami Rustan. À Ispahan.}

\noindent Nous n’avons séjourné que huit jours à Tocat : après trente-cinq jours de marche, nous sommes arrivés à Smyrne.\par
De Tocat à Smyrne, on ne trouve pas une seule ville qui mérite qu’on la nomme. J’ai vu avec étonnement la faiblesse de l’empire des Osmanlins. Ce corps malade ne se soutient pas par un régime doux et tempéré, mais par des remèdes violents, qui l’épuisent et le minent sans cesse.\par
Les pachas, qui n’obtiennent leurs emplois qu’à force d’argent, entrent ruinés dans les provinces, et les ravagent comme des pays de conquête. Une milice insolente n’est soumise qu’à ses caprices. Les places sont démantelées, les villes désertes, les campagnes désolées, la culture des terres et le commerce entièrement abandonnés.\par
L’impunité règne dans ce gouvernement sévère : les chrétiens qui cultivent les terres, les juifs qui lèvent les tributs, sont exposés à mille violences.\par
La propriété des terres est incertaine, et, par conséquent, l’ardeur de les faire valoir ralentie : il n’y a ni titre, ni possession, qui vaillent contre le caprice de ceux qui gouvernent.\par
Ces barbares ont tellement abandonné les arts, qu’ils ont négligé jusques à l’art militaire. Pendant que les nations d’Europe se raffinent tous les jours, ils restent dans leur ancienne ignorance, et ils ne s’avisent de prendre leurs nouvelles inventions qu’après qu’elles s’en sont servies mille fois contre eux.\par
Ils n’ont nulle expérience sur la mer, nulle habileté dans la manœuvre. On dit qu’une poignée de chrétiens sortis d’un rocher\footnote{Ce sont apparemment les chevaliers de Malte.} font suer tous les Ottomans, et fatiguent leur empire.\par
Incapables de faire le commerce, ils souffrent presque avec peine que les Européens, toujours laborieux et entreprenants, viennent le faire : ils croient faire grâce à ces étrangers de permettre qu’ils les enrichissent.\par
Dans toute cette vaste étendue de pays que j’ai traversée, je n’ai trouvé que Smyrne qu’on puisse regarder comme une ville riche et puissante. Ce sont les Européens qui la rendent telle, et il ne tient pas aux Turcs qu’elle ne ressemble à toutes les autres.\par
Voilà, cher Rustan, une juste idée de cet empire, qui, avant deux siècles, sera le théâtre des triomphes de quelque conquérant.\par
À Smyrne, le 2 de la lune de Rhamazan, 1711.
\section[{Lettre XX. Usbek à Zachi, sa femme. Au sérail d’Ispahan.}]{Lettre XX. Usbek à Zachi, sa femme. Au sérail d’Ispahan.}\renewcommand{\leftmark}{Lettre XX. Usbek à Zachi, sa femme. Au sérail d’Ispahan.}

\noindent Vous m’avez offensé, Zachi ; et je sens dans mon cœur des mouvements que vous devriez craindre, si mon éloignement ne vous laissait le temps de changer de conduite, et d’apaiser la violente jalousie dont je suis tourmenté.\par
J’apprends qu’on vous a trouvée seule avec Nadir, eunuque blanc, qui payera de sa tête son infidélité et sa perfidie. Comment vous êtes-vous oubliée jusqu’à ne pas sentir qu’il ne vous est pas permis de recevoir dans votre chambre un eunuque blanc, tandis que vous avez de noirs destinés à vous servir ? Vous avez beau me dire que des eunuques ne sont pas des hommes, et que votre vertu vous met au-dessus des pensées que pourrait faire naître en vous une ressemblance imparfaite ; cela ne suffit ni pour vous ni pour moi : pour vous, parce que vous faites une chose que les lois du sérail vous défendent ; pour moi, en ce que vous m’ôtez l’honneur, en vous exposant à des regards ; que dis-je, à des regards ? peut-être aux entreprises d’un perfide qui vous aura souillée par ses crimes et plus encore par ses regrets et le désespoir de son impuissance.\par
Vous me direz peut-être que vous m’avez été toujours fidèle. Eh ! pouviez-vous ne l’être pas ? Comment auriez-vous trompé la vigilance des eunuques noirs, qui sont si surpris de la vie que vous menez ? Comment auriez-vous pu briser ces verrous et ces portes qui vous tiennent enfermée ? Vous vous vantez d’une vertu qui n’est pas libre : et peut-être que vos désirs impurs vous ont ôté mille fois le mérite et le prix de cette fidélité que vous vantez tant.\par
Je veux que vous n’ayez point fait tout ce que j’ai lieu de soupçonner ; que ce perfide n’ait point porté sur vous ses mains sacrilèges ; que vous ayez refusé de prodiguer à sa vue les délices de son maître ; que, couverte de vos habits, vous ayez laissé cette faible barrière entre lui et vous ; que, frappé lui-même d’un saint respect, il ait baissé les yeux ; que, manquant à sa hardiesse, il ait tremblé sur les châtiments qu’il se prépare : quand tout cela serait vrai, il ne l’est pas moins que vous avez fait une chose qui est contre votre devoir. Et, si vous l’avez violé gratuitement sans remplir vos inclinations déréglées, qu’eussiez-vous fait pour les satisfaire ? Que feriez-vous encore si vous pouviez sortir de ce lieu sacré, qui est pour vous une dure prison, comme il est pour vos compagnes un asile favorable contre les atteintes du vice, un temple sacré où votre sexe perd sa faiblesse, et se trouve invincible, malgré tous les désavantages de la nature ? Que feriez-vous si, laissée à vous-même, vous n’aviez pour vous défendre que votre amour pour moi, qui est si grièvement offensé, et votre devoir, que vous avez si indignement trahi ? Que les mœurs du pays où vous vivez sont saintes, qui vous arrachent à l’attentat des plus vils esclaves ! Vous devez me rendre grâce de la gêne où je vous fais vivre, puisque ce n’est que par là que vous méritez encore de vivre.\par
Vous ne pouvez souffrir le chef des eunuques, parce qu’il a toujours les yeux sur votre conduite, et qu’il vous donne ses sages conseils. Sa laideur, dites-vous, est si grande que vous ne pouvez le voir sans peine : comme si, dans ces sorte de postes, on mettait de plus beaux objets. Ce qui vous afflige est de n’avoir pas à sa place l’eunuque blanc qui vous déshonore.\par
Mais que vous a fait votre première esclave ? Elle vous a dit que les familiarités que vous preniez avec la jeune Zélide étaient contre la bienséance : voilà la raison de votre haine.\par
Je devrais être, Zachi, un juge sévère ; je ne suis qu’un époux qui cherche à vous trouver innocente. L’amour que j’ai pour Roxane, ma nouvelle épouse, m’a laissé toute la tendresse que je dois avoir pour vous, qui n’êtes pas moins belle. Je partage mon amour entre vous deux ; et Roxane n’a d’autre avantage que celui que la vertu peut ajouter à la beauté.\par

\dateline{À Smyrne, le 12 de la lune de Zilcadé, 1711.}
\section[{Lettre XXI. Usbek au premier eunuque blanc.}]{Lettre XXI. Usbek au premier eunuque blanc.}\renewcommand{\leftmark}{Lettre XXI. Usbek au premier eunuque blanc.}

\noindent Vous devez trembler à l’ouverture de cette lettre, ou plutôt vous le deviez lorsque vous souffrîtes la perfidie de Nadir. Vous qui, dans une vieillesse froide et languissante, ne pouvez sans crime lever les yeux sur les redoutables objets de mon amour ; vous à qui il n’est jamais permis de mettre un pied sacrilège sur la porte du lieu terrible qui les dérobe à tous les regards, vous souffrez que ceux dont la conduite vous est confiée aient fait ce que vous n’auriez pas la témérité de faire, et vous n’apercevez pas la foudre toute prête à tomber sur eux et sur vous ?\par
Et qui êtes-vous, que de vils instruments que je puis briser à ma fantaisie ; qui n’existez qu’autant que vous savez obéir ; qui n’êtes dans le monde que pour vivre sous mes lois, ou pour mourir dès que je l’ordonne ; qui ne respirez qu’autant que mon bonheur, mon amour, ma jalousie même, ont besoin de votre bassesse ; et enfin qui ne pouvez avoir d’autre partage que la soumission, d’autre âme que mes volontés, d’autre espérance que ma félicité ?\par
Je sais que quelques-unes de mes femmes souffrent impatiemment les lois austères du devoir : que la présence continuelle d’un eunuque noir les ennuie ; qu’elles sont fatiguée de ces objets affreux, qui leurs sont donnés pour les ramener à leur époux ; je le sais : mais vous qui vous prêtez à ce désordre, vous serez puni d’une manière à faire trembler tous ceux qui abusent de ma confiance.\par
Je jure par tous les prophètes du ciel, et par Ali, le plus grand de tous, que, si vous vous écartez de votre devoir, je regarderai votre vie comme celle des insectes que je trouve sous mes pieds.\par

\dateline{À Smyrne, le 13 de la lune de Zilcadé, 1711.}
\section[{Lettre XXII. Jaron au premier eunuque.}]{Lettre XXII. Jaron au premier eunuque.}\renewcommand{\leftmark}{Lettre XXII. Jaron au premier eunuque.}

\noindent À mesure qu’Usbek s’éloigne du sérail, il tourne sa tête vers ses femmes sacrées ; il soupire, il verse des larmes ; sa douleur s’aigrit, ses soupçons se fortifient. Il veut augmenter le nombre de leurs gardiens. Il va me renvoyer, avec tous les noirs qui l’accompagnent. Il ne craint plus pour lui ; il craint pour ce qui lui est mille fois plus cher que lui-même.\par
Je vais donc vivre sous tes lois, et partager tes soins. Grand Dieu ! qu’il faut de choses pour rendre un seul homme heureux !\par
La nature semblait avoir mis les femmes dans la dépendance, et les en avoir retirées : le désordre naissait entre les deux sexes, parce que leurs droits étaient réciproques. Nous sommes entrés dans le plan d’une nouvelle harmonie : nous avons mis entre les femmes et nous la haine ; et entre les hommes et les femmes, l’amour.\par
Mon front va devenir sévère. Je laisserai tomber des regards sombres. La joie fuira de mes lèvres. Le dehors sera tranquille, et l’esprit inquiet. Je n’attendrai point les rides de la vieillesse pour en montrer les chagrins.\par
J’aurais eu du plaisir à suivre mon maître dans l’Occident ; mais ma volonté est son bien. Il veut que je garde ses femmes ; je les garderai avec fidélité. Je sais comment je dois me conduire avec ce sexe qui, quand on ne lui permet pas d’être vain, commence à devenir superbe, et qu’il est moins aisé d’humilier que d’anéantir. Je tombe sous tes regards.\par

\dateline{De Smyrne, le 12 de la lune de Zilcadé, 1711.}
\section[{Lettre XXIII. Usbek à son ami Ibben.}]{Lettre XXIII. Usbek à son ami Ibben.}\renewcommand{\leftmark}{Lettre XXIII. Usbek à son ami Ibben.}

\noindent Nous sommes arrivés à Livourne dans quarante jours de navigation. C’est une ville nouvelle ; elle est un témoignage du génie des ducs de Toscane, qui ont fait d’un village marécageux la ville d’Italie la plus florissante.\par
Les femmes y jouissent d’une grande liberté : elles peuvent voir les hommes à travers certaines fenêtres qu’on nomme jalousies, elles peuvent sortir tous les jours avec quelques vieilles qui les accompagnent : elles n’ont qu’un voile\footnote{(Les Persanes en ont quatre.)}. Leurs beaux-frères, leurs oncles, leurs neveux peuvent les voir sans que le mari s’en formalise presque jamais.\par
C’est un grand spectacle pour un mahométan de voir pour la première fois une ville chrétienne. Je ne parle pas des choses qui frappent d’abord tous les yeux, comme la différence des édifices, des habits, des principales coutumes : il y a, jusque dans les moindres bagatelles, quelque chose de singulier que je sens et que je ne sais pas dire.\par
Nous partirons demain pour Marseille : notre séjour n’y sera pas long. Le dessein de Rica et le mien est de nous rendre incessamment à Paris, qui est le siège de l’empire d’Europe. Les voyageurs cherchent toujours les grandes villes, qui sont une espèce de patrie commune à tous les étrangers. Adieu. Sois persuadé que je t’aimerai toujours.\par

\dateline{À Livourne, le 12 de la lune de Saphar, 1712.}
\section[{Lettre XXIV. Rica à Ibben. À Smyrne.}]{Lettre XXIV. Rica à Ibben. À Smyrne.}\renewcommand{\leftmark}{Lettre XXIV. Rica à Ibben. À Smyrne.}

\noindent Nous sommes à Paris depuis un mois, et nous avons toujours été dans un mouvement continuel. Il faut bien des affaires avant qu’on soit logé, qu’on ait trouvé les gens à qui on est adressé, et qu’on se soit pourvu des choses nécessaires, qui manquent toutes à la fois.\par
Paris est aussi grand qu’Ispahan : les maisons y sont si hautes, qu’on jugerait qu’elles ne sont habitées que par des astrologues. Tu juges bien qu’une ville bâtie en l’air, qui a six ou sept maisons les unes sur les autres, est extrêmement peuplée ; et que, quand tout le monde est descendu dans la rue, il s’y fait un bel embarras.\par
Tu ne le croirais pas peut-être, depuis un mois que je suis ici, je n’y ai encore vu marcher personne. Il n’y a pas de gens au monde qui tirent mieux partie de leur machine que les Français ; ils courent, ils volent : les voitures lentes d’Asie, le pas réglé de nos chameaux, les feraient tomber en syncope. Pour moi, qui ne suis point fait à ce train, et qui vais souvent à pied sans changer d’allure, j’enrage quelquefois comme un chrétien : car encore passe qu’on m’éclabousse depuis les pieds jusqu’à la tête ; mais je ne puis pardonner les coups de coude que je reçois régulièrement et périodiquement. Un homme qui vient après moi et qui me passe me fait faire un demi-tour ; et un autre qui me croise de l’autre côté me remet soudain où le premier m’avait pris ; et je n’ai pas fait cent pas, que je suis plus brisé que si j’avais fait dix lieues.\par
Ne crois pas que je puisse, quant à présent, te parler à fond des mœurs et des coutumes européennes : je n’en ai moi-même qu’une légère idée, et je n’ai eu à peine que le temps de m’étonner.\par
Le roi de France est le plus puissant prince de l’Europe. Il n’a point de mines d’or comme le roi d’Espagne son voisin ; mais il a plus de richesses que lui, parce qu’il les tire de la vanité de ses sujets, plus inépuisable que les mines. On lui a vu entreprendre ou soutenir de grandes guerres, n’ayant d’autres fonds que des titres d’honneur à vendre ; et, par un prodige de l’orgueil humain, ses troupes se trouvaient payées, ses places munies, et ses flottes équipées.\par
D’ailleurs ce roi est un grand magicien : il exerce son empire sur l’esprit même de ses sujets ; il les fait penser comme il veut. S’il n’a qu’un million d’écus dans son trésor et qu’il en ait besoin de deux, il n’a qu’à leur persuader qu’un écu en vaut deux, et ils le croient. S’il a une guerre difficile à soutenir, et qu’il n’ait point d’argent, il n’a qu’à leur mettre dans la tête qu’un morceau de papier est de l’argent, et ils en sont aussitôt convaincus. Il va même jusqu’à leur faire croire qu’il les guérit de toutes sortes de maux en les touchant, tant est grande la force et la puissance qu’il a sur les esprits.\par
Ce que je dis de ce prince ne doit pas t’étonner : il y a un autre magicien plus fort que lui, qui n’est pas moins maître de son esprit qu’il l’est lui-même de celui des autres. Ce magicien s’appelle le pape : tantôt il lui fait croire que trois ne sont qu’un ; que le pain qu’on mange n’est pas du pain, ou que le vin qu’on boit n’est pas du vin, et mille autres choses de cette espèce\footnote{Il faut qu’un Turc voie, parle et pense en Turc: c’est à quoi des gens ne font point attention en lisant les {\itshape Lettres Persanes}. (Mont., {\itshape Lettre à l’abbé de Guasco, du 4 octobre 1752.})}.\par
Et, pour le tenir toujours en haleine et ne point lui laisser perdre l’habitude de croire, il lui donne de temps en temps, pour l’exercer, de certains articles de croyance. IL y a deux ans qu’il lui envoya un grand écrit qu’il appela {\itshape constitution}, et voulut obliger, sous de grandes peines, ce prince et ses sujets de croire tout ce qui y était contenu. Il réussit à l’égard du prince, qui se soumit aussitôt, et donna l’exemple à ses sujets ; mais quelques-uns d’entre eux se révoltèrent, et dirent qu’ils ne voulaient rien croire de tout ce qui était dans cet écrit. Ce sont les femmes qui ont été les motrices de toute cette révolte qui divise toute la cour, tout le royaume et toutes les familles. Cette {\itshape constitution} leur défend de lire un livre que tous les chrétiens disent avoir été apporté du ciel : c’est proprement leur Alcoran. Les femmes, indignées de l’outrage fait à leur sexe, soulèvent tout contre la {\itshape constitution} : elles ont mis les hommes de leur parti, qui, dans cette occasion, ne veulent point avoir de privilège. Il faut pourtant avouer que ce moufti ne raisonne pas mal ; et, par le grand Ali, il faut qu’il ait été instruit des principes de notre sainte loi : car, puisque les femmes sont d’une création inférieure à la nôtre, et que nos prophètes nous disent qu’elles n’entreront point dans le paradis, pourquoi faut-il qu’elles se mêlent de lire un livre qui n’est fait que pour apprendre le chemin du paradis ?\par
J’ai ouï raconter du roi des choses qui tiennent du prodige, et je ne doute pas que tu ne balances à les croire.\par
On dit que, pendant qu’il faisait la guerre à ses voisins, qui s’étaient tous ligués contre lui, il avait dans son royaume un nombre innombrable d’ennemis invisibles qui l’entouraient ; on ajoute qu’il les a cherchés pendant plus de trente ans, et que, malgré les soins infatigables de certains dervis qui ont sa confiance, il n’en a pu trouver un seul. Ils vivent avec lui : ils sont à sa cour, dans sa capitale, dans ses troupes, dans ses tribunaux ; et cependant on dit qu’il aura le chagrin de mourir sans les avoir trouvés. On dirait qu’ils existent en général, et qu’ils ne sont plus rien en particulier : c’est un corps ; mais point de membres. Sans doute que le ciel veut punir ce prince de n’avoir pas été assez modéré envers les ennemis qu’il a vaincus, puisqu’il lui en donne d’invisibles, et dont le génie et le destin sont au-dessus du sien.\par
Je continuerai à t’écrire, et je t’apprendrai des choses bien éloignées du caractère et du génie persan. C’est bien la même terre qui nous porte tous deux ; mais les hommes du pays où je vis, et ceux du pays où tu es, sont des hommes bien différents.\par

\dateline{De Paris, le 4 de la lune de Rebiab 2, 1712.}
\section[{Lettre XXV. Usbek à Ibben. À Smyrne.}]{Lettre XXV. Usbek à Ibben. À Smyrne.}\renewcommand{\leftmark}{Lettre XXV. Usbek à Ibben. À Smyrne.}

\noindent J’ai reçu une lettre de ton neveu Rhédi : il me mande qu’il quitte Smyrne, dans le dessein de voir l’Italie ; que l’unique but de son voyage est de s’instruire, et de se rendre par là plus digne de toi. Je te félicite d’avoir un neveu qui sera quelque jour la consolation de ta vieillesse.\par
Rica t’écrit une longue lettre ; il m’a dit qu’il te parlait beaucoup de ce pays-ci. La vivacité de son esprit fait qu’il saisit tout avec promptitude : pour moi, qui pense plus lentement, je ne suis pas en état de te rien dire.\par
Tu es le sujet de nos conversations les plus tendres : nous ne pouvons assez parler du bon accueil que tu nous as fait à Smyrne, et des services que ton amitié nous rend tous les jours. Puisses-tu, généreux Ibben, trouver partout des amis aussi reconnaissants et aussi fidèles que nous !\par
Puissé-je te revoir bientôt, et retrouver avec toi ces jours heureux qui coulent si doucement entre deux amis ! Adieu.\par

\dateline{De Paris, le 4 de la lune de Rebiab 2, 1712.}
\section[{Lettre XXVI. Usbek à Roxane. Au sérail d’Ispahan.}]{Lettre XXVI. Usbek à Roxane. Au sérail d’Ispahan.}\renewcommand{\leftmark}{Lettre XXVI. Usbek à Roxane. Au sérail d’Ispahan.}

\noindent Que vous êtes heureuse, Roxane, d’être dans le doux pays de Perse, et non pas dans ces climats empoisonnés où l’on ne connaît ni la pudeur ni la vertu ! Que vous êtes heureuse ! Vous vivez dans mon sérail comme dans le séjour de l’innocence, inaccessible aux attentats de tous les humains ; vous vous trouver avec joie dans une heureuse impuissance de faillir ; jamais homme ne vous a souillée de ses regards lascifs : votre beau-père même, dans la liberté des festins, n’a jamais vu votre belle bouche : vous n’avez jamais manqué de vous attacher un bandeau sacré pour la couvrir. Heureuse Roxane, quand vous avez été à la campagne, vous avez toujours eu des eunuques, qui ont marché devant vous, pour donner la mort à tous les téméraires qui n’ont pas fui votre vue. Moi-même, à qui le ciel vous a donnée pour faire mon bonheur, quelle peine n’ai-je pas eue pour me rendre maître de ce trésor, que vous défendiez avec tant de constance ! Quel chagrin pour moi, dans les premiers jours de notre mariage, de ne pas vous voir ! Et quelle impatience quand je vous eus vue ! Vous ne la satisfaisiez pourtant pas ; vous l’irritiez, au contraire, par les refus obstinés d’une pudeur alarmée : vous me confondiez avec tous ces hommes à qui vous vous cachez sans cesse. Vous souvient-il de ce jour où je vous perdis parmi vos esclaves, qui me trahirent, et vous dérobèrent à mes recherches ? Vous souvient-il de cet autre où, voyant vos larmes impuissantes, vous employâtes l’autorité de votre mère pour arrêter les fureurs de mon amour ? Vous souvient-il, lorsque toutes les ressources vous manquèrent, de celles que vous trouvâtes dans votre courage ? Vous mîtes le poignard à la main, et menaçâtes d’immoler un époux qui vous aimait, s’il continuait à exiger de vous ce que vous chérissiez plus que votre époux même. Deux mois se passèrent dans ce combat de l’amour et de la vertu. Vous poussâtes trop loin vos chastes scrupules : vous ne vous rendîtes pas même après avoir été vaincue ; vous défendîtes jusqu’à la dernière extrémité une virginité mourante : vous me regardâtes comme un ennemi qui vous avait fait un outrage ; non pas comme un époux qui vous avait aimée ; vous fûtes plus de trois mois que vous n’osiez me regarder sans rougir : votre air confus semblait me reprocher l’avantage que j’avais pris. Je n’avais pas même une possession tranquille ; vous me dérobiez tout ce que vous pouviez de ces charmes et de ces grâces ; et j’étais enivré des plus grandes faveurs sans avoir obtenu les moindres.\par
Si vous aviez été élevée dans ce pays-ci, vous n’auriez pas été si troublée : les femmes y ont perdu toute retenue : elles se présentent devant les hommes à visage découvert, comme si elles voulaient demander leur défaite ; elles les cherchent de leurs regards ; elles les voient dans les mosquées, les promenades, chez elles même ; l’usage de se faire servir par des eunuques leur est inconnu. Au lieu de cette noble simplicité et de cette aimable pudeur qui règne parmi vous, on voit une impudence brutale à laquelle il est impossible de s’accoutumer.\par
Oui, Roxane, si vous étiez ici, vous vous sentiriez outragée dans l’affreuse ignominie où votre sexe est descendu ; vous fuiriez ces abominables lieux, et vous soupireriez pour cette douce retraite, où vous trouvez l’innocence, où vous êtes sûre de vous-même, où nul péril ne vous fait trembler, où enfin vous pouvez m’aimer sans craindre de perdre jamais l’amour que vous me devez.\par
Quand vous relevez l’éclat de votre teint par les plus belles couleurs ; quand vous vous parfumez tout le corps des essences les plus précieuses ; quand vous vous parez de vos plus beaux habits ; quand vous cherchez à vous distinguer de vos compagnes par les grâces de la danse et par la douceur de votre chant ; que vous combattez gracieusement avec elles de charmes, de douceur et d’enjouement, je ne puis pas m’imaginer que vous ayez d’autre objet que celui de me plaire ; et quand je vous vois rougir modestement, que vos regards cherchent les miens, que vous vous insinuez dans mon cœur par des paroles douces et flatteuses, je ne saurais, Roxane, douter de votre amour.\par
Mais que puis-je penser des femmes d’Europe ? L’art de composer leur teint, les ornements dont elles se parent, les soins qu’elles prennent de leur personne, le désir continuel de plaire qui les occupe, sont autant de tâches faites à leur vertu et d’outrages à leurs époux.\par
Ce n’est pas, Roxane, que je pense qu’elles poussent l’attentat aussi loin qu’une pareille conduite devrait le faire croire, et qu’elles portent la débauche à cet excès horrible, qui fait frémir, de violer absolument la foi conjugale. Il y a bien peu de femmes assez abandonnées pour porter le crime si loin : elles portent toutes dans leur cœur un certain caractère de vertu qui y est gravé, que la naissance donne et que l’éducation affaiblit, mais ne détruit pas. Elles peuvent bien se relâcher des devoirs extérieurs que la pudeur exige ; mais, quand il s’agit de faire les derniers pas, la nature se révolte. Aussi, quand nous vous enfermons si étroitement, que nous vous faisons garder par tant d’esclaves, que nous gênons si fort vos désirs lorsqu’ils volent trop loin, ce n’est pas que nous craignions la dernière infidélité, mais c’est que nous savons que la pureté ne saurait être trop grande, et que la moindre tache peut la corrompre.\par
 Je vous plains, Roxane. Votre chasteté, si longtemps éprouvée, méritait un époux qui ne vous eût jamais quittée, et qui pût lui-même réprimer les désirs que votre seule vertu sait soumettre.\par

\dateline{De Paris, le 7 de la lune de Rhégeb, 1712.}
\section[{Lettre XXVII. Usbek à Nessir. À Ispahan.}]{Lettre XXVII. Usbek à Nessir. À Ispahan.}\renewcommand{\leftmark}{Lettre XXVII. Usbek à Nessir. À Ispahan.}

\noindent Nous sommes à présent à Paris, cette superbe rivale de la ville du soleil\footnote{Ispahan (c’est \emph{ville des cheveux} qu’il eût fallu dire).}.\par
Lorsque je partis de Smyrne, je chargeai mon ami Ibben de te faire tenir une boîte où il y avait quelques présents pour toi : tu recevras cette lettre par la même voie. Quoique éloigné de lui de cinq ou six cents lieues, je lui donne de mes nouvelles, et je reçois des siennes aussi facilement que s’il était à Ispahan, et moi à Com. J’envoie mes lettres à Marseille, d’où il part continuellement des vaisseaux pour Smyrne ; de là il envoie celles qui sont pour la Perse par les caravanes d’Arméniens qui partent tous les jours pour Ispahan.\par
Rica jouit d’une santé parfaite : la force de sa constitution, sa jeunesse et sa gaieté naturelle, le mettent au-dessus de toutes les épreuves.\par
Mais, pour moi, je ne me porte pas bien : mon corps et mon esprit sont abattus ; je me livre à des réflexions qui deviennent tous les jours plus tristes ; ma santé, qui s’affaiblit, me tourne vers ma patrie, et me rend ce pays-ci plus étranger.\par
Mais, cher Nessir, je te conjure, fais en sorte que mes femmes ignorent l’état où je suis. Si elles m’aiment, je veux épargner leurs larmes ; et si elles ne m’aiment pas, je ne veux point augmenter leur hardiesse.\par
Si mes eunuques me croyaient en danger, s’ils pouvaient espérer l’impunité d’une lâche complaisance, ils cesseraient bientôt d’être sourds à la voix flatteuse de ce sexe qui se fait entendre aux rochers, et remue les choses inanimées.\par
Adieu, Nessir ; j’ai du plaisir à te donner des marques de ma confiance.\par

\dateline{De Paris, le 5 de la lune de Chahban, 1712.}
\section[{Lettre XXVIII. Rica à ***.}]{Lettre XXVIII. Rica à ***.}\renewcommand{\leftmark}{Lettre XXVIII. Rica à ***.}

\noindent Je vis hier une chose assez singulière, quoique elle se passe tous les jours à Paris.\par
Tout le peuple s’assemble sur la fin de l’après-dînée, et va jouer une espèce de scène que j’ai entendu appeler comédie. Le grand mouvement est sur une estrade, qu’on nomme le théâtre. Aux deux côtés on voit, dans de petits réduits, qu’on nomme loges, des hommes et des femmes qui jouent ensemble des scènes muettes, à peu près comme celles qui sont en usage en notre Perse.\par
 Tantôt c’est une amante affligée qui exprime sa langueur ; tantôt une autre, avec des yeux vifs et un air passionné, dévore des yeux son amant, qui la regarde de même : toutes les passions sont peintes sur les visages, et exprimées avec une éloquence qui n’en est que plus vive pour être muette. Là les acteurs ne paraissent qu’à demi-corps, et ont ordinairement un manchon, par modestie, pour cacher leurs bras. Il y a en bas une troupe de gens debout qui se moquent de ceux qui sont en haut sur le théâtre, et ces derniers rient à leur tour de ceux qui sont en bas.\par
Mais ceux qui prennent le plus de peine sont quelques jeunes gens, qu’on prend pour cet effet dans un âge peu avancé pour soutenir à la fatigue. Ils sont obligés d’être partout ; ils passent par des endroits qu’eux seuls connaissent, montent avec une adresse surprenante d’étage en étage ; ils sont en haut, en bas, dans toutes les loges ; ils plongent pour ainsi dire ; on les perd, ils reparaissent ; souvent ils quittent le lieu de la scène, et vont jouer dans un autre. On en voit même qui, par un prodige qu’on n’aurait osé espérer de leurs béquilles, marchent et vont comme les autres. Enfin on se rend à des salles où l’on joue une comédie particulière : on commence par des révérences, on continue par des embrassades. On dit que la connaissance la plus légère met un homme en droit d’en étouffer un autre : il semble que le lieu inspire de la tendresse. En effet, on dit que les princesses qui y règnent ne sont point cruelles ; et si on excepte deux ou trois heures par jour, où elles sont assez sauvages, on peut dire que le reste du temps elles sont traitables, et que c’est une ivresse qui les quitte aisément.\par
Tout ce que je te dis ici se passe à peu près de même dans un autre endroit qu’on nomme l’Opéra : toute la différence est que l’on parle à l’un, et chante à l’autre. Un de mes amis me mena l’autre jour dans la loge où se déshabillait une des principales actrices. Nous fîmes si bien connaissance, que le lendemain je reçus d’elle cette lettre :\par
"Monsieur,\par
Je suis la plus malheureuse fille du monde ; j’ai toujours été la plus vertueuse actrice de l’Opéra. Il y a sept ou huit mois, que j’étais dans la loge où vous me vîtes hier ; comme je m’habillais en prêtresse de Diane, un jeune abbé vint m’y trouver ; et, sans respect pour mon habit blanc, mon voile et mon bandeau, il me ravit mon innocence. J’ai beau exagérer le sacrifice que je lui ai fait, il se met à rire, et me soutient qu’il m’a trouvée très profane. Cependant je suis si grosse, que je n’ose plus me présenter sur le théâtre : car je suis, sur le chapitre de l’honneur, d’une délicatesse inconcevable ; et je soutiens toujours qu’à une fille bien née il est plus facile de faire perdre la vertu que la modestie. Avec cette délicatesse, vous jugez bien que ce jeune abbé n’eût jamais réussi, s’il ne m’avait promis de se marier avec moi : un motif si légitime me fit passer sur les petites formalités ordinaires, et commencer par où j’aurais dû finir. Mais, puisque son infidélité m’a déshonorée, je ne veux plus vivre à l’Opéra, où, entre vous et moi, l’on ne me donne guère de quoi vivre : car, à présent que j’avance en âge, et que je perds du côté des charmes, ma pension, qui est toujours la même, semble diminuer tous les jours. J’ai appris par un homme de votre suite que l’on faisait un cas infini, dans votre pays, d’une bonne danseuse, et que, si j’étais à Ispahan, ma fortune serait aussitôt faite. Si vous vouliez m’accorder votre protection, et m’emmener avec vous dans ce pays-là, vous auriez l’avantage de faire du bien à une fille qui, par sa vertu et sa conduite, ne se rendrait pas indigne de vos bontés. Je suis…"\par

\dateline{De Paris, le 2 de la lune de Chaval, 1712}
\section[{Lettre XXIX. Rica à Ibben. À Smyrne.}]{Lettre XXIX. Rica à Ibben. À Smyrne.}\renewcommand{\leftmark}{Lettre XXIX. Rica à Ibben. À Smyrne.}

\noindent Le pape est le chef des chrétiens. C’est une vielle idole qu’on encense par habitude. Il était autrefois redoutable aux princes mêmes, car il les déposait aussi facilement que nos magnifiques sultans déposent les rois d’Irimette et de Géorgie. Mais on ne le craint plus. Il se dit successeur d’un des premiers chrétiens, qu’on appelle saint Pierre : et c’est certainement une riche succession, car il a des trésors immenses et un grand pays sous sa domination.\par
Les évêques sont des gens de loi qui lui sont subordonnés, et ont sous son autorité deux fonctions bien différentes. Quand ils sont assemblés, ils font, comme lui, des articles de foi ; quand ils sont en particulier, ils n’ont guère d’autre fonction que de dispenser d’accomplir la loi. Car tu sauras que la religion chrétienne est chargée d’une infinité de pratiques très difficiles ; et, comme on a jugé qu’il est moins aisé de remplir ses devoirs que d’avoir des évêques qui en dispensent, on a pris ce dernier parti pour l’utilité publique : ainsi, si on ne veut pas faire de rahmazan, si on ne veut pas s’assujettir aux formalités des mariages, si on veut rompre ses vœux, si on veut se marier contre les défenses de la loi, quelquefois même si on veut revenir contre son serment, on va à l’évêque ou au pape, qui donne aussitôt la dispense.\par
Les évêques ne font pas des articles de foi de leur propre mouvement. Il y a un nombre infini de docteurs, la plupart dervis, qui soulèvent entre eux mille questions nouvelles sur la religion : on les laisse disputer longtemps, et la guerre dure jusqu’à ce qu’une décision vienne la terminer.\par
Aussi puis-je t’assurer qu’il n’y a jamais eu de royaume où il y ait eu tant de guerres civiles que dans celui du Christ.\par
Ceux qui mettent au jour quelque proposition nouvelle sont d’abord appelés hérétiques. Chaque hérésie a son nom, qui est, pour ceux qui y sont engagés, comme le mot de ralliement. Mais n’est hérétique qui ne veut : il n’y a qu’à partager le différend par la moitié, et donner une distinction à ceux qui accusent d’hérésie ; et, quelle que soit la distinction, intelligible ou non, elle rend un homme blanc comme de la neige, et il peut se faire appeler orthodoxe.\par
Ce que je te dis est bon pour la France et l’Allemagne : car j’ai ouï dire qu’en Espagne et en Portugal il y a de certains dervis qui n’entendent point raillerie, et qui font brûler un homme comme de la paille. Quand on tombe entre les mains de ces gens-là, heureux celui qui a toujours prié Dieu avec de petits grains de bois à la main, qui a porté sur lui deux morceaux de drap attachés à deux rubans, et qui a été quelquefois dans une province qu’on appelle la Galice ! Sans cela un pauvre diable est bien embarrassé. Quand il jurerait comme un païen qu’il est orthodoxe, on pourrait bien ne pas demeurer d’accord des qualités, et le brûler comme hérétique : il aurait beau donner sa distinction ; point de distinction ; il serait en cendres avant que l’on eût seulement pensé à l’écouter.\par
Les autres juges présument qu’un accusé est innocent : ceux-ci le présument toujours coupable. Dans le doute, ils tiennent pour règle de se déterminer du côté de la rigueur : apparemment parce qu’ils croient les hommes mauvais ; mais, d’un autre côté, ils en ont si bonne opinion, qu’ils ne les jugent jamais capables de mentir ; car ils reçoivent le témoignage des ennemis capitaux, des femmes de mauvaise vie, de ceux qui exercent une profession infâme. Ils font dans leur sentence un petit compliment à ceux qui sont revêtus d’une chemise de soufre, et leur disent qu’ils sont bien fâchés de les voir si mal habillés, qu’ils sont doux et qu’ils abhorrent le sang, et sont au désespoir de les avoir condamnés ; mais, pour se consoler, ils confisquent tous les biens de ces malheureux à leur profit.\par
Heureuse la terre qui est habitée par les enfants des prophètes ! Ces tristes spectacles y sont inconnus\footnote{Les Persans sont les plus tolérants de tous les mahométans.}. La sainte religion que les anges y ont apportée se défend par sa vérité même ; elle n’a point besoin de ces moyens violents pour se maintenir.\par

\dateline{À Paris, le 4 de la lune de Chalval, 1712}
\section[{Lettre XXX. Rica au même. À Smyrne.}]{Lettre XXX. Rica au même. À Smyrne.}\renewcommand{\leftmark}{Lettre XXX. Rica au même. À Smyrne.}

\noindent Les habitants de Paris sont d’une curiosité qui va jusqu’à l’extravagance. Lorsque j’arrivai, je fus regardé comme si j’avais été envoyé du ciel : vieillards, hommes, femmes, enfants, tous voulaient me voir. Si je sortais, tout le monde se mettait aux fenêtres ; si j’étais aux Tuileries, je voyais aussitôt un cercle se former autour de moi ; les femmes mêmes faisaient un arc-en-ciel nuancé de mille couleurs, qui m’entourait. Si j’étais aux spectacles, je voyais aussitôt cent lorgnettes dressées contre ma figure : enfin jamais homme n’a tant été vu que moi. Je souriais quelquefois d’entendre des gens qui n’étaient presque jamais sortis de leur chambre, qui disaient entre eux : Il faut avouer qu’il a l’air bien persan. Chose admirable ! Je trouvais de mes portraits partout ; je me voyais multiplié dans toutes les boutiques, sur toutes les cheminées, tant on craignait de ne m’avoir pas assez vu.\par
Tant d’honneurs ne laissent pas d’être à la charge : je ne me croyais pas un homme si curieux et si rare ; et quoique j’aie très bonne opinion de moi, je ne me serais jamais imaginé que je dusse troubler le repos d’une grande ville où je n’étais point connu. Cela me fit résoudre à quitter l’habit persan, et à en endosser un à l’européenne, pour voir s’il resterait encore dans ma physionomie quelque chose d’admirable. Cet essai me fit connaître ce que je valais réellement. Libre de tous les ornements étrangers, je me vis apprécié au plus juste. J’eus sujet de me plaindre de mon tailleur, qui m’avait fait perdre en un instant l’attention et l’estime publique ; car j’entrai tout à coup dans un néant affreux. Je demeurais quelquefois une heure dans une compagnie sans qu’on m’eût regardé, et qu’on m’eût mis en occasion d’ouvrir la bouche ; mais, si quelqu’un par hasard apprenait à la compagnie que j’étais Persan, j’entendais aussitôt autour de moi un bourdonnement : Ah ! ah ! monsieur est Persan ? C’est une chose bien extraordinaire ! Comment peut-on être Persan ?\par

\dateline{À Paris, le 6 de la lune de Chalval, 1712}
\section[{Lettre XXXI. Rhédi à Usbek. À Paris.}]{Lettre XXXI. Rhédi à Usbek. À Paris.}\renewcommand{\leftmark}{Lettre XXXI. Rhédi à Usbek. À Paris.}

\noindent Je suis à présent à Venise, mon cher Usbek. On peut avoir vu toutes les villes du monde, et être surpris en arrivant à Venise : on sera toujours étonné de voir une ville, des tours et des mosquées sortir de dessous l’eau, et de trouver un peuple innombrable dans un endroit où il ne devrait y avoir que des poissons.\par
Mais celle ville profane manque du trésor le plus précieux qui soit au monde, c’est-à-dire d’eau vive : il est impossible d’y accomplir une seule ablution légale. Elle est en abomination à notre saint prophète, et il ne la regarde jamais du haut du ciel qu’avec colère.\par
Sans cela, mon cher Usbek, je serais charmé de vivre dans une ville où mon esprit se forme tous les jours. Je m’instruis des secrets du commerce, des intérêts des princes, de la forme de leur gouvernement ; je ne néglige pas même les superstitions européennes ; je m’applique à la médecine, à la physique, à l’astronomie ; j’étudie les arts : enfin je sors des nuages qui couvraient mes yeux dans le pays de ma naissance.\par

\dateline{À Venise, le 16 de la lune de Chalval, 1712}
\section[{Lettre XXXII. Rica à ***.}]{Lettre XXXII. Rica à ***.}\renewcommand{\leftmark}{Lettre XXXII. Rica à ***.}

\noindent J’allai l’autre jour voir une maison où l’on entretient environ trois cents personnes assez pauvrement. J’eus bientôt fait, car l’église ni les bâtiments ne méritent pas d’être regardés. Ceux qui sont dans cette maison étaient assez gais ; plusieurs d’entre eux jouaient aux cartes, ou à d’autres jeux que je ne connais point. Comme je sortais, un de ces hommes sortait aussi ; et, m’ayant entendu demander le chemin du Marais, qui est le quartier le plus éloigné de Paris : J’y vais, me dit-il, et je vous y conduirai ; suivez-moi. Il me mena à merveille, me tira de tous les embarras, et me sauva adroitement des carrosses et des voitures. Nous étions près d’arriver, quand la curiosité me prit. Mon bon ami, lui dis-je, ne pourrais-je point savoir qui vous êtes ? Je suis aveugle, monsieur, me répondit-il. Comment ! lui dis-je, vous êtes aveugle ! Et que ne priiez-vous cet honnête homme qui jouait aux cartes avec vous de nous conduire ? Il est aveugle aussi, me répondit-il : il y a quatre cents ans que nous sommes trois cents aveugles dans cette maison où vous m’avez trouvé. Mais il faut que je vous quitte : voilà la rue que vous demandiez ; je vais me mettre dans la foule ; j’entre dans cette église, où, je vous jure, j’embarrasserai plus les gens qu’ils ne m’embarrasseront.\par

\dateline{À Paris, le 17 de la lune de Chalval, 1712}
\section[{Lettre XXXIII. Usbek à Rhédi. À Venise.}]{Lettre XXXIII. Usbek à Rhédi. À Venise.}\renewcommand{\leftmark}{Lettre XXXIII. Usbek à Rhédi. À Venise.}

\noindent Le vin est si cher à Paris, par les impôts que l’on y met, qu’il semble qu’on ait entrepris d’y faire exécuter les préceptes du divin Alcoran, qui défend d’en boire.\par
Lorsque je pense aux funestes effets de cette liqueur, je ne puis m’empêcher de la regarder comme le présent le plus redoutable que la nature ait fait aux hommes. Si quelque chose à flétri la vie et la réputation de nos monarques, ç’a été leur intempérance ; c’est la source la plus empoisonnée de leurs injustices et de leur cruautés.\par
Je le dirai, à la honte des hommes : la loi interdit à nos princes l’usage du vin, et ils en boivent avec un excès qui les dégrade de l’humanité même ; cet usage, au contraire, est permis aux princes chrétiens, et on ne remarque pas qu’il leur fasse aucune faute. L’esprit humain est la contradiction même : dans une débauche licencieuse, on se révolte avec fureur contre les préceptes ; et la loi faite pour nous rendre plus justes ne sert souvent qu’à nous rendre plus coupables.\par
Mais quand je désapprouve l’usage de cette liqueur qui fait perdre la raison, je ne condamne pas de même ces boissons qui l’égayent. C’est la sagesse des Orientaux de chercher des remèdes contre la tristesse avec autant de soin que contre les maladies les plus dangereuses. Lorsqu’il arrive quelque malheur à un Européen, il n’a d’autre ressource que la lecture d’un philosophe qu’on appelle Sénèque ; mais les Asiatiques, plus sensés qu’eux et meilleurs physiciens en cela, prennent des breuvages capables de rendre l’homme gai, et de charmer le souvenir de ses peines.\par
Il n’y a rien de si affligeant que les consolations tirées de la nécessité du mal, de l’inutilité des remèdes, de la fatalité du destin, de l’ordre de la Providence, et du malheur de la condition humaine. C’est se moquer de vouloir adoucir un mal par la considération que l’on est né misérable ; il vaut bien mieux enlever l’esprit hors de ses réflexions, et traiter l’homme comme sensible, au lieu de le traiter comme raisonnable.\par
L’âme, unie avec le corps, en est sans cesse tyrannisée. Si le mouvement du sang est trop lent, si les esprits ne sont pas assez épurés, s’ils ne sont pas en quantité suffisante, nous tombons dans l’accablement et dans la tristesse ; mais, si nous prenons des breuvages qui puissent changer cette disposition de notre corps, notre âme redevient capable de recevoir des impressions qui l’égayent, et elle sent un plaisir secret de voir sa machine reprendre, pour ainsi dire, son mouvement et sa vie.\par

\dateline{À Paris, le 25 de la lune de Zilcadé, 1713}
\section[{Lettre XXXIV. Usbek à Ibben. À Smyrne.}]{Lettre XXXIV. Usbek à Ibben. À Smyrne.}\renewcommand{\leftmark}{Lettre XXXIV. Usbek à Ibben. À Smyrne.}

\noindent Les femmes de Perse sont plus belles que celles de France ; mais celles de France sont plus jolies. Il est difficile de ne point aimer les premières, et de ne se point plaire avec les secondes : les unes sont plus tendres et plus modestes, les autres sont plus gaies et plus enjouées.\par
Ce qui rend le sang si beau en Perse, c’est la vie réglée que les femmes y mènent : elles ne jouent ni ne veillent, elles ne boivent point de vin, et ne s’exposent presque jamais à l’air. Il faut avouer que le sérail est plutôt fait pour la santé que pour les plaisirs : c’est une vie unie, qui ne pique point ; tout s’y ressent de la subordination et du devoir ; les plaisirs mêmes y sont graves, et les joies sévères ; et on ne les goûte presque jamais que comme des marques d’autorité et de dépendance.\par
Les hommes mêmes n’ont pas en Perse la même gaieté que les Français : on ne leur voit point cette liberté d’esprit et cet air content que je trouve ici dans tous les états et dans toutes les conditions.\par
C’est bien pis en Turquie, où l’on pourrait trouver des familles où, de père en fils, personne n’a ri depuis la fondation de la monarchie.\par
Cette gravité des Asiatiques vient du peu de commerce qu’il y a entre eux : ils ne se voient que lorsqu’ils y sont forcés par la cérémonie ; l’amitié, ce doux engagement du cœur, qui fait ici la douceur de la vie, leur est presque inconnue : ils se retirent dans leurs maisons, où ils trouvent toujours une compagnie qui les attend ; de manière que chaque famille est, pour ainsi dire, isolée des autres.\par
Un jour que je m’entretenais là-dessus avec un homme de ce pays-ci, il me dit : Ce qui me choque le plus de vos mœurs, c’est que vous êtes obligés de vivre avec des esclaves dont le cœur et l’esprit sec sentent toujours de la bassesse de leur condition. Ces gens lâches affaiblissent en vous les sentiments de la vertu, que l’on tient de la nature, et ils les ruinent depuis l’enfance qu’ils vous obsèdent.\par
Car, enfin, défaites-vous des préjugés : que peut-on attendre de l’éducation qu’on reçoit d’un misérable qui fait consister son honneur à garder les femmes d’un autre, et s’enorgueillit du plus vil emploi qui soit parmi les humains, qui est méprisable par sa fidélité même, qui est la seule de ses vertus, parce qu’il y est porté par envie, par jalousie et par désespoir ; qui, brûlant de se venger des deux sexes dont il est le rebut, consent à être tyrannisé par le plus fort, pourvu qu’il puisse désoler le plus faible ; qui, tirant de son imperfection, de sa laideur et de sa difformité, tout l’éclat de sa condition, n’est estimé que parce qu’il est indigne de l’être ; qui enfin, rivé pour jamais à la porte où il est attaché, plus dur que les gonds et les verrous qui la tiennent, se vante de cinquante ans de vie dans ce poste indigne, où, chargé de la jalousie de son maître, il a exercé toute sa bassesse ?\par

\dateline{À Paris, le 14 de la lune Zilhagé, 1713}
\section[{Lettre XXXV. Usbek à Gemchid, son cousin, dervis du brillant monastère de Tauris.}]{Lettre XXXV. Usbek à Gemchid, son cousin, dervis du brillant monastère de Tauris.}\renewcommand{\leftmark}{Lettre XXXV. Usbek à Gemchid, son cousin, dervis du brillant monastère de Tauris.}

\noindent Que penses-tu des chrétiens, sublime dervis ? Crois-tu qu’au jour du jugement ils seront comme les infidèles Turcs, qui serviront d’ânes aux Juifs, et seront menés par eux au grand trot en enfer ? Je sais bien qu’ils n’iront point dans le séjour des prophètes, et que le grand Ali n’est point venu pour eux. Mais, parce qu’ils n’ont pas été assez heureux pour trouver des mosquées dans leur pays, crois-tu qu’ils soient condamnés à des châtiments éternels, et que Dieu les punisse pour n’avoir pas pratiqué une religion qu’il ne leur a pas fait connaître ? Je puis te le dire : j’ai souvent examiné ces chrétiens ; je les ai interrogés pour voir s’ils avaient quelque idée du grand Ali, qui était le plus beau de tous les hommes ; j’ai trouvé qu’ils n’en avaient jamais ouï parler.\par
Ils ne ressemblent point à ces infidèles que nos saints prophètes faisaient passer au fil de l’épée, parce qu’ils refusaient de croire aux miracles du ciel ; ils sont plutôt comme ces malheureux qui vivaient dans les ténèbres de l’idolâtrie avant que la divine lumière vînt éclairer le visage de notre grand prophète.\par
D’ailleurs, si on examine de près leur religion, on y trouvera comme une semence de nos dogmes. J’ai souvent admiré les secrets de la Providence, qui semble les avoir voulu préparer par là à la conversion générale. J’ai ouï parler d’un livre de leurs docteurs, intitulé la {\itshape Polygamie triomphante}, dans lequel il est prouvé que la polygamie est ordonnée aux chrétiens. Leur baptême est l’image de nos ablutions légales ; et les chrétiens n’errent que dans l’efficacité qu’ils donnent à cette première ablution, qu’ils croient devoir suffire pour toutes les autres. Leurs prêtres et les moines prient comme nous sept fois le jour. Ils espèrent de jouir d’un paradis où ils goûteront mille délices par le moyen de la résurrection des corps. Ils ont, comme nous, des jeûnes marqués, des mortifications avec lesquelles ils espèrent fléchir la miséricorde divine. Ils rendent un culte aux bons anges, et se méfient des mauvais. Ils ont une sainte crédulité pour les miracles que Dieu opère par le ministère de ses serviteurs. Ils reconnaissent, comme nous, l’insuffisance de leurs mérites, et les besoins qu’ils ont d’un intercesseur auprès de Dieu. Je vois partout le mahométisme, quoique je n’y trouve point Mahomet. On a beau faire, la vérité s’échappe, et perce toujours les ténèbres qui l’environnent. Il viendra un jour où l’Éternel ne verra sur la terre que de vrais croyants. Le temps, qui consume tout, détruira les erreurs mêmes. Tous les hommes seront étonnés de se voir sous le même étendard : tout, jusqu’à la loi, sera consommé ; les divins exemplaires seront enlevés de la terre, et portés dans les célestes archives.\par

\dateline{À Paris, le 20 de la lune de Zilhagé, 1713.}
\section[{Lettre XXXVI. Usbek à Rhédi. À Venise.}]{Lettre XXXVI. Usbek à Rhédi. À Venise.}\renewcommand{\leftmark}{Lettre XXXVI. Usbek à Rhédi. À Venise.}

\noindent Le café est très en usage à Paris : il y a un grand nombre de maisons publiques où on le distribue. Dans quelques-unes de ces maisons, on dit des nouvelles ; dans d’autres, on joue aux échecs. Il y en a une où l’on apprête le café de telle manière qu’il donne de l’esprit à ceux qui en prennent : au moins, de tous ceux qui en sortent, il n’y a personne qui ne croie qu’il en a quatre fois plus que lorsqu’il y est entré.\par
Mais ce qui me choque de ces beaux esprits, c’est qu’ils ne se rendent pas utiles à leur patrie, et qu’ils amusent leurs talents à des choses puériles. Par exemple, lorsque j’arrivai à Paris, je les trouvai échauffés sur une dispute la plus mince qu’il se puisse imaginer : il s’agissait de la réputation d’un vieux poète grec dont, depuis deux mille ans, on ignore la patrie, aussi bien que le temps de sa mort. Les deux partis avouaient que c’était un poète excellent : il n’était question que du plus ou du moins de mérite qu’il fallait lui attribuer. Chacun en voulait donner le taux ; mais, parmi ces distributeurs de réputation, les uns faisaient meilleur poids que les autres : voilà la querelle. Elle était bien vive, car on se disait cordialement de part et d’autre des injures si grossières, on faisait des plaisanteries si amères, que je n’admirais pas moins la manière de disputer que le sujet de la dispute. Si quelqu’un, disais-je en moi-même, était assez étourdi pour aller devant l’un de ces défenseurs du poète grec attaquer la réputation de quelque honnête citoyen, il ne serait pas mal relevé ; et je crois que ce zèle si délicat sur la réputation des morts s’embraserait bien pour défendre celle des vivants ! Mais, quoi qu’il en soit, ajoutais-je, Dieu me garde de m’attirer jamais l’inimitié des censeurs de ce poète, que le séjour de deux mille ans dans le tombeau n’a pu garantir d’une haine si implacable ! Ils frappent à présent des coups en l’air : mais que serait-ce si leur fureur était animée par la présence d’un ennemi ?\par
Ceux dont je te viens de parler disputent en langue vulgaire ; et il faut les distinguer d’une autre sorte de disputeurs qui se servent d’une langue barbare qui semble ajouter quelque chose à la fureur et à l’opiniâtreté des combattants. Il y a des quartiers où l’on voit comme une mêlée noire et épaisse de ces sortes de gens ; ils se nourrissent de distinctions, ils vivent de raisonnements obscurs et de fausses conséquences. Ce métier, où l’on devrait mourir de faim, ne laisse pas de rendre. On a vu une nation entière chassée de son pays, traverser les mers pour s’établir en France, n’emportant avec elle, pour parer aux nécessités de la vie, qu’un redoutable talent pour la dispute. Adieu.\par

\dateline{À Paris, le dernier de la lune de Zilhagé, 1713.}
\section[{Lettre XXXVII. Usbek à Ibben. À Smyrne.}]{Lettre XXXVII. Usbek à Ibben. À Smyrne.}\renewcommand{\leftmark}{Lettre XXXVII. Usbek à Ibben. À Smyrne.}

\noindent Le roi de France est vieux. Nous n’avons point d’exemple dans nos histoires d’un monarque qui ait si longtemps régné. On dit qu’il possède à un très haut degré le talent de se faire obéir : il gouverne avec le même génie sa famille, sa cour, son état. On lui a souvent entendu dire que, de tous les gouvernements du monde, celui des Turcs, ou celui de notre auguste sultan, lui plairait le mieux : tant il fait cas de la politique orientale.\par
J’ai étudié son caractère, et j’y ai trouvé des contradictions qu’il m’est impossible de résoudre : par exemple, il a un ministre qui n’a que dix-huit ans, et une maîtresse qui en a quatre-vingts ; il aime sa religion, et il ne peut souffrir ceux qui disent qu’il la faut observer à la rigueur ; quoiqu’il fuie le tumulte des villes, et qu’il se communique peu, il n’est occupé depuis le matin jusqu’au soir qu’à faire parler de lui ; il aime les trophées et les victoires, mais il craint autant de voir un bon général à la tête de ses troupes qu’il aurait sujet de le craindre à la tête d’une armée ennemie. Il n’est, je crois, jamais arrivé qu’à lui d’être en même temps comblé de plus de richesses qu’un prince n’en saurait espérer, et accablé d’une pauvreté qu’un particulier ne pourrait soutenir.\par
Il aime à gratifier ceux qui le servent ; mais il paie aussi libéralement les assiduités, ou plutôt l’oisiveté de ses courtisans, que les campagnes laborieuses de ses capitaines : souvent il préfère un homme qui le déshabille, ou qui lui donne la serviette lorsqu’il se met à table, à un autre qui lui prend des villes ou lui gagne des batailles : il ne croit pas que la grandeur souveraine doive être gênée dans la distribution des grâces ; et, sans examiner si celui qu’il comble de biens est homme de mérite, il croit que son choix va le rendre tel ; aussi lui a-t-on vu donner une petite pension à un homme qui avait fui des lieues, et un beau gouvernement à un autre qui en avait fui quatre.\par
Il est magnifique, surtout dans ses bâtiments : il y a plus de statues dans les jardins de son palais que de citoyens dans une grande ville. Sa garde est aussi forte que celle du prince devant qui tous les trônes se renversent ; ses armées sont aussi nombreuses, ses ressources aussi grandes, et ses finances aussi inépuisables.\par

\dateline{À Paris, le 7 de la lune de Maharran, 1713.}
\section[{Lettre XXXVIII. Rica à Ibben. À Smyrne.}]{Lettre XXXVIII. Rica à Ibben. À Smyrne.}\renewcommand{\leftmark}{Lettre XXXVIII. Rica à Ibben. À Smyrne.}

\noindent C’est une grande question parmi les hommes de savoir s’il est plus avantageux d’ôter aux femmes la liberté que de la leur laisser. Il me semble qu’il y a bien des raisons pour et contre. Si les Européens disent qu’il n’y a pas de générosité à rendre malheureuses les personnes que l’on aime, nos Asiatiques répondent qu’il y a de la bassesse aux hommes de renoncer à l’empire que la nature leur a donné sur les femmes. Si on leur dit que le grand nombre des femmes enfermées est embarrassant, ils répondent que dix femmes qui obéissent embarrassent moins qu’une qui n’obéit pas. Que s’ils objectent à leur tour que les Européens ne sauraient être heureux avec des femmes qui ne leur sont pas fidèles, on leur répond que cette fidélité qu’ils vantent tant n’empêche pas le dégoût qui suit toujours les passions satisfaites ; que nos femmes sont trop à nous ; qu’une possession si tranquille ne nous laisse rien à désirer ni à craindre ; qu’un peu de coquetterie est un sel qui pique et prévient la corruption. Peut-être qu’un homme plus sage que moi serait embarrassé de décider : car, si les Asiatiques font fort bien de chercher des moyens propres à calmer leurs inquiétudes, les Européens font fort bien aussi de n’en point avoir.\par
Après tout, disent-ils, quand nous serions malheureux en qualité de maris, nous trouverions toujours moyen de nous dédommager en qualité d’amants. Pour qu’un homme pût se plaindre avec raison de l’infidélité de sa femme, il faudrait qu’il n’y eût que trois personnes dans le monde ; ils seront toujours à but quand il y en aura quatre.\par
C’est une autre question de savoir si la loi naturelle soumet les femmes aux hommes. Non, me disait l’autre jour un philosophe très galant : la nature n’a jamais dicté une telle loi ; l’empire que nous avons sur elles est une véritable tyrannie ; elles ne nous l’ont laissé prendre que parce qu’elles ont plus de douceur que nous, et par conséquent, plus d’humanité et de raison : ces avantages, qui devaient sans doute leur donner la supériorité si nous avions été raisonnables, la leur ont fait perdre, parce que nous ne le sommes point.\par
Or, s’il est vrai que nous n’avons sur les femmes qu’un pouvoir tyrannique, il ne l’est pas moins qu’elles ont sur nous un empire naturel, celui de la beauté, à qui rien ne résiste. Le nôtre n’est pas de tous les pays ; mais celui de la beauté est universel. Pourquoi aurions-nous donc un privilège ? Est-ce parce que nous sommes les plus forts ? Mais c’est une véritable injustice. Nous employons toutes sortes de moyens pour leur abattre le courage ; les forces seraient égales, si l’éducation l’était aussi ; éprouvons-les dans les talents que l’éducation n’a point affaiblis, et nous verrons si nous sommes si forts.\par
Il faut l’avouer, quoique cela choque nos mœurs : chez les peuples les plus polis, les femmes ont toujours eu de l’autorité sur leurs maris ; elle fut établie par une loi chez les Égyptiens en l’honneur d’Isis et chez les Babyloniens en l’honneur de Sémiramis. On disait des Romains qu’ils commandaient à toutes les nations, mais qu’ils obéissaient à leurs femmes. Je ne parle point des Sauromates, qui étaient véritablement dans la servitude du sexe ; ils étaient trop barbares pour que leur exemple puisse être cité.\par
Tu verras, mon cher Ibben, que j’ai pris le goût de ce pays-ci, où l’on aime à soutenir des opinions extraordinaires et à réduire tout en paradoxe. Le prophète a décidé la question, et a réglé les droits de l’un et de l’autre sexe. Les femmes, dit-il, doivent honorer leurs maris : leurs maris les doivent honorer ; mais ils ont l’avantage d’un degré sur elles.\par

\dateline{À Paris, le 26 de la lune de Gemmadi 2, 1713.}
\section[{Lettre XXXIX. Hagi Ibbi au juif Ben Josue, Prosélyte Mahométan. À Smyrne.}]{Lettre XXXIX. Hagi\protect\footnotemark  Ibbi au juif Ben Josue, Prosélyte Mahométan. À Smyrne.}\renewcommand{\leftmark}{Lettre XXXIX. Hagi Ibbi au juif Ben Josue, Prosélyte Mahométan. À Smyrne.}

\footnotetext{Hagi est un homme qui a fait le pèlerinage de la Mecque.}
\noindent Il me semble, Ben Josué, qu’il y a toujours des signes éclatants qui préparent à la naissance des hommes extraordinaires ; comme si la nature souffrait une espèce de crise, et que la puissance céleste ne produisît qu’avec effort.\par
Il n’y a rien de si merveilleux que la naissance de Mahomet. Dieu, qui par les décrets de sa providence avait résolu dès le commencement d’envoyer aux hommes ce grand prophète pour enchaîner Satan, créa une lumière deux mille ans avant Adam, qui, passant d’élu en élu, d’ancêtre en ancêtre de Mahomet, parvint enfin jusques à lui comme un témoignage authentique qu’il était descendu des patriarches.\par
Ce fut aussi à cause de ce même prophète que Dieu ne voulut pas qu’aucun enfant fût conçu, que la nature de la femme ne cessât d’être immonde, et que le membre viril ne fût livré à la circoncision.\par
Il vint au monde circoncis, et la joie parut sur son visage dès sa naissance ; la terre trembla trois fois, comme si elle eût enfanté elle-même ; toutes les idoles se prosternèrent ; les trônes des rois furent renversés ; Lucifer fut jeté au fond de la mer ; et ce ne fut qu’après avoir nagé pendant quarante jours qu’il sortit de l’abîme, et s’enfuit sur le mont Cabès, d’où, avec une voix terrible, il appela les anges.\par
Cette nuit, Dieu posa un terme entre l’homme et la femme, qu’aucun d’eux ne pût passer. L’art des magiciens et nécromants se trouva sans vertu. On entendit une voix du ciel qui disait ces paroles : J’ai envoyé au monde mon ami fidèle.\par
Selon le témoignage d’Isben Aben, historien arabe, les générations des oiseaux, des nuées, des vents, et tous les escadrons des anges, se réunirent pour élever cet enfant, et se disputèrent cet avantage. Les oiseaux disaient dans leurs gazouillements qu’il était plus commode qu’ils l’élevassent, parce qu’ils pouvaient plus facilement rassembler plusieurs fruits de divers lieux. Les vents murmuraient, et disaient : C’est plutôt à nous, parce que nous pouvons lui apporter de tous les endroits les odeurs les plus agréables. Non, non, disaient les nuées, non ; c’est à nos soins qu’il sera confié, parce que nous lui ferons part à tous les instants de la fraîcheur des eaux. Là-dessus les anges indignés s’écriaient : Que nous restera-t-il donc à faire ? Mais une voix du ciel fut entendue, qui termina toutes les disputes : Il ne sera point ôté d’entre les mains des mortels, parce que heureuses les mamelles qui l’allaiteront, et les mains qui le toucheront, et la maison qu’il habitera, et le lit où il reposera.\par
Après tant de témoignages si éclatants, mon cher Josué, il faut avoir un cœur de pierre pour ne pas croire sa sainte loi. Que pouvait faire davantage le ciel pour autoriser sa mission divine, à moins que de renverser la nature, et de faire périr les hommes mêmes qu’il voulait convaincre ?\par

\dateline{À Paris, le 20 de la lune de Rhégeb, 1713.}
\section[{Lettre XL. Usbek à Ibben. À Smyrne.}]{Lettre XL. Usbek à Ibben. À Smyrne.}\renewcommand{\leftmark}{Lettre XL. Usbek à Ibben. À Smyrne.}

\noindent Dès qu’un grand est mort, on s’assemble dans une mosquée, et l’on fait son oraison funèbre, qui est un discours à sa louange, avec lequel on serait bien embarrassé de décider au juste du mérite du défunt.\par
Je voudrais bannir les pompes funèbres : il faut pleurer les hommes à leur naissance, et non pas à leur mort. À quoi servent les cérémonies et tout l’attirail lugubre qu’on fait paraître à un mourant dans ses derniers moments, les larmes mêmes de sa famille, et la douleur de ses amis, qu’à lui exagérer la perte qu’il va faire ?\par
Nous sommes si aveugles, que nous ne savons quand nous devons nous affliger ou nous réjouir : nous n’avons presque jamais que de fausses tristesses ou de fausses joies.\par
Quand je vois le Mogol, qui toutes les années va sottement se mettre dans une balance et se faire peser comme un bœuf, quand je vois les peuples se réjouir de ce que ce prince est devenu plus matériel, c’est-à-dire moins capable de les gouverner, j’ai pitié, Ibben, de l’extravagance humaine.\par

\dateline{De Paris, le 20 de la lune de Rhégeb, 1713.}
\section[{Lettre XLI. Le premier eunuque noir à Usbek.}]{Lettre XLI. Le premier eunuque noir à Usbek.}\renewcommand{\leftmark}{Lettre XLI. Le premier eunuque noir à Usbek.}

\noindent Ismaël, un de tes eunuques noirs, vient de mourir, magnifique seigneur ; et je ne puis m’empêcher de le remplacer. Comme les eunuques sont extrêmement rares à présent, j’avais pensé de me servir d’un esclave noir que tu as à la campagne ; mais je n’ai pu jusqu’ici le porter à souffrir qu’on le consacrât à cet emploi. Comme je vois qu’au bout du compte c’est son avantage, je voulus l’autre jour user à son égard d’un peu de rigueur ; et, de concert avec l’intendant de tes jardins, j’ordonnai que, malgré lui, on le mît en état de te rendre les services qui flattent le plus ton cœur, et de vivre comme moi dans ces redoutables lieux qu’il n’ose pas même regarder : mais il se mit à hurler comme si on avait voulu l’écorcher, et fit tant qu’il échappa de nos mains, et évita le fatal couteau. Je viens d’apprendre qu’il veut t’écrire pour te demander grâce, soutenant que je n’ai conçu ce dessein que par un désir insatiable de vengeance sur certaines railleries piquantes qu’il dit avoir faites de moi. Cependant je te jure par les cent mille prophètes que je n’ai agi que pour le bien de ton service, la seule chose qui me soit chère, et hors laquelle je ne regarde rien. Je me prosterne à tes pieds.\par

\dateline{Du sérail de Fatmé, le 7 de la lune de Maharram, 1713.}
\section[{Lettre XLII. Pharan à Usbek, son souverain seigneur.}]{Lettre XLII. Pharan à Usbek, son souverain seigneur.}\renewcommand{\leftmark}{Lettre XLII. Pharan à Usbek, son souverain seigneur.}

\noindent Si tu étais ici, magnifique seigneur, je paraîtrais à ta vue tout couvert de papier blanc ; et il n’y en aurait pas assez encore pour écrire toutes les insultes que ton premier eunuque noir, le plus méchant de tous les hommes, m’a faites depuis ton départ.\par
Sous prétexte de quelques railleries qu’il prétend que j’ai faites sur le malheur de sa condition, il exerce sur ma tête une vengeance inépuisable ; il a animé contre moi le cruel intendant de tes jardins, qui depuis ton départ m’oblige à des travaux insurmontables, dans lesquels j’ai pensé mille fois laisser la vie sans perdre un moment l’ardeur de te servir. Combien de fois ai-je dis en moi-même : J’ai un maître rempli de douceur, et je suis le plus malheureux esclave qui soit sur la terre !\par
Je te l’avoue, magnifique seigneur, je ne me croyais pas destiné à de plus grandes misères, mais ce traître d’eunuque a voulu mettre le comble à sa méchanceté. Il y a quelques jours que, de son autorité privée, il me destina à la garde de tes femmes sacrées, c’est-à-dire à une exécution qui serait pour moi mille fois plus cruelle que la mort. Ceux qui en naissant ont eu le malheur de recevoir de leurs parents un traitement pareil, se consolent peut-être sur ce qu’ils n’ont jamais connu d’autre état que le leur ; mais qu’on me fasse descendre de l’humanité et qu’on m’en prive, je mourrais de douleur si je ne mourais pas de cette barbarie.\par
J’embrasse tes pieds, sublime seigneur, dans une humilité profonde : fais en sorte que je sente les effets de cette vertu si respectée, et qu’il ne soit pas dit que par ton ordre il y a sur la terre un malheureux de plus.\par

\dateline{Des jardins de Fatmé, le 7 de la lune de Maharram, 1713.}
\section[{Lettre XLIII. Usbek à Pharan. Aux jardins de Fatmé.}]{Lettre XLIII. Usbek à Pharan. Aux jardins de Fatmé.}\renewcommand{\leftmark}{Lettre XLIII. Usbek à Pharan. Aux jardins de Fatmé.}

\noindent Recevez la joie dans votre cœur, et reconnaissez ces sacrés caractères ; faites-les baiser au grand eunuque et à l’intendant de mes jardins. Je leur défends de mettre la main sur vous jusqu’à mon retour ; dites-leur d’acheter l’eunuque qui manque. Acquittez-vous de votre devoir comme si vous m’aviez toujours devant les yeux ; car sachez que plus mes bontés sont grandes, plus vous serez puni si vous en abusez.\par

\dateline{De Paris, le 25 de la lune de Rhégeb, 1713.}
\section[{Lettre XLIV. Usbek à Rhédi. À Venise.}]{Lettre XLIV. Usbek à Rhédi. À Venise.}\renewcommand{\leftmark}{Lettre XLIV. Usbek à Rhédi. À Venise.}

\noindent Il y a en France trois sortes d’état : l’Église, l’épée et la robe. Chacun a un mépris souverain pour les deux autres : tel, par exemple, que l’on devrait mépriser parce qu’il est un sot, ne l’est souvent que parce qu’il est un homme de robe.\par
Il n’y a pas jusqu’aux plus vils artisans qui ne disputent sur l’excellence de l’art qu’ils ont choisi : chacun s’élève au-dessus de celui qui est d’une profession différente, à proportion de l’idée qu’il s’est faite de la supériorité de la sienne.\par
Les hommes ressemblent tous, plus ou moins, à cette femme de la province d’Erivan qui, ayant reçu quelque grâce d’un de nos monarques, lui souhaita mille fois, dans les bénédictions qu’elle lui donna, que le ciel le fît gouverneur d’Erivan.\par
J’ai lu, dans une relation, qu’un vaisseau français ayant relâché à la côte de Guinée, quelques hommes de l’équipage voulurent aller à terre acheter quelques moutons. On les mena au roi, qui rendait la justice à ses sujets sous un arbre. Il était sur son trône, c’est-à-dire sur un morceau de bois, aussi fier que s’il eût été assis sur celui du grand Mogol ; il avait trois ou quatre gardes avec des piques de bois ; un parasol en forme de dais le couvrait de l’ardeur du soleil ; tous ses ornements et ceux de la reine sa femme consistaient en leur peau noire et quelques bagues. Ce prince, plus vain encore que misérable, demanda à ces étrangers si l’on parlait beaucoup de lui en France. Il croyait que son nom devait être porté d’un pôle à l’autre ; et, à la différence de ce conquérant de qui on dit qu’il avait fait taire toute la terre, il croyait, lui, qu’il devait faire parler tout l’univers.\par
Quand le kan de Tartarie a dîné, un héraut crie que tous les princes de la terre peuvent aller dîner, si bon leur semble ; et ce barbare, qui ne mange que du lait, qui n’a pas de maison, qui ne vit que de brigandages, regarde tous les rois du monde comme ses esclaves, et les insulte régulièrement deux fois par jour.\par

\dateline{De Paris, le 28 de la lune de Rhégeb, 1713.}
\section[{Lettre XLV. Rica à Usbek. À ***.}]{Lettre XLV. Rica à Usbek. À ***.}\renewcommand{\leftmark}{Lettre XLV. Rica à Usbek. À ***.}

\noindent Hier matin, comme j’étais au lit, j’entendis frapper rudement à ma porte, qui fut soudain ouverte ou enfoncée par un homme avec qui j’avais lié quelque société, et qui me parut tout hors de lui-même.\par
Son habillement était beaucoup plus modeste, sa perruque de travers n’avait pas même été peignée ; il n’avait pas eu le temps de faire recoudre son pourpoint noir, et il avait renoncé, pour ce jour-là, aux sages précautions avec lesquelles il avait coutume de déguiser le délabrement de son équipage.\par
Levez-vous, me dit-il ; j’ai besoin de vous tout aujourd’hui ; j’ai mille emplettes à faire, et je serai bien aise que ce soit avec vous : il faut premièrement que nous allions à la rue Saint-Honoré parler à un notaire qui est chargé de vendre une terre de cinq cent mille livres ; je veux qu’il m’en donne la préférence. En venant ici, je me suis arrêté un moment au faubourg Saint-Germain, où j’ai loué un hôtel deux mille écus, et j’espère passer le contrat aujourd’hui.\par
Dès que je fus habillé, ou peu s’en fallait, mon homme me fit précipitamment descendre : Commençons par aller acheter un carrosse, et établissons d’abord l’équipage. En effet, nous achetâmes non seulement un carrosse, mais encore pour cent mille francs de marchandises, en moins d’une heure ; tout cela se fit promptement, parce que mon homme ne marchanda rien, et ne compta jamais : aussi ne déplaça-t-il pas. Je rêvais sur tout ceci ; et quand j’examinais cet homme, je trouvais en lui une complication singulière de richesse et de pauvreté : de manière que je ne savais que croire. Mais enfin je rompis le silence, et, le tirant à quartier, je lui dis : Monsieur, qui est-ce qui payera tout cela ? Moi, me dit-il ; venez dans ma chambre ; je vous montrerai mes trésors immenses, et des richesses enviées des plus grands monarques ; mais elles ne le seront pas de vous, qui les partagerez toujours avec moi. Je le suis. Nous grimpons à son cinquième étage, et par une échelle nous nous guidons à un sixième, qui était un cabinet ouvert aux quatre vents, dans lequel il n’y avait que deux ou trois douzaines de bassins de terre remplis de diverses liqueurs. Je me suis levé de grand matin, me dit-il, et j’ai fait d’abord ce que je fais depuis vingt-cinq ans, qui est d’aller visiter mon œuvre : j’ai vu que le grand jour était venu qui devait me rendre plus riche qu’homme qui soit sur la terre. Voyez-vous cette liqueur vermeille ? elle a à présent toutes les qualités que les philosophes demandent pour faire la transmutation des métaux. J’en ai tiré ces grains que vous voyez, qui sont de vrai or par la couleur, quoiqu’un peu imparfaits par leur pesanteur. Ce secret, que Nicolas Flamel trouva, mais que Raimond Lulle et un million d’autres cherchèrent toujours, est venu jusque à moi, et je me trouve aujourd’hui un heureux adepte. Fasse le ciel que je me serve de tant de trésors qu’il m’a communiqués, que pour sa gloire !\par
Je sortis, et je descendis, ou plutôt je me précipitai par cet escalier, transporté de colère, et laissai cet homme si riche dans son hôpital. Adieu, mon cher Usbek. J’irai te voir demain, et, si tu veux, nous reviendrons ensemble à Paris.\par

\dateline{À Paris, le dernier de la lune de Rhégeb, 1713.}
\section[{Lettre XLVI. Usbek à Rhédi. À Venise.}]{Lettre XLVI. Usbek à Rhédi. À Venise.}\renewcommand{\leftmark}{Lettre XLVI. Usbek à Rhédi. À Venise.}

\noindent Je vois ici des gens qui disputent sans fin sur la religion, mais il semble qu’ils combattent en même temps à qui l’observera le moins.\par
Non seulement ils ne sont pas meilleurs chrétiens, mais même meilleurs citoyens ; et c’est ce qui me touche : car, dans quelque religion qu’on vive, l’observation des lois, l’amour pour les hommes, la piété envers les parents, sont toujours les premiers actes de religion.\par
En effet, le premier objet d’un homme religieux ne doit-il pas être de plaire à la divinité qui a établi la religion qu’il professe ? Mais le moyen le plus sûr pour y parvenir est sans doute d’observer les règles de la société et les devoirs de l’humanité. Car, en quelque religion qu’on vive, dès qu’on en suppose une, il faut bien que l’on suppose aussi que Dieu aime les hommes, puisqu’il établit une religion pour les rendre heureux ; que s’il aime les hommes, on est sûr de lui plaire en les aimant aussi, c’est-à-dire en exerçant envers eux tous les devoirs de la charité et de l’humanité, en ne violant point les lois sous lesquelles ils vivent.\par
On est bien plus sûr par là de plaire à Dieu qu’en observant telle ou telle cérémonie ; car les cérémonies n’ont point un degré de bonté par elles-mêmes ; elles ne sont bonnes qu’avec égard, et dans la supposition que Dieu les a commandées ; mais c’est la matière d’une grande discussion : on peut facilement s’y tromper, car il faut choisir les cérémonies d’une religion entre celles de deux mille.\par
Un homme faisait tous les jours à Dieu cette prière : Seigneur, je n’entends rien dans les disputes que l’on fait sans cesse à votre sujet ; je voudrais vous servir selon votre volonté ; mais chaque homme que je consulte veut que je vous serve à la sienne. Lorsque je veux vous faire ma prière, je ne sais en quelle langue je dois vous parler. Je ne sais non plus en quelle posture je dois me mettre : l’un dit que je dois vous prier debout ; l’autre veut que je sois assis ; l’autre exige que mon corps porte sur mes genoux. Ce n’est pas tout : il y en a qui prétendent que je dois me laver tous les matins avec de l’eau froide ; d’autres soutiennent que vous me regarderez avec horreur, si je ne me fais pas couper un petit morceau de chair. Il m’arriva l’autre jour de manger un lapin dans un caravansérail : trois hommes qui étaient auprès de là me firent trembler ; ils me soutinrent tous trois que je vous avais grièvement offensé : l’un (un juif), parce que cet animal était immonde ; l’autre (un Turc), parce qu’il était étouffé ; l’autre enfin (un Arménien), parce qu’il n’était pas un poisson. Un brachmane qui passait par là, et que je pris pour juge, me dit : Ils ont tort, car apparemment vous n’avez pas tué vous-même cet animal. Si fait, lui dis-je. Ah ! vous avez commis une action abominable, et que Dieu ne vous pardonnera jamais, me dit-il d’une voix sévère : que savez-vous si l’âme de votre père n’était pas passée dans cette bête ? Toutes ces choses, Seigneur, me jettent dans un embarras inconcevable : je ne puis remuer la tête que je ne sois menacé de vous offenser ; cependant je voudrais vous plaire, et employer à cela ma vie que je tiens de vous. Je ne sais si je me trompe ; mais je crois que le meilleur moyen pour y parvenir est de vivre en bon citoyen dans la société où vous m’avez fait naître, et en bon père dans la famille que vous m’avez donnée.\par

\dateline{À Paris, le 8 de la lune de Chahban, 1713.}
\section[{Lettre XLVII. Zachi à Usbek. À Paris.}]{Lettre XLVII. Zachi à Usbek. À Paris.}\renewcommand{\leftmark}{Lettre XLVII. Zachi à Usbek. À Paris.}

\noindent J’ai une grande nouvelle à t’apprendre : je me suis réconciliée avec Zéphis ; le sérail, partagé entre nous, s’est réuni. Il ne manque que toi dans ces lieux, où la paix règne : viens, mon cher Usbek, viens y faire triompher l’amour.\par
Je donnai à Zéphis un grand festin, où ta mère, tes femmes et tes principales concubines furent invitées ; tes tantes et plusieurs de tes cousines s’y trouvèrent aussi ; elles étaient venues à cheval, couvertes du sombre nuage de leurs voiles et de leurs habits.\par
Le lendemain nous partîmes pour la campagne, où nous espérions être plus libres ; nous montâmes sur nos chameaux, et nous nous mîmes quatre dans chaque loge. Comme la partie avait été faite brusquement, nous n’eûmes pas le temps d’envoyer à la ronde annoncer le courouc ; mais le premier eunuque, toujours industrieux, prit une autre précaution : car il joignit à la toile qui nous empêchait d’être vues un rideau si épais, que nous ne pouvions absolument voir personne.\par
Quand nous fûmes arrivées à cette rivière qu’il faut traverser, chacune de nous se mit, selon la coutume, dans une boîte, et se fit porter dans le bateau ; car on nous dit que la rivière était pleine de monde. Un curieux, qui s’approcha trop près du lieu où nous étions enfermées, reçut un coup mortel qui lui ôta pour jamais la lumière du jour ; un autre, qu’on trouva se baignant tout nu sur le rivage, eut le même sort ; et tes fidèles eunuques sacrifièrent à ton honneur et au nôtre ces deux infortunés.\par
Mais écoute le reste de nos aventures. Quand nous fûmes au milieu du fleuve, un vent si impétueux s’éleva et un nuage si affreux couvrit les airs, que nos matelots commencèrent à désespérer. Effrayées de ce péril, nous nous évanouîmes presque toutes. Je me souviens que j’entendis la voix et la dispute de nos eunuques, dont les uns disaient qu’il fallait nous avertir du péril et nous tirer de notre prison ; mais leur chef soutint toujours qu’il mourrait plutôt que de souffrir que son maître fût ainsi déshonoré, et qu’il enfoncerait un poignard dans le sein de celui qui ferait des propositions si hardies. Une de mes esclaves, toute hors d’elle, courut vers moi déshabillée, pour me secourir : mais un eunuque noir la prit brutalement, et la fit rentrer dans l’endroit d’où elle était sortie. Pour lors je m’évanouis, et ne revins à moi que lorsque le péril fut passé.\par
Que les voyages sont embarrassants pour les femmes ! Les hommes ne sont exposés qu’aux dangers qui menacent leur vie, et nous sommes à tous les instants dans la crainte de perdre notre vie ou notre vertu. Adieu, mon cher Usbek. Je t’adorerai toujours.\par

\dateline{Du sérail de Fatmé, le 2 de la lune de Rhamazan, 1713.}
\section[{Lettre XLVIII. Usbek à Rhédi. À Venise.}]{Lettre XLVIII. Usbek à Rhédi. À Venise.}\renewcommand{\leftmark}{Lettre XLVIII. Usbek à Rhédi. À Venise.}

\noindent Ceux qui aiment à s’instruire ne sont jamais oisifs : quoique je ne sois chargé d’aucune affaire importante, je suis cependant dans une occupation continuelle. Je passe ma vie à examiner ; j’écris le soir ce que j’ai remarqué, ce que j’ai vu, ce que j’ai entendu dans la journée ; tout m’intéresse, tout m’étonne : je suis comme un enfant, dont les organes encore tendres sont vivement frappés par les moindres objets.\par
Tu ne le croirais pas peut-être ; nous sommes reçus agréablement dans toutes les compagnies et dans toutes les sociétés : je crois devoir beaucoup à l’esprit vif et à la gaieté naturelle de Rica, qui fait qu’il recherche tout le monde, et qu’il en est également recherché. Notre air étranger n’offense plus personne ; nous jouissons même de la surprise où l’on est de nous trouver quelque politesse : car les Français n’imaginent pas que notre climat produise des hommes. Cependant, il faut l’avouer, ils valent la peine qu’on les détrompe.\par
J’ai passé quelques jours dans une maison de campagne auprès de Paris, chez un homme de considération, qui est ravi d’avoir de la compagnie chez lui. Il a une femme fort aimable, et qui joint à une grande modestie une gaieté que la vie retirée ôte toujours à nos dames de Perse.\par
Étranger que j’étais, je n’avais rien de mieux à faire que d’étudier, selon ma coutume, sur cette foule de gens qui y abordait sans cesse, dont les caractères me présentaient toujours quelque chose de nouveau. Je remarquai d’abord un homme dont la simplicité me plut : je m’attachai à lui, il s’attacha à moi : de sorte que nous nous trouvions toujours l’un auprès de l’autre.\par
Un jour que, dans un grand cercle, nous nous entretenions en particulier, laissant les conversations générales à elles-mêmes : Vous trouverez peut-être en moi, lui dis-je, plus de curiosité que de politesse ; mais je vous supplie d’agréer que je vous fasse quelques questions ; car je m’ennuie de n’être au fait de rien, et de vivre avec des gens que je ne saurais démêler. Mon esprit travaille depuis deux jours : il n’y a pas un seul de ces homme qui ne m’ait donné la torture plus de deux cent fois ; et cependant je ne les devinerais de mille ans : ils me sont plus invisibles que les femmes de notre grand monarque. Vous n’avez qu’à dire, me répondit-il, et je vous instruirai de tout ce que vous souhaiterez : d’autant mieux que je vous crois homme discret, et que vous n’abuserez pas de ma confiance,\par
Qui est cet homme, lui dis-je, qui nous a tant parlé des repas qu’il a donnés aux grands, qui est si familier avec vos ducs, et qui parle si souvent à vos ministres qu’on me dit être d’un accès si difficile ? Il faut bien que ce soit un homme de qualité : mais il a la physionomie si basse, qu’il ne fait guère honneur aux gens de qualité ; et d’ailleurs je ne lui trouve point d’éducation. Je suis étranger ; mais il me semble qu’il y a en général une certaine politesse commune à toutes les nations ; je ne lui trouve point de celle-là : est-ce que vos gens de qualité sont plus mal élevés que les autres ? Cet homme, me répondit-il en riant, est un fermier : il est autant au-dessus des autres par ses richesses qu’il est au-dessus de tout le monde par sa naissance ; il aurait la meilleure table de Paris, s’il pouvait se résoudre à ne manger jamais chez lui. Il est bien impertinent, comme vous voyez, mais il excelle par son cuisiner : aussi n’en est-il pas ingrat ; car vous avez entendu qu’il l’a loué tout aujourd’hui.\par
Et ce gros homme vêtu de noir, lui dis-je, que cette dame a fait placé auprès d’elle, comment a-t-il un habit si lugubre avec un air si gai et un teint si fleuri ? Il sourit gracieusement dès qu’on lui parle ; sa parure est plus modeste, mais plus arrangée que celle de vos femmes. C’est, me répondit-il, un prédicateur, et, qui pis est, un directeur. Tel que vous le voyez, il en sait plus que les maris ; il connaît le faible des femmes : elles savent aussi qu’il a le sien. Comment, dis-je, il parle toujours de quelque chose qu’il appelle la grâce ? Non pas toujours, me répondit-il : à l’oreille d’une jolie femme, il parle encore plus volontiers de sa chute : il foudroie en public, mais il est doux comme un agneau en particulier. Il me semble, dis-je pour lors, qu’on le distingue beaucoup, et qu’on a de grands égards pour lui. Comment ! si on le distingue ! C’est un homme nécessaire ; il fait la douceur de la vie retirée ; petits conseils, soins vicieux, visites marquées ; il dissipe un mal de tête mieux qu’un homme du monde ; c’est un homme excellent.\par
Mais, si je ne vous importune pas, dites-moi qui est celui qui est vis-à-vis de nous, qui est si mal habillé ; qui fait quelquefois des grimaces, et a un langage différent des autres ; qui n’a pas d’esprit pour parler, mais parle pour avoir de l’esprit ? C’est, me répondit-il, un poète, et le grotesque du genre humain. Ces gens-là disent qu’ils sont nés ce qu’ils sont ; cela est vrai, et aussi ce qu’ils seront toute leur vie, c’est-à-dire presque toujours les plus ridicules de tous les hommes : aussi ne les épargne-t-on point ; on verse sur eux le mépris à pleines mains. La famine a fait entrer celui-ci dans cette maison ; et il y est bien reçu du maître et de la maîtresse, dont la bonté et la politesse ne se démentent à l’égard de personne ; il fit leur épithalame lorsqu’ils se marièrent : c’est ce qu’il a fait de mieux dans sa vie ; car il s’est trouvé que le mariage a été aussi heureux qu’il l’a prédit.\par
Vous ne le croiriez pas peut-être, ajouta-t-il, entêté comme vous êtes des préjugés \\
de l’Orient : il y a parmi nous des mariages heureux, et des femmes dont la vertu est un gardien sévère. Les gens dont nous parlons goûtent entre eux une paix qui ne peut être troublée ; ils sont aimés et estimés de tout le monde : il n’y a qu’une chose ; c’est que leur bonté naturelle leur fait recevoir chez eux toute sorte de monde ; ce qu’il fait qu’il y a quelquefois mauvaise compagnie. Ce n’est pas que je les désapprouve ; il faut vivre avec les gens tels qu’ils sont : les gens qu’on dit être de bonne compagnie ne sont souvent que ceux dont le vice est plus raffiné ; et peut-être qu’il en est comme les poisons, dont les plus subtils sont aussi les plus dangereux.\par
Et ce vieux homme, lui dis-je tout bas, qui a l’air si chagrin ? je l’ai pris d’abord pour un étranger ; car outre qu’il est habillé autrement que les autres, il censure tout ce qui se fait en France, et n’approuve pas votre gouvernement. C ‘est un vieux guerrier, me dit-il, qui se rend mémorable à tous ses auditeurs par la longueur de ses exploits. Il ne peut souffrir que la France ait gagné des batailles ou il ne se soit pas trouvé, ou qu’on vente un siège ou il n’ait pas monté à la tranchée : il se croit si nécessaire à notre histoire, qu’il s’imagine qu’elle finit où il a fini ; il regarde quelques blessures qu’il a reçues, comme la dissolution de la monarchie ; et, à la différence de ces philosophes qui disent qu’on ne jouit que du présent, et que le passé n’est rien, il ne jouit, au contraire, que du passé, et n’existe que dans les campagnes qu’il a faites : il respire dans les temps qui se sont écoulés, comme les héros doivent vivre dans ceux qui passeront après eux. Mais pourquoi, dis-je, a-t-il quitté le service ? Il ne l’a point quitté, me répondit-il ; mais le service l’a quitté ; on l’a employé dans une petite place où il racontera le reste de ses jours, mais il n’ira jamais plus loin : le chemin des honneurs lui est fermé. Et pourquoi cela ? lui dis-je. Nous avons une maxime en France, me répondit-il : c’est de n’élever jamais les officiers dont la patience a langui dans les emplois subalternes ; nous les regardons comme des gens dont l’esprit s’est comme rétréci dans les détails, et qui, par une habitude de petites choses, sont devenus incapables des plus grandes. Nous croyons qu’un homme qui n’a pas les qualités d’un général à trente ans ne les aura jamais ; que celui qui n’a pas ce coup d’œil qui montre tout d’un coup un terrain de plusieurs lieues dans toutes ses situations différentes, cette présence d’esprit qui fait que dans une victoire on se sert de tous ses avantages, et dans un échec de toutes ses ressources, n’acquerra jamais ces talents : c’est pour cela que nous avons des emplois brillants pour ces hommes grands et sublimes que le ciel a partagés non seulement d’un cœur, mais aussi d’un génie héroïque ; et des emplois subalternes pour ceux dont les talents le sont aussi. De ce nombre sont ces gens qui ont vieilli dans une guerre obscure ; ils ne réussissent tout au plus qu’à faire ce qu’ils ont fait ; et il ne faut point commencer à les charger dans le temps qu’ils s’affaiblissent.\par
Un moment après, la curiosité me reprit, et je lui dis : Je m’engage à ne plus vous faire de questions, si vous voulez encore souffrir celle-ci. Qui est ce grand jeune homme qui a des cheveux, peu d’esprit et tant d’impertinence ? D’où vient qu’il parle plus haut que les autres, et se sait si bon gré d’être au monde ? C’est un homme à bonnes fortunes, me répondit-il. À ces mots, des gens entrèrent, d’autres sortirent, on se leva, quelqu’un vint parler à mon gentilhomme, et je restai aussi peu instruit qu’auparavant. Mais un moment après, je ne sais par quel hasard ce jeune homme se trouva auprès de moi, et, m’adressant la parole : Il fait beau ; voudriez-vous, monsieur, faire un tour de parterre ? Je lui répondis le plus civilement qu’il me fut possible, et nous sortîmes ensemble. Je suis venu à la campagne, me dit-il, pour faire plaisir à la maîtresse de la maison, avec laquelle je ne suis pas mal : il y a bien certaine femme dans le monde qui pestera un peu, mais qu’y faire ? Je vois les plus jolies femmes de Paris ; mais je ne me fixe pas à une, et je leur en donne bien à garder : car, entre vous et moi, je ne vaux pas grand-chose. Apparemment, monsieur, lui dis-je, que vous avez quelque charge ou quelque emploi, qui vous empêche d’être plus assidu auprès d’elles. Non, monsieur, je n’ai d’autre emploi que de faire enrager un mari, ou désespérer un père ; j’aime à alarmer une femme qui croit me tenir, et la mettre à deux doigts de ma perte. Nous sommes quelques jeunes gens qui partageons ainsi tout Paris, et l’intéressons à nos moindres démarches. À ce que je comprends, lui dis-je, vous faites plus de bruit que le guerrier le plus valeureux, et vous êtes plus considéré qu’un grave magistrat. Si vous étiez en Perse, vous ne jouiriez pas de tous ces avantages ; vous deviendriez plus propre à garder nos dames qu’à leur plaire. Le feu me monta au visage ; et je crois que, pour peu que j’eusse parlé, je n’aurais pu m’empêcher de le brusquer.\par
Que dis-tu d’un pays où l’on tolère de pareilles gens, et où l’on laisse vivre un homme qui un tel métier ? où l’infidélité, la trahison, le rapt, la perfidie et l’injustice conduisent à la considération ? où l’on estime un homme parce qu’il ôte une fille à son père, une femme à son mari, et trouble les sociétés les plus douces et les plus saintes ? Heureux les enfants d’Ali, qui défendent leurs familles de l’opprobre et de la séduction ! La lumière du jour n’est pas plus pure que le feu qui brûle dans le cœur de nos femmes : nos filles ne pensent qu’en tremblant au jour qui doit les priver de cette vertu, qui les rend semblables aux anges et aux puissances incorporelles. Terre natale et chérie, sur qui le soleil jette ses premiers regards, tu n’es point souillée par les crimes horribles qui obligent cet astre à se cacher dès qu’il paraît dans le noir Occident.\par

\dateline{À Paris, le 5 de la lune de Rhamazan, 1713.}
\section[{Lettre XLIX. Rica à Usbek. À ***.}]{Lettre XLIX. Rica à Usbek. À ***.}\renewcommand{\leftmark}{Lettre XLIX. Rica à Usbek. À ***.}

\noindent Étant l’autre jour dans ma chambre, je vis entrer un dervis extraordinairement habillé : sa barbe descendait jusqu’à sa ceinture de corde ; il avait les pieds nus ; son habit était gris, grossier, et en quelques endroits pointus. Le tout me parut si bizarre, que ma première idée fut d’envoyer chercher un peintre pour en faire une fantaisie.\par
Il me fit d’abord un grand compliment, dans lequel il m’apprit qu’il était homme de mérite, et de plus capucin. On m’a dit, ajouta-t-il, monsieur, que vous retournez bientôt à la cour de Perse, où vous tenez un rang distingué : je viens vous demander protection, et vous prier de nous obtenir du roi une petite habitation, auprès de Casbin, pour deux ou trois religieux. Mon père, lui dis-je, vous voulez donc aller en Perse ? Moi, monsieur ! me dit-il ; je m’en donnerai bien de garde. Je suis ici provincial, et je ne troquerais pas ma condition contre celle de tous les capucins de ce monde. Et que diable me demandez-vous donc ? C’est, me répondit-il, que si nous avions cet hospice, nos pères d’Italie y enverraient deux ou trois de leurs religieux : Vous les connaissez apparemment, lui dis-je, ces religieux ? Non monsieur, je ne les connais pas. Eh morbleu ! que vous importe donc qu’ils aillent en Perse ? C’est un beau projet de faire respirer l’air de Casbin à deux capucins : cela sera très utile et à l’Europe et à l’Asie ; il est fort nécessaire d’intéresser là-dedans des monarques : voilà ce qui s’appelle de bonnes colonies ! Allez, vous et vos semblables n’êtes point faits pour être transplantés, et vous ferez bien de continuer à ramper dans les endroits où vous vous êtes engendrés.\par

\dateline{À Paris, le 15 de la lune de Rhamazan, 1713.}
\section[{Lettre L. Rica à ***.}]{Lettre L. Rica à ***.}\renewcommand{\leftmark}{Lettre L. Rica à ***.}

\noindent J’ai vu des gens chez qui la vertu était si naturelle, qu’elle ne se faisait pas même sentir : ils s’attachaient à leur devoir sans s’y plier, et s’y portaient comme par instinct ; bien loin de relever par leurs discours leurs rares qualités, il semblait qu’elles n’avaient pas percé jusqu’à eux. Voilà les gens que j’aime ; non pas ces gens vertueux qui semblent être étonnés de l’être, et qui regardent une bonne action comme un prodige dont le récit doit surprendre.\par
Si la modestie est une vertu nécessaire à ceux à qui le ciel a donné de grands talents, que peut-on dire de ces insectes qui osent faire paraître un orgueil qui déshonorerait les plus grands hommes ?\par
Je vois de tous côtés des gens qui parlent sans cesse d’eux-mêmes : leurs conversations sont un miroir qui présente toujours leur impertinente figure ; ils vous parleront des moindres choses qui leur sont arrivées, et ils veulent que l’intérêt qu’ils y prennent les grossisse à vos yeux ; ils ont tout fait, tout vu, tout dit, tout pensé : ils sont un modèle universel, un sujet de comparaison inépuisable, une source d’exemples qui ne tarit jamais. Oh ! que la louange est fade lorsqu’elle réfléchit vers le lieu d’où elle part !\par
Il y a quelques jours qu’un homme de ce caractère nous accabla pendant deux heures de lui, de son mérite et de ses talents ; mais comme il n’y a point de mouvement perpétuel dans le monde, il cessa de parler ; la conversation nous revint donc, et nous la prîmes.\par
Un homme qui paraissait assez chagrin commença par se plaindre de l’ennui répandu dans les conversations. Quoi ! toujours des sots qui se peignent eux-mêmes, et qui ramènent tout à eux ? Vous avez raison, reprit brusquement notre discoureur : il n’y a qu’à faire comme moi ; je ne me loue jamais ; j’ai du bien, de la naissance, je fais de la dépense, mes amis disent que j’aie quelque esprit ; mais je ne parle jamais de tout cela : si j’ai quelques bonnes qualités, celle dont je fais le plus de cas, c’est ma modestie.\par
J’admirais cet impertinent ; et pendant qu’il parlait tout haut, je disais tout bas : Heureux celui qui a assez de vanité pour ne dire jamais de bien de lui ; qui craint ceux qui l’écoutent ; et ne compromet point son mérite avec l’orgueil des autres !\par

\dateline{À Paris, le 20 de la lune de Rhamazan, 1713.}
\section[{Lettre LI. Nargum, envoyé de Perse en Moscovie, à Usbek. À Paris.}]{Lettre LI. Nargum, envoyé de Perse en Moscovie, à Usbek. À Paris.}\renewcommand{\leftmark}{Lettre LI. Nargum, envoyé de Perse en Moscovie, à Usbek. À Paris.}

\noindent On m’a écrit d’Ispahan que tu avais quitté la Perse, et que tu étais actuellement à Paris. Pourquoi faut-il que j’apprenne de tes nouvelles par d’autres que par toi ?\par
Les ordres du roi des rois me retiennent depuis cinq ans dans ce pays-ci, où j’ai terminé plusieurs négociations importantes.\par
Tu sais que le czar est le seul des princes chrétiens dont les intérêts soient mêlés avec ceux de la Perse, parce qu’il est ennemi des Turcs comme nous.\par
Son empire est plus grand que le nôtre : car on compte deux mille lieues depuis Moscou jusqu’à la dernière place de ses États du côté de la Chine.\par
Il est le maître absolu de la vie et des biens de ses sujets, qui sont tous esclaves, à la réserve de quatre familles. Le lieutenant des prophètes, le roi des rois, qui a le ciel pour marchepied, ne fait pas un exercice plus redoutable de sa puissance.\par
À voir le climat affreux de la Moscovie, on ne croirait jamais que ce fût une peine d’en être exilé : cependant, dès qu’un grand est disgracié, on le relègue en Sibérie.\par
Comme la loi de notre prophète nous défend de boire du vin, celle du prince le défend aux Moscovites.\par
Ils ont une manière de recevoir leurs hôtes, qui n’est point du tout persane. Dès qu’un étranger entre dans une maison, le mari lui présente sa femme ; l’étranger la baise ; et cela passe pour une politesse faite au mari.\par
Quoique les pères, au contrat de mariage de leurs filles, stipulent ordinairement que le mari ne les fouettera pas, cependant on ne saurait croire combien les femmes moscovites aiment à être battues : elles ne peuvent comprendre qu’elles possèdent le cœur de leur mari, s’il ne les bat comme il faut ; une conduite opposée, de sa part, est une marque d’indifférence impardonnable. Voici une lettre qu’une d’elles écrivit dernièrement à sa mère :\par
"Ma chère mère,\par
Je suis la plus malheureuse femme du monde ; il n’y a rien que je n’aie fait pour me faire aimer de mon mari, et je n’ai jamais pu y réussir. Hier, j’avais mille affaires dans la maison ; je sortis, et je demeurais tout le jour dehors : je crus, à mon retour, qu’il me battrait bien fort ; mais il ne me dit pas un seul mot. Ma sœur est bien autrement traitée : son mari la roue de coups tous les jours ; elle ne peut pas regarder un homme, sans qu’il ne l’assomme soudain : ils s’aiment beaucoup aussi, et ils vivent de la meilleure intelligence du monde.\par
C’est ce qui la rend si fière ; mais je ne lui donnerai pas longtemps sujet de me mépriser. J’ai résolu de me faire aimer de mon mari, à quelque prix que ce soit : je le ferai si bien enrager, qu’il faudra bien qu’il me donne des marques d’amitié. Il ne sera pas dit que je ne serai pas battue, et que je vivrai dans la maison sans que l’on pense à moi. La moindre chiquenaude qu’il me donnera, je crierai de toute ma force, afin qu’on s’imagine qu’il y va tout de tout bon ; et je crois que, si quelque voisin venait au secours, je l’étranglerais. Je vous supplie, ma chère mère, de vouloir bien représenter à mon mari qu’il me traite d’une manière indigne. Mon père, qui est un si honnête homme, n’agissait pas de même ; et il me souvient, lorsque j’étais petite fille, qu’il me semblait quelquefois qu’il vous aimait trop. Je vous embrasse, ma chère mère."\par
Les Moscovites ne peuvent point sortir de l’empire, quand ce serait pour voyager. Ainsi, séparés des autres nations par les lois du pays, ils ont conservé leurs anciennes coutumes avec d’autant plus d’attachements qu’ils ne croyaient pas qu’il fût possible qu’on en pût avoir d’autres.\par
Mais le prince qui règne à présent a voulu tout changer ; il y a eu de grands démêlés avec eux au sujet de leur barbe : le clergé et les moines n’ont pas moins combattu en faveur de leur ignorance.\par
Il s’attache à faire fleurir les arts, et ne néglige rien pour porter dans l’Europe et l’Asie la gloire de sa nation, oubliée jusqu’ici, et presque uniquement connue d’elle-même.\par
Inquiet et sans cesse agité, il erre dans ses vastes États, laissant partout des marques de sa sévérité naturelle.\par
Il les quitte, comme s’ils ne pouvaient le contenir, et va chercher dans l’Europe d’autres provinces et de nouveaux royaumes.\par
Je t’embrasse, mon cher Usbek : donne-moi de tes nouvelles, je te conjure.\par

\dateline{De Moscou, le 2 de la lune de Chalval, 1713.}
\section[{Lettre LII. Rica à Usbek. À ***.}]{Lettre LII. Rica à Usbek. À ***.}\renewcommand{\leftmark}{Lettre LII. Rica à Usbek. À ***.}

\noindent J’étais l’autre jour dans une société où je me divertis assez bien. Il y avait là des femmes de tous les âges : une de quatre-vingts ans, une de soixante, une de quarante, laquelle avait une nièce qui pouvait en avoir vingt ou vingt-deux. Un certain instinct me fit approcher de cette dernière, et elle me dit à l’oreille : Que dites-vous de ma tante, qui à son âge veut avoir des amants, et fait la jolie ? Elle a tort, lui dis-je : c’est un dessein qui ne convient qu’à vous. Un moment après, je me trouvai auprès de sa tante, qui me dit : Que dites-vous de cette femme qui a pour le moins soixante ans, qui a passé aujourd’hui plus d’une heure à sa toilette ? C’est du temps perdu, lui dis-je ; et il faut avoir vos charmes pour devoir y songer. J’allai à cette malheureuse femme de soixante ans, et la plaignais dans mon âme, lorsqu’elle me dit à l’oreille : Y a-t-il rien de si ridicule ? Voyez-vous cette femme qui a quatre-vingts ans, et qui met des rubans couleur de feu ; elle veut faire la jeune, et elle y réussit : car cela approche de l’enfance. Ah ! bon Dieu, dis-je en moi-même, ne sentirons-nous jamais que le ridicule des autres ? C’est peut-être un bonheur, disais-je ensuite, que nous trouvions de la consolation dans les faiblesses d’autrui. Cependant j’étais en train de me divertir, et je dis : Nous avons assez monté, descendons à présent, et commençons par la vieille qui est au sommet. Madame, vous vous ressemblez si fort, cette dame à qui je viens de parler et vous, qu’il semble que vous soyez deux sœurs ; et je ne crois pas que vous soyez plus âgées l’une que l’autre. Eh ! vraiment, monsieur, me dit-elle, lorsque l’une mourra, l’autre devra avoir grand-peur : je ne crois pas qu’il y ait d’elle à moi deux jours de différence. Quand je tins cette femme décrépite, j’allai à celle de soixante ans : Il faut, madame, que vous décidiez un pari que j’ai fait ; j’ai gagé que cette femme et vous (lui montrant la femme de quarante ans) étiez de même âge. Ma foi, dit-elle, je ne crois pas qu’il y ait six mois de différence. Bon, m’y voilà ; continuons. Je descendis encore, et j’allai à la femme de quarante ans. Madame, faites-moi la grâce de me dire si c’est pour rire que vous appelez cette demoiselle, qui est à l’autre table, votre nièce ? Vous êtes aussi jeune qu’elle ; elle a même quelque chose dans le visage de passé, que vous n’avez certainement pas ; et ces couleurs vives qui paraissent sur votre teint… Attendez, me dit-elle : je suis sa tante, mais sa mère avait pour le moins vingt-cinq ans plus que moi : nous n’étions pas de même lit ; j’ai ouï dire à feu ma sœur que sa fille et moi naquîmes la même année. Je le disais bien, madame, et je n’avais pas tort d’être étonné.\par
Mon cher Usbek, les femmes qui se sentent finir d’avance par la perte de leurs agréments voudraient reculer vers la jeunesse. Eh ! comment ne chercheraient-elles pas à tromper les autres ? elles font tous leurs efforts pour se tromper elles-mêmes, et pour se dérober à la plus affligeante de toutes les idées.\par

\dateline{À Paris, le 3 de la lune de Chalval, 1713.}
\section[{Lettre LIII. Zélis à Usbek. À Paris.}]{Lettre LIII. Zélis à Usbek. À Paris.}\renewcommand{\leftmark}{Lettre LIII. Zélis à Usbek. À Paris.}

\noindent Jamais passion n’a été plus forte et plus vive que celle de Cosrou, eunuque blanc, pour mon esclave Zélide ; il la demande en mariage avec tant de fureur, que je ne puis la lui refuser. Et pourquoi ferais-je de la résistance, lorsque sa mère n’en fait pas, et que Zélide elle-même paraît satisfaite de l’idée de ce mariage imposteur, et de l’ombre vaine qu’on lui présente ?\par
Que veut-elle faire de cet infortuné, qui n’aura d’un mari que la jalousie ; qui ne sortira de sa froideur que pour entrer dans un désespoir inutile ; qui se rappellera toujours la mémoire de ce qu’il a été, pour la faire souvenir de ce qu’il n’est plus ; qui, toujours prêt à se donner, et ne se donnant jamais, se trompera, la trompera sans cesse, et lui fera essuyer à chaque instant tous les malheurs de sa condition ?\par
Hé quoi ! être toujours dans les images et dans les fantômes ? ne vivre que pour imaginer ? se trouver toujours auprès des plaisirs et jamais dans les plaisirs ? languissante dans les bras d’un malheureux, au lieu de répondre à ses soupirs, ne répondre qu’à ses regrets ?\par
Quel mépris ne doit-on pas avoir pour un homme de cette espèce, fait uniquement pour garder, et jamais pour posséder ? Je cherche l’amour, et je ne le vois pas.\par
Je te parle librement, parce que tu aimes ma naïveté, et que tu préfères mon air libre et ma sensibilité pour les plaisirs à la pudeur feinte de mes compagnes.\par
Je t’ai ouï dire mille fois que les eunuques goûtent avec les femmes une sorte de volupté qui nous est inconnue ; que la nature se dédommage de ses pertes ; qu’elle a des ressources qui réparent le désavantage de leur condition ; qu’on peut bien cesser d’être homme, mais non pas d’être sensible ; et que, dans cet état, on est comme dans un troisième sens, où l’on ne fait, pour ainsi dire, que changer de plaisirs.\par
Si cela était, je trouverais Zélide moins à plaindre ; c’est quelque chose de vivre avec des gens moins malheureux.\par
Donne-moi tes ordres là-dessus, et fais-moi savoir si tu veux que le mariage s’accomplisse dans le sérail. Adieu.\par

\dateline{Du sérail d’Ispahan, le 5 de la lune de Chalval, 1713.}
\section[{Lettre LIV. Rica à Usbek. À ***.}]{Lettre LIV. Rica à Usbek. À ***.}\renewcommand{\leftmark}{Lettre LIV. Rica à Usbek. À ***.}

\noindent J’étais ce matin dans ma chambre, laquelle, comme tu sais, n’est séparée des autres que par une cloison fort mince, et percée en plusieurs endroits ; de manière qu’on entend tout ce qui se dit dans la chambre voisine. Un homme, qui se promenait à grand pas, disait à un autre : Je ne sais ce que c’est, mais tout se tourne contre moi ; il y a plus de trois jours que je n’ai rien dit qui m’ait fait honneur ; et je me suis trouvé confondu pêle-mêle dans toutes les conversations, sans qu’on ait fait la moindre attention à moi, et qu’on m’ait deux fois adressé la parole. J’avais préparé quelques saillies pour relever mon discours ; jamais on n’a voulu souffrir que je les fisse venir : j’avais un conte fort joli à faire ; mais à mesure que j’ai voulu l’approcher, on l’a esquivé comme si on l’avait fait exprès : j’ai quelques bons mots, qui depuis quatre jours vieillissent dans ma tête, sans que j’en aie pu faire le moindre usage. Si cela continue, je crois qu’à la fin je serai un sot : il semble que ce soit mon étoile, et que je ne puisse m’en dispenser. Hier, j’avais espéré de briller avec trois ou quatre vieilles femmes qui certainement ne m’imposent point, et je devais dire les plus jolies choses du monde : je fus plus d’un quart d’heure à diriger ma conversation ; mais elles ne tinrent jamais un propos suivi, et elles coupèrent, comme des parques fatales, le fil de tous mes discours. Veux-tu que je te dise ? La réputation de bel esprit coûte bien à soutenir. Je ne sais comment tu as fait pour y parvenir. Il me vient dans l’idée une chose, reprit l’autre : travaillons de concert à nous donner de l’esprit ; associons-nous pour cela. Nous nous dirons chacun tous les jours de quoi nous devons parler ; et nous nous secourrons si bien que, si quelqu’un vient nous interrompre au milieu de nos idées, nous l’attirerons nous-mêmes ; et s’il ne veut pas venir de bon gré, nous lui ferons violence. Nous conviendrons des endroits où il faudra approuver, de ceux où il faudra sourire, des autres où il faudra rire tout à fait, et à gorge déployée. Tu verras que nous donnerons le ton à toutes les conversations, et qu’on admirera la vivacité de notre esprit et le bonheur de nos reparties. Nous nous protégerons par des signes de tête mutuels. Tu brilleras aujourd’hui, demain tu seras mon second. J’entrerai avec toi dans une maison, et je m’écrierai en te montrant : Il faut que je vous dise une réponse bien plaisante que monsieur vient de faire à un homme que nous avons trouvé dans la rue ; et je me tournerai vers toi ; il ne s’y attendait pas ; il a été bien étonné. Je réciterai quelques-uns de mes vers, et tu diras : J’y étais quand il les fit ; c’était dans un souper, et il ne rêva pas un moment. Souvent même nous nous raillerons toi et moi ; et l’on dira : Voyez comme ils s’attaquent, comme ils se défendent ; ils ne s’épargnent pas ; voyons comme il sortira de là ; à merveille ! quelle présence d’esprit ! voilà une véritable bataille. Mais on ne dira pas que nous nous étions escarmouchés dès la veille. Il faudra acheter de certains livres qui sont des recueils de bons mots composés à l’usage de ceux qui n’ont pas d’esprit et qui en veulent contrefaire : tout dépend d’avoir des modèles. Je veux qu’avant six mois nous soyons en état de tenir une conversation d’une heure toute remplie de bons mots. Mais il faudra avoir une attention ; c’est de soutenir leur fortune : ce n’est pas tout que de dire un bon mot, il faut le répandre et le semer partout ; sans cela, autant de perdu ; et je t’avoue qu’il n’y a rien de si désolant que de voir une jolie chose qu’on dite mourir dans l’oreille d’un sot qui l’entend. Il est vrai que souvent il y a une compensation, et que nous disons aussi bien des sottises qui passent {\itshape incognito} ; et c’est la seule chose qui peut nous consoler dans cette occasion. Voilà, mon cher, le parti qu’il nous faut prendre. Fais ce que je te dirai, et je te promets avant six mois une place à l’Académie : c’est pour te dire que le travail ne sera pas long, car pour lors tu pourras renoncer à ton art ; tu seras homme d’esprit, malgré que tu en aies. On remarque en France que, dès qu’un homme entre dans une compagnie, il prend d’abord ce qu’on appelle l’esprit du corps : tu en seras de même ; et je ne crains pour toi que l’embarras des applaudissements.\par

\dateline{À Paris, le 6 de la lune de Zilcadé, 1714.}
\section[{Lettre LV. Rica à Ibben. À Smyrne.}]{Lettre LV. Rica à Ibben. À Smyrne.}\renewcommand{\leftmark}{Lettre LV. Rica à Ibben. À Smyrne.}

\noindent Chez les peuples d’Europe, le premier quart d’heure du mariage aplanit toutes les difficultés ; les dernières faveurs sont toujours de même date que la bénédiction nuptiale : les femmes n’y font point comme nos Persanes, qui disputent le terrain quelquefois des mois entiers ; il n’y a rien de si plénier : si elles ne perdent rien, c’est qu’elles n’ont rien à perdre ; mais on sait toujours, chose honteuse ! le moment de leur défaite ; et, sans consulter les astres, on peut prédire au juste l’heure de la naissance de leurs enfants.\par
Les Français ne parlent presque jamais de leurs femmes : c’est qu’ils ont peur d’en parler devant des gens qui les connaissent mieux qu’eux.\par
Il y a parmi eux des hommes très malheureux que personne ne console : ce sont les maris jaloux ; il y en a que tout le monde hait : ce sont les maris jaloux ; il y en a que tous les hommes méprisent : ce sont encore les maris jaloux.\par
Aussi n’y a-t-il point de pays où ils soient en si petit nombre que chez les Français. Leur tranquillité n’est pas fondée sur la confiance qu’ils ont en leurs femmes ; c’est au contraire sur la mauvaise opinion qu’ils en ont : toutes les sages précautions des Asiatiques, les voiles qui les couvrent, les prisons où elles sont détenues, la vigilance des eunuques, leur paraissent des moyens plus propres à exercer l’industrie du sexe qu’à la lasser. Ici les maris prennent leur parti de bonne grâce, et regardent les infidélités comme des coups d’une étoile inévitable. Un mari qui voudrait seul posséder sa femme serait regardé comme perturbateur de la joie publique, et comme un insensé qui voudrait jouir de la lumière du soleil à l’exclusion des autres hommes.\par
Ici un mari qui aime sa femme est un homme qui n’a pas assez de mérite pour se faire aimer d’une autre ; qui abuse de la nécessité de la loi, pour suppléer aux agréments qui lui manquent ; qui se sert de tous ses avantages au préjudice d’une société entière ; qui s’approprie ce qui ne lui avait été donné qu’en engagement, et qui agit autant qu’il est en lui pour renverser une convention tacite qui fait le bonheur de l’un et de l’autre sexe. Ce titre de mari d’une jolie femme, qui se cache en Asie avec tant de soin, se porte ici sans inquiétude : on se sent en état de faire diversion partout. Un prince se console de la perte d’une place par la prise d’une autre : dans le temps que le Turc nous prenait Bagdad, n’enlevions-nous pas au Mogol la forteresse de Candahar ?\par
Un homme qui, en général, souffre les infidélités de sa femme n’est point désapprouvé ; au contraire, on le loue de sa prudence : il n’y a que les cas particuliers qui déshonorent.\par
Ce n’est pas qu’il n’y ait des dames vertueuses, et on peut dire qu’elles sont distinguées ; mon conducteur me les faisait toujours remarquer : mais elles étaient toutes si laides, qu’il faut être un saint pour ne pas haïr la vertu.\par
Après ce que je t’ai dit des mœurs de ce pays-ci, tu t’imagines facilement que les Français ne s’y piquent guère de constance : ils croient qu’il est aussi ridicule de jurer à une femme qu’on l’aimera toujours, que de soutenir qu’on se portera toujours bien, ou qu’on sera toujours heureux. Quand ils promettent à une femme qu’ils l’aimeront toujours, ils supposent qu’elle, de son côté, leur promet d’être toujours aimable ; et si elle manque à sa parole, ils ne se croient plus engagés à la leur.\par

\dateline{À Paris, le 7 de la lune de Zilcadé, 1714.}
\section[{Lettre LVI. Usbek à Ibben. À Smyrne.}]{Lettre LVI. Usbek à Ibben. À Smyrne.}\renewcommand{\leftmark}{Lettre LVI. Usbek à Ibben. À Smyrne.}

\noindent Le jeu est très en usage en Europe : c’est un état que d’être joueur ; ce seul titre tient lieu de naissance, de bien, de probité : il met tout homme qui le porte au rang des honnêtes gens, sans examen ; quoiqu’il n’y ait personne qui ne sache qu’en jugeant ainsi il s’est trompé très souvent : mais on est convenu d’être incorrigible.\par
Les femmes y sont surtout très adonnées ; il est vrai qu’elles ne s’y livrent guère dans leur jeunesse que pour favoriser une passion plus chère ; mais, à mesure qu’elles vieillissent, leur passion pour le jeu semble rajeunir, et cette passion remplit tout le vide des autres.\par
Elles veulent ruiner leurs maris ; et pour y parvenir, elles ont des moyens pour tous les âges, depuis leur plus tendre jeunesse jusqu’à la vieillesse la plus décrépite : les habits et les équipages commencent le dérangement, la coquetterie l’augmente, le jeu l’achève.\par
J’ai vu souvent neuf ou dix femmes, ou plutôt neuf ou dix siècles, rangées autour d’une table ; je les ai vues dans leurs espérances, dans leurs craintes, dans leurs joies, surtout dans leurs fureurs : tu aurais dit qu’elles n’auraient jamais le temps de s’apaiser, et que la vie allait les quitter avant leur désespoir ; tu aurais été en doute si ceux qu’elles payaient étaient leurs créanciers, ou leurs légataires.\par
Il semble que notre saint prophète ait eu principalement en vue de nous priver de tout ce qui peut troubler notre raison : il nous a interdit l’usage du vin, qui la tient ensevelie ; il nous a, par un précepte exprès, défendu les jeux de hasard ; et quand il lui a été impossible d’ôter la cause des passions, il les a amorties. L’amour parmi nous ne porte ni trouble ni fureur : c’est une passion languissante qui lasse notre âme dans le calme ; la pluralité des femmes nous sauve de leur empire ; elle tempère la violence de nos désirs.\par

\dateline{À Paris, le 10 de la lune de Zilhagé, 1714.}
\section[{Lettre LVII. Usbek à Rhédi. À Venise.}]{Lettre LVII. Usbek à Rhédi. À Venise.}\renewcommand{\leftmark}{Lettre LVII. Usbek à Rhédi. À Venise.}

\noindent Les libertins entretiennent ici un nombre infini de filles de joie ; et les dévots un nombre innombrable de dervis. Ces dervis font trois vœux, d’obéissance, de pauvreté, et de chasteté. On dit que le premier est le mieux observé de tous ; quant au second, je te réponds qu’il ne l’est point : je te laisse à juger du troisième.\par
Mais, quelque riches que soient ces dervis, ils ne quittent jamais la qualité des pauvres ; notre glorieux sultan renoncerait plutôt à ses magnifiques et sublimes titres : ils ont raison ; car ce titre de pauvreté les empêche de l’être.\par
Les médecins, et quelques-uns de ces dervis, qu’on appelle confesseurs, sont toujours ici ou trop estimés ou trop méprisés ; cependant on dit que les héritiers s’accommodent mieux des médecins que des confesseurs.\par
Je fus l’autre jour dans un couvent de ces dervis ; un d’entre eux, vénérable par ses cheveux blancs, m’accueillit fort honnêtement ; et après m’avoir fait voir toute la maison, il me mena dans le jardin, où nous nous mîmes à discourir. Mon père, lui dis-je, quel emploi avez-vous dans la communauté ? Monsieur, me répondit-il avec un air très content de ma question, je suis casuiste. Casuiste ? repris-je : depuis que je suis en France, je n’ai pas ouï parler de cette charge. Quoi ! vous ne savez pas ce que c’est qu’un casuiste ? Eh bien ! écoutez, je vais vous en donner une idée qui ne vous laissera rien à désirer. Il y a deux sortes de péchés : de mortels, qui excluent absolument du paradis ; de véniels, qui offensent Dieu à la vérité, mais ne l’irritent pas au point de nous priver de la béatitude. Or tout notre art consiste à bien distinguer ces deux sortes de péchés : car, à la réserve de quelques libertins, tous les chrétiens veulent gagner le paradis ; mais il n’y a guères personne qui ne le veuille gagner à meilleur marché qu’il est possible. Quand on connaît bien les péchés mortels, on tâche de ne pas commettre de ceux-là, et l’on fait son affaire. Il y a des hommes qui n’aspirent pas à une si grande perfection ; et comme ils n’ont point d’ambition, ils ne se soucient pas des premières places : aussi ils entrent en paradis le plus juste qu’ils peuvent ; pourvu qu’ils y soient, cela leur suffit : leur but est de n’en faire ni plus ni moins. Ce sont des gens qui ravissent le ciel plutôt qu’ils ne l’obtiennent, et qui disent à Dieu : Seigneur, j’ai accompli les conditions à la rigueur ; vous ne pouvez vous empêcher de tenir vos promesses : comme je n’en ai pas fait plus que vous n’en avez demandé, je vous dispense de m’en accorder plus que vous n’en avez promis.\par
Nous sommes donc des gens nécessaires, monsieur. Ce n’est pas tout pourtant ; vous allez bien voir autre chose. L’action ne fait pas le crime, c’est la connaissance de celui qui la commet, celui qui fait un mal, tandis qu’il peut croire que ce n’en est pas un, est en sûreté de conscience ; et comme il y a un nombre infini d’actions équivoques, un casuiste peut leur donner un degré de bonté qu’elles n’ont point, en les qualifiant telles ; et pourvu qu’il puisse persuader qu’elles n’ont pas de venin, il le leur ôte tout entier.\par
Je vous dis ici le secret d’un métier où j’ai vieilli ; je vous en fais voir les raffinements : il y a un tour à donner à tout, même aux choses qui en paraissent le moins susceptibles. Mon père, lui dis-je, cela est fort bon ; mais comment vous accommodez-vous avec le ciel ? Si le grand sophi avait à sa cour un homme qui fît à son égard ce que vous faites contre votre Dieu, qui mît de la différence entre ses ordres, et qui apprît à ses sujets dans quel cas ils doivent les exécuter, et dans quel autre ils peuvent les violer, il le ferait empaler sur l’heure. Là-dessus, je saluai mon dervis, et le quittai sans attendre sa réponse.\par

\dateline{À Paris, le 23 de la lune de Maharram, 1714.}
\section[{Lettre LVIII. Rica à Rhédi. À Venise.}]{Lettre LVIII. Rica à Rhédi. À Venise.}\renewcommand{\leftmark}{Lettre LVIII. Rica à Rhédi. À Venise.}

\noindent À Paris, mon cher Rhédi, il y a bien des métiers. Là un homme obligeant vient, pour un peu d’argent, vous offrir le secret de faire de l’or.\par
Un autre vous promet de vous faire coucher avec les esprits aériens, pourvu que vous soyez seulement trente ans sans voir de femmes.\par
Vous trouverez ensuite des devins si habiles, qu’ils vous diront toute votre vie, pourvu qu’ils aient eu un quart d’heure de conversation avec vos domestiques.\par
Des femmes adroites font de la virginité une fleur qui périt et renaît tous les jours, et se cueille la centième fois plus douloureusement que la première.\par
Il y en a d’autres qui, réparant par la force de leur art toutes les injures du temps, savent rétablir sur un visage une beauté qui chancelle, et même rappeler une femme du sommet de la vieillesse pour la faire redescendre jusqu’à la jeunesse la plus tendre.\par
Tous ces gens-là vivent ou cherchent à vivre dans une ville qui est la mère de l’invention.\par
Les revenus des citoyens ne s’y afferment point : ils ne consistent qu’en esprit et en industrie ; chacun a la sienne, qu’il fait valoir de son mieux.\par
Qui voudrait nombrer tous les gens de loi qui poursuivent le revenu de quelque mosquée, aurait aussitôt compté les sables de la mer, et les esclaves de notre monarque.\par
Un nombre infini de maître de langues, d’arts et de sciences, enseignent ce qu’ils ne savent pas ; et ce talent est bien considérable : car il ne faut pas beaucoup d’esprit pour montrer ce qu’on sait ; mais il en faut infiniment pour enseigner ce qu’on ignore.\par
 On ne peut mourir ici que subitement ; la mort ne saurait autrement exercer son empire : car il y a dans tous les coins des gens qui ont des remèdes infaillibles contre toutes les maladies imaginables.\par
Toutes les boutiques sont tendues de filets invisibles où se vont prendre tous les acheteurs. L’on en sort pourtant quelquefois à bon marché : une jeune marchande cajole un homme une heure entière, pour lui faire acheter un paquet de cure-dents.\par
Il n’y a personne qui sorte de cette ville plus précautionné qu’il n’y est entré : à force de faire part de son bien aux autres, on apprend à le conserver ; seul avantage des étrangers dans cette ville enchanteresse.\par

\dateline{De Paris, le 10 de la lune de Saphar, 1714.}
\section[{Lettre LIX. Rica à Usbek. À ***.}]{Lettre LIX. Rica à Usbek. À ***.}\renewcommand{\leftmark}{Lettre LIX. Rica à Usbek. À ***.}

\noindent J’étais l’autre jour dans une maison où il y avait un cercle de gens de toute espèce : je trouvai la conversation occupée par deux vieilles femmes, qui avaient en vain travaillé tout le matin à se rajeunir. Il faut avouer, disait une d’entre elles, que les hommes d’aujourd’hui sont bien différents de ceux que nous voyions dans notre jeunesse : ils étaient polis, gracieux, complaisants ; mais à présent je les trouve d’une brutalité insupportable. Tout est changé, dit pour lors un homme qui paraissait accablé de goutte, le temps n’est plus comme il était : il y a quarante ans, tout le monde se portait bien, on marchait, on était gai, on ne demandait qu’à rire et à danser ; à présent tout le monde est d’une tristesse insupportable. Un moment après, la conversation tourna du côté de la politique. Morbleu ! dit un vieux seigneur, l’État n’est plus gouverné, trouvez-moi à présent un ministre comme Monsieur Colbert. Je le connaissais beaucoup ce Monsieur Colbert ; il était de mes amis, il me faisait toujours payer de mes pensions avant qui que ce fût : le bel ordre qu’il y avait dans les finances ! tout le monde était à son aise ; mais aujourd’hui je suis ruiné. Monsieur, dit pour lors un ecclésiastique, vous parlez là du temps le plus miraculeux de notre invincible monarque ; y a-t-il rien de si grand que ce qu’il faisait alors pour détruire l’hérésie ? Et comptez-vous pour rien l’abolition des duels ? dit d’un air content un autre homme qui n’avait point encore parlé. La remarque est judicieuse, me dit quelqu’un à l’oreille : cet homme est charmé de l’édit, et il l’observe si bien, qu’il y a six mois il reçut cent coups de bâton pour ne le pas violer.\par
Il me semble, Usbek, que nous ne jugeons jamais des choses que par un retour secret que nous faisons sur nous-mêmes. Je ne suis pas surpris que les Nègres peignent le diable d’une blancheur éblouissante, et leurs dieux noirs comme du charbon ; que la Vénus de certains peuples ait des mamelles qui lui pendent jusqu’aux cuisses ; et qu’enfin tous les idolâtres aient représenté leurs dieux avec une figure humaine, et leur aient fait part de toutes leurs inclinations. On a dit fort bien que si les triangles faisaient un dieu, ils lui donneraient trois côtés.\par
Mon cher Usbek, quand je vois des hommes qui rampent sur un atome, c’est-à-dire la terre, qui n’est qu’un point de l’univers, se proposer directement pour modèles de la Providence, je ne sais comment accorder tant d’extravagance avec tant de petitesse.\par

\dateline{De Paris, le 14 de la lune de Saphar, 1714.}
\section[{Lettre LX. Usbek à Ibben. À Smyrne.}]{Lettre LX. Usbek à Ibben. À Smyrne.}\renewcommand{\leftmark}{Lettre LX. Usbek à Ibben. À Smyrne.}

\noindent Tu me demandes s’il y a des Juifs en France ? Sache que, partout où il y a de l’argent, il y a des Juifs. Tu me demandes ce qu’ils y font ? Précisément ce qu’ils font en Perse : rien ne ressemble plus à un Juif d’Asie qu’un Juif européen.\par
Ils font paraître chez les chrétiens ; comme parmi nous, une obstination invincible pour leur religion qui va jusqu’à la folie.\par
La religion juive est un vieux tronc qui a produit deux branches qui ont couvert toute la terre, je veux dire le mahométisme et le christianisme ; ou plutôt c’est une mère qui a engendré deux filles qui l’ont accablée de mille plaies : car, en fait de religion, les plus proches sont les plus grandes ennemies. Mais, quelques mauvais traitements qu’elle en ait reçus, elle ne laisse pas de se glorifier de les avoir mises au monde ; elle se sert de l’une et de l’autre pour embrasser le monde entier, tandis que d’un autre côté sa vieillesse vénérable embrasse tous les temps.\par
Les Juifs se regardent donc comme source de toute sainteté et l’origine de toute religion ; ils nous regardent au contraire comme des hérétiques qui ont changé la loi, ou plutôt comme des Juifs rebelles.\par
Si le changement s’était fait insensiblement, ils croient qu’ils auraient été facilement séduits : mais comme il s’est fait tout à coup et d’une manière violente, comme ils peuvent marquer le jour et l’heure de l’une et de l’autre naissance, ils se scandalisent de trouver en nous des âges, et se tiennent fermes à une religion que le monde même n’a pas précédée.\par
Ils n’ont jamais eu dans l’Europe un calme pareil à celui dont ils jouissent. On commence à se défaire parmi les chrétiens de cet esprit d’intolérance qui les animait : on s’est mal trouvé en Espagne de les avoir chassés, et en France d’avoir fatigué des chrétiens dont la croyance différait un peu de celle du prince. On s’est aperçu que le zèle pour les progrès de la religion est différent de l’attachement qu’on doit avoir pour elle ; et que, pour l’aimer et pour l’observer, il n’est pas nécessaire de haïr et de persécuter ceux qui ne l’observent pas.\par
Il serait à souhaiter que nos musulmans pensassent aussi sensément sur cet article que les chrétiens ; que l’on pût une bonne fois faire la paix entre Ali et Abubeker, et laisser à Dieu le soin de décider des mérites de ces saints prophètes : je voudrais qu’on les honorât par des actes de vénération et de respect, et non pas par de vaines préférences ; et qu’on cherchât à mériter leur faveur, quelque place que Dieu leur ait marquée, soit à sa droite, ou bien sous le marchepied de son trône.\par

\dateline{À Paris, le 18 de la lune de Saphar, 1714.}
\section[{Lettre LXI. Usbek à Rhédi. À Venise.}]{Lettre LXI. Usbek à Rhédi. À Venise.}\renewcommand{\leftmark}{Lettre LXI. Usbek à Rhédi. À Venise.}

\noindent J’entrai l’autre jour dans une église fameuse qu’on appelle Notre-Dame : pendant que j’admirais ce superbe édifice, j’eus occasion de m’entretenir avec un ecclésiastique que la curiosité y avait attiré comme moi. La conversation tomba sur la tranquillité de sa profession. La plupart des gens, me dit-il, envient le bonheur de notre état, et ils ont raison : cependant il a ses désagréments ; nous ne sommes point si séparés du monde, que nous n’y soyons appelés en mille occasions : là, nous avons un rôle très difficile à soutenir.\par
Les gens du monde sont étonnants ; ils ne peuvent souffrir notre approbation, ni nos censures ; si nous les voulons corriger, ils nous trouvent ridicules ; si nous les approuvons, ils nous regardent comme des gens au-dessous de notre caractère. Il n’y a rien de si humiliant de penser qu’on a scandalisé les impies mêmes : nous sommes donc obligés de tenir une conduite équivoque, et d’imposer aux libertins, non pas par un caractère décidé, mais par l’incertitude où nous les mettons de la manière dont nous recevons leurs discours. Il faut avoir beaucoup d’esprit pour cela ; cet état de neutralité est difficile : les gens du monde, qui hasardent tout, qui se livrent à toutes leurs saillies, qui, selon le succès, les poussent ou les abandonnent, réussissent bien mieux.\par
Ce n’est pas tout : cet état si heureux et si tranquille, que l’on vante tant, nous ne le conservons pas dans le monde. Dès que nous y paraissons, on nous fait disputer ; on nous fait entreprendre, par exemple, de prouver l’utilité de la prière à un homme qui ne croit pas en Dieu, la nécessité du jeûne à un autre qui a nié toute sa vie l’immortalité de l’âme : l’entreprise est laborieuse, et les rieurs ne sont pas pour nous. Il y a plus : une certaine envie d’attirer les autres dans nos opinions nous tourmente sans cesse, et est pour ainsi dire attachée à notre profession. Cela est aussi ridicule que si on voyait les Européens travailler, en faveur de la nature humaine, à blanchir le visage des Africains. Nous troublons l’État, nous nous tourmentons nous-mêmes, pour faire recevoir des points de religion qui ne sont point fondamentaux ; et nous ressemblons à ce conquérant de la Chine, qui poussa ses sujets à une révolte générale pour les avoir voulu obliger à se rogner les cheveux ou les ongles.\par
Le zèle même que nous avons pour faire remplir à ceux dont nous sommes chargés les devoirs de notre sainte religion est souvent dangereux, et ne saurait être accompagné de trop de prudence. Un empereur nommé Théodose fit passer au fil de l’épée tous les habitants d’une ville, même les femmes et les petits enfants : s’étant ensuite présenté pour entrer dans une église, un évêque nommé Ambroise lui fit fermer les portes, comme à un meurtrier et un sacrilège ; et en cela il fit une action héroïque. Cet empereur ayant ensuite fait la pénitence qu’un tel crime exigeait, ayant été admis dans l’église s’alla placer parmi les prêtres ; le même évêque l’en fit sortir ; et en cela il commit l’action d’un fanatique et d’un fou : tant il est vrai que l’on doit se défier de son zèle. Qu’importait à la religion ou à l’État que ce prince eût, ou n’eût pas, une place parmi les prêtres ?\par

\dateline{De Paris, le 1er de la lune de Rebiab, 1714.}
\section[{Lettre LXII. Zélis à Usbek. À Paris.}]{Lettre LXII. Zélis à Usbek. À Paris.}\renewcommand{\leftmark}{Lettre LXII. Zélis à Usbek. À Paris.}

\noindent Ta fille ayant atteint sa septième année, j’ai cru qu’il était temps de la faire passer dans les appartements intérieurs du sérail, et de ne point attendre qu’elle ait dix ans pour la confier aux eunuques noirs. On ne saurait de trop bonne heure priver une jeune personne des libertés de l’enfance, et lui donner une éducation sainte dans les sacrés murs où la pudeur habite.\par
Car je ne puis être de l’avis de ces mères qui ne renferment leurs filles que lorsqu’elles sont sur le point de leur donner un époux ; qui, les condamnant au sérail plutôt qu’elles ne les y consacrent, leur font embrasser violemment une manière de vie qu’elles auraient dû leur inspirer. Faut-il tout attendre de la force de la raison, et rien de la douceur de l’habitude ?\par
C’est en vain que l’on nous parle de la subordination où la nature nous a mises : ce n’est pas assez de nous la faire sentir ; il faut nous la faire pratiquer, afin qu’elle nous soutienne dans ce temps critique où les passions commencent à naître, et à nous encourager à l’indépendance.\par
Si nous étions attachées à vous que par le devoir, nous pourrions quelques fois l’oublier ; si nous n’y étions entraînées que par le penchant, peut-être un penchant plus fort pourrait l’affaiblir. Mais quand les lois nous donnent à un homme, elles nous dérobent à tous les autres, et nous mettent aussi loin d’eux que si nous en étions à cent mille lieux.\par
La nature, industrieuse en faveur des hommes, ne s’est pas bornée à leur donner des désirs ; elle a voulu que nous en eussions nous-mêmes, et que nous fussions des instruments animés de leur félicité : elle nous a mises dans le feu des passions, pour les faire vivre tranquilles ; s’ils sortent de leur insensibilité, elle nous a destinées à les y faire rentrer, sans que nous puissions jamais goûter cet heureux état où nous les mettons.\par
Cependant, Usbek, ne t’imagine pas que ta situation soit plus heureuse que la mienne : j’ai goûté ici mille plaisirs que tu ne connais pas : notre imagination a travaillé sans cesse à m’en faire connaître le prix : j’ai vécu, et tu n’as fait que languir.\par
Dans la prison même où tu me retiens, je suis plus libre que toi : tu ne saurais redoubler tes attentions pour me faire garder, que je ne jouisse de tes inquiétudes ; et tes soupçons, ta jalousie, tes chagrins, sont autant de marques de ta dépendance.\par
Continue, cher Usbek : fais veiller sur moi nuit et jour ; ne te fie pas même aux précautions ordinaires ; augmente mon bonheur en assurant le tien ; et sache que je ne redoute rien, que ton indifférence.\par

\dateline{Du sérail d’Ispahan, le 2 de la lune de Rebiab, 1714.}
\section[{Lettre LXIII. Rica à Usbek. À ***.}]{Lettre LXIII. Rica à Usbek. À ***.}\renewcommand{\leftmark}{Lettre LXIII. Rica à Usbek. À ***.}

\noindent Je crois que tu veux passer ta vie à la campagne. Je ne te perdais au commencement que pour deux ou trois jours ; et en voilà quinze que je ne t’ai vu : il est vrai que tu es dans une maison charmante, que tu y trouves une société qui te convient, que tu y raisonnes tout à ton aise : il n’en faut pas davantage pour te faire oublier tout l’univers.\par
Pour moi je mène à peu près la même vie que tu m’as vu mener ; je me répands dans le monde, et je cherche à le connaître : mon esprit perd insensiblement tout ce qui lui reste d’asiatique, et se plie sans effort aux mœurs européennes. Je ne suis plus si étonné de voir dans une maison cinq ou six femmes avec cinq ou six hommes : et je trouve que cela n’est pas mal imaginé.\par
Je le puis dire, je ne connais les femmes que depuis que je suis ici ; j’en ai plus appris dans un mois que je n’aurais fait en trente ans dans un sérail.\par
Chez nous les caractères sont tous uniformes, parce qu’ils sont forcés : on ne voit pas les gens tels qu’ils sont, mais tels qu’on les oblige d’être ; dans cette servitude du cœur et de l’esprit on entend parler que la crainte, qui n’a qu’un langage, et non pas la nature, qui s’exprime si différemment, et qui paraît sous tant de formes.\par
La dissimulation, cet art parmi nous si pratiqué et si nécessaire, est ici inconnue : tout parle, tout se voit, tout s’entend ; le cœur se montre comme le visage ; dans les mœurs, dans la vertu, dans le vice même, on aperçoit toujours quelque chose de naïf.\par
Il faut, pour plaire aux femmes, un certain talent différent de celui qui lui plaît encore davantage : il consiste dans une espèce de badinage dans l’esprit, qui les amuse en ce qu’il semble leur promettre à chaque instant ce qu’on ne peut tenir que dans de trop longs intervalles.\par
Ce badinage, naturellement fait pour les toilettes, semble être venu à former le caractère général de la nation : on badine au conseil, on badine à la tête d’une armée, on badine avec un ambassadeur ; les professions ne paraissent ridicules qu’à proportion du sérieux qu’on y met : un médecin ne le serait plus, si ses habits étaient moins lugubres, et s’il tuait ses malades en badinant.\par

\dateline{À Paris, le 10 de la lune de Rebiab, 1714.}
\section[{Lettre LXIV. Le chef des eunuques noirs à Usbek. À Paris.}]{Lettre LXIV. Le chef des eunuques noirs à Usbek. À Paris.}\renewcommand{\leftmark}{Lettre LXIV. Le chef des eunuques noirs à Usbek. À Paris.}

\noindent Je suis dans un embarras que je ne saurais t’exprimer, magnifique seigneur : le sérail est dans un désordre et une confusion épouvantable ; la guerre règne entre tes femmes ; tes eunuques sont partagés ; on n’entend que plaintes, que murmures, que reproches ; mes remontrances sont méprisées ; tout semble permis dans ce temps de licence, et je n’ai plus qu’un vain titre dans le sérail.\par
Il n’y a aucune de tes femmes qui ne se juge au-dessus des autres par sa naissance, par sa beauté, par ses richesses, par son esprit, par ton amour ; et qui ne fasse valoir quelques-uns de ces titres-là pour avoir toutes les préférences : je perds à chaque instant cette longue patience, avec laquelle néanmoins j’ai eu le malheur de les mécontenter toutes ; ma prudence, ma complaisance même, vertu si rare et si étrangère dans le poste que j’occupe, ont été inutiles.\par
Veux-tu que je te découvre, magnifique seigneur, la cause de tous ces désordres ? Elle est toute dans ton cœur, et dans les tendres égards que tu as pour elles. Si tu ne me retenais par la main ; si, au lieu de la voie des remontrances, tu me laissais celle des châtiments ; si, sans te laisser attendrir à leurs plaintes et à leurs larmes, tu les envoyais pleurer devant moi, qui ne m’attendris jamais, je les façonnerais bientôt au joug qu’elles doivent porter, et je laisserais leur humeur impérieuse et indépendante.\par
Enlevé dès l’âge de quinze ans du fond de l’Afrique, ma patrie, je fus d’abord vendu à un maître qui avait plus de vingt femmes, ou concubines. Ayant jugé à mon air grave et taciturne que j’étais propre au sérail, il ordonna qu’on achevât de me rendre tel ; et me fit faire une opération pénible dans les commencements, mais qui me fut heureuse dans la suite, parce qu’elle m’approcha de l’oreille et de la confiance de mes maîtres. J’entrai dans ce sérail, qui fut pour moi un nouveau monde. Le premier eunuque, l’homme le plus sévère que j’ai vu de ma vie, y gouvernait avec un empire absolu. On y entendait parler ni de divisions, ni de querelles : un silence profond régnait partout ; toutes ces femmes étaient couchées à la même heure d’un bout de l’année à l’autre, et levées à la même heure ; elles entraient dans le bain tour à tour, elles en sortaient au moindre signe que nous leur en faisions ; le reste du temps elles étaient presque toujours enfermées dans leurs chambres. Il avait une règle, qui était de les faire tenir dans une grande propreté, et il avait pour cela des attentions inexprimables : le moindre refus d’obéir était puni sans miséricorde. Je suis, disait-il, esclave ; mais je le suis d’un homme qui est votre maître, et le mien ; et j’use du pouvoir qu’il m’a donné sur vous : c’est lui qui vous châtie, et non moi qui ne fait que prêter ma main. Ces femmes n’entraient jamais dans la chambre de mon maître qu’elles n’y fussent appelées ; elles recevaient cette grâce avec joie, et s’en voyaient privées sans se plaindre. Enfin moi, qui étais le dernier des noirs dans ce sérail tranquille, j’étais mille fois plus respecté que je ne le suis dans le tien, où je les commande tous.\par
Dès que ce grand eunuque eut connu mon grand génie, il tourna les yeux de mon côté ; il parla de moi à mon maître, comme un homme capable de travailler selon ses vues, et de lui succéder dans le poste qu’il remplissait ; il ne fut point étonné de ma grande jeunesse, il crut que mon attention me tiendrait lieu d’expérience. Que te dirai-je ? Je fis tant de progrès dans ma confiance, qu’il ne faisait plus de difficulté de me confier les clefs des lieux terribles qu’il gardait depuis si longtemps. C’est sous ce grand maître que j’appris l’art de commander, et que je me formais aux maximes d’un gouvernement inflexible : j’étudiai sous lui le cœur des femmes ; il m’apprit à profiter de leurs faiblesses et à ne point m’étonner de leurs hauteurs. Souvent il me plaisait de me les faire exercer même, et de les conduire jusqu’au dernier retranchement de l’obéissance ; il les faisait ensuite revenir insensiblement, et voulu que je parusse pour quelques temps plier moi-même. Mais il fallait le voir dans ces moments, où il les trouvait tout près du désespoir, entre les prières et les reproches : il soutenait leurs larmes sans s’émouvoir. Voilà, disait-il d’un air content, comment il faut gouverner les femmes : leur nombre ne m’embarrasse pas ; je conduirais de même toutes celles de notre grand monarque. Comment un homme peut-il espérer de captiver leur cœur, si ses fidèles eunuques n’ont commencé par soumettre leur esprit ?\par
Il avait non seulement de la fermeté, mais aussi de la pénétration : il lisait leurs pensées et leurs dissimulations ; leurs gestes étudiés, leur visage feint ne lui dérobaient rien ; il savait toutes leurs actions les plus cachées et leurs paroles les plus secrètes ; il se servait des unes pour connaître les autres, et il se plaisait à récompenser la moindre confidence. Comme elles n’abordaient leur mari que lorsqu’elles étaient averties, l’eunuque y appelait qui il voulait, et tournait les yeux de son maître sur celles qu’il avait en vue ; et cette distinction était la récompense de quelque secret révélé : il avait persuadé à son maître qu’il était du bon ordre qu’il lui laissât ce choix, afin de lui donner une autorité plus grande. Voilà comme on gouvernait, magnifique seigneur, dans un sérail qui était, je crois, le mieux réglé qu’il n’y eût en Perse.\par
Laisse-moi les mains libres : permets que je me fasse obéir ; huit jours remettront l’ordre dans le sein de la confusion ; c’est ce que ta gloire demande, et que ta sûreté exige.\par

\bibl{De ton sérail d’Ispahan, le 9 de la lune de Rebiab I, 1714.}
\section[{Lettre LXV. Usbek à ses femmes. Au sérail d’Ispahan.}]{Lettre LXV. Usbek à ses femmes. Au sérail d’Ispahan.}\renewcommand{\leftmark}{Lettre LXV. Usbek à ses femmes. Au sérail d’Ispahan.}

\noindent J’apprends que le sérail est dans le désordre, et qu’il est rempli de querelles et de divisions intestines. Que vous recommandai-je en partant, que la paix et la bonne intelligence ? Vous me le promîtes ; était-ce pour me tromper ?\par
C’est vous qui seriez trompées, si je voulais suivre les conseils que me donne le grand eunuque, si je voulais employer mon autorité pour vous faire vivre comme mes exhortations le demandaient de vous.\par
Je ne sais me servir de ces moyens violents que lorsque j’ai tenté tous les autres : faites donc en votre considération ce que vous n’avez pas voulu faire à la mienne.\par
Le premier eunuque a grand sujet de se plaindre : il dit que vous n’avez aucun égard pour lui. Comment pouvez-vous accorder cette conduite avec la modestie de votre état ? N’est-ce pas à lui que, pendant mon absence, votre vertu est confiée ? C’est un trésor sacré, dont il est le dépositaire. Mais ces mépris que vous lui témoignez sont une marque que ceux qui sont chargés de vous faire vivre dans les lois de l’honneur vous sont à charge.\par
Changez donc de conduite, je vous prie ; et faites en sorte que je puisse une autre fois rejeter les propositions que l’on me fait contre votre liberté et votre repos.\par
Car je voudrais vous faire oublier que je suis votre maître, pour me souvenir seulement que je suis votre époux.\par

\dateline{À Paris, le 5 de la lune de Chahban, 1714.}
\section[{Lettre LXVI. Rica à ***.}]{Lettre LXVI. Rica à ***.}\renewcommand{\leftmark}{Lettre LXVI. Rica à ***.}

\noindent On s’attache ici beaucoup aux sciences, mais je ne sais si on est fort savant. Celui qui doute de tout comme un philosophe n’ose rien nier comme théologien ; cet homme contradictoire est toujours content de lui, pourvu qu’on convienne des qualités.\par
La fureur de la plupart des Français, c’est d’avoir de l’esprit ; et la fureur de ceux qui veulent avoir de l’esprit, c’est de faire des livres.\par
Cependant il n’y a rien de si mal imaginé : la nature semblait avoir sagement pourvu à ce que les sottises des hommes fussent passagères, et les livres les immortalisent. Un sot devrait être content d’avoir ennuyé tous ceux qui ont vécu avec lui : il veut encore tourmenter les races futures ; il veut que sa sottise triomphe de l’oubli dont il aurait pu jouir comme du tombeau ; il veut que la postérité soit informée qu’il a vécu, et qu’elle sache à jamais qu’il a été un sot.\par
De tous les auteurs, il n’y en a point que je méprise plus que les compilateurs, qui vont de tous côtés, chercher des lambeaux des ouvrages des autres, qu’ils plaquent dans les leurs, comme des pièces de gazon dans un parterre : ils ne sont point au-dessus de ces ouvriers d’imprimerie qui rangent des caractères, qui, combinés ensemble, font un livre où ils n’ont fourni que la main. Je voudrais qu’on respectât les livres originaux ; et il me semble que c’est une espèce de profanation de tirer les pièces qui les composent du sanctuaire où elles sont, pour les exposer à un mépris qu’elles ne méritent point.\par
Quand un homme n’a rien à dire de nouveau, que ne se tait-il ? Qu’a-t-on affaire de ces doubles emplois ? Mais je veux donner un nouvel ordre. Vous êtes un habile homme : c’est-à-dire que vous venez dans ma bibliothèque et vous mettez en bas les livres qui sont en haut, et en haut ceux qui sont en bas : vous avez fait un chef-d’œuvre.\par
Je t’écris sur ce sujet, ***, parce que je suis outré d’un livre que je viens de quitter, qui est si gros qu’il semblait contenir la science universelle ; mais il m’a rompu la tête sans m’avoir rien appris. Adieu.\par

\dateline{À Paris, le 8 de la lune de Chahban 1714.}
\section[{Lettre LXVII. Ibben à Usbek. À Paris.}]{Lettre LXVII. Ibben à Usbek. À Paris.}\renewcommand{\leftmark}{Lettre LXVII. Ibben à Usbek. À Paris.}

\noindent Trois vaisseaux sont arrivés ici sans m’avoir apporté aucune de tes nouvelles. Es-tu malade ? ou te plais-tu à m’inquiéter ?\par
Si tu ne m’aimes pas dans un pays où tu n’es lié à rien, que sera-ce au milieu de la Perse, et dans le sein de ta famille ? Mais peut-être que je me trompe : tu es assez aimable pour trouver partout des amis ; le cœur est citoyen de tous les pays : comment une âme bien faite peut-elle s’empêcher de former des engagements ? Je te l’avoue, je respecte les anciennes amitiés ; mais je ne suis pas fâché d’en faire partout de nouvelles.\par
En quelque pays que j’aie été, j’y ai vécu comme si j’avais dû y passer ma vie ; j’ai eu le même empressement pour les gens vertueux, la même compassion ou plutôt la même tendresse pour les malheureux, la même estime pour ceux que la prospérité n’a point aveuglés. C’est mon caractère, Usbek ; partout où je trouverai des hommes, je me choisirai des amis.\par
Il y a ici un Guèbre qui, après toi, a, je crois, la première place dans mon cœur : c’est l’âme de la probité même. Des raisons particulières l’ont obligé de se retirer dans cette ville, où il vit tranquille du produit d’un trafic honnête avec une femme qu’il aime. Sa vie est toute marquée d’actions généreuses ; et, quoiqu’il cherche la vie obscure, il y a plus d’héroïsme dans son cœur que dans celui des plus grands monarques.\par
Je lui ai parlé mille fois de toi, je lui montre toutes tes lettres ; je remarque que cela lui fait plaisir, et je vois déjà que tu as un ami qui t’est inconnu.\par
Tu trouveras ici ses principales aventures ; quelques répugnances qu’il eût à les écrire, il n’a pu les refuser à mon amitié, et je les confie à la tienne.\par

\begin{center}
\noindent \centerline{HISTOIRE D’APHÉRIDON ET D’ASTARTE.}\par
\end{center}

\noindent Je suis né parmi les Guèbres, d’une religion qui est peut-être la plus ancienne qui soit au monde. Je fus si malheureux que l’amour me vint avant la raison. J’avais à peine six ans, que je ne pouvais vivre qu’avec ma sœur ; mes yeux s’attachaient toujours sur elle ; et lorsqu’elle me quittait un moment, elle les retrouvait baignés de larmes : chaque jour n’augmentait pas plus mon âge que mon amour. Mon père, étonné d’une si forte sympathie, aurait bien souhaité de nous marier ensemble, selon l’ancien usage des Guèbres introduit par Cambyse ; mais la crainte des mahométans, sous le joug desquels nous vivons, empêche ceux de notre nation de penser à ces alliances saintes, que notre religion ordonne plutôt qu’elle ne permet, et qui sont des images si naïves de l’union déjà formée par la nature.\par
Mon père, voyant donc qu’il aurait été dangereux de suivre mon inclination et la sienne, résolut d’éteindre une flamme qu’il croyait naissante, mais qui était déjà à son dernier période ; il prétexta un voyage et m’emmena avec lui, laissant ma sœur entre les mains d’une de ses parentes ; car ma mère était morte depuis deux ans. Je ne vous dirai point quel fut le désespoir de cette séparation : j’embrassai ma sœur toute baignée de larmes ; mais je n’en versai point, car la douleur m’avait rendu comme insensible. Nous arrivâmes à Tiflis ; et mon père, ayant confié mon éducation à un de nos parents, m’y laissa et s’en retourna chez lui.\par
Quelques temps après j’appris qu’il avait, par le crédit d’un de ses amis, fait entrer ma sœur dans le beïram du roi, où elle était au service d’une sultane. Si l’on m’avait appris sa mort, je n’en aurais pas été plus frappé : car, outre que je n’espérais plus de la revoir, son entrée dans le beïram l’avait rendu mahométane ; et elle ne pouvait plus, suivant le préjugé de cette religion, me regarder qu’avec horreur. Cependant, ne pouvant plus vivre à Tefflis, las de moi-même et de la vie, je retournai à Ispahan. Mes premières paroles furent amères à mon père ; je lui reprochai d’avoir mis sa fille en un lieu où l’on ne peut entrer qu’en changeant de religion. Vous avez attiré sur votre famille, lui dis-je, la colère de Dieu et du Soleil qui vous éclaire ; vous avez plus fait que si vous aviez souillé les Éléments, puisque vous avez souillé l’âme de votre fille, qui n’est pas moins pure : j’en mourrai de douleur et d’amour ; mais puisse ma mort être la seule peine que Dieu vous fasse sentir ! À ces mots, je sortis ; et pendant deux ans je passai ma vie à aller regarder les murailles du beïram, et considérer le lieu où ma sœur pouvait être, m’exposant tous les jours mille fois à être égorgé par les eunuques qui font la ronde autour de ces redoutables lieux.\par
Enfin mon père mourut ; et la sultane que ma sœur servait, la voyant tous les jours croître en beauté, en devint jalouse, et la maria à un eunuque qui la souhaitait avec passion. Par ce moyen, ma sœur sortit du sérail, et prit avec son eunuque une maison à Ispahan.\par
Je fus plus de trois mois sans pouvoir lui parler ; l’eunuque, le plus jaloux de tous les hommes, me remettant toujours, sous divers prétextes. Enfin j’entrai dans son beïram, et il me lui fit parler au travers d’une jalousie : des yeux de lynx ne l’auraient pas pu découvrir, tant elle était enveloppée d’habits et de voiles ; et je ne la pus reconnaître qu’au son de sa voix. Quelle fut mon émotion quand je me vis si près et si éloigné d’elle ! Je me contraignis, car j’étais examiné. Quant à elle, il me parut qu’elle versa quelques larmes. Son mari voulut me faire quelques mauvaises excuses ; mais je le traitai comme le dernier des esclaves. Il fut bien embarrassé quand il vit que je parlai à ma sœur une langue qui lui était inconnu ; c’était l’ancien persan, qui est notre langue sacrée. Quoi ! ma sœur, lui dis-je, est-il vrai que vous avez quitté la religion de vos pères ? Je sais qu’en entrant au beïram, vous avez dû faire profession du mahométisme ; mais, dites-moi, votre cœur a-t-il pu consentir, comme votre bouche, à quitter une religion qui me permet de vous aimer ? Et pour qui la quittez-vous, cette religion, qui doit nous être si cher ? Pour un misérable encore flétri de fers qu’il a portés ; qui, s’il était homme, serait le dernier de tous ! Mon frère, dit-elle, cet homme dont vous parlez est mon mari ; il faut que je l’honore, tout indigne qu’il vous paraît ; et je serais aussi la dernière des femmes si… Ah ! Ma sœur, lui dis-je, vous êtes guèbre ; il n’est ni votre époux, ni ne peut l’être : si vous êtes fidèle comme vos pères, vous ne devez le regarder que comme un monstre. Hélas ! dit-elle, que cette religion se montre à moi de loin ! À peine en savais-je les préceptes, qu’il fallut oublier. Vous voyez que cette langue que je vous parle ne m’est plus familière, et que j’ai toutes les peines du monde à m’exprimer : mais comptez que le souvenir de notre enfance me charme toujours ; que depuis ce temps-là, je n’ai eu que de fausses joies ; qu’il ne s’est pas passé de jour que je n’aie pas pensez à vous ; que vous avez eu plus de part que vous ne croyez à mon mariage, et que je n’y ai été déterminée que par l’espérance de vous revoir. Mais que ce jour qui m’a tant coûté va me coûter encore. Je vous vois tout hors de vous-même : mon mari frémit de rage et de jalousie : je ne vous verrai plus ; je vous parle sans doute pour la dernière fois de ma vie : si cela était, mon frère, elle ne serait pas longue. À ces mots, elle s’attendrit ; et, se voyant hors d’état de tenir la conversation, elle me quitta le plus désolé des hommes.\par
Trois ou quatre jours après je demandai à voir ma sœur : le barbare eunuque aurait bien voulu m’en empêcher ; mais, outre que ces sortes de maris n’ont pas sur leurs femmes la même autorité que les autres, il aimait si éperdument ma sœur, qu’il ne savait lui rien refuser. Je la vis encore dans le même lieu et dans le même équipage, accompagnée de deux esclaves ; ce qui me fit avoir recours à notre langue particulière. Ma sœur, lui dis-je, d’où vient que je ne puis vous voir sans me trouver dans une situation affreuse ? Les murailles qui vous tiennent enfermée, ces verrous et ces grilles, ces misérables gardiens qui vous observent, me mettent en fureur. Comment avez-vous perdu la douce liberté dont jouissaient nos ancêtres ? Votre mère, qui était si chaste, ne donnait à son mari, pour garant de vertu, que sa vertu même : ils vivaient heureux l’un et l’autre dans une confiance mutuelle ; et la simplicité de leurs mœurs était pour eux une richesse plus précieuse mille fois que le faux éclat dont vous semblez jouir dans cette maison somptueuse. En perdant votre religion, vous avez perdu votre liberté, votre bonheur, et cette précieuse égalité qui fait l’honneur de votre sexe. Mais ce qu’il y a de pis encore, c’est que vous êtes, non pas la femme, car vous ne pouvez pas l’être ; mais l’esclave d’un esclave, qui a été dégradé de l’humanité. Ah ! mon frère, dit-elle, respectez mon époux, respectez la religion que j’ai embrassée : selon cette religion, je n’ai pu vous entendre ni vous parler sans crime. Quoi ! Ma sœur, lui dis-je tout transporté, vous la croyez donc véritable, cette religion ? Ah ! dit-elle, qu’il me serait avantageux qu’elle ne le fût pas ! Je fais pour elle un trop grand sacrifice, pour que je puisse ne pas la croire ; et si mes doutes… À ces mots elle se tut. Oui, vos doutes, ma sœur, sont bien fondés, quels qu’ils soient. Qu’attendez-vous d’une religion qui vous rend malheureuse dans ce monde-ci, et ne vous laisse point d’espérance pour l’autre ? Songez que la nôtre est la plus ancienne qui soit au monde ; qu’elle a toujours fleuri dans la Perse ; et n’a pas d’autre origine que cet empire, dont les commencements ne sont point connus ; que ce n’est que le hasard qui y a introduit le mahométisme ; que cette secte y a été établie, non par la voie de la persuasion, mais de la conquête. Si nos princes naturels n’avaient pas été faibles, vous verriez régner encore le culte de ces anciens mages. Transportez-vous dans ces siècles reculés : tout vous parlera du magisme, et rien de la secte mahométane, qui, plusieurs milliers d’années après, n’était pas même pas dans son enfance. Mais, dit-elle, quand ma religion serait plus moderne que la vôtre, elle est au moins plus pure, puisqu’elle n’adore que Dieu ; au lieu que vous adorez encore le Soleil, les Étoiles, le Feu, et même les Éléments. Je vois ma sœur, que vous avez apprise parmi les musulmans à calomnier notre sainte religion. Nous n’adorons ni les Astres ni les Éléments, et nos pères ne les ont jamais adorés : jamais ils ne leur ont élevé des temples, jamais ils ne leur ont offert des sacrifices ; ils leur ont seulement rendu un culte religieux, mais inférieur, comme à des ouvrages et des manifestations de la Divinité. Mais, ma sœur, au nom de Dieu qui nous éclaire, recevez ce livre sacré que je vous porte ; c’est le livre de notre législateur Zoroastre ; lisez-le sans prévention : recevez dans votre cœur les rayons de lumière qui vous éclaireront en le lisant ; souvenez-vous de vos pères, qui ont si longtemps honoré le Soleil dans la ville sainte de Balk ; et enfin souvenez-vous de moi, qui n’espère de repos, de fortune, de vie, que de votre changement. Je la quittai tout transporté, et la laissai seule décider la plus grande affaire que je pusse avoir de ma vie.\par
J’y retournai deux jours après : je ne lui parlai point : j’attendis dans le silence l’arrêt de ma vie ou de ma mort. Vous êtes aimé, mon frère, me dit-elle, et par une Guèbre. J’ai longtemps combattu : mais dieux ! Que l’amour lève de difficultés ! que je suis soulagée ! Je ne crains plus de vous trop aimer, je puis ne mettre point de bornes à mon amour ; l’excès même en est légitime. Ah ! Que ceci convient bien à l’état de mon cœur ! Mais vous, qui avez su rompre les chaînes que mon esprit s’était forgées, quand romprez-vous celles qui me lient les mains ? Dès ce moment je me donne à vous : faites voir, par la promptitude avec laquelle vous m’accepterez, combien ce présent vous est cher. Mon frère, la première fois que je pourrai vous embrasser, je crois que je mourrai dans vos bras. Je n’exprimerais jamais bien la joie que je sentis à ces douces paroles, je me crus et je me vis en effet, en un instant, le plus heureux de tous les hommes ; je vis presque accomplir tous les désirs que j’avais formés en vingt-cinq ans de vie, et évanouir tous les chagrins qui me l’avaient rendue si laborieuse. Mais, quand je me fus un peu accoutumé à ces douces idées, je trouvai que je n’étais pas si près de mon bonheur que je m’étais figuré tout à coup, quoique j’eusse surmonté le plus grand de tous les obstacles. Il fallait surprendre la vigilance des gardiens ; je n’osais confier à personne le secret de ma vie : il fallait que nous fissions tout, elle et moi : si je manquais mon coup, je courais risque d’être empalé ; mais je ne voyais pas de peine plus cruelle que de le manquer. Nous convînmes qu’elle m’enverrait demander une horloge, que son père lui avait laissée, et que j’y mettrais une lime, pour scier les jalousies de sa fenêtre qui donnaient dans la rue, et une corde nouée pour descendre ; que je ne la verrais plus dorénavant, mais que j’irais toutes les nuits sous cette fenêtre attendre qu’elle pût exécuter son dessein. Je passai quinze nuits entières sans voir personne, parce qu’elle n’avait pas trouvé le temps favorable. Enfin, la seizième, j’entendis une scie qui travaillait ; de temps en temps, l’ouvrage était interrompu, et dans ces intervalles ma frayeur était inexprimable. Enfin, après une heure de travail, je la vis qui attachait la corde ; elle se laissa aller, et glissa dans mes bras. Je ne connus plus le danger, et je restai longtemps sans bouger de là ; je la conduisis hors de la ville, où j’avais un cheval tout prêt ; je la mis en croupe derrière moi, et m’éloignai, avec toute la promptitude imaginable, d’un lieu qui pouvait nous être si funeste. Nous arrivâmes avant le jour chez un Guèbre, dans un lieu désert où il était retiré, vivant frugalement du travail de ses mains : nous ne jugeâmes pas à propos de rester chez lui ; et, par son conseil, nous entrâmes dans une épaisse forêt, et nous nous mîmes dans le creux d’un vieux chêne, jusqu’à ce que le bruit de notre évasion se fût dissipé. Nous vivions tous deux dans ce séjour écarté, sans témoins, nous répétant sans cesse que nous nous aimerions toujours, attendant l’occasion que quelque prêtre guèbre pût faire la cérémonie du mariage prescrite par nos livres sacrés. Ma sœur, lui dis-je, que cette union est sainte ! La nature nous avait unis, notre sainte loi va nous unir encore. Enfin un prêtre vint calmer notre impatience amoureuse. Il fit dans la maison du paysan toutes les cérémonies du mariage ; il nous bénit, et nous souhaita mille fois toute la vigueur de Gustaspe et la sainteté de l’Hohoraspe. Bientôt après nous quittâmes la Perse, où nous n’étions pas en sûreté, et nous nous retirâmes en Géorgie. Nous y vécûmes un an, tous les jours plus charmés l’un de l’autre : mais comme mon argent allait finir, et que je craignais la misère pour ma sœur, non pas pour moi, je la quittai pour aller chercher quelque secours chez nos parents. Jamais adieu ne fut plus tendre. Mais mon voyage me fut non seulement inutile, mais funeste : car, ayant trouvé d’un côté tous nos biens confisqués, de l’autre mes parents presque dans l’impuissance de me secourir, je ne rapportai d’argent précisément ce qu’il fallait pour mon retour. Mais quel fut mon désespoir ! Je ne trouvai plus ma sœur. Quelques jours avant mon arrivée, des Tartares avaient fait une excursion dans la ville où elle était ; et, comme ils la trouvèrent belle, ils la prirent, et la vendirent à des Juifs qui allaient en Turquie, et ne laissèrent qu’une petite fille dont elle était accouchée quelques mois auparavant. Je suivis ces Juifs, et je les joignis à trois lieues de là : mes prières, mes larmes furent vaines ; ils me demandèrent toujours trente tomans, et ne se relâchèrent jamais d’un seul. Après m’être adressé à tout le monde, avoir imploré la protection des prêtres trucs et chrétiens, je m’adressai à un marchand arménien ; je lui vendis ma fille, et me vendis aussi pour trente-cinq tomans ; et j’allai aux Juifs, je leur donnai trente tomans, et portai les cinq autres à ma sœur, que je n’avais pas encore vue. Vous êtes libre, lui dis-je, ma sœur, et je puis vous embrasser : voilà cinq tomans que je vous porte ; j’ai du regret qu’on ne m’ait pas acheté davantage. Quoi ! dit-elle, vous vous êtes vendu ? Oui, lui dis-je. Ah ! malheureux, qu’avez-vous fait ? n’étais-je pas assez infortunée, sans que vous travaillassiez à me la rendre davantage ? Votre liberté me consolait, et votre esclavage va me mettre au tombeau. Ah ! mon frère, que votre amour est cruel ! Et ma fille ? je ne la vois point. Je l’ai vendu aussi, lui, dis-je. Nous fondîmes tous deux en larmes, et n’eûmes pas la force de nous rien dire. Enfin j’allai trouver mon maître, et ma sœur y arriva presque aussitôt que moi ; elle se jeta à ses genoux. Je vous demande, dit-elle, la servitude comme les autres vous demandent la liberté : prenez-moi, vous me vendrez plus cher que mon mari. Ce fut alors qu’il se fit un combat qui arracha des larmes des yeux de mon maître. Malheureux ! dit-elle, as-tu pensé que je pusse accepter ma liberté aux dépens de la tienne ? Seigneur, vous voyez deux infortunés qui mourront si vous nous séparez : je me donne à vous, payez-moi ; peut-être que cet argent et mes services pourront quelque jour obtenir de vous ce que je n’ose vous demander : il est de votre intérêt de ne nous point séparer ; comptez que je dispose de sa vie. L’Arménien était un homme doux, qui fut touché de nos malheurs. Servez-moi l’un et l’autre avec fidélité et avec zèle, et je vous promets que dans un an je vous donnerai votre liberté : je vois que vous ne méritez, ni l’un ni l’autre, les malheurs de votre condition ; si, lorsque vous serez libres, vous êtes aussi heureux que vous le méritez, si la fortune vous rit, je suis certain que vous me satisferez de la perte que je souffrirai. Nous embrassâmes tous deux ses genoux, et le suivîmes dans son voyage. Nous nous soulagions l’un et l’autre dans les travaux de la servitude, et j’étais charmé lorsque j’avais pu faire l’ouvrage qui était tombé à ma sœur.\par
La fin de l’année arriva : notre maître tint sa parole, et nous délivra. Nous retournâmes à Tefflis : là je trouvai un ancien ami de mon père, qui exerçait avec succès la médecine dans cette ville ; il me prêta quelque argent avec lequel je fis quelque négoce. Quelques affaires m’appelèrent ensuite à Smyrne, où je m’établis. J’y vis depuis six ans, et j’y jouis de la plus aimable et de la plus douce société du monde : l’union règne dans ma famille, et je ne changerais pas ma condition pour celle de tous les rois du monde. J’ai été assez heureux pour retrouver le marchand arménien à qui je dois tout, et lui ai rendu des services signalés.\par

\dateline{À Smyrne, le 27 de la lune de Gemmadi 2, 1714.}
\section[{Lettre LXVIII. Rica à Usbek. À ***.}]{Lettre LXVIII. Rica à Usbek. À ***.}\renewcommand{\leftmark}{Lettre LXVIII. Rica à Usbek. À ***.}

\noindent J’allai l’autre jour dîner chez un homme de robe, qui m’en avait prié plusieurs fois. Après avoir parlé de bien des choses, je lui dis : Monsieur, il me paraît que votre métier est bien pénible. Pas tant que vous vous imaginez, répondit-il : de la manière dont nous le faisons, ce n’est qu’un amusement. Mais comment ! n’avez-vous pas toujours la tête remplie d’affaires d’autrui ? n’êtes-vous pas toujours occupé de choses qui ne sont point intéressantes ? Vous avez raison : ces choses ne sont point intéressantes, car nous nous y intéressons si peu que rien ; et cela même fait que le métier n’est pas si fatigant que vous dites. Quand je vis qu’il prenait la chose d’une manière si dégagée, je continuai, et lui dis : Monsieur, je n’ai point vu votre cabinet. Je le crois, car je n’en ai point. Quand je pris cette charge, j’eus besoin d’argent pour payer mes provisions ; je vendis ma bibliothèque ; et le libraire qui la prit, d’un nombre prodigieux de volumes, ne me laissa que mon livre de raison. Ce n’est pas que je les regrette : nous autres juges ne nous enflons point d’une vaine science. Qu’avons-nous affaire de tous ces volumes de loi ? Presque tous les cas sont hypothétiques et sortent de la règle générale. Mais ne serait-ce pas, monsieur, lui dis-je, parce que vous les en faites sortir ? Car enfin pourquoi chez tous les peuples du monde y aurait-il des lois si elles n’avaient pas leur application ? et comment peut-on les appliquer si on ne les sait pas ? Si vous connaissiez le Palais, reprit le magistrat, vous ne parleriez pas comme vous faites : nous avons des livres vivants, qui sont les avocats ; ils travaillent pour nous, et se chargent de nous instruire. Et ne se chargent-ils pas aussi quelquefois de vous tromper ? lui repartis-je. Vous ne feriez donc pas mal de vous garantir de leurs embûches ; ils ont des armes avec lesquelles ils attaquent votre équité ; il serait bon que vous en eussiez aussi pour la défendre, et que vous n’allassiez pas vous mettre dans la mêlée, habillés à la légère, parmi des gens cuirassés jusqu’aux dents.\par

\dateline{À Paris, le 13 de la lune de Chahban, 1714.}
\section[{Lettre LXIX. Usbek à Rhédi. À Venise.}]{Lettre LXIX. Usbek à Rhédi. À Venise.}\renewcommand{\leftmark}{Lettre LXIX. Usbek à Rhédi. À Venise.}

\noindent Tu ne te serais jamais imaginé que je fusse devenu plus métaphysicien que je ne l’étais : cela est pourtant, et tu en seras convaincu quand tu auras essuyé ce débordement de ma philosophie.\par
Les philosophes les plus sensés qui ont réfléchi sur la nature de Dieu ont dit qu’il était un être souverainement parfait ; mais ils ont extrêmement abusé de cette idée : ils ont fait une énumération de toutes les perfections différentes que l’homme est capable d’avoir et d’imaginer, et en ont chargé l’idée de la divinité, sans songer que souvent ces attributs s’entrempêchent, et qu’ils ne peuvent subsister dans un même sujet sans se détruire.\par
Les poètes d’Occident disent qu’un peintre ayant voulu faire le portrait de la déesse de la beauté, assembla les plus belles Grecques, et prit de chacune ce qu’elle avait de plus gracieux, dont il fit un tout pour ressembler à la plus belle de toutes les déesses. Si un homme en avait conclu qu’elle était blonde ou brune, qu’elle avait les yeux noirs ou bleus, qu’elle était douce et fière, il aurait passé pour ridicule.\par
Souvent Dieu manque d’une perfection qui pourrait lui donner une grande imperfection : mais il n’est jamais limité que par lui-même : il est lui-même sa nécessité : ainsi, quoique Dieu soit tout-puissant, il ne peut pas violer ses promesses, ni tromper les hommes. Souvent même l’impuissance n’est pas dans lui, mais dans les choses relatives ; et c’est la raison pourquoi il ne peut pas changer les essences.\par
Ainsi il n’y a point sujet de s’étonner que quelques-uns de nos docteurs aient osé nier la prescience infinie de Dieu, sur ce fondement qu’elle est incompatible avec sa justice.\par
Quelque hardie que soit cette idée, la métaphysique s’y prête merveilleusement. Selon ses principes, il n’est pas possible que Dieu prévoie les choses qui dépendent de la détermination des causes libres, parce que ce qui n’est point arrivé n’est point, et par conséquent ne peut être connu ; car le rien, qui n’a point de propriétés, ne peut être aperçu : Dieu ne peut point lire dans une volonté qui n’est point, et voir dans l’âme une chose qui n’existe point en elle ; car, jusqu’à ce qu’elle se soit déterminée, cette action qui la détermine n’est point en elle.\par
L’âme est l’ouvrière de sa détermination ; mais il y a des occasions où elle est tellement indéterminée qu’elle ne sait pas même de quel côté se déterminer. Souvent même elle ne le fait que pour faire usage de sa liberté ; de manière que Dieu ne peut voir cette détermination par avance ni dans l’action de l’âme, ni dans l’action que les objets font sur elle.\par
Comment Dieu pourrait-il prévoir les choses qui dépendent de la détermination des causes libres ? Il ne pourrait les voir que de deux manières : par conjecture, ce qui est contradictoire avec la prescience infinie ; ou bien il les verrait comme des effets nécessaires qui suivraient infailliblement d’une cause qui les produirait de même, ce qui est encore plus contradictoire : car l’âme serait libre par la supposition ; et, dans le fait, elle ne le serait pas plus qu’une boule de billard n’est libre de se remuer, lorsqu’elle est poussée par une autre.\par
Ne crois pas pourtant que je veuille borner la science de Dieu. Comme il fait agir les créatures à sa fantaisie, il connaît tout ce qu’il veut connaître. Mais, quoiqu’il puisse voir tout, il ne se sert pas toujours de cette faculté ; il laisse ordinairement à la créature la faculté d’agir ou de ne pas agir, pour lui laisser celle de mériter ou de démériter : c’est pour lors qu’il renonce au droit qu’il a d’agir sur elle, et de la déterminer. Mais quand il veut savoir quelque chose, il le sait toujours, parce qu’il n’a qu’à vouloir qu’elle arrive comme il la voit, et déterminer les créatures conformément à sa volonté. C’est ainsi qu’il tire ce qui doit arriver du nombre des choses purement possibles, en fixant par ses décrets les déterminations futures des esprits, et les privant de la puissance qu’il leur a donnée d’agir ou de ne pas agir.\par
Si l’on peut se servir d’une comparaison dans une chose qui est au-dessus des comparaisons : un monarque ignore ce que son ambassadeur fera dans une affaire importante : s’il le veut savoir, il n’a qu’à lui ordonner de se comporter d’une telle manière, et il pourra assurer que la chose arrivera comme il la projette.\par
L’Alcoran et les livres des Juifs s’élèvent sans cesse contre le dogme de la prescience absolue : Dieu y paraît partout ignorer la détermination future des esprits ; et il semble que ce soit la première vérité que Moïse ait enseignée aux hommes.\par
Dieu met Adam dans le paradis terrestre, à condition qu’il ne mangera pas d’un certain fruit : précepte absurde dans un être qui connaîtrait les déterminations futures des âmes ; car enfin un tel être peut-il mettre des conditions à ses grâces, sans les rendre dérisoires ? C’est comme si un homme qui aurait su la prise de Bagdad avait dit à un autre : Je vous donne mille écus si Bagdad n’est pas pris. Ne ferait-il pas là une mauvaise plaisanterie ?\par
Mon cher Rhédi, pourquoi tant de philosophie ? Dieu est si haut, que nous n’apercevons pas même ses nuages. Nous ne le connaissons bien que dans ses préceptes. Il est immense, spirituel, infini. Que sa grandeur nous ramène à notre faiblesse. S’humilier toujours, c’est l’adorer toujours.\par

\dateline{À Paris, le dernier de la lune de Chahban, 1714.}
\section[{Lettre LXX. Zélis à Usbek. À Paris.}]{Lettre LXX. Zélis à Usbek. À Paris.}\renewcommand{\leftmark}{Lettre LXX. Zélis à Usbek. À Paris.}

\noindent Soliman, que tu aimes, est désespéré d’un affront qu’il vient de recevoir. Un jeune étourdi, nommé Suphis, recherchait depuis trois mois sa fille en mariage : il paraissait content de la figure de la fille, sur le rapport et la peinture que lui en avaient faits les femmes qui l’avaient vue dans son enfance ; on était convenu de la dot, et tout s’était passé sans aucun incident. Hier, après les premières cérémonies, la fille sortit à cheval, accompagnée de son eunuque, et couverte, selon la coutume, depuis la tête jusqu’aux pieds. Mais, dès qu’elle fut arrivée devant la maison de son mari prétendu, il lui fit fermer la porte, et il jura qu’il ne la recevrait jamais si on n’augmentait la dot. Les parents accoururent, de côté et d’autre, pour accommoder l’affaire ; et après bien de la résistance, ils firent convenir Soliman de faire un petit présent à son gendre. Enfin, les cérémonies du mariage accomplies, on conduisit la fille dans le lit avec assez de violence ; mais, une heure après, cet étourdi se leva furieux, lui coupa le visage en plusieurs endroits, soutenant qu’elle n’était pas vierge, et la renvoya à son père. On ne peut pas être plus frappé qu’il l’est de cette injure. Il y a des personnes qui soutiennent que cette fille est innocente. Les pères sont bien malheureux d’être exposés à de tels affronts : si pareil traitement arrivait à ma fille, je crois que j’en mourrais de douleur. Adieu.\par

\dateline{Du sérail de Fatmé, le 9 de la lune de Gemmadi I, 1714.}
\section[{Lettre LXXI. Usbek à Zélis.}]{Lettre LXXI. Usbek à Zélis.}\renewcommand{\leftmark}{Lettre LXXI. Usbek à Zélis.}

\noindent Je plains Soliman, d’autant plus que le mal est sans remède, et que son gendre n’a fait que se servir de la liberté de la loi. Je trouve cette loi bien dure, d’exposer ainsi l’honneur d’une famille aux caprices d’un fou. On a beau dire que l’on a des indices certains pour connaître la vérité, c’est une vieille erreur dont on est aujourd’hui revenu parmi nous ; et nos médecins donnent des raisons invincibles de l’incertitude de nos preuves. Il n’y a pas jusqu’aux chrétiens qui ne les regardent comme chimériques, quoiqu’elles soient clairement établies par leurs livres sacrés, et que leur ancien législateur en ait fait dépendre l’innocence ou la condamnation de toutes les filles.\par
J’apprends avec plaisir le soin que tu te donnes de l’éducation de la tienne. Dieu veuille que son mari la trouve aussi belle et aussi pure que Fatima ; qu’elle ait dix eunuques pour la garder ; qu’elle soit l’honneur et l’ornement du sérail où elle est destinée ; qu’elle n’ait sur sa tête que des lambris dorés, et ne marche que sur des tapis superbes ; et, pour comble de souhaits. Puissent mes yeux la voir dans toute sa gloire !\par

\dateline{À Paris, le 5 de la lune de Chalval, 1714.}
\section[{Lettre LXXII. Rica à Usbek. A***.}]{Lettre LXXII. Rica à Usbek. A***.}\renewcommand{\leftmark}{Lettre LXXII. Rica à Usbek. A***.}

\noindent Je me trouvai l’autre jour dans une compagnie où je vis un homme bien content de lui. Dans un quart d’heure, il décida trois questions de morale, quatre problèmes historiques, et cinq points de physique : je n’ai jamais vu un décisionnaire si universel ; son esprit ne fut jamais suspendu par le moindre doute. On laissa les sciences ; on parla des nouvelles du temps : il décida sur les nouvelles du temps. Je voulus l’attraper, et je dis en moi-même : il faut que je me mette dans mon fort ; je vais me réfugier dans mon pays. Je lui parlai de la Perse ; mais à peine lui eus-je dis quatre mots, qu’il me donna deux démentis, fondés sur l’autorité de messieurs Tavernier et Chardin. Ah ! bon Dieu ! dis-je en moi-même, quel homme est-ce là ? Il connaîtra tout à l’heure les rues d’Ispahan mieux que moi ! Mon parti fut bientôt pris : je me tus, je le laissai parler, et il décide encore.\par

\dateline{À Paris, le 8 de la lune de Zilcadé, 1715.}
\section[{Lettre LXXIII. Rica à ***.}]{Lettre LXXIII. Rica à ***.}\renewcommand{\leftmark}{Lettre LXXIII. Rica à ***.}

\noindent J’ai ouï parler d’une espèce de tribunal qu’on appelle l’Académie française : il n’y en a point de moins respecté dans le monde ; car on dit qu’aussitôt qu’il a décidé, le peuple casse ses arrêts, et lui impose des lois qu’il est obligé de suivre.\par
Il y a quelques temps que, pour fixer son autorité, il donna un code de ses jugements. Cet enfant de tant de pères était presque vieux quand il naquit ; et quoiqu’il fût légitime, un bâtard, qui avait déjà paru, l’avait presque étouffé dans sa naissance.\par
Ceux qui le composent n’ont d’autre fonction que de jaser sans cesse : l’éloge va se placer comme de lui-même dans leur babil éternel ; et sitôt qu’ils sont initiés dans ses mystères, la fureur du panégyrique vient les saisir, et ne les quitte plus.\par
Ce corps a quarante têtes, toutes remplies de figures, de métaphores et d’antithèse ; tant de bouches ne parlent presque que par exclamation ; ses oreilles veulent toujours être frappées par la cadence et l’harmonie. Pour les yeux, il n’en est pas question : il semble qu’il soit fait pour parler, et non pas pour voir. Il n’est point ferme sur ses pieds ; car le temps, qui est son fléau, l’ébranle à tous les instants, et détruit tout ce qu’il a fait. On a dit autrefois que ses mains étaient avides ; je ne t’en dirai rien, et je laisse décider cela à ceux qui le savent mieux que moi.\par
Voilà des bizarreries, ***, que l’on ne voit point dans notre Perse. Nous n’avons point l’esprit porté à ces établissements singuliers et bizarres ; nous cherchons toujours la nature dans nos coutumes simples et nos manières naïves.\par

\dateline{À Paris, le 27 de la lune de Zilhagé, 1715.}
\section[{Lettre LXXIV. Rica à Usbek. À ***.}]{Lettre LXXIV. Rica à Usbek. À ***.}\renewcommand{\leftmark}{Lettre LXXIV. Rica à Usbek. À ***.}

\noindent Il y a quelques jours qu’un homme de ma connaissance me dit : Je vous ai promis de vous produire dans les bonnes maisons de Paris ; je vous mène à présent chez un grand seigneur qui est un des hommes du royaume qui représentent le mieux.\par
Que cela veut-il dire, monsieur ? est-ce qu’il est plus poli, plus affable qu’un autre ? Ce n’est pas cela, me dit-il. Ah ! J’entends ; il fait sentir à tous les instants la supériorité qu’il a sur tous ceux qui l’approchent ; si cela est, je n’ai faire d’y aller ; je prends déjà condamnation, et je la lui passe tout entière.\par
Il fallut pourtant marcher ; et je vis un petit homme si fier, il prit une prise de tabac avec tant de hauteur, il se moucha si impitoyablement, il cracha avec tant de flegme, il caressa ses chiens d’une manière si offensante pour les hommes, que je ne pouvais me laisser de l’admirer. Ah ! bon Dieu ! dis-je en moi-même, si lorsque j’étais à la course de la Perse, je représentais ainsi, je représentais un grand sot ! Il aurait fallu, Usbek, que nous eussions eu un bien mauvais naturel pour aller faire cent petites insultes à des gens qui venaient tous les jours chez nous nous témoigner leur bienveillance ; ils savaient bien que nous étions au-dessus d’eux ; et s’ils l’avaient ignoré, nos bienfaits le leur auraient appris chaque jour. N’ayant rien à faire pour nous faire respecter, nous faisions tout pour nous rendre aimables : nous nous communiquions aux plus petits ; au milieu des grandeurs, qui endurcissent toujours, ils nous trouvaient sensibles ; ils ne voyaient que notre cœur au-dessus d’eux ; nous descendions jusqu’à leurs besoins. Mais lorsqu’il fallait soutenir la majesté du prince dans les cérémonies publiques ; lorsqu’il fallait faire respecter la nation aux étrangers ; lorsque enfin, dans les occasions périlleuses, il fallait animer les soldats, nous remontions cent fois plus haut que nous n’étions descendus ; nous ramenions la fierté sur notre visage ; et l’on trouvait quelquefois que nous représentions assez bien.\par

\dateline{De Paris, le 10 de la lune de Saphar, 1715.}
\section[{Lettre LXXV. Usbek à Rhédi. À Venise.}]{Lettre LXXV. Usbek à Rhédi. À Venise.}\renewcommand{\leftmark}{Lettre LXXV. Usbek à Rhédi. À Venise.}

\noindent Il faut que je te l’avoue, je n’ai point remarqué chez les chrétiens cette persuasion vive de leur religion qui se trouve parmi les musulmans ; il y a bien loin chez eux de la profession à la croyance, de la croyance à la conviction, de la conviction à la pratique. La religion est moins un sujet de sanctification qu’un sujet de disputes qui appartient à tout le monde : les gens de cour, les gens de guerre, les femmes même, s’élèvent contre les ecclésiastiques, et leur demandent de leur prouver ce qu’ils sont résolus de ne pas croire. Ce n’est pas qu’ils se soient déterminés par raison, et qu’ils aient pris la peine d’examiner la vérité ou la fausseté de cette religion qu’ils rejettent : ce sont des rebelles qui ont senti le joug, et l’ont secoué avant de l’avoir connu. Aussi ne sont-ils pas plus fermes dans leur incrédulité que dans leur foi : ils vivent dans un flux et reflux qui les porte sans cesse de l’un à l’autre. Un d’eux me disait un jour : Je crois l’immortalité de l’âme par semestre ; mes opinions dépendent absolument de la constitution de mon corps ; selon que j’ai plus ou moins d’esprits animaux, que mon estomac digère bien ou mal, que l’air que je respire est subtil ou grossier, que les viandes dont je me nourris sont légères ou solides, je suis spinoziste, socinien, catholique, impie ou dévot. Quand le médecin est auprès de mon lit, le confesseur me trouve à son avantage. Je sais bien empêcher la religion de m’affliger quand je me porte bien ; mais je lui permets de me consoler quand je suis malade : lorsque je n’ai plus rien à espérer d’un côté, la religion se présente et me gagne par ses promesses ; je veux bien m’y livrer, et mourir du côté de l’espérance. \\
Il y a longtemps que les princes chrétiens affranchirent tous les esclaves de leurs États, parce, disaient-ils, que le christianisme rend tous les hommes égaux. Il est vrai que cet acte de religion leur était très utile : ils abaissaient par là les seigneurs, de la puissance desquels ils retiraient le bas peuple. Ils ont ensuite fait des conquêtes dans des pays où ils ont vu qu’il leur était avantageux d’avoir des esclaves ; et ils ont permis d’en acheter et d’en vendre, oubliant ce principe de religion qui les touchait tant. Que veux-tu que je te dise ? vérité dans un temps, erreur dans un autre. Que ne faisons-nous comme les chrétiens ? Nous sommes bien simples de refuser des établissements et des conquêtes faciles dans des climats heureux\footnote{Les mahométans ne se soucient point de prendre Venise, parce qu’ils n’y trouveraient point d’eau pour leurs purifications.}, parce que l’eau n’y est pas assez pure pour nous laver selon les principes du saint Alcoran !\par
Je rends grâce au Dieu tout-puissant, qui a envoyé Ali son grand prophète, de ce que je professe une religion qui se fait préférer à tous les intérêts humains, et qui est pure comme le ciel, dont elle est descendue.\par

\dateline{De Paris, le 13 de la lune de Saphar, 1715.}
\section[{Lettre LXXVI. Usbek à son ami Ibben. À Smyrne.}]{Lettre LXXVI. Usbek à son ami Ibben. À Smyrne.}\renewcommand{\leftmark}{Lettre LXXVI. Usbek à son ami Ibben. À Smyrne.}

\noindent Les lois sont furieuses en Europe contre ceux qui se tuent eux-mêmes : on les fait mourir, pour ainsi dire, une seconde fois ; ils sont traînés indignement par les rues ; on les note d’infamie ; on confisque leurs biens.\par
Il me paraît, Ibben, que ces lois sont bien injustes. Quand je suis accablé de douleur, de misère, de mépris, pourquoi veut-on m’empêcher de mettre fin à mes peines, et me priver cruellement d’un remède qui est en mes mains ?\par
Pourquoi veut-on que je travaille pour une société, dont je consens de n’être plus ? que je tienne, malgré moi, une convention qui s’est faite sans moi ? La société est fondée sur un avantage mutuel : mais lorsqu’elle me devient onéreuse, qui m’empêche d’y renoncer ? La vie m’a été donnée comme une faveur ; je puis donc la rendre lorsqu’elle ne l’est plus : la cause cesse, l’effet doit donc cesser aussi.\par
Le prince veut-il que je sois son sujet quand je ne retire point les avantages de la sujétion ? Mes concitoyens peuvent-ils demander ce partage inique de leur utilité et de mon désespoir ? Dieu, différent de tous les bienfaiteurs, veut-il me condamner à recevoir des grâces qui m’accablent ?\par
Je suis obligé de suivre les lois quand je vis sous les lois : mais quand je n’y vis plus, peuvent-elles me lier encore ?\par
Mais, dira-t-on, vous troublez l’ordre de la providence. Dieu a uni votre âme avec votre corps ; et vous l’en séparez : vous vous opposez donc à ses desseins, et vous lui résistez.\par
Que veut dire cela ? Troublé-je l’ordre de la providence, lorsque je change les modifications de la matière, et que je rends carrée une boule que les premières lois du mouvement, c’est-à-dire les lois de la création et de la conservation, avaient faite ronde ? Non, sans doute : je ne fais qu’user du droit qui m’a été donné ; et, en ce sens, je puis troubler à ma fantaisie toute la nature, sans que l’on puisse dire que je m’oppose à la providence.\par
Lorsque mon âme sera séparée de mon corps, y aura-t-il moins d’ordre et moins d’arrangement dans l’univers ? croyez-vous que cette nouvelle combinaison soit moins parfaite, et moins dépendante des lois générales ? que le monde y ait perdu quelque chose ? et que les ouvrages de Dieu soient moins grands, ou plutôt moins immenses ?\par
Croyez-vous que mon corps, devenu un épi de blé, un ver, un gazon, soit changé en un ouvrage de la nature moins digne d’elle, et que mon âme, dégagée de tout ce qu’elle avait de terrestre, soit devenue moins sublime ?\par
Toutes ces idées, mon cher Ibben, n’ont d’autre source que notre orgueil : nous ne sentons point notre petitesse ; et, malgré qu’on en ait, nous voulons être comptés dans l’univers, y figurer, et y être un objet important. Nous nous imaginons que l’anéantissement d’un être aussi parfait que nous dégraderait toute la nature ; et nous ne concevons pas qu’un homme de plus ou de moins dans le monde, que dis-je ? tous les hommes ensemble, cent millions de têtes comme la nôtre, ne sont qu’un atome subtil et délié que Dieu n’aperçoit qu’à cause de l’immensité de ses connaissances.\par

\dateline{À Paris, le 15 de la lune de Saphar, 1715.}
\section[{Lettre LXXVII. Ibben à Usbek. À Paris.}]{Lettre LXXVII. Ibben à Usbek. À Paris.}\renewcommand{\leftmark}{Lettre LXXVII. Ibben à Usbek. À Paris.}

\noindent Mon cher Usbek, il me semble que, pour un vrai musulman, les malheurs sont moins des châtiments que des menaces. Ce sont des jours bien précieux que ceux qui nous portent à expier les offenses. C’est le temps des prospérités qu’il faudrait abréger. Que servent toutes ces impatiences, qu’à faire voir que nous voudrions être heureux, indépendamment de celui qui donne les félicités, parce qu’il est la félicité même ?\par
Si un être est composé de deux êtres, et que la nécessité de conserver l’union marque plus la soumission aux ordres du Créateur, on en a pu faire une loi religieuse ; si cette nécessité de conserver l’union est un meilleur garant des actions des hommes, on en a pu faire une loi civile.\par

\dateline{De Smyrne, le dernier jour de la lune de Saphar, 1715.}
\section[{Lettre LXXVIII. Rica à Usbek. À ***.}]{Lettre LXXVIII. Rica à Usbek. À ***.}\renewcommand{\leftmark}{Lettre LXXVIII. Rica à Usbek. À ***.}

\noindent Je t’envoie la copie d’une lettre qu’un Français qui est en Espagne a écrite ici ; je crois que tu seras bien aise de la voir.\par
Je parcours depuis six mois l’Espagne et le Portugal, et je vis parmi des peuples qui, méprisant tous les autres, font aux seuls Français l’honneur de les haïr.\par
La gravité est le caractère brillant des deux nations : elles se manifestent principalement de deux manières ; par les lunettes, et par la moustache.\par
Les lunettes font voir démonstrativement que celui qui les porte est un homme consommé dans les sciences et enseveli dans de profondes lectures, à tel point que sa vue s’en est affaiblie ; et tout nez qui en est orné ou chargé peut passer, sans contredit, pour le nez d’un savant.\par
Pour la moustache, elle est respectable par elle-même, et indépendamment des conséquences ; quoique pourtant on ne laisse pas d’en tirer souvent de grandes utilités pour le service du prince et l’honneur de la nation, comme le fit bien voir un fameux général portugais dans les Indes\footnote{Jean de Castro.} : car, se trouvant avoir besoin d’argent, il se coupa une de ses moustaches, et envoya demander aux habitants de Goa vingt mille pistoles sur ce gage ; elles lui furent prêtées d’abord, et dans la suite il retira sa moustache avec honneur.\par
On conçoit aisément que des peuples graves et flegmatiques comme ceux-là peuvent avoir de la vanité : aussi en ont-ils. Ils la fondent ordinairement sur deux choses bien considérables. Ceux qui vivent dans le continent de l’Espagne et du Portugal se sentent le cœur extrêmement élevé, lorsqu’ils sont ce qu’ils appellent de vieux chrétiens ; c’est-à-dire, qu’ils ne sont pas originaires de ceux à qui l’inquisition a persuadé dans ces derniers siècles d’embrasser la religion chrétienne. Ceux qui sont dans les Indes ne sont pas moins flattés lorsqu’ils considèrent qu’ils ont le sublime mérite d’être, comme ils le disent, hommes de chair blanche. Il n’y a jamais eu dans le sérail du Grand Seigneur de sultane si orgueilleuse de sa beauté que le plus vieux et le plus vilain mâtin ne l’est de la blancheur olivâtre de son teint, lorsqu’il est dans une ville du Mexique, assis sur sa porte, les bras croisés. Un homme de cette conséquence, une créature si parfaite, ne travaillerait pas pour tous les trésors du monde, et ne se résoudrait jamais, par une vile et mécanique industrie, de compromettre l’honneur et la dignité de sa peau.\par
Car il faut savoir que lorsqu’un homme a un certain mérite en Espagne, comme, par exemple, quand il peut ajouter aux qualités dont je viens de parler celle d’être le propriétaire d’une grande épée, ou d’avoir appris de son père l’art de faire jurer une discordante guitare, il ne travaille plus : son honneur s’intéresse au repos de ses membres. Celui qui reste assis dix heures par jour obtient précisément la moitié plus de considération qu’un autre qui n’en reste que cinq, parce que c’est sur les chaises que la noblesse s’acquiert.\par
Mais quoique ces invincibles ennemis du travail fassent parade d’une tranquillité philosophique, ils ne l’ont pourtant pas dans le cœur ; car ils sont toujours amoureux. Ils sont les premiers hommes du monde pour mourir de langueur sous la fenêtre de leurs maîtresses ; et tout Espagnol qui n’est pas enrhumé ne saurait passer pour galant.\par
Ils sont premièrement dévots, et secondement jaloux. Ils se garderont bien d’exposer leurs femmes aux entreprises d’un soldat criblé de coups, ou d’un magistrat décrépit ; mais ils les enfermeront avec un novice fervent qui baisse les yeux, ou un robuste franciscain qui les élève.\par
Ils connaissent mieux que les autres le faible des femmes ; ils ne veulent pas qu’on leur voie le talon, et qu’on les surprenne par le bout des pieds : ils savent que l’imagination va toujours, que rien ne l’amuse en chemin ; elle arrive, et là on était quelquefois averti d’avance.\par
On dit partout que les rigueurs de l’amour sont cruelles, elles le sont encore plus pour les Espagnols ; les femmes les guérissent de leurs peines ; mais elles ne font que leur en faire changer, et il leur reste toujours un long et fâcheux souvenir d’une passion éteinte.\par
Ils ont de petites politesses qui en France paraîtraient mal placées : par exemple, un capitaine ne bat jamais son soldat sans lui en demander la permission ; et l’inquisition ne fait jamais brûler un Juif sans lui faire ses excuses.\par
Les Espagnols qu’on ne brûle pas paraissent si attachés à l’inquisition, qu’il y aurait de la mauvaise humeur de la leur ôter : je voudrais seulement qu’on en établit une autre ; non pas contre les hérétiques, mais contre les hérésiarques qui attribuent à de petites pratiques monacales la même efficacité qu’aux sept sacrements ; qui adorent tout ce qu’ils vénèrent ; et qui sont si dévots qu’ils sont à peine chrétiens.\par
Vous pourrez trouver de l’esprit et du bon sens chez les Espagnols ; mais n’en cherchez point dans leurs livres : voyez une de leurs bibliothèques, les romans d’un côté, et les scolastiques de l’autre : vous diriez que les parties en ont été faites, et le tout rassemblé, par quelque ennemi secret de la raison humaine.\par
Le seul de leurs livres qui soit bon est celui qui a fait voir le ridicule de tous les autres.\par
Ils ont fait des découvertes immenses dans le nouveau monde, et ils ne connaissent pas encore leur propre continent : il y a sur leurs rivières tel port qui n’a pas encore été découvert, et dans leurs montagnes des nations qui leur sont inconnues.\par
Ils disent que le soleil se lève et se couche dans leur pays : mais il faut dire aussi qu’en faisant sa course il ne rencontre que des campagnes ruinées et des contrées désertes.\par
Je ne serais pas fâché, Usbek, de voir une lettre écrite à Madrid par un Espagnol qui voyagerait en France ; je crois qu’il vengerait bien sa nation. Quel vaste champ pour un homme flegmatique et pensif ! Je m’imagine qu’il commencerait ainsi la description de Paris.\par
Il y a ici une maison où l’on met les fous : on croirait d’abord qu’elle est la plus grande de la ville ; non : le remède est bien petit pour le mal. Sans doute que les Français, extrêmement décriés chez leurs voisins, enferment quelques fous dans une maison, pour persuader que ceux qui sont dehors ne le sont pas.\par
Je laisse là mon Espagnol. Adieu, mon cher Usbek.\par

\dateline{De Paris, le 17 de la lune de Saphar, 1715.}
\section[{Lettre LXXIX. Usbek à Rhédi. À Venise.}]{Lettre LXXIX. Usbek à Rhédi. À Venise.}\renewcommand{\leftmark}{Lettre LXXIX. Usbek à Rhédi. À Venise.}

\noindent La plupart des législateurs ont été des hommes bornés, que le hasard a mis à la tête des autres et qui n’ont presque consulté que leurs préjugés et leurs fantaisies. \\
Il semble qu’ils aient méconnu la grandeur et la dignité même de leur ouvrage : ils se sont amusés à faire des institutions puériles, avec lesquelles ils se sont à la vérité conformés aux petits esprits, mais décrédités auprès des gens de bon sens.\par
Ils se sont jetés dans des détails inutiles ; ils ont donné dans les cas particuliers : ce qui marque un génie étroit qui ne voit les choses que par parties, et n’embrasse rien d’une vue générale.\par
Quelques-uns ont affecté de se servir d’une autre langue que la vulgaire ; chose absurde pour un faiseur de lois : comment peut-on les observer, si elles ne sont pas connues ?\par
Ils ont souvent aboli sans nécessité celles qu’ils ont trouvées établies ; c’est-à-dire qu’ils ont jeté les peuples dans les désordres inséparables des changements.\par
Il est vrai que, par une bizarrerie qui vient plutôt de la nature que de l’esprit des hommes, il est quelquefois nécessaire de changer certaines lois. Mais le cas est rare, et lorsqu’il arrive, il n’y faut toucher que d’une main tremblante : on y doit observer tant de solennités, et apporter tant de précautions, que le peuple en conclue naturellement que les lois sont bien saintes, puisqu’il faut tant de formalités pour les abroger.\par
Souvent ils les ont faites trop subtiles, et ont suivi des idées logiciennes plutôt que l’équité naturelle. Dans la suite, elles ont été trouvées trop dures ; et par un esprit d’équité, on a cru devoir s’en écarter : mais ce remède était un nouveau mal. Quelles que soient ces lois, il faut toujours les suivre, et les regarder comme la conscience publique, à laquelle celle des particuliers doit se conformer toujours.\par
Il faut pourtant avouer que quelques-uns d’entre eux ont eu une attention qui marque beaucoup de sagesse ; c’est qu’ils ont donné aux pères une grande autorité sur leurs enfants : rien ne soulage plus les magistrats ; rien ne dégarnit plus de tranquillité dans un État, où les mœurs font toujours de meilleurs citoyens que les lois.\par
C’est de toutes les puissances celle dont on abuse le moins ; c’est la plus sacrée de toutes les magistratures ; c’est la seule qui ne dépend pas des conventions, et qui les a même précédées.\par
On remarque que, où l’on met dans les mains paternelles plus de récompenses et de punitions, les familles sont mieux réglées : les pères sont l’image du créateur de l’univers, qui, quoiqu’il puisse conduire les hommes par son amour, ne laisse pas de se les attacher encore par les motifs de l’espérance et de la crainte.\par
Je ne finirai pas cette lettre sans te faire remarquer la bizarrerie de l’esprit des Français. On dit qu’ils ont retenu des lois romaines un nombre infini de choses inutiles, et même pis ; et ils n’ont pas pris d’elle la puissance paternelle, qu’elles ont établie comme la première autorité légitime.\par

\dateline{À Paris, le 18 de la lune de Saphar, 1715.}
\section[{Lettre LXXX. Le grand eunuque à Usbek. À Paris.}]{Lettre LXXX. Le grand eunuque à Usbek. À Paris.}\renewcommand{\leftmark}{Lettre LXXX. Le grand eunuque à Usbek. À Paris.}

\noindent Hier des Arméniens menèrent au sérail une jeune esclave de Circassie, qu’ils voulaient vendre. Je la fis entrer dans les appartements secrets, je la déshabillai, je l’examinai avec les regards d’un juge ; et plus je l’examinai, plus je lui trouvai des grâces. Une pudeur virginale semblait vouloir les dérober à ma vue ; je vis tout ce qu’il lui en coûtait pour obéir : elle rougissait de se voir nue, même devant moi, qui, exempt des passions qui peuvent alarmer la pudeur, suis inanimé sous l’empire de ce sexe, et qui, ministre de la modestie dans les actions les plus libres, ne porte que de chastes regards, et ne puis inspirer que l’innocence.\par
Dès que je l’eus jugée digne de toi, je baissai les yeux : je lui jetai un manteau d’écarlate, je lui mis au doigt un anneau d’or ; je me prosternai à ses pieds, je l’adorai comme la reine de ton cœur ; je payai les Arméniens ; je la dérobai à tous les yeux. Heureux Usbek ! tu possèdes plus de beautés que n’en enferment tous les palais d’Orient. Quel plaisir pour toi de trouver, à ton retour, tout ce que la Perse a de plus ravissant, et de voir dans ton sérail renaître les grâces, à mesure que le temps et la possession travaillent à les détruire !\par

\dateline{Du sérail de Fatmé, le 1er de la lune de Rebiab I, 1715.}
\section[{Lettre LXXXI. Usbek à Rhédi. À Venise.}]{Lettre LXXXI. Usbek à Rhédi. À Venise.}\renewcommand{\leftmark}{Lettre LXXXI. Usbek à Rhédi. À Venise.}

\noindent Depuis que je suis en Europe, mon cher Rhédi, j’ai vu bien des gouvernements : ce n’est pas comme en Asie, où les règles de la politique se trouvent partout les mêmes.\par
J’ai souvent pensé en moi-même, pour savoir quel de tous les gouvernements était le plus conforme à la raison. Il m’a semblé que le plus parfait est celui qui va à son but à moins de frais ; et qu’ainsi celui qui conduit les hommes de la manière qui convient le plus à leur penchant et à leur inclination est le plus parfait.\par
Si, dans un gouvernement doux, le peuple est aussi soumis que dans un gouvernement sévère, le premier est préférable, puisqu’il est plus conforme à la raison, et que la sévérité est un motif étranger.\par
Compte, mon cher Rhédi, que dans un État les peines plus ou moins cruelles ne font pas que l’on obéisse plus aux lois. Dans les pays où les châtiments sont modérés, on les craint comme dans ceux où ils sont tyranniques et affreux.\par
Soit que le gouvernement soit doux, soit qu’il soit cruel, on punit toujours par degrés ; on inflige un châtiment plus ou moins grand à un crime plus ou moins grand. L’imagination se plie d’elle-même aux mœurs du pays où l’on vit : huit jours de prison, ou une légère amende, frappent autant l’esprit d’un Européen nourri dans un pays de douceur, que la perte d’un bras intimide un Asiatique. Ils attachent un certain degré de crainte à un certain degré de peine, et chacun la partage à sa façon : le désespoir de l’infamie vient désoler un Français qu’on vient de condamner à une peine qui n’ôterait pas un quart d’heure de sommeil à un Turc.\par
D’ailleurs je ne vois pas que la police, la justice et l’équité soient mieux observées en Turquie, en Perse, chez le Mogol, que dans les républiques de Hollande, de Venise, et dans l’Angleterre même ; je ne vois pas qu’on y commette moins de crimes ; et que les hommes, intimidés par la grandeur des châtiments, y soient plus soumis aux lois.\par
Je remarque au contraire une source d’injustice et de vexations au milieu de ces mêmes États.\par
Je trouve même le prince, qui est la loi même, moins maître que partout ailleurs.\par
Je vois que, dans ces moments rigoureux, il y a toujours des mouvements tumultueux, où personne n’est le chef ; et que, quand une fois l’autorité violente est méprisée, il n’en reste plus assez à personne pour la faire revenir ;\par
Que le désespoir même de l’impunité confirme le désordre, et le rend plus grand ;\par
Que, dans ces États, il ne se forme point de petite révolte, et qu’il n’y a jamais d’intervalle entre le murmure et la sédition ;\par
Qu’il ne faut point que les grands événements y soient préparés par de grandes causes ; au contraire, le moindre accident produit une grande révolution, souvent aussi imprévue de ceux qui la font que de ceux qui la souffrent.\par
Lorsqu’Osman, empereur des Turcs, fut déposé, aucun de ceux qui commirent cet attentat ne songeait à le commettre ; ils demandaient seulement en suppliants qu’on leur fit justice sur quelque grief : une voix, qu’on n’a jamais connue, sortit de la foule par hasard ; le nom de Mustapha fut prononcé, et soudain Mustapha fut empereur.\par

\dateline{De Paris, le 2 de la lune de Rebiab I, 1715.}
\section[{Lettre LXXXII. Nargum, envoyé de Perse en Moscovie à Usbek. À Paris.}]{Lettre LXXXII. Nargum, envoyé de Perse en Moscovie à Usbek. À Paris.}\renewcommand{\leftmark}{Lettre LXXXII. Nargum, envoyé de Perse en Moscovie à Usbek. À Paris.}

\noindent De toutes les nations du monde, mon cher Usbek, il n’y en a pas qui ait surpassé celle des Tartares, ni en gloire, ni dans la grandeur des conquêtes. Ce peuple est le vrai dominateur de l’univers ; tous les autres semblent être faits pour le servir : il est également le fondateur et le destructeur des empires ; dans tous les temps il a donné sur la terre des marques de sa puissance ; dans tous les âges, il a été le fléau des nations.\par
Les Tartares ont conquis deux fois la Chine, et ils la tiennent encore sous leur obéissance.\par
Ils dominent sur les vastes pays qui forment l’empire du Mogol.\par
Maîtres de la Perse, ils sont assis sur le trône de Cyrus et de Gustaspe. Ils ont soumis la Moscovie. Sous le nom de Turcs, ils ont fait des conquêtes immenses dans l’Europe, l’Asie et l’Afrique ; et ils dominent sur ces trois parties de l’univers.\par
Et, pour parler de temps plus reculés, c’est d’eux que sont sortis presque tous les peuples qui ont renversé l’Empire romain.\par
Qu’est-ce que les conquêtes d’Alexandre, en comparaison de celles de Genghiscan. Il n’a manqué à cette victorieuse nation que des historiens, pour célébrer la mémoire de ses merveilles.\par
Que d’actions immortelles ont été ensevelies dans l’oubli ! que d’empires par eux fondés, dont nous ignorons l’origine ! Cette belliqueuse nation, uniquement occupée de sa gloire présente, sûre de vaincre dans tous les temps, ne songeait point à se signaler dans l’avenir par la mémoire de ses conquêtes passées.\par

\dateline{De Moscou, le 4 de la lune de Rebiab I, 1715.}
\section[{Lettre LXXXIII. Rica à Ibben. À Smyrne.}]{Lettre LXXXIII. Rica à Ibben. À Smyrne.}\renewcommand{\leftmark}{Lettre LXXXIII. Rica à Ibben. À Smyrne.}

\noindent Quoique les Français parlent beaucoup, il y a cependant parmi eux une espèce de dervis taciturnes qu’on appelle chartreux : on dit qu’ils se coupent la langue en entrant dans le couvent ; et on souhaiterait fort que tous les autres dervis se retranchassent de même tout ce que leur profession leur rend inutile.\par
À propos de gens taciturnes, il y en a de bien plus singuliers que ceux-là, et qui ont un talent bien extraordinaire. Ce sont ceux qui savent parler sans rien dire, et qui amusent une conversation pendant deux heures de temps sans qu’il soit possible de les déceler, d’être leur plagiaire, ni de retenir un mot de ce qu’ils ont dit.\par
Ces sortes de gens sont adorés des femmes : mais ils ne le sont pourtant pas tant que d’autres, qui ont reçu de la nature l’aimable talent de sourire à propos, c’est-à-dire à chaque instant, et qui portent la grâce d’une gracieuse approbation sur tout ce qu’elles disent.\par
Mais ils sont au comble de l’esprit lorsqu’ils savent entendre finesse à tout, et trouver milles petits traits ingénieux dans les choses les plus communes.\par
J’en connais d’autres qui s’en sont bien trouvés d’introduire dans les conversations les choses inanimées, et d’y faire parler leur habit brodé, leur perruque blonde, leur tabatière, leur canne et leurs gants. Il est bon de commencer de la rue à se faire écouter par le bruit du carrosse, et du marteau qui frappe rudement la porte : cet avant-propos prévient pour le reste du discours ; et quand l’exorde est beau, il rend supportables toutes les sottises qui viennent ensuite, mais qui par bonheur arrivent trop tard.\par
Je te promets que ces petits talents, dont on ne fait aucun cas chez nous, servent bien ici ceux qui sont assez heureux pour les avoir, et qu’un homme de bons sens ne brille guère devant ces sortes de gens.\par

\dateline{De Paris, le 6 de la lune de Rebiab 2, 1715.}
\section[{Lettre LXXXIV. Usbek à Rhédi. À Venise.}]{Lettre LXXXIV. Usbek à Rhédi. À Venise.}\renewcommand{\leftmark}{Lettre LXXXIV. Usbek à Rhédi. À Venise.}

\noindent S’il y a un Dieu, mon cher Rhédi, il faut nécessairement qu’il soit juste : car, s’il ne l’était pas, il serait le plus mauvais et le plus imparfait de tous les êtres.\par
La justice est un rapport de convenance, qui se trouve réellement entre deux choses : ce rapport est toujours le même, quelque être qui le considère, soit que ce soit Dieu, soit que ce soit un ange, ou enfin que ce soit un homme.\par
Il est vrai que les hommes ne voient pas toujours ces rapports ; souvent même lorsqu’ils les voient, ils s’en éloignent ; et leur intérêt est toujours ce qu’ils voient le mieux. La justice élève sa voix ; mais elle a peine à se faire entendre dans le tumulte des passions.\par
Les hommes peuvent faire des injustices, parce qu’ils ont intérêt à les commettre, et qu’ils aiment mieux se satisfaire que les autres. C’est toujours par un retour sur eux-mêmes qu’ils agissent : nul n’est mauvais gratuitement ; il faut qu’il y ait une raison qui détermine, et cette raison est toujours une raison d’intérêt.\par
Mais il n’est pas possible que Dieu fasse jamais rien d’injuste : dès qu’on suppose qu’il voit la justice, il faut nécessairement qu’il la suive ; car, comme il n’a besoin de rien, et qu’il se suffit à lui-même, il serait le plus méchant de tous les êtres, puisqu’il le serait sans intérêt.\par
Ainsi, quand il n’y aurait pas de Dieu, nous devrions toujours aimer la justice ; c’est-à-dire faire nos efforts pour ressembler à cet être dont nous avons une si belle idée, et qui, s’il existait, serait nécessairement juste. Libres que nous serions du joug de la religion, nous ne devrions pas l’être de celui de l’équité.\par
Voilà, Rhédi, ce qui m’a fait penser que la justice est éternelle, et ne dépend point des conventions humaines ; et quand elle en dépendrait, ce serait une vérité terrible qu’il faudrait se dérober à soi-même.\par
Nous sommes entourés d’hommes plus forts que nous ; ils peuvent nous nuire de mille manières différentes, les trois quarts du temps ils peuvent le faire impunément : quel repos pour nous de savoir qu’il y a dans le cœur de tous ces hommes un principe intérieur qui combat en notre faveur, et nous met à couvert de leurs entreprises !\par
Sans cela nous devrions être dans une frayeur continuelle ; nous passerions devant les hommes comme devant les lions ; et nous ne serions jamais assurés un moment de notre vie, de notre bien, ni de notre honneur.\par
Toutes ces pensées m’animent contre ces docteurs qui représentent Dieu comme un être qui fait un exercice tyrannique de sa puissance ; qui le font agir d’une manière dont nous ne voudrions pas agir nous-mêmes, de peur de l’offenser ; qui le chargent de toutes les imperfections qu’il punit en nous ; et, dans leurs opinions contradictoires, le représentent tantôt comme un être mauvais, tantôt comme un être qui hait le mal et le punit.\par
Quand un homme s’examine, quelle satisfaction pour lui de trouver qu’il a le cœur juste ! Ce plaisir, tout sévère qu’il est, doit le ravir : il voit son être autant au-dessus de ceux qui ne l’ont pas qu’il se voit au-dessus des tigres et des ours. Oui, Rhédi, si j’étais sûr de suivre toujours inviolablement cette équité que j’ai devant les yeux, je me croirais le premier des hommes.\par

\dateline{À Paris, le 1er de la lune de Gemmadi 1, 1715.}
\section[{Lettre LXXXV. Rica à ***.}]{Lettre LXXXV. Rica à ***.}\renewcommand{\leftmark}{Lettre LXXXV. Rica à ***.}

\noindent Je fus hier aux Invalides : j’aimerais autant avoir fait cet établissement, si j’étais prince, que d’avoir gagné trois batailles. On y trouve partout la main d’un grand monarque. Je crois que c’est le lieu le plus respectable de la terre.\par
Quel spectacle que de voir dans un même lieu rassemblé toutes ces victimes de la patrie, qui ne respirent que pour la défendre ; et qui, se sentant le même cœur, et non pas la même force, ne se plaignent que de l’impuissance où elles sont de se sacrifier encore pour elle !\par
Quoi de plus admirable que de voir ces guerriers débiles, dans cette retraite, observer une discipline aussi exacte que s’ils étaient contraints par la présence d’un ennemi, chercher leur dernière satisfaction dans cette image de la guerre, et partager leur cœur et leur esprit entre les devoirs de la religion et ceux de l’art militaire !\par
Je voudrais que les noms de ceux qui meurent pour la patrie fussent écrits et conservés dans les temples, dans des registres qui fussent comme la source de la gloire et de la noblesse.\par

\dateline{À Paris le 15 de la lune de Gemmadi 1, 1715.}
\section[{Lettre LXXXVI. Usbek à Mirza. À Ispahan.}]{Lettre LXXXVI. Usbek à Mirza. À Ispahan.}\renewcommand{\leftmark}{Lettre LXXXVI. Usbek à Mirza. À Ispahan.}

\noindent Tu sais Mirza, que quelques ministres de Cha-Soliman avaient formé le dessein d’obliger tous les Arméniens de Perse de quitter le royaume, ou de se faire mahométans, dans la pensée que notre empire serait toujours pollué, tandis qu’il garderait dans son sein ces infidèles.\par
C’était fait de la grandeur persane, si dans cette occasion l’aveugle dévotion avait été écoutée.\par
On ne sait comment la chose manqua ; ni ceux qui firent la proposition, ni ceux qui la rejetèrent, n’en connurent les conséquences : le hasard fit l’office de la raison et de la politique, et sauva l’empire d’un péril plus grand que celui qu’il aurait pu courir de la perte de trois batailles et de la prise de deux villes.\par
En proscrivant les Arméniens, on pensa détruire en un seul jour tous les négociants, et presque tous les artisans du royaume. Je suis sûr que le grand Cha-Abas aurait mieux aimé se faire couper les deux bras que de signer un ordre pareil, et qu’en envoyant au Mogol et aux autres rois des Indes ses sujets les plus industrieux, il aurait cru leur donner la moitié de ses États.\par
Les persécutions que nos mahométans zélés ont faites aux Guèbres les ont obligés de passer en foule dans les Indes ; et ont privé la Perse de cette laborieuse nation, si appliquée au labourage, qui seule, par son travail, était en état de vaincre la stérilité de nos terres.\par
Il ne restait à la dévotion qu’un second coup à faire : c’était de ruiner l’industrie ; moyennant quoi l’empire tombait de lui-même, et avec lui, par une suite nécessaire, cette même religion qu’on voulait rendre si florissante.\par
S’il faut résonner sans prévention, je ne sais, Mirza, s’il n’est pas bon que dans un État il y ait plusieurs religions.\par
On remarque que ceux qui vivent dans des religions tolérées, se rendent ordinairement plus utiles à leur patrie que ceux qui vivent dans la religion dominante ; parce que, éloignés des honneurs, ne pouvant se distinguer que par leur opulence et leur richesses, ils sont portés à en acquérir par leur travail, et à embrasser les emplois de la société les plus pénibles.\par
D’ailleurs, comme toutes les religions contiennent des préceptes utiles à la société, il est bon qu’elles soient observées avec zèle. Or qu’y a-t-il de plus capable d’animer ce zèle que leur multiplicité ?\par
Ce sont des rivales qui ne se pardonnent rien. La jalousie descend jusque aux particuliers : chacun se tient sur ces gardes, et craint de faire des choses qui déshonoreraient son parti, et l’exposeraient aux mépris et aux censures impardonnables du parti contraire.\par
Aussi a-t-on toujours remarqué qu’une secte nouvelle introduite dans un État était le moyen le plus sûr pour corriger les abus de l’ancienne.\par
On a beau dire qu’il n’est pas dans l’intérêt du prince de souffrir plusieurs religions dans son État. Quand toutes les sectes du monde viendraient s’y rassembler, cela ne lui porterait aucun préjudice ; parce qu’il n’y en a aucune qui ne prescrive l’obéissance et ne prêche la soumission.\par
J’avoue que les histoires sont remplies de guerres de religion : mais qu’on y prenne bien garde, ce n’est point la multiplicité des religions qui a produit ces guerres, c’est l’esprit d’intolérance qui animait celle qui se croyait la dominante.\par
C’est cet esprit de prosélytisme, que les Juifs ont pris aux Égyptiens, et qui d’eux est passé, comme une maladie épidémique et populaire, aux mahométans et aux chrétiens.\par
C’est enfin cet esprit de vertige, dont les progrès ne peuvent être regardés que comme une éclipse entière de la raison humaine.\par
Car enfin, quand il n’y aurait pas de l’inhumanité à affliger la conscience des autres, quand il n’en résulterait aucun des mauvais effets qui en germent à milliers, il faudrait être fou pour s’en aviser. Celui qui veut me faire changer de religion ne le fait sans doute que parce qu’il ne changerait pas la sienne quand on voudrait l’y forcer : il trouve donc étrange que je ne fasse pas une chose qu’il ferait lui-même, peut-être pour l’empire du monde.\par

\dateline{À Paris, le 26 de la lune de Gemmadi 1, 1715.}
\section[{Lettre LXXXVII. Rica à ***.}]{Lettre LXXXVII. Rica à ***.}\renewcommand{\leftmark}{Lettre LXXXVII. Rica à ***.}

\noindent Il semble ici que les familles se gouvernent toutes seules : le mari n’a qu’une ombre d’autorité sur sa femme, le père sur ces enfants, le maître sur ces esclaves ; la justice se mêle de tous leurs différends ; et sois sûr qu’elle est toujours contre le mari jaloux, le père chagrin, le maître incommode.\par
J’allai l’autre jour dans le lieu où se rend la justice. Avant que d’y d’arriver, il faut passer sous les armes d’un nombre infini de jeunes marchandes, qui vous appellent d’une voix trompeuse. Ce spectacle d’abord est assez riant ; mais il devient lugubre lorsqu’on entre dans les grandes salles, où l’on ne voit que des gens dont l’habit est encore plus grave que la figure. Enfin on entre dans le lieu sacré où se révèlent tous les secrets des familles, et où les actions les plus cachées sont mises au grand jour.\par
Là, une fille modeste vient avouer les tourments d’une virginité trop longtemps gardée, ses combats, et sa douloureuse résistance : elle est si peu fière de sa victoire, qu’elle menace toujours d’une défaite prochaine ; et pour que son père n’ignore plus ses besoins, elle les expose à tout le peuple.\par
Une femme effrontée vient ensuite exposer les outrages qu’elle a faits à son époux, comme une raison d’en être séparée.\par
Avec une modestie pareille, une autre vient dire qu’elle est lasse de porter le titre de femme sans en jouir : elle vient révéler les mystères cachés dans la nuit du mariage ; elle veut qu’on la livre aux regards des experts les plus habiles, et qu’une sentence la rétablisse dans tous les droits de la virginité. Il y en a même qui osent défier leurs maris, et leur demander en public un combat que les témoins rendent si difficile : épreuve aussi flétrissante pour la femme qui la soutient que pour le mari qui y succombe.\par
Un nombre infini de filles ravies ou séduites font les hommes beaucoup plus mauvais qu’ils ne le sont. L’amour fait retentir ce tribunal : on y entend parler que des pères irrités, des filles abusées, d’amants infidèles, et de maris chagrins.\par
Par la loi qui y est observée, tout enfant né pendant le mariage est censé être au mari : il a beau avoir de bonnes raisons pour ne le pas croire ; la loi le croit pour lui, et le soulage de l’examen et des scrupules.\par
Dans ce tribunal, on prend les voix à la majeure ; mais on a reconnu par expérience qu’il vaudrait mieux les recueillir à la mineure : et cela est bien naturel ; car il y a très peu d’esprits justes, et tout le monde convient qu’il y en a une infinité de faux.\par

\dateline{À Paris, le 1er de la lune de Gemmadi 2, 1715.}
\section[{Lettre LXXXVIII. Rica à ***.}]{Lettre LXXXVIII. Rica à ***.}\renewcommand{\leftmark}{Lettre LXXXVIII. Rica à ***.}

\noindent On dit que l’homme est un animal sociable. Sur ce pied-là, il me paraît que le Français est plus homme qu’un autre, c’est l’homme par excellence, car il semble fait uniquement pour la société.\par
Mais j’ai remarqué des gens parmi eux qui non seulement sont sociables, mais sont eux-mêmes la société universelle. Ils se multiplient dans tous les coins, et peuplent en un instant les quatre quartiers d’une ville : cent hommes de cette espèce abondent plus que deux mille citoyens ; ils pourraient réparer aux yeux des étrangers les ravages de la peste ou de la famine. On demande dans les écoles si un corps peut être en un instant en plusieurs lieux ; ils sont une preuve de ce que les philosophes mettent en question.\par
Ils sont toujours empressés, parce qu’ils ont l’affaire importante de demander à tous ceux qu’ils rencontrent où ils vont et d’où ils viennent.\par
On ne leur ôterait jamais de la tête qu’il est de la bienséance de visiter chaque jour le public en détail, sans compter les visites qu’ils font en gros dans les lieux où l’on s’assemble ; mais, comme la vie en est trop abrégée, elles sont comptées pour rien dans les règles de leur cérémonial.\par
Ils fatiguent plus les portes des maisons à coups de marteau, que les vents et les tempêtes. Si l’on allait examiner la liste des portiers, on y trouverait chaque jour leur nom estropié de mille manières en caractères suisses. Ils passent leur vie à la suite d’un enterrement, dans des compliments de condoléance, ou dans des sollicitations de mariage. Le roi ne fait point de gratification à quelqu’un de ses sujets, qu’il ne lui en coûte une voiture pour lui en témoigner leur joie. Enfin, ils reviennent chez eux, bien fatigués, se reposer, pour pouvoir reprendre le lendemain leurs pénibles fonctions.\par
Un d’eux mourut l’autre jour de lassitude, et on mit cette épitaphe sur son tombeau : "C’est ici que repose celui qui ne s’est jamais reposé. Il s’est promené à cinq cent trente enterrements. Il s’est réjoui de la naissance de deux mille six cent quatre-vingts enfants. Les pensions dont il a félicité ses amis, toujours en des termes différents, montent à deux millions six cent mille livres ; le chemin qu’il a fait sur le pavé, à neuf mille six cent stades ; celui qu’il a fait dans la campagne, à trente-six. Sa conversation était amusante ; il avait un fonds tout fait de trois cent soixante-cinq contes : il possédait d’ailleurs, depuis son jeune âge, cent dix-huit apophtegmes tirés des anciens, qu’il employait dans les occasions brillantes. Il est mort enfin à la soixantième année de son âge. Je me tais, voyageur ; car comment pourrais-je achever de te dire ce qu’il a fait et ce qu’il a vu ?"\par

\dateline{À Paris, le 3 de la lune de Gemmadi 2, 1715.}
\section[{Lettre LXXXIX. Usbek à Rhédi. À Venise.}]{Lettre LXXXIX. Usbek à Rhédi. À Venise.}\renewcommand{\leftmark}{Lettre LXXXIX. Usbek à Rhédi. À Venise.}

\noindent À Paris règnent la liberté et l’égalité. La naissance, la vertu, le mérite même de la guerre, quelque brillant qu’il soit, ne sauvent pas un homme de la foule dans laquelle il est confondu. La jalousie des rangs y est inconnue. On dit que le premier de Paris est celui qui a les meilleurs chevaux à son carrosse.\par
Un grand seigneur est un homme qui voit le roi, qui parle aux ministres, qui a des ancêtres, des dettes et des pensions. S’il peut avec cela cacher son oisiveté par un air empressé, ou par un feint attachement pour les plaisirs, il croit être le plus heureux de tous les hommes.\par
En Perse, il n’y a de grands que ceux à qui le monarque donne quelque part au gouvernement. Ici, il y a des gens qui sont grands par leur naissance ; mais ils sont sans crédit. Les rois font comme ces ouvriers habiles qui, pour exécuter leurs ouvrages, se servent toujours des machines les plus simples.\par
La faveur est la grande divinité des Français. Le ministre est le grand prêtre, qui lui offre bien des victimes. Ceux qui l’entourent ne sont point habillés de blanc : tantôt sacrificateurs, et tantôt sacrifiés, ils se dévouent eux-mêmes à leur idole avec tout le peuple.\par

\dateline{À Paris, le 9 de la lune de Gemmadi 2, 1715.}
\section[{Lettre XC. Usbek à Ibben. À Smyrne.}]{Lettre XC. Usbek à Ibben. À Smyrne.}\renewcommand{\leftmark}{Lettre XC. Usbek à Ibben. À Smyrne.}

\noindent Le désir de la gloire n’est point différent de cet instinct que toutes les créatures ont pour leur conservation. Il semble que nous augmentons notre être, lorsque nous pouvons le porter dans la mémoire des autres : c’est une nouvelle vie que nous acquérons, et qui nous devient aussi précieuse que celle que nous avons reçue du ciel.\par
Mais comme tous les hommes ne sont pas également attachés à la vie, ils ne sont pas aussi également sensibles à la gloire. Cette noble passion est bien toujours gravée dans leur cœur ; mais l’imagination et l’éducation la modifient de mille manières.\par
Cette différence, qui se trouve d’homme à homme, se fait encore plus sentir de peuple à peuple.\par
On peut poser pour maxime que, dans chaque État, le désir de la gloire croît avec la liberté des sujets, et diminue avec elle : la gloire n’est jamais compagne de la servitude.\par
Un homme de bon sens me disait l’autre jour : On est en France, à bien des égards, plus libre qu’en Perse ; aussi y aime-t-on plus la gloire. Cette heureuse fantaisie fait faire à un Français, avec plaisir et avec goût, ce que votre sultan n’obtient de ses sujets qu’en leur mettant sans cesse devant les yeux les supplices et les récompenses.\par
Aussi, parmi nous, le prince est-il jaloux de l’honneur du dernier de ses sujets. Il y a pour le maintenir des tribunaux respectables : c’est le trésor sacré de la nation, et le seul dont le souverain n’est pas le maître, parce qu’il ne peut l’être sans choquer ses intérêts. Ainsi, si un sujet se trouve blessé dans son honneur par son prince, soit par quelque préférence, soit par la moindre marque de mépris, il quitte sur-le-champ sa cour, son emploi, son service, et se retire chez lui.\par
La différence qu’il y a des troupes françaises aux vôtres, c’est que les unes, composées d’esclaves naturellement lâches, ne surmontent la crainte de la mort que par celle du châtiment ; ce qui produit dans l’âme un nouveau genre de terreur qui la rend comme stupide : au lieu que les autres se présentent aux coups avec délice, et bannissent la crainte par une satisfaction qui lui est supérieure.\par
Mais le sanctuaire de l’honneur, de la réputation et de la vertu, semble être établi dans les républiques, et dans les pays où l’on peut prononcer le mot de patrie. À Rome, à Athènes, à Lacédémone, l’honneur payait seul les services les plus signalés. Une couronne de chêne ou de laurier, une statue, un éloge, était une récompense immense pour une bataille gagnée ou une ville prise.\par
Là, un homme qui avait fait une belle action se trouvait suffisamment récompensé par cette action même. Il ne pouvait voir un de ses compatriotes qu’il ne ressentît le plaisir d’être son bienfaiteur ; il comptait le nombre de ses services par celui de ses concitoyens. Tout homme est capable de faire du bien à un homme : mais c’est ressembler aux dieux que de contribuer au bonheur d’une société entière.\par
Mais cette noble émulation ne doit-elle point être entièrement éteinte dans le cœur de vos Persans, chez qui les emplois et les dignités ne sont que des attributs de la fantaisie du souverain ? La réputation et la vertu y sont regardées comme imaginaires, si elles ne sont accompagnées de la faveur du prince, avec laquelle elles naissent et meurent de même. Un homme qui a pour lui l’estime publique n’est jamais sûr de ne pas être déshonoré demain : le voilà aujourd’hui général d’armée ; peut-être que le prince le va faire son cuisinier, et qu’il n’aura plus à espérer d’autre éloge que celui d’avoir fait un bon ragoût.\par

\dateline{À Paris, le 15 de la lune de Gemmadi 2, 1715.}
\section[{Lettre XCI. Usbek au même. À Smyrne.}]{Lettre XCI. Usbek au même. À Smyrne.}\renewcommand{\leftmark}{Lettre XCI. Usbek au même. À Smyrne.}

\noindent De cette passion générale que la nation française a pour la gloire, il s’est formé dans l’esprit des particuliers un certain je ne sais quoi qu’on appelle point d’honneur : c’est proprement le caractère de chaque profession ; mais il est plus marqué chez les gens de guerre, et c’est le point d’honneur par excellence. Il me serait bien difficile de te faire sentir ce que c’est ; car nous n’en avons point précisément d’idée.\par
Autrefois les Français, surtout les nobles, ne suivaient guère d’autres lois que celles de ce point d’honneur : elles réglaient toute la conduite de leur vie ; et elles étaient si sévères qu’on ne pouvait, sans une peine plus cruelle que la mort, je ne dis pas les enfreindre, mais en éluder la plus petite disposition.\par
Quand il s’agissait de régler les différends, elles ne prescrivaient guère qu’une manière de décision, qui tranchait toutes les difficultés ; mais ce qu’il avait de mal, c’est que souvent le jugement se rendait entre d’autres parties que celles qui y étaient intéressées.\par
Pour peu qu’un homme fût connu d’un autre, il fallait qu’il entrât dans la dispute, et qu’il payât de sa personne, comme s’il avait été lui-même en colère. Il se sentait toujours honoré d’un tel choix et d’une préférence si flatteuse ; et tel qui n’aurait pas voulu donner quatre pistoles à un homme pour le sauver de la potence, lui et toute sa famille, ne faisait aucune difficulté d’aller risquer pour lui mille fois sa vie.\par
Cette manière de décider était assez mal imaginée, car de ce qu’un homme était plus adroit ou plus fort qu’un autre, il ne s’ensuivait pas qu’il eût de meilleures raisons.\par
Aussi les rois l’ont-ils défendue sous des peines très sévères ; mais c’est en vain : l’honneur, qui veut toujours régner, se révolte, et il ne reconnaît point de lois.\par
Ainsi les Français sont dans un état bien violent : car les mêmes lois de l’honneur obligent un honnête homme de se venger quand il a été offensé ; mais, d’un autre côté, la justice le punit des plus cruelles peines lorsqu’il se venge. Si l’on suit les lois de l’honneur, on périt sur un échafaud : si l’on suit celles de la justice, on est banni pour jamais de la société des hommes : il n’y a donc que cette cruelle alternative, ou de mourir, ou d’être indigne de vivre.\par

\dateline{À Paris, le 18 de la lune de Gemmadi 2, 1715.}
\section[{Lettre XCII. Usbek à Rustan. À Ispahan.}]{Lettre XCII. Usbek à Rustan. À Ispahan.}\renewcommand{\leftmark}{Lettre XCII. Usbek à Rustan. À Ispahan.}

\noindent Il paraît ici un personnage travesti en ambassadeur de Perse, qui se joue insolemment des deux plus grands rois du monde. Il apporte au monarque des Français des présents que le nôtre ne saurait donner à un roi d’Irimette ou de Géorgie ; et, par sa lâche avarice, il a flétri la majesté des deux empires.\par
Il s’est rendu ridicule devant un peuple qui prétend être le plus poli de l’Europe, et il a fait dire en Occident que le roi des rois ne domine que sur des barbares.\par
Il a reçu des honneurs qu’il semblait avoir voulu se faire refuser lui-même : et, comme si la cour de France avait eu plus à cœur la grandeur persane que lui, elle l’a fait paraître avec dignité devant un peuple dont il est le mépris.\par
Ne dis point ceci à Ispahan : épargne la tête d’un malheureux. Je ne veux pas que nos ministres le punissent de leur propre imprudence, et de l’indigne choix qu’ils ont fait.\par

\dateline{De Paris, le dernier de la lune de Gemmadi 2, 1715.}
\section[{Lettre XCIII. Usbek à Rhédi. À Venise.}]{Lettre XCIII. Usbek à Rhédi. À Venise.}\renewcommand{\leftmark}{Lettre XCIII. Usbek à Rhédi. À Venise.}

\noindent Le monarque qui a si longtemps régné n’est plus\footnote{Il mourut le 1\textsuperscript{er} septembre 1715.}. Il a bien fait parler des gens pendant sa vie ; tout le monde s’est tu à sa mort. Ferme et courageux dans ce dernier moment, il a paru ne céder qu’au destin. Ainsi mourut le grand Cha-Abas, après avoir rempli toute la terre de son nom.\par
Ne crois pas que ce grand événement n’ait fait faire ici que des réflexions morales. Chacun a pensé à ses affaires, et à prendre ses avantages dans ce changement. Le roi, arrière-petit-fils du monarque défunt, n’ayant que cinq ans, un prince son oncle a été déclaré régent du royaume.\par
Le feu roi avait fait un testament qui bornait l’autorité du régent. Ce prince habile a été au parlement : et, y exposant tous les droits de sa naissance, il a fait casser la disposition du monarque, qui voulant se survivre à lui-même, semblait avoir prétendu régner encore après sa mort.\par
Les parlements ressemblent à ces ruines que l’on foule aux pieds, mais qui rappellent toujours l’idée de quelque temple fameux par l’ancienne religion des peuples. Ils ne se mêlent guère plus que de rendre la justice ; et leur autorité est toujours languissante, à moins que quelque conjoncture imprévue ne vienne lui rendre la force et la vie. Ces grands corps ont suivi le destin des choses humaines : ils ont cédé au temps, qui détruit tout ; à la corruption des mœurs, qui a tout affaibli ; à l’autorité suprême, qui a tout abattu.\par
Mais le régent, qui a voulu se rendre agréable au peuple, a paru d’abord respecter cette image de la liberté publique ; et, comme s’il avait pensé à relever de terre le temple et l’idole, il a voulu qu’on les regardât comme l’appui de la monarchie et le fondement de toute autorité légitime.\par

\dateline{À Paris, le 4 de la lune de Rhégeb, 1715.}
\section[{Lettre XCIV. Usbek à son frère, Santon au monastère de Casbin.}]{Lettre XCIV. Usbek à son frère, Santon au monastère de Casbin.}\renewcommand{\leftmark}{Lettre XCIV. Usbek à son frère, Santon au monastère de Casbin.}

\noindent Je m’humilie devant toi, sacré santon, et je me prosterne : je regarde les vestiges de tes pieds comme la prunelle de mes yeux. Ta sainteté est si grande, qu’il semble que tu aies le cœur de notre saint prophète ; tes austérités étonnent le ciel même ; les anges t’ont regardé du sommet de la gloire, et on dit : Comment est-il encore sur la terre, puisque son esprit est avec nous, et vole autour du trône qui est soutenu par les nuées ?\par
Et comment ne t’honorerais-je pas, moi qui ai appris de nos docteurs que les dervis, même infidèles, ont toujours un caractère de sainteté qui les rend respectables aux vrais croyants : et que Dieu s’est choisi dans tous les coins de la terre des âmes plus pures que les autres, qu’il a séparées du monde impie, afin que leurs mortifications et leurs prières ferventes suspendissent sa colère prête à tomber sur tant de peuples rebelles ?\par
Les chrétiens disent des merveilles de leurs premiers santons, qui se réfugièrent à milliers dans les déserts affreux de la Thébaïde, et eurent pour chefs Paul, Antoine et Pacôme. Si ce qu’ils en disent est vrai, leurs vies sont aussi pleines de prodiges que celles de nos plus sacrés immaums. Ils passaient quelquefois dix ans entiers sans voir un seul homme : mais ils habitaient la nuit et le jour avec des démons ; ils étaient sans cesse tourmentés par ces esprits malins ; ils les trouvaient au lit, ils les trouvaient à table ; jamais d’asile contre eux. Si tout ceci est vrai, santon vénérable, il faudrait avouer que personne n’aurait jamais vécu en plus mauvaise compagnie.\par
Les chrétiens sensés regardent toutes ces histoires comme une allégorie bien naturelle, qui nous peut servir à nous faire sentir le malheur de la condition humaine. En vain cherchons-nous dans le désert un état tranquille ; les tentations nous suivent toujours : nos passions, figurées par les démons, ne nous quittent point encore ; ces monstres du cœur, ces illusions de l’esprit, ces vains fantômes de l’erreur et du mensonge, se montrent toujours à nous pour nous séduire, et nous attaquent jusque dans les jeûnes et les cilices, c’est-à-dire jusque dans notre force même.\par
Pour moi, santon vénérable, je sais que l’envoyé de Dieu a enchaîné Satan, et l’a précipité dans les abîmes : il a purifié la terre, autrefois pleine de son empire, et l’a rendue digne du séjour des anges et des prophètes.\par

\dateline{À Paris, le 9 de la lune de Chahban, 1715.}
\section[{Lettre XCV. Usbek à Rhédi. À Venise.}]{Lettre XCV. Usbek à Rhédi. À Venise.}\renewcommand{\leftmark}{Lettre XCV. Usbek à Rhédi. À Venise.}

\noindent Je n’ai jamais ouï parler du droit public, qu’on n’ait commencé par rechercher soigneusement quelle est l’origine des sociétés ; ce qui me paraît ridicule. Si les hommes n’en formaient point, s’ils se quittaient et se fuyaient les uns les autres, il faudrait en demander la raison, et chercher pourquoi ils se tiennent séparés : mais ils naissent tous liés les uns aux autres ; un fils est né auprès de son père, et il s’y tient : voilà la société, et la cause de la société.\par
Le droit public est plus connu en Europe qu’en Asie : cependant on peut dire que les passions des princes, la patience des peuples, la flatterie des écrivains, en ont corrompu tous les principes.\par
Ce droit, tel qu’il est aujourd’hui, est une science qui apprend aux princes jusqu’à quel point ils peuvent violer la justice sans choquer leurs intérêts. Quel dessein, Rhédi, de vouloir, pour endurcir leur conscience, mettre l’iniquité en système, d’en donner des règles, d’en former des principes, et d’en tirer des conséquences !\par
La puissance illimitée de nos sublimes sultans, qui n’a d’autre règle qu’elle-même, ne produit pas plus de monstres que cet art indigne qui veut faire plier la justice, tout inflexible qu’elle est.\par
On dirait, Rhédi, qu’il y a deux justices toutes différentes : l’une qui règle les affaires des particuliers, qui règne dans le droit civil ; l’autre qui règle les différends qui surviennent de peuple à peuple, qui tyrannise dans le droit public : comme si le droit public n’était pas lui-même un droit civil, non pas à la vérité d’un pays particulier, mais du monde.\par
Je t’expliquerai dans une autre lettre mes pensées là-dessus.\par

\dateline{À Paris, le 1er de la lune de Zilhagé, 1716.}
\section[{Lettre XCVI. Usbek au même.}]{Lettre XCVI. Usbek au même.}\renewcommand{\leftmark}{Lettre XCVI. Usbek au même.}

\noindent Les magistrats doivent rendre la justice de citoyen à citoyen : chaque peuple la doit rendre lui-même de lui à un autre peuple. Dans cette seconde distribution de justice, on ne peut employer d’autres maximes que dans la première.\par
De peuple à peuple, il est rarement besoin de tiers pour juger, parce que les sujets de disputes sont presque toujours clairs et faciles à terminer. Les intérêts de deux nations sont ordinairement si séparés, qu’il ne faut qu’aimer la justice pour la trouver : on ne peut guère se prévenir dans sa propre cause.\par
Il n’en est pas de même des différends qui arrivent entre particuliers. Comme ils vivent en société, leurs intérêts sont si mêlés et si confondus, il y en a tant de sortes différentes, qu’il est nécessaire qu’un tiers débrouille ce que la cupidité des parties cherche à obscurcir.\par
Il n’y a que deux sortes de guerres justes : les unes qui se font pour repousser un ennemi qui attaque ; les autres pour secourir un allié qui est attaqué.\par
Il n’y aurait point de justice de faire la guerre pour des querelles particulières du prince, à moins que le cas ne fût si grave qu’il méritât la mort du prince, ou du peuple qui l’a commis. Ainsi un prince ne peut fait la guerre parce qu’on lui aura refusé un honneur qui lui est dû, ou parce qu’on aura eu quelque procédé peu convenable à l’égard de ses ambassadeurs, et autres choses pareilles ; non plus qu’un particulier ne peut tuer celui qui lui refuse le pas. La raison en est que, comme la déclaration de guerre doit être un acte de justice, dans lequel il faut toujours que la peine soit proportionnée à la faute, il faut voir si celui à qui on déclare la guerre mérite la mort. Car faire la guerre à quelqu’un, c’est vouloir le punir de mort.\par
Dans le droit public, l’acte de justice le plus sévère c’est la guerre : puisque son but est la destruction de la société.\par
Les représailles sont du second degré. C’est une loi que les tribunaux n’ont pu s’empêcher d’observer, de mesurer la peine par le crime.\par
Un troisième acte de justice est de priver un prince des avantages qu’il peut tirer de nous, proportionnant toujours la peine à l’offense.\par
Le quatrième acte de justice, qui doit être le plus fréquent, est la renonciation à l’alliance du peuple dont on a à se plaindre. Cette peine répond à celle du bannissement établi dans les tribunaux, qui retranche les coupables de la société. Ainsi un prince à l’alliance duquel nous renonçons est retranché par là de notre société, et n’est plus un de nos membres.\par
On ne peut pas faire de plus grand affront à un prince que de renoncer à son alliance, ni lui faire de plus grand honneur que de la contracter. Il n’y a rien parmi les hommes qui leur soit plus glorieux, et même plus utile, que d’en voir d’autres toujours attentifs à leur conservation.\par
Mais pour que l’alliance nous lie, il faut qu’elle soit juste : ainsi une alliance faite entre deux nations pour en opprimer une troisième n’est pas légitime, et on peut la violer sans crime.\par
Il n’est pas même de l’honneur et de la dignité du prince de s’allier avec un tyran. On dit qu’un monarque d’Égypte fit avertir le roi de Samos de sa cruauté et de sa tyrannie, et le somma de s’en corriger : comme il ne le fit pas, il lui envoya dire qu’il renonçait à son amitié et à son alliance.\par
La conquête ne donne point un droit par elle-même. Lorsque le peuple subsiste, elle est un gage de la paix et de la réparation du tort ; et, si le peuple est détruit ou dispersé, elle est le monument d’une tyrannie.\par
Les traités de paix sont si sacrés parmi les hommes, qu’il semble qu’ils soient la voix de la nature, qui réclame ses droits. Ils sont tous légitimes, lorsque les conditions en sont telles que les deux peuples peuvent se conserver : sans quoi, celle des deux sociétés qui doit périr, privée de sa défense naturelle par la paix, la peut chercher dans la guerre.\par
Car la nature, qui a établi les différents degrés de force et de faiblesse parmi les hommes, a encore souvent égalé la faiblesse à la force par le désespoir.\par

\dateline{À Paris, le 4 de la lune de Zilhagé, 1716.}
\section[{Lettre XCVII. Le premier eunuque à Usbek. À Paris.}]{Lettre XCVII. Le premier eunuque à Usbek. À Paris.}\renewcommand{\leftmark}{Lettre XCVII. Le premier eunuque à Usbek. À Paris.}

\noindent Il est arrivé ici beaucoup de femmes jaunes du royaume de Visapour : j’en ai acheté une pour ton frère le gouverneur de Mazenderan, qui m’envoya il y a un mois son commandement sublime et cent tomans.\par
Je me connais en femmes, d’autant mieux qu’elles ne me surprennent pas, et qu’en moi les yeux ne sont point troublés par les mouvements du cœur.\par
Je n’ai jamais vu de beauté si régulière et si parfaite : ses yeux brillants portent la vie sur son visage, et relèvent l’éclat d’une couleur qui pourrait effacer tous les charmes de la Circassie.\par
Le premier eunuque d’un négociant d’Ispahan la marchandait avec moi ; mais elle se dérobait dédaigneusement à ses regards, et semblait chercher les miens, comme si elle avait voulu me dire qu’un vil marchand n’était pas digne d’elle, et qu’elle était destinée à un plus illustre époux.\par
Je te l’avoue, je sens dans moi-même une joie secrète quand je pense aux charmes de cette belle personne : il me semble que je la vois entrer dans le sérail de ton frère ; je me plais à prévoir l’étonnement de toutes ses femmes ; la douleur impérieuse des unes ; l’affliction muette, mais plus douloureuse, des autres ; la consolation maligne de celles qui n’espèrent plus rien, et l’ambition irritée de celles qui espèrent encore.\par
Je vais d’un bout du royaume à l’autre faire changer tout un sérail de face. Que de passions je vais émouvoir ! Que de craintes et de peines je prépare !\par
Cependant, dans le trouble de dedans, le dehors ne sera pas moins tranquille : les grandes révolutions seront cachées dans le fond du cœur ; les chagrins seront dévorés, et les joies contenues ; l’obéissance ne sera pas moins exacte, et les règles moins inflexibles ; la douleur, toujours contrainte de paraître, sortira du fond même de désespoir.\par
Nous remarquons que, plus nous avons de femmes sous nos yeux, moins elles nous donnent d’embarras. Une plus grande nécessité de plaire, moins de facilité de s’unir, plus d’exemples de soumission, tout cela leur forme des chaînes. Les unes sont sans cesse attentives sur les démarches des autres : il semble que, de concert avec nous, elles travaillent à se rendre plus dépendantes ; elles font presque la moitié de notre office, et nous ouvrent les yeux quand nous les fermons. Que dis-je ? elles irritent sans cesse le maître contre leurs rivales ; et elles ne voient pas combien elles se trouvent près de celles qu’on punit.\par
Mais tout cela, magnifique seigneur, tout cela n’est rien sans la présence du maître. Que pouvons-nous faire avec ce vain fantôme d’une autorité qui ne se communique jamais tout entière ? Nous ne représentons que faiblement la moitié de toi-même : nous ne pouvons que leur montrer une odieuse sévérité. Toi, tu tempères la crainte par les espérances : plus absolu quand tu caresses, que tu ne l’es quand tu menaces.\par
Reviens donc, magnifique seigneur, reviens dans ces lieux porter partout les marques de ton empire. Viens adoucir des passions désespérées : viens ôter tout prétexte de faillir ; viens apaiser l’amour qui murmure, et rendre le devoir même aimable ; viens enfin soulager tes fidèles eunuques d’un fardeau qui s’appesantit chaque jour.\par

\dateline{Du sérail d’Ispahan, le 8 de la lune de Zilhagé, 1716.}
\section[{Lettre XCVIII. Usbek à Hassein, dervis de la montagne de Jaron.}]{Lettre XCVIII. Usbek à Hassein, dervis de la montagne de Jaron.}\renewcommand{\leftmark}{Lettre XCVIII. Usbek à Hassein, dervis de la montagne de Jaron.}

\noindent Ô toi, sage dervis, dont l’esprit curieux brille de tant de connaissances, écoute ce que je vais te dire.\par
Il y a ici des philosophes qui, à la vérité n’ont point atteint jusqu’au faîte de la sagesse orientale : ils n’ont point été ravis jusqu’au trône lumineux ; ils n’ont, ni entendu les paroles ineffables dont les concerts des anges retentissent, ni senti les formidables accès d’une fureur divine : mais, laissés à eux-mêmes, privés de saintes merveilles, ils suivent dans le silence les traces de la raison humaine.\par
Tu ne saurais croire jusqu’où ce guide les a conduits. Ils ont débrouillé le chaos ; et ont expliqué, par une mécanique simple, l’ordre de l’architecture divine. L’auteur de la nature a donné du mouvement à la matière : il n’en a pas fallu davantage pour produire cette prodigieuse variété d’effets, que nous voyons dans l’univers.\par
Que les législateurs ordinaires nous proposent des lois pour régler les sociétés des hommes ; des lois aussi sujettes au changement que l’esprit de ceux qui les proposent et des peuples qui les observent : ceux-ci ne nous parlent que de lois générales, immuables, éternelles, qui s’observent sans aucune exception, avec un ordre, une régularité, et une promptitude infinie, dans l’immensité des espaces.\par
Et que crois-tu, homme divin, que soient ces lois ? Tu t’imagines peut-être qu’entrant dans le conseil de l’Éternel, tu vas être étonné par la sublimité des mystères : tu renonces par avance à comprendre ; tu ne te proposes que d’admirer.\par
Mais tu changeras bientôt de pensée : elles n’éblouissent point par un faux respect ; leur simplicité les a fait longtemps méconnaître, et ce n’est qu’après bien des réflexions qu’on en a connu toute la fécondité et toute l’étendue.\par
La première est que tout corps tend à décrire une ligne droite, à moins qu’il ne rencontre quelque obstacle qui l’en détourne ; et la seconde, qui n’en est qu’une suite, c’est que tout corps qui tourne autour d’un centre tend à s’en éloigner, parce que, plus il en est loin, plus la ligne qu’il décrit approche de la ligne droite.\par
Voilà, sublime dervis, la clef de la nature : voilà des principes féconds, dont on tire des conséquences à perte de vue, comme je te le ferai voir dans une lettre particulière.\par
La connaissance de cinq ou six vérités a rendu leur philosophie pleine de miracles, et leur a fait faire plus de prodiges et de merveilles que tout ce qu’on nous raconte de nos saints prophètes.\par
Car enfin je suis persuadé qu’il n’y a aucun de nos docteurs qui n’eût été embarrassé, si on lui eût dit de peser dans une balance tout l’air qui est autour de la terre, ou de mesurer toute l’eau qui tombe chaque année sur sa surface ; et qui n’eût pensé plus de quatre fois, avant de dire combien de lieues le son fait dans une heure ; quel temps un rayon de lumière emploie à venir du soleil à nous ; combien de toises il y a d’ici à Saturne ; quelle est la courbe selon laquelle un vaisseau doit être taillé, pour être le meilleur voilier qu’il soit possible.\par
Peut-être que si quelque homme divin avait orné les ouvrages de ces philosophes de paroles hautes et sublimes ; s’il y avait mêlé des figures hardies et des allégories mystérieuses, il aurait fait un bel ouvrage qui n’aurait cédé qu’au saint Alcoran.\par
Cependant, s’il te faut dire ce que je pense, je ne m’accommode guères du style figuré. Il y a dans notre Alcoran un grand nombre de choses puériles, qui me paraissent toujours telles, quoiqu’elles soient relevées par la force et la vie de l’expression. Il semble d’abord que les livres inspirés ne sont que les idées divines rendues en langage humain : au contraire, dans nos livres saints, on trouve le langage de Dieu, et les idées des hommes ; comme si, par un admirable caprice, Dieu y avait dicté les paroles, et que l’homme eût fourni les pensées.\par
Tu diras peut-être que je parle trop librement de ce qu’il y a de plus saint parmi nous ; tu croiras que c’est le fruit de l’indépendance où l’on vit dans ce pays. Non, grâces au ciel, l’esprit n’a pas corrompu le cœur ; et, tandis que je vivrai, Ali sera mon prophète.\par

\dateline{À Paris, le 15 de la lune de Chahban, 1716.}
\section[{Lettre XCIX. Usbek à Ibben. À Smyrne.}]{Lettre XCIX. Usbek à Ibben. À Smyrne.}\renewcommand{\leftmark}{Lettre XCIX. Usbek à Ibben. À Smyrne.}

\noindent Il n’y a point de pays au monde où la fortune soit si inconstante que dans celui-ci. Il arrive tous les dix ans des révolutions qui précipitent le riche dans la misère, et enlèvent le pauvre avec des ailes rapides au comble des richesses. Celui-ci est étonné de sa pauvreté ; celui-là l’est de son abondance. Le nouveau riche admire la sagesse de la providence ; le pauvre, l’aveugle fatalité du destin.\par
Ceux qui lèvent les tributs nagent au milieu des trésors : parmi eux il y a peu de Tantales. Ils commencent pourtant ce métier par la dernière misère : ils sont méprisés comme de la boue pendant qu’ils sont pauvres ; quand ils sont riches, on les estime assez : aussi ne négligent-ils rien pour acquérir de l’estime.\par
Ils sont à présent dans une situation bien terrible. On vient d’établir une chambre qu’on appelle de justice, parce qu’elle va leur ravir tout leur bien. Ils ne peuvent ni détourner ni cacher leurs effets ; car on les oblige de les déclarer au juste, sous peine de la vie : ainsi on les fait passer par un défilé bien étroit, je veux dire entre la vie et leur argent. Pour comble d’infortune, il y a un ministre connu par son esprit, qui les honore de ses plaisanteries, et badine sur toutes les délibérations du conseil. On ne trouve pas tous les jours des ministres disposés à faire rire le peuple ; et l’on doit savoir bon gré à celui-ci de l’avoir entrepris.\par
Le corps des laquais est plus respectable en France qu’ailleurs : c’est un séminaire de grands seigneurs ; il remplit le vide des autres états. Ceux qui le composent prennent la place des grands malheureux, des magistrats ruinés, des gentilshommes tués dans les fureurs de la guerre ; et quand ils ne peuvent pas suppléer par eux-mêmes, ils relèvent toutes les grandes maisons par le moyen de leurs filles, qui sont comme une espèce de fumier qui engraisse les terres montagneuses et arides.\par
Je trouve, Ibben, la providence admirable dans la manière dont elle a distribué les richesses : si elle ne les avait accordées qu’aux gens de bien, on ne les aurait pas assez distinguées de la vertu, et on n’en aurait plus senti tout le néant. Mais quand on examine qui sont les gens qui en sont les plus chargés, à force de mépriser les riches, on en vient enfin à mépriser les richesses.\par

\dateline{À Paris, le 26 de la lune de Maharram, 1717.}
\section[{Lettre C. Rica à Rhédi. À Venise.}]{Lettre C. Rica à Rhédi. À Venise.}\renewcommand{\leftmark}{Lettre C. Rica à Rhédi. À Venise.}

\noindent Je trouve les caprices de la mode, chez les Français, étonnants. Ils ont oublié comment ils étaient habillés cet été ; ils ignorent encore plus comment ils le seront cet hiver : mais surtout on ne saurait croire combien il en coûte à un mari, pour mettre sa femme à la mode.\par
Que me servirait de te faire une description exacte de leur habillement et de leurs parures ? une mode nouvelle viendrait détruire tout mon ouvrage, comme celui de leurs ouvriers ; et, avant que tu eusses reçu ma lettre, tout serait changé.\par
Une femme qui quitte Paris pour aller passer six mois à la campagne en revient aussi antique que si elle s’y était oubliée trente ans. Le fils méconnaît le portrait de sa mère, tant l’habit avec lequel elle est peinte lui parait étranger ; il s’imagine que c’est quelque Américaine qui y est représentée, ou que le peintre a voulu exprimer quelqu’une de ses fantaisies.\par
Quelquefois les coiffures montent insensiblement ; et une révolution les fait descendre tout à coup. Il a été un temps que leur hauteur immense mettait le visage d’une femme au milieu d’elle-même : dans un autre, c’était les pieds qui occupaient cette place ; les talons faisaient un piédestal, qui les tenait en l’air. Qui pourrait le croire ? les architectes ont été souvent obligés de hausser, de baisser et d’élargir leurs portes, selon que les parures des femmes exigeaient d’eux ce changement ; et les règles de leur art ont été asservies à ces fantaisies. On voit quelquefois sur un visage une quantité prodigieuse de mouches, et elles disparaissent toutes le lendemain. Autrefois les femmes avaient de la taille, et des dents ; aujourd’hui il n’en est pas question. Dans cette changeante nation, quoi qu’en dise le critique, les filles se trouvent autrement faites que leurs mères.\par
Il en est des manières et de la façon de vivre comme des modes : les Français changent de mœurs selon l’âge de leur roi. Le monarque pourrait même parvenir à rendre la nation grave, s’il l’avait entrepris. Le prince imprime le caractère de son esprit à la cour, la cour à la ville, la ville aux provinces. L’âme du souverain est un moule qui donne la forme à toutes les autres.\par

\dateline{De Paris, le 8 de la lune de Saphar, 1717.}
\section[{Lettre CI. Rica au même.}]{Lettre CI. Rica au même.}\renewcommand{\leftmark}{Lettre CI. Rica au même.}

\noindent Je te parlais l’autre jour de l’inconstance prodigieuse des Français sur leurs modes. Cependant il est inconcevable à quel point ils en sont entêtés : c’est la règle avec laquelle ils jugent de tout ce qui se fait chez les autres nations ; ils y rappellent tout ; ce qui est étranger leur parait toujours ridicule. Je t’avoue que je ne saurais guères ajuster cette fureur pour leurs costumes avec l’inconstance avec laquelle ils en changent tous les jours.\par
Quand je te dis qu’ils méprisent tout ce qui est étranger, je ne te parle que des bagatelles ; car, sur les choses importantes, ils semblent s’être méfiés d’eux-mêmes jusqu’à se dégrader. Ils avouent de bon cœur que les autres peuples sont plus sages, pourvu qu’on convienne qu’ils sont mieux vêtus : ils veulent bien s’assujettir aux lois d’une nation rivale, pourvu que les perruquiers français décident en législateurs sur la forme des perruques étrangères. Rien ne leur parait si beau que de voir le goût de leurs cuisiniers régner du septentrion au midi, et les ordonnances de leurs coiffeuses portées dans toutes les toilettes de l’Europe.\par
Avec ces nobles avantages, que leur importe que le bon sens leur vienne d’ailleurs, et qu’ils aient pris de leurs voisins tout ce qui concerne le gouvernement politique et civil ?\par
Qui peut penser qu’un royaume, le plus ancien et le plus puissant de l’Europe, soit gouverné, depuis plus de dix siècles, par des lois qui ne sont pas faites pour lui ? Si les Français avaient été conquis, ceci ne serait pas difficile à comprendre : mais ils sont les conquérants.\par
Ils ont abandonné les lois anciennes, faites par leurs premiers rois dans les assemblées générales de la nation : et ce qu’il y a de singulier, c’est que les lois romaines, qu’ils ont prises à la place, étaient en partie faites et en partie rédigées par des empereurs contemporains de leurs législateurs.\par
Et afin que l’acquisition fût entière, et que tout le bon sens leur vînt d’ailleurs, ils ont adopté toutes les constitutions des papes, et en ont fait une nouvelle partie de leur droit : nouveau genre de servitude.\par
Il est vrai que, dans les derniers temps, on a rédigé par écrit quelques statuts des villes et des provinces : mais ils sont presque tous pris du droit romain.\par
Cette abondance de lois adoptées, et pour ainsi dire naturalisées, est si grande qu’elle accable également la justice et les juges. Mais ces volumes de lois ne sont rien en comparaison de cette armée effroyable de glossateurs, de commentateurs, de compilateurs ; gens aussi faibles par le peu de justesse de leur esprit qu’ils sont forts par leur nombre prodigieux.\par
Ce n’est pas tout : ces lois étrangères ont introduit des formalités qui sont la honte de la raison humaine. Il serait assez difficile de décider si la forme s’est rendue plus pernicieuse, lorsqu’elle est entrée dans la jurisprudence, ou lorsqu’elle s’est logée dans la médecine ; si elle a fait plus de ravages sous la robe d’un jurisconsulte que sous le large chapeau d’un médecin ; et si dans l’une elle a plus ruiné de gens qu’elle n’en a tués dans l’autre.\par

\dateline{De Paris, le 17 de la lune de Saphar, 1717.}
\section[{Lettre CII. Usbek à ***.}]{Lettre CII. Usbek à ***.}\renewcommand{\leftmark}{Lettre CII. Usbek à ***.}

\noindent On parle toujours ici de la constitution. J’entrai l’autre jour dans une maison où je vis d’abord un gros homme avec un teint vermeil, qui disait d’une voix forte : J’ai donné mon mandement ; je n’irai point répondre à tout ce que vous dites ; mais lisez-le, ce mandement ; et vous verrez que j’y ai résolu tous vos doutes. Il m’a fallu bien suer pour le faire, dit-il en portant la main sur le front ; j’ai eu besoin de toute ma doctrine ; et il m’a fallu lire bien des auteurs latins. Je le crois, dit un homme qui se trouva là, car c’est un bel ouvrage ; et je défie ce jésuite qui vient si souvent vous voir d’en faire un meilleur. Eh bien, lisez-le donc, reprit-il, et vous serez plus instruit sur ces matières dans un quart d’heure, que si je vous en avais parlé deux heures. Voilà comme il évitait d’entrer en conversation, et de commettre sa suffisance. Mais, comme il se vit pressé, il fut obligé de sortir de ses retranchements ; et il commença à dire théologiquement force sottises, soutenu d’un dervis qui les lui rendait très respectueusement. Quand deux hommes qui étaient là lui niaient quelque principe, il disait d’abord : Cela est certain, nous l’avons jugé ainsi ; et nous sommes des juges infaillibles. Et comment, lui dis-je pour lors, êtes-vous des juges infaillibles ? Ne voyez-vous pas, reprit-il, que le Saint-Esprit nous éclaire ? Cela est heureux, lui répondis-je ; car, de la manière dont vous avez parlé tout aujourd’hui, je reconnais que vous avez grand besoin d’être éclairé.\par

\dateline{De Paris, le 18 de la lune de Rebiab I, 1717.}
\section[{Lettre CIII. Usbek à Ibben. À Smyrne.}]{Lettre CIII. Usbek à Ibben. À Smyrne.}\renewcommand{\leftmark}{Lettre CIII. Usbek à Ibben. À Smyrne.}

\noindent Les plus puissants États de l’Europe sont ceux de l’empereur, des rois de France, d’Espagne et d’Angleterre. L’Italie et une grande partie de l’Allemagne sont partagées en un nombre infini de petits États, dont les princes sont, à proprement parler, les martyrs de la souveraineté. Nos glorieux sultans ont plus de femmes que la plupart de ces princes n’ont de sujets. Ceux d’Italie, qui ne sont pas si unis, sont plus à plaindre : leurs États sont ouverts comme des caravansérails, où ils sont obligés de loger les premiers qui viennent : il faut donc qu’ils s’attachent aux grands princes, et leur fassent part de leur frayeur, plutôt que de leur amitié.\par
La plupart des gouvernements d’Europe sont monarchiques, ou plutôt sont ainsi appelés : car je ne sais pas s’il y en a jamais eu véritablement de tels, au moins est-il impossible qu’ils aient subsisté longtemps dans leur pureté. C’est un état violent, qui dégénère toujours en despotisme, ou en république : la puissance ne peut jamais être également partagée entre le peuple et le prince ; l’équilibre est trop difficile à garder : il faut que le pouvoir diminue d’un côté pendant qu’il augmente de l’autre ; mais l’avantage est ordinairement du côté du prince, qui est à la tête des armées.\par
Aussi le pouvoir des rois d’Europe est-il bien grand, et on peut dire qu’ils l’ont tel qu’ils le veulent : mais ils ne l’exercent point avec tant d’étendue que nos sultans ; premièrement, parce qu’ils ne veulent point choquer les mœurs et la religion des peuples ; secondement, parce qu’il n’est pas de leur intérêt de le porter si loin.\par
Rien ne rapproche plus les princes de la condition de leurs sujets, que cet immense pouvoir qu’ils exercent sur eux ; rien ne les soumet plus aux revers, et aux caprices de la fortune.\par
L’usage où ils sont de faire mourir tous ceux qui leur déplaisent, au moindre signe qu’ils font, renverse la proportion qui doit être entre les fautes et les peines, qui est comme l’âme des États et l’harmonie des empires ; et cette proportion, scrupuleusement gardée par les princes chrétiens, leur donne un avantage infini sur nos sultans.\par
Un Persan qui, par imprudence ou par malheur, s’est attiré la disgrâce du prince, est sûr de mourir : la moindre faute ou le moindre caprice le met dans cette nécessité. Mais s’il avait attenté à la vie de son souverain, s’il avait voulu livrer ses places aux ennemis, il en serait aussi quitte pour perdre la vie : il ne court donc pas plus de risque dans ce dernier cas que dans le premier.\par
Aussi, dans la moindre disgrâce, voyant la mort certaine, et ne voyant rien de pis, il se porte naturellement à troubler l’État, et à conspirer contre le souverain ; seule ressource qui lui reste.\par
Il n’en est pas de même des grands d’Europe, à qui la disgrâce n’ôte rien que la bienveillance et la faveur. Ils se retirent de la cour, et ne songent qu’à jouir d’une vie tranquille et des avantages de leur naissance. Comme on ne les fait guères périr que pour le crime de lèse-majesté, ils craignent d’y tomber, par la considération de ce qu’ils ont à perdre, et du peu qu’ils ont à gagner : ce qui fait qu’on voit peu de révoltes, et peu de princes morts d’une mort violente.\par
Si, dans cette autorité illimitée qu’ont nos princes, ils n’apportaient pas tant de précautions pour mettre leur vie en sûreté, ils ne vivraient pas un jour ; et s’ils n’avaient à leur solde un nombre innombrable de troupes, pour tyranniser le reste de leurs sujets, leur empire ne subsisterait pas un mois.\par
Il n’y a que quatre ou cinq siècles qu’un roi de France prit des gardes, contre l’usage de ces temps-là, pour se garantir des assassins qu’un petit prince d’Asie avait envoyés pour le faire périr : jusque-là les rois avaient vécu tranquilles au milieu de leurs sujets, comme des pères au milieu de leurs enfants.\par
Bien loin que les rois de France puissent de leur propre mouvement ôter la vie à un de leurs sujets, comme nos sultans, ils portent au contraire toujours avec eux la grâce de tous les criminels ; il suffit qu’un homme ait été assez heureux pour voir l’auguste visage de son prince, pour qu’il cesse d’être indigne de vivre. Ces monarques sont comme le soleil, qui porte partout la chaleur et la vie.\par

\dateline{De Paris, le 8 de la lune de Rebiab 2, 1717.}
\section[{Lettre CIV. Usbek au même.}]{Lettre CIV. Usbek au même.}\renewcommand{\leftmark}{Lettre CIV. Usbek au même.}

\noindent Pour suivre l’idée de ma dernière lettre, voici à peu près ce que me disait l’autre jour un Européen assez sensé :\par
Le plus mauvais parti que les princes d’Asie aient pu prendre, c’est de se cacher comme ils font. Ils veulent se rendre plus respectables : mais ils font respecter la royauté, et non pas le roi ; et attachent l’esprit des sujets à un certain trône, et non pas à une certaine personne.\par
Cette puissance invisible qui gouverne est toujours la même pour le peuple. Quoique dix rois, qu’il ne connaît que de nom, se soient égorgés l’un après l’autre, il ne sent aucune différence : c’est comme s’il avait été gouverné successivement par des esprits.\par
Si le détestable parricide de notre grand roi Henri IV avait porté ce coup sur un roi des Indes ; maître du sceau royal et d’un trésor immense, qui aurait semblé amassé pour lui, il aurait pris tranquillement les rênes de l’empire, sans qu’un seul homme eût pensé à réclamer son roi, sa famille et ses enfants.\par
On s’étonne de ce qu’il n’y a presque jamais de changement dans le gouvernement des princes d’Orient ; et d’où vient cela, si ce n’est de ce qu’il est tyrannique et affreux ?\par
Les changements ne peuvent être faits que par le prince, ou par le peuple ; mais là, les princes n’ont garde d’en faire, parce que, dans un si haut degré de puissance, ils ont tout ce qu’ils peuvent avoir ; s’ils changeaient quelque chose, ce ne pourrait être qu’à leur préjudice.\par
Quant aux sujets, si quelqu’un d’eux forme quelque résolution, il ne saurait l’exécuter sur l’État ; il faudrait qu’il contrebalançât tout à coup une puissance redoutable et toujours unique ; le temps lui manque comme les moyens : mais il n’a qu’à aller à la source de ce pouvoir ; et il ne lui faut qu’un bras et qu’un instant.\par
Le meurtrier monte sur le trône pendant que le monarque en descend, tombe, et va expirer à ses pieds.\par
Un mécontent en Europe songe à entretenir quelque intelligence secrète, à se jeter chez les ennemis, à se saisir de quelque place, à exciter quelques vains murmures parmi les sujets. Un mécontent en Asie va droit au prince, étonne, frappe, renverse : il en efface jusqu’à l’idée ; dans un instant l’esclave et le maître ; dans un instant usurpateur et légitime.\par
Malheureux le roi qui n’a qu’une tête ! il semble ne réunir sur elle toute sa puissance, que pour indiquer au premier ambitieux l’endroit où il la trouvera tout entière.\par

\dateline{De Paris, le 17 de la lune de Rebiab 2, 1717.}
\section[{Lettre CV. Usbek au même.}]{Lettre CV. Usbek au même.}\renewcommand{\leftmark}{Lettre CV. Usbek au même.}

\noindent Tous les peuples d’Europe ne sont pas également soumis à leurs princes : par exemple, l’humeur impatiente des Anglais ne laisse guère à leur roi le temps d’appesantir son autorité ; la soumission et l’obéissance sont les vertus dont ils se piquent le moins. Ils disent là-dessus des choses bien extraordinaires. Selon eux, il n’y a qu’un lien qui puisse attacher les hommes, qui est celui de la gratitude : un mari, une femme, un père et un fils, ne sont liés entre eux que par l’amour qu’ils se portent, ou par les bienfaits qu’ils se procurent ; et ces motifs divers de reconnaissance sont l’origine de tous les royaumes, et de toutes les sociétés.\par
Mais si un prince, bien loin de faire vivre ses sujets heureux, veut les accabler et les détruire, le fondement de l’obéissance cesse ; rien ne les lie, rien ne les attache à lui ; et ils rentrent dans leur liberté naturelle. Ils soutiennent que tout pouvoir sans bornes ne saurait être légitime, parce qu’il n’a jamais pu avoir d’origine légitime. Car nous ne pouvons pas, disent-ils, donner à un autre plus de pouvoir sur nous que nous n’en avons nous-mêmes : or nous n’avons pas sur nous-mêmes un pouvoir sans bornes ; par exemple, nous ne pouvons pas nous ôter la vie : personne n’a donc, concluent-ils, sur la terre un tel pouvoir.\par
Le crime de lèse-majesté n’est autre chose, selon eux, que le crime que le plus faible commet contre le plus fort, en lui désobéissant, de quelque manière qu’il lui désobéisse. Aussi le peuple d’Angleterre, qui se trouva le plus fort contre un de leurs rois, déclara-t-il que c’était un crime de lèse-majesté à un prince de faire la guerre à ses sujets. Ils ont donc grande raison, quand ils disent que le précepte de leur Alcoran, qui ordonne de se soumettre aux puissances, n’est pas bien difficile à suivre, puisqu’il leur est impossible de ne le pas observer ; d’autant que ce n’est pas au plus vertueux qu’on les oblige de se soumettre, mais à celui qui est le plus fort.\par
Les Anglais disent qu’un de leurs rois, qui avait vaincu et pris prisonnier un prince qui s’était révolté et lui disputait la couronne ; ayant voulu lui reprocher son infidélité et sa perfidie : Il n’y a qu’un moment, dit le prince infortuné, qu’il vient d’être décidé lequel de nous deux est le traître.\par
Un usurpateur déclare rebelles tous ceux qui n’ont point opprimé la patrie comme lui : et, croyant qu’il n’y a pas de loi là où il ne voit point de juges, il fait révérer, comme des arrêts du ciel, les caprices du hasard et de la fortune.\par

\dateline{De Paris, le 20 de la lune de Rebiab 2, 1717.}
\section[{Lettre CVI. Rhédi à Usbek. À Paris.}]{Lettre CVI. Rhédi à Usbek. À Paris.}\renewcommand{\leftmark}{Lettre CVI. Rhédi à Usbek. À Paris.}

\noindent Tu m’as beaucoup parlé, dans une de tes lettres, des sciences et des arts cultivés en Occident. Tu me vas regarder comme un barbare ; mais je ne sais si l’utilité que l’on en retire dédommage les hommes du mauvais usage que l’on en fait tous les jours.\par
J’ai ouï dire que la seule invention des bombes avait ôté la liberté à tous les peuples d’Europe. Les princes ne pouvant plus confier la garde des places aux bourgeois, qui, à la première bombe, se seraient rendus, ont eu un prétexte pour entretenir de gros corps de troupes réglées, avec lesquelles ils ont dans la suite opprimé leurs sujets.\par
Tu sais que, depuis l’invention de la poudre, il n’y a plus de places imprenables ; c’est-à-dire, Usbek, qu’il n’y a plus d’asile sur la terre contre l’injustice et la violence.\par
Je tremble toujours qu’on ne parvienne à la fin à découvrir quelque secret qui fournisse une voie plus abrégée pour faire périr les hommes, détruire les peuples et les nations entières.\par
Tu as lu les historiens ; fais-y bien attention : presque toutes les monarchies n’ont été fondées que sur l’ignorance des arts, et n’ont été détruites que parce qu’on les a trop cultivés. L’ancien empire de Perse peut nous en fournir un exemple domestique.\par
Il n’y a pas longtemps que je suis en Europe ; mais j’ai ouï parler à des gens sensés des ravages de la chimie : il semble que ce soit un quatrième fléau qui ruine les hommes et les détruit en détail, mais continuellement ; tandis que la guerre, la peste, la famine, les détruisent en gros, mais par intervalles.\par
Que nous a servi l’invention de la boussole, et la découverte de tant de peuples, qu’à nous communiquer leurs maladies plutôt que leurs richesses ? L’or et l’argent avaient été établis, par une convention générale, pour être le prix de toutes les marchandises et un gage de leur valeur, par la raison que ces métaux étaient rares, et inutiles à tout autre usage : que nous importait-il donc qu’ils devinssent plus communs, et que, pour marquer la valeur d’une denrée, nous eussions deux ou trois signes au lieu d’un ? Cela n’en était que plus incommode.\par
Mais, d’un autre côté, cette invention a été bien pernicieuse aux pays qui ont été découverts. Les nations entières ont été détruites ; et les hommes qui ont échappé à la mort ont été réduits à une servitude si rude, que le récit en a fait frémir les musulmans.\par
Heureuse l’ignorance des enfants de Mahomet ! Aimable simplicité, si chérie de notre saint prophète, vous me rappelez toujours la naïveté des anciens temps, et la tranquillité qui régnait dans le cœur de nos premiers pères.\par

\dateline{De Venise, le 2 de la lune de Rhamazan, 1717.}
\section[{Lettre CVII. Usbek à Rhédi. À Venise.}]{Lettre CVII. Usbek à Rhédi. À Venise.}\renewcommand{\leftmark}{Lettre CVII. Usbek à Rhédi. À Venise.}

\noindent Ou tu ne penses pas ce que tu dis, ou bien tu fais mieux que tu ne penses. Tu as quitté ta patrie pour t’instruire, et tu méprises toute instruction : tu viens pour te former dans un pays où l’on cultive les beaux-arts, et tu les regardes comme pernicieux. Te le dirai-je, Rhédi ? je suis plus d’accord avec toi que tu ne l’es avec toi-même.\par
As-tu bien réfléchi à l’état barbare et malheureux où nous entraînerait la perte des arts ? Il n’est pas nécessaire de se l’imaginer, on peut le voir. Il y a encore des peuples sur la terre chez lesquels un singe passablement instruit pourrait vivre avec honneur ; il s’y trouverait à peu près à la portée des autres habitants : on ne lui trouverait point l’esprit singulier, ni le caractère bizarre ; il passerait tout comme un autre, et serait distingué même par sa gentillesse.\par
Tu dis que les fondateurs des empires ont presque tous ignoré les arts. Je ne te nie pas que des peuples barbares n’aient pu, comme des torrents impétueux, se répandre sur la terre, et couvrir de leurs armées féroces les royaumes les mieux policés. Mais, prends-y garde, ils ont appris les arts pu les ont fait exercer aux peuples vaincus ; sans leur puissance aurait passé comme le bruit du tonnerre et des tempêtes.\par
Tu crains, dis-tu, que l’on n’invente quelque manière de destruction plus cruelle que celle qui est en usage. Non : si une fatale invention venait à se découvrir, elle serait bientôt prohibée par le droit des gens ; et le consentement unanime des nations ensevelirait cette découverte. Il n’est point de l’intérêt des princes de faire des conquêtes par de pareilles voies : ils doivent chercher des sujets, et non pas des terres.\par
Tu te plains de l’invention de la poudre et des bombes : tu trouves étrange qu’il n’y ait plus de place imprenable : c’est-à-dire que tu trouves étrange que les guerres soient aujourd’hui terminées plus tôt qu’elles ne l’étaient autrefois.\par
Tu dois avoir remarqué, en lisant les histoires, que, depuis l’invention de la poudre, les batailles sont beaucoup moins sanglantes qu’elles ne l’étaient, parce qu’il n’y a presque plus de mêlée.\par
Et quand il se serait trouvé quelque cas particulier où un art aurait été préjudiciable, doit-on pour cela le rejeter ? Penses-tu, Rhédi, que la religion que notre saint prophète a apportée du ciel soit pernicieuse, parce qu’elle servira quelques jours à confondre les perfides chrétiens ?\par
Tu crois que les arts amollissent les peuples, et par là sont cause de la chute des empires. Tu parles de la ruine de celui des anciens Perses, qui fut l’effet de leur mollesse ; mais il s’en faut bien que cet exemple décide, puisque les Grecs, qui les subjuguèrent, cultivaient les arts avec infiniment plus de soin qu’eux.\par
Quand on dit que les arts rendent les hommes efféminés, on ne parle pas du moins des gens qui s’y appliquent, puisqu’ils ne sont jamais dans l’oisiveté, qui, de tous les vices, est celui qui amollit le plus le courage.\par
Il n’est donc question que de ceux qui en jouissent. Mais comme dans un pays policé ceux qui jouissent des commodités d’un art sont obligés d’en cultiver un autre, à moins que de se voir réduits à une pauvreté honteuse, il s’ensuit que l’oisiveté et la mollesse sont incompatibles avec les arts.\par
Paris est peut-être la ville du monde la plus sensuelle, et où l’on raffine le plus sur les plaisirs ; mais c’est peut-être celle où l’on mène une vie plus dure. Pour qu’un homme vive délicieusement, il faut que cent autres travaillent sans relâche. Une femme s’est mis dans la tête qu’elle devait paraître à une assemblée avec une certaine parure ; il faut que dès ce moment cinquante artisans ne dorment plus, et n’aient plus le loisir de boire et de manger : elle commande, et elle est obéie plus promptement que ne serait notre monarque ; parce que l’intérêt est le plus grand monarque de la terre.\par
Cette ardeur pour le travail, cette passion de s’enrichir, passe de condition en condition, depuis les artisans jusqu’aux grands. Personne n’aime à être plus pauvre que celui qu’il vient de voir immédiatement au-dessous de lui. Vous voyez à Paris un homme qui a de quoi vivre jusqu’au jour du jugement, qui travail sans cesse, et court le risque d’accourcir ses jours pour amasser, dit-il, de quoi vivre.\par
Le même esprit gagne la nation ; on n’y voit que travail et qu’industrie : où est donc ce peuple efféminé dont tu parles tant ?\par
Je suppose, Rhédi, qu’on ne souffrît dans un royaume que les arts absolument nécessaires à la culture des terres, qui sont pourtant en grand nombre, et qu’on en bannît tous ceux qui ne servent qu’à la volupté ou à la fantaisie, je le soutiens, cet État serait le plus misérable qu’ils y eût au monde.\par
Quand les habitants auraient assez de courage pour se passer de tant de choses qu’ils doivent à leurs besoins, le peuple dépérirait tous les jours ; et l’État deviendrait si faible, qu’il n’y aurait si petite puissance qui ne fût en état de le conquérir.\par
Je pourrais entrer ici dans un long détail, et te faire voir que les revenus des particuliers cesseraient presque absolument, et par conséquent ceux du prince. Il n’y aurait presque plus de relation de facultés entre les citoyens ; cette circulation de richesses et cette progression de revenus, qui vient de la dépendance où sont les arts les uns des autres, cesseraient absolument ; chacun ne tirerait de revenu que de sa terre, et n’en tirerait précisément que ce qu’il lui faut pour ne pas mourir de faim. Mais, comme ce n’est pas la centième partie du revenu d’un royaume, il faudrait que le nombre des habitants diminuât à proportion, et qu’il n’en restât que la centième partie.\par
Fais bien attention jusqu’où vont les revenus de l’industrie. Un fonds ne produit annuellement à son maître que la vingtième partie de sa valeur ; mais, avec une pistole de couleur, un peintre fera un tableau qui lui en vaudra cinquante. On en peut dire de même des orfèvres, des ouvriers en laine, en soie, et de toutes sortes d’artisans.\par
De tout ceci il faut conclure, Rhédi, que pour qu’un prince soit puissant, il faut que ses sujets vivent dans les délices ; il faut qu’il travaille à leur procurer toutes sortes de superfluités avec autant d’attention que les nécessités de la vie.\par

\dateline{De Paris, le 14 de la lune de Chalval, 1717.}
\section[{Lettre CVIII. Rica à Ibben. À Smyrne.}]{Lettre CVIII. Rica à Ibben. À Smyrne.}\renewcommand{\leftmark}{Lettre CVIII. Rica à Ibben. À Smyrne.}

\noindent J’ai vu le jeune monarque : sa vie est bien précieuse à ses sujets ; elle ne l’est pas moins à \\
toute l’Europe, par les grands troubles que sa mort pourrait produire. Mais les rois sont comme les dieux ; et, pendant qu’ils vivent, on doit les croire immortels. Sa physionomie est majestueuse, mais charmante : une belle éducation semble concourir avec un heureux naturel, et promet déjà un grand prince.\par
On dit que l’on ne peut jamais connaître le caractère des rois d’Occident jusqu’à ce qu’ils aient passé par les deux grandes épreuves, de leur maîtresse, et de leur confesseur. On verra bientôt l’un et l’autre travailler à se saisir de l’esprit de celui-ci ; et il se livrera pour cela de grands combats. Car, sous un jeune prince, ces deux puissances sont toujours rivales ; mais elles se concilient et se réunissent sous un vieux. Sous un jeune prince, le dervis a un rôle bien difficile à soutenir : la force du roi fait sa faiblesse ; mais l’autre triomphe également de sa faiblesse et de sa force.\par
Lorsque j’arrivai en France, je trouvai le feu roi absolument gouverné par les femmes ; et cependant, dans l’âge où il était, je crois que c’était, le monarque de la terre qui en avait le moins de besoin. J’entendis un jour une femme qui disait : Il faut que l’on fasse quelque chose pour ce jeune colonel, sa valeur m’est connue ; j’en parlerai au ministre. Une autre disait ; Il est surprenant que ce jeune abbé ait été oublié ; il faut qu’il soit évêque ; il est homme de naissance, et je pourrais répondre de ses mœurs. Il ne faut pas pourtant que tu t’imagines que celles qui tenaient ces discours fussent des favorites du prince ; elles ne lui avaient peut-être pas parlé deux fois en leur vie : chose pourtant très facile à faire chez les princes européens. Mais c’est qu’il n’y a personne qui ait quelque emploi à la cour, dans Paris ou dans les provinces, qui n’ait une femme par les mains de laquelle passent toutes les grâces et quelquefois les injustices qu’il peut faire. Ces femmes ont toutes des relations les unes avec les autres, et forment une espèce de république, dont les membres toujours actifs se secourent et se servent mutuellement : c’est comme un nouvel État dans l’État ; et celui qui est à la cour, à Paris, dans les provinces, qui voit agir des ministres, des magistrats, des prélats, s’il ne connaît les femmes qui les gouvernent, est comme un homme qui voit bien une machine qui joue, mais qui n’en connaît point les ressorts.\par
Crois-tu, Ibben, qu’une femme s’avise d’être la maîtresse d’un ministre pour coucher avec lui ? Quelle idée ! c’est pour lui présenter cinq ou six placets tous les matins ; et la bonté de leur naturel paraît dans l’empressement qu’elles ont de faire du bien à une infinité de gens malheureux, qui leur procurent cent mille livres de rente.\par
On se plaint en Perse de ce que le royaume est gouverné par deux ou trois femmes : c’est bien pis en France, où les femmes en général gouvernent, et prennent non seulement en gros, mais même se partagent en détail, toute l’autorité.\par

\dateline{De Paris, le dernier de la lune de Chalval, 1717.}
\section[{Lettre CIX. Usbek à ***.}]{Lettre CIX. Usbek à ***.}\renewcommand{\leftmark}{Lettre CIX. Usbek à ***.}

\noindent Il y a une espèce de livres que nous ne connaissons point en Perse, et qui me paraissent ici fort à la mode : ce sont les journaux. La paresse se sent flattée en les lisant : on est ravi de pouvoir parcourir trente volumes en un quart d’heure.\par
Dans la plupart des livres, l’auteur n’a pas fait les compliments ordinaires, que les lecteurs sont aux abois : il les fait entrer à demi morts dans une matière noyée au milieu d’une mer de paroles. Celui-ci veut s’immortaliser par un in-douze ; celui-là, par un in-quarto ; un autre, qui a de plus belles inclinations, vise à l’in-folio ; il faut donc qu’il étende son sujet à proportion ; ce qu’il fait sans pitié, comptant pour rien la peine du pauvre lecteur, qui se tue à réduire ce que l’auteur a pris tant de peine à amplifier.\par
Je ne sais, ***, quel mérite il y a à faire pareils ouvrages : j’en ferais bien autant si je voulais ruiner ma santé et un libraire.\par
Le grand tort qu’on les journalistes, c’est qu’ils ne parlent que des livres nouveaux : comme si la vérité était jamais nouvelle. Il me semble que, jusqu’à ce qu’un homme ait lu tous les livres anciens, il n’a aucune raison de leur préférer les nouveaux.\par
Mais lorsqu’ils s’imposent la loi de ne parler que des ouvrages encore tout chauds de la forge, ils s’en imposent un autre, qui est d’être très ennuyeux. Ils n’ont garde de critiquer les livres dont ils font les extraits, quelque raison qu’ils en aient ; et, en effet, quel est l’homme assez hardi pour vouloir se faire dix ou douze ennemis dans les mois ?\par
La plupart des auteurs ressemblent aux poètes, qui souffriront une volée de coups de bâton sans se plaindre ; mais qui, peu jaloux de leurs épaules, le sont si fort de leurs ouvrages, qu’ils ne sauraient soutenir la moindre critique. Il faut donc bien se donner de garde de les attaquer par un endroit si sensible ; et les journalistes le savent bien. Ils font donc tout le contraire ; ils commencent par louer la matière qui est traitée : première fadeur. De là ils passent aux louanges de l’auteur ; louanges forcées : car ils ont affaire à des gens qui sont encore en haleine, tout prêts à se faire faire raison, et à foudroyer à coup de plume un téméraire journaliste.\par

\dateline{De Paris, le 5 de la lune de Zilcadé, 1718.}
\section[{Lettre CX. Rica à ***.}]{Lettre CX. Rica à ***.}\renewcommand{\leftmark}{Lettre CX. Rica à ***.}

\noindent L’université de Paris est la fille aînée des rois de France, et très aînée ; car elle a plus de neuf cents ans : aussi rêve-t-elle quelquefois.\par
On m’a compté qu’elle eut, il y a quelque temps, un grand démêlé avec quelques docteurs à l’occasion de la lettre Q\footnote{Il veut parler de la querelle de Ramus.}, qu’elle voulait que l’on prononçât comme un K. La dispute s’échauffa si fort, que quelques-uns furent dépouillés de leurs biens : il fallut que le parlement terminât le différend ; et il accorda permission, par un arrêt solennel, à tous les sujets du roi de France de prononcer cette lettre à leur fantaisie. Il faisait beau de voir les deux corps de l’Europe les plus respectables occupés à décider du sort d’une lettre de l’alphabet.\par
Il semble, mon cher ***, que les têtes des plus grands hommes s’étrécissent lorsqu’elles sont assemblées ; et que, là où il y a plus de sages, il y ait aussi moins de sagesses. Les grands corps s’attachent toujours si fort aux minuties, aux formalités, aux vains usages, que l’essentiel ne va jamais qu’après. J’ai ouï dire qu’un roi d’Aragon\footnote{C'était en 1610.} ayant assemblé les états d’Aragon et de Catalogne, les premières séances s’employèrent à décider en quelle langue les délibérations seraient conçues : la dispute était vive, et les états se seraient rompus mille fois, si l’on n’avait imaginé un expédient, qui était que la demande serait faite en langage catalan, et la réponse en aragonais.\par

\dateline{De Paris, le 25 de la lune de Zilhagé, 1718.}
\section[{Lettre CXI. Rica à ***.}]{Lettre CXI. Rica à ***.}\renewcommand{\leftmark}{Lettre CXI. Rica à ***.}

\noindent Le rôle d’une jolie femme est beaucoup plus grave que l’on ne pense. Il n’y a rien de plus sérieux que ce qui se passe le matin à sa toilette, au milieu de ses domestiques ; un général d’armée n’emploie pas plus d’attention à placer sa droite ou son corps de réserve, qu’elle en met à poster une mouche qui peut manquer, mais dont elle espère ou prévoit le succès.\par
Quelle gêne d’esprit, quelle attention, pour concilier sans cesse les intérêts de deux rivaux, pour paraître neutre à tous les deux, pendant qu’elle est livrée à l’un et à l’autre, et se rendre médiatrice sur tous les sujets de plainte qu’elle leur donne !\par
Quelle occupation pour faire venir parties de plaisir sur parties, les faire succéder et renaître sans cesse, et prévenir tous les accidents qui pourraient les rompre !\par
Avec tout cela, la plus grande peine n’est pas de se divertir ; c’est de la paraître : ennuyez-les tant que vous voudrez, elles vous le pardonneront pourvu que l’on puisse croire qu’elles se sont bien réjouies.\par
Je fus, il y a quelques jours, d’un souper que des femmes firent à la campagne. Dans le chemin elles disaient sans cesse : Au moins, il faudra bien rire et bien nous divertir.\par
Nous nous retrouvâmes assez mal assortis, et par conséquent assez sérieux. Il faut avouer, dit une de ces femmes, que nous nous divertissons bien : il n’y a pas aujourd’hui dans Paris une partie aussi gaie que la nôtre. Comme l’ennui me gagnait, une femme me secoua, et me dit : Eh bien ! ne sommes-nous pas de bonne humeur ? Oui, lui répondis-je en bâillant ; je crois que je crèverai à force de rire. Cependant la tristesse triomphait toujours des réflexions ; et, quant à moi, je me sentis conduit de bâillement en bâillement dans un sommeil léthargique, qui finit tous mes plaisirs.\par

\dateline{De Paris, le 11 de la lune de Maharram, 1718.}
\section[{Lettre CXII. Usbek à ***.}]{Lettre CXII. Usbek à ***.}\renewcommand{\leftmark}{Lettre CXII. Usbek à ***.}

\noindent Le règne du feu roi a été si long, que la fin en avait fait oublier le commencement. C’est aujourd’hui la mode de ne s’occuper que des événements arrivés dans sa minorité ; et on ne lit plus que les mémoires de ces temps-là.\par
Voici le discours qu’un des généraux de la ville de Paris prononça dans un conseil de guerre : et j’avoue que j’y comprends pas grand-chose.\par
"Messieurs, quoique nos troupes aient été repoussées avec perte, je crois qu’il nous sera facile de réparer cet échec. J’ai six couplets de chanson tout prêts à mettre au jour, qui, je m’assure, remettrons toutes choses dans l’équilibre. J’ai fait choix de quelques voix très nettes, qui, sortant de la cavité de certaines poitrines très fortes, émouvront merveilleusement le peuple. Ils sont sur un air qui a fait, jusqu’à présent, un effet tout particulier.\par
Si cela ne suffit pas, nous ferons paraître une estampe qui fera voir Mazarin pendu.\par
Par bonheur pour nous, il ne parle pas bien français ; et il l’écorche tellement, qu’il n’est pas possible que ses affaires ne déclinent. Nous ne manquons pas de faire bien remarquer au peuple le ton ridicule dont il prononce. Nous relevâmes, il y a quelques jours, une faute de grammaire si grossière, qu’on en fit des farces par tous les carrefours.\par
J’espère qu’avant qu’il soit huit jours, le peuple fera du nom de Mazarin un mot générique pour exprimer toutes les bêtes de somme, et celles qui servent à tirer.\par
Depuis notre défaite, notre musique l’a si furieusement vexé sur le péché originel, que, pour ne pas voir ses partisans réduits à la moitié, il a été obligé de renvoyer tous ses pages.\par
Ranimez-vous donc ; reprenez courage, et soyez sûrs que nous lui ferons repasser les monts à coups de sifflets."\footnote{[Note de l’auteur, extraite de l’{\itshape édition} de 1721 2\textsuperscript{e}, Cologne, Pierre Marteau] Le cardinal Mazarin, voulant prononcer l’{\itshape arrêt d’Union}, dit devant les députés du parlement: l’{\itshape Arrêt d’Ognon}, de quoi le peuple fit force plaisanteries.}\par

\dateline{De Paris, le 4 de la lune de Chahban, 1718.}
\section[{Lettre CXIII. Rhédi à Usbek. À Paris.}]{Lettre CXIII. Rhédi à Usbek. À Paris.}\renewcommand{\leftmark}{Lettre CXIII. Rhédi à Usbek. À Paris.}

\noindent Pendant le séjour que je fais en Europe, je lis les historiens anciens et modernes : je compare tous les temps ; j’ai du plaisir à les voir passer, pour ainsi dire, devant moi ; et j’arrête surtout mon esprit à ces grands changements qui ont rendu les âges si différents des âges, et la terre si peu semblable à elle-même.\par
Tu n’as peut-être pas fait attention à une chose qui cause tous les jours ma surprise. Comment le monde est-il si peu peuplé, en comparaison de ce qu’il était autrefois ? Comment la nature a-t-elle pu perdre cette prodigieuse fécondité des premiers temps ? serait-elle déjà dans sa vieillesse, et tomberait-elle de langueur ?\par
J’ai resté plus d’un an en Italie, où je n’ai vu que le débris de cette ancienne Italie si fameuse autrefois. Quoique tout le monde habite les villes, elles sont entièrement désertes et dépeuplées : il semble qu’elles ne subsistent encore, que pour marquer le lieu où étaient ces cités puissantes dont l’histoire a tant parlé.\par
Il y a des gens qui prétendent que la seule ville de Rome contenait autrefois plus de peuples que le plus grand royaume de l’Europe n’en a aujourd’hui. Il y a eu tel citoyen romain, qui avait dix, et même vingt mille esclaves, sans compter ceux qui travaillaient dans les maisons de campagne ; et, comme on y comptait quatre ou cinq cent mille citoyens, on ne peut fixer le nombre de ses habitants sans que l’imagination ne se révolte.\par
Il y avait autrefois dans la Sicile de puissants royaumes, et des peuples nombreux, qui en ont disparu depuis : cette île n’a plus rien de considérable que ses volcans.\par
La Grèce est si déserte, qu’elle ne contient pas la centième partie de ses anciens habitants.\par
L’Espagne, autrefois si remplie, ne fait voir aujourd’hui que des campagnes inhabitées ; et la France n’est rien, en comparaison de cette ancienne Gaule dont parle César.\par
Les pays du Nord sont fort dégarnis ; et ils s’en faut bien que les peuples y soient, comme autrefois, obligés de se partager et d’envoyer dehors, comme des essaims, des colonies et des nations entières chercher de nouvelles demeures.\par
La Pologne et la Turquie en Europe n’ont presque plus de peuples.\par
On ne saurait trouver dans l’Amérique la deux-centième partie des hommes qui y formaient de si grands empires.\par
L’Asie n’est guères en meilleur état. Cette Asie mineure, qui contenait tant de puissantes monarchies, et un nombre si prodigieux de grandes villes, n’en a plus que deux ou trois. Quant à la grande Asie, celle qui est soumise au Turc n’est pas plus pleine, et pour celle qui est sous la domination de nos rois, si on la compare à l’état florissant où elle était autrefois, on verra qu’elle n’a qu’une très petite partie des habitants qui y étaient sans nombre du temps des Xerxès et des Darius.\par
Quant aux petits États, qui sont autour de ces grands empires, ils sont réellement déserts : tels sont les royaumes d’Irimette, de Circassie et de Guriel. Tous ces princes, avec de vastes États, comptent à peine cinquante mille sujets.\par
L’Égypte n’a pas moins manqué que les autres pays.\par
Enfin je parcours la terre, et je n’y trouve que délabrement : je crois la voir sortir des ravages de la peste et de la famine.\par
L’Afrique a toujours été si inconnue, qu’on ne peut en parler si précisément que des autres parties du monde : mais, à ne faire attention qu’aux côtes de la Méditerranée connues de tout temps, on voit qu’elle a extrêmement déchu de ce qu’elle était, lorsqu’elle était province romaine. Aujourd’hui ses princes sont si faibles, que ce sont les plus petites puissances du monde.\par
Après un calcul aussi exact qu’il peut l’être dans ces sortes de choses, j’ai trouvé qu’il y a à peine sur la terre la cinquantième partie des hommes qui y étaient du temps de César. Ce qu’il y a d’étonnant, c’est qu’elle se dépeuple tous les jours ; et si cela continue, dans dix siècles, elle ne sera qu’un désert.\par
Voilà, mon cher Usbek, la plus terrible catastrophe qui soit jamais arrivée dans le monde ; mais à peine s’en est-on aperçu, parce qu’elle est arrivée insensiblement, et dans le cours d’un grand nombre de siècle ; ce qui marque un vice intérieur, un venin secret et caché, une maladie de langueur, qui afflige la nature humaine.\par

\dateline{De Venise, le 10 de la lune de Rhégeb, 1718.}
\section[{Lettre CXIV. Usbek à Rhédi. À Venise.}]{Lettre CXIV. Usbek à Rhédi. À Venise.}\renewcommand{\leftmark}{Lettre CXIV. Usbek à Rhédi. À Venise.}

\noindent Le monde, mon cher Rhédi, n’est point incorruptible ; les cieux mêmes ne le sont pas : les astronomes sont des témoins oculaires de tous leurs changements, qui sont les effets bien naturels du mouvement universel de la matière.\par
La terre est soumise, comme les autres planètes, aux mêmes lois des mouvements ; elle souffre au dedans d’elle un combat perpétuel de ses principes : la mer et le continent semblent être dans une guerre éternelle ; chaque instant produit de nouvelles combinaisons.\par
Les hommes, dans une demeure si sujette aux changements, sont dans un état aussi \\
incertain : cent mille causes peuvent agir, dont la plus petite peut les détruire, et à plus forte raison augmenter ou diminuer leur nombre.\par
Je ne te parlerai pas de ces catastrophes particulières, si communes chez les historiens, qui ont détruit des villes et des royaumes entiers : il y en a de générales, qui ont mis bien des fois le genre humain à deux doigts de sa perte.\par
Les histoires sont pleines de ces pestes universelles, qui ont tour à tour désolé l’univers. Elles parlent d’une, entre autres, qui fut si violente qu’elle brûla jusqu’à la racine des plantes, et se fit sentir dans tout le monde connu, jusqu’à l’empire du Catay : un degré de plus de corruption aurait peut-être, dans un seul jour, détruit toute la nature humaine.\par
Il n’y a pas deux siècles que la plus honteuse de toutes les maladies se fit sentir en Europe, en Asie et en Afrique ; elle fit dans très peu de temps des effets prodigieux : c’était fait des hommes, si elle avait continué ses progrès avec la même furie. Accablés de maux dès leur naissance, incapables de soutenir le poids des charges de la société, ils auraient péri misérablement.\par
Qu’aurait-ce été si le venin eût été un peu plus exalté ? et il le serait devenu sans doute, si l’on n’avait été assez heureux pour trouver un remède aussi puissant que celui qu’on a découvert. Peut-être que cette maladie, attaquant les parties de la génération, aurait attaqué la génération même.\par
Mais pourquoi parler de la destruction qui aurait pu arriver au genre humain ? N’est-elle pas arrivée en effet, et le déluge ne le réduisit-il pas à une seule famille ?\par
Ceux qui connaissent la nature, et qui ont de Dieu une idée raisonnable, peuvent-ils comprendre que la matière et les choses créées n’aient que six mille ans ? que Dieu ait différé pendant toute l’éternité ses ouvrages, et n’ai usé que hier de sa puissance créatrice ? Serait-ce parce qu’il ne l’aurait pas pu, ou parce qu’il ne l’aurait pas voulu ? Mais s’il ne l’a pas pu dans un temps, il ne l’a pas pu dans l’autre. C’est donc parce qu’il ne l’a pas voulu. Mais, comme il n’y a pas de succession dans Dieu, si l’on admet qu’il ait voulu quelque chose une fois, il l’a voulu toujours, et dès le commencement.\par
Il ne faut donc pas compter les années du monde ; le nombre des grains de sable de la mer ne leur est pas plus comparable qu’un instant.\par
Cependant tous les historiens nous parlent d’un premier père : ils nous font voir la nature humaine naissante. N’est-il pas naturel de penser qu’Adam fut sauvé d’un malheur commun, comme Noé le fut du déluge : et que ces grands événements ont été fréquents sur la terre, depuis la création du monde ?\par
Mais toutes les destructions ne sont pas violentes. Nous voyons plusieurs parties de la terre se lasser de fournir à la subsistance des hommes : que savons-nous si la terre entière n’a pas des causes générales, lentes, et imperceptibles, de lassitude ?\par
J’ai été bien aise de te donner ces idées générales, avant de répondre plus particulièrement à ta lettre sur la diminution des peuples, arrivée depuis dix-sept à dix-huit siècles. Je te ferai voir dans une lettre suivante, qu’indépendamment des causes physiques, il y en a de morales qui ont produit cet effet.\par

\dateline{De Paris, le 8 de la lune de Chahban, 1718.}
\section[{Lettre CXV. Usbek au même.}]{Lettre CXV. Usbek au même.}\renewcommand{\leftmark}{Lettre CXV. Usbek au même.}

\noindent Tu cherches la raison pourquoi la terre est moins peuplée qu’elle ne l’était autrefois : et si tu y fais bien attention, tu verras que la grande différence vient de celle qui est arrivée dans les mœurs.\par
Depuis que la religion chrétienne et la mahométane ont partagé le monde romain, les choses sont bien changées : il s’en faut bien que ces deux religions soient aussi favorables à la propagation de l’espèce que celle de ces maîtres de l’univers.\par
Dans cette dernière, la polygamie était défendue : et en cela elle avait un très grand avantage sur la religion mahométane ; le divorce y était permis : ce qui lui en donnait un autre, non moins considérable, sur la chrétienne.\par
Je ne trouve rien de si contradictoire que cette pluralité des femmes permise par le saint Alcoran, et l’ordre de les satisfaire ordonné par le même livre. Voyez vos femmes, dit le prophète, parce que vous leur êtes nécessaire comme leurs vêtements, et qu’elles vous sont nécessaires comme vos vêtements. Voilà un précepte qui rend la vie d’un véritable musulman bien laborieuse. Celui qui a les quatre femmes établies par la loi, et seulement autant de concubines et d’esclaves, ne doit-il pas être accablé de tant de vêtements ?\par
Vos femmes sont vos labourages, dit encore le prophète ; approchez-vous donc de vos labourages : faites du bien pour vos âmes ; et vous le trouverez un jour.\par
Je regarde un bon musulman comme un athlète, destiné à combattre sans relâche ; mais qui bientôt, faible et accablé de ses premières fatigues, languit dans le champ même de la victoire ; et se trouve, pour ainsi dire, enseveli sous ses propres triomphes.\par
La nature agit toujours avec lenteur, et pour ainsi dire avec épargne : ses opérations ne sont jamais violentes ; jusque dans ses productions elle veut de la tempérance ; elle ne va jamais qu’avec règle et mesure ; si on la précipite, elle tombe bientôt dans la langueur : elle emploie toute la force qui lui reste à se conserver, perdant absolument sa vertu productrice et sa puissance générative.\par
C’est dans cet état de défaillance que nous met toujours ce grand nombre de femmes, plus propres à nous épuiser qu’à nous satisfaire. Il est très ordinaire parmi nous de voir un homme dans un sérail prodigieux avec un très petit nombre d’enfants : ces enfants mêmes sont la plupart du temps faibles et malsains, et se sentent de la langueur de leur père.\par
Ce n’est pas tout : ces femmes, obligées à une continence forcée, on besoin d’avoir des gens pour les garder, qui ne peuvent être que des eunuques : la religion, la jalousie, et la raison même, ne permettent pas dans laisser approcher d’autres ; ces gardiens doivent être en grand nombre, soit afin de maintenir la tranquillité au dedans parmi les guerres que ces femmes se font sans cesse, soit enfin pour empêcher les entreprises du dehors. Ainsi un homme qui a dix femmes ou concubines n’a pas trop d’autant d’eunuques pour les garder. Mais quelle perte pour la société que ce grand nombre d’hommes morts dès leur naissance ! quelle dépopulation ne doit-il pas s’ensuivre !\par
Les filles esclaves qui sont dans le sérail pour servir avec les eunuques ce grand nombre de femmes, y vieillissent presque toujours dans une affligeante virginité : elles ne peuvent pas se marier pendant qu’elles y restent ; et leurs maîtresses, une fois accoutumée à elles, ne s’en défont presque jamais.\par
Voilà comme un seul homme occupe lui seul tant de sujets de l’un et l’autre sexe à ses plaisirs, les fait mourir pour l’État, et les rend inutiles à la propagation de l’espèce.\par
Constantinople et Ispahan sont les capitales des deux plus grands empires du monde : c’est là que tout doit aboutir, et que les peuples, attirés de mille manières, se rendent de toutes parts. Cependant elles périssent d’elles-mêmes, et elles seraient bientôt détruites, si les souverains n’y faisaient venir, presque à chaque siècle, des nations entières pour les repeupler. J’épuiserai ce sujet dans une autre lettre.\par

\dateline{De Paris, le 13 de la lune de Chahban, 1718.}
\section[{Lettre CXVI. Usbek au même.}]{Lettre CXVI. Usbek au même.}\renewcommand{\leftmark}{Lettre CXVI. Usbek au même.}

\noindent Les Romains n’avaient pas moins d’esclaves que nous ; ils en avaient même plus : mais ils en faisaient un meilleur usage.\par
Bien loin d’empêcher, par des voies forcées, la multiplication de ces esclaves, ils la favorisaient au contraire de tout leur pouvoir ; ils les associaient le plus qu’ils pouvaient par des espèces de mariages : par ce moyen, ils remplissaient leurs maisons de domestiques de tous les sexes, de tous les âges, et l’État d’un peuple innombrable.\par
Ces enfants, qui faisaient à la longue la richesse d’un maître, naissaient sans nombre autour de lui : il était seul chargé de leur nourriture et de leur éducation ; les pères, libres de ce fardeau, suivaient uniquement le penchant de la nature, et multipliaient sans craindre une trop nombreuse famille.\par
Je t’ai dit que, parmi nous, tous les esclaves sont occupés à garder nos femmes, et à rien de plus ; qu’ils sont, à l’égard de l’État, dans une perpétuelle léthargie : de manière qu’il faut restreindre à quelques hommes libres, à quelques chefs de familles, la culture des arts et des terres, lesquels même s’y donnent le moins qu’ils peuvent.\par
Il n’en était pas de même chez les Romains : la république se servait avec un avantage infini de ce peuple d’esclaves. Chacun d’eux avait son pécule, qu’il possédait aux conditions que son maître lui imposait ; avec ce pécule il travaillait et se tournait du côté où le portait son industrie. Celui-ci faisait la banque : celui-là se donnait au commerce de la mer ; l’un vendait des marchandises en détail ; l’autre s’appliquait à quelque art mécanique, ou bien affermait et faisait valoir des terres : mais il n’y en avait aucun qui ne s’attachât de tout son pouvoir à faire profiter son pécule, qui lui procurait en même temps l’aisance dans la servitude présente, et l’espérance d’une liberté future : cela faisait un peuple laborieux, animait les arts et l’industrie.\par
Ces esclaves, devenus riches par leurs soins et leur travail, se faisaient affranchir, et devenaient citoyens. La république se réparait sans cesse, et recevait dans son sein de nouvelles familles, à mesure que les anciennes se détruisaient.\par
J’aurai peut-être, dans mes lettres suivantes, occasion de te prouver que plus il y a des hommes dans un État, plus le commerce y fleurit ; je prouverai aussi facilement que plus le commerce y fleurit, plus le nombre des hommes y augmente : ces deux choses s’entraident, et se favorisent nécessairement.\par
Si cela est, combien ce nombre prodigieux d’esclaves, toujours laborieux, devait-il s’accroître et s’augmenter ! L’industrie et l’abondance les faisaient naître ; et eux, de leur côté, faisaient naître l’abondance et l’industrie.\par

\dateline{De Paris, le 16 de la lune de Chahban, 1718.}
\section[{Lettre CXVII. Usbek au même.}]{Lettre CXVII. Usbek au même.}\renewcommand{\leftmark}{Lettre CXVII. Usbek au même.}

\noindent Nous avons jusqu’ici parlé des pays mahométans, et cherché la raison pourquoi ils étaient moins peuplés que ceux qui étaient soumis à la domination des Romains : examinons à présent ce qui a produit cet effet chez les chrétiens.\par
Le divorce était permis dans la religion païenne, et il fut défendu aux chrétiens. Ce changement, qui parut d’abord de si petite conséquence, eut insensiblement des suites terribles, et telles qu’on peut à peine les croire.\par
On ôta non seulement toute la douceur du mariage, mais aussi l’on donna atteinte à sa fin : en voulant resserrer ses nœuds, on les relâcha ; et au lieu d’unir les cœurs, comme on le prétendait, on les sépara pour jamais.\par
Dans une action si libre, et où le cœur doit avoir tant de part, on mit la gêne, la nécessité, et la fatalité du destin même. On compta pour rien les dégoûts, les caprices, et l’insociabilité des humeurs ; on voulut fixer le cœur, c’est-à-dire ce qu’il y a de plus variable et de plus inconstant dans la nature : on attacha sans retour et sans espérance des gens accablés l’un de l’autre, et presque toujours mal assortis ; et l’on fit comme ces tyrans qui faisaient lier des hommes vivants à des corps morts.\par
Rien ne contribuait plus à l’attachement mutuel que la faculté du divorce : un mari et une femme étaient portés à soutenir patiemment les peines domestiques, sachant qu’ils étaient maîtres de les faire finir : et ils gardaient souvent ce pouvoir en main toute leur vie sans en user, par cette seule considération qu’ils étaient libres de le faire.\par
Il n’en est pas de même des chrétiens, que leurs peines présentes désespèrent pour l’avenir : ils ne voient, dans les désagréments du mariage, que leur durée et, pour ainsi dire, leur éternité : de là viennent les dégoûts, les discordes, les mépris ; et c’est autant de perdu pour la postérité. À peine a-t-on trois ans de mariage, qu’on en néglige l’essentiel ; on passe ensemble trente ans de froideur : il se forme des séparations intestines aussi fortes, et peut-être plus pernicieuses, que si elles étaient publiques : chacun vit et reste de son côté ; et tout cela au préjudice des races futures. Bientôt un homme dégoûté d’une femme éternelle, se livrera aux filles de joie : commerce honteux et si contraire à la société ; lequel, sans remplir l’objet du mariage, n’en représente tout au plus que les plaisirs.\par
Si, de deux personnes ainsi liées, il y en a une qui n’est pas propre au dessein de la nature et à la propagation de l’espèce, soit par son tempérament, soit par son âge, elle ensevelit l’autre avec elle, et la rend aussi inutile qu’elle l’est elle-même.\par
Il ne faut donc pas s’étonner si l’on voit chez les chrétiens tant de mariages fournir un si petit nombre de citoyens. Le divorce est aboli : les mariages mal assortis ne se raccommodent plus ; les femmes ne passent plus, comme chez les Romains, successivement dans les mains de plusieurs maris, qui en tiraient, dans le chemin, le meilleur parti qu’il était possible.\par
J’ose le dire : si, dans une république comme Lacédémone, où les citoyens étaient sans cesse gênés par des lois singulières et subtiles, et dans laquelle il n’y avait qu’une famille, qui était la république, il avait été établi que les maris changeassent de femme tous les ans, il en serait né un peuple innombrable.\par
Il est assez difficile de faire bien comprendre la raison qui a porté les chrétiens à abolir le divorce. Le mariage, chez toutes les nations du monde, est un contrat susceptible de toutes les conventions, et on n’en a dû bannir que celles qui auraient pu en affaiblir l’objet ; mais les chrétiens ne le regardent pas dans ce point de vue : aussi ont-ils bien de la peine à dire ce que c’est. Ils ne le font pas consister dans le plaisir des sens ; au contraire, comme je te l’ai déjà dit, il semble qu’ils veulent l’en bannir autant qu’ils le peuvent : mais c’est une image, une figure, et quelque chose de mystérieux, que je ne comprends point.\par

\dateline{De Paris, le 19 de la lune de Chahban, 1718.}
\section[{Lettre CXVIII. Usbek au même}]{Lettre CXVIII. Usbek au même}\renewcommand{\leftmark}{Lettre CXVIII. Usbek au même}

\noindent La prohibition du divorce n’est pas la seule cause de la dépopulation des pays chrétiens : le grand nombre d’eunuques qu’ils ont parmi eux n’en est pas une moins considérable.\par
Je parle des prêtres et des dervis de l’un et de l’autre sexe, qui se vouent à une continence éternelle : c’est chez les chrétiens la vertu par excellence ; en quoi je ne les comprends pas, ne sachant ce que c’est qu’une vertu dont il ne résulte rien.\par
Je trouve que leurs docteurs se contredisent manifestement quand ils disent que le mariage est saint, et que le célibat, qui lui est opposé, l’est encore d’avantage, sans compter qu’en fait de préceptes et de dogmes fondamentaux, le bien est toujours le mieux.\par
Le nombre de ces gens faisaient profession de célibat est prodigieux. Les pères y condamnaient autrefois les enfants dès le berceau ; aujourd’hui ils s’y vouent eux-mêmes dès l’âge de quatorze ans : ce qui revient à peu près à la même chose.\par
Ce métier de continence a anéanti plus d’hommes que les pestes et les guerres les plus sanglantes n’ont jamais fait. On voit dans chaque maison religieuse une famille éternelle, où il ne naît personne, et qui s’entretient aux dépens de toutes les autres. Ces maisons sont toujours ouvertes, comme autant de gouffres où s’ensevelissent les races futures.\par
Cette politique est bien différente de celle des Romains, qui établissaient des lois pénales contre ceux qui se refusaient aux lois du mariage, et voulaient jouir d’une liberté si contraire à l’utilité publique.\par
Je ne te parle ici que des pays catholiques. Dans la religion protestante, tout le monde est en droit de faire des enfants : elle ne souffre ni prêtres ni dervis ; et si, dans l’établissement de cette religion qui ramenait tout aux premiers temps, ses fondateurs n’avaient été accusés sans cesse d’intempérance, il ne faut pas douter qu’après avoir rendu la pratique du mariage universelle, ils n’en eussent encore adouci le joug, et achevé d’ôter toute la barrière qui sépare, en ce point, le Nazaréen et Mahomet.\par
Mais, quoi qu’il en soit, il est certain que la religion donne aux protestants un avantage infini sur les catholiques.\par
J’ose le dire : dans l’état présent où est l’Europe, il n’est pas possible que la religion catholique y subsiste cinq cents ans.\par
Avant l’abaissement de la puissance d’Espagne, les catholiques étaient beaucoup plus forts que les protestants. Ces derniers sont peu à peu parvenus à un équilibre, et aujourd’hui la balance commence à l’emporter de leur côté. Cette supériorité augmentera tous les jours : les protestants deviendront plus riches et plus puissants, et les catholiques plus faibles.\par
Les pays protestants doivent être, et sont réellement, plus peuplés que les catholiques : d’où il suit, premièrement, que les tribus y sont plus considérables, parce qu’ils augmentent à proportion de ceux qui les payent ; secondement, que les terres y sont mieux cultivées ; enfin que le commerce y fleurit davantage, parce qu’il y a plus de gens qui ont une fortune à faire, et qu’avec plus de besoins on y a plus de ressources pour les remplir. Quand il n’y a que le nombre de gens suffisants pour la culture des terres, il faut que le commerce périsse ; et lorsqu’il n’y a que celui qui est nécessaire pour entretenir le commerce, il faut que la culture des terres manque, c’est-à-dire il faut que tous les deux tombent en même temps, parce que l’on ne s’attache jamais à l’un, que ce ne soit aux dépens de l’autre.\par
Quant aux pays catholiques, non seulement la culture des terres y est abandonnée, mais même l’industrie y est pernicieuse ; elle ne consiste qu’à apprendre cinq ou six mots d’une langue morte. Dès qu’un homme a cette provision par devers lui, il ne doit plus s’embarrasser de sa fortune : il trouve dans le cloître une vie tranquille, qui dans le monde lui aurait coûté des sueurs et des peines.\par
Ce n’est pas tout : les dervis ont en leurs mains presque toutes les richesses de l’État ; c’est une société de gens avares, qui prennent toujours, et ne rendent jamais : ils accumulent sans cesse des revenus pour acquérir des capitaux. Tant de richesses tombent, pour ainsi dire, en paralysie, plus de circulation, plus de commerce, plus d’arts, plus de manufactures.\par
Il n’y a point de prince protestant qui ne lève sur ses peuples dix fois plus d’impôts que le pape n’en lève sur ses sujets ; cependant ces derniers sont misérables, pendant que les autres vivent dans l’opulence. Le commerce ranime tout chez les uns, et le monachisme porte la mort partout chez les autres.\par

\dateline{De Paris, le 26 de la lune de Chahban, 1718.}
\section[{Lettre CXIX. Usbek au même.}]{Lettre CXIX. Usbek au même.}\renewcommand{\leftmark}{Lettre CXIX. Usbek au même.}

\noindent Nous n’avons plus rien à dire de l’Asie et de l’Europe : passons à l’Afrique. On ne peut guère parler que de ses côtes, parce qu’on n’en connaît pas l’intérieur.\par
Celles de Barbarie, où la religion mahométane est établie, ne sont plus si peuplées qu’elles étaient du temps des Romains, par les raisons que nous avons déjà dites. Quant aux côtes de la Guinée, elles doivent être furieusement dégarnies depuis deux cents ans que les petits rois, ou chefs de villages, vendent leurs sujets aux princes d’Europe, pour les porter dans leurs colonies en Amérique.\par
Ce qu’il y a de singulier, c’est que cette Amérique qui reçoit tous les ans tant de nouveaux habitants, est elle-même déserte, et ne profite point des pertes continuelles de l’Afrique. Ces esclaves, qu’on transporte dans un autre climat, y périssent par milliers ; et les travaux des mines, où l’on occupe sans cesse et les naturels du pays et les étrangers, les exhalaisons malignes qui en sortent, le vif-argent dont il faut faire un continuel usage, les détruisent sans ressource.\par
Il n’y a rien de si extravagant que de faire périr un nombre innombrable d’hommes pour tirer du fond de la terre l’or et l’argent, ces métaux d’eux-mêmes absolument inutiles, et qui ne sont des richesses que parce qu’on les a choisis pour en être les signes.\par

\dateline{De Paris, le dernier de la lune de Chahban, 1718.}
\section[{Lettre CXX. Usbek au même.}]{Lettre CXX. Usbek au même.}\renewcommand{\leftmark}{Lettre CXX. Usbek au même.}

\noindent La fécondité d’un peuple dépend quelquefois des plus petites circonstances du monde : de manière qu’il ne faut souvent qu’un nouveau tour dans son imagination pour le rendre beaucoup plus nombreux qu’il n’était.\par
Les Juifs, toujours exterminés et toujours renaissants, ont réparé leurs pertes et leurs destructions continuelles, par cette seule espérance qu’on parmi eux toutes les familles, d’y voir naître un roi puissant qui sera le maître de la terre.\par
Les anciens rois de Perse n’avaient tant de milliers de sujets qu’à cause de ce dogme de la religion des mages, que les actes les plus agréables à Dieu que les hommes puissent faire, c’était de faire un enfant, labourer un champ, et planter un arbre.\par
Si la Chine a dans son sein un peuple si prodigieux, cela ne vient que d’une certaine manière de penser : car, comme les enfants regardent leurs pères comme des dieux, qu’ils les respectent comme tels dès cette vie, qu’ils les honorent après leur mort par des sacrifices dans lesquels ils croient que leurs âmes, anéanties dans le Tyen, reprennent une nouvelle vie, chacun est porté à augmenter une famille si soumise dans cette vie, et si nécessaire dans l’autre.\par
D’un autre côté, les pays des mahométans deviennent tous les jours déserts, à cause d’une opinion qui, toute sainte qu’elle est, ne laisse pas d’avoir des effets très pernicieux lorsqu’elle est enracinée dans les esprits. Nous nous regardons comme des voyageurs qui ne doivent penser qu’à une autre patrie ; les travaux utiles et durables, les soins pour assurer la fortune de nos enfants, les projets qui tendent au-delà d’une vie courte et passagère, nous paraissent quelque chose d’extravagant. Tranquilles pour le présent, sans inquiétude pour l’avenir, nous ne prenons la peine ni de réparer les édifices publics, ni de défricher les terres incultes, ni de cultiver celles qui sont en état de recevoir nos soins : nous vivons dans une insensibilité générale, et nous laissons tout faire à la providence.\par
C’est un esprit de vanité qui a établi chez les Européens l’injuste droit d’aînesse, si défavorable à la propagation, en ce qu’il porte l’attention d’un père sur un seul de ses enfants, et détourne ses yeux de tous les autres ; en ce qu’il l’oblige, pour rendre solide la fortune d’un seul, de s’opposer à l’établissement de plusieurs ; enfin en ce qu’il détruit l’égalité des citoyens, qui en fait toute l’opulence.\par

\dateline{De Paris, le 4 de la lune de Rhamazan, 1718.}
\section[{Lettre CXXI. Usbek au même.}]{Lettre CXXI. Usbek au même.}\renewcommand{\leftmark}{Lettre CXXI. Usbek au même.}

\noindent Les pays habités par les sauvages sont ordinairement peu peuplés, par l’éloignement qu’ils ont presque tous pour le travail et la culture de la terre. Cette malheureuse aversion est si forte que, lorsqu’ils font quelque imprécation contre quelqu’un de leurs ennemis, ils ne lui souhaitent autre chose que d’être réduit à labourer un champ, croyant qu’il n’y a que la chasse et la pêche qui soit un exercice noble et digne d’eux.\par
Mais, comme il y a souvent des années où la chasse et la pêche rendent très peu, ils sont désolés par des famines fréquentes ; sans compter qu’il n’y a pas de pays si abondant en gibier et en poisson qui puisse donner la subsistance à un grand peuple, parce que les animaux fuient toujours les endroits trop habités.\par
D’ailleurs, les bourgades de sauvages, au nombre de deux ou trois cents habitants, isolées les unes des autres, ayant des intérêts aussi séparés que ceux de deux empires, ne peuvent pas se soutenir, parce qu’elles n’ont pas la ressource des grands États, dont toutes les parties se répondent et se secourent mutuellement.\par
Il y a chez les sauvages une autre coutume qui n’est pas moins pernicieuse que la première : c’est la cruelle habitude où sont les femmes de se faire avorter, afin que leur grossesse ne les rende pas désagréables à leurs maris.\par
Il y a ici des lois terribles contre ce désordre ; elles vont jusqu’à la fureur. Toute fille qui n’a point été déclarer sa grossesse au magistrat est punie de mort si son fruit périt : la pudeur et la honte, les accidents même, ne l’excusent jamais.\par

\dateline{De Paris, le 9 de la lune de Rhamazan, 1718.}
\section[{Lettre CXXII. Usbek au même.}]{Lettre CXXII. Usbek au même.}\renewcommand{\leftmark}{Lettre CXXII. Usbek au même.}

\noindent L’effet ordinaire des colonies est d’affaiblir les pays d’où on les tire, sans peupler ceux où on les envoie.\par
Il faut que les hommes restent où ils sont : il y a des maladies qui viennent de ce qu’on change un bon air contre un mauvais ; d’autres qui viennent précisément de ce qu’on en change.\par
L’air se charge, comme les plantes, des particules de la terre de chaque pays. Il agit tellement sur nous que notre tempérament en est fixé. Lorsque nous sommes transportés dans un autre pays, nous devenons malades. Les liquides étant accoutumés à une certaine consistance, les solides à une certaine disposition, tous les deux, à un certain degré de mouvement, n’en peuvent plus souffrir d’autres, et ils résistent à un nouveau pli.\par
Quand un pays est désert, c’est un préjugé de quelque vice particulier de la nature du climat : ainsi, quand on ôte les hommes d’un ciel heureux pour les envoyer dans un tel pays, on fait précisément le contraire de ce qu’on se propose.\par
Les Romains savaient cela par expérience ; ils reléguaient tous les criminels en Sardaigne, et ils y faisaient passer des Juifs. Il fallut se consoler de leur perte ; chose que le mépris qu’ils avaient pour ces misérables rendait très facile.\par
Le grand Cha-Abas, voulant ôter aux Turcs, le moyen d’entretenir de grosses armées sur les frontières, transporta presque tous les Arméniens hors de leur pays, et en envoya plus de vingt mille familles dans la province de Guilan, qui périrent presque toutes en très peu de temps.\par
Tous les transports de peuples faits à Constantinople n’ont jamais réussi.\par
Ce nombre prodigieux de nègres dont nous avons parlé n’a point rempli l’Amérique.\par
Depuis la destruction des Juifs sous Adrien, la Palestine est sans habitants.\par
Il faut donc avouer que les grandes destructions sont presque irréparables, parce qu’un peuple qui manque à un certain point reste dans le même état ; et si, par hasard il se rétablit, il faut des siècles pour cela.\par
Que si, dans un état de défaillance, la moindre des circonstances dont nous avons parlé vient à concourir, non seulement il ne se répare pas, mais il dépérit tous les jours, et tend à son anéantissement.\par
L’expulsion des Maures d’Espagne se fait encore sentir comme le premier jour : bien loin que ce vide se remplisse, il devient tous les jours plus grand.\par
Depuis la dévastation de l’Amérique, les Espagnols, qui ont pris la place de ses anciens habitants, n’ont pu la repeupler ; au contraire, par une fatalité que je ferais mieux de nommer une justice divine, les destructeurs se détruisent eux-mêmes, et se consument tous les jours.\par
Les princes ne doivent donc point songer à peupler de grands pays par des colonies. Je ne dis pas qu’elles ne réussissent quelquefois, il y a des climats si heureux, que l’espèce s’y multiplie toujours : témoin ces îles\footnote{L'auteur parle peut-être de l’île Bourbon.} qui ont été peuplées par des malades que quelques vaisseaux y avaient abandonnés, et qui y recouvraient aussitôt la santé.\par
Mais quand ces colonies réussiraient, au lieu d’augmenter la puissance, elles ne feraient que la partager, à moins qu’elles n’eussent très peu d’étendue, comme sont celles que l’on envoie pour occuper quelque place pour le commerce.\par
Les Carthaginois avaient, comme les Espagnols, découvert l’Amérique, ou au moins de grandes îles dans lesquelles ils faisaient un commerce prodigieux : mais quand ils virent le nombre de leurs habitants diminuer, cette sage république défendit à ses sujets ce commerce et cette navigation.\par
J’ose le dire : au lieu de faire passer les Espagnols dans les Indes, il faudrait faire repasser les Indiens et les métifs en Espagne ; il faudrait rendre à cette monarchie tous ses peuples dispersés ; et, si la moitié seulement de ces grandes colonies se conservait, l’Espagne deviendrait la puissance de l’Europe la plus redoutable.\par
On peut comparer les empires à un arbre dont les branches trop étendues ôtent tout le suc du tronc, et ne servent qu’à faire de l’ombrage.\par
Rien ne devrait corriger les princes de la fureur des conquêtes lointaines que l’exemple des Portugais et des Espagnols.\par
Ces deux nations ayant conquis, avec une rapidité inconcevable, des royaumes immenses, plus étonnées de leurs victoires que les peuples vaincus de leur défaite, songèrent aux moyens de les conserver, et prirent chacune pour cela une voie différente.\par
Les Espagnols, désespérant de retenir les nations vaincues dans la fidélité, prirent le parti de les exterminer, et d’y envoyer d’Espagne des peuples fidèles : jamais dessein horrible ne fut plus ponctuellement exécuté. On vit un peuple, aussi nombreux que tous ceux de l’Europe ensemble, disparaître de la terre à l’arrivée de ces barbares, qui semblèrent, en découvrant les Indes, avoir voulu en même temps découvrir aux hommes quel était le dernier période de la cruauté.\par
Par cette barbarie, ils conservèrent ce pays sous leur domination. Juge par là combien les conquêtes sont funestes, puisque les effets en sont tels : car enfin ce remède affreux était unique. Comment auraient-ils pu retenir tant de millions d’hommes dans l’obéissance ? Comment soutenir une guerre civile de si loin ? Que seraient-ils devenus, s’ils avaient donné le temps à ces peuples de revenir de l’admiration où ils étaient de l’arrivée de ces nouveaux dieux et de la crainte de leurs foudres ?\par
Quant aux Portugais, ils prirent une voie tout opposée ; ils n’employèrent pas les cruautés : aussi furent-ils bientôt chassés de tous les pays qu’ils avaient découverts. Les Hollandais favorisèrent la rébellion de ces peuples, et en profitèrent.\par
Quel prince envierait le sort de ces conquérants ? Qui voudrait de ces conquêtes à ces conditions ? Les uns en furent aussitôt chassés ; les autres en firent des déserts, et rendirent de même leur propre pays.\par
C’est le destin des héros de se ruiner à conquérir des pays qu’ils perdent soudain, ou à soumettre des nations qu’ils sont obligés eux-mêmes de détruire ; comme cet insensé qui se consumait à acheter des statues qu’il jetait dans la mer, et des glaces qu’il brisait aussitôt.\par

\dateline{De Paris, le 18 de la lune de Rhamazan, 1718}
\section[{Lettre CXXIII. Usbek au même.}]{Lettre CXXIII. Usbek au même.}\renewcommand{\leftmark}{Lettre CXXIII. Usbek au même.}

\noindent La douceur du gouvernement contribue merveilleusement à la propagation de l’espèce. Toutes les républiques en sont une preuve constante ; et, plus que toutes, la Suisse et la Hollande, qui sont les deux plus mauvais pays de l’Europe, si l’on considère la nature du terrain, et qui cependant sont les plus peuplés.\par
Rien n’attire plus les étrangers que la liberté, et l’opulence qui la suit toujours : l’une se fait rechercher par elle-même, et les besoins attirent dans les pays où l’on trouve l’autre.\par
L’espèce se multiplie dans un pays où l’abondance fournit aux enfants, sans rien diminuer de la subsistance des pères.\par
L’égalité même des citoyens, qui produit ordinairement de l’égalité dans les fortunes, porte l’abondance et la vie dans toutes les parties du corps politique, et la répand partout.\par
Il n’en est pas de même des pays soumis au pouvoir arbitraire : le prince, les courtisans, et quelques particuliers, possèdent toutes les richesses, pendant que tous les autres gémissent dans une pauvreté extrême.\par
Si un homme est mal à son aise, et qu’il sente qu’il fera des enfants plus pauvres que lui, il ne se mariera pas ; ou s’il se marie, il craindra d’avoir un trop grand nombre d’enfants, qui pourraient achever de déranger sa fortune, et qui descendraient de la condition de leur père.\par
J’avoue que le rustique ou paysan, étant une fois marié, peuplera indifféremment, soit qu’il soit riche, soit qu’il soit pauvre : cette considération ne le touche pas : il a toujours un héritage sûr à laisser à ses enfants, qui est son hoyau ; et rien ne l’empêche jamais de suivre aveuglément l’instinct de la nature.\par
Mais à quoi sert dans un État ce nombre d’enfants qui languissent dans la misère ? Ils périssent presque tous à mesure qu’ils naissent ; ils ne prospèrent jamais : faibles et débiles, ils meurent en détail de mille manières, tandis qu’ils sont emportés en gros par les fréquentes maladies populaires, que la misère et la mauvaise nourriture produisent toujours ; ceux qui en échappent atteignent l’âge viril sans en avoir la force, et languissent tout le reste de leur vie.\par
Les hommes sont comme les plantes, qui ne croissent jamais heureusement si elles ne sont bien cultivées : chez les peuples misérables, l’espèce perd, et même quelquefois dégénère.\par
La France peut fournir un grand exemple de tout ceci. Dans les guerres passées, la crainte où étaient tous les enfants de famille qu’on ne les enrôlât dans la milice les obligeait de se marier, et cela dans un âge trop tendre, et dans le sein de la pauvreté. De tant de mariages il naissait bien des enfants, que l’on cherche encore en France, et que la misère, la famine et les maladies en ont fait disparaître.\par
Que si, dans un ciel aussi heureux, dans un royaume aussi policé que la France, on fait de pareilles remarques, que sera-ce dans les autres États ?\par

\dateline{De Paris, le 23 de la lune de Rhamazan, 1718.}
\section[{Lettre CXXIV. Usbek au Mollak Mehemet Ali, gardien des Trois tombeaux. À Com.}]{Lettre CXXIV. Usbek au Mollak Mehemet Ali, gardien des Trois tombeaux. À Com.}\renewcommand{\leftmark}{Lettre CXXIV. Usbek au Mollak Mehemet Ali, gardien des Trois tombeaux. À Com.}

\noindent Que nous servent les jeûnes des immaums et les cilices des mollaks ? La main de Dieu s’est deux fois appesantie sur les enfants de la loi, le soleil s’obscurcit, et semble n’éclairer plus que leurs défaites ; leurs armées s’assemblent, et elles sont dissipées comme la poussière.\par
L’empire des Osmanlins est ébranlé par les deux plus grands échecs qu’il ait jamais reçus : un moufti chrétien ne le soutient qu’avec peine ; le grand vizir d’Allemagne est le fléau de Dieu, envoyé pour châtier les sectateurs d’Omar ; il porte partout la colère du ciel, irrité contre leur rébellion et leur perfidie.\par
Esprit sacré des immaums, tu pleures nuit et jour sur les enfants du prophète que le détestable Omar a dévoyés ; tes entrailles s’émeuvent à la vue de leurs malheurs ; tu désires leur conversion, et non pas leur perte ; tu voudrais les voir réunis sous l’étendard d’Ali, par les larmes des saints ; et non pas dispersés dans les montagnes et dans les déserts par la terreur des infidèles.\par

\dateline{De Paris, le 1er de la lune de Chalval, 1718.}
\section[{Lettre CXXV. Usbek à Rhédi. À Venise.}]{Lettre CXXV. Usbek à Rhédi. À Venise.}\renewcommand{\leftmark}{Lettre CXXV. Usbek à Rhédi. À Venise.}

\noindent Quel peut être le motif de ces libéralités immenses que les princes versent sur leurs courtisans ? veulent-ils se les attacher ? ils leur sont déjà acquis autant qu’ils peuvent l’être. Et d’ailleurs, s’ils acquièrent quelques-uns de leurs sujets en les achetant, il faut bien, par la même raison, qu’ils en perdent une infinité d’autres en les appauvrissant.\par
Quand je pense à la situation des princes, toujours entourés d’hommes avides et insatiables, je ne puis que les plaindre : et je les plains encore davantage lorsqu’ils n’ont pas la force de résister à des demandes toujours onéreuses à ceux qui ne demandent rien.\par
Je n’entends jamais parler de leurs libéralités, des grâces et des pensions qu’ils accordent, que je ne me livre à mille réflexions : une foule d’idées se présente à mon esprit ; il me semble que j’entends publier cette ordonnance :\par
"Le courage infatigable de quelques-uns de nos sujets à nous demander des pensions ayant exercé sans relâche notre magnificence royale, nous avons enfin cédé à la multitude des requêtes qu’ils nous ont présentées, lesquelles ont fait jusqu’ici la plus grande sollicitude du trône. Ils nous ont représenté qu’ils n’ont point manqué, depuis notre avènement à la couronne, de se trouver à notre lever ; que nous les avons toujours vus sur notre passage immobiles comme des bornes ; et qu’ils se sont extrêmement élevés pour regarder, sur les épaules les plus hautes, notre sérénité. Nous avons même reçu plusieurs requêtes de la part de quelques personnes du beau sexe, qui nous ont suppliés de faire attention qu’il était notoire qu’elles sont d’un entretien très difficile ; quelques-unes même très surannées nous ont prié, branlant la tête, de faire attention qu’elles ont fait l’ornement de la cour des rois nos prédécesseurs ; et que, si les généraux de leurs armées ont rendu l’État redoutable par leurs faits militaires, elles n’ont point rendu la cour moins célèbre par leurs intrigues. Ainsi, désirant traiter les suppliants avec bonté, et leur accorder toutes leurs prières, nous avons ordonné ce qui suit :\par
Que tout laboureur ayant cinq enfants retranchera journellement la cinquième partie du pain qu’il leur donne. Enjoignons aux pères de famille de faire la diminution, sur chacun d’eux, aussi juste que faire se pourra.\par
Défendons expressément à tous ceux qui s’appliquent à la culture de leurs héritages, ou qui les ont donnés à titre de ferme, d’y faire aucune réparation, de quelque espèce qu’elle soit.\par
Ordonnons que toutes personnes qui s’exercent à des travaux vils et mécaniques, lesquelles n’ont jamais été au lever de notre majesté, n’achètent désormais d’habits, à eux, à leurs femmes et à leurs enfants, que de quatre ans en quatre ans ; leur interdisons en outre très étroitement ces petites réjouissances qu’ils avaient coutume de faire, dans leurs familles, les principales fêtes de l’année.\par
Et, d’autant que nous demeurons averti que la plupart des bourgeois de nos bonnes villes sont entièrement occupés à pourvoir à l’établissement de leurs filles, lesquelles ne se sont rendues recommandables, dans notre État, que par une triste et ennuyeuse modestie, nous ordonnons qu’ils attendent à les marier, jusqu’à ce qu’ayant atteint l’âge limité par les ordonnances, elles viennent à les y contraindre. Défendons à nos magistrats de pourvoir à l’éducation de leurs enfants."\par

\dateline{De Paris, le 1er de la lune de Chalval, 1718.}
\section[{Lettre CXXVI. Rica à ***.}]{Lettre CXXVI. Rica à ***.}\renewcommand{\leftmark}{Lettre CXXVI. Rica à ***.}

\noindent On est bien embrassé dans toutes les religions, quand il s’agit de donner une idée des plaisirs qui sont destinés à ceux qui ont bien vécu. On épouvante facilement les méchants par une longue suite de peines, dont on les menace : mais, pour les gens vertueux, on ne sait que leur promettre. Il semble que la nature des plaisirs soit d’être d’une courte durée ; l’imagination a peine à en représenter d’autres.\par
J’ai vu des descriptions du paradis, capables d’y faire renoncer tous les gens de bon sens : les uns font jouer sans cesse de la flûte ces ombres heureuses ; d’autres les condamnent au supplice de se promener éternellement ; d’autres enfin, qui les font rêver là-haut aux maîtresses d’ici-bas, n’ont pas cru que cent millions d’années fussent un terme assez long pour leur ôter le goût de ces inquiétudes amoureuses.\par
Je me souviens à ce propos d’une histoire que j’ai ouï raconter à un homme qui avait été dans le pays du Mogol ; elle fait voir que les prêtres indiens ne sont pas moins stériles que les autres dans les idées qu’ils ont des plaisirs du paradis.\par
Une femme qui venait de perdre son mari vint en cérémonie chez le gouverneur de la ville lui demander permission de se brûler : mais, comme, dans les pays soumis aux mahométans, on abolit tant qu’on peut cette cruelle coutume, il la refusa absolument.\par
Lorsqu’elle vit ses prières impuissantes, elle se jeta dans un furieux emportement. Voyez, disait-elle, comme on est gêné ! Il ne sera seulement pas permis à une pauvre femme de se brûler quand elle en a envie ! A-t-on jamais vu rien de pareil ? Ma mère, ma tante, mes sœurs, se sont bien brûlées ? Et, quand je vais demander permission à ce maudit gouverneur, il se fâche, et se met à crier comme un enragé.\par
Il se trouva là, par hasard, un jeune bonze : Homme infidèle, lui dit le gouverneur, est-ce toi qui a mis dans l’esprit de cette femme cette fureur ? Non, dit-il, je ne lui ai jamais parlé : mais, si elle m’en croit, elle consommera son sacrifice ; elle fera une action agréable au dieu Brama : aussi en sera-t-elle bien récompensée ; car elle retrouvera dans l’autre monde son mari, et elle recommencera avec lui un second mariage. Que dites-vous ? dit la femme surprise. Je retrouverai mon mari ? Ah ! je ne me brûle pas. Il était jaloux, chagrin, et d’ailleurs si vieux, que, si le dieu Brama n’a point fait sur lui quelque réforme, sûrement il n’a pas besoin de moi. Me brûler pour lui !… pas seulement le bout du doigt pour le retirer du fond des enfers. Deux vieux bonzes, qui me séduisaient, et qui savaient de quelle manière je vivais avec lui, n’avaient garde de me tout dire : mais si le dieu Brama n’a que ce présent à me faire, je renonce à cette béatitude. Monsieur le gouverneur, je me fais mahométane. Et pour vous, dit-elle en regardant le bonze, vous pouvez, si vous voulez, aller dire à mon mari que je me porte fort bien.\par

\dateline{De Paris, le 2 de la lune de Chalval, 1718.}
\section[{Lettre CXXVII. Rica à Usbek. À ***.}]{Lettre CXXVII. Rica à Usbek. À ***.}\renewcommand{\leftmark}{Lettre CXXVII. Rica à Usbek. À ***.}

\noindent Je t’attends ici demain : cependant je t’envoie tes lettres d’Ispahan. Les miennes portent que l’ambassadeur du Grand Mogol a reçu ordre de sortir du royaume. On ajoute qu’on a fait arrêter le prince, oncle du roi, qui est chargé de son éducation ; qu’on l’a fait conduire dans un château, où il est très étroitement gardé, et qu’on l’a privé de tous ses honneurs. Je suis touché du sort de ce prince, et je le plains.\par
Je te l’avoue, Usbek, je n’ai jamais vu couler les larmes de personne sans en être attendri : je sens de l’humanité pour les malheureux, comme s’il n’y avait qu’eux qui fussent hommes ; et les grands même, pour lesquels je trouve dans mon cœur de la dureté quand ils sont élevés, je les aime sitôt qu’ils tombent.\par
En effet, qu’ont-ils à faire, dans la prospérité, d’une inutile tendresse ? elle approche trop de l’égalité : ils aiment bien mieux du respect, qui ne demande point de retour. Mais, sitôt qu’ils sont déchus de leur grandeur, il n’y a que nos plaintes qui puissent leur en rappeler l’idée.\par
Je trouve quelque chose de bien naïf, et même de bien grand, dans les paroles d’un prince qui, près de tomber entre les mains de ses ennemis, voyant ses courtisans autour de lui qui pleuraient : Je sens, leur dit-il, à vos larmes que je suis encore votre roi.\par

\dateline{De Paris, le 3 de la lune de Chalval, 1718.}
\section[{Lettre CXXVIII. Rica à Ibben. À Smyrne.}]{Lettre CXXVIII. Rica à Ibben. À Smyrne.}\renewcommand{\leftmark}{Lettre CXXVIII. Rica à Ibben. À Smyrne.}

\noindent Tu as ouï parler mille fois du fameux roi de Suède : il assiégeait une place dans un royaume qu’on nomme la Norvège ; comme il visitait la tranchée, seul avec un ingénieur, il a reçu un coup dans la tête, dont il est mort. On a fait sur-le-champ arrêter son premier ministre : les états se sont assemblés, et l’ont condamné à perdre la tête.\par
Il était accusé d’un grand crime : c’était d’avoir calomnié la nation, et de lui avoir fait perdre la confiance de son roi : forfait qui, selon moi, mérite mille morts.\par
Car enfin, si c’est une mauvaise action de noircir dans l’esprit du prince le dernier de ses sujets, qu’est-ce, lorsque l’on noircit la nation entière, et qu’on lui ôte la bienveillance de celui que la providence a établi pour faire son bonheur ?\par
Je voudrais que les hommes parlassent aux rois comme les anges parlent à notre saint prophète.\par
Tu sais que, dans les banquets sacrés où le seigneur des seigneurs descend du plus sublime trône du monde pour se communiquer à ses esclaves, je me suis fait une loi sévère de captiver une langue indocile ; on ne m’a jamais vu abandonner une seule parole qui pût être amère au dernier de ses sujets. Quand il m’a fallu cesser d’être sobre, je n’ai point cessé d’être un honnête homme ; et, dans cette épreuve de notre fidélité, j’ai risqué ma vie, et jamais ma vertu.\par
Je ne sais comment il arrive qu’il n’y a presque jamais de prince si méchant, que son ministre ne le soit encore davantage ; s’il fait quelque action mauvaise, elle a presque toujours été suggérée ; de manière que l’ambition des princes n’est jamais si dangereuse que la bassesse d’âme de ses conseillers. Mais comprends-tu qu’un homme, qui n’est que d’hier dans le ministère, qui peut-être n’y sera pas demain, puisse devenir dans un moment l’ennemi de lui-même, de sa famille, de sa patrie, et du peuple qui naîtra à jamais de celui qu’il va faire opprimer ?\par
Un prince a des passions ; le ministre les remue : c’est de ce côté-là qu’il dirige son ministère ; il n’a point d’autre but, ni n’en veut connaître. Les courtisans le séduisent par leurs louanges ; et lui le flatte plus dangereusement par ses conseils, par les desseins qu’il lui inspire, et par les maximes qu’il lui propose.\par

\dateline{De Paris, le 25 de la lune de Saphar, 1719.}
\section[{Lettre CXXIX. Rica à Usbek. À ***.}]{Lettre CXXIX. Rica à Usbek. À ***.}\renewcommand{\leftmark}{Lettre CXXIX. Rica à Usbek. À ***.}

\noindent Je passais l’autre jour sur le Pont-Neuf avec un de mes amis : il rencontra un homme de sa connaissance, qu’il me dit être un géomètre ; et il n’y avait rien qui n’y parût, car il était d’une rêverie profonde ; il fallut que mon ami le tirât longtemps par la manche, et le secouât pour le faire descendre jusqu’à lui ; tant il était occupé d’une courbe qui le tourmentait peut-être depuis plus de huit jours. Ils se firent tous deux beaucoup d’honnêtetés, et s’apprirent réciproquement quelques nouvelles littéraires. Ces discours les menèrent jusque sur la porte d’un café où j’entrai avec eux.\par
Je remarquai que notre géomètre y fut reçu de tout le monde avec empressement, et que les garçons du café en faisaient beaucoup plus de cas que de deux mousquetaires qui étaient dans un coin. Pour lui, il parut qu’il se trouvait dans un lieu agréable : car il dérida un peu son visage, et se mit à rire comme s’il n’avait pas eu la moindre teinture de géométrie.\par
Cependant son esprit régulier toisait tout ce qui se disait dans la conversation. Il ressemblait à celui qui, dans un jardin, coupait avec son épée la tête des fleurs qui s’élevaient au-dessus des autres : martyr de sa justesse, il était offensé d’une saillie, comme une vue délicate est offensée par une lumière trop vive. Rien pour lui n’était indifférent, pourvu qu’il fût vrai : aussi sa conversation était-elle singulière. Il était arrivé ce jour-là de la campagne avec un homme qui avait vu un château superbe et des jardins magnifiques ; et il n’avait vu, lui, qu’un bâtiment de soixante pieds de long sur trente-cinq de large, et un bosquet barlong de dix arpents : il aurait fort souhaité que les règles de la perspective eussent été tellement observées, que les allées des avenues eussent paru partout de même largeur ; et il aurait donné pour cela une méthode infaillible. Il parut fort satisfait d’un cadran qu’il y avait démêlé, d’une structure fort singulière ; et il s’échauffa fort contre un savant qui était auprès de moi, qui lui demanda si ce cadran marquait les heures babyloniennes. Un nouvelliste parla du bombardement du château de Fontarabie ; et il nous donna soudain les propriétés de la ligne que les bombes avaient décrite en l’air ; et, charmé de savoir cela, il voulut en ignorer entièrement le succès. Un homme se plaignait d’avoir été ruiné l’hiver d’auparavant par une inondation. Ce que vous me dites là m’est fort agréable, dit alors le géomètre : je vois que je ne me suis pas trompé dans l’observation que j’ai faite, et qu’il est au moins tombé sur la terre deux pouces d’eau plus que l’année passée.\par
Un moment après il sortit, et nous le suivîmes. Comme il allait assez vite, et qu’il négligeait de regarder devant lui, il fut rencontré directement par un autre homme : ils se choquèrent rudement ; et de ce coup ils rejaillirent, chacun de son côté, en raison réciproque de leur vitesse et de leurs masses. Quand ils furent un peu revenus de leur étourdissement, cet homme, portant la main sur le front, dit au géomètre : Je suis bien aise que vous m’ayez heurté ; car j’ai une grande nouvelle à vous apprendre : je viens de donner mon Horace au public. Comment ! dit le géomètre, il y a deux mille ans qu’il y est. Vous ne m’entendez pas, reprit l’autre : c’est une traduction de cet ancien auteur, que je viens de mettre au jour ; il y a vingt ans que je m’occupe à faire des traductions.\par
Quoi ! monsieur, dit le géomètre, il y a vingt ans que vous ne pensez pas ! Vous parlez pour les autres, et ils pensent pour vous ! Monsieur, dit le savant, croyez-vous que je n’aie pas rendu un grand service au public, de lui rendre la lecture des bons auteurs familière ? Je ne dis pas tout à fait cela : j’estime autant qu’un autre les sublimes génies que vous travestissez ; mais vous ne leur ressemblerez point ; car si vous traduisez toujours, on ne vous traduira jamais.\par
Les traductions sont comme ces monnaies de cuivre qui ont bien la même valeur qu’une pièce d’or, et même sont d’un plus grand usage pour le peuple ; mais elles sont toujours faibles et d’un mauvais aloi.\par
Vous voulez, dites-vous, faire renaître parmi nous ces illustres morts ; et j’avoue que vous leur donnez bien un corps : mais vous ne leur rendez pas la vie ; il y manque toujours un esprit pour les animer.\par
Que ne vous appliquez-vous plutôt à la recherche de tant de belles vérités qu’un calcul facile nous fait découvrir tous les jours ? Après ce petit conseil, ils se séparèrent, je crois, très mécontents l’un de l’autre.\par

\dateline{De Paris, le dernier de la lune de Rebiab 2, 1719.}
\section[{Lettre CXXX. Rica à ***.}]{Lettre CXXX. Rica à ***.}\renewcommand{\leftmark}{Lettre CXXX. Rica à ***.}

\noindent Je te parlerai dans cette lettre d’une certaine nation qu’on appelle les nouvellistes, qui s’assemblent dans un jardin magnifique, où leur oisiveté est toujours occupée. Ils sont très inutiles à l’État, et leurs discours de cinquante ans n’ont pas un effet différent de celui qu’aurait pu produire un silence aussi long : cependant ils se croient considérables, parce qu’ils s’entretiennent de projets magnifiques, et traitent de grands intérêts.\par
La base de leurs conversations est une curiosité frivole et ridicule : il n’y a point de cabinet si mystérieux qu’ils ne prétendent pénétrer ; ils ne sauraient consentir à ignorer quelque chose ; ils savent combien notre auguste sultan a de femmes, combien il fait d’enfants toutes les années ; et quoiqu’ils ne fassent aucune dépense en espions, ils sont instruits des mesures qu’il prend pour humilier l’empereur des Turcs et celui des Mogols.\par
À peine ont-ils épuisé le présent, qu’ils se précipitent dans l’avenir ; et, marchant au-devant de la providence, la préviennent sur toutes les démarches des hommes. Ils conduisent un général par la main ; et, après l’avoir loué de mille sottises qu’il n’a pas faites, ils lui en préparent mille autres qu’il ne fera pas.\par
Ils font voler les armées comme les grues, et tomber les murailles comme des cartons ; ils ont des ponts sur toutes les rivières, des routes secrètes dans toutes les montagnes, des magasins immenses dans les sables brûlants : il ne leur manque que le bon sens.\par
Il y a un homme avec qui je loge, qui reçut cette lettre d’un nouvelliste ; comme elle m’a paru singulière, je la gardai ; la voici :\par
"Monsieur,\par
Je me trompe rarement dans mes conjectures sur les affaires du temps. Le 1er janvier 1711, je prédis que l’empereur Joseph mourrait dans le cours de l’année : il est vrai que, comme il se portait fort bien, je crus que je me ferais moquer de moi si je m’expliquais d’une manière bien claire ; ce qui fit que je me servis de termes un peu énigmatiques ; mais les gens qui savent raisonner m’entendirent bien. Le 17 avril de la même année, il mourut de la petite vérole.\par
Dès que la guerre fut déclarée entre l’empereur et les Turcs, j’allai chercher nos messieurs dans tous les coins des Tuileries ; je les assemblai près du bassin, et leur prédis qu’on ferait le siège de Belgrade, et qu’il serait pris. J’ai été assez heureux pour que ma prédiction ait été accomplie. Il est vrai que, vers le milieu du siège, je pariai cent pistoles qu’il serait pris le 18 août\footnote{1717.} ; il ne fut pris que le lendemain : peut-on perdre à si beau jeu ?\par
Lorsque je vis que la flotte d’Espagne débarquait en Sardaigne, je jugeai qu’elle en ferait la conquête : je le dis, et cela se trouva vrai. Enflé de ce succès, j’ajoutai que cette flotte victorieuse irait débarquer à Final pour faire la conquête du Milanais. Comme je trouvai de la résistance à faire recevoir cette idée, je voulus la soutenir glorieusement : je pariai cinquante pistoles, et je les perdis encore ; car ce diable d’Alberoni, malgré la foi des traités, envoya sa flotte en Sicile, et trompa tout à la fois deux grands politiques, le duc de Savoie et moi.\par
Tout cela, monsieur, me déroute si fort, que j’ai résolu de prédire toujours et de ne parier jamais. Autrefois nous ne connaissions point aux Tuileries l’usage des paris, et feu M. le comte de L. ne les souffrait guère ; mais, depuis qu’une troupe de petits-maîtres s’est mêlée parmi nous, nous ne savons plus où nous en sommes. À peine ouvrons-nous la bouche pour dire une nouvelle, qu’un de ces jeunes gens propose de parier contre.\par
L’autre jour, comme j’ouvrais mon manuscrit, et accommodais mes lunettes sur mon nez, un de ces fanfarons, saisissant justement l’intervalle du premier mot au second, me dit : Je parie cent pistoles que non. Je fis semblant de n’avoir pas fait d’attention à cette extravagance ; et, reprenant la parole d’une voix plus forte, je dis : M. le maréchal de *** ayant appris… Cela est faux, me dit-il, vous avez toujours des nouvelles extravagantes ; il n’y a pas le sens commun à tout cela. Je vous prie, monsieur, de me faire le plaisir de me prêter trente pistoles ; car je vous avoue que ces paris m’ont fort dérangé. Je vous envoie la copie de deux lettres que j’ai écrites au ministre. Je suis, etc."\par

\begin{center}
\noindent \centerline{LETTRE D’UN NOUVELLISTE AU MINISTRE.}\par
\end{center}

\noindent "MONSEIGNEUR,\par
Je suis le sujet le plus zélé que le roi ait jamais eu : c’est moi qui obligeai un de mes amis d’exécuter le projet que j’avais formé d’un livre pour démontrer que Louis le Grand était le plus grand de tous les princes qui ont mérité le nom de Grand. Je travaille depuis longtemps à un autre ouvrage qui fera encore plus d’honneur à notre nation, si Votre Grandeur veut m’accorder un privilège : mon dessein est de prouver que, depuis le commencement de la monarchie, les Français n’ont jamais été battus, et que ce que les historiens ont dit jusqu’ici de nos désavantages sont de véritables impostures. Je suis obligé de les redresser en bien des occasions ; et j’ose me flatter que je brille surtout dans la critique. Je suis, monseigneur, etc."\par
"MONSEIGNEUR,\par
Depuis la perte que nous avons faite de M. le comte de L., nous vous supplions d’avoir la bonté de nous permettre d’élire un président. Le désordre se met dans nos conférences, et les affaires d’État n’y sont pas traitées avec la même discussion que par le passé ; nos jeunes gens vivent absolument sans égard pour les anciens, et entre eux sans discipline : c’est le véritable conseil de Roboam, où les jeunes imposent aux vieillards. Nous avons beau leur représenter que nous étions paisibles possesseurs des Tuileries vingt ans avant qu’ils fussent au monde ; je crois qu’ils nous en chasseront à la fin, et qu’obligés de quitter ces lieux où nous avons tant de fois évoqué les ombres de nos héros français, il faudra que nous allions tenir nos conférences au Jardin du Roi ou dans quelque lieu plus écarté. Je suis…"\par

\dateline{De Paris, le 7 de la lune de Gemmadi 2, 1719.}
\section[{Lettre CXXXI. Rhédi à Rica. À Paris.}]{Lettre CXXXI. Rhédi à Rica. À Paris.}\renewcommand{\leftmark}{Lettre CXXXI. Rhédi à Rica. À Paris.}

\noindent Une des choses qui a le plus exercé ma curiosité en arrivant en Europe, c’est l’histoire et l’origine des républiques. Tu sais que la plupart des Asiatiques n’ont pas seulement d’idée de cette sorte de gouvernement, et que l’imagination ne les a pas servis jusqu’à leur faire comprendre qu’il puisse y en avoir sur la terre d’autre que le despotique.\par
Les premiers gouvernements du monde furent monarchiques : ce ne fut que par hasard et par la succession des siècles que les républiques se formèrent.\par
La Grèce ayant été abîmée par un déluge, de nouveaux habitants vinrent la peupler : elle tira presque toutes ses colonies d’Égypte et des contrées de l’Asie les plus voisines ; et, comme ces pays étaient gouvernés par des rois, les peuples qui en sortirent furent gouvernés de même. Mais la tyrannie de ces princes devenant trop pesante, on secoua le joug ; et du débris de tant de royaumes s’élevèrent ces républiques qui firent si fort fleurir la Grèce, seule polie au milieu des barbares.\par
L’amour de la liberté, la haine des rois, conserva longtemps la Grèce dans l’indépendance, et étendit au loin le gouvernement républicain. Les villes grecques trouvèrent des alliées dans l’Asie mineure : elles y envoyèrent des colonies aussi libres qu’elles, qui leur servirent de remparts contre les entreprises des rois de Perse. Ce n’est pas tout : la Grèce peupla l’Italie ; l’Italie, l’Espagne, et peut-être les Gaules. On sait que cette grande Hespérie, si fameuse chez les anciens, était au commencement la Grèce, que ses voisins regardaient comme un séjour de félicité : les Grecs qui ne trouvaient point chez eux ce pays heureux, l’allèrent chercher en Italie ; ceux de l’Italie, en Espagne ; ceux d’Espagne, dans la Bétique ou le Portugal : de manière que toutes ces régions portèrent ce nom chez les anciens. Ces colonies grecques apportèrent avec elles un esprit de liberté qu’elles avaient pris dans ce doux pays. Ainsi, on ne voit guère, dans ces temps reculés, de monarchies dans l’Italie, l’Espagne, les Gaules. On verra bientôt que les peuples du Nord et d’Allemagne n’étaient pas moins libres : et, si l’on trouve des vestiges de quelque royauté parmi eux, c’est qu’on a pris pour des rois les chefs des armées ou des républiques.\par
Tout ceci se passait en Europe : car, pour l’Asie et l’Afrique, elles ont toujours été accablées sous le despotisme, si vous en exceptez quelques villes de l’Asie mineure dont nous avons parlé, et la république de Carthage en Afrique.\par
Le monde fut partagé entre deux puissantes républiques : celle de Rome et celle de Carthage. Il n’y a rien de si connu que les commencements de la République romaine, et rien qui le soit si peu que l’origine de celle de Carthage. On ignore absolument la suite des princes africains depuis Didon, et comment ils perdirent leurs puissance. C’eût été un grand bonheur pour le monde que l’agrandissement prodigieux de la république romaine, s’il n’y avait pas eu cette différence injuste entre les citoyens romains et les peuples vaincus ; si l’on avait donné au gouverneur des provinces une autorité moins grande ; si les lois si saintes pour empêcher leur tyrannie avaient été observées, et s’ils ne s’étaient pas servis, pour les faire taire, des mêmes trésors que leur injustice avait amassés.\par
Il semble que la liberté soit faite pour le génie des peuples d’Europe, et la servitude pour celui des peuples d’Asie. C’est en vain que les Romains offrirent aux Cappadociens ce précieux trésor : cette nation lâche la refusa, et elle courut à la servitude avec le même empressement que les autres peuples couraient à la liberté.\par
César opprima la République romaine, et la soumit à un pouvoir arbitraire.\par
L’Europe gémit longtemps sous un gouvernement militaire et violent, et la douceur romaine fut changée en une cruelle oppression.\par
Cependant une infinité de nations inconnues sortirent du Nord, se répandirent comme des torrents dans les provinces romaines ; et, trouvant autant de facilité à faire des conquêtes qu’à exercer leurs pirateries, les démembrèrent et firent des royaumes. Ces peuples étaient libres et ils bornaient si fort l’autorité de leurs rois, qu’ils n’étaient proprement que des chefs ou des généraux. Ainsi ces royaumes, quoique fondés par la force, ne sentirent point le joug du vainqueur. Lorsque les peuples d’Asie, comme les Turcs et les Tartares, firent des conquêtes, soumis à la volonté d’un seul, ils ne songèrent qu’à lui donner de nouveaux sujets, et à établir par les armes son autorité violente : mais les peuples du Nord, libres dans leur pays, s’emparant des provinces romaines, ne donnaient point à leurs chefs une grande autorité. Quelques-uns même de ces peuples, comme les vandales en Afrique, les Goths en Espagne, déposaient leurs rois dès qu’ils n’en étaient pas satisfaits ; et, chez les autres, l’autorité du prince était bornée de mille manières différentes : un grand nombre de seigneurs la partageaient avec lui ; les guerres n’étaient entreprises que de leur consentement ; les dépouilles étaient partagées entre le chef et les soldats ; aucun impôt en faveur du prince ; les lois étaient faites dans les assemblés de la nation. Voilà le principe fondamental de tous ces États, qui se formèrent des débris de l’Empire romain.\par

\dateline{De Venise, le 20 de la lune de Rhégeb, 1719.}
\section[{Lettre CXXXII. Rica à ***.}]{Lettre CXXXII. Rica à ***.}\renewcommand{\leftmark}{Lettre CXXXII. Rica à ***.}

\noindent Je fus, il y a cinq ou six mois, dans un café ; j’y remarquai un gentilhomme assez bien mis qui se faisait écouter : il parlait du plaisir qu’il y avait à vivre à Paris ; il déplorait sa situation d’être obligé de vivre dans la province. J’ai, dit-il, quinze mille livres de rente en fonds de terre, et je me croirais plus heureux si j’avais le quart de ce bien-là en argent et en effets portables partout. J’ai beau presser mes fermiers, et les accabler de frais de justice, je ne fais que les rendre plus insolvables : je n’ai jamais pu voir cent pistoles à la fois. Si je devais dix mille francs, on me ferait saisir toutes mes terres, et je serais à l’hôpital.\par
Je sortis sans avoir fait grande attention à tout ce discours ; mais, me trouvant hier dans ce quartier, j’entrai dans la même maison, et j’y vis un homme grave, d’un visage pâle et allongé, qui, au milieu de cinq ou six discoureurs, paraissait morne et pensif, jusques à ce que, prenant brusquement la parole : Oui, messieurs, dit-il en haussant la voix, je suis ruiné ; je n’ai plus de quoi vivre ; car j’ai actuellement chez moi deux cent mille livres en billets de banque, et cent mille écus d’argent : je me trouve dans une situation affreuse ; je me suis cru riche ; et me voilà à l’hôpital : au moins si j’avais seulement une petite terre où je pusse me retirer, je serais sûr d’avoir de quoi vivre ; mais je n’ai pas grand comme ce chapeau en fonds de terre.\par
Je tournai par hasard la tête d’un autre côté, et je vis un autre homme qui faisait des grimaces de possédé. À qui se fier désormais ? s’écriait-il. Il y a un traître que je croyais si fort de mes amis que je lui avais prêté mon argent : et il me l’a rendu ! quelle perfidie horrible ! Il a beau faire ; dans mon esprit il sera toujours déshonoré.\par
Tout près de là était un homme très mal vêtu, qui, élevant les yeux au ciel, disait : Dieu bénisse les projets de nos ministres ! puissé-je voir les actions à deux mille, et tous les laquais de Paris plus riches que leurs maîtres ! J’eus la curiosité de demander son nom. C’est un homme extrêmement pauvre, me dit-on ; aussi a-t-il un pauvre métier : il est généalogiste, et il espère que son art rendra, si les fortunes continuent ; et que tous ces nouveaux riches auront besoin de lui pour réformer leur nom, décrasser leurs ancêtres, et orner leurs carrosses ; il s’imagine qu’il va faire autant de gens de qualité qu’il voudra ; il tressaille de joie de voir multiplier ses pratiques.\par
Enfin, je vis entrer un vieillard pâle et sec, que je reconnus pour nouvelliste avant qu’il se fût assis ; il n’était pas du nombre de ceux qui ont une assurance victorieuse contre tous les revers, et présagent toujours les victoires et les trophées : c’était au contraire un de ces trembleurs qui n’ont que des nouvelles tristes. Les affaires vont bien mal du côté de l’Espagne, dit-il : nous n’avons point de cavalerie sur la frontière, et il est à craindre que le prince Pio, qui en a un gros corps, ne fasse contribuer tout le Languedoc. Il y avait vis-à-vis de moi un philosophe assez mal en ordre qui prenait le nouvelliste en pitié, et haussait les épaules à mesure que l’autre haussait la voix ; je m’approchai de lui, et il me dit à l’oreille : Vous voyez que ce fat nous entretient, il y a une heure, de sa frayeur pour le Languedoc ; et moi, j’aperçus hier au soir une tache dans le soleil, qui, si elle augmentait, pourrait faire tomber toute la nature en engourdissement ; et je n’ai pas dit un seul mot.\par

\dateline{De Paris, le 17 de la lune de Rhamazan, 1719.}
\section[{Lettre CXXXIII. Rica à ***.}]{Lettre CXXXIII. Rica à ***.}\renewcommand{\leftmark}{Lettre CXXXIII. Rica à ***.}

\noindent J’allai l’autre jour voir une grande bibliothèque dans un couvent de dervis, qui en sont comme les dépositaires, mais qui sont obligés d’y laisser entrer tout le monde à certaines heures.\par
En entrant, je vis un homme grave qui se promenait au milieu d’un nombre innombrable de volumes qui l’entouraient. J’allai à lui, et le priai de me dire quels étaient quelques-uns de ces livres que je voyais mieux reliés que les autres. Monsieur, me dit-il, j’habite ici une terre étrangère : je n’y connais personne : bien des gens me font de pareilles questions ; mais vous voyez bien que je n’irai pas lire tous ces livres pour les satisfaire ; mais j’ai mon bibliothécaire qui vous donnera satisfaction, car il s’occupe nuit et jour à déchiffrer tout ce que vous voyez là ; c’est un homme qui n’est bon à rien, et qui nous est très à charge, parce qu’il ne travaille point pour le couvent. Mais j’entends l’heure du réfectoire qui sonne. Ceux qui comme moi sont à la tête d’une communauté doivent être les premiers à tous les exercices. En disant cela, le moine me poussa dehors, ferma la porte, et, comme s’il eût volé, disparut à mes yeux.\par

\dateline{De Paris le 21 de la lune de Rhamazan 1719.}
\section[{Lettre CXXXIV. Rica au même.}]{Lettre CXXXIV. Rica au même.}\renewcommand{\leftmark}{Lettre CXXXIV. Rica au même.}

\noindent Je retournai le lendemain à cette bibliothèque, où je trouvai tout un autre homme que celui que j’avais vu la première fois : son air était simple, sa physionomie spirituelle, et son abord très affable. Dès que je lui eus fait connaître ma curiosité, il se mit en devoir de la satisfaire, et même en qualité d’étranger, de m’instruire.\par
Mon père, lui dis-je, quels sont ces gros volumes qui tiennent tout ce côté de bibliothèque ? Ce sont, me dit-il, les interprètes de l’Écriture. Il y en a un grand nombre ! lui repartis-je ; il faut que l’Écriture fût bien obscure autrefois, et bien claire à présent ; reste-t-il encore quelques doutes ? peut-il y avoir des points contestés ? S’il y en a, bon Dieu ! s’il y en a ! me répondit-il ; il y en a presque autant que de lignes. Oui, lui dis-je ! Et qu’ont donc fait tous ces auteurs ! Ces auteurs, me repartit-il, n’ont point cherché dans l’Écriture ce qu’il faut croire, mais ce qu’ils croient eux-mêmes ; ils ne l’ont point regardée comme un livre où étaient contenus les dogmes qu’ils devaient recevoir, mais comme un ouvrage qui pourrait donner de l’autorité à leurs propres idées : c’est pour cela qu’ils en ont corrompu tous les sens, et ont donné la torture à tous les passages. C’est un pays où les hommes de toutes les sectes font des descentes, et vont comme au pillage ; c’est un champ de bataille où les nations ennemies qui se rencontrent livrent bien des combats, où l’on s’attaque, où l’on s’escarmouche de bien des manières.\par
Tout près de là vous voyez les livres ascétiques ou de dévotion ; ensuite les livres de morale, bien plus utiles ; ceux de théologie, doublement inintelligibles, et par la matière qui y est traitée, et par la manière de la traiter ; les ouvrages des mystiques, c’est-à-dire des dévots qui ont le cœur tendre. Ah ! mon père, lui dis-je, un moment ; n’allez pas si vite ; parlez-moi de ces mystiques. Monsieur, dit-il, la dévotion échauffe un cœur disposé à la tendresse, et lui fait envoyer des esprits au cerveau qui l’échauffent de même, d’où naissent les extases et les ravissements. Cet état est le délire de la dévotion ; souvent il se perfectionne, ou plutôt dégénère en quiétisme : vous savez qu’un quiétiste n’est autre chose qu’un homme fou, dévot et libertin.\par
Voyez les casuistes, qui mettent au jour les secrets de la nuit ; qui forment dans leur imagination tous les monstres que le démon d’amour peut produire, les rassemblent, les comparent, et en font l’objet éternel de leurs pensées : heureux si leur cœur ne se met pas de la partie, et ne devient pas lui-même complice de tant d’égarements si naïvement décrits et si nuement peints !\par
Vous voyez, monsieur, que je pense librement, et que je vous dis tout ce que je pense. Je suis naturellement naïf, et plus encore avec vous, qui êtes un étranger, qui voulez savoir les choses, et les savoir telles qu’elles sont. Si je voulais, je vous parlerais de tout ceci qu’avec admiration ; je vous dirais sans cesse : Cela et divin, cela est respectable ; il y a du merveilleux. Et il en arriverait de deux choses l’une, ou que je vous tromperais, ou que je me déshonorerais dans votre esprit.\par
Nous en restâmes là ; une affaire qui survint au dervis rompit notre conversation jusqu’au lendemain.\par

\dateline{De Paris, le 23 de la lune de Rhamazan, 1719.}
\section[{Lettre CXXXV. Rica au même.}]{Lettre CXXXV. Rica au même.}\renewcommand{\leftmark}{Lettre CXXXV. Rica au même.}

\noindent Je revins à l’heure marquée, et mon homme me mena précisément dans l’endroit où nous nous étions quittés. Voici, me dit-il, les grammairiens, les glossateurs, et les commentateurs. Mon père, lui dis-je, tous ces gens-là ne peuvent-ils pas se dispenser d’avoir du bon sens ? Oui, dit-il, ils le peuvent ; et même il n’y paraît pas, leurs ouvrages n’en sont pas plus mauvais ; ce qui est très commode pour eux. Cela est vrai, lui dis-je ; et je connais bien des philosophes qui feraient bien de s’appliquer à ces sortes de sciences-là.\par
Voilà, poursuivit-il, les orateurs, qui ont le talent de persuader indépendamment des raisons ; et les géomètres, qui obligent un homme malgré lui d’être persuadé, et le convainquent avec tyrannie.\par
Voici les livres de métaphysique, qui traitent de si grands intérêts, et dans lesquels l’infini se rencontre partout : les livres de physique, qui ne trouvent pas plus de merveilleux dans l’économie du vaste univers que dans la machine la plus simple de nos artisans ; les livres de médecine, ces monuments de la fragilité de la nature et de la puissance de l’art ; qui font trembler quand ils traitent des maladies même les plus légères, tant ils nous rendent la mort présente ; mais qui nous mettent dans une sécurité entière quand ils parlent de la vertu des remèdes, comme si nous étions devenus immortels.\par
Tout près de là sont les livres d’anatomie, qui contiennent bien moins la description des parties du corps humain que les noms barbares qu’on leur a donnés : chose qui ne guérit ni le malade de son mal, ni le médecin de son ignorance\par
Voici la chimie, qui habite tantôt l’hôpital et tantôt les petites maisons, comme des demeures qui lui sont également propres.\par
Voici les livres de science, ou plutôt d’ignorance occulte : tels sont ceux qui contiennent quelque espèce de diablerie ; exécrables selon la plupart des gens ; pitoyables selon moi. Tels sont encore les livres d’astrologie judiciaire. Que dites-vous, mon père ? Les livres d’astrologie judiciaire, repartis-je avec feu ! Et ce sont ceux dont nous faisons plus de cas en Perse : ils règlent toutes les actions de notre vie, et nous déterminent dans toutes nos entreprises ; les astrologues sont proprement nos directeurs ; ils font plus, ils entrent dans le gouvernement de l’État. Si cela est, me dit-il, vous vivez sous un joug bien plus dur que celui de la raison : voilà ce qui s’appelle le plus étrange de tous les empires ; je plains bien une famille, et encore plus une nation, qui se laisse si fort dominer par les planètes. Nous nous servons, lui repartis-je, de l’astrologie, comme vous vous servez de l’algèbre. Chaque nation a sa science, selon laquelle elle règle sa politique : tous les astrologues ensemble n’ont jamais fait tant de sottises en notre Perse qu’un seul de vos algébristes en a fait ici. Croyez-vous que le concours fortuit des astres ne soit pas une règle aussi sûre que les beaux raisonnements de votre faiseur de système ? Si l’on comptait les voix là-dessus en France et en Perse, ce serait un beau sujet de triomphe pour l’astrologie ; vous verriez les mathématiciens bien humiliés : quel accablant corollaire en pourrait-on tirer contre eux !\par
Notre dispute fut interrompue, et il fallut nous quitter.\par

\dateline{De Paris, le 6 de la lune de Rhamazan, 1719.}
\section[{Lettre CXXXVI. Rica au même.}]{Lettre CXXXVI. Rica au même.}\renewcommand{\leftmark}{Lettre CXXXVI. Rica au même.}

\noindent Dans l’entrevue suivante, mon savant me mena dans un cabinet particulier. Voici les livres d’histoire moderne, me dit-il. Voyez premièrement les historiens de l’Église et des papes ; livres que je lis pour m’édifier, et qui font souvent en moi un effet tout contraire.\par
Là, ce sont ceux qui ont écrit de la décadence du formidable Empire romain, qui s’était formé du débris de tant de monarchies, et sur la chute duquel il s’en forma tant de nouvelles. Un nombre infini de peuples barbares, aussi inconnus que les pays qu’ils habitaient, parurent tout à coup, l’inondèrent, le ravagèrent, le dépecèrent, et fondèrent tous les royaumes que vous voyez à présent en Europe. Ces peuples n’étaient point proprement barbares, puisqu’ils étaient libres : mais ils le sont devenus depuis que, soumis pour la plupart à une puissance absolue, ils ont perdu cette douce liberté si conforme à la raison, à l’humanité, et à la nature.\par
Vous voyez ici les historiens de l’Allemagne, laquelle n’est qu’une ombre du premier Empire, mais qui est, je crois, la seule puissance qui soit sur la terre, que la division n’a point affaiblie ; la seule, je crois encore, qui se fortifie à mesure de ses pertes ; et qui, lente à profiter des succès, devient indomptable par ses défaites.\par
Voici les historiens de France, où l’on voit d’abord la puissance des rois se former, mourir deux fois, renaître de même, languir ensuite pendant plusieurs siècles ; mais, prenant insensiblement des forces, accrue de toutes parts, monter à son dernier période : semblable à ces fleuves qui dans leur course perdent leurs eaux, ou se cachent sous terre ; puis reparaissant de nouveau, grossis par les rivières qui s’y jettent, entraînent avec rapidité tout ce qui s’oppose à leur passage.\par
Là, vous voyez la nation espagnole sortir de quelques montagnes : les princes mahométans subjugués aussi insensiblement qu’ils avaient rapidement conquis ; tant de royaumes réunis dans une vaste monarchie, qui devint presque la seule ; jusqu’à ce qu’accablé de sa fausse opulence, elle perdit sa force et sa réputation même, et ne conserva que l’orgueil de sa première puissance.\par
Ce sont ici les historiens d’Angleterre, où l’on voit la liberté sortir sans cesse des feux de la discorde et de la sédition ; le prince toujours chancelant sur un trône inébranlable ; une nation impatiente, sage dans sa fureur même ; et qui, maîtresse de la mer (chose inouïe jusqu’alors), mêle le commerce avec l’empire.\par
Tout près de là sont les historiens de cette autre reine de la mer, la république de Hollande, si respectée en Europe, et si formidable en Asie, où ses négociants voient tant de rois prosternés devant eux.\par
Les historiens d’Italie vous représentent une nation autrefois maîtresse du monde, aujourd’hui esclave de toutes les autres ; ses princes divisés et faibles, et sans autre attribut de souveraineté qu’une vaine politique.\par
Voilà les historiens des républiques : de la Suisse, qui est l’image de sa liberté ; de Venise, qui n’a de ressources qu’en son économie ; et de Gênes, qui n’est superbe que par ses bâtiments.\par
Voici ceux du nord, et entre autres de la Pologne, qui use si mal de sa liberté et du droit qu’elle a d’élire ses rois, qu’il semble qu’elle veuille consoler par là les peuples ses voisins, qui ont perdu l’un et l’autre.\par
Là-dessus, nous nous séparâmes jusqu’au lendemain.\par

\dateline{De Paris, le 2 de la lune de Chalval, 1719.}
\section[{Lettre CXXXVII. Rica au même.}]{Lettre CXXXVII. Rica au même.}\renewcommand{\leftmark}{Lettre CXXXVII. Rica au même.}

\noindent Le lendemain, il me mena dans un autre cabinet. Ce sont ici les poète, me dit-il ; c’est-à-dire ces auteurs dont le métier est de mettre des entraves au bon sens, et d’accabler la raison sous les agréments comme on ensevelissait autrefois les femmes sous leurs parures et leurs ornements. Vous les connaissiez ; ils ne sont pas rares chez les Orientaux, où le soleil, plus ardent, semble échauffer les imaginations mêmes.\par
Voilà les poèmes épiques ? Hé ! Qu’est-ce que les poèmes épiques ? En vérité, me dit-il, je n’en sais rien ; les connaisseurs disent qu’on n’en a jamais fait que deux, et que les autres qu’on donne sous ce nom ne le sont point ; c’est aussi ce que je ne sais pas. Ils disent de plus qu’il est impossible d’en faire de nouveaux ; et cela est encore plus surprenant.\par
Voici les poètes dramatiques, qui, selon moi, sont les poètes par excellence, et les maîtres des passions. Il y en a de deux sortes : les comiques, qui nous remuent si doucement ; et les tragiques, qui nous troublent et nous agitent avec tant de violence.\par
Voici les lyriques, que je méprise autant que je fais cas des autres, et qui font de leur art une harmonieuse extravagance.\par
On voit ensuite les auteurs des idylles et des églogues, qui plaisent même aux gens de cour, par l’idée qu’ils leur donnent d’une certaine tranquillité qu’ils n’ont pas, et qu’ils leur montrent dans la condition des bergers.\par
De tous les auteurs que nous avons vus, voici les plus dangereux : ce sont ceux qui aiguisent les épigrammes, qui sont de petites flèches déliées qui font une plaie profonde et inaccessible aux remèdes.\par
Vous voyez ici les romans, qui sont des espèces de poètes, et qui outrent également le langage de l’esprit et celui du cœur ; qui passent leur vie à chercher la nature, et la manquent toujours ; et qui font des héros, qui y sont aussi étrangers que les dragons ailés et les hippocentaures.\par
J’ai vu, lui dis-je, quelques-uns de vos romans : et, si vous voyiez les nôtres, vous en seriez encore plus choqué. Ils sont aussi peu naturels, et d’ailleurs extrêmement gênés par nos mœurs : il faut dix années de passion avant qu’un amant ait pu voir seulement le visage de la maîtresse. Cependant les auteurs sont forcés de faire passer les lecteurs dans ces ennuyeux préliminaires. Or, il est impossible que les incidents soient variés ; on a recours à un artifice pire que le mal même qu’on veut guérir : c’est aux prodiges. Je suis sûr que vous ne trouverez pas bon qu’une magicienne fasse sortir une armée de dessous terre ; qu’un héros lui seul en détruise une de cent mille hommes. Cependant voilà nos romans : ces aventures froides et souvent répétées nous font languir, et ces prodiges extravagants nous révoltent.\par

\dateline{De Paris, le 6 de la lune de Chalval, 1719.}
\section[{Lettre CXXXVIII. Rica à Ibben. À Smyrne.}]{Lettre CXXXVIII. Rica à Ibben. À Smyrne.}\renewcommand{\leftmark}{Lettre CXXXVIII. Rica à Ibben. À Smyrne.}

\noindent Les ministres se succèdent et se détruisent ici comme les saisons : depuis trois ans j’ai vu changer quatre fois de système sur les finances. On lève aujourd’hui, en Perse et en Turquie, les subsides de la même manière que les fondateurs de ces monarchies les levaient : il s’en faut bien qu’il en soit ici de même. Il est vrai que nous n’y mettons pas tant d’esprit que les Occidentaux : nous croyons qu’il n’y a pas plus de différence entre l’administration des revenus du prince et de ceux d’un particulier qu’il y en a entre compter cent mille tomans ou en compter cent. Mais il y a ici bien plus de finesse et de mystère. Il faut que de grands génies travaillent nuit et jour ; qu’ils enfantent sans cesse, et avec douleur, de nouveaux projets ; qu’ils écoutent les avis d’une infinité de gens qui travaillent pour eux sans en être priés ; qu’ils se retirent et vivent dans le fond d’un cabinet impénétrable aux grands et sacré aux petits ; qu’ils aient toujours la tête remplie de secrets importants, de dessins miraculeux, de systèmes nouveaux ; et qu’absorbés dans des méditations, ils soient privés non seulement de l’usage de la parole, mais même quelquefois de la politesse.\par
Dès que le feu roi eut fermé les yeux, on pensa à établir une nouvelle administration. On sentait qu’on était mal, mais ne savait comment faire pour être mieux. On s’était mal trouvé de l’autorité sans bornes des ministres précédents : on la voulut partager. On créa pour cet effet six ou sept conseils ; et ce ministère est peut-être celui de tous qui a gouverné la France avec plus de sens : la durée en fut courte, aussi bien que celle du bien qu’il produisit.\par
La France, à la mort du feu roi, était un corps accablé de mille maux : N*** prit le fer à la main, retrancha les chairs inutiles, et appliqua quelques remèdes topiques ; mais il restait toujours un vice intérieur à guérir. Un étranger est venu, qui a entrepris cette cure. Après bien des remèdes violents, il a cru lui avoir rendu son embonpoint ; et il l’a seulement rendue bouffie.\par
Tous ceux qui étaient riches il y a six mois sont à présent dans la pauvreté, et ceux qui n’avaient pas de pain regorgent de richesses. Jamais ces deux extrémités ne se sont touchées de si près. L’étranger a tourné l’État comme un fripier tourne un habit : il fait paraître dessus ce qui était dessous ; et ce qui était dessus, il le met à l’envers. Quelles fortunes inespérées, incroyables même à ceux qui les ont faites ! Dieu ne tire pas plus rapidement les hommes du néant. Que de valets servis par leurs camarades, et peut-être demain par leurs maîtres !\par
Tout ceci produit souvent des choses bizarres. Les laquais qui avaient fait fortune sous le règne passé vantent aujourd’hui leur naissance : ils rendent à ceux qui viennent de quitter leur livrée dans une certaine rue tout le mépris qu’on avait pour eux il y a six mois ; ils crient de toute leur force : La noblesse est ruinée, quel désordre dans l’État ! quelle confusion dans les rangs ! On ne voit que des inconnus faire fortune ! Je te promets que ceux-ci prendront bien leur revanche sur ceux qui viendront après eux, et que, dans trente ans, ces gens de qualité feront bien du bruit.\par

\dateline{De Paris, le 1er de lune de Zilcadé, 1720.}
\section[{Lettre CXXXIX. Rica au même.}]{Lettre CXXXIX. Rica au même.}\renewcommand{\leftmark}{Lettre CXXXIX. Rica au même.}

\noindent Voici un grand exemple de la tendresse conjugale, non seulement dans femme, mais dans une reine. La reine de Suède, voulant à toute force associer le prince son époux à la couronne, pour aplanir toutes les difficultés, a envoyé aux états une déclaration par laquelle elle se désiste de la régence, en cas qu’il soit élu.\par
Il y a soixante et quelques années qu’une autre reine, nommée Christine, abdiqua la couronne pour se donner tout entière à la philosophie. Je ne sais lequel de ces deux exemples nous devons admirer davantage.\par
Quoique j’approuve assez que chacun se tienne ferme dans le poste où la nature l’a mis ; et que je ne puisse louer la faiblesse de ceux qui, se trouvant au-dessous de leur état, le quittent comme par une espèce de désertion ; je suis cependant frappé de la grandeur d’âme de ces deux princesses, et de voir l’esprit de l’une et le cœur de l’autre supérieurs à leur fortune. Christine a songé à connaître dans le temps que les autres ne songent qu’à jouir ; et l’autre ne veut jouir que pour mettre tout son bonheur entre les mains de son auguste époux.\par

\dateline{De Paris, le 27 de la lune de Maharram, 1720.}
\section[{Lettre CXL. Rica à Usbek. À ***.}]{Lettre CXL. Rica à Usbek. À ***.}\renewcommand{\leftmark}{Lettre CXL. Rica à Usbek. À ***.}

\noindent Le parlement de Paris vient d’être relégué dans une petite ville qu’on appelle Pontoise. Le conseil lui a envoyé enregistrer ou approuver une déclaration qui le déshonore ; et il l’a enregistrée d’une manière qui déshonore le conseil.\par
On menace d’un pareil traitement quelques parlements du royaume.\par
Ces compagnies sont toujours odieuses : elles n’approchent des rois que pour leur dire de tristes vérités ; et pendant qu’une foule de courtisans leur représentent sans cesse un peuple heureux sous leur gouvernement, elles viennent démentir la flatterie, et apporter au pied du trône les gémissements et les larmes dont elles sont dépositaires.\par
C’est un pesant fardeau, mon cher Usbek, que celui de la vérité, lorsqu’il faut la porter jusqu’aux princes : ils doivent bien penser que ceux qui le font y sont contraints, et qu’ils ne se résoudraient jamais à faire des démarches si tristes et si affligeantes pour ceux qui les font, s’ils n’y étaient forcés par leur devoir, leur respect, et même leur amour.\par

\dateline{De Paris, le 21 de la lune de Gemmadi I, 1720.}
\section[{Lettre CXLI. Rica au même. À ***.}]{Lettre CXLI. Rica au même. À ***.}\renewcommand{\leftmark}{Lettre CXLI. Rica au même. À ***.}

\noindent J’irai te voir sur la fin de la semaine : que les jours couleront agréablement avec toi !\par
Je fus présenté, il y a quelques jours, à une dame de la cour, qui avait quelque envie de voir ma figure étrangère. Je la trouvai belle, digne des regards de notre monarque, et d’un rang auguste dans le lieu sacré où son cœur repose.\par
Elle me fit mille questions sur les mœurs des Persans, et sur la manière de vivre des Persanes : il me parut que la vie du sérail n’était pas de son goût, et qu’elle trouvait de la répugnance à voir un homme partagé entre dix ou douze femmes. Elle ne put voir sans envie le bonheur de l’un, et sans pitié la condition des autres. Comme elle aime la lecture, surtout celle des poètes et des romans, elle souhaita que je lui parlasse des nôtres : ce que je lui en dis redoubla sa curiosité ; elle me pria de lui faire traduire un fragment de quelques-uns de ceux que j’ai apportés. Je le fis, et je lui envoyai, quelques jours après, un conte persan : peut-être seras-tu bien aise de le voir travesti.\par
Du temps de Cheik-Ali-Can, il y avait en Perse une femme nommée Zuléma : elle savait par cœur tout le saint Alcoran ; il n’y avait point de dervis qui entendît mieux qu’elle les traditions des saints prophètes ; les docteurs arabes n’avaient rien dit de si mystérieux qu’elle n’en comprît tous les sens ; et elle joignait à tant de connaissances un certain caractère d’esprit enjoué, qui laissait à peine deviner si elle voulait amuser ceux à qui elle parlait, ou les instruire.\par
Un jour qu’elle était avec ses compagnes dans une des salles du sérail, une d’elles lui demanda ce qu’elle pensait de l’autre vie, et si elle ajoutait foi à cette ancienne tradition de nos docteurs, que le paradis n’est fait que pour les hommes.\par
C’est le sentiment commun, leur dit-elle ; il n’y a rien que l’on n’ait fait pour dégrader notre sexe. Il y a même une nation répandue par toute la Perse, qu’on appelle la nation juive, qui soutient, par l’autorité de ses livres sacrés, que nous n’avons point d’âme.\par
Ces opinions si injurieuses n’ont d’autre origine que l’orgueil des hommes, qui veulent porter leur supériorité au-delà même de leur vie ; et ne pensent pas que, dans le grand jour, toutes les créatures paraîtront devant Dieu comme le néant, sans qu’il y ait entre elles de prérogatives que celles que la vertu y aura mises.\par
Dieu ne se bornera point dans ses récompenses : et comme les hommes qui auront bien vécu, et bien usé de l’empire qu’ils ont ici-bas sur nous, seront dans un paradis plein de beautés célestes et ravissantes, et telles que, si un mortel les avait vues, il se donnerait aussitôt la mort, dans l’impatience d’en jouir ; aussi les femmes vertueuses iront dans un lieu de délices, où elles seront enivrées d’un torrent de voluptés, avec des hommes divins qui leur seront soumis : chacune d’elles aura un sérail, dans lequel ils seront enfermés ; et des eunuques, encore plus fidèles que les nôtres, pour les garder.\par
J’ai lu, ajouta-t-elle, dans un livre arabe, qu’un homme, nommé Ibrahim, était d’une jalousie insupportable. Il avait douze femmes extrêmement belles, qu’il traitait d’une manière très dure : il ne se fiait plus à ses eunuques, ni aux murs de son sérail ; il les tenait presque toujours sous la clef, enfermées dans leur chambre, sans qu’elles pussent se voir ni se parler ; car il était même jaloux d’une amitié innocente : toutes ses actions prenaient la teinture de sa brutalité naturelle ; jamais une douce parole ne sortit de sa bouche ; et jamais il ne fit un moindre signe qui n’ajoutât quelque chose à la rigueur de leur esclavage.\par
Un jour qu’il les avait toutes assemblées dans une salle de son sérail, une d’entre elles, plus hardie que les autres, lui reprocha son mauvais naturel. Quand on cherche si fort les moyens de se faire craindre, lui dit-elle, on trouve toujours auparavant ceux de se faire haïr. Nous sommes si malheureuses, que nous ne pouvons nous empêcher de désirer un changement : d’autres, à ma place, souhaiteraient votre mort ; je ne souhaite que la mienne : et, ne pouvant espérer d’être séparée de vous que par là, il me sera encore bien doux d’en être séparée. Ce discours, qui aurait dû le toucher, le fit entrer dans une furieuse colère ; il tira son poignard, et le lui plongea dans le sein. Mes chères compagnes, dit-elle d’une voix mourante, si le ciel a pitié de ma vertu, vous serez vengées. À ces mots, elle quitta cette vie infortunée, pour aller dans le séjour des délices, où les femmes qui ont bien vécu jouissent d’un bonheur qui se renouvelle toujours.\par
D’abord elle vit une prairie riante, dont la verdure était relevée par les peintures des fleurs les plus vives : un ruisseau, dont les eaux étaient plus pures que le cristal, y faisait un nombre infini de détours. Elle entra ensuite dans des bocages charmants, dont le silence n’était interrompu que par le doux chant des oiseaux ; de magnifiques jardins se présentèrent ensuite ; la nature les avait ornés avec sa simplicité, et toute sa magnificence. Elle trouva enfin un palais superbe préparé pour elle et rempli d’hommes célestes destinés à ses plaisirs.\par
Deux d’entre eux se présentèrent aussitôt pour la déshabiller ; d’autres la mirent dans le bain, et la parfumèrent des plus délicieuses essences ; on lui donna ensuite des habits infiniment plus riches que les siens ; après quoi on la mena dans une grande salle, où elle trouva un feu fait avec des bois odoriférants, et une table couverte des mets les plus exquis. Tout semblait concourir au ravissement de ses sens : elle entendait d’un côté une musique d’autant plus divine qu’elle était plus tendre ; de l’autre, elle ne voyait que des danses de ces hommes divins, uniquement occupés à lui plaire. Cependant tant de plaisirs ne devaient servir qu’à la conduire insensiblement à des plaisirs plus grands. On la mena dans sa chambre ; et, après l’avoir encore une fois déshabillée, on la porta dans un lit superbe, où deux hommes d’une beauté charmante la reçurent dans leurs bras. C’est pour lors qu’elle fut enivrée, et que ses ravissements passèrent même ses désirs. Je suis toute hors de moi, leur disait-elle ; je croirais mourir, si je n’étais sûre de mon immortalité. C’en est trop, laissez-moi ; je succombe sous la violence des plaisirs. Oui, vous rendez un peu le calme à mes sens ; je commence à respirer et à revenir à moi-même. D’où vient que l’on a ôté les flambeaux ? Que ne puis-je à présent considérer votre beauté divine ? Que ne puis-je voir… Mais pourquoi voir ? Vous me faites rentrer dans mes premiers transports. Ô dieux ! que ces ténèbres sont aimables ! Quoi ! je serai immortelle, et immortelle avec vous ! je serai… Non, je vous demande grâce, car je vois bien que vous êtes gens à n’en demander jamais.\par
Après plusieurs commandements réitérés, elle fut obéie : mais elle ne le fut que lorsqu’elle voulut l’être bien sérieusement. Elle se reposa languissamment, et s’endormit dans leurs bras. Deux moments de sommeil réparèrent sa lassitude : elle reçut deux baisers qui l’enflammèrent soudain, et lui firent ouvrir les yeux. Je suis inquiète, dit-elle ; je crains que vous ne m’aimiez plus. C’était un doute dans lequel elle ne voulait pas rester longtemps : aussi eut-elle avec eux tous les éclaircissements qu’elle pouvait désirer. Je suis désabusée, s’écria-t-elle ; pardon, pardon ; je suis sûre de vous. Vous ne me dites rien, mais vous prouvez mieux que tout ce que vous me pourriez dire : oui, oui, je vous le confesse, on n’a jamais tant aimé. Mais quoi ! vous vous disputez tous deux l’honneur de me persuader ! Ah ! si vous vous disputez, si vous joignez l’ambition au plaisir de ma défaite, je suis perdue ; vous serez tous deux vainqueurs, il n’y aura que moi de vaincue ; mais je vous vendrai bien cher la victoire.\par
Tout ceci ne fut interrompu que par le jour. Ses fidèles et aimables domestiques entrèrent dans sa chambre, et firent lever ces deux jeunes hommes, que deux vieillards ramenèrent dans les lieux où ils étaient gardés pour ses plaisirs. Elle se leva ensuite, et parut d’abord à cette cour idolâtre dans les charmes d’un déshabillé simple, et ensuite couverte des plus somptueux ornements. Cette nuit l’avait embellie ; elle avait donné de la vie à son teint, et de l’expression à ses grâces. Ce ne fut pendant tout le jour que danses, que concerts, que festins, que jeux, que promenades ; et l’on remarquait qu’Anaïs se dérobait de temps en temps, et volait vers ses deux jeunes héros ; après quelques précieux instants d’entrevue, elle revenait vers la troupe qu’elle avait quittée, toujours avec un visage serein. Enfin, sur le soir, on la perdit tout à fait : elle alla s’enfermer dans le sérail, où elle voulait, disait-elle, faire connaissance avec ces captifs immortels qui devaient à jamais vivre avec elle. Elle visita donc les appartements de ces lieux les plus reculés et les plus charmants où elle compta cinquante esclaves d’une beauté miraculeuse : elle erra toute la nuit de chambre en chambre, recevant partout des hommages toujours différents, et toujours les mêmes.\par
Voilà comment l’immortelle Anaïs passait sa vie, tantôt dans des plaisirs éclatants, tantôt dans des plaisirs solitaires ; admirée d’une troupe brillante, ou bien aimée d’un amant éperdu : souvent elle quittait un palais enchanté pour aller dans une grotte champêtre ; les fleurs semblaient naître sous ses pas, et les jeux se présentaient en foule au-devant d’elle.\par
Il y avait plus de huit jours qu’elle était dans cette demeure heureuse, que, toujours hors d’elle-même ; elle n’avait pas fait une seule réflexion : elle avait joui de son bonheur sans le connaître, et sans avoir eu un seul de ces moments tranquilles, où l’âme se rend, pour ainsi dire, compte à elle-même, et s’écoute dans le silence des passions.\par
Les bienheureux ont des plaisirs si vifs, qu’ils peuvent rarement jouir de cette liberté d’esprit : c’est pour cela qu’attachés invinciblement aux objets présents, ils perdent entièrement la mémoire des choses passées, et n’ont plus aucun souci de ce qu’ils ont connu ou aimé dans l’autre vie.\par
Mais Anaïs, dont l’esprit était vraiment philosophe, avait passé presque toute sa vie à méditer : elle avait poussé ses réflexions beaucoup plus loin qu’on n’aurait dû l’attendre d’une femme laissée à elle-même. La retraite austère que son mari lui avait fait garder ne lui avait laissé que cet avantage. C’est cette force d’esprit qui lui avait fait mépriser la crainte dont ses compagnes étaient frappées, et la mort, qui devait être la fin de ses peines et le commencement de sa félicité.\par
Ainsi elle sortit peu à peu de l’ivresse des plaisirs, et s’enferma seule dans un appartement de son palais. Elle se laissa aller à des réflexions bien douces sur sa condition passée, et sur sa félicité présente ; elle ne put s’empêcher de s’attendrir sur le malheur de ses compagnes : on est sensible à des tourments que l’on a partagés. Anaïs ne se tint pas dans les simples bornes de la compassion : plus tendre envers ces infortunées, elle se sentit portée à les secourir.\par
Elle donna ordre à un de ces jeunes hommes qui étaient auprès d’elle de prendre la figure de son mari ; d’aller dans son sérail de s’en rendre maître ; de l’en chasser, et d’y rester à sa place jusqu’à ce qu’elle le rappelât.\par
L’exécution fut prompte : il fendit les airs, arriva à la porte du sérail d’Ibrahim, qui n’y était pas. Il frappe, tout lui est ouvert ; les eunuques tombent à ses pieds : il vole vers les appartements où les femmes d’Ibrahim étaient enfermées. Il avait, en passant, pris les clefs dans la poche de ce jaloux, à qui il s’était rendu invisible. Il entre, et les surprend d’abord par son air doux et affable ; et, bientôt après, il les surprend davantage par ses empressements et par la rapidité de ses entreprises. Toutes eurent leur part de l’étonnement ; et elles l’auraient pris pour un songe, s’il y eût eu moins de réalité.\par
Pendant que ces nouvelles scènes se jouent dans le sérail, Ibrahim heurte, se nomme, tempête, et crie. Après avoir essuyé bien des difficultés, il entre, et jette les eunuques dans un désordre extrême. Il marche à grands pas ; mais il recule en arrière, et tombe comme des nues, quand il voit le faux Ibrahim, sa véritable image, dans toutes les libertés d’un maître. Il crie au secours ; il veut que les eunuques lui aident à tuer cet imposteur ; mais il n’est pas obéi. Il n’a plus qu’une faible ressource, c’est de s’en rapporter au jugement de ses femmes. Dans une heure le faux Ibrahim avait séduit tous ses juges. Il est chassé et traîné indignement hors du sérail, et il aurait reçu la mort mille fois, si son rival n’avait ordonné qu’on lui sauvât la vie. Enfin, le nouvel Ibrahim, resté maître du champ de bataille, se montra de plus en plus digne d’un tel choix, et se signala par des miracles jusqu’alors inconnus. Vous ne ressemblez pas à Ibrahim, disaient ces femmes. Dites, dites plutôt que cet imposteur ne me ressemble pas, disait le triomphant Ibrahim : comment faut-il faire pour être votre époux, si ce que je fais ne suffit pas ?\par
Ah ! nous n’avons garde de douter, dirent les femmes. Si vous n’êtes pas Ibrahim, il nous suffit que vous ayez si bien mérité de l’être : vous êtes plus Ibrahim en un jour qu’il ne l’a été dans le cours de dix années. Vous me promettez donc, reprit-il, que vous vous déclarerez en ma faveur contre cet imposteur ? N’en doutez pas, dirent-elles d’une commune voix ; nous vous jurons une fidélité éternelle : nous n’avons été que trop longtemps abusées : le traître ne soupçonnait point notre vertu, il ne soupçonnait que sa faiblesse ; nous voyons bien que les hommes ne sont point faits comme lui ; c’est à vous sans doute qu’ils ressemblent : si vous saviez combien vous nous le faites haïr ! Ah ! je vous donnerai souvent de nouveaux sujets de haine, reprit le faux Ibrahim : vous ne connaissez point encore tout le tort qu’il vous a fait. Nous jugeons de son injustice par la grandeur de votre vengeance, reprirent-elles. Oui, vous avez raison, dit l’homme divin ; j’ai mesuré l’expiation au crime : je suis bien aise que vous soyez contentes de ma manière de punir. Mais, dirent ces femmes, si cet imposteur revient, que ferons-nous ? Il lui serait, je crois, difficile de vous tromper, répondit-il : dans la place que j’occupe auprès de vous, on ne se soutient guère par la ruse ; et d’ailleurs je l’enverrai si loin, que vous n’entendrez plus parler de lui, pour lors je prendrai sur moi le soin de votre bonheur. Je ne serai point jaloux ; je saurai m’assurer de vous, sans vous gêner ; j’ai assez bonne opinion de mon mérite pour croire que vous me serez fidèles : si vous n’étiez pas vertueuses avec moi, avec qui le seriez-vous ? Cette conversation dura longtemps entre lui et ces femmes, qui, plus frappées de la différence des deux Ibrahims que de leur ressemblance, ne songeaient pas même à se faire éclaircir de tant de merveilles. Enfin le mari désespéré revint encore les troubler ; il trouva toute sa maison dans la joie, et les femmes plus incrédules que jamais. La place n’était pas tenable pour un jaloux ; il sortit furieux ; et un instant après le faux Ibrahim le suivit, le prit, le transporta dans les airs, et le laissa à quatre cents lieues de là.\par
Ô dieux ! dans quelle désolation se trouvèrent ces femmes dans l’absence de leur cher Ibrahim ! Déjà leurs eunuques avaient repris leur sévérité naturelle ; toute la maison était en larmes ; elles s’imaginaient quelquefois que tout ce qu’il leur était arrivé n’était qu’un songe ; elles se regardaient toutes les unes les autres, et se rappelaient les moindres circonstances de ces étranges aventures. Enfin, Ibrahim revint, toujours plus aimable ; il leur parut que son voyage n’avait pas été pénible. Le nouveau maître prit une conduite si opposée à celle de l’autre qu’elle surprit tous les voisins. Il congédia tous les eunuques, rendit sa maison accessible à tout le monde : il ne voulut pas même souffrir que ses femmes se voilassent. C’était une chose singulière de les voir dans les festins, parmi des hommes, aussi libres qu’eux. Ibrahim crut avec raison que les coutumes du pays n’étaient pas faites pour des citoyens comme lui. Cependant il ne se refusait aucune dépense : il dissipa avec une immense profusion les biens du jaloux, qui, de retour trois ans après des pays lointains où il avait été transporté, ne trouva plus que ses femmes et trente-six enfants.\par

\dateline{De Paris, le 26 de la lune de Gemmadi I, 1720.}
\section[{Lettre CXLII. Rica à Usbek. À ***.}]{Lettre CXLII. Rica à Usbek. À ***.}\renewcommand{\leftmark}{Lettre CXLII. Rica à Usbek. À ***.}

\noindent Voici une lettre que je reçus hier d’un savant : elle te paraîtra singulière.\par
"MONSIEUR.\par
Il y a six mois que j’ai recueilli la succession d’un oncle très riche, qui m’a laissé cinq ou six cent mille livres, et une maison superbement meublée. Il y a plaisir d’avoir du bien lorsqu’on en sait faire un bon usage. Je n’ai point d’ambition ni de goût pour les plaisirs : je suis presque toujours enfermé dans un cabinet, où je mène la vie d’un savant. C’est dans ce lieu que l’on trouve un curieux amateur de la vénérable antiquité.\par
Lorsque mon oncle eut fermé les yeux, j’aurais fort souhaité de le faire enterrer avec les cérémonies observées par les anciens Grecs et Romains : mais je n’avais pour lors ni lacrymatoires, ni urnes, ni lampes antiques.\par
Mais depuis je me suis bien pourvu de ces précieuses raretés. Il y a quelques jours que je vendis ma vaisselle d’argent pour acheter une lampe de terre qui avait servi à un philosophe stoïcien. Je me suis défait de toutes les glaces dont mon oncle avait couvert presque tous les murs de ses appartements, pour avoir un petit miroir un peu fêlé, qui fut autrefois à l’usage de Virgile : je suis charmé d’y voir ma figure représentée, au lieu de celle du cygne de Mantoue. Ce n’est pas tout : j’ai acheté cent louis d’or cinq ou six pièces de monnaie de cuivre qui avait cours il y a deux mille ans. Je ne sache pas avoir à présent dans ma maison un seul meuble qui n’ait été fait avant la décadence de l’empire. J’ai un petit cabinet de manuscrits fort précieux et fort chers : quoique je me tue la vue à les lire, j’aime beaucoup mieux m’en servir que des exemplaires imprimés, qui ne sont pas si corrects, et que tout le monde a entre les mains. Quoique je ne sorte presque jamais, je ne laisse pas d’avoir une passion démesurée de connaître tous les anciens chemins qui étaient du temps des Romains. Il y en a un qui est près de chez moi, qu’un proconsul des Gaules fit faire il y a environ douze cents ans : lorsque je vais à ma maison de campagne, je ne manque jamais d’y passer, quoiqu’il soit très incommode, et qu’il m’allonge de plus d’une lieue ; mais ce qui me fait enrager, c’est qu’on y a mis des poteaux de bois de distance en distance, pour marquer l’éloignement des villes voisines ; je suis désespéré de voir ces misérables indices, au lieu des colonnes milliaires qui y étaient autrefois : je ne doute pas que je ne les fasse rétablir par mes héritiers, et que je ne les engage à cette dépense par mon testament. Si vous avez, monsieur, quelque manuscrit persan, vous me ferez plaisir de m’en accommoder : je vous le payerai tout ce que vous voudrez ; et je vous donnerai par-dessus le marché quelques ouvrages de ma façon, par lesquels vous verrez que je ne suis point un membre inutile de la république des lettres. Vous y remarquerez entre autres une dissertation, où je prouve que la couronne dont on se servait autrefois dans les triomphes était de chêne, et non pas de laurier : vous en admirerez une autre, où je prouve, par de doctes conjectures tirées des plus graves auteurs grecs, que Cambyses fut blessé à la jambe gauche, et non pas à la droite ; une autre, où je prouve qu’un petit front était une beauté recherchée par les Romains. Je vous enverrai encore un volume in-quarto, en forme d’explication d’un vers du sixième livre de l’Énéide de Virgile. Vous ne recevrez tout ceci que dans quelques jours ; et quant à présent, je me contente de vous envoyer ce fragment d’un ancien mythologiste grec, qui n’avait point paru jusques ici, et que j’ai découvert dans la poussière d’une bibliothèque. Je vous quitte pour une affaire importante que j’ai sur les bras : il s’agit de restituer un beau passage de Pline le naturaliste, que les copistes du cinquième siècle ont étrangement défiguré. Je suis, etc."\par

\begin{center}
\noindent \centerline{FRAGMENT D’UN ANCIEN MYTHOLOGISTE.}\par
\end{center}

\noindent "Dans une île près des Orcades, il naquit un enfant qui avait pour père Éole, dieu des vents, et pour mère une nymphe de Calédonie. On dit de lui qu’il apprit tout seul à compter avec ses doigts ; et que, dès l’âge de quatre ans, il distinguait si parfaitement les métaux, que sa mère ayant voulu lui donner une bague de laiton au lieu d’une d’or, il reconnut la tromperie, et la jeta par terre.\par
Dès qu’il fut grand, son père lui apprit le secret d’enfermer les vents dans une outre, qu’il vendait ensuite à tous les voyageurs ; mais comme la marchandise n’était pas fort prisée dans son pays, il le quitta, et se mit à courir le monde en compagnie de l’aveugle dieu du hasard.\par
Il apprit dans ses voyages que dans la Bétique l’or reluisait de toutes parts : cela fit qu’il y précipita ses pas. Il y fut fort mal reçu de Saturne, qui régnait pour lors ; mais ce dieu ayant quitté la terre, il s’avisa aller dans tous les carrefours, où il criait sans cesse d’une voix rauque : Peuples de Bétique, vous croyez être riches parce que vous avez de l’or et de l’argent ; votre erreur me fait pitié : croyez-moi, quittez le pays des vils métaux ; venez dans l’empire de l’imagination ; et je vous promets des richesses qui vous étonneront vous-mêmes. Aussitôt il ouvrit une grande partie des outres qu’il avait apportées, et il distribua de sa marchandise à qui en voulut.\par
Le lendemain il revint dans les mêmes carrefours, et il s’écria : Peuples de Bétique, voulez-vous être riches ? Imaginez-vous que je le suis beaucoup, et que vous l’êtes beaucoup aussi ; mettez-vous tous les matins dans l’esprit que votre fortune a doublé pendant la nuit ; levez-vous ensuite ; et, si vous avez des créanciers, allez les payer de ce que vous aurez imaginé ; et dites-leur d’imaginer à leur tour.\par
Il reparut quelques jours après, et il parla ainsi : Peuples de Bétique, je vois bien que votre imagination n’est pas si vive que les premiers jours ; laissez-vous conduire à la mienne ; je mettrai tous les matins devant vos yeux un écriteau qui sera pour vous la source des richesses : vous n’y verrez que quatre paroles ; mais elles seront bien significatives, car elles régleront la dot de vos femmes, la légitime de vos enfants, le nombre de vos domestiques. Et quant à vous, dit-il à ceux de la troupe qui étaient le plus près de lui ; quant à vous, mes chers enfants (je puis vous appeler de ce nom, car vous avez reçu de moi une seconde naissance), mon écriteau décidera de la magnificence de vos équipages, de la somptuosité de vos festins, du nombre et de la pension de vos maîtresses.\par
À quelques jours de là il arriva dans le carrefour, tout essoufflé ; et, transporté de colère, il s’écria : Peuples de Bétique, je vous avais conseillé d’imaginer, et je vois que vous ne le faites pas : eh bien ! à présent je vous l’ordonne. Là-dessus, il les quitta brusquement ; mais la réflexion le rappela sur ses pas. J’apprends que quelques-uns de vous sont assez détestables pour conserver leur or et leur argent. Encore passe pour l’argent ; mais pour de l’or… pour de l’or… Ah ! cela me met dans une indignation… Je jure par mes outres sacrées que, s’ils ne viennent me l’apporter, je les punirai sévèrement. Puis il ajouta d’un air tout à fait persuasif : Croyez-vous que ce soit pour garder ces misérables métaux que je vous les demande ? Une marque de ma candeur, c’est que, lorsque vous me les apportâtes il y a quelques jours, je vous en rendis sur-le-champ la moitié.\par
Le lendemain on l’aperçut de loin, et on le vit s’insinuer avec une voix douce et flatteuse : Peuples de Bétique, j’apprends que vous avez une partie de vos trésors dans les pays étrangers ; je vous prie, faites-les-moi venir ; vous me ferez plaisir, et je vous en aurai une reconnaissance éternelle.\par
Le fils d’Éole parlait à des gens qui n’avaient pas grande envie de rire ; ils ne purent pourtant s’en empêcher : ce qui fit qu’il s’en retourna bien confus. Mais, reprenant courage, il hasarda encore une petite prière : je sais que vous avez des pierres précieuses ; au nom de Jupiter, défaites-vous-en : rien ne vous appauvrit comme ces sortes de choses ; défaites-vous-en, vous dis-je : si vous ne le pouvez pas par vous-mêmes, je vous donnerai des hommes d’affaires excellents. Que de richesse vont couler chez vous, si vous faites ce que je vous conseille ! Oui, je vous promets tout ce qu’il y aura de plus pur dans mes outres.\par
Enfin il monta sur un tréteau, et, prenant une voix plus assurée, il dit : Peuples de Bétique, j’ai comparé l’heureux état dans lequel vous êtes, avec celui où je vous trouvai lorsque j’arrivai ici ; je vous vois le plus riche peuple de la terre : mais, pour achever votre fortune, souffrez que je vous ôte la moitié de vos biens. À ces mots, d’une aile légère le fils d’Éole disparut, et laissa ses auditeurs dans une consternation inexprimable ; ce qui fit qu’il revint le lendemain, et parla ainsi : Je m’aperçus hier que mon discours vous déplut extrêmement. Eh bien ! prenez que je ne vous aie rien dit. Il n’y a qu’à prendre d’autres expédients pour arriver au but que je me suis proposé. Assemblons nos richesses dans un même endroit ; nous le pouvons facilement, car elles ne tiennent pas un gros volume. Aussitôt il en disparut les trois quarts."\par

\dateline{De Paris, le 9 de la lune de Chahban, 1720.}
\section[{Lettre CXLIII. Rica à Nathanaël Levi, médecin juif. À Livourne.}]{Lettre CXLIII. Rica à Nathanaël Levi, médecin juif. À Livourne.}\renewcommand{\leftmark}{Lettre CXLIII. Rica à Nathanaël Levi, médecin juif. À Livourne.}

\noindent Tu me demandes ce que je pense de la vertu des amulettes, et de la puissance des talismans. Pourquoi t’adresses-tu à moi ? tu es Juif, et je suis mahométan : c’est-à-dire que nous sommes tous deux bien crédules.\par
Je porte toujours sur moi plus de deux mille passages du saint Alcoran ; j’attache à mes bras un petit paquet, où sont écrits les noms de plus de deux cents dervis : ceux d’Ali, de Fatmé, et de tous les Purs, sont cachés en plus de vingt endroits de mes habits.\par
Cependant je ne désapprouve point ceux qui rejettent cette vertu que l’on attribue à de certaines paroles : il nous est bien plus difficile de répondre à leurs raisonnements, qu’à eux de répondre à nos expériences.\par
Je porte tous ces chiffons sacrés par une longue habitude, pour me conformer à une pratique universelle : je crois que, s’ils n’ont pas plus de vertu que les bagues et les autres ornements dont on se pare, ils n’en ont pas moins. Mais toi, tu mets toute ta confiance sur quelques lettres mystérieuses, et, sans cette sauvegarde, tu serais dans un effroi continuel.\par
Les hommes sont bien malheureux ! ils flottent sans cesse entre de fausses espérances et des craintes ridicules : et, au lieu de s’appuyer sur la raison, ils se font des monstres qui les intimident, ou des fantômes qui les séduisent.\par
Quel effet veux-tu que produise l’arrangement de certaines lettres ? Quel effet veux-tu que leur dérangement puisse troubler ? quelle relation ont-elles avec les vents, pour apaiser les tempêtes ; avec la poudre à canon, pour en vaincre l’effort ; avec ce que les médecins appellent l’humeur peccante et la cause morbifique des maladies, pour les guérir ?\par
Ce qu’il y a d’extraordinaire, c’est que ceux qui fatiguent leur raison pour lui faire rapporter de certains événements à des vertus occultes, n’ont pas un moindre effort à faire pour s’empêcher d’en voir la véritable cause.\par
Tu me diras que de certains prestiges ont fait gagner une bataille : et moi je te dirai qu’il faut que tu t’aveugles, pour ne pas trouver dans la situation du terrain, dans le nombre ou dans le courage des soldats, dans l’expérience des capitaines, des causes suffisantes pour produire cet effet dont tu veux ignorer la cause.\par
Je te passe pour un moment qu’il y ait des prestiges : passe-moi à mon tour, pour un moment, qu’il n’y en ait point ; car cela n’est pas impossible. Cette concession que tu me fais n’empêche pas que deux armées ne puissent se battre : veux-tu que, dans ce cas-là, aucune des deux ne puisse remporter la victoire ?\par
Crois-tu que leur sort restera incertain jusqu’à ce que quelque puissance invisible vienne le déterminer ? que tous les coups seront perdus, toute la prudence vaine, et tout le courage inutile ?\par
Penses-tu que la mort, dans ces occasions, rendue présente de mille manières, ne puisse pas produire dans les esprits ces terreurs paniques que tu as tant de peine à expliquer ? Veux-tu que, dans une armée de cent mille hommes, il ne puisse pas y avoir un seul homme timide ? Crois-tu que le découragement de celui-ci ne puisse pas produire le découragement d’un autre ? que le second, qui quitte un troisième, ne lui fasse pas bientôt abandonner un quatrième ? Il n’en faut pas davantage pour que le désespoir de vaincre saisisse soudain toute une armée, et la saisisse d’autant plus facilement qu’elle se trouve plus nombreuse.\par
Tout le monde sait, et tout le monde sent, que les hommes, comme toutes les créatures qui tendent à conserver leur être, aiment passionnément la vie ; on sait cela en général : et on cherche pourquoi, dans une certaine occasion particulière, ils ont craint de la perdre ?\par
Quoique les livres sacrés de toutes les nations soient remplis de ces terreurs paniques ou surnaturelles, je n’imagine rien de si frivole, parce que, pour s’assurer qu’un effet qui peut être produit par cent mille causes naturelles est surnaturel, il faut avoir auparavant examiné si aucune de ces causes n’a agi ; ce qui est impossible.\par
Je ne t’en dirai pas davantage Nathanaël : il me semble que la matière ne mérite pas d’être si sérieusement traitée.\par

\dateline{De Paris, le 20 de la lune de Chahban, 1720.}
\noindent {\itshape P. S.} Comme je finissais, j’ai entendu crier dans la rue une lettre d’un médecin de province à un médecin de Paris (car ici toutes les bagatelles s’impriment, se publient, et s’achètent) : j’ai cru que je ferais bien de te l’envoyer, parce qu’elle a du rapport à notre sujet. Il y a bien des choses que je n’entends pas : mais toi, qui es médecin, tu dois entendre le langage de tes confrères.

\begin{center}
\noindent \centerline{{\scshape Lettre d’un médecin de province à un médecin de paris}.}\par
\end{center}

\noindent Il y avait dans notre ville un malade qui ne dormait point depuis trente-cinq jours : son médecin lui ordonna l’opium ; mais il ne pouvait se résoudre à le prendre ; et il avait la coupe à la main, qu’il était plus indéterminé que jamais. Enfin il dit à son médecin : Monsieur, je vous demande quartier seulement jusqu’à demain : je connais un homme qui n’exerce pas la médecine, mais qui a chez lui un nombre innombrable de remèdes contre l’insomnie ; souffrez que je l’envoie quérir : et, si je ne dors pas cette nuit, je vous promets que je reviendrai à vous. Le médecin congédié, le malade fit fermer les rideaux, et dit à un petit laquais : Tiens, va-t’en chez M. Anis, et dis-lui qu’il vienne me parler. M. Anis arrive. Mon cher monsieur Anis, je me meurs, je ne puis dormir : n’auriez-vous point, dans votre boutique, la C. du G., ou bien quelque livre de dévotion composé par un révérend père jésuite, que vous n’ayez pas pu vendre ? car souvent les remèdes les plus gardés sont les meilleurs. Monsieur, dit le libraire, j’ai chez moi la Cour sainte du P. Caussin, en six volumes, à votre service : je vais vous l’envoyer ; je souhaite que vous vous en trouviez bien. Si vous voulez les œuvres du révérend père Rodriguez, jésuite espagnol, ne vous en faites faute. Mais, croyez-moi, tenons-nous-en au père Caussin ; j’espère, avec l’aide de Dieu, qu’une période du père Caussin vous fera autant d’effet qu’un feuillet tout entier de la C. du G. Là-dessus M. Anis sortit, et courut chercher le remède à sa boutique. La Cour sainte arrive : on en secoue la poudre ; le fils du malade, jeune écolier, commence à la lire : il en sentit le premier l’effet, à la seconde page il ne prononçait plus que d’une voix mal articulée, et déjà toute la compagnie se sentait affaiblie : un instant après tout ronfla, excepté le malade, qui après avoir été longtemps éprouvé, s’assoupit à la fin.\par
Le médecin arrive de grand matin. Hé bien ! a-t-on pris mon opium ? On ne lui répond rien : la femme, la fille, le petit garçon, tous transportés de joie, lui montrent le père Caussin. Il demande ce que c’est ; on lui dit : Vive le père Caussin ! il faut l’envoyer relier. Qui l’eût dit ? qui l’eût cru ? c’est un miracle ! Tenez, monsieur, voyez donc le père Caussin : c’est ce volume-là qui a fait dormir mon père. Et là-dessus on lui expliqua la chose, comme elle s’était passée.\par
Le médecin était un homme subtil, rempli des mystères de la cabale, et de la puissance des paroles et des esprits : cela le frappa ; et, après plusieurs réflexions, il résolut de changer absolument sa pratique. Voilà un fait bien singulier, disait-il. Je tiens une expérience ; il faut la pousser plus loin. Hé pourquoi un esprit ne pourrait-il pas transmettre à son ouvrage les mêmes qualités qu’il a lui-même ? ne le voyons-nous pas tous les jours ? Au moins cela vaut-il bien la peine de l’essayer. Je suis las des apothicaires ; leurs sirops, leurs juleps, et toutes les drogues galéniques ruinent les malades et leur santé : changeons de méthode ; éprouvons la vertu des esprits. Sur cette idée il dressa une nouvelle pharmacie, comme vous allez voir par la description que je vous vais faire des principaux remèdes qu’il mit en pratique.\par
 {\itshape Tisane purgative.} \par
Prenez trois feuilles de la logique d’Aristote en grec ; deux feuilles d’un traité de théologie scolastique le plus aigu, comme, par exemple, du subtil Scot ; quatre de Paracelse ; une d’Avicenne ; six d’Averroès ; trois de Porphyre ; autant de Plotin ; autant de Jamblique : faites infuser le tout pendant vingt-quatre heures, et prenez-en quatre prises par jour.\par
 {\itshape Purgatif plus violent.} \par
Prenez dix {\itshape A***} du C*** concernant la B*** et la C*** des I*** ; faites-les distiller au bain-marie ; mortifiez une goutte de l’humeur âcre et piquante qui en viendra, dans un verre d’eau commune : avalez le tout avec confiance.\par
 {\itshape Vomitif.} \par
Prenez six harangues ; une douzaine d’oraisons funèbres indifféremment, prenant garde pourtant de ne point se servir de celles de M. de N. ; un recueil de nouveaux opéras ; cinquante romans ; trente mémoires nouveaux : mettez le tout dans un matras ; laissez-le en digestion pendant deux jours ; puis faites-le distiller au feu de sable. Et si tout cela ne suffit pas,\par
 {\itshape Autre plus puissant.} \par
Prenez une feuille de papier marbré, qui ait servit à couvrir un recueil des pièces des J. F. ; faites-la infuser l’espace de trois minutes ; faites chauffer une cuillerée de cette infusion ; et avalez.\par
 {\itshape Remède très simple pour guérir de l’asthme.} \par
Lisez tous les ouvrages du révérend père Maimbourg, ci-devant jésuite, prenant garde de ne vous arrêter qu’à la fin de chaque période : et vous sentirez la faculté de respirer vous revenir peu à peu, sans qu’il soit besoin de réitérer le remède.\par
 {\itshape Pour préserver de la gale, gratelle, teigne, farcin des chevaux.} \par
Prenez trois catégories d’Aristote, deux degrés métaphysiques, une distinction, six vers de Chapelain, une phrase tirée des lettres de M. l’abbé de Saint-Cyran : écrivez le tout sur un morceau de papier, que vous plierez, attacherez à un ruban, et porterez au col.\par
 {\itshape Miraculum chymicum, de violenta fermentatione \\
cum fumo, igne et flammâ.} \par
Misce Quesnellianam infusionem, cum infusione Lallemaniana ; fiat fermentatio cum magna vi, impetu et tonitru, acidis pugnantibus, et invicem penetrantibus alcalinos sales : fiet evaporatio ardentium spirituum. Pone liquorem fermentatum in alembico : nihil inde extrahes, et nihil invenies, nisi caput mortuum.\par
 {\itshape Lenitivum.} \par
Recipe Molinae anodyni chartas duas ; Escobaris relaxativi paginas sex ; Vasquii emollientis folium unum : infunde in aquae communis libras iiij. Ad consumptionem dimidiae partis colentur et exprimantur ; et in expressione dissolve Bauni detersivi et Tamburini abluentis folia iiij.\par
Fiat clyster.\par
 {\itshape In chlorosim, quam vulgus pallidos-colores, aut febrim-amatoriam, appellat.} \par
Recipe Aretini figuras quatuor ; R. Thomae Sanchii de matrimonio folia ij. Infundantur in aquae communis libras quinque.\par
Fiat ptisana aperiens.\par
Voilà les drogues que notre médecin mit en pratique, avec un succès imaginable. Il ne voulait pas, disait-il, pour ne pas ruiner ses malades, employer des remèdes rares, et qui ne se trouvent presque point : comme, par exemple, une épître dédicatoire qui n’ait fait bâiller personne ; une préface trop courte ; un mandement fait par un évêque ; et l’ouvrage d’un janséniste méprisé par un janséniste, ou bien admiré par un jésuite. Il disait que ces sortes de remèdes ne sont propres qu’à entretenir la charlatanerie, contre laquelle il avait une antipathie insurmontable.
\section[{Lettre CXLIV. Usbek à Rica.}]{Lettre CXLIV. Usbek à Rica.}\renewcommand{\leftmark}{Lettre CXLIV. Usbek à Rica.}

\noindent Je trouvai, il y a quelques jours, dans une maison de campagne où j’étais allé, deux savants qui ont ici une grande célébrité. Leur caractère me parut admirable. La conversation du premier, bien appréciée, se réduisait a ceci : Ce que j’ai dit est vrai, parce que je l’ai dit. La conversation du second portait sur autre chose : Ce que je n’ai pas dit n’est pas vrai, parce que je ne l’ai pas dit.\par
J’aimais assez le premier : car qu’un homme soit opiniâtre, cela ne me fait absolument rien ; mais qu’il soit impertinent, cela me fait beaucoup. Le premier défend ses opinions ; et c’est son bien ; le second attaque les opinions des autres ; et c’est le bien de tout le monde.\par
Oh, mon cher Usbek, que la vanité sert mal ceux qui en ont une dose plus forte que celle qui est nécessaire pour la conservation de la nature ! Ces gens-là veulent être admirés à force de déplaire. Ils cherchent à être supérieurs ; et ils ne sont pas seulement égaux.\par
Hommes modestes, venez, que je vous embrasse. Vous faites la douceur et le charme de la vie. Vous croyez que vous n’avez rien ; et moi je vous dis que vous avez tout. Vous pensez que vous n’humiliez personne ; et vous humiliez tout le monde. Et, quand je vous compare dans mon idée avec ces hommes absolus que je vois partout, je les précipite de leur tribunal, et je les mets à vos pieds.\par

\dateline{De Paris, le 22 de la lune de Chahban, 1720.}
\section[{Lettre CXLV. Usbek à ***.}]{Lettre CXLV. Usbek à ***.}\renewcommand{\leftmark}{Lettre CXLV. Usbek à ***.}

\noindent Un homme d’esprit est ordinairement difficile dans les sociétés. Il choisit peu de personnes ; il s’ennuie avec tout ce grand nombre de gens qu’il lui plaît appeler mauvaise compagnie ; il est impossible qu’il ne fasse un peu sentir son dégoût : autant d’ennemis.\par
Sûr de plaire quand il voudra, il néglige très souvent de le faire.\par
Il est porté à la critique, parce qu’il voit plus de choses qu’un autre, et les sent mieux.\par
Il ruine presque toujours sa fortune, parce que son esprit lui fournit pour cela un plus grand nombre de moyens.\par
Il échoue dans ses entreprises, parce qu’il hasarde beaucoup. Sa vue, qui se porte toujours loin, lui fait voir des objets qui sont à de trop grandes distances. Sans compter que, dans la naissance d’un projet, il est moins frappé des difficultés qui viennent de la chose, que des remèdes qui sont de lui, et qu’il tire de son propre fonds.\par
Il néglige les menus détails, dont dépend cependant la réussite de presque toutes les grandes affaires.\par
L’homme médiocre, au contraire, cherche à tirer parti de tout : il sent bien qu’il n’a rien à perdre en négligences.\par
L’approbation universelle est plus ordinairement pour l’homme médiocre. On est charmé de donner à celui-ci, on est enchanté d’ôter à celui-là. Pendant que l’envie fond sur l’un, et qu’on ne lui pardonne rien, on supplée tout en faveur de l’autre : la vanité se déclare pour lui.\par
Mais, si un homme d’esprit a tant de désavantages, que dirions-nous de la dure condition des savants ?\par
Je n’y pense jamais que je ne me rappelle une lettre d’un d’eux à un de ses amis. La voici.\par
"MONSIEUR,\par
Je suis un homme qui m’occupe, toutes les nuits, à regarder, avec des lunettes de trente pieds, ces grands corps qui roulent sur nos têtes ; et quand je veux me délasser, je prends mes petits microscopes, et j’observe un ciron ou une mite.\par
Je ne suis point riche, et je n’ai qu’une seule chambre : je n’ose même y faire du feu, parce que j’y tiens mon thermomètre, et que la chaleur étrangère le ferait hausser. L’hiver dernier, je pensai mourir de froid ; et quoique mon thermomètre, qui était au plus bas degré, m’avertît que mes mains allaient se geler, je ne me dérangeai point ; et j’ai la consolation d’être instruit exactement des changements de temps les plus insensibles de toute l’année passée.\par
Je me communique fort peu : et, de tous les gens que je vois, je n’en connais aucun. Mais il y a un homme à Stockholm, un autre à Leipsick, un autre à Londres, que je n’ai jamais vus, et que je ne verrai sans doute jamais, avec lesquels j’entretiens une correspondance si exacte, que je ne laisse pas passer un courrier sans leur écrire.\par
Mais, quoique je ne connaisse personne dans mon quartier, je suis dans une si mauvaise réputation, que je serai, à la fin, obligé de le quitter. Il y a cinq ans que je fus rudement insulté par une de mes voisines, pour avoir fait la dissection d’un chien qu’elle prétendait lui appartenir. La femme d’un boucher, qui se trouva là, se mit de la partie ; et, pendant que celle-là m’accablait d’injures, celle-ci m’assommait à coups de pierres, conjointement avec le docteur ***, qui était avec moi, et qui reçut un coup terrible sur l’os frontal et occipital, dont le siège de sa raison fut très ébranlé.\par
Depuis ce temps-là, dès qu’il s’écarte quelque chien au bout de la rue, il est aussitôt décidé qu’il a passé par mes mains. Une bonne bourgeoise qui en avait perdu un petit, qu’elle aimait, disait-elle, plus que ses enfants, vint l’autre jour s’évanouir dans ma chambre ; et, ne le trouvant pas, elle me cita devant le magistrat. Je crois que je ne serai jamais délivré de la malice importune de ces femmes qui, avec leurs voix glapissantes, m’étourdissent sans cesse de l’oraison funèbre de tous les automates qui sont morts depuis dix ans.\par
Je suis, etc."\par
Tous les savants étaient autrefois accusés de magie. Je n’en suis point étonné. Chacun disait en lui-même : j’ai porté les talents naturels aussi loin qu’ils peuvent aller ; cependant un certain savant a des avantages sur moi : il faut bien qu’il y ait là quelque diablerie.\par
À présent que ces sortes d’accusations sont tombées dans le décri, on a pris un autre tour ; et un savant ne saurait guère éviter le reproche d’irréligion ou d’hérésie. Il a beau être absous par le peuple : la plaie est faite ; elle ne se fermera jamais bien. C’est toujours pour lui un endroit malade. Un adversaire viendra, trente ans après, lui dire modestement : À Dieu ne plaise que je dise que ce dont on vous accuse soit vrai ! mais vous avez été obligé de vous défendre. C’est ainsi qu’on tourne contre lui sa justification même.\par
S’il écrit quelque histoire, et qu’il ait de la noblesse dans l’esprit, et quelque droiture dans le cœur, on lui suscite mille persécutions. On ira contre lui soulever le magistrat sur un fait qui s’est passé il y a mille ans. Et on voudra que sa plume soit captive, si elle n’est pas vénale.\par
Plus heureux cependant que ces hommes lâches, qui abandonnent leur foi pour une médiocre pension ; qui, à prendre toutes leurs impostures en détail, ne les vendent pas seulement une obole ; qui renversent la constitution de l’empire, diminuent les droits d’une puissance, augmentent ceux d’une autre, donnent aux princes, ôtent aux peuples, font revivre des droits surannés, flattent les passions qui sont en crédit de leur temps, et les vices qui sont sur le trône, imposant à la postérité, d’autant plus indignement qu’elle a moins de moyens de détruire leur témoignage.\par
Mais ce n’est point assez, pour un auteur, d’avoir essuyé toutes ces insultes ; ce n’est point assez pour lui d’avoir été dans une inquiétude continuelle sur le succès de son ouvrage. Il voit le jour enfin, cet ouvrage qui lui a tant coûté. Il lui attire des querelles de toutes parts. Et comment les éviter ? Il avait un sentiment ; il l’a soutenu par ses écrits : il ne savait pas qu’un homme, à deux cents lieues de lui, avait dit tout le contraire. Voilà cependant la guerre qui se déclare.\par
Encore s’il pouvait espérer d’obtenir quelque considération ! Non. Il n’est tout au plus estimé que de ceux qui se sont appliqués au même genre de science que lui. Un philosophe a un mépris souverain pour un homme qui a la tête chargée de faits ; et il est, à son tour, regardé comme un visionnaire par celui qui a une bonne mémoire.\par
Quant à ceux qui font profession d’une orgueilleuse ignorance, ils voudraient que tout le genre humain fût enseveli dans l’oubli où ils seront eux-mêmes.\par
Un homme à qui il manque un talent se dédommage en le méprisant : il ôte cet obstacle qu’il rencontrait entre le mérite et lui ; et, par là, se trouve au niveau de celui dont il redoute les travaux.\par
Enfin, il faut joindre, à une réputation équivoque, la privation des plaisirs et la perte de la santé.\par

\dateline{De Paris, le 20 de la lune de Chahban 1720.}
\section[{Lettre CXLVI. Usbek à Rhédi. À Venise.}]{Lettre CXLVI. Usbek à Rhédi. À Venise.}\renewcommand{\leftmark}{Lettre CXLVI. Usbek à Rhédi. À Venise.}

\noindent Il y a longtemps que l’on a dit que la bonne foi était l’âme d’un grand ministre.\par
Un particulier peut jouir de l’obscurité où il se trouve ; il ne se décrédite que devant quelques gens ; il se tient couvert devant les autres : mais un ministre qui manque à la probité a autant de témoins, autant de juges, qu’il y a de gens qu’il gouverne.\par
Oserai-je le dire ? le plus grand mal que fait un ministre sans probité n’est pas de desservir son prince et de ruiner son peuple : il y en a un autre, à mon avis, mille fois plus dangereux ; c’est le mauvais exemple qu’il donne.\par
Tu sais que j’ai longtemps voyagé dans les Indes. J’y ai vu une nation, naturellement généreuse, pervertie en un instant, depuis le dernier des sujets jusqu’aux plus grands, par le mauvais exemple d’un ministre : j’y ai vu tout un peuple, chez qui la générosité, la probité, la candeur et la bonne foi ont passé de tout temps pour les qualités naturelles, devenir tout à coup le dernier des peuples ; le mal se communiquer, et n’épargner pas même les membres les plus sains ; les hommes les plus vertueux faire des choses indignes ; et violer, dans toutes les occasions de leur vie, les premiers principes de la justice, sur ce vain prétexte qu’on la leur avait violée.\par
Ils appelaient des lois odieuses en garantie des actions les plus lâches ; et nommaient nécessité l’injustice et la perfidie.\par
J’ai vu la foi des contrats bannie, les plus saintes conventions anéanties, toutes les lois des familles renversées. J’ai vu des débiteurs avares, fiers d’une insolente pauvreté, instruments indignes de la fureur des lois et de la rigueur des temps, feindre un payement au lieu de le faire, et porter le couteau dans le sein de leurs bienfaiteurs.\par
J’en ai vu d’autres, plus indignes encore, acheter presque pour rien, ou plutôt ramasser de terre des feuilles de chêne, pour les mettre à la place de la substance des veuves et des orphelins.\par
J’ai vu naître soudain, dans tous les cœurs, une soif insatiable des richesses. J’ai vu se former en un moment une détestable conjuration de s’enrichir, non par un honnête travail et une généreuse industrie, mais par la ruine du prince, de l’État et des concitoyens.\par
J’ai vu un honnête citoyen, dans ces temps malheureux, ne se coucher qu’en disant : J’ai ruiné une famille aujourd’hui ; j’en ruinerai une autre demain.\par
Je vais, disait un autre, avec un homme noir qui porte une écritoire à la main et un fer pointu à l’oreille, assassiner tous ceux à qui j’ai de l’obligation.\par
Un autre disait : Je vois que j’accommode mes affaires il est vrai que, lorsque j’allai, il y a trois jours, faire un certain payement, je laissai toute une famille en larmes, que je dissipai la dot de deux honnêtes filles, que j’ôtai l’éducation à un petit garçon ; le père en mourra de douleur, la mère périt de tristesse : mais je n’ai fait que ce qui est permis par la loi.\par
Quel plus grand crime que celui que commet un ministre, lorsqu’il corrompt les mœurs de toute une nation, dégrade les âmes les plus généreuses, ternit l’éclat des dignités, obscurcit la vertu même, et confond la plus haute naissance dans le mépris universel ?\par
Que dira la postérité, lorsqu’il lui faudra rougir de la honte de ses pères ? Que dira le peuple naissant, lorsqu’il comparera le fer de ses aïeux avec l’or de ceux à qui il doit immédiatement le jour ? Je ne doute pas que les nobles ne retranchent de leurs quartiers un indigne degré de noblesse qui les déshonore, et ne laissent la génération présente dans l’affreux néant où elle s’est mise.\par

\dateline{De Paris, le 11 de la lune de Rhamazan, 1720.}
\section[{Lettre CXLVII. Le grand eunuque à Usbek. À Paris.}]{Lettre CXLVII. Le grand eunuque à Usbek. À Paris.}\renewcommand{\leftmark}{Lettre CXLVII. Le grand eunuque à Usbek. À Paris.}

\noindent Les choses sont venues à un état qui ne se peut plus soutenir : tes femmes se sont imaginé que ton départ leur laissait une impunité entière ; il se passe ici des choses horribles : je tremble moi-même au cruel récit que je vais te faire.\par
Zélis, allant il y a quelques jours à la mosquée, laissa tomber son voile, et parut presque à visage découvert devant tout le peuple.\par
J’ai trouvé Zachi couchée avec une de ses esclaves ; chose si défendue par les lois du sérail.\par
J’ai surpris, par le plus grand hasard du monde, une lettre que je t’envoie : je n’ai jamais pu découvrir à qui elle était adressée.\par
Hier au soir, un jeune garçon fut trouvé dans le jardin du sérail, et il se sauva par-dessus les murailles.\par
Ajoute à cela ce qui n’est pas parvenu à ma connaissance ; car sûrement tu es trahi. J’attends tes ordres : et, jusqu’à l’heureux moment que je les recevrai, je vais être dans une situation mortelle. Mais, si tu ne mets toutes ces femmes à ma discrétion, je ne te réponds d’aucune d’elles, et j’aurai tous les jours des nouvelles aussi tristes à te mander.\par

\dateline{Du sérail d’Ispahan, le 1er de la lune de Rhégeb, 1717.}
\section[{Lettre CXLVIII. Usbek au premier eunuque. Au sérail d’Ispahan.}]{Lettre CXLVIII. Usbek au premier eunuque. Au sérail d’Ispahan.}\renewcommand{\leftmark}{Lettre CXLVIII. Usbek au premier eunuque. Au sérail d’Ispahan.}

\noindent Recevez par cette lettre un pouvoir sans bornes sur tout le sérail : commandez avec autant d’autorité que moi-même ; que la crainte et la terreur marchent avec vous ; courez d’appartements en appartements porter les punitions et les châtiments ; que tout vive dans la consternation, que tout fonde en larmes devant vous ; interrogez tout le sérail ; commencez par les esclaves ; n’épargnez pas mon amour ; que tout subisse votre tribunal redoutable ; mettez au jour les secrets les plus cachés ; purifiez ce lieu infâme ; et faites-y rentrer la vertu bannie. Car, dès ce moment, je mets sur votre tête les moindres fautes qui se commettront. Je soupçonne Zélis d’être celle à qui la lettre que vous avez surprise s’adressait : examinez cela avec des yeux de lynx.\par

\dateline{De ***, le 11 de la lune de Zilhagé, 1718.}
\section[{Lettre CXLIX. Narsit à Usbek. À Paris.}]{Lettre CXLIX. Narsit à Usbek. À Paris.}\renewcommand{\leftmark}{Lettre CXLIX. Narsit à Usbek. À Paris.}

\noindent Le grand eunuque vient de mourir, magnifique seigneur : comme je suis le plus vieux de tes esclaves, j’ai pris sa place, jusques à ce que tu aies fait connaître sur qui tu veux jeter les yeux.\par
Deux jours après sa mort, on m’apporta une de tes lettres qui lui était adressée : je me suis bien gardé de l’ouvrir ; je l’ai enveloppée avec respect, et l’ai serrée, jusques à ce que tu m’aies fait connaître tes sacrées volontés.\par
Hier, un esclave vint, au milieu de la nuit, me dire qu’il avait trouvé un jeune homme dans le sérail : je me levai, j’examinai la chose, et je trouvai que c’était une vision.\par
Je te baise les pieds, sublime seigneur ; et je te prie de compter sur mon zèle, mon expérience et ma vieillesse.\par

\dateline{Du sérail d’Ispahan, le 5 de la lune de Gemmadi I, 1718.}
\section[{Lettre CL. Usbek à Narsit. Au sérail d’Ispahan.}]{Lettre CL. Usbek à Narsit. Au sérail d’Ispahan.}\renewcommand{\leftmark}{Lettre CL. Usbek à Narsit. Au sérail d’Ispahan.}

\noindent Malheureux que vous êtes, vous avez dans vos mains des lettres qui contiennent des ordres prompts et violents ; le moindre retardement peut me désespérer : et vous demeurez tranquille sous un vain prétexte !\par
Il se passe des choses horribles : j’ai peut-être la moitié de mes esclaves qui méritent la mort. Je vous envoie la lettre que le premier eunuque m’écrivit là-dessus avant de mourir. Si vous aviez ouvert le paquet qui lui est adressé, vous y auriez trouvé des ordres sanglants. Lisez-les donc ces ordres : et vous périrez, si vous ne les exécutez pas.\par

\dateline{De ***, les 25 de la lune de Chalval, 1718.}
\section[{Lettre CLI. Solim à Usbek. À Paris.}]{Lettre CLI. Solim à Usbek. À Paris.}\renewcommand{\leftmark}{Lettre CLI. Solim à Usbek. À Paris.}

\noindent Si je gardais plus longtemps le silence, je serais aussi coupable que tous ces criminels que tu as dans le sérail.\par
J’étais le confident du grand eunuque, le plus fidèle de tes esclaves. Lorsqu’il se vit près de sa fin, il me fit appeler, et me dit ces paroles : Je me meurs ; mais le seul chagrin que j’aie en quittant la vie, c’est que mes derniers regards aient trouvé les femmes de mon maître criminelles. Le ciel puisse le garantir de tous les malheurs que je prévois ! puisse, après ma mort, mon ombre menaçante venir avertir ces perfides de leur devoir, et les intimider encore ! Voilà les clefs de ces redoutables lieux ; va les porter au plus vieux des noirs. Mais si, après ma mort, il manque de vigilance, songe à en avertir ton maître. En achevant ces mots, il expira dans mes bras.\par
Je sais ce qu’il t’écrivit, quelque temps avant sa mort, sur la conduite de tes femmes : il y a dans le sérail une lettre qui aurait porté la terreur avec elle, si elle avait été ouverte ; celle que tu as écrite depuis a été surprise à trois lieues d’ici : je ne sais ce que c’est, tout se tourne malheureusement.\par
Cependant tes femmes ne gardent plus aucune retenue : depuis la mort du grand eunuque, il semble que tout leur soit permis ; la seule Roxane est restée dans le devoir, et conserve de la modestie. On voit les mœurs se corrompre tous les jours. On ne trouve plus sur le visage de tes femmes cette vertu mâle et sévère qui y régnait autrefois : une joie nouvelle, répandue dans ces lieux, est un témoignage infaillible, selon moi, de quelque satisfaction nouvelle ; dans les plus petites choses, je remarque des libertés jusqu’alors inconnues. Il règne, même parmi tes esclaves, une certaine indolence pour leur devoir et pour l’observation des règles, qui me surprend ; ils n’ont plus ce zèle ardent pour ton service, qui semblait animer tout le sérail.\par
Tes femmes ont été huit jours à la campagne, à une de tes maisons les plus abandonnées. On dit que l’esclave qui en a soin a été gagné ; et qu’un jour avant qu’elles arrivassent, il avait fait cacher deux hommes dans un réduit de pierre qui est dans la muraille de la principale chambre, d’où ils sortaient le soir lorsque nous étions retirés. Le vieux eunuque, qui est à présent à notre tête, est un imbécile, à qui l’on fait croire tout ce qu’on veut.\par
Je suis agité d’une colère vengeresse contre tant de perfidies : et si le ciel voulait, pour le bien de ton service, que tu me jugeasses capable de gouverner, je te promets que, si tes femmes n’étaient pas vertueuses, au moins elles seraient fidèles.\par

\dateline{Du sérail d’Ispahan, le 6 de la lune de Rebiab I, 1719.}
\section[{Lettre CLII. Narsit à Usbek. À Paris.}]{Lettre CLII. Narsit à Usbek. À Paris.}\renewcommand{\leftmark}{Lettre CLII. Narsit à Usbek. À Paris.}

\noindent Roxane et Zélis ont souhaité d’aller à la campagne ; je n’ai pas cru devoir le leur refuser. Heureux Usbek ! tu as des femmes fidèles et des esclaves vigilants : je commande en des lieux où la vertu semble s’être choisi un asile. Compte qu’il ne s’y passera rien que tes yeux ne puissent soutenir.\par
Il est arrivé un malheur qui me met en grande peine. Quelques marchands arméniens, nouvellement arrivés à Ispahan, avaient apporté une de tes lettres pour moi ; j’ai envoyé un esclave pour la chercher ; il a été volé à son retour, de manière que la lettre est perdue. Écris-moi donc promptement ; car je m’imagine que, dans ce changement, tu dois avoir des choses de conséquence à me mander.\par

\dateline{Du sérail de Fatmé, le 6 de la lune de Rebiab I, 1719.}
\section[{Lettre CLIII. Usbek à Solim. Au sérail d’Ispahan.}]{Lettre CLIII. Usbek à Solim. Au sérail d’Ispahan.}\renewcommand{\leftmark}{Lettre CLIII. Usbek à Solim. Au sérail d’Ispahan.}

\noindent Je te mets le fer à la main. Je te confie ce que j’ai à présent dans le monde de plus cher, qui est ma vengeance. Entre dans ce nouvel emploi : mais n’y porte ni cœur ni pitié. J’écris à mes femmes de t’obéir aveuglément : dans la confusion de tant de crimes, elles tomberont devant tes regards. Il faut que je te doive mon bonheur et mon repos : rends-moi mon sérail comme je l’ai laissé. Mais commence par l’expier ; extermine les coupables, et fais trembler ceux qui se proposaient de le devenir. Que ne peux-tu pas espérer de ton maître pour des services si signalés ? Il ne tiendra qu’à toi de mettre au-dessus de ta condition même, et de toutes les récompenses que tu as jamais désirées.\par

\dateline{De Paris, le 4 de la lune de Chahban, 1719.}
\section[{Lettre CLIV. Usbek à ses femmes. Au sérail d’Ispahan.}]{Lettre CLIV. Usbek à ses femmes. Au sérail d’Ispahan.}\renewcommand{\leftmark}{Lettre CLIV. Usbek à ses femmes. Au sérail d’Ispahan.}

\noindent Puisse cette lettre être comme la foudre qui tombe au milieu des éclairs et des tempêtes ! Solim est votre premier eunuque, non pas pour vous garder, mais pour vous punir. Que tout le sérail s’abaisse devant lui. Il doit juger vos actions passées : et, pour l’avenir, il vous fera vivre sous un joug si rigoureux, que regretterez votre liberté, si vous ne regrettez pas votre vertu.\par

\dateline{De Paris, le 4 de la lune de Chahban, 1719.}
\section[{Lettre CLV. Usbek à Nessir.}]{Lettre CLV. Usbek à Nessir.}\renewcommand{\leftmark}{Lettre CLV. Usbek à Nessir.}

\noindent Heureux celui qui, connaissant tout le prix d’une vie douce et tranquille, repose son cœur au milieu de sa famille, et ne connaît d’autre terre que celle qui lui a donné le jour.\par
Je vis dans un climat barbare, présent à tout ce qui m’intéresse. Une tristesse sombre me saisit ; je tombe dans un accablement affreux : il me semble que je m’anéantis ; et je ne me retrouve moi même que lorsqu’une sombre jalousie vient s’allumer, et enfanter dans mon âme la crainte, les soupçons, la haine et les regrets.\par
Tu me connais, Nessir ; tu as toujours vu dans mon cœur comme dans le tien : je te ferais pitié, si tu savais mon état déplorable. J’attends quelquefois six mois entiers des nouvelles du sérail ; je compte tous les instants qui s’écoulent ; mon impatience me les allonge toujours ; me les allonge toujours ; et lorsque celui qui a été tant attendu est près d’arriver, il se fait dans mon cœur une révolution soudaine ; ma main tremble d’ouvrir un lettre fatale ; cette inquiétude qui me désespérait, je la trouve l’état le plus heureux où je puisse être, et je crains d’en sortir par un coup plus cruel que moi que mille morts.\par
Mais, quelque raison que j’aie eue de sortir de ma patrie, quoique je doive ma vie à ma retraite, je ne puis plus, Nessir, rester dans cet affreux exil et ne mourrais-je pas tout de même en proie à mes chagrins ? J’ai pressé mille fois Rica de quitter cette terre étrangère ; mais il s’oppose à toutes mes résolutions ; il m’attache ici par mille prétextes : il semble qu’il ait oublié sa patrie ; ou plutôt il semble qu’il m’ait oublié moi-même, tant il est insensible à mes déplaisirs.\par
Malheureux que je suis ! je souhaite de revoir ma patrie, peut-être pour devenir plus malheureux encore ! Eh ! qu’y ferai-je ? Je vais rapporter ma tête à mes ennemis. Ce n’est pas tout : j’entrerai dans le sérail ; il faut que j’y demande compte du temps funeste de mon absence : et si j’y trouve des coupables, que deviendrai-je ? Et si la seule idée m’accable de si loin, que sera-ce, lorsque ma présence la rendra plus vive ? que sera-ce, s’il faut que je voie, s’il faut que j’entende ce que je n’ose imaginer sans frémir ? que sera-ce enfin, s’il faut que des châtiments que je prononcerai moi-même soient des marques éternelles de ma confusion et de mon désespoir ?\par
J’irai m’enfermer dans des murs plus terribles pour moi que pour les femmes qui y sont gardées ; j’y porterai tous mes soupçons ; leurs empressements ne m’en déroberont rien ; dans mon lit, dans leurs bras, je ne jouirai que de mes inquiétudes ; dans un temps si peu propre aux réflexions, ma jalousie trouvera à en faire. Rebut indigne de la nature humaine, esclaves vils dont le cœur a été fermé pour jamais à tous les sentiments de l’amour, vous ne gémiriez plus sur votre condition, si vous connaissiez le malheur de la mienne.\par

\dateline{De Paris, le 4 de la lune de Chahban, 1719.}
\section[{Lettre CLVI. Roxane à Usbek. À Paris.}]{Lettre CLVI. Roxane à Usbek. À Paris.}\renewcommand{\leftmark}{Lettre CLVI. Roxane à Usbek. À Paris.}

\noindent L’horreur, la nuit et l’épouvante règnent dans le sérail ; un deuil affreux l’environne : un tigre y exerce à chaque instant toute sa rage ; il a mis dans les supplices deux eunuques blancs, qui n’ont avoué que leur innocence ; il a vendu une partie de nos esclaves, et nous a obligées de changer entre nous celles qui nous restaient. Zachi et Zélis ont reçu dans leur chambre, dans l’obscurité de la nuit un traitement indigne ; le sacrilège n’a pas craint de porter sur elles ses viles mains. Il nous tient enfermées chacune dans notre appartement ; et, quoique nous y soyons seules, il nous y fait vivre sous le voile : il ne nous est plus permis de nous parler ; ce serait un crime de nous écrire : nous n’avons plus rien de libre que les pleurs.\par
Une troupe de nouveaux eunuques est entrée dans le sérail, où ils nous assiègent nuit et jour : notre sommeil est sans cesse interrompu par leurs méfiances feintes ou véritables. Ce qui me console, c’est que tout ceci ne durera pas longtemps, et que ces peines finiront avec ma vie : elle ne sera pas longue, cruel Usbek ! je ne te donnerai pas le temps de faire cesser tous ces outrages.\par

\dateline{Du sérail d’Ispahan, le 2 de la lune de Maharram, 1720.}
\section[{Lettre CLVII. Zachi à Usbek. À Paris.}]{Lettre CLVII. Zachi à Usbek. À Paris.}\renewcommand{\leftmark}{Lettre CLVII. Zachi à Usbek. À Paris.}

\noindent Ô ciel ! Un barbare m’a outragée jusque dans la manière de me punir ! Il m’a infligé ce châtiment qui commence par alarmer la pudeur ; ce châtiment qui met dans l’humiliation extrême ; ce châtiment qui ramène, pour ainsi dire, à l’enfance.\par
Mon âme, d’abord anéantie sous la honte, reprenait le sentiment d’elle-même, et commençait à s’indigner, lorsque mes cris firent retentir les voûtes de mes appartements. On m’entendit demander grâce au plus vil de tous les humains, et tenter sa pitié, à mesure qu’il était inexorable.\par
Depuis ce temps, son âme insolente et servile s’est élevée sur la mienne. Sa présence, ses regards, ses paroles, tous les malheurs viennent m’accabler. Quand je suis seule, j’ai du moins la consolation de verser des larmes ; mais lorsqu’il s’offre à ma vue, la fureur me saisit ; je la trouve impuissante : et je tombe dans le désespoir.\par
Le tigre ose me dire que tu es l’auteur de toutes ces barbaries. Il voudrait m’ôter mon amour, et profaner jusques aux sentiments de mon cœur. Quand il me prononce le nom de celui que j’aime, je ne sais plus me plaindre : je ne puis plus que mourir.\par
J’ai soutenu ton absence, et j’ai conservé mon amour, par la force de mon amour. Les nuits, les jours, les moments tout a été pour toi. J’étais superbe de mon amour même ; et le tien me faisait respecter ici. Mais à présent… Non, je ne puis plus soutenir l’humiliation où je suis descendue. Si je suis innocente, reviens pour m’aimer ; reviens, si je suis coupable, pour que j’expire à tes pieds.\par

\dateline{Du sérail d’Ispahan, le 2 de la lune de Maharram, 1720.}
\section[{Lettre CLVIII. Zélis à Usbek. À Paris.}]{Lettre CLVIII. Zélis à Usbek. À Paris.}\renewcommand{\leftmark}{Lettre CLVIII. Zélis à Usbek. À Paris.}

\noindent À mille lieues de moi, vous me jugez coupable : à mille lieues de moi, vous me punissez.\par
Qu’un eunuque barbare porte sur moi ses viles mains, il agit par votre ordre : c’est le tyran qui m’outrage, et non pas celui qui exerce la tyrannie.\par
Vous pouvez, à votre fantaisie, redoubler vos mauvais traitements. Mon cœur est tranquille, depuis qu’il ne peut plus vous aimer. Votre âme se dégrade, et vous devenez cruel. Soyez sûr que vous n’êtes point heureux. Adieu.\par

\dateline{Du sérail d’Ispahan, le 2 de la lune de Maharram, 1720.}
\section[{Lettre CLIX. Solim à Usbek. À Paris.}]{Lettre CLIX. Solim à Usbek. À Paris.}\renewcommand{\leftmark}{Lettre CLIX. Solim à Usbek. À Paris.}

\noindent Je me plains, magnifique seigneur, et je te plains : jamais serviteur fidèle n’est descendu dans l’affreux désespoir où je suis. Voici tes malheurs et les miens ; je ne t’en écris qu’en tremblant.\par
Je jure, par tous les prophètes du ciel, que, depuis que tu m’as confié tes femmes, j’ai veillé nuit et jour sur elles ; que je n’ai jamais suspendu un moment le cours de mes inquiétudes. J’ai commencé mon ministère par les châtiments ; et je les ai suspendus, sans sortir de mon austérité naturelle.\par
Mais que dis-je ? pourquoi te vanter ici une fidélité qui t’a été inutile ? Oublie tous mes services passés ; regarde-moi comme un traître ; et punis-moi de tous les crimes que je n’ai pu empêcher.\par
Roxane, la superbe Roxane, ô ciel ! à qui se fier désormais ? Tu soupçonnais Zachi, et tu avais pour Roxane une sécurité entière ; mais sa vertu farouche était une cruelle imposture ; c’était le voile de sa perfidie. Je l’ai surprise dans les bras d’un jeune homme, qui, dès qu’il s’est vu découvert, est venu sur moi ; il m’a donné deux coups de poignard ; les eunuques, accourus au bruit, l’ont entouré : il s’est défendu longtemps, en a blessé plusieurs ; il voulait même rentrer dans la chambre, pour mourir, disait-il, aux yeux de Roxane. Mais enfin il a cédé au nombre, et il est tombé à nos pieds.\par
Je ne sais si j’attendrai, sublime seigneur, tes ordres sévères : tu as mis ta vengeance en mes mains ; je ne dois pas la faire languir.\par

\dateline{Du sérail d’Ispahan, le 8 de la lune de Rebiab I, 1720.}
\section[{Lettre CLX. Solim à Usbek. À Paris.}]{Lettre CLX. Solim à Usbek. À Paris.}\renewcommand{\leftmark}{Lettre CLX. Solim à Usbek. À Paris.}

\noindent J’ai pris mon parti : tes malheurs vont disparaître ; je vais punir.\par
Je sens déjà une joie secrète ; mon âme et la tienne vont s’apaiser : nous allons exterminer le crime, et l’innocence va pâlir.\par
Ô vous, qui semblez n’être faites que pour ignorer tous vos sens et être indignées de vos désirs mêmes ; éternelles victimes de la honte et de la pudeur, que ne puis-je vous faire à grands flots dans ce sérail malheureux, pour vous voir étonnées de tout le sang que j’y vais répandre !\par

\dateline{Du sérail d’Ispahan, le 8 de la lune de Rebiab I, 1720.}
\section[{Lettre CLXI. Roxane à Usbek. À Paris.}]{Lettre CLXI. Roxane à Usbek. À Paris.}\renewcommand{\leftmark}{Lettre CLXI. Roxane à Usbek. À Paris.}

\noindent Oui, je t’ai trompé ; j’ai séduit tes eunuques ; je me suis jouée de ta jalousie ; et j’ai su de ton affreux sérail faire un lieu de délices et de plaisirs.\par
Je vais mourir ; le poison va couler dans mes veines : car que ferais-je ici, puisque le seul homme qui me retenait à la vie n’est plus ? Je meurs ; mais mon ombre s’envole bien accompagnée : je viens d’envoyer devant moi ces gardiens sacrilèges, qui ont répandu le plus beau sang du monde.\par
Comment as-tu pensé que je fusse assez crédule, pour m’imaginer que je ne fusse dans le monde que pour adorer tes caprices ? que, pendant que tu te permets tout, tu eusses le droit d’affliger tous mes désirs ? Non : j’ai pu vivre dans la servitude ; mais j’ai toujours été libre : j’ai réformé tes lois sur celles de la nature ; et mon esprit s’est toujours tenu dans l’indépendance.\par
Tu devrais me rendre grâces encore du sacrifice que je t’ai fait ; de ce que je me suis abaissée jusqu’à te paraître fidèle ; de ce que j’ai lâchement gardé dans mon cœur ce que j’aurais dû faire paraître à toute la terre ; enfin de ce que j’ai profané la vertu en souffrant qu’on appelât de ce nom ma soumission à tes fantaisies.\par
Tu étais étonné de ne point trouver en moi les transports de l’amour : si tu m’avais bien connue, tu y aurais trouvé toute la violence de la haine.\par
Mais tu as eu longtemps l’avantage de croire qu’un cœur comme le mien t’était soumis. Nous étions tous deux heureux ; tu me croyais trompée, et je te trompais.\par
Ce langage, sans doute, te paraît nouveau. Serait-il possible qu’après t’avoir accablé de douleurs, je te forçasse encore d’admirer mon courage ? Mais c’en est fait, le poison me consume, ma force m’abandonne ; la plume me tombe des mains ; je sens affaiblir jusqu’à ma haine ; je me meurs.\par

\dateline{Du sérail d’Ispahan, le 8 de la lune de Rebiab I, 1720.}
 


% at least one empty page at end (for booklet couv)
\ifbooklet
  \pagestyle{empty}
  \clearpage
  % 2 empty pages maybe needed for 4e cover
  \ifnum\modulo{\value{page}}{4}=0 \hbox{}\newpage\hbox{}\newpage\fi
  \ifnum\modulo{\value{page}}{4}=1 \hbox{}\newpage\hbox{}\newpage\fi


  \hbox{}\newpage
  \ifodd\value{page}\hbox{}\newpage\fi
  {\centering\color{rubric}\bfseries\noindent\large
    Hurlus ? Qu’est-ce.\par
    \bigskip
  }
  \noindent Des bouquinistes électroniques, pour du texte libre à participation libre,
  téléchargeable gratuitement sur \href{https://hurlus.fr}{\dotuline{hurlus.fr}}.\par
  \bigskip
  \noindent Cette brochure a été produite par des éditeurs bénévoles.
  Elle n’est pas faîte pour être possédée, mais pour être lue, et puis donnée.
  Que circule le texte !
  En page de garde, on peut ajouter une date, un lieu, un nom ; pour suivre le voyage des idées.
  \par

  Ce texte a été choisi parce qu’une personne l’a aimé,
  ou haï, elle a en tous cas pensé qu’il partipait à la formation de notre présent ;
  sans le souci de plaire, vendre, ou militer pour une cause.
  \par

  L’édition électronique est soigneuse, tant sur la technique
  que sur l’établissement du texte ; mais sans aucune prétention scolaire, au contraire.
  Le but est de s’adresser à tous, sans distinction de science ou de diplôme.
  Au plus direct ! (possible)
  \par

  Cet exemplaire en papier a été tiré sur une imprimante personnelle
   ou une photocopieuse. Tout le monde peut le faire.
  Il suffit de
  télécharger un fichier sur \href{https://hurlus.fr}{\dotuline{hurlus.fr}},
  d’imprimer, et agrafer ; puis de lire et donner.\par

  \bigskip

  \noindent PS : Les hurlus furent aussi des rebelles protestants qui cassaient les statues dans les églises catholiques. En 1566 démarra la révolte des gueux dans le pays de Lille. L’insurrection enflamma la région jusqu’à Anvers où les gueux de mer bloquèrent les bateaux espagnols.
  Ce fut une rare guerre de libération dont naquit un pays toujours libre : les Pays-Bas.
  En plat pays francophone, par contre, restèrent des bandes de huguenots, les hurlus, progressivement réprimés par la très catholique Espagne.
  Cette mémoire d’une défaite est éteinte, rallumons-la. Sortons les livres du culte universitaire, cherchons les idoles de l’époque, pour les briser.
\fi

\ifdev % autotext in dev mode
\fontname\font — \textsc{Les règles du jeu}\par
(\hyperref[utopie]{\underline{Lien}})\par
\noindent \initialiv{A}{lors là}\blindtext\par
\noindent \initialiv{À}{ la bonheur des dames}\blindtext\par
\noindent \initialiv{É}{tonnez-le}\blindtext\par
\noindent \initialiv{Q}{ualitativement}\blindtext\par
\noindent \initialiv{V}{aloriser}\blindtext\par
\Blindtext
\phantomsection
\label{utopie}
\Blinddocument
\fi
\end{document}
