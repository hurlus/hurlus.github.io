%%%%%%%%%%%%%%%%%%%%%%%%%%%%%%%%%
% LaTeX model https://hurlus.fr %
%%%%%%%%%%%%%%%%%%%%%%%%%%%%%%%%%

% Needed before document class
\RequirePackage{pdftexcmds} % needed for tests expressions
\RequirePackage{fix-cm} % correct units

% Define mode
\def\mode{a4}

\newif\ifaiv % a4
\newif\ifav % a5
\newif\ifbooklet % booklet
\newif\ifcover % cover for booklet

\ifnum \strcmp{\mode}{cover}=0
  \covertrue
\else\ifnum \strcmp{\mode}{booklet}=0
  \booklettrue
\else\ifnum \strcmp{\mode}{a5}=0
  \avtrue
\else
  \aivtrue
\fi\fi\fi

\ifbooklet % do not enclose with {}
  \documentclass[french,twoside]{book} % ,notitlepage
  \usepackage[%
    papersize={105mm, 297mm},
    inner=12mm,
    outer=12mm,
    top=20mm,
    bottom=15mm,
    marginparsep=0pt,
  ]{geometry}
  \usepackage[fontsize=9.5pt]{scrextend} % for Roboto
\else\ifav
  \documentclass[french,twoside]{book} % ,notitlepage
  \usepackage[%
    a5paper,
    inner=25mm,
    outer=15mm,
    top=15mm,
    bottom=15mm,
    marginparsep=0pt,
  ]{geometry}
  \usepackage[fontsize=12pt]{scrextend}
\else% A4 2 cols
  \documentclass[twocolumn]{report}
  \usepackage[%
    a4paper,
    inner=15mm,
    outer=10mm,
    top=25mm,
    bottom=18mm,
    marginparsep=0pt,
  ]{geometry}
  \setlength{\columnsep}{20mm}
  \usepackage[fontsize=9.5pt]{scrextend}
\fi\fi

%%%%%%%%%%%%%%
% Alignments %
%%%%%%%%%%%%%%
% before teinte macros

\setlength{\arrayrulewidth}{0.2pt}
\setlength{\columnseprule}{\arrayrulewidth} % twocol
\setlength{\parskip}{0pt} % classical para with no margin
\setlength{\parindent}{1.5em}

%%%%%%%%%%
% Colors %
%%%%%%%%%%
% before Teinte macros

\usepackage[dvipsnames]{xcolor}
\definecolor{rubric}{HTML}{902c20} % the tonic
\def\columnseprulecolor{\color{rubric}}
\colorlet{borderline}{rubric!30!} % definecolor need exact code
\definecolor{shadecolor}{gray}{0.95}
\definecolor{bghi}{gray}{0.5}

%%%%%%%%%%%%%%%%%
% Teinte macros %
%%%%%%%%%%%%%%%%%
%%%%%%%%%%%%%%%%%%%%%%%%%%%%%%%%%%%%%%%%%%%%%%%%%%%
% <TEI> generic (LaTeX names generated by Teinte) %
%%%%%%%%%%%%%%%%%%%%%%%%%%%%%%%%%%%%%%%%%%%%%%%%%%%
% This template is inserted in a specific design
% It is XeLaTeX and otf fonts

\makeatletter % <@@@


\usepackage{blindtext} % generate text for testing
\usepackage{contour} % rounding words
\usepackage[nodayofweek]{datetime}
\usepackage{DejaVuSans} % font for symbols
\usepackage{enumitem} % <list>
\usepackage{etoolbox} % patch commands
\usepackage{fancyvrb}
\usepackage{fancyhdr}
\usepackage{fontspec} % XeLaTeX mandatory for fonts
\usepackage{footnote} % used to capture notes in minipage (ex: quote)
\usepackage{framed} % bordering correct with footnote hack
\usepackage{graphicx}
\usepackage{lettrine} % drop caps
\usepackage{lipsum} % generate text for testing
\usepackage[framemethod=tikz,]{mdframed} % maybe used for frame with footnotes inside
\usepackage{pdftexcmds} % needed for tests expressions
\usepackage{polyglossia} % non-break space french punct, bug Warning: "Failed to patch part"
\usepackage[%
  indentfirst=false,
  vskip=1em,
  noorphanfirst=true,
  noorphanafter=true,
  leftmargin=\parindent,
  rightmargin=0pt,
]{quoting}
\usepackage{ragged2e}
\usepackage{setspace}
\usepackage{tabularx} % <table>
\usepackage[explicit]{titlesec} % wear titles, !NO implicit
\usepackage{tikz} % ornaments
\usepackage{tocloft} % styling tocs
\usepackage[fit]{truncate} % used im runing titles
\usepackage{unicode-math}
\usepackage[normalem]{ulem} % breakable \uline, normalem is absolutely necessary to keep \emph
\usepackage{verse} % <l>
\usepackage{xcolor} % named colors
\usepackage{xparse} % @ifundefined
\XeTeXdefaultencoding "iso-8859-1" % bad encoding of xstring
\usepackage{xstring} % string tests
\XeTeXdefaultencoding "utf-8"
\PassOptionsToPackage{hyphens}{url} % before hyperref, which load url package
\usepackage{hyperref} % supposed to be the last one, :o) except for the ones to follow
\urlstyle{same} % after hyperref

% TOTEST
% \usepackage{hypcap} % links in caption ?
% \usepackage{marginnote}
% TESTED
% \usepackage{background} % doesn’t work with xetek
% \usepackage{bookmark} % prefers the hyperref hack \phantomsection
% \usepackage[color, leftbars]{changebar} % 2 cols doc, impossible to keep bar left
% \usepackage[utf8x]{inputenc} % inputenc package ignored with utf8 based engines
% \usepackage[sfdefault,medium]{inter} % no small caps
% \usepackage{firamath} % choose firasans instead, firamath unavailable in Ubuntu 21-04
% \usepackage{flushend} % bad for last notes, supposed flush end of columns
% \usepackage[stable]{footmisc} % BAD for complex notes https://texfaq.org/FAQ-ftnsect
% \usepackage{helvet} % not for XeLaTeX
% \usepackage{multicol} % not compatible with too much packages (longtable, framed, memoir…)
% \usepackage[default,oldstyle,scale=0.95]{opensans} % no small caps
% \usepackage{sectsty} % \chapterfont OBSOLETE
% \usepackage{soul} % \ul for underline, OBSOLETE with XeTeX
% \usepackage[breakable]{tcolorbox} % text styling gone, footnote hack not kept with breakable



% Metadata inserted by a program, from the TEI source, for title page and runing heads
\title{\textbf{ Discours de la Méthode }\\ \medskip
\textit{ Pour bien conduire sa raison, \& chercher la vérité dans les sciences }}
\date{1637}
\author{Descartes}
\def\elbibl{Descartes. 1637. \emph{Discours de la Méthode}}
\def\elabstract{%
 
\labelblock{Les français sont-ils cartésiens ?}

 \noindent 1637, bientôt 4 siècles que le \emph{Discours de la Méthode} est écrit, et que beaucoup de français s’en réclament. Ils se flattent d’être \emph{cartésiens}, d’être rationnels et de penser par eux-mêmes, \emph{« je pense donc je suis » (\hyperref[IV1]{\dotuline{IV.1}})}, sans toujours avoir lu le texte.\par
 Descartes a cherché sa certitude, et à cette fin, il a d’abord douté de tout, faisant le tri de ses idées pour n’en retenir que les plus sûres. Persistance étrange, les français composent actuellement la société qui doute le plus des savoirs autorisés, la plus défiante à l’égard de toutes les autorités (relativement aux autres sociétés développées). Ils ne font pas confiance aux élites censées savoir ce qu’est l’économie, la politique, ou un vaccin ; ils veulent se construire leur avis par eux-même.\par
 \bigbreak
 \noindent Ce discours a la réputation d’être philosophique, il ressemble plutôt à un programme de recherche pour demander des financements.\par
 (\hyperref[I]{\dotuline{I}}) Le chercheur raconte d’abord sa vie pour se présenter, sans façon, à la Montaigne ;\par
 (\hyperref[II]{\dotuline{II}}) il présente ensuite une Méthode originale qui s’annonce prometteuse ;\par
 (\hyperref[III]{\dotuline{III}}) de cette Méthode il commence par tirer une Morale, afin de fonder l’éthique du chercheur qui n’existait pas encore ;\par
 (\hyperref[IV]{\dotuline{IV}}) il en tire aussi une Métaphysique, nécessaire à l’époque pour que les résultats de la science ne fâchent l’Église de Rome ;\par
 (\hyperref[IV]{\dotuline{V}}) il présente ensuite quelques résultats scientifiques prometteurs de Physique et de Médecine, toujours grâce à sa Méthode ;\par
 (\hyperref[IV]{\dotuline{VI}}) et enfin, il demande de l’argent pour continuer ses expériences.\par
 Le \emph{Discours} introduisait aussi 3 essais scientifiques révolutionnaires pour l’époque (la Dioptrique, les Météores et la Géométrie) ; et par ailleurs, c’est le premier livre européen de sciences qui n’est pas écrit en latin.\par
 \bigbreak
 \noindent La science de Descartes est parfois étonnante, par exemple il suppose que le sang circule dans le corps comme dans un chauffage central, par l’effet de la chaleur et de la pression ; ou il développe la théorie célèbre des animaux-machines ; mais ses hypothèses sont toujours plausibles et compatibles avec ses expériences. Il a tenté un système global de la connaissance humaine, où la science la plus moderne de l’époque resterait en accord avec la tradition de la Religion.\par
 
\begin{quoteblock}
\noindent la lecture de tous les bons livres est comme une conversation avec les plus honnêtes gens des siècles passés, qui en ont été les auteurs, \& même une conversation étudiée, en laquelle ils ne nous découvrent que les meilleures de leurs pensées (\hyperref[I7]{\dotuline{I.7}})\end{quoteblock}

 \noindent Ce détour par de la science fausse peut sembler une perte de temps, à la réserve que l’on apprend beaucoup des erreurs des autres, des préjugés qui les ont retenu, ou simplement, des conséquences d’une idée forte qui découvre des faits, mais qui mécaniquement en cache d’autres. Ce \emph{Discours} permet surtout de connaître un grand savant, une personne attachante, et une sagesse.\par
 \bigbreak
 
\begin{quoteblock}
\noindent Il y a moins de perfection dans les ouvrages composés de plusieurs pièces, \& faits de la main de divers maîtres, qu’en ceux auxquels un seul a travaillé. (\hyperref[VI2]{\dotuline{VI.2}})\end{quoteblock}

 \noindent La pensée de Descartes est révolutionnaire, mais sa révolution s’arrête avant son corps, c’est un conservateur qui ne veut surtout pas modifier l’ordre des choses. Il est limité par ses axiomes : le besoin d’une certitude absolue qui ne s’accommode pas de la négociation avec les autres, alors pour tout ce qui concerne la société, il suit la coutume. Il refuse la \emph{réforme}, qui dans le langue d’alors est une terme aussi fort que notre \emph{révolution}, car il ne s’agit pas seulement de renverser le monde cul-par-dessus-tête, mais de le reconstruire, de le re-former, comme une maison à rebâtir selon un meilleur plan, avec les matériaux de l’ancienne.\par
 
\begin{quoteblock}
\noindent « pour ce qui touche les mœurs, chacun abonde si fort en son sens, qu’il se pourrait trouver autant de réformateurs que de têtes, s’il était permis à d’autres qu’à ceux que Dieu a établis pour souverains sur ses peuples » (\hyperref[II2]{\dotuline{II.2}})\end{quoteblock}

 \noindent Descartes n’imagine pas qu’une réforme puisse être démocratique ; il préfère une monarchie justifiée par Dieu, et tempérée par la coutume. Bien que gentilhomme à l’époque, il formule très tôt l’idéal politique du rentier : libre de sa vie et de sa pensée, n’exigeant pas de participer au gouvernement, à la condition que l’état le protège des gens les plus dérangeants.\par
 
\astermono

 \noindent Depuis 1840 et le premier programme des classes de philosophie des lycées français par Victor Cousin, la conscience et le \emph{cogito} « je pense (donc je suis) », sont au fondement de la formation des citoyens français. Descartes a inspiré la grande qualité de la science française d’alors, fondé sur le matérialisme radical de l’animal-machine ; mais il a aussi conforté une classe dominante dans sa supériorité intellectuelle, fondée en raison jusqu’à un principe métaphysique ultime : un dieu rationnel validant l’existence de l’âme, du monde, et de leur accord dans les vérités de la science ; afin de \emph{nous rendre comme maîtres \& possesseurs de la Nature (\hyperref[VI2]{\dotuline{VI.2}})}.\par
 Avec la promotion de cette philosophie officielle, l’état français pensait conserver une religion rationnelle, qui ne nuise pas à la science et donc à l’industrie, sans perdre l’immortalité de l’âme et donc l’obéissance du peuple. Par ailleurs, ce système classique n’est pas historique, ce qui évite à l’état de rappeler son origine révolutionnaire, qui porte nécessairement la possibilité de son renversement, c’est-à-dire, que la Révolution puisse revenir. Enfin, le cartésianisme d’état a fondé la métaphysique de l’individualisme absolu, libre de ses opinions en droit, ce qui permet de l’attacher à ses ignorances en fait ; par exemple si le peuple a voté pour Napoléon III en 1870, alors l’Empereur est voulu par Dieu qui s’exprime à travers l’âme de tous les citoyens qui ont voté librement, mais pas éclairés par un débat collectif. \par
 \bigbreak
 \noindent Toutefois, l’homme Descartes porte aussi une sagesse profondément démocratique et universelle. Il accorde à tous le bon sens qui fait l’humanité, sans distinction sociale ou nationale. Il est fidèle en cela à la promesse catholique d’une âme égale pour tous devant Dieu ; contrairement à la prédestination protestante de l’époque, où plusieurs pouvaient s’estimer \emph{élus} (par Dieu).\par
 
\begin{quoteblock}
\noindent tous ceux qui ont des sentiments fort contraires aux nôtres, ne sont pas pour cela barbares ni sauvages, mais que plusieurs usent autant ou plus que nous de raison […] il me semblait que je pourrais rencontrer beaucoup plus de vérité, dans les raisonnements que chacun fait touchant les affaires qui lui importent, \& dont l’événement le doit punir bientôt après, s’il a mal jugé ; que dans ceux que fait un homme de lettres dans son cabinet, touchant des spéculations qui ne produisent aucun effet (\hyperref[II4]{\dotuline{II.4}})\end{quoteblock}

 \noindent Descartes est une introduction, il libère à la condition de ne pas s’y arrêter. Le prêtre Malebranche peut en tirer un piétisme radical et rationnel (l’âme pense et donc n’existe qu’en Dieu) ; le juif excommunié Spinoza souhaitera plutôt une réalisation politique de la liberté de penser, aboutissant nécessairement à une démocratie. Hegel considère en tous cas que \emph{« Descartes est de fait le véritable initiateur de la philosophie moderne »}, le \emph{Discours de la Méthode} est comme une graine en concentré de notre modernité.\par
 
\astermono

 \noindent Le \emph{Discours de la Méthode} est donné ici dans une édition nouvelle, revue sur l’imprimé original de 1637\footnote{\href{https://gallica.bnf.fr/ark:/12148/btv1b86069594}{\dotuline{Gallica [https://gallica.bnf.fr/ark:/12148/btv1b86069594]}}}. L’orthographe est modernisée, inutile de surprendre l’œil et l’oreille par des conventions graphiques comme \emph{paroistre} « paraître » ou \emph{connoiſſance} « connaissance », qui ne signifient pas plus qu’une différence d’accent.\par
 
\begin{quoteblock}
\noindent Ceux qui ont le raisonnement le plus fort, \& qui digèrent le mieux leurs pensées, afin de les rendre claires \& intelligibles, peuvent toujours le mieux persuader ce qu’ils proposent, encore qu’ils ne parlassent que bas Breton (\hyperref[I9]{\dotuline{I.9}})\end{quoteblock}

 \noindent Par contre, on a restauré les majuscules et la ponctuation originale, qui sont très significatives. Les éditeurs depuis le \textsc{XIX}\textsuperscript{e} siècle pensent corriger le texte en déplaçant les virgules, oubliant que Descartes est mathématicien, et que sa contribution majeure concerne justement la notation. Il a un usage logique de la ponctuation, qui sépare et hiérarchise les propositions. L’usage actuel met par exemple des virgules après des \emph{quoique}, ou des \emph{par exemple}, imposant une respiration de l’incise. Dans une phrase aussi longue que celles de l’époque, ces virgules modernes brouillent l’arbre des propositions en imposant un ton qui n’est pas du sens ; mais une certaine manière de dire qui ne convient ni à Descartes, ni à aujourd’hui. La chaîne de ses raisons cherche justement à éviter ces détours.\par
 L’usage de certains signes peut surprendre. Le point (.) reste le séparateur des phrases qui peuvent être parfois très longues. La virgule (,) est le séparateur de plus bas niveaux, entre les termes d’une énumération, et surtout les propositions. Le point-virgule (;) est nécessaire comme articulation supérieure à la virgule, il arrive qu’il soit suivi d’une majuscule, qui est un marqueur fort de division dans le propos. Les deux-points (:) sont utilisés comme une division plus forte que le point-virgule, mais inférieure au point. Cet usage peut surprendre au départ mais devient vite naturel.\par
 La typographie originale du \emph{Discours} est pauvre, c’est un flot continu, juste divisé par des alinéas. Les parties ne sont même pas marquées d’une ligne vide, mais indiquées par une note marginale. Ces paragraphes ont été scrupuleusement conservés, et même numérotés pour faciliter la référence. Ils sont longs, parfois plus de 100 lignes. Notre œil et notre mémoire ne sont plus exercés à articuler de si longues périodes.\par
 Comme les majuscules conservées après les ponctuations faibles pouvaient gêner l’œil actuel, elles ont été mises à la ligne, comme des versets. Il en résulte une division du propos étonnamment rigoureuse.\par
 Les passages en gras ne sont pas marqués par l’auteur, ils servent de balises pour se retrouver plus rapidement et entretenir l’attention. Il y a quelques notes de l’éditeur pour assister les lecteurs qui ne seraient pas familiers de certaines abstractions, même si Descartes les évite le plus possible.\par
 En conséquence, même ceux qui connaissent ce texte pourrait trouvé un plaisir nouveau à cette édition libre (quoique scrupuleuse).\par
 \vfill\null
 
}
\def\elsource{ \href{https://gallica.bnf.fr/ark:/12148/btv1b86069594/f11.vertical#}{\dotuline{Gallica}}\footnote{\href{https://gallica.bnf.fr/ark:/12148/btv1b86069594/f11.vertical#}{\url{https://gallica.bnf.fr/ark:/12148/btv1b86069594/f11.vertical#}}} }

% Default metas
\newcommand{\colorprovide}[2]{\@ifundefinedcolor{#1}{\colorlet{#1}{#2}}{}}
\colorprovide{rubric}{red}
\colorprovide{silver}{Gray}
\@ifundefined{syms}{\newfontfamily\syms{DejaVu Sans}}{}
\newif\ifdev
\@ifundefined{elbibl}{% No meta defined, maybe dev mode
  \newcommand{\elbibl}{Titre court ?}
  \newcommand{\elbook}{Titre du livre source ?}
  \newcommand{\elabstract}{Résumé\par}
  \newcommand{\elurl}{http://oeuvres.github.io/elbook/2}
  \author{Éric Lœchien}
  \title{Un titre de test assez long pour vérifier le comportement d’une maquette}
  \date{1566}
  \devtrue
}{}
\let\eltitle\@title
\let\elauthor\@author
\let\eldate\@date


\defaultfontfeatures{
  % Mapping=tex-text, % no effect seen
  Scale=MatchLowercase,
  Ligatures={TeX,Common},
}

\@ifundefined{\columnseprulecolor}{%
    \patchcmd\@outputdblcol{% find
      \normalcolor\vrule
    }{% and replace by
      \columnseprulecolor\vrule
    }{% success
    }{% failure
      \@latex@warning{Patching \string\@outputdblcol\space failed}%
    }
}{}

\hypersetup{
  % pdftex, % no effect
  pdftitle={\elbibl},
  % pdfauthor={Your name here},
  % pdfsubject={Your subject here},
  % pdfkeywords={keyword1, keyword2},
  bookmarksnumbered=true,
  bookmarksopen=true,
  bookmarksopenlevel=1,
  pdfstartview=Fit,
  breaklinks=true, % avoid long links
  pdfpagemode=UseOutlines,    % pdf toc
  hyperfootnotes=true,
  colorlinks=false,
  pdfborder=0 0 0,
  % pdfpagelayout=TwoPageRight,
  % linktocpage=true, % NO, toc, link only on page no
}


% generic typo commands
\newcommand{\astermono}{\medskip\centerline{\color{rubric}\large\selectfont{\syms ✻}}\medskip\par}%
\newcommand{\astertri}{\medskip\par\centerline{\color{rubric}\large\selectfont{\syms ✻\,✻\,✻}}\medskip\par}%
\newcommand{\asterism}{\bigskip\par\noindent\parbox{\linewidth}{\centering\color{rubric}\large{\syms ✻}\\{\syms ✻}\hskip 0.75em{\syms ✻}}\bigskip\par}%

% lists
\newlength{\listmod}
\setlength{\listmod}{\parindent}
\setlist{
  itemindent=!,
  listparindent=\listmod,
  labelsep=0.2\listmod,
  parsep=0pt,
  % topsep=0.2em, % default topsep is best
}
\setlist[itemize]{
  label=—,
  leftmargin=0pt,
  labelindent=1.2em,
  labelwidth=0pt,
}
\setlist[enumerate]{
  label={\bf\color{rubric}\arabic*.},
  labelindent=0.8\listmod,
  leftmargin=\listmod,
  labelwidth=0pt,
}
\newlist{listalpha}{enumerate}{1}
\setlist[listalpha]{
  label={\bf\color{rubric}\alph*.},
  leftmargin=0pt,
  labelindent=0.8\listmod,
  labelwidth=0pt,
}
\newcommand{\listhead}[1]{\hspace{-1\listmod}\emph{#1}}

\renewcommand{\hrulefill}{%
  \leavevmode\leaders\hrule height 0.2pt\hfill\kern\z@}

% General typo
\DeclareTextFontCommand{\textlarge}{\large}
\DeclareTextFontCommand{\textsmall}{\small}


% commands, inlines
\newcommand{\anchor}[1]{\Hy@raisedlink{\hypertarget{#1}{}}} % link to top of an anchor (not baseline)
\newcommand\abbr{}
\newcommand{\autour}[1]{\tikz[baseline=(X.base)]\node [draw=rubric,thin,rectangle,inner sep=1.5pt, rounded corners=3pt] (X) {\color{rubric}#1};}
\newcommand\corr{}
\newcommand{\ed}[1]{ {\color{silver}\sffamily\footnotesize (#1)} } % <milestone ed="1688"/>
\newcommand\expan{}
\newcommand\gap{}
\renewcommand{\LettrineFontHook}{\color{rubric}}
\newcommand{\initial}[2]{\lettrine[lines=2, loversize=0.3, lhang=0.3]{#1}{#2}}
\newcommand{\initialiv}[2]{%
  \let\oldLFH\LettrineFontHook
  % \renewcommand{\LettrineFontHook}{\color{rubric}\ttfamily}
  \IfSubStr{QJ’}{#1}{
    \lettrine[lines=4, lhang=0.2, loversize=-0.1, lraise=0.2]{\smash{#1}}{#2}
  }{\IfSubStr{É}{#1}{
    \lettrine[lines=4, lhang=0.2, loversize=-0, lraise=0]{\smash{#1}}{#2}
  }{\IfSubStr{ÀÂ}{#1}{
    \lettrine[lines=4, lhang=0.2, loversize=-0, lraise=0, slope=0.6em]{\smash{#1}}{#2}
  }{\IfSubStr{A}{#1}{
    \lettrine[lines=4, lhang=0.2, loversize=0.2, slope=0.6em]{\smash{#1}}{#2}
  }{\IfSubStr{V}{#1}{
    \lettrine[lines=4, lhang=0.2, loversize=0.2, slope=-0.5em]{\smash{#1}}{#2}
  }{
    \lettrine[lines=4, lhang=0.2, loversize=0.2]{\smash{#1}}{#2}
  }}}}}
  \let\LettrineFontHook\oldLFH
}
\newcommand{\labelchar}[1]{\textbf{\color{rubric} #1}}
\newcommand{\milestone}[1]{\autour{\footnotesize\color{rubric} #1}} % <milestone n="4"/>
\newcommand\name{}
\newcommand\orig{}
\newcommand\orgName{}
\newcommand\persName{}
\newcommand\placeName{}
\newcommand{\pn}[1]{\IfSubStr{-—–¶}{#1}% <p n="3"/>
  {\noindent{\bfseries\color{rubric}   ¶  }}
  {{\footnotesize\autour{ #1}  }}}
\newcommand\reg{}
% \newcommand\ref{} % already defined
\newcommand\sic{}
\def\mednobreak{\ifdim\lastskip<\medskipamount
  \removelastskip\nopagebreak\medskip\fi}
\def\bignobreak{\ifdim\lastskip<\bigskipamount
  \removelastskip\nopagebreak\bigskip\fi}

% commands, blocks
\newcommand{\byline}[1]{\bigskip{\RaggedLeft{#1}\par}\bigskip}
\newcommand{\bibl}[1]{{\RaggedLeft{#1}\par\bigskip}}
\newcommand{\biblitem}[1]{{\noindent\hangindent=\parindent   #1\par}}
\newcommand{\dateline}[1]{\medskip{\RaggedLeft{#1}\par}\bigskip}
\newcommand{\labelblock}[1]{\medbreak{\noindent\color{rubric}\bfseries #1}\par\mednobreak}
\newcommand{\salute}[1]{\bigbreak{#1}\par\medbreak}
\newcommand{\signed}[1]{\bigbreak\filbreak{\raggedleft #1\par}\medskip}

% environments for blocks (some may become commands)
\newenvironment{borderbox}{}{} % framing content
\newenvironment{citbibl}{\ifvmode\hfill\fi}{\ifvmode\par\fi }
\newenvironment{docAuthor}{\ifvmode\vskip4pt\fontsize{16pt}{18pt}\selectfont\fi\itshape}{\ifvmode\par\fi }
\newenvironment{docDate}{}{\ifvmode\par\fi }
\newenvironment{docImprint}{\vskip6pt}{\ifvmode\par\fi }
\newenvironment{docTitle}{\vskip6pt\bfseries\fontsize{18pt}{22pt}\selectfont}{\par }
\newenvironment{msHead}{\vskip6pt}{\par}
\newenvironment{msItem}{\vskip6pt}{\par}
\newenvironment{titlePart}{}{\par }


% environments for block containers
\newenvironment{argument}{\itshape\parindent0pt}{\vskip1.5em}
\newenvironment{biblfree}{}{\ifvmode\par\fi }
\newenvironment{bibitemlist}[1]{%
  \list{\@biblabel{\@arabic\c@enumiv}}%
  {%
    \settowidth\labelwidth{\@biblabel{#1}}%
    \leftmargin\labelwidth
    \advance\leftmargin\labelsep
    \@openbib@code
    \usecounter{enumiv}%
    \let\p@enumiv\@empty
    \renewcommand\theenumiv{\@arabic\c@enumiv}%
  }
  \sloppy
  \clubpenalty4000
  \@clubpenalty \clubpenalty
  \widowpenalty4000%
  \sfcode`\.\@m
}%
{\def\@noitemerr
  {\@latex@warning{Empty `bibitemlist' environment}}%
\endlist}
\newenvironment{quoteblock}% may be used for ornaments
  {\begin{quoting}}
  {\end{quoting}}

% table () is preceded and finished by custom command
\newcommand{\tableopen}[1]{%
  \ifnum\strcmp{#1}{wide}=0{%
    \begin{center}
  }
  \else\ifnum\strcmp{#1}{long}=0{%
    \begin{center}
  }
  \else{%
    \begin{center}
  }
  \fi\fi
}
\newcommand{\tableclose}[1]{%
  \ifnum\strcmp{#1}{wide}=0{%
    \end{center}
  }
  \else\ifnum\strcmp{#1}{long}=0{%
    \end{center}
  }
  \else{%
    \end{center}
  }
  \fi\fi
}


% text structure
\newcommand\chapteropen{} % before chapter title
\newcommand\chaptercont{} % after title, argument, epigraph…
\newcommand\chapterclose{} % maybe useful for multicol settings
\setcounter{secnumdepth}{-2} % no counters for hierarchy titles
\setcounter{tocdepth}{5} % deep toc
\markright{\@title} % ???
\markboth{\@title}{\@author} % ???
\renewcommand\tableofcontents{\@starttoc{toc}}
% toclof format
% \renewcommand{\@tocrmarg}{0.1em} % Useless command?
% \renewcommand{\@pnumwidth}{0.5em} % {1.75em}
\renewcommand{\@cftmaketoctitle}{}
\setlength{\cftbeforesecskip}{\z@ \@plus.2\p@}
\renewcommand{\cftchapfont}{}
\renewcommand{\cftchapdotsep}{\cftdotsep}
\renewcommand{\cftchapleader}{\normalfont\cftdotfill{\cftchapdotsep}}
\renewcommand{\cftchappagefont}{\bfseries}
\setlength{\cftbeforechapskip}{0em \@plus\p@}
% \renewcommand{\cftsecfont}{\small\relax}
\renewcommand{\cftsecpagefont}{\normalfont}
% \renewcommand{\cftsubsecfont}{\small\relax}
\renewcommand{\cftsecdotsep}{\cftdotsep}
\renewcommand{\cftsecpagefont}{\normalfont}
\renewcommand{\cftsecleader}{\normalfont\cftdotfill{\cftsecdotsep}}
\setlength{\cftsecindent}{1em}
\setlength{\cftsubsecindent}{2em}
\setlength{\cftsubsubsecindent}{3em}
\setlength{\cftchapnumwidth}{1em}
\setlength{\cftsecnumwidth}{1em}
\setlength{\cftsubsecnumwidth}{1em}
\setlength{\cftsubsubsecnumwidth}{1em}

% footnotes
\newif\ifheading
\newcommand*{\fnmarkscale}{\ifheading 0.70 \else 1 \fi}
\renewcommand\footnoterule{\vspace*{0.3cm}\hrule height \arrayrulewidth width 3cm \vspace*{0.3cm}}
\setlength\footnotesep{1.5\footnotesep} % footnote separator
\renewcommand\@makefntext[1]{\parindent 1.5em \noindent \hb@xt@1.8em{\hss{\normalfont\@thefnmark . }}#1} % no superscipt in foot


% orphans and widows
\clubpenalty=9996
\widowpenalty=9999
\brokenpenalty=4991
\predisplaypenalty=10000
\postdisplaypenalty=1549
\displaywidowpenalty=1602
\hyphenpenalty=400
% Copied from Rahtz but not understood
\def\@pnumwidth{1.55em}
\def\@tocrmarg {2.55em}
\def\@dotsep{4.5}
\emergencystretch 3em
\hbadness=4000
\pretolerance=750
\tolerance=2000
\vbadness=4000
\def\Gin@extensions{.pdf,.png,.jpg,.mps,.tif}
% \renewcommand{\@cite}[1]{#1} % biblio

\makeatother % /@@@>
%%%%%%%%%%%%%%
% </TEI> end %
%%%%%%%%%%%%%%


%%%%%%%%%%%%%
% footnotes %
%%%%%%%%%%%%%
\renewcommand{\thefootnote}{\bfseries\textcolor{rubric}{\arabic{footnote}}} % color for footnote marks

%%%%%%%%%
% Fonts %
%%%%%%%%%
\usepackage[]{roboto} % SmallCaps, Regular is a bit bold
% \linespread{0.90} % too compact, keep font natural
\newfontfamily\fontrun[]{Roboto Condensed Light} % condensed runing heads
\ifav
  \setmainfont[
    ItalicFont={Roboto Light Italic},
  ]{Roboto}
\else\ifbooklet
  \setmainfont[
    ItalicFont={Roboto Light Italic},
  ]{Roboto}
\else
\setmainfont[
  ItalicFont={Roboto Italic},
]{Roboto Light}
\fi\fi
\renewcommand{\LettrineFontHook}{\bfseries\color{rubric}}
% \renewenvironment{labelblock}{\begin{center}\bfseries\color{rubric}}{\end{center}}

%%%%%%%%
% MISC %
%%%%%%%%

\setdefaultlanguage[frenchpart=false]{french} % bug on part


\newenvironment{quotebar}{%
    \def\FrameCommand{{\color{rubric!10!}\vrule width 0.5em} \hspace{0.9em}}%
    \def\OuterFrameSep{\itemsep} % séparateur vertical
    \MakeFramed {\advance\hsize-\width \FrameRestore}
  }%
  {%
    \endMakeFramed
  }
\renewenvironment{quoteblock}% may be used for ornaments
  {%
    \savenotes
    \setstretch{0.9}
    \normalfont
    \begin{quotebar}
  }
  {%
    \end{quotebar}
    \spewnotes
  }


\renewcommand{\headrulewidth}{\arrayrulewidth}
\renewcommand{\headrule}{{\color{rubric}\hrule}}

% delicate tuning, image has produce line-height problems in title on 2 lines
\titleformat{name=\chapter} % command
  [display] % shape
  {\vspace{1.5em}\centering} % format
  {} % label
  {0pt} % separator between n
  {}
[{\color{rubric}\huge\textbf{#1}}\bigskip] % after code
% \titlespacing{command}{left spacing}{before spacing}{after spacing}[right]
\titlespacing*{\chapter}{0pt}{-2em}{0pt}[0pt]

\titleformat{name=\section}
  [block]{}{}{}{}
  [\vbox{\color{rubric}\large\raggedleft\textbf{#1}}]
\titlespacing{\section}{0pt}{0pt plus 4pt minus 2pt}{\baselineskip}

\titleformat{name=\subsection}
  [block]
  {}
  {} % \thesection
  {} % separator \arrayrulewidth
  {}
[\vbox{\large\textbf{#1}}]
% \titlespacing{\subsection}{0pt}{0pt plus 4pt minus 2pt}{\baselineskip}

\ifaiv
  \fancypagestyle{main}{%
    \fancyhf{}
    \setlength{\headheight}{1.5em}
    \fancyhead{} % reset head
    \fancyfoot{} % reset foot
    \fancyhead[L]{\truncate{0.45\headwidth}{\fontrun\elbibl}} % book ref
    \fancyhead[R]{\truncate{0.45\headwidth}{ \fontrun\nouppercase\leftmark}} % Chapter title
    \fancyhead[C]{\thepage}
  }
  \fancypagestyle{plain}{% apply to chapter
    \fancyhf{}% clear all header and footer fields
    \setlength{\headheight}{1.5em}
    \fancyhead[L]{\truncate{0.9\headwidth}{\fontrun\elbibl}}
    \fancyhead[R]{\thepage}
  }
\else
  \fancypagestyle{main}{%
    \fancyhf{}
    \setlength{\headheight}{1.5em}
    \fancyhead{} % reset head
    \fancyfoot{} % reset foot
    \fancyhead[RE]{\truncate{0.9\headwidth}{\fontrun\elbibl}} % book ref
    \fancyhead[LO]{\truncate{0.9\headwidth}{\fontrun\nouppercase\leftmark}} % Chapter title, \nouppercase needed
    \fancyhead[RO,LE]{\thepage}
  }
  \fancypagestyle{plain}{% apply to chapter
    \fancyhf{}% clear all header and footer fields
    \setlength{\headheight}{1.5em}
    \fancyhead[L]{\truncate{0.9\headwidth}{\fontrun\elbibl}}
    \fancyhead[R]{\thepage}
  }
\fi

\ifav % a5 only
  \titleclass{\section}{top}
\fi

\newcommand\chapo{{%
  \vspace*{-3em}
  \centering % no vskip ()
  {\Large\addfontfeature{LetterSpace=25}\bfseries{\elauthor}}\par
  \smallskip
  {\large\eldate}\par
  \bigskip
  {\Large\selectfont{\eltitle}}\par
  \bigskip
  {\color{rubric}\hline\par}
  \bigskip
  {\Large LIVRE LIBRE À PRIX LIBRE, DEMANDEZ AU COMPTOIR\par}
  \centerline{\small\color{rubric} {hurlus.fr, tiré le \today}}\par
  \bigskip
}}


\begin{document}
\pagestyle{empty}
\ifbooklet{
  \thispagestyle{empty}
  \centering
  {\LARGE\bfseries{\elauthor}}\par
  \bigskip
  {\Large\eldate}\par
  \bigskip
  \bigskip
  {\LARGE\selectfont{\eltitle}}\par
  \vfill\null
  {\color{rubric}\setlength{\arrayrulewidth}{2pt}\hline\par}
  \vfill\null
  {\Large LIVRE LIBRE À PRIX LIBRE, DEMANDEZ AU COMPTOIR\par}
  \centerline{\small{hurlus.fr, tiré le \today}}\par
  \newpage\null\thispagestyle{empty}\newpage
  \addtocounter{page}{-2}
}\fi

\thispagestyle{empty}
\ifaiv
  \twocolumn[\chapo]
\else
  \chapo
\fi
{\it\elabstract}
\bigskip
\makeatletter\@starttoc{toc}\makeatother % toc without new page
\bigskip

\pagestyle{main} % after style

  \noindent Si ce discours semble trop long pour être tout lu en une fois, on le pourra distinguer en six parties.\par
Et, en la première, on trouvera diverses considérations touchant les sciences.\par
En la seconde, les principales règles de la Méthode que l’auteur a cherchée.\par
En la 3, quelques-unes de celles de la Morale qu’il a tirée de cette Méthode.\par
En la 4, les raisons par lesquelles il prouve l’existence de Dieu \& de l’âme humaine, qui sont les fondements de sa Métaphysique.\par
En la 5, l’ordre des questions de Physique qu’il a cherchées, \& particulièrement l’explication du mouvement du cœur \& de quelques autres difficultés qui appartiennent à la Médecine, puis aussi la différence qui est entre notre âme \& celle des bêtes.\par
Et en la dernière, quelles choses il croit être requises pour aller plus avant en la recherche de la Nature qu’il n’a été, \& quelles raisons l’ont fait écrire.

\chapteropen
\chapter[Première partie]{Première partie}\phantomsection
\label{I}\renewcommand{\leftmark}{Première partie}


\begin{argument}\noindent Et, en la première, on trouvera diverses considérations touchant les sciences.
\end{argument}


\chaptercont
\phantomsection
\label{I1}\noindent \initialiv{L}{e bon sens} est la chose du monde la mieux partagée : car chacun pense en être si bien pourvu, que ceux même qui sont les plus difficiles à contenter en toute autre chose, n’ont point coutume d’en désirer plus qu’ils en ont.\par
\pn{-}En quoi il n’est pas vraisemblable que tous se trompent :\par
Mais plutôt cela témoigne que la puissance de bien juger, \& distinguer le vrai d’avec le faux, qui est proprement ce qu’on nomme le bon sens ou la raison, est naturellement égale en tous les hommes ;\par
Et ainsi que la diversité de nos opinions ne vient pas de ce que les uns sont plus raisonnables que les autres, mais seulement de ce que nous conduisons nos pensées par diverses voies, \& ne considérons pas les mêmes choses.\par
\textbf{Car ce n’est pas assez d’avoir l’esprit bon, mais le principal est de l’appliquer bien.}\par
Les plus grandes âmes sont capables des plus grands vices, aussi bien que des plus grandes vertus :\par
Et ceux qui ne marchent que fort lentement peuvent avancer beaucoup davantage, s’ils suivent toujours le droit chemin, que ne font ceux qui courent, \& qui s’en éloignent.\par
\bigbreak
\phantomsection
\label{I2}\noindent \pn{2}Pour moi je n’ai jamais présumé que mon esprit fût en rien plus parfait que ceux du commun : même j’ai souvent souhaité d’avoir la pensée aussi prompte, ou l’imagination aussi nette \& distincte, ou la mémoire aussi ample, ou aussi présente, que quelques autres.\par
Et je ne sache point de qualités que celles-ci, qui servent à la perfection de l’esprit : car pour la raison, ou le sens, d’autant qu’elle est la seule chose qui nous rend hommes, \& nous distingue des bêtes, je veux croire qu’elle est tout entière en un chacun ; \& suivre en ceci l’opinion commune des Philosophes, qui disent qu’il n’y a du plus \& du moins qu’entre les {\itshape accidents}, \& non point entre les {\itshape formes}, ou natures, des {\itshape individus} d’une même {\itshape espèce}\footnote{On peut avoir plus ou moins d’une qualité \emph{accidentelle}, comme la mémoire, l’imagination, ou la force : mais le bon sens est un caractère de l’espèce humaine, elle en est douée par nature, en entier et sans degrés.}.\par
\bigbreak
\phantomsection
\label{I3}\noindent \pn{3}Mais je ne craindrai pas de dire que je pense avoir eu beaucoup d’heur, de m’être rencontré dès ma jeunesse en certains chemins, qui m’ont conduit à des considérations \& des maximes, dont j’ai formé une Méthode, par laquelle il me semble que j’ai moyen d’augmenter par degrés ma connaissance, \& de l’élever peu à peu au plus haut point, auquel la médiocrité de mon esprit \& la courte durée de ma vie lui pourront permettre d’atteindre.\par
Car j’en ai déjà recueilli de tels fruits, qu’encore qu’aux jugements que je fais de moi-même, je tâche toujours de pencher vers le côté de la défiance, plutôt que vers celui de la présomption ; \& que, regardant d’un œil de Philosophe les diverses actions \& entreprises de tous les hommes, il n’y en ait quasi aucune qui ne me semble vaine \& inutile, Je ne laisse pas de recevoir une extrême satisfaction du progrès que je pense avoir déjà fait en la recherche de la vérité, \& de concevoir de telles espérances pour l’avenir, que si entre les occupations des hommes purement hommes, il y en a quelqu’une qui soit solidement bonne \& importante, j’ose croire que c’est celle que j’ai choisie.\par
\bigbreak
\phantomsection
\label{I4}\noindent \pn{4}Toutefois il se peut faire que je me trompe, \& ce n’est peut-être, qu’un peu de cuivre \& de verre que je prends pour de l’or \& des diamants. \par
\textbf{Je sais combien nous sommes sujets à nous méprendre en ce qui nous touche ; \& combien aussi les jugements de nos amis nous doivent être suspects, lorsqu’ils sont en notre faveur.}\par
Mais je serai bien aise de faire voir en ce discours quels sont les chemins que j’ai suivis, \& d’y représenter ma vie comme en un tableau, afin que chacun en puisse juger, \& qu’apprenant du bruit commun les opinions qu’on en aura, ce soit un nouveau moyen de m’instruire, que j’ajouterai à ceux dont j’ai coutume de me servir.\par
\bigbreak
\phantomsection
\label{I5}\noindent \pn{5}Ainsi mon dessein n’est pas d’enseigner ici la Méthode que chacun doit suivre pour bien conduire sa raison : mais seulement de faire voir en quelle sorte j’ai tâché de conduire la mienne. Ceux qui se mêlent de donner des préceptes, se doivent estimer plus habiles, que ceux auxquels ils les donnent, \& s’ils manquent en la moindre chose, ils en sont blâmables. Mais ne proposant cet écrit, que comme une histoire, ou si vous l’aimez mieux que comme une fable, en laquelle parmi quelques exemples qu’on peut imiter, on en trouvera peut-être aussi plusieurs autres qu’on aura raison de ne pas suivre ;\par
J’espère qu’il sera utile à quelques-uns, sans être nuisible à personne, \& que tous me sauront gré de ma franchise.\par
\bigbreak
\phantomsection
\label{I6}\noindent \pn{6}J’ai été nourri aux lettres dès mon enfance, \& parce qu’on me persuadait que, par leur moyen, on pouvait acquérir une connaissance claire \& assurée de tout ce qui est utile à la vie, j’avais un extrême désir de les apprendre. Mais sitôt que j’eus achevé tout ce cours d’études, au bout duquel on a coutume d’être reçu au rang des doctes, je changeai entièrement d’opinion, car je me trouvais embarrassé de tant de doutes \& d’erreurs, qu’il me semblait n’avoir fait autre profit en tâchant de m’instruire, sinon que j’avais découvert de plus en plus mon ignorance.\par
\pn{-}Et néanmoins j’étais en l’une des plus célèbres écoles de l’Europe, où je pensais qu’il devait y avoir de savants hommes, s’il y en avait en aucun endroit de la terre :\par
J’y avais appris tout ce que les autres y apprenaient ; \& même ne m’étant pas contenté des sciences qu’on nous enseignait, j’avais parcouru tous les livres, traitant de celles qu’on estime les plus curieuses \& les plus rares, qui avaient pu tomber entre mes mains :\par
Avec cela je savais les jugements que les autres faisaient de moi ; \& je ne voyais point qu’on m’estimât inférieur à mes condisciples, bien qu’il y en eût déjà entre eux quelques-uns, qu’on destinait à remplir les places de nos maîtres :\par
Et enfin notre siècle me semblait aussi fleurissant, \& aussi fertile en bons esprits, qu’ait été aucun des précédents. Ce qui me faisait prendre la liberté de juger par moi de tous les autres, \& de penser qu’il n’y avait aucune doctrine dans le monde qui fût telle qu’on m’avait auparavant fait espérer.\par

\astermono

\phantomsection
\label{I7}\noindent \pn{7}Je ne laissais pas toutefois d’estimer les exercices, auxquels on s’occupe dans les écoles.\par
Je savais que les \textbf{langues} qu’on y apprend sont nécessaires pour l’intelligence des livres anciens ; Que la gentillesse des \textbf{fables} réveille l’esprit ; Que les actions mémorables des histoires le relèvent, \& qu’étant lues avec discrétion, elles aident à former le jugement ;\par
Que la lecture de tous les bons livres est comme une conversation avec les plus honnêtes gens des siècles passés, qui en ont été les auteurs, \& même une conversation étudiée, en laquelle ils ne nous découvrent que les meilleures de leurs pensées ;\par
Que l’\textbf{Éloquence} a des forces \& des beautés incomparables ; Que la \textbf{Poésie} a des délicatesses \& des douceurs très ravissantes ;\par
Que les \textbf{Mathématiques} ont des inventions très subtiles \& qui peuvent beaucoup servir, tant à contenter les curieux, qu’à faciliter tous les arts \& diminuer le travail des hommes ;\par
Que les écrits qui traitent des mœurs contiennent plusieurs enseignements \& plusieurs exhortations à la vertu qui sont fort utiles ; Que la \textbf{Théologie} enseigne à gagner le ciel ;\par
Que la \textbf{Philosophie} donne moyen de parler vraisemblablement de toutes choses, \& se faire admirer des moins savants ;\par
Que la Jurisprudence, la Médecine \& les \textbf{autres sciences} apportent des honneurs \& des richesses à ceux qui les cultivent ; Et enfin, qu’il est bon de les avoir toutes examinées, même les plus superstitieuses \& les plus fausses, afin de connaître leur juste valeur, \& se garder d’en être trompé.\par
\bigbreak
\phantomsection
\label{I8}\noindent \pn{8}Mais je croyais avoir déjà donné assez de temps aux \textbf{langues}, \& même aussi à la lecture des livres anciens, \& à leurs histoires, \& à leurs \textbf{fables}.\par
Car c’est quasi le même de converser avec ceux des autres siècles, que de voyager.\par
Il est bon de savoir quelque chose des mœurs de divers peuples, afin de juger des nôtres plus sainement, \& que nous ne pensions pas que tout ce qui est contre nos modes soit ridicule, \& contre raison ; ainsi qu’ont coutume de faire ceux qui n’ont rien vu :\par
Mais lorsqu’on. emploie trop de temps à voyager, on devient enfin étranger en son pays ; \& lorsqu’on est trop curieux des choses qui se pratiquaient aux siècles passés, on demeure ordinairement fort ignorant de celles qui se pratiquent en celui-ci.\par
Outre que les fables font imaginer plusieurs événements comme possibles qui ne le sont point ;\par
Et que même les histoires les plus fidèles, si elles ne changent ni n’augmentent la valeur des choses, pour les rendre plus dignes d’être lues, au moins en omettent-elles presque toujours les plus basses \& moins illustres circonstances, d’où vient que le reste ne paraît pas tel qu’il est, \& que ceux qui règlent leurs mœurs par les exemples qu’ils en tirent, sont sujets à tomber dans les extravagances des Paladins de nos romans, \& à concevoir des desseins qui passent leurs forces.\par
\bigbreak
\phantomsection
\label{I9}\noindent \pn{9}J’estimais fort l’\textbf{Éloquence}, \& j’étais amoureux de la \textbf{Poésie} :\par
Mais je pensais que l’une \& l’autre étaient des dons de l’esprit, plutôt que des fruits de l’étude.\par
Ceux qui ont le raisonnement le plus fort, \& qui digèrent le mieux leurs pensées, afin de les rendre claires \& intelligibles, peuvent toujours le mieux persuader ce qu’ils proposent, encore qu’ils ne parlassent que bas Breton, \& qu’ils n’eussent jamais appris de Rhétorique.\par
Et ceux qui ont les inventions les plus agréables, \& qui les savent exprimer avec le plus d’ornement \& de douceur, ne laisseraient pas d’être les meilleurs poètes, encore que l’art poétique leur fût inconnu.\par
\bigbreak
\phantomsection
\label{I10}\noindent \pn{10}Je me plaisais surtout aux \textbf{Mathématiques}, à cause de la certitude \& de l’évidence de leurs raisons, mais je ne remarquais point encore leur vrai usage, \& pensant qu’elles ne servaient qu’aux arts Mécaniques, je m’étonnais de ce que, leurs fondements étant si fermes \& si solides, on n’avait rien bâti dessus de plus relevé.\par
Comme, au contraire, je comparais les écrits des anciens païens, qui traitent des mœurs, à des palais fort superbes \& fort magnifiques, qui n’étaient bâtis que sur du sable \& sur de la boue ;\par
Ils élèvent fort haut les vertus, \& les font paraître estimables par-dessus toutes les choses qui sont au monde, mais ils n’enseignent pas assez à les connaître, \& souvent ce qu’ils appellent d’un si beau nom n’est qu’une insensibilité, ou un orgueil, ou un désespoir, ou un parricide.\par
\bigbreak
\phantomsection
\label{I11}\noindent \pn{11}Je révérais notre \textbf{Théologie}, \& prétendais, autant qu’aucun autre, à gagner le ciel ; mais ayant appris, comme chose très assurée, que le chemin n’en est pas moins ouvert aux plus ignorants qu’aux plus doctes, \& que les vérités révélées, qui y conduisent, sont au-dessus de notre intelligence, je n’eusse osé les soumettre à la faiblesse de mes raisonnements, \& je pensais que, pour entreprendre de les examiner \& y réussir, il était besoin d’avoir quelque extraordinaire assistance du ciel, \& d’être plus qu’homme.\par
\bigbreak
\phantomsection
\label{I12}\noindent \pn{12}Je ne dirai rien de la \textbf{Philosophie}, sinon que, voyant qu’elle a été cultivée par les plus excellents esprits qui aient vécu depuis plusieurs siècles, \& que néanmoins il ne s’y trouve encore aucune chose dont on ne dispute, \& par conséquent qui ne soit douteuse, je n’avais point assez de présomption pour espérer d’y rencontrer mieux que les autres ;\par
Et que, considérant combien il peut y avoir de diverses opinions, touchant une même matière, qui soient soutenues par des gens doctes, sans qu’il y en puisse avoir jamais plus d’une seule qui soit vraie, je réputais presque pour faux tout ce qui n’était que vraisemblable.\par
\bigbreak
\phantomsection
\label{I13}\noindent \pn{13}Puis pour les \textbf{autres sciences}, d’autant qu’elles empruntent leurs principes de la Philosophie, je jugeais qu’on ne pouvait avoir rien bâti, qui fût solide, sur des fondements si peu fermes ;\par
Et ni l’honneur, ni le gain qu’elles promettent, n’étaient suffisants pour me convier à les apprendre :\par
Car je ne me sentais point, grâces à Dieu, de condition, qui m’obligeât à faire un métier de la science, pour le soulagement de ma fortune ;\par
Et quoique je ne fisse pas profession de mépriser la gloire en Cynique, je faisais néanmoins fort peu d’état de celle que je n’espérais point pouvoir acquérir qu’à faux titres.\par
Et enfin, pour les mauvaises doctrines, je pensais déjà connaître assez ce qu’elles valaient, pour n’être plus sujet à être trompé, ni par les promesses d’un alchimiste, ni par les prédictions d’un astrologue, ni par les impostures d’un magicien, ni par les artifices ou la vanterie d’aucun de ceux qui font profession de savoir plus qu’ils ne savent.\par

\astermono

\phantomsection
\label{I14}\noindent \pn{14}C’est pourquoi, sitôt que l’âge me permit de sortir de la sujétion de mes précepteurs, je quittai entièrement l’étude des lettres. Et me résolvant de ne chercher plus d’autre science, que celle qui se pourrait trouver en moi-même, ou bien dans le grand livre du monde, J’employai le reste de ma jeunesse à voyager, à voir des cours, \& des armées, à fréquenter des gens de diverses humeurs \& conditions, à recueillir diverses expériences, à m’éprouver moi-même dans les rencontres que la fortune me proposait, \& partout à faire telle réflexion sur les choses qui se présentaient, que j’en pusse tirer quelque profit.\par
\textbf{Car il me semblait que je pourrais rencontrer beaucoup plus de vérité dans les raisonnements que chacun fait, touchant les affaires qui lui importent, \& dont l’événement le doit punir bientôt après, s’il a mal jugé ; que dans ceux que fait un homme de lettres dans son cabinet, touchant des spéculations qui ne produisent aucun effet}, \& qui ne lui sont d’autre conséquence, sinon que peut-être il en tirera d’autant plus de vanité qu’elles seront plus éloignées du sens commun : à cause qu’il aura dû employer d’autant plus d’esprit \& d’artifice à tâcher de les rendre vraisemblables. Et j’avais toujours un extrême désir d’apprendre à distinguer le vrai d’avec le faux, pour voir clair en mes actions, \& marcher avec assurance en cette vie.\par
\bigbreak
\phantomsection
\label{I15}\noindent \pn{15}Il est vrai que, pendant que je ne faisais que considérer les mœurs des autres hommes, je n’y trouvais guère de quoi m’assurer ; \& que j’y remarquais quasi autant de diversité que j’avais fait auparavant entre les opinions des Philosophes.\par
En sorte que le plus grand profit que j’en retirais, était que voyant plusieurs choses, qui bien qu’elles nous semblent fort extravagantes \& ridicules, ne laissent pas d’être communément reçues \& approuvées par d’autres grands peuples, j’apprenais à ne rien croire trop fermement de ce qui ne m’avait été persuadé que par l’exemple \& par la coutume :\par
Et ainsi je me délivrais peu à peu de beaucoup d’erreurs, qui peuvent offusquer notre lumière naturelle, \& nous rendre moins capables d’entendre raison.\par
Mais après que j’eus employé quelques années à étudier ainsi dans le livre du monde, \& à tâcher d’acquérir quelque expérience, je pris un jour résolution d’étudier aussi en moi-même, \& d’employer toutes les forces de mon esprit à choisir les chemins que je devais suivre. Ce qui me réussit beaucoup mieux, ce me semble, que si je ne me fusse jamais éloigné, ni de mon pays, ni de mes livres.
\chapterclose


\chapteropen
\chapter[Seconde partie]{Seconde partie}\phantomsection
\label{II}\renewcommand{\leftmark}{Seconde partie}


\begin{argument}\noindent En la seconde, les principales règles de la Méthode que l’auteur a cherchée.
\end{argument}


\chaptercont
\phantomsection
\label{II1}\noindent \initialiv{J\kern-0.08em{’}}{étais alors en Allemagne,} où l’occasion des guerres qui n’y sont pas encore finies m’avait appelé ; \& comme je retournais du couronnement de l’Empereur vers l’armée, le commencement de l’hiver m’arrêta en un quartier où ne trouvant aucune conversation qui me divertît, \& n’ayant d’ailleurs par bonheur, aucuns soins ni passions qui me troublassent, je demeurais tout le jour enfermé seul dans un poêle, où j’avais tout loisir de m’entretenir de mes pensées.\par
Entre lesquelles l’une des premières, fut que je m’avisai de considérer, que souvent il n’y a pas tant de perfection dans les ouvrages composés de plusieurs pièces, \& faits de la main de divers maîtres, qu’en ceux auxquels un seul a travaillé. Ainsi voit-on que les bâtiments qu’un seul Architecte a entrepris \& achevés, ont coutume d’être plus beaux \& mieux ordonnés, que ceux que plusieurs ont tâché de raccommoder, en faisant servir de vieilles murailles qui avaient été bâties à d’autres fins. Ainsi ces anciennes cités, qui n’ayant été au commencement que des bourgades, sont devenues par succession de temps de grandes villes, sont ordinairement si mal compassées, au prix de ces places régulières qu’un Ingénieur trace à sa fantaisie dans une plaine, qu’encore que considérant leurs édifices chacun à part on y trouve souvent autant ou plus d’art qu’en ceux des autres ; toutefois à voir comme ils sont arrangés, ici un grand, là un petit, \& comme ils rendent les rues courbées \& inégales, on dirait que c’est plutôt la fortune, que la volonté de quelques hommes usant de raison, qui les a ainsi disposés. Et si on considère qu’il y a eu néanmoins de tout temps quelques officiers, qui ont eu charge de prendre garde aux bâtiments des particuliers, pour les faire servir à l’ornement du public, on connaîtra bien qu’\textbf{il est malaisé, en ne travaillant que sur les ouvrages d’autrui, de faire des choses fort accomplies}.\par
Ainsi je m’imaginai que les peuples, qui ayant été autrefois demi-sauvages, \& ne s’étant civilisés que peu à peu, n’ont fait leurs lois qu’à mesure que l’incommodité des crimes \& des querelles les y a contraints, ne sauraient être si bien policés, que ceux qui dès le commencement qu’ils se sont assemblés, ont observé les constitutions de quelque prudent Législateur. Comme il est bien certain que l’État de la vraie religion, dont Dieu seul a fait les ordonnances, doit être incomparablement mieux réglé que tous les autres. Et pour parler des choses humaines, je crois que, si Sparte a été autrefois très florissante, ce n’a pas été à cause de la bonté de chacune de ses lois en particulier, vu que plusieurs étaient fort étranges, \& même contraires aux bonnes mœurs, mais à cause que, n’ayant été inventées que par un seul, elles tendaient toutes à même fin.\par
Et ainsi je pensai que les sciences des livres, au moins celles dont les raisons ne sont que probables, \& qui n’ont aucunes démonstrations, s’étant composées \& grossies peu à peu des opinions de plusieurs diverses personnes, ne sont point, si approchantes de la vérité, que les simples raisonnements que peut faire naturellement un homme de bon sens touchant les choses qui se présentent.\par
Et ainsi encore je pensai, que pour ce que nous avons tous été enfants avant que d’être hommes, \& qu’il nous a fallu longtemps être gouvernés par nos appétits \& nos Précepteurs, qui étaient souvent contraires les uns aux autres, \& qui ni les uns ni les autres, ne nous conseillaient peut-être pas toujours le meilleur, il est presque impossible que nos jugements soient si purs, ni si solides qu’ils auraient été, si nous avions eu l’usage entier de notre raison dès le point de notre naissance, \& que nous n’eussions jamais été conduits que par elle.\par
\bigbreak
\phantomsection
\label{II2}\noindent \pn{2}Il est vrai que nous ne voyons point qu’on jette par terre toutes les maisons d’une ville, pour le seul dessein de les refaire d’autre façon, \& d’en rendre les rues plus belles ; mais on voit bien que plusieurs font abattre les leurs pour les rebâtir, \& que même quelquefois ils y sont contraints, quand elles sont en danger de tomber d’elles-mêmes, \& que les fondements n’en sont pas bien fermes.\par
À l’exemple de quoi je me persuadai, qu’il n’y aurait véritablement point d’apparence, qu’un particulier fît dessein de réformer un État, en y changeant tout dès les fondements, \& en le renversant pour le redresser ;\par
Ni même aussi de réformer le corps des sciences, ou l’ordre établi dans les écoles pour les enseigner.\par
\pn{-}Mais que \textbf{pour toutes les opinions que j’avais reçues jusques alors en ma créance, je ne pouvais mieux faire que d’entreprendre une bonne fois, de les en ôter, afin d’y en remettre par après, ou d’autres meilleures, ou bien les mêmes, lorsque je les aurais ajustées au niveau de la raison}.\par
Et je crus fermement que, par ce moyen, je réussirais à conduire ma vie beaucoup mieux que si je ne bâtissais que sur de vieux fondements, \& que je ne m’appuyasse que sur les principes que je m’étais laissé persuader en ma jeunesse, sans avoir jamais examiné s’ils étaient vrais.\par
\pn{-}Car, bien que je remarquasse en ceci diverses difficultés, elles n’étaient point toutefois sans remède, ni comparables à celles qui se trouvent en la réformation des moindres choses qui touchent le public. Ces grands corps sont trop malaisés à relever, étant abattus, ou même à retenir, étant ébranlés, \& leurs chutes ne peuvent être que très rudes. Puis pour leurs imperfections, s’ils en ont, comme la seule diversité qui est entre eux suffit pour assurer que plusieurs en ont, l’usage les a sans doute fort adoucies ; \& même il en a évité, ou corrigé insensiblement quantité, auxquelles en ne pourrait si bien pourvoir par prudence. Et enfin, elles sont quasi toujours plus supportables que ne serait leur changement :\par
En même façon que les grands chemins, qui tournoient entre des montagnes, deviennent peu à peu si unis \& si commodes, à force d’être fréquentés, qu’il est beaucoup meilleur de les suivre, que d’entreprendre d’aller plus droit, en grimpant au-dessus des rochers, \& descendant jusques au bas des précipices.\par
\bigbreak
\phantomsection
\label{II3}\noindent \pn{3}C’est pourquoi \textbf{je ne saurais aucunement approuver ces humeurs brouillonnes, \& inquiètes, qui n’étant appelées, ni par leur naissance, ni par leur fortune, au maniement des affaires publiques, ne laissent pas d’y faire toujours en Idée quelque nouvelle réformation.}\par
Et si je pensais qu’il y eût la moindre chose en cet écrit, par laquelle on me pût soupçonner de cette folie, je serais très marri de souffrir qu’il fût publié. Jamais mon dessein ne s’est étendu plus avant que de tâcher à réformer mes propres pensées, \& de bâtir dans un fonds qui est tout à moi.\par
\pn{-}Que si mon ouvrage m’ayant assez plu, je vous en fais voir ici le modèle, ce n’est pas pour cela que je veuille conseiller à personne de l’imiter :\par
Ceux que Dieu a mieux partagés de ses grâces auront peut-être des desseins plus relevés, mais je crains bien que celui-ci ne soit déjà que trop hardi pour plusieurs. \par
La seule résolution de se défaire de toutes les opinions qu’on a reçues auparavant en sa créance, n’est pas un exemple que chacun doive suivre :\par
Et le monde n’est quasi composé que de deux sortes d’esprits auxquels il ne convient aucunement.\par
\pn{-}À savoir, de ceux qui, se croyant plus habiles qu’ils ne sont, ne se peuvent empêcher de précipiter leurs jugements, ni avoir assez de patience pour conduire par ordre toutes leurs pensées : d’où vient que, s’ils avaient une fois pris la liberté de douter des principes qu’ils ont reçus, \& de s’écarter du chemin commun, jamais ils ne pourraient tenir le sentier qu’il faut prendre pour aller plus droit, \& demeureraient égarés toute leur vie.\par
\pn{-}Puis, de ceux qui, ayant assez de raison, ou de modestie, pour juger qu’ils sont moins capables de distinguer le vrai d’avec le faux, que quelques autres par lesquels ils peuvent être instruits, doivent bien plutôt se contenter de suivre les opinions de ces autres, qu’en chercher eux-mêmes de meilleures.\par
\bigbreak
\phantomsection
\label{II4}\noindent \pn{4}Et pour moi, j’aurais été sans doute du nombre de ces derniers, si je n’avais jamais eu qu’un seul maître, ou que je n’eusse point su les différences qui ont été de tout temps entre les opinions des plus doctes.\par
\pn{-}Mais ayant appris dès le Collège, qu’on ne saurait rien imaginer de si étrange \& si peu croyable, qu’il n’ait été dit par quelqu’un des Philosophes ;\par
Et depuis en voyageant ayant reconnu, que tous ceux qui ont des sentiments fort contraires aux nôtres, ne sont pas pour cela barbares ni sauvages, mais que plusieurs usent autant ou plus que nous de raison ;\par
Et ayant considéré combien un même homme, avec son même esprit, étant nourri dès son enfance entre des Français ou des Allemands, devient différent de ce qu’il serait, s’il avait toujours vécu entre des Chinois ou des Cannibales ;\par
Et comment, jusques aux modes de nos habits, la même chose qui nous a plu il y a dix ans, \& qui nous plaira peut-être encore avant dix ans, nous semble maintenant extravagante \& ridicule :\par
\pn{-}En sorte que c’est bien plus la coutume \& l’exemple qui nous persuadent, qu’aucune connaissance certaine ;\par
Et que néanmoins la pluralité des voix n’est pas une preuve qui vaille rien, pour les vérités un peu malaisées à découvrir, à cause qu’il est bien plus vraisemblable qu’un homme seul les ait rencontrées que tout un peuple ;\par
Je ne pouvais choisir personne dont les opinions me semblassent devoir être préférées à celles des autres, \& je me trouvai comme contraint d’entreprendre moi-même de me conduire.\par
\bigbreak
\phantomsection
\label{II5}\noindent \pn{5}Mais comme un homme qui marche seul, \& dans les ténèbres, je me résolus d’aller si lentement, \& d’user de tant de circonspection en toutes choses, que si je n’avançais que fort peu, je me garderais bien au moins, de tomber. Même je ne voulus point commencer à rejeter tout à fait aucune des opinions, qui s’étaient pu glisser autrefois en ma créance sans y avoir été introduites par la raison, que je n’eusse auparavant employé assez de temps à faire le projet de l’ouvrage que j’entreprenais, \& à chercher la vraie Méthode pour parvenir à la connaissance de toutes les choses dont mon esprit serait capable.\par
\bigbreak
\phantomsection
\label{II6}\noindent \pn{6}J’avais un peu étudié, étant plus jeune, entre les parties de la Philosophie à la \textbf{Logique}, \& entre les Mathématiques à l’\textbf{Analyse} des géomètres, \& à l’\textbf{Algèbre}, trois arts ou sciences qui semblaient devoir contribuer quelque chose à mon dessein.\par
\pn{-}Mais, en les examinant je pris garde, que pour la \textbf{Logique} ses syllogismes, \& la plupart de ses autres instructions servent plutôt à expliquer à autrui les choses qu’on sait, ou même, comme l’art de Lulle, à parler sans jugement de celles qu’on ignore, qu’à les apprendre. Et bien qu’elle contienne en effet beaucoup de préceptes très vrais \& très bons, il y en a toutefois tant d’autres mêlés parmi, qui sont ou nuisibles ou superflus, qu’il est presque aussi malaisé de les en séparer, que de tirer une Diane ou une Minerve hors d’un bloc de marbre qui n’est point encore ébauché. \par
\pn{-}Puis, pour l’\textbf{analyse} des anciens, \& l’\textbf{algèbre} des modernes, outre qu’elles ne s’étendent qu’à des matières fort abstraites, \& qui ne semblent d’aucun usage,\par
La première est toujours si astreinte à la considération des figures, qu’elle ne peut exercer l’entendement sans fatiguer beaucoup l’imagination,\par
Et on s’est tellement assujetti en la dernière à certaines règles, \& à certains chiffres, qu’on en a fait un art confus \& obscur qui embarrasse l’esprit, au lieu d’une science qui le cultive.\par
\pn{-}Ce qui fut cause que je pensai qu’il fallait chercher quelque autre Méthode, qui comprenant les avantages de ces trois, fût exempte de leurs défauts. \par
\textbf{Et comme la multitude des lois fournit souvent des excuses aux vices ; en sorte qu’un État est bien mieux réglé, lorsque n’en ayant que fort peu, elles y sont fort étroitement observées} :\par
Ainsi, au lieu de ce grand nombre de préceptes dont la Logique est composée, je crus que j’aurais assez des quatre suivants, pourvu que je prisse une ferme \& constante résolution de ne manquer pas une seule fois à les observer.\par

\astermono

\phantomsection
\label{II7}\noindent \pn{7}\textbf{Le premier} était de ne recevoir jamais aucune chose pour vraie, que je ne la connusse évidemment être telle : c’est-à-dire, d’éviter soigneusement la Précipitation, \& la Prévention ; \& de ne comprendre rien de plus en mes jugements, que ce qui se présenterait si \textbf{clairement} \& si \textbf{distinctement} à mon esprit, que je n’eusse aucune occasion de le mettre en doute.\par
\phantomsection
\label{II8}\pn{8}\textbf{Le second}, de \textbf{diviser} chacune des difficultés que j’examinerais en autant de parcelles qu’il se pourrait, \& qu’il serait requis pour les mieux résoudre.\par
\phantomsection
\label{II9}\pn{9}\textbf{Le troisième}, de conduire par \textbf{ordre} mes pensées, en commençant par les objets les plus simples, \& les plus aisés à connaître, pour monter peu à peu comme par degrés jusques à la connaissance des plus composés : Et supposant même de l’ordre entre ceux qui ne se précèdent point naturellement les uns les autres.\par
\phantomsection
\label{II10}\pn{10}Et \textbf{le dernier}, de faire partout des \textbf{dénombrements} si entiers, \& des revues si générales, que je fusse assuré de ne rien omettre.\par

\astermono

\phantomsection
\label{II11}\noindent \pn{11}Ces longues chaînes de raisons, toutes simples \& faciles, dont les Géomètres ont coutume de se servir, pour parvenir à leurs plus difficiles démonstrations, m’avaient donné occasion de m’imaginer que toutes les choses qui peuvent tomber sous la connaissance des hommes s’entre-suivent en même façon, \& que, pourvu seulement qu’on s’abstienne d’en recevoir aucune pour vraie qui ne le soit, \& qu’on garde toujours l’ordre qu’il faut pour les déduire les unes des autres, il n’y en peut avoir de si éloignées auxquelles enfin on ne parvienne, ni de si cachées qu’on ne découvre.\par
Et je ne fus pas beaucoup en peine de chercher par lesquelles il était besoin de commencer : car je savais déjà que c’était par les plus simples \& les plus aisées à connaître ; \& considérant qu’entre tous ceux qui ont ci-devant recherché la vérité dans les sciences, il n’y a eu que les seuls Mathématiciens qui ont pu trouver quelques démonstrations, c’est-à-dire quelques raisons certaines \& évidentes, je ne doutais point que ce ne fût par les mêmes qu’ils ont examinées ; bien que je n’en espérasse aucune autre utilité, sinon qu’elles accoutumeraient mon esprit à se repaître de vérités, \& ne se contenter point de fausses raisons.\par
Mais je n’eus pas dessein, pour cela, de tâcher d’apprendre toutes ces sciences particulières, qu’on nomme communément Mathématiques : \& voyant qu’encore que leurs objets soient différents, elles ne laissent pas de s’accorder toutes, en ce qu’elles n’y considèrent autre chose que les divers rapports ou proportions qui s’y trouvent, je pensai qu’il valait mieux que j’examinasse seulement ces proportions en général, \& sans les supposer que dans les sujets qui serviraient à m’en rendre la connaissance plus aisée ; même aussi sans les y astreindre aucunement, afin de les pouvoir d’autant mieux appliquer après à tous les autres auxquels elles conviendraient.\par
Puis, ayant pris garde que pour les connaître, j’aurais quelquefois besoin de les considérer chacune en particulier ; \& quelquefois seulement de les retenir, ou de les comprendre plusieurs ensemble : je pensai que pour les considérer mieux en particulier, je les devais supposer en des lignes, à cause que je ne trouvais rien de plus simple, ni que je pusse plus distinctement représenter à mon imagination \& à mes sens ; mais que pour les retenir, ou les comprendre plusieurs ensemble, il fallait que je les expliquasse par quelques chiffres les plus courts qu’il serait possible.\par
Et que, par ce moyen, j’emprunterais tout le meilleur de l’Analyse Géométrique, \& de l’Algèbre, \& corrigerais tous les défauts de l’une par l’autre.\par
\bigbreak
\phantomsection
\label{II12}\noindent \pn{12}Comme en effet j’ose dire, que l’exacte observation de ce peu de préceptes que j’avais choisis, me donna telle facilité à démêler toutes les questions auxquelles ces deux sciences s’étendent, qu’en deux ou trois mois que j’employai à les examiner, ayant commencé par les plus simples \& plus générales, \& chaque vérité que je trouvais étant une règle qui me servait après à en trouver d’autres, non seulement je vins à bout de plusieurs que j’avais jugées autrefois très difficiles, mais il me sembla aussi vers la fin que je pouvais déterminer, en celles même que j’ignorais, par quels moyens, \& jusques où, il était possible de les résoudre.\par
En quoi je ne vous paraîtrai peut-être pas être fort vain, si vous considérez que n’y ayant qu’une vérité de chaque chose, quiconque la trouve en sait autant qu’on en peut savoir :\par
Et que, par exemple un enfant instruit en l’Arithmétique ayant fait une addition suivant ses règles, se peut assurer d’avoir trouvé, touchant la somme qu’il examinait, tout ce que l’esprit humain saurait trouver.\par
Car enfin la Méthode qui enseigne à suivre le vrai ordre, \& à dénombrer exactement toutes les circonstances de ce qu’on cherche, contient tout ce qui donne de la certitude aux règles d’Arithmétique.\par
\bigbreak
\phantomsection
\label{II13}\noindent \pn{13}Mais ce qui me contentait le plus de cette Méthode, était que par elle j’étais assuré d’user en tout de ma raison, sinon parfaitement, au moins le mieux qui fût en mon pouvoir : outre que je sentais, en la pratiquant, que mon esprit s’accoutumait peu à peu à concevoir plus nettement \& plus distinctement ses objets ; \& que ne l’ayant point assujettie à aucune matière particulière, je me promettais de l’appliquer aussi utilement aux difficultés des autres sciences, que j’avais fait à celles de l’Algèbre.\par
\pn{-}Non que pour cela j’osasse entreprendre d’abord d’examiner toutes celles qui se présenteraient, car cela même eût été contraire à l’ordre qu’elle prescrit :\par
Mais ayant pris garde que leurs principes devaient tous être empruntés de la philosophie, en laquelle je n’en trouvais point encore de certains, je pensai qu’il fallait avant tout que je tâchasse d’y en établir ; \& que cela étant la chose du monde la plus importante, \& où la Précipitation \& la Prévention étaient le plus à craindre, je ne devais point entreprendre d’en venir à bout, que je n’eusse atteint un âge bien plus mûr que celui de vingt-trois ans que j’avais alors ;\par
Et que je n’eusse auparavant employé beaucoup de temps à m’y préparer, tant en déracinant de mon esprit toutes les mauvaises opinions que j’y avais reçues avant ce temps-là, qu’en faisant amas de plusieurs expériences, pour être après la matière de mes raisonnements, \& en m’exerçant toujours en la Méthode que je m’étais prescrite, afin de m’y affermir de plus en plus.
\chapterclose


\chapteropen
\chapter[Troisième partie]{Troisième partie}\phantomsection
\label{III}\renewcommand{\leftmark}{Troisième partie}


\begin{argument}\noindent En la 3, quelques-unes de celles de la Morale qu’il a tirée de cette Méthode.
\end{argument}


\chaptercont
\phantomsection
\label{III1}\noindent \initialiv{E}{t enfin} comme ce n’est pas assez, avant de commencer à rebâtir le logis où on demeure, que de l’abattre, \& de faire provision de matériaux \& d’Architectes, ou s’exercer soi-même à l’Architecture, \& outre cela d’en avoir soigneusement tracé le dessin ; mais qu’il faut aussi s’être pourvu de quelque autre, où on puisse être logé commodément pendant le temps qu’on y travaillera. Ainsi afin que je ne demeurasse point irrésolu en mes actions, pendant que la raison m’obligerait de l’être en mes jugements, \& que je ne laissasse pas de vivre dès lors le plus heureusement que je pourrais, je me formai une morale par provision, qui ne consistait qu’en trois ou quatre maximes, dont je veux bien vous faire part.\par
\bigbreak
\phantomsection
\label{III2}\noindent \pn{2}La \textbf{première} était d’\textbf{obéir aux lois \& aux coutumes de mon pays}, retenant constamment la religion en laquelle Dieu m’a fait la grâce d’être instruit dès mon enfance, \& me gouvernant, en toute autre chose suivant les opinions les plus modérées, \& les plus éloignées de l’excès qui fussent communément reçues en pratique, par les mieux sensés de ceux avec lesquels j’aurais à vivre.\par
Car commençant dès lors à ne compter pour rien les miennes propres, à cause que je les voulais remettre toutes à l’examen, j’étais assuré de ne pouvoir mieux que de suivre celles des mieux sensés. Et encore qu’il y en ait peut-être d’aussi bien sensés parmi les Perses ou les Chinois que parmi nous, il me semblait que le plus utile était de me régler selon ceux avec lesquels j’aurais à vivre ;\par
Et que, pour savoir quelles étaient véritablement leurs opinions, je devais plutôt prendre garde à ce qu’ils pratiquaient qu’à ce qu’ils disaient ; non seulement à cause qu’en la corruption de nos mœurs il y a peu de gens qui veuillent dire tout ce qu’ils croient ; mais aussi à cause que plusieurs l’ignorent eux-mêmes, car l’action de la pensée par laquelle on croit une chose étant différente de celle par laquelle on connaît qu’on la croit, elles sont souvent l’une sans l’autre.\par
Et entre plusieurs opinions également reçues, je ne choisissais que les plus modérées ; tant à cause que ce sont toujours les plus commodes pour la pratique, \& vraisemblablement les meilleures, tous excès ayant coutume d’être mauvais ; comme aussi afin de me détourner moins du vrai chemin, en cas que je faillisse, que si ayant choisi l’un des extrêmes, c’eût été l’autre qu’il eût fallu suivre.\par
\phantomsection
\label{III2col}Et, particulièrement je mettais entre les excès toutes les promesses par lesquelles on retranche quelque chose de sa liberté :\par
Non que je désapprouvasse les lois, qui pour remédier à l’inconstance des esprits faibles, permettent lorsqu’on a quelque bon dessein, ou même pour la sûreté du commerce, quelque dessein qui n’est qu’indifférent, qu’on fasse des vœux ou des contrats qui obligent à y persévérer :\par
Mais à cause que je ne voyais au monde aucune chose qui demeurât toujours en même état, \& que pour mon particulier je me promettais de perfectionner de plus en plus mes jugements, \& non point de les rendre pires, j’eusse pensé commettre une grande faute contre le bon sens, si pour ce que j’approuvais alors quelque chose, je me fusse obligé de la prendre pour bonne encore après, lorsqu’elle aurait peut-être cessé de l’être, ou que j’aurais cessé de l’estimer telle.\par
\bigbreak
\phantomsection
\label{III3}\noindent \pn{3}Ma \textbf{seconde maxime} était d’\textbf{être le plus ferme \& le plus résolu en mes actions que je pourrais}, \& de ne suivre pas moins constamment les opinions les plus douteuses, lorsque je m’y serais une fois déterminé, que si elles eussent été très assurées.\par
Imitant en ceci les voyageurs qui se trouvant égarés en quelque forêt, ne doivent pas errer en tournoyant tantôt d’un côté tantôt d’un autre, ni encore moins s’arrêter en une place, mais marcher toujours le plus droit qu’ils peuvent vers un même côté, \& ne le changer point pour de faibles raisons, encore que ce n’ait peut-être été au commencement que le hasard seul qui les ait déterminés à le choisir : car, par ce moyen, s’ils ne vont justement où ils désirent, ils arriveront au moins à la fin quelque part, où vraisemblablement ils seront mieux que dans le milieu d’une forêt.\par
Et ainsi, les actions de la vie ne souffrant souvent aucun délai, c’est une vérité très certaine, que lorsqu’il n’est pas en notre pouvoir de discerner les plus vraies opinions, nous devons suivre les plus probables,\par
Et même, qu’encore que nous ne remarquions point davantage de probabilité aux unes qu’aux autres, nous devons néanmoins nous déterminer à quelques-unes,\par
Et les considérer après non plus comme douteuses, en tant qu’elles se rapportent à la pratique, mais comme très vraies \& très certaines, à cause que la raison qui nous y a fait déterminer se trouve telle.\par
Et ceci fut capable dès lors de me délivrer de tous les repentirs \& les remords, qui ont coutume d’agiter les consciences de ces esprits faibles \& chancelants, qui se laissent aller inconstamment à pratiquer comme bonnes, les choses qu’ils jugent après être mauvaises.\par
\bigbreak
\phantomsection
\label{III4}\noindent \pn{4}Ma \textbf{troisième maxime} était\textbf{ de tâcher toujours plutôt à me vaincre que la fortune}, \& à changer mes désirs que l’ordre du monde ;\par
Et généralement de m’accoutumer à croire qu’il n’y a rien qui soit entièrement en notre pouvoir que nos pensées, en sorte qu’après que nous avons fait notre mieux, touchant les choses qui nous sont extérieures, tout ce qui manque de nous réussir est au regard de nous absolument impossible.\par
Et ceci seul me semblait être suffisant pour m’empêcher de rien désirer à l’avenir que je n’acquisse, \& ainsi pour me rendre content :\par
Car notre volonté ne se portant naturellement à désirer que les choses que notre entendement lui représente en quelque façon comme possibles, il est certain que si nous considérons tous les biens qui sont hors de nous comme également éloignés de notre pouvoir, nous n’aurons pas plus de regrets de manquer de ceux qui semblent être dus à notre naissance, lorsque nous en serons privés sans notre faute, que nous avons de ne posséder pas les royaumes de la Chine ou du Mexique :\par
Et que faisant, comme on dit, de nécessité vertu, nous ne désirerons pas davantage d’être sains étant malades ; ou d’être libres étant en prison, que nous faisons maintenant d’avoir des corps d’une matière aussi peu corruptible que les diamants, ou des ailes pour voler comme les oiseaux.\par
Mais j’avoue qu’il est besoin d’un long exercice, \& d’une méditation souvent réitérée, pour s’accoutumer à regarder de ce biais toutes les choses :\par
Et je crois que c’est principalement en ceci, que consistait le secret de ces Philosophes, qui ont pu autrefois se soustraire de l’empire de la Fortune, \& malgré les douleurs \& la pauvreté, disputer de la félicité avec leurs Dieux.\par
Car s’occupant sans cesse à considérer les bornes qui leur étaient prescrites par la Nature, ils se persuadaient si parfaitement que rien n’était en leur pouvoir que leurs pensées, que cela seul était suffisant pour les empêcher d’avoir aucune affection pour d’autres choses ; \& ils disposaient d’elles si absolument, qu’ils avaient en cela quelque raison de s’estimer plus riches, \& plus puissants, \& plus libres, \& plus heureux, qu’aucun des autres hommes qui, n’ayant point cette Philosophie, tant favorisés de la Nature \& de la Fortune qu’ils puissent être, ne disposent jamais ainsi de tout ce qu’ils veulent.\par
\bigbreak
\phantomsection
\label{III5}\noindent \pn{5}Enfin pour \textbf{conclusion} de cette Morale je m’avisai de faire une revue sur les diverses occupations qu’ont les hommes en cette vie, pour tâcher à faire choix de la meilleure, \& sans que je veuille rien dire de celles des autres, je pensai que je ne pouvais mieux que de continuer en celle-là même où je me trouvais, c’est-à-dire, que d’\textbf{employer toute ma vie à cultiver ma raison, \& m’avancer autant que je pourrais en la connaissance de la vérité} suivant la Méthode que je m’étais prescrite.\par
J’avais éprouvé de si extrêmes contentements depuis que j’avais commencé à me servir de cette Méthode, que je ne croyais pas qu’on en pût recevoir de plus doux, ni de plus innocents, en cette vie ;\par
Et découvrant tous les jours par son moyen quelques vérités, qui me semblaient assez importantes, \& communément ignorées des autres hommes, la satisfaction que j’en avais remplissait tellement mon esprit que tout le reste ne me touchait point.\par
\pn{-}Outre que les trois maximes précédentes n’étaient fondées que sur le dessein que j’avais de continuer à m’instruire :\par
Car Dieu nous ayant donné à chacun quelque lumière pour discerner le vrai d’avec le faux, je n’eusse pas cru me devoir contenter des opinions d’autrui un seul moment, si je ne me fusse proposé d’employer mon propre jugement à les examiner, lorsqu’il serait temps ;\par
Et je n’eusse su m’exempter de scrupule en les suivant, si je n’eusse espéré de ne perdre pour cela aucune occasion d’en trouver de meilleures, en cas qu’il y en eût ;\par
Et enfin je n’eusse su borner mes désirs ni être content, si je n’eusse suivi un chemin par lequel pensant être assuré de l’acquisition de toutes les connaissances dont je serais capable, je le pensais être par même moyen de celle de tous les vrais biens qui seraient jamais en mon pouvoir :\par
D’autant que, notre volonté ne se portant à suivre ni à fuir aucune chose, que selon que notre entendement lui représente bonne ou mauvaise, il suffit de bien juger pour bien faire, \& de juger le mieux qu’on puisse, pour faire aussi tout son mieux, c’est-à-dire pour acquérir toutes les vertus, \& ensemble tous les autres biens, qu’on puisse acquérir ; \& lorsqu’on est certain que cela est, on ne saurait manquer d’être content.\par

\astermono

\phantomsection
\label{III6}\noindent \pn{6}Après m’être ainsi assuré de ces maximes, \& les avoir mises à part, avec les vérités de la foi, qui ont toujours été les premières en ma créance, je jugeai que pour tout le reste de mes opinions je pouvais librement entreprendre de m’en défaire.\par
\pn{-}Et d’autant que j’espérais en pouvoir mieux venir à bout en conversant avec les hommes, qu’en demeurant plus longtemps renfermé dans le poêle où j’avais eu toutes ces pensées, l’hiver n’était pas encore bien achevé que je me remis à voyager.\par
Et en toutes les neuf années suivantes \textbf{je ne fis autre chose que rouler çà \& là dans le monde, tâchant d’y être spectateur plutôt qu’acteur en toutes les comédies qui s’y jouent} ;\par
Et faisant particulièrement réflexion en chaque matière sur ce qui la pouvait rendre suspecte, \& nous donner occasion de nous méprendre, je déracinais cependant de mon esprit toutes les erreurs qui s’y étaient pu glisser auparavant.\par
\pn{-}Non que j’imitasse pour cela les sceptiques, qui ne doutent que pour douter, \& affectent d’être toujours irrésolus :\par
Car au contraire tout mon dessein ne tendait qu’à m’assurer, \& à rejeter la terre mouvante \& le sable, pour trouver le roc ou l’argile.\par
Ce qui me réussissait ce me semble assez bien, d’autant que tâchant à découvrir la fausseté ou l’incertitude des propositions que j’examinais, non par de faibles conjectures, mais par des raisonnements clairs \& assurés, je n’en rencontrais point de si douteuses, que je n’en tirasse toujours quelque conclusion assez certaine, quand ce n’eût été que cela même qu’elle ne contenait rien de certain.\par
Et comme en abattant un vieux logis, on en réserve ordinairement les démolitions, pour servir à en bâtir un nouveau : ainsi en détruisant toutes celles de mes opinions que je jugeais être mal fondées, je faisais diverses observations, \& acquérais plusieurs expériences, qui m’ont servi depuis à en établir de plus certaines.\par
\pn{-}Et de plus je continuais à m’exercer en la Méthode que je m’étais prescrite, car outre que j’avais soin de conduire généralement toutes mes pensées selon ses règles, je me réservais de temps en temps quelques heures que j’employais particulièrement à la pratiquer en des difficultés de Mathématique, ou même aussi en quelques autres que je pouvais rendre quasi semblables à celles des Mathématiques, en les détachant de tous les principes des autres sciences que je ne trouvais pas assez fermes, comme vous verrez que j’ai fait en plusieurs qui sont expliquées en ce volume.\par
Et ainsi, sans vivre d’autre façon en apparence, que ceux qui n’ayant aucun emploi qu’à passer une vie douce \& innocente, s’étudient à séparer les plaisirs des vices ; \& qui, pour jouir de leur loisir sans s’ennuyer, usent de tous les divertissements qui sont honnêtes, je ne laissais pas de poursuivre en mon dessein, \& de profiter en la connaissance de la vérité, peut-être plus, que si je n’eusse fait que lire des livres, ou fréquenter des gens de lettres.\par
\bigbreak
\phantomsection
\label{III7}\noindent \pn{7}Toutefois ces neuf ans s’écoulèrent avant que j’eusse encore pris aucun parti, touchant les difficultés qui ont coutume d’être disputées entre les doctes, ni commencé à chercher les fondements d’aucune Philosophie plus certaine que la vulgaire.\par
Et l’exemple de plusieurs excellents esprits, qui en ayant eu ci-devant le dessein me semblaient n’y avoir pas réussi, m’y faisait imaginer tant de difficulté, que je n’eusse peut-être pas encore sitôt osé l’entreprendre, si je n’eusse vu que quelques-uns faisaient déjà courre le bruit que j’en étais venu à bout. Je ne saurais pas dire sur quoi ils fondaient cette opinion ; \& si j’y ai contribué quelque chose par mes discours, ce doit avoir été en confessant plus ingénument ce que j’ignorais que n’ont coutume de faire ceux qui ont un peu étudié, \& peut-être aussi en faisant voir les raisons que j’avais de douter de beaucoup de choses que les autres estiment certaines ; plutôt qu’en me vantant d’aucune doctrine.\par
\pn{-}Mais ayant le cœur assez bon pour ne vouloir point qu’on me prît pour autre que je n’étais, je pensai qu’il fallait que je tâchasse par tous moyens à me rendre digne de la réputation qu’on me donnait :\par
Et il y a justement huit ans que ce désir me fit résoudre à m’éloigner de tous les lieux où je pouvais avoir des connaissances, \& à me retirer ici en un pays où la longue durée de la guerre a fait établir de tels ordres, que les armées qu’on y entretient ne semblent servir qu’à faire qu’on y jouisse des fruits de la paix avec d’autant plus de sûreté ; \& où parmi la foule d’un grand peuple fort actif, \& plus soigneux de ses propres affaires, que curieux de celles d’autrui, sans manquer d’aucune des commodités qui sont dans les villes les plus fréquentées, j’ai pu vivre aussi solitaire \& retiré que dans les déserts les plus écartés.
\chapterclose


\chapteropen
\chapter[Quatrième partie]{Quatrième partie}\phantomsection
\label{IV}\renewcommand{\leftmark}{Quatrième partie}


\begin{argument}\noindent En la 4, les raisons par lesquelles il prouve l’existence de Dieu \& de l’âme humaine, qui sont les fondements de sa Métaphysique.
\end{argument}


\chaptercont
\phantomsection
\label{IV1}\noindent \initialiv{J}{e ne sais si} je dois vous entretenir des premières \textbf{méditations} que j’y ai faites, car elles sont si \textbf{Métaphysiques}  \& si peu communes, qu’elles ne seront peut-être pas au goût de tout le monde :\par
Et toutefois, afin qu’on puisse juger si les fondements que j’ai pris sont assez fermes, je me trouve en quelque façon contraint d’en parler. J’avais dès longtemps remarqué que, pour les mœurs il est besoin quelquefois de suivre des opinions qu’on sait fort incertaines, tout de même que si elles étaient indubitables, ainsi qu’il a été dit ci-dessus : mais pour ce qu’alors je désirais vaquer seulement à la recherche de la vérité, je pensai qu’il fallait que je fisse tout le contraire, \& que je rejetasse comme absolument faux tout ce en quoi je pourrais imaginer le moindre doute, afin de voir s’il ne resterait point après cela, quelque chose en ma créance qui fût entièrement indubitable. Ainsi à cause que nos sens nous trompent quelquefois, je voulus supposer qu’il n’y avait aucune chose qui fût telle qu’ils nous la font imagine :\par
Et pour ce qu’il y a des hommes qui se méprennent en raisonnant, même touchant les plus simples matières de Géométrie, \& y font des Paralogismes, jugeant que j’étais sujet à faillir autant qu’aucun autre, je rejetai comme fausses toutes les raisons que j’avais prises auparavant pour Démonstrations :\par
Et enfin considérant que toutes les mêmes pensées que nous avons étant éveillés, nous peuvent aussi venir quand nous dormons sans qu’il y en ait aucune pour lors qui soit vraie, je me résolus de feindre que toutes les choses qui m’étaient jamais entrées en l’esprit n’étaient non plus vraies que les illusions de mes songes.\par
Mais aussitôt après je pris garde, que pendant que je voulais ainsi penser que tout était faux, il fallait nécessairement que moi qui le pensais fusse quelque chose :\par
Et remarquant que cette vérité, \emph{je pense, donc je suis}, était si ferme \& si assurée que toutes les plus extravagantes suppositions des Sceptiques n’étaient pas capables de l’ébranler, je jugeai que je pouvais la recevoir sans scrupule pour le premier principe de la philosophie que je cherchais.\par

\astermono


\labelblock{Méditation première : Des choses que l'on peut révoquer en doute\footnote{Cet intertitre et les 5 suivants sont tirés des \emph{Méditations métaphysiques} de Descartes.}}

\phantomsection
\label{IV2}\noindent \pn{2}Puis, examinant avec attention ce que j’étais, \& voyant que je pouvais feindre que je n’avais aucun corps \& qu’il n’y avait aucun monde ni aucun lieu où je fusse ; mais que je ne pouvais pas feindre pour cela que je n’étais point ; \& qu’au contraire, de cela même que je pensais à \textbf{douter} de la vérité des autres choses, il suivait très évidemment \& très certainement que j’étais : au lieu que si j’eusse seulement cessé de penser, encore que tout le reste de ce que j’avais jamais imaginé eût été vrai, je n’avais aucune raison de croire que j’eusse été :\par
Je connus de là que j’étais une substance dont toute l’essence ou la nature n’est que de penser, \& qui pour être n’a besoin d’aucun lieu ni ne dépend d’aucune chose matérielle,\par
En sorte que ce Moi, c’est-à-dire l’Âme par laquelle je suis ce que je suis, est entièrement distincte du corps, \& même qu’elle est plus aisée à connaître que lui, \& qu’encore qu’il ne fût point, elle ne laisserait pas d’être tout ce qu’elle est.\par

\labelblock{Méditation seconde : De la nature de l'esprit humain, et qu'il est plus aisé à connaître que le corps}

\phantomsection
\label{IV3}\noindent \pn{3}Après cela je considérai en général ce qui est requis à une proposition pour être vraie \& certaine ; car puisque je venais d’en trouver une que je savais être telle, je pensai que je devais aussi savoir en quoi consiste cette certitude. Et ayant remarqué qu’il n’y a rien du tout en ceci : \textbf{je pense, donc je suis}, qui m’assure que je dis la vérité, sinon que je vois très clairement que, pour penser, il faut être :\par
Je jugeai que je pouvais prendre pour règle générale, que les choses que nous concevons fort clairement \& fort distinctement sont toutes vraies ; mais qu’il y a seulement quelque difficulté à bien remarquer quelles sont celles que nous concevons distinctement.\par

\labelblock{Méditation troisième : De Dieu ; qu'il existe.}

\phantomsection
\label{IV4}\noindent \pn{4}En suite de quoi, faisant réflexion sur ce que je doutais, \& que par conséquent mon être n’était pas tout parfait ; car je voyais clairement que c’était une plus grande perfection de connaître que de douter :\par
Je m’avisai de chercher d’où j’avais appris à penser à quelque chose de plus parfait que je n’étais ; \& je connus évidemment que ce devait être de quelque nature qui fût en effet plus parfaite. Pour ce qui est des pensées que j’avais de plusieurs autres choses hors de moi, comme du ciel, de la terre, de la lumière, de la chaleur, \& de mille autres, je n’étais point tant en peine de savoir d’où elles venaient, à cause que, ne remarquant rien en elles qui me semblât les rendre supérieures à moi, je pouvais croire que, si elles étaient vraies, c’étaient des dépendances de ma nature, en tant qu’elle avait quelque perfection ; \& si elles ne l’étaient pas, que je les tenais du néant, c’est-à-dire qu’elles étaient en moi, parce que j’avais du défaut. Mais ce ne pouvait être le même de l’idée d’un être plus parfait que le mien :\par
Car, de la tenir du néant, c’était chose manifestement impossible ;\par
Et parce qu’il n’y a pas moins de répugnance que le plus parfait soit une suite \& une dépendance du moins parfait, qu’il y en a que de rien procède quelque chose, je ne la pouvais tenir non plus de moi-même ;\par
De façon qu’il restait qu’elle eût été mise en moi par une nature qui fût véritablement plus parfaite que je n’étais, \& même qui eût en soi toutes les perfections dont je pouvais avoir quelque idée, c’est-à-dire, pour m’expliquer en un mot, qui fût \textbf{Dieu}.\par
À quoi j’ajoutai que puisque ; je connaissais quelques perfections que je n’avais point, je n’étais pas le seul être qui existât (j’userai s’il vous plaît, ici librement des mots de l’École) Mais qu’il fallait de nécessité qu’il y en eût quelque autre plus parfait, duquel je dépendisse, \& duquel j’eusse acquis tout ce que j’avais :\par
Car, si j’eusse été seul \& indépendant de tout autre, en sorte que j’eusse eu de moi-même tout ce peu que je participais de l’être parfait, j’eusse pu avoir de moi par même raison tout le surplus que je connaissais me manquer, \& ainsi être moi-même infini, éternel, immuable, tout connaissant, tout puissant, \& enfin avoir toutes les perfections que je pouvais remarquer être en Dieu. \par
Car suivant les raisonnements que je viens de faire, pour connaître la nature de Dieu autant que la mienne en était capable, je n’avais qu’à considérer de toutes les choses dont je trouvais en moi quelque idée, si c’était perfection, ou non, de les posséder, \& j’étais assuré qu’aucune de celles qui marquaient quelque imperfection n’était en lui, mais que toutes les autres y étaient. Comme je voyais que le doute, l’inconstance, la tristesse, \& choses semblables, n’y pouvaient être, vu que j’eusse été moi-même bien aise d’en être exempt. Puis outre cela j’avais des idées de plusieurs choses sensibles \& corporelles : car quoique je supposasse que je rêvais, \& que tout ce que je voyais ou imaginais était faux, je ne pouvais nier toutefois que les idées n’en fussent véritablement en ma pensée :\par
Mais pour ce que j’avais déjà connu en moi très clairement que la nature intelligente est distincte de la corporelle, considérant que toute composition témoigne de la dépendance, \& que la dépendance est manifestement un défaut, je jugeais de là, que ce ne pouvait être une perfection en Dieu d’être composé de ces deux natures, \& que par conséquent, il ne l’était pas ;\par
Mais que, s’il y avait quelques corps dans le monde, ou bien quelques intelligences, ou autres natures, qui ne fussent point toutes parfaites, leur être devait dépendre de sa puissance, en telle sorte qu’elles ne pouvaient subsister sans lui un seul moment.\par

\labelblock{Méditation quatrième : Du vrai et du faux.}

\phantomsection
\label{IV5}\noindent \pn{5}Je voulus chercher après cela d’autres \textbf{vérités}, \& m’étant proposé l’objet des Géomètres, que je concevais comme un corps continu, ou un espace indéfiniment étendu en longueur largeur \& hauteur ou profondeur, divisible en diverses parties, qui pouvaient avoir diverses figures, \& grandeurs, \& être mues ou transposées en toutes sortes, car les Géomètres supposent tout cela du leur objet, je parcourus quelques-unes de leurs plus simples démonstrations ;\par
Et ayant pris garde que cette grande certitude, que tout le monde leur attribue, n’est fondée que sur ce qu’on les conçoit évidemment, suivant la règle que j’ai tantôt dite ;\par
Je pris garde aussi qu’il n’y avait rien du tout en elles qui m’assurât de l’existence de leur objet :\par
Car par exemple je voyais bien, que supposant un triangle il fallait que ses trois angles fussent égaux à deux droits, mais je ne voyais rien pour cela qui m’assurât qu’il y eût au monde aucun triangle :\par
Au lieu que, revenant à examiner l’idée que j’avais d’un Être parfait, je trouvais que l’existence y était comprise, en même façon qu’il est compris en celles d’un triangle que ses trois angles sont égaux à deux droits, ou en celle d’une Sphère que toutes ses parties sont également distantes de son centre, ou même encore plus évidemment,\par
Et que, par conséquent, il est pour le moins aussi certain, que Dieu, qui est cet Être parfait, est ou existe, qu’aucune démonstration de géométrie le saurait être.\par
\bigbreak
\phantomsection
\label{IV6}\noindent \pn{6}Mais ce qui fait qu’il y en a plusieurs qui se persuadent qu’il y a de la difficulté à le connaître, \& même aussi à connaître ce que c’est que leur âme, c’est qu’ils n’élèvent jamais leur esprit au delà des choses sensibles, \& qu’ils sont tellement accoutumés à ne rien considérer qu’en l’imaginant, qui est une façon de penser particulière pour les choses matérielles, que tout ce qui n’est pas imaginable leur semble n’être pas intelligible.\par
Ce qui est assez manifeste de ce que même les Philosophes tiennent pour maxime dans les Écoles, qu’il n’y a rien dans l’entendement qui n’ait premièrement été dans le sens, où toutefois il est certain que les idées de Dieu \& de l’âme n’ont jamais été,\par
Et il me semble que ceux qui veulent user de leur imagination pour les comprendre, font tout de même que si pour ouïr les sons, ou sentir les odeurs, ils se voulaient servir de leurs yeux :\par
Sinon qu’il y a encore cette différence, que le sens de la vue ne nous assure pas moins de la vérité de ses objets, que font ceux de l’odorat ou de l’ouïe ; au lieu que ni notre imagination ni nos sens ne nous sauraient jamais assurer d’aucune chose, si notre entendement n’y intervient.\par

\labelblock{Méditation cinquième : De l'essence des choses matérielles ; et, derechef de Dieu, qu'il existe.}

\phantomsection
\label{IV7}\noindent \pn{7}Enfin s’il y a encore des hommes, qui ne soient pas assez persuadés de l’existence de \textbf{Dieu} \& de leur âme, par les raisons que j’ai apportées,\par
Je veux bien qu’ils sachent que toutes les autres choses, dont ils se pensent peut-être plus assurés, comme d’avoir un corps, \& qu’il y a des astres, \& une terre, \& choses semblables, sont moins certaines :\par
Car encore qu’on ait une assurance morale de ces choses, qui est telle, qu’il semble qu’à moins que d’être extravagant, on n’en peut douter ;\par
Toutefois aussi à moins que d’être déraisonnable, lorsqu’il est question d’une certitude métaphysique, on ne peut nier, que ce ne soit assez de sujet pour n’en être pas entièrement assuré, que d’avoir pris garde qu’on peut en même façon, s’imaginer étant endormi qu’on a un autre corps, \& qu’on voit d’autres astres, \& une autre terre, sans qu’il en soit rien.\par
Car d’où sait-on que les pensées qui viennent en songe sont plutôt fausses que les autres, vu que souvent elles ne sont pas moins vives \& expresses ? Et que les meilleurs Esprits y étudient tant qu’il leur plaira, je ne crois pas qu’ils puissent donner aucune raison qui soit suffisante pour ôter ce doute ; s’ils ne présupposent l’existence de Dieu.\par
Car premièrement cela même que j’ai tantôt pris pour une règle, à savoir que les choses que nous concevons très clairement \& très distinctement sont toutes vraies, n’est assuré qu’à cause que Dieu est ou existe, \& qu’il est un être parfait, \& que tout ce qui est en nous vient de lui :\par
D’où il suit que nos idées ou notions, étant des choses réelles, \& qui viennent de Dieu, en tout ce en quoi elles sont claires \& distinctes, ne peuvent en cela être que vraies.\par
En sorte que si nous en avons assez souvent qui contiennent de la fausseté, ce ne peut être que de celles, qui ont quelque chose de confus \& obscur, à cause qu’en cela elles participent du néant, c’est-à-dire, qu’elles ne sont en nous ainsi confuses qu’à cause que nous ne sommes pas tout parfaits.\par
Et il est évident qu’il n’y a pas moins de répugnance que la fausseté ou l’imperfection procède de Dieu en tant que telle, qu’il y en a que la vérité ou la perfection procède du néant. Mais si nous ne savions point que tout ce qui est en nous de réel, \& de vrai, vient d’un être parfait \& infini, pour claires \& distinctes que fussent nos idées, nous n’aurions aucune raison qui nous assurât, qu’elles eussent la perfection d’être vraies.\par

\labelblock{Méditation sixième : De l'existence des choses matérielles ; et de la réelle distinction entre l'âme et le corps de l'homme.}

\phantomsection
\label{IV8}\noindent \pn{8}Or, après que la connaissance de Dieu \& de l’âme nous a ainsi rendus certains de cette règle, il est bien aisé à connaître que les rêveries que nous imaginons étant endormis, ne doivent aucunement nous faire douter de la vérité des pensées que nous avons étant éveillés. Car s’il arrivait même en dormant qu’on eût quelque idée fort distincte, comme, par exemple qu’un Géomètre inventât quelque nouvelle démonstration, son sommeil ne l’empêcherait pas d’être vraie :\par
Et pour l’erreur la plus ordinaire de nos songes, qui consiste en ce qu’ils nous représentent divers objets en même façon que font nos sens extérieurs, n’importe pas qu’elle nous donne occasion de nous défier de la vérité de telles idées, à cause qu’elles peuvent aussi nous tromper assez souvent, sans que nous dormions : comme lorsque ceux qui ont la jaunisse voient tout de couleur jaune, ou que les astres ou autres corps fort éloignes nous paraissent beaucoup plus petits qu’ils ne sont. Car enfin, soit que nous veillions, soit que nous dormions, nous ne nous devons jamais laisser persuader qu’à l’évidence de notre raison. Et il est à remarquer que je dis, de notre raison, \& non point, de notre imagination ni de nos sens. Comme encore que nous voyons le soleil très clairement, nous ne devons pas juger pour cela qu’il ne soit que de la grandeur que nous le voyons ;\par
Et nous pouvons bien imaginer distinctement une tête de lion entée\footnote{greffée} sur le corps d’une chèvre, sans qu’il faille conclure, pour cela, qu’il y ait au monde une Chimère :\par
Car la raison ne nous dicte point que ce que nous voyons ou imaginons ainsi soit véritable. Mais elle nous dicte bien que toutes nos idées ou notions doivent avoir quelque fondement de vérité, car il ne serait pas possible que Dieu qui est tout parfait \& tout véritable les eût mises en nous sans cela ;\par
Et pour ce que nos raisonnements ne sont jamais si évidents ni si entiers pendant le sommeil que pendant la veille, bien que quelquefois nos imaginations soient alors autant ou plus vives \& expresses, elle nous dicte aussi que nos pensées ne pouvant être toutes vraies, à cause que nous ne sommes pas tout parfaits, ce qu’elles ont de vérité doit infailliblement se rencontrer en celles que nous avons étant éveillés, plutôt qu’en nos songes.
\chapterclose


\chapteropen
\chapter[Cinquième partie]{Cinquième partie}\phantomsection
\label{V}\renewcommand{\leftmark}{Cinquième partie}


\begin{argument}\noindent En la 5, l’ordre des questions de Physique qu’il a cherchées, \& particulièrement l’explication du mouvement du cœur \& de quelques autres difficultés qui appartiennent à la Médecine, puis aussi la différence qui est entre notre âme \& celle des bêtes.
\end{argument}


\chaptercont
\phantomsection
\label{V1}\noindent \initialiv{J}{e serais bien aise} de poursuivre, \& de faire voir ici toute la chaîne des autres vérités que j’ai déduites de ces premières :\par
Mais à cause que, pour cet effet, il serait maintenant besoin que je parlasse de plusieurs questions, qui sont en controverse entre les doctes\footnote{1633, condamnation de Galilée par l’Église de Rome, pour avoir défendu le système \emph{héliocentrique} (que la terre tourne autour du soleil).}, avec lesquels je ne désire point me brouiller, je crois qu’il sera mieux que je m’en abstienne\footnote{1632, Descartes avait écrit le \emph{Traité du monde et de la lumière}, défendant entre autres, la thèse de l’héliocentrisme, en la rendant compatible avec les dogmes de l’Église de Rome. Après la condamnation de Galilée, il renonce à publier son traité du monde, qui n’est paru qu’après sa mort (1664). Cette cinquième partie du \emph{discours de la Méthode} en détaille les contenus, à part ce qui concerne la cosmologie.} ; \& que je dise seulement en général quelles elles sont, afin de laisser juger aux plus sages, s’il serait utile que le public en fût plus particulièrement informé.\par
Je suis toujours demeuré ferme en la résolution que j’avais prise, de ne supposer aucun autre principe, que celui dont je viens de me servir pour démontrer l’existence de Dieu \& de l’âme, \& de ne recevoir aucune chose pour vraie, qui ne me semblât plus claire \& plus certaine que n’avaient fait auparavant les démonstrations des Géomètres :\par
Et néanmoins j’ose dire, que non seulement j’ai trouvé moyen de me satisfaire en peu de temps, touchant toutes les principales difficultés dont on a coutume de traiter en la Philosophie ;\par
Mais aussi que j’ai remarqué certaines lois, que Dieu a tellement établies en la nature, \& dont il a imprimé de telles notions en nos âmes, qu’après y avoir fait assez de réflexion, nous ne saurions douter qu’elles ne soient exactement observées, en tout ce qui est ou qui se fait dans le monde.\par
Puis en considérant la suite de ces lois, il me semble avoir découvert plusieurs vérités plus utiles \& plus importantes, que tout ce que j’avais appris auparavant, ou même espéré d’apprendre.\par
\bigbreak
\phantomsection
\label{V2}\noindent \pn{2}Mais pour ce que j’ai tâché d’en expliquer les principales dans un traité, que quelques considérations m’empêchent de publier, je ne les saurais mieux faire connaître, qu’en disant ici sommairement ce qu’il contient.\par
\pn{-}J’ai eu dessein d’y comprendre tout ce que je pensais savoir, avant que de l’écrire, touchant la nature des choses matérielles :\par
Mais, tout de même que les peintres, ne pouvant également bien représenter dans un tableau plat toutes les diverses faces d’un corps solide, en choisissent une des principales qu’ils mettent seule vers le jour, \& ombrageant les autres, ne les font paraître qu’en tant qu’on les peut voir en la regardant :\par
Ainsi craignant de ne pouvoir mettre en mon discours tout ce que j’avais en la pensée, j’entrepris seulement d’y exposer bien amplement ce que je concevais de la lumière ;\par
Puis à son occasion d’y ajouter quelque chose du Soleil \& des Étoiles fixes, à cause qu’elle en procède presque toute ; des cieux, à cause qu’ils la transmettent ; des Planètes, des Comètes, \& de la terre, à cause qu’elles la font réfléchir ; \& en particulier de tous les corps qui sont sur la terre, à cause qu’ils sont ou colorés, ou transparents, ou lumineux, \& enfin de l’Homme, à cause qu’il en est le spectateur.\par
Même pour ombrager un peu toutes ces choses, \& pouvoir dire plus librement ce que j’en jugeais, sans être obligé de suivre ni de réfuter les opinions qui sont reçues entre les doctes, je me résolus de laisser tout ce monde ici à leurs disputes, \& de parler seulement de ce qui arriverait dans un nouveau, si Dieu créait maintenant quelque part, dans les espaces imaginaires assez de matière pour le composer, \& qu’il agitât diversement \& sans ordre les diverses parties de cette matière, en sorte qu’il en composât un Chaos aussi confus que les Poètes en puissent feindre,\par
Et que par après il ne fît autre chose que prêter son concours ordinaire à la Nature, \& la laisser agir suivant les lois qu’il a établies.\par
\pn{-}Ainsi premièrement je décrivis cette \textbf{matière}, \& tâchai de la représenter telle qu’il n’y a rien au monde ce me semble, de plus clair ni plus intelligible, excepté ce qui a tantôt été dit de Dieu \& de l’âme :\par
Car même je supposai expressément, qu’il n’y avait en elle aucune de ces formes ou qualités dont on dispute dans les Écoles, ni généralement aucune chose, dont la connaissance ne fût si naturelle à nos âmes, qu’on ne pût pas même feindre de l’ignorer.\par
\pn{-}De plus, je fis voir quelles étaient les \textbf{lois de la Nature} ;\par
Et sans appuyer mes raisons sur aucun autre principe que sur les perfections infinies de Dieu, je tâchai à démontrer toutes celles dont on eût pu avoir quelque doute,\par
Et à faire voir qu’elles sont telles, qu’encore que Dieu aurait créé plusieurs mondes, il n’y en saurait avoir aucun où elles manquassent d’être observées.\par
\pn{-}Après cela je montrai comment la plus grande part de la matière de ce Chaos devait, en suite de ces lois, se disposer \& s’arranger d’une certaine façon qui la rendait semblable à nos cieux ;\par
Comment, cependant, quelques-unes de ses parties devaient composer une Terre, \& quelques-unes des \textbf{Planètes} \& des Comètes, \& quelques autres un Soleil \& des Étoiles fixes :\par
Et ici m’étendant sur le sujet de la lumière, j’expliquai bien au long quelle était celle qui se devait trouver dans le Soleil \& les \textbf{Étoiles},\par
Et comment de là elle traversait en un instant les immenses espaces des cieux,\par
Et comment elle se réfléchissait des planètes \& des comètes vers la terre.\par
\pn{-}J’y ajoutai aussi plusieurs choses, touchant la\textbf{ substance}, la \textbf{situation}, les \textbf{mouvements}, \& toutes les diverses qualités de ces cieux \& de ces astres ;\par
En sorte que je pensais en dire assez pour faire connaître qu’il ne se remarque rien en ceux de ce monde, qui ne dût, ou du moins qui ne pût, paraître tout semblable en ceux du monde que je décrivais.\par
\pn{-}De là je vins à parler particulièrement de la \textbf{Terre} :\par
Comment, encore que j’eusse expressément supposé que Dieu n’avait mis aucune pesanteur en la matière dont elle était composée, toutes ses parties ne laissaient pas de tendre exactement vers son centre :\par
Comment, y ayant de l’eau \& de l’air sur sa superficie, la disposition des cieux \& des astres, principalement de la lune y devait causer un flux \& reflux, qui fût semblable en toutes ses circonstances, à celui qui se remarque dans nos mers ;\par
Et outre cela un certain cours, tant de l’eau que de l’air, du levant vers le couchant tel qu’on le remarque aussi entre les Tropiques :\par
Comment les montagnes, les mers, les fontaines \& les rivières pouvaient naturellement s’y former ;\par
Et les métaux y venir dans les mines ;\par
Et les plantes y croître dans les campagnes ;\par
Et généralement tous les corps qu’on nomme mêlés ou composés s’y engendrer :\par
Et entre autres choses à cause qu’après les astres je ne connais rien au monde que le feu qui produise de la lumière je m’étudiai à faire entendre bien clairement tout ce qui appartient à sa nature, comment il se fait, comment il se nourrit, comment il n’a quelquefois que de la chaleur sans lumière, \& quelquefois que de la lumière sans chaleur, comment il peut introduire diverses couleurs en divers corps, \& diverses autres qualités, comment il en fond quelques-uns, \& en durcit d’autres, comment il les peut consumer presque tous, ou convertir en cendres \& en fumée ;\par
Et enfin comment de ces cendres par la seule violence de son action il forme du verre :\par
Car cette transmutation de cendres en verre me semblant être aussi admirable qu’aucune autre qui se fasse en la nature, je pris particulièrement plaisir à la décrire.\par
\bigbreak
\phantomsection
\label{V3}\noindent \pn{3}Toutefois, je ne voulais pas inférer, de toutes ces choses, que ce monde ait été créé en la façon que je proposais :\par
Car il est bien plus vraisemblable que dès le commencement Dieu l’a rendu tel qu’il devait être.\par
Mais il est certain, \& c’est une opinion communément reçue entre les Théologiens, que \textbf{l’action par laquelle maintenant il le conserve, est toute la même que celle par laquelle il l’a créé\footnote{Descartes évoque ici sa théorie de la \emph{création continuée}, selon laquelle Dieu a non seulement créé le monde, mais en plus il continue à le faire durer. Le cartésianisme est un mysticisme radical. Il suppose que le temps est une série d’instants que seul Dieu continue. Ainsi par exemple, Descartes constate et admet la conservation du mouvement d’un corps dans un monde sans frottement, l’inertie d’une boule de billard par exemple, mais au lieu d’en attribuer comme nous la cause à la masse du corps en mouvement (son énergie), il en fait la preuve que Dieu est partout et conduit tout dans sa création. C’est par cet artifice que Descartes pensait sauver la théorie de Galilée tout en augmentant la majesté divine. Ces thèses cartésiennes ont plus tard justifié des résistances à la physique de Newton (1642, 1727).}} :\par
De façon qu’encore qu’il ne lui aurait point donné au commencement, d’autre forme que celle du Chaos, pourvu qu’ayant établi les lois de la Nature, il lui prêtât son concours pour agir ainsi qu’elle a de coutume, on peut croire, sans faire tort au miracle de la création, que par cela seul toutes les choses qui sont purement matérielles auraient pu avec le temps s’y rendre telles que nous les voyons à présent :\par
Et leur nature est bien plus aisée à concevoir lorsqu’on les voit naître peu à peu en cette sorte, que lorsqu’on ne les considère que toutes faites.\par
\bigbreak
\phantomsection
\label{V4}\noindent \pn{4}De la description des corps inanimés \& des plantes, je passai à celle des animaux \& particulièrement à celle des hommes. Mais pour ce que je n’en avais pas encore assez de connaissance pour en parler du même style que du reste, c’est-à-dire, en démontrant les effets par les causes, \& faisant voir de quelles semences, \& en quelle façon la Nature les doit produire,\par
Je me contentai de supposer, que Dieu formât le corps d’un homme, entièrement semblable à l’un des nôtres, tant en la figure extérieure de ses membres, qu’en la conformation intérieure de ses organes, sans le composer d’autre matière que de celle que j’avais décrite, \& sans mettre en lui, au commencement, aucune âme raisonnable, ni aucune autre chose pour y servir d’âme végétante ou sensitive,\par
Sinon qu’il excitait en son cœur un de ces feux sans lumière que j’avais déjà expliqués, \& que je ne concevais point d’autre nature que celui qui échauffe le foin, lorsqu’on l’a renfermé avant qu’il fût sec, ou qui fait bouillir les vins nouveaux, lorsqu’on les laisse cuver sur la râpe.\par
\pn{-}Car examinant les fonctions, qui pouvaient en suite de cela être en ce corps, j’y trouvais exactement toutes celles, qui peuvent être en nous sans que nous y pensions, ni par conséquent que notre âme, c’est-à-dire, cette partie distincte du corps dont il a été dit ci-dessus que la nature n’est que de penser, y contribue,\par
Et qui sont toutes les mêmes, en quoi on peut dire que les animaux sans raison nous ressemblent :\par
Sans que j’y en pusse pour cela trouver aucune, de celles qui, étant dépendantes de la pensée, sont les seules qui nous appartiennent en tant qu’hommes ;\par
Au lieu que je les y trouvais toutes par après, ayant supposé que Dieu créât une âme raisonnable, \& qu’il la joignît à ce corps en certaine façon que je décrivais.\par
\bigbreak
\phantomsection
\label{V5}\noindent \pn{5}Mais, afin qu’on puisse voir en quelle sorte j’y traitais cette matière, je veux mettre ici l’explication du \textbf{mouvement du cœur \& des artères}, qui étant le premier \& le plus général qu’on observe dans les animaux, on jugera facilement de lui ce qu’on doit penser de tous les autres.\par
Et afin qu’on ait moins de difficulté à entendre ce que j’en dirai, je voudrais que ceux qui ne sont point versés dans l’Anatomie prissent la peine, avant que de lire ceci, de faire couper devant eux le cœur de quelque grand animal qui ait des poumons, car il est en tous assez semblable à celui de l’homme ;\par
Et qu’il se fissent montrer les deux chambres ou concavités qui y sont,\par
\pn{-}Premièrement, celle qui est dans son \textbf{côté droit}, à laquelle répondent deux tuyaux fort larges :\par
À savoir la veine cave, qui est le principal réceptacle du sang, \& comme le tronc de l’arbre dont toutes les autres veines du corps sont les branches ;\par
Et la veine artérieuse\footnote{Artère pulmonaire.}, qui a été ainsi mal nommée pour ce que c’est en effet une artère, laquelle prenant son origine du cœur, se divise, après en être sortie, en plusieurs branches qui se vont répandre partout dans les poumons.\par
\pn{-}Puis, celle qui est dans son \textbf{côté gauche}, à laquelle répondent en même façon deux tuyaux, qui sont autant ou plus larges que les précédents :\par
À savoir l’artère veineuse\footnote{Veine pulmonaire.}, qui a été aussi mal nommée à cause qu’elle n’est autre chose qu’une veine, laquelle vient des poumons, où elle est divisée en plusieurs branches, entrelacées avec celles de la veine artérieuse, \& celles de ce conduit qu’on nomme le sifflet, par où entre l’air de la respiration ;\par
Et la grande artère, qui sortant du cœur envoie ses branches par tout le corps.\par
\pn{-}Je voudrais aussi qu’on leur montrât soigneusement les onze \textbf{petites peaux}, qui comme autant de petites portes ouvrent \& ferment les quatre ouvertures qui sont en ces deux concavités :\par
À savoir, trois à l’entrée de la veine cave, où elles sont tellement disposées, qu’elles ne peuvent aucunement empêcher que le sang qu’elle contient ne coule dans la concavité droite du cœur, \& toutefois empêchent exactement qu’il n’en puisse sortir ;\par
Trois à l’entrée de la veine artérieuse, qui étant disposées tout au contraire, permettent bien au sang, qui est dans cette concavité, de passer dans les poumons, mais non pas à celui qui est dans les poumons d’y retourner ;\par
Et ainsi deux autres à l’entrée de l’artère veineuse, qui laissent couler le sang des poumons vers la concavité gauche du cœur, mais s’opposent à son retour ;\par
Et trois à l’entrée de la grande artère, qui lui permettent de sortir du cœur, mais l’empêchent d’y retourner.\par
Et il n’est point besoin de chercher d’autre raison du nombre de ces peaux, sinon que l’ouverture de l’artère veineuse, étant en ovale à cause du lieu où elle se rencontre, peut être commodément fermée avec deux, au lieu que les autres, étant rondes, le peuvent mieux être avec trois.\par
\pn{-}De plus, je voudrais qu’on leur fît considérer que la grande\textbf{ artère} \& la veine artérieuse sont d’une composition beaucoup plus dure \& plus ferme, que ne sont l’artère veineuse \& la \textbf{veine} cave ;\par
Et que ces deux dernières s’élargissent avant que d’entrer dans le cœur, \& y font comme deux bourses, nommées les oreilles du cœur, qui sont composées d’une chair semblable à la sienne ;\par
\pn{-}Et qu’il y a toujours plus de chaleur dans le cœur qu’en aucun autre endroit du corps ;\par
\pn{-}Et enfin, que cette chaleur est capable de faire, que s’il entre quelque goutte de sang en ses concavités, elle s’enfle promptement \& se dilate, ainsi que font généralement toutes les liqueurs, lorsqu’on les laisse tomber goutte à goutte en quelque vaisseau qui est fort chaud.\par
\bigbreak
\phantomsection
\label{V6}\noindent \pn{6}Car, après cela je n’ai besoin de dire autre chose pour expliquer le mouvement du cœur, sinon que, lorsque ses concavités ne sont pas pleines de sang, il y en coule nécessairement de la veine cave dans la droite, \& de l’artère veineuse dans la gauche ;\par
D’autant que ces deux vaisseaux en sont toujours pleins, \& que leurs ouvertures, qui regardent vers le cœur, ne peuvent alors être bouchées.\par
\pn{-}Mais que sitôt qu’il est entré ainsi deux gouttes de sang, une en chacune de ses concavités,\par
Ces gouttes, qui ne peuvent être que fort grosses, à cause que les ouvertures par où elles entrent sont fort larges, \& les vaisseaux d’où elles viennent fort pleins de sang, se raréfient \& se dilatent, à cause de la chaleur qu’elles y trouvent,\par
Au moyen de quoi, faisant enfler tout le cœur, elles poussent \& ferment les cinq petites portes qui sont aux entrées des deux vaisseaux d’où elles viennent, empêchant ainsi qu’il ne descende davantage de sang dans le cœur ;\par
Et continuant à se raréfier de plus en plus, elles poussent \& ouvrent les six autres petites portes qui sont aux entrées des deux autres vaisseaux par où elles sortent, faisant enfler par ce moyen toutes les branches de la veine artérieuse \& de la grande artère, quasi au même instant que le cœur,\par
Lequel incontinent après se désenfle, comme font aussi ces artères, à cause que le sang qui y est entré s’y refroidit, \& leurs six petites portes se referment, \& les cinq de la veine cave \& de l’artère veineuse se rouvrent, \& donnent passage à deux autres gouttes de sang, qui font derechef enfler le cœur \& les artères, tout de même que les précédentes.\par
Et pour ce que le sang, qui entre ainsi dans le cœur, passe par ces deux bourses qu’on nomme ses oreilles, de là vient que leur mouvement est contraire au sien, \& qu’elles se désenflent lorsqu’il s’enfle.\par
\pn{-}Au reste afin que ceux qui ne connaissent pas la force des démonstrations Mathématiques, \& ne sont pas accoutumés à distinguer les vraies raisons des vraisemblables, ne se hasardent pas de nier ceci sans l’examiner,\par
Je les veux avertir que ce mouvement que je viens d’expliquer, suit aussi nécessairement de la seule disposition des organes qu’on peut voir à l’œil dans le cœur, \& de la chaleur qu’on y peut sentir avec les doigts, \& de la nature du sang qu’on peut connaître par expérience,\par
Que fait celui d’une horloge, de la force, de la situation \& de la figure de ses contrepoids \& de ses roues.\par
\bigbreak
\phantomsection
\label{V7}\noindent \pn{7}Mais si on demande comment le sang des veines ne s’épuise point, en coulant ainsi continuellement dans le cœur, \& comment les artères n’en sont point trop remplies, puisque tout celui qui passe par le cœur s’y va rendre,\par
Je n’ai pas besoin d’y répondre autre chose, que ce qui a déjà été écrit par un \textbf{médecin d’Angleterre\footnote{Harvey, William (1578, 1657), \emph{De Motu Cordis} (1628) « le mouvement du cœur »}} auquel il faut donner la louange d’avoir rompu la glace en cet endroit, \& d’être le premier qui a enseigné, qu’il y a plusieurs petits passages aux extrémités des artères, par où le sang qu’elles reçoivent du cœur entre dans les petites branches des veines, d’où il se va rendre derechef vers le cœur,\par
En sorte que son cours n’est autre chose qu’une circulation perpétuelle.\par
\pn{-}Ce qu’il prouve fort bien, par l’expérience ordinaire des Chirurgiens, qui ayant lié le bras médiocrement fort, au-dessus de l’endroit où ils ouvrent la veine, font que le sang en sort plus abondamment que s’ils ne l’avaient point lié :\par
Et il arriverait tout le contraire, s’ils le liaient au-dessous entre la main \& l’ouverture, ou bien qu’ils le liassent très fort au-dessus.\par
Car il est manifeste que le lien médiocrement serré, pouvant empêcher que le sang qui est déjà dans le bras ne retourne vers le cœur par les veines, n’empêche pas pour cela qu’il n’y en vienne toujours de nouveau par les artères :\par
À cause qu’elles sont situées au-dessous des veines ;\par
Et que leurs peaux étant plus dures sont moins aisées à presser ;\par
Et aussi que le sang qui vient du cœur tend avec plus de force à passer par elles vers la main, qu’il ne fait à retourner de là vers le cœur par les veines ;\par
Et puisque ce sang sort du bras par l’ouverture qui est en l’une des veines, il doit nécessairement y avoir quelques passages au-dessous du lien, c’est-à-dire, vers les extrémités du bras, par où il y puisse venir des artères.\par
\pn{-}Il prouve aussi fort bien ce qu’il dit du cours du sang, par certaines \textbf{petites peaux}, qui sont tellement disposées en divers lieux le long des veines, qu’elles ne lui permettent point d’y passer du milieu du corps vers les extrémités, mais seulement de retourner des extrémités vers le cœur ;\par
Et de plus par l’expérience qui montre, que tout celui qui est dans le corps en peut sortir en fort peu de temps par une seule artère lorsqu’elle est coupée, encore même qu’elle fût étroitement liée fort proche du cœur, \& coupée entre lui \& le lien,\par
En sorte qu’on n’eût aucun sujet d’imaginer que le sang qui en sortirait vînt d’ailleurs.\par
\bigbreak
\phantomsection
\label{V8}\noindent \pn{8}Mais il y a plusieurs autres choses qui témoignent que la vraie cause de ce mouvement du sang est celle que j’ai dite.\par
\pn{-}Comme premièrement la différence, qu’on remarque entre celui qui sort des \textbf{veines} \& celui qui sort des \textbf{artères}, ne peut procéder que de ce qu’étant raréfié, \& comme distillé, en passant par le cœur, il est plus subtil \& plus vif \& plus chaud incontinent après en être sorti, c’est-à-dire, étant dans les artères, qu’il n’est un peu devant que d’y entrer, c’est-à-dire, étant dans les veines :\par
Et si on y prend garde, on trouvera que cette différence ne paraît bien que vers le cœur, \& non point tant aux lieux qui en sont les plus éloignés.\par
Puis la dureté des peaux, dont la veine artérieuse \& la grande artère sont composées, montre assez, que le sang bat contre elles avec plus de force que contre les veines. Et pourquoi la concavité gauche du cœur \& la grande artère, seraient-elles plus amples \& plus larges que la concavité droite \& la veine artérieuse ? Si ce n’était que le sang de l’artère veineuse, n’ayant été que dans les poumons depuis qu’il a passé par le cœur, est plus subtil \& se raréfie plus fort \& plus aisément que celui qui vient immédiatement de la veine cave. Et qu’est-ce que les médecins peuvent deviner en tâtant le pouls, s’ils ne savent que selon que le sang change de nature, il peut être raréfié par la chaleur du cœur plus ou moins fort, \& plus ou moins vite qu’auparavant.\par
\pn{-}Et si on examine comment cette \textbf{chaleur} se communique aux autres membres, ne faut-il pas avouer que c’est par le moyen du sang, qui passant par le cœur s’y réchauffe, \& se répand de là par tout le corps :\par
D’où vient que, si on ôte le sang de quelque partie, on en ôte par même moyen la chaleur ;\par
Et encore que le cœur fût aussi ardent qu’un fer embrasé, il ne suffirait pas pour réchauffer les pieds \& les mains tant qu’il fait, s’il n’y envoyait continuellement de nouveau sang.\par
Puis aussi on connaît de là, que le vrai usage de la respiration, est d’apporter assez d’air frais dans le poumon, pour faire que le sang, qui y vient de la concavité droite du cœur, où il a été raréfié \& comme changé en vapeurs, s’y épaississe, \& convertisse en sang derechef, avant que de retomber dans la gauche ; sans quoi il ne pourrait être propre à servir de nourriture au feu qui y est.\par
Ce qui se confirme parce qu’on voit que les animaux qui n’ont point de poumons, n’ont aussi qu’une seule concavité dans le cœur ;\par
Et que les enfants, qui n’en peuvent user pendant qu’ils sont renfermés au ventre de leurs mères, ont une ouverture par où il coule du sang de la veine cave en la concavité gauche du cœur,\par
Et un conduit par où il en vient de la veine artérieuse en la grande artère, sans passer par le poumon.\par
\pn{-}Puis la coction, comment se ferait-elle en l’\textbf{estomac} ? si le cœur n’y envoyait de la chaleur par les artères, \& avec cela quelques-unes des plus coulantes parties du sang qui aident à dissoudre les viandes qu’on y a mises.\par
Et l’action qui convertit le suc de ces viandes en sang, n’est-elle pas aisée à connaître, si on considère qu’il se distille, en passant \& repassant par le cœur, peut-être par plus de cent ou deux cents fois en chaque jour.\par
Et qu’a-t-on besoin d’autre chose, pour expliquer la nutrition, \& la production des diverses humeurs qui sont dans le corps, sinon de dire que la force, dont le sang en se raréfiant passe du cœur vers les extrémités des artères, fait que quelques-unes de ses parties s’arrêtent entre celles des membres où elles se trouvent, \& y prennent la place de quelques autres qu’elles en chassent ;\par
Et que selon la situation, ou la figure, ou la petitesse des pores qu’elles rencontrent, les unes se vont rendre en certains lieux plutôt que les autres ;\par
En même façon que chacun peut avoir vu divers cribles, qui étant diversement percés servent à séparer divers grains les uns des autres.\par
\pn{-}Et enfin ce qu’il y a de plus remarquable en tout ceci, c’est la génération des \textbf{esprits animaux}, qui sont comme un vent très subtil, ou plutôt comme une flamme très pure \& très vive, qui montant continuellement en grande abondance du cœur dans le cerveau, se va rendre de là par les nerfs dans les muscles, \& donne le mouvement à tous les membres :\par
Sans qu’il faille imaginer d’autre cause, qui fasse que les parties du sang, qui étant les plus agitées \& les plus pénétrantes sont les plus propres à composer ces esprits, se vont rendre plutôt vers le cerveau que vers ailleurs ;\par
Sinon que les artères, qui les y portent, sont celles qui viennent du cœur le plus en ligne droite de toutes ;\par
Et que selon les règles des Mécaniques ; qui sont les mêmes que celles de la nature, lorsque plusieurs choses tendent ensemble à se mouvoir vers un même côté où il n’y a pas assez de place pour toutes, ainsi que les parties du sang qui sortent de la concavité gauche du cœur tendent vers le cerveau, les plus faibles \& moins agitées en doivent être détournées par les plus fortes, qui par ce moyen s’y vont rendre seules.\par

\astermono

\phantomsection
\label{V9}\noindent \pn{9}J’avais expliqué assez particulièrement toutes ces choses dans le traité que j’avais eu ci-devant dessein de publier. Et ensuite j’y avais montré, quelle doit être la fabrique des \textbf{nerfs} \& des \textbf{muscles} du corps humain, pour faire que les esprits animaux, étant dedans, aient la force de mouvoir ses membres :\par
Ainsi qu’on voit que les têtes, un peu après être coupées, se remuent encore, \& mordent la terre, nonobstant qu’elles ne soient plus animées ;\par
Quels changements se doivent faire dans le cerveau pour causer la veille, \& le sommeil, \& les songes ;\par
Comment la lumière, les sons, les odeurs, les goûts, la chaleur, \& toutes les autres qualités des objets extérieurs y peuvent imprimer diverses idées par l’entremise des sens ;\par
Comment la faim, la soif, \& les autres passions intérieures, y peuvent aussi envoyer les leurs ;\par
Ce qui doit y être pris pour le sens commun, où ces idées sont reçues ; pour la mémoire, qui les conserve ; \& pour la fantaisie, qui les peut diversement changer, \& en composer de nouvelles, \& par même moyen, distribuant les esprits animaux dans les muscles, faire mouvoir les membres de ce corps, en autant de diverses façons, \& autant à propos des objets qui se présentent à ses sens, \& des passions intérieures qui sont en lui, que les nôtres se puissent mouvoir, sans que la volonté les conduise.\par
Ce qui ne semblera nullement étrange, à ceux qui, sachant combien de divers automates, ou machines mouvantes, l’industrie des hommes peut faire, sans y employer que fort peu de pièces, à comparaison de la grande multitude des os, des muscles, des nerfs, des artères, des veines, \& de toutes les autres parties qui sont dans le corps de chaque animal,\par
\textbf{Considéreront ce corps comme une machine}, qui, ayant été faite des mains de Dieu, est incomparablement mieux ordonnée, \& a en soi des mouvements plus admirables, qu’aucune de celles qui peuvent être inventées par les hommes.\par
\pn{-}Et je m’étais ici particulièrement arrêté à faire voir, que s’il y avait de telles machines, qui eussent les organes \& la figure d’un singe, ou de quelque autre animal sans raison nous n’aurions aucun moyen pour reconnaître, qu’elles ne seraient pas en tout de même nature que ces animaux :\par
Au lieu que s’il y en avait qui eussent la ressemblance de nos corps, \& imitassent autant nos actions que moralement il serait possible, nous aurions toujours deux moyens très certains pour reconnaître qu’elles ne seraient point pour cela de vrais hommes.\par
\pn{-}Dont le premier est que jamais elles ne pourraient user de \textbf{paroles}, ni d’autres signes en les composant, comme nous faisons pour déclarer aux autres nos pensées.\par
Car on peut bien concevoir, qu’une machine soit tellement faite qu’elle profère des paroles, \& même qu’elle en profère quelques-unes à propos des actions corporelles qui causeront quelque changement en ses organes :\par
Comme si on la touche en quelque endroit, qu’elle demande ce qu’on lui veut dire ; si en un autre, qu’elle crie qu’on lui fait mal, \& choses semblables ;\par
Mais non pas qu’elle les arrange diversement, pour répondre au sens de tout ce qui se dira en sa présence, ainsi que les hommes les plus hébétés peuvent faire.\par
\pn{-}Et le second est, que bien qu’elles fissent plusieurs choses, aussi bien, ou peut-être mieux, qu’aucun de nous, elles manqueraient infailliblement en quelques autres, par lesquelles on découvrirait qu’elles n’agiraient pas par \textbf{connaissance}, mais seulement par la disposition de leurs organes :\par
Car au lieu que la raison est un instrument universel, qui peut servir en toutes sortes de rencontres, ces organes ont besoin de quelque particulière disposition pour chaque action particulière ; d’où vient qu’il est moralement impossible, qu’il y en ait assez de divers en une machine, pour la faire agir en toutes les occurrences de la vie, de même façon que notre raison nous fait agir.\par
\pn{-}Or, par ces deux mêmes moyens, on peut aussi connaître \textbf{la différence, qui est entre les hommes \& les bêtes}. Car c’est une chose bien remarquable, qu’il n’y a point d’hommes si hébétés \& si stupides, sans en excepter même les insensés, qu’ils ne soient capables d’arranger ensemble diverses paroles, \& d’en composer un discours par lequel ils fassent entendre leurs pensées ;\par
Et qu’au contraire, il n’y a point d’autre animal, tant parfait \& tant heureusement né qu’il puisse être, qui fasse le semblable.\par
Ce qui n’arrive pas de ce qu’ils ont faute d’organes, car on voit que les pies \& les perroquets peuvent proférer des paroles ainsi que nous, \& toutefois ne peuvent parler ainsi que nous, c’est-à-dire en témoignant qu’ils pensent ce qu’ils disent ;\par
Au lieu que les hommes, qui étant nés sourds \& muets, sont privés des organes qui servent aux autres pour parler ; autant ou plus que les bêtes, ont coutume d’inventer d’eux-mêmes quelques signes, par lesquels ils se font entendre à ceux qui, étant ordinairement avec eux, ont loisir d’apprendre leur langue.\par
\pn{-}Et ceci ne témoigne pas seulement que les bêtes ont moins de raison que les hommes, mais qu’elles n’en ont point du tout :\par
Car on voit qu’il n’en faut que fort peu pour savoir parler ; \& d’autant qu’on remarque de l’inégalité entre les animaux d’une même espèce, aussi bien qu’entre les hommes, \& que les uns sont plus aisés à dresser que les autres, il n’est pas croyable qu’un singe ou un perroquet, qui serait des plus parfaits de son espèce, n’égalât en cela un enfant des plus stupides, ou du moins un enfant qui aurait le cerveau troublé, si leur âme n’était d’une nature du tout différente de la nôtre.\par
Et on ne doit pas confondre les paroles, avec les mouvements naturels, qui témoignent les passions, \& peuvent être imités par des machines aussi bien que par les animaux : ni penser, comme quelques Anciens, que les bêtes parlent, bien que nous n’entendions pas leur langage : car s’il était vrai, puisqu’elles ont plusieurs organes qui se rapportent aux nôtres, elles pourraient aussi bien se faire entendre à nous qu’à leurs semblables.\par
C’est aussi une chose fort remarquable, que bien qu’il y ait plusieurs animaux qui témoignent plus d’industrie que nous en quelques-unes de leurs actions, on voit toutefois que les mêmes n’en témoignent point du tout en beaucoup d’autres :\par
De façon que ce qu’ils font mieux que nous, ne prouve pas qu’ils ont de l’esprit, car à ce compte ils en auraient plus qu’aucun de nous, \& feraient mieux en toute chose ;\par
Mais plutôt qu’ils n’en ont point, \& que c’est la Nature qui agit en eux, selon la disposition de leurs organes :\par
Ainsi qu’on voit qu’une horloge, qui n’est composée que de roues \& de ressorts, peut compter les heures, \& mesurer le temps, plus justement que nous avec toute notre prudence.\par
\bigbreak
\phantomsection
\label{V10}\noindent \pn{10}J’avais décrit après cela l’âme raisonnable, \& fait voir qu’elle ne peut aucunement être tirée de la puissance de la matière, ainsi que les autres choses dont j’avais parlé, mais qu’elle doit expressément être créée ;\par
Et comment il ne suffit pas, qu’elle soit \textbf{logée dans le corps humain ainsi qu’un pilote en son navire}, sinon peut-être pour mouvoir ses membres,\par
Mais qu’il est besoin qu’elle soit jointe, \& unie plus étroitement avec lui, pour avoir, outre cela des sentiments, \& des appétits semblables aux nôtres, \& ainsi composer un vrai homme.\par
Au reste je me suis ici un peu étendu sur le sujet de l’âme, à cause qu’il est des plus importants :\par
Car après l’erreur de ceux qui nient Dieu, laquelle je pense avoir ci-dessus assez réfutée, il n’y en a point qui éloigne plutôt les esprits faibles du droit chemin de la vertu, que d’imaginer que l’âme des bêtes soit de même nature que la nôtre, \& que par conséquent nous n’avons rien à craindre, ni à espérer, après cette vie, non plus que les mouches \& les fourmis :\par
Au lieu que lorsqu’on sait combien elles diffèrent, on comprend beaucoup mieux les raisons, qui prouvent que la nôtre est d’une nature entièrement indépendante du corps, \& par conséquent qu’elle n’est point sujette à mourir avec lui : puis, d’autant qu’on ne voit point d’autres causes qui la détruisent, on est naturellement porté à juger de là qu’elle est immortelle.
\chapterclose


\chapteropen
\chapter[Sixième partie]{Sixième partie}\renewcommand{\leftmark}{Sixième partie}


\begin{argument}\noindent Et en la dernière, quelles choses il croit être requises pour aller plus avant en la recherche de la Nature qu’il n’a été, \& quelles raisons l’ont fait écrire.
\end{argument}


\chaptercont
\phantomsection
\label{VI1}\noindent \pn{1}Or, il y a maintenant trois ans que j’étais parvenu à la fin du traité qui contient toutes ces choses, \& que je commençais à le revoir afin de le mettre entre les mains d’un imprimeur,\par
Lorsque j’appris que des personnes à qui je défère ; \& dont l’autorité ne peut guère moins sur mes actions, que ma propre raison sur mes pensées, avaient désapprouvé une opinion de Physique publiée un peu auparavant par quelque autre\footnote{Galilée, 1633, \emph{ Dialogue sur les deux grands systèmes du monde} (systèmes géocentrique de Ptolémée, et héliocentrique de Copernic).}, de laquelle je ne veux pas dire que je fusse, mais bien que je n’y avais rien remarqué, avant leur censure, que je pusse imaginer être préjudiciable ni à la religion ni à l’état, ni par conséquent qui m’eût empêché de l’écrire, si la raison me l’eût persuadée ;\par
Et que cela me fit craindre qu’il ne s’en trouvât tout de même quelqu’une entre les miennes, en laquelle je me fusse mépris : nonobstant le grand soin que j’ai toujours eu, de n’en point recevoir de nouvelles en ma créance, dont je n’eusse des démonstrations très certaines ; \& de n’en point écrire qui pussent tourner au désavantage de personne. Ce qui a été suffisant pour m’obliger à changer la résolution que j’avais eue de les publier.\par
Car encore que les raisons, pour lesquelles je l’avais prise auparavant, fussent très fortes, mon inclination, qui m’a toujours fait haïr le métier de faire des livres, m’en fit incontinent trouver assez d’autres pour m’en excuser. Et ces raisons de part \& d’autre sont telles, que non seulement j’ai ici quelque intérêt de les dire, mais peut-être aussi que le public en a de les savoir.\par
\bigbreak
\phantomsection
\label{VI2}\noindent \pn{2}Je n’ai jamais fait beaucoup d’état des choses qui venaient de mon esprit, \& pendant que je n’ai recueilli d’autres fruits de la méthode dont je me sers, sinon que je me suis satisfait touchant quelques difficultés qui appartiennent aux sciences spéculatives, ou bien que j’ai tâché de régler mes mœurs par les raisons qu’elle m’enseignait, je n’ai point cru être obligé d’en rien écrire.\par
Car, pour ce qui touche les mœurs, chacun abonde si fort en son sens, qu’il se pourrait trouver autant de réformateurs que de têtes, s’il était permis à d’autres qu’à ceux que Dieu a établis pour souverains sur ses peuples, ou bien auxquels il a donné assez de grâce \& de zèle pour être prophètes, d’entreprendre d’y rien changer ;\par
\textbf{Et bien que mes spéculations me plussent fort, j’ai cru que les autres en avaient aussi qui leur plaisaient peut-être davantage.}\par
\pn{-}Mais sitôt que j’ai eu acquis quelques notions générales touchant la Physique, \& que commençant à les éprouver en diverses difficultés particulières, j’ai remarqué jusques où elles peuvent conduire, \& combien elles diffèrent des principes dont on s’est servi jusques à présent, j’ai cru que je ne pouvais les tenir cachées, sans pécher grandement contre la loi qui nous oblige à procurer, autant qu’il est en nous, le bien général de tous les hommes :\par
Car elles m’ont fait voir qu’il est possible de parvenir à des connaissances qui soient fort utiles à la vie ;\par
Et qu’au lieu de cette Philosophie spéculative qu’on enseigne dans les écoles, on en peut trouver une pratique, par laquelle connaissant la force \& les actions du feu, de l’eau, de l’air, des astres, des cieux \& de tous les autres corps qui nous environnent, aussi distinctement que nous connaissons les divers métiers de nos artisans, nous les pourrions employer en même façon à tous les usages auxquels ils sont propres, \& ainsi \textbf{nous rendre comme maîtres \& possesseurs de la Nature}.\par
\pn{-}Ce qui n’est pas seulement à désirer pour l’invention d’une infinité d’artifices, qui feraient qu’on jouirait sans aucune peine des fruits de la terre, \& de toutes les commodités qui s’y trouvent :\par
Mais principalement aussi pour la conservation de la santé, laquelle est sans doute le premier bien, \& le fondement de tous les autres biens de cette vie ;\par
Car même l’esprit dépend si fort du tempérament, \& de la disposition des organes du corps, que s’il est possible de trouver quelque moyen, qui rende communément les hommes plus sages, \& plus habiles qu’ils n’ont été jusques ici, je crois que c’est dans la Médecine qu’on doit le chercher.\par
\pn{-}Il est vrai que celle qui est maintenant en usage contient peu de choses dont l’utilité soit si remarquable ;\par
Mais, sans que j’aie aucun dessein de la mépriser, je m’assure qu’il n’y a personne, même de ceux qui en font profession, qui n’avoue que tout ce qu’on y sait n’est presque rien, a comparaison de ce qui reste à y savoir, \& qu’on se pourrait exempter d’une infinité de maladies, tant du corps que de l’esprit, \& même aussi peut-être de l’affaiblissement de la vieillesse, si on avait assez de connaissance de leurs causes, \& de tous les remèdes dont la Nature nous a pourvus.\par
\pn{-}Or, ayant dessein d’employer toute ma vie à la recherche d’une science si nécessaire, \& ayant rencontré un chemin qui me semble tel qu’on doit infailliblement la trouver en le suivant, si ce n’est qu’on en soit empêché, ou par la brièveté de la vie, ou par le défaut des expériences, Je jugeais qu’il n’y avait point de meilleur remède contre ces deux empêchements, que de communiquer fidèlement au public tout le peu que j’aurais trouvé, \& de convier les bons esprits à tâcher de passer plus outre, en contribuant, chacun selon son inclination \& son pouvoir, aux expériences qu’il faudrait faire, \& communiquant aussi au public toutes les choses qu’ils apprendraient, afin que les derniers commençant où les précédents auraient achevé, \& ainsi, joignant les vies \& les travaux de plusieurs, nous allassions tous ensemble beaucoup plus loin, que chacun en particulier ne saurait faire.\par
\bigbreak
\phantomsection
\label{VI3}\noindent \pn{3}Même je remarquais, touchant les expériences, qu’elles sont d’autant plus nécessaires qu’on est plus avancé en connaissance.\par
\pn{-}Car pour le commencement, il vaut mieux ne se servir que de celles qui se présentent d’elles-mêmes a nos sens, \& que nous ne saurions ignorer pourvu que nous y fassions tant soit peu de réflexion, que d’en chercher de plus rares \& étudiées : \par
Dont la raison est que ces plus rares trompent souvent, lorsqu’on ne sait pas encore les causes des plus communes ; \& que les circonstances dont elles dépendent sont quasi toujours si particulières, \& si petites, qu’il est très malaisé de les remarquer.\par
Mais l’ordre que j’ai tenu en ceci a été tel.\par
\pn{-}Premièrement j’ai tâché de trouver en général les principes ou premières causes de tout ce qui est ou qui peut être dans le monde, sans rien considérer pour cet effet que Dieu seul qui l’a créé, ni les tirer d’ailleurs que de certaines semences de vérités qui sont naturellement en nos âmes.\par
Après cela j’ai examiné quels étaient les premiers \& plus ordinaires effets qu’on pouvait déduire de ces causes ;\par
Et il me semble que par là j’ai trouvé des cieux, des astres, une terre, \& même sur la terre, de l’eau, de l’air, du feu, des minéraux, \& quelques autres telles choses, qui sont les plus communes de toutes, \& les plus simples, \& par conséquent les plus aisées à connaître.\par
\pn{-}Puis lorsque j’ai voulu descendre à celles qui étaient plus particulières, il s’en est tant présenté à moi de diverses, que je n’ai pas cru qu’il fût possible à l’esprit humain de distinguer les formes ou espèces de corps qui sont sur la terre, d’une infinité d’autres qui pourraient y être si c’eût été le vouloir de Dieu de les y mettre ;\par
Ni par conséquent de les rapporter à notre usage, si ce n’est qu’on vienne au-devant des causes par les effets, \& qu’on se serve de plusieurs expériences particulières.\par
\pn{-}En suite de quoi repassant mon esprit sur tous les objets qui s’étaient jamais présentés à mes sens, j’ose bien dire que je n’y ai remarqué aucune chose que je ne pusse assez commodément expliquer par les principes que j’avais trouvés :\par
Mais il faut aussi que j’avoue, que la puissance de la Nature est si ample, \& si vaste, \& que ces principes sont si simples \& si généraux, que je ne remarque quasi plus aucun effet particulier, que d’abord je ne connaisse qu’il peut en être déduit en plusieurs diverses façons ;\par
Et que ma plus grande difficulté est d’ordinaire de trouver en laquelle de ces façons il en dépend,\par
Car à cela je ne sais point d’autre expédient, que de chercher derechef quelques expériences, qui soient telles, que leur événement ne soit pas le même si c’est en l’une de ces façons qu’on doit l’expliquer, que si c’est en l’autre.\par
\pn{-}Au reste j’en suis maintenant là, que je vois ce me semble assez bien de quel biais on se doit prendre à faire la plupart de celles qui peuvent servir à cet effet :\par
Mais je vois aussi qu’elles sont telles \& en si grand nombre, que ni mes mains, ni mon revenu, bien que j’en eusse mille fois plus que je n’en ai, ne sauraient suffire pour toutes :\par
En sorte que, selon que j’aurai désormais la commodité d’en faire plus ou moins, j’avancerai aussi plus ou moins en la connaissance de la Nature.\par
\pn{-}Ce que je me promettais de faire connaître par le traité que j’avais écrit, \& d’y montrer si clairement l’utilité que le public en peut recevoir, que j’obligerais tous ceux qui désirent en général le bien des hommes, c’est-à-dire, tous ceux qui sont en effet vertueux, \& non point par faux semblant, ni seulement par opinion, tant à me communiquer celles qu’ils ont déjà faites, qu’à m’aider en la recherche de celles qui restent à faire.\par
\bigbreak
\phantomsection
\label{VI4}\noindent \pn{4}Mais j’ai eu depuis ce temps-là d’autres raisons qui m’ont fait changer d’opinion, \& penser que je devais véritablement continuer d’écrire toutes les choses que je jugerais de quelque importance, à mesure que j’en découvrirais la vérité, \& y apporter le même soin que si je les voulais faire imprimer :\par
Tant afin d’avoir d’autant plus d’occasion de les bien examiner ;\par
Comme sans doute on regarde toujours de plus près à ce qu’on croit devoir être vu par plusieurs, qu’à ce qu’on ne fait que pour soi-même,\par
Et souvent les choses, qui m’ont semblé vraies lorsque j’ai commencé à les concevoir, m’ont paru fausses lorsque je les ai voulu mettre sur le papier ;\par
Qu’afin de ne perdre aucune occasion de profiter au public si j’en suis capable, \& que si mes écrits valent quelque chose, ceux qui les auront après ma mort en puissent user, ainsi qu’il sera le plus à propos.\par
\pn{-}Mais que je ne devais aucunement consentir qu’ils fussent publiés pendant ma vie, afin que ni les oppositions \& controverses auxquelles ils seraient peut-être sujets, ni même la réputation telle quelle qu’ils me pourraient acquérir, ne me donnassent aucune occasion de perdre le temps que j’ai dessein d’employer à m’instruire.\par
Car bien qu’il soit vrai que chaque homme est obligé de procurer autant qu’il est en lui le bien des autres, \& que c’est proprement ne valoir rien que de n’être utile à personne ;\par
Toutefois il est vrai aussi que nos soins se doivent étendre plus loin que le temps présent, \& qu’il est bon d’omettre les choses qui apporteraient peut-être quelque profit à ceux qui vivent, lorsque c’est à dessein d’en faire d’autres qui en apportent davantage à nos neveux.\par
\pn{-}Comme, en effet, je veux bien qu’on sache que le peu que j’ai appris jusqu’ici n’est presque rien, à comparaison de ce que j’ignore, \& que je ne désespère pas de pouvoir apprendre :\par
\textbf{Car c’est quasi le même de ceux qui découvrent peu à peu la vérité dans les sciences, que de ceux qui commençant à devenir riches ont moins de peine à faire de grandes acquisitions, qu’ils n’ont eu auparavant étant plus pauvres à en faire de beaucoup moindres.}\par
\pn{-}Ou bien on peut les comparer aux chefs d’armée, dont les forces ont coutume de croître à proportion de leurs victoires, \& qui ont besoin de plus de conduite pour se maintenir après la perte d’une bataille, qu’ils n’ont, après l’avoir gagnée à prendre des villes \& des provinces.\par
Car c’est véritablement donner des batailles, que de tâcher à vaincre toutes les difficultés \& les erreurs qui nous empêchent de parvenir à la connaissance de la vérité ; \& c’est en perdre une, que de recevoir quelque fausse opinion, touchant une matière un peu générale \& importante :\par
Il faut après beaucoup plus d’adresse pour se remettre au même état qu’on était auparavant, qu’il ne faut à faire de grands progrès, lorsqu’on a déjà des principes qui sont assurés. Pour moi, si j’ai ci-devant trouvé quelques vérités dans les sciences (\& j’espère que les choses qui sont contenues en ce volume feront juger que j’en ai trouvé quelques-unes) je puis dire que ce ne sont que des suites \& des dépendances de cinq ou six principales difficultés que j’ai surmontées, \& que je compte pour autant de batailles où j’ai eu l’heur de mon côté :\par
Même je ne craindrai pas de dire que je pense n’avoir plus besoin d’en gagner que deux ou trois autres semblables pour venir entièrement à bout de mes desseins ;\par
Et que mon âge n’est point si avancé, que selon le cours ordinaire de la Nature, je ne puisse encore avoir assez de loisir pour cet effet.\par
\pn{-}Mais je crois être d’autant plus obligé à ménager le temps qui me reste, que j’ai plus d’espérance de le pouvoir bien employer ;\par
Et j’aurais sans doute plusieurs occasions de le perdre, si je publiais les fondements de ma Physique.\par
Car, encore qu’ils soient presque tous si évidents, qu’il ne faut que les entendre pour les croire, \& qu’il n’y en ait aucun, dont je ne pense pouvoir donner des démonstrations, toutefois à cause qu’il est impossible qu’ils soient accordants avec toutes les diverses opinions des autres hommes, je prévois que je serais souvent diverti par les oppositions qu’ils feraient naître.\par
\bigbreak
\phantomsection
\label{VI5}\noindent \pn{5}On peut dire que ces oppositions seraient utiles, tant afin de me faire connaître mes fautes, qu’afin que si j’avais quelque chose de bon, les autres en eussent par ce moyen plus d’intelligence, \&, comme plusieurs peuvent plus voir qu’un homme seul, que commençant dès maintenant à s’en servir, ils m’aidassent aussi de leurs inventions.\par
\pn{-}Mais, encore que je me reconnaisse extrêmement sujet à faillir, \& que je ne me fie quasi jamais aux premières pensées qui me viennent, toutefois l’expérience que j’ai des objections qu’on me peut faire m’empêche d’en espérer aucun profit :\par
Car j’ai déjà souvent éprouvé les jugements, tant de ceux que j’ai tenus pour mes amis, que de quelques autres à qui je pensais être indifférent, \& même aussi de quelques-uns dont je savais que la malignité \& l’envie tâcheraient assez à découvrir ce que l’affection cacherait à mes amis ;\par
Mais il est rarement arrivé qu’on m’ait objecté quelque chose que je n’eusse point du tout prévue, si ce n’est qu’elle fût fort éloignée de mon sujet.\par
En sorte que je n’ai quasi jamais rencontré aucun Censeur de mes opinions qui ne me semblât ou moins rigoureux, ou moins équitable que moi-même.\par
\pn{-}Et je n’ai jamais remarqué non plus que, par le moyen des disputes qui se pratiquent dans les Écoles, on ait découvert aucune vérité qu’on ignorât auparavant.\par
Car pendant que chacun tâche de vaincre, on s’exerce bien plus à faire valoir la vraisemblance, qu’à peser les raisons de part \& d’autre :\par
Et \textbf{ceux qui ont été longtemps bons avocats, ne sont pas pour cela par après, meilleurs juges}.\par
\bigbreak
\phantomsection
\label{VI6}\noindent \pn{6}Pour l’utilité que les autres recevraient de la communication de mes pensées, elle ne pourrait aussi être fort grande, d’autant que je ne les ai point encore conduites si loin, qu’il ne soit besoin d’y ajouter beaucoup de choses, avant que de les appliquer à l’usage.\par
\pn{-}Et je pense pouvoir dire sans vanité, que s’il y a quelqu’un qui en soit capable, ce doit être plutôt moi qu’aucun autre :\par
Non pas qu’il ne puisse y avoir au monde plusieurs esprits incomparablement meilleurs que le mien ; mais pour ce qu’\textbf{on ne saurait si bien concevoir une chose, \& la rendre sienne, lorsqu’on l’apprend de quelque autre, que lorsqu’on l’invente soi-même}.\par
\pn{-}Ce qui est si véritable en cette matière, que bien que j’aie souvent expliqué quelques-unes de mes opinions à des personnes de très bon esprit, \& qui pendant que je leur parlais semblaient les entendre fort distinctement, toutefois lorsqu’ils les ont redites, j’ai remarqué qu’ils les ont changées presque toujours en telle sorte que je ne les pouvais plus avouer pour miennes.\par
À l’occasion de quoi je suis bien aise de prier ici nos neveux de ne croire jamais que les choses qu’on leur dira viennent de moi, lorsque je ne les aurai point moi-même divulguées :\par
\pn{-}Et je ne m’étonne aucunement des extravagances qu’on attribue à tous ces anciens Philosophes dont nous n’avons point les écrits, ni ne juge pas pour cela que leurs pensées aient été fort déraisonnables, vu qu’ils étaient des meilleurs esprits de leurs temps ;\par
Mais seulement qu’on nous les a mal rapportées.\par
Comme on voit aussi que presque jamais il n’est arrivé qu’aucun de leurs sectateurs les ait surpassés :\par
Et je m’assure que les plus passionnés de ceux qui suivent maintenant Aristote, se croiraient heureux, s’ils avaient autant de connaissance de la Nature qu’il a en eu, encore même que ce fût à condition qu’ils n’en auraient jamais davantage.\par
\textbf{Ils sont comme le lierre qui ne tend point à monter plus haut que les arbres qui le soutiennent, \& même souvent qui redescend après qu’il est parvenu jusques à leur faîte} :\par
Car il me semble aussi que ceux-la redescendent, c’est-à-dire, se rendent en quelque façon moins savants que s’ils s’abstenaient d’étudier, lesquels non contents de savoir tout ce qui est intelligiblement expliqué dans leur Auteur ; veulent outre cela y trouver la solution de plusieurs difficultés, dont il ne dit rien, \& auxquelles il n’a peut-être jamais pensé.\par
\pn{-}Toutefois, leur façon de philosopher est fort commode, pour ceux qui n’ont que des esprits fort médiocres : car l’obscurité des distinctions \& des principes dont ils se servent, est cause qu’ils peuvent parler de toutes choses aussi hardiment que s’ils les savaient, \& soutenir tout ce qu’ils en disent contre les plus subtils \& les plus habiles, sans qu’on ait moyen de les convaincre :\par
En quoi \textbf{ils me semblent pareils à un aveugle, qui pour se battre sans désavantage contre un qui voit, l’aurait fait venir dans le fond de quelque cave fort obscure} :\par
Et je puis dire que ceux-ci ont intérêt que je m’abstienne de publier les principes de la Philosophie dont je me sers, car étant très simples \& très évidents, comme ils sont, je ferais quasi le même en les publiant, que si j’ouvrais quelques fenêtres, \& faisais entrer du jour dans cette cave où ils sont descendus pour se battre.\par
Mais même les meilleurs esprits n’ont pas occasion de souhaiter de les connaître : car s’ils veulent savoir parler de toutes choses, \& acquérir la réputation d’être doctes, ils y parviendront plus aisément en se contentant de la vraisemblance, qui peut être trouvée sans grande peine en toutes sortes de matières ; qu’en cherchant la vérité, qui ne se découvre que peu à peu en quelques-unes, \& qui lorsqu’il est question de parler des autres oblige à confesser franchement qu’on les ignore.\par
Que s’ils préfèrent la connaissance de quelque peu de vérités à la vanité de paraître n’ignorer rien, comme sans doute elle est bien préférable, \& qu’ils veuillent suivre un dessein semblable au mien, ils n’ont pas besoin pour cela que je leur dise rien davantage que ce que j’ai dit en ce discours. Car s’ils sont capables de passer plus outre que je n’ai fait, ils le seront aussi à plus forte raison, de trouver d’eux-mêmes tout ce que je pense avoir trouvé.\par
\pn{-}D’autant que, n’ayant jamais rien examiné que par ordre, il est certain que ce qui me reste encore à découvrir est de soi plus difficile \& plus caché, que ce que j’ai pu ci-devant rencontrer, \& ils auraient bien moins de plaisir à l’apprendre de moi que d’eux-mêmes :\par
Outre que l’habitude qu’ils acquerront, en cherchant premièrement des choses faciles, \& passant peu à peu par degrés à d’autres plus difficiles, leur servira plus, que toutes mes instructions ne sauraient faire.\par
Comme pour moi je me persuade, que si on m’eût enseigné dès ma jeunesse toutes les vérités dont j’ai cherché depuis les démonstrations, \& que je n’eusse eu aucune peine à les apprendre, je n’en aurais peut-être jamais su aucunes autres, \& du moins que jamais je n’aurais acquis l’habitude, \& la facilité que je pense avoir, d’en trouver toujours de nouvelles, à mesure que je m’applique à les chercher.\par
\pn{-}Et en un mot s’il y a au monde quelque ouvrage, qui ne puisse être si bien achevé par aucun autre, que par le même qui l’a commencé, c’est celui auquel je travaille.\par
\bigbreak
\phantomsection
\label{VI7}\noindent \pn{7}Il est vrai que pour ce qui est des expériences qui peuvent y servir, un homme seul ne saurait suffire à les faire toutes ;\par
Mais il n’y saurait aussi employer utilement d’autres mains que les siennes, sinon celles des artisans, ou telles gens qu’il pourrait payer, \& à qui l’espérance du gain, qui est un moyen très efficace, ferait faire exactement toutes les choses qu’il leur prescrirait.\par
Car pour les volontaires, qui par curiosité ou désir d’apprendre, s’offriraient peut-être de lui aider, outre qu’ils ont pour l’ordinaire plus de promesses que d’effet, \& qu’ils ne font que de belles propositions dont aucune jamais ne réussit,\par
Ils voudraient infailliblement être payés par l’explication de quelques difficultés, ou du moins par des compliments \& des entretiens inutiles, qui ne lui sauraient coûter si peu de son temps qu’il n’y perdît.\par
\pn{-}Et pour les expériences que les autres ont déjà faites, quand bien même ils les lui voudraient communiquer, ce que ceux qui les nomment des secrets ne feraient jamais, elles sont pour la plupart composées de tant de circonstances, ou d’ingrédients superflus, qu’il lui serait très malaisé d’en déchiffrer la vérité :\par
Outre qu’il les trouverait presque toutes si mai expliquées, ou même si fausses, à cause que ceux qui les ont faites se sont efforcés de les faire paraître conformes à leurs principes, que s’il y en avait quelques-unes qui lui servissent, elles ne pourraient derechef valoir le temps qu’il lui faudrait employer à les choisir.\par
\pn{-}De façon que s’il y avait au monde quelqu’un, qu’on sût assurément être capable de trouver les plus grandes choses, \& les plus utiles au public qui puissent être, \& que pour cette cause les autres hommes s’efforçassent par tous moyens de l’aider à venir à bout de ses desseins :\par
Je ne vois pas qu’ils pussent autre chose pour lui, sinon fournir aux frais des expériences dont il aurait besoin, \& du reste empêcher que son loisir ne lui fût ôté par l’importunité de personne.\par
Mais outre que je ne présume pas tant de moi-même, que de vouloir rien promettre d’extraordinaire ; ni ne me repais point de pensées si vaines, que de m’imaginer que le public se doive beaucoup intéresser en mes desseins :\par
Je n’ai pas aussi l’âme si basse, que je voulusse accepter de qui que ce fût aucune faveur, qu’on pût croire que je n’aurais pas méritée.\par
\bigbreak
\phantomsection
\label{VI8}\noindent \pn{8}Toutes ces considérations jointes ensemble furent cause il y a trois ans que je ne voulus point divulguer le traité que j’avais entre les mains ;\par
Et même que je pris résolution de n’en faire voir aucun autre pendant ma vie, qui fût si général, ni duquel on pût entendre les fondements de ma Physique :\par
Mais il y a eu depuis derechef deux autres raisons, qui m’ont obligé à mettre ici quelques essais particuliers, \& à rendre au public quelque compte de mes actions, \& de mes desseins.\par
\pn{-}La première est, que si j’y manquais, plusieurs, qui ont su l’intention que j’avais eue ci-devant de faire imprimer quelques écrits, pourraient s’imaginer que les causes pour lesquelles je m’en abstiens seraient plus à mon désavantage qu’elles ne sont.\par
Car bien que je n’aime pas la gloire par excès ; ou même, si je l’ose dire, que je la haïsse, en tant que je la juge contraire au repos, lequel j’estime sur toutes choses :\par
Toutefois aussi je n’ai jamais tâché de cacher mes actions comme des crimes, ni n’ai usé de beaucoup de précautions pour être inconnu ; tant à cause que j’eusse cru me faire tort ; qu’à cause que cela m’aurait donné quelque espèce d’inquiétude, qui eût derechef été contraire au parfait repos d’esprit que je cherche.\par
Et pour ce que m’étant toujours ainsi tenu indifférent entre le soin d’être connu ou ne l’être pas, je n’ai pu empêcher que je n’acquisse quelque sorte de réputation, j’ai pensé que je devais faire mon mieux pour m’exempter au moins de l’avoir mauvaise.\par
\pn{-}L’autre raison qui m’a obligé à écrire ceci, est que voyant tous les jours de plus en plus le retardement que souffre le dessein que j’ai de m’instruire, à cause d’une infinité d’expériences dont j’ai besoin, \& qu’il est impossible que je fasse sans l’aide d’autrui,\par
Bien que je ne me flatte pas tant que d’espérer que le public prenne grande part en mes intérêts,\par
Toutefois je ne veux pas aussi me défaillir tant à moi-même, que de donner sujet a ceux qui me survivront de me reprocher quelque jour, que j’eusse pu leur laisser plusieurs choses beaucoup meilleures que je n’aurai fait, si je n’eusse point trop négligé de leur faire entendre en quoi ils pouvaient contribuer à mes desseins.\par
\bigbreak
\phantomsection
\label{VI9}\noindent \pn{9}Et j’ai pensé qu’il m’était aisé de choisir quelques matières, qui sans être sujettes à beaucoup de controverses, ni m’obliger à déclarer davantage de mes principes que je ne désire, ne laisseraient pas de faire voir assez clairement ce que je puis, ou ne puis pas, dans les sciences.\par
En quoi je ne saurais dire si j’ai réussi, \& je ne veux point prévenir les jugements de personne, en parlant moi-même de mes écrits ;\par
Mais je serai bien aise qu’on les examine, \& afin qu’on en ait d’autant plus d’occasion, je supplie tous ceux qui auront quelques objections à y faire de prendre la peine de les envoyer à mon libraire, par lequel en étant averti, je tâcherai d’y joindre ma réponse en même temps, \& par ce moyen les lecteurs, voyant ensemble l’un \& l’autre, jugeront d’autant plus aisément de la vérité.\par
Car je ne promets pas d’y faire jamais de longues réponses, mais seulement d’avouer mes fautes fort franchement, si je les connais ; ou bien si je ne les puis apercevoir, de dire simplement ce que je croirai être requis, pour la défense des choses que j’ai écrites, sans y ajouter l’explication d’aucune nouvelle matière, afin de ne me pas engager sans fin de l’une en l’autre.\par
\bigbreak
\phantomsection
\label{VI10}\noindent \pn{10}Que si quelques-unes de celles dont j’ai parlé au commencement de la Dioptrique \& des Météores, choquent d’abord, à cause que je les nomme des suppositions, \& que je ne semble pas avoir envie de les prouver,\par
Qu’on ait la patience de lire le tout avec attention, \& j’espère qu’on s’en trouvera satisfait :\par
Car il me semble que les raisons s’y entre-suivent en telle sorte, que comme les dernières sont démontrées, par les premières qui sont leurs causes ; ces premières le sont réciproquement, par les dernières qui sont leurs effets.\par
\pn{-}Et on ne doit pas imaginer que je commette en ceci la faute que les Logiciens nomment un cercle ; car l’expérience rendant la plupart de ces effets très certains, les causes dont je les déduis ne servent pas tant à les prouver qu’à les expliquer ; mais tout au contraire ce sont elles qui sont prouvées par eux.\par
Et je ne les ai nommées des suppositions, qu’afin qu’on sache que je pense les pouvoir déduire de ces premières vérités que j’ai ci-dessus expliquées ;\par
Mais que j’ai voulu expressément ne le pas faire, pour empêcher que certains esprits, qui s’imaginent qu’ils savent en un jour tout ce qu’un autre a pensé en vingt années, sitôt qu’il leur en a seulement dit deux ou trois mots, \& qui sont d’autant plus sujets à faillir, \& moins capables de la vérité, qu’ils sont plus pénétrants \& plus vifs,\par
Ne puissent de là prendre occasion, de bâtir quelque Philosophie extravagante sur ce qu’ils croiront être mes principes, \& qu’on m’en attribue la faute.\par
\pn{-}Car pour les opinions qui sont toutes miennes, je ne les excuse point comme nouvelles, d’autant que si on en considère bien les raisons, je m’assure qu’on les trouvera si simples, \& si conformes au sens commun, qu’elles sembleront moins extraordinaires, \& moins étranges, qu’aucunes autres qu’on puisse avoir sur mêmes sujets. Et je ne me vante point d’être le premier inventeur d’aucunes, mais bien que je ne les ai jamais reçues, ni pour ce qu’elles avaient été dites par d’autres, ni parce qu’elles ne l’avaient point été, mais seulement pour ce que la raison me les a persuadées.\par
\bigbreak
\phantomsection
\label{VI11}\noindent \pn{11}Que si les artisans ne peuvent si tôt exécuter l’invention qui est expliquée en la Dioptrique, je ne crois pas qu’on puisse dire pour cela, qu’elle soit mauvaise :\par
Car d’autant qu’il faut de l’adresse \& de l’habitude, pour faire, \& pour ajuster, les machines que j’ai décrites, sans qu’il y manque aucune circonstance, je ne m’étonnerais pas moins s’ils rencontraient du premier coup, que si quelqu’un pouvait apprendre en un jour à jouer du luth excellemment, par cela seul, qu’on lui aurait donné de la tablature qui serait bonne.\par
\pn{-}Et si j’écris en Français, qui est la langue de mon pays ; plutôt qu’en Latin, qui est celle de mes précepteurs ; c’est à cause que j’espère que ceux qui ne se servent que de leur raison naturelle toute pure jugeront mieux de mes opinions, que ceux qui ne croient qu’aux livres anciens :\par
Et pour ceux qui joignent le bon sens avec l’étude, lesquels seuls je souhaite pour mes juges, ils ne seront point je m’assure, si partiaux pour le Latin, qu’ils refusent d’entendre mes raisons pour ce que je les explique en langue vulgaire.\par
\bigbreak
\phantomsection
\label{VI12}\noindent \pn{12}Au reste, je ne veux point parler ici en particulier des progrès, que j’ai espérance de faire à l’avenir dans les sciences,\par
Ni m’engager envers le public d’aucune promesse que je ne sois pas assuré d’accomplir ;\par
Mais je dirai seulement que j’ai résolu, de n’employer le temps qui me reste à vivre, à autre chose, qu’à tâcher d’acquérir quelque connaissance de la Nature, qui soit telle, qu’on en puisse tirer des règles pour la Médecine, plus assurées que celles qu’on a eues jusques à présent ;\par
Et que mon inclination m’éloigne si fort de toute sorte d’autres desseins, principalement de ceux qui ne sauraient être utiles aux uns qu’en nuisant aux autres, que si quelques occasions me contraignaient de m’y employer, je ne crois point que je fusse capable d’y réussir.\par
\pn{-}De quoi je fais ici une déclaration, que je sais bien ne pouvoir servir à me rendre considérable dans le monde, mais aussi n’ai-je aucunement envie de l’être ;\par
Et je me tiendrai toujours plus obligé à ceux, par la faveur desquels je jouirai sans empêchement de mon loisir ; que je ne ferais à ceux qui m’offriraient les plus honorables emplois de la terre.
\chapterclose

 


% at least one empty page at end (for booklet couv)
\ifbooklet
  \newpage\null\thispagestyle{empty}\newpage
\fi

\ifdev % autotext in dev mode
\fontname\font — \textsc{Les règles du jeu}\par
(\hyperref[utopie]{\underline{Lien}})\par
\noindent \initialiv{A}{lors là}\blindtext\par
\noindent \initialiv{À}{ la bonheur des dames}\blindtext\par
\noindent \initialiv{É}{tonnez-le}\blindtext\par
\noindent \initialiv{Q}{ualitativement}\blindtext\par
\noindent \initialiv{V}{aloriser}\blindtext\par
\Blindtext
\phantomsection
\label{utopie}
\Blinddocument
\fi
\end{document}
