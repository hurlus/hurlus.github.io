%%%%%%%%%%%%%%%%%%%%%%%%%%%%%%%%%
% LaTeX model https://hurlus.fr %
%%%%%%%%%%%%%%%%%%%%%%%%%%%%%%%%%

% Needed before document class
\RequirePackage{pdftexcmds} % needed for tests expressions
\RequirePackage{fix-cm} % correct units

% Define mode
\def\mode{a4}

\newif\ifaiv % a4
\newif\ifav % a5
\newif\ifbooklet % booklet
\newif\ifcover % cover for booklet

\ifnum \strcmp{\mode}{cover}=0
  \covertrue
\else\ifnum \strcmp{\mode}{booklet}=0
  \booklettrue
\else\ifnum \strcmp{\mode}{a5}=0
  \avtrue
\else
  \aivtrue
\fi\fi\fi

\ifbooklet % do not enclose with {}
  \documentclass[french,twoside]{book} % ,notitlepage
  \usepackage[%
    papersize={105mm, 297mm},
    inner=12mm,
    outer=12mm,
    top=20mm,
    bottom=15mm,
    marginparsep=0pt,
  ]{geometry}
  \usepackage[fontsize=9.5pt]{scrextend} % for Roboto
\else\ifav
  \documentclass[french,twoside]{book} % ,notitlepage
  \usepackage[%
    a5paper,
    inner=25mm,
    outer=15mm,
    top=15mm,
    bottom=15mm,
    marginparsep=0pt,
  ]{geometry}
  \usepackage[fontsize=12pt]{scrextend}
\else% A4 2 cols
  \documentclass[twocolumn]{report}
  \usepackage[%
    a4paper,
    inner=15mm,
    outer=10mm,
    top=25mm,
    bottom=18mm,
    marginparsep=0pt,
  ]{geometry}
  \setlength{\columnsep}{20mm}
  \usepackage[fontsize=9.5pt]{scrextend}
\fi\fi

%%%%%%%%%%%%%%
% Alignments %
%%%%%%%%%%%%%%
% before teinte macros

\setlength{\arrayrulewidth}{0.2pt}
\setlength{\columnseprule}{\arrayrulewidth} % twocol
\setlength{\parskip}{0pt} % classical para with no margin
\setlength{\parindent}{1.5em}

%%%%%%%%%%
% Colors %
%%%%%%%%%%
% before Teinte macros

\usepackage[dvipsnames]{xcolor}
\definecolor{rubric}{HTML}{800000} % the tonic 0c71c3
\def\columnseprulecolor{\color{rubric}}
\colorlet{borderline}{rubric!30!} % definecolor need exact code
\definecolor{shadecolor}{gray}{0.95}
\definecolor{bghi}{gray}{0.5}

%%%%%%%%%%%%%%%%%
% Teinte macros %
%%%%%%%%%%%%%%%%%
%%%%%%%%%%%%%%%%%%%%%%%%%%%%%%%%%%%%%%%%%%%%%%%%%%%
% <TEI> generic (LaTeX names generated by Teinte) %
%%%%%%%%%%%%%%%%%%%%%%%%%%%%%%%%%%%%%%%%%%%%%%%%%%%
% This template is inserted in a specific design
% It is XeLaTeX and otf fonts

\makeatletter % <@@@


\usepackage{blindtext} % generate text for testing
\usepackage[strict]{changepage} % for modulo 4
\usepackage{contour} % rounding words
\usepackage[nodayofweek]{datetime}
% \usepackage{DejaVuSans} % seems buggy for sffont font for symbols
\usepackage{enumitem} % <list>
\usepackage{etoolbox} % patch commands
\usepackage{fancyvrb}
\usepackage{fancyhdr}
\usepackage{float}
\usepackage{fontspec} % XeLaTeX mandatory for fonts
\usepackage{footnote} % used to capture notes in minipage (ex: quote)
\usepackage{framed} % bordering correct with footnote hack
\usepackage{graphicx}
\usepackage{lettrine} % drop caps
\usepackage{lipsum} % generate text for testing
\usepackage[framemethod=tikz,]{mdframed} % maybe used for frame with footnotes inside
\usepackage{pdftexcmds} % needed for tests expressions
\usepackage{polyglossia} % non-break space french punct, bug Warning: "Failed to patch part"
\usepackage[%
  indentfirst=false,
  vskip=1em,
  noorphanfirst=true,
  noorphanafter=true,
  leftmargin=\parindent,
  rightmargin=0pt,
]{quoting}
\usepackage{ragged2e}
\usepackage{setspace} % \setstretch for <quote>
\usepackage{tabularx} % <table>
\usepackage[explicit]{titlesec} % wear titles, !NO implicit
\usepackage{tikz} % ornaments
\usepackage{tocloft} % styling tocs
\usepackage[fit]{truncate} % used im runing titles
\usepackage{unicode-math}
\usepackage[normalem]{ulem} % breakable \uline, normalem is absolutely necessary to keep \emph
\usepackage{verse} % <l>
\usepackage{xcolor} % named colors
\usepackage{xparse} % @ifundefined
\XeTeXdefaultencoding "iso-8859-1" % bad encoding of xstring
\usepackage{xstring} % string tests
\XeTeXdefaultencoding "utf-8"
\PassOptionsToPackage{hyphens}{url} % before hyperref, which load url package

% TOTEST
% \usepackage{hypcap} % links in caption ?
% \usepackage{marginnote}
% TESTED
% \usepackage{background} % doesn’t work with xetek
% \usepackage{bookmark} % prefers the hyperref hack \phantomsection
% \usepackage[color, leftbars]{changebar} % 2 cols doc, impossible to keep bar left
% \usepackage[utf8x]{inputenc} % inputenc package ignored with utf8 based engines
% \usepackage[sfdefault,medium]{inter} % no small caps
% \usepackage{firamath} % choose firasans instead, firamath unavailable in Ubuntu 21-04
% \usepackage{flushend} % bad for last notes, supposed flush end of columns
% \usepackage[stable]{footmisc} % BAD for complex notes https://texfaq.org/FAQ-ftnsect
% \usepackage{helvet} % not for XeLaTeX
% \usepackage{multicol} % not compatible with too much packages (longtable, framed, memoir…)
% \usepackage[default,oldstyle,scale=0.95]{opensans} % no small caps
% \usepackage{sectsty} % \chapterfont OBSOLETE
% \usepackage{soul} % \ul for underline, OBSOLETE with XeTeX
% \usepackage[breakable]{tcolorbox} % text styling gone, footnote hack not kept with breakable


% Metadata inserted by a program, from the TEI source, for title page and runing heads
\title{\textbf{ L’unique et sa propriété (traduction Reclaire, 1899) }}
\date{1844}
\author{Max Stirner}
\def\elbibl{Max Stirner. 1844. \emph{L’unique et sa propriété (traduction Reclaire, 1899)}}
\def\elsource{ \href{http://books.google.com/books?id=5J8DAAAAMAAJ}{\dotuline{http://books.google.com/books?id=5J8DAAAAMAAJ}}\footnote{\href{http://books.google.com/books?id=5J8DAAAAMAAJ}{\url{http://books.google.com/books?id=5J8DAAAAMAAJ}}}  \href{http://efele.net/ebooks/livres/000405}{\dotuline{http://efele.net/ebooks/livres/000405}}\footnote{\href{http://efele.net/ebooks/livres/000405}{\url{http://efele.net/ebooks/livres/000405}}} }

% Default metas
\newcommand{\colorprovide}[2]{\@ifundefinedcolor{#1}{\colorlet{#1}{#2}}{}}
\colorprovide{rubric}{red}
\colorprovide{silver}{lightgray}
\@ifundefined{syms}{\newfontfamily\syms{DejaVu Sans}}{}
\newif\ifdev
\@ifundefined{elbibl}{% No meta defined, maybe dev mode
  \newcommand{\elbibl}{Titre court ?}
  \newcommand{\elbook}{Titre du livre source ?}
  \newcommand{\elabstract}{Résumé\par}
  \newcommand{\elurl}{http://oeuvres.github.io/elbook/2}
  \author{Éric Lœchien}
  \title{Un titre de test assez long pour vérifier le comportement d’une maquette}
  \date{1566}
  \devtrue
}{}
\let\eltitle\@title
\let\elauthor\@author
\let\eldate\@date


\defaultfontfeatures{
  % Mapping=tex-text, % no effect seen
  Scale=MatchLowercase,
  Ligatures={TeX,Common},
}


% generic typo commands
\newcommand{\astermono}{\medskip\centerline{\color{rubric}\large\selectfont{\syms ✻}}\medskip\par}%
\newcommand{\astertri}{\medskip\par\centerline{\color{rubric}\large\selectfont{\syms ✻\,✻\,✻}}\medskip\par}%
\newcommand{\asterism}{\bigskip\par\noindent\parbox{\linewidth}{\centering\color{rubric}\large{\syms ✻}\\{\syms ✻}\hskip 0.75em{\syms ✻}}\bigskip\par}%

% lists
\newlength{\listmod}
\setlength{\listmod}{\parindent}
\setlist{
  itemindent=!,
  listparindent=\listmod,
  labelsep=0.2\listmod,
  parsep=0pt,
  % topsep=0.2em, % default topsep is best
}
\setlist[itemize]{
  label=—,
  leftmargin=0pt,
  labelindent=1.2em,
  labelwidth=0pt,
}
\setlist[enumerate]{
  label={\bf\color{rubric}\arabic*.},
  labelindent=0.8\listmod,
  leftmargin=\listmod,
  labelwidth=0pt,
}
\newlist{listalpha}{enumerate}{1}
\setlist[listalpha]{
  label={\bf\color{rubric}\alph*.},
  leftmargin=0pt,
  labelindent=0.8\listmod,
  labelwidth=0pt,
}
\newcommand{\listhead}[1]{\hspace{-1\listmod}\emph{#1}}

\renewcommand{\hrulefill}{%
  \leavevmode\leaders\hrule height 0.2pt\hfill\kern\z@}

% General typo
\DeclareTextFontCommand{\textlarge}{\large}
\DeclareTextFontCommand{\textsmall}{\small}

% commands, inlines
\newcommand{\anchor}[1]{\Hy@raisedlink{\hypertarget{#1}{}}} % link to top of an anchor (not baseline)
\newcommand\abbr[1]{#1}
\newcommand{\autour}[1]{\tikz[baseline=(X.base)]\node [draw=rubric,thin,rectangle,inner sep=1.5pt, rounded corners=3pt] (X) {\color{rubric}#1};}
\newcommand\corr[1]{#1}
\newcommand{\ed}[1]{ {\color{silver}\sffamily\footnotesize (#1)} } % <milestone ed="1688"/>
\newcommand\expan[1]{#1}
\newcommand\foreign[1]{\emph{#1}}
\newcommand\gap[1]{#1}
\renewcommand{\LettrineFontHook}{\color{rubric}}
\newcommand{\initial}[2]{\lettrine[lines=2, loversize=0.3, lhang=0.3]{#1}{#2}}
\newcommand{\initialiv}[2]{%
  \let\oldLFH\LettrineFontHook
  % \renewcommand{\LettrineFontHook}{\color{rubric}\ttfamily}
  \IfSubStr{QJ’}{#1}{
    \lettrine[lines=4, lhang=0.2, loversize=-0.1, lraise=0.2]{\smash{#1}}{#2}
  }{\IfSubStr{É}{#1}{
    \lettrine[lines=4, lhang=0.2, loversize=-0, lraise=0]{\smash{#1}}{#2}
  }{\IfSubStr{ÀÂ}{#1}{
    \lettrine[lines=4, lhang=0.2, loversize=-0, lraise=0, slope=0.6em]{\smash{#1}}{#2}
  }{\IfSubStr{A}{#1}{
    \lettrine[lines=4, lhang=0.2, loversize=0.2, slope=0.6em]{\smash{#1}}{#2}
  }{\IfSubStr{V}{#1}{
    \lettrine[lines=4, lhang=0.2, loversize=0.2, slope=-0.5em]{\smash{#1}}{#2}
  }{
    \lettrine[lines=4, lhang=0.2, loversize=0.2]{\smash{#1}}{#2}
  }}}}}
  \let\LettrineFontHook\oldLFH
}
\newcommand{\labelchar}[1]{\textbf{\color{rubric} #1}}
\newcommand{\milestone}[1]{\autour{\footnotesize\color{rubric} #1}} % <milestone n="4"/>
\newcommand\name[1]{#1}
\newcommand\orig[1]{#1}
\newcommand\orgName[1]{#1}
\newcommand\persName[1]{#1}
\newcommand\placeName[1]{#1}
\newcommand{\pn}[1]{\IfSubStr{-—–¶}{#1}% <p n="3"/>
  {\noindent{\bfseries\color{rubric}   ¶  }}
  {{\footnotesize\autour{ #1}  }}}
\newcommand\reg{}
% \newcommand\ref{} % already defined
\newcommand\sic[1]{#1}
\newcommand\surname[1]{\textsc{#1}}
\newcommand\term[1]{\textbf{#1}}

\def\mednobreak{\ifdim\lastskip<\medskipamount
  \removelastskip\nopagebreak\medskip\fi}
\def\bignobreak{\ifdim\lastskip<\bigskipamount
  \removelastskip\nopagebreak\bigskip\fi}

% commands, blocks
\newcommand{\byline}[1]{\bigskip{\RaggedLeft{#1}\par}\bigskip}
\newcommand{\bibl}[1]{{\RaggedLeft{#1}\par\bigskip}}
\newcommand{\biblitem}[1]{{\noindent\hangindent=\parindent   #1\par}}
\newcommand{\dateline}[1]{\medskip{\RaggedLeft{#1}\par}\bigskip}
\newcommand{\labelblock}[1]{\medbreak{\noindent\color{rubric}\bfseries #1}\par\mednobreak}
\newcommand{\salute}[1]{\bigbreak{#1}\par\medbreak}
\newcommand{\signed}[1]{\bigbreak\filbreak{\raggedleft #1\par}\medskip}

% environments for blocks (some may become commands)
\newenvironment{borderbox}{}{} % framing content
\newenvironment{citbibl}{\ifvmode\hfill\fi}{\ifvmode\par\fi }
\newenvironment{docAuthor}{\ifvmode\vskip4pt\fontsize{16pt}{18pt}\selectfont\fi\itshape}{\ifvmode\par\fi }
\newenvironment{docDate}{}{\ifvmode\par\fi }
\newenvironment{docImprint}{\vskip6pt}{\ifvmode\par\fi }
\newenvironment{docTitle}{\vskip6pt\bfseries\fontsize{18pt}{22pt}\selectfont}{\par }
\newenvironment{msHead}{\vskip6pt}{\par}
\newenvironment{msItem}{\vskip6pt}{\par}
\newenvironment{titlePart}{}{\par }


% environments for block containers
\newenvironment{argument}{\itshape\parindent0pt}{\vskip1.5em}
\newenvironment{biblfree}{}{\ifvmode\par\fi }
\newenvironment{bibitemlist}[1]{%
  \list{\@biblabel{\@arabic\c@enumiv}}%
  {%
    \settowidth\labelwidth{\@biblabel{#1}}%
    \leftmargin\labelwidth
    \advance\leftmargin\labelsep
    \@openbib@code
    \usecounter{enumiv}%
    \let\p@enumiv\@empty
    \renewcommand\theenumiv{\@arabic\c@enumiv}%
  }
  \sloppy
  \clubpenalty4000
  \@clubpenalty \clubpenalty
  \widowpenalty4000%
  \sfcode`\.\@m
}%
{\def\@noitemerr
  {\@latex@warning{Empty `bibitemlist' environment}}%
\endlist}
\newenvironment{quoteblock}% may be used for ornaments
  {\begin{quoting}}
  {\end{quoting}}

% table () is preceded and finished by custom command
\newcommand{\tableopen}[1]{%
  \ifnum\strcmp{#1}{wide}=0{%
    \begin{center}
  }
  \else\ifnum\strcmp{#1}{long}=0{%
    \begin{center}
  }
  \else{%
    \begin{center}
  }
  \fi\fi
}
\newcommand{\tableclose}[1]{%
  \ifnum\strcmp{#1}{wide}=0{%
    \end{center}
  }
  \else\ifnum\strcmp{#1}{long}=0{%
    \end{center}
  }
  \else{%
    \end{center}
  }
  \fi\fi
}


% text structure
\newcommand\chapteropen{} % before chapter title
\newcommand\chaptercont{} % after title, argument, epigraph…
\newcommand\chapterclose{} % maybe useful for multicol settings
\setcounter{secnumdepth}{-2} % no counters for hierarchy titles
\setcounter{tocdepth}{5} % deep toc
\markright{\@title} % ???
\markboth{\@title}{\@author} % ???
\renewcommand\tableofcontents{\@starttoc{toc}}
% toclof format
% \renewcommand{\@tocrmarg}{0.1em} % Useless command?
% \renewcommand{\@pnumwidth}{0.5em} % {1.75em}
\renewcommand{\@cftmaketoctitle}{}
\setlength{\cftbeforesecskip}{\z@ \@plus.2\p@}
\renewcommand{\cftchapfont}{}
\renewcommand{\cftchapdotsep}{\cftdotsep}
\renewcommand{\cftchapleader}{\normalfont\cftdotfill{\cftchapdotsep}}
\renewcommand{\cftchappagefont}{\bfseries}
\setlength{\cftbeforechapskip}{0em \@plus\p@}
% \renewcommand{\cftsecfont}{\small\relax}
\renewcommand{\cftsecpagefont}{\normalfont}
% \renewcommand{\cftsubsecfont}{\small\relax}
\renewcommand{\cftsecdotsep}{\cftdotsep}
\renewcommand{\cftsecpagefont}{\normalfont}
\renewcommand{\cftsecleader}{\normalfont\cftdotfill{\cftsecdotsep}}
\setlength{\cftsecindent}{1em}
\setlength{\cftsubsecindent}{2em}
\setlength{\cftsubsubsecindent}{3em}
\setlength{\cftchapnumwidth}{1em}
\setlength{\cftsecnumwidth}{1em}
\setlength{\cftsubsecnumwidth}{1em}
\setlength{\cftsubsubsecnumwidth}{1em}

% footnotes
\newif\ifheading
\newcommand*{\fnmarkscale}{\ifheading 0.70 \else 1 \fi}
\renewcommand\footnoterule{\vspace*{0.3cm}\hrule height \arrayrulewidth width 3cm \vspace*{0.3cm}}
\setlength\footnotesep{1.5\footnotesep} % footnote separator
\renewcommand\@makefntext[1]{\parindent 1.5em \noindent \hb@xt@1.8em{\hss{\normalfont\@thefnmark . }}#1} % no superscipt in foot
\patchcmd{\@footnotetext}{\footnotesize}{\footnotesize\sffamily}{}{} % before scrextend, hyperref


%   see https://tex.stackexchange.com/a/34449/5049
\def\truncdiv#1#2{((#1-(#2-1)/2)/#2)}
\def\moduloop#1#2{(#1-\truncdiv{#1}{#2}*#2)}
\def\modulo#1#2{\number\numexpr\moduloop{#1}{#2}\relax}

% orphans and widows
\clubpenalty=9996
\widowpenalty=9999
\brokenpenalty=4991
\predisplaypenalty=10000
\postdisplaypenalty=1549
\displaywidowpenalty=1602
\hyphenpenalty=400
% Copied from Rahtz but not understood
\def\@pnumwidth{1.55em}
\def\@tocrmarg {2.55em}
\def\@dotsep{4.5}
\emergencystretch 3em
\hbadness=4000
\pretolerance=750
\tolerance=2000
\vbadness=4000
\def\Gin@extensions{.pdf,.png,.jpg,.mps,.tif}
% \renewcommand{\@cite}[1]{#1} % biblio

\usepackage{hyperref} % supposed to be the last one, :o) except for the ones to follow
\urlstyle{same} % after hyperref
\hypersetup{
  % pdftex, % no effect
  pdftitle={\elbibl},
  % pdfauthor={Your name here},
  % pdfsubject={Your subject here},
  % pdfkeywords={keyword1, keyword2},
  bookmarksnumbered=true,
  bookmarksopen=true,
  bookmarksopenlevel=1,
  pdfstartview=Fit,
  breaklinks=true, % avoid long links
  pdfpagemode=UseOutlines,    % pdf toc
  hyperfootnotes=true,
  colorlinks=false,
  pdfborder=0 0 0,
  % pdfpagelayout=TwoPageRight,
  % linktocpage=true, % NO, toc, link only on page no
}

\makeatother % /@@@>
%%%%%%%%%%%%%%
% </TEI> end %
%%%%%%%%%%%%%%


%%%%%%%%%%%%%
% footnotes %
%%%%%%%%%%%%%
\renewcommand{\thefootnote}{\bfseries\textcolor{rubric}{\arabic{footnote}}} % color for footnote marks

%%%%%%%%%
% Fonts %
%%%%%%%%%
\usepackage[]{roboto} % SmallCaps, Regular is a bit bold
% \linespread{0.90} % too compact, keep font natural
\newfontfamily\fontrun[]{Roboto Condensed Light} % condensed runing heads
\ifav
  \setmainfont[
    ItalicFont={Roboto Light Italic},
  ]{Roboto}
\else\ifbooklet
  \setmainfont[
    ItalicFont={Roboto Light Italic},
  ]{Roboto}
\else
\setmainfont[
  ItalicFont={Roboto Italic},
]{Roboto Light}
\fi\fi
\renewcommand{\LettrineFontHook}{\bfseries\color{rubric}}
% \renewenvironment{labelblock}{\begin{center}\bfseries\color{rubric}}{\end{center}}

%%%%%%%%
% MISC %
%%%%%%%%

\setdefaultlanguage[frenchpart=false]{french} % bug on part


\newenvironment{quotebar}{%
    \def\FrameCommand{{\color{rubric!10!}\vrule width 0.5em} \hspace{0.9em}}%
    \def\OuterFrameSep{\itemsep} % séparateur vertical
    \MakeFramed {\advance\hsize-\width \FrameRestore}
  }%
  {%
    \endMakeFramed
  }
\renewenvironment{quoteblock}% may be used for ornaments
  {%
    \savenotes
    \setstretch{0.9}
    \normalfont
    \begin{quotebar}
  }
  {%
    \end{quotebar}
    \spewnotes
  }


\renewcommand{\headrulewidth}{\arrayrulewidth}
\renewcommand{\headrule}{{\color{rubric}\hrule}}

% delicate tuning, image has produce line-height problems in title on 2 lines
\titleformat{name=\chapter} % command
  [display] % shape
  {\vspace{1.5em}\centering} % format
  {} % label
  {0pt} % separator between n
  {}
[{\color{rubric}\huge\textbf{#1}}\bigskip] % after code
% \titlespacing{command}{left spacing}{before spacing}{after spacing}[right]
\titlespacing*{\chapter}{0pt}{-2em}{0pt}[0pt]

\titleformat{name=\section}
  [block]{}{}{}{}
  [\vbox{\color{rubric}\large\raggedleft\textbf{#1}}]
\titlespacing{\section}{0pt}{0pt plus 4pt minus 2pt}{\baselineskip}

\titleformat{name=\subsection}
  [block]
  {}
  {} % \thesection
  {} % separator \arrayrulewidth
  {}
[\vbox{\large\textbf{#1}}]
% \titlespacing{\subsection}{0pt}{0pt plus 4pt minus 2pt}{\baselineskip}

\ifaiv
  \fancypagestyle{main}{%
    \fancyhf{}
    \setlength{\headheight}{1.5em}
    \fancyhead{} % reset head
    \fancyfoot{} % reset foot
    \fancyhead[L]{\truncate{0.45\headwidth}{\fontrun\elbibl}} % book ref
    \fancyhead[R]{\truncate{0.45\headwidth}{ \fontrun\nouppercase\leftmark}} % Chapter title
    \fancyhead[C]{\thepage}
  }
  \fancypagestyle{plain}{% apply to chapter
    \fancyhf{}% clear all header and footer fields
    \setlength{\headheight}{1.5em}
    \fancyhead[L]{\truncate{0.9\headwidth}{\fontrun\elbibl}}
    \fancyhead[R]{\thepage}
  }
\else
  \fancypagestyle{main}{%
    \fancyhf{}
    \setlength{\headheight}{1.5em}
    \fancyhead{} % reset head
    \fancyfoot{} % reset foot
    \fancyhead[RE]{\truncate{0.9\headwidth}{\fontrun\elbibl}} % book ref
    \fancyhead[LO]{\truncate{0.9\headwidth}{\fontrun\nouppercase\leftmark}} % Chapter title, \nouppercase needed
    \fancyhead[RO,LE]{\thepage}
  }
  \fancypagestyle{plain}{% apply to chapter
    \fancyhf{}% clear all header and footer fields
    \setlength{\headheight}{1.5em}
    \fancyhead[L]{\truncate{0.9\headwidth}{\fontrun\elbibl}}
    \fancyhead[R]{\thepage}
  }
\fi

\ifav % a5 only
  \titleclass{\section}{top}
\fi

\newcommand\chapo{{%
  \vspace*{-3em}
  \centering % no vskip ()
  {\Large\addfontfeature{LetterSpace=25}\bfseries{\elauthor}}\par
  \smallskip
  {\large\eldate}\par
  \bigskip
  {\Large\selectfont{\eltitle}}\par
  \bigskip
  {\color{rubric}\hline\par}
  \bigskip
  {\Large TEXTE LIBRE À PARTICPATION LIBRE\par}
  \centerline{\small\color{rubric} {hurlus.fr, tiré le \today}}\par
  \bigskip
}}

\newcommand\cover{{%
  \thispagestyle{empty}
  \centering
  {\LARGE\bfseries{\elauthor}}\par
  \bigskip
  {\Large\eldate}\par
  \bigskip
  \bigskip
  {\LARGE\selectfont{\eltitle}}\par
  \vfill\null
  {\color{rubric}\setlength{\arrayrulewidth}{2pt}\hline\par}
  \vfill\null
  {\Large TEXTE LIBRE À PARTICPATION LIBRE\par}
  \centerline{{\href{https://hurlus.fr}{\dotuline{hurlus.fr}}, tiré le \today}}\par
}}

\begin{document}
\pagestyle{empty}
\ifbooklet{
  \cover\newpage
  \thispagestyle{empty}\hbox{}\newpage
  \cover\newpage\noindent Les voyages de la brochure\par
  \bigskip
  \begin{tabularx}{\textwidth}{l|X|X}
    \textbf{Date} & \textbf{Lieu}& \textbf{Nom/pseudo} \\ \hline
    \rule{0pt}{25cm} &  &   \\
  \end{tabularx}
  \newpage
  \addtocounter{page}{-4}
}\fi

\thispagestyle{empty}
\ifaiv
  \twocolumn[\chapo]
\else
  \chapo
\fi
{\it\elabstract}
\bigskip
\makeatletter\@starttoc{toc}\makeatother % toc without new page
\bigskip

\pagestyle{main} % after style

  \frontmatter  \section[{Préface du traducteur}]{Préface du traducteur}\renewcommand{\leftmark}{Préface du traducteur}

\noindent « Wer ein ganzer Mensch ist, braucht keine Autorität zu sein. »\par
M. Stirner.
\noindent « Moi, Johann-Caspar Schmidt, de la confession évangélique, je suis né à Bayreuth, ville appartenant alors à la Prusse et rattachée aujourd’hui à la Bavière, le 25\textsuperscript{e} jour du mois d’octobre de l’an 1806, d’un père fabricant de flûtes qui mourut peu de jours après ma naissance. Ma mère épousa trois ans plus tard l’apothicaire Ballerstedt, et, s’étant, après des chances diverses, transportée à Kulm, ville située sur la Vistule dans la Prusse occidentale, elle m’appela bientôt auprès d’elle en l’an 1810.\par
« C’est là que je fus instruit dans les premiers rudiments des lettres ; j’en revins à l’âge de douze ans à Bayreuth pour y fréquenter le très florissant gymnase de cette ville. J’y fus pendant près de sept ans sous la discipline de maîtres très doctes, parmi lesquels je cite avec un souvenir  pieux et reconnaissant Pausch, Kieffer, Neubig, Kloeter, Held et Gabler, qui méritèrent toute ma gratitude par leur science des humanités et par la bienveillance qu’ils me témoignèrent.\par
« Préparé par leurs préceptes, j’étudiai pendant les années 1826-1828 la philologie et la théologie à l’Académie de Berlin, où je suivis les leçons de Boeckh, Hégel, Marheinecke, C. Ritter, H. Ritter et Schleiermacher. Je fréquentai ensuite pendant un semestre les cours de Rapp et de Winer à Erlangen, puis j’abandonnai l’université pour faire en Allemagne un voyage auquel je consacrai près d’une année. Des affaires domestiques m’obligèrent alors à passer une année à Kulm, une autre à Kœnigsberg ; mais, s’il me fut impossible pendant ce temps de poursuivre mes études dans une académie, je ne négligeai cependant pas l’étude des lettres et je m’adonnai d’un esprit studieux aux sciences philosophiques et philologiques.\par
« L’an 1833, au mois d’Octobre, j’étais retourné à Berlin pour y reprendre le cours de mes études, lorsque je fus atteint d’une maladie qui me tint pendant un semestre éloigné des leçons. Après ma guérison, je suivis les cours de Boeckh, de Lachmann et de Michelet. Mon triennium étant ainsi achevé, je me propose de subir, Dieu aidant, l’examen \emph{pro facultate docendi}. »\par
\bigbreak
\noindent Quelques noms, quelques dates, une maladie, un voyage, nous ne connaissons rien de plus des premières années de celui qui devait un jour s’appeler Max Stirner. Ce « curriculum vitæ », qu’il rédigea en 1834 lorsqu’il s’apprêtait à terminer ses longues et pénibles études universitaires résume à peu près tout ce que nous savons de sa jeunesse, de ses études et de la formation de son esprit. Le reste de sa vie est plongé dans la même obscurité. Il publie en 1844 l’\emph{Unique et sa propriété}, puis il disparaît. Le court et violent scandale qu’avaient soulevé son intraitable franchise et l’audace de sa critique est étouffé par la rumeur grandissante des événements de 48 qui approchent ; et lorsqu’il meurt en 1856, les rares contemporains  qui se rappellent encore le titre de son œuvre apprennent avec quelque surprise que l’auteur vient seulement de s’éteindre dans la misère et dans l’oubli.\par
Pendant cinquante ans, l’ombre s’amasse sur son nom et sur son œuvre ; seuls, quelques curieux que leurs études forcent à fouiller les coins poudreux des bibliothèques ont feuilleté d’un doigt soupçonneux ce livre réprouvé ; s’ils en parlent parfois, en passant, c’est comme d’un paradoxe impudent ou d’une gageure douteuse. — Les idées marchent, et un jour vient où l’on s’avise que ce solitaire inconnu a été un des penseurs les plus vigoureux de son époque ; on s’aperçoit qu’il a prononcé les paroles décisives dont nous cherchions hier encore la formule, et cet isolé retrouve chez nous une famille. Il sort de l’oubli, et des mains pieuses cherchent à retrouver sous la poussière d’un demi-siècle les traces de ce passant hautain en qui palpitaient déjà nos haines et nos amours d’aujourd’hui.\par
Le poète J.-H. Mackay, dont nos lecteurs connaissent le roman « \emph{Anarchistes} », publié dans cette même \emph{bibliothèque Sociologique} a pendant dix ans recueilli avec un soin jaloux tous les documents, tous les renseignements, tous les indices capables de jeter quelque clarté sur la vie de Max Stirner ; mais la consciencieuse enquête à laquelle il s’est livré, les fouilles laborieuses qu’il a pratiquées dans les registres des facultés, les publications de l’époque et les souvenirs de ceux qui avaient croisé son héros dans la vie — nous osons à peine dire de ceux qui l’avaient connu — n’ont malheureusement point réussi à faire sortir Stirner de « l’ombre de son esprit ». L’ouvrage, fruit de ses patientes recherches\footnote{ \noindent J.-H. M{\scshape ackay}, Max Stirner, sein Leben und sein Werk \emph{(Berlin, Schuster et Loeffler, 1898}).
 }, nous donne une description exacte jusqu’à la minutie du milieu dans lequel dut évoluer l’auteur de l’\emph{Unique}, ses tableaux abondants et sympathiques font revivre les hommes qu’il dut fréquenter, les êtres et les choses parmi lesquels il vécut ; mais cette esquisse, encore pleine de lacunes,  de la vie extérieure de J.-Caspar Schmidt, Max Stirner ne la traverse que comme un étranger. C’est un cadre, mais le portrait manque et manquera vraisemblablement toujours.\par
Ce cadre, c’est l’Allemagne des « années quarante », grosse de rêves et d’espoirs démesurés, pleine du juvénile sentiment qu’il suffisait de volonté et d’enthousiasme pour faire éclore le monde nouveau qu’elle sentait tressaillir dans ses flancs. La jeune Allemagne, nourrie des doctrines de Hégel mais que ne satisfaisait plus la scolastique pétrifiée du maître, s’était jetée dans la mêlée philosophique et sociale qui devait aboutir aux orages de 1848-1849, et se pressait sous les drapeaux du radicalisme et du socialisme, ou combattait autour de Br. Bauer, de Feuerbach et des « Nachhegelianer », avec, pour centres de ralliement, les \emph{Annales de Halle} de Ruge et la \emph{Gazette du Rhin} du jeune docteur Karl Marx.\par
C’est sur ce fond tumultueux et lourd de menaces, où chaque livre est une arme, où toute parole est un acte, où l’un sort de prison quand l’autre part pour l’exil, que nous voyons passer la silhouette effacée, l’ombre fugitive du grand penseur oublié.\par
Cet homme silencieux et discret, sans passions vives ni attaches profondes dans la vie, qui contemple d’un œil serein les événements politiques se dérouler devant lui, avec parfois un mince sourire derrière ses lunettes d’acier, c’est J.-C. Schmidt.\par
Ceux qui le coudoient au milieu des promptes et chaudes camaraderies du champ de bataille le connaissent peu. Ils savent que la vie lui est dure, que dès sa jeunesse la chance lui fut hostile, que des « affaires de famille » pénibles troublèrent ses études, et qu’un mariage conclu en 1837 le laissa après six mois veuf et seul, sans autres relations que sa mère « dont l’esprit est dérangé ». Ils savent que, son examen pro facultate docendi passé, il a fait un an de stage pédagogique à Berlin, puis que, renonçant à acquérir le grade de docteur et à entrer dans l’enseignement officiel, il a accepté, en 1839, une  place de professeur dans un établissement privé d’instruction pour jeunes filles. Mais nul n’a pénétré dans l’intimité de sa vie et de sa pensée, et il n’est pas de ceux à qui l’on peut dire : pourquoi ?\par
De 1840 à 1844, « les meilleures années de sa vie », on le voit fréquenter assidûment, plutôt en spectateur qu’en acteur, les cercles radicaux où trône Br. Bauer ; il publie, en 1842 et 1843, quelques articles de philosophie sociale sous le pseudonyme de Max Stirner, mais n’occupe qu’une place effacée dans les réunions turbulentes de la jeunesse de Berlin. En 1843, il se remarie et la vie semble un instant vouloir sourire au pauvre « professeur privé  ».\par
En 1844, paraît chez l’éditeur Otto Wigand, de Leipzig « \emph{l’Unique et sa Propriété}. » Stupeur de ceux qui, voyant sans cesse l’auteur au milieu d’eux, le croyaient des leurs, et scandale violent dans le public lettré dont il renverse les idoles avec une verve d’iconoclaste. Le livre, répandu en cachette chez les libraires, est interdit par la censure qui, quelques jours après, revient sur sa condamnation, jugeant l’ouvrage « trop absurde pour pouvoir être dangereux ». Les anciens compagnons s’écartent, le livre est oublié et la solitude se fait.\par
Dès ce moment, commence la longue agonie du penseur. L’année même de la publication de son œuvre, « cette œuvre laborieuse des plus belles années de sa jeunesse », l’établissement où il professait lui ferme ses portes, et la gêne s’installe à son foyer ; l’éditeur Wigand qui resta un ami fidèle du proscrit moral, lui confie, pour l’aider, quelques traductions, et il publie en allemand, de 1846 à 1847, le \emph{Dictionnaire d’économie politique} de J.-B. Say et les \emph{Recherches sur la richesse des nations} de Smith. Mais les embarras d’argent vont croissant ; une tentative commerciale malheureuse achève de fondre en peu de mois les quelques milliers de francs qui avaient formé la dot de sa femme, et celle-ci se sépare de lui en 1846. Dès lors, c’est la misère de plus en plus profonde. Ceux qui l’avaient connu le perdent complètement de vue, Wigand lui-même ignore où il cache son orgueilleuse détresse ; les  événements de 48 se déroulent sans qu’on voie Stirner y prendre aucune part.\par
En 1852, paraît encore une \emph{Histoire de la Réaction} en deux volumes, entreprise de librairie sans intérêt où la part de collaboration de Stirner est d’ailleurs mal définie. — Et puis, plus rien, à peine quelques lueurs : en 1852, il est commissionnaire, et son biographe a retrouvé les traces de deux séjours de J.-C. Schmidt dans la prison pour dettes en 1852 et 53.................\par
Il achève de mourir le 25 juin 1856, âgé de quarante-neuf ans et huit mois.\par
On peut voir aujourd’hui sur sa tombe, grâce aux soins pieux de J.-H. Mackay, une dalle de granit portant ces seuls mots : M{\scshape ax} S{\scshape tirner}. Et sur la façade de la maison où il mourut, Philippstrasse, 19, à Berlin, on lit cette inscription :\par


\begin{verse}
C’est dans cette maison\\
que vécut ses derniers jours\\
MAX STIRNER\\
(D\textsuperscript{r} Caspar Schmidt, 1806-1856)\\
l’auteur du livre immortel\\
L’UNIQUE ET SA PROPRIÉTÉ\\
1845\\
\end{verse}


\asterism

\noindent « \emph{De ceux que nous jugeons grands comme de ceux que nous aimons, avait dit Stirner, tout nous intéresse, même ce qui n’a aucune importance ; celui qui vient nous parler d’eux est toujours le bienvenu. » Cela suffirait, son mérite mis à part, pour expliquer l’intérêt du livre de Mackay et pour en faire regretter vivement les lacunes. Mais il a probablement tout dit, et personne n’achèvera de soulever le voile que cinquante ans d’oubli ont épaissi sur la vie et la personnalité de l’auteur de l’\emph{Unique}.}\par
 \emph{C’est vers son œuvre que nous devons nous tourner, et lui demander comment il se fait que, si vite oubliée lorsqu’elle parut, elle se relève aujourd’hui si vivante et si actuelle.}\par
\emph{Les œuvres originales de Stirner (nous laissons de côté ses traductions de Say et de Smith et son « Histoire de la réaction »), sont peu nombreuses. Outre l’\emph{Unique et sa Propriété}, son œuvre capitale, et deux articles polémiques (\emph{Recensenten Stirners}, 1845 — \emph{Die philosophischen Reactionäre}, 1847), en réponse aux critiques que l’apparition de son livre avait provoquées de la part d’écrivains de différents partis, il n’existe de lui que quelques essais publiés de 1842 à 1844 dans la Reinische Zeitung de Marx et dans la Berliner Monatschrifft de Buhl.}\par
\emph{Ces articles\footnote{ \noindent \emph{Réunis en un volume pour la première fois en 1898 par J.-H. Mackay, avec les Réponses aux objections, sous le titre : \emph{Max Stirner’s kleinere Schriften} (Berlin, Schuster et Loeffler).}
 }, esquisses de son grand ouvrage, sont : 1\textsuperscript{o} Le faux principe de notre éducation, ou Humanisme et Réalisme (\emph{Das unwahre Prinzip unserer Erziehung oder der Humanismus und Realismus}, avril 1842), 2\textsuperscript{o} l’Art et la Religion (\emph{Kunst und Religion}, juin 1842), et 3\textsuperscript{o} de l’Amour dans l’Etat (\emph{Einiges Vorläufige von Liebesstaat}, 1844), ce dernier, simple esquisse d’un travail plus considérable que la censure supprima. Ajoutons-y deux études philosophiques sur des œuvres littéraires alors célèbres : Sur les « esquisses Koenigsbergiennes » de Rosenkranz (1842), et sur les « Mystères de Paris » d’E. Sue (1844).}\par
\emph{Il serait à désirer que ces études préliminaires fussent mises à la portée du lecteur français ; elles sont une introduction naturelle à la lecture du chef d’œuvre de Stirner, comme ses réponses aux objections, qui achèvent de préciser sa pensée en sont le complément précieux.}\par
\emph{Nous leur demanderons de nous aider à comprendre ce que Stirner a voulu, ce qu’il a fait et ce qu’il est aujourd’hui  pour nous. Que signifiait l’\emph{Unique et sa Propriété} lorsqu’il parut, et quelle est sa signification actuelle ?}\par
\emph{Et d’abord — car appeler ce livre \emph{unique} n’est qu’un jeu de mot très vain — quelles sont ses racines dans la pensée allemande contemporaine ? Tout l’effort de la philosophie pratique du XIX\textsuperscript{e} siècle a eu pour but de séculariser les bases de la vie sociale et d’arracher à la théologie les notions de droit, de morale et de justice. En Allemagne, c’est cette lutte contre la transcendance qui fait le caractère fondamental des travaux philosophiques des penseurs qui suivirent Hégel. La critique historique des sources religieuses y aboutit bientôt à la critique philosophique du sentiment religieux, et les travaux d’exégèse chrétienne préludèrent à l’étude de la morale du Christianisme.}\par
\emph{Strauss avait ouvert la voie par sa \emph{Vie de Jésus} en s’attaquant au caractère révélé des Evangiles ; Br. Bauer, dans sa \emph{Critique des Evangiles}, se donna pour tâche de détruire le fond de religiosité et de mysticisme que la mythique de Strauss laissait subsister dans la légende chrétienne, et s’attaqua à l’esprit théologique en général.}\par
\emph{C’est à Br. Bauer que succède logiquement Feuerbach, dont l’\emph{Essence du Christianisme} eut un retentissement considérable. Comme Feuerbach le dit lui-même, « il étudie le Christianisme en général, et, comme conséquence, la philosophie chrétienne ou la théologie. » Sans plus s’attaquer en historien au mythe chrétien comme Strauss ou à l’esprit évangélique comme Bauer, il étudie le Christianisme tel qu’il s’est transmis jusqu’à nous et se borne à en rechercher l’essence en le débarrassant « des innombrables mailles du réseau de mensonges, de contradictions et de mauvaise foi dont la théologie l’avait enveloppé. » Il en vint à conclure que « l’être infini ou divin est l’être spirituel de l’homme, projeté par l’homme en dehors de lui-même et contemplé comme un être indépendant... L’Homme est le Dieu du Christianisme, l’anthropologie est le secret de la religion chrétienne. L’histoire du Christianisme n’a pas eu d’autre tendance ni d’autre tâche que de dévoiler ce mystère, d’humaniser Dieu et de résoudre la  théologie en anthropologie. » C’est en ramenant Dieu à n’être plus que la partie la plus haute de l’être humain, séparée de lui et élevée au rang d’être particulier, que la philosophie spéculative parvient à rendre à l’homme [{\corr tous}] les prédicats divins dont il avait été arbitrairement [{\corr dépouillé}] au profit d’un être imaginaire. \emph{Homo homini deus} est la conclusion de la philosophie de Feuerbach\footnote{ \noindent Essence du Christianisme, \emph{trad. fr., pp. 5, 310, 323, 376.}
 }.}\par
\emph{On sait l’enthousiasme que souleva chez la jeunesse allemande en rébellion contre la théologie hégélienne le pieux athéisme de l’auteur de l’\emph{Essence du Christianisme}. — « Tu es qui restitues mihi hœreditatem meam ! » lui disaient volontiers les jeunes hégéliens chez lesquels cette religion de l’Humanité trouvait de fervents adeptes.}\par
\emph{Quoique formellement opposé à Feuerbach et à Br. Bauer, contre lesquels est dirigée presque toute la partie polémique de l’\emph{Unique}, Stirner est en réalité leur continuateur immédiat.}\par
\emph{Stirner est essentiellement anti-chrétien. Son individualisme même est une conséquence de ce premier caractère. Tout son livre est une critique des bases religieuses de la vie humaine. Esprit infiniment plus rigoureux que ses prédécesseurs, la conception, au fond très religieuse, de l’Homme, ne peut le satisfaire, et sa critique impitoyable ne s’arrête que lorsqu’il a dressé sur les ruines du monde religieux et « hiérarchique » l’individu autonome, sans autre règle que son égoïsme.}\par
\emph{D’après Feuerbach, les attributs de l’homme jugés à tort ou à raison les plus élevés lui avaient été arrachés pour en doter un être imaginaire « supérieur » ou « suprême », nommé Dieu. Mais qu’est-ce que l’Homme de Feuerbach, reprend Stirner, sinon un nouvel être imaginaire formé en séparant de l’individu certains de ses attributs, et qu’est-ce que l’Homme, sinon un nouvel « être suprême »  ? L’Homme n’a aucune réalité, tout ce qu’on lui attribue est un vol fait à l’individu. Peu importe  que vous fondiez ma moralité et mon droit et que vous régliez mes relations avec le monde des choses et des hommes sur une volonté divine révélée ou sur l’essence de l’homme ; toujours vous me courbez sous le joug \emph{étranger} d’une puissance supérieure, vous humiliez ma volonté aux pieds d’une sainteté quelconque, vous me proposez comme un devoir, une vocation, un idéal sacrés cet esprit, cette raison et cette vérité qui ne sont en réalité que mes instruments. « L’au-delà extérieur est balayé, mais l’au-delà intérieur reste et nous appelle à de nouveaux combats. » La prétendue « immanence » n’est qu’une forme déguisée de l’ancienne « transcendance ». Le libéralisme politique qui me soumet à l’Etat, le socialisme qui me subordonne à la Société, et l’humanisme de Br. Bauer, de Feuerbach et de Ruge qui me réduit à n’être plus qu’un rouage de l’humanité ne sont que les dernières incarnations du vieux sentiment chrétien qui toujours soumet l’individu à une généralité abstraite ; ce sont les dernières formes de la domination de l’esprit, de la Hiérarchie. « Les plus récentes révoltes contre Dieu ne sont que des insurrections théologiques.} »\par
\emph{En face de ce rationalisme chrétien dont il a exposé la genèse et l’épanouissement dans la première partie de son livre, Stirner, dans la seconde, dresse l’individu, le moi corporel et unique de qui tout ce dont on avait fait l’apanage de Dieu et de l’Homme redevient la \emph{propriété}.}\par
\emph{Le Dieu, avait dit Feuerbach, n’est autre chose que l’Homme. — Mais l’Homme lui-même, répond Stirner, est un fantôme, qui n’a de réalité qu’en Moi et par Moi ; l’humain n’est qu’un des éléments constitutifs de mon individualité et est le mien, de même que l’Esprit est mon esprit et que ma chair est ma chair. Je suis le centre du monde, et le monde (monde des choses, des hommes et des idées), n’est que ma propriété, dont mon égoïsme souverain use selon son bon plaisir et selon ses forces.}\par
\emph{Ma propriété est ce qui est en mon pouvoir ; mon droit, s’il n’est pas une permission que m’accorde un être extérieur  et « supérieur » à moi, n’a d’autre limite que ma force et n’est que ma force. Mes relations avec les hommes, que nulle puissance religieuse, c’est-à-dire extérieure, ne peut régler, sont celles d’égoïste à égoïste : je les emploie et ils m’emploient, nous sommes l’un pour l’autre un instrument ou un ennemi}.\par
\emph{Ainsi se clot par une négation radicale la lutte de la gauche hégélienne contre l’esprit théologique ; et, du même coup, sont convaincus de devoir tourner sans fin dans un cercle vicieux ceux qui attaquent l’Eglise ou l’Etat au nom de la morale ou de la justice : tous en appelant à une autorité extérieure à la volonté égoïste de l’individu en appellent en dernière analyse à la volonté d’un « dieu » : « Nos athées sont de pieuses gens »\footnote{ \noindent \emph{Remarquons en passant que Stirner, qui ne connut — et assez superficiellement — que les premiers travaux de Proudhon, répond par avance à la pensée fondamentale de sa \emph{Justice dans la Révolution et dans l'Église} et repousse toute opposition entre la justice purement humaine de la Révolution et la radicale incapacité de justice de l'Eglise. La dignité humaine, source de justice de Proudhon, vaut la dignité, source de moralité de Mill ; à moins d'être un retour à la révélation, elles sont l'une et l'autre également incapables de justifier toute idée de sanction et d'obligation ; la justice de l'un comme la morale de l'autre sont religieuses ou ne sont ni morale ni justice. Comme le dit excellemment Guyau, la morale des utilitaires (et la justice de Proudhon est dans le même cas) n'a jamais pu expliquer que le \emph{faire} moral (la possibilité d'être amené à poser des actes conformes à la moralité), et non le \emph{vouloir} moral (la moralité) ; la physique des mœurs ne peut devenir une morale que si elle en appelle inconsciemment à une « table des valeurs » religieuse. Il n'est d'autre réfutation de la morale théologique que la suppression de la théologie — et de la morale.}
 }.}\par
\emph{Si Feuerbach s’était rallié théoriquement à la « morale de l’égoïsme », ce n’avait été de sa part qu’une inadvertance, résultant de sa polémique anti-religieuse, car sa doctrine de l’amour devait l’en tenir éloigné. Stirner ne tombe pas dans de pareilles inconséquences, et il tire avec une logique impitoyable toutes les conclusions renfermées dans les prémisses posées par ses prédécesseurs. Une fois renversé [{\corr le}]  monde de l’esprit, du sacré et de l’amour, en un mot le monde chrétien, l’intraitable droiture de sa pensée devait le conduire à ne plus voir dans les rapports entre les hommes que le choc des individualités égoïstes et la lutte de tous contre tous. Son individualisme anti-chrétien et anti-idéaliste peut légitimement taxer de faux individualisme toutes les doctrines auxquelles on attribue généralement ce caractère ; en effet, si elles affranchissent l’individu des dogmes et secouent en apparence toute autorité, elles ne le laissent pas moins serviteur de l’esprit, de la vérité et de l’\emph{objet :} pour l’Unique, l’esprit n’est que mon arme, la vérité est ma créature, et l’objet n’est que mon objet. Libéraux, socialistes, humanitaires, tous ces amants de la liberté n’ont jamais compris le mot « ni dieu ni maître » : « Possesseurs d’esclaves aux rires méprisants, ils sont eux-mêmes — des esclaves}\footnote{ \noindent Das unwahre Prinzip unserer Erziehung, \emph{Kl. Schriften, éd. Mackay, p. 24.}
 }. »\par

\asterism

\noindent \emph{Il est superflu de nous étendre longuement sur les détails de la pensée de Stirner ; une simple lecture de son livre les fera connaître mieux qu’aucune analyse. Mais toute lecture est une traduction en une langue qui va s’écartant de plus en plus de celle de l’auteur ; les œuvres philosophiques les plus solidement pensées, si elles n’ont pas à en craindre d’autre, ne peuvent échapper à cette « réfutation ». L’induction scientifique, impuissante contre le réseau serré des déductions, en ronge chaque maille tour à tour, les points de vue se modifient, les termes reçoivent des définitions nouvelles, et, finalement, l’ossature logique de l’œuvre demeure, mais la chair et le sang en ont changé et elle vit d’une vie toute nouvelle. Tel est le sort habituel de tous les travaux purement dialectiques, et Stirner y est soumis. Si l’œuvre du moraliste reste inattaquable, il faut aujourd’hui, pour juger les conclusions de son livre, faire subir une espèce de remise au point à son principe, l’individu}.\par
 \emph{Il importerait donc de dégager la véritable signification de l’unique et de son égoïsme, et de nous demander ce qu’\emph{est} à proprement parler, c’est-à-dire dans le domaine de l’action et de la vie et non plus de la théorie et du livre, l’individualisme de Stirner. Nous comprendrons ainsi ce qu’il peut devenir en nous, à quelle tendance il répond et quel rôle il peut remplir dans le mouvement actuel des esprits. Nous savons ce qu’il a détruit ; mais quel sol a-t-il mis à nu sous les ruines du monde moral ? Pouvons-nous espérer y voir lever encore une moisson, ou bien son « égoïsme » a-t-il creusé sous la vie sociale un gouffre impossible à combler à moins de nouveaux mensonges et de nouvelles illusions ? Que faut-il entendre, en un mot, par le « nihilisme » de Stirner ?}\par
\emph{Et d’abord, qu’est-ce que l’Unique ? La polémique qui suivit l’apparition de l’œuvre de Stirner est précieuse, en ce quelle achève de fixer le sens exact qu’y attachait son auteur.}\par
\emph{L’Unique est-il une conception nouvelle du Moi, le principe nouveau d’une doctrine nouvelle (un complément, par exemple, de la philosophie de Fichte\footnote{ \noindent \emph{Stirner avait nettement répudié toute parenté entre son moi corporel et passager et le Moi absolu de Fichte, ce qui n'empêche pas Ed. von Hartmann de voir « dans son \emph{absolutisation} du Moi la véritable conséquence pratique du monisme subjectif de Fichte ». C'est une des plus lourdes méprises dont notre penseur ait été victime. Je ne cite que pour mémoire l'opinion du critique auquel Stirner et l'Unique rappellent Machiavel et le Prince, et le critique français dont je ne retrouve pas le nom et dont tous ceux qui s'occupent de Stirner répètent la phrase sur « ce livre qu'on quitte monarque ». Voyez aussi, à titre de curiosité, un discours de H. von Bülow où Stirner est comparé à Bismarck.}
 }) ? C’est ainsi que le comprirent les critiques. Feuerbach, Szeliga et Hess en 1845, Kuno Fischer en 1847, attaquèrent Stirner en se plaçant à ce point de vue, et parlèrent à l’envi d’un « moi principe », d’un « égoïsme absolu », d’une « dogmatique de l’égoïsme », d’un « égoïsme en système », etc., tous virent dans l’individu une idée, un principe ou un idéal qui s’opposait à l’Homme.}\par
\emph{Stirner leur répond : Le moi que tu penses n’est  qu’un agrégat de prédicats, aussi peux-tu le concevoir, c’est-à-dire le définir et le distinguer d’autres concepts voisins. Mais \emph{toi} tu n’es pas définissable, \emph{toi} tu n’es pas un concept, car tu n’as aucun contenu logique ; et c’est de toi, l’indisable et l’impensable, que je parle ; l’Unique ne fait que te désigner, comme te désigne le nom qu’on t’a donné en te baptisant, sans dire ce que tu es ; dire que tu es unique revient à dire que tu es toi ; l’unique n’est pas un concept, une notion, car il n’a aucun contenu logique : tu es son contenu, toi, le « qui » et le « il » de la phrase. Dans la réalité, l’unique c’est toi, toi contre qui vient se briser le royaume des pensées ; dans ce royaume des pensées, l’unique n’est qu’une phrase — et une phrase vide, c’est-à-dire pas même une phrase ; mais « cette phrase est la pierre sous laquelle sera scellée la tombe de notre monde des phrases, de ce monde au commencement duquel était le mot. » Et l’individu n’étant pas une idée que j’oppose à l’Homme, l’unique n’étant que toi, ton « égoïsme » n’est nullement un impératif, un devoir ou une vocation ; c’est, comme l’unique, une — phrase, « mais c’est la dernière des phrases possible, et destinée à mettre fin au règne des phrases.} »\par
\emph{L’Unique est donc pour Stirner le moi \emph{gedankenlos}, qui n’offre aucune prise à la pensée et s’épanouit en deçà ou au delà de la pensée logique ; c’est le néant logique d’où sortent comme d’une source féconde mes pensées et mes volontés. — Traduisons, et poursuivant l’idée de Stirner un peu plus loin qu’il ne le fit, nous ajouterons : c’est ce moi profond et non rationnel dont un penseur magnifique et inconsistant a dit par la suite : « O mon frère, derrière tes sentiments et tes pensées se cache un maître puissant, un sage inconnu ; il se nomme toi-même (Selbst). Il habite ton corps, il est ton corps}\footnote{ \noindent \emph{Fr. Nietzsche, \emph{Also sprach Zarathustra}, p. 47. Nous laissons de côté tout parallèle entre Nietzsche et Stirner ; il y a de telles affinités entre l'\emph{Unique et sa Propriété} et la partie critique de l'œuvre du chantre de Zarathustra qu'il est difficile de se convaincre, quoique le fait soit à peu près prouvé, que ce dernier ne connut point Stirner. Sa destruction de la « table des valeurs » actuellement admises est d’un Stirner qui, au lieu de Hégel, aurait eu Schopenhauer pour éducateur. Remarquons en passant que ce que Lange appelle volonté, « volonté à laquelle, dit-il (\emph{Hist. du Mat.}, tr.fr., II, p. 98), Stirner donne une valeur telle qu’elle nous apparaît comme la force fondamentale de l’être humain », semble correspondre exactement à la « Wille zur Macht » de Nietzsche.}
 }. »\par
 \emph{Telle est la source vive que Stirner a fait jaillir de la « dure roche » de l’individualité, et tel est, je pense, le fond positif et fécond de sa pensée. Mais ce fond, le logicien, ancien disciple de Hégel, ne fit que l’entrevoir et l’affirmer ; il soupçonne « la signification d’un cri de joie sans pensée, signification formidable qui ne put être reconnue tant que dura la longue nuit de la pensée et de la foi » ; mais si, matelot aventureux errant sur l’océan de la pensée, il a senti passer la grande voix venue de la terre qui clame que les dieux sont morts, s’il a entendu les flots se briser contre la côte prochaine, son œil n’a pas aperçu la terre à travers les brumes de l’aube ; un autre y posera son pied de rêveur dionysien, mais c’est à nous, Anarchistes, à aborder au port.}\par
\emph{C’est ainsi, semble-t-il, qu’il faut comprendre le « nihilisme » de Stirner : à la conception chrétienne du monde, à la philosophie « qui inscrit sur son bouclier la négation de la vie », à cette « pratique du nihilisme\footnote{ \noindent \emph{Fr. Nietzsche}, Der Antichrist, \emph{VII.}
 } », il répond en inscrivant sur le sien la négation de l’esprit et aboutit à un nihilisme purement théorique.}\par
\emph{Mais « est-ce à dire, demande-t-il, que par son égoïsme Stirner prétende nier toute généralité et faire table rase, par une simple dénégation, de toutes les propriétés organiques dont pas un individu ne peut s’affranchir ? Est-ce à dire qu’il veuille rompre tout commerce avec les hommes et se suicider en se mettant pour ainsi dire en chrysalide en lui-même ? » Et il répond : « Il y a dans le livre de Stirner un « par conséquent » capital, une conclusion importante qu’il est en vérité possible de lire entre les lignes, mais qui a échappé aux yeux des philosophes, parce que  les dits philosophes ne connaissent pas l’homme réel et ne se connaissent pas comme hommes réels, mais qu’ils ne s’occupent que de l’Homme, de l’Esprit en soi, a priori, des noms et jamais des choses et des personnes. C’est ce que Stirner \emph{exprime négativement} dans sa critique acérée et irréfutable, lorsqu’il analyse les illusions de l’idéalisme et démasque les mensonges du dévouement et de l’abnégation...\footnote{ \noindent Die philosophischen Reactionaere, \emph{Kl. Schriften, édit. Mackay, p. 182, 183.}
 } »}\par
\emph{Je souligne ces mots \emph{expression négatives}, et je demande : quelle serait donc la traduction positive de son œuvre ? Quel « par conséquent » peut-on logiquement en déduire, et de quelle suite positive est-elle susceptible ?}\par
\emph{Telle est la question que s’est posée entre autres Lange, qui regrette que Stirner n’ait pas complété son livre par une seconde partie et suppose que « pour sortir de mon moi limité, je puis, à mon tour, créer une espèce quelconque d’idéalisme comme l’expression de ma volonté et de mon idée. » M. Lichtenberger, de son côté, dans une courte notice consacrée à Stirner\footnote{ \noindent Nouvelle revue, \emph{15 [{\corr juillet}] 1894.}
 } se demande quelle forme sociale pourrait résulter de la mise en pratique de ses idées.}\par
\emph{Ce sont là, je crois — et j’aborde ici le point le plus délicat de cette étude — des questions que l’on ne peut pas se poser ; je pense que du livre de Stirner aucun système social ne peut logiquement sortir (en entendant par logiquement ce que lui-même aurait pu en tirer et non ce que nous bâtissons sur le terrain par lui déblayé) ; comme Samson, il s’est enseveli lui-même sous les ruines du monde religieux renversé. Pourquoi ? C’est ce qu’il me reste à montrer}.\par
« \emph{Amis, dit-il quelque part\footnote{ \noindent Die Mysterien von Paris, \emph{Kl. Schriften, éd. Mackay, p. 101.}
 }, notre temps n’est pas malade, mais il est vieux et sa dernière heure a sonné ; ne le tourmentez donc point de vos remèdes, mais soulagez  son agonie en l’abrégeant et laissez-le — mourir. » Cette société lasse qui meurt de ses mensonges et dont il compte les pulsations qui s’éteignent, que viendra-t-il à sa suite, il l’ignore. Il a pu détruire les anciennes valeurs, mais il ne peut en créer de nouvelles ; du temple du dieu qu’il a renversé, c’est à d’autres à reconnaître les matériaux épars dont il ne connut que l’ordonnance, et à rebâtir avec les décombres la maison des hommes.}\par
\emph{Cette impuissance, un dernier exemple va nous la faire toucher du doigt et nous en livrer le secret. Dans un chapitre auquel il attachait la plus grande importance, Stirner nous trace les grandes lignes de l’association des égoïstes, telle qu’il la conçoit résultant du libre choc des individus, opposée à la société actuelle religieuse et hiérarchique. Or, il a surabondamment démontré auparavant que l’amour, le désintéressement, le loyalisme, etc., ne sont que des travestissements de l’égoïsme, que la piété du croyant, le souci de légalité du bourgeois et la tendresse de l’amant ne sont que des procédés, à vrai dire souvent méconnus, par lesquels l’un exploite son dieu, l’autre l’état ou sa maîtresse ; de sorte que la société actuelle réalise en somme l’état de lutte de tous contre tous auquel son analyse le conduit. Elle ne diffère de l’association des égoïstes que par le caractère des armes employées : l’égoïsme de ses uniques s’est simplement débarrassé de son vieil appareil de guerre ; ses combattants, comme les soldats des armées modernes, ne marchent plus à l’ennemi en brandissant des boucliers ornés de figures terribles destinées à effrayer l’ennemi quand elles ne les épouvantent pas eux-mêmes ; aucun dieu, aucune déesse ne descend plus du ciel pour combattre à leurs côtés sous les traits augustes de la Morale, de la Justice ou de l’Amour. L’égoïsme de Stirner est, pour tout dire en un mot, un égoïsme — rationnel.}\par
\emph{Le destructeur du rationalisme est lui-même, par la forme logique de son esprit, un rationaliste, et l’adversaire passionné du libéralisme reste un libéral. Stirner rationaliste poursuit jusque dans ses derniers retranchements  l’idée de Dieu et en démasque les dernières métamorphoses, mais il n’aboutit fatalement qu’à une négation : l’individu et l’égoïsme, Stirner \emph{libéral} sape au nom de l’individu les fondements de l’Etat, mais, ce dernier détruit, il n’aboutit qu’à une nouvelle négation : anarchie ne pouvant signifier pour lui que désordre, si l’Etat, régulateur de la concurrence, vient à disparaître, à celle-ci ne peut succéder que la guerre de tous contre tous.}\par
\emph{Cette conception toute formelle de l’individu nous explique le caractère purement négatif de ce qu’on pourrait appeler la « doctrine » de Stirner, c’est-à-dire de la partie logiquement critique de son œuvre ; c’est ce rationalisme et ce libéralisme conséquents, c’est-à-dire radicalement destructeurs, qui me permettaient tantôt de nier la possibilité de donner à l’Unique le « complément positif » dont Lange regrette l’absence. Mais ce serait, je crois, mutiler la pensée de son auteur et méconnaître l’importance de l’\emph{Unique et sa Propriété} de n’y voir que l’œuvre du logicien nihiliste. « Stirner, dit-il lui-même\footnote{ \noindent Die philosophischen Reactionaere, \emph{Kl. Schriften, éd. Mackay, p. 183.}
 }, ne présente son livre que comme l’expression souvent maladroite et incomplète de ce qu’il voulut ; ce livre est l’œuvre laborieuse des meilleures années de sa vie et il convient cependant que ce n’est qu’un à peu près. Tant il eut à lutter contre une langue que les philosophes ont corrompue, que tous les dévôts de l’Etat, de l’Eglise, etc. ont faussée, et qui est devenue susceptible de confusions d’idées sans fin.} »\par
\emph{D’autres ont mis en lumière l’importance formidable qu’ont prise dans l’Etat les \emph{facteurs régulateurs sociaux} aux dépens des facteurs actifs et producteurs. En démontant la « machine de l’Etat » rouage par rouage et en montrant dans cette police sociale qui s’étend du roi jusqu’au garde-champêtre et au juge de village un instrument de guerre au service des vainqueurs contre les vaincus, sans autre rôle que de défendre l’état de choses existant, c’est-à-dire  de perpétuer l’écrasement du faible actuel par le fort actuel, ils ont mis en évidence son caractère essentiellement inhibiteur et stérilisant. Loin de pouvoir être un ressort pour l’activité individuelle, l’Etat [{\corr ne}] peut que comprimer, paralyser et annihiler les efforts de l’individu.}\par
\emph{Stirner, de son côté, met en lumière l’étouffement des forces vives de l’individu par la végétation parasite et stérile des facteurs régulateurs moraux. Il dénonce dans la justice, la moralité et tout l’appareil des sentiments « chrétiens » une nouvelle police, une police morale, ayant même origine et même but que la police de l’Etat : prohiber, réfréner et immobiliser. Les veto de la conscience s’ajoutent aux veto de la loi ; grâce à elle, la force d’autrui est sanctifiée et s’appelle le droit, la crainte devient respect et vénération, et le chien apprend à lécher le fouet de son maître.}\par
\emph{Les premiers disaient : que l’individu puisse se réaliser librement sans qu’aucune contrainte extérieure s’oppose à la mise en œuvre de ses facultés : l’activité libre seule est féconde. Stirner répond : que l’individu puisse vouloir librement et ne cherche qu’en lui seul sa règle, sans qu’aucune contrainte intérieure s’oppose à l’épanouissement de sa personnalité : seule l’individuelle volonté est créatrice.}\par
\emph{Mais l’individualisme ainsi compris n’a encore que la valeur négative d’une révolte, et n’est que la réponse de ma force à une force ennemie. L’individu n’est que le bélier logique à l’aide duquel on renverse les bastilles de l’autorité : il n’a aucune réalité et, n’est qu’un dernier fantôme rationnel, le fantôme de l’Unique.}\par
\emph{Cet Unique où Stirner aborda sans reconnaître le sol nouveau sur lequel il posait le pied, croyant toucher le dernier terme de la critique et l’écueil où doit sombrer toute pensée, nous avons aujourd’hui appris à le connaître : Dans le moi non rationnel fait d’antiques expériences accumulées, gros d’instincts héréditaires et de passions, et siège de notre « grande volonté » opposée à la « petite volonté » de l’individu égoïste, dans cet « Unique » du logicien, la science nous fait entrevoir le fond commun à tous  sur lequel doivent se lever, par delà les mensonges de la fraternité et de l’amour chrétiens une solidarité nouvelle, et par delà les mensonges de l’autorité et du droit un ordre nouveau.}\par
\emph{C’est sur cette terre féconde que Stirner met à nu, que le grand négateur tend par-dessus cinquante ans la main aux anarchistes d’aujourd’hui.}\par


\signed{R.-L. R{\scshape eclaire}}

\dateline{\emph{Décembre 1899.}}
 \mainmatter  \section[{Je n’ai basé ma cause sur rien}]{Je n’ai basé ma cause sur rien}\renewcommand{\leftmark}{Je n’ai basé ma cause sur rien}

\noindent Quelle cause n’ai-je pas à défendre ? Avant tout, ma cause est la bonne cause, c’est la cause de Dieu, de la Vérité, de la Liberté, de l’Humanité, de la Justice ; puis, celle de mon Prince, de mon Peuple, de ma Patrie ; ce sera ensuite celle de l’Esprit, et mille autres encore. Mais que la cause que je défends soit \emph{ma} cause, ma cause à Moi, jamais ! « Fi, l’égoïste qui ne pense qu’à lui ! »\par
Mais ceux-là dont nous devons prendre à cœur les intérêts, ceux-là pour qui nous devons nous dévouer et nous enthousiasmer, comment donc entendent-ils leur cause ? Voyons un peu.\par
Vous qui savez de Dieu tant et de si profondes choses, vous qui pendant des siècles avez « exploré les profondeurs de la Divinité » et avez plongé vos regards jusqu’au fond de son cœur, vous pourrez bien nous dire comment Dieu entend la « divine cause » que nous sommes appelés à servir. Ne nous celez point les desseins du Seigneur. Que veut-il ? Que poursuit-il ? A-t-il, comme ce nous est prescrit à nous, embrassé une cause étrangère et s’est-il fait le champion de la Vérité et de l’Amour ? Cette absurdité vous révolte ; vous nous enseignez que Dieu étant lui-même tout Amour et toute Vérité, la cause de la  Vérité et celle de l’Amour se confondent avec la sienne et ne lui sont pas étrangères. Il vous répugne d’admettre que Dieu puisse être comme nous, pauvres vers, et faire sienne la cause d’un autre. « Mais Dieu embrasserait-il la cause de la Vérité, s’il n’était pas lui-même la Vérité ? » Dieu ne s’occupe que de \emph{sa }cause, seulement il est tout dans tout, de sorte que tout est \emph{sa} cause. Mais nous ne sommes pas tout dans tout et notre cause est bien mince, bien méprisable ; aussi devons-nous « servir une cause supérieure \emph{ ».} — Voilà qui est clair : Dieu ne s’inquiète que du sien, Dieu ne s’occupe que de lui-même, ne pense qu’à lui-même et n’a que lui-même en vue ; malheur à ce qui contrarie ses desseins. Il ne sert rien de supérieur et ne cherche qu’à se satisfaire. La Cause qu’il défend est purement — \emph{égoïste !}\par
Et l’Humanité, dont nous devons aussi défendre les intérêts comme les nôtres ; quelle cause défend-elle ? Celle d’un autre ? Une supérieure ? Non. L’Humanité ne voit qu’elle-même, l’Humanité n’a d’autre but que l’Humanité ; sa cause, c’est elle-même. Pourvu qu’elle se développe, peu lui importe que les individus et les peuples succombent à son service ; elle tire d’eux ce qu’elle en peut tirer et lorsqu’ils ont accompli la tâche qu’elle réclamait d’eux, elle les jette en guise de remerciement dans la hotte de l’histoire. La cause que défend l’Humanité n’est-elle pas purement — \emph{égoïste ?}\par
Inutile de poursuivre, et de montrer à propos de chacune de ces choses qui nous appellent à leur défense qu’il ne s’agit pour elles que d’elles et non de nous, de leur bien et non du nôtre. Passez vous-mêmes les autres en revue, et dites si la Vérité, la Liberté, la Justice, etc., s’inquiètent de vous autrement que pour réclamer votre enthousiasme et vos services. Soyez des serviteurs zélés, rendez leur hommage, c’est tout ce qu’elles demandent.\par
Voyez ce Peuple que sauvent des patriotes dévoués ; les patriotes tombent sur le champ de bataille ou  crèvent de faim et de misère ; qu’en dit le Peuple ? Le Peuple ? fumé de leurs cadavres il devient un « peuple florissant » ! Les individus sont morts « pour la grande cause du Peuple », qui leur envoie quelques tardives phrases de reconnaissance et — garde pour lui tout le profit. Cela me paraît d’un égoïsme assez lucratif.\par
Mais contemplez maintenant ce Sultan qui soigne si tendrement « les siens » ; N’est-il pas l’image du plus pur dévouement, et sa vie n’est-elle pas un perpétuel sacrifice pour les siens ? Hé oui, pour « les siens » ! En veux-tu faire l’essai ? Montre que tu n’es pas « le sien », mais « le tien » ; refuse-toi à son égoïsme : tu iras aux galères. Le Sultan n’a basé sa cause sur rien d’autre que sur lui-même ; il est tout dans tout, il est l’Unique et ne permet à personne de ne pas être un des « siens ».\par
Ces illustres exemples ne vous suggèrent-ils rien ? Ne vous invitent-ils pas à penser que l’Egoïste pourrait bien avoir raison ? Pour ma part, j’y vois une leçon ; au lieu de continuer à servir avec désintéressement ces grands égoïstes, je serai plutôt moi-même l’Egoïste.\par
Dieu et l’Humanité n’ont basé leur cause sur rien, sur rien qu’eux-mêmes. Je baserai donc ma cause sur \emph{Moi :} aussi bien que Dieu, je suis la négation de tout le reste, je suis pour moi tout, je suis l’Unique.\par
Si Dieu et l’Humanité sont, comme vous rassurez, riches de ce qu’ils renferment au point d’être pour eux-mêmes tout dans tout, je m’aperçois qu’il me manque à moi beaucoup moins encore et que je n’ai pas à me plaindre de ma « vanité ». Je ne suis pas rien dans le sens de « rien que vanité », mais je suis le Rien créateur, le Rien dont je tire tout.\par
Foin donc de toute cause qui n’est pas entièrement, exclusivement la Mienne ! Ma cause, dites-vous, devrait au moins être la « bonne cause » ? Qu’est-ce qui est bon, qu’est-ce qui est mauvais ? Je suis moi-même  ma cause, et je ne suis ni bon ni mauvais, ce ne sont là pour moi que des mots.\par
Le divin regarde Dieu, l’humain regarde l’Homme. Ma cause n’est ni divine ni humaine, ce n’est ni le vrai, ni le bon, ni le juste, ni le libre, c’est — le \emph{Mien ;} elle n’est pas générale, mais — \emph{unique}, comme je suis unique.\par
Rien n’est, pour Moi, au-dessus de Moi !
 \section[{A. Première partie. L’Homme}]{A. Première partie \\
L’Homme}\renewcommand{\leftmark}{A. Première partie \\
L’Homme}

 \noindent « L’Homme est pour l’homme l’Etre suprême, » dit Feuerbach.\par
« L’Homme vient seulement d’être découvert, » dit Bruno Bauer.\par
Examinons de plus près cet Etre suprême et cette nouvelle découverte.\par
 \subsection[{A.I. une vie d’homme}]{A.I. \\
une vie d’homme}
\noindent Dès l’instant ou il ouvre les yeux à la lumière, l’homme cherche à se dégager et à se conquérir au milieu du chaos où il roule confondu avec le reste du monde. Mais tout ce que touche l’enfant se rebelle contre ses tentatives et affirme son indépendance. Chacun faisant de soi le centre et se heurtant de toutes parts à la même prétention chez tous les autres, le conflit, la lutte pour l’autonomie et la suprématie est inévitable.\par
\emph{Vaincre} ou \emph{être vaincu}, — pas d’autre alternative. Le vainqueur sera le \emph{maître}, le vaincu sera l’\emph{esclave : }l’un jouira de la souveraineté et des « droits du seigneur », l’autre remplira plein de respect et de crainte ses « devoirs de sujet ».\par
Mais les \emph{adversaires} ne désarment pas ; chacun d’eux reste aux aguets, épiant les faiblesses de l’autre, les enfants celles des parents, les parents celles des enfants (la peur, par exemple) ; celui qui ne donne pas les coups les reçoit.\par
Voici la voie qui dès l’enfance nous conduit à l’affranchissement : nous cherchons à pénétrer au fond des choses, ou « derrière les choses » ; pour cela nous épions leur point faible (en quoi les enfants sont, comme on le sait, guidés par un instinct qui ne  les trompe pas), nous nous plaisons à briser ce qui nous tombe sous la main, nous prenons plaisir à touiller les coins interdits, à explorer tout ce qui est voilé et soustrait à nos regards ; nous essayons sur tout nos forces. Et, le secret enfin découvert, nous nous sentons sûrs de nous ; si par exemple nous sommes arrivés à nous convaincre que le fouet ne peut rien contre notre obstination, nous ne le craignons plus, « nous avons passé l’âge de la férule ».\par
Derrière les verges se dressent, plus puissantes qu’elles, notre audace, notre obstinée volonté ! Nous nous glissons doucement derrière tout ce qui nous semblait inquiétant, derrière la force redoutée du fouet, derrière la mine fâchée de notre père, et derrière tout nous découvrons notre — Ataraxie, c’est-à-dire que plus rien ne nous trouble, plus rien ne nous effraie ; nous prenons conscience de notre pouvoir de résister et de vaincre, nous découvrons que rien ne peut nous contraindre.\par
Ce qui nous inspirait crainte et respect, loin de nous intimider, nous encourage ; derrière le rude commandement des supérieurs et des parents, plus obstinée se redresse notre volonté, plus artificieuse notre ruse. Plus nous apprenons à \emph{nous} connaître, plus nous nous rions de ce que nous avions cru insurmontable.\par
Mais que sont notre adresse, notre ruse, notre courage, notre audace, sinon — l’\emph{Esprit ?} Pendant longtemps nous échappons à une lutte qui plus tard nous mettra hors d’haleine, la lutte contre la \emph{Raison}. La plus belle enfance se passe sans que nous ayons à nous débattre contre la raison. Nous ne nous soucions point d’elle, nous n’avons avec elle nul commerce et elle n’a sur nous aucune prise. On n’obtient rien de nous en essayant de nous \emph{convaincre ;} sourds aux bonnes raisons et aux meilleurs arguments, nous réagissons au contraire vivement sous les caresses, les châtiments et tout ce qui y ressemble.\par
 Ce n’est que plus tard que commence le rude combat contre la \emph{raison}, et avec lui s’ouvre une nouvelle phase de notre vie. Enfants, nous nous étions trémoussés sans beaucoup rêver.\par
L’\emph{Esprit} est le premier aspect sous lequel se révèle à nous notre être intime, le premier nom sous lequel nous divinisons le divin, c’est-à-dire l’objet de nos inquiétudes, le fantôme, la « puissance supérieure ». Rien ne s’impose plus désormais à notre respect ; nous sommes pleins du juvénile sentiment de notre force, et le monde perd à nos yeux tout crédit, car nous nous sentons supérieurs à lui, nous nous sentons \emph{Esprit.} Nous commençons à nous apercevoir que nous avions jusqu’ici regardé le monde sans le voir, que nous ne l’avions jamais encore contemplé avec les yeux de l’Esprit.\par
C’est sur les \emph{puissances de la nature} que nous essayons nos premières forces. Nos parents nous en imposent comme des puissances naturelles ; plus tard on dit « il faudrait abandonner père et mère pour que toute puissance naturelle fût brisée ! » Un jour vient où on les quitte et où le lien se rompt. Pour l’homme qui pense, c’est-à-dire pour l’homme « spirituel », la famille n’est pas une puissance naturelle et il doit faire abstraction des parents, des frères et sœurs, etc. Si ces parents « renaissent » dans la suite comme \emph{puissances spirituelles et rationnelles}, ces puissances nouvelles ne sont plus du tout ce qu’elles étaient à l’origine.\par
Ce n’est pas seulement le joug des parents, c’est toute autorité humaine que le jeune homme secoue : les hommes ne sont plus un obstacle devant lequel il daigne s’arrêter, car « il faut obéir à Dieu plutôt qu’aux hommes ». Le nouveau point de vue auquel il se place est le — \emph{céleste}, et, vu de cette hauteur, tout le « \emph{terrestre} » recule, se rapetisse et s’efface dans une lointaine brume de mépris.\par
De là, changement radical dans l’orientation intellectuelle du jeune homme et souci chez lui exclusif  du \emph{spirituel}, tandis que l’enfant, qui ne se sentait pas encore Esprit, demeurait confiné dans la lettre des livres entre lesquels il grandissait. Le jeune homme ne s’attache plus aux choses, mais cherche à saisir les \emph{pensées} que ces choses récèlent ; ainsi, par exemple, il cesse d’accumuler pêle-mêle dans sa tête les \emph{faits} et les dates de l’histoire, pour s’efforcer d’en pénétrer l’\emph{esprit ; }l’enfant, au contraire, s’il peut bien comprendre l’\emph{enchaînement} des faits, est incapable d’en dégager les idées, l’esprit ; aussi entasse-t-il les connaissances qu’il acquiert sans suivre de plan \emph{a priori}, sans s’astreindre à une méthode théorique, bref, sans poursuivre d’idées.\par
Dans l’enfance, on avait à surmonter la résistance des \emph{lois du monde ;} à présent, quoi qu’on se propose, on se heurte à une objection de l’esprit, de la raison, de la \emph{conscience.} « Cela n’est pas raisonnable, pas chrétien, pas patriotique ! » nous crie la conscience, et — nous nous abstenons. Ce que nous redoutons, ce n’est ni la puissance vengeresse de Euménides, ni la colère de Poséidon, ni le Dieu qui verrait les choses cachées, ni la correction paternelle, c’est — la Conscience.\par
Nous sommes désormais « les serviteurs de nos pensées » ; nous obéissons à leurs ordres comme naguère à ceux des parents ou des hommes. Ce sont elles (idées, représentations, \emph{croyances}) qui remplacent les injonctions paternelles, et qui gouvernent notre vie.\par
Enfants, nous pensions déjà, mais nos pensées alors n’étaient pas incorporelles, abstraites, \emph{absolues ; }ce n’étaient point \emph{rien que des pensées}, un ciel pour soi, un pur monde de pensées, ce n’étaient point des pensées \emph{logiques}.\par
Nous n’avions au contraire d’autres pensées que celles que nous inspiraient les événements ou les choses : nous jugions qu’une chose donnée était de telle ou telle nature. Nous pensions bien « c’est Dieu  qui a créé ce monde que nous voyons », mais notre pensée n’allait pas plus loin, nous ne « scrutions » pas « les profondeurs mêmes de la Divinité ». Nous disions bien « ceci est vrai, ceci est la vérité », mais sans nous enquérir du Vrai en soi, de la Vérité en soi, sans nous demander si « Dieu est la vérité ». Peu nous importaient « les profondeurs de la Divinité, laquelle est la vérité ». Pilate ne s’arrête pas à des questions de pure logique (ou, en d’autres termes, de pure théologie) comme « Qu’est-ce que la vérité ? » et cependant, à l’occasion, il n’hésitera pas à distinguer « ce qu’il y a de vrai et ce qu’il y a de faux dans une affaire », c’est-à-dire si \emph{telle chose} déterminée est vraie.\par
Toute pensée inséparable d’un \emph{objet} n’est pas encore \emph{rien qu’une pensée}, une pensée absolue.\par
Il n’y a pas pour le jeune homme de plus vif plaisir que de découvrir et de faire sienne la \emph{pensée pure ; }la Vérité, la Liberté, l’Humanité, l’Homme, etc., ces astres brillants qui éclairent le monde des pensées, illuminent et exaltent les âmes juvéniles.\par
Mais, l’Esprit une fois reconnu comme l’essentiel, apparaît une différence : l’esprit peut être riche ou pauvre, et on s’efforce par conséquent de devenir riche en esprit : l’esprit veut s’étendre, fonder son royaume, royaume qui n’est pas de ce monde mais le dépasse ; aussi aspire-t-il à résumer en soi toute spiritualité. Tout esprit que je suis, je ne suis pas esprit \emph{parfait,} et je dois commencer par rechercher cet esprit parfait.\par
Moi qui tout à l’heure m’étais découvert en me reconnaissant esprit, je me perds de nouveau, aussitôt que, pénétré de mon inanité, je m’humilie devant l’esprit parfait en reconnaissant qu’il n’est pas à moi et en moi, mais \emph{au delà} de moi.\par
Tout dépend de l’Esprit ; mais tout esprit est-il « bon » ? L’esprit bon et vrai est l’idéal de l’esprit, le « Saint-Esprit ». Ce n’est ni le mien, ni le tien, c’est un esprit idéal, supérieur : c’est « Dieu ». « Dieu est  l’Esprit. » Et « Dieu qui est dans le ciel donnera le bon esprit à ceux qui le demandent\footnote{ \noindent Luc. {\scshape xi}, 13.
 }. »\par
L’homme fait diffère du jeune homme en ce qu’il prend le monde comme il est, sans y voir partout du mal à corriger, des torts à redresser, et sans prétendre le modeler sur son idéal. En lui se fortifie l’opinion qu’on doit agir envers le monde suivant son \emph{intérêt}, et non suivant un \emph{idéal}.\par
Tant qu’on ne voit en soi que l’Esprit, et qu’on met tout son mérite à être esprit (il ne coûte guère au jeune homme de risquer sa vie, le « corporel », pour un rien, pour la plus niaise blessure d’amour-propre), aussi longtemps qu’on n’a que des pensées, des idées qu’on espère pouvoir réaliser un jour, lorsqu’on aura trouvé sa voie, trouvé un débouché à son activité, ces pensées, ces idées que l’on possède restent provisoirement inaccomplies, irréalisées : on n’a qu’un \emph{Idéal.}\par
Mais dès qu’on se met (ce qui arrive ordinairement dans l’âge mûr,) à prendre en affection « sa guenille » et à éprouver un plaisir à être tel qu’on est, à vivre sa vie, on cesse de poursuivre l’idéal pour s’attacher à un intérêt personnel, \emph{égoïste}, c’est-à dire à un intérêt qui ne vise plus la satisfaction du seul esprit, mais le contentement de tout l’individu ; l’intérêt devient dès lors vraiment \emph{intéressé.}\par
Comparez donc l’homme fait au jeune homme. Ne vous paraît-il pas plus âpre, plus égoïste, moins généreux ? Sans doute ! Est-il pour cela plus mauvais ? Non, n’est-ce pas ? Il est simplement devenu plus positif, ou, comme vous dites aussi, plus « pratique ». Le grand point est qu’il fait de soi le centre de tout plus résolument que ne le fait le jeune homme, distrait par un tas de choses qui ne sont pas lui, Dieu, la Patrie, et autres prétextes à « enthousiasme ».\par
L’homme ainsi se découvre lui-même une \emph{seconde}  fois. Le jeune homme avait aperçu sa \emph{spiritualité}, et s’était ensuite égaré à la poursuite de l’Esprit \emph{universel} et parfait, du Saint-Esprit, de l’Homme, de l’Humanité, bref de tous les Idéaux. L’homme se ressaisit, et retrouve son esprit \emph{incarné} en lui, fait chair, et devenu quelqu’un.\par
Un enfant ne met dans ses désirs ni idée ni pensée, un jeune homme ne poursuit que des intérêts spirituels, mais les intérêts de l’homme sont matériels, personnels et égoïstes. Lorsque l’enfant n’a aucun \emph{objet} dont il puisse s’occuper, il s’ennuie, car il ne sait pas encore s’occuper de lui-même. Le jeune homme au contraire se lasse vite des objets, parce que de ces objets s’élèvent pour lui des \emph{pensées}, et qu’il s’intéresse avant tout à ses pensées, à ses rêves, qui l’occupent spirituellement : a son esprit est occupé ».\par
En tout ce qui n’est pas spirituel, le jeune homme ne voit avec mépris que des « futilités ». S’il lui arrive de prendre au sérieux les plus minces enfantillages (par exemple les cérémonies de la vie universitaire et autres formalités), c’est qu’il en saisit l’\emph{esprit,} c’est-à-dire qu’il y voit des \emph{symboles}.\par
Je me suis retrouvé derrière les choses et m’y suis découvert Esprit ; de même plus tard je me retrouve \emph{derrière les pensées}, et me découvre leur créateur et leur \emph{possesseur}.\par
A l’âge des visions, mes pensées faisaient de l’ombre sur mon cerveau, comme l’arbre sur le sol qui le nourrit ; elles planaient autour de moi comme des rêves de fiévreux, et me troublaient de leur effroyable puissance. Les pensées avaient elles-mêmes revêtu une forme corporelle, et ces fantômes je les voyais : ils s’appelaient Dieu, l’Empereur, le Pape, la Patrie, etc.\par
Aujourd’hui, je détruis ces incarnations mensongères, je rentre en possession de mes pensées, et je dis : \emph{Moi} seul ai un corps et suis quelqu’un. Je ne vois plus dans le monde que ce qu’il est pour moi ; il est à moi, il est ma propriété. Je rapporte tout à  moi. Naguère j’étais esprit et le monde était à mes yeux digne seulement de mon mépris ; aujourd’hui je suis Moi, je suis propriétaire, et je repousse ces esprits ou ces idées dont j’ai mesuré la vanité. Tout cela n’a pas plus de pouvoir sur Moi qu’aucune « puissance de la terre » n’en a sur l’Esprit.\par
L’enfant était réaliste, embarrassé par les choses de ce monde jusqu’à ce qu’il parvînt peu à peu à pénétrer derrière elles. Le jeune homme est idéaliste, tout occupé de ses pensées, jusqu’au jour où il devient homme fait, homme égoïste qui ne poursuit à travers les choses et les pensées que la joie de son cœur, et met au-dessus de tout son intérêt personnel. Quant au vieillard... lorsque j’en serai un il sera encore temps d’en parler.
 \subsection[{A.II. les anciens et les modernes}]{A.II. \\
les anciens et les modernes}
\noindent Comment s’est développé chacun de nous ? Quels furent ses efforts ? A-t-il réussi ? A-t-il échoué ? Quel but poursuivait-il jadis ? Quels désirs et quels projets occupent aujourd’hui son cœur ? Quels changements ont subis ses opinions ? Quels ébranlements ont éprouvés ses principes ? Bref, comment chacun de nous est-il devenu ce qu’il est aujourd’hui, ce qu’il n’était point hier ou jadis ? Chacun peut se le rappeler plus ou moins facilement, mais chacun sent surtout vivement les changements dont il a été le théâtre, quand c’est la vie d’un autre qui se déroule à ses yeux.\par
Interrogeons donc l’histoire, demandons lui quels furent le but et les efforts de nos ancêtres.\par
\subsubsection[{A.II.1. Les Anciens.}]{A.II.1. Les Anciens.}
\noindent Puisque l’usage a imposé à nos aïeux d’avant le Christ le nom d’ « Anciens », nous ne soutiendrons pas que comparés à nous, gens d’expérience, ils seraient à plus juste titre appelés des enfants\footnote{ \noindent Cf. D{\scshape escartes} : \emph{Ce qu’on nomme l’Antiquité n’était que l’enfance et la jeunesse du genre humain ; « à nous plutôt convient le nom d’anciens ; car le monde est plus vieux qu’alors, et nous avons une plus grande expérience ». (N. D. Tr.)}
 } ; nous  préférons nous incliner devant eux comme devant de vieux parents. Mais comment donc purent-ils finir par vieillir, et quel est celui dont la prétendue nouveauté parvint à les supplanter ?\par
Nous le connaissons, le novateur révolutionnaire, l’héritier impie qui profana de ses propres mains le sabbat de ses pères pour sanctifier son dimanche, et qui interrompit le cours du temps pour faire dater de lui une ère nouvelle : nous le connaissons, et nous savons que ce fut — le Chrétien. Mais reste-t-il lui-même éternellement jeune, est-il encore aujourd’hui le « Moderne », ou son tour est-il venu de vieillir, à lui qui fit vieillir les « Anciens » ?\par
Ce furent les Anciens eux-mêmes qui enfantèrent l’homme moderne qui devait les supplanter ; examinons cette genèse.\par
« Pour les Anciens, » dit Feuerbach, « le monde était une vérité. » Mais il néglige d’ajouter, ce qui est important, « une vérité derrière la fausseté de laquelle ils cherchaient, et finalement parvinrent à pénétrer. » On reconnaît bientôt ce qu’il faut penser de ces mots de Feuerbach, quand on en rapproche la parole chrétienne « ce monde vain et périssable. »\par
Jamais le Chrétien n’a pu se convaincre de la vanité de la parole divine ; il croit à son éternelle et inébranlable véracité, dont les plus profondes méditations ne peuvent que rendre le triomphe plus éclatant ; les Anciens par contre étaient pénétrés de ce sentiment que le monde et les lois du monde (les liens du sang par exemple), étaient la vérité, vérité devant laquelle devait s’incliner leur impuissance. C’est précisément ce que les Anciens avaient estimé du plus haut prix que les Chrétiens rejetèrent comme sans valeur ; c’est ce que les uns avaient proclamé  vrai, que les autres flétrirent comme un mensonge : l’idée tant exaltée de patrie perd son importance, et le Chrétien ne doit plus se regarder que comme « un étranger sur la terre \footnote{ \noindent Hébreux, {\scshape xi}, 13.
 } » ; l’ensevelissement des morts, ce devoir sacré qui inspira un chef-d’œuvre, l’\emph{Antigone} de Sophocle, ne paraît plus qu’une misère (« Laissez les morts enterrer leurs morts ») ; l’indissolubilité des liens de famille devient un préjugé dont on ne saurait assez tôt se défaire\footnote{ \noindent Marc, {\scshape x}, 29.
 }, et ainsi de suite.\par
Nous voyons donc que ce que les anciens tinrent pour la vérité était le contraire même de ce qui passa pour la vérité aux yeux des modernes ; les uns crurent au naturel, les autres au spirituel ; les uns aux choses et aux lois de la terre, les autres à celles du ciel (la patrie céleste, « la Jérusalem de là haut », etc.). Etant donné que la pensée moderne ne fut que l’aboutissement et le produit de la pensée antique, reste à examiner comment était possible une telle métamorphose.\par
Ce furent les Anciens eux-mêmes qui finirent par faire de leur vérité un mensonge.\par
Remontons aux plus belles années de l’antiquité, au siècle de Périclès : c’est alors que commença la sophistique, et que la Grèce fit un jouet de ce qui avait été pour elle jusqu’alors l’objet des plus graves méditations.\par
Les pères avaient été trop longtemps courbés sous le joug inexorable des réalités pour que ces dures expériences n’apprissent à leurs descendants à \emph{se connaître}. C’est avec une assurance hardie que les S{\scshape ophistes} poussent le cri de ralliement : « Ne t’en laisse pas imposer ! », et qu’ils exposent leur doctrine : « Use en toute occasion de ton intelligence, de la finesse, de l’ingéniosité de ton esprit ; c’est grâce à  une intelligence solide et bien exercée qu’on se tire le mieux d’affaire dans le monde, qu’on s’y assure le meilleur \emph{sort}, la plus belle \emph{vie.} » Ils reconnaissent donc dans l’esprit la véritable arme de l’homme contre le monde ; c’est ce qui leur fait tant priser la souplesse dialectique, l’adresse oratoire, l’art de la controverse. Ils proclament qu’il faut en toute occasion recourir à l’esprit, mais ils sont encore bien loin de sanctifier l’esprit, car ce dernier n’est pour eux qu’une arme, un \emph{moyen}, ce que sont pour les enfants la ruse et l’audace. L’esprit c’est pour eux l’\emph{intelligence}, l’infaillible raison.\par
On jugerait aujourd’hui cette éducation intellectuelle incomplète, unilatérale, et l’on ajouterait : Ne formez pas uniquement votre intelligence, formez aussi votre cœur. C’est ce que fit S{\scshape ocrate}.\par
Si le cœur, en effet, n’était point affranchi de ses aspirations naturelles, s’il restait empli de son contenu fortuit, d’impulsions désordonnées soumises à toutes les influences extérieures, il ne serait que le foyer des convoitises les plus diverses, et il arriverait fatalement que la libre intelligence, asservie à ce « mauvais cœur », se prêterait à réaliser tout ce qu’en souhaiterait la malice.\par
Aussi Socrate déclare-t-il qu’il ne suffit pas d’employer en toutes circonstances son intelligence, mais que la question est de savoir à quel but il sied de l’appliquer. Nous dirions aujourd’hui que ce but doit être « le Bien » ; mais poursuivre le bien c’est être — moral : Socrate est donc le fondateur de l’éthique.\par
Le principe de la sophistique conduisait à admettre pour l’homme le plus aveuglément esclave de ses passions la possibilité d’être un sophiste redoutable, capable, grâce à la puissance de son esprit, de tout ordonner et façonner au gré de son cœur grossier. Quelle est l’action en faveur de laquelle on ne peut invoquer « de bonnes raisons » ? Tout n’est-il pas soutenable ?\par
 C’est pour cela que Socrate ajoute : pour que l’on puisse priser votre sagesse, il faut que vous ayez « un cœur pur. » Alors commence la seconde période de l’affranchissement de la pensée grecque, la période de la \emph{pureté du cœur}. La première finit avec les Sophistes, lorsqu’ils eurent proclamé la puissance illimitée de l’intelligence.\par
Mais le cœur prend toujours le parti du monde ; il est son serviteur, toujours agité de passions terrestres. Il fallait dès lors dégrossir ce cœur inculte : ce fut le temps de l’\emph{éducation du cœur}. Mais quelle éducation convient au cœur ?\par
L’intelligence en est arrivée à se jouer librement de tout le contenu de l’esprit, dont elle est une face ; c’est là aussi ce qui menace le cœur : devant lui va bientôt s’écrouler tout ce qui appartient au monde, si bien que, finalement, famille, chose publique, patrie, tout sera abandonné pour lui, c’est-à-dire pour la \emph{Félicité,} pour la félicité du cœur.\par
L’expérience journalière enseigne que la raison peut avoir depuis longtemps renoncé à une chose, alors que le cœur bat et battra pour elle encore pendant bien des années. De même, si complètement que l’intelligence sophistique se fût rendue maîtresse es antiques Forces naguère toutes-puissantes, il restait encore, afin qu’elles n’eussent plus aucune prise sur l’homme, à les expulser du cœur où elles régnaient sans conteste.\par
Cette guerre, ce fut Socrate qui la déclara, et la paix ne fut signée que le jour où mourut le monde antique.\par
Avec Socrate commence l’examen du cœur, et tout son contenu va être passé au crible. Les derniers, les suprêmes efforts des Anciens aboutirent à rejeter du cœur tout son contenu, et à le laisser battre à vide : ce fut l’œuvre des S{\scshape ceptiques}. Ainsi fut atteinte cette pureté du cœur qui était parvenue, au temps des Sophistes, à s’opposer à l’intelligence.  Le résultat de la culture sophistique fut que l’intelligence ne se laisse plus \emph{arrêter} par rien, celui de l’éducation sceptique, que le cœur ne se laisse plus \emph{émouvoir} par rien.\par
Aussi longtemps que l’homme reste pris dans l’engrenage du monde et embarrassé par ses relations avec lui — et il le reste jusqu’à la fin de l’antiquité parce que son cœur a jusqu’alors dû lutter pour s’affranchir du monde, — il n’est pas encore esprit ; l’esprit en effet est immatériel, sans rapports avec le monde et la matière, il n’existe pour lui ni nature, ni lois de la nature, mais uniquement le spirituel et les liens spirituels.\par
C’est pourquoi l’homme dut devenir aussi insoucieux et aussi détaché de tout que l’avait fait l’éducation sceptique, assez indifférent envers le monde pour que son écroulement même ne le pût émouvoir, avant de pouvoir se sentir indépendant du monde, c’est-à-dire se sentir esprit. Et c’est à l’œuvre de géants accomplie par les Anciens que l’homme doit de se savoir un être sans liaison avec le monde, un \emph{Esprit.}\par
Lorsque tout souci du monde l’a abandonné, et alors seulement, l’homme est pour lui-même tout dans tout ; il n’est plus que pour lui-même, il est esprit pour l’esprit ; ou, plus clairement : il ne se soucie plus que du spirituel.\par
Les Anciens tendirent vers l’\emph{Esprit} et s’efforcèrent de parvenir à la \emph{spiritualité}. Mais l’homme qui veut être actif comme esprit sera entraîné à des tâches tout autres que celles qu’il pouvait d’avance se tracer, à des tâches qui mettront en œuvre l’esprit et non plus seulement l’intelligence pratique, la \emph{perspicacité} capable uniquement de se rendre maître des \emph{choses.} L’esprit poursuit uniquement le spirituel et cherche en tout les « traces de l’esprit » : pour l’esprit \emph{croyant }« toute chose procède de Dieu » et ne l’intéresse que pour autant que cette origine divine s’y révèle ; tout paraît à l’esprit \emph{philosophique} marqué du sceau de la  raison, et ne l’intéresse que s’il peut y découvrir la raison, c’est-à-dire le contenu spirituel.\par
Cet esprit qui ne s’applique à rien de non spirituel, à aucune \emph{chose}, mais uniquement à l’être qui existe derrière et au-dessus des choses, aux \emph{pensées}, cet esprit, les Anciens ne le possédaient pas encore. Mais ils luttaient pour l’acquérir, ils le désiraient ardemment et, par là même, ils l’aiguisaient en silence pour le tourner contre leur tout-puissant ennemi, le Monde ; en attendant, c’est leur sens pratique, leur sagacité qu’ils opposaient à ce monde sensible, qui d’ailleurs n’était pas encore devenu sensible pour eux, car Jéhovah et les dieux du paganisme étaient encore bien loin de la notion « Dieu est esprit », et la patrie « céleste » n’avait pas encore remplacé la patrie sensible. Aujourd’hui encore, les Juifs, ces héritiers de la sagesse antique, ne se sont pas élevés plus haut, et sont incapables, en dépit de toute la subtilité et de toute la puissance de raisonnement qui les rendent si aisément maîtres des choses, de concevoir l’esprit \emph{pour lequel les choses ne sont rien.}\par
Le Chrétien a des intérêts spirituels, parce qu’il ose être homme \emph{par l’esprit ;} le Juif ne peut comprendre ces intérêts dans toute leur pureté parce qu’il ne peut prendre sur lui de n’accorder aucune valeur aux choses : la \emph{spiritualité} pure, cette spiritualité qui trouve par exemple son expression religieuse dans « la foi que ne justifie aucune œuvre » des Chrétiens lui est fermée. Leur \emph{réalisme} éloigne toujours les Juifs des Chrétiens, car le spirituel est aussi inintelligible pour le réaliste que le réel est méprisable aux yeux de l’esprit. Les Juifs n’ont que « l’esprit de ce monde ».\par
La perspicacité et la profondeur antiques sont aussi éloignées de l’esprit et de la spiritualité du monde chrétien, que la terre l’est du ciel.\par
Les choses de ce monde ne frappent ni n’angoissent celui qui se sent un libre esprit ; il n’en a cure, car il  faudrait, pour qu’il continuât à sentir leur poids, qu’il fût assez borné pour leur accorder encore quelque importance, ce qui témoignerait manifestement qu’il n’a pas encore complètement perdu de vue « la chère vie ».\par
Celui qui s’applique exclusivement à se savoir et à se sentir un pur esprit s’inquiète peu des éventualités fâcheuses qui peuvent l’attendre et ne songe nullement aux dispositions à prendre pour s’assurer une \emph{vie} libre et agréable.\par
Les désagréments que les hasards de la vie font naître des choses ne l’affectent point, car il ne vit que par l’esprit et d’aliments tout spirituels. Sans doute, comme le premier animal venu, mais sans presque s’en apercevoir, il boit, il mange, et quand la pâture vient à lui faire défaut, son corps succombe ; mais en tant qu’esprit il se sait immortel, et ses yeux se ferment au milieu d’une méditation ou d’une prière.\par
Toute sa vie tient dans ses rapports avec le spirituel : \emph{il pense}, et le reste n’est rien ; quelque direction que prenne son activité dans le domaine de l’esprit, prière, contemplation, ou spéculation philosophique, toujours ses efforts se réalisent sous la forme d’une pensée. Aussi Descartes, lorsqu’il fut parvenu à la parfaite conscience de cette vérité, pouvait-il s’écrier « je pense, c’est-à-dire je suis ». Cela signifie que c’est ma pensée qui est mon être et ma vie, que je n’ai d’autre vie que ma vie spirituelle, que je n’ai d’autre existence que mon existence en tant qu’esprit, ou, enfin, que je suis absolument esprit, et rien qu’esprit. L’infortuné Peter Schlemihl, qui avait perdu son ombre, est le portrait de cet homme devenu esprit, car le corps de l’esprit ne fait pas d’ombre. Il en était tout autrement chez les Anciens ! Si énergique, si virile que pût être leur attitude vis-à-vis de la puissance des choses, ils ne pouvaient faire autrement que de reconnaître cette puissance, et leur pouvoir se bornait à protéger autant que possible leur vie contre elle.  Ce n’est que bien tard qu’ils reconnurent que leur « véritable vie » n’était point celle qui prenait part à la lutte contre les choses du monde, mais la vie « spirituelle », « qui se détourne des choses », et le jour où ils s’en avisèrent, ils étaient — Chrétiens, ils étaient des « modernes » et des novateurs vis-à-vis du monde antique. La vie spirituelle, étrangère aux choses d’ici-bas, n’a plus de racines dans la nature, « elle ne vit que de pensées », et n’est par conséquent plus la « vie » mais — la \emph{pensée}.\par
Il ne faudrait pas croire, toutefois, que les Anciens vivaient sans penser ; ce serait aussi faux que de s’imaginer l’homme spirituel comme pensant sans vivre. Les Anciens avaient leurs pensées, leurs vues sur tout, sur le monde, sur l’homme, sur les dieux, etc., et montraient le plus grand empressement à acquérir des lumières nouvelles. Mais ce qu’ils ne connaissaient pas, c’était la Pensée, bien qu’ils pensassent d’ailleurs à toutes sortes de choses, et qu’ils pussent être « tourmentés par leurs pensées ». Rappelez-vous en songeant à eux, la phrase de l’Evangile : « mes pensées ne sont pas vos pensées ; autant le ciel est plus haut que la terre, autant mes pensées sont plus hautes que vos pensées », et rappelez-vous ce qui a été dit plus haut de nos pensées d’enfants.\par
Que cherche donc l’Antiquité ? La véritable \emph{joie, }la \emph{joie de vivre}, et c’est à la « véritable vie » qu’elle finit par aboutir.\par
Le poète grec Simonide chante : « Pour l’homme mortel, le plus noble et le premier des biens est la santé ; le suivant est la beauté ; le troisième, la richesse acquise sans fraude, le quatrième est de jouir de ces biens en compagnie de jeunes amis. » Ce sont là les \emph{biens de la vie}, les joies de la vie. Et que cherchait d’autre Diogène de Sinope, sinon cette véritable joie de vivre qu’il crut trouver dans le plus strict dénûment ! Que cherchait d’autre Aristippe, qui la trouva dans une inaltérable tranquillité d’âme ? Ce  qu’ils cherchaient tous, c’était le calme et imperturbable \emph{désir de vivre}, c’était la \emph{sérénité ;} ils cherchaient à être « de bonnes choses ».\par
Les Stoïciens veulent réaliser l’idéal de la \emph{sagesse dans la vie}, être des hommes qui \emph{savent vivre}. Cet idéal, ils le trouvent dans le dédain du monde, dans une vie immobile et stagnante, isolée et nue, sans expansion, sans rapports cordiaux avec le monde. Le stoïque vit, mais il est seul à vivre : pour lui tout le reste est mort. Les Epicuriens au contraire demandaient une vie active.\par
Les Anciens, en voulant être de bonnes choses, aspirent au \emph{bien vivre} (les Juifs notamment désirent vivre longuement, comblés d’enfants et de richesses), à l’Eudémonie, au bien-être sous toutes ses formes. Démocrite, par exemple, vante la paix du cœur de celui « qui coule ses jours dans le repos, loin des agitations et des soucis. »\par
L’Ancien songe donc à traverser la vie sans encombre, en se garant des chances mauvaises et, des hasards du monde. Comme il ne peut s’affranchir du monde, puisque toute son activité est tournée vers l’effort, il doit se borner à le \emph{repousser}, mais son mépris ne le détruit pas ; aussi ne peut-il atteindre tout au plus qu’un haut degré d’affranchissement, et il n’y aura jamais, entre lui et le moins affranchi, qu’une différence de degré. Qu’il parvienne même à tuer en lui le dernier reste de sensibilité aux choses terrestres que trahit encore le monotone chuchottement du mot « Brahm », rien cependant ne le séparera essentiellement de l’homme des sens, de l’homme de la chair.\par
Le stoïcisme, la vertu virile même, n’ont d’autre raison d’être que la nécessité de s’affirmer et de se soutenir envers et contre le monde ; l’éthique des stoïciens n’est point une doctrine de l’esprit, mais une doctrine du mépris du monde et de l’affirmation de soi vis-à-vis du monde. Et cette doctrine  s’exprime dans « l’impassibilité et le calme de la vie », c’est-à-dire dans la pure vertu romaine.\par
Les Romains ne dépassèrent pas cette \emph{sagesse dans la vie} (Horace, Cicéron, etc.).\par
La \emph{prospérité} épicurienne (Hédonè) n’est que le \emph{savoir vivre} stoïcien, mais affiné, plus artificieux ; les épicuriens enseignent simplement une autre conduite dans le monde ; ils conseillent de ruser avec lui au lieu de le heurter de front : il faut tromper le monde, car il est mon ennemi.\par
Le divorce définitif avec le monde fut consommé par les \emph{Sceptiques.} Toutes nos relations avec lui sont « sans valeur et sans vérité ». « Les sensations et les pensées que nous puisons dans le monde ne renferment, dit Timon, aucune vérité. » — « Qu’est-ce que la vérité ? » s’écrie Pilate. La doctrine de Pyrrhon nous enseigne que le monde n’est ni bon ni mauvais, ni beau ni laid, etc., que ce sont là-de simples \emph{prédicats }que nous lui attribuons. « Une chose n’est ni bonne ni mauvaise en soi, c’est l’homme qui la juge telle ou telle » (Timon). Il n’y a d’autre attitude possible devant le monde que l’Ataraxie (l’indifférence) et l’Aphasie (le silence, ou en d’autres termes l’\emph{isolement intérieur}).\par
Il n’y a dans le monde « aucune vérité à saisir » ; les choses se contredisent, nos jugements sur elles n’ont aucun critérium (une chose est bonne ou mauvaise suivant que l’un la trouve bonne ou que l’autre la trouve mauvaise) ; mettons de côté toute recherche de la « Vérité » ; que les hommes renoncent à trouver dans le monde aucun objet de connaissance, et qu’ils cessent de s’inquiéter d’un monde sans vérité.\par
Ainsi l’Antiquité vint à bout du \emph{monde des choses}, de l’ordre de la nature et de l’univers ; mais cet ordre embrasse non seulement les lois de la nature, mais encore toutes les relations dans lesquelles la nature place l’homme, la famille, la chose publique, et tout ce qu’on nomme les « liens naturels ».\par
 Avec le \emph{monde de l’Esprit} commence le Christianisme. L’homme qui se tient encore en armes vis-à-vis du monde est l’Ancien, le — \emph{Païen} (le Juif l’est resté parce que non chrétien) ; l’homme que ne guide plus que la « joie du cœur », la compassion, la sympathie, l’\emph{Esprit,} est le Moderne, le — Chrétien.\par
Les Anciens travaillèrent à \emph{soumettre le monde}, et s’efforcèrent de dégager l’homme des lourdes chaînes de sa dépendance vis-à-vis de ce qui n’était pas lui ; ils aboutirent ainsi à la dissolution de l’Etat et à la prépondérance du « privé ». Chose publique, Famille, etc., sont des liens \emph{naturels}, et comme tels d’importunes entraves qui rabaissent ma \emph{liberté spirituelle.}
\subsubsection[{A.II.2. Les Modernes.}]{A.II.2. Les Modernes.}
\noindent « Si quelqu’un est en Jésus-Christ, il est devenu \emph{une nouvelle créature ;} ce qui était devenu vieux est passé, voyez, tout est devenu \emph{nouveau} » \footnote{ \noindent 2\textsuperscript{e} aux Corinthiens, V, 17.
 }.\par
Nous avons dit plus haut que « pour les Anciens le Monde était une vérité » ; nous pouvons dire maintenant : « pour les Modernes, l’Esprit était une vérité », mais à condition d’ajouter comme précédemment : « une vérité, derrière la fausseté de laquelle ils cherchèrent et finalement parvinrent à pénétrer ».\par
Le Christianisme suit la route qu’avait prise l’Antiquité ; courbée durant tout le moyen-âge sous la discipline de dogmes chrétiens, l’intelligence se fait sophiste pendant le siècle qui précède la Réforme, et joue un jeu hérétique avec tous les fondements de la foi. Cela se produit surtout en Italie, et particulièrement à la cour de Rome ; mais quel mal y a-t-il à ce que l’esprit se divertisse, pourvu que le cœur reste chrétien ?\par
 Longtemps avant la Réforme, on était si bien accoutumé aux subtiles controverses, que le Pape, et presque tous avec lui, ne crurent d’abord assister, lorsque Luther entra en scène, qu’à une simple « querelle de moines ».\par
L’humanisme répond à la sophistique : c’est au temps des Sophistes que la vie grecque atteignit son plein épanouissement (siècle de Périclès) ; de même l’époque de l’humanisme, que l’on pourrait peut-être appeler aussi l’époque du machiavélisme, fut un apogée dans l’histoire de la civilisation (découverte de l’imprimerie, du Nouveau Monde, etc.).\par
Le cœur était, à ce moment, bien éloigné encore de toute velléité de se débarrasser de son contenu chrétien. Mais la Réforme prit enfin, comme l’avait fait Socrate, le cœur au sérieux, et les cœurs à dater de ce jour ont à vue d’œil — cessé d’être chrétiens. Du moment qu’on commençait avec Luther à remettre le cœur en question, ce premier pas dans la voie de la Réforme devait aboutir à ce que lui aussi s’allégeât du fardeau écrasant des sentiments chrétiens. De jour en jour moins chrétien, le cœur perdit ce qui l’avait rempli et occupé jusque-là, si bien qu’il ne lui resta plus enfin qu’une \emph{cordialité} vide, l’amour tout général de l’Homme, de l’Humanité, le sentiment de la liberté, la « Conscience ».\par
Le Christianisme atteint ainsi le terme de son évolution, parce qu’il s’est dénudé, atrophié et vidé. Le cœur n’a plus rien en lui qui ne le révolte, à moins de surprise ou d’inconscience. Il soumet à une critique mortelle tout ce qui prétend l’émouvoir ; il n’a ni ménagements ni pitié. Il n’est capable ni d’amitié ni d’amour. Et que pourrait-il aimer chez les hommes ? Tous sont des « égoïstes », nul n’est vraiment l’\emph{Homme,} le \emph{pur esprit ;} le Chrétien n’aime que l’esprit, mais où est-il le pur esprit ?\par
Aimer l’homme corporel, en chair et en os, ne serait plus un amour « spirituel », ce serait une trahison  envers l’amour « pur », « l’intérêt théorique ». Ne confondez pas en effet avec l’amour pur cette cordialité qui serre amicalement la main à chacun ; il en est précisément le contraire, il ne se livre en toute sincérité à personne, il n’est qu’une sympathie toute théorique, un intérêt qui s’attache à l’homme en tant qu’homme et non en tant que personne. La personne repousse cet amour, parce qu’elle est égoïste, qu’elle n’est pas l’Homme, l’idée à laquelle seule peut s’attacher l’intérêt théorique.\par
Les hommes comme vous et moi, ne fournissent à l’amour pur, à la pure théorie, qu’un sujet de critique, de raillerie et de radical mépris ; il ne sont pour lui, comme pour les prêtres fanatiques, que de l’ « ordure » et pis encore.\par
Arrivés à ce premier sommet de l’amour désintéressé, nous devons nous apercevoir que cet Esprit auquel s’adresse l’amour exclusif du Chrétien n’est rien, — ou est un leurre.\par
Ce qui, dans ce résumé, pourrait encore paraître obscur et n’être pas compris, s’éclaircira, nous l’espérons, par la suite. Acceptons l’héritage que nous ont légué les Anciens, et tâchons, ouvriers laborieux, d’en tirer — tout ce qu’on en peut tirer. La terre gît méprisée à nos pieds, loin en dessous de nous et de notre ciel ; ses bras puissants ne nous étreignent plus, nous avons publié son souffle enivrant ; si séduisante qu’elle soit, elle ne peut égarer que nos \emph{sens ; }notre esprit, et nous ne sommes en vérité rien qu’esprit, elle ne saurait le tromper. Une fois parvenu \emph{derrière} les choses, l’Esprit est aussi \emph{au-dessus} d’elles, il est délivré de leurs liens et plane, affranchi, librement dans l’au-delà. Ainsi parle la « liberté spirituelle ».\par
Pour l’esprit que de longs efforts ont dégagé et affranchi du monde, il ne reste plus, le monde et la matière déchus, que l’Esprit et le spirituel.\par
Toutefois, en devenant essentiellement différent et indépendant du monde, l’esprit n’a fait que s’en  éloigner, sans pouvoir en réalité l’anéantir ; aussi ce monde lui oppose-t-il, du fond du discrédit où il est tombé, des obstacles sans cesse renaissants, et l’Esprit est-il condamné à traîner perpétuellement le mélancolique désir de spiritualiser le monde, de le « racheter » ; de là, les plans de rédemption, les projets d’amélioration du monde qu’il bâtit comme un jeune homme.\par
Les Anciens, nous l’avons vu, étaient esclaves du naturel, du terrestre ; ils s’inclinaient devant l’ordre naturel des choses, mais en se demandant sans cesse s’il n’existait pas de moyen d’esquiver cette servitude ; lorsqu’ils se furent mortellement fatigués à des tentatives de révolte toujours renouvelées, de leur dernier soupir naquit le \emph{Dieu}, le « vainqueur du monde ».\par
Toute l’activité de leur pensée avait été dirigée vers la connaissance du monde, elle n’avait été qu’un long effort pour le pénétrer et le dépasser. Quel but s’est donné la pensée pendant les siècles qui suivirent ? Derrière quoi les Modernes ont-ils tenté de pénétrer ? Derrière le monde ? Non, car cette tâche, les Anciens l’avaient accomplie ; mais derrière le dieu que ces derniers leur léguaient, derrière le dieu « qui est esprit », derrière tout ce qui tient à l’esprit, derrière le spirituel.\par
L’activité de l’esprit qui explore « les profondeurs mêmes de la divinité » aboutit à la \emph{Théologie.}\par
Si les Anciens n’ont produit qu’une \emph{Cosmologie}, les Modernes n’ont jamais dépassé et ne dépassent pas la \emph{Théologie}.\par
Nous verrons par la suite que les plus récentes révoltes contre Dieu ne sont elles-mêmes que les dernières convulsions de cette « théologie » ; ce sont des insurrections théologiques.\par
 \paragraph[{§ 1. — L’esprit.}]{§ 1. — \emph{L’esprit.}}
\noindent Le monde des esprits est prodigieusement vaste, celui du spirituel est infini ; examinons donc ce qu’est au juste cet Esprit que nous ont légué les Anciens. Ils l’enfantèrent dans les douleurs, mais ils ne purent se reconnaître en lui ; ils avaient pu le mettre au monde, mais il devait parler lui-même. Le « Dieu-né, » le « fils de l’Homme » exprima le premier cette pensée que l’Esprit, c’est-à-dire Lui, le Dieu, n’a nulle attache avec les choses terrestres et leurs rapports, mais uniquement avec les choses spirituelles et leurs rapports.\par
Mon inébranlable fermeté dans l’adversité, mon inflexibilité et mon audace, sont elles déjà l’Esprit, dans la pleine acception du mot ? Le Monde en effet ne peut rien contre elles ? Mais s’il en était ainsi, l’Esprit serait encore en opposition avec le monde, et tout son pouvoir se bornerait à ne point s’y soumettre. Non, tant qu’il ne s’occupe point exclusivement de lui-même, tant qu’il n’a pas uniquement à faire à \emph{son }monde, au monde spirituel, l’Esprit n’est pas encore le \emph{libre} Esprit, il demeure l’ « esprit du monde » enchaîné à ce monde. L’Esprit n’est libre Esprit, c’est-à-dire réellement Esprit que dans le monde qui lui est propre ; ici-bas, dans « ce » monde, il reste un étranger. Ce n’est que dans un monde spirituel que l’Esprit se complète, et prend possession de soi, car « ce bas monde » ne le comprend pas et ne peut garder auprès de lui « la fille de l’étranger ».\par
Mais où trouvera-t-il ce domaine spirituel ? où, sinon en lui même ? Il doit se manifester, et les mots qu’il prononce, les révélations par lesquelles il se découvre, c’est là \emph{son} monde. Comme l’extravagant ne vit et ne possède son monde que dans les figures fantastiques que crée son imagination, comme le fou engendre  son propre monde de rêves, sans lequel il ne serait pas fou, ainsi l’esprit doit créer son monde de fantômes, et, tant qu’il ne l’a pas créé, il n’est pas Esprit. Ce sont ses créations qui le font Esprit, c’est à elles qu’on le reconnaît, lui leur créateur : il vit en elles, elles sont son monde.\par
Qu’est-ce donc que l’Esprit ? L’Esprit est le créateur d’un monde spirituel. On reconnaît sa présence en toi et en moi dès que l’on constate que nous nous sommes approprié quelque chose de spirituel, c’est-à-dire des pensées ; que ces pensées nous aient été suggérées, peu importe, pourvu que nous leur ayons donné la vie ; car, aussi longtemps que nous étions enfants, on eût pu nous proposer les maximes les plus édifiantes sans que nous eussions la volonté ou que nous fussions en état de les recréer en nous. Ainsi donc, l’Esprit n’existe que lorsqu’il crée du spirituel, et son existence résulte de son union avec le spirituel, sa création.\par
Comme c’est à ses œuvres que nous le reconnaissons, il faut nous demander ce que sont ces œuvres : les œuvres, les enfants de l’Esprit ne sont autres que — des Esprits, des fantômes.\par
Si j’avais devant moi des Juifs, des Juifs de vieille roche, je pourrais m’arrêter ici et les laisser méditer sur le mystère de leur incrédulité et de leur incompréhension de vingt siècles. Mais comme toi, mon cher lecteur, tu n’es pas un Juif, du moins pas un Juif pur sang — nul d’entre eux ne se serait égaré jusqu’ici — nous ferons encore ensemble un bout de chemin, jusqu’à ce que toi aussi peu-être tu me tournes le dos, croyant que je me moque de toi.\par
Si quelqu’un te disait que tu es tout Esprit, tu te tâterais et ne le croirais pas, mais tu répondrais : « Je ne manque en vérité pas d’esprit, j’en \emph{ai,} mais je n’existe pas uniquement comme esprit, je suis un homme en chair et en os ». Tu ferais encore toujours une distinction entre toi et « ton esprit ». — « Mais,  réplique ton interlocuteur, c’est ta destinée, quoique tu sois encore à présent le prisonnier d’un corps, de devenir quelque jour un esprit bienheureux ; et si tu peux te figurer l’aspect futur de cet esprit, il est également certain que dans la mort tu abandonneras ce corps, et que ce que tu garderas pour l’éternité ce sera toi, c’est-à-dire ton Esprit. Par conséquent, ce qu’il y a de véritable et d’éternel en toi, c’est ton esprit ; le corps n’est que ta demeure en ce monde, demeure que tu peux abandonner et peut-être échanger pour une autre ».\par
Te voilà convaincu ! Pour le moment, en vérité, tu n’es pas un pur Esprit, mais lorsque tu auras émigré de ce corps périssable, tu pourras te tirer d’affaire sans lui ; aussi est-il nécessaire de prendre tes précautions et de soigner à temps ton « moi » par excellence : « Que servirait-il à l’homme de conquérir l’univers, s’il devait pour cela faire tort à son âme ? »\par
De graves doutes se sont élevés au cours des temps contre les dogmes chrétiens, et t’ont dépouillé de ta foi en l’immortalité de ton esprit. Mais un dogme est resté debout : tu es toujours fermement convaincu que l’Esprit est ce qu’il y a de meilleur en toi et que le spirituel doit primer en toi tout le reste. Quel que soit ton athéisme, tu communies avec les croyants en l’immortalité dans leur zèle contre l’\emph{Egoïsme.}\par
Qu’entends-tu donc par un égoïste ? Celui qui, au lieu de vivre pour une idée, c’est-à-dire pour quelque chose de spirituel, et de sacrifier à cette idée son intérêt personnel, sert au contraire ce dernier. Un bon patriote, par exemple, porte son offrande sur l’autel de la patrie, et que la patrie soit une pure idée, cela ne fait pas de doute, car il n’y a ni patrie, ni patriotisme pour les animaux, auxquels l’esprit est interdit, ou pour les enfants encore sans esprit. Celui qui ne se montre pas bon patriote décèle son égoïsme  vis-à-vis de la patrie. Il en est de même dans une infinité d’autres cas : jouir d’un privilège aux dépens du reste de la société, c’est pécher par égoïsme contre l’idée d’égalité ; détenir le pouvoir, c’est violer en égoïste l’idée de liberté, etc., etc.\par
Telle est la cause de ton aversion pour l’égoïste : il subordonne le spirituel au personnel, et c’est à lui qu’il songe, alors que tu préférerais le voir agir pour l’amour d’une idée. Ce qui vous distingue, c’est que tu rapportes à ton esprit tout ce qu’il rapporte à lui-même ; en d’autres termes, tu scindes ton Moi, et ériges ton « Moi proprement dit », l’Esprit, en maître souverain du reste que tu juges sans valeur, tandis que lui ne veut rien entendre d’un tel partage et poursuit à \emph{son} gré \emph{ses} intérêts tant spirituels que matériels. Tu crois ne t’insurger que contre ceux qui ne conçoivent aucun intérêt spirituel, mais en fait tu embrasses dans ta malédiction tous ceux qui ne tiennent pas ces intérêts spirituels pour « les vrais, les suprêmes intérêts ». Tu pousses si loin ton office de chevalier servant de cette belle, que tu la proclames l’unique beauté qui soit au monde. Ce n’est pas pour \emph{toi} que tu vis, mais pour ton \emph{Esprit} et pour ce qui tient à l’Esprit, c’est-à-dire pour des Idées.\par
Puisque l’Esprit n’existe qu’en tant qu’il crée le spirituel, tâchons donc de découvrir sa première création. De celle-ci découle naturellement une génération indéfinie de créations, comme, à en croire le mythe, il suffit que les premiers humains fussent créés pour que la race se multipliât spontanément. Quant à cette première création, elle doit être tirée « du néant », c’est-à-dire que l’Esprit, pour se réaliser, ne dispose que de lui-même ; bien plus : il ne dispose pas même encore de lui, mais il doit se créer ; l’Esprit est par conséquent lui-même sa première création. Quelque mystique que le fait paraisse, sa réalité n’en est pas moins attestée par une expérience de tous les jours. Es-tu un penseur avant d’avoir pensé ?  Ce n’est que par le fait que tu crées ta première pensée que tu crées en toi le penseur, car tu ne penses point tant que tu n’as point eu une pensée. N’est-ce pas ton premier chant qui fait de toi un chanteur, ta première parole qui te fait homme parlant ? De même c’est ta première production spirituelle qui fait de toi un Esprit. Si tu \emph{te} distingues du penseur et du chanteur que tu es, tu devrais te distinguer également de l’Esprit et sentir clairement que tu es encore autre chose qu’Esprit. Mais de même que le Moi pensant perd aisément la vue et l’ouïe dans son enthousiasme de penser, de même « l’enthousiasme de l’Esprit » t’a saisi, et tu aspires maintenant de toutes tes forces à devenir tout Esprit et à te fondre dans l’Esprit. L’Esprit est ton \emph{Idéal,} l’inaccessible, l’au-delà ; tu appelles l’Esprit — Dieu : « Dieu, c’est l’Esprit » !\par
Ton zèle t’excite contre tout ce qui n’est pas Esprit, aussi t’insurges-tu contre \emph{toi-même}, qui n’es pas exempt d’un reste de non spiritualité. Au lieu de dire : « je suis \emph{plus} qu’Esprit », tu dis avec contrition : « Je suis \emph{moins} qu’Esprit. L’Esprit, le pur Esprit, je ne puis que le concevoir, mais je ne le suis point ; et puisque je ne le suis pas, c’est qu’un autre l’est, et cet autre je l’appelle — Dieu ».\par
L’Esprit, pour exister comme pur Esprit, doit nécessairement être un au-delà, car, puisque je ne le suis pas, il ne peut être qu’\emph{en dehors de moi}, et puisque nul homme ne réalise intégralement la notion d’ « Esprit », l’Esprit pur, l’Esprit en soi ne peut être qu’en dehors des hommes, au delà du monde humain, non terrestre, mais céleste.\par
Cette discordance entre moi et l’Esprit, qui éclate en ce fait que « Moi » et « Esprit » ne sont pas deux noms applicables à une seule et même chose, mais deux noms différents pour deux choses différentes, que je ne suis pas Esprit et que l’Esprit n’est pas moi, cela seul suffît pour nous montrer sur quelle tautologie repose l’apparente nécessité pour l’Esprit  d’habiter l’au-delà, c’est-à-dire d’être Dieu.\par
Cela seul aussi suffit pour nous faire apprécier la base totalement théologique sur laquelle Feuerbach\footnote{ \noindent \emph{Wesen des Christentums} (Essence du Christianisme).\par
 \emph{Une traduction de cet ouvrage a été publiée par J. Roy (Paris, Lacroix}, 1864). \emph{N. du Tr}.
 } édifie la solution qu’il s’efforce de nous faire accepter. Autrefois, dit-il, nous ne cherchions et n’apercevions notre essence que dans l’au-delà, tandis qu’à présent que nous comprenons que Dieu n’est que notre essence humaine, nous devons reconnaître cette dernière, comme nôtre, et la transposer de nouveau de l’autre monde en ce monde. Ce Dieu, qui est Esprit, Feuerbach l’appelle « notre essence ». Pouvons-nous accepter cette opposition entre « notre essence » et \emph{nous}, et admettre notre division en un moi essentiel et un moi non essentiel ? Ne sommes-nous pas ainsi de nouveau condamnés à nous voir misérablement bannis de nous-mêmes ?\par
Que gagnons-nous donc à métamorphoser le divin extérieur à nous en un divin intérieur ? \emph{Sommes-nous} ce qui est en nous ? Pas plus que ce qui est hors de nous. Je ne suis pas plus mon cœur que je ne suis ma maîtresse, cet « autre moi ». C’est précisément parce que nous ne sommes pas l’Esprit qui habite en nous, que nous étions obligés de projeter cet Esprit hors de nous : il n’était pas nous, ne faisant qu’un avec nous, aussi ne pouvions-nous lui accorder d’autre existence que hors de nous, au delà de nous, dans l’au-delà.\par
Feuerbach étreint avec l’énergie du désespoir tout le contenu du Christianisme, non pour le jeter bas, mais pour s’en emparer, pour arracher de son ciel par un dernier effort cet idéal toujours désiré, jamais atteint, et le garder éternellement. N’est-ce point là un suprême effort, une entreprise désespérée sur la vie et la mort, et n’est-ce point en même temps  la dernière convulsion de l’esprit chrétien altéré d’au-delà ? Le Héros ne tente pas d’escalader le Ciel, mais de l’attirer à lui, de le forcer à devenir terrestre ! Et que crie le monde entier depuis ce jour-là ? Qu’appellent ses vœux plus ou moins conscients ? Qu’il vienne, cet « au-delà », que le ciel descende sur la terre, qu’il s’ouvre dès maintenant à nous !\par
A la doctrine théologique de Feuerbach, opposons en quelques mots les objections qu’elle nous suggère : « l’être de l’homme est pour l’homme l’\emph{être suprême}. Cet être suprême, la religion l’appelle Dieu et en fait un être \emph{objectif ;} mais il n’est en réalité que le propre être de l’homme ; et nous sommes à un tournant de l’histoire du monde, parce que désormais pour l’homme ce n’est plus Dieu, mais l’Homme qui incarne la divinité\footnote{ \noindent Cf. par exemple : F{\scshape euerbach}, \emph{Wesen des Christentums}, p. 402.
 } ».\par
A cela nous répondons : l’Être suprême est l’être ou l’essence de l’homme, je vous l’accorde ; mais c’est précisément parce que cette essence suprême est « son essence » et non « lui », qu’il est totalement indifférent que nous la voyions hors de lui et en fassions « Dieu », ou que nous la voyions en lui et en fassions l’ « Essence de l’homme » ou l’ « Homme ». Je ne suis ni Dieu, ni l’Homme, je ne suis ni l’essence suprême, ni mon essence, et c’est au fond tout un que je conçoive l’essence en moi ou hors de moi. Bien plus, toujours l’essence suprême a été conçue dans ce double au-delà, au-delà intérieur et au-delà extérieur ; car, d’après la doctrine chrétienne, l’ « esprit de Dieu » est aussi « notre esprit » et « habite en nous\footnote{ \noindent Cf. Romains {\scshape viii}, 9 ; 1\textsuperscript{re} aux Corinth., {\scshape iii}, 16 ; Jean {\scshape xx}, 22, etc., etc.
 } ». Il habite le ciel et habite en nous, nous ne sommes que sa « demeure ». Si Feuerbach détruit sa demeure céleste et le force à venir s’installer chez nous avec  armes et bagages, nous serons, nous, son terrestre logis, singulièrement encombrés.\par
Cette digression, nous nous avisons après coup qu’il eût mieux valu la réserver pour plus tard, pour éviter une répétition. Revenons à la première création de l’Esprit, c’est-à-dire à l’Esprit lui-même. L’Esprit est quelque chose d’autre que moi ; ce quelque chose d’autre, quel est-il ?
\paragraph[{§ 2. — Les Possédés.}]{§ 2. — \emph{Les Possédés}.}
\noindent As-tu déjà vu un Esprit ? — Moi ? non, mais ma grand’mère en a vu. — C’est comme moi : je n’en ai jamais vu, mais ma grand’mère en avait qui lui couraient sans cesse dans les jambes ; et, par respect pour le témoignage de nos grand’mères, nous croyons à l’existence des esprits.\par
Mais n’avions-nous pas aussi des grand-pères, et ne haussaient-ils pas les épaules chaque fois que nos grand’mères entamaient leurs histoires de revenants ? Hélas oui, c’étaient des incrédules et ils ont fait grand tort à notre bonne religion, tons ces philosophes ! Nous le verrons bien par la suite !\par
Qu’y a-t-il au fond de cette foi profonde dans les revenants, sinon la foi dans l’existence « d’êtres spirituels » en général ? Et la seconde ne serait-elle pas déplorablement ébranlée, s’il était établi que tout homme qui pense doit hausser les épaules devant la première ?\par
Les Romantiques, sentant combien l’abandon de la croyance aux esprits ou revenants compromettrait la croyance en Dieu même, s’efforcèrent de conjurer cette conséquence fâcheuse ; dans ce but, non seulement ils ressuscitèrent le monde merveilleux des légendes, mais ils finirent par exploiter le « monde supérieur » avec leurs somnambules, leurs voyantes, etc.\par
Les bons croyants et les Pères de l’Eglise ne soupçonnaient guère que la croyance aux revenants  s’effondrant, c’était le sol même qui se dérobait sous la Religion, désormais flottante et sans appui. Celui qui ne croit plus à aucun revenant n’a qu’à être conséquent avec lui-même pour que son incrédulité le conduise à s’apercevoir qu’il ne se cache derrière les choses aucun être à part, aucun revenant, ou (pour employer un mot dont on a naïvement fait un synonyme de ce dernier) — aucun « \emph{Esprit} ».\par
« Mais il existe des Esprits ! » Contemple le monde qui t’entoure, et dis moi si derrière toute chose ne t’apparaît pas un Esprit. La fleur, l’humble fleur te dit l’Esprit du créateur qui en fit une petite merveille ; les étoiles proclament l’Esprit qui ordonne leur cours ; un Esprit de sublimité plane au sommet des monts ; l’Esprit de la mélancolie et du désir murmure sous les eaux ; — et dans les hommes parlent des millions d’Esprits. Que les montagnes s’affaissent, que le monde des étoiles tombe en poussière, que les fleurs se flétrissent et que meurent les hommes, que survit-il à la ruine de ces corps visibles ? L’Esprit, invisible, éternel ! Oui, tout dans ce monde est hanté ! Que dis-je ? Ce monde lui-même est hanté ; masque décevant, il est la forme errante d’un Esprit, il est un fantôme.\par
Qu’est-ce qu’un fantôme, sinon un corps apparent, mais un [{\corr Esprit}] réel ? Tel est le monde, « vain », « nul », illusoire apparence sans autre réalité que l’Esprit, dont il est l’enveloppe visible. Regarde : ici, là, de toutes parts t’entoure un monde de fantômes ; tu es assiégé sans cesse de visions, d’ « apparitions ». Tout ce qui se montre à toi n’est que le reflet de l’Esprit qui l’habite, une apparition spectrale ; le monde entier n’est qu’une fantasmagorie derrière laquelle s’agite l’Esprit. Tu « vois des Esprits ».\par
Vas-tu peut-être te comparer aux Anciens, qui voyaient partout des dieux ? Les dieux, mon cher Moderne, ne sont pas des Esprits ; les dieux ne réduisent  pas le monde à n’être qu’une apparence et ne le spiritualisent pas.\par
A tes yeux, le monde entier est spiritualisé ; il est devenu un énigmatique fantôme ; aussi ne songes-tu même plus à t’étonner de ne trouver en toi qu’un fantôme. Ton Esprit ne hante-t-il pas ton corps, et n’est-il pas, lui, le vrai, le réel, tandis que ton corps n’est qu’une « apparence », quelque chose de « périssable » et « sans valeur » ? Ne sommes-nous point tous des spectres, de pauvres êtres tourmentés qui attendent la « délivrance » ; ne sommes-nous pas des « Esprits » ?\par
Depuis que l’Esprit a paru dans le monde, depuis que « le Verbe s’est fait chair », ce monde spiritualisé et livré aux enchantements n’est plus qu’une maison hantée.\par
Tu as un esprit, car tu as des pensées. Mais que sont ces pensées ? — Des êtres spirituels. — Elles ne sont donc point des choses ? — Non, mais l’Esprit des choses, ce qu’il y a en elles de plus intime, de plus essentiel, leur idée. — Ce que tu penses n’est donc pas simplement ta pensée ? — Au contraire, c’est ce qu’il y a de plus réel, de proprement vrai dans le monde : c’est la vérité même ; quand je pense juste, je pense la vérité. Je puis me tromper au sujet de la vérité, je puis la \emph{méconnaître ;} mais lorsque ma \emph{connaissance} est véridique, c’est la vérité qui en est l’objet. — Aspires-tu donc à connaître la vérité ? — La Vérité m’est sacrée. Il peut arriver que je trouve \emph{une} vérité imparfaite et que je doive la remplacer par une meilleure, mais je ne puis supprimer \emph{la }Vérité. Je \emph{crois} à la Vérité et c’est pourquoi je la recherche ; rien ne la dépasse, elle est éternelle.\par
La vérité est sacrée et éternelle ! Mais toi, qui t’emplis de cette sainteté et en fais ton guide, tu seras toi-même sanctifié. Le Sacré ne se manifeste jamais à tes sens, ce n’est jamais comme être matériel que tu en découvres la trace ; il ne se révèle qu’à ta foi, ou  plus exactement à ton \emph{Esprit,} car il est lui-même quelque chose de spirituel, un Esprit ; il est Esprit pour l’Esprit.\par
La notion de sainteté ne se laisse pas extirper aussi facilement que beaucoup semblent le croire, qui se refusent à employer encore ce mot « impropre ». A quelque point de vue qu’on se mette pour m’accuser d’égoïsme, ce reproche sous entend toujours que l’on a en vue quelque chose d’autre que moi, que je devrais servir de préférence à moi-même, que je devrais estimer plus important que tout le reste, bref, quelque chose en quoi je devrais chercher mon véritable salut, c’est-à-dire quelque chose de « saint », de « sacré ». Que ce sacro-saint soit d’ailleurs si humain que l’on voudra, qu’il soit l’Humain même, cela ne lui enlève rien de son caractère, et fait tout au plus de ce sacré supraterrestre un sacré terrestre, de ce sacré divin un sacré humain.\par
Rien n’est sacré que pour l’Égoïste qui ne se rend pas compte de son égoïsme, pour l’\emph{Égoïste involontaire}. J’appelle ainsi celui qui, incapable de dépasser jamais les bornes de son moi, ne le tient cependant pas pour l’être suprême ; qui ne sert que lui en croyant servir un être supérieur, et qui, ne connaissant rien de supérieur à lui-même, rêve pourtant quelque chose de supérieur ; bref, c’est l’Égoïste qui voudrait n’être pas Égoïste, qui s’humilie, et qui combat son égoïsme, mais qui ne s’humilie que « pour être élevé » c’est-à-dire pour satisfaire son égoïsme. Comme il voudrait cesser d’être égoïste, il interroge ciel et terre, en quête de quelque être supérieur auquel il puisse offrir ses services et ses sacrifices ; mais il a beau s’agiter et se mortifier, il ne le fait en définitive que par amour de lui-même, et l’Égoïsme, l’odieux Égoïsme ne le lâche pas. Voilà pourquoi je l’appelle un égoïste involontaire. Tous ses soins, toutes ses peines pour s’affranchir \emph{de son moi} ne sont qu’un effort mal compris pour affranchir \emph{son moi.}\par
 Es-tu lié à ton heure passée ? Dois-tu faire aujourd’hui ce que tu fis hier\footnote{ 
\begin{verse}
Wie sie klingeln, die Pfaffen, wie angelegen Sie’s machen,\\
 Dass man komme, nur ja plappre, wie gestern, so heut.\\
 Scheltet mir nicht die Pfaffen ! Sie kennen des Menschen Bedürfniss ;\\
 Denn wie ist er beglückt, plappert er morgen wie heut.\\
\end{verse}
 

\signed{G{\scshape œthe}, \emph{(Épigrammes vénitiennes, 11).}}
 } ? Ne peux-tu te transformer à chaque instant ? S’il en était ainsi, tu te sentirais enchaîné et paralysé. Mais à chaque minute de ton existence, une nouvelle minute de l’avenir te fait signe et t’appelle ; en te développant, tu te dégages « de toi », de ton moi actuel. Ce que tu es à chaque instant est ton œuvre, et tu dois à cette œuvre de ne pas te perdre, toi, son auteur. Tu es toi-même un être supérieur à ce que tu es, tu te dépasses toi-même. Mais ce fait que tu es supérieur à toi, que tu n’es pas seulement une créature, mais en même temps ton créateur, t’échappe en ta qualité d’Égoïste involontaire, et c’est pourquoi l’ « être supérieur » reste pour toi un étranger. Tout être supérieur, Vérité, Humanité, etc. est un être \emph{au-dessus} de nous.\par
Il nous est étranger ; c’est là un signe auquel nous reconnaissons ce qui est « sacré ». Il y a dans tout ce qui est sacré quelque chose d’inconnu, de différent, qui nous met mal à l’aise et nous empêche de nous sentir chez nous. Ce qui m’est sacré \emph{ne m’appartient pas ;} si la propriété d’autrui, par exemple, ne m’était pas sacrée, je la regarderais comme \emph{mienne} et ne laisserais pas échapper une occasion de m’en saisir ; si au contraire la figure de l’empereur de Chine m’est sacrée, elle reste étrangère à mes yeux et je les baisse devant lui.\par
Pourquoi une vérité mathématique indiscutable, qu’on pourrait dans le sens usuel du mot appeler éternelle, ne m’est-elle pas — sacrée ? Parce qu’elle n’est pas révélée ; elle n’est point la révélation d’un  être supérieur. Entendre uniquement par révélées les « vérités religieuses » serait absolument erroné, ce serait méconnaître complètement la valeur du concept « être supérieur ». Les athées tournent en dérision cet être supérieur auquel on a voué un culte sous le nom d’ « être suprême\footnote{ \noindent \emph{En français dans le texte.} (N. D. T.)
 } », et réduisent en poussière l’une après l’autre toutes les « preuves de son existence », sans remarquer qu’eux-mêmes obéissent ainsi à leur besoin d’un être supérieur, et qu’ils ne détruisent l’ancien que pour faire place à un nouveau. A côté d’un individu humain, « l’Homme » n’est-il pas un être supérieur ? Et les Vérités, les Droits, les Idées qui découlent de son concept ne doivent-ils pas, comme révélations de ce concept être respectés et tenus pour — sacrés ? Supposez même qu’on vienne à démontrer la fausseté de telle vérité qui passait pour une de ses manifestations : cela témoignera uniquement d’une fausse interprétation de votre part, sans causer le moindre préjudice à la notion sacrée elle-même et sans rien enlever de leur sainteté aux autres vérités, à celles qui doivent « à juste titre » être regardées comme ses révélations. L’Homme dépasse tout homme pris individuellement, et s’il est l’ « essence » de l’individu, il n’est en réalité pas \emph{son} essence (car l’être ou l’essence de l’individu devrait être aussi unique que l’individu même), mais une essence, un être « supérieur », et, pour les athées eux-mêmes, l’essence ou l’Etre « suprêmes ».\par
De même que les révélations divines ne furent pas écrites de la main de Dieu, mais publiées par les « instruments du Seigneur », de même l’Homme ne publie pas lui-même ses révélations, mais nous les fait connaître par l’intermédiaire de « véritables, hommes ». Seulement ce nouvel être suprême trahit une conception bien plus spiritualisée que celle de l’ancien Dieu ; ce dernier pouvait encore être représenté  sous une forme corporelle, on pouvait lui imaginer une certaine figure, tandis que l’Homme, au contraire, reste purement spirituel, et on ne peut lui prêter aucun corps matériel particulier. Il est vrai qu’il ne manque pas d’une certaine corporalité, d’autant plus séduisante qu’elle paraît plus naturelle et plus terrestre, et c’est tout bonnement chaque homme corporel, ou, plus simplement encore, la « race humaine » ou « tous les hommes ». L’Esprit réintègre ainsi sa forme de fantôme et redevient compact et populaire à souhait.\par
Saint donc est l’être suprême, saint est tout ce en quoi il se révèle ou se révélera, et sanctifiés sont ceux qui reconnaissent cet être suprême dans leur propre être, c’est-à-dire dans les manifestations de cet être. Ce qui est saint sanctifie en retour son adorateur ; le culte qu’il lui rend le sanctifie et sanctifie ce qu’il fait : un saint commerce, de saintes pensées et de saintes actions, etc.\par
L’objet qui doit être honoré comme l’être suprême ne peut, on le conçoit, être discuté avec fruit que pour autant que les contradicteurs les plus acharnés soient d’accord sur le point essentiel et qu’ils admettent l’existence d’un être suprême auquel s’adressent notre culte et nos sacrifices. Si quelqu’un souriait dédaigneusement devant toute controverse au sujet de l’être suprême comme un Chrétien peut sourire devant les discussions d’un Chiite et d’un Sunnite ou d’un Brahmine et d’un Bouddhiste, ce serait qu’à ses yeux l’hypothèse d’un être suprême est nulle et que toute contestation à ce propos est un jeu puéril. Que votre être suprême soit le Dieu unique en trois personnes, le Dieu de Luther, l’ « Etre suprême\footnote{ \noindent \emph{En français dans le texte.} (N. D. T.)
 } » du Déiste, ou qu’il ne soit nullement Dieu mais « l’Homme », c’est tout un pour qui nie l’être suprême lui-même : Vous tous qui servez un  Etre suprême quel qu’il soit, vous n’êtes que des — gens pieux, l’athée le plus frénétique comme le plus fervent chrétien.\par
Dans la sainteté vient au premier rang l’Etre suprême, et avec lui notre « sainte croyance ».\par

\labelblock{Le Fantôme.}

\noindent Avec les revenants nous entrons dans le royaume des esprits, dans le royaume des Etres, des Essences.\par
L’être énigmatique et « incompréhensible » qui hante et trouble l’univers est le fantôme mystérieux que nous nommons être suprême. Pénétrer ce fantôme, le saisir, découvrir la \emph{réalité} qui est en lui (prouver l’ « existence de Dieu ») est la tâche à laquelle les hommes se sont attelés pendant des siècles ; ils se sont ingéniés à venir à bout de cette terrible impossibilité, de cet interminable travail de Danaïdes, de changer le fantôme en un non-fantôme, l’irréel en réel, l’esprit en une personne entière et corporelle. Derrière le monde existant ils cherchèrent la « chose en soi », l’être, l’essence ; derrière la chose ils cherchèrent la \emph{non-chose}.\par
Qu’on examine à fond le moindre phénomène, qu’on en recherche l’essence, et l’on y découvrira souvent tout autre chose que ce qui \emph{paraissait} à première vue : une parole mielleuse et un cœur faux, un discours pompeux et des pensées mesquines, etc. Et par là-même qu’on fait ressortir l’essence, on réduit l’aspect jusqu’alors mal compris à une mensongère apparence.\par
L’essence de ce monde superbe est, pour celui qui en scrute les profondeurs, la — vanité. Celui qui est religieux ne s’occupe point de l’apparence trompeuse, des vains phénomènes, mais recherche l’essence, et quand il tient cette essence il tient — la Vérité.\par
Les essences qui se manifestent sous certaines apparences sont les mauvaises essences, celles qui se  manifestent sous d’autres sont les bonnes. L’essence du sentiment humain, par exemple, est l’amour, l’essence de la volonté humaine est le bien, celle de la pensée est le vrai, etc.\par
Ce qui passe d’abord pour existant, comme le monde et ce qui s’y rapporte, apparaît maintenant comme une pure illusion, et \emph{ce qui existe vraiment}, c’est l’essence, dont le royaume s’emplit de dieux, d’Esprits, de démons, c’est-à-dire de bonnes et de mauvaises essences. Ce monde retourné, le monde des essences, est désormais seul vraiment existant. Le cœur humain peut être sans amour, mais son essence existe : le dieu, « qui est l’amour » ; la pensée humaine peut s’égarer dans l’erreur, mais son essence, la vérité, n’en existe pas moins : « Dieu est la vérité ». Ne connaître et ne reconnaître que les essences, tel est le propre de la religion ; son royaume est un royaume des essences, des fantômes, des revenants.\par
L’effort pour rendre saisissable le fantôme, ou pour réaliser le « non-sens\footnote{ \noindent « \emph{Non sens} » \emph{en français dans le texte.} (N. D. Tr.)
 } » a abouti à produire un \emph{fantôme corporel}, un fantôme ou un esprit pourvu d’un corps réel, un fantôme fait chair. Comment les plus puissants génies du christianisme se sont mis l’esprit à la torture pour saisir cette apparence fantomatique, chacun le sait ; mais en dépit de leurs efforts la contradiction des deux natures est restée irréductible : d’une part la divine, d’autre part l’humaine, d’une part le fantôme, de l’autre le corps sensible. Le plus extraordinaire des fantômes est resté une « non-chose ». Celui qui se martyrisait l’âme n’était point encore un Esprit, et nul Chaman qui se torture jusqu’au délire furieux et à la frénésie pour exorciser un esprit n’a éprouvé les angoisses que ce spectre insaisissable procura aux Chrétiens.\par
C’est le Christ qui mit en lumière cette vérité que  le véritable Esprit, le fantôme par excellence, est — l’Homme. L’esprit \emph{fait chair}, c’est l’Homme ; il est lui-même l’effroyable essence, il en est à la fois l’apparence et l’existence. Depuis lors l’homme ne s’épouvante plus devant les revenants qui sont hors de lui, mais devant lui-même ; il est pour lui-même un objet d’effroi. Au fond de sa poitrine habite l’\emph{Esprit de péché}, la pensée la plus douce (et cette pensée est elle-même un Esprit) est peut être un diable, etc.\par
Le fantôme a pris un corps, le Dieu s’est fait homme, mais l’homme est maintenant lui-même le terrifiant fantôme derrière lequel il s’efforce de pénétrer, qu’il cherche à exorciser, à comprendre et à exprimer ; l’homme est — \emph{Esprit.} Que le corps se dessèche pourvu que l’esprit soit sauvé ; l’esprit est désormais l’unique souci, et le salut de l’esprit ou de l’ « âme » est le but unique. L’homme est lui-même devenu un revenant, un fantôme obscur et décevant, auquel une place déterminée est assignée dans le corps (Discussions sur le siège de l’âme : est-elle dans la tête ? etc.)\par
Tu n’es pas pour moi un être supérieur, et je n’en suis point un pour toi. Il se peut toutefois que chacun de nous recèle un être supérieur, qui exige de nous un respect mutuel. Ainsi, pour prendre comme exemple ce qu’il y a en nous de plus général, en toi et en moi vit l’Homme. Si je ne voyais pas l’Homme en toi, qu’aurais-je à y respecter ? En vérité, tu n’es pas l’Homme, tu n’es pas sa vraie et adéquate figure, tu n’es que l’enveloppe périssable que l’Homme revêt pour quelques heures et dont il peut sortir sans cesser d’être lui-même. Cependant cet être général et supérieur demeure pour le moment en toi ; aussi m’apparais-tu, toi dont un esprit immortel a revêtu la forme passagère, toi en qui un esprit se manifeste sans être lié à ton corps et à ce mode d’apparition, comme un fantôme.\par
Aussi ne te considérai-je point comme un être supérieur ; ce que je respecte en toi, ce n’est que l’être  supérieur que tu héberges, c’est-à-dire l’ « Homme ». Les Anciens n’avaient à ce point de vue aucun respect pour leurs esclaves, parce qu’ils n’en avaient guère pour l’être supérieur que nous honorons aujourd’hui sous le nom d’ « Homme ». Ils apercevaient chez autrui d’autres fantômes, d’autres Esprits. Le peuple est un être supérieur à l’individu, c’est un Esprit qui hante l’individu, c’est l’Esprit du peuple. C’est cet Esprit qu’honoraient les Anciens et l’individu n’avait pour eux d’importance qu’au service de cet Esprit ou d’un Esprit voisin, l’Esprit de famille. C’est par amour de cet être supérieur, le Peuple, qu’on accordait quelque valeur à chaque citoyen. De même que tu es à nos yeux sanctifié par l’ « Homme » qui te hante, de même on était en ce temps-là sanctifié par tel ou tel autre être supérieur : le peuple, la famille, etc.\par
Si je te prodigue mes attentions et mes soins, c’est que tu m’es cher, c’est que je trouve en toi l’aliment de mon cœur, l’apaisement de ma détresse ; si je t’aime, ce n’est point par amour d’un être supérieur dont tu serais l’enveloppe consacrée, ce n’est pas que je voie en toi un fantôme et que j’y devine un esprit ; c’est par égoïsme que je t’aime : c’est toi-même, avec \emph{ton} essence, qui m’es cher, car ton essence n’est rien de supérieur, elle n’est ni plus haute ni plus générale que toi, elle est unique comme toi-même, c’est toi-même.\par
Mais l’homme n’est pas seul un fantôme ; tout est hanté. L’être supérieur, l’Esprit qui s’agite en toutes choses n’est lié à rien et ne fait que « paraître » dans les choses. Fantômes dans tous les coins !\par
Ce serait ici le lieu de faire défiler ces fantômes ; mais nous aurons l’occasion dans la suite de les évoluer de nouveau pour les voir s’envoler devant l’égoïsme. Nous pouvons donc nous borner à en citer quelques-uns en guise d’exemples : ainsi le S\textsuperscript{t} Esprit, ainsi la Vérité, le Roi, la Loi, le Bien, la  Majesté, l’Honneur, le Bien public, l’Ordre, la Patrie, etc., etc.\par

\labelblock{La Marotte.}

\noindent Homme, ta cervelle est hantée, tu bats la campagne ! Dans tes rêves démesurés, tu te forges tout un monde divin, un royaume des Esprits qui t’attend, un Idéal qui t’invite. Tu as une idée fixe !\par
Ne crois pas que je plaisante ou que je parle par métaphore, quand je déclare radicalement fous, fous à lier, tous ceux que l’infini, le surhumain tourmente, c’est-à-dire, à en juger par l’unanimité de ses vœux, à peu près toute la race humaine. Qu’appelle-t-on en effet une « idée fixe » ? Une idée à laquelle l’homme est asservi. Lorsque vous reconnaissez l’insanité d’une telle idée, vous enfermez son esclave dans une maison de santé. Mais que sont donc la Vérité religieuse dont il n’est pas permis de douter, la Majesté (celle du Peuple par exemple) que l’on ne peut secouer sans lèse-majesté, la Vertu, à laquelle le censeur, gardien de la moralité, ne tolère pas la moindre atteinte ? Ne sont-ce point autant d’ « idées fixes » ? Et qu’est-ce par exemple, que ce radotage qui remplit la plupart de nos journaux, sinon le langage de fous que hante une idée fixe de légalité, de moralité, de christianisme, fous qui n’ont l’air d’être en liberté que grâce à la grandeur du préau où ils prennent leurs ébats ? Essayez donc d’entreprendre un tel fou au sujet de sa manie, immédiatement vous aurez à protéger votre échine contre sa méchanceté ; car ces fous de grande envergure ont encore cette ressemblance avec les pauvres gens dûment déclarés fous, qu’ils se ruent haineusement sur quiconque touche à leur marotte. Ils vous volent d’abord votre arme, ils vous volent la liberté de la parole, puis ils se jettent sur vous les griffes en avant. Chaque jour montre mieux la lâcheté et la rage de ces maniaques, et le peuple, comme  un imbécile, leur prodigue ses applaudissements. Il suffit de lire les gazettes d’aujourd’hui, et d’écouter parler les philistins pour acquérir bien vite la désolante conviction qu’on est enfermé avec des fous dans une maison de santé. « Tu ne traiteras pas ton frère de fou, sinon... etc. ! » Mais la menace me laisse froid, et je répète : mes frères sont des fous fieffés.\par
Qu’un pauvre fou dans son cabanon se nourrisse de l’illusion qu’il est Dieu le Père, l’Empereur du Japon, le Saint-Esprit, ou qu’un brave bourgeois s’imagine qu’il est appelé par sa destinée à être bon chrétien, fidèle protestant, citoyen loyal, homme vertueux, — c’est identiquement la même « idée fixe ». Celui qui ne s’est jamais risqué à n’être ni bon chrétien, ni fidèle protestant, ni homme vertueux, est enfermé et enchaîné dans la foi, la vertu, etc. C’est ainsi que les scolastiques ne philosophaient que dans les limites de la foi de l’Église, et que le pape Benoît XIV écrivit de volumineux bouquins dans les limites de la superstition papiste, sans que le moindre doute effleurât leur croyance ; c’est ainsi que les écrivains entassent in-folios sur in-folios traitant de l’Etat, sans jamais mettre en question l’idée fixe d’Etat elle-même ; c’est ainsi que nos gazettes regorgent de politique parce qu’elles sont infectées de cette illusion que l’homme est fait pour être un « zoon politicon ». Et les sujets végètent dans leur servitude, les gens vertueux dans la vertu, les Libéraux dans les « éternels principes de 89 », sans jamais porter dans leur idée fixe le scalpel de la critique. Ces idoles restent inébranlables sur leurs larges pieds comme les manies d’un fou, et celui qui les met en doute joue avec les vases de l’autel ! Redisons-le encore : une idée fixe, voilà ce qu’est le vrai sacro-saint !\par
Ne nous heurtons-nous qu’à des possédés du Diable, ou rencontrons-nous aussi souvent des possédés d’espèce contraire, possédés par le Bien, la Vertu, la Morale, la Loi, ou n’importe quel autre « principe » ?  Les possessions diaboliques ne sont point les seules : si le Diable nous tire par une manche, Dieu nous tire par l’autre ; d’un côté la « tentation », de l’autre la « grâce » ; mais quelle que soit celle qui opère, les possédés n’en sont pas moins acharnés dans leur opinion.\par
« Possession » vous déplaît ? Dites obsession, ou, puisque c’est l’Esprit qui vous possède et qui vous suggère tout, dites \emph{inspiration}, \emph{enthousiasme}. J’ajoute que l’enthousiasme, dans sa plénitude, car il ne peut être question de faux, de demi-enthousiasme, s’appelle — \emph{fanatisme}.\par
Le \emph{fanatisme} est spécialement propre aux gens cultivés, car la culture d’un homme est en raison de l’intérêt qu’il attache aux choses de l’esprit, et cet intérêt spirituel, s’il est fort et vivace, n’est et ne peut être que \emph{fanatisme ;} c’est un intérêt fanatique pour ce qui est sacré \emph{(fanum)}.\par
Observez nos libéraux, lisez certains de nos journaux saxons, et écoutez ce que dit Schlosser\footnote{ \noindent Achtzehntes Jahrhundert, 11, 519.
 } : « La société d’Holbach ourdit un complot formel contre la doctrine traditionnelle et l’ordre établi, et ses membres mettaient dans leur incrédulité autant de fanatisme que moines et curés, jésuites, piétistes et méthodistes ont coutume d’en mettre au service de leur piété machinale et de leur foi littérale ».\par
Examinez la façon dont se comporte aujourd’hui un homme « moral », qui pense en avoir bien fini avec Dieu, et qui rejette le Christianisme comme une guenille usée. Demandez-lui s’il lui est déjà arrivé de mettre en doute que les rapports charnels entre frère et sœur soient un inceste, que la monogamie soit la vraie loi du mariage, que la piété soit un devoir sacré, etc. : Vous le verrez saisi d’une vertueuse horreur à cette idée qu’on pourrait traiter sa sœur  en femme, etc. Et d’où lui vient cette horreur ? De ce qu’il \emph{croit} à une loi morale. Cette \emph{foi} morale est solidement ancrée en lui. Avec quelque vivacité qu’il s’insurge contre la \emph{piété} des Chrétiens, il n’en est pas moins resté également chrétien par la \emph{moralité.} Par son côté moral, le Christianisme le tient enchaîné, et enchaîné dans la \emph{foi.} La monogamie doit être quelque chose de sacré, et le bigame sera châtié comme un \emph{criminel ;} celui qui se livre à l’inceste portera le poids de son \emph{crime.} Et ceci s’applique aussi à ceux qui ne cessent de crier que la Religion n’a rien, à voir avec l’Etat, que Juif et Chrétien sont également citoyens. Inceste, monogamie, ne sont-ce point autant de \emph{dogmes ?} Qu’on s’avise d’y toucher, et l’on éprouvera qu’il y a dans cet homme moral l’étoffe d’un \emph{inquisiteur} à faire envie à un Krummacher ou à un Philippe II. Ceux-ci défendaient l’autorité religieuse de l’Eglise, lui défend l’autorité morale de l’Etat, les lois morales sur lesquelles l’Etat repose ; l’un comme l’autre condamnent au nom d’articles de foi : quiconque agit autrement que ne le permet \emph{leur foi} à eux, on lui infligera la flétrissure due à son « crime », et on l’enverra pourrir dans une maison de correction, au fond d’un cachot. La croyance morale n’est pas moins fanatique que la religieuse. Et cela s’appelle « liberté de conscience », quand un frère et une sœur sont jetés en prison au nom d’un principe que leur « conscience » avait rejeté ? — Mais ils donnaient un exemple détestable ! — Certes oui, car il se pourrait que d’autres s’avisassent grâce à eux que l’Etat n’a point à se mêler de leurs relations, et que deviendrait la « pureté des mœurs ? » D’où, tolle général : « Sainteté divine ! » crient les zélateurs de la Foi, « Vertu sacrée ! » crient les apôtres de la Morale.\par
Ceux qui s’agitent pour des intérêts sacrés se ressemblent souvent fort peu. Combien les orthodoxes stricts ou vieux croyants diffèrent des combattants pour « la Vérité, la Lumière et le Droit », des  Philalèthes, des amis de la lumière, etc. ! Et cependant rien d’essentiel, de fondamental ne les sépare. Si l’on attaque telle ou telle des vieilles vérités traditionnelles (le miracle, le droit divin,) les plus éclairés applaudissent, les vieux croyants sont seuls à gémir. Mais si l’on s’attaque à la vérité elle-même, aussitôt tous se retrouvent croyants, et on les a tous à dos. De même pour les choses de la morale : les bigots sont intolérants, les cerveaux éclairés se piquent d’être plus larges ; mais si quelqu’un s’avise de toucher à la Morale elle-même, tous font aussitôt cause commune contre lui. « Vérité, Morale, Droit » sont et doivent rester « sacrés ». Ce qu’on trouve à blâmer dans le Christianisme ne peut, disent les plus libéraux, qu’y avoir été introduit à tort, et n’est point vraiment chrétien ; mais le Christianisme doit rester au-dessus de toute discussion, c’est la « base » immuable qu’il est « criminel » d’ébranler. L’hérétique contre la croyance pure n’est plus exposé, il est vrai, à la rage de persécution de jadis, mais celle-ci s’est tournée tout entière contre l’hérétique qui touche à la morale pure.\par

\asterism

\noindent La Piété a eu depuis un siècle tant d’assauts à subir, elle a si souvent entendu reprocher à son essence surhumaine d’être tout bonnement « inhumaine », qu’on ne peut plus guère être tenté de s’attaquer à adversaires se sont présentés pour la combattre, ce fut presque toujours au nom de la Morale elle-même, pour détrôner l’Etre suprême au profit d’un — autre être suprême. Ainsi Proudhon\footnote{ \noindent P{\scshape roudhon}, \emph{De la création de l’ordre}, p. 36.
 } n’hésite pas à dire : « Les hommes sont destinés à vivre sans religion, mais la morale est éternelle et absolue ; qui oserait aujourd’hui attaquer la  morale ? ». Les moralistes ont tous passé dans le lit de la Religion, et après qu’ils se sont plongés jusqu’au cou dans l’adultère, c’est à qui dira aujourd’hui en s’essuyant la bouche : « la Religion ? Je ne connais pas cette femme-là ! »\par
Si nous montrons que la Religion est loin d’être mortellement atteinte tant qu’on se borne à incriminer son essence surnaturelle, et qu’elle en appelle en dernière instance à l’ « Esprit » (car Dieu est l’Esprit), nous aurons suffisamment fait voir son accord final avec la moralité pour qu’il nous soit permis de les laisser à leur interminable querelle.\par
Que vous parliez de la Religion ou de la Morale, il s’agit toujours d’un être suprême ; que cet être suprême soit surhumain ou humain, peu m’importe, il est en tous cas un être au-dessus de moi. Qu’il devienne en dernière analyse l’essence humaine ou l’ « Homme », il n’aura fait que quitter la peau de la vieille religion pour revêtir une nouvelle peau religieuse.\par
Voyez Feuerbach : il nous enseigne que « du moment qu’on s’en tient à la philosophie spéculative, c’est-à-dire qu’on fait systématiquement du prédicat le sujet, et, réciproquement, du sujet l’objet et le principe, on possède la vérité nue et sans voiles\footnote{ \noindent F{\scshape euerbach}, \emph{Anekdota}, 11, 64.
 } ». Sans doute, nous abandonnons ainsi le point de vue étroit de la Religion, nous abandonnons le \emph{Dieu} qui à ce point de vue est sujet ; mais nous ne faisons que le troquer pour l’autre face du point de vue religieux, le \emph{Moral.} Nous ne disons plus par exemple « Dieu est l’amour », mais bien « l’amour est divin » ; remplaçons même le prédicat « divin » par son équivalent « sacré », et nous en sommes toujours à notre point de départ, nous n’avons pas fait un pas. L’amour n’en reste pas moins pour l’homme le \emph{Bien,} ce qui le divinise, ce qui le rend respectable, sa véritable « humanité »,  ou, pour nous exprimer plus exactement, l’amour est ce qu’il y a dans l’homme de véritablement \emph{humain,} et ce qu’il y a en lui d’inhumain, c’est l’égoïste sans amour.\par
Mais, précisément, tout ce que le Christianisme, et avec lui la philosophie spéculative, c’est-à-dire la théologie, nous présente comme le bien, l’absolu, n’est proprement pas le bien (ou, ce qui revient au même, n’est \emph{que le bien}) ; de sorte que cette transmutation du prédicat en sujet ne fait qu’affirmer plus solidement encore l’\emph{être} chrétien (le prédicat lui-même postule déjà l’être). Le dieu et le divin m’enlacent plus indissolublement encore. Avoir délogé le dieu de son ciel, et l’avoir ravi à la « \emph{transcendance} » cela ne justifie nullement vos prétentions à une victoire définitive, tant que vous ne faites que le refouler dans le cœur humain et le doter d’une indéracinable « \emph{immanence} ». Il faudra dire désormais : le divin est le véritablement humain.\par
Ceux-là mêmes qui se refusent à voir dans le Christianisme le fondement de l’Etat, et qui s’insurgent contre toute formule telle que Etat chrétien, Christianisme d’Etat, etc., ne se lassent pas de répéter que la Moralité est « la base de la vie sociale et de l’Etat ». Comme si le règne de la Moralité n’était pas la domination absolue du sacré, une « Hiérarchie » !\par
A ce propos, on peut se rappeler la tentative d’explication qu’on a voulu opposer à l’ancienne doctrine des théologiens. A les en croire, la foi seule serait capable de saisir les vérités religieuses, Dieu ne se révélerait qu’aux seuls croyants, ce qui revient à dire que seuls le cœur, le sentiment, la fantaisie dévote sont religieux. A cette affirmation on répondit que l’ « intelligence naturelle », la raison humaine sont également aptes à connaître Dieu (singulière prétention de la raison, pour le dire en passant, que de vouloir rivaliser de fantaisie avec la fantaisie elle-même).\par
 C’est dans ce sens que Reimarus écrivit ses « Vornehmsten Wahrheiten der natürlichen Religion » (Principales vérités de la Religion naturelle). Il en vint à considérer l’homme \emph{entier} comme tendant à la religion par toutes ses facultés ; cœur, sentiment, intelligence, raison, sentir, savoir, vouloir, tout chez l’homme lui parut \emph{religieux}. Hégel a bien montré que la philosophie elle-même est religieuse ! Et que ne décore-t-on point de nos jours du nom de Religion ? La « Religion de l’Amour », « la Religion de la Liberté », la « Religion politique », bref tout enthousiasme. Et, au fond, on n’a pas tort !\par
Aujourd’hui encore nous employons ce mot d’origine latine « Religion », qui par son étymologie exprime l’idée de \emph{lien.} Et liés nous sommes en effet, et liés nous resterons tant que nous serons imprégnés de religion. Mais l’Esprit aussi est-il lié ? Au contraire, l’Esprit est libre ; il est l’unique maître, il n’est pas \emph{notre }Esprit, mais il est absolu. Aussi, la vraie traduction affirmative du mot Religion serait — « \emph{Liberté spirituelle.} » Celui dont l’Esprit est libre est par là même religieux, comme celui qui donne libre cours à ses appétits est sensuel ; l’Esprit lie l’un, la Chair lie l’autre. Liaison, dépendance, — \emph{Religio}, telle est la Religion par rapport à moi : je suis lié ; Liberté, voilà la Religion par rapport à l’Esprit : il est libre, il jouit de la liberté spirituelle.\par
Le mal que peut nous faire le déchaînement de nos passions, combien le connaissent pour en avoir souffert ! Mais que le libre Esprit, la radieuse spiritualité, l’enthousiasme pour des intérêts idéaux puissent nous plonger dans une détresse pire que ne le ferait la plus noire méchanceté, c’est ce que l’on ne veut pas voir ; et l’on ne peut d’ailleurs s’en aviser, si l’on n’est et ne fait profession d’être un égoïste.\par
Reimarus, et avec lui tous ceux qui ont montré que notre raison aussi bien que notre cœur, etc., conduisent à Dieu, ont montré du même coup que nous  sommes complètement et totalement possédés. Assurément, ils faisaient tort aux théologiens, auxquels ils enlevaient le monopole de l’illumination religieuse ; mais ils n’en élargissaient pas moins d’autant le domaine de la Religion et de la liberté spirituelle. En effet, si par Esprit vous n’entendez plus seulement le sentiment ou la foi, mais l’Esprit dans toutes ses manifestations, intelligence, raison, et pensée en général, et si vous lui permettez en tant qu’intelligence, etc. de participer aux vérités spirituelles et célestes, en ce cas c’est l’Esprit tout entier qui s’élève à la pure spiritualité, et qui est libre.\par
Partant de ces prémisses, la Moralité était autorisée à se mettre en opposition absolue avec la Piété. C’est cette opposition qui se fit jour révolutionnairement sous forme d’une haine brûlante contre tout ce qui ressemblait à un « commandement » (ordonnance, décret, etc.), et contre la [{\corr personne}] honnie et persécutée du « maître absolu ». Elle s’affirma dans la suite comme doctrine, et trouva d’abord sa formule dans le Libéralisme, dont la « bourgeoisie constitutionnelle » est la première expression historique, et qui éclipsa les puissances religieuses proprement dites (voir plus loin le « Libéralisme »).\par
La moralité ne dérivant plus simplement de la piété, mais ayant ses racines propres, le principe de la morale ne découle plus des commandements divins, mais des lois de la raison ; pour que ces commandements restent valables, il faut d’abord que leur valeur ait été contrôlée par la raison, et qu’ils soient contresignés par elle. Les lois de la raison sont l’expression de l’homme lui-même, car « l’Homme » est raisonnable, et « l’essence de l’homme » implique ces lois de toute nécessité. Piété et moralité diffèrent en ce que la première reconnaît Dieu et la seconde l’Homme pour législateurs. En se mettant à un certain point de vue de la moralité, on raisonne à peu près comme suit : Ou bien l’homme obéit à sa sensualité  et par là il est \emph{immoral}, ou bien il obéit au Bien, lequel, en tant que facteur agissant sur la volonté, s’appelle sens moral (sentiment, préoccupation du Bien), et dans ce cas il est \emph{moral}. Comment, à ce point de vue, peut-on appeler immoral l’acte de Sand tuant Kotzebue ? Ce qu’on appelle désintéressé, cet acte l’était sûrement autant que, par exemple, les larcins de saint Crispin au profit des pauvres. « Il ne devait pas assassiner, car il est écrit : tu ne tueras pas ! » — Poursuivre le bien, le bien public (comme Sand croyait le faire), ou le bien des pauvres (comme Crispin) est donc moral, mais le meurtre et le vol sont immoraux : but moral, moyen immoral. Pourquoi ? — « Parce que le meurtre, l’assassinat, est mal en soi, d’une manière absolue ». — Lorsque les Guérillas entraînaient les ennemis de leur pays dans les ravins et les canardaient à loisir, embusqués derrière les buissons, n’était-ce pas un assassinat ?\par
Si vous vous en tenez au principe de la morale qui prescrit de poursuivre partout et toujours le Bien, vous en êtes réduits à vous demander si en aucun cas, le meurtre ne peut arriver à réaliser ce Bien ; dans l’affirmative vous devez liciter ce meurtre dont le Bien est sorti. Vous ne pouvez condamner l’action de Sand : elle fut morale, parce que désintéressée et sans autre objectif que le Bien ; ce fut un châtiment infligé par un individu, une \emph{exécution}, pour laquelle il risquait sa vie.\par
Que voir dans l’entreprise de Sand, sinon sa volonté de supprimer de vive force certains écrits ? N’avez-vous jamais vu appliquer ce même procédé comme très « légal » et très sanctionné ? Et que répondre à cela au nom de votre principe de la Moralité ? — « C’était une exécution illégale ! » L’immoralité du fait était-elle donc dans son illégalité, dans la désobéissance à la loi ? Accordez-moi tout d’un coup que le Bien n’est autre chose que — la Loi, et que Moralité égale Légalité ! Votre moralité doit se résigner  à n’être plus qu’une vaine façade de « légalité », une fausse dévotion à l’accomplissement de la loi, bien plus tyrannique et plus révoltante que l’ancienne ; celle-ci n’exigeait que la \emph{pratique extérieure}, tandis que vous exigez en plus l’\emph{intention :} on doit porter en soi la règle et le dogme, et le plus légalement intentionné est le plus moral. La dernière clarté de la vie catholique s’éteint dans cette légalité protestante. Ainsi finalement se complète et s’absolutise la domination de la Loi. « Ce n’est pas moi qui vis, c’est la Loi qui vit en moi ». J’en arrive à n’être plus que le « vaisseau de sa gloire ». « Chaque Prussien porte son gendarme dans sa poitrine », disait en parlant de ses compatriotes un officier supérieur.\par

\asterism

\noindent D’où vient l’incurable impuissance de certaines \emph{oppositions ?} Uniquement de ce qu’elles ne veulent point s’écarter du chemin de la Moralité ou de la Légalité, ce qui les condamne à jouer cette monstrueuse comédie de dévouement, d’amour, etc., dont l’hypocrite mauvaise grâce achève d’écœurer ceux que dégoûtent la pourriture et la cafarderie de ce qui s’intitule « opposition légale ».\par
Un accord \emph{moral} conclu au nom de l’amour et de la fidélité ne laisse place à aucune volonté discordante et opposée ; la belle harmonie est rompue si l’un veut une chose et l’autre le contraire. Or, l’usage et un vieux préjugé exigent avant tout de l’opposition le respect de ce pacte moral. Que reste-t-il à l’opposition ? Peut-elle vouloir une liberté lorsque l’élu, la majorité, trouvent bon de la repousser ? Non ! Elle n’oserait \emph{vouloir} la liberté ; tout ce qu’elle peut faire, c’est la \emph{souhaiter}, et pour l’obtenir, « pétitionner » et tendre la main en la demandant par charité. Voyez-vous ce qui arriverait si l’opposition \emph{voulait} réellement,  si elle voulait de toute l’énergie de sa volonté ? Non, non : qu’elle sacrifie la Volonté à l’Amour, qu’elle renonce à la Liberté — pour les beaux yeux de la Moralité. Elle ne doit jamais « réclamer comme un droit » ce qu’il lui est seulement permis de « demander comme une grâce ». L’amour, le dévouement, etc., exigent impérieusement qu’il n’y ait qu’une seule volonté devant laquelle toutes les autres s’inclinent, à laquelle elles obéissent avec amour et soumission. Que cette volonté soit raisonnable ou déraisonnable, il est en tous cas moral de s’y soumettre et immoral de s’y soustraire.\par
La volonté qui régit la censure paraît déraisonnable à beaucoup de gens. Cependant, dans un pays où sévit la censure, celui qui lui soustrait ses écrits fait mal et celui qui les lui soumet fait bien. Que quelqu’un, dûment averti et rappelé à l’ordre par le censeur, passe outre et installe par exemple une presse clandestine, on sera en droit de l’accuser d’immoralité, et, qui plus est, de sottise s’il se fait prendre ; son aventure ne lui donnera-t-elle pas quelque titre à l’estime des « honnêtes gens » ? Qui sait ? — Peut-être s’imaginait-il servir une « moralité supérieure » ?\par
La toile de l’hypocrisie moderne est tendue aux confins des deux domaines entre lesquels, alternativement ballottée, notre époque tend les fils déliés du mensonge et de l’erreur. Trop faible désormais pour servir la morale sans hésitation et sans défaillance, trop scrupuleuse encore pour vivre tout à fait selon l’égoïsme, elle passe en tremblant, dans la toile d’araignée de l’hypocrisie, d’un principe à l’autre, et, paralysée par le fléau de l’incertitude, ne capture plus que de sottes et pauvres mouches. A-t-on eu l’audace grande de dire carrément son avis, aussitôt on énerve la liberté du propos par des protestations d’amour : — \emph{résignation hypocrite.} A-t-on au contraire eu le front de combattre une affirmation libre en invoquant \emph{moralement} la bonne foi, etc., aussitôt le courage  moral s’évanouit et l’on assure que c’est avec un plaisir tout particulier qu’on a entendu cette vaillante parole : — \emph{approbation hypocrite.} Bref, on voudrait tenir l’un, mais ne pas lâcher l’autre ; on veut vouloir \emph{librement}, mais on n’entend pas, à Dieu ne plaise, cesser de vouloir \emph{moralement.} — Voyons, Libéraux, vous voilà en présence d’un de ces adversaires dont vous méprisez la servilité ; nous vous écoutons : vous atténuez l’effet de chaque mot un peu libéral par un regard de la plus loyale fidélité ; lui habille son servilisme des plus chaudes protestations de libéralisme. Maintenant, séparez-vous ; chacun pense de l’autre : je te connais, masque ! Il a flairé en vous le Diable, aussi bien que vous avez flairé en lui le vieux Bon Dieu.\par
Un Néron n’est « mauvais » qu’aux yeux des « bons » ; à mes yeux, il est simplement un \emph{possédé}, comme les bons eux-mêmes. Les bons voient en lui un franc scélérat et le vouent à l’enfer. Comment se fait-il que rien ne se soit opposé à ses caprices ? Comment a-t-on pu tant supporter ? Les Romains domestiqués valaient-ils un liard de plus pour se laisser fouler aux pieds par un tel tyran ? Dans l’ancienne Rome, on l’eût immédiatement supprimé, et on ne fût jamais devenu son esclave. Mais les « honnêtes gens » de son temps se bornaient, dans leur moralité, à lui opposer leurs vœux, et non leur \emph{volonté}. Ils chuchotaient que leur empereur ne se soumettait pas comme eux aux lois de la Morale, mais ils restaient des « sujets moraux », en attendant que l’un d’eux osât passer franchement par dessus « ses devoirs de sujet obéissant ». Et tous ces « bons Romains », tous ces « sujets soumis » abreuvés d’outrages par leur manque de volonté, d’acclamer aussitôt l’action criminelle et immorale du révolté.\par
Où était, chez les « bons », le courage de faire la \emph{Révolution}, cette Révolution qu’ils vantent et exploitent aujourd’hui, après qu’un autre l’a faite ? Ce courage  ils ne pouvaient l’a voir, car toute révolution, toute insurrection est toujours quelque chose « d’immoral », auquel on ne peut se résoudre à moins de cesser d’être « bon » pour devenir « mauvais » ou — ni bon ni mauvais.\par
Néron n’était pas pire que le temps où il vivait ; on ne pouvait alors être que l’un des deux : bon ou mauvais. Son temps a jugé qu’il était mauvais, et aussi mauvais qu’on peut l’être, non par faiblesse, mais par scélératesse pure ; quiconque est moral doit ratifier ce jugement. On rencontre encore parfois aujourd’hui des coquins de son espèce mêlés à la foule des honnêtes gens (voyez, par exemple, les Mémoires du Chevalier de Lang). En vérité il ne fait pas bon vivre avec eux, car on n’a pas un instant de sécurité ; mais est-il plus commode de vivre au milieu des bons ? On n’y est guère plus sûr de sa vie, sauf que quand on est pendu, c’est du moins pour la bonne cause ; quant à l’honneur il est encore plus en danger, bien que le drapeau national le couvre de ses plis tutélaires. Le rude poing de la morale est sans miséricorde pour la noble essence de l’égoïsme.\par
« On ne peut cependant pas mettre sur la même ligne un gredin et un honnête homme ! » Eh ! qui donc le fait plus souvent que vous, Censeurs ? Bien mieux, l’honnête homme qui s’élève ouvertement contre l’ordre établi, contre les sacro-saintes institutions, etc., vous le coffrez comme un criminel, tandis qu’à un subtil coquin, vous confiez vos portefeuilles et des choses encore plus précieuses. Donc, \emph{in praxi}, vous n’avez rien à me reprocher. « Mais en théorie ! » En théorie, je les mets sur la même ligne, sur la ligne de la moralité, dont ils sont les deux pôles opposés. Bons et mauvais, ils n’ont de signification que dans le monde « moral », juste comme avant le Christ, être un Juif selon la Loi on non selon la Loi n’avait de signification que par rapport à la Loi mosaïque. Aux yeux du Christ, le pharisien  n’était rien de plus que « les pécheurs et les publicains », et de même, aux yeux de l’individualiste, le pharisien moral vaut le pécheur immoral.\par
Néron était un possédé très malcommode, un fou dangereux. C’eût été une sottise de perdre son temps à le rappeler au « respect des choses sacrées », pour lamenter ensuite parce que le tyran n’en tenait aucun compte et agissait à sa guise. A chaque instant on entend des gens invoquer la sacro-sainteté des imprescriptibles droits de l’Homme devant ceux-là même qui en sont les ennemis, et s’efforcer de prouver et de démontrer par anticipation que telle ou telle liberté est un des « droits sacrés de l’Homme ». Ceux qui se livrent à ces exercices méritent d’être raillés comme ils le sont, si, fût-ce [{\corr inconsciemment}], ils ne prennent pas résolument le chemin qui conduit à leur but. Ils pressentent que ce n’est que lorsque la majorité sera acquise à cette liberté qu’ils désirent, qu’elle la \emph{voudra} et la \emph{prendra}. Ce n’est pas la sainteté d’un droit et toutes les preuves qu’on peut en fournir qui en font approcher d’un pas : se lamenter, pétitionner ne convient qu’aux mendiants.\par
L’homme « moral » est nécessairement borné, en ce qu’il ne conçoit d’autre ennemi que l’ « immoral » ; ce qui n’est pas bien est « mal » et par conséquent réprouvé, odieux, etc. Aussi est-il radicalement incapable de comprendre l’égoïste. L’amour en dehors du mariage n’est-il pas immoral ? L’homme moral peut tourner et retourner la question, il n’échappera pas à la nécessité de condamner le fornicateur. L’amour libre est bien une immoralité, et cette vérité morale a coûté la vie à Emilia Galotti. Une jeune fille vertueuse vieillira fille ; un homme vertueux usera sa vie à refouler les aspirations de sa nature jusqu’à ce qu’elles soient étouffées, il se mutilera même par amour de la vertu, comme Origène par amour au ciel : ce sera honorer la sainteté du mariage, l’inviolable sainteté de la chasteté, ce sera  moral. L’impureté ne peut jamais porter un bon fruit ; avec quelque indulgence que l’honnête homme juge celui qui s’y livre, elle reste une faute, une infraction à une loi morale, et entraîne une souillure ineffaçable. La chasteté qui faisait jadis partie des vœux monastiques est entrée dans le domaine de la morale commune.\par
Pour l’égoïste, au contraire, la chasteté n’est pas un bien dont il ne puisse se passer ; elle est pour lui sans importance. Aussi, quel va être le jugement de l’homme moral à son égard ? Celui-ci : il classera l’égoïste dans la seule catégorie de gens qu’il conçoive en dehors des « moraux », dans celle des — immoraux. Il ne peut faire autrement ; l’égoïste, n’ayant aucun respect pour la moralité, doit lui paraître immoral. S’il le jugeait autrement, c’est que, sans se l’avouer, il ne serait plus un homme véritablement moral, mais un apostat de la Moralité. Ce phénomène, qui n’est plus fort rare aujourd’hui, ne doit pas nous induire en erreur ; il faut bien se dire que celui qui tolère la moindre atteinte à la moralité ne mérite pas plus le nom d’homme moral que Lessing ne méritait celui de pieux chrétien, lui qui dans une parabole bien connue compare la religion chrétienne aussi bien que la mahométane et la juive à une « bague fausse ». Souvent les gens sont déjà beaucoup plus loin qu’ils ne voudraient en convenir.\par
C’eût été de la part de Socrate une immoralité d’accueillir les offres séduisantes de Criton et de s’échapper de sa prison ; rester était le seul parti qu’il pût moralement prendre. Et c’était le seul, simplement parce que Socrate était — un homme moral.\par
Les hommes de la Révolution, « immoraux et impies », avaient, eux, juré fidélité à Louis XVI, ce qui ne les empêcha pas de décréter sa déchéance et de l’envoyer à l’échafaud ; action immorale, qui fera horreur aux honnêtes gens de toute éternité.\par
 
\asterism

\noindent Ces critiques ne s’appliquent toutefois qu’à la « morale bourgeoise », que tout esprit un peu libre fait profession de dédaigner. Cette morale, comme la bourgeoisie dont elle est la fille, est encore trop près du ciel, trop peu affranchie de la Religion, pour ne pas se borner à s’en approprier les lois. N’exigez pas d’elle de la critique, et ne lui demandez pas de tirer de son propre fond une doctrine originale.\par
C’est sous un tout autre aspect que se présente la morale, lorsque, consciente de sa dignité, elle prend pour unique règle son principe, l’essence humaine ou « l’Homme ». Ceux qui parviennent à transporter résolument le problème sur ce terrain rompent pour toujours avec la Religion : il n’y a plus de place pour son Dieu auprès de leur Homme ; de plus, comme ils coulent à fond le vaisseau de l’Etat (voir plus loin), ils anéantissent du même coup toute « moralité » procédant du seul Etat, et s’interdisent par conséquent d’en invoquer jamais même le nom. Ce que ces « Critiques » désignent sous le nom de moralité s’écarte définitivement de la morale dite « bourgeoise » ou « politique », et doit paraître aux hommes d’état et aux bourgeois une « licence effrénée ».\par
Cependant, cette conception nouvelle de la moralité n’a rien de neuf et d’inédit ; elle ne fait que s’adapter au progrès réalisé dans la « pureté du principe ». Ce dernier, lavé de la souillure de son adultère avec le principe religieux, se précise et atteint son plein épanouissement en devenant « l’Humanité ». Aussi ne faut-il pas s’étonner de voir conserver ce nom de moralité, à côté d’autres comme liberté, humanité, conscience, etc., en se contentant d’y ajouter tout au plus l’épithète « libre ». La morale devient « morale libre », comme l’Etat bourgeois,  quoique bouleversé de fond en comble, devient « Etat libre » ou même « Société libre », sans cesser d’être l’une la morale et l’autre l’Etat.\par
La morale étant désormais purement humaine et complètement séparée de la Religion dont, historiquement, elle est sortie, rien ne s’oppose à ce qu’elle devienne elle-même une religion. En effet, la Religion ne diffère de la Morale que pour autant que nos relations avec le monde des hommes sont réglées et sanctifiées par nos rapports avec un être surhumain, et que nous n’agissons plus que par « amour de Dieu ». Mais admettez que « l’Homme est pour l’homme l’être suprême », et toute différence s’efface ; la Morale quitte son rang subalterne, elle se complète, s’absolutise et devient — Religion. L’Homme, être supérieur, jusqu’ici subordonné à un Etre suprême, s’élève à la hauteur absolue, et nous sommes dans nos rapports avec Lui ce que nous sommes aux pieds d’un être suprême, — religieux.\par
Moralité et Piété redeviennent ainsi aussi parfaitement synonymes qu’au début du Christianisme. Si le sacré n’est plus « saint » mais « humain », c’est simplement que l’être suprême a changé et que l’Homme a pris la place du Dieu. La victoire de la Moralité aboutit simplement à un \emph{changement de dynastie}.\par
La Foi détruite, Feuerbach croit trouver un asile dans l’Amour. « La première et la suprême loi doit être l’amour de l’homme pour l’homme. \emph{Homo homini Deus est}, telle est la maxime pratique la plus haute ; par elle la face du monde est changée\footnote{ \noindent \emph{Wesen des Christentums, zw. Aufl.}, p. 402.
 }. » Mais il n’y a à proprement parler que le dieu, \emph{Deus}, de changé ; l’amour reste : vous adoriez le dieu surhumain, vous adorerez le dieu humain, l’\emph{Homo qui est Deus}. L’Homme m’est — sacré, et tout ce qui est « vraiment humain » m’est — sacré ! « Le mariage est par  lui-même sacré ; de même toutes les relations de la vie morale : l’amitié, la propriété, le mariage, le bien de chacun sont et doivent être sacrés, en eux et par eux-mêmes\footnote{ \noindent P. 403.
 }. » Est-ce un prêtre qui parle ? Quel est son dieu ? L’Homme ! Qu’est-ce que le divin ? C’est l’humain ! Le prédicat n’a fait en définitive que prendre la place du sujet ; la proposition « Dieu est l’amour » devient « l’Amour est divin » ; continuez à appliquer le procédé : « Dieu s’est fait Homme » vous donnera « l’Homme s’est fait Dieu, » etc..., et voilà une nouvelle — Religion.\par
« Tous les phénomènes de la vie morale constituant les mœurs ne sont moraux, ne prennent une signification morale, que s’ils ont en eux-mêmes (sans que la bénédiction du prêtre les consacre) une valeur \emph{religieuse}. » Le sens de la proposition de Feuerbach « La théologie est une anthropologie » se précise et se réduit à « La religion doit être une éthique, l’éthique est la seule religion. » Feuerbach se contente de renverser l’ordre du prédicat et du sujet, de faire un \emph{usteron proteron} logique.\par
Comme il le dit lui-même : « L’amour n’est pas sacré (et n’a jamais passé pour sacré aux yeux des hommes) parce qu’il est un prédicat de Dieu, mais il est un prédicat de Dieu parce qu’il est par lui-même et pour lui-même divin. » Pourquoi donc ne déclare-t-il pas la guerre aux prédicats eux-mêmes, à l’amour et à toute sacro-sainteté ? Comment peut-il se flatter de détourner les hommes de Dieu, s’il leur laisse le divin ? Et si, comme il le dit, l’essentiel pour eux n’a jamais été Dieu, mais ses seuls prédicats, à quoi bon leur enlever le mot si on leur laisse la chose ?\par
Il proclame d’autre part que son but est « de détruire une illusion\footnote{ \noindent P. 408.
 } », une illusion pernicieuse « qui a si bien faussé l’homme, que l’amour même,  son sentiment le plus intime et le plus vrai, est devenu, par le fait de la religiosité, vain et illusoire, vu que l’amour religieux n’aime l’homme que par amour de Dieu, c’est-à-dire aime en apparence l’homme et en réalité Dieu. » Mais en est-il autrement de l’amour moral ? S’attache-t-il à l’homme, à \emph{tel ou tel homme} en particulier, par amour de \emph{lui}, \emph{cet} homme, ou par amour de la Moralité, de l’Homme en général, et, en définitive, — puisque \emph{Homo homini Deus,} — par. amour de Dieu ?\par

\asterism

\noindent La marotte se manifeste encore sous une foule d’autres formes ; il est nécessaire d’en énumérer ici quelques-unes.\par
Parmi elles, le \emph{renoncement}, l’\emph{abnégation} sont communs aux saints et aux non saints, aux purs et aux impurs.\par
L’impur \emph{renonce} à tout bon sentiment, « renie » toute pudeur, tout respect humain ; il obéit en esclave docile à ses appétits. Le pur renonce au commerce du monde, « renie le monde », pour se faire l’esclave de son impérieux idéal. L’avare que ronge la soif de l’or renie les avertissements de sa conscience, il renonce à tout sentiment d’honneur, à toute bienveillance et à toute pitié ; sourd à toute autre voix, il court où l’appelle son tyrannique désir. Le saint fait de même ; impitoyable aux autres et à lui-même, rigoriste et dur, il affronte la « risée du monde » et court où l’appelle son tyrannique idéal. De part et d’autre, même abnégation de \emph{soi-même : }si le non saint \emph{abdique} devant Mammon, le saint \emph{abdique} devant Dieu et les lois divines.\par
Nous vivons en un temps où l’\emph{impudence} du Sacré se fait sentir et se révèle chaque jour davantage, parce qu’elle est chaque jour plus obligée de se découvrir et de s’exposer. Peut-on rien imaginer qui surpasse en  insolence et en stupidité les arguments que l’on oppose par exemple aux « progrès du temps » ? La naïveté de leur effronterie passe depuis longtemps toute mesure et toute attente ; mais comment en serait-il autrement ? Saints et non saints, tous ceux qui pratiquent l’abnégation doivent prendre un même chemin, qui, d’abdication en abdication, conduit les uns à s’enfoncer dans la plus ignomineuse \emph{dégradation}, et les autres à s’élever à la plus déshonorante \emph{sublimité}. Le Mammon terrestre et le Dieu du ciel exigent exactement la même somme de — renoncement.\par
Le dégradé et le sublime aspirent tous deux à un « bien », l’un à un bien matériel, l’autre à un bien idéal, et finalement l’un complète l’autre, l’ « homme de la Matière » sacrifiant à sa \emph{vanité}, but idéal, ce que l’ « homme de l’Esprit » sacrifie à une jouissance matérielle, le \emph{confort}.\par
Ceux-là s’imaginent dire énormément qui placent dans le cœur de l’homme le « désintéressement ». Qu’entendent-ils par là ? Quelque chose de très voisin de l’ « abnégation de soi ». \emph{De soi ?} de qui donc ? Qui est-ce qui sera nié et dont l’intérêt sera mis de côté ? Il semble que ce doit être toi. Et au profit de qui \emph{te} recommande-t-on cette abnégation désintéressée ? De nouveau à \emph{ton} profit, à \emph{ton} bénéfice, à charge simplement de poursuivre par désintéressement ton « véritable intérêt ».\par
On doit tirer profit \emph{de soi}, mais ne pas chercher \emph{son} profit.\par
Le \emph{bienfaiteur} de l’humanité comme Franke, le créateur des orphelinats, ou O’Connel l’infatigable défenseur de la cause irlandaise, passe pour désintéressé ; de même le \emph{fanatique} comme saint Boniface qui expose sa vie pour la conversion des païens, Robespierre qui sacrifie tout à la vertu, ou Körner qui meurt pour son Dieu, son Roi et sa Patrie. Leur désintéressement est chose admise. Aussi les adversaires d’O’Connel, par exemple, s’efforçaient-ils de  le représenter comme un homme cupide (accusations auxquelles sa fortune donnait quelque vraisemblance), sachant bien que s’ils parvenaient à rendre suspect son désintéressement, il leur serait facile de détacher de lui ses partisans. Tout ce qu’ils pouvaient prouver, c’est que O’Connel visait un autre but que celui qu’il avouait. Mais qu’il eût en vue un avantage pécuniaire ou la liberté de son peuple, il est en tout cas évident qu’il poursuivait un but et même \emph{son }but : dans un cas comme dans l’autre il avait un intérêt, seulement il se trouvait que son intérêt national était utile \emph{à d’autres,} ce qui en faisait un \emph{intérêt commun.}\par
N’existe-t-il donc pas de désintéressement et ne peut-on jamais en rencontrer ? Au contraire, rien n’est plus commun ! On pourrait appeler le désintéressement un article de mode du monde civilisé et on le tient pour si nécessaire que lorsqu’il coûte trop cher en étoffe solide on s’en paie un de camelote : on singe le désintéressement.\par
Où commence le désintéressement ? Précisément au moment où un but cesse d’être \emph{notre} but et notre \emph{propriété} et où nous cessons de pouvoir en disposer à notre guise, en propriétaire, lorsque ce but devient un but fixe ou une — idée fixe, et commence à nous inspirer, à nous enthousiasmer, à nous fanatiser, bref, quand il devient — notre maître. On n’est pas désintéressé tant qu’on tient le but en son pouvoir ; on le devient lorsqu’on pousse le cri du cœur des possédés : « Je suis comme ça, je ne saurais être autrement », et qu’on applique à un but \emph{sacré} un zèle sacré.\par
Je ne suis pas désintéressé, tant que mon but reste \emph{à moi}, et que je le laisse perpétuellement en question au lieu de me faire l’instrument aveugle de son accomplissement. Je peux ne pas déployer pour cela moins de zèle que le fanatique, mais tout mon zèle me laisse en face de mon but froid, calculateur, incroyant  et hostile ; je reste son \emph{juge}, parce que je suis son propriétaire.\par
Le désintéressement pullule là où règne la « possession », aussi bien sur les possessions du Diable que sur celles du bon Esprit : là, vice, folie, etc., ici, résignation, soumission, etc.\par
Où tourner ses regards sans rencontrer quelque victime du renoncement ?\par
En face de chez moi habite une jeune fille qui depuis tantôt dix ans offre à son âme de sanglants holocaustes. C’était jadis une adorable créature, mais une lassitude mortelle courbe aujourd’hui son front, et sa jeunesse saigne et meurt lentement sous ses joues pâles.\par
Pauvre enfant, que de fois les passions ont dû frapper à ton cœur, et réclamer pour ton printemps une part de soleil et de joie ! Quand tu posais ta tête sur l’oreiller, comme la nature en éveil faisait tressaillir tes membres, comme ton sang bondissait dans tes artères ! Toi seule le sais, et toi seule pourrais dire les ardentes rêveries qui faisaient s’allumer dans tes yeux la flamme du désir.\par
Mais, soudain, à ton chevet se dressait un fantôme : l’Ame, le salut éternel !\par
Effrayée, tu joignais les mains, tu levais vers le ciel ton regard éploré, tu — priais. Le tumulte de la nature s’apaisait et le calme immense de la mer s’apesantissait sur les flots mouvants de tes désirs. Peu à peu la vie s’éteignait dans tes yeux, tu fermais tes paupières meurtries, le silence se taisait dans ton cœur, tes mains jointes retombaient inertes sur ton sein sans révolte, un dernier soupir s’exhalait de tes lèvres, et — \emph{l’âme était en repos}. Tu t’endormais, et le lendemain c’étaient de nouveaux combats et — une nouvelle prière.\par
Aujourd’hui, l’habitude du renoncement a glacé l’ardeur de tes désirs et les roses de ton printemps pâlissent au vent desséchant de ta félicité future.  L’âme est sauve, le corps peut périr. O Laïs, ô Ninon, que vous eûtes raison de mépriser cette blême sagesse ! Une grisette, libre et joyeuse, pour mille vieilles filles blanchies dans la vertu !\par

\asterism

\noindent « Axiome, principe, point d’appui moral, » autres formes sous lesquelles s’exprime l’idée fixe.\par
Archimède demandait, pour soulever la terre un point d’appui \emph{en dehors} d’elle. C’est ce \emph{point d’appui} que les hommes ont sans cesse cherché et que chacun a pris où il l’a trouvé et comme il l’a trouvé. Ce point d’appui étranger est le \emph{monde de l’Esprit,} le monde des idées, des pensées, des concepts, des essences, etc., c’est le \emph{Ciel}. C’est sur le ciel qu’on s’appuie pour ébranler la terre, et c’est du ciel qu’on se penche pour contempler les agitations terrestres, et — les mépriser. S’assurer le ciel, s’assurer solidement et pour toujours le point d’appui céleste, combien a peiné pour cela la douloureuse et inlassable humanité !\par
Le Christianisme s’est proposé de nous délivrer du déterminisme de la nature et de la fatalité des appétits. Son but était donc que l’homme ne se laissât plus déterminer par ses désirs et ses passions, ce qui n’implique pas que l’homme ne doit pas \emph{avoir }de désirs, de passions, etc., mais qu’il ne doit pas se laisser posséder par eux, qu’ils ne doivent pas être dans sa vie des facteurs \emph{fixes}, incoercibles et inéluctables.\par
Mais ce que le Christianisme (la Religion) a machiné contre les appétits, ne serions-nous pas en droit de le retourner contre l’Esprit (pensées, représentations, idées, croyance, etc.), par lequel il prétend que nous soyons déterminés ? Ne pourrions-nous exiger que l’Esprit, les représentations, les idées, ne pussent plus nous déterminer, cessassent d’être fixes et hors d’atteinte, autrement dit « sacrées » ? Cela  aurait pour effet de nous \emph{affranchir de l’Esprit,} de nous délier du joug des représentations et des idées.\par
Le Christianisme disait : « Nous devons bien posséder des appétits, mais ces appétits ne doivent pas nous posséder ». Nous lui répondons : « Nous devons bien posséder un esprit, mais l’Esprit ne doit pas nous posséder. » Si cette dernière phrase ne vous offre pas de prime abord un sens satisfaisant, réfléchissez au cas de celui chez qui, par exemple une pensée devient « maxime » de telle sorte qu’il s’en fait lui-même le prisonnier : ce n’est plus lui qui possède la maxime, c’est plutôt elle qui le possède. Et lui en revanche possède dans cette maxime un « solide point d’appui, » Les leçons du catéchisme deviennent peu à peu, sans qu’on s’en aperçoive, des \emph{axiomes} qui ne permettent plus le moindre doute ; leurs pensées ou leur — Esprit deviennent tout puissants et aucune objection de la « chair » ne prévaudra plus contre eux.\par
Ce n’est cependant que par la « chair » que je puis secouer la tyrannie de l’Esprit, car ce n’est que quand un homme comprend aussi sa chair qu’il se comprend entièrement, et ce n’est que quand il \emph{se} comprend entièrement qu’il est intelligent ou raisonnable.\par
Le Chrétien ne comprend pas la détresse de sa nature asservie, l’ « humilité » est sa vie ; c’est pourquoi il ne murmure point contre l’iniquité lorsque sa \emph{personne} en est victime : il se croit satisfait de la « liberté spirituelle. » Mais si la chair élève la voix, et si son ton est, comme il doit l’être, « passionné », « inconvenant », « mal intentionné », « malicieux », etc., le Chrétien croit ouïr des voix diaboliques, des voix \emph{contre l’Esprit} (car la bienséance, l’absence de passion, les bonnes intentions, etc., sont — Esprit) ; il fulmine contre elles, et avec raison : il ne serait pas chrétien s’il les écoutait sans révolte. N’obéissant qu’à la moralité, il stigmatise l’immoralité ;  n’obéissant qu’à la légalité, il bâillonne, il musèle la voix de l’anarchie : l’\emph{Esprit} de moralité et de légalité, maître inflexible et inexorable, le tient captif. C’est là ce qu’ils appellent la « royauté de l’Esprit » — c’est en même temps le \emph{point d’appui }de l’Esprit.\par
Et qui Messieurs les Libéraux veulent-ils libérer ? Quelle est la liberté qu’ils appellent de tous leurs vœux ? Celle de l’\emph{Esprit}, de l’esprit de moralité, de légalité, de piété, etc. Mais Messieurs les Anti-libéraux n’ont pas d’autre désir, et le seul objet de la dispute, c’est l’avantage que chacun ambitionne, d’avoir seul la parole. L’\emph{Esprit} reste le \emph{maître} absolu des uns et des autres, et s’ils se querellent, c’est uniquement pour savoir qui s’assiéra sur le trône héréditaire de « lieutenant du Seigneur. »\par
Ce qu’il y a de meilleur dans l’affaire c’est qu’on peut rester tranquille spectateur de la lutte, avec la certitude que les bêtes féroces de l’histoire s’entre déchirent juste comme celles de la nature ; leurs cadavres en se putréfiant engraisseront le sol pour — nos moissons.\par
Nous reviendrons par la suite sur une foule d’autres marottes : Vocation, Véracité, Amour, etc.\par

\asterism

\noindent Si j’oppose la spontanéité de l’inspiration à la passivité de la suggestion, et ce qui nous est \emph{propre} à ce qui nous est \emph{donné}, on aurait tort de me répondre que, tout tenant à tout et l’univers entier formant un tout solidaire, rien de ce que nous sommes ou de ce que nous avons n’est par conséquent isolé, mais nous vient des influences ambiantes et nous est en somme « donné » ; l’objection porterait à faux, car il y a une grande différence entre les sentiments ou les pensées que ce qui m’entoure \emph{éveille en moi}, et les sentiments et les pensées qu’on \emph{me fournit tout faits.}  Dieu, immortalité, liberté, humanité, sont de ces derniers : on nous les inculque dès l’enfance et ils enfoncent en nous plus ou moins profondément leurs racines ; mais, soit qu’ils gouvernent les uns à leur insu, soit que chez les autres, natures plus riches, ils s’épanouissent et deviennent le point de départ de systèmes ou d’œuvres d’art, ce n’en sont pas moins des sentiments que nous avons toujours \emph{reçus }tels quels, et jamais \emph{produits ;} la preuve en est que nous y croyons et qu’ils s’imposent à nous.\par
Qu’il y ait un Absolu, et que cet Absolu puisse être perçu, senti et pensé, c’est un article de foi pour ceux qui consacrent leurs veilles à le pénétrer et le définir. Le \emph{sentiment} de l’Absolu est pour eux un \emph{datum, }le texte sur lequel toute leur activité se borne à broder les gloses les plus diverses. De même le sentiment religieux était pour Klopstock une « donnée » qu’il ne fit que traduire sous forme d’œuvre d’art dans sa \emph{Messiade}. Si la Religion n’avait fait que le stimuler à sentir et à penser, et s’il avait pu prendre \emph{lui même }position en face d’elle, il eût abouti à analyser et finalement à détruire l’objet de ses pieuses effusions. Mais, devenu homme, il ne fit que ressasser les sentiments dont avait été farci son cerveau d’enfant, et il gaspilla son talent et ses forces à habiller ses vieilles poupées.\par
On comprendra à présent de quelle valeur pratique est la différence que nous faisons entre les sentiments qui nous sont donnés et ceux dont les circonstances extérieures ne font que provoquer en nous l’éclosion. Ces derniers nous sont \emph{propres}, il sont égoïstes, parce qu’on ne nous les a pas soufflés et imposés \emph{en tant que sentiments ;} les premiers au contraire nous ont été donnés, nous les soignons comme un héritage, nous les cultivons et ils nous \emph{possèdent.}\par
Qui a pu ne pas remarquer ou tout au moins éprouver que toute notre éducation consiste à greffer dans notre cervelle certains sentiments déterminés, au lieu  d’y laisser germer au petit bonheur ceux qui y auraient trouvé un sol convenable ? Lorsque nous entendons le nom de Dieu, nous devons éprouver de la crainte ; que l’on prononce devant nous le nom de Sa Majesté le Prince, nous devons nous sentir pénétrés de respect, de vénération et de soumission ; si l’on nous parle de moralité, nous devons entendre quelque chose d’inviolable ; si l’on nous parle du mal ou des méchants, nous ne pouvons nous dispenser de frémir, et ainsi de suite. Ces \emph{sentiments} sont le but de l’éducateur, ils sont obligatoires : si l’enfant se délectait par exemple au récit des hauts faits des méchants, ce serait au fouet à le punir et à le « ramener dans la bonne voie. »\par
Lorsque nous sommes ainsi bourrés de \emph{sentiments donnés}, nous parvenons à la majorité et nous pouvons être « émancipés ». Notre équipement consiste en « sentiments élevés, pensées sublimes, maximes édifiantes, éternels principes, » etc. Les jeunes sont majeurs quand ils gazouillent comme les vieux ; on les pousse dans les écoles pour qu’ils y apprennent les vieux refrains, et quand ils les savent par cœur l’heure de l’émancipation a sonné.\par
Il ne nous est \emph{pas permis} d’éprouver, à l’occasion de chaque objet et de chaque nom qui se présentent à nous, le premier sentiment venu ; le nom de Dieu, par exemple, ne doit pas éveiller en nous d’images risibles ou de sentiments irrespectueux ; ce que nous devons en penser et ce que nous devons sentir nous est d’avance tracé et prescrit.\par
Tel est le sens de ce qu’on appelle la « \emph{charge d’âme} » : mon âme et mon esprit doivent être façonnés d’après ce qui convient aux autres, et non d’après ce qui pourrait me convenir à moi-même.\par
On sait combien il faut se donner de peine pour acquérir une façon \emph{à soi} de sentir vis-à-vis de bien des noms que l’on prononce même tous les jours ; on sait aussi combien il est difficile de rire au nez de celui  qui attend de nous, lorsqu’il nous parle, un air pénétré et un ton de bonne compagnie.\par
Ce qui nous est donné nous est \emph{étranger}, ne nous appartient pas en propre ; aussi est-ce « sacré » et est-il malaisé de se dépouiller du « saint émoi » que cela nous inspire.\par
On entend beaucoup vanter aujourd’hui le « sérieux », la « gravité dans les sujets et les affaires de haute importance », la « gravité allemande », etc. Cette façon de prendre les choses au sérieux montre clairement combien déjà invétérées et graves sont devenues la folie et la possession. Car il n’y a rien de plus sérieux que le fou lorsqu’il se met à chevaucher sa chimère favorite ; devant son zèle il ne s’agit plus de plaisanter (Voyez les maisons de fous).
\paragraph[{§ 3. — La Hiérarchie.}]{§ 3. — \emph{La Hiérarchie}.}
\noindent Les réflexions historiques sur notre hérédité mongole que j’intercale ici sous forme de digression, n’ont aucune prétention à la profondeur ni à la solidité. Si je les présente au lecteur, c’est simplement parce qu’il me semble qu’elles peuvent contribuer à l’éclaircissement du reste.\par
L’histoire de l’humanité, qui tient à proprement parler tout entière dans l’histoire de la race caucasique, paraît avoir parcouru jusqu’à présent deux périodes ; à la première, durant laquelle nous eûmes à nous dépouiller de notre originelle nature \emph{nègre}, succéda la période \emph{mongole} (chinoise), à laquelle il faudra également mettre fin par la violence. La période nègre représente l’\emph{Antiquité}, les siècles de dépendance vis-à-vis des \emph{objets} (repas des poulets sacrés, vol des oiseaux, éternuement, tonnerre et éclairs, bruissement des arbres, etc.) ; la période mongole représente les siècles de dépendance vis-à-vis des \emph{pensées}, l’\emph{Ere chrétienne.} C’est à l’avenir que sont réservées ces paroles : « Je suis possesseur du monde des objets, et je suis possesseur du monde des pensées. »\par
 Il est impossible de faire grand cas de la valeur du \emph{moi,} tant que le dur diamant du \emph{non-moi} (que ce non-moi soit le dieu ou soit le monde) reste à un prix aussi exorbitant. Le non-moi est encore trop vert et trop dur pour qu’il soit possible au moi de l’entamer et de l’absorber. Les hommes, avec une activité extraordinaire d’ailleurs, ne font que ramper sur cet \emph{immuable}, c’est-à-dire sur cette \emph{substance}, tels des insectes sur un cadavre dont ils font servir les sucs à leur nourriture, sans pour cela le détruire. Cette activité de vermine est toute l’industrie des Mongols. Chez les Chinois, en effet, tout reste comme avant ; une révolution ne supprime rien d’« essentiel » ou de « substantiel », et ne fait que les rendre plus affairés autour de ce qui reste debout et qui porte le nom d’« antiquité », d’« aïeux », etc.\par
C’est pourquoi, dans la période mongole que nous traversons, tout changement n’a jamais été qu’une réforme, une amélioration, et jamais une destruction, un bouleversement, un anéantissement. La substance, l’objet, demeure. Toute notre industrie n’a été qu’activité de fourmis et sauts de puces, jongleries sur la corde tendue de l’Objectif, et corvées sous le bâton de garde-chiourme de l’immuable ou « Eternel ». Les Chinois sont bien le plus \emph{positif} des peuples, et cela parce qu’ils sont ensevelis sous les dogmes ; mais l’ère chrétienne non plus n’est pas sortie du \emph{positif}, c’est-à-dire de la « liberté restreinte », de la liberté « jusqu’à une certaine limite ». Aux degrés les plus élevés de la civilisation, cette activité est dite \emph{scientifique} et se traduit par un travail reposant sur une supposition fixe, une \emph{hypothèse} inébranlable.\par
La Moralité, sous sa première et sa plus inintelligible forme, se présente comme \emph{habitude}. Agir conformément aux mœurs et aux coutumes de son pays, c’est être moral. Aussi est-il plus facile au Chinois qu’à tout autre d’agir moralement et de parvenir à une pure et naturelle moralité : il n’a qu’à s’en tenir  aux vieilles coutumes, aux vieilles mœurs, et à haïr toute innovation comme un crime méritant la mort ; l’\emph{innovation} est en effet l’ennemie mortelle de \emph{l’habitude,} de la \emph{tradition} et de la \emph{routine}. Il est hors de doute que l’habitude cuirasse l’homme contre l’importunité des choses, et lui crée un monde spécial, le seul où il se sente chez lui, c’est-à-dire un \emph{ciel.} Qu’est-ce qu’un « ciel », en effet, sinon la patrie propre de l’homme, où plus rien d’étranger ne le sollicite et ne le domine, où aucune influence terrestre ne le rend plus étranger à lui-même, bref, où, purifié des souillures de la terre et vainqueur dans sa lutte contre le monde, il n’est plus obligé de \emph{renoncer} à rien. Le ciel est la fin du \emph{renoncement}, la \emph{libre jouissance.} L’homme ne s’y interdit plus rien, car rien ne lui est plus étranger ni hostile.\par
L’habitude est donc une « seconde nature » qui délie et délivre l’homme de sa nature primitive et le met à l’abri des hasards de cette nature.\par
Les traditions de la civilisation chinoise ont paré à toutes les éventualités ; tout est « prévu » ; quoi qu’il arrive, le Chinois sait toujours comment il doit se comporter, il n’a jamais besoin de prendre conseil des circonstances. Jamais un événement inattendu ne le précipite du ciel de son repos. Le Chinois qui a vécu dans la moralité et qui y est parfaitement acclimaté ne peut être ni surpris ni déconcerté ; en toute occasion il garde son sang-froid, c’est-à-dire le calme du cœur et de l’esprit, parce que son cœur et son esprit, grâce à la prévoyance des vieilles coutumes traditionnelles, ne peuvent en aucun cas être bouleversés ni troublés : l’improviste n’existe plus. C’est donc grâce à l’habitude que l’humanité gravit le premier échelon de l’échelle de la civilisation (ou de la culture) ; et, comme elle s’imagine qu’atteindre la civilisation sera atteindre en même temps le ciel ou royaume de la culture et de la seconde nature, elle gravit en réalité par l’habitude le premier échelon de l’échelle du ciel.\par
 Si les Mongols ont affirmé l’existence d’êtres spirituels et créé un ciel, un monde des Esprits, les Caucasiens d’autre part ont, pendant des milliers d’années, lutté contre ces êtres spirituels pour les pénétrer et les comprendre. Ils ne faisaient en cela que bâtir sur le terrain mongol. Ils bâtissaient non sur le sable mais dans les airs ; ils ont lutté contre la tradition mongole et assailli le ciel mongol, le « Thian ». Quand donc finiront-ils par l’anéantir ? Quand se ressaisiront-ils et redeviendront-ils enfin de \emph{véritables Caucasiens ?} Quand l’ « immortalité de l’âme », qui dans ces derniers temps crut s’affirmer encore plus solidement en se présentant comme « immortalité de l’Esprit », se transformera-t-elle enfin en \emph{mortalité de l’Esprit ?}\par
Grâce aux industrieux efforts de la race mongole, les hommes avaient construit un ciel, quand ceux de la race Caucasique, pour autant qu’un reste d’hérédité mongole leur laisse quelque souci du ciel, se donnèrent une tâche opposée, la tâche de monter à l’assaut de ce ciel de la moralité et de le conquérir. Renverser tout dogme pour en élever sur le terrain dévasté un nouveau — et un meilleur, détruire les mœurs pour mettre à leur place des mœurs nouvelles — et meilleures, c’est là toute leur œuvre. Mais cette œuvre est-elle réellement ce qu’elle se propose d’être, et atteint-elle vraiment son but ? Non : dans cette poursuite du « \emph{meilleur} » elle est entachée de « mongolisme » ; elle ne conquiert le ciel que pour en créer un nouveau, elle ne renverse une ancienne puissance que pour en légitimer une nouvelle, elle ne fait en somme qu’— \emph{améliorer.}\par
Et cependant, le but suprême vers lequel on marche et que chaque coude de la route fait perdre de vue, n’en demeure pas moins invariable ; c’est la destruction vraie et complète du ciel, de la tradition, etc., c’est, en un mot, la fin de l’homme assuré uniquement contre le monde, la fin de son \emph{isolement}, de sa solitaire  \emph{intériorité.} L’homme cherche dans le ciel de la civilisation à s’isoler du monde et à en briser la puissance hostile. Mais ce céleste isolement doit être à son tour brisé, et la véritable fin de la conquête du ciel est — la ruine et l’anéantissement du ciel. Le Caucasien qui \emph{améliore} et qui \emph{réforme} agit en Mongol, car il ne fait que rétablir ce qui était, c’est-à-dire un \emph{dogme}. un absolu, un ciel. Lui qui a voué au ciel une haine implacable, il édifie néanmoins chaque jour de nouveaux cieux : échafaudant ciel sur ciel, il ne fait que les écraser l’un sous l’autre ; le ciel des Juifs détruit celui des Grecs, celui des Chrétiens détruit celui des Juifs, celui des Protestants celui des Catholiques, etc.\par
Si ces Titans humains parviennent à affranchir leur sang caucasien de son hérédité mongole, ils enseveliront l’homme spirituel sous les cendres de son prodigieux monde spirituel, l’homme isolé sous son monde isolé, et tout ceux qui construisent un ciel, sous les ruines de ce ciel. Et le ciel c’est le \emph{royaume des Esprits}, le domaine de la \emph{liberté spirituelle.}\par
Le royaume des cieux, le royaume des Esprits et des fantômes, a trouvé la place qui lui convenait dans la philosophie spéculative. Il y est devenu royaume des pensées, des concepts et des idées : le ciel est peuplé d’idées et de pensées, et ce « royaume des Esprits » est la réalité même.\par
Vouloir affranchir l’\emph{Esprit} est du pur « mongolisme » ; liberté de l’esprit, du sentiment, de la morale, sont des libertés mongoles.\par
On prend le mot « moralité » pour synonyme d’activité spontanée, de libre disposition de soi-même. Pourtant il n’en est rien ; au contraire, si le Caucasien a fait preuve de quelque activité personnelle, ç’a été en dépit de la moralité qu’il tenait de ses attaches mongoles. Le ciel mongol ou tradition morale est resté une imprenable forteresse, et le Caucasien a fait preuve de moralité rien que par les assauts répétés  qu’il lui a livrés, car s’il n’avait plus eu aucun souci de la moralité, s’il n’avait pas vu en cette dernière son perpétuel et invincible ennemi, le rapport entre lui et la tradition, c’est-à-dire sa moralité, aurait disparu.\par
Le fait que ses impulsions naturelles sont encore morales est précisément ce qui lui reste de son hérédité mongole ; c’est un signe qu’il ne s’est pas encore ressaisi. Les impulsions « morales » correspondent exactement à la philosophie « religieuse et orthodoxe », à la monarchie « constitutionnelle », à l’Etat « chrétien », à la liberté « modérée », ou, pour employer une image, au Héros cloué sur son lit de douleur.\par
L’homme n’aura réellement vaincu le Chamanisme et le cortège de fantômes qu’il traîne à sa suite que lorsqu’il aura la force de rejeter non seulement la superstition, mais la foi, — non seulement la croyance aux esprits, mais la croyance à l’Esprit.\par
Celui qui croit aux revenants ne s’incline pas plus profondément devant « l’intervention d’un monde supérieur » que ne le fait celui qui croit à l’Esprit, et tous deux cherchent un monde spirituel derrière le monde sensible. En d’autres termes, ils engendrent un \emph{autre} monde et y croient ; cet autre monde, \emph{création de leur esprit}, est un monde spirituel : leurs sens ne perçoivent et ne connaissent rien de cet autre monde immatériel, leur esprit seul vit en lui. Lorsque l’on croit comme un Mongol à l’\emph{existence d’êtres spirituels}, on n’est pas loin de conclure que l’\emph{être réel} chez l’homme est son \emph{esprit}, et qu’on doit réserver tous ses soins à ce seul esprit, au « salut de l’âme ». On affirme ainsi la possibilité d’agir sur l’Esprit, ce qu’on appelle « influence morale ».\par
Il saute donc aux yeux que le « Mongolisme » représente la négation radicale des sens et le règne du non-sens et du contre-nature, et que le péché et le remords du péché ont été pendant des milliers d’années, un fléau mongol.\par
 Mais qui fera maintenant rentrer l’Esprit dans son \emph{néant ?} Celui qui prouva par l’Esprit que la nature aussi est vaine, bornée et périssable, celui-là seul peut prouver la vanité de l’Esprit. \emph{Je} le puis, et ceux d’entre vous le peuvent dont le \emph{Moi} ordonne et règne souverain ; celui qui le peut, c’est, en un mot, — \emph{l’Egoïste.}\par

\asterism

\noindent Devant ce qui est sacré, on perd tout sentiment de sa puissance et tout courage ; on se sent \emph{impuissant }et on \emph{s’humilie.} Rien cependant n’est par soi-même sacré ; moi seul je consacre : ce qui \emph{canonise}, c’est ma pensée, mon jugement, mes génuflexions, bref, ma conscience.\par
Est sacré ce qui est inaccessible à l’égoïste, soustrait à ses atteintes, hors de sa \emph{puissance}, c’est-à-dire au-dessus de \emph{lui ;} en un mot, sacrée est toute — \emph{affaire de conscience :} « Ce m’est une affaire de conscience » ne signifie rien d’autre que « je tiens cela pour sacré ».\par
Pour les petits enfants comme pour les animaux il n’est rien de sacré, car pour s’élever à des notions de ce genre, l’intelligence doit s’être assez développée pour être capable de distinctions telles que « bon et mauvais, permis et défendu, » etc. ; ce n’est qu’à ce degré de réflexion ou de compréhension, — degré auquel correspond précisément le point de vue de la Religion, — que la \emph{crainte} naturelle peut faire place à la \emph{vénération} (non naturelle celle-ci, parce qu’elle n’a de racines que dans la pensée) et à la « terreur sacrée ». Il faut pour cela que l’on tienne quelque chose d’extérieur à soi pour plus puissant, plus grand, plus autorisé, meilleur que soi ; en d’autres termes, il faut que l’on sente planer au-dessus de sa tête une puissance étrangère, et que non seulement on éprouve cette puissance, mais qu’on la reconnaisse formellement, qu’on l’accepte, qu’on s’y soumette, qu’on se livre à elle pieds et poings liés (résignation, humilité,  soumission, obéissance, etc.). Ici défilent comme autant de fantômes toute la collection des « vertus chrétiennes ».\par
Tout ce qui inspire le respect ou la vénération mérite d’être appelé sacré ; vous dites vous-mêmes que ce n’est pas sans une « \emph{sainte terreur} » que vous y touchez. Et c’est un frisson analogue que provoque chez vous le contraire du sacré (le gibet, le crime, etc.) parce que cela aussi récèle le même « quelque chose » d’inquiétant, d’étrange et d’\emph{étranger.}\par
« S’il n’y avait rien de sacré pour l’homme, la porte serait grande ouverte au caprice, à l’arbitraire et à une subjectivité illimitée ! » La crainte est bien un commencement, on peut bien se faire craindre de l’homme le plus grossier, et c’est là déjà une digue à opposer à son insolence. Mais au fond de toute crainte couve toujours la tentation de s’affranchir de l’objet de cette crainte par finesse, ruse, tromperie, etc. Il en est tout autrement de la Vénération : vénérer, ce n’est pas seulement redouter, c’est de plus honorer ; l’objet de la crainte devient une puissance intérieure à laquelle je ne puis plus me soustraire ; ce que j’honore me saisit, m’attache, me possède, le respect dont je le paie me met complètement en son pouvoir et ne me laisse plus aucune velléité de m’en affranchir ; j’y adhère avec toute l’énergie de la foi, — \emph{je crois}. L’objet de ma crainte et moi ne faisons qu’un : « ce n’est pas moi qui vis, mais ce que je respecte vit en moi ». De plus, l’esprit étant infini, rien pour lui ne peut avoir de fin, il reste forcément stationnaire : il redoute les décadences, les dissolutions, la vieillesse et la mort, il ne sait plus se défaire de son petit Jésus, son œil que l’éternel éblouit devient incapable de reconnaître la grandeur propre aux choses qui passent. L’objet de crainte devenu objet de culte est dorénavant inviolable. Le respect devient éternel, l’objet du respect devient dieu.\par
L’homme désormais ne crée plus, il \emph{apprend} (étudie,  examine, etc.), c’est-à-dire que toute son activité se concentre sur un objet immuable, dans lequel il s’enfonce sans retour sur lui-même. Cet objet, il arrivera à le connaître, à l’approfondir, à le démontrer, mais il ne peut et ne tentera point de l’analyser et de le détruire. « L’homme doit être religieux », c’est chose convenue : toute la question est de savoir comment on parviendra à être religieux, quel est le vrai sens de la religiosité, etc. Il en est tout autrement si on remet en question l’axiome lui-même, et si l’on en doute, au risque de devoir finalement le rejeter. La Moralité est aussi une de ces conceptions sacrées : « on doit être moral » ; comment être moral, quelle est la vraie façon de l’être, c’est tout ce qu’on doit se demander. On ne se risque pas à demander si par hasard la Moralité elle-même ne serait pas une illusion, un mirage : elle reste au-dessus de tout doute, immuable. Et ainsi on gravit, étage par étage, tous les degrés du temple, depuis le « saint » jusqu’au « saint des saints ».\par

\asterism

\noindent On range les hommes en deux classes : les cultivés et les non-cultivés, les \emph{civilisés} et les \emph{barbares.} Les premiers, en tant que méritant leur nom, s’occupaient de pensées, vivaient par l’Esprit, et comme pendant l’ère chrétienne, qui eut la pensée pour principe, ils étaient les maîtres, ils exigèrent de tous, envers les pensées reconnues par eux, la plus respectueuse soumission. Etat, Empereur, Eglise, Moralité, Ordre, etc., sont de ces pensées, de ces fantômes qui n’existent que pour l’Esprit.\par
Un être simplement vivant, un animal, s’inquiète d’eux aussi peu qu’un enfant. Mais les Barbares ne sont en réalité que des enfants, et celui qui ne songe qu’à pourvoir aux besoins de sa vie est indifférent à tous ces fantômes ; comme il est d’autre part sans  force contre eux, il finit par succomber à leur puissance et par être régi par des — pensées.\par
Tel est le sens de la Hiérarchie : \emph{La Hiérarchie est la domination de la pensée, la royauté de l’Esprit}.\par
Jusqu’à ce jour nous sommes restés hiérarchiques, opprimés par ceux qui s’appuient sur des pensées. Les pensées sont le sacré.\par
Mais à chaque instant le civilisé se heurte au barbare et le barbare se heurte au civilisé, et cela non seulement à l’occasion de la rencontre de deux hommes, mais chez un seul et même homme. Car nul docte n’est si docte qu’il ne prenne quelque plaisir aux choses, et en ce faisant il agit en barbare, et nul barbare n’est absolument sans pensée. C’est par Hégel qu’a été mise en lumière l’ardente aspiration de l’homme le plus cultivé, le plus intellectuel, vers les \emph{objets}, et son horreur pour toute « théorie creuse ». Aussi la réalité, le monde des objets, doit-elle correspondre complètement à la pensée, et nul concept ne doit-il être sans réalité. C’est ce qui a fait appeler objectif le système de Hégel, de préférence à toute autre doctrine, parce que la pensée et l’objet, l’idéal et le réel, y fêtaient leur réunion. Ce système n’est néanmoins que l’apothéose de la pensée, son ascension à l’empire suprême et universel ; c’est le triomphe de l’Esprit, et en même temps le triomphe de la \emph{Philosophie}. La philosophie ne peut s’élever plus haut, elle atteint le point culminant de sa course lorsqu’elle aboutit à la \emph{toute-puissance, l’omnipotence de l’Esprit}\footnote{ \noindent Rousseau, les Philanthropes et d’autres ont combattu la culture de l’esprit et de l’intelligence, mais ils oubliaient que cette intelligence est le fond de \emph{toute} âme chrétienne, et leur critique n’atteignait que l’excès de civilisation, les raffinements de la culture spirituelle.
 }.\par
Les hommes selon l’esprit se sont \emph{mis en tête} un but qui doit être réalisé. Ayant les \emph{notions} d’Amour, de Bien, etc., ils voudraient faire de ces concepts des  \emph{réalités ;} ils veulent, en effet, fonder sur terre un royaume de l’amour, dans lequel nul n’agira plus par intérêt égoïste, mais par « amour ». L’amour doit régner. Ce qu’ils se sont mis en tête n’a qu’un nom : c’est une — \emph{idée fixe}. « Leur cervelle est hantée », et le plus importun, le plus obstiné des fantômes qui y ont élu domicile est \emph{l’Homme.} Rappelez-vous le proverbe : « Le chemin de l’enfer est pavé de bonnes résolutions. » La résolution de réaliser complètement en soi l’Homme est un de ces excellents pavés du chemin de la perdition, et les fermes propos d’être bon, noble, charitable, etc., sortent de la même carrière.\par
Br. Bauer dit quelque part\footnote{ \noindent B{\scshape r}. B{\scshape auer}, « Denkwurdigkeiten », {\scshape vi}, p. 7.
 } : « Cette classe bourgeoise qui a pris dans l’histoire contemporaine une si redoutable importance n’est capable d’aucun sacrifice, d’aucun enthousiasme pour une idée, d’aucune élévation : elle ne s’attache qu’à ce qui intéresse sa médiocrité, c’est-à-dire qu’elle ne voit pas plus loin qu’elle-même ; si elle est victorieuse, ce n’est, en définitive, que grâce à sa masse, dont l’inertie a lassé les efforts de la passion, de l’enthousiasme et de la logique, et grâce à sa surface qui a absorbé une partie des idées nouvelles. » Et plus loin\footnote{ \noindent \emph{Id.}, p. 6.
 } : « Elle a accaparé pour elle seule le bénéfice des idées révolutionnaires auxquelles d’autres, désintéressés ou passionnés, s’étaient sacrifiés, et elle a changé l’esprit en argent. — Mais en vérité, avant de faire siennes ces idées, elle a commencé par les châtrer de ce qui en était l’extrême, mais aussi la stricte conséquence, de l’ardeur fanatique de destruction contre tout égoïsme. »\par
C’est entendu : ces gens-là sont incapables de dévouement et d’enthousiasme ; ils n’ont ni idéal ni logique ; ce sont, au sens vulgaire du mot, des  égoïstes ne songeant qu’à leurs intérêts, prosaïques, calculateurs, etc.\par
Qui donc « se sacrifie » ? Celui qui subordonne tout le reste à \emph{un} but, à \emph{une} volonté, à \emph{une} passion, etc. L’amant ne se sacrifie-t-il pas, lorsqu’il abandonne père et mère, brave tous les dangers et supporte toutes les privations pour atteindre son but ? Et que fait d’autre l’ambitieux qui sacrifie à son unique passion tout désir, tout souhait, toute joie ? Et l’avare, qui se prive de tout pour amasser un trésor ? Et l’ivrogne ? Tous, une unique passion les domine, et ils lui sacrifient toutes les autres.\par
Mais ces sacrifices les empêchent-ils d’être intéressés ? Ne sont-ils point des égoïstes ? S’ils n’ont qu’une seule passion maîtresse, ils ne cherchent pas moins à la satisfaire et à se satisfaire, ils n’y mettent même que plus d’ardeur : leur passion les absorbe. Tous leurs actes, tous leurs efforts sont égoïstes, mais d’un égoïsme non épanoui, unilatéral et borné : ils sont possédés.\par
« Ce ne sont là, dites-vous, que des passions mesquines, misérables, par lesquelles l’homme ne doit au contraire pas se laisser enchaîner. L’homme doit se dévouer à une grande cause, à une grande idée ! » Une « idée élevée », une « bonne cause », c’est par exemple la gloire de Dieu, pour laquelle d’innombrables victimes ont cherché et trouvé la mort ; c’est le Christianisme, qui. a trouvé des martyrs prêts au supplice ; c’est cette Eglise hors laquelle il n’est pas de salut, si avide d’hécatombes d’hérétiques ; c’est la Liberté et l’Egalité, dont les sanglantes guillotines furent les servantes.\par
Celui qui vit pour une grande idée, pour une bonne cause, pour une doctrine, un système, une mission sublime, ne doit se laisser effleurer par aucune convoitise terrestre, il doit dépouiller tout intérêt égoïste. Ceci nous amène à la notion du \emph{Sacerdoce}, qu’on pourrait encore, eu égard à son rôle pédagogique,  appeler la pionnerie (car un idéal n’est qu’un pion !)\par
La vocation du prêtre l’appelle à vivre exclusivement pour l’Idée, à n’agir qu’en vue de l’Idée, de la bonne cause ; aussi le peuple sent-il combien il sied mal aux gens du clergé de laisser percer un orgueil mondain, de n’être \emph{pas} insensible à la bonne chère, de se livrer à des plaisirs comme la danse et le jeu, bref de s’intéresser à ce qui n’est pas un « intérêt sacré ». On pourrait peut-être expliquer également ainsi la maigreur des traitements que reçoivent les professeurs : ils doivent se sentir amplement récompensés par la sainteté de leur mission, et faire fi des voluptés passées. Nous ne manquons pas de catalogues de ces idées sacrées dont l’homme doit regarder une ou plusieurs comme sa mission. Famille, Patrie, Science, etc., peuvent trouver en moi un serviteur fidèle à remplir ses devoirs. Nous nous heurtons ici à l’antique erreur du monde, qui n’a pas encore appris à se passer de sacerdoce ; vivre et agir \emph{pour une idée} est encore toujours pour l’homme une vocation, et c’est à la fidélité avec laquelle il s’y consacre que se mesure sa valeur \emph{humaine.}\par
Telle étant la domination des idées ou le sacerdoce, Robespierre, par exemple, Saint-Just, etc., étaient bien des prêtres ; c’étaient des inspirés, des enthousiastes, les instruments conséquents d’une idée, des champions de l’idéal. Saint-Just s’écrie dans un de ces discours : « Il y a quelque chose de terrible dans l’amour sacré de la patrie ; il est tellement exclusif, qu’il immole tout sans pitié, sans frayeur, sans respect humain à l’intérêt public ; il précipite Manlius, il immole ses affaires privées, il entraîne Régulus à Carthage, jette un Romain dans un abîme et met Marat au Panthéon, victime de son dévouement\footnote{ \noindent S{\scshape aint}-J{\scshape ust}, Rapport à la Convention, 11 germinal, an II.
 }. »\par
 Contre ces représentants d’intérêts idéaux ou sacrés se dresse l’innombrable multitude des intérêts profanes, « personnels ». Nulle idée, nulle doctrine, nulle cause sainte n’est si grande qu’elle ne doive jamais être vaincue ou modifiée par ces intérêts personnels. S’il arrive que ceux-ci sombrent momentanément aux heures de rage et de fanatisme, le « bon sens populaire » les ramène bientôt à la surface. La victoire des idées n’est complète que lorsqu’elles cessent d’être en contradiction avec les intérêts personnels, c’est-à-dire lorsqu’elles donnent satisfaction à l’égoïsme.\par
Le marchand de harengs saurs qui crie en ce moment sa marchandise sous ma fenêtre a un intérêt personnel à la bien vendre, et quand sa femme ou n’importe qui ferait des vœux pour la prospérité de son petit commerce, son intérêt n’en resterait pas moins tout personnel. Mais qu’un voleur lui dérobe sa corbeille, aussitôt va s’éveiller l’intérêt de plusieurs, du grand nombre, de toute la ville, de tout le pays, bref, de tous ceux qui abhorrent le vol ; la personne de notre marchand passe à l’arrière plan, et s’efface devant la catégorie de « volé » à laquelle s’attache l’intérêt public.\par
Mais, ici encore, tout se ramène en définitive à un intérêt personnel : si tous ceux qui compatissent à l’infortune du volé croient devoir applaudir au châtiment du voleur, c’est que si le vol restait impuni il pourrait se généraliser, et qu’eux-mêmes pourraient à leur tour en être victimes. Il est toutefois difficile d’admettre que beaucoup de gens puissent avoir conscience d’un tel calcul, et vous entendrez le plus souvent proclamer que le voleur est un « criminel ». Nous avons devant nous un jugement rendu, qui qualifie le vol une fois pour toutes, et le range dans la classe des « crimes ».\par
Le problème qui se pose maintenant est celui-ci : A supposer qu’un crime ne causât pas le moindre préjudice  ni à moi, ni à quiconque m’intéresse, je devrais néanmoins déployer tout mon zèle à le combattre. Pourquoi ? Parce que la \emph{Moralité} m’inspire, parce que je suis plein de l’\emph{idée} de moralité, et que je dois m’opposer à tout ce qui la blesse. C’est parce que le vol lui paraît \emph{a priori} abominable que Proudhon croit flétrir la propriété en disant que « la propriété c’est le vol ». Aux yeux des prêtres, le vol est toujours un \emph{crime}, ou tout au moins un délit.\par
Ici finit l’intérêt personnel. Cette personne déterminée qui a dérobé la corbeille du marchand, m’est, à moi personnellement, complètement indifférente ; ce qui m’intéresse, c’est uniquement le voleur, l’espèce dont cette personne est un exemplaire. Voleur et Homme sont dans mon esprit deux termes inconciliables, car on n’est pas vraiment Homme quand on est voleur ; en volant on avilit en soi l’\emph{Homme} ou « l’humanité ».\par
Nous sortons de l’intérêt personnel pour tomber dans la \emph{Philanthropie}. Celle-ci est généralement si mal comprise qu’on croit y voir un amour pour les hommes, pour chaque individu en particulier, alors qu’elle n’est que l’amour de \emph{l’Homme}, du concept abstrait et irréel, du fantôme. Ce n’est pas τούς ἀνθρώπους, les hommes, mais τὸν ἄνθρωπον, l’homme, que le philanthrope porte dans son cœur. Certes il compatit à l’infortune de l’individu, mais ce n’est que parce qu’il voudrait voir partout réalisé son idéal bien-aimé. Ne lui parlez pas de sollicitude pour moi, pour toi, pour nous : intérêt personnel que tout cela ; cela rentre dans le chapitre de l’ « amour mondain ». La Philanthropie est un amour céleste, spirituel — \emph{clérical}. Ce qu’il lui faut, c’est faire fleurir en nous \emph{l’Homme}, quand nous, pauvres diables, en devrions crever. C’est le même esprit clérical qui a dicté le célèbre \emph{Fiat justitia, pereat mundus :} Homme, Justice, sont des idées, des fantômes, à l’amour desquels tout doit être sacrifié ; — du reste, l’homme  de tous les sacrifices est, comme on sait, le prêtre.\par
Celui qui rêve de \emph{l’Homme} perd de vue les personnes à mesure que s’étend sa rêverie ; il nage en plein intérêt sacré, idéal. \emph{L’Homme} n’est pas une personne, mais un idéal, un fantôme.\par
On peut prêter à l’homme les attributs les plus divers ; s’il paraît que le premier, le plus essentiel de ses attributs est la piété, le sacerdoce religieux se lève ; semble t-il que c’est la moralité qui lui est avant tout nécessaire, le sacerdoce moral dresse la tête. Les esprits hiérarchiques de nos jours voudraient faire de tout une « Religion » ; nous avons déjà une « religion de la Liberté, » une « religion de l’Egalité », etc., et ils sont en train de faire une « cause sacrée » de toutes les idées ; nous entendrons un jour parler d’une religion de la Bourgeoisie, de la Politique, de la Publicité, de la Liberté de la presse, de la Cour d’assises, etc.\par
Cela dit, qu’est-ce donc que le « désintéressement » ? Être désintéressé, c’est n’avoir qu’un intérêt idéal, devant lequel s’efface toute considération de personne.\par
L’orgueil de l’homme pratique se révolte contre cette manière de voir. Mais depuis des milliers d’années on a si bien travaillé à le dompter, qu’il doit aujourd’hui courber sa tête rebelle et « adorer la puissance supérieure » : le Sacerdoce a vaincu. Lorsque l’égoïste mondain était parvenu à secouer le joug d’une puissance supérieure, comme par exemple la Loi de l’Ancien Testament, le Pape romain, etc., il s’en élevait immédiatement au-dessus de lui une autre, dix fois supérieure : la Foi prenait la place de la Loi, l’élévation de tous les laïques au sacerdoce remplaçait le clergé fermé, etc. C’est l’histoire du possédé qu’une demi-douzaine de diables harcelaient dès qu’il croyait en avoir chassé un.\par
Le passage de Br. Bauer que nous citions plus haut dénie à la classe bourgeoise tout idéalisme, etc. Il est  indubitable qu’elle a falsifié les conséquences idéales que Robespierre eût tirées de son principe. L’instinct de son intérêt l’a avertie que ces conséquences juraient avec ce qu’elle avait en vue, et que ce serait un jeu de dupes que de vouloir se plier aux déductions de la théorie. Devait-elle peut-être pousser le désintéressement jusqu’à abjurer tout ce qui avait été son but pour conduire au triomphe une rigide théorie ?\par
Cela fait merveilleusement l’affaire des prêtres, quand les gens prêtent l’oreille à leurs exhortations : « Abandonne tout et suis-moi ! » ou « Vends tout ce que tu possèdes et donnes-en l’argent aux pauvres, cela te vaudra un trésor dans le ciel ; viens et suis-moi ! » Quelques rares idéalistes écoutent cet appel, mais la plupart font comme Ananias et Saphira, se conduisent à moitié suivant l’Esprit ou la Religion, à moitié suivant le monde, et partagent leurs offrandes entre Dieu et Mammon.\par
Je ne blâme pas la Bourgeoisie de ne pas s’être laissé détourner de son but par Robespierre et d’avoir pris conseil de son égoïsme pour savoir jusqu’à quel point elle devait s’assimiler les idées révolutionnaires. Mais ceux que l’on pourrait blâmer (si toutefois il peut être question ici de blâmer quelqu’un ou quelque chose) ce sont ceux qui se laissent imposer comme leurs intérêts, les intérêts de la classe bourgeoise. Ne finiront-ils pas un jour par comprendre de quel côté est leur avantage ?\par
« Pour conquérir à sa cause les producteurs (Prolétaires), » dit Auguste Becker\footnote{ \noindent Volksphilosophie unserer Tage, p. 22.
 }, « il ne suffit pas d’une négation des notions traditionnelles du droit. Les gens s’inquiètent malheureusement assez peu de la victoire théorique d’une idée. Ce qu’il faut, c’est leur démontrer \emph{ad oculos} le bénéfice pratique que l’on peut retirer de cette victoire » ; et il ajoute \footnote{ \noindent \emph{Idem}. p. 32.
 } :  « Vous devez empoigner les gens par leurs intérêts réels, si vous voulez avoir prise sur eux. » Il nous montre de plus la sage immoralité qui se propage déjà chez nos paysans, parce qu’ils aiment mieux suivre leurs intérêts réels que de s’astreindre aux commandements de la Morale.\par
Les Pères de l’Eglise révolutionnaire, ses pédagogues, coupaient le cou \emph{aux} hommes pour servir l’Homme ; les laïques, les profanes de la Révolution n’avaient pas, en vérité, une plus grande horreur pour cette opération, mais ils. se souciaient moins des droits de l’homme et de l’humanité que de leurs propres droits.\par
Comment se fait-il donc que l’égoïsme de ceux qui confessent et qui consultent en tout temps leur intérêt personnel succombe fatalement devant un intérêt sacerdotal ou pédagogique ? Leur personne leur semble à eux-mêmes trop mince, trop insignifiante (ce qu’elle est donc en effet) pour oser prétendre à tout er pouvoir se réaliser entièrement. Ce qui prouve qu’il en est bien ainsi, c’est qu’ils se scindent eux-mêmes en deux personnes, une éternelle et une temporelle, et qu’ils ne favorisent jamais que l’une à l’exclusion de l’autre : le dimanche l’éternelle, le reste de la semaine la temporelle, la première dans la prière, la seconde au travail. Ils portent le prêtre en eux, et c’est pourquoi ils n’en sont jamais quittes : ils s’entendent intérieurement prêcher chaque dimanche.\par
Combien les hommes ont travaillé et médité pour arriver à concilier cette dualité de leur essence ! Ils ont entassé idée sur idée, principe sur principe, système sur système, et rien n’est parvenu à la longue à résoudre la contradiction que renferme l’homme « temporel » appelé « l’égoïste ». Cela ne prouve-t-il pas que toutes ces idées étaient impuissantes à embrasser ma volonté tout entière et à la satisfaire ? Elles étaient et me sont demeurées ennemies, bien que cette inimitié se soit longtemps dissimulée. — En  sera-t-il de même de l’\emph{individualité ?} N’est-elle, elle aussi, qu’un essai de conciliation ?\par
A quelque principe que je me sois adressé, à celui de la \emph{Raison}, par exemple, j’ai toujours été finalement obligé de le rejeter. Ou bien puis-je être perpétuellement raisonnable et régler en toutes choses ma vie sur la raison ? Je puis \emph{m efforcer} d’être raisonnable, je puis \emph{aimer} la raison, comme je puis aimer Dieu ou toute autre idée. Je, puis être philosophe, être l’amant de la sagesse comme je suis l’adorateur de Dieu. Mais l’objet de mon amour et de mes aspirations n’existe que dans mon esprit, dans mon imagination, dans ma pensée ; il est dans mon cœur, dans mon cerveau, il est en moi, comme mon cœur est en moi, mais il n’est pas moi et je ne suis pas lui.\par

\asterism

\noindent Ce qu’on entend sous le nom d’\emph{influence morale} est tout spécialement du ressort des esprits sacerdotaux.\par
L’influence morale commence où commence l’\emph{humiliation ;} elle n’est que cette humiliation même, sous laquelle l’orgueil, forcé de plier ou de rompre, fait place à la soumission. Lorsque je crie à quelqu’un de s’éloigner d’un rocher prêt à sauter, je n’exerce sur lui par cet avertissement aucune influence morale. Si je dis à l’enfant : « tu auras faim si tu ne veux pas manger de ce qui est sur la table », il n’y a là non plus rien qui ressemble à l’ « influence morale ». Mais si je lui dis : « Il faut prier, honorer père et mère, respecter le crucifix, dire la vérité, etc., car cela est humain, car tel est le devoir de l’homme, ou mieux encore la volonté de Dieu », j’aurai cette fois exercé sur lui une action morale. C’est grâce à cette pédagogie morale que l’homme se pénètre de la mission de l’homme, qu’il devient humble et obéissant et qu’il soumet sa volonté à une volonté étrangère qui lui est imposée comme la règle et la loi ; il  doit \emph{s’incliner} devant une \emph{supériorité :} Humiliation volontaire. « Celui qui s’abaisse sera élevé. »\par
Oui, oui, il est bon d’exhorter de bonne heure les enfants à la piété, à la dévotion, à l’honnêteté. L’homme bien élevé est celui auquel les bons principes ont été enseignés, inculqués, serinés et entonnés à force de coups ou de sermons.\par
Si cela vous fait sourire, aussitôt les Bons de s’écrier en se tordant les mains de désespoir : « Mais, pour l’amour de Dieu, si nous ne donnons pas de bons principes à nos enfants, ils se jetteront tout droit dans la gueule du péché, et ils deviendront de mauvais garnements ! » Doucement, prophètes de malheur ! « Mauvais », dans votre sens, certes ils le deviendront, mais votre sens est précisément un très mauvais sens. Les effrontés ne s’en laisseront plus imposer par vos bavardages et vos lamentations et ne sympathiseront plus avec toutes les absurdités qui vous font rêver et radoter de temps immémorial ; ils aboliront le droit de succession en refusant d’hériter des sottises que vous ont léguées vos pères, et ils extirperont le \emph{péché originel.} Si vous leur dites : « Incline-toi devant l’Être suprême ! » ils répondront : « S’il veut nous faire plier, qu’il vienne lui-même et qu’il le fasse, car nous ne nous inclinons pas de notre plein gré ! » Et si vous les menacez de sa colère et de ses châtiments, ce sera comme si vous les menaciez du loup-garou. Quand vous ne parviendrez plus à leur inculquer la peur des revenants, le règne des revenants touchera à sa fin, et les contes de nourrice ne trouveront plus de créance.\par
Mais ne sont-ce pas encore une fois les Libéraux qui insistent sur la bonne éducation et sur la nécessité d’améliorer l’instruction publique ? Comment d’ailleurs leur Libéralisme, leur « liberté dans les limites de la loi, » pourrait-elle se réaliser sans le secours de la discipline ? Si l’éducation telle qu’ils l’entendent ne repose pas précisément sur la crainte de  Dieu, elle n’en fait que plus énergiquement appel au \emph{Respect humain}, c’est-à-dire à la crainte de \emph{l’Homme, }et c’est à la discipline à inspirer « l’enthousiasme pour la véritable mission humaine ».\par

\asterism

\noindent On se contenta pendant longtemps de l’illusion de posséder la \emph{vérité,} sans qu’il vînt à l’esprit de personne de se demander sérieusement s’il ne serait peut-être pas nécessaire, avant de posséder la vérité, d’être soi-même vrai. Ce temps fut le \emph{Moyen âge.} On se figura pouvoir comprendre l’abstrait, l’immatériel, au moyen de la conscience commune, de cette conscience qui n’a de prise que sur les objets, c’est-à-dire sur le sensible et le matériel. De même qu’on doit longuement exercer son œil avant d’arriver à saisir la perspective des objets éloignés, et qu’il faut que la main fasse de pénibles efforts avant que les doigts aient acquis la dextérité nécessaire pour frapper les touches selon les règles de l’art, de même on s’est soumis aux mortifications les plus variées afin de devenir capable d’embrasser entièrement le suprasensible. Mais ce qu’on mortifiait n’était rien d’autre que l’homme matériel, la conscience commune, l’intelligence restreinte à la perception des rapports sensibles. Et comme cette intelligence, cette pensée, dont Luther « faisait fi » sous le nom de raison, est inapte à concevoir le divin, le régime de \emph{mortifications} auquel on la soumit ne contribua en rien à la découverte de la vérité ; autant eût valu exercer ses pieds à la danse pendant des années dans l’espoir de leur apprendre à jouer de la flûte.\par
Luther, avec qui finit ce qu’on nomme le Moyen âge, fut le premier à comprendre que si l’homme veut embrasser la vérité, il doit commencer par devenir autre qu’il n’est, et par devenir aussi vrai que la vérité. Celui qui possède déjà la vérité parmi ses croyances, celui qui \emph{croit} à la vérité peut seul y avoir part, c’est-à-dire qu’elle n’est accessible qu’au croyant,  et que le croyant seul peut en explorer les profondeurs. Il n’y a que l’organe de l’homme capable de produire le souffle qui puisse aussi parvenir à jouer de la flûte, et il n’y a que l’homme possédant le véritable organe de la vérité qui puisse participer à la vérité. Celui dont la pensée n’atteint que le sensible, le positif, le concret, ne saisira non plus dans la vérité que son apparence concrète ; or, la vérité est esprit, fondamentalement immatérielle et n’est par conséquent du ressort que de la « conscience supérieure », et non de celle qui « n’est ouverte qu’aux choses de la terre ».\par
Luther met donc en lumière ce principe que la Vérité, étant \emph{pensée}, n’existe que pour l’homme \emph{pensant. }Et cela revient à dire que l’homme doit simplement se placer, désormais, à un point de vue différent, au point de vue céleste, croyant, scientifique, au point de vue du \emph{penser} en face de son objet, la \emph{pensée}, ou de l’Esprit en face de l’Esprit. L’égal seul reconnaît l’égal. « Tu es l’égal de l’Esprit que tu comprends\footnote{ \noindent \emph{Faust, I} \emph{N. d. Tr.}
 } ».\par
Le Protestantisme ayant abattu la hiérarchie du Moyen âge, cette opinion put s’enraciner que toute hiérarchie, la hiérarchie en général, avait été par lui détruite, et on put ne pas s’apercevoir qu’il avait été justement une « Réforme », c’est-à-dire la remise à neuf de la hiérarchie vieillie. Cette hiérarchie du Moyen-âge n’avait jamais été qu’infirme et débile, car elle avait été obligée de tolérer autour d’elle toute la barbarie des profanes ; il fallut la Réforme pour retremper les forces de la hiérarchie et lui donner toute son inflexible rigueur.\par
« La Réforme, dit Bruno Bauer, fut avant tout le divorce théorique du principe religieux avec l’Art, l’Etat et la Science, c’est-à-dire son affranchissement vis-à-vis de ces puissances auxquelles il avait  été intimement lié durant les premiers temps de l’Eglise et la hiérarchie du Moyen-âge ; et les institutions théologiques et religieuses issues de la Réforme ne sont que les conséquences logiques de cette séparation du principe religieux d’avec les autres puissances de l’humanité ». C’est précisément le contraire qui me paraît exact ; je pense que jamais la domination de l’esprit, ou — ce qui revient au même — la liberté de l’esprit, n’a été aussi étendue et aussi toute-puissante que depuis la Réforme, attendu que, loin de rompre avec l’Art, l’Etat et la Science, le principe religieux n’a fait que les pénétrer davantage, leur enlever ce qui leur restait de séculier, pour les amener au « royaume de l’esprit » et les rendre religieux.\par
On a, non sans raison, rapproché Luther de Descartes, et le « celui qui croit est un dieu » du « je pense donc je suis » \emph{(cogito ergo sum).} Le ciel de l’homme est le \emph{penser}, — l’Esprit. Tout peut lui être retranché, sauf la pensée, sauf la foi. On peut détruire une foi \emph{déterminée}, comme la foi en Zeus, Astarté, Jéhovah, Allah, etc., mais la foi elle-même est indestructible. Penser, c’est être libre. Ce dont j’ai besoin, ce dont j’ai faim, je ne l’attends plus d’aucune \emph{grâce,} ni de la Vierge Marie, ni de l’intercession des Saints, ni de l’Eglise qui lie et délie, mais je me le dispense moi-même. Bref, mon \emph{être} (le \emph{sum}) est une vie dans le ciel de la pensée, de l’Esprit, c’est un \emph{cogitare.} Moi-même je ne suis rien d’autre qu’Esprit : — Esprit pensant, dit Descartes, — Esprit croyant, dit Luther. Je ne suis pas ce qu’est mon corps ; ma chair peut être tourmentée de convoitises et de passions. Je ne suis pas ma chair, mais je suis Esprit, rien qu’Esprit.\par
Cette pensée traverse toute l’histoire de la Réforme jusqu’à nos jours.\par
Ce n’est que depuis Descartes que la philosophie moderne s’est appliquée sérieusement à tirer toutes  leurs conclusions des prémisses chrétiennes, en faisant de la « connaissance scientifique » la seule connaissance vraie et valable. C’est pourquoi elle commence par le \emph{doute} absolu, le \emph{dubitare}, par l’humiliation du savoir vulgaire, et la négation de tout ce qui n’est pas légitimé par l’esprit, par la pensée. Elle compte pour rien la \emph{Nature}, les opinions des hommes et le « consentement général », elle n’a point de repos tant qu’elle n’a pas mis en tout la Raison, et tant qu’elle ne peut pas dire : « Le réel est le rationnel et le rationnel seul est réel ». Elle est ainsi parvenue au triomphe de l’Esprit ou de la Raison, et tout est Esprit parce que tout est raisonnable : la Nature tout entière, aussi bien que les opinions des hommes, même les plus absurdes, renferment de la Raison, car « il faut tout faire servir à sa meilleure fin », c’est-à-dire au triomphe de la Raison.\par
Le \emph{dubitare} cartésien implique ce jugement que seul le \emph{cogitare,} le penser, l’Esprit — \emph{est.} C’est une rupture complète avec le sens « commun » qui accorde une réalité aux objets indépendamment de leurs rapports avec la raison ! Seuls l’Esprit, la pensée existent. Tel est le principe de la philosophie moderne, et c’est le principe chrétien dans toute sa pureté. Descartes séparait déjà nettement le corps de l’esprit, et « c’est l’esprit qui se bâtit un corps », dit Gœthe.\par
Mais cette philosophie elle-même, philosophie toute chrétienne, ne s’écarte pas du raisonnable ; aussi se tourne-t-elle contre le « pur subjectif », contre « les caprices, les hasards, l’arbitraire », etc. ; elle veut que le \emph{divin} devienne visible en tout, que toute connaissance soit une reconnaissance de Dieu, et que [{\corr l’homme}] contemple Dieu partout ; mais il n’y a jamais de dieu sans son diable.\par
On ne donne pas le titre de philosophe à celui qui, les yeux large ouverts aux choses du monde et le regard clair et assuré, porte sur le monde un jugement droit, s’il ne voit dans le monde que tout juste  le monde, dans les objets que les seuls objets, bref, s’il voit prosaïquement tout comme il est. Celui-là seul est un philosophe qui voit, montre et démontre dans le monde le ciel, dans le terrestre le supraterrestre, et dans l’humain le \emph{divin.} « Ce que ne voit pas l’intelligence des intelligents, dans sa simplesse une âme d’enfant le voit », et c’est cette âme d’enfant, cet œil pour le divin, qui fait avant tout le philosophe. Les autres n’ont qu’un sens « commun » ; lui, qui voit et sait exprimer le divin, a une conscience « scientifique ». C’est pour cette raison qu’on a exclu Bacon du royaume des philosophes, et tout ce qu’on nomme philosophie anglaise ne paraît d’ailleurs pas avoir dépassé, dans la suite, ce qu’avaient découvert ces « cerveaux lucides », qu’étaient Bacon et Hume. Les Anglais n’ont pas su magnifier l’ingénuité de l’âme des enfants et l’élever à la signification de philosophie ; ils n’ont pas su faire, avec des âmes d’enfants, — des philosophes. Cela revient à dire que leur philosophie fut incapable de devenir une philosophie \emph{théologique}, une \emph{théologie ;} et cependant ce n’est que comme théologie que la philosophie peut atteindre au terme de son évolution. C’est sur le champ de bataille de la théologie qu’elle rendra le dernier soupir. Bacon ne s’est pas plus mis martel en tête pour les questions théologiques que pour les points cardinaux.\par
L’objet de la connaissance est la vie. La pensée allemande, plus que toute autre, cherche à atteindre les commencements et les sources de la vie, et ne voit la vie que dans la connaissance elle-même. Le \emph{cogito, ergo sum} de Descartes signifie : on ne vit que si on pense. Vie pensante signifie « vie spirituelle ». L’Esprit seul vit, sa vie est la véritable vie. De même pour la Nature : ses « lois éternelles », l’Esprit ou la raison de la Nature, en sont toute la véritable vie. Dans l’homme comme dans la Nature, seule la pensée vit, tout le reste est mort. L’histoire de l’Esprit aboutit nécessairement à cette abstraction,  à la vie des généralités abstraites ou du \emph{non-vivant}. Dieu, qui est Esprit, est seul vivant : rien ne vit que le fantôme.\par
Comment peut-on soutenir que la philosophie moderne et l’époque moderne sont parvenues à la liberté, puisqu’elles ne nous délivrent pas du joug de l’\emph{objectivité ?} Est-ce que par hasard je serais affranchi d’un despote, lorsqu’au lieu de le redouter personnellement je me mets à redouter toute atteinte à la vénération que je m’imagine lui devoir ? C’est pourtant là que nous en sommes actuellement. La pensée moderne n’a fait que transformer les objets \emph{existants}, le despote réel, etc., en objets \emph{imaginaires}, c’est-à-dire en \emph{idées}. Et que devient l’ancien respect, vis-à-vis de ces idées ? Disparaît-il ? Au contraire, il ne fait que redoubler de ferveur. On s’est moqué de Dieu et du Diable, sous leur forme épaisse et vulgairement réelle d’autrefois, mais ce n’a été que pour prendre d’autant plus au sérieux leur notion abstraite. « Affranchi du Méchant, on a gardé le mal ».\par
On ne se fit aucun scrupule de se révolter contre l’état de choses existant et de renverser les lois régnantes, lorsqu’on eut pris une fois pour toutes la résolution de ne plus s’en laisser imposer par l’actuel et le palpable ; mais qui se serait permis de pécher contre l’\emph{idée} de l’Etat, et de ne pas se soumettre à l’\emph{idée} de la Loi ? Et l’on resta « citoyen », on resta homme « légal », loyal ; on se crut même d’autant plus « légal » qu’on abolissait plus rationalistement les vieilles lois boiteuses pour rendre hommage à l’ « esprit de la Loi ». En somme, les objets n’avaient fait que se transformer, sans rien perdre de leur puissance et de leur souveraineté, et l’on resta plongé dans l’obéissance, on resta possédé ; on vécut dans la \emph{réflexion}, il y eut toujours un Objet auquel on réfléchit, que l’on respecta, et devant lequel on se sentit plein de vénération et de crainte. On n’avait fait que transmuer les \emph{choses} en \emph{images} ou en représentations des  choses, en idées, en concepts, et on ne leur fut que plus intimement et indissolublement \emph{lié.}\par
Il n’est pas difficile, par exemple, de se soustraire aux ordres des parents, de fermer l’oreille aux conseils des oncles et des tantes et aux prières des frères et des sœurs, mais l’obéissance ainsi congédiée se réfugie dans la conscience ; moins on se plie aux exigences des siens, parce que rationnellement et au nom de sa propre raison on les juge déraisonnables, plus scrupuleusement on s’attache, en revanche, à la piété filiale, à l’amour de la Famille : on ne se pardonnerait plus d’offenser l’\emph{idée} qu’on s’est faite de l’amour familial et des devoirs qu’il impose. Affranchis de notre dépendance envers la famille existante, nous tombons sous la dépendance plus assujettissante de l’idée de la famille : l’esprit de famille s’empare de nous et nous domine. La famille composée de Hans, de Grete, etc., dont l’autorité est devenue impuissante, ne fait que se transposer en nous, s’intérioriser, si l’on veut ; elle reste toujours la « Famille », mais on lui applique le vieux précepte : « il vaut mieux obéir à Dieu qu’aux hommes », précepte qui, dans le cas présent, se traduit ainsi : Je ne puis, en vérité, me plier à [{\corr vos}] absurdes exigences, mais vous êtes ma « famille » et comme tels vous restez malgré tout l’objet de mon amour et de ma sollicitude, car la « famille » est une notion sacrée que l’individu ne peut offenser. — Et cette famille, ainsi rendue intérieure et immatérielle, devenue pensée et représentation, passe au rang de chose « sacro-sainte » ; son despotisme en est centuplé, car c’est ma conscience qu’elle va, désormais, remplir de ses clameurs. Pour que le despotisme de la famille fût vraiment brisé, il faudrait que cette famille idéale elle-même, devînt d’abord un \emph{néant}. Les phrases chrétiennes « Femme, qu’ai-je à faire avec toi\footnote{ \noindent Jean, {\scshape ii}, 4.
 } » ? — « Je suis venu  pour soulever l’homme contre son père et la fille contre sa mère\footnote{ \noindent Mathieu, {\scshape x}, 35.
 } », et d’autres semblables, doivent s’entendre comme un appel à la famille céleste, à la vraie famille. L’Etat ne dit pas autre chose, lorsqu’il exige qu’en tout conflit entre la famille et lui on obéisse à \emph{ses} ordres, à lui Etat.\par
Il en est de la Moralité comme de la Famille. Beaucoup ne se laissent plus arrêter par la morale, qui auraient grand’peine à se dégager du concept « Moralité ». La Moralité est l’\emph{idée} de la morale, sa force spirituelle, sa puissance sur les consciences ; la morale, au contraire, est trop matérielle pour dominer l’esprit, et ne peut enchaîner un homme « spirituel », un soi-disant « libre-penseur ».\par
Le Protestant a beau faire, la « Sainte Ecriture », la « parole de Dieu » lui reste sacrée. Celui pour qui elle n’est plus « sacrée » a cessé d’être un Protestant. Il doit, du même coup, tenir pour sacré tout ce qui est par elle « ordonné », l’autorité instituée par Dieu, etc. Tout cela reste pour lui inexpugnable, intangible, « au-dessus de toute espèce de doute » et par conséquent (le doute étant par excellence le propre de l’homme,) — « au dessus » de lui-même. Celui qui ne peut s’en \emph{détacher} y — \emph{croit}, car y croire signifie y être \emph{lié}. Du fait que par le Protestantisme la \emph{foi} est devenue plus intérieure, la \emph{servitude} également est devenue plus intérieure ; on a attiré à soi, on s’est approprié tout ce qu’il y avait de sainteté dans les objets, on en a imprégné ses pensées et ses actes, on s’est fait des \emph{cas de conscience} et on s’est tracé des \emph{devoirs sacrés}. Aussi, tout ce dont la conscience du Protestant ne peut s’affranchir lui est-il sacré, et le Protestant est-il \emph{consciencieux ;} c’est le trait le plus saillant de son caractère.\par
Le Protestantisme a proprement organisé en l’homme un véritable service de « police occulte ».  L’espion, le guetteur « Conscience », surveille chaque mouvement de l’esprit, et tout geste, toute pensée est à ses yeux une « affaire de conscience », c’est-à-dire une affaire de police. C’est cette scission de l’homme en « instincts naturels » et « conscience » (canaille intérieure et police intérieure) qui fait le Protestant. La « sagesse de la Bible », (au lieu de la catholique « sagesse de l’Eglise »), passe pour sacrée, et ce sentiment, cette conviction que la parole biblique est sainte se nomme — conscience. La sainteté a ainsi un trône dans le cœur de chacun. Si on ne se libère pas de la conscience, de l’idée du saint ou du sacré, on peut bien agir contre sa conscience, mais non indépendamment de la conscience ; on sera immoral mais non \emph{amoral}.\par
Le Catholique peut aller en paix, du moment qu’il a rempli les « commandements » ; le Protestant, lui, « fait de son mieux ». Le Catholique n’est qu’un \emph{laïque}, tandis que tout Protestant est lui-même un \emph{prêtre}. Cet \emph{ecclésiat} universel, cette ascension de tous à la prêtrise est le progrès réalisé par la Réforme sur le Moyen-âge, et sa malédiction.\par
Qu’était la morale jésuitique, sinon la continuation de la vente des indulgences, à cette différence près que celui qu’on renvoyait absous avait désormais en plus la faculté de \emph{contrôler} la remise de ses péchés, et pouvait s’assurer que ses fautes lui étaient réellement pardonnées, attendu que dans tel ou tel cas déterminé (casuistes), son péché n’en était pas un ? La vente des indulgences avait autorisé tous les péchés et tous les crimes et réduit au silence tous les murmures de la conscience. La sensualité pouvait se donner libre carrière, sauf à être achetée à l’Eglise. Les Jésuites continuèrent à encourager la sensualité et prévinrent ainsi la dépréciation de l’homme selon les sens, tandis que les Protestants, austères, sombres, fanatiques, repentants, contrits et priants, les Protestants, véritables. continuateurs du Christianisme, n’accordaient  de valeur qu’à l’homme selon l’esprit, au prêtre. Cette indulgence du Catholicisme, et spécialement des Jésuites, pour l’égoïsme, trouva au sein même du Protestantisme une involontaire et inconsciente adhésion, et nous sauva de la déchéance et de la ruine de la \emph{sensualité.} Toutefois, l’influence de l’esprit protestant ne cesse de s’étendre, et l’esprit jésuitique, qui, près de cet esprit « divin », représente le « diabolique » inséparable de toute divinité, ne parvient nulle part à se maintenir seul ; il est le témoin forcé, en France notamment, de la victoire du philistinisme protestant et de l’allégresse de l’Esprit triomphant.\par

\asterism

\noindent On a coutume de louer le Protestantisme de ce qu’il a remis en honneur le temporel, comme par exemple le mariage, l’Etat, etc. Mais en réalité le temporel en tant que temporel, le profane, lui est bien plus indifférent encore qu’au Catholicisme ; non seulement le catholique laisse subsister le monde profane, mais il ne s’interdit pas de goûter aux jouissances mondaines, tandis que le protestant, lorsqu’il raisonne et qu’il est conséquent, travaille à anéantir le temporel par le seul fait qu’il le \emph{sanctifie}. C’est ainsi que le mariage a perdu son ingénuité naturelle en devenant sacré — non pas sacré tel que le fait le Sacrement catholique, qui implique qu’il est en lui-même profane et ne reçoit que de l’Eglise sa consécration — mais sacré au sens protestant, sacré par essence, un lien sacré. Il en est de même de l’Etat : jadis le Pape consacrait l’Etat et ses princes en les bénissant ; aujourd’hui l’Etat, la Majesté sont par eux-mêmes sacrés sans qu’au préalable la main du prêtre ait dû s’étendre sur eux.\par
En somme, l’ordre de la Nature, ou Droit naturel, a été sanctifié sous le nom d’ « ordre divin ». La Confession d’Augsbourg, art. 11, dit par exemple :  « Tenons-nous-en simplement à la sage sentence des jurisconsultes : il est de droit naturel que l’homme et la femme vivent ensemble. Or, ce qui est \emph{un droit naturel est l’ordre de Dieu} transporté dans la nature, et est donc aussi un \emph{droit divin.} »\par
Et qu’est Feuerbach, sinon un protestant éclairé, lorsqu’il déclare sacrées toutes les relations morales, non point en vérité comme conformes à la volonté divine, mais en raison de l’\emph{Esprit} qui habite en elles ? « Le mariage — naturellement en tant qu’union libre dans l’amour, — est \emph{sacré par lui-même}, par sa nature même de contrat. Le mariage n’est \emph{religieux} que lorsqu’il est \emph{vrai} et répond à l’essence du mariage, qui est l’amour. Il en est de même pour toutes les relations du monde moral ; elles ne sont \emph{morales,} elles n’ont de valeur au point de vue de la moralité, que si elles sont \emph{par elles-mêmes religieuses.} Il n’y a de véritable amitié que là où les \emph{bornes} de l’amitié sont religieusement observées, avec autant de scrupules que le croyant en met à sauvegarder la dignité de son Dieu. \emph{Sacrés} sont et nous doivent être l’amitié, la propriété, le mariage, le bien de chaque homme, mais sacrés \emph{en eux-mêmes et par eux-mêmes}\footnote{ \noindent \emph{Wesen des Christentums}, p. 408.
 }. »\par
C’est là un point essentiel, sur lequel je veux insister. D’après le Catholicisme, le mondain, le séculier peut bien être \emph{consacré} ou \emph{sanctifié}, mais il n’est pas saint sans cette bénédiction sacerdotale ; d’après le Protestantisme, au contraire, le temporel est saint \emph{par lui-même}, du fait de sa seule existence.\par
A cette consécration ecclésiastique, source de toute sainteté, est intimement liée la maxime jésuitique : « la fin justifie les moyens ». Un moyen n’est en soi ni saint ni non-saint, mais appliqué aux besoins de l’Eglise, utile à l’Eglise, le voilà sanctifié. Le régicide, par exemple, est un de ces moyens : lorsqu’il  a été accompli pour le bien de l’Eglise il a toujours été sûr d’obtenir, parfois sans aveu public, sa canonisation. Pour le Protestant, la Majesté est sacrée ; pour le Catholique, elle ne peut l’être qu’après avoir reçu du pontife sa consécration, et si le Catholique la tient pour sacrée, c’est que la sainteté lui a été implicitement conférée une fois pour toutes par le Pape. Mais que le Pape vienne à retirer sa consécration, et le roi anathème ne sera plus pour ses sujets catholiques qu’un « homme du siècle », un « laïque », un « profane ».\par
Si le Protestant s’efforce de découvrir quelque sainteté dans tout ce qui touche aux sens, à la matière, pour ne plus s’attacher ensuite qu’à son côté sacré, le Catholique, lui, relègue le « matériel » dans un domaine à part où il conserve, comme tout le reste de la nature, sa valeur propre. L’Eglise catholique a jugé le mariage incompatible avec l’état ecclésiastique et a privé les membres du clergé des joies de la famille ; le mariage et la famille, même bénits, restent mondains. L’Eglise protestante, au contraire, tenant le mariage et les liens de la famille pour sacrés, fait abstraction de ce qu’ils ont de mondain et n’y voit rien qui ne puisse convenir à ses prêtres.\par
Un Jésuite, en sa qualité de bon catholique, peut tout sanctifier. Il lui suffit par exemple de se dire : je suis prêtre et comme tel nécessaire à l’Eglise. Mais je la servirai avec bien plus de zèle si je puis dûment assouvir mes passions ! Je vais donc séduire cette jeune fille, faire empoisonner mon ennemi, etc. Mon but est saint, étant celui d’un prêtre ; par conséquent il sanctifie le moyen. Je n’agis en somme que pour le bien de l’Eglise. Pourquoi le prêtre catholique craindrait-il de tendre à l’empereur Henri VII l’hostie empoisonnée — pour le salut de l’Eglise ?\par
Les Protestants vraiment selon le cœur de l’Eglise ont prohibé tous les « plaisirs innocents », parce que seul le sacré, le spirituel, pouvait être innocent.  Ils ont été obligés de condamner tout ce en quoi ils n’apercevaient pas le Saint-Esprit : danse, théâtre, luxe (dans l’église p. ex.) etc. C’est là le fait du Calvinisme puritain ; mais parallèlement à lui le Luthéranisme évolue dans un sens plus religieux, parce qu’il est d’un spiritualisme plus radical.\par
Le Calvinisme met en interdit une foule de choses qu’il considère à première vue comme sensuelles ou profanes ; il \emph{purifie} l’Eglise par exclusion. Le Luthéranisme, au contraire, ne rejette rien et cherche autant que possible à reconnaître en tout l’Esprit, l’opération du Saint-Esprit : il \emph{sanctifie} le profane. « Un baiser en tout bien tout honneur n’est pas chose défendue », l’esprit d’honnêteté le sanctifie. C’est ainsi que le luthérien Hégel (il déclare lui-même quelque part qu’il veut rester luthérien) en est venu à identifier complètement l’ordre naturel avec l’ordre logique. En tout est la Raison, c’est-à-dire le Saint-Esprit ; « le réel est rationnel », et le réel c’est, en fait, tout, attendu qu’en toute chose, par exemple dans chaque mensonge, on peut découvrir de la vérité : il n’y a pas de mensonge absolu, pas de mal absolu, etc.\par
Les Protestants presque seuls ont produit les grandes « œuvres de l’esprit », parce qu’eux seuls sont les vrais apôtres de l’\emph{Esprit.}\par

\asterism

\noindent Combien l’empire de l’homme est borné ! Il doit laisser le soleil poursuivre sa carrière, la mer soulever et abaisser ses flots, la montagne se dresser vers le ciel. Il est sans force devant l’\emph{Insurmontable.} Ce monde gigantesque est soumis à une \emph{loi} immuable à laquelle l’homme doit se soumettre et qui règle sa \emph{destinée ;} comment pourrait-il se défendre devant lui d’un sentiment d’\emph{impuissance ?}\par
Quel fut le but des efforts de l’humanité avant le  Christ ? Se garantir contre les coups du sort, et ne plus être à leur merci. Les Stoïciens y parvinrent par l’apathie, en considérant comme \emph{indifférents} les hasards de la nature et en ne se laissant pas affecter par eux. Horace, par son célèbre \emph{nil mirari}, proclame également son indifférence vis-à-vis de « l’Autre », du Monde, qui ne doit ni influer sur nous ni exciter notre étonnement. Et l’\emph{impavidum ferient ruinœ} du poète exprime précisément la même \emph{impassibilité} que le troisième verset du psaume XLV : « Nous ne craindrons pas, quand la terre sera renversée... etc. ». En tout cela est en germe l’aphorisme chrétien sur la vanité du monde, et aussi le chrétien \emph{mépris du monde}.\par
L’\emph{impassibilité} d’esprit du « sage », par laquelle le monde antique préparait sa ruine, reçut une secousse intérieure contre laquelle ni ataraxie ni stoïcisme ne purent la protéger. L’Esprit, soustrait à l’influence du monde, insensible à ses coups, \emph{élevé} au dessus de ses attaques, cet Esprit qui ne s’étonnait plus de rien et que l’écroulement du monde eût été incapable d’émouvoir, vint à déborder irrésistiblement, distendu par les « gaz » (\emph{spiritus,} gaz, vapeur) nés à son intérieur ; et lorsque les \emph{chocs mécaniques} venus du dehors furent devenus impuissants contre lui, les \emph{affinités chimiques} excitées en son sein entrèrent en jeu et commencèrent à exercer leur merveilleuse action.\par
L’histoire ancienne est virtuellement close le jour où Je parviens à faire du monde ma propriété. « Mon Père m’a mis toutes choses entre les mains\footnote{ \noindent Mathieu, {\scshape xi}, 27.
 }. » Le monde cesse de m’écraser de sa puissance, il n’est plus inaccessible, sacré, divin, etc., « \emph{les dieux sont morts} », et je traite si bien le monde selon mon bon plaisir qu’il ne tiendrait qu’à moi d’y opérer des miracles (qui sont des œuvres de l’Esprit) ; je pourrais renverser des montagnes, « ordonner à ce mûrier de se déraciner et de s’aller jeter dans la  mer\footnote{ \noindent Luc, {\scshape xvii}, 6.
 } », tout ce qui est \emph{pensable} est possible : « Toutes choses sont possibles à celui qui croit\footnote{ \noindent Marc, {\scshape ix}, 22.
 } ». Je suis le \emph{maître} du monde, la « \emph{majesté} » est à moi. Le monde est devenu \emph{prosaïque}, car le divin en a disparu : il est ma propriété, et j’en use comme il me plaît (savoir, comme il plait à l’esprit).\par
Par le fait que le Moi s’était élevé à ce titre de \emph{possesseur du monde,} l’Egoïsme avait remporté sa première victoire, et une victoire décisive : il avait vaincu le monde et l’avait « supprimé », et il confisqua à son profit l’œuvre d’une longue suite de siècles.\par
La première propriété, le premier « trône » est conquis.\par
Mais le maître du monde n’est pas encore maître de ses pensées, de ses sentiments et de sa volonté : il n’est pas le maître et le possesseur de l’Esprit, car l’Esprit est encore sacré, il est le Saint-Esprit. Le Chrétien qui a « nié le monde » ne peut pas « nier Dieu ».\par
L’Antiquité avait lutté contre le \emph{monde ;} le combat du Moyen-âge fut un combat contre \emph{soi-même}, contre l’Esprit. L’ennemi des Anciens avait été extérieur, celui des Chrétiens fut intérieur, et le champ de bataille où ils en vinrent aux mains fut l’intimité de leur pensée, de leur conscience.\par
Toute la sagesse des Anciens est \emph{Cosmologie,} science du monde ; toute la sagesse des Modernes est \emph{Théologie,} science de Dieu.\par
Les Païens (y compris les Juifs) avaient eu raison du \emph{monde ;} il s’agit dans la suite d’avoir aussi raison de soi-même, l’\emph{Esprit}, et de nier l’Esprit, c’est-à-dire de nier Dieu.\par
Pendant près de deux mille ans nous avons travaillé à nous asservir le Saint-Esprit, et nous avons petit à petit déchiré et foulé aux pieds maint lambeau  de la sainteté ; mais le formidable adversaire se relève toujours sous d’autres formes ou d’autres noms. L’Esprit n’a point encore cessé d’être divin, saint, sacré. Il y a longtemps en vérité qu’il ne plane plus au-dessus de nos têtes comme une colombe, il y a longtemps qu’il ne descend plus sur ses seuls élus : il se laisse saisir aussi par des laïques, etc ; mais en tant qu’Esprit de l’humanité, c’est-à-dire Esprit de l’Homme, il demeure pour toi comme pour moi un Esprit \emph{étranger}, bien loin d’être une \emph{propriété} dont nous puissions disposer selon notre bon plaisir.\par
Un fait est néanmoins certain, qui a visiblement dirigé la marche de l’histoire depuis Jésus-Christ : c’est la tendance à rendre le Saint-Esprit plus \emph{humain}, à le rapprocher des hommes ou à rapprocher les hommes de lui ; de là vint qu’il put être finalement conçu comme l’« Esprit de l’humanité », et qu’il nous parut d’un commerce plus facile et plus familier sous ses divers noms d’idée de l’humanité, genre humain, humanisme, philanthropie, etc.\par
Ne devrait-on pas penser que chacun peut aujourd’hui posséder le Saint-Esprit, interpréter l’idée d’humanité et réaliser en soi le genre humain ?\par
Non : l’Esprit n’a perdu ni sa sainteté ni son inviolabilité ; il ne nous est pas accessible et n’est pas notre propriété, car l’Esprit de l’humanité n’est pas \emph{mon} Esprit. Il peut être mon \emph{idéal}, et en tant que je le pense je l’appellerai mien : la \emph{pensée} de l’humanité est ma propriété, et je le prouve surabondamment par le seul fait que j’en fais ce qu’il me plaît, et lui donne aujourd’hui telle forme et demain telle autre. Nous nous représentons l’Esprit sous les aspects les plus divers, mais il est toutefois un fidéicommis que je ne puis ni aliéner ni supprimer.\par
A la longue et après de multiples avatars, le Saint-Esprit est devenu l’ « \emph{Idée absolue} », laquelle, à son tour, se divisant et se subdivisant, a donné naissance aux  diverses idées de philanthropie, de bon sens, de vertu civique, etc.\par
Mais puis-je nommer l’idée ma propriété, tant qu’elle est l’idée de l’humanité et puis-je considérer l’Esprit comme vaincu, tant que je dois le servir et « me sacrifier » pour lui ?\par
L’Antiquité à son déclin ne parvint à faire du monde sa propriété qu’après avoir brisé sa suprématie et sa « divinité » et s’être pénétré de son impuissance et de sa « vanité ». Ma position en face de l’Esprit est identique : si je puis le réduire à n’être plus qu’un \emph{fantôme}, et rabaisser la puissance qu’il exerce sur moi au rang de \emph{marotte}, il ne paraîtra plus ni saint, ni sacré, ni divin, et je \emph{me servirai de lui} au lieu de le servir, comme je me sers de la nature à ma guise et sans le moindre scrupule.\par
La « nature des choses », la « notion des rapports » doivent me guider : la nature des choses m’enseigne comment je dois me comporter envers elles, la notion des rapports m’apprend à en conclure.\par
Comme si l’« idée d’une chose » existait par elle-même et n’était pas plutôt l’idée qu’on se fait d’une chose ! Comme si le rapport que je conçois n’était pas unique, par le fait que moi qui le conçois je suis unique ! Qu’importe la rubrique sous laquelle les autres le rangent ? Mais de même qu’on a séparé « l’essence de l’homme » de l’homme réel, et qu’on juge celui-ci d’après celle-là, de même on a séparé l’homme réel de ses actes, auxquels on applique comme critérium la « dignité humaine ». Les \emph{Idées} doivent décider de tout, ce sont des idées qui gouvernent la vie, ce sont des Idées qui règnent. Tel est le monde religieux, auquel Hégel a donné une expression systématique, lorsque, mettant de la méthode dans l’absurdité, il assit sur les lois de la logique les fondations profondes de tout son édifice dogmatique. Les idées nous font la loi, et l’homme réel, c’est-à-dire Moi, je suis forcé de vivre d’après ces lois de la logique. Peut-il y avoir  une domination pire, et le Christianisme dès le début ne convint-il pas qu’il ne poursuivait d’autre but que rendre plus rigoureuse la domination de la loi judaïque ? (« Pas une lettre de la Loi ne doit être perdue. »)\par
Le Libéralisme n’a fait que mettre d’autres idées à l’ordre du jour ; il a remplacé le divin par l’humain, l’Eglise par l’Etat et le fidèle par le « savant », ou, en général, les dogmes bruts et les aphorismes surannés par des idées réelles et des lois éternelles.\par
Aujourd’hui, rien ne règne plus dans le monde que l’Esprit. Une innombrable foule d’idées bourdonnent en tous sens dans les têtes ; et que font ceux qui veulent avancer ? Ils nient ces idées pour en mettre d’autres à la place ! Ils disent : vous vous faites une fausse idée du Droit, de l’Etat, de l’Homme, de la Liberté, de la Vérité, de l’Honneur etc. ; l’idée qu’il faut se faire du Droit, etc. est bien plutôt celle-ci, que nous proposons. Ainsi la confusion des idées va croissant.\par
L’histoire du monde nous est cruelle, et l’Esprit a conquis une puissance souveraine. Tu dois respecter mes misérables souliers qui pourraient protéger ton pied nu, tu dois respecter mon sel, grâce auquel tes pommes de terre seraient moins fades, et mon superbe carrosse dont la possession te mettrait pour toujours à l’abri du besoin ; tu ne peux allonger la main vers eux. Toutes ces choses et d’innombrables autres sont \emph{indépendantes} de toi, et l’homme doit les reconnaître telles ; il doit les tenir pour intangibles et inaccessibles, les honorer, les respecter : malheur à lui s’il y porte la main, nous appelons cela « avoir les doigts crochus ».\par
Que nous reste-il ? Bien peu de chose, hélas, autant vaudrait dire rien ! Tout nous est enlevé, et nous ne pouvons tenter de rien obtenir de ce qui ne nous a pas été donné ; si nous vivons, ce n’est plus que par la \emph{clémence} du donateur qui nous a accordé cette \emph{grâce}. Il ne t’est pas même permis de ramasser une épingle si  tu n’as pas d’abord demandé la permission, et si tu n’y es \emph{autorisé.} Et autorisé par qui ? Par le \emph{Respect ! }Ce n’est que lorsqu’il t’aura accordé la propriété de cette épingle, lorsque tu pourras la \emph{respecter} comme une propriété, que tu pourras te baisser et la prendre. Bien plus, tu ne dois avoir aucune pensée, prononcer aucune syllabe, poser aucun acte qui aient en toi seul leur sanction, au lieu de la recevoir de la Moralité, de la Raison ou de l’Humanité.\par
Bienheureuse \emph{ingénuité} de l’homme qui ne connaît que ses appétits, avec quelle cruauté on a cherché à t’immoler sur l’autel de la \emph{Contrainte !}\par
Autour de l’autel se dresse une église, et cette église grandit, et ses murailles s’écartent chaque jour davantage. Ce que couvre l’ombre de ses voûtes est — \emph{sacré}, inaccessible à tes désirs, soustrait à tes atteintes. Le ventre creux, tu rôdes au pied de ces murailles, cherchant pour apaiser ta faim quelques restes de profane, et les cercles de ta course vont sans cesse s’élargissant. Bientôt cette église couvrira la terre entière, et tu seras refoulé à ses plus lointaines limites ; encore un pas, et le \emph{monde du sacré} a vaincu, tu t’enfonces dans l’abîme. Courage donc, paria, puisqu’il est temps encore ! Cesse d’errer, criant famine, à travers les champs fauchés du profane ; risque tout, et rue-toi à travers les portes au cœur même du sanctuaire ! Si tu \emph{consommes le sacré}, tu l’auras fait \emph{tien ! }Digère l’hostie, et tu en es quitte !
\subsubsection[{A.II.3. — Les Affranchis.}]{A.II.3. — Les Affranchis.}
\noindent Comme nous avons consacré deux chapitres distincts aux Anciens et aux Modernes, on pourrait juger convenable que nous en consacrions spécialement un aux Affranchis, comme aux représentants d’un troisième moment de l’évolution de la pensée humaine.  Mais il n’en est rien. Les Affranchis ne sont que des « Modernes », les plus modernes d’entre les modernes ; si nous leur faisons l’honneur d’une étude à part, c’est uniquement parce qu’ils sont le présent et que le présent mérite avant tout de fixer notre attention. J’emploie le nom d’Affranchis comme un synonyme de Libéraux, mais je dois remettre à plus tard l’examen de l’idée de Liberté comme de plusieurs autres d’ailleurs, auxquelles je ne pourrai non plus éviter de faire dès à présent allusion.\par
\paragraph[{A.II.3§1. Le Libéralisme politique.}]{A.II.3§1. Le Libéralisme politique.}
\noindent Au {\scshape xviii}\textsuperscript{e} siècle, lorsqu’on eût vidé jusqu’à la lie la coupe du pouvoir dit absolu, on s’aperçut trop nettement que le breuvage qu’elle offrait aux hommes ne pouvait être de leur goût, pour ne pas sentir le désir de boire à un autre verre.\par
Etant des « Hommes », nos pères voulurent être considérés comme des hommes. Quiconque voit en nous autre chose, nous le regardons comme étranger à l’humanité, inhumain ; pourquoi le traiterions-nous humainement ? Celui au contraire qui reconnaît en nous des hommes et nous garantit contre le danger d’être traités autrement que des hommes, nous l’honorons comme notre soutien et notre protecteur.\par
Unissons-nous donc, et soutenons-nous mutuellement ; notre \emph{association} nous assure la protection dont nous avons besoin, et nous, les \emph{associés}, formons une communauté dont les membres reconnaissent leur qualité d’hommes, et dont ce nom d’ « hommes » est le signe de ralliement. Le produit de notre association est l’\emph{Etat ;} nous, ses membres, nous formons la \emph{Nation}.\par
En tant que réunis dans la Nation ou l’Etat, nous ne sommes que des hommes. Qu’en outre, en tant  qu’individus nous fassions nos propres affaires et poursuivions nos intérêts personnels, peu importe à l’Etat ; cela concerne exclusivement notre \emph{vie privée ; }purement, uniquement humaine est notre \emph{vie publique} ou sociale. Ce qu’il y a en nous d’inhumain, d’ « égoïste », doit rester confiné dans le cercle inférieur des « affaires privées », et nous distinguons soigneusement l’Etat de la « société civile », domaine de l’ « égoïsme ».\par
Le véritable Homme, c’est la Nation ; l’individu, lui, est toujours un égoïste. Dépouillez donc cette individualité qui vous isole, cet individualisme qui ne souffle qu’inégalité égoïste et discorde, et consacrez-vous entièrement au véritable Homme, à la Nation, à l’Etat. Alors seulement vous acquerrez votre pleine valeur d’hommes et vous jouirez de ce qu’il appartient à l’Homme de posséder : l’Etat, qui est le véritable Homme, vous fera place à la table commune, et vous conférera les « droits de l’Homme », les droits que l’Homme seul donne et que seul l’Homme reçoit.\par
Tel est le principe civique.\par
Le civisme, c’est l’idée que l’Etat est tout, qu’il est l’Homme par excellence et que la valeur de l’individu comme homme dérive de sa qualité de citoyen. A ce point de vue, le mérite, suprême est d’être bon citoyen ; il n’est rien de supérieur, à moins que le vieil idéal — bon chrétien.\par
La bourgeoisie se développa au cours de la lutte contre les castes privilégiées, par lesquelles elle était, sous le nom de « tiers-état » cavalièrement traitée et confondue avec la « canaille ». Jusqu’alors avait prévalu dans l’Etat le principe de l’ « inégalité des personnes ». Le fils d’un noble était, de droit, appelé à remplir des charges auxquelles aspiraient en vain les bourgeois les plus instruits, etc. Le sentiment de la bourgeoisie se souleva contre cette situation : plus de prérogatives personnelles, plus de privilèges, plus  de hiérarchie déclassés ! Que tous soient égaux ! Aucun \emph{intérêt privé} ne peut entrer en ligne de compte avec l’\emph{intérêt général}. L’Etat doit être une réunion d’hommes libres et égaux, et chacun doit se consacrer au « bien public », se solidariser avec l’Etat, faire de l’Etat son but et son idéal. L’Etat ! L’Etat ! Tel fut le cri général, et dès lors on chercha à « bien organiser l’Etat » et l’on s’enquit de la meilleure constitution, c’est-à-dire de la meilleure forme à lui donner. La pensée de l’Etat pénétra dans tous les cœurs et y excita l’enthousiasme ; servir ce Dieu terrestre devint un culte nouveau. L’ère de la politique s’ouvrait. Servir l’Etat ou la Nation fut l’idéal suprême, l’intérêt public l’intérêt suprême, et jouer un rôle dans l’Etat (ce qui n’impliquait nullement que l’on fût fonctionnaire) le suprême honneur.\par
Par là, les intérêts privés, personnels, furent perdus de vue, et leur sacrifice sur l’autel de l’Etat devint un schibboleth. Il faut pour toute chose s’en remettre à l’Etat, et vivre pour lui ; l’activité doit être « désintéressée », n’avoir d’autre objectif que l’Etat. L’Etat devint ainsi la véritable Personne devant laquelle s’efface la personnalité de l’individu ; ce n’est pas \emph{moi} qui vis, c’est \emph{lui} qui vit en moi. D’où nécessité de bannir l’égoïsme d’autrefois et de devenir le désintéressement et l’\emph{impersonnalité} mêmes.\par
Devant l’Etat-Dieu, tout égoïsme disparaissait, tous se trouvaient égaux, tous étaient, sans que rien permît de les distinguer les uns des autres, des Hommes et rien que des Hommes.\par
La propriété fut l’étincelle qui mit le feu à la Révolution. Le gouvernement avait besoin d’argent. Il devait dès lors, pour être logique, montrer qu’il était \emph{absolu}, et par conséquent maître de toute propriété, en \emph{reprenant possession} de \emph{son} argent, dont les sujets avaient la jouissance, mais non la propriété. Au lieu de cela, il convoqua des Etats généraux, pour se  faire \emph{accorder} l’argent nécessaire. En n’osant pas être conséquent jusqu’au bout, on détruisit l’illusion du pouvoir \emph{absolu :} le gouvernement qui doit se faire « accorder » quelque chose ne saurait plus passer pour absolu. Les sujets s’aperçurent que les \emph{véritables propriétaires} étaient eux, et que c’était \emph{leur} argent qu’on exigeait d’eux.\par
Ceux qui n’avaient été jusque là que des sujets se réveillèrent \emph{propriétaires ;} c’est ce que Bailly exprime en peu de mots : « Vous ne pouvez sans mon consentement disposer de ma propriété, et vous disposeriez de ma personne, de tout ce qui constitue ma position morale et sociale ! Tout cela est ma propriété, au même titre que le champ que je cultive : c’est mon droit, c’est mon intérêt de faire moi-même les lois... »\par
Les paroles de Bailly semblent vouloir dire que \emph{chacun} est un propriétaire ; mais en réalité, au lieu du gouvernement, au lieu des princes, le possesseur et maître fut — la Nation. A partir de ce moment, l’idéal est « la liberté du peuple, un peuple libre », etc.\par
Dès le 8 juillet 1789, les explications de l’évêque d’Autun et de Barrères dissipèrent cette illusion que chacun, chaque volonté \emph{individuelle} a son importance dans la législation ; elles montrèrent la radicale impuissance des commettants : la \emph{majorité des représentants} fait la loi. Le 9 juillet, quand vient à l’ordre du jour le projet de loi sur la répartition des travaux de la constitution, Mirabeau fait remarquer que « le gouvernement dispose de la force, et non du droit, que c’est dans le \emph{Peuple} seul que doit être cherchée la source de tout \emph{droit} ». Le 16 juillet, le même Mirabeau s’écrie : « Le peuple n’est-il pas la source de toute \emph{puissance !} » Digne peuple ! source de tout droit et de toute puissance ! Soit dit en passant, on entrevoit ici le contenu du « droit » : c’est la force. « La raison du plus fort... »\par
La bourgeoisie est l’héritière des classes privilégiées. En fait, les droits des barons, qui leur furent enlevés  comme « usurpés », ne firent que retourner à la bourgeoisie, qui s’appelait à présent la « Nation ». Tous les \emph{privilèges} retombèrent « dans les mains de la Nation » ; aussi cessèrent-ils d’être des « privilèges » pour devenir des « droits ». Désormais c’est la Nation qui percevra les dîmes et les corvées ; c’est elle qui a hérité des droits seigneuriaux, du droit de chasse — et des serfs. La nuit du 4 août fut la nuit de mort des privilèges (les villes, les communes, les magistratures étaient privilégiées, dotées de privilèges et de droits seigneuriaux) et lorsqu’elle prit fin, se leva l’aube du Droit, des droits de l’Etat, des droits de la Nation.\par
Le despotisme n’avait été dans la main des rois qu’une règle complaisante et lâche, au prix de ce qu’en fit la « Nation souveraine ». Cette \emph{monarchie} nouvelle se révéla cent fois plus sévère, plus rigoureuse et plus conséquente que l’ancienne ; devant elle, plus de droits, plus de privilèges ; combien, en comparaison, paraît tempérée la « royauté absolue » de l’ancien régime ! La Révolution, en réalité, substitua à la \emph{monarchie tempérée} la véritable \emph{monarchie absolue}. Désormais, tout droit que ne concède pas le Monarque-Etat est une « usurpation », tout privilège qu’il accorde devient un « droit ». L’esprit du temps exigeait la \emph{royauté absolue}, et c’est ce qui causa la chute de ce qu’on avait appelé jusqu’alors royauté absolue, mais qui avait consenti à être si peu absolue qu’elle se laissait rogner et limiter par mille autorités subalternes.\par
La bourgeoisie a accompli le rêve de tant de siècles ; elle a découvert un maître absolu auprès duquel d’autres maîtres ne peuvent plus se dresser comme autant de restrictions. Elle a produit le maître qui seul accorde des « titres légitimes » et sans le consentement duquel \emph{rien n’est légitime}. « Nous savons que les idoles ne sont rien dans le monde, et qu’il n’y a d’autre dieu que le seul Dieu\footnote{ \noindent 1\textsuperscript{re} aux Corinthiens, {\scshape viii}, 4.
 } ».\par
 On ne peut plus attaquer \emph{le} Droit, comme on attaquait \emph{un} droit, en soutenant qu’il est « injuste ». tout ce qu’on peut désormais dire c’est qu’il est un non-sens, une illusion. Si on l’accusait d’être contraire au droit, on serait obligé de lui opposer un \emph{autre} droit, et de les comparer. Mais si l’on rejette totalement le Droit, le Droit en soi, on nie du même coup la possibilité de le violer, et on fait table rase de tout concept de Justice (et par conséquent d’injustice).\par
Nous jouissons tous de l’ « égalité des droits politiques ». Que signifie cela ? Simplement ceci, que l’Etat ne tolère nulle acception de personne, que je ne suis à ses yeux, comme le premier venu, qu’un homme, et n’ai aucun autre titre à son attention. Peu lui importe que je sois gentilhomme et fils de noble, peu lui importe que je sois l’héritier d’un homme en place dont la charge (comme au Moyen-âge les comtés, etc., et, plus tard, sous la royauté absolue, certaines fonctions sociales) me revient à titre héréditaire. Aujourd’hui l’Etat a une multitude de droits à conférer, tels, par exemple, le droit de commander un bataillon, une compagnie, le droit d’enseigner dans une université ; il lui appartient d’en disposer parce qu’ils sont à lui, que ce sont des droits de l’Etat, des droits « politiques ». Peu lui importe d’autre part à qui ils échoient, pourvu que le bénéficiaire remplisse les devoirs que lui impose sa fonction. Nous sommes à ce point de vue tous égaux devant lui, et nul n’a plus ou moins de droits qu’un autre (à une place vacante). Je n’ai pas à savoir, dit l’Etat-Souverain, qui exerce le commandement de l’armée, du moment que celui que j’investis de ce commandement possède les capacités nécessaires. « Egalité des droits politiques » signifie donc que chacun peut acquérir tous les droits que l’Etat a à distribuer, s’il remplit les conditions requises ; et ces conditions dépendent de la nature de l’emploi et ne  peuvent être dictées par des préférences personnelles \emph{(persona grata).} Le droit d’être officier, par exemple, exige, de par sa nature, qu’on possède des membres sains et certaines connaissances spéciales, mais ne pose pas comme condition qu’on soit d’origine noble ; si une carrière pouvait être fermée au citoyen le plus apte, ce serait l’inégalité, et la négation des droits politiques. De nos États modernes, les uns ont poussé plus loin, les autres moins loin l’application de ce principe d’égalité.\par
La « Castocratie » (je nomme ainsi la royauté absolue, le système des rois antérieurs à la Révolution) ne subordonne l’individu qu’à de petites monarchies, qui sont les confréries (corps) : corporations, noblesse, clergé, bourgeoisie, villes, communes, etc. Partout, l’individu devait avant tout se considérer comme membre de la petite société à laquelle il appartenait, et se plier sans réserve à son esprit, l’esprit du corps\footnote{ \noindent \emph{ En français dans le texte.} (N. d. Tr.)
 }, comme devant une autorité sans limites. Ainsi le noble devait regarder sa famille, l’honneur de sa race comme plus que lui-même. Ce n’est que par l’intermédiaire de sa corporation, de son « \emph{état} », que l’individu se rattachait à la corporation supérieure, à l’Etat, comme dans le catholicisme l’individu ne communique avec Dieu que par l’organe du prêtre.\par
C’est à cet état de choses que le Tiers-Etat mit fin, lorsqu’il prit sur lui de nier son existence en tant qu’\emph{état séparé ;} il résolut de ne plus être un état auprès d’autres états, mais de s’affirmer comme la « Nation ». Par là il instaura une Monarchie bien plus parfaite et plus absolue, et le \emph{principe des castes }jusqu’alors régnant, le principe des petites monarchies dans la grande s’écroula du même coup. Les castes et leur tyrannie renversées (et le roi n’était que roi des castes et non roi des citoyens), les individus  se trouvèrent affranchis de l’inégalité inhérente à la hiérarchie des corps sociaux. Mais les individus, ainsi sortis des castes et des cadres qui les enfermaient, n’étaient-ils réellement plus liés à aucun état \emph{(status)}, étaient-ils détachés du reste ? Non : si le Tiers s’était proclamé Nation, c’était précisément afin de ne plus être un état à côté d’autres états, mais pour devenir l’unique état, l’Etat national \emph{(status)}. Que devenait par là l’individu ? Un protestant politique, désormais en relations immédiates avec son Dieu, l’Etat. Il n’appartenait plus comme gentilhomme à la caste noble, ou comme artisan au corps des métiers ; il ne reconnaissait plus, comme tous les autres individus, \emph{qu’un seul et unique maître}, l’Etat, décernant à tous ceux qui le servaient le même titre de « citoyens ».\par
La Bourgeoisie est la \emph{noblesse du mérite :} « Au mérite sa couronne » est sa devise.\par
Elle lutta contre la noblesse « pourrie », car pour elle, laborieuse, anoblie par le labeur et le mérite, ce n’est pas « l’homme bien né » qui est libre, ni d’ailleurs \emph{moi} qui suis libre ; est libre « celui qui le mérite », \emph{le serviteur} intègre (de son Roi, de l’Etat, ou du Peuple dans nos Etats constitutionnels). C’est par les \emph{services} rendus que l’on acquiert la liberté, autrement dit le mérite — et ses profits, fût-ce d’ailleurs en servant Mammon. Il faut avoir bien mérité de l’Etat, c’est-à-dire du principe de l’Etat, de son esprit moral. Celui qui sert cet esprit de l’Etat, celui-là, de quelque branche de l’industrie qu’il vive, est un bon citoyen ; à ses yeux, les « novateurs » font « un triste métier ». Seul, le « boutiquier » est « pratique », et c’est le même esprit de trafic qui fait qu’on chasse aux emplois, qu’on cherche à faire fortune dans le commerce, et qu’on s’efforce de se rendre de n’importe quelle façon utile à soi et aux autres.\par
Si c’est le mérite de l’homme qui fait sa liberté (et  que manque-t-il à la liberté que réclame le cœur du bon bourgeois ou du fonctionnaire fidèle ?), servir c’est être libre. Le serviteur obéissant, voilà l’homme libre ! — Et voilà une rude absurdité ! Cependant tel est le sens intime de la bourgeoisie ; son poète, Goethe, comme son philosophe, Hégel, ont célébré la dépendance du sujet vis-à-vis de l’objet, la soumission au monde objectif, etc. Celui qui s’incline devant les événements et se découvre devant le fait accompli possède la vraie liberté. Et le fait, pour quiconque fait profession de penser, c’est — la \emph{Raison}, la Raison qui, comme l’Etat et l’Eglise, promulgue des lois générales, et fait communier les individus dans l’\emph{idée de l’Humanité.} Elle détermine ce qui est « vrai », et la règle sur laquelle on doit se guider. Pas de gens plus « raisonnables » que les loyaux serviteurs, et, avant tous, ceux qui, serviteurs de l’Etat, s’appellent bons citoyens et bons bourgeois.\par
Sois ce que tu pourras, un Crésus ou un gueux, — l’Etat bourgeois te laisse le choix, — mais aie de « bonnes opinions » ! ceci, l’Etat l’exige de toi, et il regarde comme son devoir le plus urgent de faire germer chez tous ces « bonnes opinions ». Dans ce but, il te protégera contre les « suggestions mauvaises » ; il tiendra en bride les « mal pensants », il étouffera leurs discours subversifs sous les sanctions de la censure et des lois sur la presse ou derrière les murs d’un cachot : d’autre part, il choisira comme censeurs des gens « d’opinions sûres », et il te soumettra à l’\emph{influence moralisatrice} de ceux qui sont « bien pensants et bien intentionnés ». Lorsqu’il t’aura rendu sourd aux mauvaises suggestions, il te rouvrira les oreilles toutes grandes aux bonnes.\par
Avec l’ère de la bourgeoisie s’ouvre celle du \emph{Libéralisme.} On veut instaurer partout le « raisonnable », l’ « opportun ». La définition suivante du Libéralisme, d’ailleurs toute en son honneur, le caractérise parfaitement : « Le Libéralisme est l’application du  bon sens aux circonstances, à mesure qu’elles se présentent ». Son idéal est « un ordre raisonnable », une « conduite morale », une « liberté modérée », et non l’anarchie, l’absence de lois, l’individualisme. Mais si la raison règne, la \emph{personne} succombe. L’Art ne fait point que tolérer le Laid ; longtemps il l’a revendiqué comme étant de son domaine, et en a fait un de ses ressorts : le monstre lui est nécessaire, etc. Les libéraux extrêmes vont tout aussi loin sur le terrain de la Religion, si loin même, qu’ils veulent voir considérer et traiter en citoyen l’homme le plus religieux, c’est-à-dire le monstre religieux ; ils ne veulent plus entendre parler de l’inquisition. Mais nul ne doit se révolter contre la « loi raisonnable », sous peine des plus sévères châtiments. Ce que veut le Libéralisme, c’est la libre évolution, la mise en valeur non point de la personne ou du moi, mais de la Raison ; c’est en un mot la dictature de la Raison, et, en somme, une dictature. Les Libéraux sont des \emph{apôtres,} non pas précisément les apôtres de la foi, de Dieu, etc., mais de la \emph{Raison}, leur évangile. Leur rationalisme ne laissant aucune latitude au caprice, exclut en conséquence toute spontanéité dans le développement et la réalisation du moi : leur \emph{tutelle} vaut celle des maîtres les plus absolus.\par
« Liberté politique ! » Que faut-il entendre par là ? Serait-ce l’indépendance de l’individu vis-à-vis de l’Etat et de ses lois ? Nullement ; c’est au contraire l’\emph{assujettissement} de l’individu à l’Etat et aux lois de l’Etat. — Pourquoi donc « liberté » ? Parce que nul intermédiaire ne s’interpose plus entre moi et l’Etat, mais que je suis directement en relation avec lui ; parce que je suis citoyen, et non plus sujet d’un autre, cet autre fût-il le roi : ce n’est pas devant la personne royale que je m’incline, mais devant sa qualité de « chef d’Etat ». La liberté politique, maxime fondamentale du Libéralisme, n’est qu’une seconde phase du — Protestantisme, et la « liberté religieuse » lui fait exactement pendant. En  effet\footnote{ \noindent L{\scshape ouis} B{\scshape lanc} dit en parlant de la Restauration, (Histoire de dix ans, 1, p. 138) : « Le protestantisme devint le fond des idées et des mœurs. »
 }, qu’implique cette dernière ? Affranchissement \emph{de} toute religion ? Evidemment non, mais uniquement affranchissement de toute personne interposée entre le ciel et vous. Suppression de l’intermédiaire du prêtre, abolition de l’opposition entre le « laïque » et le « clerc », et mise en communication directe et immédiate du fidèle avec la Religion ou le Dieu, tel est le sens de la liberté religieuse. On n’en peut jouir qu’à condition d’être religieux, et loin de signifier irréligion elle signifie intimité de la foi, commerce immédiat et cœur à cœur avec Dieu.\par
Pour l’ « affranchi religieux », la Religion est une affaire de cœur, c’est \emph{son affaire} et il s’y consacre avec une « sainte ferveur ». Il en est de même de l’ « affranchi politique » : il prend l’Etat à cœur, l’Etat est son affaire de cœur, la première de toutes ses affaires, celle qui entre toutes lui est propre.\par
Liberté politique et liberté religieuse sous entendent l’une que l’Etat, la πολις est libre, et l’autre que la Religion est libre, de même que liberté de conscience sous entend que la conscience est libre ; y voir ma liberté, mon indépendance vis-à-vis de l’Etat, de la Religion ou de la conscience serait un contre sens absolu. Il ne s’agit point ici de \emph{ma} liberté, mais de la liberté d’une force qui me gouverne et m’opprime ; ce sont mes\emph{tyrans} Etat, Religion ou conscience qui sont libres, et \emph{leur }liberté fait \emph{mon} esclavage. Il va de soi qu’ils mettent en pratique pour me réduire le proverbe « la fin justifie les moyens ». Si le bien de l’Etat est le but, le moyen d’y pourvoir, la guerre, se trouve sanctifié ; si la justice est le but de l’Etat, le meurtre comme moyen devient légitime et porte le nom sacré d’« exécution », etc. La \emph{sainteté} de l’Etat déteint sur tout ce qui lui est utile.\par
 La « liberté individuelle », sur laquelle le Libéralisme issu de 89 veille avec un soin jaloux n’implique nullement la parfaite et totale autonomie de l’individu, (autonomie grâce à laquelle tous mes actes seraient exclusivement miens), mais uniquement l’indépendance vis-à-vis des \emph{personnes.} Posséder la liberté individuelle, c’est n’être responsable envers aucun \emph{homme.} Si l’on comprend ainsi la liberté, — et on ne peut la comprendre autrement —, ce n’est pas seulement le maître qui est libre, c’est-à-dire \emph{irresponsable devant les hommes} (il se reconnaît responsable « devant Dieu »,) mais tous sont libres, parce qu’ils « ne sont responsables que devant la Loi ». C’est le mouvement révolutionnaire du siècle qui a conquis cette forme de la liberté, l’indépendance vis-à-vis de l’arbitraire, du « tel est notre plaisir ». En conséquence, il fallait que le prince constitutionnel dépouillât toute personnalité, toute volonté individuelle pour ne point léser, comme \emph{individu}, la « liberté individuelle » d’autrui. La \emph{regis voluntas} n’existe plus chez le prince constitutionnel, et c’est poussés par un sentiment très juste que tous les princes absolus se défendent contre cette mutilation. Notez que ce sont justement ces derniers qui se prétendent des « princes chrétiens », dans le vrai sens du mot ; mais il faudrait, pour en être, que chacun d’eux devint une puissance \emph{purement spirituelle}, attendu que le Chrétien n’est le sujet que de l’Esprit (« Dieu est Esprit »). Cette pure puissance spirituelle, c’est le prince constitutionnel qui seul la représente ; la perte de toute signification personnelle l’a si bien spiritualisé qu’on peut avec raison ne plus voir en lui qu’un « esprit », l’ombre fantomatique et inquiétante d’une \emph{Idée.} Le roi constitutionnel est le véritable roi \emph{chrétien}, la stricte conséquence du principe chrétien. La monarchie constitutionnelle a été le tombeau de la domination personnelle, c’est-à-dire du Maître capable de \emph{vouloir }réellement ; aussi y voyons-nous régner la \emph{liberté individuelle, } l’indépendance vis-à-vis de tout commandement émanant d’un individu et vis-à-vis de quiconque pourrait donner un ordre en disant « tel est notre plaisir\footnote{ \noindent \emph{En français dans le texte.} (N. d. Tr.)
 } ». La monarchie constitutionnelle est la réalisation parfaite de la vie sociale chrétienne, d’une vie spiritualisée.\par
La bourgeoisie est par toute sa conduite foncièrement libérale. Tout empiètement sur le domaine d’autrui lui est odieux. Dès que le bourgeois soupçonne qu’il dépend du caprice, du bon plaisir, de la volonté de quelqu’un que n’autorise pas une « puissance supérieure », il brandit son libéralisme et crie à l’ « arbitraire ». Aussi défend-il énergiquement sa liberté contre ce qu’on appelle \emph{décret} ou \emph{ordonnance\footnote{ \noindent \emph{Idem}.
 } :} « Je n’ai d’ordres à recevoir de personne ! » Une \emph{ordonnance} implique que mon devoir peut m’être tracé par la volonté d’un autre homme, et nous savons que la \emph{loi }s’oppose à toute suprématie personnelle. La liberté personnelle ou individuelle, l’indépendance du citoyen vis à-vis des individus ou des personnes, ne peut exister que si nul ne peut disposer de ce qui est mien, et tracer à son gré la limite entre ce qui m’est permis et ce qui m’est défendu.\par
La liberté de la presse est une des conquêtes du Libéralisme ; mais s’il combat la censure comme un instrument au service du bon plaisir gouvernemental, il n’éprouve cependant aucun [{\corr scrupule}] à exercer à son tour la tyrannie à l’aide de « lois sur la presse » ; d’où il appert que si les Libéraux tiennent à la liberté de la presse, c’est \emph{pour eux :} leurs écrits ne sortant pas de la légalité ne risquent pas de tomber sous le coup de la loi. Ce qui est libéral, c’est-à-dire \emph{légal, }peut seul être imprimé ; pour le reste, gare aux « délits de presse » !\par
Hé oui, la liberté de la presse est assurée, la liberté  personnelle est garantie, cela saute aux yeux, mais ce qu’on ne voit pas, c’est que la conséquence de toutes ces libertés est un criant esclavage. Fini des ordonnances, fini du bon plaisir et de l’arbitraire, « nous n’avons plus d’ordres à recevoir de personne ! » — et nous n’en sommes que plus étroitement asservis à la \emph{Loi.} Nous sommes les forçats du Droit.\par
Il n’y a plus dans l’Etat que des « gens libres », qu’oppriment mille contraintes (respects, convictions, etc.). Mais qu’importe ? Celui qui les écrase s’appelle l’Etat, la Loi, et jamais « un tel » ou « un tel ».\par
D’où vient l’hostilité acharnée de la bourgeoisie contre tout commandement personnel, c’est-à-dire n’émanant point des « faits », de la « raison » etc ? C’est qu’elle ne lutte que dans l’intérêt des « faits », contre la domination des « personnes » ! Mais l’intérêt de l’Esprit, c’est le raisonnable, le vertueux, le légal, etc. : c’est là « la bonne cause ». La bourgeoisie veut un maître \emph{impersonnel.}\par
Voici d’ailleurs le principe : l’intérêt des faits doit seul gouverner l’homme, notamment l’intérêt de la moralité, de la légalité, etc. Aussi nul ne peut-il être lésé dans ses intérêts par un autre (comme c’était le cas jadis, par exemple, lorsque les charges nobles étaient fermées aux bourgeois, les métiers fermés aux nobles, etc.). Que la \emph{Concurrence} soit \emph{libre}, et si quelqu’un est lésé ce ne pourra plus être que par un fait et non par une personne (le riche par exemple opprime le pauvre par l’argent, qui est un « fait »).\par
Il n’y a donc plus qu’un seul maître, l’autorité de l’Etat ; nul n’est personnellement le maître d’autrui. Dès sa naissance, l’enfant appartient à l’Etat ; ses parents ne sont que les représentants de ce dernier, et c’est lui, par exemple, qui ne tolère pas l’infanticide, qui vaque aux soins du baptême, etc.\par
Aux yeux paternels de l’Etat, tous ses enfants sont égaux (égalité civile ou politique), et libres d’aviser  aux moyens de l’emporter sur les autres : ils n’ont qu’à \emph{concourir.}\par
La libre concurrence n’est rien d’autre que le droit que possède chacun de prendre position contre les autres, de se faire valoir, de lutter. Le parti de la féodalité s’est naturellement défendu contre elle, son existence n’étant possible que par la non-concurrence. Les luttes de la Restauration en France n’avaient pas d’autre objet ; la bourgeoisie voulait la concurrence libre, l’aristocratie cherchait à restaurer le système corporatif et le monopole.\par
Aujourd’hui la concurrence est victorieuse, comme elle devait l’être, dans sa lutte contre le système corporatif (voir la suite plus loin).\par
La Révolution a abouti à une Réaction, et cela montre ce qu’était \emph{en réalité} la Révolution. Toute aspiration aboutit en effet à une réaction lorsqu’elle fait un retour sur elle-même, et commence à réfléchir ; elle ne pousse à l’action qu’aussi longtemps qu’elle est une ivresse, une « irréflexion ». « La réflexion », tel est le mot d’ordre de toute réaction, parce que la réflexion pose des bornes, et dégage du « déchaînement » et du « dérèglement » primitifs le but précis qu’on a poursuivi, c’est-à-dire le principe.\par
Les mauvais sujets, les étudiants tapageurs et mécréants qui bravent toutes les convenances ne sont à proprement parler que des « philistins » : comme ces derniers ils ont pour unique objectif les convenances. Les braver par fanfaronnade comme ils le font, c’est encore s’y conformer, c’est, si vous voulez, s’y conformer négativement ; devenus « philistins » ils s’y soumettront un jour et s’y conformeront positivement. Tous leurs actes, toutes leurs pensées aux uns comme aux autres visent à la « considération », mais le philistin est \emph{réactionnaire} comparé au garnement. L’un est un mauvais sujet rassis, venu à résipiscence, l’autre est un philistin en herbe. L’expérience journalière démontre  la vérité de cette remarque : les cheveux des pires mauvais sujets grisonnent sur des crânes de philistins.\par
Ce qu’on appelle en Allemagne la Réaction apparaît également comme le prolongement réfléchi de l’accès d’enthousiasme provoqué par la guerre pour la liberté.\par
La Révolution n’était pas [{\corr dirigée}] contre l’\emph{ordre} en général, mais contre l’ordre \emph{établi,} contre un état de choses \emph{déterminé.} Elle renversa \emph{un certain} gouvernement et non \emph{le} gouvernement ; les Français ont, au contraire, été depuis écrasés sous le plus inflexible des despotismes. La Révolution tua de vieux abus immoraux, pour établir solidement des usages moraux, c’est-à-dire qu’elle ne fit que mettre la vertu à la place du vice (vice et vertu diffèrent comme le mauvais sujet et le philistin). Jusqu’à ce jour, le principe révolutionnaire n’a pas changé : ne s’attaquer qu’à l’une ou l’autre institution déterminée, en un mot, \emph{réformer.} Plus on a \emph{amélioré}, plus la réflexion qui vient ensuite met de soins à \emph{conserver} le progrès réalisé. Toujours un nouveau maître est mis à la place de l’ancien, on ne démolit que pour reconstruire, et toute révolution est une — restauration. C’est toujours la différence entre le jeune et le vieux philistin. La Révolution a commencé en petite bourgeoise par [{\corr l’élévation}] du Tiers-Etat, de la classe moyenne, et elle monte en graine sans être sortie de son arrière-boutique.\par
Celui qui est libre, ce n’est pas l’homme en tant qu’\emph{individu} — et lui seul \emph{l}’homme —, mais c’est le \emph{bourgeois}, le « citoyen\footnote{ \noindent \emph{En français dans le texte.} (N. d. Tr.)
 } », l’homme \emph{politique,} lequel n’est pas un homme, mais un exemplaire de la race humaine et plus spécialement un exemplaire de l’espèce bourgeoise, un \emph{citoyen libre.}\par
Dans la Révolution, ce ne fut pas l’\emph{individu} qui agit et dont l’action eut une valeur historique, mais  un \emph{Peuple :} la Nation souveraine, voulut tout faire. C’est une entité artificielle, imaginaire, une Idée (la Nation n’est rien de plus) qui s’y révèle agissante ; les individus n’y sont que les instruments au service de cette idée et ne sortent pas du rôle de « citoyens ».\par
La Bourgeoisie tient sa puissance, et en même temps ses limites, de la « constitution de l’Etat », d’une charte, d’un prince « légitime » ou « légitimé » qui se dirige et qui gouverne d’après des « lois raisonnables », bref de la \emph{légalité}. La période bourgeoise est dominée par l’esprit de légalité, d’importation anglaise. Une réunion des états, par exemple, n’oublie jamais que ses droits ne sont pas illimités, qu’on lui a fait une grâce en la convoquant, et qu’une disgrâce peut la dissoudre. Elle ne perd jamais de vue le but de sa convocation, sa \emph{vocation.} On ne peut en vérité pas nier que mon père m’a engendré ; mais aujourd’hui que c’est chose faite, les intentions qu’il avait en procédant à cette opération ne me regardent plus, et quel que soit le but dans lequel il m’a \emph{appelé} à la vie, je fais ce qu’il me plaît. De même les Etats généraux convoqués au début de la Révolution française jugèrent très justement qu’une fois réunis ils étaient indépendants de celui qui les avait convoqués : ils \emph{existaient}, et ils eussent été bien bêtes de ne pas faire valoir leurs droits à l’existence, et de se croire à la merci de leur père.\par
Celui qui est convoqué n’a plus à se demander : « Que voulait-on de moi, en m’appelant ? » — mais bien : « Que veux-je, maintenant que je suis présent à l’appel ? » Ni l’auteur de la convocation, ni la charte en vertu de laquelle on l’a appelé, ni ses commettants, ni ses « cahiers », rien n’est pour lui une puissance sacrée, soustraite à ses atteintes. Il est \emph{autorisé }à tout ce qui est en son pouvoir, il ne reconnaîtra aucun « mandat » impératif ou restrictif, et ne prétendra pas être \emph{loyal} (rester dans la légalité). La conséquence de ceci serait — si toutefois on pouvait attendre d’un  Parlement rien de pareil, — de nous donner des Chambres parfaitement \emph{égoïstes,} dont le cordon ombilical moral serait coupé, et qui ne garderaient plus aucun ménagement. Mais les Chambres sont toujours dévotes, à la dévotion de quelqu’un ou de quelque chose ; comment s’étonner d’y voir toujours étaler tant de demi-égoïsme, d’égoïsme inavoué et hypocrite ?\par
Les membres des parlements ne peuvent franchir les \emph{limites} que leur tracent la charte, la volonté royale, etc. ; dépasser ces limites ou tenter de les dépasser, serait « empiéter ». Quel homme fidèle à ses devoirs oserait outrepasser sa mission en mettant en première ligne lui, ses convictions ou sa volonté ? Qui serait assez immoral pour \emph{se} faire valoir et pour imposer son individualité au risque de faire crouler le corps auquel il appartient et tout le reste avec lui ? On se tient respectueusement dans les limites de ses \emph{droits, }et il faut bien d’autre part qu’on reste dans les limites de ses \emph{forces}, personne ne pouvant plus qu’il ne peut. — Que ma force ou mon impuissance soit mon seul frein, et que mandats, missions, vocations ne soient que des dogmes qui m’entravent ? Qui pourrait souscrire à une doctrine d’un aussi audacieux nihilisme ? Pas moi, en tout cas : je suis un citoyen-légal !\par
La Bourgeoisie se reconnaît à ce qu’elle pratique une morale étroitement liée à son essence. Ce qu’elle exige avant tout, c’est qu’on ait une occupation sérieuse, une profession honorable, une conduite morale. Le chevalier d’industrie, la fille de joie, le voleur, le brigand et l’assassin, le joueur, le bohème sont immoraux, et le brave bourgeois éprouve à l’égard de ces « gens sans mœurs » la plus vive répulsion. Ce qui leur manque à tous, c’est cette espèce de droit de domicile dans la vie que donnent un commerce \emph{solide, }des moyens d’existence \emph{assurés,} des revenus \emph{stables}, etc. ; comme leur vie ne repose pas sur une \emph{base sûre,} ils appartiennent au clan des « individus » dangereux,  au dangereux \emph{prolétariat :} ce sont des « particuliers » qui n’offrent aucune « garantie », et n’ont « rien à perdre » et rien à risquer.\par
La famille ou le mariage, par exemple, lient l’homme, et ce lien le case dans la société et lui sert de garant ; — mais qui répond de la courtisane ? Le joueur risque tout son avoir sur une carte, il ruine lui et les autres : — pas de garantie !\par
On pourrait réunir sous le nom de « Vagabonds » tous ceux que le bourgeois tient pour suspects, hostiles et dangereux.\par
Tout vagabondage déplaît d’ailleurs au bourgeois, et il existe aussi des vagabonds de l’esprit, qui, étouffant sous le toit qui abritait leurs pères, s’en vont chercher au loin plus d’air et plus d’espace. Au lieu de rester au coin de l’âtre familial à remuer les cendres d’une opinion modérée, au lieu de tenir pour des vérités indiscutables ce qui a consolé et apaisé tant de générations avant eux, ils franchissent la barrière qui clôt le champ paternel, et s’en vont, par les chemins audacieux de la critique, où les mène leur indomptable curiosité de douter. Ces extravagants vagabonds rentrent, eux aussi, dans la classe des gens inquiets, instables et sans repos que sont les prolétaires, et quand ils laissent soupçonner leur manque de domicile moral, on les appelle des « brouillons », des « têtes chaudes » et des « exaltés ».\par
Tel est le sens étendu qu’il faut attacher à ces mots de Prolétariat et de Paupérisme. Combien on se tromperait, si l’on croyait la Bourgeoisie capable de désirer l’extinction de la misère (du paupérisme) et de consacrer à ce but tous ses efforts ! Rien au contraire ne réconforte le bon bourgeois comme cette conviction incomparablement consolante qu’ « un sage décret de la Providence a réparti une bonne fois et pour toujours les richesses et le bonheur ». La misère qui encombre les rues autour de lui ne trouble pas le vrai citoyen au point de le solliciter à  faire plus que de s’acquitter envers elle, comme en lui jetant l’aumône, ou en fournissant le travail et la pitance à quelque « brave garçon laborieux. » Mais il n’en sent que plus vivement combien sa paisible jouissance est troublée par les grondements de la misère \emph{remuante} et \emph{avide de changement}, par ces pauvres qui ne souffrent et ne peinent plus en silence mais qui commencent à s’agiter et à \emph{extravaguer}. Enfermez le vagabond ! jetez le perturbateur dans les plus sombres oubliettes ! « Il veut attiser les mécontentements et renverser l’ordre établi ! » Tuez ! Tuez !\par
Mais justement ces trouble-fête font à peu près le raisonnement suivant : Les « bons bourgeois » s’inquiètent peu qui les protège eux et leurs principes ; roi absolu, roi constitutionnel ou république leur sont bons pourvu qu’ils soient protégés. Et quel est leur principe, ce principe dont ils « aiment » toujours le protecteur ? Ce n’est pas le travail, ce n’est pas non plus la naissance ; mais c’est la \emph{médiocrité}, le juste milieu, un peu de travail et un peu de naissance, en deux mots, un \emph{capital} qui \emph{produit des intérêts.}\par
Le capital est ici le fonds, la mise, l’héritage (naissance) ; l’intérêt est la peine prise pour faire valoir (travail) : \emph{le capital travaille}. Mais pas d’excès, pas d’ultra, pas de radicalisme ! Evidemment, il faut que le nom, la naissance, puissent donner quelque avantage, mais ce ne peut être là qu’un capital, une mise de fonds ; évidemment il faut du travail, mais que ce travail soit peu ou point personnel, que ce soit le travail du capital — et des travailleurs asservis.\par
Lorsqu’une époque est plongée dans une erreur, toujours les uns bénéficient de cette erreur, tandis que les autres en pâtissent. Au moyen-âge, l’erreur universellement répandue parmi les Chrétiens était que l’Eglise, toute puissante, doit être sur terre la surintendante et la dispensatrice de tous biens. Les  ecclésiastiques admettaient cette « vérité » tout comme les laïques, la même erreur était également ancrée chez tous. Mais le \emph{bénéfice,} la puissance, était pour les prêtres, et le \emph{dommage}, l’asservissement, pour les laïques. « Le malheur, dit-on, rend sage » ; aussi les laïques assagis finirent-ils par ne plus admettre cette « vérité » du moyen-âge.\par
Il en est exactement de même pour la Bourgeoisie et le Prolétariat. Bourgeois et ouvriers croient à la « réalité » de l’argent ; ceux qui n’en ont pas sont aussi pénétrés de cette « réalité » que ceux qui en ont, les profanes que les clercs. « L’argent régit le monde » est la tonique de l’époque bourgeoise. Un gentilhomme sans le sou et un travailleur sans le sou sont des « meurt de faim » également sans valeur politique. La \emph{valeur} ne va pas sans \emph{les valeurs ;} l’argent seul la donne, naissance et travail n’y peuvent rien. Ceux qui possèdent gouvernent, mais l’Etat élit parmi les non-possédants ses « serviteurs » et leur distribue avec une sage économie quelques sommes (traitements, appointements) pour gouverner en son nom ; il en fait ses régisseurs.\par
Je reçois tout de l’Etat. Puis-je avoir quelque chose sans la permission de l’Etat ? Non, tout ce que je pourrais avoir ainsi, il me l’enlève dès qu’il s’aperçoit que les « titres de propriété » me font défaut. Tout ce que je possède, je le dois à sa clémence. C’est uniquement là-dessus, sur les \emph{titres}, que repose la bourgeoisie ; le bourgeois n’est ce qu’il est que grâce à la bienveillante protection de l’Etat. Il aurait tout à perdre si la puissance de l’Etat venait à s’effondrer.\par
Mais quelle est la situation de celui qui n’a rien à perdre dans cette banqueroute sociale, du Prolétaire ? Comme tout ce qu’il a et ce qu’il pourrait perdre se chiffre par zéro, il n’a pour ce zéro nul besoin de la protection de l’Etat. Il ne pourrait au contraire qu’y gagner si cette protection venait à manquer aux protégés.\par
 Aussi celui qui ne possède pas considère-t-il l’Etat comme une puissance tutélaire de ceux qui possèdent ; cet ange gardien des capitalistes est — un vampire qui lui suce le sang.\par
L’Etat est un \emph{Etat-Bourgeois}, c’est le \emph{status} de la Bourgeoisie. Il accorde sa protection à l’homme non en raison de son travail, mais en raison de sa docilité (loyalisme), suivant qu’il use des droits que l’Etat lui accorde en se conformant à la volonté, autrement dit aux lois de l’Etat.\par
Le régime bourgeois livre les travailleurs aux possesseurs, c’est-à-dire à ceux qui détiennent quelque bien de l’Etat (et toute fortune est un bien de l’Etat, appartient à l’Etat et n’est donnée qu’en fief à l’individu) et particulièrement à ceux entre les mains desquels est l’argent, aux capitalistes.\par
L’ouvrier ne peut tirer de son travail un prix en rapport avec la valeur qu’a le produit de ce travail pour celui qui le consomme. « Le travail est mal payé ! » Le plus gros bénéfice en va au capitaliste. — Mais bien payés, et plus que bien payés sont les travaux de ceux qui contribuent à relever l’éclat et la puissance de l’Etat, les travaux des hauts \emph{serviteurs} de l’Etat. L’Etat paie bien, pour que les « bons citoyens », les possesseurs, puissent impunément payer mal. Il s’assure, en les payant bien, la fidélité de ses serviteurs, et fait d’eux, pour la sauvegarde des bons citoyens, une « Police » (à la police appartiennent les soldats, les fonctionnaires de tout acabit, juges, pédagogues, etc., bref toute la « machine de l’Etat ».) Les « bons citoyens » de leur côté lui paient sans faire la grimace de gros impôts, afin de pouvoir payer d’autant plus misérablement les ouvriers à leur service.\par
Mais les ouvriers ne sont, en tant qu’ouvriers, pas protégés par l’Etat ; en tant que sujets de l’Etat, ils ont simplement la cojouissance de la « police », qui leur assure ce qu’on appelle une « garantie légale » ; aussi la classe des travailleurs reste-t-elle une puissance  hostile vis-à-vis de cet Etat, l’Etat des riches, le « royaume de la Bourgeoisie. » Leur principe, le travail, n’est pas estimé à sa valeur, mais exploité ; il est le \emph{butin de guerre} des riches, de l’ennemi.\par
Les ouvriers disposent d’une puissance formidable ; qu’ils parviennent à s’en rendre bien compte et se décident à en user, rien ne pourra leur résister : il suffirait qu’ils cessent tout travail et s’approprient tous les produits, ces produits de leur travail qu’ils s’apercevraient être à eux comme ils viennent d’eux. Tel est d’ailleurs le sens des émeutes ouvrières que nous voyons éclater un peu partout.\par
L’Etat est fondé sur — l’\emph{esclavage du travail}. Que le travail soit \emph{libre}, et l’Etat s’écroule.
\paragraph[{A.II.3§2. Le Libéralisme social.}]{A.II.3§2. Le Libéralisme social.}
\noindent Nous sommes des hommes, nous sommes nés libres, et de quelque côté que nous tournions les yeux nous nous voyons réduits en servitude par des égoïstes ! Devons-nous donc nous aussi devenir des égoïstes ? Le ciel nous en préserve ! Nous préférons rendre tout égoïsme impossible, et, pour cela, faire de tous des « gueux » ; si personne n’a rien, « tous » auront.\par
Ce sont des Socialistes qui parlent.\par
— Qui est cette personne que vous nommez « tous » ? — C’est la « Société » ! — A-t-elle donc un corps ? — Nous sommes son corps. — Vous ? allons donc ! vous n’êtes pas un corps ; toi, tu as un corps, et toi aussi, et ce troisième là-bas également ; mais vous tous ensemble vous êtes des corps, et non un corps. Par conséquent, la Société, en admettant que ce soit quelqu’un, aurait bien \emph{des} corps à son service, mais non pas \emph{un} corps unique, lui appartenant en propre. Comme la « Nation » des politiciens, elle n’est qu’un  Esprit, un fantôme, et son corps n’est qu’une apparence.\par
La liberté de l’homme est, pour le Libéralisme politique, la liberté vis-à-vis des \emph{personnes}, de la domination personnelle, du \emph{Maître ;} c’est la liberté personnelle, garantissant chaque individu contre les autres individus. Nul n’a le droit d’ordonner, seule la Loi ordonne. Mais si les personnes sont égales, ce qu’elles \emph{possèdent} n’est pas égal. Le pauvre a besoin du riche comme le riche du pauvre ; le premier a besoin de la richesse du second, et celui-ci au travail du premier ; si chacun a besoin de l’autre, ce n’est toutefois pas de cet autre comme \emph{personne}, mais comme \emph{fournisseur}, comme ayant quelque chose à donner, comme détenant ou possédant quelque chose. C’est donc ce qu’il a qui fait l’homme. Et, par leur \emph{avoir,} les hommes sont inégaux.\par
Le Socialisme conclut que nul ne doit \emph{posséder}, de même que le Libéralisme politique concluait que nul ne doit \emph{commander}. Si pour l’un l’\emph{Etat} seul commandait, pour l’autre la \emph{Société} seule possède.\par
Par là même qu’il protège contre les autres la personne et la propriété de chacun, l’Etat \emph{isole} les individus : ce que je \emph{suis} et ce que j’\emph{ai} ne regarde que moi. Celui qui se contente de ce qu’il est et de ce qu’il a n’essaie pas d’aller plus loin ; mais celui qui voudrait être et avoir plus cherche ce surplus et le trouve au pouvoir d’autres \emph{personnes.} Nous aboutissons à une contradiction que les Socialistes ne se font pas faute de relever : l’un n’est personnellement pas plus que l’autre, et cependant l’un a ce que l’autre n’a pas et désirerait avoir ; donc, l’un est personnellement plus que l’autre, puisque l’un possède ce qu’il lui faut et l’autre pas, puisque l’un est riche et l’autre pauvre.\par
Devons-nous donc, continuent les Socialistes, laisser ressusciter ce que nous avions enterré avec tant de raison, et devons-nous laisser restaurer par  un subterfuge cette inégalité des personnes que nous avons voulu abolir ? Non, il faut au contraire achever la besogne qui n’a été qu’à moitié faite. Il manque encore à notre liberté vis-à-vis des personnes la liberté vis-à-vis de ce qui leur permet d’opprimer celle d’autrui, de ce qui est le fondement de la puissance personnelle, c’est-à-dire la liberté vis-à-vis de la « propriété personnelle ». Supprimons donc la \emph{propriété personnelle.} Que nul ne possède plus rien, que chacun soit un — gueux. Que la propriété soit \emph{impersonnelle}, qu’elle appartienne à la — \emph{Société.}\par
Devant le \emph{Maître} suprême, l’unique \emph{commandant}, nous étions tous devenus égaux, nous étions tous devenus des personnes égales, c’est-à-dire des zéros.\par
Devant le \emph{Propriétaire} suprême, nous devenons tous des — gueux égaux ; jusqu’à présent on pouvait n’être, à côté de son voisin, qu’un « gueux », un « pauvre diable » : désormais toute distinction s’efface, tous étant des gueux, et la Société communiste se résume dans ce qu’on peut appeler la « gueuserie » générale.\par
Quand le prolétaire aura réussi à réaliser la « Société » qu’il a en vue, et dans laquelle doit disparaître toute différence entre riche et pauvre, il sera un gueux ; mais être un gueux est pour lui être quelque chose, et il pourrait faire de ce mot « gueux » un titre aussi honorable qu’est devenu le titre de « bourgeois » grâce à la Révolution. Le gueux est son idéal, et nous devons devenir tous des gueux.\par
Tel est le second vol fait à la « personnalité » au profit de l’ « humanité ». On ne laisse à l’individu ni le droit de commander ni le droit de posséder : l’Etat prend l’un, la Société prend l’autre.\par
La société actuelle présentant les inconvénients les plus choquants, ceux qui ont le plus à en souffrir, c’est-à-dire les membres des régions inférieures de la société, en sont aussi le plus frappés, et ils croient pouvoir attribuer tout le mal à la société elle-même ;  aussi se donnent-ils pour tâche de découvrir la \emph{société telle quelle doit être}. On reconnaît là l’illusion, vieille comme le monde, qui fait que l’on commence toujours par rejeter la faute commise sur un autre que soi-même ; dans le cas présent, on incrimine l’Etat, l’égoïsme des riches, etc., alors que c’est bien notre faute s’il y a un Etat et s’il y a des riches.\par
Les réflexions et les conclusions du Communisme paraissent des plus simples : Dans l’état actuel des choses, les uns sont lésés par les autres, et, en fait, c’est la majorité qui souffre à cause de la minorité. Les uns jouissent du bien-être, les autres sont dans le besoin ; la \emph{situation} présente, c’est-à-dire l’Etat \emph{(status = }situation) ne peut subsister. Que mettre à sa place ? — Le \emph{bien-être général}, le \emph{bien-être de tous}, au lieu du bien-être de quelques-uns.\par
La Révolution a rendu la Bourgeoisie toute-puissante, et a supprimé toute inégalité en ce sens que chacun a été, selon sa position antérieure, élevé ou abaissé au rang de « citoyen » ; le plébéien a été — élevé et le noble — abaissé ; le tiers-état est devenu l’unique état, l’état des citoyens.\par
A cela, le Communisme répond : Ce qui fait notre valeur, notre dignité, ce n’est pas notre qualité d’\emph{enfants tous égaux} de notre mère l’Etat, et nés tous avec les mêmes droits à son amour et à sa protection, mais le fait que nous existons \emph{les uns pour les autres}. Notre égalité, ou ce qui nous fait égaux, consiste en ce que moi, toi, nous tous, tant que nous sommes, nous agissons ou « travaillons » ; autrement dit, si nous sommes égaux c’est parce que chacun de nous est un \emph{travailleur}. L’essentiel en nous n’est pas ce que nous sommes \emph{pour l’Etat}, c’est-à-dire notre qualité de citoyen ou notre \emph{bourgeoisie}, mais ce que nous sommes \emph{les uns pour les autres :} chacun existe par et pour autrui ; vous soignez mes intérêts et réciproquement je veille sur les vôtres. Ainsi, par exemple, vous travaillez à me vêtir (tailleur), moi à vous amuser  (poète dramatique, danseur de corde, etc.) ; vous travaillez à me nourrir (aubergiste, etc.), moi à vous instruire (savant, etc.). C’est le travail qui fait notre dignité et notre — égalité.\par
Quel avantage retirons-nous de la Bourgeoisie ? Des charges ! Et comment estime-t-on notre travail ? Aussi bas que possible. Le travail fait cependant notre unique valeur ; le \emph{travailleur} est en nous ce qu’il y a de meilleur, et si nous avons une signification dans le monde, c’est comme travailleurs. Que ce soit donc d’après notre travail qu’on nous apprécie, et que ce soit notre travail qu’on évalue.\par
Que pouvez-vous nous opposer ? Du travail, et rien que du travail. Si nous vous devons une récompense, c’est à cause du travail que vous fournissez, de la peine que vous vous donnez, et non simplement parce que vous existez ; c’est en raison de ce que vous êtes \emph{pour nous} et non de ce que vous êtes \emph{pour vous}. Sur quoi sont fondés vos droits sur nous ? Sur votre haute naissance, etc. ? Nullement ! rien que sur ce que vous faites pour satisfaire nos besoins ou nos désirs. Convenons donc de ceci : vous ne nous évaluerez que d’après ce que nous ferons pour vous, et nous en userons de même à votre égard. \emph{Le travail crée la valeur}, et la valeur se mesure par le travail, nous entendons le travail qui nous profite, la peine qu’on se donne \emph{les uns pour les autres}, le \emph{travail d’utilité générale}. Que chacun soit aux yeux des autres un \emph{travailleur}. Celui qui accomplit une besogne utile n’est inférieur à personne ; en d’autres termes, — tous les travailleurs (dans le sens, naturellement, de producteurs pour la communauté, travailleurs communistes) sont égaux. Si le travailleur est digne de son sort, que son sort soit digne de lui.\par

\asterism

\noindent Tant que la Foi suffit pour assurer à l’homme sa dignité et son rang, on n’eut rien à objecter au travail,  quelqu’absorbant qu’il fût, du moment qu’il ne détournait pas l’homme de la foi. Mais aujourd’hui que chacun a en soi une humanité à cultiver, la relégation de l’homme dans un travail de machine n’a plus qu’un nom : c’est de l’esclavage. Si l’ouvrier de fabrique doit se tuer à travailler pendant douze heures et plus par jour, qu’on ne parle plus pour lui de dignité humaine ! Toute besogne doit avoir un but qui satisfasse l’homme, et il faut pour cela que chaque ouvrier puisse devenir \emph{maître} dans son art, et que l’œuvre qu’il produit soit un tout. Dans une fabrique d’épingles, par exemple, l’ouvrier qui ne fabrique que des têtes, ou qui ne fait que passer à la filière le fil de laiton est ravalé au rang de machine, c’est un forçat et ce ne sera jamais un artiste ; son travail ne saurait l’intéresser et le \emph{satisfaire}, et il ne peut que l’\emph{éreinter.} Son œuvre, considérée en elle-même, ne signifie rien, n’a aucun but \emph{en soi}, n’est rien de définitif ; c’est le fragment d’un tout qu’un autre emploie — en exploitant le producteur.\par
Tout plaisir d’un esprit cultivé est interdit aux ouvriers au service d’autrui ; il ne leur reste que les plaisirs grossiers, toute culture leur est fermée. Pour être bon chrétien, il suffit de \emph{croire,} et croire est possible en quelque situation qu’on se trouve ; aussi les gens à convictions chrétiennes n’ont-ils en vue que la piété des travailleurs asservis, leur patience, leur résignation, etc. Les classes opprimées purent à la rigueur supporter toute leur misère aussi longtemps qu’elles furent chrétiennes, car le Christianisme est un merveilleux étouffoir de tous les murmures et de toutes les révoltes. Mais il ne s’agit plus aujourd’hui \emph{à étouffer} les désirs, il faut \emph{les satisfaire.} La Bourgeoisie, qui a proclamé l’évangile de la \emph{joie de vivre}, de la jouissance matérielle, s’étonne de voir cette doctrine trouver des adhérents parmi nous, les pauvres ; elle a montré que ce qui rend heureux, ce n’est ni la foi ni la pauvreté, mais l’instruction et la richesse : et c’est  bien ainsi que nous l’entendons aussi, nous autres prolétaires !\par
La Bourgeoisie s’est affranchie du despotisme et de l’arbitraire individuels, mais elle a laissé subsister l’arbitraire qui résulte du concours des circonstances et qu’on peut appeler la fatalité des événements ; il y a toujours une chance qui favorise et « des gens qui ont de la chance ».\par
Lorsque, par exemple, une branche de l’industrie vient à s’arrêter et que des milliers d’ouvriers sont sur le pavé, on pense assez juste pour reconnaître que l’individu n’est pas responsable, mais que « c’est la faute des circonstances » ; changeons donc ces circonstances, et changeons-les assez radicalement pour qu’elles ne soient plus à la merci de pareilles éventualités ; qu’elles obéissent désormais à une \emph{loi !} Ne soyons pas plus longtemps les esclaves du hasard. Créons un nouvel ordre de choses qui mette fin à toutes les \emph{fluctuations,} et que cet ordre soit sacré !\par
Jadis, pour obtenir quelque chose, il fallait « complaire à son \emph{maître} » ; depuis la Révolution il faut « avoir de la chance ». Une poursuite de la chance, un jeu de hasard, telle est la vie bourgeoise ; de là le précepte qu’il ne faut pas risquer de nouveau au jeu ce qu’on est parvenu à y gagner.\par
Contradiction bizarre, et pourtant toute naturelle : la concurrence, thème unique autour duquel se déroulent toutes les variations de la vie civile et politique, est devenue une pure loterie, depuis la [{\corr spéculation}] à la bourse jusqu’à la chasse aux clients, aux places, au travail, à l’avancement et aux décorations, et jusqu’au misérable petit négoce des usuriers juifs. Si l’on réussit à battre et à évincer ses concurrents, on a fait « un heureux coup ». Ce ne peut être en effet que par une faveur du sort, que le vainqueur est doué (quelque application qu’il ait d’ailleurs mise à les acquérir) de facultés contre lesquelles les autres n’ont  pu lutter ; il a eu la chance de ne rencontrer sur sa route personne de mieux — doué.\par
Ces gens qui, sans y voir de mal, passent leur vie ballottés par le flux et le reflux de la « veine » sont saisis de la plus vertueuse indignation quand leur propre principe se révèle sous son vrai jour de jeu de hasard en leur « portant malheur ». Un cornet de dés est une image de la concurrence beaucoup trop nette, trop peu déguisée ; comme toute nudité elle offense la décence et la pudeur.\par
C’est à ces caprices de la fortune que les Socialistes veulent mettre un terme, en fondant une société où les hommes ne soient plus le jouet de la chance. Tout naturellement, cette tendance se manifeste tout d’abord par la haine des « malheureux » contre les « heureux », c’est-à-dire de ceux pour lesquels le hasard n’a que peu ou rien fait contre ceux qu’il a comblés. Mais la mauvaise humeur du malchanceux ne s’adresse pas tant à celui qui a de la chance qu’à la chance elle-même, cette colonne pourrie de l’édifice bourgeois.\par
Les Communistes, partant de ce principe que l’activité libre est l’essence de l’homme, ont besoin du dimanche qu’exige comme compensation leur pensée des jours ouvrables. Il leur faut le dieu, l’élévation et l’édification que réclame tout effort matériel pour mettre un peu d’esprit dans leur travail de machines.\par
Si le Communiste voit en toi un homme et un frère, ce n’est là que sa manière de voir des dimanches ; les autres jours de la semaine il ne te regarde nullement comme un homme tout court, mais comme un travailleur humain ou un homme qui travaille. Si le premier point de vue s’inspire du principe libéral, le second recèle l’illibéralité. Si tu étais un « fainéant », il ne reconnaîtrait pas en toi l’homme, il y verrait un « homme paresseux » à corriger de sa paresse, et à catéchiser pour le convertir à la \emph{croyance}  que le travail est la « destination et la vocation » de l’homme.\par
Aussi le Communisme s’offre-t-il sous un double aspect : d’une part il attache grande importance à la satisfaction de l’homme spirituel, et d’autre part il avise aux moyens de satisfaire l’homme matériel ou charnel. Il pourvoit l’homme d’un double \emph{bénéfice}, à la fois matériel et spirituel.\par
La Bourgeoisie avait \emph{proclamé libres} les biens spirituels et matériels, et s’en était remise à chacun du soin de chercher à obtenir ce qu’il convoitait. Le Communisme donne réellement ces biens à chacun, les lui impose, et l’oblige à en tirer parti ; considérant que ce ne sont que les biens matériels et spirituels qui font de nous des hommes, il regarde comme essentiel que nous puissions acquérir ces [{\corr biens}] sans que rien ne nous fasse obstacle, afin d’être hommes. La Bourgeoisie rendait la production libre, le Communisme \emph{force} à la production, et n’admet que les \emph{producteurs}, les \emph{artisans.} Il ne suffit pas que les professions te soient ouvertes, il faut que tu en pratiques une.\par
Il ne reste plus à la Critique qu’à démontrer que l’acquisition de ces biens ne fait encore nullement de nous des hommes.\par
Le postulat du Libéralisme en vertu duquel chacun doit faire de soi un homme et acquérir une « humanité » implique la nécessité pour chacun d’avoir le temps de se consacrer à cette « humanisation » et de travailler \emph{à soi-même.}\par
Le Libéralisme politique pensait avoir fait le nécessaire en livrant à la concurrence tout le champ de l’activité humaine et en permettant à l’individu de tendre vers tout ce qui est humain. « Que tous puissent lutter contre tous. »\par
Le Libéralisme social juge cette permission insuffisante, parce que « permis » signifie simplement « qui n’est défendu à personne » et non « qui est rendu  possible à chacun ». Il part de là pour soutenir que la Bourgeoisie n’est libérale qu’en paroles, mais en fait suprêmement illibérale. Lui, de son côté, prétend nous fournir à tous le moyen de travailler à nous mêmes.\par
Le principe du travail supprime évidemment celui de la chance et de la concurrence. Mais il a également pour effet de maintenir le travailleur dans ce sentiment que l’essentiel en lui est le « travailleur » dégagé de tout égoïsme ; le travailleur se soumet à la suprématie d’une société de travailleurs, comme le bourgeois acceptait sans objection la concurrence.\par
Le beau rêve d’un « devoir social » est aujourd’hui encore le rêve de bien des gens, et l’on s’imagine encore que la Société nous donnant ce dont nous avons besoin, nous sommes ses obligés, à elle à qui nous devons tout\footnote{ \noindent P{\scshape roudhon}, par exemple, s’écrie (De la création de l’ordre, p. 414) : « Dans l’industrie comme dans la science, rendre publique une découverte est le premier et \emph{le plus sacré des devoirs !} »
 }. On persiste à vouloir \emph{servir} un « dispensateur suprême de tout bien ».\par
Que la société n’est pas un « moi » capable de donner, de prêter ou de permettre, mais uniquement un moyen, un instrument dont nous nous servons, — que nous n’avons aucun devoir social, mais uniquement des intérêts à la poursuite desquels nous faisons servir la société, — que nous ne devons à la société aucun sacrifice, mais que si nous sacrifions quelque chose ce n’est jamais qu’à nous-mêmes, — ce sont là des choses dont les Socialistes ne peuvent s’aviser : ils sont « libéraux », et, comme tels, imbus d’un principe religieux ; la Société qu’il rêvent est ce qu’était auparavant l’Etat : — sacrée !\par
La Société dont nous tenons tout est un nouveau maître, un nouveau fantôme, un nouvel « être suprême » qui nous impose « service et devoir ».\par
L’examen plus approfondi du Libéralisme tant  politique que social trouvera sa place plus loin. Contentons-nous pour le moment de les appeler au tribunal du Libéralisme humanitaire ou Libéralisme critique.
\paragraph[{A.II.3§3. Le Libéralisme humanitaire.}]{A.II.3§3. Le Libéralisme humanitaire.}
\noindent Le Libéralisme trouve son expression complète et définitive dans le Libéralisme « critique », qui se soumet lui-même à l’examen, le Critique toutefois restant un libéral et ne dépassant pas le principe du Libéralisme, l’Homme. C’est cette dernière incarnation du principe qui mérite excellemment de porter le nom de l’Homme et d’être appelée Libéralisme « humain » ou « humanitaire ».\par
Le travailleur passe pour le plus matériel et le plus égoïste des hommes ; il ne fait rien pour l’humanité et n’agit qu’exclusivement pour lui-même, en vue de satisfaire ses propres besoins.\par
La Bourgeoisie, en ne faisant l’homme libre que par sa naissance, l’a, pour le reste de la vie, laissé entre les griffes de l’inhumain (de l’égoïste). Aussi l’égoïsme possède-t-il, sous le régime du Libéralisme politique, un champ d’action extraordinairement étendu. Comme le citoyen emploie l’Etat, le travailleur emploiera la Société dans un but égoïste. « Tu n’as qu’un but égoïste, ton bien-être ! crie l’Humanitaire au Socialiste : embrasse \emph{un intérêt purement humain,} si tu veux que je sois ton compagnon. » Mais il faudrait pour cela une conscience plus ferme et plus compréhensive qu’une conscience de pur travailleur.\par
« Le travailleur ne fait rien, aussi n’a t-il rien ; mais s’il ne fait rien, c’est parce que son travail, restant toujours individuel et commandé par le besoin immédiat, est sans lendemain\footnote{ \noindent B{\scshape runo} B{\scshape auer}, \emph{Lit. Ztg.} V. 18.
 } ». On pourrait penser  le contraire : l’œuvre de Guttemberg n’est pas restée isolée, elle a engendré une innombrable postérité, et elle est encore aujourd’hui bien vivante ; elle répondait à un besoin de l’humanité, aussi est-elle éternelle, impérissable.\par
La conscience humanitaire méprise aussi bien la conscience du bourgeois que celle du travailleur : le bourgeois « s’indigne » contre les vagabonds (tous ceux qui n’ont pas une position stable) et leur « immoralité » ; le travailleur « est révolté » par les « fainéants » et leurs maximes « immorales » parce que antisociales et exploiteuses. L’Humanitaire leur répond : Le manque d’établissement de la plupart est ton œuvre, philistin ! Mais si toi, prolétaire, tu veux que tous se tuent à la besogne, si tu exiges que tous portent le bât, c’est que tu n’as été jusqu’ici qu’une bête de somme. Tu prétends en vérité, en nous condamnant tous aux travaux forcés, alléger la peine elle-même, mais c’est uniquement pour que tous disposent des mêmes loisirs. Et que feront-ils de ces loisirs ? Comment ta « Société » s’y prendra-t-elle pour que les loisirs ainsi conquis soient \emph{humainement }employés ? Elle devra bien les abandonner comme une proie à l’égoïsme, et tout le bénéfice de ta société c’est l’égoïste qui l’accaparera. A quoi a abouti l’affranchissement de l’homme de tout bon plaisir personnel, cette conquête si vantée de la bourgeoisie ? L’Etat n’ayant pas pu donner à cette liberté une valeur humaine a dû l’abandonner à l’arbitraire.\par
Certes, il faut que l’homme n’ait pas de maître, mais il faut pour cela que l’égoïste ne redevienne pas son maître et qu’il soit, lui, le maître de l’égoïste. Il n’est pas moins nécessaire que l’homme jouisse de loisirs, mais si c’est l’égoïste qui détourne ces loisirs à son profit, ils seront perdus pour l’homme : aussi devez-vous donner aux loisirs une signification humaine. Mais votre travail même, vous autres ouvriers, vous ne vous y livrez que dans en but égoïste, parce que vous  voulez manger, boire, vivre ; comment pourriez-vous être moins égoïstes dans votre repos ? Vous ne travaillez que parce qu’une fois la besogne finie il est doux de se récréer, — de flâner ; quant à la façon dont vous occuperez vos heures de loisir, le hasard seul en décidera.\par
Pour verrouiller toutes les portes par où l’égoïsme peut s’introduire dans la place, il faudrait s’efforcer de parvenir au complet « désintéressement ». Le désintéressement seul est humain, vu que l’homme seul est désintéressé, tandis que l’égoïste ne l’est jamais.\par

\asterism

\noindent Admettons provisoirement le désintéressement, et demandons : Veux-tu donc ne t’intéresser à rien, ne t’enthousiasmer pour rien, pas même pour la Liberté, l’Humanité, etc ? — Oh ! si ! Mais ce n’est pas là un intérêt égoïste, un bas calcul d’intérêt, c’est un intérêt humain, \emph{théorique}, c’est-à-dire un intérêt qui ne s’attache ni à un individu ni aux individus (à tous), mais à l’\emph{idée,} à l’Homme.\par
Mais ne remarques-tu pas que ce qui t’enthousiasme n’est que \emph{ton} idée, \emph{ton} idée de la Liberté, par exemple ? Et ne remarques-tu pas en outre que ton prétendu désintéressement n’est, comme le désintéressement religieux, qu’une spéculation sur le Ciel ?\par
Les besoins de l’individu te laissent froid, et tu serais capable de t’écrier abstraitement « \emph{fiat libertas, pereat mundus} ». Tu ne te soucies pas du lendemain, et tu ne prends surtout pas sérieusement à cœur les appétits individuels, ton bien-être à toi et aux autres ; tout cela t’importe peu, parce que tu es un — rêveur.\par
L’Humanitaire sera-t-il peut-être assez libéral pour considérer comme humain tout le possible humain ? Au contraire ! En vérité, il ne nourrit pas contre la prostituée les mêmes préventions morales que le philistin, mais « penser que cette femme fait de son  corps une machine à gagner de l’argent\footnote{ \noindent \emph{Lit. Ztg.} V, 26.
 } » la lui rend méprisable en tant qu’« Être humain »,\par
Son jugement est celui-ci : la prostituée n’est pas un être humain, ou : par le fait qu’une femme se livre à la prostitution elle se déshumanise, elle se met au ban de l’humanité. Puis : le Juif, le Chrétien, le Théologien, etc. n’est pas Homme ; plus tu es Juif etc., plus tu es loin d’être Homme. Et voici de nouveau le postulat impératif : rejette loin de toi tout \emph{a parte}, que ta critique le détruise ! Ne sois ni juif, ni chrétien, sois Homme et rien qu’Homme. Mets ton \emph{humanité} au-dessus de toute spécification limitative, sois par elle un homme \emph{sans restriction}, un « homme libre » ; autrement dit, reconnais dans l’humanité l’\emph{essence} déterminante de tous tes prédicats.\par
Je réponds : Certes, tu es plus que juif, plus que chrétien, etc., mais tu es aussi plus qu’Homme. Tout cela ce sont des idées, tandis que toi tu as un corps. Penses-tu donc pouvoir jamais devenir « homme en soi » ? Penses-tu que nos descendants ne trouveront plus aucun préjugé, aucune barrière à renverser, contre lesquels nos forces n’auront pas suffi ? Ou t’imagines-tu que tes quarante ou cinquante ans t’ont mené si loin que les jours qui suivront n’auront plus rien à te retrancher et que tu es dès à présent un homme ? Les hommes de l’avenir lutteront encore pour mainte liberté que nous ne sentons pas même nous manquer. Que fais-tu de cette liberté future ? Si tu voulais ne t’estimer rien avant d’être devenu homme, tu attendrais jusqu’au « jugement dernier », jusqu’au jour où l’homme et l’humanité auront atteint la perfection. Mais d’ici là tu seras sûrement mort : où est le prix de ta victoire ?\par
Renverse donc résolument les termes, et dis toi : \emph{Je suis homme ;} je n’ai pas à commencer par acquérir  la qualité d’homme, car elle m’appartient déjà, au même titre que tous mes attributs.\par
Mais, demande le Critique, comment peut-on être simultanément juif et homme ? Primo, répondrai-je, on ne peut être ni juif ni homme, s’il faut pour cela que « on » signifie identiquement la même chose que juif ou homme ; car « on » étant logiquement de compréhension supérieure, vous ne pourrez jamais dire « on = juif » ; que Schmoule soit aussi juif qu’il veut, il ne sera jamais juif et rien que juif, attendu qu’il est déjà au moins \emph{tel} juif. Secundo, étant juif, on ne peut certes pas être homme, si « être homme » signifie n’être rien de particulier. Et tertio — et c’est à quoi je veux en venir, il se peut que je sois, en étant juif, tout ce que — je puis être : Il vous serait difficile d’exiger de Samuel, de Moïse et d’autres, qu’ils se fussent élevés au-dessus du judaïsme, bien que vous puissiez dire que ce n’étaient pas encore là des « hommes ». Ils furent en vérité tout ce qu’ils pouvaient être. En est-il autrement des Juifs d’aujourd’hui ? De ce que vous avez découvert l’idée d’humanité, s’ensuit-il que chacun d’eux s’y puisse convertir ? S’il le pouvait, il ne s’en ferait pas faute, et s’il s’en abstient c’est — qu’il ne le peut pas. Que lui importent vos exhortations et cette \emph{vocation} d’être Homme que vous lui attribuez ?\par

\asterism

\noindent Dans la société humaine que nous promet l’Humanitaire, il n’y a évidemment pas de place pour ce que toi et moi avons de « particulier », et rien ne peut plus entrer en ligne de compte qui porte le cachet d’« affaire privée ». Ainsi se complète le cycle du Libéralisme ; son bon principe est l’Homme et la liberté humaine, et son mauvais principe est l’Egoïste et tout ce qui est privé : là est son dieu, ici son diable.\par
 La \emph{personne} particulière ou privée ayant perdu toute valeur dans l’« Etat » (plus de privilèges), et la \emph{propriété} particulière ou privée ayant été dépouillée de sa légitimité par la« Société des travailleurs » ou « Société des gueux », vient la « Société humaine » qui, elle, met de côté indistinctement \emph{tout} le particulier ou le privé. Ce n’est que le jour où la « critique pure » aura terminé sa laborieuse enquête que nous serons enfin fixés, et que nous saurons au juste ce que nous devons tenir pour privé et, « pénétrés de sa vanité et de son néant » — laisser debout juste comme devant.\par
Ni l’Etat ni la Société ne satisfont le libéral humanitaire ; aussi les nie-t-il tous deux, quitte à les conserver tous deux. En réalité, la Société humanitaire est à la fois Etat universel et Société universelle ; ce n’est qu’à l’Etat limité qu’on reproche de faire trop de cas des intérêts privés spirituels (convictions religieuses des gens, par exemple), et à la Société limitée, des intérêts privés matériels. Tous deux doivent s’en remettre aux particuliers du soin des intérêts privés, et, devenant Société humaine, s’inquiéter uniquement des intérêts humains généraux.\par
Lorsque les Politiques s’efforçaient de supprimer la \emph{volonté personnelle}, (l’arbitraire et le bon plaisir,) ils ne s’apercevaient pas que la \emph{propriété} lui offrait un sûr asile.\par
Lorsque les Socialistes à leur tour abolissent la \emph{propriété}, ils négligent de remarquer que cette propriété se perpétue sous forme d’\emph{individualité.} N’y a-t-il donc point d’autre propriété que l’argent et les biens au soleil ? Chacune de mes pensées, chacune de mes opinions\footnote{ \noindent \emph{Jeu de mots intraduisible sur les mots « Meinung » (opinion) et « Mein » (mien). (N. du Tr.)}
 } ne m’est-elle pas également propre, n’est-elle pas mienne ?\par
Pas d’autre alternative donc pour la pensée que de  disparaître ou de devenir impersonnelle. Il n’appartient pas à la personne d’avoir des opinions à elle, tout ce qu’elle pourrait avoir en propre doit faire retour à quelque chose de plus \emph{général} qu’elle : de même que l’Etat a confisqué la volonté, et « que la Société a accaparé la propriété, l’« Homme » à son tour doit totaliser les pensées individuelles et en faire de la pensée humaine, purement et universellement humaine.\par
Si on permet aux opinions individuelles de subsister, j’aurai \emph{mon} dieu, (Dieu ne saurait être que « mon dieu », c’est mon opinion ou ma « croyance ») et si j’ai \emph{mon} dieu, j’aurai ma foi, \emph{ma} religion, \emph{mes} pensées, \emph{mes} idéaux. Substituons à ces opinions particulières une foi commune à tous les hommes, « le fanatisme de la liberté ». Ce sera là une foi étroitement conforme à l’« essence de l’homme », et ce sera enfin, l’Homme seul étant raisonnable (toi et moi pouvons être très déraisonnables), une foi raisonnable.\par
Pour réduire à l’impuissance la volonté et la propriété privées, il faut avant tout dompter l’individualisme ou l’égoïsme. Après cette victoire de principe, étape suprême dans l’évolution de l’ « homme libre », on verra les buts d’ordre inférieur, tels que le « bien-être » social des Socialistes, s’évanouir devant la sublime, la radieuse « idée de l’Humanité ». Tout ce qui n’est pas « universellement humain » est un \emph{a parte} qui ne satisfait que quelques-uns ou un seul, ou qui, s’il satisfait tout le monde, ne satisfait chacun qu’en tant qu’individu et non en tant qu’homme ; autrement dit, tout ce qui n’est pas humanité pure est « égoïsme ».\par
Le \emph{bien-être} est encore le but suprême des Socialistes, comme le libre concours, l’\emph{émulation} est celui des Libéraux politiques. Maintenant aussi on est libre de bien vivre et de faire pour cela le nécessaire, de même qu’il est permis d’entrer dans la lice à celui que tente le concours (concurrence). Mais il suffit pour  prendre part au concours d’être \emph{citoyen,} et pour avoir sa part de jouissance d’être \emph{travailleur :} citoyen et travailleur ne sont encore ni l’un ni l’autre synonymes d’« homme ». L’homme ne parvient au « vrai bien » (n’est « souverainement bien ») que lorsqu’il est aussi « spirituellement libre » ! Car l’Homme est Esprit, et c’est pourquoi toutes les puissances étrangères à lui, à l’Esprit, toutes les puissances supra-humaines, célestes, non-humaines, doivent être détruites, et le nom d’« Homme » doit s’élever rayonnant au-dessus de tous les noms.\par
Ainsi l’époque moderne (époque des Modernes) finit par revenir à son point de départ, et fait de nouveau de la « liberté spirituelle » son principe et sa fin.\par
Le Libéral humanitaire, s’adressant particulièrement au Socialiste, lui dit : En te faisant de l’activité un devoir, la Société affranchit, il est vrai, cette activité de l’influence des individus, c’est-à-dire des égoïstes, mais elle ne te prescrit encore nullement une activité \emph{purement humaine}, et rien ne t’oblige encore à faire de toi sans réserve un organe de l’Humanité. Quelle espèce d’activité la Société exige-t-elle de toi ? Le hasard des circonstances seul en décidera ; elle pourrait t’employer à bâtir un temple ou quelque chose d’équivalent ; ne le fît-elle pas, tu pourrais de ton propre mouvement t’appliquer à une sottise, autrement dit à quelque chose de non-humain. Bien plus, si tu travailles, c’est uniquement pour pourvoir à tes besoins, et en somme pour vivre, pour l’amour de ta chère vie, et nullement pour la plus grande gloire de l’humanité. Que faut-il donc, pour que tu puisses te flatter d’une activité vraiment libre ? Il faut que tu te libères de toutes sottises, que tu t’affranchisses de tout ce qui est non pas humain, mais égoïste (relatif à l’individu et non à l’homme que l’individu incarne), il faut que tu dépouilles toutes les idées dont la non-vérité obscurcit l’Homme ou l’idée d’humanité, bref il faut que tu ne sois pas seulement libre d’agir, mais  que de plus le contenu de ton activité soit exclusivement humain, et que tu n’agisses et ne vives que pour l’humanité. Tu en es loin, tant que tes efforts ne tendent vers d’autre but que le bien-être, la prospérité de toi et de tous : ce que tu fais pour ta société de gueux n’est rien pour la « Société humaine ».\par
Le travail à lui seul ne suffit pas pour faire de toi un homme, car le travail est quelque chose de formel, et la matière en est à la merci des circonstances ; la question est de savoir qui tu es, toi qui travailles. Tu peux parfaitement travailler talonné par des besoins égoïstes (matériels), rien que pour te procurer le vivre, etc. : le travail doit être commandé par l’humanité, viser le bien de l’humanité, être profitable à son évolution historique ; bref, le travail doit être \emph{humain.} Cela suppose deux choses : 1\textsuperscript{o} qu’il soit utile à l’humanité, et 2\textsuperscript{o} qu’il soit le fait d’un « Homme ». La première de ces deux conditions peut être remplie par tout travail quel qu’il soit, car les œuvres de la nature elles-mêmes, les animaux par exemple, sont mises à contribution par l’humanité, et servent aux recherches scientifiques, etc. ; mais la seconde condition implique que le travailleur connaisse le but humain de son labeur ; or, ce but il ne peut s’en rendre compte que s’il \emph{se sait homme}, et qui l’instruira de sa dignité d’homme ? — La \emph{Conscience.}\par
Certes, c’est déjà beaucoup d’avoir cessé de t’attacher comme une brute à produire un fragment d’une œuvre que tu ne verras point, mais tu ne fais encore qu’embrasser du regard l’ensemble de ta tâche, et la conscience de ton œuvre que tu as acquise est encore bien loin de la conscience de toi, de la conscience de ton véritable « moi » ou de ton « essence », l’Homme. Le travailleur sent donc encore le besoin d’une « conscience supérieure » qui lui fait défaut, et ce besoin qu’il ne peut satisfaire par la pratique de son métier, il en cherche la satisfaction en dehors des heures de travail, pendant ses loisirs. Aussi la  récréation, le congé, restent-ils le complément nécessaire de son travail ; il se voit forcé de tenir à la fois pour humains le travail et la flânerie, et même de donner la première place au paresseux, à celui qui se repose. Il ne travaille que pour être quitte de son travail, il ne veut affranchir le travail que pour s’affranchir du travail.\par
Bref, son travail ne le satisfait point parce qu’il en est simplement chargé par la Société ; ce n’est qu’un pensum, un devoir, une tâche ; et réciproquement sa Société ne le satisfait point parce qu’elle ne lui fournit que du travail. Le travail devrait le satisfaire en tant qu’homme, tandis qu’il ne satisfait que la Société ; la Société devrait l’employer comme homme, tandis qu’elle ne l’emploie que comme — un travailleur gueux, ou un gueux qui travaille.\par
Travail et Société ne lui sont profitables qu’en tant qu’il a les besoins d’un « égoïste » et non d’un « homme ».\par
Telle est la position que prend la Critique en face du problème ouvrier. Elle en appelle à l’ « Esprit », elle conduit le combat de l’ « Esprit contre la masse\footnote{ \noindent \emph{Lit. Ztg.}, V, 24.
 } » et déclare que le travail communiste est une corvée sans la moindre trace d’esprit. La masse qui craint le travail se rend le travail facile. Dans la littérature dont nous sommes aujourd’hui inondés, cette horreur du travail a pour conséquence cette \emph{superficialité} bien connue qui refuse de se donner « la peine de chercher\footnote{ \noindent \emph{Ibid.}
 } ».\par
Aussi le Libéralisme humanitaire dit-il : Vous voulez le travail, c’est parfait : nous le voulons aussi, mais nous le voulons intégral. Nous n’y cherchons pas un moyen d’avoir des loisirs, mais nous prétendons trouver en lui pleine satisfaction, nous voulons le travail parce que travailler c’est nous développer, nous réaliser.\par
 Mais il faut pour cela que ce qu’on appelle travail soit digne de ce nom. Le seul travail qui honore l’homme est le travail humain et conscient, qui n’a pas un but égoïste, mais qui a pour but l’Homme, l’épanouissement des énergies humaines, de telle sorte qu’il permet de dire : \emph{laboro}, \emph{ergo sum}, je travaille, donc je suis homme. L’Humanitaire veut le travail de l’\emph{Esprit} mettant en œuvre toute matière, il veut que l’Esprit ne laisse aucun objet en repos, qu’il ne se repose devant rien, qu’il analyse et remette sans cesse sur le métier de sa critique les résultats obtenus. Cet esprit inquiet et sans repos fait le véritable travailleur ; c’est lui qui détruit les préjugés, qui abat toutes les barrières et les limitations et exalte l’homme au-dessus de tout ce qui pourrait le dominer, tandis que le Communiste qui ne travaille que pour lui, jamais librement mais toujours contraint par la nécessité, ne s’affranchit pas de l’esclavage du travail : il reste un travailleur esclave.\par
Le travailleur tel que le conçoit l’Humanitaire n’a rien d’un « égoïste », car il ne produit pas pour des individus, ni pour lui-même ni pour d’autres ; son labeur ne vise point la satisfaction de besoins privés, mais a pour objet l’Humanité et son progrès ; il ne s’attarde point à soulager les souffrances individuelles, et à s’inquiéter des désirs de chacun : il abat les barrières qui enserrent l’humanité, il déracine les préjugés séculaires, balaie les obstacles qui embarrassent la route, les erreurs qui font trébucher les hommes, et les vérités qu’il découvre, c’est pour tous et pour toujours qu’il les met en lumière ; bref — il vit et travaille pour l’humanité.\par
Je réponds à cela :\par
En premier lieu, celui qui découvre une vérité importante sait qu’elle peut être utile aux autres hommes, et comme la cacher jalousement ne lui procurerait aucune jouissance, il leur en fait part et la partage avec eux ; mais s’il a même conscience que ce partage est  précieux pour les autres, ce n’est cependant nullement pour l’amour des autres, mais uniquement pour lui-même qu’il a cherché et trouvé, parce que le problème l’attirait, et que l’obscurité et l’erreur ne lui auraient pas laissé de repos s’il n’avait de son mieux débrouillé le chaos et déchiffré l’énigme.\par
Il travaille donc pour lui-même, pour satisfaire \emph{son} désir. Que son œuvre se trouve être utile aux autres et même à la postérité, cela n’enlève point à son travail son caractère \emph{égoïste.}\par
En second lieu, puisque lui aussi ne faisait que travailler pour lui-même, pourquoi son œuvre serait-elle humaine, alors que celle des autres est inhumaine, c’est-à-dire égoïste ? Serait-ce parce que ce livre, ce tableau, cette symphonie est l’œuvre de tout son être, qu’il y a mis ce qu’il y avait de meilleur en lui, qu’il s’y est exprimé tout entier et qu’on peut l’y retrouver tout entier, tandis que l’œuvre de l’artisan ne reflète que l’artisan, c’est-à-dire l’habileté professionnelle et non « l’homme » ? Par ses poèmes, nous connaissons tout Schiller, tandis que des centaines et des milliers de poêles ne nous apprennent à connaître que le fumiste et non « l’homme ».\par
Mais cela revient simplement à dire que telle œuvre me révèle aussi complètement que possible, tandis que telle autre ne témoigne que de la connaissance que j’ai de mon métier. N’est-ce pas encore une fois \emph{moi }qu’exprime le fruit de mes veilles ? Et n’est-il pas plus égoïste de faire de son œuvre le socle sur lequel on s’expose aux yeux du monde, sur lequel on s’étale dans toutes les postures possibles, que de rester dissimulé derrière elle ? Tu me diras que ce que tu exposes ainsi c’est l’Homme ! Mais remarque que cet homme que tu nous montres c’est toi : tu ne nous montres que toi, et si quelque chose te distingue de l’artisan, c’est que celui-ci n’entend pas s’exprimer en raccourci dans une seule et unique œuvre, mais a besoin pour être reconnu comme étant lui-même d’être considéré  sous tous les autres aspects qui constituent sa vie ; le désir pour la satisfaction duquel est née son œuvre était — théorique.\par
Tu vas répliquer que tu révèles un tout autre homme, un homme plus digne, plus haut, plus grand, en un mot plus Homme que tel ou tel autre. Soit, je veux admettre que tu réalises tout le possible humain, que tu es parvenu où nul autre ne peut atteindre. En quoi consiste ta grandeur ? Précisément en ce que tu es plus que d’autres hommes (que la « masse »), plus que ne sont les « hommes ordinaires » ; ce qui te fait grand, c’est ton élévation au-dessus des hommes. Si tu te distingues au milieu d’eux, ce n’est nullement parce que tu es un homme, mais parce que tu es un homme « unique ». Ton œuvre témoigne bien de ce dont un homme est capable, mais de ce que toi qui es un homme tu l’as accomplie il ne résulte pas que d’autres, également hommes, en puissent faire autant : ce n’est que parce que tu es un homme \emph{unique} que tu as pu l’accomplir, et en cela tu es unique.\par
Ce n’est pas l’homme qui fait ta grandeur, c’est toi qui la fais parce que tu es plus qu’un homme et plus puissant que d’autres — hommes.\par
On s’imagine ne pouvoir être plus qu’homme. Etre moins qu’homme serait pourtant bien plus difficile.\par
On s’imagine en outre que tout ce que l’on fait de bien, de beau, de remarquable fait honneur à l’homme. Mais si je suis homme, c’est comme Schiller était souabe, Kant prussien et Gustave Adolphe myope, et mes mérites et les leurs font de nous un homme, un souabe, un prussien et un myope distingués. Tous ces qualificatifs valent au fond la canne de Frédéric le Grand, qui n’est célèbre que parce que Frédéric l’est.\par
A l’ancien « rendez hommage à Dieu », le Moderne répond « rendez hommage à l’Homme ». Mais mes hommages je compte les garder pour moi.\par
 Lorsque la Critique exhorte les hommes à être « humains », elle formule la condition indispensable de la sociabilité ; car ce n’est qu’en tant qu’on est homme parmi les hommes que l’on peut vivre avec eux \emph{en société}. Elle montre ainsi son but \emph{social,} la fondation de la « société humaine ».\par
La Critique est incontestablement la plus parfaite de toutes les théories sociales, parce qu’elle écarte et annihile tout ce qui \emph{sépare} l’homme de l’homme : tous les privilèges, et jusqu’au privilège de la foi. Elle a achevé de purifier et a systématisé le vrai principe social, le principe d’amour du Christianisme, et c’est elle qui aura fait la dernière tentative possible pour dépouiller les hommes de leur exclusivisme et de leur foncière. inimitié, en luttant corps à corps avec l’Egoïsme sous sa forme la plus primitive et par conséquent la plus dure, l’\emph{unicité} ou l’exclusivisme.\par
« Comment pouvez-vous vivre d’une vie vraiment sociale, tant qu’il reste en vous la moindre trace d’exclusivisme, » la moindre chose qui n’est que vous et rien que vous ?\par
Je demande au contraire : Comment pouvez-vous être vraiment uniques, tant qu’il reste entre vous la moindre trace de dépendance, la moindre chose qui n’est pas vous et rien que vous ? Tant que vous restez enchaînés les uns aux autres, vous ne pouvez parler de vous au singulier ; tant qu’un « lien » vous unit, vous restez un \emph{pluriel}, à vous douze vous faites la douzaine, à mille vous formez un peuple, et à quelques millions l’humanité !\par
« Ce n’est que par votre humanité que vous pouvez avoir commerce les uns avec les autres en tant qu’hommes, de même que grâce seulement à votre patriotisme vous pouvez vous entendre comme patriotes ! »\par
Soit, mais je réponds : Ce n’est que si vous êtes uniques que vous pouvez avoir commerce les uns  avec les autres en votre nom propre, et être les uns pour les autres — ce que vous êtes.\par
Le Critique le plus radical est précisément celui que frappe le plus cruellement la malédiction qui pèse sur son principe. A mesure qu’il se dépouille d’un exclusivisme après l’autre, et qu’il secoue successivement zèle religieux, patriotisme, etc., il dénoue un lien après l’autre, et se sépare des dévôts, des patriotes, etc. ; si bien que finalement, tous les liens étant tombés, il se trouve — seul. Il est forcé de rejeter tout ce qui a quelque chose d’exclusif ou de privé, mais qu’est-ce qui peut être en définitive plus exclusif que l’exclusive, l’unique personne elle-même ?\par
Peut-être pense-t-il qu’il vaudrait mieux que \emph{tous }devinssent des « hommes » et abandonnassent leur exclusivisme ? Mais « tous » ne signifie précisément rien d’autre que « chaque individu », de sorte que la contradiction reste aussi aiguë qu’auparavant, car chaque « individu » est l’exclusivisme même. Comme l’Humanitaire ne laisse plus à l’individu rien de privé ou d’exclusif, ni pensées privées ni sottise privée, il finit par le laisser complètement nu, car sa haine absolue et fanatique du privé ne permet à son égard aucune tolérance, tout privé étant essentiellement \emph{inhumain}. L’Humanitaire est cependant impuissant à détruire la personne privée elle-même, car sa critique se briserait les dents avant d’entamer la dure écorce de la personnalité ; aussi est-il bien obligé de se contenter de déclarer que cette personne est une « personne privée », et de se résigner à lui rendre, en réalité tout le domaine du privé.\par
Que fera la Société, si elle ne s’inquiète plus de rien de privé ? Va-t-elle rendre le privé impossible ? Non, mais elle « le subordonnera aux intérêts de la Société, et permettra par exemple aux volontés individuelles  de s’accorder autant de jours de congé qu’elles le jugeront bon, pourvu que ces volontés individuelles ne se mettent pas en contradiction avec l’intérêt général\footnote{ \noindent B{\scshape runo} B{\scshape auer}, \emph{Judenfrage}, p. 66.
 } ». Tout le privé sera \emph{abandonné à lui-même :} il ne présente aucun intérêt pour la Société.\par
« L’irréductible opposition faite à la Science par l’Eglise et la religiosité prouve qu’elles sont (ce qu’elles ont toujours été, quelque illusion qu’on ait pu se faire à leur égard tant qu’elles passèrent pour la base et le fondement de l’Etat)... une pure affaire privée. Jadis même, si elles furent étroitement unies à l’Etat et si l’Etat fut chrétien, cette union prouva simplement que l’Etat n’avait point encore développé son idée politique générale et ne reconnaissait d’autre droit que des droits privés[{\corr ...}] elles témoignèrent d’une façon irrécusable que l’Etat était affaire privée et ne s’occupait que d’affaires privées. Si l’Etat a enfin le courage et la force de rompre avec le passé et d’accomplir sa mission universelle, s’il parvient à remettre à leur place les intérêts particuliers et les affaires privées... la Religion et l’Eglise seront libres comme elles ne l’ont jamais été. Elles seront abandonnées à elles-mêmes au même titre que les plus pures affaires privées et que la satisfaction des besoins strictement personnels, et chaque individu, chaque communauté ou communion de fidèles, pourra travailler au salut de son âme comme il lui plaira et de la façon qui lui paraîtra la plus efficace ; chacun pourvoira à sa félicité selon qu’il en sentira personnellement le besoin, et choisira et salariera pour veiller sur son âme celui qui lui semblera offrir le plus de garanties de succès. Et la Science enfin sera hors de question\footnote{ \noindent \emph{Id.}, p. 60.
 } ».\par
Qu’arrivera-t-il ? La vie sociale va-t-elle donc finir, et toute sociabilité, toute fraternité, tout ce qui a été édifié sur le principe d’amour ou de société va-t-il s’effondrer ?\par
 Comme si l’un ne devait pas fatalement toujours rechercher l’autre parce qu’il en \emph{a besoin}, comme si l’autre pouvait ne pas toujours s’offrir à l’un parce qu’il en \emph{a besoin !} Le seul changement est que désormais l’individu s’\emph{unira} réellement à l’individu, tandis qu’auparavant il lui était \emph{lié}. Le père et le fils, qu’un lien enchaîne l’un à l’autre jusqu’à la majorité de ce dernier, peuvent dans la suite continuer à faire spontanément route ensemble ; avant que le fils soit majeur, ils sont sous la \emph{dépendance} l’un de l’autre en tant que membres de la famille ; après, ils s’unissent en tant qu’égoïstes ; l’un reste le fils, l’autre reste le père, mais ce n’est plus comme fils et père qu’ils tiennent l’un à l’autre.\par
Le dernier privilège est en vérité l’ « Homme », et tous ont ce privilège, tous en jouissent. Car, comme le dit Bruno Bauer lui-même : « le privilège subsiste quand même tous y ont part\footnote{ \noindent B{\scshape runo} B{\scshape auer}, \emph{Die gute Sache der Freiheit}, p. 62-63.
 } ».\par
Résumons donc les étapes parcourues par le Libéralisme :\par
Primo : L’individu n’\emph{est} pas l’Homme, aussi la personnalité individuelle n’a-t-elle aucune valeur : donc, pas de volonté personnelle, pas d’arbitraire, plus d’ordres ni d’ordonnances ;\par
Secundo : L’individu n’\emph{a} rien d’humain, aussi le mien et le tien n’ont-ils aucun fondement dans la réalité : donc, plus de propriété ;\par
Tertio : Attendu que l’individu n’est pas Homme et n’a rien d’humain, il ne doit être rien du tout ; c’est un égoïste, et la critique doit supprimer lui et son égoïsme pour faire place à l’Homme « qui vient seulement d’être découvert ».\par
Mais si l’individu n’est pas Homme, l’Homme cependant est en puissance dans l’individu, et a chez ce dernier l’existence virtuelle qu’y ont tout fantôme et tout divin. Aussi le Libéralisme politique accorde-t-il  à l’individu tout ce qui lui revient en tant qu’il est « né homme », c’est-à-dire liberté de conscience, droit de propriété, etc., en un mot tout ce qu’on range sous le nom de « droits de l’homme ». Le Socialisme à son tour accorde à l’individu tout ce qui lui revient en tant qu’il « agit en homme » c’est-à-dire qu’il « travaille ». Vient enfin le Libéralisme humanitaire, qui gratifie l’individu de tout ce qu’il a en tant qu’Homme, c’est-à-dire de tout ce qui appartient à l’humanité. Conséquence : l’unique n’a rien, l’humanité a tout ; d’où l’évidente et absolue nécessité de cette « renaissance » que prêche le Christianisme : Deviens une nouvelle créature, deviens « Homme » !\par
Tout cela ne fait-il pas songer au \emph{Pater Noster ?} C’est à l’Homme qu’appartient la \emph{Puissance} (la force, \emph{dunamis}), et c’est pourquoi aucun individu ne peut être maître : c’est l’Homme qui est le maître des individus. — C’est à l’Homme qu’appartient la \emph{Royauté}, c’est-à-dire le monde, et c’est pourquoi l’individu ne doit pas être propriétaire : c’est l’Homme, c’est « Tous » qui possède le monde comme une propriété. — Enfin c’est à l’Homme qu’appartient la gloire, la \emph{Glorification (doxa)}, car l’Homme, c’est-à-dire l’humanité est le but de l’individu, but pour lequel il travaille, pour lequel il pense et vit, pour la glorification duquel il doit devenir « Homme ».\par
Les hommes se sont jusqu’à présent toujours efforcés de découvrir une forme sociale dans laquelle leurs anciennes inégalités ne fussent plus « essentielles ». Le but de leurs efforts fut un nivellement produisant l’\emph{égalité}, et cette prétention d’être autant de têtes sous le même bonnet ne signifiait rien moins que ceci : ils cherchaient un maître, un lien, une foi (« Nous croyons tous en un Dieu »). Si quelque chose est commun aux hommes et égal chez tous, c’est bien l’Homme, et grâce à cette communauté le besoin d’amour a trouvé sa satisfaction : il ne se reposa pas jusqu’à ce qu’il eut réalisé ce dernier nivellement, aplani toute  inégalité et jeté l’homme dans les bras de l’homme. Mais c’est justement ce trait d’union qui rend la rupture et l’antagonisme plus criants : une Société limitée mettait aux prises le Français et l’Allemand, le Chrétien et le Mahométan, etc., tandis que maintenant l’Homme s’oppose aux hommes, ou, puisque les hommes ne sont pas l’Homme, au Non-homme.\par
A cette proposition « Dieu est devenu homme » succède à présent cette autre : « l’Homme est devenu moi. » C’est là le \emph{moi humain}. Mais nous disons au contraire : je n’ai pas pu me trouver tant que je me suis cherché comme Homme. Si l’homme tente aujourd’hui de devenir moi et de gagner grâce à moi un corps, je remarque qu’en somme tout repose sur moi, et que sans moi l’Homme est perdu. Je ne puis cependant me sacrifier sur l’autel de ce Saint des Saints, et désormais je ne me demanderai plus si mes manifestations sont d’un Homme ou d’un Non-homme : que cet \emph{Esprit} me laisse en paix !\par
Le Libéralisme humanitaire n’y va pas de main morte. Que tu veuilles, à n’importe quel point de vue, être ou avoir quelque chose de particulier, que tu prétendes au moindre avantage que n’ont pas les autres, que tu veuilles t’autoriser d’un droit qui n’est pas un des « droits généraux de l’humanité », et tu es un égoïste.\par
Soit, je ne prétends avoir ou être rien de particulier qui me fasse passer avant les autres, je ne veux bénéficier à leurs dépens d’aucun privilège, mais, — je ne me mesure pas à la mesure des autres, et si je ne veux pas de passe-droit en ma faveur, je ne veux non plus d’aucune sorte de \emph{droit.} Je veux être tout ce que je puis être, avoir tout ce que je puis avoir. Que les autres soient ou aient quelque chose d’\emph{analogue,} que m’importe ? Avoir ce que j’ai, être ce que je suis, ils ne le peuvent. Je ne leur fais aucun \emph{tort,} pas plus que je ne fais de tort au rocher en ayant sur lui le « privilège » du mouvement. S’il \emph{pouvait} l’avoir, il l’aurait.\par
 Ne pas faire de \emph{tort} aux autres hommes ! De là découlent la nécessité de ne posséder aucun privilège, de renoncer à tout « avantage », et la plus rigoureuse doctrine de renoncement. On ne doit point se regarder comme « quelque chose de spécial », par exemple comme juif ou comme chrétien. Fort bien, moi non plus je ne me prends pas pour quelque chose de particulier ! Je me tiens pour \emph{unique !} J’ai bien quelque \emph{analogie} avec les autres, mais cela n’a d’importance que pour la comparaison et la réflexion ; en fait, je suis incomparable, unique. Ma chair n’est pas leur chair, mon esprit n’est pas leur esprit ; que vous les rangiez dans des catégories générales, « la Chair, l’Esprit », ce sont là de vos \emph{pensées}, qui n’ont rien de commun avec \emph{ma} chair et \emph{mon} esprit, et ne peuvent le moins du monde prétendre à me dicter une « vocation ».\par
Je ne veux respecter en toi rien, ni le propriétaire, ni le gueux, ni même l’Homme, mais je veux \emph{t’employer.} J’apprécie que le sel me fait mieux goûter mes aliments, aussi ne me fais-je pas faute d’en user ; je reconnais dans le poisson une nourriture qui me convient, et j’en mange ; j’ai découvert en toi le don d’ensoleiller et d’égayer ma vie, et j’ai fait de toi ma compagne. Il se pourrait aussi que j’étudiasse dans le sel la cristallisation, dans le poisson l’animalité, et chez toi l’humanité, mais tu n’es jamais à mes yeux que ce que tu es pour moi, c’est-à-dire mon objet, et en tant que \emph{mon} objet, tu es ma propriété.\par
Le Libéralisme humanitaire est l’apogée de la gueuserie. Nous devons commencer par descendre jusqu’au dernier échelon du dénuement et de la gueuserie si nous voulons parvenir à l’individualité ; mais est-il rien de plus misérable que — l’Homme tout nu ?\par
C’est toutefois dépasser la gueuserie que de me dépouiller même de l’Homme, après m’être aperçu que lui aussi m’est étranger et n’est pas un titre sur lequel je puisse rien fonder. Mais ce n’est plus là de la  gueuserie pure : ses dernières guenilles tombées, le gueux se dressant dans sa nudité, dépouillé de toute enveloppe étrangère, se trouve avoir rejeté même sa gueuserie, et cesser d’être un gueux.\par
Je ne suis plus un gueux, mais j’en fus un.\par

\asterism

\noindent Si l’on n’est pas, jusqu’à cette heure, parvenu à s’entendre, c’est que toute la bataille s’est livrée entre les partisans d’une « liberté » parcimonieusement mesurée, et ceux qui veulent « pleine mesure » de liberté, c’est-à-dire entre les \emph{modérés} et les \emph{immodérés}. Tout dépend de la réponse que l’on fera à la question : Comment et jusqu’à quel point faut-il que l’homme soit libre ? Que l’homme doive être libre, tous le pensent, aussi tous sont libéraux. Mais ce non-homme qui se cache au fond de chaque individu, quelle barrière lui opposer ? Comment faire pour libérer l’homme sans, du même coup, mettre en liberté le non-homme ?\par
Le Libéralisme, quelle que soit sa nuance, a un ennemi mortel, qui lui est aussi irréductiblement opposé que le Diable l’est à Dieu : toujours à côté de l’homme se dresse le non-homme, et l’égoïste à côté de l’individu. Etat, Société, Humanité, rien ne parvient à déloger ce diable de ses positions.\par
Le Libéralisme humanitaire a pris à tâche de prouver aux autres Libéraux qu’ils n’ont pas encore la moindre idée de ce que c’est que vouloir la « liberté ».\par
Les autres Libéraux n’apercevaient que l’égoïsme individuel, et le plus grave leur échappait ; le Libéralisme radical, lui, dirige ses batteries contre l’égoïsme « en bloc », et renie « en bloc » tous ceux qui n’embrassent pas comme leur propre cause la cause de la liberté ; d’où, grâce à lui, opposition aujourd’hui complète et hostilité implacable entre l’homme et le non-homme,  représentés l’un par la « Critique » et l’autre par la « masse\footnote{ \noindent \emph{Lit. Ztg.}, v, 23, v. 12 sqs.
 } », ou, sur le terrain de la théorie, l’un par ce qu’on appellera la « Critique libre et humaine » (Judenfrage, p. 114) et l’autre par les critiques superficielles et grossières, telle que, par exemple, la critique religieuse.\par
La Critique proclame son ferme espoir de vaincre la « masse » et de lui donner un « certificat d’indigence »\footnote{ \noindent \emph{Lit. Ztg.}, v. 15.
 }. Elle prétend finir par avoir raison, et par rabaisser toutes les dissensions des « tièdes et des timides » à n’être plus qu’une \emph{chicane} égoïste, une querelle misérable et mesquine. Et, en fait, toute dispute va perdre son importance, les mesquines querelles vont être oubliées, car un ennemi commun s’avance, et cet ennemi c’est la Critique. « Tous, autant que vous êtes, vous n’êtes que des égoïstes, et l’un ne vaut pas mieux que l’autre ! » Et voilà tous les égoïstes ligués contre la Critique.\par
Les égoïstes ? Sont-ce vraiment les égoïstes ? Non : s’ils s’insurgent contre la Critique, c’est précisément parce qu’elle les accuse d’égoïsme, et que cet égoïsme ils ne veulent pas en convenir. Aussi la Critique et la masse ont-elles la même base d’opérations : toutes deux combattent l’égoïsme, le désavouent, et s’en accusent mutuellement.\par
La Critique et la masse poursuivent le même but : affranchissement vis-à-vis de l’égoïsme, et ne se disputent que pour savoir laquelle des deux approche le plus de ce but ou même l’atteint.\par
Juifs, Chrétiens, absolutistes, hommes des ténèbres et hommes du grand jour, Politiques, Communistes, tous se défendent énergiquement contre le reproche d’égoïsme ; et quand vient la Critique, qui les en accuse carrément et sans ménagements, tous se \emph{disculpent} de cette accusation, et se mettent à guerroyer  contre — l’égoïsme, l’ennemi même auquel la Critique fait la guerre.\par
Ennemis des égoïstes, tous le sont, aussi bien la masse que la Critique, et l’une comme l’autre s’efforce de repousser l’égoïsme, tant en \emph{se} prétendant blanche comme neige qu’en noircissant la partie adverse.\par
Le Critique est le vrai « porte-paroles de la masse » ; il lui fournit de l’égoïsme « une notion simple et les mots pour l’exprimer », tandis que les anciens porte-paroles, ceux auxquels la \emph{Gazette littéraire} (\emph{Lit. Ztg.}, v. 24) refuse l’espoir de jamais triompher, n’étaient que des interprètes de rencontre, des apprentis. Le Critique est le prince et le conducteur de la masse dans la guerre faite à l’égoïsme au nom de la Liberté. Ce que le Critique combat, la masse le combat également. Mais il est en même temps l’ennemi de la masse ; non qu’il lui veuille du mal, mais il est envers elle l’ennemi bien intentionné, qui suit les peureux le fouet à la main pour les forcer à montrer qu’ils ont du cœur.\par
Aussi toute l’opposition entre la Critique et la masse se réduit-elle au dialogue suivant : « Vous êtes des égoïstes ! — Non, nous n’en sommes pas ! — je vais vous le prouver ! — Tu ne peux nous condamner sans nous entendre ! »\par
Prenons-les donc, les uns comme les autres, et pour ce qu’ils se prétendent, pour des non-égoïstes, et pour ce qu’ils se croient mutuellement, pour des égoïstes : ce sont des égoïstes et ce n’en sont pas.\par
La Critique dit bien : tu dois affranchir si complètement ton moi de toute limitation qu’il devienne un moi \emph{humain.} Mais Moi je dis : affranchis-toi tant que tu peux, tu n’arriveras à renverser que tes barrières à toi, car il n’appartient pas à chacun isolément de les renverser toutes ; ou plus explicitement : ce qui est une barrière pour l’un n’en est pas une pour l’autre ; ne t’épuise donc pas contre celles des autres, il suffit que tu abattes les tiennes. Qui a jamais eu le bonheur  de reculer la moindre borne, de lever le moindre obstacle qui fût une barrière pour tous les hommes ?\par
Celui qui renverse une de \emph{ses} barrières peut avoir par là montré aux autres la route et le procédé à suivre ; mais renverser \emph{leurs} barrières reste leur affaire. Personne, d’ailleurs, ne fait autre chose. Exhorter les gens à être intégralement hommes, revient à exiger d’eux qu’ils abattent toutes les barrières humaines ; or c’est impossible, car \emph{l’Homme} n’a pas de barrières et de limites ; Moi, j’en ai, c’est vrai, mais celles là seules, les \emph{miennes}, me concernent, et elles seules peuvent être par moi renversées. Je ne puis être un moi \emph{humain, }parce que je suis Moi et non purement homme.\par
Mais examinons encore une fois si dans ce que nous enseigne la Critique nous ne découvrirons rien à quoi nous puissions nous rallier ! Je ne suis pas libre tant que je ne me dépouille pas de tout intérêt, et je ne suis pas homme tant que je ne suis pas désintéressé. Soit, mais il m’importe en somme assez peu d’être homme et d’être libre, tandis qu’il m’importe beaucoup de ne laisser échapper sans en profiter aucune occasion de m’affirmer et de \emph{me} mettre en valeur. De ces occasions, le Critique m’en fournit une en professant que lorsque quelque chose s’implante en moi et devient indéracinable, je deviens le prisonnier et le serviteur de cette chose, autrement dit son possédé. Tout intérêt pour quoi que ce soit fait de moi, quand je ne sais plus m’en dégager, son esclave, et n’est plus ma propriété : c’est moi qui suis la sienne. C’est la Critique qui nous y invite : ne laissons s’ancrer, devenir stable, aucune partie de notre propriété, et ne nous trouvons bien que lorsque nous — \emph{détruisons}.\par
Tu n’es homme, dit la Critique, que si tu critiques, analyses et détruis sans repos ni trêve ! Et nous disons : je suis homme sans cela, et qui plus est je suis Moi. Aussi ne veux je prendre d’autre souci que celui de m’assurer ma propriété ; et pour me la bien assurer,  je la ramène perpétuellement à moi, je supprime en elle toute velléité d’indépendance, et je la « consomme » avant qu’elle ait le temps de se cristalliser et de devenir « idée fixe » ou « manie ».\par
Et si j’agis ainsi, ce n’est pas parce que « l’Humanité m’y convie » et m’en fait un devoir, mais parce que je m’y convie moi-même. Je ne me raidis point pour renverser tout ce qu’il est théoriquement possible à un homme de renverser ; tant que je n’ai pas encore dix ans, par exemple, je ne critique pas l’absurdité du décalogue ; en suis-je moins homme ? peut-être même que si mes dix ans agissent humainement, c’est précisément en ne la critiquant pas ! Bref, je n’ai pas de vocation et je n’en suis aucune, pas même celle d’être homme.\par
Est-ce à dire que je refuse les bénéfices réalisés dans les différentes directions par les efforts du Libéralisme ? Oh que non ! Gardons-nous de rien laisser perdre de ce qui est acquis. Seulement, à présent que, grâce au Libéralisme, voilà « l’Homme » libéré, je tourne les yeux vers moi-même, et je le proclame hautement : ce que l’homme a l’air d’avoir gagné, c’est \emph{Moi}, et Moi seul qui l’ai gagné.\par
L’homme est libre quand « l’Homme est pour l’homme l’être suprême ». Il faut donc pour que l’œuvre du Libéralisme soit complète et parachevée, que tout autre être suprême soit anéanti, que la Théologie soit détrônée par l’Anthropologie, qu’on se moque de Dieu et de la Providence, et que l’ « athéisme » devienne universel.\par
Que « mon Dieu » même en arrive à n’avoir plus aucun sens, c’est la dernière perte que puisse faire l’égoïsme de la propriété, car Dieu n’existe que s’il a à cœur le salut de l’individu, comme celui-ci cherche en lui son salut.\par
Le Libéralisme politique abolit l’inégalité du maître et du serviteur, et fit l’homme \emph{sans maître}, anarchique. Le maître, séparé de l’individu, de l’égoïste, devint  un fantôme : la Loi ou l’Etat. — Le Libéralisme social à son tour supprima l’inégalité résultant de la possession, l’inégalité du riche et du pauvre, et fit l’homme \emph{sans biens} ou sans propriété. La propriété retirée à l’individu revint au fantôme : la Société. — Enfin, le Libéralisme humain ou humanitaire fait l’homme \emph{sans dieu}, athée : le dieu de l’individu, « mon Dieu », doit donc disparaître. Où cela nous mène-t-il ? La suppression du pouvoir personnel entraîne nécessairement suppression du servage, la suppression de la propriété entraîne suppression du besoin, et la suppression du dieu implique suppression des préjugés, car, avec le maître déchu s’en vont les serviteurs, la propriété emporte les soucis qu’elle procurait, et le dieu qui chancelle et s’abat comme un vieil arbre arrache du sol ses racines, les préjugés. Mais attendons la fin : Le maître ressuscite sous forme Etat, et le serviteur reparaît : c’est le citoyen, l’esclave de la loi, etc. — Les biens sont devenus la propriété de la Société, et la peine, le souci renaissent : ils se nomment travail. — Enfin Dieu étant devenu l’Homme, c’est un nouveau préjugé qui se lève, et l’aurore d’une nouvelle foi : la foi dans l’Humanité et la Liberté. Au dieu de l’individu succède le dieu de tous, « l’Homme » : « le degré suprême où nous puissions aspirer à nous élever serait d’être Homme ! » Mais comme nul ne peut réaliser complètement l’idée d’Homme, l’Homme reste pour l’individu un au-delà sublime, un être suprême inaccessible, un dieu. De plus, celui-ci est le « vrai dieu », parce qu’il nous est parfaitement adéquat, étant proprement « \emph{nous-même} » : nous-même, mais séparé de nous et élevé au-dessus de nous.
\paragraph[{Post-Scriptum.}]{Post-Scriptum.}
\noindent Les observations qui précèdent sur la « libre critique humaine » et celles que j’aurai encore à faire  par la suite sur les écrits de tendance parallèle ont été notées au jour le jour, à mesure que paraissaient les livres auxquels elles se rapportent ; je n’ai guère fait ici que mettre bout à bout les appréciations fragmentaires que m’avaient suggérées mes lectures. Mais la Critique est en perpétuel progrès et chaque jour il se trouve qu’elle a fait quelques pas en avant ; aussi est-il nécessaire, aujourd’hui que j’ai écrit le mot fin au bout de mon livre, de jeter un coup d’œil en arrière et d’intercaler ici quelques remarques en forme de post-scriptum.\par
J’ai devant moi le huitième et dernier fascicule paru de l’\emph{Allgemeine Litteraturzeitung} (Revue générale de la littérature) de Bruno Bauer.\par
Dès les premières lignes, il nous est de nouveau parlé des « intérêts généraux de la Société ». Mais la Critique s’est recueillie, et donne à cette « Société » une signification nouvelle, par laquelle elle se sépare radicalement de l’ « Etat » avec lequel elle était restée jusqu’à présent plus ou moins confondue. L’Etat, naguère encore célébré sous le nom d’ « Etat libre » est définitivement abandonné, comme foncièrement incapable de remplir le rôle de « Société humaine ». La Critique s’est vue en 1842 « momentanément obligée d’identifier les intérêts humains et les intérêts politiques », mais elle s’est aperçue depuis que l’Etat, même sous la forme d’ « Etat libre », n’est pas la société humaine, ou, pour parler sa langue, que le peuple n’est pas « l’Homme ».\par
Nous avons vu la Critique faire table rase de la théologie, et prouver clairement que le Dieu succombe devant l’Homme ; nous la voyons à présent jeter par dessus bord la politique, et démontrer que devant l’Homme, peuples et nationalités s’évanouissent. Aujourd’hui qu’elle a rompu avec l’Eglise et l’Etat en les déclarant tous deux inhumains, nous ne tarderons pas à la voir se faire forte de prouver qu’à côté de l’Homme, la « masse », qu’elle-même appelle  un « être spirituel » est sans valeur ; et ce nouveau divorce ne sera pas pour nous surprendre, car nous pouvons déjà entrevoir des symptômes précurseurs de cette évolution. Comment, en effet, des « êtres spirituels » de rang inférieur pourraient-ils tenir devant l’Esprit suprême ? « L’Homme » renverse de leur piédestal les idoles fausses.\par
Ce que la Critique se propose pour le moment, c’est l’étude de la « masse », qu’elle campe en face de l’ « Homme » pour la combattre au nom de ce dernier. Quel est actuellement l’objet de la Critique ? — La masse, un être spirituel ! La Critique « apprendra à la connaître » et découvrira qu’elle est en contradiction avec l’Homme ; elle démontrera que la masse est inhumaine, et n’aura pas plus de peine à faire cette preuve qu’elle n’en a eu à démontrer que le divin et le national, autrement dit l’Eglise et l’Etat sont la négation même de l’humanité.\par
On définira la masse en disant qu’elle est le produit le plus important et le plus significatif de la Révolution ; c’est la foule abusée pour laquelle les illusions de la philosophie politique et surtout de toute la philosophie du {\scshape xviii}\textsuperscript{e} siècle n’ont abouti qu’à une cruelle déception. La Révolution a, par ses résultats, contenté les uns et laissé les autres mécontents. La partie satisfaite est la classe moyenne (bourgeois, philistins, etc.), la non satisfaite est — la masse. Et s’il en est ainsi, le Critique lui-même ne fait-il pas partie de la masse ?\par
Mais les non satisfaits tâtonnent encore en pleine obscurité, et leur déplaisir se traduit par une « mauvaise humeur sans bornes ». C’est de ceux-là que le Critique, non moins mécontent, doit à cette heure se rendre maître ; tout ce qu’il peut ambitionner et tout ce qu’il peut atteindre, c’est de tirer cet « être spirituel » qu’est la masse de sa mauvaise humeur et de l’ « élever », c’est-à-dire de lui donner la place qu’auraient dû légitimement lui assurer les trop  triomphants résultats de la Révolution ; il peut devenir la tête de la masse, son interprète par excellence. Aussi veut-il « combler l’abîme qui le sépare de la foule ». Il se distingue de ceux qui « prétendent élever les classes inférieures du peuple » en ce que ce n’est pas seulement elles, mais lui-même dont il doit apaiser les rancunes.\par
Toutefois, l’instinct ne le trompe pas, quand il tient la masse pour « naturellement opposée à la théorie » et lorsqu’il prévoit que « plus cette théorie prendra d’ampleur, plus la masse deviendra « compacte ». Car le Critique ne peut, avec son \emph{hypothèse} de l’Homme, ni éclairer ni satisfaire la masse. Si, en face de la Bourgeoisie, elle n’est déjà qu’une « couche sociale inférieure », une masse politiquement sans valeur, c’est en face de l’Homme, à plus forte raison, qu’elle va n’être plus qu’une simple « masse », un ramassis inhumain ou un troupeau de non-hommes.\par
Le Critique en arrive à nier toute humanité : parti de cette hypothèse que l’humain est le vrai, il se retourne lui-même contre cette hypothèse en contestant le caractère d’humanité à tout ce à quoi on l’avait jusqu’alors accordé. Il aboutit simplement à prouver que l’humain n’a d’existence que dans sa tête, tandis que l’inhumain est partout. L’inhumain est le réel, le partout existant ; en s’évertuant à démontrer qu’il n’est « pas humain », le Critique ne fait que formuler explicitement cette tautologie que l’inhumain n’est pas humain.\par
Quand l’inhumain se sera résolument tourné le dos à lui-même, que dira-t-il au critique qui le harcèle, avant de s’éloigner de lui sans s’être laissé émouvoir par ses objections ? Tu m’appelles inhumain, pourrait-il lui aire, et inhumain je suis en effet — pour toi ; mais je ne le suis que parce que tu m’opposes à l’humain et je n’ai pu avoir honte de moi qu’aussi longtemps que je me suis laissé ravaler à ce rôle de repoussoir. J’étais méprisable parce que je  cherchais mon « meilleur moi-même » hors de moi ; j’étais l’inhumain, parce que je rêvais de « l’humain » : j’imitais les pieux que tantalise leur « vrai moi » et qui restent toujours de « pauvres pécheurs » ; je ne me concevais que par contraste avec un autre ; cela suffit, je n’étais pas tout dans tout, je n’étais pas — \emph{Unique.} Mais aujourd’hui je cesse de me regarder comme l’inhumain, je cesse de me mesurer et de me laisser mesurer à l’aune de l’Homme, je cesse de m’incliner devant quelque chose de supérieur à moi, et ainsi — adieu, ô Critique humain ! J’ai été l’inhumain, mais je n’ai fait que passer par là, et je ne le suis plus : je suis l’Unique, je suis l’Egoïste, cet égoïste qui te fait horreur ; mais mon égoïsme n’est pas de ceux que l’on peut peser à la balance de l’humanité, du désintéressement, etc., c’est l’égoïsme de — l’Unique !\par
Il faut nous arrêter encore à un autre passage du même fascicule : « La Critique ne pose aucun dogme et ne veut rien connaître d’autre que les \emph{objets}. »\par
La Critique redoute d’être « dogmatique » et d’édifier des dogmes. Naturellement, car ce serait là passer aux antipodes de la critique, au dogmatisme, et, comme critique, de bon devenir mauvais, de désintéressé égoïste, etc. « Pas de dogmes ! » tel est — le sien. Car Critique et Dogmatique restent sur le même terrain, celui des \emph{pensées}. Comme le dogmatique, le critique a toujours pour point de départ une pensée, mais il se distingue de son adversaire en ce qu’il ne cesse de maintenir la pensée qui lui sert de principe sous l’empire d’un \emph{processus mental} qui ne lui permet d’acquérir aucune stabilité. Il fait simplement prévaloir en elle le processus intellectuel sur la foi, et le progrès dans le penser sur l’immobilité. Aux yeux du Critique, aucune pensée n’est assurée, car toute pensée est elle-même le penser ou l’esprit pensant.\par
C’est pourquoi, je le répète, le monde religieux, — qui est précisément le monde des pensées, — atteint  sa perfection dans la Critique, où le penser est supérieur à toute pensée, dont aucune ne peut se fixer « égoïstement ». Que deviendrait la « pureté de la critique », la pureté du penser, si une seule pensée pouvait échapper au procès intellectuel ? Cela nous explique le fait que le Critique lui-même se laisse aller, de temps à autre à railler doucement les idées d’Homme et d’Humanité : il pressent que ce sont là des pensées qui approchent de la cristallisation dogmatique. Mais il ne peut détruire une pensée tant qu’il n’a point découvert une pensée — supérieure, en laquelle la première se résout. Cette pensée supérieure pourrait s’appeler celle du mouvement de l’esprit ou du procès intellectuel, c’est-à-dire la pensée du penser ou de la critique.\par
La liberté de penser est en fait ainsi devenue complète ; c’est le triomphe de la liberté spirituelle, car les pensées individuelles, « égoïstes », perdent leur caractère dogmatique d’impératif. Une seule le conserve, le — dogme du penser libre ou de la critique.\par
Contre tout ce qui appartient au monde de la pensée, la Critique a le droit, c’est-à-dire la force, pour elle : elle est victorieuse. La Critique, et la Critique seule « domine notre époque ». Au point de vue de la pensée il n’est aucune puissance capable de surpasser la sienne, et c’est plaisir de voir avec quelle aisance ce dragon dévore comme en se jouant toute autre pensée ; tous ces vermisseaux de pensées se tordent, mais elle les broie malgré leurs contorsions et leurs « détours ».\par
Je ne suis pas un antagoniste de la critique, autrement dit je ne suis pas un dogmatique, et je ne me sens pas atteint par les dents du Critique. Si j’étais un dogmatique, je poserais en première ligne un dogme, c’est-à-dire une pensée, une idée, un principe, et je compléterais ce dogme en me faisant « systématique » et en bâtissant un système, c’est-à-dire une édifice de pensées.\par
 Réciproquement, si j’étais un Critique et le contradicteur du Dogmatique, je conduirais le combat du penser libre contre la pensée qui enchaîne, et je défendrais le penser contre le pensé. Mais je ne suis le champion ni du penser ni d’une pensée, car mon point de départ est Moi, qui ne suis pas plus une pensée que je ne consiste pas dans le fait de penser. Contre Moi, l’innommable, se brise le royaume des pensées, du penser et de l’esprit.\par
La Critique est la lutte du possédé contre la possession comme telle, contre toute possession ; elle naît de la conscience que partout règne la possession ou, comme l’appelle le Critique, le rapport religieux et théologique. Il sait que ce n’est pas seulement envers Dieu qu’on se comporte religieusement et qu’on agit en croyant ; il sait que l’on peut être également religieux et croyant en face d’autres idées, telles que Droit, Etat, Loi, etc. ; autrement dit, il reconnaît que la possession est partout et revêt toutes les formes. Il en appelle au penser contre les pensées ; mais moi je dis que seul le non-penser me sauve des pensées. Ce n’est pas le penser qui peut me délivrer de la possession, mais bien mon absence de pensée, ou Moi, l’impensable, l’insaisissable.\par
Un haussement d’épaules me rend les mêmes services que la plus laborieuse méditation, allonger mes membres dissipe l’angoisse des pensées, un saut, un bond renverse l’Alpe au monde religieux qui pèse sur ma poitrine, un hourra d’allégresse jette à terre des fardeaux sous lesquels on pliait depuis des années. Mais la signification formidable d’un cri de joie sans pensée ne pouvait être comprise tant que dura la longue nuit du penser et de la foi.\par
« Quelle frivolité, et quelle grossière frivolité, de vouloir, par un coq-à-l’âne, résoudre les plus difficiles problèmes et s’acquitter des plus vastes devoirs ! »\par
Mais as-tu des devoirs, si tu ne te les imposes pas ? Tant que tu t’en imposes tu ne peux en démordre, et  je ne nie pas, note le bien, que tu penses, et qu’en pensant tu crées des milliers de pensées. Mais toi qui t’es imposé ces devoirs, ne dois-tu pouvoir jamais les renverser ? Dois-tu y rester lié et doivent-ils devenir des devoirs absolus ?\par
Dernière remarque : on a fait au gouvernement un grief de recourir à la force contre la pensée, de braquer contre la presse les foudres policières de la censure et de transformer des batailles littéraires en combats personnels. Comme s’il ne s’agissait que des pensées et comme si l’on devait aux pensées du désintéressement, de l’abnégation et des sacrifices ! Ces pensées n’attaquent-elles pas les gouvernants eux-mêmes, et n’appellent-elles pas une riposte de l’égoïsme ? Et ceux qui pensent n’émettent-ils pas cette prétention \emph{religieuse} de voir honorer la puissance de la pensée, des idées ? Ceux auxquels ils s’adressent doivent succomber de leur plein gré et sans résistance, parce que la divine puissance de la pensée, la Minerve, combat aux côtés de leurs adversaires. Ce serait déjà là l’acte d’un possédé, un sacrifice religieux. Les gouvernants sont en vérité eux-mêmes pétris de préventions religieuses et guidés par la puissance d’une idée ou d’une croyance, mais ils sont en même temps des égoïstes inavoués, et c’est surtout lorsqu’on est en face de l’ennemi qu’éclate l’égoïsme latent : ils sont possédés quant à leur foi, mais quand il s’agit de la foi de leurs adversaires ils ne sont plus possédés et se retrouvent égoïstes. Si on veut leur faire un reproche, ce ne peut être que le reproche opposé, celui d’être possédés par leurs idées.\par
Aucune force égoïste, nulle puissance policière et rien de semblable ne doit entrer en jeu contre les pensées. C’est ce que croient les dévots de la pensée. Mais le penser et les pensées ne me sont pas sacrés ; lorsque je les attaque, c’est \emph{ma peau} que je défends contre elles. Il se peut que cette lutte ne soit pas raisonnable ; mais si la raison m’était un devoir, c’est ce  que j’ai de plus cher que je devrais, nouvel Abraham, lui sacrifier.\par
Dans le royaume de la Pensée, qui, comme celui de la foi, est le royaume céleste, celui-là a assurément tort qui recourt à la force \emph{sans pensée}, juste comme a tort celui qui dans le royaume de l’amour agit sans amour et celui qui, quoique chrétien, n’agit pas en chrétien : dans ces royaumes auxquels ils pensent appartenir tout en se soustrayant à leurs lois l’un comme l’autre sont des « pécheurs » et des « égoïstes ». Mais, d’autre part, ils y seraient des \emph{criminels} s’ils prétendaient en sortir et ne plus s’en reconnaître les sujets.\par
Il en résulte encore que dans leur lutte contre le gouvernement, ceux qui pensent ont pour eux le droit, autrement dit la force, tant qu’ils ne combattent que les pensées du gouvernement (ce dernier reste court et ne trouve à répondre rien qui vaille, littérairement parlant), tandis qu’ils ont tort, autrement dit ils sont impuissants, lorsqu’ils entreprennent de mener des pensées à l’assaut d’une puissance personnelle (la puissance égoïste ferme la bouche aux raisonneurs). Ce n’est pas sur le champ de bataille de la théorie qu’on peut remporter une victoire décisive, et la puissance sacrée de la pensée succombe sous les coups de l’égoïsme. Seul le combat égoïste, le combat entre égoïstes peut trancher un différend et tirer une question au clair.\par
Mais c’est là réduire le penser lui-même à n’être plus qu’affaire de bon plaisir égoïste, l’affaire de l’unique, ni plus ni moins qu’un simple passe-temps ou qu’une amourette ; c’est lui enlever sa dignité de ce dernier et suprême arbitre », et cette dépréciation, cette profanation du penser, cette égalisation du moi qui pense et du moi qui ne pense pas, cette grossière mais réelle « égalité » — il est interdit à la critique de l’instaurer, parce qu’elle n’est que la prêtresse du penser, et qu’elle n’aperçoit par delà le penser que — le déluge.\par
 La Critique soutient bien, par exemple, que la libre critique doit vaincre l’Etat, mais elle se défend contre le reproche que lui fait le gouvernement de l’Etat de « provoquer à l’indiscipline et à la licence » ; elle pense que l’indiscipline et la licence ne devraient pas triompher, et qu’elle seule le doit. C’est bien plutôt le contraire : ce n’est que par l’audace ennemie de toute règle et de toute discipline que l’Etat peut être vaincu.\par
Concluons : Nous en avons assez dit pour qu’il paraisse évident que la nouvelle évolution qu’a subie le Critique n’est pas une métamorphose et qu’il n’a fait que « rectifier quelques jugements hasardés » et « mettre un objet au point » ; il se vante quand il dit que« la Critique se critique elle-même » : elle ou plutôt il ne critique que les « erreurs » de la critique, et se borne à la purger de ses « inconséquences ». S’il voulait critiquer la Critique, il devrait commencer par examiner s’il y a réellement quelque chose dans l’hypothèse sur laquelle elle est bâtie.\par
Moi aussi je pars d’une hypothèse, attendu que je \emph{me} suppose ; mais mon hypothèse ne tend pas à se parfaire comme « l’Homme tend à sa perfection », elle ne me sert qu’à en jouir et à la consommer. Je ne me nourris précisément que de cette seule hypothèse, et je ne suis que pour autant que je m’en nourris. Aussi cette hypothèse n’en est-elle pas une ; étant l’Unique, je ne sais rien de la dualité d’un moi postulant et d’un moi postulé (d’un moi « imparfait » et d’un moi « parfait » ou Homme). Je ne me \emph{suppose} pas, parce qu’à chaque instant je me \emph{pose} ou me crée ; je ne suis que parce que je suis posé et non supposé, et, encore une fois, je ne suis posé que du moment où je me pose, c’est-à-dire que je suis à la fois le créateur et la créature.\par
Si les hypothèses qui ont eu cours jusqu’à présent doivent se désorganiser et disparaître, elles ne doivent pas se résoudre simplement en une hypothèse supérieure,  telle que la pensée ou le penser même, la Critique.\par
Leur destruction doit m’être profitable \emph{à Moi}, sinon la solution nouvelle qui naîtra de leur mort rentrerait dans la série innombrables de toutes celles qui jusqu’à présent n’ont jamais déclaré fausses les anciennes vérités et fait crouler des hypothèses depuis longtemps acceptées que pour édifier sur leurs ruines le trône d’un \emph{étranger}, d’un \emph{intrus :} Homme, Dieu, Etat ou Morale.
 \section[{B. Deuxième partie. Moi}]{B. Deuxième partie \\
Moi}\renewcommand{\leftmark}{B. Deuxième partie \\
Moi}

 \noindent {\small A l’aube des temps nouveaux se dresse l’Homme-Dieu. A leur déclin, le Dieu seul se sera-t-il évanoui et l’Homme-Dieu peut-il vraiment mourir si le Dieu seul meurt en lui ? On ne s’est pas posé cette question ; on crut avoir tout fait lorsqu’on eut de nos jours victorieusement mené à bout l’œuvre de lumière et vaincu le Dieu ; on ne remarqua pas que l’Homme n’a tué le Dieu que pour devenir à son tour « le seul Dieu qui règne dans les cieux ». L’\emph{au-delà extérieur} est balayé et l’œuvre colossale de la philosophie est accomplie ; mais l’\emph{au-delà intérieur} est devenu un nouveau Ciel et nous appelle à de nouveaux assauts : le Dieu a dû faire place à — l’Homme et non — à Nous. Comment pouvez-vous croire que l’Homme-Dieu soit mort, aussi longtemps qu’en lui, outre le Dieu, l’Homme ne sera pas mort aussi ?}\par
 \subsection[{B.I. La propriété (l’Individualité)}]{B.I. \\
La propriété (l’Individualité)}\phantomsection
\label{p17}
\noindent « L’esprit n’aspire-t-il pas à la Liberté ? » — Hélas ! ce n’est pas mon seul Esprit, c’est toute ma chair qui brûle sans cesse du même désir ! Lorsque, devant la cuisine embaumée du château, mon nez parle à mon palais des plats savoureux qui s’y préparent, je trouve mon pain sec terriblement amer ; lorsque mes yeux vantent à mon dos calleux les moelleux coussins sur lesquels il lui serait bien plus doux de s’étendre que sur sa paille foulée, le dépit et la rage me saisissent ; lorsque... mais à quoi bon évoquer plus de douleurs ? — Et c’est cela que tu appelles ton ardente soif de liberté ? De quoi donc veux-tu être \emph{délivré ?} De ton pain de munition et de ton lit de paille ? Hé bien ! jette-les au feu ! Mais tu n’en serais guère plus avancé ; ce que tu veux, c’est plutôt la liberté de jouir d’une bonne table et d’un bon lit. Les hommes te le permettront-ils ? Te donneront-ils cette « liberté » ? Tu n’attends pas cela de leur amour des hommes, car tu sais qu’ils pensent tous — comme toi : chacun est pour soi-même le prochain !\par
Comment feras-tu donc pour jouir de ces mets et de ces coussins qui te font envie ? Il n’y a pas d’autre moyen que d’en faire ta — propriété !\par
 Lorsque tu y penses bien, ce que tu veux, ce n’est pas la liberté d’avoir toutes ces belles choses, car cette liberté ne te les donne pas encore ; ce que tu veux, ce sont ces choses elles-mêmes ; tu veux les appeler \emph{tiennes} et les posséder comme \emph{ta propriété}. A quoi te sert une liberté, si elle ne te donne rien ? D’ailleurs si tu étais délivré de tout, tu n’aurais plus rien, car la liberté est, par essence, vide de tout contenu. Elle n’est qu’une vaine permission pour celui qui ne sait pas s’en servir ; et si je m’en sers, la manière dont j’en use ne dépend que de moi, de mon individualité.\par
Je ne trouve rien à redire à la liberté, mais je te souhaite plus que de la liberté ; tu ne devrais pas être tout bonnement \emph{quitte} de ce que tu ne veux pas, tu devrais aussi \emph{avoir} ce que tu veux ; il ne te suffit pas d’être « libre », tu dois être plus, tu dois être « propriétaire ». Tu veux être libre ? — Et de quoi ? De quoi ne peut-on s’affranchir ? On peut secouer le joug du servage, du pouvoir souverain, de l’aristocratie et des princes, on peut secouer la domination des appétits et des passions et jusqu’à l’empire de la volonté propre et personnelle ; l’abnégation totale, le complet renoncement ne sont que de la liberté, liberté vis-à-vis de soi-même, de son arbitre et de ses déterminations. Ce sont nos efforts vers la liberté comme vers quelque chose d’absolu, d’un prix infini, qui nous dépouillèrent de l’individualité en créant l’abnégation.\par
Plus je suis libre, plus la contrainte s’élève comme une tour devant mes yeux et plus je me sens impuissant. Le sauvage, dans sa simplicité, ne connaît encore rien des barrières qui enferment le civilisé : il lui semble qu’il est plus libre que ce dernier. Plus j’acquiers de liberté, plus je me crée de nouvelles limites et de nouveaux devoirs. Ai-je inventé les chemins de fer, aussitôt je me sens faible parce que je ne puis encore fendre les airs comme l’oiseau ; ai-je résolu un problème dont l’obscurité angoissait mon esprit,  déjà mille autres questions surgissent, mille énigmes nouvelles embarrassent mes pas, déconcertent mes regards et me font plus douloureusement sentir les bornes de ma liberté. « Ainsi, ayant été affranchis du péché, vous êtes devenus \emph{esclaves} de la justice\footnote{ \noindent Ep. aux Romains, {\scshape vi}, 18.
 } ».\par
Les républicains, dans leur large liberté, ne sont-ils pas esclaves de la Loi ? Avec quelle avidité les cœurs vraiment chrétiens désirèrent de tout temps « être libres », et combien il leur tardait de se voir délivrés des « liens de cette vie terrestre » ! Ils cherchaient des yeux la terre promise de la liberté. « La Jérusalem de là-haut est libre, et c’est elle qui est notre mère à tous » (Galates, {\scshape iv}, 26).\par
Etre libre de quelque chose signifie simplement en être quitte ou exempt. « Il est libre de tout mal de tête » égale : « il en est exempt, il n’a pas mal à la tête » ; « il est libre de préjugés » égale : « il n’en a pas » ou « il s’en est débarrassé ». La liberté qu’entrevit et salua le Christianisme, nous la complétons par la négation qu’exprime le « sans », le « in » négatif : \emph{sans} péché, \emph{in}ocent ; \emph{sans} Dieu, \emph{impie ; sans }mœurs, immoral.\par
La liberté est la doctrine du Christianisme : « Vous êtes, chers frères, appelés à la liberté\footnote{ \noindent 1\textsuperscript{e} Ep. de Pierre, {\scshape ii}, 16.
 } » ; « réglez donc vos paroles et vos actions comme devant être jugées par la loi de liberté\footnote{ \noindent Ep. de Jacques, {\scshape ii}, 12.
 }. »\par
Devons-nous rejeter la liberté parce qu’elle se trahit comme un idéal chrétien ? Non, il s’agit de ne rien perdre, pas plus la liberté qu’autre chose ; seulement elle doit nous devenir propre, ce qui lui est impossible sous sa forme de liberté.\par
Quelle différence entre la liberté et l’Individualité ! On peut être \emph{sans} bien des choses mais on ne peut être \emph{sans} rien ; on peut être libre de bien des choses,  mais non être libre de tout. L’esclave même peut être intérieurement libre, mais seulement vis-à vis de certaines choses et non de toutes ; comme esclave, il n’est pas \emph{libre} vis-à-vis du fouet, des caprices impérieux du maître, etc. « La Liberté n’existe que dans le royaume des songes ! » L’Individualité, c’est-à-dire ma propriété, est au contraire toute mon existence et ma réalité, c’est moi-même. Je suis libre vis-à-vis de ce que je n’ai pas ; je suis propriétaire de ce qui est en mon \emph{pouvoir}, ou de ce dont je suis \emph{capable}. Je suis en tout temps et en toutes circonstances \emph{à moi}, du moment que j’entends être à moi et que je ne me prostitue pas à autrui. L’état de liberté, je ne puis vraiment le \emph{vouloir}, vu que je ne puis pas le réaliser, le créer ; tout ce que je puis faire, c’est le désirer et y — rêver, car il reste un idéal, un fantôme. Les chaînes de la réalité infligent à chaque instant à ma chair les plus cruelles meurtrissures, mais je demeure \emph{mon bien propre}. Livré en servage à un maître, je n’ai en vue que moi et mon avantage ; ses coups, en vérité, m’atteignent, je n’en suis pas \emph{libre}, mais je ne les supporte que dans \emph{mon propre intérêt}, soit que je veuille le tromper par une feinte soumission, soit que je craigne de m’attirer pis par ma résistance. Mais comme je n’ai en vue que moi et mon intérêt personnel, je saisirai la première occasion qui se présentera et j’écraserai mon maître. Et si je suis alors \emph{libre} de lui et de son fouet, ce ne sera que la conséquence de mon égoïsme antérieur.\par
On me dira peut-être que, même esclave, j’étais « libre », que je possédais la liberté « en soi », la liberté « intérieure ». Malheureusement, être « libre en soi » n’est pas être « réellement libre », et « intérieur » n’est pas « extérieur ». Ce que j’étais, en revanche, c’est \emph{mien}, \emph{mien propre}, et je l’étais totalement, extérieurement comme intérieurement. Sous la domination d’un maître cruel, mon corps n’est pas « libre » vis-à-vis de la torture et des coups de fouet ;  mais ce sont \emph{mes} os qui gémissent dans la torture, ce sont \emph{mes} fibres qui tressaillent sous les coups, et \emph{je }gémis parce que \emph{mon} corps gémit. Si je soupire et si je frémis, c’est que je suis encore mien, que je suis toujours mon bien. Ma jambe n’est pas « libre » sous le bâton du maître, mais elle demeure \emph{ma} jambe et ne peut m’être arrachée. Qu’il la coupe donc, et qu’il dise s’il tient encore ma jambe ? Il n’aura plus dans la main que le — cadavre de ma jambe, et ce cadavre n’est pas plus ma jambe qu’un chien mort n’est un chien : un chien a un cœur qui bat, et ce qu’on appelle un chien mort n’en a plus et n’est plus un chien.\par
Dire qu’un esclave peut être, malgré tout, intérieurement libre, c’est, en réalité, émettre la plus vulgaire et la plus triviale des banalités. Qui pourrait en effet s’aviser de soutenir qu’un homme peut n’avoir \emph{aucune} liberté ? Que je sois le plus rampant des valets, ne serai-je pas cependant libre d’une infinité de choses ? De la foi en Zeus, par exemple, ou de la soif de renommée, etc. ? Et pourquoi donc un esclave fouetté ne pourrait-il être, lui aussi, « intérieurement libre » de toute pensée peu chrétienne, de toute haine pour ses ennemis, etc. ? Il est, dans ce cas, « chrétiennement libre », pur de tout ce qui n’est pas chrétien ; mais est-il absolument libre, est-il délivré de tout, de l’illusion chrétienne, de la douleur corporelle, etc ?\par
Il peut sembler, au premier abord, que tout ceci s’attaque au nom plutôt qu’à la chose. Mais le nom est-il donc chose si indifférente, et n’est-ce pas toujours par un mot, un schibboleth, que les hommes ont été inspirés et — trompés ? Il existe d’ailleurs entre la Liberté et la Propriété ou l’Individualité, un gouffre plus profond qu’une pure différence de mots.\par
Tout le monde tend vers la liberté, tous appellent son règne à grands cris. Qui n’a été bercé par toi, ô rêve enchanteur d’un « règne de la Liberté », d’une radieuse a humanité libre » ? Ainsi donc, les  hommes seront libres, complètement libres, affranchis de toute contrainte ? Vraiment de toute contrainte ? Il ne pourront même plus se contraindre eux-mêmes ? — Ah, si, parfaitement, mais cela n’est pas une contrainte ! Ce dont ils seront délivrés, c’est de la foi religieuse, des rigoureux devoirs de la moralité, de la sévérité de la loi, de... — Voilà une terrible absurdité ! Mais \emph{de quoi} donc devez-vous et de quoi ne devez-vous pas être libres ?\par
Le beau rêve s’envole ; réveillé, on se frotte les yeux et on regarde fixement le prosaïque questionneur. « De quoi les hommes doivent être libres ? » — De la crédulité aveugle ! dit l’un. — Eh ! s’écrie un autre, toute foi est aveugle ! C’est de la foi qu’ils doivent être délivrés. — Non, non, pour l’amour de Dieu, réplique le premier, ne rejetez pas loin de vous toute croyance, mais mettez un terme à la puissance de la brutalité. — Un troisième prend la parole : Nous devons, dit-il, fonder la république et nous affranchir de tous les maîtres. — Nous n’en serons pas plus avancés, répond un quatrième ; nous n’arriverons qu’à nous donner un nouveau maître, une « majorité régnante » ; débarrassons-nous plutôt de cette intolérable inégalité... — O malheureuse égalité, j’entends de nouveau tes grossières clameurs ! Quel beau rêve je faisais naguère d’une \emph{liberté} paradisiaque, et quelle impudence, quelle licence effrénée troublent maintenant mon Eden de leurs sauvages hurlements ! Ainsi s’exclame le premier de nos interlocuteurs ; il se redresse et brandit son sabre contre cette « liberté sans mesure ». Bientôt, nous n’entendrons plus que le cliquetis des épées rivales de tous ces amants de la liberté.\par
Les luttes pour la liberté n’ont eu de tout temps pour objectif que la conquête d’une liberté \emph{déterminée}, comme par exemple la liberté religieuse : l’homme religieux voulait être libre et indépendant. De quoi ? De la foi ? Nullement, mais des inquisiteurs de la foi.  Il en est de même aujourd’hui de la liberté « politique ou civile ». Le citoyen veut être affranchi non de sa qualité de citoyen, mais de l’oppression des fermiers et des traitants, de l’arbitraire royal, etc. Le comte de Provence émigra de France précisément au moment où cette même France tentait d’inaugurer le règne de la liberté et voici ses paroles : « Ma captivité m’était devenue insupportable ; je n’avais qu’une passion : conquérir la — \emph{liberté ;} je ne pensais qu’à elle. »\par
L’aspiration vers une liberté déterminée implique toujours la perspective d’une nouvelle domination ; la Révolution pouvait bien « inspirer à ses défenseurs le sublime orgueil de combattre pour \emph{la} Liberté », mais elle n’avait cependant en vue \emph{qu’une certaine} liberté ; aussi en résulta-t-il une \emph{domination} nouvelle, celle de la Loi.\par
De la liberté, vous en voulez tous ; veuillez donc \emph{la }liberté ! Pourquoi marchander pour un plus ou un moins ? La liberté ne peut être que la liberté tout entière ; un bout de liberté n’est pas \emph{la} liberté. Vous n’espérez pas qu’il soit possible d’atteindre la liberté totale, la liberté vis-à-vis de tout ; vous pensez que c’est folie de la seulement souhaiter ? — Cessez donc de poursuivre un fantôme et tournez vos efforts vers un but meilleur que — \emph{l’inaccessible.}\par
« Non ! Rien ne vaut la liberté ! »\par
Qu’aurez-vous donc, quand vous aurez la liberté ? — bien entendu, je parle ici de la liberté complète et non de vos miettes de liberté ? — Vous serez débarrassés de tout, d’absolument tout ce qui vous gêne, et rien dans la vie ne pourra plus vous gêner et vous importuner. Et pour l’amour de qui voulez-vous être délivrés de ces ennuis ? Pour l’amour \emph{de vous-mêmes, }parce qu’ils contrecarrent vos désirs. Mais supposez que quelque chose ne vous soit pas pénible, mais au contraire agréable ; — par exemple, le regard, très doux sans doute, mais irrésistiblement impérieux  de votre maîtresse : voudrez-vous aussi en être débarrassés ? Non, et vous renoncerez sans regret à votre liberté. Pourquoi ? De nouveau \emph{pour l’amour de vous-mêmes}. Ainsi donc vous faites de \emph{vous} la mesure et le juge de tout. Vous mettez volontiers de côté votre liberté, lorsque la non-liberté, le doux « \emph{esclavage }d’amour » a pour \emph{vous} plus de charmes, et vous la reprenez à l’occasion lorsqu’elle recommence à vous plaire, en supposant, ce qui n’est pas à examiner ici, que d’autres motifs (par exemple religieux) ne vous en détournent pas.\par
Pourquoi donc ne pas prendre votre courage à deux mains et ne pas faire résolument de vous le centre et le principe ? Pourquoi bayer à la liberté, votre rêve ? Etes-vous votre rêve ? Ne prenez pas conseil de vos rêves, de vos imaginations, de vos pensées, car tout cela n’est que de la « creuse théorie ». Interrogez-vous, et faites cas de vous, — cela est \emph{pratique} et il ne vous déplaît pas d’être pratiques.\par
Mais l’un se demande ce que dira son Dieu (naturellement son Dieu est ce qu’il désigne sous ce nom) ; un autre se demande ce que diront son sens moral, sa conscience, son sentiment du devoir ; un troisième s’inquiète de ce que les gens vont penser, et quand chacun a interrogé son oracle (« les gens » sont un oracle aussi sûr et plus compréhensible que celui de là-haut : \emph{vox populi}, \emph{vox dei}), chacun obéit à la volonté de son maître et n’écoute plus le moins du monde ce que \emph{lui-même} aurait pu dire et décider.\par
Adressez-vous donc à vous-mêmes, plutôt qu’à vos dieux ou à vos idoles : découvrez en vous ce qui y est caché, amenez-le à la lumière, et révélez-vous !\par
Comme chacun n’agit que d’aprés lui-même et ne s’inquiète de rien au delà, les Chrétiens se sont imaginés qu’il ne pouvait en être autrement de Dieu. Il agit « comme il lui plaît ». Et l’homme, l’insensé, qui pourrait faire de même, doit bien s’en garder et doit agir « comme il plaît à Dieu ». — Vous dites que Dieu  se conduit d’après des lois éternelles ? Vous pouvez tout aussi bien le dire de moi : moi non plus je ne puis sortir de ma peau, mais c’est dans toute ma nature, c’est-à-dire en moi, qu’est écrite ma loi.\par
Mais il suffit de vous rappeler à vous pour vous plonger tous dans le désespoir. « Que suis-je ? » se demande chacun de vous. Un abîme où bouillonnent, sans règle et sans loi, les instincts, les appétits, les désirs, les passions ; un chaos sans clarté et sans étoile ! Si je n’ai d’égards ni pour les commandements de Dieu, ni pour les devoirs que me prescrit la morale, ni pour la voix de la raison qui, dans le cours de l’histoire, a érigé en loi après de dures expériences le meilleur et le plus sage, si je n’écoute que moi seul, comment oserais-je compter sur une réponse judicieuse ? Mes passions me conseilleront précisément les pires folies ! — Ainsi, chacun de vous se prend pour le — Diable ; et cependant s’il se tenait simplement (pour autant que la religion, etc., ne l’en détournent pas), pour une bête, il remarquerait très aisément que la bête, qui n’a pourtant d’autre conseiller que son instinct, ne court pas droit à l’ « absurde » et marche très posément.\par
Mais l’habitude de penser religieusement nous a si bien faussé l’esprit, que notre nudité, notre naturel nous épouvantent ; elle nous a tellement dégradés que nous nous imaginons naître diables, souillés par le péché originel. Naturellement, vous allez immédiatement penser que votre devoir exige que vous pratiquiez le « bien », la morale, la justice. Et comment, si c’était à vous seul que vous demandiez ce que vous avez à faire, pourrait résonner en vous la bonne voix, la voix qui indique le chemin du bien, du juste, du vrai, etc. ? Comment parlent Dieu et Bélial ?\par
Que penseriez-vous, si quelqu’un vous répondait que Dieu, la conscience, le devoir, la loi, etc., sont des mensonges dont on vous a farci la tête et le cœur jusqu’à vous hébéter ? Et si quelqu’un vous demandait  d’où vous savez de science si certaine que la voix de la nature est une tentatrice ? Et s’il vous instigait à renverser les rôles et à tenir franchement la voix de Dieu et de la conscience pour l’œuvre du Diable ? Il y a des hommes assez scélérats pour cela ; comment en viendriez-vous à bout ? Vous ne pourriez en appeler contre eux à vos prêtres, à vos aïeux et à vos honnêtes gens, car ils les regardent justement comme vos séducteurs : ce sont eux, disent-ils, qui ont véritablement corrompu et souillé la jeunesse, en semant à pleines mains l’ivraie du mépris de soi et du respect des dieux ; ce sont eux qui ont envasé les jeunes cœurs et abruti les jeunes cerveaux.\par
Mais ils vont plus loin et vous demandent : pour l’amour de quoi vous inquiétez-vous de Dieu et des autres commandements ? Vous savez bien que vous n’agissez pas par pure complaisance envers Dieu ; pour l’amour de qui prenez-vous donc tant de soucis ? De nouveau — \emph{pour l’amour de vous}. Ici encore vous êtes le principal et chacun doit se dire : je suis pour moi tout, et tout ce que je fais, je le fais \emph{à cause de moi.} S’il vous arrivait, ne fût-ce qu’une fois, de voir clairement que le Dieu, la loi, etc., ne font que vous nuire, qu’ils vous amoindrissent et vous corrompent, il est certain que vous les rejetteriez loin de vous, comme les Chrétiens renversèrent jadis les images de l’Apollon et de la Minerve et la morale païenne. Il est vrai qu’ils dressèrent à leur place le Christ, et plus tard la Marie, ainsi qu’une morale chrétienne, mais ils ne le firent qu’en vue du salut de leur âme, c’est-à-dire, encore une fois, par égoïsme ou individualisme.\par
Et ce fut ce même égoïsme, ce même individualisme qui les débarrassa et les \emph{affranchit} de l’antique monde des dieux. L’individualité fut la source d’une \emph{liberté} nouvelle, car l’individualité est l’universelle créatrice ; et, depuis longtemps déjà, on regarde une de ses formes, le génie (qui toujours est singularité ou  originalité) comme le créateur de toutes les œuvres qui marquent dans l’histoire du monde.\par
Si la liberté est le but de vos efforts, sachez vouloir sans vous arrêter à mi-chemin ! Qui donc doit être libre ? Toi, Moi, Nous ! Et libres de quoi ? De tout ce qui n’est pas Toi, Moi, Nous ! Je suis le noyau, je suis l’amande qui doit être — délivrée de toutes ses enveloppes, de la coquille où elle est à l’étroit. Et que restera-t-il, quand je serai délivré de tout ce qui n’est pas moi ? Moi, toujours et rien que Moi ! Mais la liberté n’a plus rien à voir à ce Moi ; que deviendrai-je une fois libre ? Sur ce sujet, la liberté reste muette ; elle est comme nos lois pénales, qui, à l’expiation de sa peine, ouvrent au prisonnier la porte de la geôle et lui disent « va-t-en. »\par
Cela étant, pourquoi, puisque si je recherche la liberté ce n’est que dans mon intérêt, pourquoi, dis-je, ne pas me regarder comme le commencement, le milieu et la fin ? Est-ce que je ne vaux pas plus que la liberté ? N’est-ce pas moi qui me fais libre, et ne suis-je donc pas le premier ? Même esclave, même couvert de mille chaînes, j’existe ; je ne suis pas, comme la liberté, quelque chose à venir qu’on espère, je suis — actuel.\par
Pensez-y mûrement et décidez si vous inscrirez sur votre bannière la « Liberté », ce rêve, ou l’ « Egoïsme », l’ « Individualisme », cette résolution. La liberté attise vos \emph{colères} contre tout ce qui n’est pas vous ; l’égoïsme vous appelle à la jouissance de vous-même, à la \emph{joie} d’être ; la liberté est et demeure une aspiration, une élégie romantique, un espoir chrétien d’avenir et d’au-delà ; l’individualité est une réalité, qui \emph{d’elle-même} supprime toute entrave à la liberté, pour autant qu’elle vous gêne et vous barre la route. Vous ne tenez pas à être délivré de ce qui ne vous fait aucun tort, et si quelque chose commence à vous gêner, sachez que « c’est \emph{vous} qu’il faut écouter, plutôt que les hommes ».\par
 La liberté vous dit : faites-vous libres, allégez-vous de tout ce qui vous pèse ; elle ne vous enseigne pas ce que vous êtes vous-mêmes. « Libre ! Libre ! » est son cri de ralliement, et vous pressant avidement sur ses pas, vous vous faites libres de vous-mêmes, « vous faites abnégation de vous-mêmes ». L’Individualité, elle, vous rappelle à vous, elle vous crie « reviens à toi ! » Sous l’égide de la liberté, vous êtes quittes de bien des choses, mais voici que quelque chose vous opprime de nouveau : « Affranchissez-vous du mal, le mal est resté ». Comme \emph{individu}, vous êtes \emph{réellement libres} de tout ; ce qui vous reste inhérent, vous l’avez \emph{accepté} de votre plein choix et de votre plein gré. L’ « \emph{individuel} » est foncièrement libre, \emph{libre de naissance ;} le « libre » n’est qu’\emph{en mal de liberté}, c’est un rêveur et un enthousiaste.\par
Le premier est originellement, essentiellement libre, parce qu’il ne reconnaît que \emph{lui ;} il n’a pas à commencer par s’affranchir, parce que, \emph{a priori}, il rejette tout hors de lui, parce qu’il n’apprécie que lui, ne met rien au-dessus de lui ; bref, parce qu’il part de \emph{lui-même} et arrive à \emph{lui-même}. Dès l’enfance, contenu par le respect, il lutte déjà pour s’affranchir de cette contrainte. L’individualité se met à l’œuvre chez le petit égoïste et lui procure ce qu’il désire, la — liberté.\par
Des siècles de culture ont obscurci à vos yeux votre vraie signification et vous ont fait croire que vous n’êtes pas des égoïstes, que votre \emph{vocation} est d’être des idéalistes, « de braves gens ». Secouez tout cela ! Ne cherchez pas dans l’abnégation une liberté qui vous dépouille de vous-mêmes, mais cherchez vous \emph{vous-mêmes}, devenez des égoïstes, et que chacun de vous devienne un \emph{Moi tout-puissant.} Plus nettement : refaites connaissance avec vous-mêmes, apprenez à connaître ce que vous êtes réellement et abandonnez vos efforts hypocrites, votre manie insensée d’être autre chose que ce que vous êtes. J’appelle vos efforts  de l’hypocrisie parce que, pendant des siècles, vous êtes restés des égoïstes endormis, qui se trompent eux-mêmes, et dont la démence fait des \emph{héautontimoroumènes} et leurs propres bourreaux. Jamais encore une religion n’a pu se passer de promesses payables en ce monde-ci ou dans l’autre (vivre longuement, etc.) car l’homme exige un \emph{salaire} et ne fait rien \emph{pro deo. } — Mais, cependant, « on fait le bien pour l’amour du bien » sans avoir en vue une récompense ? — Comme si la récompense n’était pas contenue dans la satisfaction même que procure une bonne action ! La religion elle-même est fondée sur notre égoïsme, et elle — l’exploite ; basée sur nos \emph{appétits}, elle étouffe les uns pour satisfaire les autres. Elle nous donne le spectacle de l’égoïste leurré, de l’égoïste qui ne se satisfait pas, mais qui satisfait un de ses appétits, par exemple la soif de félicité. La religion me promet le « bien suprême » ; pour le gagner, je cesse de prêter l’oreille à mes autres désirs et je ne les assouvis plus. — Tous vos actes, tous vos efforts sont de l’égoïsme inavoué, secret, caché, dissimulé. Mais comme cet égoïsme dont vous ne voulez pas convenir et que vous célez à vous-mêmes ne s’étale ni ne s’affiche et reste inconscient, ce n’est \emph{pas de l’égoïsme}, mais de la servitude, du dévouement, de l’abnégation. Vous êtes égoïstes et vous ne l’êtes pas, parce que vous reniez l’égoïsme. Et c’est précisément vous qui avez voué ce mot « égoïste » à l’exécration et au mépris, vous à qui il s’applique si bien !\par
J’assure ma liberté contre le monde en raison de ce que je m’approprie le monde, quelque moyen d’ailleurs que j’emploie pour le conquérir et le faire mien : persuasion, prière, ordre catégorique, ou même hypocrisie, fourberie, etc. Les moyens auxquels je m’adresse ne dépendent que de ce que je suis. Suis-je faible, je ne disposerai que de moyens faibles, tels que ceux que j’ai cités, et qui suffisent cependant pour venir à bout d’une certaine partie du monde. Aussi la  tromperie, la duplicité et le mensonge paraissent-ils plus noirs qu’ils ne sont. Qui refuserait de tromper la police et de tourner la loi, qui ne prendrait bien vite un air de candide loyauté lorsqu’il rencontre les sergents, pour cacher quelque illégalité qu’il vient de commettre etc. ? Celui qui ne le fait pas se laisse violenter, c’est un \emph{lâche} — par scrupules. Je sens déjà que ma liberté est rabaissée, lorsque je ne puis imposer ma volonté à un autre (que cet autre soit sans volonté, comme un rocher, ou qu’il veuille, comme un gouvernement, un individu, etc.) ; mais c’est renier mon individualité que de m’abandonner moi-même à autrui, de céder, de plier, de renoncer, par \emph{soumission} et \emph{résignation.} Abandonner une manière de faire qui ne conduit pas au but, ou quitter un mauvais chemin, c’est tout autre chose que de me soumettre. Je tourne un rocher qui barre ma route jusqu’à ce que j’aie assez de poudre pour le faire sauter ; je tourne les lois de mon pays tant que je n’ai pas la force de les détruire. Si je ne puis saisir la lune, doit-elle pour cela m’être « sacrée », m’être une Astarté ? Si je pouvais seulement t’empoigner, je n’hésiterais certes pas, et si je trouvais un moyen de parvenir jusqu’à toi, tu ne me ferais pas peur ! Tu es l’inaccessible, mais tu ne le resteras que jusqu’à ce que j’aie conquis la puissance qu’il faut pour t’atteindre, et ce jour-là tu seras \emph{mienne ;} je ne m’incline pas devant toi, attends que mon heure soit venue !\par
C’est ainsi que depuis toujours ont agi les hommes forts. Les « soumis » avaient-ils mis bien haut la puissance de leur maître, et, prosternés, exigeaient-ils de tous l’adoration, venait un de ces fils de la nature qui refusait de s’humilier et chassait la puissance suppliée de son inaccessible Olympe. Il criait au soleil : « arrête-toi » et faisait tourner la terre, les soumis devaient bien s’y résigner ; il mettait la cognée au tronc des chênes sacrés, et les « soumis » s’étonnaient de ne pas le voir dévoré par le feu céleste ;  il renversait le Pape du siège de saint Pierre, et les « soumis » ne savaient pas l’en empêcher ; il rase aujourd’hui l’auberge de la grâce de Dieu, les « soumis » croassent, mais ils finiront par se taire, impuissants.\par
Ma liberté ne devient complète que lorsqu’elle est ma — \emph{puissance ;} c’est par cette dernière seule que je cesse d’être simplement libre pour devenir individu et possesseur. Pourquoi la liberté des peuples est-elle « un vain mot » ? Parce qu’ils n’ont pas de puissance ! Le souffle d’un Moi vivant suffit pour renverser des peuples, que ce soit le souffle d’un Néron, d’un empereur de Chine ou d’un pauvre écrivain. Pourquoi les Chambres a..... languissent-elles inutilement en rêvant à la liberté et se font-elles rappeler à l’ordre par les ministres ? Parce qu’elles ne sont pas « puissantes ». La force est une belle chose, et utile dans bien des cas, car « on va plus loin avec une main pleine de force qu’avec un sac plein de droit ». Vous aspirez à la liberté ? Fous ! Ayez la force, et la liberté viendra toute seule. Voyez : celui qui a la force est « au-dessus des lois » ! Cette remarque est-elle à votre goût, gens « légaux » ? Mais vous n’avez pas de goût !\par
Partout retentissent des appels à la « liberté ». Mais sent-on et sait-on ce que signifie une liberté donnée, octroyée ? On ne reconnaît pas que toute liberté est, dans la pleine acception du mot, essentiellement — un auto-affranchissement, c’est-à-dire que je ne puis avoir qu’autant de liberté que m’en crée mon individualité. Les moutons seront bien avancés que personne ne rogne leur franc-parler ! Il en restent au bêlement. Donnez à celui qui, au fond du cœur, est mahométan, juif ou chrétien la permission de dire ce qui lui passe par la tête : il bêlera comme avant. Mais si certains vous ravissent la liberté de parler et d’écouter, c’est qu’ils voient très nettement leur avantage actuel, car vous pourriez peut-être bien être  tentés de dire ou d’entendre quelque chose qui ébrécherait le crédit de ces « Certains ».\par
S’ils vous donnent cependant la liberté, ce ne sont que des fripons qui donnent plus qu’ils n’ont. Ils ne vous donnent rien de ce qui leur appartient, mais bien une marchandise volée ; ils vous donnent votre propre liberté, la liberté que vous auriez pu prendre vous-mêmes, et s’ils vous la \emph{donnent}, ce n’est que pour que vous ne la preniez pas et pour que vous ne demandiez pas par dessus le marché des comptes aux voleurs. Rusés comme ils le sont, ils savent bien qu’une liberté qui se donne (ou qui s’octroie) n’est pas la liberté et que seule la liberté qu’on \emph{prend}, celle des égoïstes, vogue à pleines voiles. Une liberté reçue en cadeau cargue ses voiles dès que la tempête s’élève — ou que le vent tombe ; elle doit toujours être poussée par une brise douce et modérée.\par
Ceci nous montre la différence entre l’auto-affranchissement et l’émancipation. Quiconque aujourd’hui « appartient à l’opposition » réclame à cor et à cri l’ « émancipation ». Les princes doivent proclamer leurs peuples « majeurs », c’est-à-dire les émanciper ! Si par votre façon de vous comporter vous êtes majeurs, vous n’avez que faire d’être émancipés ; si vous n’êtes pas majeurs, vous n’êtes pas dignes de l’émancipation, et ce n’est pas elle qui hâtera votre maturité. Les Grecs majeurs chassèrent leurs tyrans et le fils majeur se détache de son père ; si les Grecs avaient attendu que leurs tyrans leur fissent la grâce de les mettre hors tutelle, ils auraient pu attendre longtemps ; le père dont le fils ne veut pas devenir majeur le jette, s’il est sensé, à la porte de chez lui, et l’imbécile n’a que ce qu’il mérite.\par
Celui à qui on a accordé la liberté n’est qu’un esclave affranchi, un \emph{libertinus}, un chien qui traîne un bout de chaîne ; c’est un serf vêtu en homme libre comme l’âne sous la peau du lion. Des Juifs qu’on a émancipés n’en valent pas plus pour cela, ils  sont simplement soulagés en tant que Juifs ; il faut toutefois reconnaître que celui qui allège leur sort est plus qu’un Chrétien religieux, car ce dernier ne pourrait le faire sans inconséquence. Mais, émancipé ou non, un Juif reste un Juif ; celui qui ne s’affranchit pas lui-même n’est qu’un — émancipé. L’Etat protestant a eu beau libérer (émanciper) les Catholiques ; comme ils ne s’affranchissent pas eux-mêmes, ils restent — des Catholiques.\par
Nous avons déjà traité plus haut de l’intérêt personnel et du désintéressement. Les amis de la liberté jettent feu et flammes contre l’intérêt personnel, parce qu’ils ne sont pas parvenus, dans leurs religieux efforts pour conquérir la liberté, à se libérer de la noble, de la sublime « abnégation ». C’est à l’égoïsme que le Libéral en veut, car l’égoïste ne s’attache jamais à une chose pour l’amour de cette chose, mais bien pour l’amour de lui-même : la chose doit lui servir. Il est égoïste de n’accorder à la chose aucune valeur propre ou « absolue » et de se faire soi-même la mesure de cette valeur. On entend souvent citer comme un trait ignoble d’égoïsme pratique, le cas de ceux qui font de leurs études un gagne-pain (Brodstudium) c’est là, dit-on, une honteuse profanation de la science. Mais je me demande à quoi d’autre la science pourrait servir ? Franchement, celui qui ne sait l’employer à rien de meilleur qu’à gagner sa vie ne trahit qu’un égoïsme assez mince, car sa puissance d’égoïsme est des plus limitées ; mais il faut être un possédé pour blâmer en cela l’égoïsme et la prostitution de la science.\par
Le Christianisme, incapable d’apprécier l’Unique dans l’individu, chez qui il ne voit que la dépendance et la relativité, ne fut à proprement parler qu’une \emph{théorie sociale}, une doctrine de la vie en commun, tant de l’homme avec Dieu que de l’homme avec l’homme ; aussi en vint-il à mépriser profondément tout ce qui est « propre », particulier à l’individu.  Rien de moins chrétien que les idées exprimées par les mots allemands \emph{Eigennutz} (intérêt égoïste), \emph{Eigensinn} et \emph{Eigenwille} (caprice, obstination, entêtement, etc.), \emph{Eigenheit} (individualité, particularité), \emph{Eigenliebe} (amour-propre), etc., qui tous renferment l’idée de \emph{eigen} (propre, particulier).\par
L’optique chrétienne a peu à peu déformé le sens d’une foule de mots qui, primitivement honorables, sont devenus des termes de blâme ; pourquoi ne les réhabiliterait-on pas ? Ainsi le mot \emph{schimpf}, qui signifiait jadis « raillerie », signifie aujourd’hui « outrage, affront », car le zèle chrétien n’entend pas la plaisanterie, et tout passe-temps est à ses yeux une perte de temps ; \emph{frech}, « insolent, audacieux » voulait simplement dire « hardi, courageux » ; \emph{Frevel}, le « forfait » n’était que l’ « audace ». On sait pendant combien de temps le mot « Raison » a été regardé de travers.\par
Notre langue s’est ainsi façonnée peu à peu sur le point de vue chrétien, et la conscience universelle est encore trop chrétienne pour ne pas reculer avec effroi devant le non-chrétien comme devant quelque chose d’imparfait ou de mauvais ; c’est ce qui fait que l’intérêt personnel, égoïste, est si peu estimé.\par
« Egoïsme », au sens chrétien du mot, signifie quelque chose comme intérêt exclusif pour ce qui est utile à l’homme charnel. Mais cette qualité d’homme charnel est-elle donc ma seule propriété ? Est-ce que je m’appartiens, lorsque je suis livré à la sensualité ? Est-ce à moi-même, à ma \emph{propre} décision que j’obéis, lorsque j’obéis à la chair, à mes sens ? Je ne suis vraiment \emph{mien} que lorsque je suis soumis à ma propre puissance et non à celle des sens, pas plus d’ailleurs qu’à celle de quiconque n’est pas moi (Dieu, les hommes, l’autorité, la loi, l’Etat, l’Eglise, etc.). Ce que poursuit \emph{mon égoïsme}, c’est ce qui m’est utile à moi, l’autonome et l’autocrate.\par
On est d’ailleurs, à chaque instant, obligé de s’incliner devant cet intérêt personnel tant décrié  comme devant le moteur universel et tout-puissant. A la séance du 10 février 1844, Welcker invoque à l’appui d’une motion le manque d’indépendance des juges et prononce tout un discours pour démontrer que des magistrats congédiables, destituâmes, déplaçables et pensionnâmes, autrement dit exposés à se voir remerciés et mis à pied par voie administrative, perdent toute autorité et tout crédit ; le peuple même leur refuse son respect et sa confiance. « Toute la magistrature, s’écrie Welcker, est démoralisée par cette dépendance ! » Pour qui sait lire entre les lignes, cela veut dire que les justiciers trouvent mieux leur compte à prononcer un arrêt conforme aux intentions ministérielles qu’à s’attacher au sens de la loi. Comment y remédier ? Peut-être pourrait-on faire sentir aux juges tout ce que leur vénalité a d’ignominieux, dans l’espoir de les voir rentrer en eux-mêmes et mettre désormais la justice au-dessus de leur égoïsme ? Mais non, le peuple ne s’élève pas à une aussi romanesque confiance ; il sent trop bien que l’égoïsme est le plus puissant de tous les motifs. Laissons donc leurs fonctions de juges à ceux qui les ont exercées jusqu’à présent, quelque convaincus que nous soyons qu’ils n’ont jamais cessé et ne cesseront jamais d’agir en égoïstes. Seulement, faisons en sorte qu’ils ne voient pas plus longtemps leur égoïsme encouragé par la vénalité du droit ; qu’ils soient au contraire assez indépendants du gouvernement pour n’avoir point à chaque instant à opter entre la justice et leurs intérêts ; que leur « intérêt bien entendu » ne soit jamais compromis par la légalité des jugements qu’ils rendent, et qu’il leur soit rendu facile de recevoir un bon traitement sans s’aliéner la considération publique.\par
Ainsi donc Welcker et les Badois n’ont tous leurs apaisements que lorsqu’ils ont mis l’égoïsme dans leur jeu. Que penser dès lors des belles phrases sur le désintéressement dont ils ont sans cesse la bouche pleine ?\par
 Mes rapports avec une cause que je défends par égoïsme ne sont pas les mêmes que mes rapports avec la cause que je sers par désintéressement. Voici la pierre de touche qui permet de les distinguer : envers cette dernière je puis \emph{être coupable,} je puis commettre un \emph{péché,} tandis que je ne puis que \emph{perdre} la première, l’éloigner de moi, c’est-à-dire commettre à son égard une \emph{maladresse}. La liberté du commerce participe de cette double manière de voir ; elle passe en partie pour une liberté qui peut être accordée ou retirée selon les circonstances, en partie pour une liberté qui doit être sacrée en toutes circonstances.\par
Si une chose ne m’intéresse pas elle-même et pour elle-même, si je ne la désire pas pour l’amour d’elle, je la désirerai simplement à cause de son \emph{opportunité}, de son utilité, et en vue d’un autre but ; telles, par exemple, les huîtres que j’aime pour leur goût agréable. Pour l’égoïste, toute chose ne sera qu’un moyen, dont il est, en dernière analyse, lui-même le but ; doit-il protéger ce qui ne lui sert à rien ? — Le prolétaire, par exemple, doit-il protéger l’Etat ?\par
L’\emph{individualité} renferme en elle-même toute \emph{propriété,} et réhabilite ce que le langage chrétien avait déshonoré. Mais l’individualité n’a aucune mesure extérieure, car elle n’est nullement, comme la liberté, la moralité, l’humanité, etc., une \emph{idée :} — \emph{Somme des propriétés de l’individu,} elle n’est que le signalement de son — \emph{propriétaire}.
 \subsection[{B.II. Le propriétaire (l’Individu).}]{B.II. \\
Le propriétaire (l’Individu).}
\noindent Et Moi ? — Parviendrai-je au Moi et au Mien grâce au Libéralisme ?\par
Qui le Libéral tient-il pour son semblable ? L’Homme ! Sois seulement un Homme — et tu en es un —, le Libéral t’appellera son frère. Il s’inquiète peu de tes opinions et de tes sottises privées, du moment qu’il peut ne voir en toi que « l’Homme ».\par
Peu lui importe ce que tu es \emph{privatim}, car s’il est logique, ses principes lui interdisent rigoureusement d’y attacher la moindre valeur ; il ne voit en toi que ce que tu es \emph{generatim ;} en d’autres termes, il voit en toi non pas \emph{toi} mais l’\emph{espèce}, non pas Pierre ou Paul, mais l’Homme, non pas le réel ou l’unique, mais ton essence ou ton concept, non pas l’individu en chair et en os, mais l’\emph{Esprit.}\par
En tant que tu es Pierre, tu n’es pas son semblable, car il est Paul et non pas Pierre ; en tant qu’Homme, tu es ce qu’il est. S’il est vraiment un Libéral et non un égoïste inconscient, toi, Pierre, tu es, à ses yeux, comme si tu n’existais pas, ce qui, entre parenthèses, lui rend assez léger son « amour fraternel » ; ce qu’il aime en toi, ce n’est pas Pierre, dont il ne sait rien et ne veut rien savoir, mais uniquement l’Homme.\par
 Ne rien voir en toi et en moi de plus que l’ « Homme », c’est pousser à l’extrême la façon de voir chrétienne, d’après laquelle chacun n’est pour les autres qu’un \emph{concept} (par exemple un aspirant à la félicité, etc.).\par
Le Christianisme proprement dit nous réunit encore dans un concept moins général : ainsi, nous sommes « les enfants de Dieu », ceux « que conduit l’Esprit de Dieu »\footnote{ \noindent Ep. aux Romains, {\scshape viii}, 14.
 } ; tous ne peuvent toutefois se vanter d’être les enfants de Dieu, mais « le même Esprit qui témoigne devant notre Esprit que nous sommes enfants de Dieu montre aussi ceux qui sont enfants du Diable »\footnote{ \noindent I\textsuperscript{re} Ep. de Jean, {\scshape iii} 10.
 }. Pour être enfant de Dieu, il fallait n’être pas enfant du Diable, la famille divine excluait certains hommes. Il nous suffît, au contraire, pour être \emph{enfants des hommes}, c’est-à-dire hommes, d’appartenir à l’espèce humaine, d’être des exemplaires de cette espèce. Ce que je suis, moi, ne te regarde pas si tu es un bon libéral, tu ignores et dois ignorer mes \emph{affaires privées ;} il suffit que nous soyons tous deux enfants d’une même mère, l’espèce humaine : en tant que « né de l’homme », je suis ton semblable.\par
Que suis-je donc pour toi ? Suis-je ce \emph{moi} en chair et en os qui va et vient ? Du tout ! Ce \emph{moi}, avec ses pensées, ses déterminations et ses passions est à tes yeux « quelque chose de privé » qui ne te regarde pas, une « chose pour soi ». Comme « chose pour toi » il n’existe que mon concept, le concept de l’espèce à la-quelle j’appartiens, l’Homme, lequel s’appelle peut-être Pierre, mais pourrait aussi bien s’appeler Jean ou Michel. Tu vois en moi non pas moi, le réel et le corporel, mais l’irréel, le fantôme, — un \emph{Homme}.\par
Au cours des siècles chrétiens, toutes sortes de gens ont tour à tour passé pour « nos semblables », mais toujours nous les avons jugés selon cet \emph{esprit} que  nous attendions d’eux ; notre semblable fut par exemple celui dont l’esprit manifestait un besoin de rédemption ; plus tard, celui qui possédait l’esprit de bonne volonté, puis enfin celui qui montre un esprit et un visage humains. Ainsi varia le fondement de l’ « égalité ».\par
Du moment que l’on conçoit l’égalité comme égalité de l’\emph{esprit humain}, on a découvert une égalité qui embrasse véritablement \emph{tous} les hommes ; car qui oserait nier que nous, hommes, nous possédions un esprit humain, c’est-à-dire que nous n’ayons d’autre esprit qu’un esprit humain ?\par
Sommes-nous pour cela plus avancés qu’au début du Christianisme ? Notre esprit devait alors être divin, aujourd’hui il doit être humain ; mais si le divin ne suffisait pas à nous exprimer, comment l’humain pourrait-il exprimer tout ce que \emph{nous} sommes ? Feuerbach, par exemple, croit avoir découvert la vérité lors qu’il humanise le divin. Si le Dieu nous a fait cruellement souffrir, « l’Homme » est à même de nous martyriser plus cruellement encore.\par
Disons-le en quelques mots : si nous sommes hommes, cette qualité d’homme n’est que le moindre en nous, et n’a de signification et d’importance que comme une de nos \emph{propriétés,} comme contribuant à former notre \emph{individualité}. Certes je suis, entre autres qualités, un homme, de même que je suis, par exemple, un être vivant, un animal, un européen, un berlinois, etc. ; mais l’estime de celui qui ne priserait en moi que l’homme ou le berlinois me serait fort indifférente. Pourquoi ? Parce qu’il apprécierait une de mes \emph{propriétés} et non \emph{Moi}.\par
De même pour l’esprit. Je puis compter au nombre de mes attributs un esprit chrétien, un esprit loyal, etc., et cet esprit est ma propriété ; mais je ne suis pas cet esprit : il est à moi, et je ne suis pas à lui.\par
Nous retrouvons donc chez les Libéraux l’ancien mépris des Chrétiens pour le Moi, pour le Pierre ou  le Paul en chair et en os. Au lieu de me prendre pour ce que je suis, on ne considère que ma propriété, mes attributs, et si l’on conclut avec moi une alliance honorable, ce n’est que pour l’amour de — ma bourse : on épouse ce que j’ai et non ce que je suis. Le Chrétien s’attaque à mon esprit et le Libéral à mon humanité.\par
Mais si l’esprit, cet esprit qu’on ne regarde pas comme la \emph{propriété} du Moi réel et corporel, mais comme le Moi lui-même, est un fantôme, l’Homme dans lequel on veut reconnaître non un de mes attributs mais le Moi proprement dit n’est, lui non plus, qu’un fantôme, une pensée, un concept.\par
C’est pourquoi le Libéral tourne éternellement, sans pouvoir en sortir, dans le même cercle où est enfermé le Chrétien. Comme l’Esprit de l’humanité, c’est-à-dire l’Homme, habite en toi, tu es un homme, de même que tu es un chrétien si l’Esprit du Christ habite en toi. Mais, attendu que l’Homme qui est en toi n’y est qu’un second Moi, bien que ton véritable et ton « meilleur » Moi, il reste au delà de toi, et tu dois t’efforcer de devenir pleinement Homme. Effort aussi stérile que celui du Chrétien pour devenir pleinement esprit bienheureux !\par
Aujourd’hui que le Libéralisme a proclamé l’Homme, on peut dire que le Christianisme a été ainsi poussé à ses dernières conséquences, et que dès l’origine le Christianisme ne s’est proposé d’autre tâche que celle de réaliser l’Homme, le « véritable Homme ». On comprend donc que c’est une erreur de croire que le Christianisme accorde au Moi une valeur infinie, comme le feraient penser par exemple la doctrine de l’immortalité, le soin du salut, etc. Non : cette valeur, il ne l’attribue qu’à \emph{l’Homme.} Seul l’Homme est immortel, et ce n’est qu’en tant qu’Homme que je le suis aussi. Le Christianisme enseigna bien que nul ne périt tout entier, et de même le Libéralisme déclare que tous les hommes sont  égaux ; mais éternité d’une part, égalité d’autre part ne concernent que l’\emph{Homme} qui est en moi et non \emph{Moi}. Ce n’est qu’en tant que je suis le support, l’hôte de l’Homme que je ne meurs pas ; c’est ainsi, comme on le sait, que « le Roi ne meurt pas ». Louis meurt, mais le Roi survit ; et je meurs, mais mon esprit, l’Homme survit. On a trouvé une formule pour identifier complètement le Moi et l’Homme, et l’on émet ce vœu : « devenez conformes à la véritable essence de l’espèce ».\par
La \emph{religion de l’Humanité} n’est que la dernière métamorphose de la religion chrétienne. Le Libéralisme, en effet, est une religion, attendu qu’il me sépare de mon essence et la place au-dessus de moi, attendu qu’il élève l’Homme à la hauteur où toute autre religion fait planer son dieu ou son idole, qu’il fait un au-delà de ce qui est mien et ne devrait être autre, qu’il fait de mes attributs, de ma propriété, quelque chose d’étranger à moi, c’est-à-dire un « être », une entité ; bref, le Libéralisme est une religion, parce qu’il m’humilie aux pieds de l’Homme et me crée ainsi une « vocation ». Par les formes mêmes qu’il revêt, le Libéralisme trahit encore sa nature de religion : il réclame une dévotion fervente à l’être suprême, l’Homme, « une foi qui agisse et donne des preuves de son zèle, une ferveur qui ne s’attiédisse point »\footnote{ \noindent B{\scshape runo} B{\scshape auer}, \emph{Judenfrage}, p. 61.
 }. Mais comme le Libéralisme est une religion humaine, ses adeptes font profession d’être \emph{tolérants} envers les adeptes des autres religions (juive, chrétienne, etc.) ; c’est de cette même tolérance que Frédéric le Grand faisait preuve envers quiconque remplissait ses devoirs de sujet, de quelque façon d’ailleurs qu’il jugeât bon de faire son salut. Cette religion doit s’élever à une universalité assez haute pour se séparer de toutes les autres comme de pures « sottises privées »,  envers lesquelles on se comporte très \emph{libéralement} en considération de leur insignifiance même.\par
On peut la nommer la \emph{Religion d’Etat,} la religion de l’ « Etat libre », non dans l’ancien sens de religion prônée et privilégiée par l’Etat, mais parce qu’elle est la religion que l’ « Etat libre » est non seulement autorisé mais obligé d’exiger de chacun des siens, qu’il soit d’ailleurs \emph{privatim} juif, chrétien, ou tout ce qui lui plaît. Elle joue dans l’Etat le même rôle que la piété dans la famille. Pour que la famille soit acceptée et maintenue telle qu’elle est par chacun de ceux qui en font partie, il faut que chacun d’eux tienne le lien du sang pour sacré, et qu’il éprouve, envers ce lien de la piété, un respect qui sanctifie chacun de ses parents.\par
Quelle est l’idée la plus haute que l’Etat puisse se proposer de réaliser ? C’est bien celle d’être une véritable Société humaine, une société dans laquelle puisse être admis comme membre quiconque est vraiment Homme, c’est-à-dire \emph{n’est pas non-homme. }Si large que soit la tolérance d’un Etat, elle s’arrête devant le non-homme et devant l’inhumain. Et cependant ce « non-homme » est homme, cet inhumain est lui-même quelque chose d’humain, quelque chose de possible uniquement à un homme et non à un animal, cet inhumain est un « possible humain ». Mais bien que tout non-homme soit un homme, l’Etat l’exclut de son sein ou l’emprisonne, et fait d’un hôte de l’Etat l’hôte d’une prison (d’une maison de fous ou d’une maison de santé, d’après le Communisme).\par
Il est facile de définir en termes sèchement techniques ce qu’on entend par un \emph{non-homme :} c’est un homme qui ne correspond pas au \emph{concept} Homme, comme l’inhumain est quelque chose d’humain qui ne coïncide pas avec l’ensemble d’attributs qui forment la notion d’humain. C’est là ce que la logique appelle une « tautologie ». Peut-on en effet émettre ce jugement  qu’un homme peut ne pas être un Homme, à moins d’admettre cette hypothèse que le concept Homme peut être séparé de l’homme existant, l’essence du phénomène ? On dit : Il \emph{paraît} un homme, mais n’en \emph{est} pas un.\par
Il y a bien des siècles que les hommes se contentent de cette « pétition de principes » ! Et, ce qu’il y a de plus fort, c’est que pendant tout ce temps il n’a existé que des \emph{non-hommes}. Quel l’individu a jamais coïncidé avec son schème ? Le Christianisme ne connaît qu’un seul et unique Homme, — le Christ —, et celui-là même n’est, au point de vue opposé, qu’un \emph{non-homme :} c’est un homme surhumain, un « Dieu ». L’homme \emph{réel} n’est que le — non-homme.\par
Ces hommes qui ne sont pas des hommes, que pourraient-ils être d’autre que des \emph{fantômes ?} Chaque homme réel, ne correspondant pas au concept « Homme » ou n’étant pas « conforme au génie de l’espèce », est un spectre. Mais si je fais mien, si je réduis à n’être plus qu’un de mes attributs, une de mes \emph{propriétés}, cet Homme qui était jusqu’ici exclusivement mon idéal, mon devoir, mon essence ou mon concept et qui planait comme tel au-dessus de moi et au delà de moi, si je fais en sorte que l’Homme ne soit plus que mon humanité, ma manière d’être, et que ce que je fais ne soit plus humain que par la seule raison que c’est moi qui le fais et non parce que cela répond à la notion d’ « Homme », resterai-je encore un non-homme ? Je suis en réalité l’Homme et le non-homme tout ensemble, car je suis à la fois homme et plus qu’homme : je suis le Moi de cette individualité qui est ma et rien que ma propriété.\par
On devait finalement en venir à ne plus nous exhorter simplement à être des Chrétiens, mais à exiger que nous devinssions des Hommes. Nous n’avions en vérité jamais été réellement des Chrétiens, nous étions toujours restés de « pauvres pécheurs »  (le Chrétien aussi est un idéal inaccessible), mais l’absurdité de ces vœux n’était pas aussi frappante, il était plus facile de se faire illusion qu’aujourd’hui qu’on nous demande, à nous qui sommes hommes, qui agissons en hommes et ne saurions être autre chose ni agir autrement, d’être Hommes et « réellement Hommes ».\par
Nos Etats modernes, encore suspendus aux jupes de leur mère l’Eglise, nous imposent bien encore diverses obligations (celle d’appartenir à une confession religieuse, par exemple) qui ne sont pas strictement de leur ressort ; mais en somme ils ne méconnaissent pas leur signification lorsqu’ils se donnent pour des \emph{sociétés humaines} dont tout homme peut être membre en tant qu’homme, quand même il jouirait de moins de privilèges que tels de ses co-affiliés ; la plupart des Etats accueillent les adeptes de toutes les religions et enrégimentent les gens sans distinction de race ou de nationalité : Juifs, Turcs, Mores, etc. peuvent devenir citoyens français. L’Etat n’y met d’autre condition que celle d’être hommes. L’Eglise, qui est une réunion de fidèles, ne pouvait admettre dans son sein quiconque est homme ; l’Etat, qui est une société d’hommes, le peut. Mais le jour où l’Etat s’avisera d’être rigoureusement conséquent avec son principe, et de ne plus tenir compte chez les siens que de leur qualité d’Hommes (jusqu’à présent les Américains du Nord eux-mêmes admettent encore tacitement que ceux qu’ils agréent parmi eux pratiquent une religion, ne fût-ce que la religion du bien et de l’honneur) il ne lui restera plus qu’à descendre dans sa tombe. Les siens, qu’il suppose purement Hommes, se révéleront de purs égoïstes, et chacun d’eux usera de lui dans un but égoïste, de toutes ses forces d’égoïste. Avec les égoïstes, la « Société humaine » a vécu ; car ils ne se traitent plus mutuellement en Hommes, et quand ils agissent c’est égoïstement, comme un Moi, envers un Toi ou un Vous radicalement distincts et opposés.\par
 Dire que l’Etat doit faire état de notre humanité, revient à dire qu’il doit compter sur notre \emph{moralité}. Voir en autrui un homme et se comporter en homme à son égard c’est agir moralement ; tout l’ « amour spirituel » du Christianisme n’est rien d’autre. Si je vois en toi l’Homme, de même que je vois en moi l’Homme et uniquement l’Homme, je ferai pour toi ce que je ferais pour moi, car nous sommes en ce cas ce que les mathématiciens appellent deux quantités égales à une même troisième : A = C et B = C, d’où A = B, autrement dit « je = Homme » et « tu = Homme » d’où « je = tu » ; toi et moi sommes la même chose.\par
La moralité est incompatible avec l’égoïsme, parce que ce n’est pas à Moi mais seulement à l’Homme que je suis qu’elle accorde une valeur. Si l’Etat est une société d’hommes, et non une réunion de \emph{Moi }dont chacun n’a en vue que lui-même, il ne peut subsister sans la Moralité et doit être fondé sur elle.\par
Aussi l’Etat et Moi sommes-nous ennemis. Le bien de cette « Société humaine » ne me tient pas au cœur, à moi, l’égoïste ; je ne me dévoue pas pour elle, je ne fais que l’employer ; mais afin de pouvoir pleinement en user, je la convertis en ma propriété, j’en fais ma créature, c’est-à-dire que je l’anéantis et que j’édifie à sa place l’\emph{association des Egoïstes}.\par
L’Etat de son côté trahit son hostilité à mon égard en exigeant que je sois un Homme, ce qui sous entend que je pourrais n’en pas être un et passer à ses yeux pour un « non-homme » : il me fait de l’humanité un \emph{devoir}. Il exige en outre que je m’abstienne de toute action susceptible de compromettre son existence ; l’existence de l’Etat, l’état de choses régnant doit m’être sacré. Aussi ne dois-je pas être un égoïste, mais un homme « bien pensant » et « bien faisant », autrement dit moral. Devant l’Etat et son état je dois être impuissant, respectueux, etc.\par
 Cet Etat, qui n’a d’ailleurs actuellement aucune réalité et doit encore être fondé, est l’idéal du Libéralisme progressiste. Il sera une véritable « société humaine » où trouvera place quiconque est « Homme ». Le Libéralisme se propose comme but de réaliser l’ « Homme », c’est-à-dire de lui créer un monde, monde qui sera le monde \emph{humain}, ou la société humaine universelle (communiste). « L’Eglise, disait-on, ne pouvait s’occuper que de l’esprit ; l’Etat doit se charger de l’homme tout entier\footnote{ \noindent H{\scshape ess}, \emph{Triarchie}, p. 76.
 }. » Mais l’Homme n’est-il pas Esprit ? Le noyau de l’Etat est « l’Homme », cette irréalité, et l’Etat lui-même n’est qu’une société d’Hommes. Le monde que crée le croyant (Esprit croyant) s’appelle Eglise ; le monde que crée l’Homme (Esprit humain) s’appelle Etat. Mais ce n’est point là \emph{mon} monde. Ce que j’exécute n’est jamais humain \emph{in abstracto}, mais m’est toujours \emph{propre ; mon} œuvre d’homme est différente de toutes les autres œuvres d’hommes, et ce n’est que grâce à cette différence qu’elle est réelle et \emph{qu’elle m’appartient.} L’humain en soi est une abstraction, et, par conséquent, un fantôme, un être imaginaire.\par
Bruno Bauer exprime quelque part (Judenfrage, p.84) cette opinion que la dernière vérité à laquelle soit parvenue la Critique, et la vérité que le Christianisme lui-même avait toujours cherchée est « l’Homme ». « L’histoire du monde chrétien est, dit-il, l’histoire du plus grand des combats qui aient jamais été livrés pour la vérité ; car cette histoire — et elle seule — est l’histoire de la découverte de la première et de l’ultime vérité — de l’Homme et de la liberté ! »\par
Soit, acceptons-en le bénéfice, et admettons que l’\emph{Homme} est le résultat auquel aboutit l’histoire de la pensée chrétienne et, d’ailleurs, tout l’effort des hommes vers la religion ou l’idéal. Qu’est-ce donc que l’Homme ? C’est \emph{Moi ! Je \emph{suis} l’Homme,} fin et  aboutissant du Christianisme, et \emph{Je} suis le point de départ et la matière d’une histoire nouvelle, d’une histoire de la jouissance après l’histoire du sacrifice, d’une histoire non plus de l’Homme et de l’Humanité, mais du — \emph{Moi}. L’Homme passe pour l’universel ; mais s’il est quelque chose de réellement universel, c’est le Moi et son égoïsme, car chacun est un égoïste et fait de soi le centre de tout. Le pur égoïste n’est pas le juif, car le Juif se soumet encore à Jéhovah ; ce n’est pas non plus le chrétien, car le Chrétien ne vit que par la grâce de Dieu et se prosterne à ses pieds. Comme Juif, comme Chrétien, un homme ne satisfait que certains de ses désirs, tel appétit déterminé et non \emph{lui-même ;} son égoïsme n’est qu’un \emph{demi}-égoïsme, parce que c’est l’égoïsme d’un demi-homme, à moitié « lui » et à moitié « juif », ou à moitié son propriétaire et à moitié un esclave. Aussi Juif et Chrétien sont-ils toujours partiellement opposés : la moitié de l’un est la négation d’une moitié de l’autre ; comme hommes ils se reconnaissent, comme esclaves ils se repoussent parce qu’ils servent deux maîtres différents. S’ils pouvaient être complètement des égoïstes, ils s’excluraient \emph{totalement }et n’en tiendraient que plus fermement l’un à l’autre. Ce qui les avilit, ce n’est pas de se contredire, c’est de ne le faire qu’à demi.\par
Bruno Bauer pense au contraire que Juifs et Chrétiens pourraient se regarder comme des « Hommes » et se traiter mutuellement comme tels, s’ils dépouillaient cette manière d’être particulière qui les sépare et leur fait un devoir de perpétuer cette séparation, pour reconnaître dans « l’Homme » leur « véritable essence. » A l’en croire, l’erreur tant des Juifs que des Chrétiens serait de prétendre être et avoir quelque chose « à part », au lieu d’être simplement des Hommes et de tendre vers l’humain, c’est-à-dire vers les « droits universels de l’Homme. » Leur erreur fondamentale serait de se croire des  « élus », de se croire en possession de « privilèges » et, d’une façon générale, de croire à l’existence du \emph{privilège}. Il leur répond en leur objectant les droits de l’Homme. Les droits de l’homme !\par
L’Homme, c’est \emph{l’Homme en général,} et chacun est homme. Chacun donc doit posséder les droits éternels en question et doit en jouir, de l’avis des Communistes, dans la complète « démocratie », ou, comme il serait plus exact de l’appeler — anthropocratie. Mais Moi seul j’ai tout ce que je me — procure ; comme Homme, je n’ai rien ; on voudrait voir chaque homme jouir de tous les biens, simplement parce qu’il porte le titre d’Homme. Mais je mets l’accent sur « Je » et non sur le fait que je « suis homme ».\par
L’Homme n’est quelque chose que pour autant qu’il est mon \emph{attribut} (ma propriété) ; il en est de l’humanité comme de la virilité et de la fémininité. L’idéal des Anciens était la virilité ; la vertu était pour eux \emph{virtus} et \emph{arete}, le courage mâle. Que penser d’une femme qui ne voudrait être que parfaitement « femme » ? Etre femme n’est pas donné à tout le monde, et je sais pas mal de gens qui se proposeraient là un idéal fort inaccessible. Mais la femme, elle, est en tout cas féminine, elle l’est de nature, la « fémininité » est un des éléments de son individualité, et elle n’a que faire du « vrai féminin » Je suis homme juste comme la terre est étoile. Il n’est pas moins ridicule de m’imposer comme une mission d’être « véritablement homme » qu’il ne le serait de faire à la terre un devoir d’être « vraiment étoile ».\par
Lorsque Fichte dit : « le Moi est tout », cela semble parfaitement en harmonie avec ma théorie. Seulement le Moi \emph{n’est} pas tout, mais il \emph{détruit} tout, et seul le Moi qui se décompose lui-même, le Moi qui \emph{n’est} jamais, le Moi — \emph{finale} est réellement Moi. Fichte parle d’un Moi « absolu », tandis que je parle de Moi, du Je périssable.\par
On est bien près d’admettre que \emph{Homme} et \emph{Moi}  sont synonymes ! Et nous voyons pourtant Feuerbach, par exemple, déclarer que le terme « Homme » ne doit s’appliquer qu’au Moi absolu, à \emph{l’espèce}, et non au Moi individuel, éphémère et caduc. Egoïsme et humanisme devraient signifier la même chose ; cependant, d’après Feuerbach, si l’individu « peut franchir les limites de son individualité, il ne peut néanmoins s’élever au-dessus des lois et des caractères essentiels de l’espèce à laquelle il appartient\footnote{ \noindent \emph{Wesen des Christentums}, p. 401.
 } ». Seulement l’espèce n’est rien, et l’individu qui franchit les bornes de son individualité n’en est justement que plus lui-même, plus individuel. Il n’est lui, il n’est individu que pour autant qu’il s’élève, qu’il franchisse, qu’il ne reste pas ce qu’il est ; sinon il est fini, mort. L’Homme n’est qu’un idéal, et l’espèce n’est qu’une pensée. Etre \emph{un} homme ne signifie pas représenter l’idéal de \emph{l’}Homme, mais être soi, l’individu. Qu’ai-je à faire de réaliser \emph{l’humain en général ?} Ma tâche est de me contenter, de me suffire à moi-même. C’est Moi qui suis mon espèce ; je suis sans règle, sans loi, sans modèle, etc. Il se peut que je ne puisse faire de moi que fort peu de chose, mais ce peu est tout, ce peu vaut mieux que ce que pourrait faire de moi une force étrangère, le dressage de la Morale, de la Religion, de la Loi, de l’Etat, etc. Mieux vaut — s’il peut toutefois être question ici de mieux et de pire — mieux vaut, dis-je, un enfant indiscipliné qu’un enfant « modèle », mieux vaut l’homme qui se refuse à tout et à tous que celui qui consent toujours ; le récalcitrant, le rebelle peuvent encore se façonner à leur gré, tandis que le bien stylé, le bénévole, jetés dans le moule général de l’ « espèce » sont par elle déterminés : elle leur est une loi. Je dis \emph{déterminés, }c’est-à-dire \emph{destinés}, car qu’est-ce que l’espèce pour eux, sinon la destinée, — et leur « destination » ou leur « vocation » ?\par
 Que je me propose pour idéal l’humanité, l’espèce, et que je tende vers ce but, ou que je fasse le même effort vers Dieu et le Christ, je n’y vois aucune différence essentielle ; ma vocation est tout au plus, dans le premier cas, plus indéterminée, plus vague et plus flottante.\par
De même que l’individu est toute la nature, il est toute l’espèce.\par
Ce que je suis \emph{détermine} nécessairement tout ce que je fais, pense, etc., bref toutes mes manifestations. Le Juif, par exemple, ne peut vouloir que telle chose, ne peut « se montrer » que tel et non autre ; le Chrétien, quoi qu’il fasse, ne peut que se montrer et se manifester chrétien. S’il t’était possible d’être Juif ou Chrétien, tu ne produirais plus que du juif ou du chrétien ; mais cela n’est pas possible, toute ta conduite est celle d’un \emph{égoïste}, d’un pécheur contre les concepts juif, chrétien, etc., car tu n’est pas = Juif. Comme le bout de l’oreille de l’égoïsme dépasse toujours, on s’est informé d’un concept assez vaste et assez compréhensif pour exprimer réellement tout ce que tues, d’un concept qui, étant ta vraie nature, impliquât toutes les lois qui règlent ton activité. Ce qu’on a trouvé de plus parfait dans ce genre est « l’Homme ». En étant Juif tu es trop peu, et le juif n’est pas ton devoir ; être un Grec, être un Allemand ne suffit pas. Mais sois un Homme, et tu auras tout ; choisis l’humain comme ta vocation.\par
Nous savons désormais où est le devoir et nous pourrions rédiger le nouveau catéchisme. De nouveau le sujet est subordonné au prédicat, et le particulier immolé en général ; la domination est de [{\corr nouveau}] assurée à une \emph{Idée}, et le sol est préparé pour une nouvelle \emph{religion}. Nous avons progressé dans le domaine de la religion et particulièrement du Christianisme, mais nous n’avons pas fait un pas pour en sortir.\par
Ce pas franchi nous conduirait à l’\emph{indicible}, car la  langue indigente n’a pas de mot pour Me dire, et le « verbe », le \emph{logos}, n’est, lorsqu’il s’applique à Moi, qu’un « vain mot. »\par
On cherche \emph{mon essence}. Ce n’est pas le Juif, l’Allemand, etc. ; c’est — l’Homme. « L’Homme est mon essence ».\par
Je me suis désagréable ou antipathique, je me répugne, je me dégoûte et me fais horreur, ou bien ne suis jamais assez et ne fais jamais assez pour moi. De tels sentiments naît soit l’auto-négation, soit l’auto-critique. La religiosité commence avec l’abnégation et finit par la critique radicale.\par
Je suis possédé et je veux exorciser l’ « Esprit malin ». Que faire ? — Commettre hardiment le péché le plus noir aux yeux des Chrétiens : blasphémer le Saint-Esprit. « Si quelqu’un blasphème contre le Saint-Esprit, il n’en recevra jamais le pardon et restera chargé d’une condamnation éternelle\footnote{ \noindent Marc, {\scshape iii}, 29.
 }. » Je ne veux pas de pardon et ne crains pas le châtiment.\par
L’Homme est le dernier des mauvais Esprits, le dernier fantôme et le plus fécond en impostures et en tromperies ; c’est le plus subtil menteur qui se soit jamais caché sous un masque d’honnêteté, c’est le père » des mensonges.\par
L’Egoïste qui s’insurge contre les devoirs, les aspirations et les idées qui ont cours commet impitoyablement la suprême \emph{profanation :} rien ne lui est sacré !\par
Il serait absurde de soutenir qu’il n’est point de puissances supérieures à la mienne. Mais la position que je prendrai à leur égard sera toute différente de ce qu’elle eût été dans les âges religieux : je serai l’ennemi de toute puissance supérieure, tandis que la religion nous enseigne à nous en faire une amie et à être humbles envers elle.\par
Le \emph{sacrilège} concentre ses forces contre toute \emph{crainte  de Dieu,} car la crainte de Dieu lui enlèverait tout empire sur ce dont il laisserait subsister le caractère sacré. Que ce soit le Dieu ou l’Homme qui exerce en l’Homme-Dieu la puissance sanctifiante, que ce soit à la sainteté de Dieu ou à celle de l’Homme que nous adressions nos hommages, cela ne change en rien la crainte de Dieu : l’Homme devenu « Etre suprême » sera l’objet de la même vénération que le Dieu, Etre suprême de la religion \emph{sensu strictiori ;} tous deux exigent de nous crainte et respect.\par
La crainte de Dieu proprement dite est depuis longtemps ébranlée et la mode est à un « athéisme » plus ou moins conscient, reconnaissable extérieurement à un abandon général des exercices du culte. Mais on a reporté sur l’Homme tout ce qu’on a enlevé à Dieu, et la puissance de l’Humanité s’est accrue de tout ce que la piété a perdu en importance : l’Homme est le dieu d’aujourd’hui et la crainte de l’Homme a pris la place de l’ancienne crainte de Dieu.\par
Mais comme l’Homme ne représente qu’un autre Etre suprême, l’Etre suprême n’a subi en somme qu’une simple métamorphose, et la crainte de l’Homme n’est qu’un aspect différent de la crainte de Dieu.\par
Nos athées sont de pieuses gens.\par
Si durant les temps dits féodaux nous recevions tout en fief de Dieu, la période libérale nous a mis dans le même état de vasselage vis-à-vis de l’Homme. Dieu était le Maître, à présent l’Homme est le Maître ; Dieu était le Médiateur, à présent c’est l’Homme ; Dieu était l’Esprit, et l’Homme aujourd’hui est l’Esprit. Sous ce triple rapport la vassalité s’est transformée : en premier lieu, nous tenons de l’Homme tout-puissant notre \emph{puissance}, et cette puissance, émanant d’une autorité supérieure, ne s’appelle pas puissance ou force, mais s’appelle le Droit : le « droit de l’Homme ». En second lieu, nous tenons de lui notre situation dans le monde, car il est le médiateur qui  ordonne nos \emph{relations} et celles-ci ne peuvent par conséquent être qu’ « humaines ». Enfin nous tenons de lui \emph{nous-mêmes,} c’est-à-dire notre valeur propre ou tout ce dont nous sommes dignes, car nous n’avons aucune valeur s’il n’habite en nous et si nous ne sommes pas « humains ». — La puissance est à l’Homme, le monde est à l’Homme et je suis à l’Homme.\par
Mais en quels termes déclarer que \emph{Je} suis mon Justificateur, mon Médiateur et mon Propriétaire ? Je dirai :\par
Ma puissance \emph{est} ma propriété.\par
Ma puissance me \emph{donne} la propriété.\par
Je \emph{suis} moi-même ma puissance, et je suis par elle ma propriété.\par
\subsubsection[{B.II.1. Ma Puissance}]{B.II.1. Ma Puissance}
\noindent Le \emph{Droit} est l’\emph{Esprit de la Société.} Si la Société a une \emph{volonté}, c’est précisément cette volonté qui constitue le Droit : la société n’existe que par le Droit. Mais comme elle n’existe que par le fait d’exercer une \emph{souveraineté} sur l’individu, on peut dire que le Droit est sa \emph{volonté souveraine. La justice est l’utilité de la société}, disait Aristore.\par
Tout droit établi est un droit \emph{étranger}, un droit que l’on « m’accorde », dont on me « permet de jouir ». Aurais-je le bon droit de mon côté parce que le monde entier me donnerait raison ? Que sont donc mes droits dans l’Etat ou dans la Société, sinon des droits extérieurs, des droits que je tiens d’\emph{autrui ?} Qu’un imbécile me donne raison, et mon droit aussitôt me deviendra suspect, car je ne me soucie pas de son approbation. Mais que ce soit même un sage qui m’approuve, et je n’aurai pas encore pour cela raison. Le fait d’avoir raison ou d’avoir tort est absolument  indépendant de l’approbation et du fou et du sage.\par
C’est cependant ce droit, qui n’est que l’approbation d’autrui, que nous avons jusqu’à présent travaillé à obtenir. Lorsque nous cherchons notre droit, nous nous adressons à un tribunal. A quel tribunal ? A un tribunal royal, papal, populaire, etc. Le tribunal du Sultan peut-il être l’organe d’un autre droit que celui qu’il a plu au Sultan de désigner comme étant le droit ? Peut-il me donner raison lorsque je réclame un droit qui ne correspond pas à ce que le Sultan appelle le droit ? Peut-il, par exemple, m’accorder le droit de haute trahison, si cette dernière n’est pas un droit aux yeux du Sultan ? Ce tribunal, — le tribunal de la censure, par exemple, — peut-il me reconnaître le droit d’exprimer librement mon opinion, si le Sultan ne veut pas entendre parler de ce \emph{mien} droit ? Que demandai-je donc à ce tribunal ? Je lui demande le droit du Sultan et non mon droit, je lui demande un droit — \emph{étranger. }Il est vrai que pour autant que ce droit d’autrui concorde avec le mien, je pourrai y trouver aussi ce dernier.\par
L’Etat ne permet pas que deux hommes en viennent aux mains ; il s’oppose au duel. La moindre rixe est punie, alors même qu’aucun des combattants n’appellerait la police à son secours ; exception faite toutefois du cas où le battant et le battu, au lieu d’être toi et moi, sont un \emph{chef de famille} et son enfant : la \emph{famille,} et le père en son nom, a des droits que moi, l’individu, je n’ai pas.\par
La \emph{Gazette de Voss} (Vossige Zeitung) nous présente le type de l’Etat fondé sur le Droit. Ici, tout doit être tranché par le juge et par un tribunal. Le tribunal suprême de la censure est pour elle un tribunal qui « fixe le Droit » et qui « dit la Justice ». Quel droit ? Quelle justice ? Ceux de la censure. Pour que nous reconnaissions ses arrêts comme justes, il faudrait que nous regardions la censure comme juste. On  s’imagine cependant que ce tribunal constitue une garantie. Oui, il en est une, il est une garantie, mais contre l’erreur individuelle d’un censeur ; il ne protège que le magistrat chargé d’appliquer la loi, dont il met la volonté à l’abri des interprétations erronées de cette loi ; quant à l’écrivain, il n’en est que plus étroitement soumis à la loi dont l’autorité est renforcée de toute la « puissance sacrée du Droit ».\par
Que j’aie le droit pour moi ou contre moi, nul autre que moi-même n’en peut être juge. Tout ce que les autres peuvent faire, c’est juger si mon droit est ou n’est pas d’accord avec le leur, et apprécier si, pour eux aussi, il est un droit.\par
Envisageons encore la question à un autre point de vue. Je dois dans un sultanat respecter le droit du Sultan, en république le droit du peuple, dans la communauté catholique le droit canon, etc. Je dois me soumettre à ces droits, les tenir pour sacrés. Ce « sens du droit », cet « esprit de justice » est si solidement enraciné dans la tête des gens que les plus radicaux des révolutionnaires actuels ne se proposent rien de plus que de nous asservir à un nouveau « Droit » tout aussi sacré que l’ancien : au droit de la Société, au droit de l’Humanité, au droit de tous, etc. Le droit de « tous » doit avoir le pas sur \emph{mon} droit. Ce droit de « tous » devrait être aussi mon droit, puisque je fais partie de « tous » ; mais remarquez que ce n’est point parce qu’il est le droit des autres et même de tous les autres que je me sens poussé à travailler à sa conservation. Je ne le défendrai pas parce qu’il est un droit \emph{de tous}, mais uniquement parce qu’il est \emph{mon} droit ; que chacun veille à se le conserver de même ! Le droit de tous (celui de manger, par exemple) est le droit de chaque individu. Si chacun veille à \emph{se} le garder intact, tous l’exerceront d’eux-mêmes ; que l’individu ne s’inquiète donc pas de tous, et défende son droit sans se faire le zélateur d’un droit de tous !\par
Mais les réformateurs sociaux nous prêchent un  « \emph{droit de la} Société ». Par lui, l’individu devient l’esclave de la Société, il n’a de droits que si la Société lui en donne, c’est-à-dire s’il vit selon les lois de la Société, en homme \emph{légal}. Que je sois légal sous un gouvernement despotique ou dans une société telle que la rêve Weitling, je n’en ai pas moins aucun droit, car dans un cas comme dans l’autre tout ce que je puis avoir n’est pas \emph{mon} droit, mais un droit \emph{étranger} à moi.\par
Lorsqu’on parle de droit, il est une question qu’on se pose toujours : « Qui, ou quelle chose me donne le droit de faire ceci ou cela ? » Réponse : « Dieu, l’Amour, la Raison, l’Humanité, etc ! » Hé non, mon ami : ce qui te le donne, ce droit, c’est \emph{ta force}, ta puissance, et rien d’autre (\emph{ta} raison, par exemple peut te le donner).\par
Le Communisme, qui admet que les hommes « ont naturellement des droits égaux », se contredit en soutenant que les hommes ne tiennent de la nature aucun droit ; en effet, il n’admet pas, par exemple, que la nature donne aux parents des droits sur leurs enfants et à ces derniers des droits sur leurs parents : il supprime la famille. La nature ne donne absolument aucun droit aux parents, aux frères et aux sœurs, etc.\par
Au fond, ce principe nettement révolutionnaire ou babouviste repose sur une conception religieuse, autrement dit fausse. Qui peut s’enquérir du « Droit » s’il ne se place au point de vue religieux ? Le « Droit » n’est-il pas une notion religieuse, c’est-à-dire quelque chose de sacré ? L’ « égalité des droits » que proclama la Révolution n’est, sous un autre nom, que « l’égalité chrétienne », l’égalité fraternelle qui règne entre les enfants de Dieu, entre les Chrétiens ; c’est en un mot, la \emph{fraternité}\footnote{ \noindent \emph{En français dans le texte, N. d. Tr.}
 }.\par
 Toute controverse sur le Droit mérite d’être flagellée de ces paroles de Schiller :\par


\begin{verse}
« Il y a bien des années déjà que je me sers de mon nez pour sentir ;\\
« ai-je donc réellement sur lui un droit indiscutable ? »\\
\end{verse}

\noindent En donnant à l’égalité l’estampille du Droit, la Révolution prenait position sur le terrain de la religion, dans le domaine du sacré, de l’idéal. De là, depuis, la lutte pour les « sacrés et imprescriptibles » droits de l’Homme. En opposition avec l’ « éternel droit de l’Homme », on fait valoir, ce qui est tout naturel et tout aussi légitime, les « droits acquis » et les titres que donne l’occupation. Droit contre droit ! Chacun essaie naturellement de convaincre l’autre d’ « injustice ». Tel est le \emph{procès} qui est pendant depuis la Révolution.\par
Vous voulez que le droit soit pour vous et contre les autres ; mais ce n’est pas possible : vis-à-vis d’eux vous restez éternellement « dans votre tort », car ils ne seraient pas vos adversaires s’ils n’avaient pas eux aussi le droit de leur côté ; toujours ils vous \emph{donneront tort.} Mais, me direz-vous, mon droit est plus élevé, plus grand, plus \emph{puissant} que celui des autres. Pas du tout : votre droit n’est pas plus fort que le leur tant que vous-même n’êtes pas plus fort qu’eux. Les sujets chinois ont-ils droit à la liberté ? Faites leur en donc cadeau, et vous jugerez de votre erreur : ils n’ont aucun droit à la liberté parce qu’ils sont incapables d’en user, — ou, plus clairement : c’est justement parce qu’ils n’ont pas la liberté qu’ils n’y ont aucun droit. Les enfants n’ont aucun droit à la « majorité » parce que, étant des enfants, ils ne sont pas majeurs. Les peuples qui se laissent maintenir en tutelle n’ont pas droit à l’émancipation : ce n’est qu’en cessant d’être en tutelle qu’ils acquerront le droit d’être émancipés.\par
Tout cela revient simplement à ceci : Ce que tu as  la \emph{force} d’être, tu as aussi le \emph{droit} de l’être. C’est de moi seul que dérive tout droit et toute justice ; j’ai le droit de tout faire dès que j’en ai la force. J’ai le droit de renverser Zeus, Jéhovah, Dieu etc, si je le \emph{puis ;} si je ne le puis pas, ces dieux demeureront debout devant moi, forts de leur droit et de leur puissance ; la « crainte de Dieu » courbera mon impuissance, je suivrai leurs commandements, et je croirai marcher droit tant que j’agirai en tout conformément à \emph{leur }droit : tels ces garde-frontières russes qui se croient en droit d’abattre à coups de fusil les fuyards suspects, du moment qu’ils les assassinent au nom d’une « autorité supérieure », c’est-à-dire conformément au droit. Moi, au contraire, je me donne le droit de tuer, du moment que je ne m’interdis pas moi-même le meurtre, et que je ne recule pas devant lui avec horreur en le jugeant « contraire au droit ». Cette idée fait le fond d’un poème de Chamisso, « Das Mordenthal », qui nous montre un vieil Indien meurtrier forçant au respect le blanc dont il a massacré les compagnons. S’il est une chose que je n’ai pas le droit de faire, c’est que je ne la fais pas de propos délibéré, c’est-à-dire que je ne m’y autorise pas moi-même.\par
C’est à Moi de décider ce qui est pour moi le droit. Hors de moi, pas de droit. Ce qui \emph{m}’ « est juste » est juste. Il se peut que les autres ne jugent pas pour cela que c’est juste, mais c’est leur affaire et non la mienne : à eux de se garder ! Alors même qu’une chose paraîtrait injuste à tout le monde, si cette chose m’était juste, c’est-à-dire si je la voulais, je me soucierais peu de tout le monde. Ainsi en usent, plus ou moins selon leur degré d’égoïsme, tous ceux qui savent s’estimer eux-mêmes ; car la force prime le droit, comme c’est d’ailleurs pleinement — son droit. Etant « de ma nature » un homme, j’ai, dit Babœuf, un droit égal à la jouissance de tous les biens. Ne devait-il pas dire de même : étant « de ma nature »  prince et premier né, j’ai droit au trône ? Les droits de l’homme et les droits acquis par héritage ont la même origine, ils émanent de la \emph{nature.} Je suis né homme équivaut à : Je suis né fils de roi. L’homme naturel n’a qu’un droit naturel, sa force, et des prétentions naturelles : il a un droit de par sa naissance et des prétentions de par sa naissance. Mais la nature ne peut me donner un droit, c’est-à-dire une aptitude ou une puissance, que seul mon acte peut me donner. Si le prince se place au-dessus des autres enfants, c’est déjà là son acte, qui lui assure l’avantage ; si les autres enfants agréent et reconnaissent cet acte, c’est leur acte à eux qui les rend dignes d’être sujets.\par
Que ce soit la nature qui me donne un droit, ou que ce soient Dieu, le suffrage populaire, etc., ce droit sera toujours le même, ce sera un droit \emph{étranger, }que je ne me donne pas ou que je ne prends pas.\par
D’après les Communistes, la même somme de travail donne droit à la même somme de jouissance. Auparavant on s’était demandé si ce n’est pas l’homme « vertueux » qui doit être « heureux » sur terre. Et les Juifs conclurent réellement dans ce sens : « pour que tu sois heureux sur la terre. » Non, ce n’est pas une somme égale de travail, mais une somme égale de jouissance qui seule te donne droit à cette somme de jouissance. Jouis, et tu auras le droit de jouir. Mais si après avoir travaillé tu te laisses ravir la jouissance — « ce n’est que justice ».\par
\emph{Emparez-vous} de la jouissance, et elle vous appartiendra de droit ; mais quelle que soit l’ardeur de vos désirs, si vous ne la [{\corr saisissez}] pas, elle restera le « droit bien acquis » de ceux dont elle est le privilège. Elle est \emph{leur} droit, comme elle eut été \emph{votre} droit si vous la leur aviez arrachée.\par
Sur la question du droit de propriété, la lutte est ardente et tumultueuse. Les Communistes soutiennent\footnote{ \noindent A. B{\scshape ecker}, \emph{Volksphilosophie}, p. 22 sq.
 }  que « la terre appartient à celui qui la cultive, et ses produits à ceux qui les font naître. » Je pense qu’elle appartient à celui qui sait la prendre ou qui ne se la laisse pas enlever. S’il s’en empare et la fait sienne, il aura non seulement la terre, mais encore le droit à sa possession. C’est là le \emph{droit égoïste}, qui peut se formuler ainsi : « Je le Veux, donc c’est juste. »\par
Autrement compris, le droit est une chose dont on fait ce qu’on veut. Le tigre qui m’attaque est dans son droit, et moi qui l’abats, je suis également dans mon droit. Ce n’est pas mon \emph{droit} que je défends contre lui, c’est \emph{moi}.\par
Le droit humain étant toujours un droit accordé, il n’est jamais autre chose qu’un don, une « concession » que les hommes se font l’un à l’autre. Si l’on reconnaît par exemple aux nouveau-nés le droit à l’existence, ce droit leur appartiendra ; si on ne le leur reconnaît pas (comme chez les Spartiates et les anciens Romains) il ne leur appartiendra pas. La société seule peut en effet le leur donner, le leur « accorder », puisqu’ils ne peuvent le prendre ou se le donner eux-mêmes. On m’objectera que ces enfants avaient « naturellement » le droit de vivre, mais que ce droit, les Spartiates refusaient de le leur \emph{reconnaître}. C’est donc, répondrai-je, qu’ils n’avaient aucun droit à cette reconnaissance, pas plus qu’ils n’avaient droit à ce que les bêtes sauvages auxquelles on les jetait reconnûssent leur droit à la vie.\par
On parle beaucoup de droits \emph{innés,} et on se plaint :\par


\begin{verse}
« Du droit qui est né avec nous,\\
« de celui-là, hélas, il n’est pas question. »\\
\end{verse}

\noindent Mais quelle espèce de droit pourrait bien être né avec moi ? Est-ce le droit d’aînesse, le droit d’hériter d’un trône, de recevoir une éducation princière, — ou bien encore, si je suis né de parents pauvres, est-ce le droit de fréquenter l’école gratuite, d’être vêtu  par l’assistance publique, et enfin de gagner ma croûte et mon hareng dans une houillière ou une filature ? Ne sont-ce pas là des droits innés, congénitaux, des droits que mes parents m’ont transmis par le fait même qu’ils m’ont donné naissance ? Vous répondrez que non, que c’est là un emploi abusif du mot droits, et que ce sont précisément ces prétendus droits que vous vous efforcez de remplacer par le \emph{véritable droit de naissance}. Pour fonder celui-ci, vous le réduisez à sa plus simple expression, et vous soutenez que chacun est de par sa naissance l’égal de son voisin, autrement dit un homme Je vous accorde que tous naissent hommes, et qu’en cela tous sont égaux. Mais pourquoi le sont-ils ? Pour cette seule raison qu’ils ne se montrent, se manifestent encore que comme de simples — enfants des hommes, de petits hommes nus. Et c’est en cela qu’ils se distinguent immédiatement de ceux qui ont déjà pu tirer d’eux-mêmes quelque chose, qui se sont faits quelque chose, et ont cessé d’être uniquement « enfants des hommes » pour devenir fils de — leur propre activité créatrice. Ces derniers possèdent plus que les simples droits trouvés dans leur berceau : ils ont \emph{acquis} des droits. Quel sujet de discussions, et quel champ de bataille ! C’est le vieux combat des droits innés de l’homme et des droits acquis qui se rallume. Alléguez vos droits innés, et quelqu’un ne manquera pas de vous objecter les droits acquis ; vous vous appuiez tous deux sur le « terrain du droit », car chacun a un « droit » qui s’oppose à celui des autres : l’un a un droit inné ou naturel, l’autre un droit acquis et « bien acquis ».\par
Tant que vous vous tiendrez sur le terrain, du droit, vous ne sortirez pas de la — chicane et vous ergoterez indéfiniment. Autrui ne peut ni vous donner raison, ni faire que vous ayez raison. Celui qui a pour lui la force a pour lui — le droit ; si l’une vous manque, vous n’aurez pas non plus l’autre. Contemplez  donc les puissants, regardez-les agir ! Nous ne parlons ici, bien entendu, que de la Chine et du Japon. Essayez un peu, Chinois ou Japonais, de donner \emph{tort} à ces puissants, et vous verrez comme ils vous jetteront en prison ! (Ne confondons pas, toutefois : les « avertissements bienveillants » sont — en Chine et au Japon — bien accueillis, parce que loin d’être des bâtons dans les roues du « char de la puissance », ce sont un coup de fouet, une stimulation et une adhésion tacite). Une seule voie vous est ouverte si vous voulez donner tort aux puissants : c’est la force ; dépouillez les de leur puissance, vous les aurez \emph{réellement }mis dans leur tort et privés de leurs droits ; sinon, vous ne pouvez rien, vous vous ferez de la bile en silence ou vous serez sacrifiés comme des fous encombrants.\par
Bref, Chinois, mes amis, n’invoquez pas le droit ; ne parlez pas du « droit qui est né avec vous » ; c’est aussi inutile que de parler de vos « droits acquis ».\par
Vous reculez avec effroi devant les autres parce que vous croyez voir se dresser auprès d’eux le \emph{spectre du droit}, combattant à leur côté comme une déesse secourable des combats homériques. Et que faites-vous ? Jetez-vous votre lance ? Non ; vous vous prosternez devant le fantôme, dans l’espoir de le gagner à votre cause, afin qu’il combatte pour vous : vous briguez les faveurs du fantôme. Un autre se demanderait simplement : Est-ce que je veux ce que veut mon adversaire ? « Non ! » Hé bien, quand mille diables ou mille dieux combattraient avec lui, je l’attaque !\par
Un « Etat fondé sur le Droit », Etat tel que la \emph{Gazette de Voss} entre autres en pourrait être l’organe, exige qu’un employé ne puisse être révoqué que par un \emph{juge} et jamais par l’\emph{administration.} Vaine illusion ! Supposez qu’on veuille, conformément à la loi, expulser de sa place un fonctionnaire qui a été rencontré ivre : ce sera au juge à entendre les témoins, à condamner, etc. Bref, le législateur devrait déterminer rigoureusement toutes les infractions capables de  motiver un retrait d’emploi, quelque drolatiques qu’elles puissent être (par exemple, avoir ri au nez de son supérieur hiérarchique, n’être pas allé tous les dimanches à la messe et tous les mois à la communion, fréquenter où il ne sied pas, manquer de tenue, faire des dettes, etc.) ; tous les motifs possibles d’exclusion dûment catalogués (en s’aidant s’il le faut des conseils d’un tribunal d’honneur, etc.), la tâche du législateur prendrait fin, et le rôle du juge commencerait : celui-ci aurait uniquement à rechercher si l’accusé s’est « rendu coupable » d’un des « crimes » prévus par la loi, et, le cas échéant, après preuve faite, de prononcer contre lui la révocation « de par la loi ».\par
Le juge est perdu s’il cesse d’être « mécanique » et s’écarte de « la lettre du code ». Car s’il ne fait pas abstraction de toute opinion qu’il peut avoir en tant qu’homme privé, s’il se laisse influencer par cette opinion, il cesse de faire \emph{acte de magistrat ;} comme juge, il ne peut être que l’organe impersonnel de la loi. Mais parlez-moi de ces vieux Parlements français qui ne prétendaient pas qu’un texte eût force de loi avant d’avoir subi leur examen et reçu leur approbation ! Ceux-là au moins jugeaient suivant leur droit à eux, et ne se prêtaient pas à n’être que des machines dans les mains du législateur, encore qu’ils fussent en somme, comme juges, leurs propres machines.\par
On dit, lorsqu’un criminel est puni, qu’il n’a que ce qu’il mérite : le châtiment est son droit ; mais l’impunité est tout autant son droit. Si son entreprise réussit, il est juste qu’il en bénéficie comme il est juste qu’il lui en cuise si elle échoue. Comme on fait son lit on se couche. Lorsque quelqu’un s’expose étourdiment à un danger et y périt, nous disons très bien : il l’a voulu, il n’a que ce qu’il mérite ; s’il avait triomphé du danger, c’est-à-dire si sa \emph{puissance }en avait triomphé, il eut eu également ce qu’il méritait. Si un enfant joue avec un couteau et se coupe,  ce n’est que justice ; mais s’il ne se coupe pas, c’est encore justice. Il est juste, et conforme au droit, que celui qui viole la loi, ayant accepté de courir un risque en subisse les conséquences : pourquoi risquait-il, puisqu’il connaissait les suites possibles de son acte ? Mais le châtiment que nous lui infligeons n’est que notre droit à nous et non le sien. Notre droit réagit contre le sien, et si le droit est contre lui, c’est que — nous avons le dessus.\par

\asterism

\noindent Ce qui dans une société est conforme au droit, ce qui est juste, est formulé par la \emph{Loi}.\par
Quelle que soit la loi, le devoir de tout citoyen loyal est de la respecter. Ainsi l’esprit de légalité de la vieille Angleterre est célèbre. Que l’on rapproche de ce respect de la loi le mot d’Euripide, (Orreste, 412) : « Nous servons les dieux quels qu’ils soient ». \emph{La Loi quelle que’lle soit}, \emph{Dieu quel qu’il soit,} nous en sommes encore là aujourd’hui.\par
On s’efforce de distinguer la \emph{Loi} de l’\emph{ordre} arbitraire, ukase, ordonnance ou décret, en disant que la première émane d’une autorité légitime. Mais toute loi qui régit des actions humaines (loi morale, loi de l’Etat, etc.) est l’\emph{expression d’une volonté}, et, par conséquent, un ordre. Oui, si même c’était moi qui me donnais ces lois, elles ne seraient encore que des ordres que je me serais donnés et auxquels je pourrais un instant après refuser d’obéir. Chacun est libre de déclarer que telle chose lui convient, de s’interdire ensuite par une loi de faire le contraire et de considérer comme son ennemi quiconque transgresse cette loi ; mais nul n’a d’ordres à \emph{me} donner, nul ne peut me prescrire ce que j’ai à faire et m’en faire une loi. Je dois bien accepter qu’il me traite en \emph{ennemi}, mais jamais je ne tolérerai qu’il use de moi comme de sa \emph{créature} et qu’il me fasse une règle de \emph{sa} raison ou de sa déraison.\par
 Les Etats ne peuvent subsister qu’à condition qu’il y ait une \emph{volonté souveraine,} considérée comme traduisant la volonté individuelle. La volonté du maître est — la Loi. A quoi te servent tes lois, si personne ne les suit ? tes ordres, si personne ne se les laisse imposer ? L’Etat ne peut renoncer à la prétention de régner sur la volonté de l’individu, de compter et de spéculer dessus. Il lui est absolument indispensable que nul n’ait de \emph{volonté propre ;} celui qui en aurait une, l’Etat serait obligé de l’exclure (emprisonner, bannir, etc.), et si tous en avaient une, ils supprimeraient l’Etat. On ne peut concevoir l’Etat sans la domination et la servitude, car l’Etat doit nécessairement vouloir être le maître de tous ses membres ; et cette volonté porte le nom de « volonté de l’Etat ».\par
Celui qui doit, pour exister, compter sur le manque de volonté des autres, est tout bonnement un produit de ces autres, comme le maître est un produit du serviteur. Si la soumission venait à cesser, c’en serait fait de la domination.\par
Ma \emph{volonté d’individu} est destructrice de l’Etat ; aussi la flétrit-il du nom d’indiscipline. La volonté individuelle et l’Etat sont des puissances ennemies, entre lesquelles aucune « paix éternelle » n’est possible. Tant que l’Etat se maintient, il proclame que la volonté individuelle, son irréconciliable adversaire, est déraisonnable, mauvaise, etc. Et la volonté individuelle se laisse convaincre, ce qui prouve qu’elle l’est en effet : elle n’a pas encore pris possession d’elle-même, ni pris conscience de sa valeur ; aussi est-elle encore incomplète, malléable, etc.\par
Tout Etat est \emph{despotique}, que le despote soit un, qu’il soit plusieurs, ou que (et c’est ainsi qu’on peut se représenter une république), tous étant maîtres, l’un soit le despote de l’autre. Ce dernier cas se présente par exemple lorsqu’à la suite d’un vote, une volonté exprimée par une assemblée du peuple devient pour l’individu une loi à laquelle il doit obéissance  ou à laquelle son \emph{devoir} est de se conformer. Imaginez même le cas où chacun des individus composant le peuple aurait exprimé la même volonté, supposez qu’il y ait eu parfaite « unanimité » : la chose reviendrait encore au même. Ne serais-je pas lié, aujourd’hui et toujours, à ma volonté d’hier ? Ma volonté dans ce cas serait immobilisée, paralysée. Toujours cette malheureuse \emph{stabilité !} Un acte de volonté déterminé, ma création, deviendrait mon maître ! Et moi qui ai voulu, moi le créateur, je me verrais entravé dans ma course sans pouvoir rompre mes liens ? Parce que j’étais hier un fou, j’en devrais être un toute ma vie ? Ainsi donc, être l’esclave de moi-même est ce que je puis attendre de mieux — je pourrais tout aussi bien dire de pire — de ma participation à la vie de l’Etat. Parce que hier j’ai voulu, aujourd’hui je n’aurai plus de volonté ; maître hier, je serai aujourd’hui esclave.\par
Quel remède à cela ? Un seul : ne reconnaître aucun devoir, c’est-à-dire ne pas me lier et ne pas me regarder comme lié. Si je n’ai pas de devoir, je ne connais pas non plus de loi.\par
« Mais on me liera ! » — Personne ne peut enchaîner ma volonté, et je resterai toujours libre de ne pas vouloir.\par
« Mais tout serait bien vite sens dessus dessous, si chacun pouvait faire ce qu’il veut ! » Et qui vous dit que chacun pourrait tout faire ? N’êtes-vous pas là, et êtes-vous obligé de laisser tout faire ? Défendez-vous et on ne vous fera rien ! Celui qui veut briser votre volonté est votre \emph{ennemi ;} traitez le comme tel. Si quelques millions d’autres sont derrière vous et vous soutiennent, vous êtes une puissance imposante et vous n’aurez pas grande peine à vaincre. Mais si même, grâce à votre puissance, vous parvenez à imposer à l’adversaire, il ne vous considérera pas pour cela, à moins qu’il ne soit un pauvre sire, comme une autorité sacrée. Il ne vous doit ni respect ni hommages, bien  qu’il doive se tenir sur ses gardes en mesurant votre puissance.\par
Nous classons habituellement les états suivant la façon dont le « pouvoir suprême » y est partagé ; s’il appartient à un seul c’est une Monarchie ; s’il appartient à tous, une Démocratie, etc. Ce pouvoir suprême, contre qui s’exerce-t-il ? Contre l’individu et sa volonté d’individu. La puissance de l’Etat se manifeste sous forme de contrainte ; il emploie la « force », à laquelle l’individu, lui, n’a pas le droit de recourir. Aux mains de l’Etat, la force s’appelle « droit », aux mains de l’individu elle s’appelle « crime ». Crime signifie : emploi de sa force par l’individu ; ce n’est que par le crime que l’individu peut détruire la puissance de l’Etat, quand il est d’avis que c’est lui qui est au-dessus de l’Etat et non l’Etat qui est au-dessus de lui.\par
Et maintenant, si je voulais rire, je pourrais, avec une grimace d’orthodoxie, vous exhorter à ne point faire de loi qui contrarie mon développement individuel, ma spontanéité et ma personnalité créatrices. Je ne donne pas ce conseil, car si vous le suiviez vous seriez naïfs, et moi je serais volé. Je ne vous demande absolument rien, car si peu que je vous demande, vous seriez toujours des faiseurs de lois autoritaires ; vous le seriez et vous devez l’être, parce qu’un corbeau ne sait pas chanter, et qu’un voleur ne peut pas vivre sans voler. Je me tournerai plutôt vers ceux qui veulent être égoïstes, et je leur demanderai ce qui leur semble le plus égoïste : se faire donner par vous des lois et, ces lois une fois données, les respecter, ou bien se résoudre à l’\emph{insubordination}, au catégorique refus d’obéir ?\par
De bonnes âmes disent que les lois ne devraient prescrire que ce que le sentiment du peuple estime bon et juste. Mais que m’importe la valeur qu’ont les choses dans le peuple et pour le peuple ? Le peuple sera peut-être ennemi des blasphémateurs ; de là, loi  contre le blasphème. Sera-ce une raison pour que je ne blasphème pas ? Cette loi sera-t-elle pour moi plus qu’un d’ordre » ? Je vous le demande ?\par
Toutes les formes de gouvernement reposent sur ce seul principe que tout droit et toute puissance émanent de la \emph{totalité du peuple}. Car aucun gouvernement n’omet d’en appeler à la foule, et le despote comme le président, l’aristocratie, etc., agissent et ordonnent « au nom de l’Etat ». Ils sont les dépositaires de l’ « autorité publique », et il est parfaitement indifférent que cette autorité soit exercée par le peuple lui-même, c’est-à-dire (à supposer que ce fut pratiquement possible) par tous les individus réunis en \emph{comices,} ou seulement par les représentants de ces individus, représentants nombreux comme dans les aristocraties ou représentant unique comme dans une monarchie. Toujours la totalité est supérieure à l’individu, et sa puissance, qu’on dit \emph{légitime,} est le \emph{droit}.\par
En face de la sacro-sainteté de l’Etat, l’individu isolé n’est qu’un vase d’iniquité ou foisonnent « l’orgueil, la malice, la soif de scandale, la frivolité, » etc., tant qu’il ne s’est pas prosterné devant l’arche sainte, l’Etat. La superbe ecclésiastique des serviteurs et sujets de l’Etat a des châtiments exquis pour l’ « orgueil » séculier.\par
Quand le gouvernement déclare punissable tout jeu d’esprit \emph{contre} l’Etat, les Libéraux modérés viennent nous dire : Cependant, la fantaisie, la satire, l’esprit, l’humour, etc., devraient pouvoir jaillir ! On devrait accorder la liberté au \emph{génie !} Ainsi, ce n’est pas l’\emph{homme individuel,} mais seulement le \emph{génie} qui doit être libre ?\par
L’Etat est pleinement dans son droit lorsqu’il nous dit, ou plutôt lorsque le gouvernement nous dit en son nom : Celui qui n’est pas pour moi est contre moi. Les chansons, les caricatures, tous ces jeux d’esprit qui prennent l’Etat pour plastron ont jadis  porté les Etats en terre et ne sont pas du tout des « jeux innocents ». Où est d’ailleurs la limite entre la plaisanterie nuisible et la plaisanterie inoffensive ? Cette question jette les Modérés dans une grande perplexité ; ils finissent par rabattre de leurs prétentions et par prier tout bonnement l’Etat (le gouvernement) de ne pas être aussi \emph{susceptible}, aussi \emph{chatouilleux,} de ne pas soupçonner de malveillance là où il n’y en a pas la moindre, et d’être en général un peu plus « tolérant ». Une susceptibilité exagérée est certes une faiblesse ; en être exempt peut être une vertu louable. Mais en temps de guerre on ne peut pas faire le généreux, et ce qu’on pouvait laisser passer en fermant les yeux tant que régnait le calme cesse d’être permis sitôt l’état de siège proclamé. Les Libéraux modérés le savent si bien qu’ils se hâtent de déclarer que, vu la « soumission du peuple », aucun danger n’est à craindre. Mais le gouvernement est trop fin pour s’y laisser prendre ; il sait trop bien comment on paie les gens de belles paroles pour se contenter, lui, de cette monnaie de singe.\par
Cependant, on veut avoir comme à l’école un préau où l’on puisse jouer, car on est en somme un enfant, et on ne peut pas toujours être aussi posé qu’un vieillard. Jeunesse et sagesse ne vont guère de compagnie.\par
C’est pour ce lieu de récréation, pour ces quelques heures de joyeux ébats que l’on marchande. Tout ce qu’on demande, c’est que l’Etat ne se montre pas trop grondeur comme un vieux papa bougon, qu’il tolère ce que l’Eglise tolérait au moyen-âge, quelques cortèges de l’âne et quelques fêtes des fous. Mais le temps n’est plus où l’on pouvait sans danger traîner la Mère Sotte sur les tréteaux. Les enfants d’aujourd’hui, dès qu’ils ont eu une heure de \emph{sortie}, dès qu’ils ont vécu une heure sans voir le fouet, ne veulent plus rentrer dans la « \emph{boîte} ». Car à présent la « sortie » n’est plus un complément de la « boîte », elle n’est plus un  délassement, une relâche entre deux pensums, mais l’opposé, la négation du pensum : \emph{aut} — \emph{aut.} Bref, l’Etat doit aujourd’hui ou bien ne plus rien tolérer, ou bien tolérer tout et s’effondrer ; il doit choisir entre une extrême irritabilité et l’insensibilité de la mort. Le temps de la tolérance est passé. Si l’Etat tend un doigt, on prendra immédiatement toute la main. Ce n’est plus le moment de « rire », et toute plaisanterie, esprit, fantaisie, humour, etc., devient une chose amèrement grave.\par
Quand les « esprits libéraux » réclament la liberté de la presse, ils se mettent en contradiction avec leur propre principe et leur volonté formelle. Ils veulent ce qu’ils ne veulent pas : ils souhaitent que —, ils aimeraient à —, etc. De là leur inconsistance : sitôt la liberté de la presse accordée, ils demandent la censure. C’est tout naturel, l’Etat leur étant sacré, de même que la morale, etc. Leur façon d’agir envers lui est celle de gamins mal élevés, d’enfants gâtés qui cherchent à mettre à profit les faiblesses de leurs parents. Papa l’Etat doit leur permettre de dire un tas de choses désagréables, mais papa l’Etat a aussi le droit de leur imposer silence d’un coup d’œil sévère, et de biffer d’un trait de censure tout leur impertinent babil. S’ils le reconnaissent comme leur papa, ils doivent, comme des enfants, soumettre à sa censure toutes leurs paroles.\par

\asterism

\noindent Si tu permets à un autre de te donner raison, tu dois consentir de même à ce qu’il te donne tort. Si tu acceptes son approbation et ses récompenses, tu dois accepter également à ses reproches et à ses châtiments. Le non-droit marche à côté, du droit, et le \emph{crime} suit la légalité comme son ombre. Qu’es-tu ? — Tu es un \emph{criminel !} Bettina\footnote{ \noindent \emph{Elisabeth d’Arnim, N. d. Tr.}
 } dit : « Le criminel est  le crime de l’Etat lui-même\footnote{ \noindent \emph{Dies Buch gehört dem König}, p. 376.
 }. » On peut adopter la phrase, sans toutefois l’entendre exactement comme celle qui l’écrivit. En effet, le moi sans frein, Moi, tel que je m’appartiens à moi seul, je ne puis me compléter et me réaliser dans l’Etat. Chaque moi est foncièrement criminel envers le peuple, et l’Etat. Aussi l’Etat les surveille-t-il tous ; il voit en chacun un — égoïste et il redoute l’égoïste. Il présume de chacun le pire et prend toutes ses précautions, précautions policières, pour que « aucun tort ne soit fait à l’Etat », \emph{ne quid respublica detrimenti capiat.}\par
Le Moi sans frein — que nous sommes tous de naissance et que nous restons toujours dans notre for intérieur — est dans l’Etat un criminel incorrigible. Quand un homme prend pour guides son audace, sa volonté, son absence de scrupules et son ignorance de la peur, l’Etat et le peuple l’entourent d’espions. Le peuple ? — Oui, braves gens, le peuple ! — vous ne savez guère tout ce que vous lui devez ! — le peuple est policier dans l’âme, et celui-là seul qui renie son moi, qui pratique le « renoncement » obtient ses suffrages.\par
Dans le livre en question, Bettina, qui a bon cœur, ne considère l’Etat que comme un malade, et compte sur sa guérison, guérison qu’elle attend de la thérapeutique des « démagogues\footnote{ \noindent P. 376.
 } ». Seulement l’Etat n’est pas malade ; il est au contraire en parfaite santé, du moment qu’il tient à distance les démagogues qui veulent le saigner au profit des individus, au profit de « tous ». Ses fidèles sont les meilleurs démagogues, les meilleurs pasteurs du peuple qu’il peut désirer. A en croire Bettina\footnote{ \noindent P. 374.
 }, « l’Etat doit développer le germe de liberté que renferme l’humanité, sinon il n’est qu’une marâtre ». Il ne saurait être autre  chose, car par le fait même qu’il se préoccupe du bien de l’ « humanité » l’individu n’est plus pour lui que......\footnote{ \noindent \emph{Jeu de mot intraduisible : « Rabenmutter », mère corbeau marâtre, — «  Rabenfutter », nourriture du corbeau, charogne. N. d. T.}
 }. Qu’elle est donc juste, la réponse du bourgmestre\footnote{ \noindent P. 381.
 } : « Quoi ! l’Etat n’aurait d’autre devoir que celui de garder des malades incurables ? — Allons donc ! De tout temps, l’Etat sain s’est amputé de ses membres malades et ne s’est pas fait infirmier. Il n’a pas besoin d’être si économe de sa sève : coupons sans hésiter les branches gourmandes, afin que les autres fleurissent. — Que l’on ne s’étonne pas de la dureté de l’Etat : sa morale, sa politique et sa religion lui en font une loi ; qu’on ne l’accuse pas d’insensibilité : son cœur souffre de faire des victimes, mais l’expérience lui a appris, que cette dureté est le seul salut. Il est des maladies qu’on ne peut guérir que par l’emploi de remèdes héroïques. Le médecin qui, ayant reconnu une de ces maladies, ne recourt qu’à un palliatif anodin ne la guérira jamais, et laissera mourir son patient après de plus ou moins longues souffrances ! » C’est parfait, mais je n’en dirai pas autant de la question de Madame la Conseillère\footnote{ \noindent \emph{Die Frau Rath, la Conseillère Goethe (la mère du poète) est le personnage principal du livre d’Elisabeth d’Arnim. N. d. Tr.}\par
 P. 385.
 } : « Est-ce guérir que d’employer la mort comme moyen curatif ? » Eh, chère Madame, ce n’est pas à lui-même que l’Etat donne la mort, mais à un membre gangrené ! Il arrache l’œil qui le scandalise, etc.\par
« La seule voie de salut pour l’Etat malade est de faire prospérer en lui-même l’homme\footnote{ \noindent P. 385.
 } ». Si, comme le fait Bettina, on entend ici par l’homme l’idée « Homme », elle a raison : l’Etat « malade » guérira à mesure que « l’Homme » sera plus florissant, car plus les individus sont possédés de « l’Homme » plus l’Etat s’affermit. Mais si par l’homme on entend l’individu, la foule des unités humaines (et c’est vers ce  sens là que l’auteur incline peu à peu, parce qu’elle n’a pas clairement défini en partant le terme qu’elle emploie), la phrase citée plus haut équivaut à peu près à ce qui suit : La seule voie de salut pour une bande de brigands malades est de faire prospérer en elle le bourgeois loyal. Par là même, la bande de brigands, en tant que bande de brigands, périrait ; comme elle s’en doute, elle juge préférable de fusiller celui de ses membres qui trahit quelques velléités de rentrer dans « le droit chemin ».\par
Bettina, dans ce livre, est patriote et même philanthrope ; elle a en vue le bonheur des hommes. Elle est aussi mécontente de l’ordre établi que son héroïne l’est de tous ceux qui voudraient ramener les bonnes vieilles croyances avec tout ce qui s’en suit. Mais elle attribue la corruption de l’État aux gens de la politique, de l’administration, de la diplomatie, tandis que ceux-ci font ce reproche aux méchants, aux « séducteurs du peuple ».\par
Qu’est-ce que le criminel de droit commun ? C’est celui qui commet l’erreur fatale de toucher à ce qui est au peuple au lieu de chercher ce qui est à lui. Il a convoité le bien méprisable, le bien \emph{d’autrui ;} il a fait ce que font les dévots : aspiré à ce qui appartient à Dieu. Que fait le prêtre qui exhorte un criminel ? Il lui représente le tort immense qu’il a eu de profaner ce que l’Etat avait consacré, de porter une main sacrilège sur la propriété de l’Etat (on comprend également sous ce titre la vie de ceux qui font partie de l’Etat). Ne vaudrait-il pas mieux que le prêtre lui fit comprendre que s’il s’est dégradé, c’est en ne \emph{méprisant} pas le bien d’autrui, en le tenant pour digne d’être volé ? C’est ce qu’il pourrait faire, s’il n’était pas un prêtre. Parlez au dit criminel comme vous parleriez à un égoïste, et il aura honte, non pas d’avoir attenté à vos lois et à vos biens, mais d’avoir jugé vos lois dignes d’être violées et vos biens dignes d’être convoités ; il aura honte de  ne pas avoir — méprisé et vous et le vôtre, et d’avoir été trop peu égoïste. Mais vous ne sauriez lui parler la langue de l’égoïsme, car vous n’êtes pas aussi grands qu’un criminel, vous qui ne — profanez rien. Vous ne savez pas que le moi ne peut cesser d’être criminel, que, pour lui, vivre c’est transgresser. Vous devriez pourtant le savoir, vous qui croyez que « tous sans exception nous sommes des pécheurs » ; mais vous pensez pouvoir d’un coup d’aile vous élever au dessus du péché, vous ne saisissez pas, dans votre peur du Diable, qu’à sa culpabilité se mesure la valeur d’un homme. Ah ! si vous étiez coupables ! Mais non, vous êtes des « justes ». Hé bien, — tâchez que votre maître soit content de vous !\par
Dans un code criminel rédigé par la conscience chrétienne ou par l’homme selon le Christ, la notion de \emph{crime} est intimement liée à celle de — \emph{manque de cœur} et ne peut en être séparée. Est crime toute atteinte à l’amour, toute rupture d’un \emph{rapport sentimental}, toute marque d’insensibilité envers un être sacré. Plus un rapport implique de cordialité, plus celui qui le foule aux pieds est coupable et plus son crime est digne de châtiment. Le maître a droit à l’amour de chacun de ses sujets ; renier cet amour est un crime de haute trahison qui mérite la mort.\par
L’adultère est un manque de cœur punissable ; il faut pour le commettre n’avoir pas de cœur, n’avoir ni enthousiasme ni pathos pour la sainteté du mariage. Tant que c’est le cœur ou le sentiment qui dicte les lois, l’homme de cœur ou l’homme sentimental jouit seul de leur protection. Dire que l’homme sentimental fait les lois, revient proprement à dire que c’est l’homme \emph{moral} qui les fait : ce qu’elles condamnent, c’est ce qui choque le « sens moral » de cet homme. Comment, par exemple, l’infidélité, l’apostasie, le parjure, en un mot toute \emph{rupture radicale}, tout ce qui tranche brutalement les liens les plus vénérables ne serait-il pas scélérat et  criminel ? Celui qui rompt avec ces exigences du cœur se fait des ennemis de tous les hommes moraux, de tous les hommes de sentiment.\par
Les Krummacher et consorts sont les gens qu’il faut pour rédiger un code pénal du cœur et lui donner de l’homogénéité ; un certain projet de loi en témoigne. Pour être conséquente, la législation de l’Etat chrétien doit être exclusivement l’œuvre de \emph{ — prêtres ;} elle ne sera homogène, n’aura d’esprit de suite qu’à la condition d’avoir été élaborée par des — \emph{serviteurs des prêtres} qui ne sont jamais que des \emph{demi-prêtres}. Alors, et alors seulement, tout défaut de sensibilité, tout manque de cœur sera qualifié crime inexpiable, toute révolte de sentiment sera condamnable, toute objection de la critique et du doute sera frappée d’anathème ; alors enfin l’individu sera convaincu devant le tribunal de la conscience chrétienne d’être foncièrement — criminel.\par
Les hommes de la Révolution ont souvent parlé des « justes représailles » du peuple comme de son « droit ». Ici vengeance et droit se confondent. Est-ce là le rapport d’un Moi à un autre Moi ? Le peuple s’écrie que le parti adversaire a commis envers lui un « crime ». Puis-je admettre explicitement que quelqu’un est criminel à mon égard sans admettre implicitement qu’il est tenu d’agir comme bon me semble ? C’est cette dernière façon d’agir que j’appelle bonne, juste, etc., et je nomme crime la conduite contraire. Je pense donc que les autres devraient viser au même but que moi, c’est-à-dire que je ne les traite pas comme des Uniques ayant en eux-mêmes leur loi et leur norme de vie, mais comme des êtres qui doivent obéir à une loi « rationnelle » quelconque. Je définis ce qu’on doit entendre par « Homme » et par « vraiment humain », et j’exige que chacun fasse de ce que j’ai érigé en loi sa règle et son idéal, sous peine de n’être qu’ « un pécheur et un criminel ». Et le « coupable » tombe sous « le coup de la loi ».\par
 Il appert que c’est de nouveau « l’Homme » qui engendre les concepts de crime, de péché, et, par suite, de droit. Un homme en qui je ne reconnais pas l’Homme est un « pécheur », un « coupable ».\par
On ne peut être criminel qu’envers quelque chose de sacré. Envers Moi, tu ne seras jamais criminel, tu ne peux être que mon adversaire. Mais il y a déjà crime à ne point haïr celui qui offense une chose sacrée, l’apostrophe de Saint Just à Danton en témoigne : « N’es-tu point criminel et responsable de n’avoir pas haï les ennemis de la patrie ? »\par
Si l’on adopte l’idée de la Révolution et si l’on entend par « Homme » le « bon citoyen », de cette conception de l’Homme vont découler tous les « délits et crimes politiques ».\par
En tout cela, c’est l’Individu, l’homme individuel qui passe pour le monstre, tandis que l’homme abstrait est décoré du titre d’ « Homme ». Quelque nom qu’on donne à ce fantôme, qu’on l’appelle Chrétien, Juif, Musulman, bon citoyen, sujet loyal, affranchi, patriote, etc., devant l’ « Homme » victorieux tombent aussi bien ceux qui voudraient réaliser une conception différente de l’Homme que ceux qui veulent \emph{se} réaliser eux-mêmes.\par
Et avec quelle onction on se coupe la gorge au nom de la loi, du peuple souverain, de Dieu, etc. !\par
Lorsque les persécutés ont recours à la ruse pour échapper à la sévérité de leurs cafards de juges, on les accuse d’ « hypocrisie » ; c’est par exemple le reproche que fait Saint Just à ceux qu’il accuse dans son discours contre Danton. On doit être un fou et se livrer à leur Moloch.\par
Les crimes ont leur source dans les \emph{idées fixes}. La sainteté du mariage est une idée fixe. De ce que la foi conjugale est sacrée, il s’ensuit que la trahir est \emph{criminel ;} et, en conséquence, une certaine loi matrimoniale frappe l’adultère d’une \emph{peine} plus ou moins grave. Mais ceux qui proclament la « liberté » sacrée  doivent considérer cette peine comme un crime contre la liberté, et ce n’est qu’à ce point de vue que l’opinion publique réprouve la loi en question.\par
La Société veut, il est vrai, que \emph{chacun} obtienne son droit, mais ce droit n’est que celui que la Société a sanctionné, c’est le droit de la Société et non \emph{de chacun}. Moi, au contraire, c’est fort de ma propre puissance que je prends ou que je me donne un droit, et, vis-à-vis de toute puissance supérieure à la mienne, je suis un criminel incorrigible. Possesseur et créateur de mon droit, je ne reconnais d’autre source du droit que — Moi, et non Dieu, ni l’Etat, ni la Nature, ni même l’Homme avec ses « éternels droits de l’Homme » ; je ne connais pas plus de droit humain que de droit divin.\par
Droit « en soi et pour soi » : Donc, nullement relatif à moi ! — Droit « absolu » : Donc, séparé de Moi ! Un être en soi et pour soi ! Un Absolu ! Un Droit éternel à côté d’une Vérité éternelle !\par
Le Droit, tel que le conçoivent les Libéraux, m’oblige, parce qu’il est une émanation de la \emph{Raison humaine,} en face de laquelle ma raison n’est que « déraison ». C’est au nom de la Raison divine que l’on condamnait jadis la faible raison humaine ; c’est au nom de la puissante Raison humaine que l’on condamne aujourd’hui la raison égoïste sous le nom méprisant de « déraison ». Et cependant il n’y a de raison réelle que précisément cette « déraison ». Ni la raison divine ni la raison humaine n’ont de réalité ; seules ta raison et ma raison sont réelles, de même que et parce que toi et moi seuls sommes réels.\par
Par son origine, le Droit est une pensée ; c’est ma pensée, c’est-à-dire qu’elle a sa source en moi. Mais sitôt qu’elle a jailli hors de moi, sitôt le « mot » prononcé, « le verbe se fait chair » et cette pensée devient \emph{idée fixe}. Dès lors, je ne puis plus m’en  débarrasser ; de quelque côté que je me tourne, elle se dresse devant moi. C’est ainsi que les hommes en sont venus à ne plus être capable de maîtriser cette idée de Droit qu’eux-mêmes avaient créée ; leur propre créature les a réduits en esclavage. C’est là le Droit absolu, \emph{absolutum}, délié, détaché de moi. Tant que nous le respectons comme absolu, nous ne pouvons plus l’employer, le « consommer », il nous dépouille de notre puissance de créateurs : la créature est plus que le créateur, elle est « en soi et pour soi ».\par
Ne laisse donc plus le droit vaguer en liberté, ramène-le à sa source, c’est-à-dire à toi, et il sera \emph{ton }droit : sera juste ce qui te sera — « juste ».\par

\asterism

\noindent Le Droit a été attaqué sur son propre terrain et avec ses propres armes, lorsque le Libéralisme a déclaré la guerre au « privilège. »\par
\emph{Privilège} et \emph{égalité des droits} — autour de ces deux idées se livre un combat acharné.\par
Mais est-il au monde une puissance, une seule, que ce soit une puissance imaginaire comme Dieu, la Loi, etc., ou une puissance réelle comme toi et moi, devant laquelle tous ne soient pas parfaitement « égaux en droits », c’est-à-dire équivalents sans acception de personne ? Dieu aime également chacun de ceux qui l’adorent, la Loi regarde d’un œil également favorable quiconque est légal ; que l’adorateur de Dieu ou de la Loi soit bossu ou paralytique, pauvre ou riche, peu importe à Dieu et à la Loi ; de même, si tu es en train de te noyer, tu aimes autant avoir pour sauveur un nègre que le plus pur des caucasiens ; un chien même n’a pas pour toi en ce moment moins de valeur qu’un homme.\par
Mais, inversement, est-il au monde quelqu’un qui puisse ne pas éprouver pour chacun soit une « prédilection » soit une « répulsion » ? Dieu poursuit les méchants de sa colère, la Loi punit celui qui sort de  la légalité ; et toi-même, ta porte n’est-elle pas ouverte à toute heure à l’un et toujours fermée à l’autre ?\par
L’ « égalité des droits » n’est qu’un leurre, car droit ne signifiant ni plus ni moins qu’autorisation, le droit qu’on nous reconnaît n’est qu’une \emph{faveur }qu’on nous accorde. Cette faveur, on peut d’ailleurs la devoir à son mérite, car le mérite et la grâce ne sont nullement contradictoires, la grâce qu’on nous fait devant être elle aussi « méritée » : nous n’accordons la faveur d’un sourire qu’à celui qui a su nous l’extorquer.\par
On rêve de voir mettre « tous les citoyens sur un même pied d’égalité. » Evidemment, en tant que citoyens ils sont tous égaux devant l’Etat, mais celui-ci, suivant le but particulier qu’il poursuit, distingue déjà entre eux, choisit les uns et néglige les autres ; il doit en outre distinguer encore entre eux pour séparer les bons citoyens des mauvais, etc.\par
Br. Bauer se base, pour résoudre sa Question juive, sur l’illégitimité du « privilège. » Comme le Juif et le Chrétien ont chacun quelque chose que n’a pas l’autre, un \emph{avantage} sur l’autre, et comme c’est exclusivement cet avoir particulier, cet avantage qui fait de chacun ce qu’il est et l’oppose à l’autre, leur valeur est nulle aux yeux de la Critique. Le reproche qu’on peut leur adresser s’adresse également à l’Etat qui légitime cet avantage et le consacre comme un « privilège », s’interdisant par là même tout espoir de devenir jamais « Etat libre ».\par
Mais si l’un a quelque chose de plus que l’autre, c’est soi-même, c’est son unicité : par là seulement chacun reste exceptionnel, exclusif.\par
Chacun fait de son mieux valoir sa caractéristique devant un tiers, et tâche, s’il veut se le rendre favorable, de la lui faire paraître aussi attrayante que possible.\par
Ce tiers doit-il être insensible à la différence qu’il  constate entre le premier et le second ? Est-ce là ce qu’on exige de l’Etat libre ou de l’Humanité ? Ils devraient en ce cas ne s’intéresser absolument à rien, être fermés à toute sympathie pour quoi que ce soit. On ne s’imagine une telle indifférence ni de la part de Dieu qui sépare les siens des méchants, ni de la part de l’Etat qui distingue les bons citoyens des mauvais.\par
Ce qu’on cherche, pourtant, c’est précisément ce tiers qui n’accorderait plus aucun « privilège ». Et on l’appelle Etat libre, Humanité ou autrement.\par
Les Juifs et les Chrétiens que Br. Bauer a foudroyés de son mépris pour leur prétention à des « prérogatives » devraient pouvoir et vouloir renoncer, par abnégation ou désintéressement, au point de vue où ils se cantonnent. S’ils dépouillaient leur « égoïsme », ce serait la fin de leurs torts réciproques, et, du même coup, la fin de toute religiosité juive et chrétienne. Il suffirait qu’ils cessassent de se prétendre des êtres « à part ».\par
Mais à supposer qu’ils renonçassent à leur exclusivisme, ils n’abandonneraient pas pour cela le champ de bataille où leur hostilité s’est si longtemps exercée ; ils trouveraient tout au plus un terrain neutre sur lequel ils pourraient s’embrasser : une « religion universelle », une « religion de l’Humanité », que sais-je, — un accommodement, en un mot, et on se contenterait du premier venu ; déjà, par exemple, la conversion de tous les Juifs au christianisme (autre moyen de mettre fin aux « privilèges » des uns et des autres) serait un compromis très sortable ! On supprimerait en vérité la \emph{discordance,} mais cette discordance n’est pas l’essence des deux antagonistes, elle n’est que l’effet de leur rapprochement. Etant différents, ils doivent nécessairement être en désaccord, et l’inégalité subsistera toujours. Ce n’est pas un tort, il est vrai, de te roidir contre moi, et d’affirmer ta particularité, ton individualité :  tu n’as pas à céder ni à te renier toi-même.\par
On prend l’antithèse dans un sens trop formel et trop restreint lorsqu’on s’attache simplement à la « résoudre » pour faire place à une « synthèse. » Il faudrait au contraire accentuer encore l’opposition. En tant que juif et chrétien, vous n’êtes pas encore assez radicalement opposés, vous n’êtes en désaccord qu’au sujet de la religion, et c’est comme si vous querelliez pour la barbe de l’empereur ou quelqu’autre bagatelle. Ennemis au point de vue de la religion, vous êtes, quant au reste bons amis, et, comme hommes, par exemple, égaux. Cependant ce \emph{reste }lui-même diffère du tout au tout, et ce n’est que lorsque vous vous connaîtrez à fond, quand chacun de vous s’affirmera \emph{unique} des pieds à la tête, que pourra cesser cette opposition que vous n’avez fait jusqu’à présent que dissimuler. Alors enfin l’antithèse sera résolue, mais pour cette seule raison qu’une plus forte l’aura absorbée.\par
Notre faiblesse n’est pas d’être opposés aux autres, mais bien de ne pas leur être radicalement opposés, c’est-à-dire de ne pas en être totalement \emph{distincts}, ou encore de chercher un « trait d’union », un \emph{lien}, et de faire de ce trait d’union notre idéal. Une Foi, un Dieu, une Idée, un seul chapeau pour toutes les têtes ! Si toutes les têtes étaient sous le même bonnet, personne, il est vrai, n’aurait plus à se découvrir devant les autres !\par
La dernière opposition et la plus radicale, celle de l’Unique à l’Unique, est au fond bien éloignée de ce qu’on entend par opposition, sans pour cela retomber dans l’unité et l’identité. En tant qu’Unique, tu n’as plus rien de commun avec personne, et par la même plus rien d’inconciliable ou d’hostile. Tu ne demandes plus contre lui ton droit à un \emph{tiers}, et ne te tiens plus avec lui sur le « terrain du droit » ni sur aucun autre terrain commun. L’opposition se résout en une séparation, en une unicité radicale.  Celle-ci, il est vrai, pourrait passer pour un nouveau trait commun, pour un trait de ressemblance ; mais la ressemblance consisterait ici précisément dans la dissemblance et ne serait elle-même que dissemblance ; une dissemblance semblable, mais aux yeux seulement de ceux qui s’amusent à faire des comparaisons.\par
La polémique contre le privilège est un des traits caractéristiques du Libéralisme ; il excommunie le « privilège » par dévotion pour le « Droit » ; mais il doit s’en tenir à cette excommunication platonique, car les privilèges, n’étant que des aspects particuliers du droit, ne tomberont pas avant le droit lui-même. Le Droit ne rentrera dans son néant que lorsqu’il aura été absorbé par la Force, c’est-à-dire quand on aura compris que « la force prime le droit ». Alors, tout droit se révélera privilège, et le privilège lui-même apparaîtra sous son vrai jour comme puissance, — \emph{puissance prépondérante.} Mais ce combat de la puissance contre la puissance plus grande ne se présentera-t-il pas sous un tout autre aspect que ce timide combat contre le privilège qui ne se livre que devant un juge, le « Droit », et selon l’esprit de ce juge ?\par

\asterism

\noindent Il me reste, pour finir, à rayer de mon vocabulaire ce mot Droit dont je n’ai voulu faire usage qu’aussi longtemps que, fouillant les entrailles de la chose, je ne pouvais faire autrement que d’en laisser provisoirement subsister au moins le nom. Mais à présent le mot n’a plus de sens et peut aller rejoindre la notion. Ce que j’appelais « mon droit » n’est plus nullement un « droit », car un droit ne peut être conféré que par un Esprit, que cet Esprit soit celui de la nature, celui de l’espèce, de l’humanité, ou de Dieu, de Sa Sainteté, de Son Eminence, etc. Ce que je possède indépendamment de la sanction de l’Esprit, je le possède  sans droit, je le possède uniquement par ma \emph{puissance.} Je ne réclame aucun droit, et n’ai donc à en reconnaître aucun. Ce dont je puis m’emparer, je le saisis et je me l’approprie ; ce qui m’échappe, je n’y ai aucun droit, et ce ne sont pas mes droits imprescriptibles dont je m’enorgueillis ou qui me consolent.\par
Le Droit absolu entraîne dans sa chute les droits eux-mêmes, et avec eux s’écroule la souveraineté de l’ « idée de droit ». Car il ne faut pas oublier que nous avons été jusqu’ici gouvernés par des idées, des notions, des principes, et que parmi tant de maîtres l’idée de droit ou l’idée de justice a joué un des principaux rôles.\par
Légitime ou illégitime, juste ou injuste, que m’importe ? Ce que me permet ma \emph{puissance}, personne d’autre n’a besoin de me le \emph{permettre ;} elle me donne la seule autorisation qu’il me faille. Le droit est une marotte dont nous a gratifié un fantôme ; la force, c’est moi-même, moi qui suis puissant, qui suis possesseur de la puissance.\par
Le droit est au-dessus de moi, il est absolu, il n’existe que chez un être supérieur qui me l’accorde comme une faveur ; c’est une grâce que me fait le juge. La puissance et la force n’existent qu’en Moi, qui suis le Puissant et le Fort.
\subsubsection[{B.II.2. Mes Relations}]{B.II.2. Mes Relations}
\noindent Dans le monde et dans la société, il nous est tout au plus permis de satisfaire les exigences de l’homme ; celles de l’égoïste doivent toujours être sacrifiées.\par
Il n’est personne qui n’ait remarqué l’intérêt passionné que l’époque actuelle témoigne pour la « question sociale » de préférence à toute autre question, et qui n’ait en conséquence dirigé spécialement son attention sur la société. Pourtant, si cet intérêt était  moins aveuglé par la passion, on ne perdrait pas de vue l’individu pour ne plus voir que la Société, et on reconnaîtrait qu’une société ne peut guère se renouveler tant que ses éléments vieillis ne sont pas remplacés par d’autres. A supposer, par exemple, qu’il se formât au sein du peuple juif une société destinée à répandre sur la terre un nouvel évangile, il ne faudrait pas que ces apôtres demeurassent des Pharisiens.\par
Tel tu es, et tel tu te montreras et te comporteras envers les hommes : hypocrite, en hypocrite, chrétien, en chrétien. C’est pourquoi le caractère d’une société est déterminé par le caractère de ses membres : ils en sont les créateurs. C’est là un point dont il serait bon de tenir compte, si l’on ne veut pas analyser la notion même de société.\par
Les hommes, bien loin de chercher à atteindre \emph{leur} complet développement et à se faire valoir n’ont pas même su jusqu’à présent fonder leurs sociétés sur \emph{eux-mêmes ;} bien plus, ils n’ont su jusqu’ici que fonder des « sociétés » et vivre en société. Ces sociétés furent de tous temps des personnes, personnes puissantes, personnes « morales », qui inspiraient à l’individu la folie congrue, la peur des fantômes.\par
Tout fantôme qui se respecte porte un nom ; ceux-ci s’appellent « Peuples ». Il y eut un Peuple des aïeux, un Peuple des Hellènes, etc., et il y a aujourd’hui le Peuple des Humains, l’Humanité (Anacharsis Clootz rêvait une « Nation de l’humanité ») ; vinrent ensuite les subdivisions de ce Peuple, qui renfermait ses sociétés particulières : peuple espagnol, peuple français, etc. ; ces dernières comprirent à leur tour les castes, les villes, les corporations, bref tous les groupements possibles, jusqu’au minuscule petit peuple qu’est la \emph{Famille.}\par
Au lieu de dire que la personne qui a jusqu’à présent hanté toutes les sociétés fut le Peuple, on pourrait nommer à sa place l’un des deux extrêmes, la Famille ou l’Humanité, qui sont les deux unités les  plus naturelles. Nous préférons cependant le mot « Peuple » ; d’abord, parce que son étymologie le rattache au mot grec \emph{Polloi}, le nombre, la foule, ensuite et surtout parce que les « revendications populaires » sont aujourd’hui à l’ordre du jour et que les révolutionnaires les plus contemporains n’ont pas encore dépouillé leur idolâtrie pour cette illusoire personne ; cette dernière considération serait cependant plutôt faite pour nous incliner à choisir le terme « Humanité », l’ « Humanité » étant aujourd’hui ce que tout le monde rêve.\par
Ainsi donc, le Peuple, — l’Humanité ou la Famille, — ont jusqu’à présent, semble-t-il, occupé la scène de l’histoire : aucun intérêt \emph{égoïste} n’était toléré dans ces sociétés ; seuls, les intérêts d’une collectivité, intérêts nationaux ou intérêts du peuple, intérêts de caste, intérêts de famille et « intérêts généraux de l’humanité » pouvaient y jouer un rôle. Mais qui donc a conduit à leur perte les peuples dont l’histoire nous conte la chute ? L’égoïste qui cherche sa propre satisfaction. Chaque fois qu’un intérêt égoïste s’y insinua, la société fut « corrompue » et marcha à sa désorganisation ; le peuple romain et le perfectionnement de son droit privé, ou le christianisme et son incessante invasion par le « libre examen », la « conscience de soi », l’ « autonomie de l’esprit », etc. en sont d’illustres exemples.\par
Le Peuple chrétien a produit deux sociétés qui dureront ce que lui-même durera : l’\emph{Etat} et l’\emph{Eglise. }Peut-on les appeler des associations d’égoïstes ? Poursuivons-nous en elles un intérêt égoïste, personnel, individuel, ou y poursuivons-nous un intérêt populaire (parce que du peuple chrétien) sous le nom d’intérêt de l’Etat ou d’intérêt de l’Eglise ? M’est-il en elles possible, m’est-il par elles permis d’être moi-même ? Puis-je penser et agir comme je veux, puis-je me manifester, m’affirmer, vivre ma vie à moi ? Ne dois-je pas laisser intactes la majesté de l’Etat et la sainteté de l’Eglise ?\par
 Ainsi donc, ce que je veux je ne le puis pas. Mais est-il une société, quelle qu’elle soit, où je puisse espérer trouver cette liberté d’action illimitée ? Non ! Et une société est-elle capable de nous satisfaire ? En aucune façon ! C’est tout autre chose de me heurter à un autre Moi, ou de me heurter à un peuple, à une généralité. Dans le premier cas, mon adversaire et moi combattons égal à égal ; dans le second, je suis un adversaire méprisé, enchaîné et tenu en tutelle. Contre un autre Moi, je suis un mâle en face d’un mâle ; en face du Peuple, je suis un écolier qui ne peut s’attaquer à son camarade sans que ce dernier appelle à son secours le papa et la maman : lorsqu’il s’est mis à l’abri derrière une jupe, moi, l’enfant mal appris, on me gronde et on me « défend de raisonner ». Un Moi est un ennemi en chair et en os ; allez donc étreindre et terrasser l’Humanité, une abstraction, une « Majesté », un fantôme ! Mais il n’est pas de Majesté, pas de Sainteté qui puisse me dire : tu n’iras pas plus loin ; rien ne peut m’empêcher de passer dont je puis me rendre maître. Ce que je ne puis vaincre, voilà la seule limite à mon pouvoir ; et tant que mon pouvoir est borné, je ne suis qu’un Moi borné ; — borné, non pas par la puissance \emph{extérieure}, mais parce qu’il me manque encore de puissance propre, par ma propre impuissance. Mais « la garde meurt et ne se rend pas ! » Donnez-moi seulement un adversaire vivant !\par


\begin{verse}
« Je me mesurerai avec tout adversaire\\
» que je puis voir et toiser du regard\\
» qui, lui-même plein de courage enflamme mon courage... etc. »\\
\end{verse}

\noindent Le temps a aboli maint privilège, mais toujours en vue du seul bien public, en vue de l’Etat, du bien de l’Etat, et en aucune façon en vue de mon bien à moi. Le servage, par exemple, ne fut aboli qu’afin de renforcer la puissance d’un maître unique,  du maître du peuple, la puissance monarchique ; le servage ainsi monopolisé n’en devint que plus écrasant. Ce n’est jamais qu’en faveur d’un monarque, que ce monarque s’appelle Prince ou s’appelle Loi, que les privilèges sont tombés. En France, il est vrai, les citoyens ne sont pas serfs du roi, mais en revanche ils sont serfs de la loi (la charte). La subordination persiste ; seulement, l’Etat chrétien ayant reconnu que l’homme ne peut servir deux maîtres à la fois (son suzerain et le prince, par exemple) tous les privilèges ont été donnés à un seul : il peut de nouveau \emph{placer} l’un au-dessus de l’autre, et créer des « haut placés ».\par
Mais que m’importe le bien public ? Par là même qu’il est le bien public, il n’est pas mon bien, mais le suprême degré de l’\emph{abnégation}. Je puis avoir le ventre vide pendant que le bien public festoie, l’Etat illumine peut-être tandis que je crève de faim. La vraie folie des Libéraux politiques, c’est d’opposer le Peuple au gouvernement, et de parler de droits du Peuple. Ils veulent que le Peuple soit majeur, etc. Comme si un mot pouvait être majeur\footnote{ \noindent \emph{Jeu de mot intraduisible : Als könnte mündig sein, wer keinen Mund hat : « Comme si on pouvait être majeur (mündig) quand on n’a pas de bouche (Mund) ». N. d. Tr.}
 } ! Seul l’individu peut être majeur. De même, toute la question de la liberté de la presse n’a plus ni queue ni tête, si on prétend faire de cette liberté un « droit du Peuple » : elle est simplement un droit ou pour mieux dire une conséquence de la force de l’\emph{individu. }Si c’est le Peuple qui a la liberté de la presse, moi, bien que je fasse partie de ce Peuple, je n’en jouis pas ; une liberté du Peuple n’est pas \emph{ma} liberté, et si la liberté de la presse est une liberté du Peuple, elle sera nécessairement flanquée d’une loi sur la presse dirigée contre moi.\par
Ce qu’il faut surtout bien faire valoir en présence des tendances libérales actuelles, c’est que :\par
 La liberté \emph{du Peuple} n’est pas \emph{ma} liberté.\par
Admettons ces catégories, liberté du Peuple et droit du Peuple, et soit par exemple le droit « du Peuple » que chacun a de porter des armes. N’est-ce pas là un droit dont on peut être privé ? Si c’était mon propre droit, je ne pourrais pas le perdre ; mais ce droit ne m’appartient pas, il appartient au Peuple, aussi peut-il m’être enlevé. Je puis être emprisonné en vertu de la liberté du Peuple, et je puis, à la suite d’une condamnation, être déchu du droit de porter des armes.\par
Le Libéralisme me paraît être une dernière tentative pour instaurer la liberté du Peuple, la liberté de la communauté, de la « Société », de la généralité, de l’Humanité. J’y vois le rêve d’une humanité majeure, d’un peuple majeur, d’une communauté, d’une « Société » majeures.\par
Un peuple ne peut être libre qu’aux dépens de l’individu, car sa liberté ne touche que lui et n’est pas l’affranchissement de l’individu ; plus le peuple est libre, plus l’individu est lié. C’est à l’époque de sa plus grande liberté que le peuple grec établit l’ostracisme, bannit les athées et fit boire la cigüe au plus probe de ses penseurs.\par
Combien n’a-t-on pas vanté chez Socrate le scrupule de probité qui lui fit repousser le conseil de s’enfuir de son cachot ! Ce fut de sa part une pure folie de donner aux Athéniens le droit de le condamner. Aussi n’a-t-il été traité que comme il le méritait ; pourquoi se laissait-il entraîner par les Athéniens à engager la lutte sur le terrain où ils s’étaient placés ? Pourquoi ne pas rompre avec eux ? S’il avait su, s’il avait pu savoir ce qu’il était, il n’eût reconnu à de tels juges aucune autorité, aucun droit. S’il fut faible, ce fut précisément \emph{en ne fuyant pas}, en gardant cette illusion qu’il avait encore quelque chose de commun avec les Athéniens, et en s’imaginant n’être qu’un membre, un simple membre de ce peuple. Il était bien  plutôt ce peuple même en personne, et, seul, il pouvait être son juge. Il n’y avait point de juge \emph{au-dessus de lui :} n’avait-il pas d’ailleurs prononcé sa sentence ? Il s’était, lui, jugé digne du Prytanée. Il aurait dû s’en tenir là, et, n’ayant prononcé contre lui-même aucune sentence de mort, il aurait dû mépriser celle des Athéniens et s’enfuir. Mais il se subordonna, et il accepta le Peuple pour juge : il se sentait petit devant la majesté du Peuple. S’incliner comme devant un « droit » devant la force qu’il n’aurait dû reconnaître qu’en y succombant c’était se trahir soi-même, et c’était de la \emph{vertu}. La légende attribue les mêmes scrupules au Christ, qui, dit-on, ne voulut pas se servir de sa puissance sur les légions célestes. Luther fut plus sage ; il eut raison de se faire délivrer un sauf-conduit en bonne forme avant de se hasarder à la diète de Worms, et Socrate aurait dû savoir que les Athéniens n’étaient que ses ennemis et que lui seul était son juge. L’illusion d’une « justice », d’une « légalité », etc., devait se dissiper devant cette considération que toute relation est un rapport de \emph{force}, une lutte de puissance à puissance.\par
La liberté grecque périt misérablement au milieu des chicanes et des intrigues. Pourquoi ? Parce que les conclusions que n’avait pas su tirer Socrate, leur maître dans l’art de penser, le commun des Grecs étaient bien moins capables encore de s’y élever. Qu’est-ce que la chicane, sinon l’art d’exploiter l’actuel sans le détruire ? (Je pourrais ajouter d’exploiter « à son profit », mais ça va de soi.) Parmi ces gens de chicane nous comptons les théologiens qui « commentent et interprètent » la parole de Dieu ; à quoi accrocheraient-ils leurs gloses si la parole divine n’était pas un texte « présent » et censé immuable ? Tels sont encore les Libéraux, qui ne s’attaquent au « présent » que pour le critiquer et l’interpréter ; tous ne sont que des falsificateurs, juste comme ceux qui falsifient le droit. Socrate s’était incliné devant le droit et devant la loi, et les Grecs continuèrent  à admettre l’autorité du droit et de la loi ; mais ils n’entendaient pas prodiguer gratuitement leurs respects, il fallait qu’ils y trouvassent leur compte, et ils durent bien en venir à le chercher dans l’intrigue, et à fausser le droit. Avec Alcibiade, cet intrigant de génie, commence la période de la « décadence » athénienne ; le Spartiate Lysandre et bien d’autres témoignent que l’intrigue s’était répandue comme une lèpre dans toute la Grèce. Le Droit grec, sur lequel reposaient les Etats grecs, fut, dans ces Etats mêmes, miné et ébranlé par les égoïstes, et les Etats croulèrent, mettant en liberté les Individus. Le Peuple grec tomba, parce que les individus faisaient moins de cas de lui que d’eux-mêmes.\par
Etats, Constitutions, Eglises, etc., se sont toujours évanouis dès que l’individu a levé la tête, car l’individu est l’ennemi irréconciliable de tout ce qui tend à submerger sa volonté sous une volonté \emph{générale}, de tout \emph{lien,} c’est-à-dire de toute chaîne. Cependant on s’imagine aujourd’hui encore que l’homme ne peut se passer de « liens sacrés » ! L’homme, cet ennemi mortel de tout « lien » ! L’histoire des peuples nous montre qu’aucun lien n’a pu échapper à la dissolution ; elle nous démontre que l’homme lutte infatigablement contre toute chaîne, quelle qu’elle soit ; et cependant on ferme les yeux devant l’évidence, et l’on en rêve encore et toujours de nouvelles : on croit par exemple avoir trouvé quelque chose de neuf et de solide quand on impose à l’homme ce qu’on appelle une « constitution libérale », une belle chaîne constitutionnelle. Les rubans, les cordons et tous les liens de confiance et d’amour qui unissent les sujets à finissent en vérité par montrer quelque peu la corde ; mais quelques progrès qu’on ait faits en matière de liens, on n’est arrivé, en partant des lisières, qu’à la bretelle ou à la cravate.\par
Tout ce qui est sacré est un lien, une chaîne.\par
Tout ce qui est sacré est falsifié par des faussaires,  et il ne pourrait en être autrement ; aussi trouve-t-on à notre époque une foule de ces faussaires dans toutes les sphères. Ils préparent la rupture avec le droit, la suppression du droit.\par
Pauvres Athéniens, qu’on accuse de chicane et de sophistique ! Pauvre Alcibiade, que l’on accuse d’intrigue ! C’est là justement ce que vous aviez de meilleur, c’était votre premier pas vers la liberté. Votre Eschyle, votre Hérodote et les autres ne rêvaient qu’un Peuple grec libre : vous eûtes, les premiers, un pressentiment de \emph{votre} liberté.\par
Tout Peuple opprime ceux qui s’élèvent au-dessus de sa majesté ; l’ostracisme menace le citoyen trop puissant, l’inquisition de l’Eglise guette l’hérétique, et — l’inquisition également guette le [{\corr traître}] envers l’Etat.\par
Car le Peuple n’a cure que de se maintenir et de s’affermir ; il réclame de chacun un « patriotique dévouement ». L’individu \emph{en soi} lui est donc indifférent, c’est un zéro, et le Peuple ne peut faire ni même permettre que l’individu accomplisse ce qu’il est seul capable d’accomplir, sa \emph{réalisation}. Tout Peuple, tout Etat est « injuste » envers les égoïstes.\par
Tant qu’il reste debout une seule institution qu’il n’est pas permis à l’individu d’abolir, le Moi est encore bien loin d’être sa propriété et d’être autonome. Comment parler de liberté, tant que je dois par exemple me lier par serment à une constitution, à une charte, à une loi, tant que je dois jurer d’appartenir « corps et âme » à mon Peuple ? Comment être moi-même s’il n’est permis à mes facultés de se développer que pour autant qu’elles « ne troublent pas l’harmonie de la Société » (Weitling) ?\par
La chute des peuples et de l’humanité sera le signal de mon élévation.\par
Ecoute ! Au moment même où j’écris ces lignes, les cloches se sont mises à sonner ; elles portent au loin un joyeux message : demain on célèbre le millième  anniversaire de notre chère Allemagne. Sonnez, sonnez, ô cloches, cloches des funérailles ! Votre voix est si solennelle et si grave qu’il semble que vos langues de bronze soient mues par un pressentiment, et que vous escortiez un mort. Peuple allemand et peuples allemands ont derrière eux dix siècles d’histoire ; quelle longue vie ! Descendez donc au tombeau pour ne vous relever jamais, et qu’ils soient libres, ceux que vous avez tenus enchaînés si longtemps ! — Le Peuple est mort, Je me lève.\par
O toi qui as tant souffert, ô mon Peuple allemand, quelle a été ta souffrance ? C’était le tourment d’une pensée qui ne peut se créer un corps, le tourment d’un Esprit errant qui s’évanouit lorsque le coq chante et qui aspire cependant à sa délivrance et à sa réalisation. En moi aussi, tu as longtemps vécu, chère — pensée, cher — fantôme ! Déjà je croyais avoir trouvé la parole magique qui doit te délivrer, déjà je croyais avoir découvert une chair et des membres pour vêtir l’Esprit errant, — et voilà que j’entends le glas des cloches qui te conduisent au repos éternel ; voilà que le dernière espérance s’envole, que le dernier amour s’éteint. Je dis adieu à la maison déserte des morts et je retourne parmi les vivants,\par

« Car seuls les vivants ont raison. »\\

\noindent Adieu donc, rêve de tant de millions d’hommes ; adieu, toi qui pendant mille ans as tyrannisé tes enfants !\par
Demain on te portera en terre ; bientôt tes sœurs les nations te suivront. Quand toutes seront parties à ta suite, l’humanité sera enterrée, et sur sa tombe, Moi, mon seul maître enfin, Moi, son héritier, je rirai.\par
 
\asterism

\noindent Le mot « Gesellschaft », société, a pour étymologie le mot « Saal », salle. Lorsqu’une salle renferme plusieurs personnes, c’est elle qui fait que ces personnes sont en société. Elles \emph{sont} en société, mais elles ne \emph{font} pas la société ; elles en font tout au plus une société de salon si elles parlent le langage que l’on parle dans un salon. Quant aux véritables \emph{relations}, elles sont indépendantes de la société, elles peuvent exister ou ne pas exister sans que la nature de ce qu’on appelle « société » en soit altérée. La société, ce sont les personnes qui se trouvent dans la salle, et peu importe qu’elles soient muettes ou ne prononcent que de banales phrases de politesse. Les \emph{relations}, au contraire, impliquent réciprocité, c’est le commerce \emph{(commercium) }des individus. La société n’est que l’occupation en commun d’une salle ; des statues, dans une salle de musée, sont en société, elles sont « groupées ». Telle étant la signification naturelle du mot société, il s’en suit que la société n’est pas l’œuvre de toi et de moi, mais d’un tiers ; c’est ce tiers qui fait de nous des compagnons, et qui est le vrai fondateur, le créateur de la société.\par
Il en est de même pour une société ou compagnie de prisonniers (ceux qui jouissent d’une même prison). Le tiers que nous rencontrons ici est déjà plus complexe que ne l’était celui de tantôt, le simple local, la salle. Prison ne désigne plus simplement un lieu, mais un lieu en rapport avec ses habitants : la prison n’est prison que parce qu’elle est destinée à des prisonniers, sans lesquels elle serait un bâtiment quelconque. Qui imprime un caractère commun à ceux qui y sont assemblés ? Il est clair que c’est la prison, car c’est à cause d’elle qu’ils sont des prisonniers. Qui détermine la \emph{manière de vivre} de la société de prisonniers ?  Encore la prison. Mais qui détermine leurs \emph{relations ? }Est-ce aussi la prison ? Halte ! ici je vous arrête : Evidemment, s’ils entrent en relations, ce ne peut être que comme des prisonniers, c’est-à-dire que pour autant que le permettent les règlements de la prison ; mais ces relations, c’est \emph{eux-mêmes} et eux seuls qui les créent, c’est le Je qui se met en rapport avec le Tu ; non seulement ces relations ne peuvent pas être le fait de la prison, mais celle-ci doit veiller à s’opposer à toutes relations égoïstes, purement personnelles (les seules qui puissent s’établir réellement entre un Je et un Tu).\par
La prison consent à ce que nous fassions un travail en commun, elle nous voit avec plaisir manœuvrer ensemble une machine ou partager n’importe quelle besogne. Mais si \emph{j}’oublie que je suis un prisonnier et si je noue des relations avec \emph{toi}, également oublieux de ton sort, voilà qui met la prison en péril : non seulement elle ne peut créer de pareilles relations, mais elle ne peut même pas les tolérer. Et voilà pourquoi la Chambre française, saintement et moralement pensante, a adopté le système de la « prison cellulaire » ; les autres, non moins vertueusement intentionnées feront de même pour mettre un obstacle aux « relations démoralisantes ». Dès que l’emprisonnement est une affaire faite, il est sacré, il n’est plus permis de s’y attaquer. La moindre tentative de ce genre est punissable, comme l’est toute révolte contre une des sacro-saintetés auxquelles l’homme doit se livrer pieds et poings liés.\par
La prison, comme la salle, produit une société, une compagnie, une communauté (communauté de travail, par exemple), mais non des \emph{relations}, une réciprocité, une \emph{association.} Au contraire, toute association entre individus née à l’ombre de la prison porte en elle le germe dangereux d’un « complot », et cette semence de rébellion peut, si les circonstances sont favorables, germer et porter des fruits.\par
Ce n’est guère l’usage d’aller volontairement en  prison, et il est également peu commun que l’on y reste volontairement ; on y nourrit plutôt un égoïste désir de liberté. Il est donc à présumer que toutes les \emph{relations} personnelles entre prisonniers seront hostiles à la société réalisée par la prison, et ne tendront à rien moins qu’à dissoudre cette société qui résulte de la captivité commune.\par
Adressons-nous donc à d’autres sociétés, à des sociétés où il semble que nous demeurions volontiers et de notre plein gré, sans vouloir en compromettre l’existence par nos manœuvres égoïstes.\par
Comme communauté remplissant ces conditions se présente en premier lieu la \emph{famille}. Parents, époux, enfants, frères et sœurs forment un tout, ou constituent une famille dont des alliances viennent peu à peu grossir les rangs. La famille n’est réellement une communauté que si tous ses membres en observent la loi, la piété ou l’amour familial. Un fils à qui père, mère, frères et sœurs sont devenus indifférents \emph{a été }fils, mais sa qualité de fils ne se manifestant plus activement a aussi peu d’importance que la liaison depuis longtemps détruite de la mère et de l’enfant par le cordon ombilical. Cette dernière liaison a existé autrefois, elle est un fait qu’il n’est plus possible de défaire et en vertu duquel on reste irrévocablement le fils de cette mère et le frère de ses autres enfants ; mais une dépendance permanente ne peut résulter que de la permanence de la piété, de l’esprit de famille. Les individus ne sont dans toute l’acception du mot membres d’une famille que s’ils se font un devoir de \emph{maintenir} la famille, et s’ils se gardent, comme ses \emph{conservateurs,} d’en remettre le fondement en question. Il est pour tout membre de la famille une chose inébranlable et sacrée : c’est la famille elle-même, ou, plus exactement, la piété. La famille \emph{doit subsister :} telle est, pour celui de ses membres qui ne s’est laissé envahir par aucun égoïsme antifamilial, la vérité fondamentale, celle qu’aucun doute ne peut effleurer. En un mot,  si la famille est sacrée, aucun de ceux qui lui appartiennent ne peut s’en détacher, sous peine d’être « criminel » envers elle. Il ne pourra jamais poursuivre un intérêt contraire à celui de la famille ; se mésallier, par exemple, lui est interdit. Celui qui le fait « déshonore sa famille », en « fait la honte », etc.\par
L’individu chez qui l’instinct égoïste n’est pas assez fort se soumet : il conclut le mariage qui satisfait les prétentions de sa famille, il choisit une profession en rapport avec sa position, etc. ; bref, il « fait honneur à sa famille ».\par
Si, au contraire, le sang égoïste bout avec assez d’ardeur dans ses veines, il préfère devenir « criminel » envers la famille, et se soustrait à ses lois.\par
Lequel m’est le plus cher, de bien de la famille ou du mon bien ? Il est des cas innombrables où les deux peuvent marcher amicalement côte à côte, où ce qui est utile à ma famille peut être pour moi une source de profits ou réciproquement. Il est alors malaisé de décider si je poursuis le bien commun ou mon bien à moi, et je me flatterai peut-être avec complaisance de mon désintéressement. Mais vient un jour où se dresse devant moi l’alternative redoutable : ou ceci, ou cela ! il faut choisir, et par mon choix, je vais peut-être déshonorer mon arbre généalogique, offenser père, mère, frères et sœurs, tous mes parents. Que faire ? C’est ici que va se montrer à nu le fond de mon cœur, et qu’on va savoir si j’ai jamais mis la piété au-dessus de l’égoïsme ; l’égoïsme ne peut plus se dissimuler sous le voile du désintéressement. Un désir s’allume dans mon âme, et, grandissant d’heure en heure, il devient passion. La plus fugitive pensée contraire à l’esprit de famille, à la piété, porte déjà en elle le germe d’un crime contre la famille ; mais qui s’avise de cela, et qui pourrait au premier moment en avoir une perception nette ? C’est l’histoire de la Juliette de \emph{Roméo et Juliette :} la passion déchaînée finit par  ne plus pouvoir être domptée et par renverser tout l’édifice de la piété.\par
Vous me direz que c’est purement dans son intérêt que la famille rejette de son sein ces égoïstes qui obéissent à leurs passions plus qu’à la piété. C’est ce même argument que les Protestants ont invoqué avec beaucoup de succès contre les Catholiques, et ils sont bien convaincus que c’est ainsi que les choses se sont passées en ce qui les concerne. Mais c’est là une échappatoire de gens qui sentent le besoin de se disculper, et rien de plus. Les Catholiques qui tenaient à l’unité de l’Eglise n’ont repoussé ces hérétiques que parce que ceux-ci ne tenaient pas assez à l’unité pour lui faire le sacrifice de leurs convictions. Les uns ont énergiquement détendu l’unité parce que le lien, l’Eglise catholique (autrement dit unique et universelle) leur était sacré ; les autres au contraire ont mis le lien de côté. Il en est de même de ceux qui s’affranchissent de la piété ; ce n’est pas la famille qui les exclut de son sein, ils s’en excluent d’eux-mêmes en mettant leur passion ou leur volonté individuelle au-dessus du lien familial.\par
Mais il peut arriver que le désir s’allume dans un cœur moins passionné et moins volontaire que celui de Juliette. Alors celle qui se soumet se \emph{sacrifie} à la paix de la famille. On pourrait dire qu’ici encore c’est l’égoïsme qui la fait agir, car la décision que prend celle qui cède vient de ce que l’union de la famille la satisfait mieux que ne le ferait l’accomplissement de son désir. C’est possible, mais comment le croire, s’il reste un signe certain du sacrifice de l’égoïsme à la piété ? Comment le croire, si le désir contraire à la paix de la famille reste, une fois le sacrifice consommé, comme le souvenir et le témoin d’un « sacrifice » fait à un lien sacré, et si celle qui se soumet a conscience de n’avoir pas réalisé sa volonté propre, de s’être humblement inclinée devant une puissance supérieure ? — soumise et sacrifiée, parce que la superstition  de la piété a exercé sur elle son empire !\par
Là, l’égoïsme avait vaincu ; ici la piété est victorieuse et le cœur égoïste saigne ; là l’égoïsme était fort, ici il a été faible. Des faibles : voilà, nous le savons depuis longtemps, ce que sont les désintéressés. La famille a soin d’eux ; elle entoure ces faibles membres de sa sollicitude, parce qu’ils lui \emph{appartiennent}, et qu’ils ne s’appartiennent pas ni ne songent à eux-mêmes. C’est de cette faiblesse que Hégel, par exemple, fait l’éloge, quand il demande que le mariage des enfants soit subordonné au choix des parents.\par
La famille étant une communauté sacrée à laquelle l’individu doit obéissance, la fonction de juge lui appartient de droit. Le Cabanis de Wilibald Alexis, par exemple, nous décrit un « tribunal de famille ». Le père, au nom du « conseil de famille », envoie à l’armée son fils insoumis, et l’expulse de la famille afin de purifier par cet acte de rigueur la famille souillée. Le droit chinois donne à la responsabilité de la famille une sanction très logique en faisant expier par toute la famille la faute d’un de ses membres.\par
De nos jours, toutefois, le bras de l’autorité familiale s’étend rarement assez loin pour pouvoir efficacement châtier le rebelle (l’Etat protège même dans la plupart des cas contre l’exhérédation). Le criminel contre la famille n’a qu’à se réfugier dans le giron de l’Etat pour devenir libre, de même que le criminel envers l’Etat qui s’enfuit en Amérique y trouve un asile contre les lois de son pays ; le fils dénaturé, opprobre de sa famille, est à l’abri de tout châtiment parce que l’Etat protecteur enlève à la vindicte familiale toute sainteté et la profane en déclarant qu’elle n’est que « vengeance ». L’Etat s’oppose au châtiment, à l’exercice d’un droit sacré de la famille, parce que la sainteté de la famille étant, dans la hiérarchie, inférieure à sa propre sainteté, pâlit et perd tout prestige dès qu’elle entre en lutte avec cette sainteté supérieure. Lorsqu’il n’y a pas conflit entre les deux, l’Etat laisse toute sa valeur à l’autorité sacrée, encore  que moins sacrée, de la famille ; mais dans le cas contraire, il va jusqu’à ordonner le crime envers la famille : il fait un devoir au fils, par exemple, de refuser d’obéir à ses parents si ceux-ci veulent l’entraîner à pécher contre l’Etat.\par
Supposons que l’égoïste ait rompu les liens familiaux et trouvé dans l’Etat un protecteur contre l’esprit de famille gravement offensé. A quoi en arrive-t-il ? A faire partie d’une nouvelle société, dans laquelle son égoïsme va rencontrer les mêmes pièges, les mêmes filets que ceux auxquels il vient a échapper. L’Etat aussi est une société et n’est pas une association : il est l’extension de la famille (« père du peuple, — mère du peuple, — enfants du peuple »).\par

\asterism

\noindent Ce qu’on nomme Etat est un tissu, un entrelacement de dépendances et d’attachements ; c’est une \emph{solidarité}, une \emph{réciprocité} ayant pour effet que tous ceux entre lesquels s’établit cette coordination s’accordent entre eux et dépendent les uns des autres : l’Etat est l’ordre, le régime de cette \emph{dépendance} mutuelle. Que le roi, dont l’autorité se répercute sur tous ceux qui détiennent le moindre emploi public jusque sur le valet du bourreau, vienne à disparaître, l’ordre n’en sera pas moins maintenu en face du désordre de la bestialité par tous ceux chez qui veille le sens de l’ordre. Si le désordre l’emportait, l’Etat aurait vécu.\par
Mais cette bonne entente, cet attachement réciproque, cette dépendance mutuelle, cette pensée d’amour est-elle réellement capable de nous gouverner ? A ce compte, l’Etat serait l’\emph{amour} réalisé, vivre dans l’Etat serait être pour autrui et vivre pour autrui. Mais que devient l’individualité, quand règne l’esprit d’ordre ? Ne trouvera-t-on pas que tout est pour le mieux pourvu que l’on parvienne par la force à faire régner l’ordre, c’est-à-dire à disloquer et à parquer judicieusement le troupeau de façon que nul ne « marche sur les pieds du  voisin » ? Tout est ainsi mis en « bon ordre », et c’est ce bon ordre qu’on appelle Etat.\par
Nos sociétés et nos Etats \emph{sont} sans que nous les \emph{fassions ;} ils peuvent s’allier sans qu’il y ait alliance entre nous, ils sont prédestinés et ils ont une \emph{existence }propre, indépendante ; en face de nous, les égoïstes, ils sont l’état de choses existant et indissoluble. Toutes les luttes d’aujourd’hui sont bien, comme on le dit, dirigées contre l’ordre établi, et visent à renverser l’état de choses régnant, Mais leur véritable but est universellement méconnu ; il semblerait, à entendre nos réformateurs, qu’il s’agit simplement de substituer à ce qui existe actuellement un nouvel ordre meilleur. C’est bien plutôt à l’ordre lui-même, c’est à dire à tout Etat \emph{(status)} quel qu’il soit, que la guerre devrait être déclarée, et non pas à tel état déterminé, à la forme actuelle de l’Etat. Le but à atteindre n’est pas un autre état (l’Etat démocratique, par exemple), mais l’alliance, l’union, l’harmonie toujours instable et changeante de tout ce qui est et n’est qu’à condition de changer sans cesse.\par
Un Etat se passe de mon entremise et de mon consentement ; je nais en lui, j’y grandis, j’ai envers lui des devoirs et je lui dois « foi et hommage ». Il me prend sous son aile tutélaire, et je vis de sa « grâce ». Ainsi l’existence indépendante de l’Etat fonde ma dépendance ; sa vie comme organisme exige que je ne croisse pas en liberté mais que je sois taillé pour lui ; afin de pouvoir s’épanouir suivant sa nature, il m’applique les ciseaux de la « culture », il me donne une éducation et une instruction mesurées sur lui et non sur moi, et m’apprend par exemple à respecter les lois, à me garder d’attenter à la propriété de l’Etat (c’est-à-dire à la propriété privée), à vénérer une Altesse divine ou terrestre, etc. ; en un mot il m’enseigne à être — \emph{irréprochable,} en sacrifiant mon individualité sur l’autel de la « sainteté » (saint ou sacré est tout ce qu’on peut  imaginer : propriété, vie d’autrui, etc.). Telle est l’espèce de culture que l’Etat est capable de me donner : il me dresse à être un « bon instrument », un « membre utile de la Société ».\par
C’est ce que doit faire tout Etat, qu’il soit démocratique, absolu ou constitutionnel. Et il le fera tant que nous ne nous serons pas défaits de cette idée erronnée qu’il est un « moi » et, comme tel, une « personne » morale, mystique ou politique. C’est de cette peau du lion du moi que je dois, Moi qui suis véritablement un moi, dépouiller le vaniteux mangeur de chardons. A quel pillage mon moi n’est-il pas livré, depuis que le monde est monde ! Ce furent d’abord le soleil, la lune et les étoiles, les chats et les crocodiles qui eurent l’honneur de passer pour Moi ; ce furent ensuite Jehovah, Allah, Notre Père qui usurpèrent mon titre ; puis les familles, les tribus, les peuples, et jusqu’à l’humanité ; vinrent enfin l’Etat et l’Eglise, toujours avec la même prétention d’être Moi ; et Moi, je les regardais paisiblement faire. Quoi d’étonnant, alors, que, toujours de la même façon, un Moi réel se soit présenté et m’ait affirmé en face qu’il ne m’était pas un « toi », mais bel et bien mon propre moi ? C’est ce que fit le Fils de l’homme par excellence\footnote{ \noindent \emph{« Par excellence », en français dans le texte. N. d. Tr}.
 }, et je me demande ce qui empêcherait le premier fils de l’homme venu d’en faire autant ? Voyant ainsi mon moi toujours au-dessus et en dehors de moi, je ne suis jamais parvenu à être réellement Moi-même.\par
Je n’ai jamais cru à Moi, je n’ai jamais cru à mon actualité, et je n’ai jamais su me voir que dans l’avenir. L’enfant croit qu’il sera vraiment lui, lorsqu’il sera devenu autre, lorsqu’il sera « un grand » ; l’homme pense qu’au delà de cette vie seulement il pourra être vraiment quelque chose ; et, pour prendre un exemple plus près de nous, les meilleurs ne prétendent-ils pas aujourd’hui  encore qu’il faut, avant d’être réellement un moi, un « citoyen libre », un « citoyen de l’Etat », un « homme libre » ou un « véritable homme » s’être au préalable incorporé l’Etat, son Peuple, l’Humanité, et que je sais-je encore ? Eux non plus ne conçoivent de vérité et de réalité pour le moi que dans l’acceptation d’un moi étranger auquel on se dévoue. Et qu’est-il, ce moi ? Un moi qui n’est ni un moi ni un toi, un moi imaginaire, un fantôme.\par
Tandis qu’au Moyen-âge, l’Eglise admettait parfaitement que plusieurs Etats vécussent côte-à-côte sous son aile, quand vint la Réforme et plus particulièrement la guerre de trente ans, ce fut aux Etats à apprendre la tolérance et à permettre à diverses Eglises (confessions) de vivre réunies sous une même couronne. Mais tous les Etats sont religieux ; tous sont des « Etats chrétiens », et ils se font un devoir de courber les indépendants et les « égoïstes » sous le joug du surnaturel, c’est-à-dire de les christianiser. Toutes les institutions de l’Etat chrétien visent à la \emph{christianisation} du peuple. Le but de tout l’appareil judiciaire est de forcer les gens à la justice, celui de l’école est de leur imposer la culture intellectuelle, etc. ; bref, le but de l’Etat est invariable : protéger celui qui agit chrétiennement contre celui qui n’agit pas chrétiennement, le rendre fort et lui assurer la \emph{suprématie. }L’Eglise elle-même devint dans les mains de l’Etat un instrument de contrainte, et il exigea de chacun une religion déterminée. « L’enseignement et l’éducation appartiennent à l’Etat », disait dernièrement Dupin en parlant du clergé.\par
Tout ce qui touche au principe de la moralité est affaire d’Etat. De là, les perpétuelles immixtions de l’Etat chinois dans les affaires de famille : en Chine, on n’est rien si l’on n’est pas avant toutes choses un bon enfant de ses parents. Chez nous aussi, les affaires de famille sont foncièrement des affaires d’Etat ; seulement, l’ingérence de l’Etat y est moins visible, parce  qu’il se fie à la famille et ne la soumet pas à une trop étroite surveillance. Il la tient liée par le mariage dont seul lui peut dénouer les liens.\par
L’Etat me demande compte de mes principes et m’en impose certains ; cela pourrait m’induire à demander : « Que lui importe ma marotte (mon principe) ? » — Beaucoup, car il est, lui, le — principe suprême. C’est une opinion courante, que toute la question du divorce et du droit matrimonial en général roule sur le départ à faire entre les droits de l’Eglise et les droits de l’Etat. Le problème est plutôt celui-ci : étant donné que l’homme doit être gouverné par une Sainteté, celle-ci s’appelle-t-elle Foi ou Loi morale (moralité). La domination de l’Etat ne diffère pas de celle de l’Eglise : l’une s’appuie sur la piété, l’autre sur la moralité.\par
On parle de la tolérance, et l’on vante comme un caractère des Etats civilisés la liberté qu’y ont les tendances les plus opposées de se manifester, etc. Il est vrai que si quelques-uns lancent leurs policiers aux trousses des fumeurs de pipes, d’autres sont assez forts pour ne pas se laisser émouvoir par les meetings les plus turbulents. Mais pour apprécier cette longanimité, il faut remarquer que pour tout Etat le jeu réciproque des individualités, les hauts et les bas de leur vie quotidienne sont en quelque sorte une part laissée au hasard, part qu’il doit bien leur abandonner faute de pouvoir la canaliser utilement. Certains Etats font comme le Pharisien, qui gobait des chameaux et faisait la grimace devant une mouche, tandis que d’autres sont plus judicieux ; dans ces derniers, les individus sont « plus libres » parce qu’ils sont moins menés à la baguette. Mais libre, je ne le suis dans aucun Etat. Leur fameuse tolérance ne s’exerce qu’en faveur de ce qui est « inoffensif » et « sans danger » ; elle n’est que leur haussement d’épaules devant ce qui ne vaut pas qu’ils en tiennent compte, et n’est qu’un — despotisme plus imposant, plus auguste et plus  orgueilleux. Certain Etat a manifesté pendant quelque temps des velléités de s’élever au-dessus des querelles littéraires, et de permettre à tous de s’y livrer à cœur joie ; l’Angleterre, elle, porte la tête trop haut pour entendre la rumeur de la foule et sentir la — fumée de tabac. Mais malheur à la littérature qui s’attaque à l’Etat même, malheur aux soulèvements populaires qui mettent l’Etat en danger ! Dans l’Etat auquel nous faisions allusion, on rêve d’une « science libre », et en Angleterre on rêve d’une « vie populaire libre ».\par
L’Etat laisse autant que possible les individus jouer librement, pourvu qu’ils ne prennent pas leur jeu au sérieux, et ne le perdent pas de vue, lui, l’Etat. Il ne peut s’établir d’homme à homme de relations qui ne soient inquiétées, sans « surveillance et intervention supérieures ». Je ne puis pas faire tout ce dont je serais capable, mais seulement ce que l’Etat me permet de faire ; je ne puis faire valoir ni \emph{mes} pensées, ni \emph{mon} travail, ni en général rien de ce qui est \emph{à moi}.\par
L’Etat ne poursuit jamais qu’un but : limiter, enchaîner, assujettir l’individu, le subordonner à une \emph{généralité} quelconque. Il ne peut subsister qu’à condition que l’individu ne soit pas pour soi-même tout dans tout ; il implique de toute nécessité la \emph{limitation du moi,} ma mutilation et mon esclavage. Jamais l’Etat ne se propose de stimuler la libre activité de l’individu ; la seule activité qu’il encourage est celle qui se rattache au but que lui-même poursuit. Jamais non plus l’Etat n’est capable de produire rien de \emph{collectif ; }on ne peut pas dire qu’un tissu est l’œuvre « collective » des différentes parties d’une machine, il est plutôt l’œuvre de toute la machine considérée comme une unité : il en est de même de tout ce qui sort de la \emph{machine de l’Etat}, car l’Etat est le ressort qui met en mouvement les rouages des esprits individuels dont aucun ne suit sa propre impulsion. L’Etat cherche par sa censure, sa surveillance et  sa police à enrayer toute activité libre ; en jouant ce rôle de bâton dans les roues, il croit (avec raison d’ailleurs, car sa conservation est à ce prix) remplir son devoir. L’Etat veut faire de l’homme quelque chose, il veut le façonner ; aussi l’homme, en tant que vivant dans l’Etat, n’est-il qu’un homme \emph{factice ;} quiconque veut être soi-même est l’adversaire de l’Etat et n’est rien. « Il n’est rien » signifie : l’Etat ne l’utilise pas, ne lui accorde aucun titre, aucun emploi, aucune commission, etc.\par
E. Bauer, dans ses \emph{Liberalen Bestrebungen} (Revendications libérales, II, 50), rêve d’ « un gouvernement qui, issu du Peuple, ne puisse jamais se trouver en opposition avec lui ». Il est vrai qu’il retire lui-même (p. 69) le mot « gouvernement » : « Dans une république, il ne peut y avoir de gouvernement, il n’y a de place que pour un pouvoir exécutif. Pure et simple émanation du Peuple, ce pouvoir ne pourrait lui opposer ni une puissance indépendante, ni des principes et des fonctionnaires à lui ; il n’aurait d’autre fondement et son autorité et ses principes n’auraient d’autre source que le Peuple, unique et suprême puissance de l’Etat. La notion de gouvernement est incompatible avec celle d’Etat démocratique. » Mais cela revient au même. Tout ce qui émane, découle, ou dérive d’une chose en devient indépendant, et, comme l’enfant sorti du sein de la mère, se met immédiatement en opposition avec elle. Le gouvernement, sans ce caractère d’indépendance et d’opposition, ne serait rien du tout.\par
« Dans l’Etat libre il n’y a pas de gouvernement, etc. » (p. 94). Ceci veut simplement dire que le Peuple, lorsqu’il est le souverain, ne se laisse pas régenter par une puissance supérieure. Mais en est-il autrement dans la monarchie absolue ? Qui dit souverain exclut toute idée d’une puissance supérieure. Que le souverain s’appelle Prince ou Peuple, jamais il ne peut y avoir un gouvernement \emph{au-dessus} de lui, ça va de soi. Mais  dans tout Etat, absolu, républicain ou « libre », il y aura toujours un gouvernement au-dessus de Moi ; et je ne me trouverai pas mieux de l’un que de l’autre.\par
La République n’est qu’une — monarchie absolue, car peu importe que le souverain s’appelle Prince ou Peuple : l’un et l’autre sont une « Majesté ».\par
Le régime constitutionnel démontre précisément que personne ne veut et ne peut se résigner à n’être qu’un instrument. Les ministres dominent leur maître, le Prince, et les députés dominent leur maître, le Peuple. Le Prince doit se conformer à la volonté des ministres et le Peuple doit se laisser mener par le bout du nez où il plait aux Chambres de le conduire. Le constitutionnalisme va plus loin que la république, attendu que l’\emph{Etat} y est conçu comme en \emph{dissolution.}\par
E. Bauer nie (p. 56) que dans l’Etat constitutionnel le Peuple était une « personnalité » ; — Et dans la République ? Dans l’Etat constitutionnel, le Peuple est un — parti, et un parti est bien une « personne », s’il vous plait de parler d’une personne morale ou « politique » (p. 76). Le fait est qu’une personne morale, qu’on la baptise parti populaire, Peuple ou encore « le Seigneur » n’est nullement une personne, mais un fantôme.\par
Plus loin, E. Bauer ajoute (p. 69) : « La tutelle est la caractéristique de tout gouvernement ». En vérité, elle est plus encore celle d’un Peuple et d’un « Etat démocratique » ; elle est le caractère essentiel toute — \emph{archie.} Un Etat démocratique qui « résume en lui toute puissance », qui est « maître absolu » ne peut pas me laisser devenir majeur et user de mes forces. Et quel enfantillage de ne plus vouloir donner aux fonctionnaires élus par le peuple le nom de « serviteurs » et d’« instruments », sous prétexte qu’ils sont « les exécuteurs de la volonté libre et raisonnable que le Peuple exprime dans ses lois » (p. 73) !\par
« Il ne peut être mis d’unité dans l’Etat, dit-il encore (p. 74), qu’en subordonnant toutes les administrations  aux intentions du gouvernement. » Mais son Etat démocratique doit, lui aussi, avoir de l’ « unité » ; comment s’y passer de la subordination, de la soumission à la — volonté du Peuple ?\par
« Dans un Etat constitutionnel, tout l’édifice gouvernemental repose en définitive sur le Régent, et dépend de son \emph{sentiment} (p. 130). Comment pourrait-il en être autrement dans un « Etat démocratique » ? N’y serais-je pas également régi par le \emph{sentiment} populaire, et cela fait-il pour Moi une grande différence, de dépendre des sentiments d’un Prince ou de dépendre des sentiments du Peuple, de ce qu’on nomme l’ « opinion publique » ? Si, comme E. Bauer le dit avec raison, dépendance équivaut à « rapport Religieux », le Peuple restera pour Moi, dans un Etat démocratique, une puissance supérieure, une « majesté » (la « majesté » est proprement l’essence du Dieu et du Prince), avec laquelle je serai dans un rapport religieux. — Et le Peuple souverain serait irresponsable comme l’est le régent constitutionnel. Tous les efforts d’E. Bauer aboutissent à un changement de maître. Au lieu de vouloir libérer le Peuple, il aurait dû s’occuper de la seule liberté réalisable, de la sienne.\par
Dans l’Etat constitutionnel, l’\emph{absolutisme} a fini par entrer en lutte avec lui-même, parce qu’il a abouti à un antagonisme : le gouvernement veut être absolu, et le Peuple veut être absolu. Ces deux absolus se détruiront l’un l’autre.\par
E. Bauer s’indigne de ce que le roi constitutionnel soit donné par la naissance, c’est-à-dire par le hasard. Mais quand « le Peuple sera devenu l’unique puissance dans l’Etat » (p. 132), n’est-ce pas à un hasard pareil que nous devrons de l’avoir pour maître ? Qu’est-ce donc que le Peuple ? Le Peuple n’a jamais été que le \emph{corps} du gouvernement ; c’est plusieurs corps sous un même bonnet (couronne de prince) ou plusieurs corps sous une même constitution. Et la constitution  est le — prince. Princes et Peuples ne peuvent subsister que tant qu’ils ne \emph{s’identifient} pas. Quand plusieurs Peuples sont réunis sous une même constitution, comme cela s’est vu par exemple dans l’ancienne monarchie perse et se voit encore aujourd’hui, ces « Peuples » ne comptent plus que pour des « provinces ». En face de Moi, en tout cas, le Peuple n’est qu’une puissance — fortuite ; c’est une force de la nature, un ennemi que je dois vaincre.\par
Que faut-il entendre par un peuple « organisé» (\emph{id.}, p.132) ? Un Peuple « qui n’a plus de gouvernement » et qui se gouverne lui-même. Donc, dans lequel aucun Moi ne dépasse le niveau, un Peuple organisé par l’ostracisme. L’ostracisme, le bannissement des « Moi » fait du peuple son propre gouverneur.\par
Si vous parlez d’un Peuple, il faut aussi parler d’un prince, car pour être, pour vivre et pour faire de l’histoire, le Peuple doit, comme tout ce qui agit, avoir une \emph{tête}, un « chef ». C’est ce que Proud’hon exprime en disant : « Une société pour ainsi dire acéphale ne peut vivre\footnote{ \noindent \emph{Création de l’ordre}, p. 485.
 }. »\par
On invoque à chaque instant aujourd’hui la \emph{vox populi ;} l’ « opinion publique » doit gouverner les princes. Il est bien certain que la \emph{vox populi} est en même temps \emph{vox dei ;} mais à quoi bon l’une et l’autre ? Et la \emph{vox principis} n’est-elle pas aussi \emph{vox dei ?}\par
On peut ici se rappeler les « Nationalistes ». Vouloir faire des trente-huit Etats de l’Allemagne \emph{une nation }est aussi absurde que d’entreprendre de réunir en un seul essaim trente-huit essaims d’abeilles conduits par leurs trente-huit reines. Toutes sont des abeilles, mais ce n’est pas en tant qu’abeilles qu’elles tiennent les unes aux autres et peuvent s’unir : elles sont simplement, comme abeilles \emph{sujettes}, liées à leurs \emph{souveraines}, les reines. Abeilles et Peuples sont sans  volonté, et l’\emph{instinct} de leurs reines les conduit.\par
En rappelant aux abeilles la qualité d’abeilles qui leur est commune, on ferait exactement ce que l’on fait si bruyamment aujourd’hui lorsqu’on rappelle aux Allemands leur qualité d’Allemands. Le fait d’être Allemands a de commun avec le fait d’être abeilles qu’il renferme en soi la nécessité de scissions et de séparations, sans cependant impliquer la [{\corr séparation}] dernière, celle qui, en accomplissant la séparation radicale, en ferait sortir en même temps la fin de toute séparation. J’entends la séparation de l’homme d’avec l’homme.\par
La qualité d’Allemands est partagée par divers peuples et diverses tribus, c’est-à-dire par diverses ruches d’abeilles ; mais l’individu qui a la propriété d’être un Allemand est encore tout aussi impuissant que l’abeille isolée. Et cependant, seuls les individus peuvent s’allier ; toutes les alliances et toutes les ligues entre peuples sont et resteront des assemblages mécaniques, parce que ceux qui sont ainsi unis (du moment que l’on considère que ce sont les Peuples qui s’unissent) n’ont \emph{pas de volonté}. Ce n’est que lorsque l’ultime séparation aura eu lieu, que la séparation elle-même cessera pour se transformer en alliance.\par
Les Nationalistes s’efforcent de faire une unité abstraite et sans vie de tout ce qui est abeille. Les individualistes, eux, lutteront pour l’unité personnellement voulue qui naît de l’association. C’est la marque de toutes les tendances réactionnaires de vouloir instaurer quelque chose de \emph{général,} d’abstrait, \emph{un concept} creux et sans vie, tandis que les vœux des égoïstes tendent à délivrer les individus pleins de vie et de vigueur du faix des généralités abstraites. Les réactionnaires voudraient faire jaillir de terre un Peuple, une Nation ; les égoïstes n’ont en vue qu’eux-mêmes. Au fond, les deux tendances actuellement à l’ordre du jour, la tendance particulariste au rétablissement des droits provinciaux et des anciennes distinctions de races (Francs, Bavarois,  etc., Lusace, etc.), et la tendance unitaire au rétablissement de l’unité Nationale ont même origine et même signification. Les Allemands ne seront unis, c’est-à-dire ne s’uniront, que du jour où ils auront envoyé au diable leur qualité d’abeilles et jeté par terre toutes leurs ruches, ou, en d’autres termes, du jour où ils seront plus qu’— Allemands ; alors seulement ils pourront former une « association allemande ». Ce n’est ni dans leur nationalité ni dans le ventre de leur mère qu’ils doivent rentrer pour parvenir à une renaissance ; que chacun rentre en \emph{soi-même ! }N’est-ce pas un spectacle sentimental prodigieusement ridicule, que celui d’un Allemand qui serre la main à son voisin avec une sainte émotion parce que « lui aussi est un Allemand » ? Le voilà bien avancé ! Mais ne riez pas, ça passera pour fort touchant tant qu’on rêvera encore de « fraternité » et que la \emph{voix du sang }ne se sera pas tue. Nos Nationalistes qui prétendent fabriquer une grande \emph{famille allemande}, sont incapables de se délivrer de la superstition de la « piété », de la « fraternité », de l’ « amour filial », et de tous les poncifs sentimentaux qui composent le répertoire de l’\emph{esprit de famille.}\par
Il suffirait pourtant aux susdits Nationalistes de bien comprendre eux-mêmes ce qu’ils veulent pour ne plus se livrer aux ambrassades des teutomanes à romances, car la coalition en vue de résultats et d’intérêts matériels qu’ils prônent aux Allemands n’est qu’une association volontaire, active et spontanée.\par
L’impersonnalité de ce qu’on nomme Peuple et Nation éclate dans ce fait qu’un Peuple qui veut faire tout son possible pour mettre son Moi en valeur place à sa tête un chef sans volonté. Il ne peut échapper à ce dilemme : ou bien, être asservi à un prince qui ne réalise que lui-même et son bon plaisir personnel — et dans ce cas il ne reconnaîtra pas dans ce « maître absolu » sa propre volonté, la volonté populaire, — ou bien, hisser sur le trône un prince soliveau qui ne témoigne d’aucune  volonté personnelle — et dans ce second cas il aura un prince sans volonté qu’on pourrait, sans aucun inconvénient, remplacer par un mécanisme d’horlogerie bien réglé. De ces considérations il résulte, clair comme le jour, que le moi du Peuple est une puissance impersonnelle, « spirituelle », — la Loi. Le moi du Peuple est par conséquent un fantôme et non un moi. Je ne suis un moi, que parce que c’est Moi qui me fais, c’est-à-dire parce que Je ne suis pas l’œuvre d’un autre, mais proprement mon œuvre. Et qu’est-ce que le moi du Peuple ? un hasard le lui donne, les circonstances lui imposent tel ou tel maître héréditaire ou lui procurent le chef qu’il élit ; il n’est pas \emph{son }produit, le produit du Peuple « souverain », comme Je suis \emph{mon} produit. Figure-toi que l’on te veuille persuader que tu n’es pas toi, mais que toi, c’est Pierre ou Paul. C’est ce qui arrive au peuple, et il ne pourrait en être autrement, attendu que le peuple n’a pas plus un moi que les onze planètes assemblées n’en ont un, encore qu’elles gravitent autour d’un centre commun.\par
Pendant longtemps, l’homme a passé pour un citoyen du Ciel. On voudrait en faire aujourd’hui comme au temps des Grecs un \emph{zoon politicon,} un citoyen de l’Etat ou homme politique. Le Grec fut enseveli sous les ruines de son Etat, et le citoyen céleste tombera avec son Ciel. Mais nous n’entendons pas que la Nation, la Nationalité ou le Peuple nous entraînent dans leur chute, nous n’entendons point n’être que des \emph{animaux politiques}. Depuis la Révolution, on cherche à faire le bonheur du Peuple, et pour faire le Peuple heureux, grand, etc., on nous rend malheureux ! Le bonheur du Peuple est — mon malheur.\par
On peut juger du vide que recouvrent de leur emphase les discours des Libéraux politiques en feuilletant l’ouvrage de Nauwerk : \emph{Ueber die Theilnahme am Staate} (Sur la participation à l’Etat). L’auteur se plaint de l’indifférence et de l’apathie qui empêchent  les gens d’être des citoyens dans toute l’acception du mot, et il s’exprime comme s’il n’était possible d’être homme qu’à condition de prendre une part active à la vie de l’Etat, c’est-à-dire à condition de jouer un rôle politique. En cela il est logique, car si on considère l’Etat comme le dépositaire et le gardien de toute « humanité », nous ne pouvons avoir rien d’humain si nous n’y participons pas. Mais en quoi cela touche-t-il l’égoïste ? En rien, car l’égoïste est lui-même le gardien de son humanité, et la seule chose qu’il demande à l’Etat, c’est de s’ôter de son soleil. Ce n’est que si l’Etat vient à toucher à sa propriété que l’égoïste sort de son indifférence. Si les affaires de l’Etat n’atteignent pas le savant enfermé dans son cabinet, doit-il s’en inquiéter parce que cette sollicitude est « le plus sacré de ses devoirs » ? Tant que l’Etat ne se met pas en travers de ses études favorites, qu’a-t-il besoin de s’en laisser distraire ? Que ceux-là s’inquiètent de la marche de l’Etat qui sont personnellement intéressés à le voir rester comme il est ou changer.\par
Ce n’est pas l’idée d’un « devoir sacré » à remplir qui pousse et qui poussera jamais personne à consacrer ses veilles à l’Etat, pas plus que ce n’est pas « par devoir » qu’on se fait disciple de la science, artiste, etc. ; l’égoïsme seul peut y conduire. Démontrez aux gens que leur égoïsme exige qu’ils offrent leur concours à l’Etat, et vous n’aurez pas besoin de les exhorter longtemps ; mais si vous faites appel à leur patriotisme, etc., vous prêcherez longtemps ce « service d’amour » à des cœurs sourds. Le fait est que jamais les égoïstes ne participeront comme vous l’entendez à la vie de l’Etat.\par
Je trouve dans Nauwerk une phrase imprégnée du plus pur Libéralisme : « L’homme n’accomplit sa mission que pour autant qu’il se sache et se sente membre de l’humanité, et qu’il agisse comme tel ». Et plus loin : « Tels que les conçoit le Théologien, les  rapports de l’homme avec l’Etat se réduisent à n’être plus qu’une pure affaire privée, ce qui équivaut à les nier et les détruire ». Et la Religion, telle que la conçoit le Politique, que devient-elle ? Une « affaire privée ».\par
Si, au lieu de leur parler du « devoir sacré », de la « destination de l’homme », de la « vocation d’être parfaitement humains » et d’autres commandements de même espèce, on représentait aux gens le tort qu’ils font à leur \emph{intérêt} en laissant aller l’Etat comme il va, on leur tiendrait, aux tirades près, le même langage qu’on leur tient dans les moments critiques quand on veut atteindre son but. Mais au lieu de cela, notre théologophobe s’écrie : « S’il fut jamais un temps où l’Etat dut pouvoir compter sur tous les siens, c’est bien le nôtre. — L’homme qui pense reconnaît dans la coopération théorique et pratique à l’Etat un \emph{devoir}, et un des devoirs les plus sacrés qui peuvent, lui incomber ». Puis il examine de plus près la « nécessité catégorique qu’il y a pour chacun de s’intéresser à l’Etat ».\par
Celui-là est un politicien et le restera de toute éternité qui loge l’Etat dans sa tête ou dans son cœur ou dans les deux à la fois ; c’est un possédé de l’Etat, il a la Foi.\par
« L’Etat est la condition indispensable du développement intégral de l’humanité. » Certes, il le fut, aussi longtemps que nous nous proposâmes de développer l’humanité ; mais maintenant que nous voulons nous développer, il ne peut plus nous être qu’un embarras.\par
Peut-on encore se proposer, aujourd’hui, de réformer et d’améliorer l’Etat et le Peuple ? Pas plus que la Noblesse, le Clergé, l’Eglise, etc. ; on peut les suspendre, les détruire, les supprimer, mais non les réformer. Ce n’est pas en la réformant qu’on fait d’une absurdité une chose sensée ; mieux vaut la mettre immédiatement au rebut.\par
Il ne doit plus, à l’avenir, être question de l’Etat  (constitution de l’Etat, etc.), mais de Moi. Toutes les questions relatives à la puissance souveraine, à la constitution, etc., retombent ainsi dans l’abîme dont elles n’auraient pas dû sortir, leur néant. Moi, ce rien, je ferai jaillir de moi-même mes \emph{créations}.\par

\asterism

\noindent Au chapitre de la Société se rattache celui du « parti » dont on a ces derniers temps chanté les louanges.\par
Il y a dans l’Etat des partis. « Mon Parti ! Qui voudrait ne pas prendre parti ! » Mais l’individu est \emph{unique }et n’est pas membre d’un parti. Il s’unit librement et se sépare ensuite librement. Un parti n’est autre chose qu’un Etat dans l’Etat, et la « paix » doit régner dans ce petit essaim d’abeilles comme dans le grand. Ceux-là même qui proclament avec le plus d’énergie qu’il faut qu’il y ait dans l’Etat une \emph{opposition}, sont les premiers à s’indigner contre la discorde des partis. Preuve qu’eux non plus ne veulent qu’un — Etat. C’est contre l’Individu, et non contre l’Etat que se brisent tous les partis.\par
Il n’est rien qu’on entende plus souvent aujourd’hui que l’exhortation à rester fidèle à son parti ; les hommes de parti ne méprisent rien tant qu’un renégat. On doit marcher les yeux fermés à la suite de son parti, et approuver et adopter sans réserve tous ses principes. En vérité, le mal n’est pas aussi grand ici que dans certaines sociétés qui lient leurs membres par des lois ou statuts fixes et immuables (par exemple, les ordres religieux, la société de Jésus, etc.). Mais le parti cesse d’être une association dès le moment où il veut rendre obligatoires certains principes et les mettre au-dessus de toute discussion et de toute atteinte ; c’est précisément ce moment qui marque la naissance du parti. Sitôt le parti constitué et en tant que parti, il est une société née, une alliance morte, une idée devenue idée fixe. Le parti de l’absolutisme  ne peut tolérer chez aucun de ses membres le moindre doute sur la vérité du principe absolutiste. Ce doute ne leur serait possible que s’ils étaient assez égoïstes pour vouloir être encore quelque chose en dehors de leur parti, c’est-à-dire pour vouloir être « impartiaux ». Ils ne peuvent être impartiaux que comme égoïstes et non comme hommes de parti. Si tu es Protestant, et si tu appartiens au parti qu’est le Protestantisme, tu ne peux que maintenir ton parti dans la bonne voie ; à la rigueur, tu pourrais le « purifier », mais non le rejeter. Es-tu Chrétien, es-tu enrôlé dans le parti chrétien, tu ne peux en sortir en tant que membre de ce parti ; si tu en transgresses la discipline, ce sera seulement lorsque ton égoïsme, c’est-à-dire ton « impartialité » t’y poussera. Quelques efforts qu’aient faits les Chrétiens, jusqu’à Hégel et aux Communistes inclusivement, pour fortifier leur parti, ils en sont restés à ceci : le Christianisme devant renfermer la vérité éternelle, il suffit de l’en extraire, de la démontrer et de la faire accepter.\par
Bref, le parti est contradictoire à l’impartialité, et cette dernière est une manifestation de l’égoïsme. Que m’importe d’ailleurs le parti ? Je trouverai toujours assez de compagnons qui se réuniront à moi sans prêter serment à mon drapeau.\par
Si quelqu’un passe d’un parti à l’autre, on l’appelle immédiatement transfuge, déserteur, renégat, apostat, etc. La \emph{Morale}, en effet, exige que l’on adhère fermement à son parti ; le trahir, c’est se souiller du crime d’ « infidélité » ; mais l’individualité, elle, ne connaît ni « fidélité », ni « dévouement » de précepte ; elle permet tout, y compris l’apostasie, la désertion et le reste. Les Moraux eux-mêmes se laissent inconsciemment diriger par le principe égoïste quand ils ont à juger quelqu’un qui abandonne son parti pour se rallier au leur ; bien mieux, ils ne se font aucun scrupule d’aller racoler des partisans dans le camp opposé ! Ils devraient seulement avoir conscience  d’une chose, c’est qu’il faut agir d’une façon \emph{immorale} pour agir d’une façon personnelle, ce qui revient ici à dire qu’il faut savoir rompre sa foi et même son serment si l’on veut se déterminer soi-même au lieu de se laisser déterminer par des considérations morales. Un apostat se peint toujours sous des couleurs douteuses aux yeux des gens de moralité sévère ; ils ne lui accorderont pas facilement leur confiance, car il s’est souillé d’une « trahison », c’est-à-dire d’une immoralité. Ce sentiment est presque général chez les gens de culture inférieure. Les plus éclairés sont sur ce point, comme sur tous, incertains et troublés ; la confusion de leurs idées ne leur permet pas d’avoir clairement conscience de la contradiction où les accule nécessairement le principe de moralité. Ils n’osent pas accuser franchement l’apostat d’immoralité, car eux-mêmes prêchent en somme l’apostasie, le passage d’une religion à une autre, etc. ; d’autre part, ils n’osent pas abandonner leur point d’appui dans la moralité. Quelle excellente occasion, pourtant, de la jeter par-dessus bord !\par
Les Individus ou Uniques sont-ils d’un parti ? Eh ! comment pourraient-ils être uniques s’ils \emph{appartenaient }à un parti ?\par
Ne peut-on donc être d’aucun parti ? Entendons-nous : En entrant dans votre parti et dans vos cercles, je conclus avec vous une \emph{alliance}, qui durera aussi longtemps que votre parti et moi poursuivrons le même but. Mais si aujourd’hui je me rallie encore à son programme, demain peut-être je ne pourrai plus le faire et je lui deviendrai « infidèle ». Le parti n’a pour moi rien qui me lie, rien d’obligatoire et je ne le respecte pas ; s’il cesse de me plaire, je me retourne contre lui.\par
Les membres de tout parti qui tient à son existence et à sa conservation ont d’autant moins de liberté, ou, plus exactement, d’autant moins de personnalité, et ils manquent d’autant plus d’égoïsme qu’ils se soumettent  plus complètement à toutes les exigences de ce parti. L’indépendance du parti implique la dépendance de ses membres.\par
Un parti, quel qu’il soit, ne peut jamais se passer d’une \emph{profession de foi,} car ses membres doivent \emph{croire }à son principe et ne peuvent le mettre en doute ni le discuter, il doit être pour eux un axiome certain et indubitable. En d’autres termes : on doit appartenir corps et âme à son parti ; sinon, on n’est plus véritablement un homme de parti, on est plus ou moins un — égoïste. Que le moindre doute s’élève chez toi au sujet du Christianisme, et tu ne seras plus un vrai Chrétien, toi qui auras eu l’impiété grande d’examiner le dogme et de traîner le Christianisme devant le tribunal de ton égoïsme. Tu te seras rendu coupable envers le Christianisme, cette affaire de parti (affaire de parti, parce qu’il n’est pas l’affaire par exemple des Juifs, qui sont d’un autre parti). Mais tant mieux pour toi, si un péché ne t’épouvante pas : ton audacieuse impiété va t’aider à atteindre l’Individualité.\par
Ainsi donc, un égoïste ne pourra jamais embrasser un parti, il ne pourra jamais prendre parti ? Mais si, il le peut parfaitement, pourvu qu’il ne se laisse pas saisir et enchaîner par le parti ! Le parti n’est jamais pour lui qu’une partie : il est de la partie, il prend part.\par

\asterism

\noindent Le meilleur Etat est évidemment celui qui renferme les citoyens les plus fidèles à la loi. A mesure que le noble sentiment de la légalité languit et s’éteint, l’Etat, qui est un système de moralité et la vie morale elle-même, voit baisser ses forces et décroître ses biens. Avec les bons citoyens disparaît le bon Etat ; il sombre dans l’anarchie. « Respect à la Loi ! » tel est le ciment qui maintient debout tout l’édifice d’un Etat. « La loi est \emph{sacrée,} celui qui la viole est un \emph{criminel.} » Sans le crime, pas d’Etat. Le monde moral  — et c’est l’Etat — est plein de fripons, de trompeurs, de menteurs, de voleurs, etc. L’Etat étant la « souveraineté de la Loi » et sa hiérarchie, l’égoïste ne peut parvenir à ses fins que par le crime, dans tous les cas où son intérêt est opposé à celui de l’Etat.\par
L’Etat ne peut cesser d’exiger que ses lois soient tenues pour sacrées. Aussi l’Individu est-il aujourd’hui, vis-à-vis de l’Etat, exactement ce qu’il était jadis vis-à-vis de l’Eglise, un \emph{profane} (un barbare, un homme naturel, un « égoïste », etc.). Devant l’Individu, l’Etat se ceint d’une auréole de sainteté. Il fait par exemple une loi sur le duel. Deux hommes qui conviennent de risquer leur vie afin de régler une affaire (quelle qu’elle soit) ne peuvent exécuter leur convention, parce que l’Etat ne le veut pas : ils s’exposeraient à des poursuites judiciaires et à un châtiment. Que devient la liberté de l’arbitre ? Il en est tout autrement là où, comme dans l’Amérique du Nord, la société décide de faire subir aux duellistes certaines \emph{conséquences} désagréables de leur acte, et leur retire par exemple le crédit dont ils avaient joui antérieurement. Refuser son crédit est l’affaire de chacun, et s’il plaît à une société de le retirer à quelqu’un pour l’une ou l’autre raison, celui qu’elle frappe ne peut pas se plaindre d’une atteinte à sa liberté : la société n’a fait qu’user de la sienne. Il ne s’agit plus, ici, d’une expiation ni du châtiment d’un crime. Dans l’Amérique du Nord, le duel n’est pas un crime, c’est un acte contre lequel la société prend des mesures de prudence et se préserve. L’Etat, au contraire, qualifie le duel crime, c’est-à-dire violation de sa loi sacrée : il en fait une affaire criminelle. La société dont nous parlions laisse l’individu parfaitement libre de s’exposer aux suites funestes ou désagréables qu’entraînera sa manière d’agir, et laisse pleine et entière sa liberté de vouloir ; l’Etat fait précisément le contraire : il dénie toute légitimité à la volonté de l’individu, et ne reconnaît comme légitime que sa propre volonté, la loi de l’Etat. Il en résulte que celui  qui transgresse les commandements de l’Etat peut être considéré comme violant les commandements de Dieu, opinion, d’ailleurs, que l’Eglise a soutenue. Dieu est la Sainteté en soi et pour soi, et les commandements de l’Eglise comme ceux de l’Etat sont les ordres que cette Sainteté donne au monde par l’intermédiaire de ses prêtres ou de ses maîtres de droit divin. L’Eglise avait les \emph{péchés mortels}, l’Etat a les \emph{crimes qui entraînent la mort ;} elle avait ses \emph{hérétiques,} il a ses \emph{traîtres ;} elle avait des \emph{pénitences}, il a des \emph{pénalités ;} elle avait les \emph{inquisiteurs}, il a les agents du \emph{fisc ;} bref, à l’une le péché, à l’autre le crime ; là le pécheur, ici le criminel ; là l’inquisition, et ici — encore l’inquisition ! La sainteté de l’Etat ne tombera-t-elle pas comme est tombée la sainteté de l’Eglise ? La crainte de ses lois, le respect de sa majesté, la misère et l’humiliation de ses sujets, tout cela va-t-il durer ? Ne viendra-t-il pas un jour où l’on cessera de se prosterner devant l’image du saint ?\par
Quelle folie d’exiger que le pouvoir de l’Etat lutte à armes courtoises avec l’individu, et, comme on l’a dit à propos de la liberté de la presse, partage avec son adversaire le vent et le soleil ! Pour que l’Etat, cette idée, ait un pouvoir réel, il \emph{doit} être une puissance supérieure à l’individu. L’Etat est « sacré », donc il ne \emph{peut} pas prêter le flanc aux « attaques impies » des individus. Si l’Etat est sacré, il \emph{faut }une censure. Les Libéraux politiques admettent les prémisses et nient la conséquence. Toutefois, ils accordent à l’Etat les mesures de répression, car il est bien convenu, et ils n’en démordent pas, que l’Etat est plus que l’individu et que sa vengeance, qu’il appelle peine ou châtiment, est légitime.\par
Le mot \emph{peine} n’a de sens que s’il désigne la pénitence infligée au profanateur d’une chose sacrée. Celui qui tient une chose pour sacrée mérite évidemment qu’une peine lui soit infligée dès qu’il s’y attaque. Un homme qui épargne une vie humaine parce que  cette vie lui est sacrée et que l’idée d’y attenter lui fait horreur est un homme — \emph{religieux}.\par
Weitling impute au « désordre social » tous les crimes qui se commettent, et il espère que sous le régime communiste les crimes deviendront impossibles, les mobiles (l’argent par exemple) en ayant disparu. Mais son bon naturel l’égare, car la société organisée, telle qu’il l’entend, sera, elle aussi, sacrée et inviolable. Il n’y manquera pas de gens qui, la bouche pleine de professions de foi communistes, travailleront sous main à sa ruine. Somme toute, Weitling est bien obligé de s’en tenir aux « moyens curatifs à opposer aux maladies et aux faiblesses inséparables de la nature humaine » ; mais ce mot « curatif », n’indique-t-il pas déjà que l’on considère les individus comme « voués » à une certaine « cure », et qu’on va leur appliquer les remèdes qu’ « appelle » leur nature d’hommes ?\par
Le \emph{remède} et la \emph{cure} ne sont que l’autre face du \emph{châtiment} et de \emph{l’amendement}, la \emph{thérapeutique} du corps fait le pendant de la \emph{diététique} de l’âme. Si celle-ci voit dans une action un péché contre le Droit, celle-là y voit un péché de l’homme contre lui-même, le dérangement de sa santé. Ne vaudrait-il pas mieux considérer simplement ce que cette action a de favorable ou de défavorable pour Moi, et voir si elle m’est amie ou ennemie ? je la traiterais alors comme ma \emph{propriété,} c’est-à-dire que je la conserverais ou la détruirais à mon gré.\par
« Crime » et « maladie » ne sont point des noms qui s’appliquent à une conception \emph{égoïste} des choses qu’ils désignent ; ce sont des jugements portés non pas par Moi mais par un autre, sur l’offense faite au \emph{Droit} en général, ou à la \emph{Santé}, que ce soit la santé de l’individu (du malade) ou de la généralité (de la Société). On n’a aucun ménagement pour le « crime », tandis qu’on use envers la « maladie » de douceur, de compassion, etc.\par
 Le crime est suivi du châtiment. Si le Sacré disparaît, entraînant le crime avec lui, le châtiment doit disparaître également, car lui non plus n’a de signification que par rapport au Sacré. On a aboli les peines ecclésiastiques. Pourquoi ? Parce que la façon dont il se comporte envers le « saint Dieu » est l’affaire de chacun. Comme est tombée la peine ecclésiastique, doivent tomber toutes les \emph{peines.} Si le péché envers son Dieu est l’affaire personnelle de chacun, il en est de même du péché envers tout Sacré quel qu’il soit. Suivant la doctrine de notre droit pénal, que l’on s’efforce vainement de rendre moins anachronique, on \emph{punit} les hommes de telle ou telle « inhumanité », et l’on démontre ainsi par l’absurdité de leurs conséquences la niaiserie de ces théories qui font pendre les petits voleurs et laissent courir les grands. Pour un attentat contre la propriété, on a le bagne, et pour un « viol de pensées », pour l’oppression des « droits naturels de l’homme », on n’a que des représentations et des prières.\par
Le code pénal n’existe que grâce au sacré, et disparaîtra de lui-même quand on renoncera au châtiment. Partout, actuellement, on veut créer un nouveau code pénal, sans éprouver le moindre scrupule au sujet des pénalités à édicter. C’est pourtant justement la peine qui doit disparaître, pour faire place à la satisfaction : satisfaction, encore une fois, non point du Droit ou de la Justice, mais de \emph{nous.} Si quelqu’un nous fait ce que nous \emph{ne voulons pas qui nous soit fait,} nous brisons sa puissance et nous faisons prévaloir la nôtre : nous \emph{nous} donnons satisfaction à son égard sans faire la folie de vouloir donner satisfaction au Droit (au fantôme). C’est l’homme qui doit se défendre contre l’homme, et ce n’est pas le Sacré, pas plus que ce n’est Dieu qui se défend contre l’homme ; — encore que jadis et parfois aussi de nos jours on ait vu tous les « serviteurs de Dieu » lui prêter main forte pour châtier l’impie, comme ils prêtent aujourd’hui main  forte au Sacré. De ce dévouement au Sacré il résulte que, sans y avoir d’intérêt vital et personnel, on livre les malfaiteurs aux griffes de la police et des tribunaux : on donne procuration aux autorités constituées pour qu’elles « administrent pour le mieux le domaine du Sacré » et on reste neutre. Le peuple met une véritable rage à exciter la police contre tout ce qui lui semble immoral ou souvent simplement inconvenant, et cette rage de moralité qui possède le peuple est pour la police une protection bien plus sûre que celle que pourrait lui assurer le gouvernement.\par
C’est par le crime que l’Egoïste s’est toujours affirmé et a renversé d’une main sacrilège les saintes idoles de leurs piédestaux. Rompre avec le Sacré ou mieux encore, rompre le Sacré peut devenir général. Ce n’est pas une nouvelle révolution qui approche ; mais, puissant, orgueilleux, sans respect, sans honte, sans conscience, un — \emph{crime} ne gronde-t-il pas avec le tonnerre à l’horizon, et ne vois-tu pas que le ciel, lourd de pressentiments, s’obscurcit et se tait ?\par

\asterism

\noindent Celui qui se refuse à dépenser ses forces pour des sociétés aussi restreintes que la Famille, le Parti ou la Nation aspire encore toujours à une société de signification plus haute ; lorsqu'il a découvert la « Société humaine » ou « l’Humanité », il croit avoir trouvé l'objet véritablement digne de son culte auquel il mettra son honneur à se sacrifier : à partir de ce moment, « sa vie et ses services appartiennent à l'Humanité ».\par
Le Peuple est le corps, l'Etat est l'esprit de cette \emph{Personne} souveraine qui m'a jusqu'ici opprimé. On a voulu transfigurer le Peuple et l'Etat en les élargissant jusqu'à y voir respectivement l'« humanité » et la « raison universelle ». Mais ce \emph{magnificat} n'aboutit qu'à rendre la servitude plus lourde ; Philanthropes  et Humanitaires sont des maîtres aussi absolus que les Politiciens et les Diplomates.\par
Les Critiques contemporains déclament contre la Religion parce qu’en plaçant Dieu, le divin, le moral, etc., \emph{en dehors} de l’homme, elle en fait quelque chose d’objectif, tandis qu’eux, au contraire, préfèrent laisser ces sujets \emph{dans} l’homme. Mais ils n’en versent pas moins dans l’ornière religieuse, et ils imposent, eux aussi, une « vocation » à l’homme en le voulant divin, humain, etc. : Moralité, liberté, humanité, etc. doivent être son essence. La Politique comme la Religion prétendit se charger de l’ « éducation » de l’homme, le conduire à la réalisation de son « essence » et de sa « destination », en un mot \emph{faire} de lui quelque chose, c’est-à-dire en faire un \emph{véritable homme ;} l’une entend par là un « vrai croyant », l’autre un « vrai citoyen », ou un « véritable sujet ». En somme, que vous appeliez ma vocation divine ou humaine, cela revient au même.\par
Religion et Politique placent l’homme sur le terrain du \emph{devoir}. Il \emph{doit} devenir ceci ou cela, il \emph{doit} être ainsi et non autrement. Avec ce postulat, ce commandement, chacun s’élève non seulement au-dessus des autres, mais encore au-dessus de lui-même. Nos Critiques disent : « tu dois être complètement homme, tu dois être un homme libre. » Eux aussi sont en train de proclamer une nouvelle Religion et d’ériger un nouvel idéal absolu, un idéal qui sera la Liberté. Les hommes doivent être libres. Il n’y aurait pas à s’étonner de voir paraître des \emph{missionnaires} de la Liberté, semblables aux missionnaires de la foi que le Christianisme, convaincu que tous les hommes étaient destinés à devenir chrétiens, envoyait à la conquête du monde païen. Et de même que, jusqu’à présent, la Foi s’est constituée en Eglise et la Moralité en Etat, la Liberté pourrait suivre leur exemple et se constituer en une \emph{communion} nouvelle, qui pratiquerait à son tour la « propagande ». Il n’y a, évidemment, aucune raison  de s’opposer à un essai d’association quel qu’il soit, mais il faut s’opposer d’autant plus énergiquement à toute résurrection de l’ancienne \emph{charge d’âme,} de la tutelle, bref, au principe qui veut que l’on \emph{fasse de nous quelque chose}, que ce soit des chrétiens, des sujets ou des affranchis et des hommes.\par
On peut bien, avec Feuerbach et d’autres, dire que la Religion a dépouillé l’homme de l’humain, et qu’elle a transporté cet humain dans un au-delà si lointain qu’il y devient inaccessible et qu’il y acquiert une existence propre et y prend la forme d’une personne, d’un « Dieu ». Mais là n’est pas toute l’erreur de la religion. On pourrait fort bien cesser de croire à la personnalité de la part d’humanité qui fut retirée à l’homme, on pourrait fort bien transformer le dieu en divin et rester nonobstant religieux. Car être religieux c’est n’être pas pleinement satisfait de l’homme \emph{présent}, c’est imaginer une perfection » qui doit être atteinte, et se figurer l’homme comme « tendant à se parfaire\footnote{ \noindent B{\scshape r}. B{\scshape auer}, \emph{Litt. Ztg.} {\scshape viii}, 22.
 } ». (« C’est pourquoi vous \emph{devez} devenir parfaits comme l’est votre Père céleste. » Math, v, 48). Etre religieux, c’est se fixer un Idéal, un absolu. La perfection est le « suprême bien », le \emph{finis bonorum},et l’idéal de chacun est l’homme parfait, le véritable homme, l’homme libre, etc.\par
Les efforts de l’époque actuelle tendent à instaurer en guise d’idéal l’ « homme libre ». Si l’on y parvenait, cet idéal nouveau aurait pour conséquence une nouvelle — religion, de nouvelles aspirations, de nouveaux tourments, une nouvelle dévotion, une nouvelle divinité, de nouveaux remords.\par
L’idéal de la « liberté absolue » a fait divaguer comme le fait tout absolu. D’après Hess, par exemple, cette liberté absolue serait « réalisable dans la société humaine absolue », et un peu plus bas le même auteur appelle cette réalisation une « vocation  », et définit la liberté une « moralité » : il faut inaugurer le règne de la « Justice » (\emph{ id est :} Egalité) et de la « Moralité » (\emph{id. est :} Liberté).\par
Vous vous gaussez de celui qui, tandis que les membres de sa tribu, de sa famille, de sa nation, etc. peinent et méritent, se borne à se « gonfler » glorieusement des hauts faits de ses compagnons. Non moins aveugle est celui qui met toute sa gloire à être « homme ». Ni lui ni le parasite glorieux de tantôt ne fondent le sentiment de leur valeur sur une \emph{exclusion,} mais sur une \emph{connexion}, sur le « lien » qui les unit aux autres : lien du sang, lien de la nationalité ou lien de l’humanité.\par
Les « Nationalistes » d’aujourd’hui ont rallumé la discussion entre ceux qui pensent n’avoir dans les veines qu’un sang purement humain et n’être liés que par des liens purement humains et ceux qui se targuent d’un sang spécial et de liens spéciaux.\par
Si nous considérons l’orgueil comme la conscience d’une valeur (valeur qui peut être surfaite, mais peu importe), nous constatons une différence énorme entre l’orgueil d’ « appartenir » à une nation, c’est-à-dire d’être la propriété de cette nation, et l’orgueil de nommer une nationalité sa propriété. Ma nationalité est un de mes prédicats, une de mes propriétés, tandis que la nation est ma propriétaire et ma maîtresse. Si tu possèdes la force physique, tu pourras l’employer en temps utile et elle pourra te procurer cette satisfaction de connaître ta valeur qu’on appelle orgueil. Mais si c’est ton grand vigoureux corps qui te possède, il te poussera partout et aux moments les moins opportuns à exhiber sa vigueur : tu ne pourras serrer la main à personne sans la lui écraser.\par
Une fois parvenu à se convaincre que l’on est plus que le membre d’une famille, le fils d’une race, l’individu d’un peuple, etc., on en arrive finalement à dire : On est plus que tout cela parce qu’on est homme, — ou encore : l’Homme est plus que le  Juif, l’Allemand, etc. « Que chacun soit donc entièrement et uniquement — homme ! » Mais ne vaudrait-il pas mieux dire : si nous sommes plus que ne peuvent exprimer tous les noms qu’on nous donne, nous voulons être plus qu’homme pour la même raison que vous voulez être plus que Juif et plus qu’Allemand. Les Nationalistes ont raison : on ne peut pas renier sa nationalité ; mais les Humanitaires aussi ont raison : on ne doit pas se renfermer dans les bornes étroites de sa nationalité. C’est à l’\emph{individualité} à résoudre cette contradiction : la nationalité est ma propriété, mais Je ne tiens pas tout entier dans une de mes propriétés ; l’humanité aussi est ma propriété, mais c’est Moi seul qui, par mon unicité, donne à l’homme son existence.\par
L’histoire cherche l’Homme : mais l’homme, c’est toi, c’est moi, c’est nous ! Après l’avoir pris pour un Etre mystérieux, une divinité, et l’avoir cherché dans \emph{le Dieu} d’abord, puis dans \emph{l’Homme} (l’humanité, le genre humain), je l’ai enfin trouvé dans l’individu borné et passager, dans l’Unique.\par
Je suis possesseur de l’humanité, Je suis l’humanité, et Je ne fais rien pour le bien d’une autre humanité. Tu es fou, toi qui, étant une humanité unique, te guindes afin de vivre pour une autre que celle que tu es toi-même.\par

\asterism

\noindent Les relations du Moi avec le monde humain que nous avons examinées jusqu’ici se prêtent à de tels développements et nous ouvrent de si riches perspectives qu’en d’autres circonstances on ne saurait trop s’y étendre. Mais nous ne nous proposions pour le moment que d’en indiquer les grandes lignes, et nous sommes forcé de nous interrompre pour passer à l’examen de deux autres côtés de la question. Je ne  suis pas seulement en rapport avec les hommes en tant que représentants de l’idée d’ « Homme » ou en tant qu’enfants de l’Homme (pourquoi ne pas dire « enfants de l’Homme », puisqu’on dit « enfants de Dieu » ?) : je suis en outre en rapport avec ce qu’ils tiennent de l’Homme et appellent leur propriété. En d’autres termes, je suis en relations non seulement avec ce qu’ils \emph{sont} comme hommes, mais encore avec leur \emph{avoir} humain. Après avoir traité du monde des hommes, je dois donc, pour remplir le cadre que je me suis tracé, passer à l’examen du monde des sensations et des idées et dire quelques mots de ce que les hommes appellent leur propriété : les \emph{biens }tant matériels que spirituels.\par
Tandis que la notion d’Homme se développait et qu’on en acquérait une intelligence plus claire, nous eûmes à la respecter successivement sous les diverses formes \emph{personnelles} dont on la revêtit ; du dernier et du plus haut de ses avatars sortit enfin le commandement de « respecter en chacun l’Homme ». Mais si je respecte l’Homme, mon respect doit s’étendre également à tout ce qui est humain, à tout ce qui appartient à l’Homme.\par
Les hommes ont une \emph{propriété ;} devant cette propriété, Je dois m’incliner : elle est sacrée. Leur propriété consiste en un \emph{avoir} en partie extérieur et en partie intérieur. Leur avoir extérieur comprend des choses, et leur avoir intérieur est formé de pensées, de convictions, de nobles sentiments, etc. Mais je ne suis jamais tenu de respecter que leur avoir \emph{humain}, n’ai pas à ménager celui qui n’est pas humain, car les hommes ne peuvent avoir réellement en propre que ce qui est propre à l’homme. Parmi les biens intérieurs, on peut citer par exemple la \emph{religion ;} la religion étant libre, c’est-à-dire propre à l’Homme, il ne m’est pas permis d’y toucher ; un autre de ces biens intérieurs est l’\emph{honneur ;} étant libre, il m’est inviolable (diffamation, caricatures, etc.). La religion  et l’honneur sont une « propriété spirituelle ». Comme propriété matérielle, vient en premier lieu la \emph{personne :} ma personne est ma propriété ; de là résulte la liberté de la personne ; mais, bien entendu, seule la personne \emph{humaine} est libre, l’autre, la prison l’attend. Ta vie est ta propriété, mais elle n’est sacrée pour les hommes que si elle n’est pas la vie d’un \emph{non-homme}.\par
Les biens matériels dont l’homme ne peut justifier la possession par son humanité, il n’y a aucun titre et nous pouvons les lui prendre ; d’où la concurrence sous toutes ses formes. Ceux des biens spirituels qu’il ne peut revendiquer comme homme sont également à notre disposition ; d’où la liberté de la discussion, la liberté de la science et de la critique.\par
Mais les biens \emph{consacrés} sont inviolables. Qui les consacre et les garantit ? A première vue, c’est l’Etat, la Société ; mais plus proprement c’est l’Homme, ou l’ « idée » : l’idée d’une propriété sacrée implique l’idée que cette propriété est vraiment humaine ou plutôt que son possesseur ne la détient qu’en vertu de sa qualité d’Homme et non à titre de non-homme.\par
Dans le domaine spirituel, l’homme est légitime possesseur de sa foi, par exemple, et de son honneur, de sa conscience, de son sentiment du convenable et du honteux, etc. Les actes attentatoires à l’honneur (paroles, écrits,) sont punissables ; de même ceux qui portent atteinte au fondement de la religion, à la foi politique, bref toute lésion de ce à quoi l’Homme « a droit ».\par
Le Libéralisme critique ne s’est pas encore prononcé sur la question de savoir jusqu’à quel point il pourrait admettre que les biens sont sacrés ; il pense bien être l’adversaire de toute sainteté, mais comme il lutte contre l’égoïsme, il doit lui tracer des limites et il ne peut tolérer que le non-homme les franchisse au préjudice de l’homme. Sa répulsion théorique  pour la « masse » devrait, s’il arrivait au pouvoir, se traduire par des mesures de répulsion pratique.\par
Les [{\corr représentants}] des différentes nuances du Libéralisme sont en désaccord sur l’extension à donner à l’idée d’ « Homme » et sur ce qu’en doit retirer l’homme individuel, c’est-à-dire sur la définition de l’Homme et de l’humain ; l’homme politique, l’homme moral et l’homme « humain » ont revendiqué tour à tour, et toujours plus catégoriquement, le titre d’Homme. Celui qui définit le mieux ce qu’est l’ « Homme » est aussi celui qui sait le mieux ce que doit avoir « l’Homme ». Ce concept, l’Etat ne le saisit que dans son acception politique ; la Société, d’autre part, ne comprend que sa portée sociale ; seule, dit-on, l’Humanité l’embrasse tout entier : « l’histoire de l’humanité en est le développement ». Découvrez l’Homme, et vous connaîtrez par le fait même ce qui est propre à l’Homme, la propriété de l’Homme ou l’humain.\par
Mais que l’homme individuel prétende à tous les droits du monde, qu’il invoque à leur appui l’autorité de l’ « Homme » et son titre d’homme, que m’importent, à Moi, son droit et ses prétentions ? Ses droits, il ne les tient que de l’Homme, et non de Moi : aussi n’a-t-il à mes yeux aucun droit. Sa vie, par exemple, ne m’importe que pour autant qu’elle ait une valeur pour Moi. Je ne respecte pas plus son prétendu droit de propriété, ou droit sur les biens matériels, que je ne respecte son droit sur le « sanctuaire de son âme » ou droit de garder intacts ses biens spirituels, ses idoles et ses dieux. Ses biens, tant matériels que spirituels, sont \emph{à Moi}, et je les traite en propriétaire selon — mes forces.\par
La « \emph{question de la propriété} », dans les termes où on la pose n’en est pas une ; ne visant que ce qu’on nomme notre avoir, elle est trop étroite et n’est susceptible d’aucune solution : c’est à « celui de qui nous  tenons tout » à la trancher ; la propriété dépend du propriétaire et, par l’intermédiaire de ce dernier, la question de la propriété se rattache à un problème d’une portée beaucoup plus grande.\par
La Révolution dirigea ses attaques contre tout ce qui vient de la « grâce de Dieu » et, entre autres, contre le droit divin que l’on remplaça par le droit humain. A ce que la « faveur divine » nous dispense, on opposa ce qui découle de l’ « essence de l’homme ».\par
Les relations entre les hommes ayant cessé d’être fondées sur le dogme religieux qui commande « aimez-vous les uns les autres pour l’amour de Dieu », durent être édifiées sur la base humaine du « aimez-vous les uns les autres pour l’amour de l’Homme ». De même, en ce qui concerne les relations des hommes avec les choses de ce monde, la doctrine révolutionnaire ne put faire autrement que de proclamer que le monde, jusqu’alors organisé selon l’ordre de Dieu, appartiendrait dorénavant à « l’Homme ».\par
Le monde appartient à « l’Homme » et doit être par moi respecté comme sa propriété.\par
Propriété = Mien !\par
Propriété, au sens bourgeois du mot, signifie propriété \emph{sacrée,} de sorte que je dois \emph{respecter} ta propriété. « Respect à la propriété ! » Aussi les Politiques verraient-ils volontiers chacun posséder sa parcelle de propriété, et cette tendance a abouti dans certaines régions à un morcellement incroyable. Chacun doit avoir son os où il trouve quelque chose à ronger.\par
L’égoïste voit la question sous un tout autre jour. Je ne recule pas avec un religieux effroi devant ta ou votre propriété ; je la considère au contraire toujours comme \emph{ma} propriété, que je n’ai pas à « respecter ». Traitez donc de même ce que vous appelez ma propriété ! C’est en nous plaçant tous à ce point de vue qu’il nous sera le plus facile de nous entendre.\par
 Les Libéraux politiques ont à cœur d’abolir autant que possible toutes les servitudes, afin que chacun soit franc maître de son champ, ce champ n’eût-il que tout juste assez de surface pour que le fumier a un homme suffît à l’engraisser. (« Que les cultivateurs se marient de bonne heure, afin de profiter du fumier de leur femme ! ») Peu importe que le champ soit petit, pourvu qu’on ait le sien, qu’il soit une propriété, et une propriété \emph{respectée !} Plus il y aura de propriétaires, plus l’Etat sera riche en « hommes libres » et en « bons patriotes ».\par
Le Libéralisme politique, comme toute religiosité, compte sur le \emph{respect}, l’humanité, la charité ; aussi est-il perpétuellement déçu. Car dans la pratique de la vie les gens ne respectent rien. Tous les jours, on voit de grands propriétaires arrondir leur domaine en accaparant les propriétés plus petites qui l’avoisinent ; et l’on voit tous les jours de petits propriétaires dépossédés obligés de redevenir mercenaires ou fermiers sur le lopin de terre qui leur a été légalement extorqué. La concurrence couvre de son pavillon le dol et la violence, et ce n’est pas le respect de la propriété qui peut s’opposer à ce brigandage. Si, au contraire, les « petits propriétaires » s’étaient dit que la grande propriété, elle aussi, est à eux, ils ne s’en seraient pas d’eux-mêmes respectueusement écartés et on ne les expulserait pas.\par
La propriété telle que la comprennent les Libéraux bourgeois mérite les invectives des Communistes et de Proudhon : elle est insoutenable et inexistante, attendu que le citoyen propriétaire ne possède en réalité rien et est partout un \emph{banni}. Loin que le monde puisse lui appartenir, le misérable coin où il vivote n’est même pas à lui.\par
Proudhon ne veut pas entendre parler de \emph{propriétaires}, mais bien de \emph{possesseurs} ou d’\emph{usufruitiers}\footnote{ \noindent Voir par exemple : \emph{Qu’est-ce que la propriété}, p. 83.
 }.  Qu’est-ce à dire ? Il veut que nul ne puisse s’approprier le sol, mais en ait l’usage ; — mais ne lui accordât-on même que la centième partie du produit qu’il en tire, du fruit, cette fraction du moins serait sa propriété et il pourrait en user à sa guise. Celui qui n’a que l’usage d’un champ n’est évidemment pas propriétaire de ce champ ; celui-là l’est moins encore qui doit, ainsi que le veut Proudhon, abandonner de ce produit tout ce qui ne lui est pas strictement nécessaire ; seulement il est propriétaire du tantième qui lui reste. Proudhon ne nie donc que telle ou telle propriété, et non \emph{la} propriété. Si nous voulons nous approprier le sol, au lieu d’en laisser l’aubaine aux propriétaires fonciers, unissons-nous, associons-nous dans ce but, et formons une \emph{société}\footnote{ \noindent \emph{En français dans le texte. N. d. Tr.}
 } qui s’en rendra propriétaire. Si nous réussissons, ceux qui sont aujourd’hui propriétaires cesseront de l’être. Et de même que nous les aurons dépossédés de la terre et du sol, nous pourrons encore les expulser de mainte autre propriété, pour en faire la nôtre, la propriété des \emph{ravisseurs.} Les « ravisseurs » forment une société que l’on peut s’imaginer croissant et s’étendant [{\corr progressivement}] au point de finir par embrasser l’humanité entière. Mais cette humanité elle-même n’est qu’une pensée (un fantôme) et n’a de réalité que dans les individus. Et ces individus pris en masse n’en useront pas moins arbitrairement avec la terre et le sol que ne le faisait l’individu isolé, le dit « propriétaire\footnote{ \noindent \emph{Id.}
 } ».\par
Ainsi donc, la propriété ne cesse pas de subsister et ne cesse même pas d’être « exclusive » du fait que l’humanité, cette vaste société, exproprie l’individu auquel elle afferme ou donne peut-être en fief une parcelle, de même qu’elle exproprie tout ce qui n’est pas humanité (elle ne reconnaît par exemple aucun droit de propriété aux animaux). Cela revient donc exactement  au même. Ce à quoi \emph{tous} veulent avoir part sera retiré à ce même individu qui veut l’avoir pour lui seul et sera érigé en \emph{bien commun}. En tant que bien commun, chacun en a sa \emph{part}, et cette part est sa \emph{propriété}. C’est ainsi que, d’après notre vieux droit de succession, une maison qui appartient à cinq héritiers est leur bien commun, indivis, tandis qu’un cinquième seulement du revenu est la propriété de chacun. Proudhon aurait pu nous épargner son pathos, lorsqu’il dit : « Il est certaines choses qui ne sont la propriété que de quelques-uns, mais auxquelles nous prétendons et auxquelles désormais nous ferons la chasse. Prenons-les, puisque c’est en prenant qu’on devient propriétaire, et puisque ce qui nous a manqué jusqu’à présent n’est passé aux mains des propriétaires actuels que par la prise. Associons-nous pour commettre ce vol. »\par
Il tâche de nous faire accepter l’idée que la Société est le possesseur primitif et l’unique propriétaire de droits imprescriptibles ; c’est envers elle que celui qu’on nomme propriétaire est coupable de vol (« La propriété c’est le vol\footnote{ \noindent \emph{En français dans le texte. N. d. Tr.}
 } ») ; si elle retire au propriétaire actuel ce qu’il détient comme lui appartenant, elle ne le vole pas, elle ne fait que rentrer en possession de son bien et user de son droit. — Voilà où on en arrive lorsqu’on fait du fantôme Société une personne morale. Ce que l’homme peut atteindre, voilà, au contraire, ce qui lui appartient : c’est à Moi que le monde appartient. Et que dites-vous d’autre, lorsque vous déclarez que « le monde appartient à \emph{tous} » ? Tous, c’est Moi, Moi, et encore Moi. Mais vous faites de « Tous » un fantôme que vous rendez sacré, de sorte que « Tous » devient le redoutable \emph{maître} de l’individu. Et c’est à son côté que se dresse alors le spectre du « Droit ».\par
Proudhon et les Communistes combattent l’\emph{égoïsme.}  Aussi leurs doctrines sont-elles la continuation et la conséquence du principe chrétien, du principe d’amour, de sacrifice, de dévouement à une généralité abstraite, à un « étranger ». En ce qui concerne la propriété, par exemple, ils ne font que compléter et consacrer doctrinalement ce qui existe en fait depuis longtemps, l’incapacité de l’individu à être propriétaire. Lorsque la loi nous déclare que \emph{ad reges potestas omnium pertinet, ad singulos proprietas ; omnia rex imperio possidet, singuli dominio}, cela signifie : le roi est propriétaire, car lui seul peut user et disposer de « tout », il a sur tout \emph{potestas} et \emph{imperium.} Les Communistes ont rendu la chose plus claire en dotant de cet \emph{imperium} la « Société de tous ». Donc : étant des ennemis de l’égoïsme, ils sont des — Chrétiens, ou, d’une façon plus générale, des hommes religieux, des visionnaires, subordonnés et asservis à une généralité, à une abstraction quelconque (Dieu, la Société, etc.).\par
Proudhon se rapproche encore des Chrétiens en ce qu’il accorde à Dieu ce qu’il dénie aux hommes : il le nomme (\emph{loc}. \emph{cit.,} p. 90) le propriétaire de la terre. Il montre ainsi qu’il ne peut se délivrer de l’idée qu’il doit exister quelque part un \emph{propriétaire ;} il conclut en définitive à un propriétaire, qu’il place seulement dans l’au-delà.\par
Le propriétaire, ce n’est ni Dieu ni l’Homme (la « Société humaine »), c’est — l’Individu.\par

\asterism

\noindent Proudhon (comme Weitling) croit faire la pire injure à la propriété en la qualifiant de « vol ». Sans vouloir soulever cette question embarrassante : « y a-t-il une objection bien sérieuse à faire au vol ? », nous demanderons simplement : L’idée de « vol » peut-elle subsister si on ne laisse pas subsister l’idée de « propriété » ? Comment pourrait-on voler, s’il n’y avait pas de propriété ? Ce qui n’appartient à personne  ne saurait être volé, celui qui puise de l’eau dans la mer ne vole pas. Par conséquent, la propriété n’est pas un vol, ce n’est que par elle que le vol devient possible. Weitling, qui considère tout comme la \emph{propriété de tous}, doit nécessairement aboutir à la même conclusion que Proudhon : si quelque chose appartient à « tous », l’individu qui se l’approprie est un voleur.\par
La propriété privée ne vit que grâce au \emph{Droit.} Le Droit est sa seule garantie ; — car posséder un objet n’est pas encore en être propriétaire, ce que je possède ne devient « ma propriété » que par la sanction du Droit ; — elle n’est pas « un fait\footnote{ \noindent \emph{En français dans le texte. N. d. Tr.}
 } », comme le pense Proudhon, mais une fiction, une idée ; une idée, voilà ce qu’est la propriété qu’engendre le Droit, la propriété légitime, garantie. Ce n’est pas Moi qui fais que ce que je possède est ma propriété, c’est — le Droit.\par
Néanmoins, on désigne sous le nom de propriété le \emph{pouvoir illimité} que j’ai sur les choses (objet, animal ou homme) dont je puis « user et abuser à mon gré » ; le Droit romain définit la propriété \emph{jus utendi et abutendi re sua, quatenus juris ratio patitur}, un droit \emph{exclusif} et \emph{illimité ;} mais la propriété a pour condition la puissance. Ce qui est en mon pouvoir est à moi. Tant que je maintiens ma situation de possesseur d’un objet, j’en reste le propriétaire ; s’il m’échappe, quelle que soit la force qui me l’enlève (le fait par exemple que je reconnais qu’un autre y a droit), voilà la propriété éteinte. Propriété et possession reviennent donc au même. Ce n’est point un droit extérieur à ma puissance qui me fait légitime propriétaire, mais ma puissance elle-même, et elle seule : si je la perds, l’objet m’échappe. Du jour où les Romains n’eurent plus la force de s’opposer aux Germains, Rome et les dépouilles du monde que dix siècles de  toute-puissance avaient entassées dans ses murs \emph{appartinrent} aux vainqueurs, et il serait ridicule de prétendre que les Romains en demeuraient néanmoins légitimes propriétaires. Toute chose est la propriété de qui sait la prendre et la garder, et reste à lui tant qu’elle ne lui est pas reprise ; c’est ainsi que la liberté appartient à celui qui la \emph{prend}.\par
La force seule décide de la propriété ; l’Etat (que ce soit l’Etat des bourgeois, des gueux ou tout uniment des hommes), étant seul fort, est aussi seul propriétaire ; Moi, l’Unique, je n’ai rien, je ne suis qu’un métayer sur les terres de l’Etat, je suis un vassal, et par suite un serviteur. Sous la domination de l’Etat, aucune propriété n’est \emph{à Moi}.\par
Je veux [{\corr accroître}] ma valeur, je veux élever le prix de toutes les propriétés dont est faite mon individualité, et je déprécierais la propriété ? Jamais ! De même que je n’ai jamais été jusqu’à présent justement apprécié parce que l’on mettait toujours au-dessus de Moi le Peuple, l’Humanité et cent autres abstractions, on n’a jamais non plus pleinement reconnu jusqu’à ce jour la valeur de la propriété. La propriété n’était que la propriété d’un fantôme, du Peuple par exemple ; mon existence tout entière « appartenait à la patrie » : j’appartenais, et par suite tout ce que je nommais \emph{mien }appartenait à la patrie, au Peuple, à l’Etat.\par
On demande aux Etats de mettre fin au paupérisme. Autant vaudrait leur demander de se couper la tête et de la poser à leurs pieds, car tant que l’Etat est un moi, le moi individuel doit rester un pauvre diable de non-moi. L’intérêt de l’Etat est d’être riche lui-même ; peu lui chaut que Pierre soit riche et Paul pauvre, il aimerait autant que ce fût Paul le riche et Pierre le pauvre ; il regarde l’un s’enrichir et l’autre s’apauvrir sans s’émouvoir de ce jeu de bascule. Comme individu, tous sont réellement égaux devant sa face, et en cela il a raison : pauvre et riche ne sont pour lui — rien, de même  que devant Dieu nous sommes tous « de pauvres pécheurs ». D’autre part, l’Etat a un très grand intérêt à ce que ces mêmes individus qui font de lui leur moi partagent ses richesses : il les fait participer à \emph{sa }propriété. La propriété, dont il fait un appât et une récompense pour les individus, lui sert à les apprivoiser, mais elle reste \emph{sa} propriété et nul n’en a la jouissance qu’autant qu’il porte dans son cœur le moi de l’Etat, comme un « membre loyal de la Société » qu’il est ; sinon, la propriété est confisquée ou fond en procès ruineux.\par
La propriété est et reste donc la \emph{propriété de l’Etat}, sans jamais être la propriété du Moi. Dire que l’Etat ne retire pas arbitrairement à l’individu ce que l’individu tient de l’Etat revient simplement à dire que l’Etat ne se vole pas lui-même. Celui qui est un « Moi d’Etat », c’est-à-dire un bon citoyen ou un bon sujet, jouit [{\corr de}] son fief en toute sécurité, mais il en jouit comme \emph{moi d’Etat} et non comme Moi propre, comme individu. C’est ce qu’exprime le code, quand il définit la propriété : ce que je nomme mien « de par Dieu ou de par le Droit ». Mais Dieu et le Droit ne le font mien que si — l’Etat ne s’y oppose pas.\par
En cas d’expropriation, de réquisition d’armes, etc., ou encore, par exemple, lorsque le fisc recueille une succession dont les ayant droits ne se sont pas présentés dans les délais légaux, le principe, habituellement voilé, saute aux yeux de tous : le Peuple, « l’Etat » est seul propriétaire ; l’individu n’est que fermier, tenancier, vassal.\par
Je voulais dire ceci : L’Etat ne peut se proposer de faire qu’un individu soit propriétaire dans son propre intérêt à lui, individu ; il ne peut vouloir que Je sois riche ou même que Je possède seulement quelque aisance ; pour autant que je suis Moi, l’Etat ne peut rien me reconnaître, rien me permettre, rien m’accorder. L’Etat ne peut obvier au paupérisme, parce que l’indigence est \emph{mon} indigence.  Celui qui \emph{n’est} que ce que font de lui les circonstances ou la volonté d’un tiers (l’Etat) n’\emph{a} non plus, et c’est parfaitement juste, que ce que ce tiers lui accorde. Et ce tiers ne lui donnera que ce qu’il \emph{mérite}, c’est-à-dire le salaire de ses \emph{services.} Ce n’est pas lui qui se fait valoir et qui tire de soi-même le meilleur parti possible, c’est l’Etat.\par
Ce sujet est de ceux que l’économie dite politique traite avec prédilection ; il n’est cependant pas du domaine de la « politique » et dépasse de cent coudées l’horizon de l’Etat, qui ne connaît que la propriété de l’Etat et ne peut répartir qu’elle. L’Etat ne peut faire autrement que de soumettre la possession de la propriété à des \emph{conditions}, comme il y soumet tout, par exemple le mariage qu’il soustrait à ma puissance en n’admettant comme valable que le mariage par lui sanctionné. Mais une propriété n’est \emph{ma} propriété que si elle est à moi \emph{sans conditions ;} ce n’est que si je suis \emph{inconditionné}, que je puis être propriétaire, m’unir à la femme que j’aime, et me livrer librement à un « commerce ».\par
L’Etat ne s’inquiète ni de \emph{Moi} ni du \emph{mien}, il ne se préoccupe que de \emph{soi} et du \emph{sien ;} si j’ai une valeur à ses yeux, ce n’est que comme « son enfant », « enfant du pays », etc ; comme Moi, je ne lui suis rien. Ma vie, ses hauts et ses bas, ma fortune ou ma ruine ne sont pour l’intelligence de l’Etat qu’une contingence, un accident. Mais si Moi et le mien ne sommes pour lui qu’un accident, qu’est-ce que cela prouve, sinon qu’il est incapable de me comprendre ? Je dépasse sa compréhension, ou, en d’autres termes, son intelligence est trop courte pour me saisir. C’est ce qui explique d’ailleurs qu’il ne puisse rien faire pour moi.\par
Le paupérisme est un corollaire de la \emph{non-valeur du Moi}, de mon impuissance à me faire valoir. Aussi Etat et paupérisme sont-ils deux phénomènes inséparables. L’Etat n’admet pas que je me mette moi-même  à profit et il n’existe qu’à condition que je n’aie pas voix au marché : toujours il vise à \emph{tirer parti de moi}, c’est-à-dire à m’exploiter, à me dépouiller, à me faire servir à quelque chose, ne fût-ce qu’a soigner une \emph{proies} (prolétariat) ; il veut que je sois « sa créature ».\par
Le paupérisme ne pourra être enrayé que du jour où ma valeur ne dépendra plus que de moi, où je la fixerai moi-même et ferai moi-même mon prix. Si je veux me voir en hausse, c’est à moi à me hausser et à me soulever.\par
Quoi que je fasse, que je fabrique de la farine ou du coton, ou que j’extraie à grande peine du sol le fer et le charbon, c’est là \emph{mon} travail et je veux en tirer moi-même tout le profit possible. Me plaindre ne servirait de rien, mon travail ne sera pas payé ce qu’il vaut ; l’acheteur ne m’écoutera pas et l’Etat fera de même la sourde oreille jusqu’au moment où il croira nécessaire de m’ « apaiser » pour prévenir l’explosion de ma redoutable puissance. Mais ces moyens d’ « apaisement » dont il use en guise de soupape de sûreté sont tout ce que je puis attendre de lui ; si je m’avise de réclamer plus, l’Etat se tournera contre moi et me fera sentir ses griffes et ses serres, car il est le roi des animaux, le lion et l’aigle. Si le prix qu’il fixe à mon travail et à mes marchandises ne me satisfait pas, et si je tente de fixer moi-même la valeur d’échange de mes produits, c’est-à-dire de faire en sorte que « je sois payé de mes peines », je me heurterai à une fin de non recevoir absolue chez le consommateur. Si ce conflit se dénouait par un accord entre les deux parties, l’Etat n’y trouverait rien à redire, car peu lui importe comment les particuliers s’arrangent entre eux, du moment que leur entente ne lui cause aucun préjudice. Il ne se juge lésé et mis en péril que si ne parvenant pas à trouver un terrain d’entente, les antagonistes se prennent aux cheveux. Ce sont là des relations immédiates d’homme à homme que l’Etat ne peut tolérer ; il doit s’interposer comme — \emph{médiateur},  il doit — \emph{intervenir}. L’Etat, en assumant ce rôle de tampon, est devenu ce qu’était Jésus-Christ, ce qu’étaient l’Eglise et les Saints, un « entremetteur ». Il sépare les hommes et s’interpose entre eux comme « Esprit ».\par
Les ouvriers qui réclament une augmentation de salaire sont traités en criminels dès qu’ils tentent de l’arracher de force au patron. Que doivent-ils faire ? S’ils n’usent pas de leur force, ils s’en retourneront les mains vides ; mais user de sa force, recourir à la contrainte, c’est mettre en pratique le « aide-toi toi-même », c’est se faire valoir soi-même, tirer librement et réellement de sa propriété ce qu’elle vaut, toutes choses que l’Etat ne peut tolérer. Que faire donc, diront les travailleurs ? Que faire ? Vous compter, ne compter que sur vous-mêmes et ne pas vous occuper de l’Etat !\par
Voilà pour le travail de mes bras ; il en va de même du travail de mon cerveau. L’Etat me permet de tirer profit de toutes mes pensées, et d’en faire l’objet d’un commerce avec les hommes (j’en tire déjà un prix, du seul fait, par exemple, qu’elles me valent l’estime ou l’admiration des auditeurs) ; il me le permet, mais pour autant seulement que \emph{mes} pensées soient — \emph{ses} pensées. Si je nourris au contraire des pensées qu’il ne peut approuver, c’est-à-dire faire siennes, il m’interdit formellement d’en réaliser la valeur, de les \emph{échanger} et d’en \emph{commercer}. Mes pensées ne sont libres que lorsque l’Etat les agrée, c’est-à-dire lorsqu’elles sont des pensées de l’Etat. Il ne me laisse philosopher en liberté que si je me montre « philosophe d’Etat » ; mais je ne puis pas philosopher \emph{contre }l’Etat, bien qu’il me permette volontiers de remédier à ses « imperfections », de le « redresser ». — De même, donc, que je ne puis considérer mon « Moi » comme légitime que s’il porte l’estampille de l’Etat et s’il peut exhiber les certificats et passe-ports que ce dernier lui a gracieusement accordés, de même je ne suis  autorisé à faire valoir mon « Mien » que si je le tiens pour le « sien », pour un fief relevant de l’Etat. Mes chemins doivent être ses chemins, sinon il me met à l’amende, et mes pensées ses pensées, sinon il me bâillonne.\par
Rien n’est plus redoutable pour l’Etat que la valeur du Moi ; il n’est rien dont il doive plus soigneusement me tenir à l’écart que de toute occasion de m’exploiter moi-même. Je suis l’adversaire inconciliable de l’Etat, qui ne peut échapper à l’étau du dilemme : lui ou moi. Aussi s’attache-t-il non seulement à paralyser le Moi, mais encore à annihiler le Mien. Il n’y a dans l’Etat aucune — propriété, c’est-à-dire aucune propriété de l’individu : il n’y a que des propriétés de l’Etat. Ce que j’ai, je ne l’ai que par l’Etat ; ce que je suis, je ne le suis que par lui.\par
Ma propriété privée n’est que ce que l’Etat me concède du \emph{sien} en en frustrant (privant) d’autres de ses membres : c’est toujours une propriété de l’Etat.\par
Mais quoi que fasse l’Etat, je sens toujours plus clairement qu’il me reste une puissance considérable ; j’ai un pouvoir sur moi-même, c’est-à-dire sur tout ce qui n’est et ne peut être qu’à moi et qui n’existe que parce que c’est mien.\par
Que faire, quand mon chemin n’est plus le sien, quand mes pensées ne sont plus les siennes ? Passer outre, et ne compter qu’avec moi-même et sur moi-même. Ma propriété réelle, celle dont je puis disposer à mon gré, dont je puis trafiquer à ma guise, ce sont mes pensées, qui n’ont que faire d’une sanction et qu’il m’importe peu de voir légitimer par une destination, une autorisation ou une grâce. Etant miennes, elles sont mes \emph{créatures}, et je puis les abandonner pour d’autres ; si je les cède en échange d’autres, ces autres deviennent à leur tour ma propriété.\par
Qu’est-ce donc que ma propriété ? Ce qui est en  ma \emph{puissance,} et rien d’autre. A quoi suis-je légitimement \emph{autorisé ?} A tout ce dont je suis \emph{capable.} Je me donne le droit de propriété sur un objet, par le seul fait que je m’en empare, ou, en d’autres termes, je deviens propriétaire de droit chaque fois que je me fais de force propriétaire ; en me donnant le pouvoir, je me donne le titre.\par
Tant que vous ne pouvez m’arracher mon pouvoir sur une chose, cette chose demeure ma propriété. Hé bien, soit ! Que la force décide de la propriété, et j’attendrai tout de ma force ! La puissance étrangère, la puissance que je laisse à autrui a fait de moi un serf ; puisse ma propre puissance faire de moi un propriétaire ! Que je rentre donc en possession de la puissance que j’ai abandonnée aux autres, ignorant que j’étais de la valeur de mes forces. A mes yeux, ma propriété s’étend jusqu’où s’étend mon bras ; je revendiquerai comme mien tout ce que je suis capable de conquérir, et je ne verrai à ma propriété d’autre limite réelle que ma — force, unique source de mon droit.\par
Ici, c’est à l’égoïsme, à l’intérêt personnel de décider, et non pas au principe d’amour, aux raisons de sentiment telles que charité, indulgence, bienveillance ou même équité et justice (car la \emph{justitia} aussi est un phénomène d’amour, un produit de l’amour) : l’amour ne connaît que le « sacrifice » et exige le « dévouement ». Sacrifier quelque chose ? Se priver de quelque chose ? L’égoïste n’y songe pas ; il dit simplement : ce dont j’ai besoin, il me le faut, et je l’aurai !\par
Toutes les tentatives faites pour soumettre la propriété à des lois rationnelles ont leur source dans l’\emph{amour} et aboutissent à un orageux océan de réglementations et de contraintes. Le Socialisme et le Communisme ne font eux-mêmes pas exception à la règle. Chacun doit être pourvu de moyens d’existence suffisants, et peu importe que ces moyens on les trouve, selon  l’idée socialiste, dans une propriété personnelle, ou qu’avec les Communistes on les puise dans la communauté des biens. Les individus ne cesseront en aucun cas de se sentir dépendants. La \emph{cour arbitrale }que vous chargerez de répartir équitablement les biens ne m’accordera que la part que m’aura mesurée son esprit d’équité, son bienveillant souci des besoins de tous. Moi, l’individu, je ne vois pas un moindre obstacle dans la richesse de la collectivité que dans la richesse des autres individus, car ni l’une ni l’autre ne m’appartient. Que les biens soient entre les mains de la communauté qui m’en accorde une partie ou entre les mains des particuliers, il en résulte toujours pour moi la même contrainte, attendu que je ne puis en aucun cas en disposer. Bien plus, en abolissant la propriété personnelle, le Communisme ne fait que me rejeter plus profondément sous la dépendance d’autrui, autrui s’appelant désormais la généralité ou la communauté. Bien qu’il soit toujours en lutte ouverte contre l’Etat, le but que poursuit le Communisme est un nouvel « Etat », un \emph{status}, un ordre de choses destiné à paralyser la liberté de mes mouvements, un pouvoir souverain supérieur à moi ; il s’oppose avec raison à l’oppression dont je suis victime de la part des individus propriétaires, mais le pouvoir qu’il donne à la communauté est plus tyrannique encore.\par
C’est par une autre voie que l’égoïsme marche vers la suppression de la misère de la plèbe. Il ne dit pas : Attends ce que l’autorité quelconque chargée de partager les biens au nom de la communauté te donnera dans son équité (car c’est d’un don qu’il s’agit depuis toujours dans les « Etats », chacun y recevant selon ses mérites, c’est-à-dire ses services) ; il dit : Mets la main sur ce dont tu as besoin, prends le. C’est la déclaration de guerre de tous contre tous. Moi seul suis juge de ce que je veux avoir.\par
« En vérité, cette sagesse là n’est pas nouvelle, car  c’est ainsi qu’en ont de tout temps usé les égoïstes ». Peu importe que la chose ne soit pas neuve, si ce n’est que d’aujourd’hui qu’on en a \emph{conscience ;} et cette conscience ne peut prétendre à une bien haute antiquité (à moins que vous ne la fassiez remonter aux lois de l’Egypte et de Sparte) ; il suffirait de votre objection et du mépris avec lequel vous parlez de l’égoïste pour prouver qu’elle est peu répandue. Ce qu’il faut bien se dire, c’est que l’acte de mettre la main sur un objet, de s’en emparer, n’est nullement méprisable ; il est purement le fait de l’égoïste conscient et conséquent avec lui-même.\par
Ce n’est que quand je n’attendrai plus ni des individus ni de la communauté ce que je puis me donner moi-même que j’échapperai aux chaînes de — l’Amour ; la plèbe ne cessera d’être la plèbe que du jour où elle \emph{prendra.} Elle n’est plèbe que parce qu’elle a peur de prendre et peur du châtiment qui s’en suivrait. Prendre est un péché, prendre est un crime ; — voilà le dogme, et ce dogme à lui seul suffit pour créer la plèbe ; mais si la plèbe reste ce qu’elle est, à qui la faute ? A elle d’abord, qui admet ce dogme, et à ceux-là ensuite qui, par « égoïsme », (pour leur renvoyer leur injure favorite) veulent qu’il soit respecté. On n’a pas \emph{conscience} de cette « sagesse nouvelle », et c’est la vieille conscience du péché qui en est cause.\par
Si les hommes parviennent à perdre le respect de la propriété, chacun aura une propriété, de même que tous les esclaves deviennent hommes libres dès qu’ils cessent de respecter en leur maître un maître. Alors pourront se conclure des \emph{alliances} entre individus, des associations égoïstes, qui auront pour effet de multiplier les moyens d’action de chacun et d’affermir sa propriété sans cesse menacée.\par
Selon les Communistes, la communauté doit être propriétaire. C’est au contraire Moi qui suis propriétaire et je ne fais que m’entendre avec d’autres au  sujet de ma propriété. Si la communauté va à l’encontre de mes intérêts, je m’insurge contre elle et je me défends. Je suis propriétaire, mais la propriété \emph{n’est pas sacrée}. Ne serais-je donc que possesseur ? Eh non ! Jusqu’à présent on n’était que possesseur, on ne s’assurait la jouissance d’une parcelle qu’en laissant les autres jouir de la leur. Mais désormais \emph{tout }m’appartient ; je suis propriétaire de \emph{tout} ce dont j’ai besoin et dont je puis m’emparer. Si le Socialiste dit : la Société me donne ce qu’il me faut, — l’Egoïste répond : je prends ce qu’il me faut. Si les Communistes agissent en gueux, l’Egoïste agit en propriétaire.\par
Toutes les tentatives ayant pour but le soulagement des classes misérables doivent échouer si elles prennent pour principe l’Amour. C’est de l’égoïsme seul que la plèbe doit attendre quelqu’aide ; cette aide elle doit se la prêter à elle-même, et — c’est ce qu’elle fera. La plèbe est une puissance pourvu qu’elle ne se laisse pas dompter par la crainte. « Les gens perdraient tout respect si on ne les forçait pas à avoir peur », disait l’épouvantail au chat botté.\par
La propriété ne doit et ne peut donc pas être abolie ; ce qu’il faut, c’est l’arracher aux fantômes pour en faire \emph{ma} propriété. Alors s’évanouira cette illusion que je ne suis pas [{\corr autorisé}] à prendre tout ce dont j’ai besoin.\par
« Mais de combien de choses l’homme n’a-t-il pas besoin ! » Celui qui a besoin de beaucoup et qui s’entend à le prendre s’est-il jamais fait faute de se l’approprier ? Napoléon a pris l’Europe et les Français Alger. Ce qu’il faudrait, c’est que la plèbe, que le respect paralyse, apprenne enfin à se procurer ce qu’il lui faut. Si elle va trop loin et si vous vous jugez lésés, hé bien ! défendez-vous : il n’est pas nécessaire que vous lui fassiez bénévolement des cadeaux. Quand elle apprendra à se connaître, ou plutôt quand ceux de la plèbe apprendront à se connaître, ils cesseront d’en faire partie par là-même qu’ils refuseront vos aumônes.  Mais il est parfaitement ridicule de déclarer « pécheur et criminel » celui qui ne prétend plus vivre de vos bienfaits et veut se tirer d’affaire lui-même. Vos dons le trompent et lui font prendre patience. Défendez votre propriété, vous serez forts ; mais si vous voulez garder la faculté de donner et jouir d’autant plus de droits politiques que vous pouvez faire plus d’aumônes (taxe des pauvres)\footnote{ \noindent Le gouvernement anglais, dans un projet de loi électorale pour l’Irlande, proposa d’accorder l’électorat à tous ceux qui payaient 5 livres sterling de taxe des pauvres. Celui qui fait l’aumône acquiert des droits politiques !
 }, cela durera ce que ceux que vous gratifiez de vos dons permettront que cela dure.\par
La question de la propriété n’est pas, je crois l’avoir montré, aussi simple à résoudre que se l’imaginent les Socialistes et même les Communistes. Elle ne sera résolue que par la guerre de tous contre tous. Les pauvres ne deviendront libres et propriétaires que lorsqu’ils — s’insurgeront, se soulèveront, s’élèveront. Quoi que vous leur donniez, ils voudront toujours davantage, car ils ne veulent rien moins que — la suppression de tout don.\par
On demandera : Mais que se passera-t-il, quand les sans fortune auront pris courage ? Comment s’accomplira le nivellement ? Autant vaudrait me demander de tirer l’horoscope d’un enfant. Ce que fera un esclave quand il aura brisé ses chaînes ? — Attendez, et vous le saurez.\par

\asterism

\noindent La \emph{concurrence} est étroitement liée au principe de la [{\corr bourgeoisie}]. Est-elle autre chose que l’\emph{égalité ?} et l’égalité n’est-elle pas précisément un produit de cette Révolution dont la bourgeoisie ou la classe moyenne fut l’auteur ? Il n’est défendu à personne de rivaliser avec tous les autres membres de l’Etat (le Prince excepté, parce qu’il représente l’Etat) ; chacun peut  travailler à s’élever au rang des autres et à les surpasser, voire même à les ruiner, les dépouiller et leur arracher jusqu’aux derniers lambeaux de leur fortune. Cela prouve à toute évidence que devant le tribunal de l’Etat chacun n’a la valeur que d’un « simple individu » et ne doit compter sur aucune faveur. Surpassez-vous l’un l’autre, enchérissez l’un sur l’autre tant que vous voulez et tant que vous pouvez ; moi, l’Etat, je n’ai rien à y voir. Vous êtes libres de concourir entre vous, vous êtes concurrents et la concurrence est votre position \emph{sociale}. Mais devant moi, l’Etat, vous n’êtes que de « simples individus ».\par
L’égalité que l’on a théoriquement établie en principe entre tous les hommes trouve sa mise en application et sa réalisation pratique dans la concurrence, car l’égalité\footnote{ \noindent \emph{« Egalité » en français dans le texte. N. d. T.}
 } n’est que la libre concurrence. Tous sont, vis-à-vis de l’Etat, de — simples particuliers, et dans la Société, c’est-à-dire vis à-vis les uns des autres, des — concurrents.\par
Je n’ai pas à être autre chose qu’un simple particulier pour pouvoir concourir avec tout autre homme, sauf le Prince et sa famille. Cette liberté était jadis impossible, attendu qu’on ne jouissait de la liberté de se faire valoir que dans la corporation et par la corporation. Sous le système des corporations et de la féodalité, l’Etat accordait des \emph{privilèges}, tandis que sous le régime de la concurrence et du Libéralisme il se borne à accorder des \emph{patentes} (brevet donné à un candidat et établissant que telle profession lui est ouverte [patente]).\par
Mais la « libre concurrence » est-elle bien réellement « libre » ? Est-elle même vraiment une « concurrence », c’est-à-dire un concours entre les \emph{personnes ? }C’est ce qu’elle prétend être, puisqu’elle fonde justement son droit sur ce titre ; elle est née, en effet, du fait que les personnes ont été affranchies de toute domination  personnelle. Peut-on dire que la concurrence est « libre », quand l’Etat, que le principe de la bourgeoisie fait souverain, s’ingénie à la restreindre de mille façons ?\par
« Voici un riche fabricant qui fait de brillantes affaires, et je voudrais lui faire la concurrence.\par
— Fais, dit l’Etat, je ne vois, pour ma part, rien qui s’oppose à ce que tu le fasses.\par
— Oui, mais il me faudrait de la place pour mon installation, il me faudrait de l’argent !\par
— C’est regrettable, mais si tu n’as pas d’argent, tu ne peux pas songer à concourir. Et il ne s’agit pas que tu prennes rien à personne, car je protège la propriété et ses privilèges. »\par
La libre concurrence n’est pas « libre », parce que les moyens de concourir, les \emph{choses} nécessaires à la concurrence me font défaut. Contre ma personne, on n’a rien à objecter ; mais comme je n’ai pas la \emph{chose}, il faut que ma personne renonce. Et qui est en possession des moyens, qui a ces choses nécessaires ? Est-ce peut-être tel ou tel fabricant ? Non, car dans ce cas je pourrais les lui prendre ! Le seul propriétaire, c’est l’Etat ; le fabricant n’est pas propriétaire ; ce qu’il possède il ne l’a qu’à titre de concession, de dépôt.\par
— Allons, soit ! Si je ne puis rien contre le fabricant, je m’en vais faire concurrence à ce professeur de droit ; c’est un sot et j’en sais cent fois plus que lui : je ferai déserter son auditoire.\par
— As-tu fait des études, mon ami, et es-tu reçu docteur ?\par
— Non, mais à quoi bon ? Je possède largement les connaissances nécessaires à cet enseignement.\par
— J’en suis fâché, mais ici la concurrence n’est pas « libre ». Contre ta personne il n’y a rien à dire, mais la \emph{chose} essentielle te manque : le diplôme de docteur. Et ce diplôme, moi, l’Etat, je  l’exige ! Demande-le moi d’abord bien gentiment, et nous verrons ensuite ce qu’il y a à faire \emph{ ».}\par
Voilà à quoi se réduit la « liberté » de la concurrence. Il faut que l’Etat, mon seigneur et maître, me confère l’aptitude à concourir.\par
Mais aussi, sont-ce bien en réalité les \emph{personnes} qui concourent ? Non, encore une fois, ce sont les \emph{choses ! }L’argent en première ligne, etc.\par
Dans la lutte, il y aura toujours des vaincus (ainsi le poète médiocre devra céder la palme, etc.). Mais ce qu’il importe de distinguer, c’est d’abord si les moyens qui font défaut au concurrent malheureux sont personnels ou matériels, et, en second lieu, si les moyens matériels peuvent s’acquérir à force d’\emph{énergie personnelle}, ou si l’on ne peut les obtenir que par \emph{faveur}, en simples dons, le pauvre, par exemple étant forcé de laisser au riche sa richesse, c’est-à-dire de lui en faire cadeau. En somme, s’il faut que j’attende l’autorisation de l’Etat pour avoir les moyens ou les mettre en œuvre (comme c’est le cas par exemple lorsqu’il s’agit d’un diplôme), ces moyens sont une grâce que l’Etat m’accorde\footnote{ \noindent Dans les collèges, les universités etc., on voit des pauvres concourir avec des riches. Mais cela ne leur est en général possible que grâce à des bourses, qui — cela est significatif — ont pour la plupart été fondées à une époque où la libre concurrence était encore loin d’être admise en principe. Le principe de la concurrence ne fonde pas de bourses d’études, mais il signifie : Aide-toi toi-même, c’est-à-dire procure-toi les moyens. Ce que l’Etat dépense dans ce but n’est qu’un placement à intérêt, destiné à lui procurer des « serviteurs ».
 }.\par
Tel est, au fond, le sens de la libre concurrence : L’Etat considère tous les hommes comme ses enfants et comme égaux ; libre à chacun de faire tout son possible pour mériter les biens et les faveurs dont l’Etat est le dispensateur. Aussi tous se lancent à la poursuite de la fortune, des biens (argent, emplois, titres, etc.), en un mot des \emph{moyens matériels.}\par
Au sens bourgeois, tout homme possède, chacun est  « propriétaire ». Comment se fait-il donc que la plupart n’aient pour ainsi dire rien ? Cela vient de ce que la plupart sont déjà tout heureux rien que d’être propriétaires, ne fût-ce que de quelques loques, comme les enfants se font un bonheur de leur première culotte ou du premier sou qu’on leur a donné. A examiner la chose de plus près, voici comment il faut l’entendre. Le Libéralisme vint d’abord déclarer qu’il était de l’essence de l’homme d’être non pas propriété, mais propriétaire. Comme cela ne s’appliquait qu’à « l’Homme » et non à l’individu, cela abandonnait à l’individu le soin de déterminer la quotité [{\corr nécessaire}] à la satisfaction de son intérêt personnel. Il en résulta que l’égoïsme de l’individu, conservant au sujet de cette quotité la plus grande latitude, se jeta à corps perdu dans la concurrence.\par
Fatalement, l’égoïsme heureux devait porter ombrage à celui qui était moins favorisé ; ce dernier, s’appuyant toujours sur le principe de l’humanité, souleva la question du quotient de répartition des biens sociaux et la résolut ainsi : « L’homme doit avoir autant qu’il lui est nécessaire. »\par
Mais \emph{mon} égoïsme pourra-t-il se contenter de cela ? Les besoins de « l’Homme » ne sont nullement une mesure applicable à moi et à mes besoins ; car je puis avoir besoin de plus ou de moins. Non, je dois avoir autant que je suis capable de m’approprier.\par
Chacun n’a pas à sa disposition les \emph{moyens} de concourir, parce que ces moyens (et c’est là le vice [{\corr fondamental}] de la concurrence) ne dépendent pas de la personne, mais de circonstances tout à fait indépendantes de cette dernière. La plupart des hommes sont dépourvus de ces \emph{instruments} et, par suite, des \emph{biens} qu’ils pourraient en tirer.\par
Aussi les Socialistes réclament-ils pour tous les hommes les instruments, et préparent-ils une société qui fournira à tous ces instruments. Nous ne reconnaissons plus, disent-ils, tes richesses (avoir) comme  ta richesse (pouvoir)\footnote{ \noindent \emph{Le mot allemand « Vermögen » a un sens très étendu et signifie suivant les cas : force, puissance, faculté, moyen, richesse, fortune ou pécule. Nous le traduirons par richesse, en priant le lecteur de bien vouloir se rappeler que nous entendons par ce mot la richesse « instrument de production » et non « résultat de production ». C’est d’ailleurs le sens étymologique du mot français, qui, par sa racine germanique « rik » ou « reich », signifie « puissance », « Richesse c’est pouvoir » disait Hobbes. N. d. Tr.}
 }. Tu auras à te créer une autre richesse, à te pourvoir d’autres moyens d’action, qui seront ta \emph{force de travail}. Sous l’homme en possession d’un avoir, sous le « possesseur », nous apercevons l’homme, aussi avons-nous provisoirement respecté ce possesseur que nous nommions « propriétaire ». Mais il faut bien te dire que tu ne détiens les choses qu’en attendant que tu sois « exproprié ».\par
Celui qui possède est riche, mais pour autant seulement que les autres ne le soient pas. Et comme ta marchandise ne forme ta richesse qu’aussi longtemps que tu es capable de la maintenir en ta possession, c’est-à-dire aussi longtemps que nous n’avons pas de pouvoir sur elle, il faudra bien que tu cherches à te procurer d’autres moyens d’action, car notre puissance l’emporte aujourd’hui sur ta prétendue richesse.\par
Parvenir à être considéré comme possesseur réalisait déjà un progrès énorme. Le servage disparaissait et l’homme, qui jusque-là avait dû la corvée à son seigneur et avait été plus ou moins la propriété de ce dernier, devenait, à son tour, un « seigneur », un « monsieur ». Mais il ne suffit plus désormais, que tu possèdes : ton avoir est démonétisé ; par contre, ton travail augmente de prix. Tu ne vaux à nos yeux qu’en tant que tu \emph{mets en œuvre} les choses, comme autrefois en tant que tu les \emph{avais}. C’est ton travail qui est ta richesse. Tu n’es plus, désormais, maître et possesseur que de ce qui naît de ton travail, et non plus de ce que peut te donner un héritage.\par
En attendant, comme il n’existe pas de possession  qui n’ait à sa source l’héritage, comme tous les sous qui forment ton avoir sont à l’effigie de l’hérédité et non à l’effigie du travail, il faut que tout soit refondu au creuset commun.\par
Mais est-il bien vrai, comme le pensent les Communistes, que ma richesse ne consiste que dans mon travail ? Ne consiste-t-elle pas plutôt en tout ce dont je suis capable ? La Société des travailleurs elle-même est bien obligée d’en convenir, puisqu’elle vient en aide aux malades, aux enfants, aux vieillards, en un mot à ceux qui sont impropres au travail. Ceux-ci sont encore capables de bien des choses, ne fut-ce que de conserver leur vie au lieu de se l’ôter. Et s’ils sont capables de vous faire désirer leur conservation, c’est qu’ils possèdent un pouvoir sur vous. A celui qui n’exercerait absolument aucun pouvoir sur vous, vous n’accorderiez rien, il n’aurait plus qu’à disparaître.\par
Ainsi, ta \emph{richesse} consiste en tout ce dont tu es \emph{capable !} Si tu es capable de procurer un plaisir à des milliers d’hommes, ces milliers d’hommes te donneront des honoraires, parce qu’il est en ton pouvoir de cesser de leur être agréable et que cela les oblige à acheter ton travail. Mais si tu n’es capable d’intéresser personne à toi, tu es tout juste capable de disparaître.\par
Ne dois-je donc pas, moi qui suis capable de beaucoup, avoir l’avantage sur ceux qui peuvent moins ? Nous voici attablés devant l’abondance : vais-je m’abstenir de me servir de mon mieux, et attendre ce qui me reviendra d’un partage égal ?\par
Contre la concurrence se dresse le principe de la Société des gueux, le principe du \emph{partage égal}.\par
L’individu ne supporte pas de n’être considéré que comme une \emph{fraction}, un tantième de la société, parce qu’il est \emph{plus} que cela ; son unicité s’insurge contre cette conception qui le diminue et le rabaisse.\par
Aussi n’admet-il pas que les autres lui adjugent sa part ; déjà, dans la Société des travailleurs, il soupçonne  que le partage égal aura pour effet de dépouiller le fort au profit du faible. Il n’attend au contraire sa richesse que de lui-même, et il dit : ce que je suis capable de me procurer, voilà ma richesse. Quelle richesse ne possède pas l’enfant dans son sourire, dans ses gestes, dans sa voix, dans le seul fait qu’il existe ! Etes-vous capables de résister à son désir ? Toi, mère, ne lui offres-tu pas ton sein, et toi, père, ne te refuses-tu pas bien des choses pour qu’il ne manque de rien ? Il vous contraint, et par cela même, il possède ce que vous croyez à vous.\par
Si je tiens à ta personne, ta seule existence a déjà pour moi une valeur ; si je n’ai besoin que d’une de tes facultés, c’est ta complaisance ou ton assistance qui ont un prix à mes yeux, et que j’achète.\par
Il se peut aussi que tu ne saches prendre à mon estimation qu’une valeur en argent : c’était le cas des citoyens allemands vendus à beaux deniers et expédiés en Amérique, dont l’histoire raconte l’odyssée. Dira-t-on que le vendeur devait faire plus grand cas d’eux, qui se laissèrent vendre ? Il préférait l’argent comptant à cette marchandise vivante qui n’avait pas su se rendre précieuse à ses yeux. S’il ne reconnaissait pas en eux une plus grande valeur, c’est qu’en définitive sa marchandise ne valait pas grand’chose : et un fripon ne regarde pas à la qualité de ce qu’il donne. Comment leur aurait-il témoigné une estime qu’il ne ressentait pas, qu’il pouvait à peine ressentir pour un pareil bétail ?\par
La pratique égoïste consiste à ne considérer les autres ni comme des propriétaires ni comme des gueux ou des travailleurs, mais à voir en eux une partie de votre richesse, des \emph{objets qui peuvent vous servir. }Cela étant, vous ne paierez rien à celui qui possède (« au propriétaire »), vous ne paierez rien à celui qui travaille, vous ne donnerez qu’à celui dont vous \emph{avez besoin.} Avons-nous besoin d’un roi ? disent les Américains du Nord. Et ils répondent : Nous ne donnerions  pas un liard ni de lui ni de son travail.\par
Lorsqu’on dit que la concurrence met tout à la portée de tous, on s’exprime d’une façon inexacte ; il est plus juste de dire que grâce à elle tout est à vendre. En mettant tout \emph{à la disposition} de tous, elle le livre à leur appréciation et en demande un prix.\par
Mais les amateurs manquent le plus souvent du moyen de se faire acheteurs : ils n’ont pas d’argent. On peut, avec de l’argent, se procurer tout ce qui est à vendre, mais justement c’est l’argent qui fait défaut. Où prendre l’argent, cette propriété mobile ou circulante ? Sache donc que tu as autant d’argent que tu as de — puissance ; car tu as la valeur que tu sais te donner.\par
On ne paie pas avec de l’argent, dont on peut être à court, mais avec sa richesse, son « pouvoir » ; car on n’est propriétaire que de ce dont on est maître.\par
Weitling a imaginé un nouvel instrument d’échange, le travail. Mais le véritable instrument de paiement reste encore, comme toujours, notre \emph{richesse :} Tu paies avec ce que tu as « en ton pouvoir ». Songe donc à augmenter ta richesse !\par
En concédant cela, on est tout près de répéter la maxime : « A chacun selon ses moyens ». Mais qui me donnera « selon mes moyens » ? La Société ? Je devrais pour cela me soumettre à son estimation. Non. Je \emph{prendrai} selon mes moyens.\par
« Tout appartient à tous ! » Cette proposition procède aussi d’une théorie futile. A chacun appartient seulement ce qu’il \emph{peut.} Lorsque je dis : le monde est à moi, c’est là aussi une phrase vide de sens, à moins que je veuille simplement faire entendre que je ne respecte aucune propriété étrangère. Cela seul est à moi que j’ai en mon pouvoir, qui dépend de ma force.\par
On n’est pas digne d’avoir ce que par faiblesse on se laisse prendre ; on n’est pas digne de le garder parce qu’on n’est pas capable de le garder.\par
On fait grand bruit de l’« injustice séculaire » des  riches envers les pauvres. Comme si c’était la faute des riches s’il y a des pauvres, et comme si ce n’était pas aussi la faute des pauvres s’il y a des riches ! Quelle différence y a-t-il entre eux, sinon celle qui sépare la puissance de l’impuissance et ceux qui peuvent de ceux qui ne peuvent pas ? Quel crime les riches ont-ils commis ? « Ils sont durs ! » Mais qui donc a entretenu les pauvres, qui a pourvu à leur subsistance lorsqu’ils ne pouvaient plus travailler, qui a répandu à profusion les aumônes, ces aumônes dont le nom même signifie compassion (Ἐλεημοσύϛη) ? Les riches ne furent-ils pas toujours « compatissants » ? Ne furent-ils pas toujours « charitables » ? Et les taxes des pauvres, les crèches, les hospices, les établissements de bienfaisance de toute espèce, d’où viennent-ils ?\par
Mais tout cela ne vous suffit pas. Les riches devraient, n’est-ce pas, \emph{partager} avec les pauvres ? En un mot, ils devraient supprimer la misère. Sans compter qu’il y a à peine un de vous qui consentirait à partager, et que celui-là serait un fou, demandez-vous : Pourquoi les riches devraient-ils se dépouiller et se dévouer, alors que c’est aux pauvres que cette conduite profiterait, bien plus qu’à eux-mêmes ? Toi qui touches un écu par jour, tu es un riche à côté de milliers d’hommes qui vivent avec dix sous : est-il de ton intérêt de partager avec eux, ou n’est-ce pas plutôt du leur ?\par
Grâce à la concurrence, ce qu’on fait on ne le fait pas avec l’intention de le « faire de son mieux », mais avec l’intention de le faire le plus \emph{lucrativement} possible, avec le moins de frais et le plus grand bénéfice possible. Aussi n’étudie-t-on que pour se faire une position (\emph{brodstudium}), on apprend les courbettes et les belles manières, on tâche d’acquérir la routine et la « connaissance des affaires », on travaille « pour la forme ». Et tandis qu’en apparence il s’agit de « bien remplir ses fonctions », on ne vise en réalité qu’à faire « une bonne affaire », à gagner de l’argent. On  fait son métier prétendument par amour du métier, mais en réalité pour l’amour du bénéfice qu’il procure. Si l’on devient censeur, ce n’est pas que le métier soit attrayant, mais la position n’est pas déplaisante ; et puis on veut — monter en grade. On voudrait bien administrer, rendre la justice, etc. en toute conscience, mais on craint d’être déplacé ou révoqué : avant tout, il faut bien qu’on — vive.\par
Toute cette pratique est en somme une lutte pour cette \emph{chère vie}, une suite d’efforts ininterrompus pour s’élever de degré en degré jusqu’à plus ou moins de « bien-être ». Et toutes leurs peines et tous leurs soucis ne rapportent à la plupart des hommes qu’une « vie amère », une « amère indigence ». Tant d’ardeur pour si peu de chose !\par
Une infatigable âpreté à la curée ne nous laisse pas le temps de respirer et de nous arrêter à une \emph{jouissance} paisible. Nous ne connaissons pas la joie de posséder.\par
Lorsqu’on parle d’organiser le travail, on ne peut avoir en vue que celui dont d’autres peuvent s’acquitter à notre place, par exemple, celui du boucher, du laboureur, etc. ; mais il est des travaux qui restent du ressort de l’égoïsme, attendu que personne ne peut exécuter pour vous le tableau que vous peignez, produire vos compositions musicales, etc. ; personne ne peut faire l’œuvre de Raphaël. Ces derniers travaux sont ceux d’un Unique, ce sont les œuvres que cet Unique seul est à même d’exécuter, tandis que les premiers sont des travaux banaux que l’on pourrait appeler « humains », attendu que l’individualité de l’ouvrier y est sans importance et qu’on peut y dresser à peu près « tous les hommes ».\par
Comme la Société ne peut prendre en considération que les travaux qui présentent une utilité générale, les travaux \emph{humains}, sa sollicitude ne peut pas s’étendre à celui qui fait œuvre d’Unique ; son intervention dans ce cas pourrait même être nuisible.  L’Unique saura bien s’élever dans la Société par son travail, mais la Société ne peut pas élever l’Unique.\par
Il est par conséquent toujours à souhaiter que nous nous unissions pour les travaux \emph{humains}, afin qu’ils n’absorbent plus tout notre temps et tous nos efforts comme ils le faisaient sous le régime de la concurrence. A ce point de vue, le Communisme est appelé à porter des fruits. Ce dont tout le monde est capable ou peut devenir capable était, avant l’avènement de la Bourgeoisie, au pouvoir de quelques-uns et refusé à tous les autres : c’était le temps du Privilège. La bourgeoisie trouva juste de permettre à tous l’accès de ce qui paraissait convenir à quiconque est « homme ». Toutefois, ce qu’elle permettait à tous, elle ne le donnait réellement à personne : elle laissait seulement chacun libre de s’en emparer par ses efforts « humains ». Tous les yeux se dirigèrent vers ces biens humains, qui dès lors souriaient à tous les passants, et il en résulta cette tendance que l’on entend à chaque instant déplorer sous le nom de « matérialisme des mœurs ».\par
Le Communisme essaie d’y mettre un frein en répandant la croyance que les biens humains n’exigent pas que l’on se donne tant de peine pour eux, et qu’on peut, par une organisation judicieuse, se les procurer sans la grande dépense de temps et d’énergie qui a paru nécessaire jusqu’à présent.\par
Mais pour qui faut-il gagner du temps ? Pourquoi l’homme a-t-il besoin de plus de temps qu’il n’en faut pour ranimer ses forces épuisées par le travail ? Ici le Communisme se tait.\par
Pourquoi ? Hé bien ! pour jouir de soi-même comme Unique, après avoir fait sa part comme homme !\par
Dans la première joie de se voir autorisé à allonger la main vers tout ce qui est humain, on ne songea plus à désirer autre chose, et on se lança par les chemins de la concurrence à la poursuite de cet humain,  comme si sa possession était le but de tous nos vœux.\par
Mais, après une course effrénée, on s’aperçoit enfin que « la richesse ne fait pas le bonheur ». Et l’on cherche à se procurer le nécessaire à moins de frais, et à ne lui consacrer que le temps et les peines indispensables. La richesse se trouve dépréciée, et la pauvreté satisfaite, la gueuserie insouciante, devient le séduisant idéal.\par
Est-il bien nécessaire que telles fonctions humaines, auxquelles tout le monde se croit apte, soient mieux rémunérées que les autres, et qu’on dépense pour s’y élever toutes ses forces et toute son énergie ? Sans chercher plus loin, la phrase si souvent employée « Ah ! si j’étais le Ministre, si j’étais le ..., ça ne se passerait pas ainsi ! » exprime déjà la conviction qu’on se sent capable de jouer le rôle d’un de ces dignes personnages ; on sent très bien qu’il n’est pas besoin pour cela d’une personnalité exceptionnelle, mais qu’il suffit d’un degré de culture accessible en somme sinon à tout le monde, du moins au grand nombre ; pour toutes ces choses, un homme ordinaire suffit.\par
En admettant même que, si l’\emph{ordre} est essentiel à l’Etat, la nécessité d’une \emph{subordination hiérarchique }ne lui est pas moins imposée par sa nature, nous remarquerons que ceux qui trônent au sommet de la hiérarchie jouissent de biens et de privilèges démesurés en comparaison de ceux qui occupent les degrés inférieurs de l’échelle sociale.\par
Pourtant ces derniers, inspirés d’abord par la doctrine socialiste, plus tard sans doute aussi par un sentiment égoïste (dont nous donnerons dès à présent une légère teinte à leur langage) s’enhardissent à demander : Qu’est-ce donc qui fait la sécurité de votre propriété, Messieurs les privilégiés ? et ils répondent eux-mêmes : Votre propriété est sûre parce que nous nous abstenons de l’attaquer ! Donc grâce à \emph{notre} protection ! Et que nous donnez-vous en récompense ?  Vous n’avez pour le « menu peuple » que du mépris et des coups de pied, la surveillance de la police, et un catéchisme avec ce principe fondamental : Respecte ce qui n’est pas à toi, ce qui est à autrui ! Respecte les autres, et en particulier tes supérieurs !\par
A cela nous répondons : vous voulez notre respect ? Soit, \emph{achetez-le} nous, voici le prix que nous en demandons. Nous voulons bien vous laisser votre propriété, mais moyennant une compensation suffisante. Qu’est-ce qu’un général fournit en temps de paix, pour compenser les milliers d’écus de son traitement ? Et tel autre, pour ses centaines de mille ou ses millions annuels ? Quelle compensation recevons-nous de vous, pour manger des pommes de terre en vous regardant tranquillement humer vos huîtres ? Achetez-nous seulement ces huîtres au prix où nous devons vous acheter les pommes de terre, et vous pourrez continuer à les manger en paix. Vous imaginez-vous peut-être que les huîtres ne sont pas à nous comme à vous ? Vous crieriez à la violence si vous nous voyiez en remplir notre assiette et nous mettre à les consommer avec vous, — et vous auriez raison. Sans violence, nous ne les aurons pas ; mais vous, ce n’est que parce que vous nous faites violence que vous les avez.\par
Mais va pour les huîtres, et passons à une propriété qui nous touche de plus près (car tout cela n’était que possession), au travail.\par
Nous peinons douze heures par jour à la sueur de notre front, et vous nous donnez pour cela quelques sous. Hé bien ! faites vous donc payer votre travail au même prix. Cela ne vous va pas ! Vous imaginez-vous peut-être que notre travail est ainsi royalement payé, tandis que le vôtre vaut un traitement de vingt mille francs ? Mais si vous ne taxiez pas le vôtre à si haut prix, et si vous nous laissiez tirer un meilleur parti du nôtre, qui vous dit que nous ne serions pas capables de produire des choses plus  importantes que tout ce que vous avez fait jusqu’ici avec vos milliers d’écus ? Si vous ne receviez plus qu’un salaire comme le nôtre, vous deviendriez bientôt plus assidus pour gagner davantage. Si vous exécutez des choses qui nous semblent valoir dix fois, cent fois plus que notre propre travail, qu’à cela ne tienne, vous en recevrez cent fois plus. De notre côté, nous projetons aussi des travaux que vous nous paierez mieux que de notre salaire habituel. Nous serons bientôt d’accord, pourvu qu’il soit bien entendu que personne n’a plus à faire ni à recevoir de \emph{cadeaux}.\par
Qui sait ? nous pourrons même bien aller jusqu’à payer de notre poche un prix équitable aux infirmes, aux malades et aux vieillards, pour que la faim et la misère ne nous les enlèvent pas ; car si nous voulons qu’ils vivent, la satisfaction de ce désir il convient que nous l’— \emph{achetions}. Je dis bien : que nous l’ « achetions », je ne songe nullement à une misérable « aumône ». Leur vie est aussi leur propriété, à ceux-là même qui ne peuvent pas travailler ; et si nous voulons (n’importe pour quelle raison) qu’ils ne nous privent pas de cette vie qui est à eux, il n’y a pas d’autre moyen d’obtenir ce résultat qu’en l’achetant. Il se pourra même, un peu parce que nous aimons à voir autour de nous des visages souriants, que nous voulions leur bien-être.\par
Seulement, plus de cadeaux ! Gardez les vôtres, et n’en attendez plus de nous. Il y a des siècles que nous vous faisons l’aumône avec une bonne volonté — stupide, il y a des siècles que nous gaspillons l’obole du pauvre et que nous rendons au seigneur — ce qui n’est pas au seigneur. C’est fini : déliez les cordons de votre bourse, car dès à présent le prix de notre marchandise est en hausse énorme. Nous ne vous prendrons rien, rien du tout, mais vous paierez mieux ce que vous voudrez avoir.\par
Toi, quelle est ta fortune ? — J’ai un bien de  mille arpents. — Hé bien, moi je suis ton valet de charrue, et dorénavant je ne labourerai plus ton champ qu’au prix d’un écu par jour. — Alors j’en prendrai un autre. — Tu n’en trouveras pas, car nous autres laboureurs nous ne travaillons plus à d’autres conditions, et s’il s’en présente un qui demande moins, qu’il prenne garde à lui !\par
Voici la servante, qui à présent demande tout autant, et tu n’en trouveras plus en dessous de ce prix. — Mais alors je suis ruiné ! — Doucement ! Il te reviendra toujours bien autant qu’à nous ; du reste, s’il en était autrement, nous rabattrions assez pour que tu puisses vivre comme nous. — Mais je suis habitué à vivre mieux ! — Nous le voulons bien, mais cela ne nous regarde pas ; tâche de réduire ta dépense. Faut-il nous louer au rabais pour que tu puisses bien vivre ?\par
Le riche régale toujours le pauvre de ces paroles : — Est-ce que ta misère me regarde ? Tâche de te tirer d’affaire comme tu pourras : c’est ton affaire et non la mienne. — Soit, nous y veillerons, et nous ne laisserons plus les riches accaparer à leur profit les moyens que nous avons de tirer parti de nous-mêmes. — Pourtant, vous autres, gens sans instruction, vous n’avez pas autant de besoins que nous. — Qu’à cela ne tienne, nous prendrons quelque chose de plus pour être à même de nous procurer l’instruction dont nous pourrons avoir besoin. — Et si vous abattez ainsi les riches, qui est-ce donc qui soutiendra encore les arts et les sciences ? — Mais c’est au public de le faire ! Nous nous cotiserons : on fait ainsi de jolies petites sommes. D’ailleurs, nous savons comment vous autres, riches, vous encouragez les arts ; vous n’achetez que des livres insipides, ou des saintes Vierges de la plus lamentable platitude, quand ce n’est pas une paire de jambes de danseuses. — Ah ! la maudite égalité ! — Non, mon bon vieux Monsieur, il ne s’agit pas ici d’égalité. Nous voulons  tout bonnement compter pour ce que nous valons ; si vous valez plus que nous, qu’à cela ne tienne, vous compterez pour plus. Ce que nous voulons, c’est \emph{avoir une valeur}, et nous avons bien l’intention de nous montrer dignes du prix que vous payerez.\par
L’Etat est-il capable d’éveiller chez le salarié une aussi courageuse confiance, et un sentiment aussi vif de son Moi ? L’Etat peut-il faire que l’homme ait conscience de sa valeur ? Il y a plus, oserait il se proposer un tel but, peut-il vouloir que l’individu connaisse sa valeur et en tire le meilleur profit ? La question, on le voit, est double. Voyons en premier lieu ce que l’Etat est capable de réaliser dans cette direction. Il faut, nous l’avons vu, que tous les garçons de charrue marchent la main dans la main, mais aussi il n’y a que cet accord qui puisse donner un résultat : une loi de l’Etat se verrait éludée de mille manières et resterait lettre morte par l’effet de la concurrence. En second lieu, que peut permettre l’Etat ? il lui est impossible de tolérer que les gens subissent une autre contrainte que la sienne ; il ne peut donc tolérer que les garçons de charrue coalisés se fassent justice contre ceux qui voudraient se louer.à trop bas prix. Supposons pourtant que l’Etat ait fait une loi et que les valets de labour soient parfaitement d’accord, l’Etat pourrait-il, alors, consentir ?\par
Dans ce cas isolé, — oui ; mais ce cas isolé est plus que cela, il met en jeu \emph{un principe ;} ce qui est en question ici, c’est le Moi réalisant lui-même sa valeur, et par conséquent s’affirmant en face de l’Etat. Jusque-là les Communistes étaient d’accord avec nous. Mais la mise en valeur de soi-même est nécessairement en contradiction non seulement avec l’Etat mais encore avec la Société ; elle vise bien au delà du commun et du communiste, — par égoïsme.\par
Le Communisme fait du principe de la bourgeoisie, que tout homme est possesseur (« propriétaire ») une vérité indiscutable, une réalité, en mettant fin  au souci d’acquérir et en faisant que chacun ait ce dont il a besoin. C’est la puissance de travail de chacun qui forme sa richesse, et s’il n’en fait pas usage c’est sa faute : C’en est fait des compétitions infatigables, et nulle concurrence ne demeure plus, comme c’était trop souvent le cas jusqu’à aujourd’hui, stérile, attendu que tout effort de travail a pour effet de procurer à celui qui le fait le nécessaire. A présent seulement on \emph{possède réellement :} ce que quelqu’un possède en puissance dans sa capacité de travail il ne peut plus le perdre, comme, sous le régime de la concurrence, cela menaçait à chaque instant de lui échapper. On est possesseur d’une façon assurée, et \emph{sans souci}. Et on l’est précisément parce qu’on ne cherche plus sa richesse dans une marchandise, mais dans sa puissance de travail, c’est-à-dire parce qu’on est un \emph{gueux}, un homme dont la fortune n’est qu’idéale. Quant à Moi, je ne puis me contenter de la maigre pitance que me rapporterait mon labeur, parce que ma richesse ne consiste pas seulement dans mon travail.\par
Par le travail, je puis arriver par exemple à m’acquitter des fonctions d’un président ou d’un ministre ; ces emplois n’exigent que l’instruction moyenne, c’est-à-dire accessible à tout le monde (car l’instruction moyenne ne signifie pas seulement l’instruction que tout le monde possède, mais celle par exemple du médecin, du militaire, du philosophe, que tout le monde peut acquérir, et qu’un « homme cultivé » ne croit pas au dessus de ses forces), ou, en somme, qu’un savoir-faire dont tout le monde est capable.\par
Mais s’il est vrai que ces fonctions peuvent être exercées par tout homme quel qu’il soit, ce n’est pourtant que la force unique de l’individu, propre exclusivement à l’individu, qui leur donne en quelque sorte une vie et une signification. S’il ne remplit pas ses fonctions comme un « homme ordinaire », mais s’il y dépense tout le trésor de son unicité, il  n’est pas payé par le fait qu’il touche le traitement ordinaire de l’employé ou du ministre. S’il vous a pleinement satisfait, et si vous voulez continuer à bénéficier non seulement de son travail de fonctionnaire, mais en plus de sa précieuse puissance individuelle, vous ne le paierez pas seulement comme un homme ordinaire qui ne fait que de la besogne humaine, mais encore comme un producteur d’unique. Faites payer de même votre propre travail.\par
On ne peut appliquer à l’œuvre de mon unicité un prix général comme à ce que je fais en tant qu’homme. Ce n’est qu’en cette dernière qualité que je puis travailler à forfait.\par
Fixez donc, je le veux bien, une taxe générale pour les travaux humains, mais que le contrat n’ait pas pour effet d’aliéner votre unicité.\par
Tes besoins humains ou généraux peuvent être satisfaits par la Société ; mais c’est à Toi à chercher la satisfaction de tes besoins uniques. La Société ne peut ni te procurer une amitié ou le service d’un ami, ni même t’assurer les bons offices d’un individu. Et pourtant tu auras à chaque instant besoin de services de ce genre, dans les circonstances les plus insignifiantes il te faudra quelqu’un pour t’assister. Ne compte pas pour cela sur la Société, mais fais en sorte d’avoir de quoi — acheter la satisfaction de tes désirs.\par
Faut-il que l’usage de l’argent soit conservé entre égoïstes ? A l’ancienne monnaie s’attache la tare de la possession héréditaire. Ne la recevez plus en paiement, et elle est ruinée ; ne faites plus rien pour cet argent, et toute sa puissance s’évanouit. Biffez le mot \emph{héritage}, et le sceau du magistrat est sans vertu. A présent, tout est héritage, que l’héritier soit ou ne soit pas encore en possession. Si tout cela est à vous, pourquoi le laisser mettre sous scellés, pourquoi vous inquiéter des sceaux ?\par
Mais à quoi bon créer un nouvel instrument ? Anéantissez-vous donc la marchandise parce que vous  lui ôtez le cachet de l’hérédité ? Considérez la monnaie comme une marchandise ; à ce titre, elle est un précieux moyen, une \emph{richesse}. Car elle empêche l’ankylose de la richesse, la maintient en circulation et en opère l’échange. Si vous connaissez un meilleur instrument d’échange, adoptez-le, je le veux bien ; mais ce sera encore toujours l’ « argent » sous une nouvelle forme. Ce n’est pas l’argent qui vous fait du mal, mais bien votre impuissance à le prendre. Mettez en jeu tous vos moyens, faites tous vos efforts, et l’argent ne vous manquera pas : ce sera un argent à vous, une monnaie à \emph{votre} effigie. Mais travailler, ce n’est pas cela que j’appelle « mettre en jeu tous vos moyens ». Ceux qui se contentent de « chercher du travail », d’avoir « la volonté de bien travailler », ceux-là sont condamnés fatalement, et par leur faute, à devenir des — sans travail.\par
C’est de l’argent que dépend le bonheur et le malheur. Ce qui en fait une puissance dans la période bourgeoise, c’est qu’on ne fait que le courtiser comme une jeune fille, mais que personne ne l’épouse. Tous les procédés romanesques et chevaleresques pour s’attacher une femme aimée se retrouvent dans la concurrence. Et c’est par un enlèvement que les hardis chevaliers (d’industrie) conquièrent l’argent, objet de leur ardente passion.\par
Celui que la chance favorise emmène chez lui la fiancée. Le gueux introduit la jeune fille dans son ménage, qui est « la Société », et elle disparaît. Dans sa maison, elle n’est plus la fiancée, mais la femme, et avec sa virginité s’en va son nom de famille : la jeune fille s’appelait « Argent », elle s’appelle aujourd’hui « Travail », parce que « Travail » est le nom du mari. Elle est la propriété du mari. Pour en finir avec cette comparaison, l’enfant de Travail et d’Argent est de nouveau une fille, et de nouveau célibataire, c’est-à-dire Argent, mais avec une filiation certaine : elle est issue de Travail, son père. Les  traits du visage, l’ « effigie », présentent un caractère nouveau.\par
Revenons-en enfin encore une fois à la concurrence. La concurrence doit précisément son existence à ce que personne ne s’occupe de \emph{ses affaires} et ne songe à s’entendre avec les autres à leur sujet. Le pain, par exemple, est un objet de première nécessité pour tous les habitants d’une ville. Donc, rien de plus naturel que de s’accorder pour établir une boulangerie publique. Au lieu de cela, on abandonne cette indispensable fourniture à des boulangers qui se font concurrence. Et ainsi de la viande aux bouchers, du vin aux marchands de vin, etc.\par
Abolir le régime de la concurrence ne veut pas dire favoriser le régime de la corporation. Voici la différence : dans la \emph{corporation}, faire le pain, etc., est l’affaire des compagnons ; sous la \emph{concurrence}, c’est l’affaire de ceux à qui il plait de concourir ; dans l’\emph{association}, c’est l’affaire de ceux qui ont besoin de pain, par conséquent la mienne, la vôtre : ce n’est affaire ni des compagnons, ni des boulangers patentés, mais bien celle des \emph{associés}.\par
Si je ne m’inquiète pas de mes affaires, il faut bien que je me contente de ce qu’il plaît à d’autres de me donner. Or, avoir du pain est mon affaire, j’en veux, je ne puis m’en passer ; et pourtant on s’en remet aux boulangers, sans autre espoir que d’obtenir de leur discorde, de leur jalousie, de leur rivalité, en un mot de leur concurrence, un avantage sur lequel on ne pouvait pas compter avec les membres des corporations, qui étaient entièrement et exclusivement en possession du monopole de la boulangerie. Ce dont chacun a besoin, chacun aussi devrait participer à sa production ou à sa fabrication : c’est son affaire, sa propriété, et non la propriété des membres de telle corporation ou de tel patron patenté.\par
Jetons encore un regard en arrière. Le monde appartient aux enfants de ce monde, aux enfants des  hommes. Il n’est plus le monde de Dieu, mais le monde des hommes. Tout ce que chaque homme peut s’en procurer, il peut le nommer \emph{sien ;} seulement, le véritable Homme, l’Etat, la Société humaine ou l’Humanité veilleront à ce que chacun ne fasse sien que ce qu’il s’approprie en tant que Homme, c’est-à-aire d’une manière humaine. L’appropriation non humaine n’est pas autorisée par l’Homme ; elle est « criminelle », tandis que, au contraire, l’appropriation humaine est « juste », et se fait par une « voie légale ».\par
C’est ainsi qu’on parle depuis la Révolution.\par
Mais nulle chose n’est en elle-même ma propriété, vu qu’une chose a une existence indépendante de moi ; seule ma puissance est à moi. Cet arbre n’est pas à moi ; ce qui est à moi, c’est mon pouvoir sur lui, l’usage que j’en fais. Et, comment exprime-t-on ce pouvoir ? On dit : j’ai un \emph{droit} sur cet arbre ; ou bien : il est ma \emph{légitime} propriété Or, si je l’ai acquis, c’est par la force. On oublie que la propriété ne dure qu’aussi longtemps que la puissance reste agissante ; ou, plus exactement, on oublie que la puissance n’est pas une entité, mais qu’elle n’a d’existence que comme puissance du Moi, et qu’elle n’existe qu’en Moi, le \emph{puissant}.\par
On élève la puissance, comme d’autres de mes \emph{propriétés} (l’humanité, la majesté, etc.) au rang d’« être pour soi » \emph{(fürsichseiend)}, de sorte qu’elle ne cesse pas d’exister alors qu’elle a longtemps cessé d’être \emph{ma} puissance. Ainsi transformée en fantôme, la puissance est le — \emph{Droit}. Cette puissance immortalisée ne s’éteint pas même à ma mort, elle est transmissible (« héréditaire »).\par
Il suit de là qu’en réalité les choses appartiennent non pas à Moi, mais au Droit.\par
Tout cela n’est qu’une vaine apparence pour un autre motif encore : la puissance de l’individu ne devient permanente et ne devient un droit, que  pour autant que d’autres individus conjuguent leur puissance à la sienne. L’illusion consiste à croire qu’ils ne peuvent plus retirer leur puissance à ceux auxquels ils l’ont accordée. Ici reparaît le même phénomène que tantôt, le divorce de la puissance et du moi : je ne puis pas reprendre au possesseur la part de puissance qui lui vient de moi. On a donné « pleins pouvoirs », on s’est dessaisi du pouvoir, on a renoncé à celui de prendre un meilleur parti.\par
Le propriétaire peut renoncer à sa puissance et à son droit sur une chose, en en faisant don, en la dissipant, etc. Et nous, nous ne pourrions pas également abandonner la puissance que nous lui avons prêtée ?\par
L’homme selon le droit, l’ « honnête homme », ne demande pas à faire sien ce qui n’est pas à lui « de droit » ou ce à quoi il n’a pas droit ; il ne revendique que sa « propriété légitime ». Qui donc sera juge et fixera les limites de son droit ? Finalement, ce doit être l’Homme, car c’est de lui qu’on tient les droits de l’homme. Par conséquent on peut dire avec Térence, mais dans un sens infiniment plus large que Térence : « \emph{humani nihil a me alienum puto} » c’est-à-dire \emph{l’humain est ma propriété.} De quelque manière qu’on s’y prenne, sur ce terrain on aura inévitablement un juge, et de notre temps les divers juges que l’on s’était donnés ont fini par s’incarner en deux personnes mortellement ennemies : le Dieu, et l’Homme. Les uns se réclament du droit divin, les autres du droit humain ou des droits de l’homme.\par
Ce qui est clair, c’est que dans les deux cas l’individu ne crée pas lui-même son droit.\par
Trouvez-moi donc aujourd’hui une seule action qui n’offense pas un droit ! A chaque instant les droits de l’homme sont foulés aux pieds par les uns, tandis que les autres ne peuvent pas ouvrir la bouche sans blasphémer contre le droit divin. Faites l’aumône, et vous outragez un droit de l’homme, puisque le rapport de mendiant à bienfaiteur n’est  pas humain ; exprimez un doute, vous péchez contre un droit divin. Mangez votre pain sec avec contentement, votre résignation est une offense aux droits de l’homme ; mangez-le en mécontents, et vos murmures sont une insulte au droit divin. Il n’est pas un de vous qui ne commette à chaque instant un crime : tous vos discours sont des crimes, et toute entrave à votre liberté de discourir n’est pas moins un crime. Vous êtes tous des criminels.\par
Cependant, vous ne l’êtes que parce que vous vous tenez tous sur le terrain du droit, c’est-à-dire parce que vous ne savez pas que vous êtes criminels et ne savez pas vous en féliciter.\par
La propriété inviolable ou \emph{sacrée} a pris naissance sur ce même terrain ; elle est la fille spirituelle du Droit. Le chien qui voit un os en la puissance d’un autre n’y renonce que s’il se sent trop faible. Mais l’homme respecte le \emph{droit} de l’autre à son os. Ceci est considéré comme humain, cela comme brutal ou « égoïste ». Et partout, comme dans ce cas ci, ce qui est « humain » c’est de voir en tout quelque chose de spirituel (ici, le droit), c’est-à-dire de faire de toute chose un fantôme que l’on peut bien chasser dès qu’il se montre mais qu’on ne peut pas tuer. Ce qui est humain, c’est de voir dans tout objet particulier non pas quelque chose de particulier mais quelque chose de général.\par
Je ne dois plus à la nature, comme telle, aucun respect ; je sais que j’ai à son égard tous les droits. Mais je suis tenu de respecter dans l’arbre du jardin que voilà sa qualité d’objet \emph{étranger} (à un point de vue plus étroit on dit : de respecter la « propriété »), et il ne m’est pas permis d’y toucher. Et cela ne pourra changer que quand je ne verrai pas dans le fait de laisser cet arbre à autrui, autre chose que dans le fait de lui abandonner, par exemple, mon bâton, c’est-à-dire quand j’aurai cessé de considérer cet arbre comme quelque chose d’étranger \emph{a priori},  de sacré. Moi, au contraire, je ne me fais pas un crime de l’abattre si cela me plait ; il reste ma propriété quelque long qu’ait pu être le temps pendant lequel je l’ai abandonné à d’autres : il était et il reste \emph{à moi}. Je ne vois pas plus la qualité d’objet étranger dans la richesse du banquier, que Napoléon dans les provinces des rois. Nous ne nous faisons aucun scrupule d’en tenter la « conquête », et nous cherchons par tous les moyens à y arriver. Nous en exorcisons donc l’\emph{esprit d’étrangeté} qui nous avait fait d’abord reculer d’effroi devant elle.\par
Mais il est indispensable pour cela que je ne prétende à rien en qualité d’\emph{Homme,} mais seulement en qualité de \emph{Moi}, de ce Moi que je suis ; je ne prétendrai par conséquent à rien d’humain mais seulement à ce qui est mien, ou, en d’autres termes, à rien de ce qui me revient en tant qu’homme mais à — ce que je veux, et parce que je le veux.\par
Donc, une chose ne sera la juste et légitime propriété d’un autre que quand il sera juste \emph{pour toi }qu’elle soit la propriété de cet autre. Dès qu’il ne te convient plus qu’il en soit ainsi, la légitimité disparaît à tes yeux, et il ne te reste plus qu’à rire du droit absolu du propriétaire.\par
Outre la propriété au sens restreint dont nous nous sommes entretenus jusqu’à présent, il en est une autre qui s’impose à notre vénération et contre laquelle il nous est encore bien moins permis de « pécher ». Cette propriété est constituée par les biens spirituels et le « sanctuaire de la conscience ». Ce qu’un homme tient pour sacré, il n’est pas permis à un autre de s’en moquer. Si faux que soit l’objet de sa foi, et si désireux qu’on soit de l’en détacher pour le ramener « tout doucement et pour son bien » au culte d’un sacro-saint plus authentique, sa foi du moins, quelque discutable qu’en soit l’objet, est sacrée et doit toujours être respectée ; quelqu’absurde que soit l’idole, la faculté de vénération de celui qui la  tient pour sacrée est elle-même sacrée et on doit s’incliner devant elle.\par
Dans des temps plus barbares que les nôtres, on avait coutume d’exiger de chacun une certaine foi et une dévotion à un certain objet sacré ; on n’y allait pas de main morte contre les dissidents. Mais la « liberté de conscience » se répandant de plus en plus, le « Dieu jaloux » et « seul Seigneur » s’est, depuis, peu à peu transformé en ce qu’on désigne sous le nom plus vague d’ « être suprême » ; la tolérance humaine se déclare satisfaite du moment que chacun révère un « objet sacré » quel qu’il soit.\par
Ramené à son expression la plus humaine, cet objet sacré est « l’Homme lui-même » et « l’humain ». Car c’est une illusion de croire que l’humain est tout à fait nôtre et tout à fait exempt de cette teinte de surnaturel qui s’attache au divin, et de s’imaginer que dire l’Homme c’est dire Moi ou Toi. Et c’est cette erreur qui peut conduire à l’orgueilleuse illusion qu’on ne voit plus nulle part rien de « sacré », que partout nous nous sentons chez nous et délivrés de l’obsession de la sainteté, du frisson de la terreur sacrée, Mais le ravissement d’ « avoir enfin trouvé l’Homme » a empêché d’entendre le cri de douleur de l’égoïsme ; c’est ainsi qu’on a pris pour notre vrai moi un fantôme devenu si bon homme.\par
Mais « le Sacré s’appelle Humain » dit Goethe, et l’humain n’est que le sacré à sa plus haute puissance.\par
L’égoïste s’exprime tout autrement. C’est justement parce que tu tiens quelque chose pour sacré que je te trouve ridicule, et en admettant même que je veuille tout respecter en toi, c’est précisément ton sanctuaire intérieur que je ne respecterais pas.\par
A ces manières de voir si opposées correspondent naturellement des conduites différentes envers les biens spirituels : l’égoïste les attaque ; le religieux (c’est-à-dire celui qui, au-dessus de lui, place son « essence ») doit, pour être conséquent, les défendre.  Quels biens spirituels faut-il défendre et lesquels doit-on laisser sans protection ? Cela dépend entièrement de l’idée qu’on se fait de l’ « être suprême » ; celui qui craint Dieu, par exemple, a plus à défendre que celui qui craint l’Homme, que le Libéral.\par
Quand on nous offense dans nos biens spirituels, ce n’est plus comme lorsqu’on nous lésait dans nos biens matériels : ici, l’offense est spirituelle, le péché commis contre les biens spirituels consiste à les \emph{profaner} directement, tandis qu’on ne faisait que détourner ou éloigner les biens matériels. Ici les biens eux-mêmes subissent une dépréciation, une déchéance : ils ne sont pas simplement soustraits, leur caractère sacré est directement mis en jeu. On désigne sous le nom d’« impiété » ou de « sacrilège » toutes les infractions qui peuvent être commises contre les biens spirituels, c’est-à-dire envers ce que nous tenons pour sacré ; et la raillerie, l’insulte, le mépris, le scepticisme, etc., ne sont que des nuances différentes de la \emph{criminelle impiété}.\par

\asterism

\noindent Sans nous occuper des multiples façons dont le sacrilège peut se commettre, nous ne rappellerons ici que celle qui met en danger la sainteté par le fait d’une \emph{presse trop libre.}\par
Tant qu’on exigera encore du respect pour le moindre être spirituel, la parole et la presse devront être enchaînées au nom de cet être ; car l’égoïste pourrait par ses manifestations l’ « offenser » et c’est cette offense qu’on doit réprimer à l’aide de « pénalités convenables », à moins qu’on ne préfère recourir au moyen plus judicieux que fournit la puissance préventive de la police, c’est-à-dire à la censure.\par
Combien de gens nous entendons tous les jours appeler à grands cris la liberté de la presse ! Or, de  quoi la presse doit-elle être libérée ? Sans doute, d’une dépendance, d’une sujétion, d’un asservissement ! Mais c’est affaire à chacun de s’affranchir de tout cela ; on peut affirmer avec certitude que si vous avez secoué le joug des vieilles habitudes de domesticité, ce que vous écrivez et publiez vous appartient \emph{en propre}, au lieu d’avoir été conçu et formulé \emph{au service} d’un pouvoir quelconque ; mais qu’est-ce qu’un fidèle chrétien peut bien dire ou imprimer qui soit plus indépendant de la croyance chrétienne qu’il ne l’est lui-même ? S’il est des choses que je ne puis ou n’ose écrire, le premier coupable ne peut être que moi-même. — Et, quoique ceci paraisse s’éloigner du sujet, en voici pourtant l’explication : Par une loi sur la presse, je trace ou je permets qu’on trace autour de mes publications une limite au delà de laquelle commencent le délit et la répression. C’est moi-même qui restreins ma liberté.\par
Pour que la presse fût libre, il serait indispensable qu’aucune contrainte ne pût lui être imposée au \emph{nom d’une loi}. Et pour en arriver là, il faudrait que moi-même je me fusse affranchi de l’obéissance à la loi.\par
En vérité, la liberté absolue de la presse est une chimère, comme toute liberté absolue. La presse peut être libre de bien des choses, mais elle ne le sera jamais que de ce dont je serai moi-même libre. Affranchissons-nous de tout ce qui est sacré, soyons \emph{sans foi} et \emph{sans loi} et nos discours le seront aussi.\par
Nous ne pouvons pas plus affranchir nos écrits de toute contrainte que nous pouvons être nous-mêmes affranchis de tout. Mais nous pouvons les faire aussi libres que nous le sommes. Il faut pour ce la qu’ils soient notre \emph{propriété}, au lieu d’être, comme ils l’ont été jusqu’ici, au service d’un fantôme.\par
On ne se rend pas bien compte de ce qu’on demande en réclamant la liberté de la presse. Ce que prétendument on désire, c’est que l’Etat rende la presse libre ; mais ce qu’on veut en réalité et sans  s’en douter, c’est que la presse soit affranchie de l’Etat, ou n’ait plus à compter avec lui. Le vœu conscient est une \emph{pétition} que l’on adresse à l’Etat, la tendance inconsciente est une \emph{révolte} contre l’Etat. L’humble supplique comme la ferme revendication du droit à la liberté de la presse supposent que l’Etat est le \emph{dispensateur}, dont on ne peut espérer qu’un \emph{don}, une concession, un octroi. Il se pourrait qu’un Etat fût assez fou pour accorder le cadeau demandé, mais il y a tout à parier que ceux qui le recevraient ne sauraient pas s’en servir, aussi longtemps qu’ils considéreraient l’Etat comme une vérité : ils se garderaient bien d’offenser cette « chose sacrée » et appelleraient sur celui qui se le permettrait les sévérités d’une loi sur la presse.\par
En un mot, il est impossible que la presse soit libre de ce dont je ne suis pas libre moi-même.\par
Ce que j’en dis va peut-être me faire passer pour un adversaire de la liberté de la presse ? Loin de là ! J’affirme seulement qu’on ne l’obtiendra jamais tant qu’on ne voudra qu’elle, la liberté de la presse, c’est-à-dire tant qu’on n’aura en vue qu’une permission limitée. Mendiez-la tant que vous voudrez, cette permission : vous l’attendrez éternellement, car il n’y a personne au monde qui puisse vous la donner. Tant que vous voudrez voir « légitimer, autoriser, justifier » par une permission (c’est-à dire par la liberté de la presse), l’usage que vous faites de la presse, vous vivrez dans de vaines espérances et de vaines récriminations.\par
« Absurdité ! Vous qui nourrissez des pensées comme on en voit dans votre livre, vous ne parviendrez à leur donner de publicité que grâce à un heureux hasard ou à force d’artifices. Et c’est vous qui voulez vous opposer à ce qu’on harcèle, qu’on importune l’Etat jusqu’à ce qu’il accorde enfin la liberté d’imprimer ? »\par
Il se pourrait qu’un auteur à qui on tiendrait ce  langage répondît — car jusqu’où ne va pas l’insolence de ces gens ? — de la manière suivante :\par
— Réfléchissez bien à ce que vous dites ! Que fais-je donc en vue de me procurer pour mon livre la liberté de la presse ? Est-ce que je demande une permission ? Ne me voit-on pas, au contraire, sans me soucier de la légalité, guetter une occasion favorable, et la saisir sans aucun égard pour l’Etat et ses désirs ?\par
« Oui ! je trompe — puisqu’il faut que le mot terrible soit prononcé — je trompe l’Etat.\par
« Et vous, sans vous en douter, vous en faites autant. Vous lui persuadez du haut de vos tribunes qu’il doit faire le sacrifice de sa sainteté et de son invulnérabilité, qu’il doit s’exposer aux attaques des gens qui écrivent, sans avoir pour cela de danger à redouter. Hé bien ! vous l’abusez ; car c’en sera fait de son existence aussitôt qu’il aura perdu son inviolabilité.\par
« Il est vrai qu’à \emph{vous} il pourrait bien concéder la liberté d’écrire comme l’a fait l’Angleterre : Vous êtes les \emph{dévôts de l’Etat}, vous êtes incapables d’écrire contre lui, quoi que vous y puissiez voir d’abus à réformer et de « défectuosités à amender ». Mais quoi ? Si des adversaires de l’Etat profitaient de la liberté de la parole pour se déchaîner contre l’Eglise, l’Etat, les Mœurs, et pour assaillir le « sacro-saint » d’implacables arguments ? Vous seriez alors les premiers à trembler et à appeler à la vie des \emph{lois de septembre}. Vous vous repentiriez, trop tard, de la sottise qui vous aurait poussés à enjôler et à aveugler l’Etat ou le Gouvernement.\par
« Mais ma conduite à moi ne prouve que deux choses. D’abord ceci, que la liberté de la presse est toujours inséparable de « circonstances favorables » et ne peut, par conséquent jamais être une liberté absolue ; en second lieu ceci, que quiconque veut en jouir doit rechercher et au besoin créer l’occasion favorable, en faisant prévaloir contre l’Etat son propre  intérêt et en se mettant, soi et sa volonté, au-dessus de l’Etat et de toute « puissance supérieure \emph{ ».}\par
« Ce n’est pas dans l’Etat, ce n’est que contre l’Etat que la liberté de la presse peut être conquise. Et si cette liberté règne jamais, ce n’est pas à la suite d’une \emph{prière}, mais bien comme l’œuvre d’une \emph{révolte }qu’on l’aura obtenue. Toute demande, toute proposition de liberté de la presse est déjà une révolte, consciente ou inconsciente ; il n’y a que l’insuffisance philistine qui ne veuille ni ne puisse se l’avouer, tant que le résultat ne le lui aura pas, à sa grande terreur, montré d’une façon claire et évidente. La liberté de la presse obtenue à force de prières a d’abord un air amical et bienveillant, il est bien loin de ses intentions de laisser jamais surgir la \emph{licence de la presse ;} mais peu à peu son cœur s’endurcit, et elle en arrive insensiblement à conclure qu’en définitive une liberté n’est pas une liberté tant qu’elle est au \emph{service} de l’Etat, de la morale ou de la loi. Liberté vis-à-vis de la contrainte de la censure, elle n’est pas liberté vis-à-vis de la contrainte de la loi.\par
« La presse, une fois saisie du désir de la liberté, veut devenir toujours plus libre jusqu’à ce qu’enfin l’écrivain se dise : Puisque je ne suis tout à fait libre que quand je n’ai aucun ménagement à garder, mes écrits ne sont libres que quand ils sont \emph{à moi}, quand ils ne peuvent m’être dictés par aucune puissance ou autorité, par aucune foi, par aucun respect ; ce n’est pas « libre » que la presse doit être — c’est trop peu — elle doit être à Moi ! l’\emph{individualité,} la \emph{propriété de la presse}, voilà ce que je veux m’assurer.\par
« Une liberté de la presse n’est qu’un permis d’imprimer que me délivre l’Etat, et l’Etat ne permettra jamais, et il ne peut jamais librement permettre que j’emploie la presse à l’anéantir.\par
« Exprimons-nous donc plutôt de la maniéré suivante, pour éviter ce que le terme « liberté de la presse » a pu laisser jusqu’ici de vague dans nos paroles :  La liberté de la presse que revendiquent si haut les Libéraux est, sans aucun doute, possible dans l’Etat ; elle n’est même possible que dans l’Etat, attendu qu’elle est une permission, et que, par conséquent, cet \emph{imprimatur} doit être accordé par quelqu’un, qui, dans le cas présent, est l’Etat. Mais, en tant que permission, elle est limitée par cet Etat lui-même qui naturellement n’est pas tenu de tolérer plus qu’il n’est compatible avec sa conservation et sa prospérité. Il trace à la liberté de la presse une limite qui est la \emph{loi} de son existence et de son extension. Un Etat peut être plus tolérant qu’un autre, mais il n’y a là qu’une différence de quantité ; c’est pourtant cette différence qui tient tant à cœur aux politiciens libéraux : en Allemagne, par exemple, ils ne demandent qu’ « une tolérance plus large, plus étendue, de la parole libre ».\par
« La liberté de la presse qu’on sollicite est une liberté qui doit appartenir au Peuple, et tant que le Peuple (l’Etat) ne la possède pas, je ne puis en faire aucun usage. Mais si on se place au point de vue de la propriété de la presse, les choses se présentent sous un jour différent. Bien que mon Peuple soit privé de la liberté de la presse, je me procure par ruse ou par violence le moyen d’imprimer ; je ne demande la permission d’imprimer qu’à — Moi et à ma force.\par
« Dès que la presse est à Moi, il ne me faut pas plus d’autorisation de l’Etat pour en user qu’il ne m’en faut pour me moucher. Et la presse est \emph{ma propriété }à partir du moment où, pour Moi, il n’y a plus rien au-dessus de Moi, car dès lors plus d’Etat, plus d’Eglise, plus de Peuple, plus de Société : tous ne devaient leur existence qu’à mon mépris de moi-même, et tous s’évanouissent dès que l’infirmité de mon orgueil disparaît ; ils ne sont, qu’à la condition d’être au-dessus de moi, ils n’existent que s’ils sont des \emph{puissances}. — A moins qu’on ne puisse se figurer un Etat dont les sujets ne feraient aucun cas ? Ce serait  un rêve, une illusion, tout comme l’ « unité de l’Allemagne ».\par
« La presse est à Moi dès que je m’appartiens, dès que je suis mon propriétaire : Le monde est à l’égoïste, parce que l’égoïste n’appartient à aucune puissance du monde.\par
« Cela étant, il se peut très bien que la presse, quoique \emph{mienne}, soit encore très peu libre, comme c’est le cas en ce moment. Mais le monde est grand, et on se tire d’affaire comme on peut. Si je consentais à renoncer à la \emph{propriété} de ma presse, j’arriverais facilement à faire imprimer partout tout ce que ma plume produit. Mais comme je veux affirmer ma propriété, il faut bien que j’en vienne aux mains avec mes ennemis,\par
— N’accepterais-tu pas leur permission si on te l’accordait ?\par
— Oui certes, et avec plaisir ; car leur permission me prouverait que je les ai aveuglés et que je les mène à l’abîme. Ce n’est pas leur permission que je veux, mais leur aveuglement et leur défaite. Si je la sollicite, cette permission, ce n’est pas parce que j’espère, comme les politiciens libéraux, qu’eux et moi pourrions vivre en paix côte à côte, et même nous soutenir, nous entr’aider réciproquement. Non. Si je la sollicite, c’est pour m’en faire une arme contre eux, c’est pour faire disparaître ceux-là mêmes qui ne l’auront accordée.\par
« J’agis consciemment comme un ennemi, je prends mes avantages et je \emph{profite} de leur imprévoyance.\par
« La presse n’est à moi que si j’en use sans reconnaître absolument aucun juge en dehors de moi-même, c’est-à-dire que si je ne suis plus déterminé ni par la religion, ni par la morale, ni par le respect des lois de l’État, etc., mais par Moi seul et par mon égoïsme ! »\par
Qu’avez-vous à répliquer à celui qui vous fait une réponse si insolente ? Mais peut-être la question sera-t-elle mieux posée sous la forme suivante : A qui est  la presse ? Au Peuple (l’Etat), ou à Moi ? Les politiciens se proposent simplement de soustraire la presse aux entreprises personnelles et arbitraires des gouvernants ; ils ne réfléchissent pas que, pour être vraiment ouverte à tout le monde, elle devrait être affranchie des lois, c’est-à-dire indépendante de la volonté du Peuple (de la volonté de l’Etat).\par
Mais une fois devenue la propriété du Peuple, la presse est encore bien loin d’être ma propriété ; sa liberté conserve par rapport à moi le sens de \emph{permission}. C’est au Peuple qu’il appartient de juger mes idées, c’est à lui que j’en dois compte, c’est envers lui que j’en suis responsable. Or les jurés aussi, quand on attaque leurs idées fixes, ont le cœur et la tête durs, tout comme les plus farouches despotes et les esclaves qu’ils emploient.\par
E. Bauer, dans ses « Revendications libérales », soutient que la liberté de la presse est impossible dans les Etats absolus ou constitutionnels, mais qu’elle a sa place tout indiquée dans les « Etats libres ». « Dans ceux-ci, dit-il, l’individu a le droit d’exprimer tout ce qu’il pense, et ce droit ne lui est pas contesté parce qu’il n’est plus seulement un individu isolé mais bien un membre solidaire d’un tout réel et intelligent\footnote{ \noindent II, p. 91 sqs.
 }. » Ce n’est donc pas l’individu mais le membre qui jouit de la liberté de la presse. Mais si, pour jouir de la liberté de la presse, il faut que l’individu ait prouvé sa fidélité à la communauté, qui est le Peuple, cette liberté ne lui appartient pas en vertu de sa \emph{propre énergie :} elle est une \emph{liberté du peuple}, une liberté qui ne lui est accordée à lui, individu, qu’en raison de sa fidélité et de sa qualité de sociétaire.\par
Au contraire, ce n’est que comme individu que chacun peut être libre d’exprimer sa pensée. Mais il n’en a pas le « droit », et cette liberté n’est pas « son droit sacré » ; il n’en a que le \emph{pouvoir}, pouvoir qui suffit  d’ailleurs pour le mettre en possession. Pour posséder la liberté de la presse, je n’ai pas besoin de concession, je n’ai pas besoin du consentement du peuple, je n’ai pas besoin d’en avoir le « droit » ni d’y être « autorisé ». Il en est de la liberté de la presse comme de toute autre liberté, je dois la \emph{prendre} moi-même ; le Peuple, quoique « seul juge », ne peut me la \emph{donner}. Il peut applaudir à la liberté dont je m’empare ou il peut se mettre en garde et se défendre contre elle ; mais me la donner, me l’accorder, me l’octroyer lui est impossible. J’en use \emph{malgré} le Peuple, en ma seule qualité d’individu, c’est-à-dire que je lutte pour elle contre le Peuple, — mon ennemi ; je ne l’obtiens que si je la conquiers réellement, si je la \emph{prends}. Et si je la prends, c’est qu’elle est ma propriété.\par
Sander, que combat E. Bauer, considère la liberté de la presse comme « le droit et la liberté du citoyen dans l’Etat ». Bauer ne dit rien d’autre. Pour lui aussi elle n’est que le droit du \emph{citoyen} libre.\par
On réclame encore la liberté de la presse comme un « droit commun à tous les hommes ». A cela il a été objecté que tous les hommes ne savent pas en faire bon usage, attendu que tous ne sont pas vraiment hommes. A l’Homme, comme tel, jamais un gouvernement ne l’a refusée. Seulement, l’Homme n’écrit pas, pour l’excellente raison qu’il est un fantôme. Cette liberté, les gouvernements ne l’ont jamais refusée qu’à des \emph{individus}, pour l’accorder à d’autres individus par exemple à leurs organes. Donc, si on veut l’obtenir pour tout le monde, il faut précisément affirmer qu’elle appartient à l’individu, à Moi, et non pas à l’Homme ou à l’individu en tant que Homme. Dans tous les cas, ce qui n’est pas homme (l’animal par exemple) ne peut en faire usage. Le Gouvernement français, par exemple, ne conteste pas que la liberté de la presse soit un droit de l’Homme. Il exige seulement de l’individu un cautionnement établissant qu’il est vraiment Homme ; car ce n’est pas à l’individu,  c’est à l’Homme qu’il accorde la liberté de la presse.\par
C’est justement sous le prétexte que cela \emph{n’est pas humain} qu’on m’a enlevé ce qui est à Moi ! Et on m’a laissé ce qui est à l’Homme.\par
La liberté de la presse ne peut produire qu’une presse \emph{responsable.} Une presse \emph{irresponsable} ne peut naître que de la propriété de la presse.\par

\asterism

\noindent Les relations des hommes entre eux sont régies, pour tous ceux qui vivent religieusement, par une loi formelle dont on peut bien parfois, au risque de pécher, négliger l’observation, mais dont on ne s’aviserait jamais de nier la valeur absolue. C’est la loi de l’Amour, loi avec laquelle ceux-là même qui semblent combattre son principe et qui haïssent son nom n’ont pas encore su rompre ; car à eux aussi il reste de l’amour, leur amour est même plus profond et plus épuré : ils aiment l’Homme et l’Humanité.\par
Si nous tâchons de formuler le sens de cette loi, nous dirons à peu près : Chaque homme doit tenir quelque chose pour plus que lui-même. Tu dois oublier ton « intérêt privé » dès qu’il s’agit du bonheur des autres, du bien de la Patrie ou de la Société, du bien public, du bien de l’humanité, de la bonne cause, etc ! Patrie, humanité, Société, etc., doivent être pour toi plus que toi-même, et ton « intérêt privé » doit s’effacer devant leur intérêt ; car il ne faut pas être un — égoïste !\par
L’Amour est un commandement religieux d’une grande portée ; il ne se borne pas à l’amour de Dieu et des hommes, mais il préside à tous nos rapports. Quoi que nous fassions, pensions et voulions, toujours l’amour doit faire le fond de nos actions, de nos pensées et de nos désirs. Il nous est bien permis de juger, mais nous ne devons juger qu’avec amour. On peut certainement critiquer la Bible, et même  d’une manière approfondie ; mais le critique doit, avant tout, l’aimer et voir en elle le livre saint. N’est-ce pas comme si on disait : il ne faut pas que sa critique l’anéantisse, il doit la laisser subsister, et subsister en tant que chose sacrée et indestructible.\par
Il en est de même de notre critique des hommes : l’amour doit en rester la tonique invariable. Il est certain que les jugements que nous dicte la haine ne sont pas nos propres jugements, ce sont les jugements de la haine qui nous domine, des « jugements haineux » Mais les jugements dictés par l’amour sont-ils mieux les \emph{nôtres ?} Ce sont les jugements de l’amour qui nous domine, ce sont des jugements « charitables, indulgents », mais ce ne sont pas nos \emph{propres} jugements, ni, par conséquent, réellement, des jugements.\par
Celui qui brûle d’amour pour la justice s’écrie : \emph{fiat justitia, pereat mundus !} Il lui est permis de se demander et d’examiner ce que c’est, à proprement parler, que la justice, ce qu’elle exige et \emph{en quoi} elle consiste, mais non pas \emph{si} elle est quelque chose.\par
Il est bien vrai que « Celui qui demeure dans l’amour, celui-là demeure en Dieu et Dieu en lui » (1\textsuperscript{e} Ep. de Jean, IV, 16). Le Dieu demeure en lui, il ne peut s’en défaire et devenir sans dieu, et lui-même demeure en Dieu, il reste confiné dans l’amour de Dieu et ne peut devenir sans amour.\par
« Dieu est l’Amour ! » Tous les siècles et toutes les générations reconnaissent dans cette parole le fondement du Christianisme. Mais ce Dieu qui est amour est un Dieu importun : il ne peut pas laisser le monde en repos, il veut lui infuser la sainteté. « Dieu s’est fait homme pour rendre les hommes divins\footnote{ \noindent Athanase.
 } ». Sa main se retrouve partout, et rien n’arrive que par lui. En tout se révèlent ses « desseins excellents », ses « vues et ses décrets impénétrables ». La raison, qui est lui-même, doit aussi se développer et se réaliser  dans le monde entier. Sa providence paternelle ne nous laisse plus la moindre initiative ; nous ne pouvons rien faire de sensé sans que l’on dise : c’est Dieu qui l’a fait, ni nous attirer une disgrâce sans entendre dire : Dieu l’a voulu ainsi. Nous n’avons rien qui ne nous vienne de lui, tout nous est « donné » par lui. Mais ce que fait Dieu, l’homme le fait aussi. Dieu veut donner au monde la \emph{béatitude}, l’homme veut lui donner le \emph{bonheur,} et rendre tous les hommes heureux. C’est pourquoi tout homme voudrait éveiller chez les autres la raison qu’il croit avoir lui-même en partage : tout doit être totalement raisonnable. Dieu combat le Diable, le philosophe combat la déraison et l’irrationnel. Dieu ne laisse aucun être suivre la voie qui lui est \emph{propre}, et l’Homme ne veut nous permettre qu’une conduite humaine.\par
Mais celui qui est pénétré de l’amour sacré (religieux, moral, humain) n’a d’amour que pour le fantôme, pour le « véritable Homme », et il persécute l’individu, l’homme réel, aussi impitoyablement et avec la même froideur que s’il procédait juridiquement contre un monstre. Il trouve louable et nécessaire de se montrer inexorable, car l’amour du fantôme ou de la généralité abstraite lui ordonne de haïr tout ce qui n’est pas fantôme, c’est-à-dire l’égoïste ou l’individuel. Tel est le sens de cette fameuse manifestation de l’Amour qu’on nomme « Justice ».\par
L’accusé n’a aucun ménagement à espérer, pas une âme compatissante ne jettera un voile sur sa triste nudité. Sans émotion, le juge austère arrache au pauvre condamné ses derniers lambeaux d’excuse ; sans pitié, le geôlier le traîne à sa sombre prison ; et à l’expiration de sa peine, il n’a pas à espérer de réconciliation ; quand on le rejettera, flétri, parmi les hommes, ses bons, ses loyaux frères en Christianisme lui cracheront au visage avec mépris. Pas de grâce non plus pour le criminel « qui a mérité la mort ». On le conduit à l’échafaud, et la loi morale assouvit,  aux acclamations de la foule, son sublime besoin de — vengeance. Car l’un des deux seul peut vivre, la loi morale ou le criminel : où les criminels restent impunis, la loi morale succombe, et où celle-ci règne, ceux-là doivent tomber. Leur antagonisme est impérissable.\par
L’ère chrétienne est l’ère de la miséricorde, de l’amour, du souci de rendre aux hommes ce qui leur appartient et de les guider vers l’accomplissement de leur vocation humaine (divine). Aussi toutes les relations humaines ont-elles pour base cette considération : telle et telle chose constituent l’essence de l’homme, et, par conséquent, lui tracent la destinée à laquelle il est appelé soit par Dieu, soit (selon les idées d’aujourd’hui) par sa qualité d’Homme (sa race). De là le prosélytisme. Bien que les Communistes et les Humanitaires attendent de l’homme plus que les Chrétiens, leur point de vue reste le même. A l’homme doit appartenir tout ce qui est humain. S’il suffisait aux pieux que l’homme eût en partage ce qui est de Dieu, les humanitaires exigent que rien ne lui soit refusé de ce qui est de l’Homme. Quant à ce qui est de l’Egoïste, les uns et les autres le repoussent énergiquement. Cela est parfaitement naturel, car ce qui est l’œuvre de l’égoïsme ne peut être accordé ni concédé (en fief) : il faut qu’on le crée soi-même. Le reste, l’amour me l’accordait ; ceci, Moi seul puis me le donner.\par
Jusqu’à présent, les relations ont été fondées sur l’amour, les égards et les services réciproques. Si l’on se devait à soi-même de se sanctifier, c’est-à-dire d’introniser en soi l’être suprême et d’en faire une vérité\footnote{ \noindent \emph{« Vérité », en français dans le texte. N. d. Tr.}
 } et une réalité, on devait aussi aux autres de les aider à réaliser leur essence et leur destinée ; dans les deux cas on devait à l’essence de l’homme de contribuer à sa réalisation.\par
 Seulement, on ne se doit pas à soi-même de faire quelque chose de soi, ni aux autres de faire d’eux quelque chose : on ne doit rien ni à son essence ni à celle des autres. Toutes relations qui reposent sur une essence sont des relations avec un fantôme et non avec une réalité. Mes rapports avec l’être suprême ne sont pas des rapports avec Moi, et mes rapports avec l’essence de l’Homme ne sont pas des rapports avec les hommes.\par
De l’amour, tel qu’il est naturel à l’homme de le ressentir, la civilisation a fait un \emph{commandement}. Mais en tant que commandé, l’amour appartient à l’Homme comme tel, et non à moi ; il est mon essence, cette essence que l’on tient pour si « essentielle », et n’est pas ma propriété. C’est l’Homme, c’est-à-dire l’humanité, qui me l’impose : l’amour est obligatoire, aimer est mon devoir. Ainsi, au lieu d’avoir sa source réellement en Moi, il l’a dans l’Homme en général, dont il est la propriété, l’attribut particulier : « Il sied à l’Homme, c’est-à-dire à chaque homme, d’aimer : aimer est le devoir et la vocation de l’homme, etc. »\par
Il faut par conséquent que je revendique l’amour pour Moi, et que je le soustraie à la puissance de l’Homme.\par
On en est arrivé à me concéder comme un fief dont la propriété appartient à l’Homme ce qui était primitivement \emph{à moi}, mais sans raison logique, instinctivement. En aimant, je suis devenu un vassal, je suis devenu l’homme-lige de l’humanité, un simple représentant de cette espèce ; lorsque j’agis non pas comme Moi, mais comme Homme, j’agis comme un exemplaire de l’espèce humaine, c’est-à-dire humainement. Notre état de civilisation tout entier est un \emph{système féodal,} où la propriété appartient à l’Homme ou à l’humanité et où rien n’appartient au Moi. En dépouillant l’individu de tout pour attribuer tout à l’Homme, on a fondé une énorme féodalité. L’individu  n’apparaît plus en fin de compte que comme « foncièrement mauvais ».\par
Faut-il peut-être ne prendre aucun intérêt actif à la personne d’autrui ? dois-je n’avoir à cœur ni sa joie ni son intérêt, ne puis-je préférer la jouissance que je lui procure à telle ou telle de mes jouissances personnelles ? Loin de là : je puis lui sacrifier avec joie d’innombrables jouissances, je puis m’imposer des privations sans nombre pour augmenter son plaisir, et je puis, pour lui, mettre en péril ce qui, sans lui, me serait le plus cher, ma vie, ma prospérité, ma liberté. En effet, c’est pour moi un plaisir et un bonheur que le spectacle de son bonheur et de son plaisir. Mais je ne me sacrifie pas à lui, je reste égoïste et je — jouis de lui. En lui sacrifiant tout ce que, n’était mon amour pour lui, je me réserverais, je fais une chose très simple et même plus commune dans la vie qu’il ne parait, qui prouve uniquement qu’une certaine passion est plus forte chez moi que toutes les autres. Le Christianisme aussi enseigne à sacrifier toutes les autres passions à celle-là. Mais sacrifier des passions à une autre, ce n’est pas me sacrifier moi même ; je ne sacrifie rien de ce par quoi je suis vraiment moi, je ne sacrifie pas ce qui fait à proprement parler ma valeur, mon individualité. Il se pourrait que cette fâcheuse éventualité se produisît : c’est qu’il en est de l’amour comme de toute autre passion, du moment que j’y obéis aveuglément ; si l’ambitieux que sa passion entraîne reste sourd aux avertissements qu’un instant de sang-froid éveillerait en lui, c’est qu’il a laissé cette passion prendre les proportions d’une tyrannie à laquelle il a perdu le pouvoir de se soustraire. Il a abdiqué devant elle, parce qu’il ne sait plus se détacher d’elle et par conséquent s’en affranchir. Il est possédé.\par
Moi aussi, j’aime les hommes, non seulement quelques uns mais chacun d’eux. Mais je les aime avec la conscience de mon égoïsme : je les aime parce que  l’amour me rend heureux, j’aime parce qu’il m’est naturel et agréable d’aimer. Je ne connais pas d’obligation d’aimer. J’ai de la sympathie pour tout être sentant, ce qui l’afflige m’afflige et ce qui le soulage me soulage : je pourrais le tuer, je ne saurais le martyriser. Au contraire, le noble et vertueux philistin qu’est le prince Rodolphe des « Mystères de Paris » s’ingénie à martyriser les méchants parce qu’ils l’ « exaspèrent »\footnote{ \noindent \emph{Voir l’étude de Max Stirner sur les Mystères de Paris d’E. Sue, publiée dans les Berliner Monatschriften en 1843 et réimprimée par les soins de J. H. Mackay dans les}« M{\scshape ax} S{\scshape tirner’s kleinere schriften} » \emph{(Berlin, Schuster et Lœffler, 1898). N. d. Tr.}
 }. Ma sympathie prouve simplement que le sentiment de ceux qui sentent est aussi le mien, qu’il est ma propriété, — tandis que le procédé impitoyable de l’ « homme de bien » (la façon par exemple dont il traite le notaire Ferrand) rappelle l’insensibilité de ce brigand qui, selon la mesure de son lit, coupait ou étendait de force les jambes de ses prisonniers. Le lit de Rodolphe, à la mesure duquel il taille les hommes, c’est la notion du « Bien ». Le sentiment du droit, de la vertu, etc., rend dur et intolérant. Rodolphe ne sent pas comme le notaire ; il sent, au contraire, que « le scélérat a ce qu’il a mérité ». Ce n’est pas là de la sympathie.\par
Vous aimez l’Homme, et ce vous est une raison pour torturer l’individu, l’égoïste ; votre amour de Homme fait de vous les bourreaux des hommes.\par
Quand je vois souffrir celui que j’aime, je souffre avec lui, et je n’ai pas de repos que je n’aie tout tenté pour le consoler et l’égayer. Quand je le vois joyeux, sa joie me rend joyeux. Il ne suit pas de là que ce soit le même objet qui produit sa peine ou sa joie et qui éveille en moi les mêmes sentiments ; cela est surtout évident quand il s’agit de la douleur corporelle, que je ne ressens pas comme lui : c’est sa dent qui lui fait mal, et ce qui me fait mal à moi, c’est sa souffrance.\par
 Et c’est parce que je ne puis supporter ce pli douloureux sur le front aimé, c’est par conséquent dans mon intérêt, que je l’efface par un baiser. Si je ne t’aimais pas, tu pourrais froncer les sourcils tant que tu voudrais sans m’émouvoir ; je ne veux dissiper que \emph{mon} chagrin.\par
Y a-t-il maintenant quelqu’un ou quelque chose que je n’aime pas et qui a le \emph{droit} d’être aimé par moi ? Qui passe le premier, mon amour ou son droit ? Les parents, les amis, le peuple, la patrie, la ville natale, etc., enfin, en général, mes semblables (« mes frères ») prétendent avoir droit à mon amour et le réclament impérieusement. Ils le considèrent comme \emph{leur propriété,} et moi, si je ne respecte pas cette propriété, ils me considèrent comme un voleur qui leur enlève ce qui leur appartient.\par
Je dois donc aimer. Mais si l’amour est un commandement et une loi, il faut qu’on m’y forme et qu’on m’y dresse, et qu’on me punisse si je viens à l’enfreindre. On exercera donc sur moi, pour m’amener à aimer, la plus énergique « influence morale » possible. Et il est hors de doute que l’on peut exciter et induire les hommes à l’amour aussi bien qu’aux autres passions, à la haine par exemple. La haine se transmet de génération en génération, on peut se haïr uniquement parce que les ancêtres des uns étaient Guelfes et ceux des autres Gibelins.\par
Mais l’amour n’est pas un commandement. Comme tous mes autres sentiments, il est ma propriété. \emph{Méritez}, c’est-à-dire achetez ma propriété, et je vous la céderai. Je n’ai pas à aimer une religion, un peuple, une patrie, une famille, etc. qui ne savent pas mériter mon amour ; je vends ma tendresse au prix qu’il me plait de fixer.\par
L’amour intéressé est bien différent de l’amour désintéressé, mystique ou romantique. On peut aimer une foule de choses, on peut aimer non seulement l’homme, mais en général tout « objet » quel qu’il  soit (le vin, sa patrie, etc.). L’amour devient aveugle et furieux lorsque, devenant \emph{nécessité}, il échappe à ma puissance (aimer à la folie) ; — il devient romantique lorsqu’il s’y joint une idée de \emph{devoir}, c’est-à-dire quand l’objet de l’amour me devient sacré et quand je me sens lié à lui par le devoir, la conscience, le serment. Dans les deux cas, l’objet ne m’appartient plus, c’est moi qui lui appartiens.\par
Si l’amour est une possession, ce n’est pas en tant qu’il est mon sentiment (en cette qualité, au contraire, j’en reste maître comme de ma propriété), mais bien parce que son objet m’est étranger. L’amour religieux, en effet, repose sur le commandement d’aimer dans l’objet aimé une chose « sacrée » ; car il existe pour l’amour désintéressé des objets dignes d’amour d’une manière absolue, des objets pour lesquels mon cœur a le devoir de battre : tels sont par exemple les autres hommes, ou encore un époux, les parents, etc. L’amour sacré s’attache à ce qu’il y a de sacré dans l’objet aimé, aussi s’efforce-t-il de faire que ce qu’il aime approche autant que possible de la sainteté, et devienne par exemple, un « Homme ».\par
Ce que j’aime, il est de mon \emph{devoir} de l’aimer ; ce n’est pas par suite ou en raison de mon amour qu’il devient le but de ce dernier : il est de lui-même et par lui-même digne d’amour. Ce n’est pas Moi qui fais de lui un objet d’amour, il l’est par essence (qu’il puisse, dans une certaine mesure, l’être devenu par mon choix, s’il s’agit par exemple d’un époux, d’une fiancée, cela ne fait rien à l’affaire, attendu que, même dans ce cas, ma prédilection lui confère un « droit à mon amour » et que, l’ayant aimé, je suis tenu de l’aimer éternellement). Il n’est donc pas l’objet de \emph{mon} amour, mais de l’amour en général : c’est un objet qui \emph{doit} être aimé. L’amour lui revient, il lui est dû, il est son droit, et Moi, je suis obligé de l’aimer. Mon amour, c’est-à-dire l’amour dont je m’acquitte  envers lui, est en réalité un amour qui lui appartient, un tribut que je lui paie.\par
Tout amour auquel adhère la moindre tache d’obligation est un amour désintéressé ; et aussi loin que s’étend cette tache, l’amour devient \emph{servitude}. Quiconque croit \emph{devoir} quelque chose à l’objet de son amour aime d’une façon romantique Ou religieuse. L’amour de la famille, par exemple, tel qu’on le conçoit communément sous le nom de « piété », est un amour religieux ; de même, l’amour de la patrie qu’on prêche sous le nom de « patriotisme ». Tout ce que nous avons d’amour romantique se meut dans le même cercle : c’est partout et toujours le mensonge, ou plutôt l’illusion, d’un « amour désintéressé » ; c’est un intérêt que nous portons à l’objet pour l’amour de cet objet et non pour l’amour de nous et de nous seuls.\par
L’amour religieux ou romantique se distingue, il est vrai, de l’amour physique par une différence dans l’objet, mais pas par une différence dans nos rapports avec lui. A ce dernier point de vue, l’un comme l’autre est possession, servitude. Quant à l’objet, dans un cas il est profane, dans l’autre sacré. Il exerce sur moi, dans les deux cas, la même domination, seulement dans l’un il est sensible et dans l’autre spirituel (imaginaire). Mon amour n’est ma propriété que s’il consiste uniquement en un intérêt personnel et égoïste, si, par conséquent, l’objet de mon amour est réellement \emph{mon} objet ou ma propriété. Or, je ne dois rien à ma propriété, et je n’ai pas de devoirs envers elle, pas plus que je n’ai par exemple de devoirs envers mon œil. Si j’en prends le plus grand soin, c’est pour Moi que je le fais.\par
L’amour n’a pas plus manqué à l’Antiquité qu’aux siècles de Christianisme ; le dieu de l’amour est né longtemps avant le Dieu d’amour. Mais il était réservé aux Modernes de connaître l’esclavage du mysticisme.\par
 Si l’amour est servitude, c’est que son objet m’est étranger, et que je suis impuissant contre son éloignement et sa supériorité Pour l’égoïste, rien n’est assez haut pour qu’il croie devoir s’humilier, rien n’est assez indépendant pour qu’il en fasse le principe de sa vie, rien n’est assez sacré pour qu’il s’y sacrifie. L’amour de l’égoïste prend sa source dans l’intérêt personnel, coule dans le lit de l’intérêt personnel et a son embouchure dans l’intérêt personnel.\par
Est-ce encore là de l’amour, demandera-t-on ? Choisissez un autre nom si vous en savez un meilleur, et que le doux nom d’amour s’éteigne avec un monde qui n’est plus ! Pour ma part, je n’en trouve pas d’autre pour le moment dans notre langue \emph{chrétienne}, et je m’en tiens au vieux mot : « j’aime » l’objet qui est \emph{mien,} j’aime ma — propriété.\par
Je ne consens à me livrer à l’amour que pour autant qu’il ne soit qu’un de mes sentiments ; mais s’il faut qu’il soit une force supérieure à moi, une puissance divine (Feuerbach), une passion à laquelle j’ai le devoir de ne pas me soustraire, une obligation morale et religieuse, je le — méprise. Sentiment, il est à Moi ; principe auquel je dois vouer et « consacrer » mon âme, il est souverain et divin, comme la haine est diabolique : l’un ne vaut pas mieux que l’autre. En un mot, l’amour égoïste, c’est-à-dire \emph{mon} amour, n’est ni sacré, ni profane, ni divin, ni diabolique.\par
« Un amour que limite la foi est un amour faux. La seule limitation qui ne soit pas contradictoire avec l’essence de l’amour est celle que l’amour s’impose à lui-même par la raison, l’intelligence. L’amour qui repousse la rigueur et la loi de l’intelligence est théoriquement un amour faux, pratiquement un amour funeste\footnote{ \noindent Feuerbach, Wesen des Christentums, p. 394.
 } ». C’est ce que dit Feuerbach ; les croyants disent au contraire : L’amour est essentiellement du domaine de la foi. Celui-là s’élève avec véhémence  contre l’amour \emph{sans raison}, ceux-ci contre l’amour \emph{sans foi.} Pour Feuerbach comme pour le dévot, l’amour est tout au plus un \emph{splendidum vitium.} Ne sont-ils pas tous deux obligés de laisser subsister l’amour, même entaché de déraison ou d’impiété ? Ils n’osent pas dire : l’amour déraisonnable ou impie est une absurdité, n’est pas de l’amour, pas plus qu’ils n’oseraient dire : des larmes déraisonnables ou impies ne sont pas des larmes.\par
L’amour, même en dehors de la raison ou de la foi, doit bien être considéré comme de l’amour, encore qu’on doive le regarder alors comme indigne de l’homme ; tout ce qu’on peut conclure, c’est que l’essentiel n’est pas l’amour, mais la raison ou la loi, et que celui qui est sans raison ou sans foi peut bien aimer, mais qu’un amour n’a de valeur que s’il est celui d’un homme raisonnable ou d’un croyant. Feuerbach est victime d’une illusion lorsqu’il dit que l’amour emprunte à la raison « sa propre limitation » ; le croyant aurait au même titre le droit de dire que cette « limitation propre » est le fait de la foi. L’amour déraisonnable n’est ni « faux » ni « funeste » ; c’est comme amour tout court qu’il remplit son rôle.\par
Il faut qu’envers le monde, et particulièrement envers les hommes, j’adopte un sentiment déterminé, et que ce sentiment, qui dans le cas présent est l’amour, je le leur témoigne de prime abord, avant toute expérience. Je reconnais qu’en agissant ainsi, je fois preuve de plus d’arbitraire et d’autonomie que si je laisse le monde m’assaillir des sentiments les plus divers et si je me laisse envelopper par le réseau inextricable des impressions que le hasard m’apporte. En effet, j’aborde les hommes et les choses avec un sentiment fait d’avance, avec, pour ainsi dire, un parti-pris et une opinion préconçue. Je me suis au préalable tracé ma conduite envers eux, et, quoi qu’ils fassent, je ne sentirai et ne penserai à leur égard que comme j’ai, une fois pour toutes, résolu de le faire.  Le principe de l’amour m’assure contre la domination du monde ; car, quoi qu’il arrive, j’\emph{aime}. Ce qui est laid, par exemple, peut m’inspirer de la répulsion, mais comme j’ai résolu d’aimer, je surmonte cette impression désagréable comme je surmonte toute autre antipathie.\par
Mais le sentiment auquel je me suis \emph{a priori} déterminé et — condamné est, en réalité, un sentiment \emph{borné}, parce qu’il résulte d’une prédestination dont il ne m’est pas possible de m’affranchir. Etant préconçu, il est un \emph{préjugé.} Ce n’est plus Moi qui m’exprime dans mes rapports avec le monde, mais c’est mon amour qui s’exprime. De sorte que si le monde ne me domine pas, je suis en revanche d’autant plus fatalement dominé par l’esprit d’amour. J’ai vaincu le monde, pour devenir l’esclave de cet esprit.\par
Si j’ai dit d’abord : J’aime le monde, je puis tout aussi bien ajouter à présent : Je ne l’aime pas : car je l’\emph{anéantis} comme je m’anéantis ; j’en use et je l’use. Je ne m’astreins pas à n’éprouver pour les hommes qu’un seul et invariable sentiment, je donne libre carrière à tous ceux dont je suis capable. Pourquoi ne le déclarerais-je pas crûment ? Oui, j’exploite le monde et les hommes ! Je puis ainsi rester ouvert à toute espèce d’impressions, sans qu’aucune d’elles m’arrache à moi -même. Je puis aimer, aimer de toute mon âme, et laisser brûler dans mon cœur le feu dévorant de la passion, sans cependant prendre l’être aimé pour autre chose que pour l’\emph{aliment} de ma passion, un aliment qui l’aiguise sans la rassasier jamais. Tous les soins dont je l’entoure ne s’adressent qu’à l’objet de \emph{mon} amour, qu’à celui dont mon amour \emph{a besoin,} au « bien-aimé ». Combien il me serait indifférent, n’était — mon amour ! C’est mon amour que je repais de lui, il ne me sert qu’à cela, je \emph{jouis} de lui.\par
Choisissons un autre exemple, tout actuel, celui-ci : Je vois les hommes plongés dans les ténèbres de la superstition, harcelés par un essaim de fantômes. Si je  cherche, dans la mesure de mes forces, à projeter la lumière du jour sur ces apparitions de la nuit, croyez-vous que j’obéisse à mon amour pour vous ? J’écris peut-être par amour pour les hommes ? Hé non ! j’écris parce que je veux faire à des idées qui sont \emph{mes} idées une place dans le monde ; si je prévoyais que ces idées dussent vous ravir la paix et le repos, si dans ces idées que je sème je voyais les germes de guerres sanglantes et une cause de ruine pour maintes générations, je ne les répandrais pas moins. Faites-en ce que vous voudrez, faites en ce que vous pourrez, c’est votre affaire et je ne m’en inquiète pas. Peut être ne vous apporteront-elles que le chagrin, les combats, la mort, et ne seront-elles que pour bien peu d’entre vous une source de joie. Si j’avais à cœur votre bien-être, j’imiterais l’Eglise qui interdit aux laïques la lecture de la Bible, ou les gouvernements chrétiens qui se font un devoir sacré de défendre l’homme du peuple contre les « mauvais livres. »\par
Non seulement ce n’est pas pour l’amour de vous que j’exprime ce que je pense, mais ce n’est pas même pour l’amour de la vérité. Non :\par


\begin{verse}
« Je chante comme chante l’oiseau\\
« Qui habite dans le feuillage\\
« Le chant même que produit ma voix\\
« Est mon salaire, et un salaire royal\footnote{ \noindent Wilhelm Meister.
 }.\\
\end{verse}

\noindent Je chante ? Je chante parce que je suis un chanteur ! Si pour cela je me sers de vous, c’est que j’ai besoin — d’oreilles.\par
Quand le monde se trouve sur mon chemin (et il s’y trouve toujours), je le consomme pour apaiser la faim de mon égoïsme : tu n’es pour moi qu’une — nourriture ; de même, toi aussi tu me consommes et tu me fais servir à ton usage. Il n’y a entre nous  qu’un rapport, celui de l’utilité, du profit, de l’intérêt. Nous ne nous devons rien l’un à l’autre, car ce que je puis paraître te devoir, c’est tout au plus à moi que je le dois. Si pour te faire sourire, je t’aborde avec une mine joyeuse, c’est que \emph{j’ai intérêt} à ton sourire et que mon visage est au service de mon désir. A mille autres personnes que je ne désire pas faire sourire, je ne sourirai pas.\par

\asterism

\noindent Cet amour, qui se fonde sur l’ « essence de l’Homme » et qui, dans la période chrétienne et morale, pèse sur nous comme un « commandement », on doit y être dressé. C’est à l’influence morale, le principal facteur de notre éducation, à y pourvoir. Comment s’y prend-on pour régler les relations entre les hommes ? C’est ce que nous allons, du moins pour un cas particulier, étudier ici avec les yeux de l’égoïsme.\par
Ceux qui nous élèvent apportent un soin tout particulier à nous déshabituer de bonne heure du mensonge et à nous inculquer ce principe qu’il faut toujours dire la vérité. Si on fondait cette règle sur l’égoïsme, tout le monde s’en pénétrerait facilement ; on comprendrait sans peine que le menteur perd de gaîté de cœur la confiance qu’il désire inspirer aux autres, et on sentirait combien il est juste de dire que le menteur n’est pas cru même quand il dit vrai. Mais chacun sentirait en même temps qu’il ne doit la vérité qu’à celui que lui-même autorise à entendre cette vérité. Supposez qu’un espion rôde dans le camp ennemi sous un vêtement emprunté et qu’on lui demande qui il est. Ceux qui posent la question sont évidemment en droit de le faire, mais l’homme déguisé ne leur donne pas le droit d’apprendre de lui la vérité ; aussi leur dira-t-il tout ce qu’il lui plaira d’inventer mais non ce qui est vrai. Et pourtant la  loi morale dit : « Tu ne mentiras pas. » La morale donne donc à ceux qui m’interrogent le droit d’attendre de moi la vérité, mais Moi je ne le leur donne pas ; je ne reconnais d’autre droit que celui que j’accorde moi-même.\par
Autre exemple : La police pénètre dans une assemblée révolutionnaire et demande son nom à l’orateur. Tout le monde sait que la police a le droit de le faire ; seulement, ce droit elle ne le tient pas du révolutionnaire, qui est son ennemi : il lui donne un faux nom et il lui — ment. Mais la police n’est pas assez naïve pour se fier à la véracité de ses ennemis ; au contraire, elle ne croit rien sans preuve et tâche, autant que possible, d’ « établir l’identité » de l’individu qu’elle a interrogé. L’Etat lui-même agit toujours avec défiance envers les individus, parce qu’il reconnaît dans leur égoïsme son ennemi naturel ; il lui faut toujours « la preuve », et celui qui ne peut pas fournir cette preuve devient l’objet de recherches inquisitoriales, d’une enquête. L’Etat ne croit pas l’individu et n’a pas confiance en lui ; il vit avec lui sur le pied de la « défiance mutuelle » : il ne se fie à moi que quand il s’est convaincu de la véracité de mes assertions, et pour cela il n’a souvent d’autre moyen que le \emph{serment}. Ce moyen prouve que l’Etat ne se fie pas à notre amour de la vérité, à notre sincérité, mais seulement à notre \emph{intérêt}, à notre égoïsme. Il compte que nous ne voudrons pas nous brouiller avec Dieu par un parjure.\par
Imaginez-vous à présent un révolutionnaire français de 1788, qui, entre amis, ait laissé échapper la phrase devenue célèbre : « Le monde n’aura pas la paix avant qu’on ait étranglé le dernier des rois avec les boyaux du dernier des prêtres ! » A cette époque, le roi possède encore toute sa puissance. Le hasard a ébruité le propos, mais on ne saurait pourtant citer aucun témoin. On veut obtenir que l’accusé avoue. Doit-il ou ne doit-il pas avouer ? S’il nie, il ment et — reste impuni ;  s’il avoue, il est sincère et — on lui coupe la tête. Il met la vérité au-dessus de tout ? Soit, qu’il meure ! Il faudrait n’être qu’un poète bien misérable pour ramasser cette mort comme un sujet de tragédie : car quel intérêt y a-t-il à voir comment un homme meurt par lâcheté ? Si notre homme avait le courage de ne pas être esclave de la vérité et de la sincérité, voici à peu près ce qu’il dirait : « Quel besoin les juges ont-ils de savoir ce que j’ai dit à mes amis ? Si j’avais eu l’intention de leur en faire part, je le leur aurais dit comme je l’ai dit à mes amis ; mais il ne me plaît pas qu’ils le sachent. Ils prétendent s’imposer à ma confiance sans que je la leur aie accordée, sans que j’aie voulu faire d’eux mes confidents ; ils \emph{veulent }connaître ce que moi je \emph{veux} cacher. Approchez donc, vous qui croyez que votre volonté brisera la mienne, approchez, juges et bourreaux, et montrez votre savoir-faire. Vous pouvez me mettre à la torture, vous pouvez me menacer de l’enfer et de la damnation éternelle, vous me briserez peut-être au point de me faire prêter un faux serment, mais vous ne m’arracherez pas la vérité, car je veux vous tromper, car je ne vous ai donné aucune autorité, aucun droit sur ma sincérité. Et malgré les menaces du Dieu « qui est la vérité même », malgré l’amertume du mensonge, j’ai le courage de mentir. Lors même que je serais dégoûté de la vie et que rien ne me paraîtrait plus désirable que la hache du bourreau, vous n’auriez pas la joie de trouver en moi un esclave de la vérité, de me faire trahir ma volonté par vos ruses d’inquisiteurs. Si en prononçant les paroles dont on m’accuse je me suis rendu coupable de haute trahison, je ne m’adressais pas à vous et vous deviez les ignorer ; ma volonté est immuable et l’horreur du mensonge ne m’effrayera pas. »\par
Si Sigismond est un triste sire, ce n’est pas parce qu’il a violé sa parole de prince ; mais s’il a enfreint sa parole, c’est parce qu’il était un coquin. Il aurait pu  tenir parole, et n’en aurait pas moins été un plat coquin, un valet de la prêtraille. Luther, poussé par une force supérieure, a été infidèle à ses vœux monastiques : il le fut pour l’amour de Dieu. Tous deux ont violé leur serment parce qu’ils étaient asservis : Sigismond, parce qu’il voulait se montrer le disciple fidèle de la vérité divine, c’est-à-dire de la vraie foi, de la foi catholique ; Luther, pour rendre témoignage fidèlement, de tout son cœur et de toute son âme, en faveur de l’Evangile ; tous deux furent parjures, pour ne pas mentir à la « vérité supérieure ». Le premier fut absous par les prêtres, le second le fut par lui même. A quoi pensaient-ils tous deux, sinon à ce qu’exprime cette parole de l’apôtre : « Ce n’est pas aux hommes, mais à Dieu que tu as menti » ? Ils mentaient aux hommes, ils violaient leur serment aux yeux du monde, pour ne pas mentir à Dieu et pour le servir. Ils nous montrent ainsi comment on doit en user avec la vérité à l’égard des hommes.\par
En l’honneur de Dieu et pour l’amour de Dieu, un parjure, un mensonge, une parole princière violée !\par
Et si, changeant deux mots à la phrase, nous écrivions : un parjure et un mensonge — \emph{pour l’amour de moi ?} Ne serait-ce pas nous faire l’avocat de toute espèce de bassesses et d’infamies ? Peut-être, mais que fait-on d’autre en disant « pour l’amour de Dieu » ? L’amour de Dieu ? Quelle infamie n’a-t-on pas commise pour l’amour de Dieu ? Quels échafauds n’a-t-on pas inondés de sang pour l’amour de Dieu ? Quels autodafés n’a-t-on pas allumés pour l’amour de Dieu ? L’amour de Dieu ? Et pour qui donc l’intelligence humaine a t-elle été abrutie ? Pour qui, aujourd’hui encore, l’éducation religieuse enchaîne-t-elle l’esprit dès la plus tendre enfance ? N’a-t-on pas, « pour l’amour de Dieu », rompu des vœux sacrés ? et, tous les jours, des missionnaires et des prêtres ne parcourent-ils pas le monde pour amener des juifs, des païens, des protestants, des catholiques,  etc., à trahir la foi de leurs pères, — toujours pour l’amour de Dieu ? Y aurait-il un grand mal à ce que tout cela se fit \emph{pour l’amour de moi ?} Que signifie donc pour l’amour de moi ? Tout d’abord cela donne l’idée d’une « spéculation ignoble ». Celui qui spécule en vue d’un « gain sordide » le fait en effet pour l’amour de soi (puisqu’il n’est en somme rien que l’on ne fasse pour l’amour de soi, par exemple tout ce que l’on fait « à la plus grande gloire de Dieu ») ; mais ce soi-même pour lequel il recherche le gain est l’esclave du gain, il ne s’élève pas au-dessus du gain, il appartient au gain, au sac d’argent, et ne s’appartient pas, il n’est pas son maître. Un homme que gouverne la passion de l’avarice ne doit-il pas obéir aux ordres de cette maîtresse ? Si, une fois en passant, il se laisse aller à une généreuse faiblesse, cela ne paraîtra-t-il pas tout simplement une exception, juste comme lorsque de fidèles croyants à qui vient à manquer la conduite de leur maître tombent dans les embûches du « diable » ? Donc un avare n’est pas son possesseur ; il est esclave, et il ne peut rien faire pour l’amour de soi sans le faire en même temps pour l’amour de son maître, tout comme celui qui craint Dieu.\par
Le parjure de François I\textsuperscript{er} envers l’empereur Charles-Quint est célèbre. Ce n’est pas quelque temps après, en réfléchissant mûrement à la promesse faite, c’est immédiatement, au moment même où il prêtait serment, que François la rétracta tacitement par une restriction mentale à laquelle avaient d’avance souscrit ses conseillers. Le parjure fut prémédité. François était tout disposé à acheter sa liberté, mais le prix qu’en exigeait Charles lui paraissait trop élevé et déraisonnable. J’admets que Charles fut dupe de son avarice, en cherchant à soutirer de son prisonnier la plus grosse somme possible, mais il n’en fut pas moins misérable de la part du roi de vouloir racheter sa liberté au prix d’une rançon plus  faible qu’il n’était convenu ; la suite de son histoire, où s’étale un second parjure, démontre d’ailleurs à suffisance qu’il était possédé d’un esprit de trafic qui faisait de lui un bas filou.\par
Que répondre à ceux qui lui reprochent ce faux serment ? D’abord sans doute nous répéterons que s’il se déshonora ce ne fut pas tant par son parjure que par son avarice ; que ce n’est pas son parjure qui le rendit méprisable, mais que c’est parce qu’il était un méprisable personnage qu’il s’en rendit coupable.\par
Toutefois, considéré en lui-même, le parjure de François doit être autrement jugé.\par
Pourrait-on dire que François ne répondit pas à la confiance que Charles lui témoignait en lui rendant la liberté ? Si Charles avait eu réellement confiance en lui, il lui aurait dit le prix que lui semblait valoir sa mise en liberté, puis il lui eût ouvert la porte de sa prison et eût attendu que François lui envoyât la rançon convenue. Mais cette confiance, Charles ne l’éprouvait pas ; il ne se fiait qu’à la faiblesse et à la crédulité de François, lesquelles, croyait-il, ne lui permettraient pas de manquer à son serment. François ne trompa que — ce calcul trop crédule. C’est précisément en croyant trouver une garantie dans le serment de son ennemi que Charles l’affranchit de toute obligation. Il avait supposé chez le roi de France de la sottise, de l’étroitesse de conscience, et il mettait sa confiance non pas en François, mais dans la sottise, c’est-à-dire la scrupulosité de François. Il ne lui ouvrait les grilles de sa prison de Madrid que pour refermer sur lui les grilles plus sûres de la conscience, cette prison où la religion enferme l’esprit humain. Il le renvoyait en France, garrotté de liens invisibles : quoi d’étonnant à ce que François ait cherché à s’échapper et à rompre ses liens ? Personne n’eût trouvé mauvais qu’il s’évadât de Madrid, puisqu’il était au pouvoir d’un ennemi ; mais tout bon chrétien lui jette la pierre pour avoir  voulu se délivrer des liens de Dieu (Le pape ne le délia que plus tard de son serment).\par
Il est honteux de tromper une confiance que nous avons librement cherché à gagner ; mais quand un homme veut nous tenir en son pouvoir par un serment, le rendre victime de l’insuccès de sa ruse et de sa défiance n’est pas une honte pour l’égoïsme. Tu as voulu me lier ? Apprends donc que je puis rompre tes liens.\par
Est-ce Moi qui ai donné à celui qui a confiance le droit de se fier à moi ? toute la question est là. Qu’un homme qui poursuit mon ami me demande dans quelle direction il s’est enfui, je le mettrai certainement sur une fausse piste. Pourquoi vient-il s’adresser justement à moi, à l’ami de celui qu’il poursuit ? Plutôt que d’être un faux ami, plutôt que de trahir l’ami et l’amitié, je mentirai à l’ennemi. Je pourrais, il est vrai, répondre avec une courageuse droiture que je ne veux pas parler (c’est ainsi que Fichte résout la question). De cette manière mon amour de la vérité sera sauf, mais j’aurai fait, pour mon ami, tout juste — rien ; car si je ne dépiste pas l’ennemi, le hasard peut le mettre sur la bonne voie, et mon amour de la vérité aura livré mon ami, en m’ôtant — le courage du mensonge. Celui pour qui la vérité est une idole, une chose sacrée, doit s’humilier devant elle, il ne peut pas braver ses exigences et y résister vaillamment, bref il doit renoncer à l’\emph{héroïsme du mensonge.} Car le mensonge ne demande pas moins de courage que la vérité, et un courage dont sont dépourvus la plupart des jeunes gens : ils aiment mieux confesser la vérité et monter pour elle sur l’échafaud, que conserver, en ayant le courage de mentir, l’espoir de ruiner la puissance de l’ennemi. Pour eux la vérité est « sacrée », et ce qui est sacré exige toujours un culte aveugle fait de soumission et de sacrifice.\par
Si vous manquez d’audace, si vous ne vous moquez  pas du sacro-saint, il vous domestique et vous asservit. Qu’on amorce le piège d’un grain de vérité, vous vous y élancerez certainement tête baissée — et voilà un fou attrapé. Vous ne voulez pas mentir ? Hé bien, faites vous égorger sur l’autel de la vérité et soyez — martyrs ! Martyrs au profit de qui ? de vous-mêmes, de votre individualité ? Non, de votre idole, — de la vérité. Vous ne connaissez que deux espèces de services, que deux espèces de serviteurs : les serviteurs de la vérité et les serviteurs du mensonge. Servez donc la vérité, et que Dieu vous bénisse !\par
Il y a d’autres serviteurs de la vérité qui la servent « avec mesure », et qui font, par exemple, une distinction entre le mensonge simple et le mensonge sous serment. Et pourtant tout le chapitre du serment se confond avec celui du mensonge, car un serment n’est qu’une énonciation fortement affirmée. Vous vous croyez en droit de mentir parce que vous n’ajoutez pas un serment ? Ceux qui y regardent de près doivent condamner et damner le mensonge aussi sévèrement que le faux serment. Il s’est conservé dans la morale un vieux sujet de controverse que l’on a l’habitude de traiter sous le titre de « mensonge officieux ». Quiconque admet le mensonge officieux est obligé, pour être conséquent, d’admettre le « serment officieux ». Si mon mensonge se trouve justifié parce qu’il est un mensonge de nécessité, pourquoi serais-je assez pusillanime pour priver ce mensonge justifié de l’appui de la plus forte affirmation ? Quoi que je fasse, pourquoi ne le ferais-je pas tout à fait et sans restriction \emph{(reservatio mentalis) ?} Et si je me mets à mentir, pourquoi ne pas le faire complètement, en toute connaissance de cause et de toutes mes forces ? Espion, je serais obligé de confirmer par serment toutes les fausses déclarations que je ferais à l’ennemi. Résolu à lui mentir, devrais-je tout à coup sentir ma résolution et mon courage  faiblir si l’on m’acculait au serment ? C’est qu’alors j’aurais été d’avance corrompu et rendu incapable de faire un menteur ou un espion, puisque je fournirais de mon plein gré à l’ennemi le moyen de me démasquer.\par
L’Etat lui-même craint le mensonge et le serment « officieux » ; aussi n’admet-il pas l’accusé au serment. Mais vous ne justifiez pas la crainte de l’Etat : Vous mentez, mais vous ne prêtez pas de faux serment. Si vous avez, par exemple, rendu à quelqu’un un service qu’il doit ignorer, qu’il vienne à s’en douter et qu’il vous pose la question en face, vous nierez ; s’il insiste, vous direz : « Non, bien certainement, non ! » S’il fallait en venir au serment, vous reculeriez, car la crainte du sacré vous arrête toujours à moitié chemin. \emph{Contre} le sacré, vous êtes sans volonté propre. Vous mentez avec mesure, comme vous êtes libres « avec mesure », religieux « avec mesure » (voir la fade controverse actuelle de l’Université contre l’Eglise à propos des « empiètements du clergé »), monarchiques « avec mesure » (il vous faut un monarque lié par une constitution, une loi fondamentale de l’Etat). Que tout soit gentiment \emph{tempéré}, bien tiède et bien doux, tant bien que mal.\par
Il avait été convenu entre les étudiants d’une université que toute parole d’honneur qu’exigerait d’eux le juge universitaire serait nulle et non avenue. Ils ne voyaient en effet dans cette exigence qu’un piège, impossible à éviter si l’on n’enlevait pas toute signification à une parole donnée dans ces conditions. A la même université, quiconque manquait à sa parole d’honneur envers un condisciple était infâme, et quiconque avait donné sa parole au juge universitaire pouvait aller rire avec les mêmes condisciples aux dépens du juge trompé, qui se figurait qu’un serment a la même valeur entre amis et entre ennemis. Ce n’était pas tant la théorie que la nécessité pratique qui avait appris à ces étudiants à agir ainsi ; sans ce  stratagème, ils auraient inévitablement été forcés de trahir et de dénoncer leurs amis. Mais si le moyen se justifiait pratiquement, il a aussi sa justification théorique. Une parole d’honneur ou un serment ne m’engagent qu’envers celui à qui \emph{moi-même} je donne le droit de les recevoir ; contraint à jurer de dire la vérité, je ne donnerai qu’une parole \emph{contrainte}, c’est-à-dire hostile, la parole d’un ennemi ; vous n’avez pas le droit de vous y fier, car l’ennemi ne vous accorde pas ce droit.\par
D’ailleurs, les tribunaux de l’Etat eux-mêmes ne reconnaissent pas l’inviolabilité du serment. Si j’avais juré à un homme contre qui la justice instruit de ne rien révéler à sa charge, la cour, sans tenir compte du serment qui me lie, ne manquerait pas d’exiger mon témoignage, et, en cas de refus, de me faire enfermer jusqu’à ce que je me décide — à devenir parjure. La cour « me délie de mon serment ». Quelle générosité ! Seulement, s’il est au monde une puissance qui puisse me délier du serment, je suis certainement moi-même la première puissance qui ait droit de le faire.\par
Comme curiosité, et pour rappeler toutes sortes de serments usuels, il est juste de donner place ici à celui que l’empereur Paul fit prêter aux prisonniers polonais (Kosciusko, Potocki, Niemcewicz, etc.), lorsqu’il leur rendit la liberté : « Nous jurons non seulement fidélité et obéissance à l’Empereur, mais nous promettons de verser notre sang pour sa gloire. Nous nous engageons à dénoncer tout ce qui pourrait venir à notre connaissance de menaçant pour sa personne ou son empire, nous déclarons enfin qu’en quelque point du monde que nous nous trouvions, un seul mot de l’Empereur suffira pour que nous quittions tout et nous rendions à son appel ».\par
 
\asterism

\noindent Il est un domaine où il semble que le principe de l’amour ait été depuis longtemps débordé par l’égoïsme, et où il paraît ne plus manquer qu’une chose, la conscience au bon droit dans la victoire. Ce domaine est celui de la spéculation sous ses deux formes, pensée et agiotage.\par
On s’abandonne hardiment à sa pensée sans se demander ce qu’il en adviendra, et on se livre à toutes sortes d’opérations financières malgré le grand nombre de ceux qui souffriront peut-être de nos spéculations. Mais bien que l’on ait dépouillé le dernier reste de religiosité, de romantisme ou d’« humanité \emph{ »,} si une catastrophe finit par se produire, la conscience religieuse se réveille et on fait tout au moins \emph{profession} d’humanité. Le spéculateur avide laisse tomber quelques sous dans le tronc des pauvres et « fait du bien » ; le penseur téméraire se console en songeant qu’il travaille au progrès du genre humain, que l’humanité se trouvera bien des ruines qu’il a faites, ou encore, en se disant qu’il est « au service de l’Idée \emph{ ».} L’Humanité, l’Idée, sont pour lui ce quelque chose dont il est obligé de dire : cela est au-dessus de moi.\par
On a jusqu’aujourd’hui pensé et trafiqué — pour l’amour de Dieu. Ceux qui, pendant six jours, ont tout foulé aux pieds en vue de leurs intérêts égoïstes, offrent, le septième jour, un sacrifice au Seigneur ; ceux dont la pensée inexorable a bouleversé mille « bonnes causes » ne le faisaient que pour servir une autre « bonne cause », et sont obligés de penser non seulement à eux-mêmes mais à un « autre » qui doit bénéficier de leur satisfaction personnelle, au Peuple, à l’Humanité, etc. Mais cet « autre » est un être au-dessus d’eux, un être supérieur,  un être suprême, et c’est pourquoi je puis dire qu’ils travaillent « pour l’amour de Dieu ».\par
Je puis par conséquent dire aussi que le principe de toutes leurs actions est — l’Amour. Non pas, toutefois, un amour volontaire, leur propriété à eux, mais un amour obligatoire, appartenant à l’être suprême (c’est-à-dire à Dieu, qui est l’amour même) ; bref, non pas l’amour égoïste mais l’amour religieux, un amour qui naît de l’illusion qu’ils doivent payer un tribut à l’Amour, c’est-à-dire qu’il ne leur est pas permis d’être des « égoïstes ».\par
Notre désir de délivrer le monde des liens qui entravent sa liberté n’a pas sa source dans notre amour pour lui, le monde, mais dans notre amour pour nous ; n’étant ni par profession ni par « amour » les libérateurs du monde, nous voulons simplement en enlever la possession à d’autres, et le faire nôtre ; il ne faut pas qu’il reste asservi à Dieu (l’Eglise) et à la loi (l’Etat), mais qu’il devienne \emph{notre propriété. }Quand le monde est à nous, il n’exerce plus sa puissance \emph{contre} nous, mais \emph{pour} nous. Mon égoïsme a intérêt à affranchir le monde, afin qu’il devienne — ma propriété.\par
L’état primitif de l’homme n’est pas l’isolement ou la solitude, mais bien la société. Au début de notre existence nous nous trouvons déjà étroitement unis à notre mère, puisque avant même de respirer nous partageons sa vie. Lorsqu’ensuite nous ouvrons les yeux à la lumière, c’est pour reposer encore sur le sein d’un être humain qui nous bercera sur ses genoux, qui guidera nos premiers pas, et nous enchaînera à sa personne par les mille liens de son amour. La société est notre \emph{état de nature}. C’est pourquoi l’union qui a d’abord été si intime se relâche peu à peu, à mesure que nous apprenons à nous connaître, et la dissolution de la société primitive devient de plus en plus manifeste. Si la mère veut, une fois encore, avoir pour elle seule l’enfant qu’elle a porté,  il faut qu’elle aille l’arracher à la rue et à la société de ses camarades. L’enfant préfère les \emph{relations} qu’il a nouées avec ses semblables à la \emph{société} dans laquelle il n’est pas entré, où il n’a fait que naître.\par
Mais l’union ou l’association sont la dissolution de la société. Il est vrai qu’une association peut dégénérer en société, comme une pensée peut dégénérer en idée fixe ; cela a lieu quand dans la pensée s’éteint l’énergie pensante, le penser lui même, ce perpétuel désaveu de toutes les pensées qui tendent à prendre trop de consistance. Lorsqu’une association s’est cristallisée en société, elle cesse d’être une association (car l’association veut que l’action de s’associer soit permanente), elle ne consiste plus que dans le fait d’être associés, elle n’est plus que l’immobilité, la fixité, elle est — morte comme association, elle est le cadavre de l’association, c’est-à-dire qu’elle est — société, communauté. Une analogie frappante rapproche sous ce rapport l’association du parti.\par
Qu’une société, l’Etat par exemple, restreigne ma \emph{liberté}, cela ne me trouble guère. Car je sais bien que je dois m’attendre à voir ma liberté limitée par toutes sortes de puissances, par tout ce qui est plus fort que moi, même par chacun de mes voisins ; quand je serais l’autocrate de toutes les R... je ne jouirais pas de la liberté absolue. Mon \emph{individualité}, au contraire, je n’entends pas la laisser entamer. Et c’est précisément à l’individualité que la société s’attaque, c’est elle qui doit succomber sous ses coups.\par
Une société à laquelle je m’attache m’enlève bien certaines libertés ; mais en revanche elle m’en assure d’autres. Il importe de même assez peu que je me prive moi-même (par exemple par un contrat) de telle ou telle liberté. Par contre, je défendrai jalousement mon individualité.\par
Toute communauté a une tendance, plus ou moins grande d’après la somme de ses forces, à devenir pour ses membres une \emph{autorité}, et à leur imposer des limites.  Elle leur demande, et doit leur demander un certain esprit d’obéissance, elle exige que ses membres lui soient soumis, soient ses « sujets », elle n’existe que par la \emph{sujétion}. Cela ne veut pas dire qu’elle ne puisse faire preuve d’une certaine tolérance ; au contraire, elle fera bon accueil aux projets d’amélioration, aux conseils et aux critiques, pour autant qu’ils aient en vue son bénéfice ; mais la critique doit se montrer « bienveillante », on ne lui permet pas d’être « insolente et irrévérencieuse » ; en d’autres termes, il faut laisser intacte et tenir pour sacrée la substance de la société. La société ne prétend pas que ses membres s’élèvent et se placent au-dessus d’elle ; elle veut qu’ils restent « dans les bornes de la légalité », c’est-à-dire qu’ils ne se permettent que ce que leur permettent la société et ses lois.\par
Il y a loin d’une société qui ne restreint que ma liberté à une société qui restreint mon individualité. La première est une union, un accord, une association. Mais celle qui menace l’individualité est une puissance \emph{pour soi} et au-dessus de Moi, une puissance qui m’est inaccessible, que je peux bien admirer, honorer, respecter, adorer, mais que je ne puis ni dominer ni mettre à profit, parce que devant elle je me résigne et j’abdique. La société est fondée sur ma résignation, mon abnégation, ma lâcheté que l’on nomme — \emph{humilité}. Mon humilité fait sa grandeur, ma soumission sa souveraineté.\par
Mais sous le rapport de la \emph{liberté}, il n’y a pas de différence essentielle entre l’Etat et l’association. Pas plus que l’Etat n’est compatible avec une liberté illimitée, l’association ne peut naître et subsister si elle ne restreint de toutes façons la liberté. On ne peut nulle part éviter une certaine limitation de la liberté, car il est impossible de s’affranchir \emph{de tout :} on ne peut pas voler comme un oiseau pour la seule raison qu’on le désire, car on ne se débarrasse pas de sa pesanteur ; on ne peut pas vivre à son gré sous l’eau comme un poisson,  car on a besoin d’air, c’est là un besoin dont on ne peut s’affranchir, et ainsi de suite. La religion, et en particulier le Christianisme, ayant torturé l’homme en exigeant de lui qu’il réalise le contre-nature et l’absurde, c’est par une conséquence naturelle de cette impulsion religieuse extravagante que l’on en vint à élever au rang d’idéal la \emph{liberté en soi}, la \emph{liberté absolue}, ce qui était étaler au plein jour l’absurdité des vœux impossibles.\par
L’association procurant une plus grande somme de liberté, pourra être considérée comme « une nouvelle liberté » ; on y échappe en effet à la contrainte inséparable de la vie dans l’Etat ou la Société ; toutefois, les restrictions à la liberté et les obstacles à la volonté n’y manqueront pas. Car le but de l’association n’est pas précisément la liberté, qu’elle sacrifie à l’individualité, mais, cette individualité elle-même. Relativement à celle-ci, la différence est grande entre Etat et association. L’Etat est l’ennemi, le meurtrier de l’individu, l’association en est la fille et l’auxiliaire ; le premier est un Esprit, qui veut être adoré en esprit et en vérité, la seconde est mon œuvre, elle est née de Moi. L’Etat est le maître de mon esprit, il veut que je croie en lui et m’impose un \emph{credo,} le \emph{credo} de la légalité. Il exerce sur Moi une influence morale, il règne sur mon esprit, il proscrit mon Moi pour se substituer à lui comme mon \emph{vrai moi.} Bref, l’Etat est sacré, et en face de moi, l’individu, il est le véritable homme, l’esprit, le fantôme. L’association au contraire est mon œuvre, ma créature ; elle n’est pas sacrée et n’est pas une puissance spirituelle supérieure à mon esprit.\par
Je ne veux pas être l’esclave de mes maximes, mais je veux qu’elles restent, sans aucune garantie, exposées sans cesse à ma critique ; je ne leur accorde aucun droit de cité chez moi. Mais j’entends encore moins engager mon avenir à l’association et lui « vendre mon âme », comme on dit quand il s’agit du diable  et comme c’est réellement le cas quand il s’agit de l’Etat ou d’ une autorité spirituelle. Je suis et je reste pour moi plus que l’Etat, plus que l’Eglise, Dieu, etc. et par conséquent, infiniment plus aussi que l’association.\par
La Société que le Communisme se propose de fonder paraît à première vue se rapprocher extrêmement de l’association telle que je l’entends. Le but qu’elle se propose est le « bien de tous », et lorsqu’on dit de tous, il faut entendre, Weitling ne se lasse pas de le répéter, d’absolument tous, de tous sans exception. Il semble bien en réalité que personne n’y doive être désavantagé. Mais quel sera donc ce bien ? Y a-t-il un seul et même bien pour tous, tous se trouveront-ils également bien d’une seule et même chose ? S’il en est ainsi, c’est du « vrai bien » qu’il s’agit. Et nous voilà ramenés précisément au point où commence la tyrannie de la religion. Le Christianisme dit : Ne vous arrêtez pas aux vanités de ce monde, cherchez votre vrai bien, devenez de pieux chrétiens. Etre chrétien, voilà le vrai bien. C’est le vrai bien de « tous », parce que c’est le bien de l’Homme comme tel (du fantôme). Mais le bien de « tous » est-il nécessairement \emph{mon} bien et \emph{ton} bien ? Et si pour toi et pour moi ce bien là n’en est pas un, aura-t-on soin de nous procurer ce dont \emph{nous} jugeons devoir nous trouver bien ? Au contraire : la Société ayant décrété que le « vrai bien » est telle ou telle chose, par exemple la jouissance honnêtement acquise par le travail, s’il arrive que tu préfères, toi, les délices de la paresse, la jouissance sans le travail, la Société, qui veille au « bien de tous », se gardera d’étendre sa sollicitude à ce qui pour toi est le bien. Le Communisme, qui se fait le champion du bien de \emph{tous} les hommes, anéantit précisément le bien-être de ceux qui ont jusqu’à présent vécu de leurs rentes et qui s’en trouvent probablement mieux que des heures de travail strictement réglées que leur promet Weitling.\par
Le même Weitling affirme que le bien-être de  quelques milliers d’hommes ne peut être mis en balance avec le bien-être de plusieurs millions d’autres, et il exhorte les premiers à renoncer à leurs avantages particuliers « pour l’amour du bien général ». Non, n’exigez pas des gens qu’ils sacrifient la moindre partie de ce qu’ils ont à la communauté ; c’est là une façon chrétienne de présenter les choses avec laquelle vous n’aboutirez à rien. Exhortez-les au contraire à ne se laisser arracher ce qu’ils ont par personne, engagez-les à s’en assurer la possession de façon à ce qu’elle soit durable, ils vous comprendront beaucoup mieux. Ils en viendront alors d’eux-mêmes à se dire que le meilleur moyen de soigner leur bien, c’est de s’allier dans ce but avec d’autres, c’est-à-dire de « sacrifier une partie de leur liberté », non pas dans l’intérêt de tous, mais dans leur propre intérêt. Comment peut-on encore être tenté de faire appel à l’esprit de sacrifice et à l’amour désintéressé des hommes ? On ne sait que trop que ces beaux sentiments n’ont produit, après une gestation de plusieurs milliers d’années, que la présente misère. Pourquoi s’obstiner à attendre encore de l’abnégation la venue de temps meilleurs ? Pourquoi ne pas mettre plutôt son espoir dans l’\emph{usurpation ?} Ce n’est plus des débonnaires et des miséricordieux, ce n’est plus de ceux qui donnent et de ceux qui aiment que viendra le salut, mais uniquement de ceux qui \emph{prendront}, qui s’approprieront et qui sauront dire : ceci est à moi. Le Communisme compte encore toujours sur l’amour, et, conscient ou inconscient, l’Humanitaire qui bafoue l’égoïsme ne sort pas de la même ornière.\par
Quand la communauté est devenue pour l’homme un besoin, quand il trouve qu’elle l’aide à réaliser ses desseins, elle ne tarde pas, prenant rang de principe, à lui imposer ses lois, les lois de la \emph{ — société.} Le principe des hommes arrive ainsi à régner souverainement sur eux ; il devient leur être suprême, leur dieu, et, comme  tel, leur législateur. Le Communisme conduit ce principe jusqu’à ses plus rigoureuses conséquences, et le Christianisme est la religion de la société ; car, comme Feuerbach le dit justement, bien que sa pensée ne soit pas juste, l’amour est l’essence de l’Homme, c’est-à-dire l’essence de la société ou de l’Homme social (communiste). Toute religion est un culte de la société, du principe qui régit l’homme social (l’homme cultivé) ; aussi nul dieu n’est-il jamais le dieu exclusif d’un Moi ; toujours un dieu est le dieu d’une société ou d’une communauté : d’une famille (lares, pénates), d’un Peuple (dieux nationaux) ou de « tous les hommes » (« Il est le père de tous les hommes »).\par
Que l’on n’espère point arriver à détruire de fond en comble la religion, tant que l’on n’aura pas auparavant mis au rebut la société et tout ce qu’implique son principe. Or, c’est précisément à l’heure du Communisme que ce principe passe au méridien, attendu qu’alors tout doit être \emph{commun} afin que règne l’ « égalité ». Cette « égalité » une fois conquise, la « liberté » ne manquera pas non plus, mais la liberté de qui ? De la Société ! La société alors est le grand Pan, et les hommes n’existent plus que « les uns pour les autres ». C’est l’apothéose de l’ « Amour-Etat » !\par
Pour moi, j’aime mieux avoir recours à l’égoïsme des hommes qu’à leurs « services d’amour », à leur miséricorde, à leur charité, etc. L’égoïsme exige la réciprocité (donnant, donnant), il ne fait rien pour rien, et s’il offre ses services, c’est pour qu’on les — \emph{achète.} Mais le « service d’amour », comment me le procurer ? C’est le hasard qui fera que j’aurai justement affaire à un « bon cœur ». Et je ne puis émouvoir la charité qu’en \emph{mendiant} ses services, soit par mon extérieur misérable, soit par ma détresse, ma misère, ma souffrance. — Et que puis-je lui offrir en échange de son assistance ? Rien ! Il faut que je la reçoive comme un cadeau. L’amour ne se paie pas, ou, disons  mieux : l’amour peut bien se payer, mais seulement en amour (un service en vaut un autre). Quelle misère, quelle gueuserie que de recevoir d’année en année, sans jamais rien rendre en échange, les dons que nous fait par exemple régulièrement le pauvre manœuvre ! Celui qui reçoit ainsi, que peut-il faire pour l’autre, en échange de ces sous dont l’accumulation forme pourtant toute sa fortune ? Le manœuvre aurait plus de jouissance si celui qu’il engraisse de ses laborieux bienfaits n’existait pas, ni ses lois et ses institutions qu’il paie par-dessus le marché. Et malgré tout, le pauvre diable \emph{aime} encore son maître !\par
Non, la communauté comme « but » de l’histoire jusqu’à ce jour est impossible. Défaisons-nous au plus tôt de toute illusion hypocrite à ce sujet, et reconnaissons que si c’est en tant qu’Hommes que nous sommes égaux, égaux nous ne le sommes pas, attendu que nous ne sommes pas Hommes. Nous ne sommes égaux qu’en tant que pensés ; ce qu’il y a d’égal en nous, c’est « nous » tels que nous nous concevons et non tels que nous sommes en réalité et en personnes. Je suis « moi » et tu es « moi », mais Je ne suis pas ce « moi » \emph{pensé ;} il n’est, lui par qui nous sommes tous égaux, que \emph{ma pensée}. Je suis Homme et tu es homme, mais « Homme » n’est qu’une idée, une généralité abstraite. Ni Moi ni Toi ne pouvons être exprimés, nous sommes \emph{indicibles}, parce qu’il n’y a que les idées qui puissent être exprimées et se fixer par la parole.\par
Cessons donc d’aspirer à la communauté ; ayons plutôt en vue la \emph{particularité}. Ne recherchons pas la plus vaste collectivité, la « société humaine », ne cherchons dans les autres que des moyens et des organes à mettre en œuvre comme notre propriété ! Dans l’arbre et dans l’animal, nous ne voyons pas nos semblables, et l’hypothèse d’après laquelle les autres seraient nos semblables prend sa source dans une hypocrisie. Personne n’est \emph{mon semblable}, mais,  semblable à tous les autres êtres, l’homme est pour moi une propriété. On a beau me dire que je dois me comporter en homme envers « le prochain » et que je dois « respecter » mon prochain. Personne n’est pour moi un objet de respect ; mon prochain, comme tous les autres êtres, est un \emph{objet} pour lequel j’ai ou je n’ai pas de sympathie, un objet qui m’intéresse ou ne m’intéresse pas, dont je puis ou dont je ne puis pas me servir.\par
S’il peut m’être utile, je consens à m’entendre avec lui, à m’associer avec lui pour que cet accord augmente ma force, pour que nos puissances réunies produisent plus que l’une d’elles ne pourrait faire isolément. Mais je ne vois dans cette réunion rien d’autre qu’une augmentation de ma force, et je ne la conserve que tant qu’elle est \emph{ma} force multipliée. Dans ce sens-là, elle est une — association.\par
L’association n’est maintenue ni par un lien naturel ni par un lien spirituel ; elle n’est ni une société naturelle ni une société morale. Ce n’est ni l’unité de sang, ni l’unité de croyance (c’est-à-dire d’esprit) qui lui donne naissance. Dans une société naturelle, — comme une famille, une tribu, une nation, ou même l’humanité —, les individus n’ont que la valeur d’\emph{exemplaires} d’un même genre ou d’une même espèce ; dans une société morale, — comme une communauté religieuse ou une église, — l’individu ne représente qu’un \emph{membre} animé de l’esprit commun ; dans l’un comme dans l’autre cas, ce que tu es comme Unique doit passer à l’arrière plan et s’effacer. Ce n’est que dans l’association que votre unicité peut s’affirmer, parce que l’association ne vous possède pas, mais que vous la possédez et que vous vous servez d’elle.\par
Dans l’association, et dans l’association seule, la propriété prend sa véritable valeur et est réellement propriété, attendu que je n’y dois plus à personne ce qui est à moi. Les Communistes ne font que consacrer  logiquement un état de choses qui dure depuis qu’a commencé l’évolution religieuse et dont l’Etat donne la formule : une féodalité, ayant en somme à sa base la négation de la propriété.\par
L’Etat s’efforce de discipliner les appétits ; en d’autres termes, il cherche à faire en sorte qu’ils se tournent vers lui seul, et à les \emph{satisfaire} au moyen de ce qu’il a à leur offrir. Rassasier un appétit pour l’amour de celui qui l’éprouve est une idée qui ne saurait venir à l’Etat ; il flétrit du nom d’ « égoïste » celui qui manifeste des désirs déréglés, et l’ « homme égoïste » est son ennemi. Il l’est parce que l’Etat, incapable de « comprendre » l’égoïste, ne peut s’entendre avec lui. Comme l’Etat (et il ne pourrait en être autrement) ne s’occupe que de lui-même, il ne s’informe pas de mes besoins et ne s’inquiète de moi que pour me corrompre et me fausser, c’est-à-dire pour faire de moi un autre moi, un bon citoyen. Il prend une foule de mesures pour « améliorer les mœurs ». — Et par quel moyen s’attache-t-il les individus ? Au moyen de lui-même, c’est-à-dire de ce qui est à l’Etat, de la \emph{propriété de l’Etat}. Il s’occupe sans relâche à faire participer tout le monde à ses « biens », à faire profiter tout le monde des « avantages de l’instruction » : il vous donne son éducation, il vous ouvre l’accès de ses établissements d’instruction, il vous met à même d’arriver par les voies de l’industrie à la propriété, c’est à-dire à l’inféodation. Seigneur généreux, il n’exige de vous, en échange de cette investiture, que le légitime hommage d’une perpétuelle reconnaissance. Mais les vassaux ingrats et félons oublient de s’acquitter de cette redevance. — La « Société » à son tour ne peut agir d’une façon essentiellement différente.\par
Tu apportes dans l’association toute ta puissance, toute ta richesse, et tu t’y \emph{fais valoir.} Dans la société, toi et ton activité \emph{êtes utilisés.} Dans la première, tu vis en égoïste, dans la seconde tu vis en Homme,  c’est-à-dire religieusement : tu y travailles à la vigne du Seigneur. Tu dois à la société tout ce que tu as, tu es son obligé et tu es obsédé de « devoirs sociaux » ; à l’association, tu ne dois rien : elle te sert, et tu la quittes sans scrupules dès que tu n’a plus d’avantages à en tirer.\par
Si la société est plus que toi, tu la feras passer avant toi et tu t’en feras le serviteur ; l’association est ton outil, ton arme, elle aiguise et [{\corr multiplie}] ta force naturelle. L’association n’existe que pour toi et par toi, la société au contraire te réclame comme son bien et elle peut exister sans toi. Bref, la société est \emph{sacrée} et l’association est \emph{ta propriété}, la société se sert de toi et tu te sers de l’association.\par
On ne manquera probablement pas de nous objecter que l’accord que nous avons conclu peut devenir gênant et limiter notre liberté ; on dira qu’en définitive nous en venons aussi à ce que « chacun devra sacrifier une partie de sa liberté dans l’intérêt de la communauté ». Mais ce n’est nullement à la « communauté » que ce sacrifice sera fait, pas plus que ce n’est pour l’amour de la « communauté » ou de qui que ce soit que j’ai contracté ; si je m’associe, c’est dans mon intérêt, et si je sacrifiais quelque chose, ce serait encore dans mon intérêt, par pur \emph{égoïsme. }D’ailleurs, en fait de « sacrifice », je ne renonce qu’à ce qui échappe à mon pouvoir, c’est-à-dire que je ne « sacrifie » rien du tout.\par
Pour en revenir à la propriété, c’est donc le maître qui est propriétaire. Et maintenant, choisis : veux-tu être le maître, ou veux-tu que la société soit maîtresse ? Il dépendra de là que tu sois un \emph{propriétaire }ou un \emph{gueux !} L’égoïsme fait le propriétaire, la société fait le gueux. Or, gueuserie ou absence de propriété, tel est le sens de la féodalité, du régime de vasselage qui, depuis le siècle dernier, n’a fait que changer de maître en mettant l’Homme à la place du Dieu, et en faisant un fief de l’Homme de ce qui auparavant  était un fief accordé par la grâce divine.\par
Nous avons montré plus haut que la gueuserie du Communisme est, par le principe humanitaire, poussée jusqu’à la gueuserie absolue, jusqu’à la plus gueuse des gueuseries ; mais nous avons montré aussi que ce n’est que par cette voie que la gueuserie peut aboutir à l’individualité. L’ancien régime féodal a été si complètement anéanti par la Révolution, que toute réaction, quelque habileté qu’elle déploie à galvaniser le cadavre au passé, est désormais condamnée à avorter misérablement, car ce qui est mort — est mort. Mais la résurrection aussi devait, dans l’histoire du Christianisme, se montrer comme une vérité ; elle l’a fait : dans un monde nouveau, la féodalité est ressuscitée avec un corps transfiguré, féodalité nouvelle sous la haute suzeraineté de « l’Homme ».\par
Le Christianisme est loin d’être anéanti et ses fidèles ont eu raison de voir avec confiance dans les assauts qu’on lui a livrés jusqu’à présent de simples épreuves dont il ne devait sortir que plus pur et plus fort ; il n’a fait en réalité que se transfigurer, et le Christianisme « qu’on vient de découvrir » est l’— \emph{humain. }Nous vivons encore en pleine ère chrétienne ; ce sont précisément ceux que cela irrite le plus qui contribuent le plus à la faire durer. Plus la féodalité s’est faite humaine, plus elle nous est devenue chère : nous ne reconnaissons plus le caractère de féodalité dans ce que, pleins de confiance, nous prenons pour notre propriété ; et nous croyons avoir trouvé ce qui est « à nous » quand nous découvrons ce qui est « à l’Homme ».\par
Si le Libéralisme veut me donner ce qui est à moi, ce n’est point qu’il y voie le mien, mais l’humain. Comme si, sous ce déguisement, il m’était possible de l’atteindre ! Les droits de l’Homme eux-mêmes, ce produit tant vanté de la Révolution, doivent s’entendre dans ce sens : l’Homme qui est en moi me donne droit à telle et telle choses ; en tant qu’individu,  c’est-à-dire tel que je suis, je n’ai aucun droit ; les droits sont l’apanage de l’Homme, et c’est lui qui m’autorise et me justifie. Comme Homme, je puis avoir un droit, mais je suis plus qu’Homme, je suis un homme \emph{particulier}, aussi ce droit peut il m’être refusé à Moi, au particulier.\par
Mais si vous savez faire cas de votre richesse, si vous tenez à haut prix vos talents, si vous ne permettez pas qu’on vous force à les vendre au-dessous de leur valeur, si vous ne vous laissez pas mettre en tête que votre marchandise n’est pas précieuse, si vous ne vous rendez pas ridicules par un « prix dérisoire », mais si vous imitez le brave qui dit : « Je vendrai cher ma Vie (ma propriété), l’ennemi ne l’aura pas à bon marché », — alors vous aurez reconnu comme vrai le contraire du Communisme, et l’on ne pourra plus vous dire : renoncez à votre propriété ! vous répondriez : je veux en profiter.\par
Au fronton de notre siècle, on ne lit plus la maxime delphique : « Connais-toi toi-même », mais bien : « E{\scshape xploite-toi toi-même} ! »\par
Proudhon dit que la propriété c’est « le vol ». Mais la propriété d’autrui (il ne parle que de celle-là) n’existe que par le fait d’une renonciation, d’un abandon, comme une conséquence de mon humilité ; elle est un \emph{cadeau}. Que signifient alors toutes ces grimaces sentimentales ? Pourquoi faire appel à la compassion comme un pauvre volé, quand on n’est qu’un imbécile et un lâche faiseur de cadeaux ? Et pourquoi rejeter toujours la faute sur les autres et les accuser de nous voler, alors que c’est nous-mêmes qui sommes en faute en ne les volant pas ? S’il y a des riches, la faute en est aux pauvres.\par
En général, personne ne s’indigne et ne proteste contre sa propre propriété ; on ne s’irrite que contre celle d’autrui. Chacun, pour sa part, veut augmenter et non diminuer ce qu’il peut appeler \emph{sien} et voudrait pouvoir appeler tout ainsi. Ce n’est en réalité  pas à la propriété qu’on s’attaque, mais à la propriété \emph{étrangère ;} ce que l’on combat, c’est, pour former un mot qui fasse le pendant de propriété, l’\emph{aliénité}. Et comment s’y prend-on ? Au lieu de transformer l’\emph{alienum} en \emph{proprium} et de s’approprier le bien étranger, on se donne des airs d’impartialité et de détachement et l’on demande seulement que toute propriété soit abandonnée à un tiers (par exemple à la Société humaine). On revendique le bien étranger non pas en son nom à soi, mais au nom d’un tiers. Alors toute trace d’ « égoïsme, » disparaît, et tout devient on ne peut plus pur, on ne peut plus humain !\par
Radicale inhabilité de l’individu à être propriétaire, radicale gueuserie, telle est l’ « essence du Christianisme » et de toute religiosité (piété, moralité, humanité), tel est le principe jadis voilé qu’a mis en tête de son joyeux message la « religion nouvelle ». C’est l’évolution de ce nouvel Evangile que nous avons sous les yeux dans la lutte qui se livre actuellement contre la propriété et qui doit conduire l’Homme à la victoire : la victoire de l’humanité, c’est le triomphe du — Christianisme. Et ce Christianisme « qui vient seulement d’être découvert » est la féodalité parfaite, la servitude universelle, la — parfaite gueuserie.\par
Est-ce donc une nouvelle « révolution » qu’appelle cette féodalité nouvelle ?\par
Révolution et insurrection ne sont pas synonymes. La première consiste en un bouleversement de l’ordre établi, du \emph{status} de l’Etat ou de la Société, elle n’a donc qu’une portée \emph{politique} ou \emph{sociale.} La seconde entraîne bien comme conséquence inévitable le même renversement des institutions établies, mais là n’est point son but, elle ne procède que du mécontentement des hommes ; elle n’est pas une levée de boucliers, mais l’acte d’individus qui s’élèvent, qui se redressent, sans s’inquiéter des institutions qui vont craquer sous leurs efforts ni de celles qui pourront en résulter. La révolution  avait en vue un \emph{régime} nouveau, l’insurrection nous mène à ne plus nous \emph{laisser régir} mais à nous régir nous-mêmes et elle ne fonde pas de brillantes espérances sur les a institutions à venir ». Elle est une lutte contre ce qui est établi, en ce sens que, lorsqu’elle réussit, ce qui est établi s’écroule tout seul. Elle est mon effort pour me dégager du présent qui m’opprime ; et dès que je l’ai abandonné, ce présent est mort et tombe en décomposition.\par
En somme, mon but n’étant pas de renverser ce qui est, mais de m’élever au-dessus de lui, mes intentions et mes actes n’ont rien de politique ni de social ; n’ayant d’autre objet que moi et mon individualité, ils sont \emph{égoïstes}.\par
La révolution ordonne d’instituer, d’instaurer, l’insurrection veut qu’on \emph{se soulève} ou qu’on \emph{s élève.}\par
Le choix d’une constitution, tel était le problème qui préoccupait les cerveaux révolutionnaires ; toute l’histoire politique de la Révolution est remplie par des luttes constitutionnelles et des questions constitutionnelles ; de même, que les génies du Socialisme se sont montrés étonnamment féconds en institutions sociales (phalanstères, etc.). C’est au contraire à s’affranchir de toute constitution que tend l’insurgé\footnote{ \noindent Pour me garantir contre toute poursuite criminelle, je ferai, par surcroît de précaution, expressément remarquer que je prends le mot « insurrection » dans son sens étymologique et non dans l’acception restreinte sur laquelle sont suspendues les foudres du code pénal.
 }.\par
Je cherchais une comparaison afin de rendre plus clair ce que je viens de dire, et voici que ma pensée se reporte aux premiers temps de la fondation du Christianisme.\par
Dans le camp libéral, on reproche aux premiers Chrétiens d’avoir prêché l’obéissance aux lois païennes existantes, d’avoir prescrit de reconnaître l’autorité païenne, et d’avoir franchement ordonné de « rendre à César ce qui est à César ». Quel soulèvement pourtant à  ce moment contre la domination romaine, combien les Juifs, combien les Romains eux-mêmes se montraient séditieux envers le pouvoir qui régissait le monde, en un mot, combien général était le « mécontentement politique » ! Mais les Chrétiens ne voulurent pas s’en apercevoir, ni s’associer aux « tendances libérales de l’époque ». Les passions politiques étaient alors tellement surexcitées, que, comme on le voit dans les Evangiles, on ne crut pas pouvoir accuser avec plus de succès le fondateur du Christianisme qu’en lui imputant des « machinations politiques » ; les mêmes Evangiles nous apprennent pourtant que personne ne s’intéressait moins que lui aux menées politiques ambiantes. Pourquoi donc ne fut-il pas un révolutionnaire, ou un démagogue, comme les Juifs auraient voulu le faire croire ? Pourquoi ne fut-il pas un libéral ? Parce qu’il n’attendait pas le salut du remaniement des institutions, et que toute la boutique gouvernementale et administrative lui était totalement indifférente. Il n’était pas un révolutionnaire, comme le fut par exemple César, mais un insurgé ; il ne cherchait pas à renverser un gouvernement, mais à \emph{se relever lui-même.} Aussi s’en tenait-il à sa maxime : « Soyez prudents comme les serpents », dont le « rendez à César ce qui appartient à César » n’était que l’application à un cas spécial. En effet, il ne faisait pas une campagne libérale ou politique contre l’autorité établie, mais il voulait, sans s’inquiéter de cette autorité ni s’en laisser troubler, suivre \emph{sa propre voie}. Les ennemis du gouvernement ne lui étaient pas moins indifférents que le gouvernement lui-même, car de part et d’autre on ne comprenait pas ce qu’il voulait, et il lui suffisait de se tenir, avec la prudence du serpent, aussi loin que possible des uns et des autres. Mais, sans être un séditieux, un démagogue ou un révolutionnaire, il n’en fut pas moins, comme chacun des Chrétiens primitifs, un \emph{insurgé}, s’élevant au-dessus de tout ce que le gouvernement  et ses adversaires tenaient pour auguste, s’affranchissant de tous les liens qui entravaient les uns et les autres, et détruisant en même temps les sources de la vie du monde païen tout entier, devenu du reste incapable de maintenir dans son éclat le système établi. C’est précisément parce qu’il ne visait pas au renversement de l’ordre établi qu’il en fut le plus mortel ennemi et le véritable destructeur. Car il le mura dans son tombeau et, tranquille, sans un regard pour les vaincus, il éleva \emph{son} temple à lui, sans prêter l’oreille aux cris de douleur de ceux qu’il avait ensevelis sous leurs ruines.\par
Et maintenant, ce qui est arrivé au monde païen arrivera-t-il au monde chrétien ? Une révolution ne conduira certainement pas au but, si d’abord une insurrection ne s’est accomplie.\par
A quoi tendent mes relations avec le monde ? Je veux en jouir ; il faut pour cela qu’il soit ma propriété, et je veux donc le conquérir. Je ne veux pas la liberté des hommes, je ne veux pas l’égalité des hommes, je ne veux que ma puissance sur les hommes ; je veux qu’ils soient ma propriété, c’est-à-dire qu’ils servent à \emph{ma jouissance}. Et s’ils s’opposent à mes désirs, hé bien ? le droit de vie et de mort que se sont réservé l’Eglise et l’Etat, je déclare que lui aussi — est à moi.\par
Flétrissez cette veuve d’officier qui, durant la retraite de Russie, ayant eu la jambe emportée par un boulet, défit sa jarretière, étrangla son enfant, puis se coucha pour mourir à côté du cadavre ; flétrissez la mémoire de cette mère infanticide. Qui sait, si cet enfant était resté en vie, quels « services il eût pu rendre » au monde ? Et la mère le tua, parce qu’elle voulait mourir \emph{contente} et tranquille ! Cette histoire émeut peut-être encore votre sentimentalité, mais vous n’en savez rien tirer d’autre. Soit. Pour moi, je veux montrer par cet exemple que c’est \emph{mon contentement} qui décide de mes rapports avec les hommes et qu’il n’y a pas  d’accès d’humilité qui puisse me faire renoncer au pouvoir de vie et de mort.\par
Quant aux « devoirs sociaux » en général, ce n’est pas à un tiers à fixer ma position vis-à-vis des autres ; ce n’est par conséquent ni Dieu ni l’humanité qui peuvent déterminer les rapports entre moi et les hommes : c’est Moi qui prends position. Cela revient à dire plus nettement : Je n’ai pas de devoirs envers les autres, pas plus que je n’ai de devoirs envers moi (par exemple le devoir de la conservation, opposé au suicide) à moins que je ne « me » distingue Moi-même (mon âme immortelle de mon existence terrestre, etc.).\par
Je ne m’\emph{humilie} plus devant aucune puissance, je reconnais que toute puissance n’est que la mienne, et que je dois l’abattre dès qu’elle menace de devenir opposée ou supérieure à Moi. Toute puissance ne peut être considérée que comme un de \emph{mes moyens }d’arriver à mes fins, de même qu’un chien de chasse est une puissance à notre service contre l’animal sauvage, mais que nous le tuons s’il vient à nous attaquer nous-mêmes. Toutes les puissances qui furent mes maîtresses, je les rabaisse donc au rôle de mes servantes. Les idoles n’existent que par Moi : il suffit que je ne les crée plus pour qu’elles ne soient plus ; il n’y a de « puissances supérieures » que parce que je les élève et me mets au-dessous d’elles.\par
Voici donc en quoi consistent mes rapports avec le monde : Je ne fais plus rien pour lui « pour l’amour de Dieu », je ne fais plus rien « pour l’amour de l’Homme », mais ce que je fais, je le fais « pour l’amour de Moi ». Ainsi seulement le monde peut me satisfaire, tandis que pour celui qui le considère au point de vue religieux (avec lequel, notez-le bien, je confonds le point de vue moral et humain) le monde reste un « pieux désir » \emph{(pium desiderium}), c’est-à-dire un au-delà, un inaccessible. Tels sont l’universelle félicité, le monde moral où régneraient l’amour universel,  la paix éternelle, l’extinction de l’égoïsme, etc.\par
« Rien dans ce monde n’est parfait ! » — Sur cette triste parole, les bons s’en détournent et se réfugient près de Dieu dans leur oratoire, ou dans l’orgueilleux sanctuaire de leur « conscience. » Mais nous, nous demeurons dans ce monde « imparfait » : tel qu’il est, nous savons le faire servir à notre jouissance.\par
Mes relations avec le monde consistent en ce que je jouis de lui et l’emploie à ma jouissance. \emph{Relations }équivaut à \emph{jouissance du monde}, et cela rentre dans ma — jouissance de Moi.
\subsubsection[{B.II.3. Ma jouissance de Moi}]{B.II.3. Ma jouissance de Moi}
\noindent Nous sommes au tournant d’une époque. Le monde n’a jusqu’à présent songé qu’à conquérir la vie, son unique souci a été de — \emph{vivre}. Que toute activité tende vers les choses d’ici-bas ou vers l’au-delà, vers la vie temporelle ou vers l’éternelle, qu’on aspire au « pain quotidien » (« donnez-nous notre pain quotidien ») ou au « pain sacré » (« le véritable pain du Ciel », « le pain de Dieu qui est descendu du ciel et qui donne \emph{la vie} au monde », « le pain de vie », Jean, VI, 32, 33, 48), que l’on se préoccupe de la « chère vie » ou de la « vie éternelle », le but de tout effort, l’objet de toute sollicitude ne change pas : dans l’un comme dans l’autre cas, ce qu’on cherche est toujours \emph{la Vie}. Les tendances modernes témoignent-elles d’un autre souci ? On veut que les besoins de la vie ne soient plus un tourment pour personne, et l’on enseigne d’ailleurs que l’homme doit s’occuper [{\corr de}] ce monde-ci et vivre sa vie réelle sans vain souci de l’au-delà.\par
Reprenons la question à un autre point de vue. Celui dont l’unique souci est de \emph{vivre} ne peut guère  songer à \emph{jouir} de la vie. Tant que sa vie est encore en question, tant qu’il peut encore avoir à trembler pour elle, il ne peut consacrer toutes ses forces à se servir de la vie, c’est-à-dire à en jouir. Mais comment en jouir ? En l’usant, comme on brûle la chandelle qu’on emploie. On use de la vie et de soi-même en la consumant et en se consumant. \emph{Jouir} de la vie, c’est la \emph{dévorer} et la \emph{détruire}.\par
Eh bien, — que faisons-nous ? Nous cherchons la \emph{jouissance} de la vie. Et que faisait le monde religieux ? Il cherchait \emph{la vie.} « En quoi consiste la vraie vie, la vie bienheureuse, etc. ? Comment y parvenir ? Que doit faire l’homme et que doit-il être pour être un véritable vivant ? Quels devoirs lui impose cette vocation ? » Ces questions et d’autres pareilles indiquent que ceux qui les posent en sont encore à \emph{se} chercher, à chercher leur vrai sens, le sens que leur vie doit avoir pour être vraie. « Ce que je suis n’est qu’un peu d’ombre et d’écume, ce que je serai sera mon vrai moi ! » Poursuivre ce moi, le préparer, le réaliser, telle est la lourde tâche des mortels ; ils ne meurent que pour \emph{ressusciter}, ils ne vivent que pour mourir et pour trouver la vraie vie.\par
Ce n’est que quand je suis sûr de moi et quand je ne me cherche plus que je suis vraiment ma propriété. Alors je me possède, et c’est pourquoi je m’emploie et je jouis de mot. Mais tant que je crois au contraire avoir encore à découvrir mon vrai moi, tant que je pense devoir faire en sorte que celui qui vit en moi ne soit pas Moi, mais soit le Chrétien ou quelque autre moi spirituel, c’est-à-dire quelque fantôme tel que l’Homme, l’essence de l’Homme, etc., il m’est à jamais interdit de jouir de moi.\par
Il y a un abîme entre ces deux conceptions : d’après l’ancienne je suis mon but, d’après la nouvelle je suis mon point de départ ; d’après l’une je me cherche, d’après l’autre je me possède et je fais de moi ce que je ferais de toute autre de mes propriétés, — je jouis  de moi selon mon bon plaisir. Je ne tremble plus pour ma vie, je la « prodigue ».\par
La question, désormais, n’est plus de savoir comment conquérir la vie, mais comment la dépenser et en jouir ; il ne s’agit plus de faire fleurir en moi le vrai moi, mais de faire ma vendange et de consommer ma vie.\par
Qu’est-ce que l’Idéal, sinon le moi toujours cherché et jamais atteint ? Vous vous cherchez ? C’est donc que vous ne vous possédez pas encore ! Vous vous demandez ce que vous \emph{devez} être ? Vous ne l’\emph{êtes} donc pas ! Votre vie n’est qu’une longue et passionnée attente ; pendant des siècles on a soupiré vers l’avenir et vécu d’\emph{espérance}. C’est tout autre chose de vivre de — \emph{jouissance.}\par
Est-ce à ceux là seuls que l’on dit pieux que s’adressent mes paroles ? Nullement, elles s’appliquent à tous ceux qui appartiennent à cette époque finissante, et même à ses joyeux vivants. Pour eux aussi un dimanche succède aux jours ouvrables et les tracas de la vie sont suivis du rêve d’un monde meilleur, d’un bonheur universel, d’un Idéal en un mot. Mais les philosophes au moins doivent, direz-vous, être opposés aux dévots ! Eux ? Ont-ils jamais pensé à autre chose qu’à l’idéal et ont-ils jamais eu en vue autre chose que le moi absolu ? Partout attente, aspirations, partout de lointaines chimères, de longs espoirs et rien de plus. Faites-moi le plaisir d’appeler ça du romantisme !\par
Pour triompher de l’aspiration à la vie, la \emph{jouissance de la vie} doit la vaincre sous sa double forme, écraser aussi bien la détresse spirituelle que la détresse temporelle, et exterminer à la fois la soif de l’idéal et la faim du pain quotidien. Celui qui doit user sa vie à la conserver ne peut en jouir, et celui qui la cherche ne l’a pas et ne peut pas non plus en jouir : tous deux sont pauvres, mais — « bienheureux les pauvres ! »\par
Les affamés de vraie vie n’ont plus aucun pouvoir  sur leur vie présente qu’ils doivent consacrer à la conquête de la vraie vie et sacrifier à l’accomplissement de cette tâche et de ce devoir. La servitude de l’existence terrestre, tout entière subordonnée à l’existence céleste qu’ils attendent, est évidente chez les esprits religieux qui escomptent une vie future et ne voient dans la vie ici-bas qu’un simple stage ; mais il serait très faux de croire à moins de renoncement chez ceux qui se sont en apparence le plus affranchi des dogmes. Comprenez donc que la « vie vraie » a un sens bien plus étendu que votre « vie céleste » ! Et, pour en venir immédiatement à la conception libérale de la vie, la « vie vraie » n’est-elle pas « humaine » et « vraiment humaine » ? Faut-il se donner tant de peines pour parvenir à cette vie humaine, ou le premier venu la vit-il dès l’instant où il commence à respirer ? Est-elle pour chacun le présent, ce qu’il a et ce qu’il est actuellement, ou doit-il y tendre comme à une vie future qu’il ne possédera qu’après s’être « lavé de la souillure de l’égoïsme » ? A ce compte, la vie n’est que la conquête de la vie, on ne vit que pour faire vivre en soi l’essence de l’Homme et pour l’amour de cette essence. On n’a sa vie que pour en créer une « véritable vie », purifiée de tout égoïsme. Et voilà pourquoi on hésite à l’employer à sa guise : elle a son emploi, son but, et on ne peut l’en détourner.\par
Bref, on a une \emph{vocation}, un devoir ; on a, par sa vie, à réaliser, à accomplir \emph{quelque chose ;} ce « quelque chose » en vue duquel la vie n’est qu’un moyen et un instrument a plus d’importance qu’elle, et on la lui \emph{doit.} On a un dieu qui réclame des \emph{victimes vivantes.} Les sacrifices humains n’ont perdu à la longue que leurs formes barbares, ils n’ont pas disparu ; à chaque instant des criminels sont offerts en holocauste à la Justice, et nous, « pauvres pécheurs », nous nous immolons nous-mêmes sur l’autel de l’« essence humaine », de « l’Homme », de l’ « Humanité », des idoles ou des dieux quel que soit le nom qu’on leur donne.\par
 Ayant un créancier auquel nous devons notre vie, nous n’avons aucun droit de la dépenser pour nous.\par
Les tendances conservatrices du Christianisme ne permettent pas au Chrétien de songer à la mort autrement qu’avec l’intention de lui arracher son aiguillon et de se survivre bel et bien. Le Chrétien consent à ce que tout arrive, il prend ses maux en patience, du moment qu’il peut — le Juif ! compter qu’il se rattrapera au ciel et y touchera de gros intérêts. Il ne lui est pas permis de se tuer, il ne peut que se — conserver et travailler à « se préparer une place pour plus tard ». La perpétuité, le « triomphe sur la mort », voilà ce qui lui est à cœur : « La dernière ennemie qui sera vaincue, c’est la mort »\footnote{ \noindent 1\textsuperscript{re} Corinth., {\scshape xv}, 26.
 }, « Jésus-Christ a brisé la puissance de la mort, et a mis en lumière par l’Evangile la vie et l’\emph{incorruptibilité} »\footnote{ \noindent 2\textsuperscript{e} Timoth., {\scshape i}, 10.
 }. — « Incorruptibilité », stabilité !\par
L’homme moral veut le Bien, le Juste, etc. ; s’il use des moyens qui conduisent à ce but et y conduisent réellement, ces moyens ne sont pas pour cela les \emph{siens}, mais sont ceux du Bien, du Juste, etc. Ces moyens ne sont jamais immoraux, car le but auquel ils permettent d’atteindre est bon : la fin justifie les moyens ; cette maxime passe pour jésuitique, bien qu’elle soit strictement « morale ». L’homme moral est le serviteur d’un but ou d’une idée, il se fait l’\emph{instrument} du Bien comme l’homme pieux se fait gloire d’être l’ouvrier, l’outil de Dieu.\par
Les commandements de la Morale ordonnent comme étant bien d’attendre l’heure de la mort ; se donner à soi-même la mort est immoral et mauvais : le suicide n’a aucun pardon à attendre devant le tribunal de la moralité. L’homme religieux le condamnait parce que « ce n’est pas toi qui t’es donné la vie, c’est Dieu, et lui seul peut te le reprendre » (comme si, à ce  compte, ce n’était pas aussi bien Dieu qui me la reprend lorsque je me tue que lorsqu’une tuile ou une balle ennemie me cassent la tête : c’est lui aussi qui a éveillé en moi la résolution de mourir !) L’homme moral, de son côté, le condamne parce que je dois ma vie à la Patrie, etc., « et que je ne sais pas si de ma vie n’eût pas pu résulter encore quelque bien ». Si je me tue, le Bien perd naturellement en moi un instrument, comme Seigneur compte, moi mort, un ouvrier de moins à sa vigne. Si je fus immoral, le Bien bénéficiera de mon \emph{amélioration ;} si je fus impie, Dieu se réjouira de ma \emph{contrition}. Le suicide est aussi criminel envers Dieu qu’envers la vertu. Toi qui t’ôtes la vie, tu oublies Dieu si tu étais religieux et tu oublies le devoir si tu étais moral. La mort d’Emilia Galotti est-elle justifiable au point de vue de la moralité (on admet que cette mort est un suicide, et le fait est que c’en est bien un) ? On s’est mis martel en tête pour en décider. Etre assez enragée de chasteté, ce bien moral, pour lui sacrifier sa vie est certainement moral ; mais, en revanche, ne pas avoir assez de confiance en soi-même pour oser affronter les pièges de la chair est immoral. Le conflit tragique qui fait le fond de tout drame moral repose généralement sur une antinomie de ce genre ; il faut penser et sentir moralement pour être capable de s’y intéresser.\par
Tout ce que l’on peut dire au nom de la morale et de la piété à propos du suicide n’est pas moins vrai si l’on en appelle à l’humanité, attendu que l’on doit également sa vie à l’Homme, à l’humanité, au genre humain. C’est seulement quand je ne me reconnais d’obligations envers personne que la conservation de ma vie est — mon affaire, « Un saut du haut de ce pont me fait libre ! »\par
Nous devons à l’Etre quelqu’il soit que nous avons à faire vivre en nous non seulement de conserver la vie dont nous sommes les dépositaires, mais en outre de ne pas employer cette vie à \emph{notre} guise, de la  régler sur lui et de la lui conformer. Tout en moi, penser, sentir, vouloir, tous mes actes, tous mes efforts sont à lui.\par
L’idée que nous avons de cet Etre détermine ce qui lui est conforme. Mais cette idée, de combien de façons l’a-t-on conçue ? et cet Etre, sous combien de formes se l’est-on représenté ? Le Mahométan croit que l’Etre suprême exige de lui une chose et le Chrétien croit qu’il en réclame une toute autre : quel aspect différent la vie doit leur présenter ! Mais tous sont du moins unanimes à croire que c’est à l’Etre suprême à diriger leur vie.\par
Je ne m’arrêterai pas plus longtemps aux dévots qui ont en Dieu un guide et en sa parole un fil conducteur ; je ne les ai cités que pour mémoire, ils appartiennent à une faune éteinte et leur immobilité est celle des pétrifications. Ce ne sont plus aujourd’hui les pieux mais bien les libéraux qui ont le verbe haut, et la piété elle-même ne peut plus se dispenser de rougir quelque peu ses joues blêmes de fard libéral. Les libéraux n’honorent point en Dieu leur guide et ne suspendent point leur vie au fil conducteur de la parole divine ; ils se guident sur l’Homme, et ce n’est pas à une vie « divine » mais à une vie « humaine » qu’ils aspirent.\par
L’Etre suprême du libéral est « l’Homme » ; l’Homme est son mentor et l’humanité est son catéchisme. Dieu est Esprit, mais l’Homme est « l’Esprit parfait », le résultat final de la longue chasse à l’Esprit à laquelle on se livra en « sondant les profondeurs de la divinité », c’est-à-dire les profondeurs de l’Esprit.\par
Chacun de tes traits doit être humain ; toi-même tu dois l’être de la nuque aux talons, intérieurement comme extérieurement : car l’humanité est ta \emph{vocation}.\par
Vocation — destination — devoir ! —\par
Ce qu’on peut être, on l’est. La défaveur des circonstances  pourra empêcher celui qui naquit poète d’être le premier de son temps, et ne pas lui permettre de produire des chefs-d’œuvre en lui interdisant les longues mais indispensables études préliminaires ; mais il fera des vers, qu’il soit valet de ferme ou qu’il ait la chance de vivre à la cour de Weimar. Le musicien fera de la musique, dût-il, faute d’instrument, souffler dans un roseau. Une tête philosophique roulera des problèmes, qu’elle orne les épaules d’un philosophe d’université ou d’un philosophe de village. Enfin l’imbécile, qui peut être en même temps un « malin » (les deux vont très bien ensemble, quiconque a fréquenté les écoles en retrouvera dans sa mémoire plusieurs exemples, s’il passe en revue ses anciens condisciples), l’imbécile, dis-je, restera toujours un imbécile, soit qu’on l’ait dressé et exercé à être chef de bureau, ou à cirer les bottes du dit chef. Les cerveaux obtus forment la classe humaine incontestablement la plus nombreuse. Mais pourquoi n’y aurait-il pas dans l’espèce humaine les mêmes différences qu’il est impossible de méconnaître dans la première espèce animale venue ? On trouve partout des êtres plus ou moins bien doués.\par
Peu cependant sont assez obtus pour qu’on ne puisse leur insuffler quelques idées. Aussi considère-t-on ordinairement tous les hommes comme capables d’avoir de la religion. Ils sont, de plus, susceptibles d’être dans une certaine mesure dressés à d’autres idées, et on peut leur donner, par exemple, quelque compréhension musicale, une teinte de philosophie même. Ici le sacerdoce se lie à la religion, à la moralité, à la culture, à la science, etc., et les Communistes, par exemple, veulent par leur « école populaire » rendre tout accessible à tous. On soutient ordinairement que la « grande masse » ne pourrait se passer de religion ; les Communistes étendent cette affirmation, et disent que non seulement la « grande masse » mais tous sont appelés à tout.\par
 Il ne suffit pas d’avoir dressé la masse à la religion, il faut à présent la pétrir de « tout ce qui est humain ». Et le dressage devient toujours plus universel et plus étendu.\par
Pauvres êtres, qui pourriez être si heureux s’il vous était permis de gambader à votre guise ! Il faut que vous dansiez au son de la serinette des pédagogues et des montreurs d’ours et que vous appreniez à faire des tours dont vous n’eussiez jamais de la vie senti le besoin. Cela ne finit-il pas par vous révolter, de voir qu’on vous prend toujours pour autre chose que ce que vous voulez paraître ? Non ! Vous répétez mécaniquement la question qu’on vous a soufflée : « A quoi suis-je appelé ? Quel est mon \emph{devoir ?} » Et il suffit que vous posiez la question pour qu’aussitôt la réponse s’impose à vous : vous vous ordonnez ce que vous devez faire, vous vous tracez une vocation, ou vous vous donnez les ordres et vous vous imposez la vocation que l’Esprit a d’avance prescrits. Par rapport à la volonté, cela peut s’énoncer ainsi : Je veux ce que je dois.\par
Un homme n’est « appelé » à rien ; il n’a pas plus de « devoir » et de « vocation » que n’en ont une plante ou un animal. La fleur qui s’épanouit n’obéit pas à une « vocation », mais elle s’efforce de jouir du monde et de le consommer tant qu’elle peut, c’est-à-dire qu’elle puise autant de sucs de la terre, autant d’air de l’éther et autant de lumière du soleil qu’elle en peut absorber et contenir. L’oiseau ne vit pas pour remplir une vocation, mais il emploie ses forces le mieux possible, il attrape des insectes et chante à cœur joie. Les forces de la fleur et de l’oiseau sont faibles, comparées à celles d’un homme, et l’homme qui bande ses forces pour conquérir le monde l’étreint bien plus puissamment que ne le font la fleur et l’oiseau. Il n’a pas de vocation ou de mission à remplir, mais il a des forces, et ces forces se déploient, se manifestent où elles sont parce que, pour elles, être  c’est se manifester, et qu’elles ne peuvent pas plus rester inactives que ne le peut la vie, qui, si elle « s’arrêtait » une seconde, ne serait plus la vie. On pourrait donc crier à l’homme : emploie ta force ! Mais cet impératif impliquerait encore une idée de devoir là où il n’y en a pas l’ombre. Et d’ailleurs à quoi bon ce conseil ? Chacun le suit et agit, sans commencer par voir dans l’action un devoir : chacun déploie à chaque instant tout ce qu’il a de puissance. On dit bien a un vaincu qu’il aurait dû déployer plus de force ; mais on oublie que si, au moment de succomber, il avait eu le pouvoir de déployer ses forces (corporelles par exemple), il l’eût fait : il n’a eu peut-être qu’une minute de découragement, mais ce fut là, en somme, une minute d’impuissance. Les forces peuvent évidemment s’aiguiser et se multiplier, particulièrement par les bravades de l’ennemi ou par des exhortations amies ; mais là où elles restent sans effet, on peut être certain qu’elles manquaient. On peut faire jaillir des étincelles d’une pierre, mais sans le choc pas d’étincelle ; de même l’homme a besoin d’une « impulsion ».\par
Attendu donc que les forces se montrent toujours d’elles-mêmes actives, l’ordre de les mettre en œuvre serait superflu et vide de sens. Employer ses forces n’est pas la \emph{vocation} et le devoir de l’homme, mais son \emph{fait}, perpétuellement réel et actuel. Force n’est qu’un mot plus simple pour dire « manifestation de force ».\par
Cette rose est, depuis qu’elle existe, une véritable rose et ce rossignol est et a toujours été un véritable rossignol ; de même Moi : ce n’est pas seulement quand je remplis ma mission et me conforme à ma destination que je suis un « véritable homme » : j’en suis un, j’en ai toujours été et ne saurais cesser d’en être un. Mon premier vagissement fut le signe de vie d’un « véritable homme », les combats de ma vie sont les manifestations d’une force « vraiment humaine », et mon dernier soupir sera le dernier effort « de l’Homme ».\par
Le véritable homme n’est pas dans l’avenir, il n’est  pas un but, un idéal vers lequel on aspire ; mais il est ici, dans le présent, il existe réellement : quel que je sois, quoi que je sois, joyeux ou souffrant, enfant ou vieillard, dans la confiance ou dans le doute, dans le sommeil ou la veille, c’est Moi. Je suis le véritable homme.\par
Mais si je suis l’Homme, si j’ai réellement trouvé en Moi celui dont l’humanité religieuse faisait un but lointain, tout ce qui est « vraiment humain » est par là même \emph{ma propriété.} Tout ce qu’on attribuait à l’idée d’humanité m’appartient. Cette liberté de commerce, par exemple, que l’humanité est encore à espérer et que l’on remet à un avenir doré comme un rêve enchanté, je l’emporte comme ma propriété et je la pratique provisoirement sous la forme de la contrebande. Peu de contrebandiers, j’en conviens, pourraient interpréter ainsi leur conduite, mais l’instinct de l’égoïsme supplée à la conscience qui leur fait défaut. J’ai montré plus haut qu’il en va de même de la liberté de la presse.\par
Tout est à moi, aussi ressaisirai-je ce qui veut se soustraire à moi ; mais, avant tout, je me ressaisis, si une servitude quelconque m’a fait échapper à moi-même. Mais cela non plus n’est pas ma vocation, c’est ma conduite naturelle.\par
En somme, il y a donc une grande différence entre \emph{me} prendre pour point de départ ou pour point d’arrivée. Si je suis mon but, je ne me possède pas, je suis encore étranger à moi, je suis mon \emph{essence}, ma « véritable nature intime », et cette « essence vraie » prendra comme un fantôme mille noms et mille formes diverses pour se jouer de moi. Si je ne suis pas moi, c’est un autre (Dieu, le véritable Homme, le vrai dévot, l’homme raisonnable, l’homme libre, etc.) qui est moi, qui est mon moi.\par
Encore bien loin de moi, je fais de moi deux parts, dont l’une, celle qui n’est pas atteinte et que j’ai à accomplir, est la vraie. L’autre, la non-vraie, c’est-à-dire  la non spirituelle, doit être sacrifiée ; ce qu’il y a de vrai en moi, c’est-à-dire l’Esprit, doit être tout l’homme. Cela se traduit ainsi : « L’esprit est l’essentiel chez l’homme » ou « l’homme n’est Homme que par l’esprit ». On se précipite avidement pour saisir l’esprit, comme si on allait du même coup se saisir, et dans cette chasse éperdue au moi on perd de vue le moi que l’on est.\par
Dans cette poursuite furieuse d’un moi qu’on n’atteint jamais, on fait fi de la règle des sages qui conseillent de prendre les hommes comme ils sont ; on préfère les prendre comme ils devraient être, et, en conséquence, on galope sans trêve sur la piste de son « moi tel qu’il devrait être » et on « s’efforce de rendre tous les hommes également justes, estimables, moraux ou raisonnables »\footnote{ \noindent \emph{Der Kommunismus in der Schweiz}, p. 24.
 }.\par
Oui, « si les hommes étaient comme ils \emph{devraient }et comme ils \emph{pourraient} être, si tous les hommes étaient raisonnables, s’ils s’aimaient les uns les autres comme des frères », la vie serait un paradis ! — Eh mais, les hommes sont comme ils doivent être et comme ils peuvent être. Que doivent-ils être ? Ce qu’ils peuvent être et rien de plus ! Et que peuvent-ils être ? Rien de plus que ce qu’ils — peuvent, c’est-à-dire que ce qu’ils ont le pouvoir ou la force d’être. Mais cela, ils le sont réellement, attendu que ce qu’ils ne \emph{sont pas} ils ne \emph{sont pas capables} de l’être : car être capable de faire ou d’être veut dire faire ou être réellement. On n’est pas capable d’être ce qu’on n’est pas, on n’est pas capable de faire ce qu’on ne fait pas. Cet homme que la cataracte aveugle pourrait-il y voir ? Certainement, il suffirait qu’il fut opéré avec succès. Mais, pour le moment, il ne peut pas voir, parce qu’il ne voit pas. Possibilité et réalité sont inséparables. On ne peut pas faire ce qu’on ne fait pas, comme on ne fait pas ce qu’on ne peut pas faire.\par
 La singularité de cette proposition disparaît si l’on veut bien réfléchir que les mots « il est possible que... etc. » ne signifient au fond presque jamais autre chose que « je puis imaginer que... etc. », Par exemple : « Il est possible que tous les hommes vivent raisonnablement » veut dire « Je puis m’imaginer que... etc. ». Ma pensée ne peut faire, et par conséquent ne fait pas, que les hommes vivent raisonnablement, c’est là une chose qui ne dépend pas de moi mais d’eux ; la raison de tous les hommes n’est donc pour moi que pensable, elle ne m’est \emph{qu’intelligible ;} mais comme telle elle est en fait une \emph{réalité ;} si cette réalité prend le nom de possibilité, ce n’est que par rapport à ce que je ne \emph{puis} pas faire, c’est-à-dire à la raison des gens. A supposer que cela dépendît de toi, tous les hommes pourraient être raisonnables, car tu n’y vois aucun inconvénient et si loin même que s’étende ta pensée tu ne découvres peut-être rien qui s’y oppose : il en résulte qu’aucun obstacle ne s’oppose à la chose dans ta pensée : elle est pensable.\par
Mais les hommes ne sont pas tous raisonnables ; c’est donc sans doute qu’ils — ne peuvent pas l’être.\par
Lorsqu’une chose que l’on s’imaginait n’offrir aucune difficulté, être très possible etc, n’est pas ou n’arrive pas, on peut être certain qu’elle s’est heurtée à un obstacle et qu’elle est — impossible. Notre époque a son art, sa science, etc. ; il se peut que son art soit exécrable, mais pouvons-nous, dans ce cas, dire : Nous méritions d’en avoir un meilleur, et nous « aurions pu » en avoir un meilleur si nous l’avions voulu ? Nous avons tout juste autant d’art que nous pouvons en avoir ; notre art actuel est actuellement l’\emph{unique possible} et c’est pourquoi il est notre art réel.\par
Réduisez encore le sens du mot « possible » jusqu’à ce qu’il ne signifie finalement plus que « futur », et il sera encore l’équivalent de « réel ». Quand on dit par exemple : Il est possible que le soleil se lève demain,  — cela ne signifie rien de plus que : par rapport à aujourd’hui, demain est l’avenir réel ; car il est à peine besoin d’exprimer qu’un avenir n’est réellement « à venir » que s’il n’a pas encore paru.\par
A quoi bon, dites-vous, cette dissection microscopique d’un mot ? Ah ! si ce n’était pas derrière lui que se tient embusquée l’erreur qui a eu, depuis des siècles, le plus de conséquences, si ce petit mot « possible » n’était pas dans la cervelle des hommes le coin où se donnent rendez-vous tous les fantômes qui la hantent, nous ne nous serions guère inquiété de lui !\par
La pensée, nous l’avons montré plus haut, règne sur le monde possédé. Revenons à la possibilité, qui est un de ses lieutenants. Possible, disions-nous, n’est rien d’autre que pensable, intelligible, et d’innombrables victimes ont été sacrifiées à ce terrible \emph{intelligible}. Il est pensable que les hommes puissent être raisonnables, il est pensable qu’ils puissent reconnaître le Christ, pensable qu’ils puissent être inspirés par le Bien et être moraux, pensable qu’ils puissent se réfugier dans le giron de l’Eglise, qu’ils puissent ne rien faire, ne rien penser et ne rien dire qui mette l’Etat en péril, il est pensable encore qu’ils puissent être des sujets obéissants. Mais voyez où cela va nous mener : Tout cela étant pensable est possible, et cela étant possible aux hommes (c’est ici qu’est l’erreur : parce que ce \emph{m}’est intelligible, c’est possible \emph{aux hommes)} ils \emph{doivent} l’être ou doivent le faire, c’est leur \emph{vocation}. Et, enfin, il ne faut rien voir dans les hommes que leur vocation, il faut les regarder comme \emph{appelés} à quelque chose, et les tenir non pour « ce qu’ils sont » mais pour « ce qu’ils doivent être ».\par
Autre conséquence : Ce n’est pas l’individu qui est l’Homme ; l’Homme est une pensée, un idéal. L’individu n’est pas à l’Homme ce que l’enfance est à l’âge mur, mais ce qu’un point à la craie est au point mathématique,  ce qu’une créature finie est au créateur infini, ou, en termes plus modernes, ce que l’exemplaire est à l’espèce. D’où, le culte de l’Humanité « éternelle », « immortelle », à la gloire de laquelle \emph{(ad majorem humanitatis gloriam)} l’individu doit être prêt à tout sacrifier, convaincu que ce serait pour lui un « éternel honneur » d’avoir fait quelque chose pour l’ « esprit de l’humanité ».\par
Il en résulte que ceux qui \emph{pensent} gouvernent le monde tant que dure l’époque des prêtres et des pédagogues ; ce qu’ils pensent est possible et ce qui est possible doit être réalisé. Ils \emph{pensent} un idéal humain qui n’a provisoirement de réalité que dans leur pensée, mais ils pensent ensuite la possibilité de réaliser cet idéal, et il est incontestable que cette réalisation est réelle...ment pensable : c’est une — idée.\par
Il se peut qu’un Krummacher \emph{pense} que toi et moi, sommes encore capables de devenir bons chrétiens ; mais s’il s’avisait de nous « travailler » dans ce sens, nous lui ferions bientôt sentir que notre christianisation, encore que \emph{pensable}, est cependant \emph{impossible,} et s’il s’obstinait à nous assassiner de ses \emph{pensées} et de sa « bonne doctrine » dont nous n’avons que faire, il ne tarderait pas à se convaincre que nous n’avons que faire de devenir ce qu’il ne nous plaît pas d’être.\par
Et le raisonnement que nous résumions tantôt se poursuit, laissant loin derrière lui dévots et bigots : « Si tous les hommes étaient raisonnables, si tous pratiquaient la justice, si tous prenaient pour guide la charité, etc. ! » Raison, Justice, Charité leur sont présentées comme la vocation de l’homme, comme le but où doivent tendre ses efforts. Et que signifie être raisonnable ? Est-ce se raisonner soi-même, se comprendre ? Non, la Raison est un gros livre bourré d’articles de lois, tous braqués contre l’Egoïsme.\par
L’histoire n’a été jusqu’à présent que l’histoire de l’homme \emph{spirituel}. Après l’âge des sens a commencé  l’histoire proprement dite, c’est-à-dire l’âge de l’intelligence, du spirituel, du suprasensible, de l’idéal, du — non-sens. L’homme se met alors à vouloir être \emph{quelque chose}. Etre quoi ? Bon, beau, vrai, ou plus exactement moral, pieux, noble, etc. il veut faire de lui-même un « véritable homme » ; l’Homme est son but, son impératif, son devoir, sa destination, sa vocation, son — Idéal, l’Homme est pour lui un futur, un au-delà. Et s’il devient ce qu’il rêve, ce ne peut être que grâce à \emph{quelque chose}, qui s’appellera véracité, bonté, moralité, etc. Dès lors, il regarde de travers quiconque ne rend pas hommage au même « quelque chose », ne suit pas la même morale et n’a pas la même foi : il persécute les « dissidents, les hérétiques, les sectes, » etc.\par
Le mouton ne s’efforce pas de devenir un « vrai mouton, » ni le chien un « vrai chien » ; aucun animal ne prend son être pour un devoir, c’est-à-dire pour une idée qu’il doit réaliser. Il se réalise par là même qu’il vit sa vie, c’est-à-dire qu’il s’use et qu’il se détruit. Il ne demande pas à devenir quelque chose d’autre que ce qu’il est. Ce n’est pas que je veuille vous conseiller de ressembler aux animaux. Je ne le puis d’ailleurs pas, car vous exhorter à devenir des animaux serait vous proposer de nouveau une tâche, un idéal (« l’abeille peut t’en remontrer en application ») ; cela équivaudrait à souhaiter aux animaux de devenir hommes. Votre nature est, une fois pour toutes, humaine ; vous êtes des natures humaines, c’est-à-dire des hommes, et c’est justement parce que vous en êtes que vous n’avez plus besoin d’en devenir. Certains animaux aussi peuvent être « dressés », et un animal dressé exécute toutes sortes d’exercices qui ne lui sont pas naturels. Mais si le dressage rend le chien plus utile ou plus agréable pour nous, il n’en tire, lui, aucun profit ; une fois chien savant, il ne vaut pas plus \emph{pour lui-même} qu’un chien naturel.\par
On s’efforce, et la mode n’en est pas nouvelle, de faire des hommes des êtres moraux, raisonnables,  pieux, humains, etc., c’est-à-dire de les dresser. Mais ces tentatives se brisent contre l’incoercible individualité de l’égoïste. Ceux qu’on a soumis à cette discipline n’atteignent jamais leur idéal ; ils ne professent \emph{qu’en paroles} les sublimes doctrines, et se bornent à taire des \emph{professions de foi ; pratiquement}, ils doivent bien confesser qu’ils sont tous des « pécheurs » et qu’ils restent loin en dessous de leur idéal ; ils sont de « faibles hommes » et ils se consolent en ayant conscience de la « faiblesse humaine ».\par
Il en va tout autrement si tu ne poursuis pas un idéal comme ta « destination », mais que tu te consumes comme le temps consume tout. La destruction n’est pas ta « destination » car elle est le présent.\par
Il est parfaitement vrai que la culture et la religiosité des hommes les ont libérés, mais elles ne les ont délivrés d’un maître que pour les soumettre à un autre. La religion m’a appris à réfréner mes désirs, les artifices que la science met à mon service me permettent de vaincre la résistance du monde, et je ne reconnais même plus en aucun homme mon maître : « je ne suis le serviteur de personne ». Seulement, il vaut mieux obéir à Dieu qu’aux hommes. Plus je suis affranchi des impulsions déraisonnables de l’instinct et plus docilement j’obéis à la maîtresse — \emph{Raison}. J’ai gagné la « liberté spirituelle », la « liberté de l’Esprit », et je suis devenu par là même l’esclave de l’Esprit. L’Esprit me commande, la Raison me guide ; ils me conduisent et me gouvernent, et les « raisonnables », les « serviteurs de l’Esprit » sont leurs ministres. Mais si je ne suis pas chair, je ne suis pas non plus esprit. La liberté de l’Esprit est ma servitude, parce que je suis plus que chair et plus qu’esprit.\par
La culture m’a rendu \emph{puissant,} cela ne souffre non plus aucun doute. Elle m’a donné un pouvoir sur tout ce qui est force, aussi bien sur les impulsions de ma nature que sur les assauts et les violences du monde  extérieur. Je sais que rien ne m’oblige à me laisser contraindre par mes désirs, mes appétits et mes passions, et la culture m’a donné la force de les vaincre : je suis leur — \emph{maître}. De même, je suis, grâce aux sciences et aux arts, le \emph{maître} du monde rebelle ; la terre et la mer sont sous mes ordres et les étoiles même doivent me rendre des comptes. C’est l’Esprit qui m’a donné cet empire. — Mais sur l’Esprit lui-même je ne puis rien. La religion (culture) m’a bien enseigné le moyen de devenir « le vainqueur du monde » mais elle ne m’a pas appris à vaincre \emph{Dieu,} car Dieu « est l’Esprit ». Cet Esprit sur lequel je n’ai aucun pouvoir peut prendre les formes les plus diverses, il peut s’appeler Dieu ou s’appeler Esprit du peuple, Etat, Famille, Raison, ou encore Liberté, Humanité, Homme.\par
J’accepte avec reconnaissance ce que les siècles de culture m’ont acquis ; je ne veux rien rejeter ou abandonner : \emph{Je} n’ai pas vécu en vain. Ils ont découvert que j’ai un \emph{pouvoir} sur ma nature et que je ne suis pas forcé d’être l’esclave de mes appétits, et c’est là un résultat appréciable que je ne dois pas laisser se perdre. Ils ont découvert que je puis, grâce aux moyens que me fournit la culture, dompter le monde, et cette découverte fut achetée trop cher pour que je puisse l’oublier. Mais je veux plus encore.\par
On se demande ce que l’homme peut devenir, ce qu’il peut accomplir et quels biens il peut acquérir, et de celui de ces biens qu’on juge plus grand on me fait une vocation. Comme si tout m’était possible !\par
Lorsqu’on voit quelqu’un que consume un désir, une passion, etc. (par exemple l’esprit de lucre, la jalousie, etc.), on se prend à souhaiter de le délivrer de cette obsession et de l’aider à « se vaincre ». « Nous voulons faire de lui un homme ! » Ce serait fort beau, si une autre possession ne prenait pas immédiatement la place que vient de vider l’ancienne. Sitôt la cupidité exorcisée, on jette sa victime dans  les bras de la piété, de l’humanité ou de quel-qu’autre principe, et on lui fournit de nouveau un \emph{point d’appui moral} fixe.\par
Cet échange d’un point d’appui inférieur contre un point d’appui élevé s’exprime en disant : il ne faut pas tourner ses regards vers ce qui passe, mais vers ce qui ne passe pas, non vers le temporel, mais vers l’éternel, l’absolu, le divin, le pur humain, — le \emph{spirituel}.\par
On s’aperçut bientôt qu’il n’est pas indifférent de suspendre son cœur n’importe où et de s’éprendre de n’importe quoi ; on reconnut l’importance de l’\emph{objet. }Un objet élevé au-dessus de la particularité des choses est l’\emph{essence} des choses ; leur essence, en effet, est seulement ce qu’il y a de pensable en elles et n’existe que pour l’homme pensant. Ne dirige donc plus tes \emph{sens} sur la chose, mais dirige tes \emph{pensées} sur l’essence. « Bienheureux ceux qui, ne voyant pas, croient », autrement dit : bienheureux les \emph{pensants,} car eux seuls ont à faire à l’invisible et y croient. Mais un objet de penser qui passa pendant des siècles pour un critérium essentiel, finit tôt ou tard par « ne plus valoir la peine d’en parler ». On s’en rendit compte, mais on ne cessa jamais d’accorder à l’objet une importance en soi et une valeur absolue ; comme si l’essentiel n’était pas, pour l’enfant sa poupée, et pour le Turc le Coran. Tant que l’important pour moi n’est pas uniquement Moi, peu importe l’objet que je tiens pour « essentiel » : seule, la petitesse ou la grandeur de mon \emph{crime} envers lui a une valeur. La profondeur de mon attachement et de mon dévouement témoigne de ma servitude et la profondeur de mon péché donne la mesure de mon individualité.\par
Mais il faut finalement savoir tout « chasser de sa pensée » si l’on veut pouvoir — s’endormir. Rien ne doit nous occuper, dont nous ne nous occupons pas : l’ambitieux ne peut se défaire de ses projets d’ambition et celui qui craint Dieu ne peut détacher sa pensée de Dieu ; manie et obsession sont jumelles.\par
 Réaliser son essence ou vivre conformément à sa notion est ce que le croyant en Dieu appelle « être pieux » et ce qu’un croyant en l’Homme appelle « vivre humainement » ; ce but, seul l’homme sensuel ou le pécheur peut se le proposer, tant qu’il a encore le choix, le choix redoutable, entre la joie des sens et la paix de l’âme, tant qu’il est un « pauvre pécheur ». Le Chrétien n’est qu’un homme sensuel qui, connaissant la sainteté et ayant conscience de la violer, se regarde comme un pauvre pécheur : la sensualité conçue comme « iniquité » fait le fond de la conscience chrétienne et le Chrétien même. Nos modernes ne disent plus « le péché » et « l’iniquité », mais « l’égoïsme », « l’amour de soi », « l’intérêt personnel », etc. ; entre leurs mains le Diable a changé de peau et est devenu l’ « Inhumain » ou l’ « Egoïste » ; mais cela les empêche-t-il d’être chrétiens ? Le vieux dualisme du Bien et du Mal ne reste-t-il pas debout ? N’y a-t-il plus au-dessus de nous un juge : l’Homme ? N’est-il plus de vocation ? Et « faire de soi un Homme[{\corr  »}], comment appelez-vous ça ? Je le sais, vous ne dites plus vocation, vous dites « tâche », ou encore « devoir », et ce changement de nom est très juste, car l’Homme n’est pas comme Dieu une personne qui peut « appeler » \emph{(vocare)}, — mais, le nom mis à part, cela ne revient-il pas exactement au même ?\par

\asterism

\noindent Chacun de nous est en rapport avec les objets et se comporte envers eux différemment. Prenons comme exemple ce livre avec lequel des millions d’hommes ont été en rapport depuis bientôt vingt siècles : la Bible. Qu’a-t-elle été pour chacun d’eux ? \emph{Ce qu’il en a fait}, et rien d’autre. Elle n’est rien pour celui qui n’en fait rien, pour celui qui en use comme d’un amulette, elle a uniquement la valeur et la signification d’un charme, pour l’enfant qui joue avec elle, elle est un jouet, etc.\par
 Mais le Christianisme prétend que la Bible doit être pour tous la \emph{même} chose, c’est-à-dire ce qu’elle est pour lui : les « Livres Saints » ou la « Sainte Ecriture ». Cela revient à prétendre que le point de vue du Chrétien doit être celui des autres hommes et que personne ne peut avoir avec l’objet en question d’autres rapports que le Chrétien. Le rapport perd ainsi toute valeur individuelle ; une certaine opinion se substitue à la mienne, devient définitive et s’implante comme la \emph{vraie} et la « seule vraie ». Avec la liberté de faire de la Bible ce qu’il me plaît, toute liberté d’agir en général se voit entravée et est remplacée par la contrainte d’une façon de voir et de juger obligatoire. Celui qui émet le jugement que la Bible est une longue erreur de l’humanité porte un jugement — \emph{criminel.}\par
En réalité, l’enfant qui la met en pièces ou qui joue avec elle et l’Inca Atahualpaf, qui y applique l’oreille et la rejette avec une moue de dédain parce qu’elle reste muette, émettent sur la Bible un jugement aussi légitime que le prêtre qui prise en elle la « parole de Dieu » ou que la critique qui la traite comme un monument de la civilisation hébraïque. Car nous manions les choses selon notre \emph{bon plaisir} et notre \emph{caprice ;} nous en usons comme il nous \emph{plaît,} ou, plus exactement, comme nous \emph{pouvons}. D’où vient que les prêtres jettent de hauts cris lorsqu’ils voient Hégel et les théologiens spéculatifs extraire de la Bible des pensées spéculatives ? De ce qu’eux-mêmes traitent ces textes à leur guise et « en font un usage arbitraire ».\par
Rien ne plaît tant au philosophe que de dénicher en tout une « Idée », et rien ne va au dévot comme de mettre tout en œuvre (la vénération de la Bible, par exemple) pour se faire de Dieu un ami. Nous faisons tous preuve du même arbitraire dans notre commerce avec les choses et nous les traitons comme il nous plaît : aussi ne rencontrons-nous nulle part une aussi pesante tyrannie, autant de violences terribles  et d’oppression stupide que dans le domaine de notre — \emph{propre arbitraire.} Mais si nous agissons à notre guise en faisant ceci ou cela des objets sacrés, comment pourrions-nous reprocher à la prêtraille d’agir, elle aussi, à sa guise, et lui en vouloir de ce qu’elle nous juge \emph{à sa façon}, c’est-à-dire dignes du bûcher ou d’un autre châtiment, — de la censure par exemple ?\par
Ce qu’un homme est, les choses le sont à ses yeux, « le monde te voit du même œil dont tu le contemples ». D’où, immédiatement, ce sage conseil : tu ne dois le regarder que d’un œil « juste et impartial ». (Comme si l’enfant ne regardait pas la Bible avec justice et impartialité quand il s’en fait un jouet) ! Feuerbach, entre autres, nous donne ce prudent avis. Voir les choses justement, c’est tout bonnement en faire ce qu’on \emph{veut} (par choses, j’entends ici tous les objets en général : Dieu, nos confrères en humanité, une maîtresse, un livre, un animal, etc) ; ce qu’il faut mettre en première ligne, ce n’est pas les choses et leur aspect mais Moi et ma volonté. On \emph{veut} des choses extraire des pensées, on \emph{veut} découvrir de la raison dans le monde, on \emph{veut} y trouver de la sainteté : il en résulte que tout cela on le trouve : « Cherchez et vous trouverez ! » \emph{Ce} que je veux chercher, c’est \emph{Moi} qui le détermine. Si je veux chercher dans la Bible matière à édification, je trouverai ; si je veux la lire et l’examiner à fond, il en résultera pour moi une connaissance et une critique profondes — d’après mes forces. Je choisis ce qui répond à mes intentions, et, par le fait même que je choisis, je prouve mon — arbitraire.\par
De ceci naît cette considération que tout jugement que je porte sur un objet est l’œuvre, la \emph{création} de ma volonté ; je suis par là de nouveau averti de ne pas me perdre dans la \emph{créature} qu’est mon jugement, mais de rester le \emph{créateur} qui juge et qui toujours crée à nouveau. Tous les prédicats des objets sont mes affirmations, mes jugements, mes — créatures. S’ils veulent se détacher de moi et devenir quelque chose  pour eux-mêmes, ou m’en imposer le moins du monde, je n’ai rien de plus urgent que de les faire rentrer dans leur néant, c’est-à-dire en Moi, leur créateur. Dieu, Jésus-Christ, la Trinité, la Moralité, le Bien, etc., sont de ces créatures dont je ne dois pas seulement me permettre de dire qu’elles sont des vérités, mais dont je dois me permettre tout aussi bien de dire que ce sont des illusions. Si j’ai, à un moment donné, voulu et décrété leur existence, il faut de même que je puisse, à un autre moment, vouloir et décréter leur non existence. Je ne puis les laisser croître par dessus ma tête, je ne puis avoir la faiblesse de les laisser devenir quelque chose d’ « absolu », ce qui les soustrairait à ma puissance et m’interdirait de décider souverainement de leur sort. Je tomberais ainsi sous le joug du \emph{principe de stabilité}, du principe vital par excellence de la Religion, qui prend à cœur de créer des « sanctuaires inviolables », des « vérités éternelles », un « sacro-saint », en un mot, et de dépouiller chacun de ce qui est \emph{à lui}.\par
L’objet fait de nous des possédés ; cette influence, il l’exerce aussi bien lorsqu’il se présente à nous sous une forme sacrée que sous une forme non sacrée, et comme objet suprasensible que comme objet sensible. A l’objet, quel qu’il soit, répond chez nous un désir : convoitise sensuelle ou vœux idéaux, soif de l’or et aspiration vers le ciel doivent être mis sur la même ligne. Alors que les propagateurs de la lumière voulaient gagner les gens au monde sensible, Lavater prêchait l’appétit de l’invisible. Les uns veulent \emph{émouvoir }et les autres \emph{mouvoir}.\par
Chacun se fait des objets une idée particulière ; Dieu, Jésus-Christ, le monde, etc., ont été et seront conçus des façons les plus diverses. Chacun est en cela « hétérodoxe » et il a fallu des guerres sanglantes avant que des vues opposées sur un même objet en vinssent par ne plus être jugées des hérésies qui méritaient la mort. Les hétérodoxes se tolèrent. Mais  pourquoi me borner à penser \emph{autrement} au sujet d’une chose, pourquoi ne pas pousser l’hétérodoxie à ses dernières limites et ne plus \emph{rien} penser de cette chose, la supprimer de ma pensée ? Ce serait la fin de toute \emph{interprétation}, parce que plus rien ne serait à interpréter. Pourquoi dire : « Dieu n’est pas Brahma, n’est pas Jéhovah, n’est pas Allah, mais est — Dieu », et ne pas dire : « Dieu n’est rien qu’une illusion » ! Pourquoi me flétrit-on quand je suis un « négateur de Dieu » ? Parce qu’on met la créature au-dessus du créateur (« Ils honorent et servent plus la créature que le créateur\footnote{ \noindent Ep. aux Romains, {\scshape i}, 25.
 } ») et qu’on a besoin qu’un objet règne pour que le sujet serve humblement. Je dois me courber sous l’Absolu, c’est \emph{mon devoir}.\par
Par le « royaume des pensées », le Christianisme s’est complété ; la pensée est cette intériorité dans laquelle s’éteignent toutes les lumières du monde, où toute existence devient inexistante et où l’homme intérieur (le cerveau, le cœur) devient tout. Ce royaume des pensées attend sa délivrance, il attend comme le sphinx qu’Œdipe résolve l’énigme et lui permette d’entrer dans la mort. Je suis son destructeur, car dans \emph{mon} royaume, dans le royaume du créateur, il ne peut plus se former de royaumes propres et d’états dans l’état : il est une création de ma créatrice — absence de pensée. Le monde chrétien, le Christianisme et la Religion en général ne peuvent périr qu’avec le monde \emph{pensant ;} ce n’est que du jour où les pensées passeront qu’il n’y aura plus de croyants. Pour celui qui pense, le penser est « un labeur sublime, une activité sacrée » et repose sur une \emph{foi} solide, la foi dans la vérité. C’est d’abord la prière qui est une sainte activité, puis ce saint « recueillement » devient un « penser » raisonnable et raisonnant, qui, toutefois, conserve lui aussi comme base l’inébranlable foi dans la « Vérité sainte » et n’est  qu’une machine merveilleuse que l’esprit de Vérité remonte pour son service.\par
La pensée libre et la science libre s’occupent — (car ce n’est pas \emph{moi} qui suis libre et qui m’occupe, mais la pensée) — du ciel et du céleste ou « divin », c’est-à-dire, en réalité, du monde et du mondain, avec cette réserve que ce monde en est devenu un « autre » ; le monde a simplement subi un déplacement, une \emph{aliénation,} et je m’occupe de son essence, ce qui est une autre aliénation. Celui qui pense est aveugle envers les choses qui l’entourent et inapte à s’en rendre maître ; il ne mange, ni ne boit, ni ne jouit, car manger et boire n’est jamais penser ; il néglige tout, son avancement dans le monde, le soin de sa conservation, etc. pour penser. Il l’oublie comme l’oublie celui qui prie. Aussi le vigoureux fils de la nature le regarde-t-il comme un cerveau détraqué, comme un \emph{fou}, alors même qu’il le tient pour un saint ; c’est ainsi que les Anciens tenaient les frénétiques pour sacrés. La pensée libre est une frénésie, une folie, attendu qu’elle est un \emph{pur mouvement de l’être intime, }du seul \emph{homme intérieur} qui conduit et régit le reste de l’homme. Le chaman et le philosophe spéculatif sont les échelons extrêmes de l’échelle de l’homme \emph{intérieur}, du — mongol. Chaman et philosophe luttent contre des revenants, des démons, des \emph{Esprits, }des Dieux.\par
Radicalement différente de la pensée \emph{libre} est la pensée qui m’est \emph{propre}, ma pensée, qui ne me conduit pas mais que je conduis, que je tiens en laisse et que je lance ou retiens à mon gré. Cette pensée, ma propriété, diffère autant de la pensée libre que la sensualité que j’ai en mon pouvoir et que je satisfais s’il me plaît et comme il me plaît diffère de la sensualité libre, débridée, à laquelle je succombe.\par
Feuerbach, dans ses Principes de la philosophie de l’avenir \emph{(Grundsätzen der Philosophie der Zukunft)} en revient toujours à \emph{l’être}. Il reste ainsi, malgré toute  son hostilité contre Hégel et la philosophie de l’Absolu, plongé jusqu’au cou dans l’abstraction, car « l’être » est une abstraction, juste comme « le moi ». Mais \emph{Moi} qui \emph{suis}, et Moi seul, je ne suis pas purement une abstraction, \emph{je suis} tout dans tout et par conséquent je suis même abstraction et rien, je suis tout et rien. Je ne suis pas une simple pensée, mais je suis plein, entre autres choses, de pensées, je suis un monde de pensées. Hégel condamne tout ce qui m’est propre, mon avoir et mon — avis privés. La « pensée absolue » est celle qui perd de vue qu’elle n’est que \emph{ma} pensée, que c’est Moi qui la pense et qu’elle n’existe que par Moi. En tant que je suis Moi, je dévore ce qui est mien, j’en suis le maître ; la pensée n’est que \emph{mon opinion}, opinion que je puis à tout moment changer, c’est-à-dire anéantir, faire rentrer en moi et consommer. Feuerbach veut démolir la « pensée absolue » de Hégel grâce à l’\emph{être invincible.} Mais l’être ne trouve pas moins en Moi son vainqueur que la pensée : il est \emph{mon} « je suis » comme elle est \emph{mon} « je pense ».\par
Feuerbach, naturellement, n’aboutit qu’à démontrer cette thèse en soi triviale que j’ai besoin des \emph{sens} ou que je ne puis pas me passer complètement de ces organes. Il est positif que je ne puis pas penser si je ne suis pas un être sensible ; seulement, pour la pensée comme pour la sensation, pour l’abstrait comme pour le concret, j’ai avant tout besoin de Moi, et quand je dis moi, j’entends ce moi parfaitement déterminé que je suis, Moi l’\emph{unique.} Si \emph{je} n’étais pas un tel, si je n’étais pas Hégel, par exemple, je ne contemplerais pas le monde comme je le contemple, je n’y trouverais pas le système philosophique que, étant Hégel, j’y trouve, etc. J’aurais des sens comme le premier venu en a, mais je ne les emploierais pas comme je le fais.\par
Feuerbach reproche à Hégel\footnote{ \noindent \emph{Loc. cit.,} p. 47, sqs.
 } d’abuser de la langue  en détournant une foule de mots de l’acception naturelle que leur attribue la conscience ; lui-même commet pourtant la même faute lorsqu’il donne au mot « sensible » \emph{(sinnlich)} un sens aussi éminent qu’inusité. C’est ainsi qu’il déclare (p. 69) que « le sensible n’est pas le profane, l’irréfléchi, le patent, ce qui se saisit à première vue ». Mais si c’est le sacré, le réfléchi, le caché, si c’est ce qui ne se comprend qu’à force de réflexion, ce n’est plus ce qu’on appelle le sensible. Le sensible n’est que ce qui est pour \emph{les sens ;} ce dont ceux-là seuls peuvent jouir qui jouissent par \emph{plus} que les sens et qui dépassent la jouissance ou la conception sensibles, a tout au plus les sens pour intermédiaires et pour véhicules, c’est-à-dire que les sens sont la \emph{condition} de son obtention, mais qu’il n’est plus rien de sensible. Le sensible, quel qu’il soit, cesse d’être sensible en pénétrant en moi, quoiqu’il y puisse de nouveau avoir des effets sensibles tels que, par exemple, d’exciter mes passions et de faire bouillir mon sang.\par
Feuerbach réhabilite les sens ; c’est fort bien, mais il ne sait qu’affubler le matérialisme de sa « philosophie nouvelle » de la défroque qui était jusqu’à présent la propriété de la « philosophie de l’absolu ». Les gens ne se laisseront pas plus persuader qu’il suffit d’être sensible pour être tout, spirituel, intelligent, etc., qu’ils ne croient qu’on puisse vivre de « spirituel » seul, sans pain.\par
\emph{L’être} ne justifie rien. Le pensé \emph{est} aussi bien que le non-pensé, la pierre dans la rue \emph{est,} et ma représentation d’elle \emph{est} également ; la pierre et sa représentation occupent simplement des \emph{espaces} différents, l’une étant dans l’air et l’autre dans ma tête, en moi : car je suis espace comme la rue.\par
Les \emph{Membres d’une corporation} ou \emph{Privilégiés} ne tolèrent aucune liberté de penser, c’est-à-dire aucune pensée qui ne vient pas du « dispensateur de tout bien », que ce dispensateur s’appelle Dieu, le Pape,  l’Eglise ou n’importe comment. Si quelqu’un d’eux nourrit des pensées illégitimes, il doit les dire à l’oreille de son confesseur et se laisser imposer par lui pénitences et mortifications jusqu’à ce que le fouet à esclaves devienne intolérable à ces libres pensées. L’esprit de corps a d’ailleurs encore recours à d’autres procédés afin que les libres pensées n’éclosent pas du tout ; au nombre de ces moyens vient en première ligne une éducation appropriée. Celui qu’on a convenablement imprégné des principes de la morale ne redevient jamais libre de pensées morales ; le vol, le parjure, la tromperie, etc. restent pour lui des idées fixes contre lesquelles aucune liberté de pensée ne peut le protéger. Il a les pensées qui lui viennent « d’en haut », et il s’en tient là.\par
Il n’en est pas de même pour les \emph{Concessionnaires ou Patentés.} Chacun doit, selon eux, être libre d’avoir et de se faire les pensées qu’il veut. S’il a la patente, la concession d’une faculté de penser, il n’a que faire d’un privilège spécial. Comme « tous les hommes sont doués de raison », chacun est libre de se mettre en tête n’importe quelle pensée et d’amasser d’après la patente de ses capacités naturelles une plus ou moins grande richesse de pensées. Et l’on vous exhorte à « respecter toutes les opinions et toutes les convictions », on affirme que « toute conviction est légitime », qu’on doit « montrer de la tolérance pour les opinions des autres, » etc.\par
Mais « vos pensées ne sont pas mes pensées et vos chemins ne sont pas mes chemins », — ou plutôt c’est le contraire que je veux dire : vos pensées sont \emph{mes} pensées, dont je fais ce que je veux et que je puis renverser impitoyablement : elles sont ma propriété, que j’anéantis si cela me plaît. Je n’attends pas votre autorisation pour souffler en l’air ou crever les bulles de vos pensées. Peu me chaut que vous aussi appeliez ces pensées les vôtres : elles n’en restent pas moins les miennes ; mon attitude à leur égard est  \emph{mon affaire} et non une permission que je m’arroge. Il peut me plaire de vous laisser à vos pensées, et je me tairai. Croyez-vous que les pensées soient comme des oiseaux, et qu’elles voltigent si librement que chacun n’ait qu’à en saisir une pour pouvoir s’en prévaloir ensuite contre moi comme de sa propriété ? Tout ce qui vole est — \emph{à moi.}\par
Croyez-vous avoir vos pensées pour vous et n’avoir à en répondre devant personne, ou, comme vous dites, n’avoir à en rendre compte qu’à Dieu ? Il n’en est rien ; vos pensées, grandes et petites, m’appartiennent et j’en use selon mon bon plaisir.\par
La pensée ne m’est \emph{propre} que du moment que je ne me fais jamais aucun scrupule de la mettre en danger de mort et que je n’ai pas à redouter sa perte comme une \emph{perte pour moi}, une déchéance. La pensée n’est à moi que du moment que c’est moi qui l’assujettis et que jamais elle ne peut me courber sous son joug, me fanatiser et faire de moi l’instrument de sa réalisation.\par
La liberté de penser existe dès que je puis avoir toutes les pensées possibles ; mais les pensées ne deviennent une propriété qu’en perdant le pouvoir de devenir mes maîtres. Tant que la pensée est libre, ce sont les pensées (les Idées) qui règnent ; mais si je parviens à faire de ces dernières ma propriété, elles se conduisent comme mes créatures.\par
Si la Hiérarchie n’était pas aussi profondément enracinée dans le cœur de l’homme, au point de lui enlever tout courage de poursuivre des pensées libres, c’est-à-dire peut-être déplaisantes à Dieu, « liberté de penser » serait une expression aussi vide de sens que par exemple « liberté de digérer ».\par
Les gens appartenant à une confession sont d’avis que la pensée m’est \emph{donnée ;} d’après les libres penseurs \emph{je cherche} la pensée. Pour les premiers, la vérité est déjà trouvée et existante, je n’ai qu’à en — accuser réception au donateur qui me fait la grâce de me  l’accorder ; pour les seconds, la vérité est à chercher, elle est un but placé dans l’avenir et vers lequel je dois tendre.\par
Pour les uns comme pour les autres, la vérité (la pensée vraie) est en dehors de moi et je m’efforce de l’\emph{obtenir} soit comme un présent (la grâce) soit comme un gain (mérite personnel). Donc : 1\textsuperscript{o} la vérité est un \emph{privilège}. 2\textsuperscript{o} Non, le chemin qui y mène est \emph{patent} à tous ; ni la Bible, ni le Saint Père, ni l’Eglise ne sont en possession de la vérité, mais on peut spéculer sur sa possession.\par
Tous deux, comme on le voit, sont \emph{sans propriété} en fait de vérité. Ils ne peuvent la détenir qu’à titre de \emph{fief} (car le « Saint Père », par exemple, n’est pas un individu ; en tant qu’unique, il est un tel Sixte, un tel Clément, etc., et en tant que Sixte ou Clément il ne possède pas la vérité : s’il en est dépositaire, c’est comme « Saint Père », c’est-à-dire comme Esprit) — ou l’avoir pour \emph{idéal.} Si elle est un fief, elle est réservée au petit nombre (privilégiés) ; si elle est un idéal, elle est pour tous (patentés).\par
La liberté de penser a donc le sens que voici : nous errons tous dans l’obscurité sur les routes de l’erreur, mais chacun peut par ces voies se rapprocher de la vérité, et est alors dans le droit chemin (tous les chemins mènent à Rome, au bout du monde, etc.). Liberté de penser implique par conséquent que la vérité de la pensée ne m’est pas \emph{propre}, car si elle l’était, comment voudrait-on m’en exclure ?\par
Le penser est devenu tout à fait libre, et a codifié une foule de vérités auxquelles je dois me soumettre. Il cherche à se compléter par un \emph{système} et à s’élever à la hauteur d’une « constitution » absolue. Dans l’Etat, par exemple, il poursuit l’idée jusqu’à ce qu’il ait instauré l’ « Etat-raison », et dans l’homme (l’anthropologie), jusqu’à ce qu’il ait « découvert l’Homme ».\par
Celui qui pense ne diffère de celui qui croit qu’en ce qu’il croit beaucoup plus que ce dernier, qui, lui,  pense en revanche beaucoup moins à sa foi (articles de foi). Celui qui pense recourt à mille dogmes là où le croyant s’en tire avec quelques-uns ; mais il met de la \emph{liaison} entre eux, et prend cette liaison pour mesure de leur valeur. Si l’un ou l’autre ne fait pas son affaire, il le met au rebut.\par
Les aphorismes chers aux penseurs font exactement le pendant de ceux qu’affectionnent les croyants ; au lieu de « si cela vient de Dieu, vous ne le détruirez pas » ils disent : « si cela vient de la Vérité, c’est vrai » ; au lieu de « rendez hommage à Dieu » — « rendez hommage à la Vérité ». Mais peu m’importe qui de Dieu ou de la Vérité est vainqueur ; ce que je veux, c’est vaincre, Moi.\par
Comment peut-on imaginer une « liberté illimitée » dans l’Etat ou dans la Société ? L’Etat peut bien protéger l’un contre l’autre, mais il ne peut se laisser mettre lui-même en danger par une liberté. illimitée, par ce qu’on appelle la licence effrénée. L’Etat, en proclamant la « liberté de l’enseignement » proclame simplement que quiconque enseigne comme le veut l’Etat ou plus exactement comme le veut le pouvoir de l’Etat est dans son droit. La \emph{concurrence} est également soumise à ce « comme le veut l’Etat » : si le clergé, par exemple, ne veut pas comme l’Etat, il s’exclut lui-même de la concurrence (voir ce qui s’est passé en France). Les bornes que met nécessairement l’Etat à toute concurrence sont appelées la « surveillance et la haute direction de l’Etat ». Par le fait même qu’il maintient la liberté de l’enseignement dans les limites convenables, l’Etat fixe son but à la liberté de penser, car les gens, c’est la règle, ne pensent pas plus loin que leurs maîtres n’ont pensé.\par
Ecoutez ce que dit le ministre Guizot\footnote{ \noindent Chambre des Pairs, 25 avril 1844.
 } : « La grande difficulté de notre temps, c’est la direction, le gouvernement des esprits ;... vous le savez bien,  et le clergé lui-même le sait bien, ce grand corps spirituel ne peut suffire aujourd’hui à une telle destination. L’Etat a besoin qu’un grand corps laïque (l’Université) tenant de l’Etat son pouvoir, sa direction, exerce sur la jeunesse cette influence morale qui la forme à l’ordre, à la règle, et sans laquelle... », etc. « C’est notre devoir à nous, Gouvernement du roi, d’y veiller sans cesse... La Charte veut la liberté de la pensée et la liberté de conscience. »\par
Le Catholicisme appelle le candidat au forum de l’Eglise, et le Protestantisme à celui du Christianisme biblique. Le progrès réalisé serait encore assez mince si on le citait devant le tribunal de la Raison, comme le veut par exemple A. Ruge\footnote{ \noindent \emph{Anekdota}, {\scshape i}, 120.
 } : que l’\emph{autorité sacrée }soit l’Eglise, la Bible ou la Raison (à laquelle en appelaient d’ailleurs déjà Luther et Huss), cela ne fait aucune différence essentielle.\par
La « question de notre temps » ne sera pas soluble tant qu’on la posera ainsi : La légitimité a-t-elle sa source dans une généralité quelle qu’elle soit ou dans le seul individu ? Est-ce la généralité (Etat, Lois, Mœurs, Moralité, etc.) ou l’individualité qui autorise ? Questions oiseuses ! Le problème n’est soluble, et résolu, que lorsqu’on ne s’inquiète plus d’une « autorisation » et qu’on ne fait plus simplement la guerre aux « privilèges ».\par
Une liberté d’enseignement « raisonnable », qui « ne reconnaît que la conscience de la raison\footnote{ \noindent \emph{Anekdota}, {\scshape i}, 127.
 } » ne nous mène pas au but ; nous avons bien plus besoin d’une liberté d’enseigner \emph{égoïste}, se pliant à toute individualité, par laquelle \emph{je} puisse me rendre \emph{compréhensible,} et m’exposer sans que rien m’en empêche. Que je me fasse « \emph{intelligible} », cela seul est « raison » quelque déraisonnable que je sois ; si je me  fais comprendre et si je me comprends ainsi moi-même, les autres jouiront de moi comme j’en jouis, et me consommeront comme je me consomme.\par
Que gagnerait-on à voir aujourd’hui le moi raisonnable libre comme le fut autrefois le moi croyant, légal, moral, etc ? Cette liberté est-elle ma liberté ?\par
Si je ne suis libre qu’en tant que « moi raisonnable », c’est le raisonnable ou la raison qui est libre en moi, et cette liberté de la raison ou liberté de la pensée a depuis toujours été l’idéal du monde chrétien. On voulait rendre libre le penser — et, comme nous l’avons dit, le croire aussi est penser, comme le penser est croire ; — ceux qui pensent, c’est-à-dire aussi bien ceux qui croient que ceux qui raisonnent, devaient être libres, pour les autres la liberté était impossible. Mais la liberté de ceux qui pensent est la « liberté des enfants de Dieu », c’est la plus impitoyable — hiérarchie ou domination de la pensée : car je succombe sous la pensée. Si les pensées sont libres, j’en suis dominé, je n’ai sur elles aucun pouvoir et je suis leur esclave. Mais je veux jouir de la pensée, je veux être plein de pensées et cependant affranchi des pensées ; je me veux libre de pensées au lieu de libre de penser.\par
Pour me faire comprendre et pour communiquer avec les autres, je ne puis faire usage que de moyens \emph{humains}, moyens dont je dispose parce que comme eux je suis homme. Et, en réalité, en tant que homme, je n’ai que des pensées, tandis que, en tant que Moi, je suis en outre \emph{sans pensée}. Pour autant qu’on ne peut se dégager d’une pensée, on n’est rien que homme, on est l’esclave de la langue, cette production des hommes, ce trésor de la pensée humaine. La langue ou « le mot » exerce sur nous la plus affreuse tyrannie parce qu’elle conduit contre nous toute une armée d’idées fixes.\par
Examine-toi au moment précis où tu réfléchis et tu t’apercevras que tu ne peux avancer que si tu  es à chaque instant sans pensée et sans parole. Ce n’est pas seulement pendant ton sommeil que tu es sans pensée ni parole ; tu l’es dans les plus profondes méditations, et c’est même justement alors que tu l’es le plus. Et ce n’est que par cette absence de pensée, par cette « liberté de penser » méconnue ou liberté vis-à-vis du penser, que tu es à toi. C’est seulement grâce à elle que tu arriveras à user du langage comme de ta propriété.\par
Si le penser n’est pas \emph{mon} penser, il n’est que le dévidement d’un écheveau de pensées, c’est une besogne d’esclave, d’ « esclave des mots ». Le commencement de mon penser n’est pas une pensée, mais est Moi ; aussi suis-je également son but, et tout son cours n’est-il que le cours de ma jouissance de Moi. Le commencement du penser absolu ou libre est au contraire le penser libre lui-même, et le tout est, dure besogne, de faire remonter ce commencement à la suprême, la primordiale abstraction (par exemple l’être). Quand on tient le bout de cette abstraction ou de cette pensée initiale, il ne reste plus qu’à tirer sur le fil pour que tout l’écheveau se dévide.\par
Le penser absolu est le fait de l’esprit humain, et celui-ci est un Esprit saint. Aussi ce penser est-il le fait des prêtres ; eux seuls en ont « l’intelligence » et ont le sens des « intérêts suprêmes de l’humanité », de « l’Esprit ».\par
Les vérités sont pour le croyant une chose accomplie, un fait ; pour le libre-penseur, elles sont une chose qui doit encore être décidée. Quelque débarrassé de toute crédulité que soit le penser absolu, son scepticisme a des bornes, et il lui reste la foi à la vérité, à l’esprit, à l’Idée et à sa victoire finale : il ne pèche pas contre le Saint-Esprit. Mais tout penser qui ne pèche pas contre le Saint-Esprit n’est qu’une foi aux esprits et aux fantômes.\par
Je ne puis pas plus me défaire de la pensée que de la sensation, ni de l’activité de l’esprit que de l’activité  des sens. De même que le sentir est notre vision des choses, le penser est notre vision des essences (pensées). Les essences existent en tout ce qui est sensible, et particulièrement dans le « verbe ». Le pouvoir des mots succède au pouvoir des choses ; on est d’abord contraint par les verges, on l’est plus tard par la conviction. La puissance des choses est vaincue par notre courage, notre esprit ; contre la puissance d’une conviction, donc d’un mot, les chevalets et le billot perdent leur supériorité et leur force. Les hommes à convictions sont des prêtres qui résistent aux pièges de Satan.\par
Le Christianisme n’a enlevé aux choses de ce monde que leur irrésistibilité, et nous a laissés sous leur dépendance. Je fais de même à l’égard des vérités et de leur puissance, je m’élève au-dessus d’elles, je suis \emph{sur-vrai} comme je suis \emph{sur-sensible}. Les vérités me sont aussi indifférentes, aussi banales que les choses ; elles ne m’attirent ni ne m’enthousiasment. Il n’est pas une vérité, que ce soit le Droit, la Liberté, l’Humanité, etc., qui ait une existence indépendante de moi et devant laquelle je m’incline. Elles sont des \emph{mots}, et rien que des mots, comme pour le Chrétien toutes les choses ne sont que « vanités ». Dans les mots et dans les vérités (chaque mot est une vérité, et, comme Hégel le soutient, il n’est pas possible de \emph{dire} un mensonge) il n’y a point de salut pour Moi, pas plus qu’il n’y a de salut pour les Chrétiens dans les choses et dans les vanités. Pas plus que les richesses de ce monde, les vérités ne peuvent me rendre heureux. Le tentateur n’est plus Satan, mais l’Esprit, et celui-ci ne nous séduit pas au moyen des richesses du monde, mais par leurs pensées, par le « resplendissement de l’idée ».\par
Après les biens du monde, tous les biens sacrés doivent aussi être dépréciés.\par
Les vérités sont des phrases, des expressions, des mots (λόγος) ; reliés les uns aux autres, enfilés bout à  bout et rangés en lignes, ces mots forment la logique, la science, la philosophie.\par
J’emploie les vérités et les mots pour penser et pour parler comme j’emploie les aliments pour manger ; sans elles et sans eux je ne puis ni penser, ni parler, ni manger. Les vérités sont les pensées des hommes traduites en mots, et c’est ce qui fait qu’elles n’ont pas moins d’existence que les autres choses, bien qu’elles n’existent que pour l’esprit ou le penser. Elles sont des productions des hommes et des créatures humaines ; si on en fait des révélations divines, elles me deviennent étrangères, et quoique mes propres créatures, elles s’éloignent de moi aussitôt après l’acte de création.\par
L’homme chrétien est celui qui a foi dans la pensée, celui qui croit à la souveraineté des pensées et veut faire régner certaines pensées qu’il appelle « principes ». Beaucoup, il est vrai, font subir aux pensées une épreuve préalable, et n’en élisent aucune pour maître sans critique ; mais ils rappellent par là le chien qui va flairer les gens pour sentir « son maître » : ils s’adressent toujours aux pensées \emph{régnantes}. Le Chrétien peut indéfiniment réformer et bouleverser les idées qui dominent depuis des siècles, il peut même les détruire, mais ce sera toujours pour tendre vers un nouveau « principe » ou un nouveau maître ; toujours il érigera une plus haute ou plus « profonde » vérité, toujours il fondera un culte, toujours il proclamera un Esprit appelé à la souveraineté, et établira une \emph{loi} pour tous.\par
Tant qu’il reste une seule vérité à laquelle l’homme doit vouer sa vie et ses forces parce qu’il est homme, il est asservi à une règle, à une domination, à une loi, etc. : il reste serf. L’Homme, l’Humanité, la Liberté sont des vérités de ce genre.\par
On peut dire au contraire : si tu veux continuer à t’occuper des pensées, il ne tient qu’à toi ; sache seulement que si tu veux y parvenir à quelque chose de  considérable, il y a une foule de problèmes difficiles à résoudre, sans être venu à bout desquels tu n’iras pas loin. Dis-toi bien que ce ne t’est nullement un devoir ou une vocation de t’occuper de pensées (idées, vérités) ; si pourtant tu le veux, tu feras bien de mettre à profit ce que les autres ont déjà dépensé de forces pour mouvoir ces pesants objets.\par
Ainsi donc, celui qui veut penser s’impose par là même consciemment ou inconsciemment une tâche, mais cette tâche, rien ne l’oblige à l’accepter, car nul n’a le devoir de penser ou de croire. On peut lui dire : tu ne vas pas assez loin, ta curiosité est bornée et timide, tu ne vas pas au fond des choses, bref, tu ne t’en rends pas complètement maître ; mais d’autre part, si loin que tu sois arrivé, tu es toujours au bout de tâche, aucune vocation ne t’appelle à pousser plus loin, et tu es libre de faire comme tu veux ou comme tu peux. Il en est de la pensée comme de toute autre besogne que tu peux abandonner quand t’en passe l’envie. De même, lorsque tu ne peux plus croire une chose, tu n’as pas à te forcer à y croire et à continuer à t’en occuper comme d’un saint article de foi à la façon des théologiens ou des philosophes : tu peux hardiment en détourner ton intérêt et lui donner congé.\par
Les esprits prêtres traiteront assurément ce désintérêt de paresse d’esprit, d’irréflexion, d’apathie, etc. ; ne t’occupe pas de ces niaiseries. Rien, aucun « intérêt suprême de l’humanité », aucune « cause sacrée » ne vaut que tu la serves et que tu t’en occupes \emph{pour l’amour d’elle ;} ne lui cherche d’autre valeur que dans ce qu’elle vaut \emph{pour toi}. Rappelle par ta conduite la parole biblique : « Soyez comme des enfants » ; les enfants n’ont pas d’intérêts sacrés et n’ont aucune idée d’une « bonne cause ». Ils en savent d’autant mieux ce qu’ils veulent, et ils se demandent de toutes leurs forces comment ils doivent s’y prendre pour y arriver.\par
 Le penser ne peut pas plus cesser que le sentir. Mais la puissance des pensées et des idées, la domination des théories et des principes, l’empire de l’Esprit, en un mot la \emph{Hiérarchie} durera aussi longtemps que les prêtres auront la parole, — les prêtres, c’est-à-dire les théologiens, les philosophes, les hommes d’Etat, les philistins, les Libéraux, les maîtres d’école, les domestiques, les parents, les enfants, les époux, Proudhon, George Sand, Bluntschli, etc., etc. La Hiérarchie durera tant qu’on croira à des principes, tant qu’on y pensera ou même qu’on les critiquera ; car la critique même la plus corrosive, celle qui ruine tous les principes admis, le fait en définitive encore au nom d’\emph{un principe.}\par
Chacun critique, mais le critérium diffère. On est à la recherche du « véritable » critérium. Ce critérium est l’hypothèse première. Le critique part d’un axiome, d’une vérité, d’une croyance ; celle-ci n’est pas une création du critique, mais du dogmatique ; elle est ordinairement tout bonnement empruntée telle quelle à la culture du temps, ainsi, par exemple, « la liberté », « l’humanité », etc. Ce n’est pas le critique qui a « découvert l’Homme », « l’Homme » a été solidement établi comme vérité par le dogmatique, et le critique, qui peut d’ailleurs être la même personne, croit à cette vérité, à cet article de foi. C’est dans cette foi, et possédé par cette foi, qu’il critique.\par
Le secret de la critique est une « vérité » : tel est l’arcane de sa force.\par
Je fais cependant une distinction entre la critique \emph{officieuse} et la critique \emph{propre} ou égoïste. Si je critique en partant de l’hypothèse d’un Etre suprême, ma critique sert à cet Etre et s’exerce en sa faveur ; si je suis possédé de la foi en un « Etat libre », je critique tout ce qui s’y rapporte au point de vue de sa concordance, de sa convenance pour l’Etat libre, car j’\emph{aime} cet Etat ; si je suis un critique pieux,  tout se divisera pour moi en deux classes, le divin et le diabolique, la nature entière sera faite à mes yeux de traces de Dieu ou de traces du Diable (de là les lieux dits Gottesgabe, don de Dieu, Gottesberg, montagne de Dieu, Teufelskanzel, Chaire du Diable, etc.), les hommes se partageront en fidèles et infidèles, etc. ; si le critique croit à l’Homme, il commencera par tout ranger sous les deux rubriques Hommes et Non-hommes, etc.\par
La critique est jusqu’à présent restée une œuvre d’amour, car nous l’avons de tout temps exercée pour l’amour de l’un ou l’autre être. Toute critique officieuse est un produit de l’amour, une possession, et obéit au précepte du nouveau testament : « Eprouvez toute chose et retenez ce qui est bon »\footnote{ \noindent 1\textsuperscript{re} aux Thess. {\scshape v}, 21.
 }. Le « bon » est la pierre de touche, le critérium. Le bon, sous mille noms et mille formes différentes, est toujours resté l’hypothèse, le point d’appui dogmatique de la critique, l’idée fixe.\par
Le critique présuppose ingénument la « vérité » en se mettant à l’œuvre, et il la cherche, convaincu qu’elle est encore à trouver. Il veut découvrir la vérité, et il a justement pour éclairer ses recherches ce « bon » dont nous parlions tout à l’heure.\par
L’hypothèse, la supposition n’est que le fait de poser une \emph{pensée} ou de penser une certaine chose sous et avant toute autre ; partant de ce \emph{pensé,} on pensera ensuite tout le reste, c’est-à-dire qu’on l’y mesurera et le critiquera d’après lui. En d’autres termes, ceci revient à dire que le penser doit commencer avec quelque chose de déjà pensé. Si le penser commençait réellement, au lieu d’être commencé, le penser serait un sujet, une personne douée d’activité propre comme la plante déjà en est une ; dans ce cas, on ne pourrait évidemment pas nier que le penser doive commencer avec lui- même. Mais c’est précisément cette personnification du penser qui est grosse d’innombrables erreurs. Les Hégéliens s’expriment toujours comme si le penser pensait et agissait ; ils en font l’« esprit pensant » c’est-à-dire le penser personnifié, le penser devenu fantôme. Le Libéralisme critique, de son côté, vous dira : « la Critique » fait ceci ou cela, ou bien : « la conscience » juge de telle ou telle façon. Mais si vous tenez le penser pour ce qui agit personnellement, ce penser lui-même devra être supposé ; si vous tenez la Critique pour agissante, une pensée encore doit en être l’antécédent. Le penser et la Critique, pour être par eux-mêmes actifs, devraient être l’hypothèse même de leur activité, vu qu’ils ne peuvent être actifs sans être. Et le penser, en tant que « supposé », est une pensée fixe, un \emph{dogme ;} il en résulte que le penser et la Critique ne pourraient sortir que d’un \emph{dogme}, c’est-à-dire d’une pensée, d’une idée fixe, d’une hypothèse.\par
Cela nous ramène à ce que nous avons déjà dit précédemment, que le Christianisme consiste dans le développement d’un monde de pensées, ou qu’il est la véritable « liberté de pensée », la « libre pensée », le « libre Esprit ». La « vraie » Critique, que j’ai appelée « officieuse » est de même et pour la même raison la « libre » Critique, car elle n’est pas \emph{ma propriété}.\par
Il en est autrement si ce qui est à toi ne devient pas un \emph{être pour soi}, n’est pas personnifié, ne devient pas un « esprit » indépendant de toi. \emph{Ton} penser n’a pas pour hypothèse « le penser », mais Toi. Ainsi donc, tu t’es supposé ? Oui, mais ce n’est pas à moi que je me \emph{suppose}, c’est à mon penser. Avant mon penser, \emph{Je} suis. Il s’en suit que nulle pensée ne précède mon penser ou que mon penser n’a pas d’hypothèse. Car si je suis une supposition par rapport à mon penser, cette supposition n’est pas l’œuvre du penser, elle n’est pas \emph{sub-pensée}, mais est la \emph{position  même} du penser, et son \emph{possesseur ;} elle prouve simplement que le penser n’est qu’une — \emph{propriété,} c’est-à-dire qu’il n’existe ni « penser en soi » ni « esprit pensant ».\par
Ce renversement de la façon habituelle de considérer les choses pourrait sembler une jonglerie avec des abstractions, si vaine que celles mêmes contre lesquelles elle est dirigée ne risqueraient rien à se [{\corr prêter}] à cet inoffensif changement ; mais les conséquences pratiques qui en découlent sont graves.\par
La conclusion que j’en tire, c’est que l’Homme n’est pas la mesure de tout, mais que Je suis cette mesure. Le critique officieux a en vue un autre que lui, une idée qu’il veut servir ; aussi ne fait-il à son dieu que des hécatombes de fausses idoles. Ce qu’il fait pour l’amour de cet être n’est qu’une — œuvre de l’amour. Mais Moi, quand je critique, je n’ai pas seulement en vue mon but, je me procure en outre un plaisir, je m’amuse selon mon goût : suivant que cela me convient, je mâche la chose ou je me borne à en respirer le parfum.\par
On ne veut pas abandonner la « Vérité », mais la chercher. N’est-elle pas l’ « être suprême\footnote{ \noindent \emph{En français dans te texte. N. d. Tr.}
 } » ? Il ne resterait plus à la « vraie Critique » qu’à se jeter à l’eau, si elle venait à perdre la foi en la vérité. Et pourtant la vérité n’est qu’une — \emph{pensée ;} mais elle n’est pas une pensée tout court, elle est la pensée qui plane par dessus toutes les pensées, elle est la pensée \emph{irrécusable}, elle est la Pensée même, celle qui sanctifie toutes les autres, la consécration des pensées, la Pensée « absolue », « sacrée ». La vérité tient bon alors que tous les dieux s’en vont, car ce n’est que pour la servir et pour l’amour d’elle qu’on a renversé les dieux et finalement même Dieu. « La Vérité » continue à resplendir alors que le monde des dieux est rentré dans la nuit, parce qu’elle est l’âme immortelle  de ce monde périssable de dieux : elle est la divinité même.\par
Je veux répondre à la question de Pilate : « Qu’est-ce que la Vérité ? » — La vérité est la pensée libre, l’idée libre, l’esprit libre ; la vérité est ce qui est libre par rapport à toi, ce qui n’est pas à toi et n’est pas en ton pouvoir. Mais la vérité est aussi ce qui n’a pas d’existence par soi-même, ce qui est impersonnel, irréel et incorporel ; la vérité ne peut se manifester comme tu te manifestes, elle ne peut se mouvoir, ni changer, ni se développer ; la vérité attend et reçoit tout de toi, et n’est même que par toi, car elle n’existe que — dans ta tête. Tu conviens que la vérité est une pensée, mais tu n’admets pas que toute pensée en soit une vraie ; tu dis, pour m’exprimer comme toi, que chaque pensée n’est pas vraiment et réellement une pensée. Et à quoi reconnais-tu la vraie pensée ou la pensée vraie ? A \emph{ton impuissance}, c’est-à-dire à ce que tu n’as aucune prise sur elle ! Quand elle te vainc, t’enthousiasme et t’entraîne, tu la tiens pour vraie. Sa domination sur toi t’est la preuve de sa vérité ; lorsqu’elle te violente et que tu en es possédé, tu es content, tu as trouvé ton — \emph{seigneur et maître. }Quand tu cherchais la vérité, qu’appelait ton cœur ? Un \emph{maître !} Tu ne tendais pas vers \emph{ta} puissance, mais vers un puissant que tu pusses adorer (« adorez le Seigneur notre Dieu »).\par
La vérité, mon cher Pilate, est le — maître, et tous ceux qui cherchent la vérité cherchent et [{\corr glorifient}] le Seigneur. Où est-il, le Seigneur ? Où, sinon dans ta tête ? Il n’est qu’esprit, et partout où tu crois le découvrir, tu n’aperçois que — un fantôme ; le Seigneur n’est qu’une pensée, et ce n’est que le tourment, l’angoisse du Chrétien qui voulait faire l’invisible visible et donner un corps à l’esprit qui engendrèrent ce fantôme et l’effroyable misère qu’est la terreur des spectres.\par
Tant que tu crois à la vérité, tu ne crois pas à toi,  et tu es un — \emph{serf,} un \emph{homme religieux.} Toi seul tu es la vérité, ou plutôt tu es plus que la vérité, car sans toi elle n’est rien. Sans doute, toi aussi tu t’enquiers de la vérité, toi aussi tu « critiques » ; mais tu ne t’enquiers pas d’une « vérité supérieure », c’est-à-dire plus haute que toi, et tu ne critiques pas suivant le critérium d’une telle vérité. Tu ne t’adresses aux pensées et aux représentations comme aux phénomènes du monde extérieur que dans le but de les conformer à ton goût, de te les rendre agréables et de te les \emph{approprier ;} tu ne veux que t’en rendre maître et devenir leur possesseur ; tu veux t’y orienter et t’y sentir chez toi, et tu les trouves vraies ou les vois sous leur vrai jour quand elles ne peuvent plus t’échapper, quand il ne leur reste rien d’insaisi, rien d’incompris, ou que tu en \emph{jouis} et qu’elles sont ta \emph{propriété}. S’il arrive qu’elles te deviennent des servantes moins empressées, qu’elles se dérobent de nouveau à ton empire, ce sera le signe de leur fausseté, c’est-à-dire de ton impuissance. Ton impuissance est leur puissance, ton abaissement est leur élévation. Leur vérité, c’est toi, c’est le néant que tu es pour elles et dans lequel elles se dissolvent ; leur vérité est leur \emph{nullité.}\par
Ce n’est que lorsqu’ils sont ma propriété que ces esprits, les vérités, parviennent au repos ; pour qu’ils soient réels, il faut que, leur existence misérable leur ayant été enlevée, ils deviennent ma propriété et qu’on ne dise plus : la vérité grandit, gouverne, l’emporte, etc. Jamais la vérité n’a triomphé, elle a toujours été l’\emph{instrument} de ma victoire, comme le glaive (« le glaive de la vérité »). La vérité est une chose morte, c’est une lettre, un mot, un matériel que je puis employer. Toute Vérité est pour elle-même un cadavre ; si elle vit, ce n’est que comme mon poumon vit, c’est-à-dire selon la mesure de ma propre vitalité. Les vérités sont comme le bon grain et l’ivraie : sont-elles bon grain, sont-elles ivraies ? seul je puis en décider.\par
 Les objets ne sont pour moi que les matériaux que je mets en œuvre. Partout où je touche, je saisis une vérité que je m’adapte. La vérité est à moi, et je n’ai nul besoin de la désirer. Je ne me propose pas de me mettre au service de la vérité ; elle n’est qu’un aliment pour mon cerveau pensant, comme la pomme de terre en est un pour mon estomac digérant ou l’ami pour mon cœur sociable. Tant que j’ai le goût et la force de penser, toute vérité ne me sert qu’à la façonner autant qu’il m’est possible. La vérité est pour moi ce que la mondanité est pour les Chrétiens, « vaine et frivole ». Elle n’en existe pas moins, de même que les choses du monde continuent à exister quoique le Chrétien ait montré leur néant ; mais elle est vaine parce que sa \emph{valeur} n’est pas \emph{en elle} mais \emph{en moi}. Pour elle, elle est sans valeur. La vérité est une — \emph{créature}.\par
Par votre activité, vous créez d’innombrables œuvres ; vous avez changé la figure de la terre et édifié partout des monuments humains ; de même, grâce à votre pensée vous pouvez découvrir d’innombrables vérités, et nous nous en réjouirons de tout cœur. Mais je ne consentirai jamais à me faire l’esclave de vos machines nouvelles, je n’aiderai à les mettre en marche que pour mon usage ; vos vérités non plus je ne veux que les employer, sans me laisser employer par elles et pour elles.\par
Toutes les vérités \emph{en dessous} de Moi me sont les bienvenues ; de vérités \emph{au-dessus} de Moi, de vérités auxquelles je doive me plier, je m’en connais pas. Il n’y a pas de vérité au-dessus de moi, car au-dessus de Moi, il n’y a rien. Ni mon essence, ni l’essence de l’Homme ne sont au-dessus de Moi ! Oui, de Moi, cette « goutte dans la cuve », de cet être « infime » !\par
Vous croyez être d’une audace extraordinaire quand vous affirmez hardiment qu’il n’y a pas de « Vérité absolue », attendu, dites-vous, que chaque époque a sa vérité qui n’est qu’à elle. Vous accordez cependant  que chaque époque eut sa vérité ? mais par là-même vous créez proprement une « vérité absolue », une vérité qui ne manque à aucune époque parce que chacune, quelle que soit sa vérité, en a une.\par
Suffit-il de dire qu’on a de tout temps pensé et qu’on a par conséquent eu des pensées et des vérités, autres à chaque époque qu’à l’époque précédente ? Non, on doit dire que chaque époque eut sa « vérité de foi », et il est un fait, c’est qu’on n’en a jamais vu aucune où l’on ne reconnût une « vérité suprême » devant laquelle on se croyait obligé de s’incliner comme devant la « souveraine majesté ». La vérité d’une époque en est l’idée fixe ; lorsqu’un jour vient où l’on trouve une autre vérité, on ne la découvre que parce qu’on en cherchait une autre : on ne faisait que réformer sa folie et l’habiller à neuf. Car on voulait être « inspiré » par une idée, on cherchait à être domidé, — possédé par une pensée. Le dernier né de cette dynastie est « notre essence » ou « l’Homme ».\par
Pour toute critique libre, le critérium était une pensée ; pour la critique propre, égoïste, le critérium c’est Moi, Moi l’indicible et par conséquent l’impensable (car le pensé est toujours exprimable attendu que parole et pensée coïncident). Est vrai, ce qui est mien ; est faux ce dont je suis la propriété ; vraie par exemple est l’association, faux sont l’Etat et la société\footnote{ \noindent \emph{La parenté étymologique qui unit en français les mots \emph{S{\scshape ociété}} et \emph{A{\scshape ssociation}} et suppose l’une résultat de l’autre n’existe pas en allemand : « Verein » (association) exprime l’idée d’union, de coopération volontaire et active, tandis que « Gesellschaft » (société) implique par sa racine « Saal » (salle) réunion passive ou, comme dirait Stirner, parcage en un même endroit ; voyez, pour l’anatomie de la société, mot de chose, p. \hyperref[p261]{\dotuline{261}}. N. D. Tr.}
 }. La « libre et vraie » critique travaille à la domination logique d’une pensée, d’une idée, d’un Esprit ; la critique « propre » ne travaille qu’à ma \emph{jouissance}. En cela, elle se rapproche — et nous ne voudrions  pas lui épargner cette « honte » — de la critique animale de l’instinct. Il en est de moi comme de l’animal critiquant ; je ne vois dans mes affaires que moi et non elles. Je suis le critérium de la Vérité, mais je ne suis pas une idée, je suis plus qu’une idée, car je dépasse toute formule. \emph{Ma} critique n’est pas « libre », libre vis-à-vis de moi, et elle n’est pas une critique « officieuse », au service d’une idée ; elle m’est \emph{propre.}\par
La véritable critique ou, critique humaine ne découvre dans ce qu’elle examine que la \emph{convenance} à et pour l’Homme, le véritable Homme ; par ta critique propre, tu vérifies si l’objet \emph{te} convient.\par
La Critique libre s’occupe d’\emph{idées ;} aussi est-elle toujours théorétique. Quelle que soit sa rage contre les idées, elle ne s’en débarrasse pourtant pas. Elle se bat contre les fantômes, mais elle ne peut le faire qu’en les tenant pour des fantômes. Les Idées auxquelles elle s’en prend ne disparaissent pas tout-à-fait : le souffle de l’aube ne les met pas en fuite.\par
Le critique peut, il est vrai, parvenir à l’ataraxie envers les Idées, mais il n’en sera jamais quitte, c’est-à-dire qu’il ne comprendra jamais qu’il n’y à rien de supérieur à l’homme \emph{corporel}, ni son humanité, ni la liberté, etc. Il s’en tient toujours à une « vocation » de l’homme, à l’« humanité ». Si cette idée de l’humanité reste toujours irréalisée, c’est précisément parce qu’elle reste et doit rester « idée ».\par
Mais si je conçois au contraire l’idée comme \emph{mon }idée, alors elle se trouve par le fait même réalisée, attendu que \emph{je} suis sa réalité : sa réalité vient de ce que c’est Moi, le corporel, qui l’ai.\par
On dit que c’est dans l’histoire universelle que se réalise l’idée de Liberté. Cette idée est au contraire réelle dès qu’un homme la pense, et elle est réelle dans la mesure où elle est idée, c’est-à-dire pour autant que je la pense ou que je l’\emph{ai.} Ce n’est pas l’idée de Liberté qui se développe, mais ce sont les  hommes qui se développent et qui, en se développant, développent naturellement aussi leur penser.\par
En résumé, le critique n’est pas encore \emph{propriétaire}, parce qu’il combat encore dans les idées des étrangères puissantes, exactement comme le Chrétien n’est pas propriétaire de ses « mauvais désirs » aussi longtemps qu’il a à s’en défendre : pour celui qui combat le vice, le vice \emph{existe.}\par
La critique reste embourbée dans la « liberté de l’entendement », dans la liberté de l’esprit ; et l’esprit gagne vraiment sa liberté lorsqu’il s’emplit de la pure, de la vraie idée ; telle est la liberté de penser, qui ne peut être sans pensées.\par
La critique ne fait qu’abattre une idée par une autre, par exemple celle du privilège par celle de l’humanité, ou celle de l’égoïsme par celle du désintéressement.\par
En somme, c’est le commencement du Christianisme qui reparaît à sa fin dans la critique, car ici comme là l’« égoïsme » est l’ennemi. Ce n’est Moi, l’unique, mais l’idée, le général que je dois mettre en valeur.\par
La guerre du clergé contre l’\emph{égoïsme} et des spirituels contre les mondains forme tout le contenu de l’histoire chrétienne. Dans la critique contemporaine, cette guerre ne fait que s’universaliser, et le fanatisme se complète. Il faut bien qu’il vive et qu’il exhale sa rage avant de disparaître.\par

\asterism

\noindent Que m’importe que ce que je pense et ce que je fais soit chrétien ? Que ce soit humain ou inhumain, libéral ou illibéral, du moment que cela mène au but que je poursuis, du moment que cela me satisfait, c’est bien. Accablez le de tous les prédicats qu’il vous plaira, je m’en moque.\par
Il se peut que moi aussi je rompe avec les pensées que j’ai eues il n’y a qu’un instant, et il se peut que  je change brusquement ma façon d’agir ; mais ce n’est point parce que ces pensées ou ces actions ne sont pas conformes au Christianisme, ce n’est pas parce qu’elles portent atteinte aux éternels droits de l’Homme ou sont un soufflet à l’idée d’Humanité ; non, — c’est qu’elles ne sont plus conformes à Moi, c’est qu’elles ne me procurent plus une pleine jouissance, et que je doute de ma pensée de naguère ou ne me plais plus à agir comme je le faisais.\par
De même que le monde, en devenant ma propriété, est devenu un \emph{matériel} dont je fais ce que veux, l’esprit doit, en devenant ma propriété, redescendre à l’état de \emph{matériel} devant lequel je ne ressens plus la terreur du sacré. Désormais je ne foisonnerai plus d’horreur à aucune pensée, quelque téméraire ou « diabolique » qu’elle paraisse, car, pour peu qu’elle me devienne trop importune et désagréable, sa fin est en mon pouvoir ; et désormais je ne m’arrêterai plus en tremblant devant une action parce que l’esprit d’impiété, d’immoralité ou d’injustice y habite, pas plus que saint Boniface ne s’abstint par scrupule religieux d’abattre les chênes sacrés des païens. Comme les choses du monde sont devenues vaines, vaines doivent devenir les pensées de l’esprit.\par
Aucune pensée n’est sacrée, car nulle pensée n’est une « dévotion » ; aucun sentiment n’est sacré (il n’y a point de sentiment sacré de l’amitié, de saint amour maternel, etc.), aucune foi n’est sacrée. Pensées, sentiments, croyances sont révocables, et sont ma propriété, propriété \emph{précaire} que Moi-même je détruis comme c’est Moi qui la crée.\par
Le Chrétien peut se voir dépouillé de toutes les \emph{choses} ou objets, il peut perdre les personnes les plus aimées, ces « objets » de son amour, sans pour cela désespérer de lui-même, c’est-à-dire, au sens chrétien, de son esprit, de son âme. Le propriétaire peut rejeter loin de lui toutes les pensées qui étaient chères à son esprit et embrasaient son zèle, il en « regagnera  mille fois autant » car lui, leur créateur, demeure.\par
Inconsciemment et involontairement, nous tendons tous à l’individualité ; il serait difficile d’en trouver un seul parmi nous qui n’ait abandonné quelque sentiment sacré, et rompu avec quelque sainte pensée ou quelque sainte croyance ; mais nous ne rencontrerions personne qui ne pût encore s’affranchir de l’une ou l’autre de ses pensées sacrées. Chaque fois que nous nous attaquons à une conviction, nous partons de l’opinion que nous sommes capables de chasser, pour ainsi dire, l’adversaire des retranchements de sa pensée. Mais ce que je fais inconsciemment, je ne le fais qu’à moitié ; aussi, après chaque victoire sur une croyance, redeviens-je le \emph{prisonnier }(le possédé) d’une nouvelle croyance, qui me reprend tout entier à son service ; elle fait de moi un fanatique de la raison quand j’ai cessé de m’enthousiasmer pour la Bible, ou un fanatique de l’idée d’Humanité quand j’ai assez longtemps combattu pour celle de Christianisme.\par
Propriétaire des pensées, je protégerai sans doute ma propriété sous mon bouclier, juste comme, propriétaire des choses, je ne laisse pas chacun y porter la main ; mais c’est en souriant que j’accueillerai l’issue du combat, c’est en souriant que je déposerai mon bouclier sur le cadavre de mes pensées et de ma foi, et en souriant que, vaincu, je triompherai. C’est là justement qu’est l’humour de la chose. Pour laisser les gens s’égayer aux dépens des petitesses des hommes, il suffit de se sentir « trop haut pour être atteint » ; mais les laisser jouer avec toutes les « grandes pensées », avec les « sentiments sublimes », le « noble enthousiasme » et la « sainte croyance » suppose que je suis le propriétaire du tout.\par
A la sentence chrétienne « nous sommes tous des pécheurs », j’oppose celle-ci : nous sommes tous parfaits ! Car nous sommes à chaque instant tout ce que  nous pouvons être, et rien ne nous oblige jamais à être davantage. Comme nous ne traînons avec nous aucun manque, aucun défaut, le péché n’a pas de sens. Montrez-moi encore un pécheur dans un monde où nul n’a plus à satisfaire rien de supérieur à soi ! Si je ne veux que me satisfaire, en ne me satisfaisant pas je ne pèche pas, attendu que je n’offense en moi aucune « sainteté » ; au contraire, si je dois être pieux, j’ai à satisfaire Dieu, si je dois agir humainement, j’ai à satisfaire l’essence de l’Homme, l’idée d’humanité, etc. Celui que le religieux appelle un « pécheur », l’humanitaire l’appelle un « égoïste ». Mais, encore une fois, je n’ai à contenter personne ; qu’est-ce donc que l’ « Egoïste », ce Diable à la nouvelle mode que s’est payé l’Humanitarisme ? L’Egoïste devant lequel les humanitaires se signent avec effroi n’est qu’un fantôme, comme le Diable : il n’est qu’un épouvantail et une fantasmagorie de leur cerveau. S’ils n’étaient pas naïvement hantés par la vieille antithèse du bien et du mal auxquels ils ont donné respectivement les noms d’ « humain » et d’ « égoïste », ils n’auraient pas, pour le rajeunir, fait bouillir le « pécheur » grisonnant dans le chaudron de l’ « égoïsme », et n’auraient pas recousu une pièce neuve à un vieil habit. Mais ils ne pouvaient faire autrement, car ils considèrent comme leur devoir d’être « Hommes ».\par
Nous sommes tous parfaits, et il n’est pas sur toute la terre un seul homme qui soit un pécheur ! Comme il y a des fous qui s’imaginent être Dieu le père, Dieu le fils, ou l’homme de la lune, il fourmille d’insensés qui se croient des pécheurs. Les premiers ne sont pas l’homme de la lune et eux ne sont pas des pécheurs. Leur péché est chimérique.\par
Mais, objecte-t-on insidieusement, leur démence ou leur possession est du moins leur péché ? Leur possession n’est que ce qu’ils ont pu produire et le résultat de leur développement, tout comme la foi de Luther dans la Bible était tout ce qu’il avait pu produire.  Son développement mène l’un dans une maison de santé et conduit l’autre au Panthéon ou au — Walhalla.\par
Il n’y a ni pécheurs ni égoïsme pécheur !\par
Laisse-moi donc en paix, avec ton « amour de l’Homme » ! Glisse-toi, ô philanthrope, par la porte entrebâillée des « cavernes du vice », attarde-toi dans la cohue de la grande ville : ne vois-tu pas partout des péchés, des péchés et encore des péchés ? Ne gémis-tu pas sur l’humanité corrompue, ne déplores-tu pas le monstrueux épanouissement de l’égoïsme ? Verras-tu un riche sans le trouver impitoyable et « égoïste » ? Tu t’intitules peut-être athée, mais tu restes fidèle au sentiment chrétien qu’il est plus facile à un chameau de passer par le trou d’une aiguille qu’à un riche de n’être pas « inhumain ». Combien as-tu déjà rencontré de gens que tu n’aies pas rejetés dans la « masse égoïste » ? Ah, ton amour de l’Homme ! A quoi a-t-il abouti ? Tu ne vois plus que des hommes indignes d’amour ! Et d’où sortent-ils ? De ta philanthropie ! Tu t’es fourré en tête le pécheur, et de là vient que tu le trouves ou le supposes partout.\par
N’appelle pas les hommes des pécheurs et ils n’en seront pas ; toi seul es le créateur des péchés ; c’est toi, qui t’imagines aimer les hommes, qui les jettes dans la fange au crime, c’est toi qui les fais vicieux ou vertueux, hommes ou inhumains, et c’est toi qui les éclabousses de la bave de ta possession ; car tu n’aimes pas les hommes, mais l’Homme. Je te le dis : tu n’as jamais vu de pécheurs, tu n’en as que — rêvé.\par
Je gaspille ma jouissance de moi, parce que je crois devoir servir un autre que moi, parce que je me crois des devoirs envers lui, et me crois appelé au « sacrifice », au « dévouement », à l’ « enthousiasme ». Eh bien, si je ne sers plus aucune idée, aucun « être supérieur », il va de soi que je ne servirai plus non plus aucun homme, sauf — et dans tous les cas — \emph{Moi.} Et ainsi ce n’est pas seulement par l’être  ou par l’action, mais encore par la conscience que je suis l’— Unique.\par
Il \emph{te} revient plus que le divin, l’humain, etc. ; il te revient ce qui est \emph{tien.}\par
Regarde-toi comme plus puissant que tout ce pour quoi on te fait passer, et tu seras plus puissant ; regarde-toi comme plus, et tu seras plus.\par
Tu n’es pas simplement \emph{voué} à tout le divin et \emph{autorisé} à tout l’humain, mais tu es \emph{possesseur} du tien, c’est-à-dire de tout ce que tu as la force de t’approprier.\par
On a toujours cru devoir me donner une destination extérieure à moi, et c’est ainsi qu’on en vint finalement à m’exhorter à être humain et à agir humainement, parce que Je = Homme. C’est là le cercle magique chrétien. Le moi de Fichte est également un être extérieur et étranger à Moi, car ce moi est chacun et a seul des droits, de sorte qu’il est « le moi » et non Moi. Mais Moi, je ne suis pas un « moi » auprès d’autres « moi » : je suis le seul Moi, je suis Unique. Et mes besoins, mes actions, tout en Moi est unique. C’est par le seul fait que je suis ce Moi unique que je fais de tout ma propriété rien qu’en me mettant en œuvre et en me développant. Ce n’est pas comme Homme que je me développe, et je ne développe pas l’Homme : c’est \emph{Moi} qui \emph{Me} développe.\par
Tel est le sens de l’Unique.
 \subsection[{B.III. L’unique}]{B.III. L’unique}\phantomsection
\label{p22}
\noindent L’époque qui précéda le Christ et celle qui le suivit poursuivent des buts opposés ; la première voulut idéaliser le réel et la seconde veut réaliser l’idéal ; l’une chercha le « Saint-Esprit », l’autre cherche le « corps glorifié ». Aussi la première aboutit-elle à l’insensibilité à l’égard du réel, au « mépris du Monde », tandis que la seconde se clora par le renversement de l’idéal et le « mépris de l’Esprit ».\par
L’opposition du réel et de l’idéal est inconciliable, et l’un ne peut jamais devenir l’autre : si l’idéal devenait réel, il ne serait plus l’idéal, et si le réel devenait idéal il serait l’idéal et ne serait plus le réel. La contradiction des deux termes ne peut être résolue que si \emph{on} les anéantit tous deux ; c’est dans cet « on », ce tiers, qu’elle expire ; sinon, idéal et réalité ne se couvrent jamais. L’idée ne peut être réalisée et rester idée, il faut qu’elle périsse comme idée ; et il en est de même du réel qui devient idéal.\par
Les Anciens nous représentent les partisans de l’idée, et les Modernes ceux de la réalité. Ni les uns ni les autres ne parvinrent à se dégager de cette opposition, et ils se bornèrent à soupirer vers leur but : les Anciens avaient aspiré à l’Esprit, et du jour où il parut que le désir du monde antique était satisfait et que cet Esprit  était venu, les Modernes commencèrent à aspirer à la réalisation de cet esprit, réalisation qui doit rester éternellement un « pieux souhait ».\par
\emph{Le pium desiderium} des Anciens était la \emph{sainteté, }celui des Modernes est la \emph{corporalité}. Mais de même que l’Antiquité devait succomber le jour où ses vœux seraient comblés (car elle n’existait que par eux), de même il est à tout jamais impossible de parvenir à la corporalité sans sortir du cercle du Christianisme. Au courant de sanctification ou de purification qui traverse le monde antique (ablutions, etc.) fait suite et correspond le courant d’incarnation qui traverse le monde chrétien : le Dieu se précipite dans ce monde, il se fait chair et veut racheter le monde, c’est-à-dire le remplir de lui ; comme il est l’ « Idée » ou l’ « Esprit », on finit (Hégel par exemple) par introduire en toute chose l’esprit et par démontrer « que l’Idée, la Raison est dans tout ». A ce que les Stoïques du paganisme vantent comme « le Sage », répond dans la culture actuelle « l’Homme » ; l’un et l’autre deux êtres — \emph{sans chair}.\par
Le « sage » irréel, ce « saint » incorporel des Stoïques, est devenu une personne réelle et un « saint » corporel dans le Dieu « qui s’est fait chair » ; l’Homme irréel, le moi incorporel deviendra réel dans le Moi corporel que Je suis.\par
Au Christianisme est liée la question de « l’existence de Dieu » ; cette question, toujours et sans cesse reprise et débattue, prouve que le désir de l’existence, de la corporalité, de la personnalité, de la réalité était pour les cœurs un sujet de constante préoccupation, parce qu’il ne parvenait jamais à une solution satisfaisante. Enfin la question de l’existence de Dieu tomba, mais pour se relever aussitôt sous une nouvelle forme, dans la doctrine de l’existence du « divin » (Feuerbach). Mais le divin non plus n’a pas d’existence, et son dernier refuge, la réalisabilité du « purement humain » n’aura bientôt plus d’asile à lui offrir.  Nulle idée n’a d’existence, car nulle n’est susceptible de corporalité. La controverse scolastique du Réalisme et du Nominalisme n’eut pas d’autre objet ; bref, ce problème traverse d’un bout à l’autre l’histoire chrétienne et ne peut trouver en elle sa solution.\par
Le monde chrétien travaille à \emph{réaliser} des Idées dans toutes les circonstances de la vie individuelle et dans toutes les institutions et les lois de l’Eglise et de l’Etat ; mais toujours ces Idées résistent à ses tentatives et toujours il leur reste quelque chose qu’il n’est pas possible de rendre corporel (d’irréalisable) ; avec quelque ardeur qu’on s’efforce de les doter d’un corps, toujours elles demeurent sans réalité tangible.\par
Le « réaliseur » d’idées s’inquiète peu des réalités, pourvu que ces réalités incarnent une idée ; aussi examine-t-il sans relâche si l’idée qui doit en être le noyau les habite ; en éprouvant le réel il éprouve en même temps l’idée, et il vérifie si elle est bien réalisable comme il la pense, ou si elle n’est pensée par lui qu’à tort et par suite inexécutable.\par
En tant qu’\emph{existences}, la Famille, l’Etat, etc. n’intéressent plus le Chrétien ; les Chrétiens ne doivent pas, comme les Anciens, se sacrifier pour ces « divines choses », celles-ci ne doivent qu’être employées à faire \emph{vivre l’Esprit} en elles. La famille \emph{réelle} est devenue indifférente, et une famille \emph{idéale} (vraiment réelle) en doit naître : famille sainte, bénie de Dieu, ou, en style libéral « raisonnable » ou rationnelle. Pour les Anciens, la Famille, la Patrie, l’Etat, etc., sont actuellement divins ; pour les Modernes, ils attendent la divinisation, et ne sont sous leur forme présente que coupables et terrestres : ils doivent être « délivrés », et cette rédemption les fera vraiment réels. En d’autres termes, ce ne sont point la Famille, etc., qui sont le présent et le réel, mais le divin, l’idée ; la question est de savoir si \emph{telle} famille pourra devenir réelle par l’opération du véritable réel, de l’idée. L’individu n’a pas pour devoir de servir la  famille comme une divinité, mais bien de servir le divin et d’élever jusqu’à lui la famille encore non divine, c’est-à-dire de tout asservir à ridée, d’arborer partout la bannière de l’idée et d’amener l’idée à une réelle et efficace activité.\par
Le Christianisme et l’Antiquité ayant à faire au \emph{divin} finissent toujours par y revenir, quoique par les voies les plus opposées. A la fin du Paganisme, le divin devient \emph{extramondain ;} à la fin du Christianisme \emph{intramondain}. L’Antiquité ne réussit pas à le placer complètement hors du monde, et sitôt le Christianisme parvenu à accomplir cette tâche, le divin n’a rien de plus pressé que de réintégrer le monde, qu’il veut « racheter ». Mais si le Christianisme fait le divin \emph{intramondain}, il n’en fait pas et ne peut pas en faire le \emph{mondain même}, car le mauvais, l’irrationnel, le fortuit, l’égoïste sont le « mondain » dans le mauvais sens du mot, et sont et restent fermés au divin. Le Christianisme commence avec l’incarnation du Dieu qui se fait homme, et il poursuit toute son œuvre de conversion et de rédemption dans le but d’amener le Dieu à fleurir dans tous les hommes et dans tout l’humain, et de pénétrer tout de l’Esprit. Il s’en tient à préparer un siège pour l’ « Esprit ».\par
Si l’on en vint finalement à mettre l’accent sur l’Homme ou l’Humanité, ce fut de nouveau l’Idée que l’on « éternisa » : « L’Homme ne meurt pas ! » On pensa avoir trouvé la réalité de l’idée : l’Homme est le moi de l’histoire ; c’est lui, cet \emph{idéal}, qui se développe c’est-à-dire se \emph{réalise}. Il est vraiment réel et corporel, car l’histoire est son corps, dont les individus ne sont que les membres. Le Christ est le moi de l’histoire du monde, même de celle qui précède son apparition sur la terre ; pour la philosophie moderne, ce moi est l’Homme. L’image du Christ est devenue l’effigie de l’Homme, et l’Homme comme tel, l’ « Homme » tout court est le « \emph{centre} » de l’histoire. Avec l’Homme reparaît le commencement imaginaire,  car l’Homme est aussi imaginaire que le Christ. L’Homme, moi de l’histoire au monde, clôt le cycle de la pensée chrétienne.\par
Le cercle magique du Christianisme serait rompu si cessait le conflit entre l’existence et la vocation, c’est-à-dire entre Moi tel que je suis et Moi tel que je dois être ; le Christianisme ne consiste que dans l’aspiration de l’idée vers la corporalité, et il disparaît si l’abîme qui les sépare est comblé. Ce n’est qu’à condition que l’idée reste — idée (et Homme et Humanité ne sont encore non plus que des idées sans corps) que le Christianisme subsiste. L’idée devenue corporelle, l’Esprit incarné ou « parfait » flottent devant les yeux du Chrétien et représentent à son imagination le « jour dernier » ou le « but de l’histoire », mais ils ne sont pas pour lui un présent.\par
L’individu ne peut que prendre part à l’édification du royaume de Dieu — ou, en style moderne, au développement de l’histoire et de l’humanité, et c’est cette participation qui lui donne une valeur chrétienne — ou, en style moderne, humaine ; pour le reste, il n’est qu’un tas de cendres et la pâture des vers.\par
Que l’individu est pour soi une histoire du monde, et que le reste de l’histoire n’est que sa propriété, cela dépasse la vue du Chrétien. Pour ce dernier, l’histoire est supérieure, parce qu’elle est l’histoire du Christ ou de « l’Homme » ; pour l’égoïste, seule \emph{son }histoire a une valeur, parce qu’il ne veut développer que \emph{lui} et non le plan de Dieu, les desseins de la providence, la liberté, etc. Il ne se regarde pas comme un instrument de l’Idée ou un vaisseau de Dieu, il ne reconnaît aucune vocation, il ne s’imagine pas n’avoir d’autre raison d’être que de contribuer au développement de l’humanité et ne croit pas devoir y apporter son obole ; il vit sa vie sans se soucier que l’humanité en tire perte ou profit. — Eh quoi ! Suis-je au monde pour y réaliser des idées ? pour apporter par mon civisme ma pierre à la réalisation de l’idée  d’Etat, ou pour, par le mariage, donner une existence comme époux et père à l’idée de Famille ? Que me veut cette vocation ? Je ne vis pas plus d’après une vocation, que la fleur ne s’épanouit et n’exhale son parfum par devoir.\par
L’idéal « Homme » est \emph{réalisé}, lorsque la conception chrétienne se transforme et devient « Moi, cet Unique, je suis l’Homme ». La question : « Qu’est-ce que l’Homme ? » devient alors : « Qui est l’Homme ? » et c’est à Toi à répondre : « Qu’est-ce que » visait le concept à réaliser ; commençant par « qui est », la question n’en est plus une, car la réponse est personnellement présente dans celui qui interroge : la question est sa propre réponse.\par
On dit de Dieu : « Les noms ne te nomment pas ». Cela est également juste de Moi : aucun \emph{concept} ne m’exprime, rien de ce qu’on donne comme mon essence ne m’épuise, ce ne sont que des noms. On dit encore de Dieu qu’il est parfait et n’a nulle vocation de tendre vers une perfection. Et Moi ?\par
Je suis le \emph{propriétaire} de ma puissance, et je le suis quand je me sais \emph{Unique}. Dans l’\emph{Unique}, le possesseur retourne au Rien créateur dont il est sorti. Tout Etre supérieur à Moi, que ce soit Dieu ou que ce soit l’Homme, faiblit devant le sentiment de mon unicité et pâlit au soleil de cette conscience.\par
Si je base ma cause sur Moi, l’Unique, elle repose sur son créateur éphémère et périssable qui se dévore lui-même, et je puis dire :\par
Je n’ai basé ma cause sur Rien.
 


% at least one empty page at end (for booklet couv)
\ifbooklet
  \pagestyle{empty}
  \clearpage
  % 2 empty pages maybe needed for 4e cover
  \ifnum\modulo{\value{page}}{4}=0 \hbox{}\newpage\hbox{}\newpage\fi
  \ifnum\modulo{\value{page}}{4}=1 \hbox{}\newpage\hbox{}\newpage\fi


  \hbox{}\newpage
  \ifodd\value{page}\hbox{}\newpage\fi
  {\centering\color{rubric}\bfseries\noindent\large
    Hurlus ? Qu’est-ce.\par
    \bigskip
  }
  \noindent Des bouquinistes électroniques, pour du texte libre à participation libre,
  téléchargeable gratuitement sur \href{https://hurlus.fr}{\dotuline{hurlus.fr}}.\par
  \bigskip
  \noindent Cette brochure a été produite par des éditeurs bénévoles.
  Elle n’est pas faîte pour être possédée, mais pour être lue, et puis donnée.
  Que circule le texte !
  En page de garde, on peut ajouter une date, un lieu, un nom ; pour suivre le voyage des idées.
  \par

  Ce texte a été choisi parce qu’une personne l’a aimé,
  ou haï, elle a en tous cas pensé qu’il partipait à la formation de notre présent ;
  sans le souci de plaire, vendre, ou militer pour une cause.
  \par

  L’édition électronique est soigneuse, tant sur la technique
  que sur l’établissement du texte ; mais sans aucune prétention scolaire, au contraire.
  Le but est de s’adresser à tous, sans distinction de science ou de diplôme.
  Au plus direct ! (possible)
  \par

  Cet exemplaire en papier a été tiré sur une imprimante personnelle
   ou une photocopieuse. Tout le monde peut le faire.
  Il suffit de
  télécharger un fichier sur \href{https://hurlus.fr}{\dotuline{hurlus.fr}},
  d’imprimer, et agrafer ; puis de lire et donner.\par

  \bigskip

  \noindent PS : Les hurlus furent aussi des rebelles protestants qui cassaient les statues dans les églises catholiques. En 1566 démarra la révolte des gueux dans le pays de Lille. L’insurrection enflamma la région jusqu’à Anvers où les gueux de mer bloquèrent les bateaux espagnols.
  Ce fut une rare guerre de libération dont naquit un pays toujours libre : les Pays-Bas.
  En plat pays francophone, par contre, restèrent des bandes de huguenots, les hurlus, progressivement réprimés par la très catholique Espagne.
  Cette mémoire d’une défaite est éteinte, rallumons-la. Sortons les livres du culte universitaire, cherchons les idoles de l’époque, pour les briser.
\fi

\ifdev % autotext in dev mode
\fontname\font — \textsc{Les règles du jeu}\par
(\hyperref[utopie]{\underline{Lien}})\par
\noindent \initialiv{A}{lors là}\blindtext\par
\noindent \initialiv{À}{ la bonheur des dames}\blindtext\par
\noindent \initialiv{É}{tonnez-le}\blindtext\par
\noindent \initialiv{Q}{ualitativement}\blindtext\par
\noindent \initialiv{V}{aloriser}\blindtext\par
\Blindtext
\phantomsection
\label{utopie}
\Blinddocument
\fi
\end{document}
