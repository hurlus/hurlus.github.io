%%%%%%%%%%%%%%%%%%%%%%%%%%%%%%%%%
% LaTeX model https://hurlus.fr %
%%%%%%%%%%%%%%%%%%%%%%%%%%%%%%%%%

% Needed before document class
\RequirePackage{pdftexcmds} % needed for tests expressions
\RequirePackage{fix-cm} % correct units

% Define mode
\def\mode{a4}

\newif\ifaiv % a4
\newif\ifav % a5
\newif\ifbooklet % booklet
\newif\ifcover % cover for booklet

\ifnum \strcmp{\mode}{cover}=0
  \covertrue
\else\ifnum \strcmp{\mode}{booklet}=0
  \booklettrue
\else\ifnum \strcmp{\mode}{a5}=0
  \avtrue
\else
  \aivtrue
\fi\fi\fi

\ifbooklet % do not enclose with {}
  \documentclass[french,twoside]{book} % ,notitlepage
  \usepackage[%
    papersize={105mm, 297mm},
    inner=12mm,
    outer=12mm,
    top=20mm,
    bottom=15mm,
    marginparsep=0pt,
  ]{geometry}
  \usepackage[fontsize=9.5pt]{scrextend} % for Roboto
\else\ifav
  \documentclass[french,twoside]{book} % ,notitlepage
  \usepackage[%
    a5paper,
    inner=25mm,
    outer=15mm,
    top=15mm,
    bottom=15mm,
    marginparsep=0pt,
  ]{geometry}
  \usepackage[fontsize=12pt]{scrextend}
\else% A4 2 cols
  \documentclass[twocolumn]{report}
  \usepackage[%
    a4paper,
    inner=15mm,
    outer=10mm,
    top=25mm,
    bottom=18mm,
    marginparsep=0pt,
  ]{geometry}
  \setlength{\columnsep}{20mm}
  \usepackage[fontsize=9.5pt]{scrextend}
\fi\fi

%%%%%%%%%%%%%%
% Alignments %
%%%%%%%%%%%%%%
% before teinte macros

\setlength{\arrayrulewidth}{0.2pt}
\setlength{\columnseprule}{\arrayrulewidth} % twocol
\setlength{\parskip}{0pt} % classical para with no margin
\setlength{\parindent}{1.5em}

%%%%%%%%%%
% Colors %
%%%%%%%%%%
% before Teinte macros

\usepackage[dvipsnames]{xcolor}
\definecolor{rubric}{HTML}{800000} % the tonic 0c71c3
\def\columnseprulecolor{\color{rubric}}
\colorlet{borderline}{rubric!30!} % definecolor need exact code
\definecolor{shadecolor}{gray}{0.95}
\definecolor{bghi}{gray}{0.5}

%%%%%%%%%%%%%%%%%
% Teinte macros %
%%%%%%%%%%%%%%%%%
%%%%%%%%%%%%%%%%%%%%%%%%%%%%%%%%%%%%%%%%%%%%%%%%%%%
% <TEI> generic (LaTeX names generated by Teinte) %
%%%%%%%%%%%%%%%%%%%%%%%%%%%%%%%%%%%%%%%%%%%%%%%%%%%
% This template is inserted in a specific design
% It is XeLaTeX and otf fonts

\makeatletter % <@@@


\usepackage{blindtext} % generate text for testing
\usepackage[strict]{changepage} % for modulo 4
\usepackage{contour} % rounding words
\usepackage[nodayofweek]{datetime}
% \usepackage{DejaVuSans} % seems buggy for sffont font for symbols
\usepackage{enumitem} % <list>
\usepackage{etoolbox} % patch commands
\usepackage{fancyvrb}
\usepackage{fancyhdr}
\usepackage{float}
\usepackage{fontspec} % XeLaTeX mandatory for fonts
\usepackage{footnote} % used to capture notes in minipage (ex: quote)
\usepackage{framed} % bordering correct with footnote hack
\usepackage{graphicx}
\usepackage{lettrine} % drop caps
\usepackage{lipsum} % generate text for testing
\usepackage[framemethod=tikz,]{mdframed} % maybe used for frame with footnotes inside
\usepackage{pdftexcmds} % needed for tests expressions
\usepackage{polyglossia} % non-break space french punct, bug Warning: "Failed to patch part"
\usepackage[%
  indentfirst=false,
  vskip=1em,
  noorphanfirst=true,
  noorphanafter=true,
  leftmargin=\parindent,
  rightmargin=0pt,
]{quoting}
\usepackage{ragged2e}
\usepackage{setspace} % \setstretch for <quote>
\usepackage{tabularx} % <table>
\usepackage[explicit]{titlesec} % wear titles, !NO implicit
\usepackage{tikz} % ornaments
\usepackage{tocloft} % styling tocs
\usepackage[fit]{truncate} % used im runing titles
\usepackage{unicode-math}
\usepackage[normalem]{ulem} % breakable \uline, normalem is absolutely necessary to keep \emph
\usepackage{verse} % <l>
\usepackage{xcolor} % named colors
\usepackage{xparse} % @ifundefined
\XeTeXdefaultencoding "iso-8859-1" % bad encoding of xstring
\usepackage{xstring} % string tests
\XeTeXdefaultencoding "utf-8"
\PassOptionsToPackage{hyphens}{url} % before hyperref, which load url package

% TOTEST
% \usepackage{hypcap} % links in caption ?
% \usepackage{marginnote}
% TESTED
% \usepackage{background} % doesn’t work with xetek
% \usepackage{bookmark} % prefers the hyperref hack \phantomsection
% \usepackage[color, leftbars]{changebar} % 2 cols doc, impossible to keep bar left
% \usepackage[utf8x]{inputenc} % inputenc package ignored with utf8 based engines
% \usepackage[sfdefault,medium]{inter} % no small caps
% \usepackage{firamath} % choose firasans instead, firamath unavailable in Ubuntu 21-04
% \usepackage{flushend} % bad for last notes, supposed flush end of columns
% \usepackage[stable]{footmisc} % BAD for complex notes https://texfaq.org/FAQ-ftnsect
% \usepackage{helvet} % not for XeLaTeX
% \usepackage{multicol} % not compatible with too much packages (longtable, framed, memoir…)
% \usepackage[default,oldstyle,scale=0.95]{opensans} % no small caps
% \usepackage{sectsty} % \chapterfont OBSOLETE
% \usepackage{soul} % \ul for underline, OBSOLETE with XeTeX
% \usepackage[breakable]{tcolorbox} % text styling gone, footnote hack not kept with breakable


% Metadata inserted by a program, from the TEI source, for title page and runing heads
\title{\textbf{ La légende de Victor Hugo }}
\date{1885}
\author{Paul Lafargue}
\def\elbibl{Paul Lafargue. 1885. \emph{La légende de Victor Hugo}}
\def\elsource{Paul Lafargue, \emph{{\itshape La Légende de Victor Hugo}} [23 juin 1885], Paris, G. Jacques, 1902, 8-58 p. Source : \href{https://gallica.bnf.fr/ark:/12148/bpt6k82399t}{\dotuline{Gallica}}\footnote{\href{https://gallica.bnf.fr/ark:/12148/bpt6k82399t}{\url{https://gallica.bnf.fr/ark:/12148/bpt6k82399t}}}.}

% Default metas
\newcommand{\colorprovide}[2]{\@ifundefinedcolor{#1}{\colorlet{#1}{#2}}{}}
\colorprovide{rubric}{red}
\colorprovide{silver}{lightgray}
\@ifundefined{syms}{\newfontfamily\syms{DejaVu Sans}}{}
\newif\ifdev
\@ifundefined{elbibl}{% No meta defined, maybe dev mode
  \newcommand{\elbibl}{Titre court ?}
  \newcommand{\elbook}{Titre du livre source ?}
  \newcommand{\elabstract}{Résumé\par}
  \newcommand{\elurl}{http://oeuvres.github.io/elbook/2}
  \author{Éric Lœchien}
  \title{Un titre de test assez long pour vérifier le comportement d’une maquette}
  \date{1566}
  \devtrue
}{}
\let\eltitle\@title
\let\elauthor\@author
\let\eldate\@date


\defaultfontfeatures{
  % Mapping=tex-text, % no effect seen
  Scale=MatchLowercase,
  Ligatures={TeX,Common},
}


% generic typo commands
\newcommand{\astermono}{\medskip\centerline{\color{rubric}\large\selectfont{\syms ✻}}\medskip\par}%
\newcommand{\astertri}{\medskip\par\centerline{\color{rubric}\large\selectfont{\syms ✻\,✻\,✻}}\medskip\par}%
\newcommand{\asterism}{\bigskip\par\noindent\parbox{\linewidth}{\centering\color{rubric}\large{\syms ✻}\\{\syms ✻}\hskip 0.75em{\syms ✻}}\bigskip\par}%

% lists
\newlength{\listmod}
\setlength{\listmod}{\parindent}
\setlist{
  itemindent=!,
  listparindent=\listmod,
  labelsep=0.2\listmod,
  parsep=0pt,
  % topsep=0.2em, % default topsep is best
}
\setlist[itemize]{
  label=—,
  leftmargin=0pt,
  labelindent=1.2em,
  labelwidth=0pt,
}
\setlist[enumerate]{
  label={\bf\color{rubric}\arabic*.},
  labelindent=0.8\listmod,
  leftmargin=\listmod,
  labelwidth=0pt,
}
\newlist{listalpha}{enumerate}{1}
\setlist[listalpha]{
  label={\bf\color{rubric}\alph*.},
  leftmargin=0pt,
  labelindent=0.8\listmod,
  labelwidth=0pt,
}
\newcommand{\listhead}[1]{\hspace{-1\listmod}\emph{#1}}

\renewcommand{\hrulefill}{%
  \leavevmode\leaders\hrule height 0.2pt\hfill\kern\z@}

% General typo
\DeclareTextFontCommand{\textlarge}{\large}
\DeclareTextFontCommand{\textsmall}{\small}

% commands, inlines
\newcommand{\anchor}[1]{\Hy@raisedlink{\hypertarget{#1}{}}} % link to top of an anchor (not baseline)
\newcommand\abbr[1]{#1}
\newcommand{\autour}[1]{\tikz[baseline=(X.base)]\node [draw=rubric,thin,rectangle,inner sep=1.5pt, rounded corners=3pt] (X) {\color{rubric}#1};}
\newcommand\corr[1]{#1}
\newcommand{\ed}[1]{ {\color{silver}\sffamily\footnotesize (#1)} } % <milestone ed="1688"/>
\newcommand\expan[1]{#1}
\newcommand\foreign[1]{\emph{#1}}
\newcommand\gap[1]{#1}
\renewcommand{\LettrineFontHook}{\color{rubric}}
\newcommand{\initial}[2]{\lettrine[lines=2, loversize=0.3, lhang=0.3]{#1}{#2}}
\newcommand{\initialiv}[2]{%
  \let\oldLFH\LettrineFontHook
  % \renewcommand{\LettrineFontHook}{\color{rubric}\ttfamily}
  \IfSubStr{QJ’}{#1}{
    \lettrine[lines=4, lhang=0.2, loversize=-0.1, lraise=0.2]{\smash{#1}}{#2}
  }{\IfSubStr{É}{#1}{
    \lettrine[lines=4, lhang=0.2, loversize=-0, lraise=0]{\smash{#1}}{#2}
  }{\IfSubStr{ÀÂ}{#1}{
    \lettrine[lines=4, lhang=0.2, loversize=-0, lraise=0, slope=0.6em]{\smash{#1}}{#2}
  }{\IfSubStr{A}{#1}{
    \lettrine[lines=4, lhang=0.2, loversize=0.2, slope=0.6em]{\smash{#1}}{#2}
  }{\IfSubStr{V}{#1}{
    \lettrine[lines=4, lhang=0.2, loversize=0.2, slope=-0.5em]{\smash{#1}}{#2}
  }{
    \lettrine[lines=4, lhang=0.2, loversize=0.2]{\smash{#1}}{#2}
  }}}}}
  \let\LettrineFontHook\oldLFH
}
\newcommand{\labelchar}[1]{\textbf{\color{rubric} #1}}
\newcommand{\milestone}[1]{\autour{\footnotesize\color{rubric} #1}} % <milestone n="4"/>
\newcommand\name[1]{#1}
\newcommand\orig[1]{#1}
\newcommand\orgName[1]{#1}
\newcommand\persName[1]{#1}
\newcommand\placeName[1]{#1}
\newcommand{\pn}[1]{\IfSubStr{-—–¶}{#1}% <p n="3"/>
  {\noindent{\bfseries\color{rubric}   ¶  }}
  {{\footnotesize\autour{ #1}  }}}
\newcommand\reg{}
% \newcommand\ref{} % already defined
\newcommand\sic[1]{#1}
\newcommand\surname[1]{\textsc{#1}}
\newcommand\term[1]{\textbf{#1}}

\def\mednobreak{\ifdim\lastskip<\medskipamount
  \removelastskip\nopagebreak\medskip\fi}
\def\bignobreak{\ifdim\lastskip<\bigskipamount
  \removelastskip\nopagebreak\bigskip\fi}

% commands, blocks
\newcommand{\byline}[1]{\bigskip{\RaggedLeft{#1}\par}\bigskip}
\newcommand{\bibl}[1]{{\RaggedLeft{#1}\par\bigskip}}
\newcommand{\biblitem}[1]{{\noindent\hangindent=\parindent   #1\par}}
\newcommand{\dateline}[1]{\medskip{\RaggedLeft{#1}\par}\bigskip}
\newcommand{\labelblock}[1]{\medbreak{\noindent\color{rubric}\bfseries #1}\par\mednobreak}
\newcommand{\salute}[1]{\bigbreak{#1}\par\medbreak}
\newcommand{\signed}[1]{\bigbreak\filbreak{\raggedleft #1\par}\medskip}

% environments for blocks (some may become commands)
\newenvironment{borderbox}{}{} % framing content
\newenvironment{citbibl}{\ifvmode\hfill\fi}{\ifvmode\par\fi }
\newenvironment{docAuthor}{\ifvmode\vskip4pt\fontsize{16pt}{18pt}\selectfont\fi\itshape}{\ifvmode\par\fi }
\newenvironment{docDate}{}{\ifvmode\par\fi }
\newenvironment{docImprint}{\vskip6pt}{\ifvmode\par\fi }
\newenvironment{docTitle}{\vskip6pt\bfseries\fontsize{18pt}{22pt}\selectfont}{\par }
\newenvironment{msHead}{\vskip6pt}{\par}
\newenvironment{msItem}{\vskip6pt}{\par}
\newenvironment{titlePart}{}{\par }


% environments for block containers
\newenvironment{argument}{\itshape\parindent0pt}{\vskip1.5em}
\newenvironment{biblfree}{}{\ifvmode\par\fi }
\newenvironment{bibitemlist}[1]{%
  \list{\@biblabel{\@arabic\c@enumiv}}%
  {%
    \settowidth\labelwidth{\@biblabel{#1}}%
    \leftmargin\labelwidth
    \advance\leftmargin\labelsep
    \@openbib@code
    \usecounter{enumiv}%
    \let\p@enumiv\@empty
    \renewcommand\theenumiv{\@arabic\c@enumiv}%
  }
  \sloppy
  \clubpenalty4000
  \@clubpenalty \clubpenalty
  \widowpenalty4000%
  \sfcode`\.\@m
}%
{\def\@noitemerr
  {\@latex@warning{Empty `bibitemlist' environment}}%
\endlist}
\newenvironment{quoteblock}% may be used for ornaments
  {\begin{quoting}}
  {\end{quoting}}

% table () is preceded and finished by custom command
\newcommand{\tableopen}[1]{%
  \ifnum\strcmp{#1}{wide}=0{%
    \begin{center}
  }
  \else\ifnum\strcmp{#1}{long}=0{%
    \begin{center}
  }
  \else{%
    \begin{center}
  }
  \fi\fi
}
\newcommand{\tableclose}[1]{%
  \ifnum\strcmp{#1}{wide}=0{%
    \end{center}
  }
  \else\ifnum\strcmp{#1}{long}=0{%
    \end{center}
  }
  \else{%
    \end{center}
  }
  \fi\fi
}


% text structure
\newcommand\chapteropen{} % before chapter title
\newcommand\chaptercont{} % after title, argument, epigraph…
\newcommand\chapterclose{} % maybe useful for multicol settings
\setcounter{secnumdepth}{-2} % no counters for hierarchy titles
\setcounter{tocdepth}{5} % deep toc
\markright{\@title} % ???
\markboth{\@title}{\@author} % ???
\renewcommand\tableofcontents{\@starttoc{toc}}
% toclof format
% \renewcommand{\@tocrmarg}{0.1em} % Useless command?
% \renewcommand{\@pnumwidth}{0.5em} % {1.75em}
\renewcommand{\@cftmaketoctitle}{}
\setlength{\cftbeforesecskip}{\z@ \@plus.2\p@}
\renewcommand{\cftchapfont}{}
\renewcommand{\cftchapdotsep}{\cftdotsep}
\renewcommand{\cftchapleader}{\normalfont\cftdotfill{\cftchapdotsep}}
\renewcommand{\cftchappagefont}{\bfseries}
\setlength{\cftbeforechapskip}{0em \@plus\p@}
% \renewcommand{\cftsecfont}{\small\relax}
\renewcommand{\cftsecpagefont}{\normalfont}
% \renewcommand{\cftsubsecfont}{\small\relax}
\renewcommand{\cftsecdotsep}{\cftdotsep}
\renewcommand{\cftsecpagefont}{\normalfont}
\renewcommand{\cftsecleader}{\normalfont\cftdotfill{\cftsecdotsep}}
\setlength{\cftsecindent}{1em}
\setlength{\cftsubsecindent}{2em}
\setlength{\cftsubsubsecindent}{3em}
\setlength{\cftchapnumwidth}{1em}
\setlength{\cftsecnumwidth}{1em}
\setlength{\cftsubsecnumwidth}{1em}
\setlength{\cftsubsubsecnumwidth}{1em}

% footnotes
\newif\ifheading
\newcommand*{\fnmarkscale}{\ifheading 0.70 \else 1 \fi}
\renewcommand\footnoterule{\vspace*{0.3cm}\hrule height \arrayrulewidth width 3cm \vspace*{0.3cm}}
\setlength\footnotesep{1.5\footnotesep} % footnote separator
\renewcommand\@makefntext[1]{\parindent 1.5em \noindent \hb@xt@1.8em{\hss{\normalfont\@thefnmark . }}#1} % no superscipt in foot
\patchcmd{\@footnotetext}{\footnotesize}{\footnotesize\sffamily}{}{} % before scrextend, hyperref


%   see https://tex.stackexchange.com/a/34449/5049
\def\truncdiv#1#2{((#1-(#2-1)/2)/#2)}
\def\moduloop#1#2{(#1-\truncdiv{#1}{#2}*#2)}
\def\modulo#1#2{\number\numexpr\moduloop{#1}{#2}\relax}

% orphans and widows
\clubpenalty=9996
\widowpenalty=9999
\brokenpenalty=4991
\predisplaypenalty=10000
\postdisplaypenalty=1549
\displaywidowpenalty=1602
\hyphenpenalty=400
% Copied from Rahtz but not understood
\def\@pnumwidth{1.55em}
\def\@tocrmarg {2.55em}
\def\@dotsep{4.5}
\emergencystretch 3em
\hbadness=4000
\pretolerance=750
\tolerance=2000
\vbadness=4000
\def\Gin@extensions{.pdf,.png,.jpg,.mps,.tif}
% \renewcommand{\@cite}[1]{#1} % biblio

\usepackage{hyperref} % supposed to be the last one, :o) except for the ones to follow
\urlstyle{same} % after hyperref
\hypersetup{
  % pdftex, % no effect
  pdftitle={\elbibl},
  % pdfauthor={Your name here},
  % pdfsubject={Your subject here},
  % pdfkeywords={keyword1, keyword2},
  bookmarksnumbered=true,
  bookmarksopen=true,
  bookmarksopenlevel=1,
  pdfstartview=Fit,
  breaklinks=true, % avoid long links
  pdfpagemode=UseOutlines,    % pdf toc
  hyperfootnotes=true,
  colorlinks=false,
  pdfborder=0 0 0,
  % pdfpagelayout=TwoPageRight,
  % linktocpage=true, % NO, toc, link only on page no
}

\makeatother % /@@@>
%%%%%%%%%%%%%%
% </TEI> end %
%%%%%%%%%%%%%%


%%%%%%%%%%%%%
% footnotes %
%%%%%%%%%%%%%
\renewcommand{\thefootnote}{\bfseries\textcolor{rubric}{\arabic{footnote}}} % color for footnote marks

%%%%%%%%%
% Fonts %
%%%%%%%%%
\usepackage[]{roboto} % SmallCaps, Regular is a bit bold
% \linespread{0.90} % too compact, keep font natural
\newfontfamily\fontrun[]{Roboto Condensed Light} % condensed runing heads
\ifav
  \setmainfont[
    ItalicFont={Roboto Light Italic},
  ]{Roboto}
\else\ifbooklet
  \setmainfont[
    ItalicFont={Roboto Light Italic},
  ]{Roboto}
\else
\setmainfont[
  ItalicFont={Roboto Italic},
]{Roboto Light}
\fi\fi
\renewcommand{\LettrineFontHook}{\bfseries\color{rubric}}
% \renewenvironment{labelblock}{\begin{center}\bfseries\color{rubric}}{\end{center}}

%%%%%%%%
% MISC %
%%%%%%%%

\setdefaultlanguage[frenchpart=false]{french} % bug on part


\newenvironment{quotebar}{%
    \def\FrameCommand{{\color{rubric!10!}\vrule width 0.5em} \hspace{0.9em}}%
    \def\OuterFrameSep{\itemsep} % séparateur vertical
    \MakeFramed {\advance\hsize-\width \FrameRestore}
  }%
  {%
    \endMakeFramed
  }
\renewenvironment{quoteblock}% may be used for ornaments
  {%
    \savenotes
    \setstretch{0.9}
    \normalfont
    \begin{quotebar}
  }
  {%
    \end{quotebar}
    \spewnotes
  }


\renewcommand{\headrulewidth}{\arrayrulewidth}
\renewcommand{\headrule}{{\color{rubric}\hrule}}

% delicate tuning, image has produce line-height problems in title on 2 lines
\titleformat{name=\chapter} % command
  [display] % shape
  {\vspace{1.5em}\centering} % format
  {} % label
  {0pt} % separator between n
  {}
[{\color{rubric}\huge\textbf{#1}}\bigskip] % after code
% \titlespacing{command}{left spacing}{before spacing}{after spacing}[right]
\titlespacing*{\chapter}{0pt}{-2em}{0pt}[0pt]

\titleformat{name=\section}
  [block]{}{}{}{}
  [\vbox{\color{rubric}\large\raggedleft\textbf{#1}}]
\titlespacing{\section}{0pt}{0pt plus 4pt minus 2pt}{\baselineskip}

\titleformat{name=\subsection}
  [block]
  {}
  {} % \thesection
  {} % separator \arrayrulewidth
  {}
[\vbox{\large\textbf{#1}}]
% \titlespacing{\subsection}{0pt}{0pt plus 4pt minus 2pt}{\baselineskip}

\ifaiv
  \fancypagestyle{main}{%
    \fancyhf{}
    \setlength{\headheight}{1.5em}
    \fancyhead{} % reset head
    \fancyfoot{} % reset foot
    \fancyhead[L]{\truncate{0.45\headwidth}{\fontrun\elbibl}} % book ref
    \fancyhead[R]{\truncate{0.45\headwidth}{ \fontrun\nouppercase\leftmark}} % Chapter title
    \fancyhead[C]{\thepage}
  }
  \fancypagestyle{plain}{% apply to chapter
    \fancyhf{}% clear all header and footer fields
    \setlength{\headheight}{1.5em}
    \fancyhead[L]{\truncate{0.9\headwidth}{\fontrun\elbibl}}
    \fancyhead[R]{\thepage}
  }
\else
  \fancypagestyle{main}{%
    \fancyhf{}
    \setlength{\headheight}{1.5em}
    \fancyhead{} % reset head
    \fancyfoot{} % reset foot
    \fancyhead[RE]{\truncate{0.9\headwidth}{\fontrun\elbibl}} % book ref
    \fancyhead[LO]{\truncate{0.9\headwidth}{\fontrun\nouppercase\leftmark}} % Chapter title, \nouppercase needed
    \fancyhead[RO,LE]{\thepage}
  }
  \fancypagestyle{plain}{% apply to chapter
    \fancyhf{}% clear all header and footer fields
    \setlength{\headheight}{1.5em}
    \fancyhead[L]{\truncate{0.9\headwidth}{\fontrun\elbibl}}
    \fancyhead[R]{\thepage}
  }
\fi

\ifav % a5 only
  \titleclass{\section}{top}
\fi

\newcommand\chapo{{%
  \vspace*{-3em}
  \centering % no vskip ()
  {\Large\addfontfeature{LetterSpace=25}\bfseries{\elauthor}}\par
  \smallskip
  {\large\eldate}\par
  \bigskip
  {\Large\selectfont{\eltitle}}\par
  \bigskip
  {\color{rubric}\hline\par}
  \bigskip
  {\Large TEXTE LIBRE À PARTICPATION LIBRE\par}
  \centerline{\small\color{rubric} {hurlus.fr, tiré le \today}}\par
  \bigskip
}}

\newcommand\cover{{%
  \thispagestyle{empty}
  \centering
  {\LARGE\bfseries{\elauthor}}\par
  \bigskip
  {\Large\eldate}\par
  \bigskip
  \bigskip
  {\LARGE\selectfont{\eltitle}}\par
  \vfill\null
  {\color{rubric}\setlength{\arrayrulewidth}{2pt}\hline\par}
  \vfill\null
  {\Large TEXTE LIBRE À PARTICPATION LIBRE\par}
  \centerline{{\href{https://hurlus.fr}{\dotuline{hurlus.fr}}, tiré le \today}}\par
}}

\begin{document}
\pagestyle{empty}
\ifbooklet{
  \cover\newpage
  \thispagestyle{empty}\hbox{}\newpage
  \cover\newpage\noindent Les voyages de la brochure\par
  \bigskip
  \begin{tabularx}{\textwidth}{l|X|X}
    \textbf{Date} & \textbf{Lieu}& \textbf{Nom/pseudo} \\ \hline
    \rule{0pt}{25cm} &  &   \\
  \end{tabularx}
  \newpage
  \addtocounter{page}{-4}
}\fi

\thispagestyle{empty}
\ifaiv
  \twocolumn[\chapo]
\else
  \chapo
\fi
{\it\elabstract}
\bigskip
\makeatletter\@starttoc{toc}\makeatother % toc without new page
\bigskip

\pagestyle{main} % after style

   \phantomsection
\label{p1}\section[{[Avant-propos]}]{[Avant-propos]}\renewcommand{\leftmark}{[Avant-propos]}

\noindent {\itshape Victor Hugo appartient désormais à l’impartialité de l’histoire.}\par
{\itshape Dès le coup d’État de 1852 la légende s’est emparée de Hugo. Durant l’Empire, dans l’intérêt de la propagande anti-bonapartiste et républicaine, on n’osait s’opposer à cette cristallisation de la fantaisie, en quête de demi-dieux : après le 16 mai, il n’y avait pas nécessité de troubler les dernières années d’un homme âgé, dont le rôle était fini. Mais aujourd’hui que le poète, célébré par la presse, reconnu et proclamé le « grand homme du siècle » dort au Panthéon, « la colossale tombe des génies », la critique reconquiert ses droits. Elle peut sans crainte de compromettre des intérêts politiques et de blesser inutilement un vieillard devenu inoffensif étudier la vie de cet homme, au nom retentissant. Elle a le devoir de dégager la vérité enfouie sous les mensonges et les exagérations.}\par
{\itshape Les hugolâtres se scandaliseront de ce qu’une critique impie, ose porter la main sur leur idole : mais qu’ils en prennent leur parti. — La critique historique ne cherche pas à plaire et ne craint pas de déplaire.}\par
{\itshape Cette étude, écrite sur des notes recueillies en 1869, n’a pas la prétention d’épuiser le sujet, mais simplement de mettre en lumière le véritable caractère de Victor Hugo, si étrangement méconnu.}\par


\signed{P. L.}

\dateline{Sainte Pélagie, 23 juin 1885.}
\section[{La légende de Victor Hugo}]{La légende de Victor Hugo\protect\footnotemark }\renewcommand{\leftmark}{La légende de Victor Hugo}

\footnotetext{\noindent {\itshape La célébration du centenaire de Victor Hugo, qui donne de l’actualité à cette étude, nous a suggéré l’idée de la republier : écrite le lendemain de sa mort, elle n’a pas encore perdu son originalité, le côté de la vie publique qu’elle expose n’ayant été ni discuté, ni analysé.} {\scshape L’Éditeur}.
}
 \phantomsection
\label{p3}\subsection[{I}]{I}
\noindent Le premier juin 1885 Paris célébrait les plus magnifiques funérailles du siècle : il enterrait Victor Hugo {\itshape il poeta sovrano}. Pendant dix jours, la presse tout entière prépara l’opinion publique de France et d’Europe. Paris, un instant ému, par la promenade du drapeau rouge et les charges policières du Père-Lachaise, qui revivifiaient les souvenirs de la Semaine sanglante, se remit à ne s’occuper que de celui qui fut \emph{« le plus illustre représentant de la conscience humaine »}. Les journaux n’avaient pas assez de leurs trois pages — la quatrième étant prise par les annonces, — pour exalter \emph{« le génie en qui vivait l’idée humaine »}. La langue que Victor Hugo avait cependant enrichie de si nombreuses expressions laudatives, semblait pauvre aux journalistes, du moment qu’elle était appelée à traduire leur  \phantomsection
\label{p4}admiration pour \emph{« le plus gigantesque penseur de l’univers »}, on recourut à l’image. Une feuille du soir, à court de vocables, représenta sur sa première page, le soleil plongeant dans l’océan. La mort de Hugo était la mort d’un astre. « L’art était fini ! »\par
La population, brassée par l’enthousiasme journalistique, jeta trois cent mille hommes, femmes et enfants, derrière le char du pauvre qui emportait le poète au Panthéon, et un million sur les places, les rues et les trottoirs par où il passait.\par
Un vélum noir voilait de deuil l’Arc de Triomphe de la gloire impériale ; la lumière des becs de gaz et des lampadaires filtrait, lugubre, à travers le crêpe ; des couronnes d’immortelles et de peluches, des portraits de Hugo sur son lit de mort, des médailles de bronze, portant gravé : {\itshape Deuil national…}, enfin tous les symboles de la douleur désespérée avaient été réquisitionnés, et pourtant la multitude immense n’avait ni regrets pour le mort, ni souvenirs pour l’écrivain : Hugo lui était indifférent. Elle paraissait ignorer que l’on menait, sous ses yeux, au Panthéon \emph{« le plus grand poète qui eût jamais existé »}.\par
La foule houleuse et de belle humeur témoignait bruyamment sa satisfaction du temps et du spectacle ; elle s’enquérait du nom des célébrités et des délégations de villes et de pays qui défilaient pour son plaisir ; elle admirait les monumentales couronnes de fleurs portées sur des chars ; elle applaudissait les fifres des sociétés de tir, déchirant les oreilles de leurs airs discordants ; elle saluait de rires ironiques Déroulède et son sérieux en redingote verte ; et  \phantomsection
\label{p5}pour mettre le comble à sa joie, il ne manquait que le blason des {\itshape Benni-bouffe-toujours} du cortège, — le lapin sauté et leur arme, — la colossale seringue de carton.\par
Acteurs et spectateurs jubilaient. Il est vrai que les habitants des grands boulevards, désappointés de ce que l’on ne promenait pas le cadavre devant leurs portes, supputaient avec aigreur les sommes rondelettes qu’ils n’auraient pas manqué d’empocher ; le cœur ulcéré, ils se racontaient que des fenêtres et des balcons avaient été loués des centaines et des milliers de francs ; qu’en trois heures d’horloge on gagnait deux fois et plus le loyer de six mois. Mais le chagrin des grincheux disparaissait dans la réjouissance générale. Les brasseries à femmes du boulevard Saint-Michel débordaient sur le trottoir en échafaudage ; on achetait au poids de l’or le droit d’y cuire au soleil, en s’arrosant de bière frelatée. Les petites gens, installées aux bons endroits, dès la pointe du jour, qui avec une chaise, qui avec une table, un banc, une échelle, les cédaient aux curieux pour le prix de deux journées de rigolade et de vie de rentier. Les hôteliers, les cabaretiers, les fricoteurs de la race goulue souriaient d’allégresse en palpant dans leurs poches les pièces de cent sous que la fête rapportait : l’un d’eux disait d’un air très convaincu : « il faudrait qu’il meure toutes les semaines un Victor Hugo pour faire aller le commerce ! » Le commerce marchait en effet ! Commerce de fleurs et d’emblèmes mortuaires ; commerce de journaux, de gravures, de lyres en zinc bronzé, doré, argenté, de médailles en galvano, d’effigies montées en  \phantomsection
\label{p6}épingle ; commerce de crêpe noir et de brassards, d’écharpes, de rubans tricolores et multicolores ; commerce de bière, de vin, de charcuterie ; les gens affamés mangeaient et buvaient debout dans la rue, devant les comptoirs, n’importe quoi et à n’importe quel prix ; commerce d’amour, — les provinciaux et les étrangers, venus des quatre coins de l’horizon, honoraient le mort en festoyant avec les horizontales.\par
Les funérailles du premier juin ont été dignes du mort qu’on panthéonisait et dignes de la classe qui escortait le cadavre.\par
Les organisations socialistes révolutionnaires de France et de l’Étranger, qui sont la partie consciente du prolétariat, ne s’étaient pas fait représenter aux obsèques de Victor Hugo. Les anarchistes faisaient exception et pour se distinguer une fois de plus des socialistes révolutionnaires, ils essayèrent de mêler leur drapeau noir aux drapeaux multicolores du cortège ; Élisée Reclus, leur homme remarquable, pria son ami Nadar d’inscrire son nom sur le registre mortuaire. Cependant le gouvernement en frappant d’interdit le déploiement du drapeau rouge ; M. Vacquerie en déclarant que dans l’exil, Hugo avait toujours marché derrière le drapeau rouge toutes les fois qu’on portait en terre une des victimes du coup d’État, et la presse radicale en réclamant le droit à la rue pour l’étendard de la Commune et en rappelant qu’en 1871 le proscrit de l’Empire avait ouvert sa maison de Bruxelles aux vaincus de Paris, tous semblaient à l’envie convier les révolutionnaires à s’assembler  \phantomsection
\label{p7}autour du cercueil de Victor Hugo, comme centre de ralliement des partis républicains. Mais les révolutionnaires socialistes refusèrent de prendre part à la promenade carnavalesque du premier juin.\par
La Cité de Londres, invitée, n’envoya pas de délégation aux funérailles du poète : des membres de son conseil prétendirent qu’ils n’avaient rien compris à la lecture de ses ouvrages ; c’était en effet bien mal comprendre Victor Hugo que de motiver son refus par de telles raisons. Sans nul doute, les honorables Michelin, Ruel et Lyon Allemand de Londres s’imaginèrent que l’écrivain, qui venait de trépasser, était un de ces prolétaires de la plume, qui louent à la semaine et à l’année leurs cervelles aux Hachette de l’éditorat et aux Villemessant de la presse. Mais si on leur avait appris que le mort avait son compte chez Rothschild, qu’il était le plus fort actionnaire de la Banque belge, qu’en homme prévoyant, il avait placé ses fonds hors de France, où l’on fait des révolutions et où l’on parle de brûler le Grand livre, et qu’il ne se départit de sa prudence et n’acheta de l’emprunt de cinq milliards pour la libération de sa patrie, que parce que le placement était à six pour cent ; si on leur avait fait entendre que le poète avait amassé cinq millions en vendant des phrases et des mots, qu’il avait été un habile commerçant de lettres, un maître dans l’art de débattre et de dresser un contrat à son avantage, qu’il s’était enrichi en ruinant ses éditeurs, ce qui ne s’était jamais vu ; si on avait ainsi énuméré les titres du mort, certes les honorables représentants  \phantomsection
\label{p8}de la Cité de Londres, ce cœur commercial des deux mondes, n’auraient pas marchandé leur adhésion à l’importante cérémonie ; ils auraient, au contraire, tenu à honorer le millionnaire qui sut allier la poésie au {\itshape doit} et {\itshape avoir}.\par
La bourgeoisie de France, mieux renseignée, voyait dans Victor Hugo une des plus parfaites et des plus brillantes personnifications de ses instincts, de ses passions et de ses pensées.\par
La presse bourgeoise, grisée par les louanges hyperboliques qu’elle jetait à pleines colonnes sur le mort, négligea de mettre en relief le côté {\itshape représentatif} de Victor Hugo, qui sera peut-être son titre le plus réel aux yeux de la postérité. Je vais essayer de réparer cet oubli.
\subsection[{II}]{II}
\noindent Les légitimistes ne pardonnent pas à Victor Hugo, l’ultra-royaliste et l’ardent catholique d’avant 1830, d’être passé au parti républicain. Ils oublient qu’un fils de vendéen, M. de  Rochejacquelein, enrôlé dans le Sénat du second Empire, répondit cavalièrement à de semblables reproches : « Il n’y a que les imbéciles qui ne changent jamais. » Le poète, incapable de ce dédain aristocratique, ne lança jamais au parti qu’il désertait cette impertinente excuse : mais il voulut expliquer aux républicains pourquoi il avait été royaliste.\par
— \emph{Ma mère était une {\itshape brigande} de la Vendée ; à quinze ans elle fuyait à travers le Bocage, comme Madame Bonchamp, comme Madame de  \phantomsection
\label{p9}La Rochejacquelein, écrit-il en 1831, dans la préface des \emph{Feuilles d’automne}.} — \emph{Mon père, soldat de la République et de l’Empire, bivouaquait en Europe ; je vécus auprès de ma mère et subis ses opinions ; pour elle « la Révolution c’était la guillotine, Bonaparte l’homme qui prenait les fils, l’empire du sabre\footnote{Victor Hugo, \emph{Philosophie et littérature mêlées}, 1831 Vol. 1. 203.} ».} Son influence, non contrebalancée, planta dans le jeune cœur de Hugo une haine vigoureuse de Napoléon et de la Révolution, car \emph{« il était soumis en tout à sa mère et prêt à tout ce qu’elle voulait\footnote{\emph{Victor Hugo raconté par un témoin de sa vie}. Vol. 1. 147. Première édition.} »}. Le royalisme de Hugo n’était que de la piété filiale et l’on sait que personne, mieux que lui, ne mérita l’épitaphe de bon fils, bon mari, bon père.\par
Emporté par son imagination, Hugo, le converti de 1830, se figurait les opinions de sa mère, non telles qu’elles avaient été, mais telles que les besoins de son excuse les exigeaient. En effet, cette brigande, qui battait la campagne pour le {\itshape Roy} s’amouracha d’un {\itshape pataud}, du républicain J.-L.-S. Hugo, qui, pour se mettre à la mode du jour, s’était affublé du prénom de {\itshape Brutus}. Elle l’avait connu à Nantes où siégeait une commission militaire, qui, parfois, jugeait et passait par les armes, en un seul jour, des fournées de dix et douze {\itshape brigands} et {\itshape brigandes}. Brutus Hugo remplissait auprès de cette commission les fonctions de greffier. En 1796, la brigande épousa civilement le soldat républicain, qui, plus Brutus que jamais, était pour l’instant et le resta jusqu’en 1797,  \phantomsection
\label{p10}rapporteur d’un conseil de guerre, qui jugeait expéditivement les royalistes : sans autre forme de procès, il les condamnait à mort, leur identité et inscription sur la liste des suspects, constatées. La brigande suivit son mari à Madrid, orna la cour de Joseph qui sur le trône d’Espagne, remplaçait le roi légitime, et permit à son fils aîné Abel, d’endosser la livrée bonapartiste, en qualité de page. Le royalisme de Madame Hugo, si tant est qu’elle eut une opinion politique, devait être bien platonique : autrement il faudrait admettre que cette femme si courageuse, si fidèle en ses amitiés (pendant 18 mois, au risque de mille dangers, elle cacha aux Feuillantines, le général Lahorie, traqué par la police impériale), aurait ainsi renié sa foi et pactisé avec les plus cruels ennemis de son parti. Hugo a dû ne savoir à quelle excuse se vouer, pour en arriver à prêter à sa mère défunte, des opinions en contradiction si flagrante avec les actes de sa vie et à nous la montrer traître au parti, traître au roi pour qui elle aurait affronté la mort. Lui, le fils pieux, il a dû souffrir d’être réduit à flétrir la mère si dévouée à ses enfants, qui les éleva et les soigna si tendrement alors que le père les abandonnait, qui les laissa librement se développer et obéir aux impulsions de leur nature. Mais il lui fallait à tout prix trouver quelqu’un, sur qui rejeter la responsabilité de ses odes royalistes, qui l’embarrassaient davantage que le boulet ne gêne le forçat pour fuir à travers champs : il prit sa mère\footnote{De 1817 à 1826 aucun événement heureux ou malheureux ne pouvait arriver à la famille royale, sans qu’il ne saisit aussitôt sa bonne plume d’oie : tantôt c’est une naissance, un baptême, une mort ; tantôt un avènement, un sacre, qui allume sa verve. Hugo est le Belmontet de Louis XVIII et de Charles X ; il est le poète officiel, attaché au service personnel de la famille royale.}.  \phantomsection
\label{p11}Il peut invoquer des circonstances atténuantes. On utilisait, à l’époque, la mère de toutes les façons ; elle était déjà la grande ficelle dramatique : c’était le souvenir de la mère qui au théâtre paralysait le bras de l’assassin prêt à frapper ; c’était la croix de la mère, qui exhibée au moment psychologique, prévenait le viol, l’inceste et sauvait l’héroïne ; c’était la mort de sa mère, qui du Chateaubriand sceptique et disciple de Jean-Jacques de 1797, tira le Chateaubriand mystagogique d’\emph{Atala} et du \emph{Génie du Christianisme} de 1800. Victor Hugo qui ne devança jamais de 24 heures l’opinion publique, mais sut toujours lui emboîter le pas, singeait Chateaubriand son maître, et appliquait à son usage privé le truc qui ne ratait pas son effet au théâtre.\par
Que le royalisme de Hugo fût de circonstance ou d’origine maternelle, peu importe ; il est certain qu’il était grassement payé, et c’était heureux, car le public achetait avec modération ses livres : les éditeurs de \emph{Han d’Islande} lui écrivaient en 1823 qu’ils ne savaient comment se débarrasser des 500 exemplaires de la première édition, qui restaient en magasin. Louis XVIII octroyait au poète, en septembre 1822, une pension de 1 000 francs sur sa cassette particulière et, en février 1823, une seconde pension de 2 000 francs sur les fonds littéraires du ministère de l’Intérieur. Victor Hugo et ses deux frères, Abel et Eugène, faisaient avec courage et ténacité le siège de ces fonds littéraires ; en 1821, ils se plaignaient amèrement de ce que le ministère n’avait pas  \phantomsection
\label{p12}subventionné leur revue bimensuelle, \emph{Le Conservateur littéraire}\footnote{La plainte de ces intéressants et intéressés jeunes gens est touchante. \emph{« Le \emph{Conservateur} n’a reçu aucun encouragement du gouvernement, disent-ils. D’autres recueils ont trouvé moyen de faire bénéfice sur les faveurs du ministre du roi, lesquels se sont souvenus des avantages de l’économie lorsqu’il s’est agi d’encourager un ouvrage assez maladroit pour se montrer royaliste et indépendant. »} (Préface du troisième volume du \emph{Conservateur littéraire}). — Cependant page 361 du même recueil on lit : \emph{« L’ode sur \emph{la mort du duc de Berry}, insérée dans la septième livraison, ayant été communiquée par le comte de Neufchâteau au duc de Richelieu, président du conseil des ministres et zélé pour les lettres, qui l’ayant jugée digne d’être mise sous les yeux du Roi, sa Majesté daigna ordonner qu’une gratification ({\itshape sic}), de 500 fr. fût remise à l’auteur, M. V. Hugo, en témoignage de son auguste satisfaction. »}}. Ils défendaient avec âpreté le fond des reptiles en même temps qu’ils l’attaquaient avec convoitise ; ainsi le Conservateur s’indignait contre Benjamin Constant, cet \emph{« ex-homme de lettres qui a fait refuser à la Chambre une somme de 40 000 francs destinée à donner des encouragements aux gens de lettres. Le but du député libéral est, dit-il, d’empêcher que cette somme ne serve à soudoyer quelque pamphlétaire ministériel\footnote{\emph{Le Conservateur littéraire}, vol. 2, p. 245.} »}. Rogner les fonds secrets du ministre, c’était porter la main sur la propriété des Hugo. À la fin de l’année 1826, Victor réclamait au vicomte de La Rochefoucauld une augmentation de la part qui lui revenait sur ces fonds : depuis que ma pension a été accordée, écrivait-il, \emph{« quatre ans se sont écoulés et si ma pension est restée ce qu’elle était, j’ai eu du moins la joie (qui ne le réjouissait pas) de voir la bonté du roi augmenter les pensions de plusieurs hommes de lettres de mes amis et dont quelques-uns la dépassent de plus du double. Ma pension seule  \phantomsection
\label{p13}étant restée stationnaire, je pense, monsieur le vicomte, n’être pas sans quelque droit à une augmentation… Je dépose avec confiance ma demande entre vos mains, en vous priant de vouloir la mettre sous les yeux de ce roi qui veut faire des beaux-arts, le fleuron le plus éclatant de sa couronne »}. On ne tint nul compte de la demande si pressante et si motivée du fidèle serviteur, qui pour se consoler, épancha son désappointement, dans une pièce de vers, où il traita Charles X de \emph{« roi-soliveau »} et ses ministres de malandrins, qui \emph{« vendraient la France aux cosaques et l’âme aux hiboux »}. Mais afin de conserver les pensions acquises, il garda ses vers en portefeuille jusqu’en 1866 : ils sont publiés dans \emph{Les Chansons des rues et des bois} sous le titre : « Écrit en 1827 ».\par
Il est regrettable que Victor Hugo, au lieu de prêter à sa mère ses opinions royalistes pour pallier son péché de royalisme, n’ait pas simplement avoué la vérité, qui était si honorable. En effet qu’y a-t-il de plus honorable que de gagner de l’argent ! Hugo vendait au roi et à ses ministres son talent lyrique, comme l’ingénieur et le chimiste louent aux capitalistes leurs connaissances mathématiques et chimiques, il détaillait sa marchandise intellectuelle en strophes et en odes, comme l’épicier et le mercier débitent leur cotonnade au mètre et leur huile en flacons. S’il avait confessé qu’en rimant l’ode {\itshape sur la naissance du duc de Bordeaux} ou l’ode sur son \emph{Baptême}, ou n’importe quelle autre de ses odes, il avait été inspiré et soutenu par l’espoir du gain, il aurait du coup conquis la haute estime  \phantomsection
\label{p14}de la Bourgeoisie, qui ne connaît que le {\itshape donnant-donnant} et {\itshape l’égal échange} et qui n’admet pas que l’on distribue des vers, des asticots ou des savates gratis {\itshape pro deo}. Convaincue que Victor Hugo ne faisait pas de « l’art pour l’art », mais produisait des vers pour les vendre, la bourgeoisie aurait imposé silence aux plumitifs envieux qui, sous Louis-Philippe, reprochaient à l’écrivain, ses gratifications royales.\par
Si le poète avait, sans ambages et détours exposé le véritable motif de sa conduite royaliste, il aurait rendu à la poésie française un service plus réel qu’en écrivant \emph{Hernani, Ruy Blas} et surtout la préface de \emph{Cromwell} : il aurait doté la France de plusieurs Hugo, bien qu’un seul suffise et au-delà à la gloire d’un siècle.\par
Baudelaire, cet esprit mal venu dans ce siècle de mercantilisme, ce mal appris qui abominait le commerce, se lamentait de ce que lorsque :\par


\begin{verse}
Le poète apparaît en ce monde ennuyé,\\
Sa mère épouvantée et pleine de blasphèmes,\\
Crispe ses poings vers Dieu qui la prend en pitié.\\
\end{verse}

\noindent Pourquoi, dans les familles bourgeoises, des imprécations et des colères accueillent le poète à sa naissance ? Parce que, on a si souvent répété que les poètes vivent dans la pauvreté et meurent à l’hôpital, comme Gilbert, comme Malfilâtre, que les pères et mères ont dû finir par croire que poésie était synonyme de misère. Mais si on leur avait prouvé que dans ce siècle du Progrès, les romantiques  \phantomsection
\label{p15}avaient domestiqué la muse vagabonde, qu’ils lui avaient enseigné l’art de \emph{« jouer de l’encensoir, d’épanouir la rate du vulgaire, pour gagner le pain de chaque soir\footnote{Baudelaire, \emph{Les Fleurs du mal}. (\emph{Bénédiction} ; \emph{La Muse vénale}).} »}, et si on leur avait montré le chef de l’école romantique recevant à vingt ans trois mille francs de pension pour des vers « somnifères » les parents, jugeant que la poésie rapportait davantage que l’élève des lapins ou la tenue des livres auraient encouragé, au lieu de réprimer, les velléités poétiques de leur progéniture\footnote{Cette impertinente épithète est de Stendhal, qui pas plus que Baudelaire n’entendait rien au commerce des lettres. \emph{« L’\emph{Edinburgh Review}, écrit-il, s’est complètement trompé en faisant de Lamartine le poète du parti {\itshape ultra}… le véritable poète du parti, c’est M. Hugo. Ce M. Hugo a un talent dans le genre de celui de Young, l’auteur des \emph{Night Thoughts}, il est toujours exagéré à froid… L’on ne peut nier au surplus, qu’il sache bien faire des vers français, malheureusement il est somnifère. »} \emph{Correspondance inédite de Stendhal}. Vol. I. 22.}.\par
La bourgeoisie industrielle et commerciale n’aurait pas attendu sa mort pour ranger Victor Hugo, parmi les plus grands hommes de son histoire, si elle avait connu les sacrifices héroïques qu’il s’imposa et les tortures mentales qu’il supporta pour acquérir ces deux pensions.
\subsection[{III}]{III}
\noindent Madame Hugo n’aimait pas Napoléon, elle choisissait pour amis ses ennemis ; après la défaite de Waterloo, afin de fouler aux pieds la couleur de l’Empire, elle se chaussa de bottines vertes, ce  \phantomsection
\label{p16}simple fait caractérise la nature violente de ses sentiments\footnote{\emph{Victor Hugo rac.} Vol. 1. 252.}. L’oncle et le père de Hugo nourrissaient de nombreux griefs contre l’empereur, qui refusa de confirmer ce dernier dans son grade de général, conféré par Joseph Lahorie, qui pendant sa réclusion de 18 mois aux Feuillantines, apprenait au jeune Victor à « lire Tacite », ne devait pas non plus, lui inculquer l’amour de Bonaparte, contre lequel il conspirait. Hugo devait donc épouser la haine de sa mère pour Napoléon, que partageaient son mari et ses amis, en même temps qu’il endossait ses opinions royalistes, Mais il fut réfractaire à toute influence, personne ne put lui imposer ses sentiments, ni père ni mère, ni oncle, ni amis : Napoléon et son extraordinaire fortune emplissaient sa tête ; \emph{« son image sans cesse ébranlait sa pensée »}. Tous les hommes de sa génération subirent cette action troublante. Il faut lire \emph{Rouge et Noir} pour comprendre à quel point Napoléon s’empara de l’imagination des hommes de vouloir et de pouvoir. Toute sa vie, il obséda Hugo : tout enfant, il était son idéal. Ses camarades d’école jouaient des pièces de théâtre de sa composition ou de celles de son frère Eugène. \emph{« Les sujets habituels de ces pièces étaient les guerres de l’empire… c’était Victor qui jouait Napoléon. Alors il couvrait de décorations sa poitrine rayonnante d’aigles d’or et d’argent\footnote{\emph{Victor Hugo rac.} Vol. I.}. »} En ces temps il songeait fort peu à la Vendée et à ses vierges martyres, à Henri IV et aux vertus des rois légitimes : Napoléon le possédait  \phantomsection
\label{p17}tout entier ; et oubliant les jeux de l’adolescence, il étudiait ses campagnes, et suivait sur la carte, la marche de ses armées.\par
Mais que son héros, battu à Waterloo, soit emprisonné à Sainte-Hélène, que son père, pour avoir refusé de rendre à l’étranger la forteresse de Thionville soit accusé de trahison, que Louis XVIII, fasse son entrée triomphale dans Paris, escorté de \emph{« cosaques énormes, roulant des yeux féroces sous des bonnets poilus, brandissant des lances rouges de sang et portant au cou des colliers d’oreilles humaines, mêlées de chaînes de montres\footnote{\emph{Victor Hugo rac.} Vol. I.} »} ; et le jeune poète, pare \emph{« sa boutonnière d’un lys d’argent »}, choisit pour sujet de sa première tragédie, une restauration, et injurie Bonaparte \emph{« ce tyran qui ravageait la terre\footnote{Pièce de vers \emph{Sur le bonheur de l’Étude}, envoyé au concours de poésie de 1817 : tout lui devenait occasion pour outrager son héros.} »}.\par
Et pendant dix ans, sans éprouver un moment de lassitude, il fit \emph{« tonner dans ses vers la malédiction des morts, comme un écho de sa fatale gloire\footnote{\emph{Odes et Ballades}. \emph{Les deux Îles}. Édit. de 1826.} »}. Il faut arriver à 1827, pour le voir dans son \emph{Ode à la Colonne}, essayer de glorifier indirectement l’Empire en glorifiant ses maréchaux ; mais pour se départir de la conduite qu’il s’était imposée et qu’il avait suivie avec tant de fermeté, Hugo avait une excuse. L’insulte faite par l’ambassade d’Autriche, aux maréchaux Soult et Oudinot, indigna si fortement l’armée et la cour, que les \emph{Débats} et les journaux royalistes prirent leur défense, en écrivant  \phantomsection
\label{p18}l’\emph{Ode à la Colonne}, il obéissait au mot d’ordre donné par le parti royaliste. \emph{Les Débats} l’insérèrent à leur troisième page.\par
Il serait difficile, si on ne connaissait les mœurs du temps et les qualités de la famille Hugo, de comprendre qu’un jeune homme, fût-il de génie, put posséder d’une manière si parfaite, l’art de se contenir et de dissimuler ses sentiments.\par
Les régimes politiques s’étaient succédés depuis 1789, avec une rapidité si vertigineuse, que l’art de renier ses opinions et de saluer le soleil levant, était cultivé comme une nécessité de la lutte pour l’existence\footnote{\noindent Les amateurs d’acrobatie politique trouveront dans le \emph{Dictionnaire des Girouettes} de Prosny d’Eppe et dans le \emph{Nouveau Dictionnaire des Girouettes de 1831}, de quoi exciter leur admiration la plus exigeante. Ils s’étonneront avec Chateaubriand \emph{« qu’il y ait des hommes, qui après avoir prêté serment à la République une et indivisible, au Directoire en cinq personnes, au Consulat en trois, à l’Empire en une seule, à la première Restauration, à l’acte additionnel, à la seconde Restauration, ont encore quelque chose à prêter à Louis-Philippe »}.\par
— « Hé, hé, disait en souriant Talleyrand, après avoir prêté serment à Louis-Philippe, Sire, c’est le treizième ! »
}. La famille Hugo excella dans cet art précieux. Quelques détails biographiques sur le général Hugo et sur son fils aîné, Abel, diminueront peut-être l’admiration des hugolâtres pour le génie machiavélique de leur héros ; mais permettront au psychologue de s’expliquer comment tant de diplomatie pouvait se loger dans un si jeune cerveau.\par
Brutus Hugo, le farouche républicain de 1793, qui pourvoyait de chouans et de royalistes les pelotons d’exécution et la guillotine, fructidorise le Corps législatif avec Augereau, prend du service dans le  \phantomsection
\label{p19}palais de Joseph, en qualité de majordome, troque son surnom romain, contre un titre de Comte espagnol, prête serment à Louis XVIII qui le décore de la croix de Saint-Louis, se rallie à Napoléon, débarqué à Cannes, offre de reprêter serment à Louis XVIII retour de Gand, qui le met à la retraite et l’interne à Blois ; là pour occuper ses loisirs, il écrit ses \emph{Mémoires}. Abel, son fils aîné, les enrichit d’un précis historique, débutant par cet acte de foi : \emph{« Attaché par conviction à la monarchie constitutionnelle, profondément pénétré du dogme de la légitimité, dévoué par sentiment à l’auguste famille qui nous a rendu, etc. »}\par
Victor Hugo ne pouvait se lasser d’admirer les exemples de conduite loyale que léguait à ses enfants l’ex-Brutus : il lui dit :\par


\begin{verse}
Va, tes fils sont contents de ton noble héritage,\\
Le plus beau patrimoine est un nom vénéré !\\
\end{verse}


\bibl{\emph{Odes}. Livre II. VIII. Édit. 1823.}
\noindent Abel, mort en 1873, vécut jusqu’en 1815 presque toujours auprès de son père : il ne pouvait donc rendre sa mère responsable de l’ultra-royalisme qui se révéla subitement dans ses écrits après la chute de l’Empire. Ainsi que Victor, il était spécialement attaché au service personnel de la famille royale. Tandis que Victor chante en vers le sacre du roi, il publie, en prose, \emph{La vie anecdotique du comte d’Artois, aujourd’hui Charles X} : \emph{« Aucun prince ne fut plus séduisant que le comte d’Artois… il est rempli de grâce, de franchise, de noblesse,  \phantomsection
\label{p20}etc. »} et cela continue ainsi pendant des dizaines de pages. Le roi encensé, il allonge son coup de pied à \emph{« cette révolution, qui se plongeait dans tous les crimes et rampait sous tous les maîtres »}, il insulte Buonaparte, se pâme à la lecture de {\itshape la proclamation à l’armée} du Comte d’Artois, lieutenant-général du royaume, envoyé à Lyon pour arrêter la marche de Napoléon, et il la commente ainsi : \emph{« Plus le langage était noble et délicat, moins il était propre à faire impression sur des esprits qui ne semblaient accessibles, qu’à celui de la séduction. Les traîtres n’y opposèrent qu’un rire moqueur. »} Son père, le général Hugo, était parmi ces traîtres. — Charles X exilé, Abel décoré par Louis-Philippe pour « services rendus par la plume », écrivit l’\emph{Histoire populaire de Napoléon} (1853), elle lui valut les chauds compliments du prince Napoléon.\par
Abel joignait à cette remarquable souplesse de conduite, un esprit commercial, fécond en ressources. Il publia pour répondre aux engouements du public et pour satisfaire ses goûts, des études sur le théâtre Espagnol, une édition du \emph{Romancero}, une brochure sur \emph{le Guano, sa valeur comme engrais}, un guide perpétuel de Paris : \emph{Tout Paris pour 12 sous}, un mémoire sur \emph{la période de Disette, qui menace la France}, une \emph{Histoire de France illustrée} ; il composa un vaudeville en collaboration avec Romieu ; il étudia \emph{L’Afrique} au point de vue agricole, créa le \emph{Journal du Soir}, inventa les publications illustrées, par livraison, etc. Abel était un habile industriel de lettres.\par
Mais ce à quoi on ne devait s’attendre, c’est de  \phantomsection
\label{p21}rencontrer chez le soldat des guerres de l’empire, cette humanitairie qui, sur la lyre de Victor devait se substituer au roi et au catholicisme. Sous le pseudonyme de Genty, le général Hugo publiait en 1818 une brochure où se mêlent avec bonheur les préoccupations de l’industriel et du philanthrope\footnote{\emph{Mémoire sur les moyens de suppléer à la traite des nègres par des individus libres, d’une manière qui garantisse pour l’avenir la sûreté des colons et la dépendance des colonies}, par Genty, in-8, janvier 1818. Blois, imprimerie Verdier.}. Il y résout ce double problème : donner une dot aux enfants trouvés, et procurer des travailleurs blancs aux planteurs, qui ne pouvaient plus, comme par le passé, aller chercher des noirs sur la côte africaine.\par
Les travailleurs blancs seraient pris aux Enfants trouvés. Le gouvernement élevant ces enfants à ses frais, peut en disposer à son gré : \emph{« il se chargerait de fournir aux colons, des enfants dans l’âge de 9 à 10 ans pour les filles, et de 10 à 11 ans pour les garçons. L’engagement pour tous prendrait la date même de leur embarquement et ne pourrait excéder 15 années, à l’expiration desquelles il cesserait de droit. L’administration ferait alors compter à ces enfants à titre de dot, savoir aux hommes 600 francs, et aux femmes 500 francs »}. Ce projet satisferait tout le monde, et lierait étroitement les colonies à la métropole. Les colons achetaient leurs négrillons des 2 et 4 cents francs : la mère patrie leur fournit les petits blancs gratis. Les enfants blancs qui résisteraient au régime des coups de fouet et de travail forcé des planteurs, recevraient au bout de 15 ans, une dot de 5 à 6 cents francs. La philanthropie bourgeoise qui a inventé la  \phantomsection
\label{p22}prison cellulaire, le travail forcé des femmes et des enfants dans les ateliers, qui valse et minaude dans les bals de charité pour apaiser la faim des affamés, devrait reprendre le projet du général Hugo et en faire le complément de la loi des récidivistes\footnote{\noindent Monsieur Belton qui a fait des recherches sur la famille Hugo, a découvert que le vieux général écrivait et rimait en diable. À sa mort il a laissé une liste de manuscrits : \emph{La Duchesse d’Alba}, \emph{le Tambour Robin}, \emph{l’Hermite du lac}, \emph{l’Épée de Brennus}, \emph{Perrine ou la Nouvelle Nina}, \emph{l’Intrigue de cour}, comédie en trois actes, \emph{la Permission}, \emph{Joseph ou l’Enfant trouvé}, etc., ces ouvrages sont perdus ou égarés.\par
Bien que Victor Hugo ne mentionne jamais les productions poétiques et romantiques de son père, il les admirait beaucoup. Dans une lettre adressée au général, et citée par M. Belton, il parle d’une pièce qui l’a \emph{« pénétré jusqu’au fond de l’âme »}, dans une autre, il mentionne un poème, \emph{Lucifer}, qui l’a \emph{« transporté »}. Si l’on ne connaissait sa piété filiale, on s’étonnerait qu’il ne se soit jamais occupé de sauver de l’oubli les œuvres « remarquables » de son père, lui qui a recueilli et si précieusement conservé ses moindres excréments littéraires, que pour leur péché d’hugolâtrie, Messieurs Vacquerie, Meurice et Lefebvre sont condamnés à publier, sinon à lire.
}.
\subsection[{IV}]{IV}
\noindent La révolution de 1830 désarçonne Victor Hugo, mais ne l’empêche pas de continuer, comme par le passé, à toucher ses trois mille francs de pension si honorablement gagnés. La préface des \emph{Feuilles d’Automne}, publiée en 1831, le montre hésitant, il avait noué des relations avec de jeunes et ardents républicains qui, pour l’attirer, le flattaient : ainsi la \emph{Biographie des contemporains} de Rabbe, dit que \emph{« Hugo avait chanté les trois jours dans les  \phantomsection
\label{p23}plus beaux vers qu’ils avaient inspirés »}. Mais les doctrines républicaines, qui ne savaient se donner du poids par des gratifications, pénétraient difficilement dans son cerveau : il n’eut pas besoin, comme le Marius des \emph{Misérables}, de monter sur les barricades et d’y recevoir des blessures pour se guérir de son néo-républicanisme. Dès qu’il comprend que le trône de Louis-Philippe est affermi, il déclare : \emph{« il nous faut la chose {\itshape république} et le mot {\itshape monarchie}\footnote{Victor Hugo. \emph{Philosophie et littérature mêlées}. 1834. Journal d’un révolutionnaire de 1830.} »}. Cette phrase qui paraîtra un plagiat du mot historique de Béranger, est une profession de foi : elle voulait dire, qu’il allait accepter les grâces et faveurs de la monarchie, tout en restant républicain dans son for intérieur. Sous Louis XVIII et Charles X, il adorait Napoléon dans son cœur, et l’insultait dans les vers publiés, pour plaire à ses patrons légitimistes. Le républicain flatta Louis-Philippe pour obtenir la pairie, comme le napoléonien adula les Bourbons pour arracher des pensions.\par
Le 21 juillet 1842, il eut le courage de jeter à la face de Louis-Philippe des phrases de ce calibre : \emph{« Sire, vous êtes le gardien auguste et infatigable de la nationalité et de la civilisation… Votre sang est le sang du pays, votre famille et la France ont le même cœur… Sire, vous vivrez longtemps encore, car Dieu et la France ont besoin de vous. »} Victor Hugo a toujours été cosmopolite : il unissait tous les rois d’Europe dans son adulation. Plus tard, après 1848, il parlera des États-Unis  \phantomsection
\label{p24}d’Europe. Mais auparavant il avait \emph{« béni l’avènement de la reine Victoria »} et célébré le Czar Nicolas \emph{« le noble et pieux empereur\footnote{Victor Hugo. \emph{Le Rhin}. Tom. III. 288, 331.} »}. En 1846, il priait le baron de Humboldt de remettre un de ses discours académiques \emph{« à son auguste roi, pour lequel, vous connaissez ma sympathie et mon admiration »}. Cette majesté si admirée était Guillaume IV, roi de Prusse et frère de l’empereur d’Allemagne, couronné à Versailles\footnote{Ces détails biographiques, que par une modestie déplacée, Victor Hugo supprima dans l’autobiographie, qu’il dicta à sa femme, ont été rétablis dans l’étude si érudite et si spirituellement écrite de M. Ed. Biré, \emph{Victor Hugo avant 1830}, J. Gervais, édit. 1883. On ne saurait trop en recommander la lecture aux Hugolâtres qui désirent connaître intimement leur héros.}. L’histoire ne raconte pas si le poète reçut des gratifications des Majestés-Unies d’Europe.\par
Enfin arrive le grand jour : Hugo reconquérant la liberté de sa pensée, ne sera plus obligé de flatter les rois en public et de chérir la république dans son for intérieur. La révolution de 1848 chasse \emph{« l’auguste gardien de la civilisation »} et juche au pouvoir les républicains du \emph{National}. Un instant on croit la régence possible, Victor Hugo s’empresse de la demander, place des Vosges ; on proclame la république, Victor Hugo, sans perdre une minute, se métamorphose en républicain. Les personnes qui s’arrêtent aux apparences, l’accuseront d’avoir varié, parce que tour à tour il fut bonapartiste, légitimiste, orléaniste, républicain ; mais une étude un peu attentive montre au contraire que sous tous ces  \phantomsection
\label{p25}régimes, il n’a jamais modifié sa conduite, que toujours, sans se laisser détourner par les avènements et les renversements de gouvernement, il poursuivit un seul objet, son intérêt personnel, que toujours il resta hugoïste, ce qui est pire qu’égoïste, disait cet impitoyable railleur de Heine, que Victor Hugo, incapable d’apprécier le génie, ne put jamais sentir.\par
Est-ce la faute à ce pauvre homme, si pour faire fortune, le but sérieux de la vie bourgeoise, il dut mettre à son chapeau toutes ces cocardes ? Si faute il y a, qu’elle retombe sur la bourgeoisie qui acclama et renversa successivement tous ces gouvernements. Hugo pâtit de ces variations politiques : jusqu’en 1830, il dut étouffer son ardente admiration pour Napoléon ; et jusqu’en 1848, il dut ensevelir son républicanisme sous des flatteries au roi, comme Harmodius cachait son poignard tyrannicide sous des fleurs.\par
Ils comprennent bien mal Hugo, ceux qui voient en lui un homme voué à la réalisation d’une idée : à ce compte sa vie serait un tissu de contradictions irréductibles. Il laissa ce rôle aux idéologues, aux hurluberlus qui rêvent leur vie ; il se contenta d’être un homme raisonnable, ne s’inquiétant, ni de l’effigie de ses pièces de cent sous, ni de la forme du gouvernement qui maintient l’ordre dans la rue, fait marcher le commerce, et donne des pensions et des places. Dans son autobiographie il déclare explicitement que \emph{« la forme du gouvernement lui semblait la question secondaire »}. Dans la préface des \emph{Voix intérieures} de 1837, il avait pris pour devise : \emph{« Être de tous les partis par leurs côtés généreux,  \phantomsection
\label{p26}(c’est-à-dire qui rapportent) ; n’être d’aucun par leurs mauvais côtés (c’est-à-dire qui occasionnent des pertes). »}\par
Hugo a été un ami de l’ordre : il n’a jamais conspiré contre aucun gouvernement, celui de Napoléon III excepté, il les a tous acceptés et soutenus de sa plume et de sa parole et ne les a abandonnés que le lendemain de leur chute. Sa conduite est celle de tout commerçant, sachant son métier : une maison ne prospère, que si son maître sacrifie ses préférences politiques et accepte le fait accompli. Les Dollfuss, les Kœchlin, les Scheurer-Kestner, ces républicains modèles de Mulhouse, la cité libre jusqu’en 1793, ne se sont-ils pas accommodés à tous les régimes qui, depuis près d’un siècle, se sont succédés en Alsace ; n’ont-ils pas reçu des subventions de l’empire et ne lui ont-ils pas réclamé des franchises douanières pour leur industrie et des mesures répressives contre leurs ouvriers ? Les affaires d’abord, la politique ensuite.\par
De 1848 à 1885, Hugo se comporte en « républicain honnête et modéré » et l’on peut défier ses adversaires de découvrir pendant ces longues années, un seul jour de défaillance.\par
En 1848, les conservateurs et les réactionnaires les plus compromis se prononcèrent pour la république que l’on venait de proclamer : Victor Hugo n’hésita pas une minute à suivre leur noble exemple. \emph{« Je suis prêt, dit-il, dans sa profession de foi aux électeurs, à dévouer ma vie pour établir la République qui multipliera les chemins de fer… décuplera la valeur du sol… dissoudra l’émeute… fera  \phantomsection
\label{p27}de l’ordre, la loi des citoyens… grandira la France, conquerra le monde, sera en un mot le majestueux embrassement du genre humain sous le regard de Dieu satisfait. »} Cette république est la bonne, la vraie, la république des affaires, qui présente « les côtés généreux » de sa devise de 1837.\par
— Je suis prêt continua-t-il, à dévouer ma vie pour \emph{« empêcher l’établissement de la république qui abattra le drapeau tricolore sous le drapeau rouge, fera des gros sous avec la colonne, jettera à bas la statue de Napoléon et dressera la statue de Marat, détruira l’Institut, l’École Polytechnique et la Légion d’honneur ; ajoutera à l’illustre devise : {\itshape Liberté, Égalité, Fraternité}, l’option sinistre : {\itshape ou la mort} ; fera banqueroute, ruinera les riches sans enrichir les pauvres, anéantira le crédit qui est la fortune de tous et le travail qui est le pain de chacun, abolira la propriété et la famille, promènera des têtes sur des piques, remplira les prisons par le soupçon et les videra par le massacre, mettra l’Europe en feu et la civilisation en cendres, fera de la France la patrie des ténèbres, égorgera la liberté, étouffera les arts, décapitera la pensée, niera Dieu »}. Cette république est la république sociale.\par
Victor Hugo a loyalement tenu parole. Il était de ceux qui fermaient les ateliers nationaux, qui jetaient les ouvriers dans la rue, pour noyer dans le sang les idées sociales, qui mitraillaient et déportaient les insurgés de juin, qui votaient les poursuites contre les députés soupçonnés de socialisme, qui soutenaient le prince Napoléon, qui voulaient un pouvoir fort pour contenir les masses, terroriser les  \phantomsection
\label{p28}socialistes, rassurer les bourgeois et protéger la famille, la religion, la propriété menacées par les communistes, ces barbares de la civilisation. Avec un courage héroïque, qu’aucune pitié pour les vaincus, qu’aucun sentiment pour la justice de leur cause n’ébranlèrent, Victor Hugo, digne fils du Brutus Hugo de 1793, vota avec la majorité, maîtresse de la force. Ses votes glorieux et ses paroles éloquentes sont bien connus ; ils sont recueillis dans les annales de la réaction qui accoucha de l’empire ; mais on ignore la conduite, non moins admirable de son journal, l’\emph{Événement} fondé le 30 juillet 1848, avec le concours de Vacquerie, de Théophile Gautier, et de ses fils ; elle mérite d’être signalée.\par
\emph{L’Événement} prenait cette devise, qui, après juin, était de saison : \emph{« Haine à l’anarchie — tendre et profond amour du peuple. »} Et pour qu’on ne se méprît pas sur le sens de la deuxième sentence, le numéro spécimen disait que \emph{L’Événement} \emph{« vient parler au pauvre des droits du riche, à chacun de ses devoirs}. Le numéro du premier novembre annonçait \emph{« qu’il est bon que le \emph{National} qui s’adresse à l’aristocratie de la République se donne pour 15 centimes, que l’\emph{Événement} qui veut parler au pauvre se vende pour un sou »}. Le poète commençait à comprendre que dans les petites bourses des pauvres, se trouvaient de meilleures rentes que dans les fonds secrets des gouvernements et les coffres-forts des riches.\par
Suivant l’exemple donné par les Thiers de la rue de Poitiers, car Victor Hugo imita toujours quelqu’un, l’\emph{Événement} endoctrine le peuple, répand  \phantomsection
\label{p29}dans les masses ouvrières les saines et consolantes théories de l’économie politique, réfute Proudhon, combat \emph{« le langage des flatteurs du peuple, qui le calomnient. Le peuple écoute ceux qui l’entretiennent des principes et des devoirs plus volontiers que ceux qui lui parlent de ses intérêts et de ses droits »}. (Numéro du 1\textsuperscript{er} novembre). Il se fait l’apôtre du libéralisme, cette religion bourgeoise qui amuse le peuple avec des principes, lui inculque des devoirs, et le détourne de ses intérêts et de ses droits ; qui lui fait abandonner la proie pour l’ombre.\par
Après l’insurrection de juin, il ne restait, selon Hugo, qu’un moyen de sauver la République : — la livrer à ses ennemis. Thiers pensait ainsi après la Commune. La \emph{Réforme} incapable de s’élever jusqu’à l’intelligence de cette machiavélique tactique, se plaignait de ce que \emph{« les républicains sont mis à l’index. On les fuit, on les renie, tandis qu’il n’y a pas de légitimistes ou d’orléanistes, si décriés, dont on n’épaule l’ignorance et qu’on n’essaie de réhabiliter à tout prix »}. L’\emph{Événement} lui rive son clou avec cette frappante réplique : « Si les républicains sont à ce point suspects, n’est-ce pas la faute des républicains ?… Le christianisme n’a été réellement puissant que lorsque les prêtres en ont perdu la direction. » (Numéro du 1\textsuperscript{er} août). Et pour protéger la République contre les républicains le journal de Victor Hugo entre en campagne contre Caussidière parce qu’il n’est pas \emph{« la tête, mais la main »} ; contre Louis Blanc, parce que \emph{« son crime, ce sont ses idées ; ses livres, ses discours ; ses complices, ce  \phantomsection
\label{p30}sont ses trois cent mille auditeurs ! »} (Numéro du 27 août) ; contre Proudhon parce qu’il est \emph{« un petit homme à figure commune ; un misérable avocat du peuple »} ; contre Ledru-Rollin parce que \emph{« ses circulaires ont plongé la civilisation dans une guerre civile de quatre jours. Depuis le 24 février jusqu’au 24 juin M. Ledru-Rollin a été un de ceux qui ont le plus contribué à frayer la route à l’abîme »}. (Numéro du 6 août).\par
Mais c’est en poursuivant de ses injures, de ses colères et de ses dénonciations les vaincus de juin, que l’\emph{Événement} donne la mesure de son profond amour pour la République. Écoutez, c’est l’auteur des \emph{Châtiments} qui parle : « Hier, au sortir de la plus douloureuse corruption, ce qui se déchaîna, ce fut la cupidité ; ceux qui avaient été les pauvres n’eurent qu’une idée, dépouiller les riches. On ne demanda plus la vie, on demanda la bourse. La propriété fut traitée de vol ; l’État fut sommé de nourrir à grands frais la fainéantise ; le premier soin des gouvernants fut de distribuer, non le pouvoir du roi, mais les millions de la liste civile, et de parler au peuple non de l’intelligence et de la pensée mais de la nourriture et du ventre… Oui, nous sommes arrivés à ce point que tous les honnêtes gens, le cœur navré et le front pâle, en sont réduits à admettre les conseils de guerre en permanence, les transportations lointaines, les clubs fermés, les journaux suspendus et la mise en accusation des représentants du peuple. » (Numéro du 28 août).\par
La dure nécessité qui navrait le cœur des honnêtes gens et l’endurcissait pour la répression impitoyable,  \phantomsection
\label{p31}obligeait Hugo à mentir impudemment.\par
Le 28 août 1848, Victor Hugo, pour exciter les conseils de guerre à condamner sans pitié, dénonce les vaincus comme des \emph{« pauvres qui n’eurent qu’une idée : dépouiller les riches »}. Deux mois auparavant, les pillards de juin avaient envahi sa maison. Ils savaient qu’il était \emph{« un des soixante représentants envoyés par la Constituante pour réprimer l’insurrection et diriger les colonnes d’attaques »}. Ils fouillèrent les appartements pour chercher des armes ; ils virent pendu au mur \emph{« un yatagan turc, dont la poignée et le fourreau étaient en argent massif… rangés sur une table, des bijoux, des cachets précieux en or et en argent… quand ils furent partis, on constata… que ces mains noires de poudre n’avaient touché à rien. Pas un objet précieux ne manquait »}. Ce sont là les propres paroles de Victor Hugo, narrant le sac de sa maison par les pillards de juin. Mais pour raconter la scène, il attendit que les conseils de guerre eussent terminé leur œuvre de répression ; il était alors exilé. — Victor Hugo reste toujours le même, au milieu des circonstances les plus diverses : pendant la restauration légitimiste, il insulte Napoléon, qui l’enthousiasme, pendant la réaction bourgeoise, il calomnie les insurgés, dont il admire les actes de délicate probité.\par
Une étrange fatalité pesa sur Victor Hugo ; toute sa vie, il fut condamné à dire et à écrire le contraire de ce qu’il pensait et ressentait.\par
En exil, pour plaire à son entourage, il pérora sur la liberté de la presse, de la parole et bien d’autres  \phantomsection
\label{p32}libertés encore ; cependant il ne détestait rien plus que cette liberté, qui permet \emph{« aux démagogues forcenés, de semer dans l´âme du peuple des rêves insensés, des théories perfides… et des idées de révolte »}. (\emph{Événement} du 3 novembre). L’insurrection abattue, la Chambre vota le cautionnement qui commandait \emph{« silence aux pauvres ! »} selon l’expression de Lamennais. L’\emph{Événement} s’empressa, ainsi que les \emph{Débats}, le \emph{Constitutionnel} et le \emph{Siècle} d’approuver cette \emph{« mesure si favorable à la presse sérieuse. Nous la considérons… comme nécessaire… la Société avait une liberté gangrenée ; le cautionnement ce chirurgien redouté vient d’opérer le corps social »}. (Numéro du 11 août). Le libertaire Hugo n’était pas homme à hésiter devant l’amputation de toute liberté qui inquiète la classe possédante et trouble les cours de la bourse.\par
Victor Hugo commit alors la grande bévue de sa vie politique : il prit le prince Napoléon pour un imbécile, dont il espérait faire un marchepied. D’ailleurs c’était l’opinion générale des politiciens sur celui que Rochefort devait surnommer le Perroquet mélancolique : car même dans l’erreur, Hugo ne fut pas original, en se trompant il imitait quelqu’un. Il était si absorbé par le désir de se caser dans un ministère bonapartiste, qu’il ne s’aperçut pas que les Morny, les Persigny et les autres Cassagnac de la bande avaient accaparé l’imbécile et qu’ils entendaient s’en réserver l’exploitation. Ces messieurs, avec un sans-gêne qui l’étonna et le choqua grandement, l’envoyèrent potiner dans sa petite succursale de la rue de Poitiers et escamotèrent à son nez et à  \phantomsection
\label{p33}sa barbe le ministère si ardemment convoité. Au lieu d’embourser son mécompte et de contenir son indignation comme s’était son habitude, il s’oublia et se jeta impétueusement dans l’opposition. Les républicains de la Chambre, manquant d’hommes, l’accueillirent malgré son passé compromettant et le sacrèrent chef. Grisé il rêva la présidence.\par
Le coup d’État qui surprit au lit les chefs républicains, dérangea ses plans, il dut suivre en exil ses partisans, puisqu’ils l’avaient promu chef. Les chenapans qui, à l’improviste, s’étaient emparés du gouvernement, étaient si tarés, leur pouvoir semblait si précaire, que les bourgeois républicains balayés de France, ne crurent pas à la durée de l’Empire. Durant des semaines et des mois, tous les matins, tremblants d’émotion, ils dépliaient leur journal pour y lire la chute du gouvernement de décembre et leur rappel triomphal ; ils tenaient leurs malles bouclées pour le voyage. Ces républicains bourgeois qui avaient massacré et déporté en masse les ouvriers, assez naïfs, pour réclamer à l’échéance les réformes sociales qui devaient acquitter les trois mois de misère, mis au service de la République, ne comprenaient pas que le Deux Décembre était la conséquence logique des journées de juin. Ils ne s’apercevaient pas encore que lorsqu’ils avaient cru ne mitrailler que des communistes et des ouvriers, ils avaient tué les plus énergiques défenseurs de leur République. Victor Hugo, qui était incapable de débrouiller une situation politique, partagea leur aveuglement ; il injuria en prose et en vers le peuple parce qu’il ne renversait pas à l’instant l’Empire  \phantomsection
\label{p34}que lui et ses amis avaient fondé et consolidé dans le sang populaire.\par
Jeté à bas de ses rêves ambitieux et enfiévré par l’attente incessante de la chute immédiate de Napoléon III, Hugo pour la première et l’unique fois de sa vie lâche la bride aux passions turbulentes qui angoissaient son cœur. Déçu dans ses ambitions personnelles, il s’attaque furibondement aux personnes, aux Rouher, aux Maupas, aux Troplong, qui culbutèrent ses projets : il les prend à bras le corps, les couvre de crachats, les mord, les frappe, les terrasse, les piétine avec une fureur épileptique. Le poète est sincère dans les \emph{Châtiments} : il est là tout entier avec sa vanité blessée, son ambition trompée, sa colère jalouse et son envie rageuse. Ses vers que les amplifications oiseuses et des comparaisons étourdissantes rendent d’ordinaire si froids, s’animent et vibrent de passion. On y dégage, sous des charretées de fatras romantique, des vers acérés comme des poignards et brûlants comme des fers rouges ; des vers que répétera l’histoire. \emph{Les Châtiments}, l’ouvrage le plus populaire de Victor Hugo, apprit à la jeunesse de l’Empire la haine et le mépris des hommes de l’Empire.\par
Il est des hugolâtres de bonne compagnie, monarchistes, voire même républicains, qui s’effarent aux engueulades des \emph{Châtiments} : ils n’en parlent jamais ou si parfois ils les mentionnent, c’est avec des précautions oratoires et des réticences infinies. Leur pudibonderie les empêche de reconnaître les services que ce pamphlet enragé rendit et rend encore aux conservateurs de toute provenance. Hugo agonise  \phantomsection
\label{p35}d’insultes les Canrobert et les Saint-Arnaud de la troupe bonapartiste de décembre ; mais il ne décoche pas un seul vers aux Cavaignac, aux Bréa et aux Clément Thomas de la bande bourgeoise de juin. Massacrer les socialistes en blouse, lui semble dans l’ordre des choses, mais charger sur le boulevard Montmartre, emporter d’assaut la maison Sallandrouze, canarder quelques bourgeois en frac et chapeau gibus ! Oh ! le plus abominable des crimes ! \emph{Les Châtiments} ignorent Juin et ne dénoncent que Décembre : en concentrant les haines sur Décembre, ils jettent l’oubli sur Juin.\par
Dans sa préface du \emph{18 Brumaire}, Karl Marx dit à propos de \emph{Napoléon le Petit} : \emph{« Victor Hugo se borne à des invectives amères et spirituelles contre l’éditeur responsable du coup d’État. Dans son livre l’événement semble n’être qu’un coup de foudre dans un ciel serein, que l’acte de violence d’un seul individu. Il ne remarque pas qu’il grandit cet individu, au lieu de le rapetisser, en lui attribuant une force d’initiative propre, telle qu’elle serait sans exemple dans l’histoire du monde. »} Mais en magnifiant, sans s’en douter, Napoléon le Petit en Napoléon le Grand, en empilant sur sa tête les crimes de la classe bourgeoise, Hugo disculpe les républicains bourgeois qui préparèrent l’empire et innocente les institutions sociales qui créent l´antagonisme des classes, fomentent la guerre civile, nécessitent les coups de force contre les socialistes et permettent les coups d’État contre la bourgeoisie parlementaire. En accumulant les colères sur les individus, sur Napoléon et ses acolytes, il détourne  \phantomsection
\label{p36}l’attention populaire de la recherche des causes de la misère sociale, qui sont l’accaparement des richesses sociales par la classe capitaliste ; il détourne l’action populaire de son but révolutionnaire, qui est l’expropriation de la classe capitaliste et la socialisation des moyens de production. — Peu de livres ont été plus utiles à la classe possédante que \emph{Napoléon le Petit} et \emph{Les Châtiments}.\par
D’autres hugolâtres, panégyristes maladroits, prenant au sérieux les déclarations de dévouement et de désintéressement du poète, le représentent comme un héros d’abnégation ; — ils le dépouillent de son prestige bourgeois, par simplicité. À les entendre, il aurait été un de ces maniaques dangereux, entichés d’idées sociales et politiques, au point de leur sacrifier les intérêts matériels ; ils voudraient l’assimiler à ces Blanqui, à ces Garibaldi, à ces Varlin, à ces fous qui n’avaient qu’un but dans la vie, la réalisation de leur idéal. — Non, Victor Hugo n’a jamais été assez bête pour mettre au service de la propagande républicaine, même quelques milliers de francs de ses millions ; — s’il avait sacrifié n’importe quoi pour ses idées, un cortège de bourgeois, aussi nombreux, ne l’aurait pas accompagné au Panthéon ; M. Jules Ferry lui souhaitant sa fête, deux ans avant sa mort, ne l’aurait pas salué du nom de Maître. Si Victor Hugo avait fait de cette politique de casse-cou, il serait sorti de la tradition bourgeoise. Car la caractéristique de l’évolution politique dans les pays civilisés, est de débarrasser la politique des dangers qu’elle présentait et des sacrifices qu’elle exigeait autrefois.  \phantomsection
\label{p37}En France, en Angleterre, aux États-Unis les ministres au pouvoir et les élus à la Chambre et aux Conseils municipaux, ne se ruinent plus, mais s’enrichissent : dans ces pays on ne condamne plus des ministres pour tripotages boursicotiers, malversations financières et abus de pouvoir. La responsabilité parlementaire couvre leurs fautes et les protège contre toute poursuite. La France républicaine a donné un mémorable exemple de cette politique raisonnable et agréable le jour qu’elle éleva au rang de sénateurs MM. Broglie et Buffet pour les consoler d’avoir échoué dans leur tentative du coup d’État monarchiste. La politique parlementaire est une carrière lucrative : elle n’offre aucun des risques pécuniers du commerce et de l’industrie ; un petit capital d’établissement, une bonne provision de bagout, un brin de chance et beaucoup d’entregent y assurent le succès. Hugo ne connaissait que cette politique positive. Dès qu’il se convainquit que l’existence de l’empire était assurée pour un long temps, il éteignit ses foudres justiciardes et concentra toute son activité à son commerce d’adjectifs et de phrases rimées et rythmées.\par
Il avait dans son aveugle emportement lancé des déclarations si catégoriques, et pour son malheur elles eurent un retentissement si considérable ; il avait marqué les hommes du coup d’État de vers si cuisants, qu’il était impossible de les faire oublier ; il lui fallut rester républicain et renoncer à la politique ; il jugea qu’il valait mieux accepter  \phantomsection
\label{p38}bravement le rôle de martyr de la République, de victime du Devoir. Le rôle séduisait sa vanité. S’il n’était pas né dans une île, ainsi que Napoléon, il allait vivre exilé dans une île ainsi que lui. Imiter Napoléon, devenir le Napoléon des lettres, berça l’ambition de toute sa vie.\par
Les proscrits coudoient toutes les misères, disait le grand Florentin ; mais Hugo avait plus d’intelligence que Dante. Avec un art que n’égala jamais Barnum, il fit de l’exil la plus retentissante des réclames. L’exil était l’enseigne criarde et aveuglante accrochée à sa boutique de librairie de Haute-Ville House. Les rois ne l’avaient pensionné que d’une somme de 3 000 francs ; sa clientèle bourgeoise lui valait cinquante mille francs par an. Il n’avait pas perdu au change. Il trouva que l’Empire avait du bon : \emph{« Napoléon a fait ma fortune »}, avouait-il dans un de ces rares moments, où il déposait sa couronne d’épines. Comment la bourgeoisie bourgeoisante ne s’extasierait-elle pas devant cet homme, qui avait su rendre l’exil si doux et si profitable ? — Les génies que l’on renomme ne savent trouver que douleurs dans l’exil, les commerçants qui s’expatrient au Sénégal, aux Indes, ces pays de fièvres et d’hépatites, après des dix et vingt ans d’exil ne parviennent à amasser qu’une pelote de quelques centaines de mille francs, s’ils ont en poupe le vent de la chance ; et lui Victor Hugo, le Prométhée moderne, vit dans une île délicieuse, où les médecins envoient leurs invalides, il s’entoure d’une cour d’adulateurs empressés, qui le font mousser, il voyage  \phantomsection
\label{p39}tranquillement en Europe, il thésaurise des millions et il obtient la palme du martyre !…\par
Les amis et les adversaires de Victor Hugo, ont accrédité des jugements téméraires portés sur lui par la crainte et l’admiration : dans l’intérêt de sa gloire il est nécessaire de les réviser.\par
La phraséologie fulgurante du Hugo des trente-cinq dernières années donne la chair de poule aux trembleurs qu’épouvantent les mots ; aux Prudhommes, pour qui tout saltimbanque, jonglant avec les vocables Liberté, Égalité, Fraternité, Humanité, Cosmopolitisme, États-Unis d’Europe, Révolution et autres balançoires du libéralisme, est un révolutionnaire, un socialiste bon à coffrer, sinon à fusiller. Mais Hugo, et c’est là son plus sérieux titre à la gloire, sut mettre en contradiction si flagrante ses actes et ses paroles, qu’il ne s’est pas encore rencontré en Europe et en Amérique un politicien pour démontrer d’une manière plus éclatante la parfaite innocuité des truculentes expressions du libéralisme.\par
Ainsi que l’on se nourrit de pain et de viande, Hugo se repaît d’Humanité et de Fraternité. — Le 14 août 1848, huit jours après le départ du premier convoi, qui transportait 581 insurgés, il fonda à côté de la Réunion de la rue de Poitiers la {\itshape Réunion de la Fraternité}. La peur de perdre leur cher argent, que les Pereire et les Mirès de la finance impériale, devaient confisquer si allègrement, avait enragé les petits bourgeois de 1848. La presse honnête et modérée racontait sur les insurgés des histoires épouvantables : — Maisons pillées, mobiles sciés entre deux planches, crânes qu’on emplissait de vin  \phantomsection
\label{p40}et qu’on vidait en chantant des obscénités… Hugo savait que si les insurgés envahissaient les maisons, ils ne les pillaient pas ; il les avait vus se battre en héros. La simple humanité lui commandait de protester contre ces idiotes calomnies et d’essayer d’apaiser ces bourgeois apeurés, réclamant une impitoyable répression. Mais la Fraternité hugoïste n’était pas de composition si humaine, elle n’entendait pas suspendre l’action des conseils de guerre, « mais tempérer l’œil du juge par les pleurs du frère… et tâcher de faire sentir jusque dans la punition la fraternité de l’assemblée ». (\emph{Événement}, nº 14). — Et dans presque tous les numéros, l’\emph{Événement} continuait à exciter les colères et les peurs contre les vaincus\footnote{Cette fraternité pleurarde de crocodile reprocha à un poète qui ne se dégrada jamais jusqu’à pincer de la guitare philanthropique, à Alfred de Musset, d’avoir envoyé « aux victimes de juin » un prix de 1 300 francs que venait de lui accorder l’Académie. L’\emph{Événement} du 23 août commentait ainsi l’acte : \emph{« qu’il nous soit permis de faire observer à M. de Musset que sa détermination ne remplit nullement le but du legs fait par M. le comte de Latour-Landry. C’était à un poète peu favorisé de la fortune et non à une œuvre patriotique que le don devait appartenir »}.}.\par
La liberté était un des Pégases, qu’enfourchait Hugo. Mais il faut être par trois fois Prudhomme pour ne pas s’apercevoir que le Pégase hugoïste était trop gonflé de vent pour prendre le mors aux dents et lancer des pétarades. La fougueuse liberté de Hugo était un humble bidet, qu’il remisait dans l’écurie de tous les gouvernements. Depuis l’immortelle révolution de 1789, Liberté, Liberté ché-ri-e, est le refrain à la mode. Tous les politiciens depuis Polignac jusqu’à Napoléon le Petit l’ont répété sur  \phantomsection
\label{p41}tous les tons. Hugo le chantait à plein gosier quand il approuvait le cautionnement qui amputait du corps social la « liberté gangrenée » de la presse.\par
Hugo planta dans ses vers la rouge cocarde de l’Égalité. Mais il y a égalité et égalité comme poètes et poètes : il en existe autant que de morales. Toute classe, tout corps social fabrique à l’usage de ses membres une morale spéciale. La morale du commerçant, l’autorise à vendre sa marchandise dix et vingt fois au-dessus de sa valeur, s’il le peut ; celle du juge d’instruction l’incite à user de la ruse et du mensonge pour forcer le prévenu à s’accuser ; celle de l’agent de mœurs l’oblige à faire violer médicalement les femmes qu’il soupçonne de travailler avec leur sexe ; celle du rentier le dispense d’obéir au commandement biblique : — \emph{« Tu gagneras ton pain à la sueur de ton front… »} La mort établit à sa façon une égalité ; la grosse et la petite vérole en créent d’autres ; les inégalités sociales ont mis au monde deux égalités de belle venue : l’égalité du ciel, qui pour les chrétiens compense les inégalités de la société et l’égalité civile, cette très sublime conquête de la Révolution sert aux mêmes usages. Cette égalité civile, qui conserve aux Rothschild leurs millions et leurs parcs, et aux pauvres leurs haillons et leurs poux, est la seule égalité que connaisse Hugo. Il aimait trop ses rentes et les antithèses pour désirer l’égalité des biens qui du coup lui eût enlevé ses millions et dérobé les plus faciles et les plus brillants contrastes de sa poétique.\par
Bien au contraire, l’\emph{Événement} du 9 septembre 1848 prenait la défense du \emph{« luxe que calomniait la  \phantomsection
\label{p42}fausse philanthropie de nos jours »} et démontrait triomphalement la nécessité de la misère pour arriver à l’équilibre social. — \emph{« L’opulence oisive est la meilleure amie de l’indigence laborieuse, développe le journal hugoïste. Qui est-ce qui fournit à la richesse ce ruineux superflu ; cette recherche, ce colifichet dont se compose la mode et le plaisir ? Le travail, l’industrie, l’art, c’est-à-dire la pauvreté. Le luxe est la plus certaine des aumônes, c’est une aumône involontaire. Les caprices du riche sont les meilleurs revenus du pauvre. Plus le salon aura de plaisir, plus l’atelier aura de bien-être. Mystérieuses balances qui mesurent les plus lourdes nécessités d’une partie de la société aux plus légères frivolités de l’autre ! Équilibre étrange qui s’établit entre les fantaisies d’en haut, et les besoins d’en bas ! Plus il y a de fleurs et de dentelles dans le plateau qui monte, plus il y a de pain dans le plateau qui descend ! »} — Le gaspillage le plus inutile et le plus ridicule devient une des voies mystérieuses de la divine providence pour créer l’harmonie sociale, basée sur la misère besogneuse et la richesse oisive. Jamais le luxe n’a été plus magnifiquement glorifié. Lorsque l’\emph{Événement}, l’organe de la Fraternité hugoïste, publia son apologie du luxe, deux mois à peine s’étaient écoulés depuis l’insurrection de juin, ce \emph{« protêt de la misère »} et le sang de la guerre civile rougissait encore le pavé des rues.\par
Les mots dont Hugo enrichit son vocabulaire après 1848, lui portèrent tort dans l’esprit des Prudhommes : ils les ahurissaient au point de leur faire prendre des vessies pour des lanternes et l’écrivain  \phantomsection
\label{p43}pour un socialiste, pour un partageux. Victor Hugo partageux ! — Mais plutôt que de partager quoi que ce soit avec qui que ce soit, il aurait immolé de sa main tous ses exécuteurs testamentaires et tout le premier son cher, son bien-aimé Vacquerie, qui ne pouvant se tuer sur son catafalque ainsi que les serviteurs sur les bûchers des héros antiques voulut être enseveli en effigie dans le tombeau du maître. Le poète était digne d’un tel sacrifice : Hugo fut en effet un héros de la phrase.\par
La révolution de 1848 lança dans la langue honnête et modérée un peuple nouveau de mots ; depuis la réaction littéraire commencée sous le consulat, ils dormaient dans les discours, les pamphlets, les journaux et les proclamations de la grande époque révolutionnaire et ne s’aventuraient en plein jour que timidement, dans le langage populaire. Les bravaches du romantisme, les Janin, les Gautier, reculèrent épouvantés ; mais Hugo ne cligna pas de l’œil, il empoigna les substantifs et les adjectifs horrifiants, qui envahissaient la langue écrite dans les journaux et parlée à la tribune des assemblées populaires ; et prestidigitateur merveilleux il jongla à étourdir les badauds, avec les immortels principes de 1789 et les mots teints encore du sang des nobles et des prêtres. Il ouvrit alors au romantisme une carrière qu’il fut seul à parcourir ; ses compagnons littéraires de 1832, plus timides que les bourgeois dont ils s’étaient moqués, n’osèrent pas suivre celui qu’ils appelaient leur maître.\par
Victor Hugo, lui-même, semble, avoir été intimidé par les expressions révolutionnaires qu’il maniait et  \phantomsection
\label{p44}dont il ne comprenait pas exactement le sens. Il voulut s’assurer de n’avoir commis, par erreur, même en pensée, de péché socialiste ; il fit son examen de conscience dans son autobiographie et il se convainquit que lui qui avait écrit sur les pauvres gens, la misère, et autres sujets de compositions rhétoriciennes, des tirades à paver le Palais-Bourbon, il n’avait demandé qu’une seule réforme sociale, l’abolition de la peine de mort \emph{« la première de toutes, — peut-être\footnote{\emph{Victor Hugo raconté}, etc. Tome II.} »}. Et encore il pouvait se dire qu’il n’avait fait que suivre l’exemple de tous les apôtres de l’humanitairie, depuis Guizot jusqu’à Louis-Philippe ; et que tout d’abord il n’avait envisagé la peine de mort qu’à un point de vue littéraire et fantaisiste, comme un excellent thème à déclamation verbeuse, à ajouter aux « croix de ma mère » — « la voix du sang » et autres trucs du romantisme qui commençaient à s’user et à perdre leur action sur le gros public.\par
Un socialisme qui se limite à cette réforme sociale pratique : l’abolition de la peine de mort, n’est de nature qu’à inquiéter les bourreaux, dont il menace les droits acquis. Et cela ne doit pas étonner, si lors de la publication de la « bible socialiste » de Hugo, \emph{les Misérables}, il ne se soit trouvé que Lamartine vieilli pour se scandaliser, que, trente ans après Eugène Sue, \emph{« le seul homme, qui selon Th. De Banville avait quelque chose à dire »}, osât s’apitoyer sur un homme envoyé aux galères pour le vol d’un pain et sur une pauvre fille se prostituant pour  \phantomsection
\label{p45}nourrir le bâtard du bourgeois qui l’a abandonnée enceinte. C’était en effet vieillot et enfantin. Mais là où Victor Hugo étale grossièrement son esprit bourgeois, c’est lorsqu’il personnifie ces deux institutions de toute société bourgeoise, la police et l’exploitation, dans deux types ridicules : Javert, la vertu faite mouchard, et Jean Valjean, le galérien qui se réhabilite en amassant en quelques années une fortune sur le dos de ses ouvriers. La fortune lave toutes les taches et tient lieu de toutes les vertus. Hugo, ainsi que tout bourgeois, ne peut comprendre l’existence d’une société sans police et sans exploitation ouvrière.\par
L’adoration du Dieu-Propriété, c’est la religion de Victor Hugo. À ses yeux, la confiscation des biens de la famille d’Orléans est un des plus affreux crimes de Napoléon III. Et s’il avait été membre de l’assemblée de Versailles, il aurait, sur la proposition de M. Thiers, voté les 50 millions d’indemnité aux d’Orléans, par respect pour la propriété. Sa haine des socialistes, qu’il dénonça si férocement en 1848, est si intense, que dans sa classification des êtres, qui troublent la société, il place au dernier échelon Lacenaire, l’assassin, et immédiatement au-dessus, Babeuf\footnote{[NdE] Orthographié \emph{Babœuf.}}, le communiste\footnote{\emph{« Plus bas que Marat, plus bas de Babeuf, il y a la dernière sape et de cette cave sort Lacenaire. »} \emph{Les Misérables}. Tome VI, page 61-62, première édition.}.\par
Des gens qui seraient de la plus atroce mauvaise foi, s’ils n’étaient des ignorants et des oublieux, ont prétendu que l’homme qui, en novembre 1848, écrivait que \emph{« l’insurrection de juin est criminelle et  \phantomsection
\label{p46}sera condamnée par l’histoire, comme elle l’a été par la société… ; si elle avait réussi, elle n’aurait pas consacré le travail, mais le pillage »} (\emph{Événement}, nº 94), que cet homme avait déserté la cause de la sacrée propriété et pris la défense de l’insurrection du 18 mars. Et cela parce qu’il avait ouvert sa maison de Bruxelles aux réfugiés de la Commune. Mais dans sa bruyante lettre, tout chez Hugo est réclame, et plus tard dans son \emph{Année terrible}, n’a-t-il pas protesté avec indignation contre les actes de guerre de la Commune ; n’a-t-il pas injurié les Communards aussi violemment qu’autrefois les Bonapartistes, les stigmatisant avec les épithètes de fusilleurs d’enfants de quinze ans, de voleurs, d’assassins, d’incendiaires ? Mais les radicaux et le si hugolâtre Camille Pelletan, ont dû trouver que Victor Hugo les compromettait par son incontinence d’insultes et de calomnies contre les vaincus de la Semaine sanglante.\par
Qu’y avait-il donc de si extraordinaire dans l’acte de Victor Hugo, pour troubler ainsi les Pessard de la presse versaillaise. Est-ce que malgré les pressantes sollicitations de MM. Thiers et Favre, les ministres de la reine Victoria et du roi Amédée n’ont pas ouvert leurs pays, l’Angleterre et l’Espagne, à ces vaincus, qu’ils n’ont jamais insultés ainsi que Victor Hugo. Personne n’accusera ces hommes d’État de pactiser avec les socialistes et les ennemis de la propriété. En Suisse, en Belgique, en Angleterre, partout enfin, des bourgeois, tout ce qu’il y a de plus bourgeois, n’ont-ils pas ouvert leurs bourses, pour secourir les proscrits sans pain et sans travail, ce que n’a jamais fait Victor Hugo, l’ex-proscrit millionnaire ?
 \phantomsection
\label{p47}\subsection[{[V]}]{[V]}
\noindent Que les légitimistes, qui avaient nourri, choyé, prôné, décoré Victor Hugo, conservent pieusement une amère rancune contre le jeune Éliacin, qui les lâche dès que la révolution de 1830 leur arrache des mains la clef de la cassette aux pensions, rien de plus naturel. Qu’ils l’accusent de désertion, de trahison, rien de plus juste. Cependant, le pair de France de la monarchie orléaniste, qui faisait porter à sa mère le poids de son royalisme, eût pu expliquer son orléanisme par son amour de la morale et leur dire : « Moi, l’homme toujours fidèle au devoir j’ai dû obéir aux commandements d’une morale plus haute que la reconnaissance : j’ai obéi aux injonctions de la morale pratique : pas d’argent, pas de suisse, ni de poète. » Mais les anciens patrons de l’écrivain dépassent toute mesure, quand pour nuire à l’écoulement de sa marchandise parmi les gens pieux, ils le calomnient et l’appellent un impie. Rien de plus faux.\par
Victor Hugo eut le malheur de naître de parents impies, et d’être élevé au milieu des impies. Sa mère ne lui permit pas de manger du Bon Dieu\footnote{La brigande Vendéenne était une Voltairienne décidée : À Madrid, elle plaça ses enfants au collège des nobles, mais \emph{« s’opposa énergiquement, malgré la résistance des prêtres directeurs, à ce qu’ils servissent la messe comme les autres élèves et défendit même qu’on fît confesser et communier ses enfants »}. (\emph{Victor Hugo rac.} Vol. I. 194).}, mais lui donna, en revanche, pour  \phantomsection
\label{p48}professeurs, des prêtres sceptiques, qui pendant la Révolution avaient jeté aux orties la soutane et le bréviaire. Et cependant une foi ardente s’éveille subitement dans son âme, le jour même que le trône et l’autel, l’un supportant l’autre, sont replacés sur leurs pieds. Il étrangle alors son voltairianisme et chante la religion catholique, ses pompes et ses pensions\footnote{Dans une épître en vers de 1818, mais publiée en 1863, Hugo dit en parlant de lui-même : \emph{« … J’ai seize ans… Je lis l’Esprit des lois et j’admire Voltaire. »} (\emph{Victor Hugo rac.} Tome I. 308).}. Les légitimistes ne reconnaissent-ils pas là le signe certain d’une foi sincèrement opportuniste ? Ils se montrent exigeants à l’extrême, quand ils demandent que ce catholicisme d’occasion survive aux causes qui l’avaient engendré. Ils n’avaient qu’à rester les maîtres du pouvoir, pour que Hugo conservât jusqu’à sa quatre-vingt-troisième année, la foi au Dieu des prêtres : mais il dût se rendre à l’évidence et suspendre son culte pour ce Dieu qui cessait de révéler sa présence réelle par la distribution de pensions. C’est ainsi qu’un banquier coupe le crédit de son client ruiné, filant sur la Belgique.\par
La Révolution de 1830 remit à la mode Voltaire et la libre-pensée ; Victor Hugo, ce tournesol, que sa nature condamnait à tourner avec le soleil, déposa, comme une cuisinière son tablier, son légitimisme et son catholicisme de circonstance. Il avait de nouveaux maîtres à satisfaire. Il adora le \emph{Dieu des bonnes gens} de Béranger et brûla Jéhovah, le Dieu farouche et sombre, qui cependant  \phantomsection
\label{p49}convenait mieux à son cerveau romantique. Ce changement de Dieux prouve la sincérité de son déisme. Il lui fallait à tout prix un Dieu ; il en avait besoin pour son usage personnel, pour être un prophète, pour être un trépied\footnote{\emph{« Le Poète est lui-même un trépied. Il est le trépied de Dieu. »} \emph{William Shakespeare}, par V. Hugo, p. 53.}.\par
Il s’éleva sans difficulté jusqu’au niveau de la grossière irréligion de ses lecteurs : car on ne lui demandait pas de sacrifier les effets de banale poésie que le romantisme tirait de l’idée de Dieu et de la Charité chrétienne, sur qui les libres-penseurs se déchargent du soin de soulager les misères que crée leur exploitation ; il put même continuer à faire l’éloge du prêtre et de la religieuse, ces gendarmes moraux que la bourgeoisie salarie pour compléter l’œuvre répressive du sergot et du soldat\footnote{\noindent \emph{« Rien ne se pénètre, ne s’amalgame plus aisément qu’un vieux prêtre et un vieux soldat. Au fond c’est le même homme. L’un s’est dévoué pour la patrie d’en bas ; l’autre pour la patrie d’en haut ; pas d’autre différence. »} (\emph{Les Misérables}).\par
\emph{« Il n’y a pas d’œuvre plus sublime, peut-être, que celle que font ces âmes (les religieuses). Et nous ajoutons : il n’y a pas de travail plus utile. Il faut ceux qui prient pour ceux qui ne prient jamais. »} (\emph{Les Misérables}). Victor Hugo a eu l’heureuse chance d’être beaucoup acheté, ce à quoi il tenait surtout, et d’être peu lu, il le sera de moins en moins, autrement il y aurait beau jour que le \emph{Siècle} et Léo Taxil auraient été forcés de le laisser pour compte aux catholiques.
}.\par
Victor Hugo est mort sans prêtres, ni prières ; sans confession ni communion, les catholiques en sont scandalisés ; mais les gens à bon Dieu, ne peuvent lui reprocher d’avoir jamais eu une pensée impie. Son gigantesque cerveau resta hermétiquement bouché à la critique démolisseuse des  \phantomsection
\label{p50}encyclopédistes et aux théories philosophiques de la science moderne. En 1831, un débat scientifique passionna l’Europe intellectuelle : Cuvier et Geoffroy Saint-Hilaire discutaient sur l’origine et la formation des êtres et des mondes. Le vieux Goethe\footnote{[NdE] Orthographié Gœthe.}, que Hugo appelle dédaigneusement \emph{« le poète de l’indifférence »}, l’âme remplie d’un sublime enthousiasme, écoutait raisonner ces deux puissants génies. — Hugo, indifférent à la philosophie et à la science, consacrait son \emph{« immense génie »} qui \emph{« embrassait dans son immensité le visible et l’invisible, l’idéal et le réel, les monstres de la mer et les créatures de la terre… »} à basculer la \emph{« balance hémistiche »} et à rimer nombril et avril, juif et suif, gouine et baragouine, Marengo et lumbago.\par
Trente ans plus tard, Charles Darwin reprenait la théorie de G. Saint-Hilaire et de Lamarck, son maître ; il la fécondait de son vaste savoir et de ses découvertes géniales ; et, triomphante, il l’implantait dans la science naturelle et renouvelait la conception humaine de la création. Hugo, « le penseur du \textsc{xix}\textsuperscript{e} siècle », que les hugolâtres nomment « le siècle de Hugo » ; Hugo, qui portait dans son crâne « l’idée humaine » vécut indifférent au milieu de ce prodigieux mouvement d’idées. {\itshape Il poeta sovrano}, qui passa la plus grande partie de sa vie à courir dans les catalogues de vente et les dictionnaires d’histoire et de géographie, après les rimes riches, ne daigne pas s’apercevoir que Lamarckisme, Darwinisme, Transformisme, rimaient plus richement encore que {\itshape faim} et {\itshape génovéfain}.
 \phantomsection
\label{p51}\subsection[{VI}]{VI}
\noindent On se souviendra de la débauche d’hyperboles de la presse parisienne, qui dura dix longues journées. Déjà on commence à revenir de cette exubérance d’admiration forcée ; et l’on arrivera bientôt à considérer ces jours d’enthousiasme et d’apothéose, comme un moment de folie inexplicable.\par
Il serait oiseux de discuter si dans un avenir prochain les œuvres de Victor Hugo vivront dans la mémoire des hommes, comme celles de Molière et de La Fontaine\footnote{[NdE] Orthographié Lafontaine.} en France ; de Heine et de Goethe, en Allemagne ; de Shakespeare en Angleterre ; de Cervantès, en Espagne ; ou bien si elles dormiront d’un sommeil profond à côté des poèmes du Cavalier Marin, feuilletés avec lassitude, seulement par quelques érudits, étudiant les origines de la littérature classique. Cependant les lettrés du \textsc{xvii}\textsuperscript{e} siècle annonçaient que l’\emph{Adone} effacerait à jamais le \emph{Roland furieux}, la \emph{Divine Comédie} et l’\emph{Iliade}\footnote{[NdE] Orthographié Illiade.}, et des foules en délire promenaient des bannières, où l’on proclamait que l’illustre Marin était \emph{« l’âme de la poésie, l’esprit des lyres, la règle des poètes… le miracle des génies… celui dont la plume glorieuse donne au poème sa vraie valeur, aux discours ses couleurs naturelles, au vers son harmonie véritable, à la prose son artifice parfait… admiré des docteurs, honoré des rois, objet des acclamations du monde, célébré par l’envie elle-même, etc., etc. »}. Shakespeare mourait oublié de son siècle.\par
 \phantomsection
\label{p52}Parfois les générations futures ne ratifient pas les jugements des contemporains. Mais la critique historique qui n’admire ni ne blâme, mais essaye de tout expliquer, adopte l’axiome populaire, il n’y a pas de fumée sans feu ; elle pense que l’écrivain acclamé par ses contemporains, n’a conquis leurs applaudissements que parce qu’il a su flatter leurs goûts et leurs passions, et exprimer leurs pensées et leurs sentiments dans la langue qu’ils pouvaient comprendre. Tout écrivain que consacre l’engouement du public, quels que soient ses mérites et démérites littéraires, acquiert par ce seul fait une haute valeur historique et devient ce que Emerson nommait un {\itshape type représentatif} d’une classe, d’une époque. — Il s’agit de rechercher comment Hugo parvint à conquérir l’admiration de la bourgeoisie.\par
La bourgeoisie, souveraine maîtresse du pouvoir social, voulut avoir une littérature qui reproduisit ses idées et ses sentiments et parlât la langue qu’elle aimait : la littérature classique élaborée pour plaire à l’aristocratie, ne pouvait lui convenir. Quand on étudiera le romantisme d’une manière critique, les études faites jusqu’ici n’ayant été que des exercices de rhétorique, où l’on s’occupait de louer ou de dénigrer, au lieu d’analyser, de comparer et d’expliquer, on verra combien exactement les écrivains romantiques satisfaisaient, par la forme et le fond, les exigences de la bourgeoisie : bien que beaucoup d’entre eux n’aient pas soupçonné le rôle qu’ils remplissaient avec tant de conscience.\par
Hugo, ne se distingue ni par les idées, ni par les sentiments, mais par la forme ; il en était conscient.  \phantomsection
\label{p53}La forme est pour lui la chose capitale, \emph{« ôtez, dit-il à tous ces grands hommes cette simple et petite chose, le style, et de Voltaire, de Pascal, de Boileau, de Bossuet, de Fénelon, de Racine, de Corneille, de La Fontaine, de Molière, de ces maîtres, que vous restera-t-il ? — Ce qui reste d’Homère après avoir passé par Bitaubé »}. — La vérité de l’observation et la force et l’originalité de la pensée, sont choses secondaires, qui ne comptent pas. — \emph{« La forme est chose plus absolue qu’on ne pense… Tout art qui veut vivre doit commencer par bien se poser à lui-même les questions de forme de langage et de style… Le style est la clef de l’avenir… Sans le style vous pouvez avoir le succès du moment, l’applaudissement, le bruit, la fanfare, les couronnes, l’acclamation enivrée des multitudes, vous n’aurez pas le vrai triomphe, la vraie gloire, la vraie conquête, le vrai laurier, comme dit Cicéron : {\itshape insignia victoriæ, non victoriam}\footnote{Victor Hugo. \emph{Philosophie et littérature mêlées}, 1831, p. 27-49-50-51.}. »}\par
Victor Cousin, le romantique de la philosophie, et Victor Hugo, le philosophe du romantisme, servirent à la bourgeoisie l’espèce de philosophie et de littérature qu’elle demandait. Les Diderot, les Voltaire, les Rousseau, les D’Alembert\footnote{[NdE] Orthographié Dalembert.} et les Condillac du \textsc{xviii}\textsuperscript{e} siècle l’avaient trop fait penser pour qu’elle ne désirât se reposer et goûter sans cassements de tête une douce philosophie et une sentimentale poésie, qui ne devaient plus mettre en jeu l’intelligence, mais amuser le lecteur, le transporter dans les nuages et le pays des rêves, et charmer ses yeux par la beauté et  \phantomsection
\label{p54}la hardiesse des images, et ses oreilles par la pompe et l’harmonie des périodes.\par
La révolution de 1789 transplanta le centre de la vie sociale de Versailles à Paris, de la cour et des salons, dans les rues, les cafés et les assemblées populaires. Les journaux, les pamphlets, les discours étaient la littérature de l’époque, tout le monde parlait et écrivait et sans nulle gêne piétinait sur les règles du goût et de la grammaire. Un peuple de mots, de néologismes, d’expressions, de tournures et d’images, venues de toutes les provinces et de toutes les couches sociales, envahirent la langue polie, élaborée par deux siècles de culture aristocratique. Le lendemain de la mort de Robespierre, les grammairiens et les puristes reprirent la férule arrachée de leurs mains et se mirent à l’œuvre pour expulser les intrus et réparer les brèches de la langue du \textsc{xviii}\textsuperscript{e} siècle, ouvertes par les sans-culottes. Ils réussirent en partie ; et imitant les précieuses de l’hôtel Rambouillet, ils châtrèrent la langue parlée et écrite de plusieurs milliers de mots, d’expressions qui ne lui ont pas encore été restitués. Heureusement que Chateaubriand, suivant l’exemple donné par les royalistes des \emph{Actes des apôtres} qui avaient soutenu le trône et l’autel dans le langage des halles, défendit, au grand scandale des puristes, la réaction et la religion avec la langue et la rhétorique enfantées par la révolution. Le succès d’\emph{Atala}, du \emph{Génie du christianisme} et des \emph{Martyrs} fut immense. L’honneur d’avoir dans ce siècle, non pas créé, mais consacré littérairement la langue romantique appartient à Chateaubriand, qui fut le maître de Victor Hugo.\par
 \phantomsection
\label{p55}Mais Chateaubriand, à l’exception d’une petite chanson fort connue et d’une pièce de théâtre justement inconnue, n’écrivit qu’en prose. Il restait encore à briser le moule du vers classique, à assouplir le vers à une nouvelle harmonie, à l’enrichir d’images, d’expressions et de mots que possédait déjà la prose courante et à ressusciter les vieilles formes de la poésie française. Victor Hugo, Lamartine, Musset, Vigny, Gautier, Banville, Baudelaire et d’autres encore se chargèrent de cette tâche. Hugo, aux yeux du gros public, accapara la gloire de la pléïade romantique, non parce qu’il fut le plus grand poète, mais parce que sa poétique embrasse tous les genres et tous les sujets, de l’ode à la satire, de la chanson d’amour au pamphlet politique : et parce que, il fut le seul qui mit en vers les tirades charlatanesques de la philanthropie et du libéralisme bourgeois. Partout il se montra virtuose habile. Ainsi que les modistes et les couturières parent les mannequins de leurs étalages des vêtements les plus brillants, pour accrocher l’œil du passant, de même Victor Hugo costuma les idées et les sentiments que lui fournissaient les bourgeois, d’une phraséologie étourdissante, calculée pour frapper l’oreille et provoquer l’ahurissement ; d’un verbiage grandiloquent, harmonieusement rythmé et rimé, hérissé d’antithèses saisissantes et éblouissantes, d’épithètes fulgurantes. Il fut, après Chateaubriand, le plus grand des étalagistes de mots et d’images du siècle.\par
Ses talents d’étalagiste littéraire n’eurent pas suffi pour lui assurer cette admiration de confiance, si  \phantomsection
\label{p56}universelle ; ses actes, plus encore que ses écrits, lui valurent la haute estime de la bourgeoisie. Hugo fut bourgeois jusque dans la moindre de ses actions.\par
Il se signait dévotement devant la formule sacramentelle du romantisme : {\itshape l’art pour l’art} ; mais, ainsi que tous bourgeois ne songeant qu’à faire fortune, il consacrait son talent à flatter les goûts du public qui paie, et selon les circonstances il chantait la royauté ou la république, proclamait la liberté ou approuvait le bâillonnement de la presse ; et quand il était besoin d’éveiller l’attention publique il tirait des coups de pistolets : — {\itshape le beau, c’est le laid} est le plus bruyant de ses pétards.\par
Il se vantait d’être l’homme immuable, attaché au devoir, comme le mollusque au rocher : mais, ainsi que tout bourgeois voulant à n’importe quel prix faire son chemin, il s’accommodait à toutes les circonstances et saluait avec empressement les pouvoirs et les opinions se levant à l’horizon. Embarqué à la légère dans une opération politique, mal combinée, il se retourna prestement, laissa ses copains conspirer et dépenser leur temps et leur argent pour la propagande républicaine, et s’attela à l’exploitation de sa renommée ; et tandis qu’il donnait à entendre qu’il se nourrissait du traditionnel pain noir de l’exil, il vendait au poids de l’or sa prose et sa poésie.\par
Il se disait simple de cœur, parlant comme il pensait et agissant comme il parlait ; mais, ainsi que tout commerçant cherchant à achalander sa boutique, il jetait de la poudre aux yeux à pleines poignées, et montait constamment des coups au public.  \phantomsection
\label{p57}La mise en scène de sa mort est le couronnement de sa carrière de comédien, si riche en effets savamment machinés. Tout y est pesé, prévu ; depuis le char du pauvre dans le but d’exagérer sa grandeur par cette simplicité et de gagner la sympathie de la foule toujours gobeuse, jusqu’aux cancans sur le million qu’il léguait pour un hôpital, sur les 50 mille francs pour ceci, et les 20 mille pour cela, dans le but de pousser le gouvernement à la générosité et d’obtenir des funérailles triomphales sans bourse délier.\par
Les bourgeois apprécièrent hautement ces qualités de Hugo, si rares à trouver réunies chez un homme de lettres : l’habileté dans la conduite de la vie et l’économie dans la gestion de la fortune\footnote{\noindent Un bout de conversation saisi au vol dans la foule du premier juin :\par
{\itshape Premier bourgeois}. — Hugo, devait être diantrement riche pour que l’État lui fasse de telles funérailles : ce n’est pas pour un génie pauvre qu’il ferait tant de dépenses.\par
{\itshape Deuxième bourgeois}. — Vous avez bien raison. Il laisse, dit-on, cinq millions.\par
{\itshape Premier bourgeois}. — Mettons en trois, car on exagère toujours, et c’est bien beau. Il faut avouer qu’il était plus intelligent que les hommes de génie, qui ne savent jamais se retourner et ne laissent jamais de fortune.\par
Le \emph{Temps} du 4 septembre 1885 fournit les renseignements suivants sur la fortune de Hugo :\par
\noindent « La succession liquidée de Victor Hugo s’élève approximativement à la somme de cinq millions de francs. On pourra se faire une idée de la rapidité avec laquelle s’accroissait la fortune du maître quand on saura que celui-ci réalisa, en 1884, onze cent mille francs de droits d’auteur.\par
« Ajoutons que celui des testaments de Victor Hugo qui contient la clause d’un don de cinquante mille francs aux pauvres de Paris est tout entier écrit de sa main, qu’il est terminé et daté, {\itshape mais non signé}. »
\par
\noindent Donner 50 000 francs aux pauvres, même après sa mort, dépassait ce que pouvait l’âme généreuse et charitable de Victor Hugo. Au moment de signer le cœur lui manqua.
}. Ils reconnurent dans Hugo, couronné de l’auréole du martyre et flamboyant des rayons de la gloire, un homme de leur espèce et plus on exaltait son dévouement au Devoir, son amour de l’idée et la profondeur de sa pensée, et plus ils s’enorgueillissaient de constater qu’il était pétri des mêmes qualités qu’eux. Ils se  \phantomsection
\label{p58}contemplaient et s’admiraient dans Hugo, ainsi que dans un miroir. La Bourgeoisie donna une preuve significative de son identification avec « le grand homme » qu’elle enterrait au Panthéon. Tandis qu’elle conviait à ses funérailles du premier juin toutes les nations ; elle ne fermait pas la Bourse et ne suspendait pas la vie commerciale et financière parce que le premier juin était jour d’échéance des effets de commerce et des coupons des valeurs publiques. Son cœur lui disait que Victor Hugo, {\itshape il poeta sovrano}, aurait désapprouvé cette mesure ; lui qui, pour rien au monde, n’aurait retardé de vingt-quatre heures l’encaissement de ses rentes et de ses créances.
 


% at least one empty page at end (for booklet couv)
\ifbooklet
  \pagestyle{empty}
  \clearpage
  % 2 empty pages maybe needed for 4e cover
  \ifnum\modulo{\value{page}}{4}=0 \hbox{}\newpage\hbox{}\newpage\fi
  \ifnum\modulo{\value{page}}{4}=1 \hbox{}\newpage\hbox{}\newpage\fi


  \hbox{}\newpage
  \ifodd\value{page}\hbox{}\newpage\fi
  {\centering\color{rubric}\bfseries\noindent\large
    Hurlus ? Qu’est-ce.\par
    \bigskip
  }
  \noindent Des bouquinistes électroniques, pour du texte libre à participation libre,
  téléchargeable gratuitement sur \href{https://hurlus.fr}{\dotuline{hurlus.fr}}.\par
  \bigskip
  \noindent Cette brochure a été produite par des éditeurs bénévoles.
  Elle n’est pas faîte pour être possédée, mais pour être lue, et puis donnée.
  Que circule le texte !
  En page de garde, on peut ajouter une date, un lieu, un nom ; pour suivre le voyage des idées.
  \par

  Ce texte a été choisi parce qu’une personne l’a aimé,
  ou haï, elle a en tous cas pensé qu’il partipait à la formation de notre présent ;
  sans le souci de plaire, vendre, ou militer pour une cause.
  \par

  L’édition électronique est soigneuse, tant sur la technique
  que sur l’établissement du texte ; mais sans aucune prétention scolaire, au contraire.
  Le but est de s’adresser à tous, sans distinction de science ou de diplôme.
  Au plus direct ! (possible)
  \par

  Cet exemplaire en papier a été tiré sur une imprimante personnelle
   ou une photocopieuse. Tout le monde peut le faire.
  Il suffit de
  télécharger un fichier sur \href{https://hurlus.fr}{\dotuline{hurlus.fr}},
  d’imprimer, et agrafer ; puis de lire et donner.\par

  \bigskip

  \noindent PS : Les hurlus furent aussi des rebelles protestants qui cassaient les statues dans les églises catholiques. En 1566 démarra la révolte des gueux dans le pays de Lille. L’insurrection enflamma la région jusqu’à Anvers où les gueux de mer bloquèrent les bateaux espagnols.
  Ce fut une rare guerre de libération dont naquit un pays toujours libre : les Pays-Bas.
  En plat pays francophone, par contre, restèrent des bandes de huguenots, les hurlus, progressivement réprimés par la très catholique Espagne.
  Cette mémoire d’une défaite est éteinte, rallumons-la. Sortons les livres du culte universitaire, cherchons les idoles de l’époque, pour les briser.
\fi

\ifdev % autotext in dev mode
\fontname\font — \textsc{Les règles du jeu}\par
(\hyperref[utopie]{\underline{Lien}})\par
\noindent \initialiv{A}{lors là}\blindtext\par
\noindent \initialiv{À}{ la bonheur des dames}\blindtext\par
\noindent \initialiv{É}{tonnez-le}\blindtext\par
\noindent \initialiv{Q}{ualitativement}\blindtext\par
\noindent \initialiv{V}{aloriser}\blindtext\par
\Blindtext
\phantomsection
\label{utopie}
\Blinddocument
\fi
\end{document}
