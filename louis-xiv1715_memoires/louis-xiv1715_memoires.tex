%%%%%%%%%%%%%%%%%%%%%%%%%%%%%%%%%
% LaTeX model https://hurlus.fr %
%%%%%%%%%%%%%%%%%%%%%%%%%%%%%%%%%

% Needed before document class
\RequirePackage{pdftexcmds} % needed for tests expressions
\RequirePackage{fix-cm} % correct units

% Define mode
\def\mode{a4}

\newif\ifaiv % a4
\newif\ifav % a5
\newif\ifbooklet % booklet
\newif\ifcover % cover for booklet

\ifnum \strcmp{\mode}{cover}=0
  \covertrue
\else\ifnum \strcmp{\mode}{booklet}=0
  \booklettrue
\else\ifnum \strcmp{\mode}{a5}=0
  \avtrue
\else
  \aivtrue
\fi\fi\fi

\ifbooklet % do not enclose with {}
  \documentclass[french,twoside]{book} % ,notitlepage
  \usepackage[%
    papersize={105mm, 297mm},
    inner=12mm,
    outer=12mm,
    top=20mm,
    bottom=15mm,
    marginparsep=0pt,
  ]{geometry}
  \usepackage[fontsize=9.5pt]{scrextend} % for Roboto
\else\ifav
  \documentclass[french,twoside]{book} % ,notitlepage
  \usepackage[%
    a5paper,
    inner=25mm,
    outer=15mm,
    top=15mm,
    bottom=15mm,
    marginparsep=0pt,
  ]{geometry}
  \usepackage[fontsize=12pt]{scrextend}
\else% A4 2 cols
  \documentclass[twocolumn]{report}
  \usepackage[%
    a4paper,
    inner=15mm,
    outer=10mm,
    top=25mm,
    bottom=18mm,
    marginparsep=0pt,
  ]{geometry}
  \setlength{\columnsep}{20mm}
  \usepackage[fontsize=9.5pt]{scrextend}
\fi\fi

%%%%%%%%%%%%%%
% Alignments %
%%%%%%%%%%%%%%
% before teinte macros

\setlength{\arrayrulewidth}{0.2pt}
\setlength{\columnseprule}{\arrayrulewidth} % twocol
\setlength{\parskip}{0pt} % classical para with no margin
\setlength{\parindent}{1.5em}

%%%%%%%%%%
% Colors %
%%%%%%%%%%
% before Teinte macros

\usepackage[dvipsnames]{xcolor}
\definecolor{rubric}{HTML}{800000} % the tonic 0c71c3
\def\columnseprulecolor{\color{rubric}}
\colorlet{borderline}{rubric!30!} % definecolor need exact code
\definecolor{shadecolor}{gray}{0.95}
\definecolor{bghi}{gray}{0.5}

%%%%%%%%%%%%%%%%%
% Teinte macros %
%%%%%%%%%%%%%%%%%
%%%%%%%%%%%%%%%%%%%%%%%%%%%%%%%%%%%%%%%%%%%%%%%%%%%
% <TEI> generic (LaTeX names generated by Teinte) %
%%%%%%%%%%%%%%%%%%%%%%%%%%%%%%%%%%%%%%%%%%%%%%%%%%%
% This template is inserted in a specific design
% It is XeLaTeX and otf fonts

\makeatletter % <@@@


\usepackage{blindtext} % generate text for testing
\usepackage[strict]{changepage} % for modulo 4
\usepackage{contour} % rounding words
\usepackage[nodayofweek]{datetime}
% \usepackage{DejaVuSans} % seems buggy for sffont font for symbols
\usepackage{enumitem} % <list>
\usepackage{etoolbox} % patch commands
\usepackage{fancyvrb}
\usepackage{fancyhdr}
\usepackage{float}
\usepackage{fontspec} % XeLaTeX mandatory for fonts
\usepackage{footnote} % used to capture notes in minipage (ex: quote)
\usepackage{framed} % bordering correct with footnote hack
\usepackage{graphicx}
\usepackage{lettrine} % drop caps
\usepackage{lipsum} % generate text for testing
\usepackage[framemethod=tikz,]{mdframed} % maybe used for frame with footnotes inside
\usepackage{pdftexcmds} % needed for tests expressions
\usepackage{polyglossia} % non-break space french punct, bug Warning: "Failed to patch part"
\usepackage[%
  indentfirst=false,
  vskip=1em,
  noorphanfirst=true,
  noorphanafter=true,
  leftmargin=\parindent,
  rightmargin=0pt,
]{quoting}
\usepackage{ragged2e}
\usepackage{setspace} % \setstretch for <quote>
\usepackage{tabularx} % <table>
\usepackage[explicit]{titlesec} % wear titles, !NO implicit
\usepackage{tikz} % ornaments
\usepackage{tocloft} % styling tocs
\usepackage[fit]{truncate} % used im runing titles
\usepackage{unicode-math}
\usepackage[normalem]{ulem} % breakable \uline, normalem is absolutely necessary to keep \emph
\usepackage{verse} % <l>
\usepackage{xcolor} % named colors
\usepackage{xparse} % @ifundefined
\XeTeXdefaultencoding "iso-8859-1" % bad encoding of xstring
\usepackage{xstring} % string tests
\XeTeXdefaultencoding "utf-8"
\PassOptionsToPackage{hyphens}{url} % before hyperref, which load url package

% TOTEST
% \usepackage{hypcap} % links in caption ?
% \usepackage{marginnote}
% TESTED
% \usepackage{background} % doesn’t work with xetek
% \usepackage{bookmark} % prefers the hyperref hack \phantomsection
% \usepackage[color, leftbars]{changebar} % 2 cols doc, impossible to keep bar left
% \usepackage[utf8x]{inputenc} % inputenc package ignored with utf8 based engines
% \usepackage[sfdefault,medium]{inter} % no small caps
% \usepackage{firamath} % choose firasans instead, firamath unavailable in Ubuntu 21-04
% \usepackage{flushend} % bad for last notes, supposed flush end of columns
% \usepackage[stable]{footmisc} % BAD for complex notes https://texfaq.org/FAQ-ftnsect
% \usepackage{helvet} % not for XeLaTeX
% \usepackage{multicol} % not compatible with too much packages (longtable, framed, memoir…)
% \usepackage[default,oldstyle,scale=0.95]{opensans} % no small caps
% \usepackage{sectsty} % \chapterfont OBSOLETE
% \usepackage{soul} % \ul for underline, OBSOLETE with XeTeX
% \usepackage[breakable]{tcolorbox} % text styling gone, footnote hack not kept with breakable


% Metadata inserted by a program, from the TEI source, for title page and runing heads
\title{\textbf{ Mémoires pour l’instruction du Dauphin }}
\date{1715}
\author{Louis 14 (roi de France ; 1638-1715)}
\def\elbibl{Louis 14 (roi de France ; 1638-1715). 1715. \emph{Mémoires pour l’instruction du Dauphin}}
\def\elsource{Louis XIV, {\itshape Mémoires} ; textes présentés par Joël Cornette, Paris, Tallandier, coll. « Texto », 2007, 355 p.}

% Default metas
\newcommand{\colorprovide}[2]{\@ifundefinedcolor{#1}{\colorlet{#1}{#2}}{}}
\colorprovide{rubric}{red}
\colorprovide{silver}{lightgray}
\@ifundefined{syms}{\newfontfamily\syms{DejaVu Sans}}{}
\newif\ifdev
\@ifundefined{elbibl}{% No meta defined, maybe dev mode
  \newcommand{\elbibl}{Titre court ?}
  \newcommand{\elbook}{Titre du livre source ?}
  \newcommand{\elabstract}{Résumé\par}
  \newcommand{\elurl}{http://oeuvres.github.io/elbook/2}
  \author{Éric Lœchien}
  \title{Un titre de test assez long pour vérifier le comportement d’une maquette}
  \date{1566}
  \devtrue
}{}
\let\eltitle\@title
\let\elauthor\@author
\let\eldate\@date


\defaultfontfeatures{
  % Mapping=tex-text, % no effect seen
  Scale=MatchLowercase,
  Ligatures={TeX,Common},
}


% generic typo commands
\newcommand{\astermono}{\medskip\centerline{\color{rubric}\large\selectfont{\syms ✻}}\medskip\par}%
\newcommand{\astertri}{\medskip\par\centerline{\color{rubric}\large\selectfont{\syms ✻\,✻\,✻}}\medskip\par}%
\newcommand{\asterism}{\bigskip\par\noindent\parbox{\linewidth}{\centering\color{rubric}\large{\syms ✻}\\{\syms ✻}\hskip 0.75em{\syms ✻}}\bigskip\par}%

% lists
\newlength{\listmod}
\setlength{\listmod}{\parindent}
\setlist{
  itemindent=!,
  listparindent=\listmod,
  labelsep=0.2\listmod,
  parsep=0pt,
  % topsep=0.2em, % default topsep is best
}
\setlist[itemize]{
  label=—,
  leftmargin=0pt,
  labelindent=1.2em,
  labelwidth=0pt,
}
\setlist[enumerate]{
  label={\bf\color{rubric}\arabic*.},
  labelindent=0.8\listmod,
  leftmargin=\listmod,
  labelwidth=0pt,
}
\newlist{listalpha}{enumerate}{1}
\setlist[listalpha]{
  label={\bf\color{rubric}\alph*.},
  leftmargin=0pt,
  labelindent=0.8\listmod,
  labelwidth=0pt,
}
\newcommand{\listhead}[1]{\hspace{-1\listmod}\emph{#1}}

\renewcommand{\hrulefill}{%
  \leavevmode\leaders\hrule height 0.2pt\hfill\kern\z@}

% General typo
\DeclareTextFontCommand{\textlarge}{\large}
\DeclareTextFontCommand{\textsmall}{\small}

% commands, inlines
\newcommand{\anchor}[1]{\Hy@raisedlink{\hypertarget{#1}{}}} % link to top of an anchor (not baseline)
\newcommand\abbr[1]{#1}
\newcommand{\autour}[1]{\tikz[baseline=(X.base)]\node [draw=rubric,thin,rectangle,inner sep=1.5pt, rounded corners=3pt] (X) {\color{rubric}#1};}
\newcommand\corr[1]{#1}
\newcommand{\ed}[1]{ {\color{silver}\sffamily\footnotesize (#1)} } % <milestone ed="1688"/>
\newcommand\expan[1]{#1}
\newcommand\foreign[1]{\emph{#1}}
\newcommand\gap[1]{#1}
\renewcommand{\LettrineFontHook}{\color{rubric}}
\newcommand{\initial}[2]{\lettrine[lines=2, loversize=0.3, lhang=0.3]{#1}{#2}}
\newcommand{\initialiv}[2]{%
  \let\oldLFH\LettrineFontHook
  % \renewcommand{\LettrineFontHook}{\color{rubric}\ttfamily}
  \IfSubStr{QJ’}{#1}{
    \lettrine[lines=4, lhang=0.2, loversize=-0.1, lraise=0.2]{\smash{#1}}{#2}
  }{\IfSubStr{É}{#1}{
    \lettrine[lines=4, lhang=0.2, loversize=-0, lraise=0]{\smash{#1}}{#2}
  }{\IfSubStr{ÀÂ}{#1}{
    \lettrine[lines=4, lhang=0.2, loversize=-0, lraise=0, slope=0.6em]{\smash{#1}}{#2}
  }{\IfSubStr{A}{#1}{
    \lettrine[lines=4, lhang=0.2, loversize=0.2, slope=0.6em]{\smash{#1}}{#2}
  }{\IfSubStr{V}{#1}{
    \lettrine[lines=4, lhang=0.2, loversize=0.2, slope=-0.5em]{\smash{#1}}{#2}
  }{
    \lettrine[lines=4, lhang=0.2, loversize=0.2]{\smash{#1}}{#2}
  }}}}}
  \let\LettrineFontHook\oldLFH
}
\newcommand{\labelchar}[1]{\textbf{\color{rubric} #1}}
\newcommand{\milestone}[1]{\autour{\footnotesize\color{rubric} #1}} % <milestone n="4"/>
\newcommand\name[1]{#1}
\newcommand\orig[1]{#1}
\newcommand\orgName[1]{#1}
\newcommand\persName[1]{#1}
\newcommand\placeName[1]{#1}
\newcommand{\pn}[1]{\IfSubStr{-—–¶}{#1}% <p n="3"/>
  {\noindent{\bfseries\color{rubric}   ¶  }}
  {{\footnotesize\autour{ #1}  }}}
\newcommand\reg{}
% \newcommand\ref{} % already defined
\newcommand\sic[1]{#1}
\newcommand\surname[1]{\textsc{#1}}
\newcommand\term[1]{\textbf{#1}}

\def\mednobreak{\ifdim\lastskip<\medskipamount
  \removelastskip\nopagebreak\medskip\fi}
\def\bignobreak{\ifdim\lastskip<\bigskipamount
  \removelastskip\nopagebreak\bigskip\fi}

% commands, blocks
\newcommand{\byline}[1]{\bigskip{\RaggedLeft{#1}\par}\bigskip}
\newcommand{\bibl}[1]{{\RaggedLeft{#1}\par\bigskip}}
\newcommand{\biblitem}[1]{{\noindent\hangindent=\parindent   #1\par}}
\newcommand{\dateline}[1]{\medskip{\RaggedLeft{#1}\par}\bigskip}
\newcommand{\labelblock}[1]{\medbreak{\noindent\color{rubric}\bfseries #1}\par\mednobreak}
\newcommand{\salute}[1]{\bigbreak{#1}\par\medbreak}
\newcommand{\signed}[1]{\bigbreak\filbreak{\raggedleft #1\par}\medskip}

% environments for blocks (some may become commands)
\newenvironment{borderbox}{}{} % framing content
\newenvironment{citbibl}{\ifvmode\hfill\fi}{\ifvmode\par\fi }
\newenvironment{docAuthor}{\ifvmode\vskip4pt\fontsize{16pt}{18pt}\selectfont\fi\itshape}{\ifvmode\par\fi }
\newenvironment{docDate}{}{\ifvmode\par\fi }
\newenvironment{docImprint}{\vskip6pt}{\ifvmode\par\fi }
\newenvironment{docTitle}{\vskip6pt\bfseries\fontsize{18pt}{22pt}\selectfont}{\par }
\newenvironment{msHead}{\vskip6pt}{\par}
\newenvironment{msItem}{\vskip6pt}{\par}
\newenvironment{titlePart}{}{\par }


% environments for block containers
\newenvironment{argument}{\itshape\parindent0pt}{\vskip1.5em}
\newenvironment{biblfree}{}{\ifvmode\par\fi }
\newenvironment{bibitemlist}[1]{%
  \list{\@biblabel{\@arabic\c@enumiv}}%
  {%
    \settowidth\labelwidth{\@biblabel{#1}}%
    \leftmargin\labelwidth
    \advance\leftmargin\labelsep
    \@openbib@code
    \usecounter{enumiv}%
    \let\p@enumiv\@empty
    \renewcommand\theenumiv{\@arabic\c@enumiv}%
  }
  \sloppy
  \clubpenalty4000
  \@clubpenalty \clubpenalty
  \widowpenalty4000%
  \sfcode`\.\@m
}%
{\def\@noitemerr
  {\@latex@warning{Empty `bibitemlist' environment}}%
\endlist}
\newenvironment{quoteblock}% may be used for ornaments
  {\begin{quoting}}
  {\end{quoting}}

% table () is preceded and finished by custom command
\newcommand{\tableopen}[1]{%
  \ifnum\strcmp{#1}{wide}=0{%
    \begin{center}
  }
  \else\ifnum\strcmp{#1}{long}=0{%
    \begin{center}
  }
  \else{%
    \begin{center}
  }
  \fi\fi
}
\newcommand{\tableclose}[1]{%
  \ifnum\strcmp{#1}{wide}=0{%
    \end{center}
  }
  \else\ifnum\strcmp{#1}{long}=0{%
    \end{center}
  }
  \else{%
    \end{center}
  }
  \fi\fi
}


% text structure
\newcommand\chapteropen{} % before chapter title
\newcommand\chaptercont{} % after title, argument, epigraph…
\newcommand\chapterclose{} % maybe useful for multicol settings
\setcounter{secnumdepth}{-2} % no counters for hierarchy titles
\setcounter{tocdepth}{5} % deep toc
\markright{\@title} % ???
\markboth{\@title}{\@author} % ???
\renewcommand\tableofcontents{\@starttoc{toc}}
% toclof format
% \renewcommand{\@tocrmarg}{0.1em} % Useless command?
% \renewcommand{\@pnumwidth}{0.5em} % {1.75em}
\renewcommand{\@cftmaketoctitle}{}
\setlength{\cftbeforesecskip}{\z@ \@plus.2\p@}
\renewcommand{\cftchapfont}{}
\renewcommand{\cftchapdotsep}{\cftdotsep}
\renewcommand{\cftchapleader}{\normalfont\cftdotfill{\cftchapdotsep}}
\renewcommand{\cftchappagefont}{\bfseries}
\setlength{\cftbeforechapskip}{0em \@plus\p@}
% \renewcommand{\cftsecfont}{\small\relax}
\renewcommand{\cftsecpagefont}{\normalfont}
% \renewcommand{\cftsubsecfont}{\small\relax}
\renewcommand{\cftsecdotsep}{\cftdotsep}
\renewcommand{\cftsecpagefont}{\normalfont}
\renewcommand{\cftsecleader}{\normalfont\cftdotfill{\cftsecdotsep}}
\setlength{\cftsecindent}{1em}
\setlength{\cftsubsecindent}{2em}
\setlength{\cftsubsubsecindent}{3em}
\setlength{\cftchapnumwidth}{1em}
\setlength{\cftsecnumwidth}{1em}
\setlength{\cftsubsecnumwidth}{1em}
\setlength{\cftsubsubsecnumwidth}{1em}

% footnotes
\newif\ifheading
\newcommand*{\fnmarkscale}{\ifheading 0.70 \else 1 \fi}
\renewcommand\footnoterule{\vspace*{0.3cm}\hrule height \arrayrulewidth width 3cm \vspace*{0.3cm}}
\setlength\footnotesep{1.5\footnotesep} % footnote separator
\renewcommand\@makefntext[1]{\parindent 1.5em \noindent \hb@xt@1.8em{\hss{\normalfont\@thefnmark . }}#1} % no superscipt in foot
\patchcmd{\@footnotetext}{\footnotesize}{\footnotesize\sffamily}{}{} % before scrextend, hyperref


%   see https://tex.stackexchange.com/a/34449/5049
\def\truncdiv#1#2{((#1-(#2-1)/2)/#2)}
\def\moduloop#1#2{(#1-\truncdiv{#1}{#2}*#2)}
\def\modulo#1#2{\number\numexpr\moduloop{#1}{#2}\relax}

% orphans and widows
\clubpenalty=9996
\widowpenalty=9999
\brokenpenalty=4991
\predisplaypenalty=10000
\postdisplaypenalty=1549
\displaywidowpenalty=1602
\hyphenpenalty=400
% Copied from Rahtz but not understood
\def\@pnumwidth{1.55em}
\def\@tocrmarg {2.55em}
\def\@dotsep{4.5}
\emergencystretch 3em
\hbadness=4000
\pretolerance=750
\tolerance=2000
\vbadness=4000
\def\Gin@extensions{.pdf,.png,.jpg,.mps,.tif}
% \renewcommand{\@cite}[1]{#1} % biblio

\usepackage{hyperref} % supposed to be the last one, :o) except for the ones to follow
\urlstyle{same} % after hyperref
\hypersetup{
  % pdftex, % no effect
  pdftitle={\elbibl},
  % pdfauthor={Your name here},
  % pdfsubject={Your subject here},
  % pdfkeywords={keyword1, keyword2},
  bookmarksnumbered=true,
  bookmarksopen=true,
  bookmarksopenlevel=1,
  pdfstartview=Fit,
  breaklinks=true, % avoid long links
  pdfpagemode=UseOutlines,    % pdf toc
  hyperfootnotes=true,
  colorlinks=false,
  pdfborder=0 0 0,
  % pdfpagelayout=TwoPageRight,
  % linktocpage=true, % NO, toc, link only on page no
}

\makeatother % /@@@>
%%%%%%%%%%%%%%
% </TEI> end %
%%%%%%%%%%%%%%


%%%%%%%%%%%%%
% footnotes %
%%%%%%%%%%%%%
\renewcommand{\thefootnote}{\bfseries\textcolor{rubric}{\arabic{footnote}}} % color for footnote marks

%%%%%%%%%
% Fonts %
%%%%%%%%%
\usepackage[]{roboto} % SmallCaps, Regular is a bit bold
% \linespread{0.90} % too compact, keep font natural
\newfontfamily\fontrun[]{Roboto Condensed Light} % condensed runing heads
\ifav
  \setmainfont[
    ItalicFont={Roboto Light Italic},
  ]{Roboto}
\else\ifbooklet
  \setmainfont[
    ItalicFont={Roboto Light Italic},
  ]{Roboto}
\else
\setmainfont[
  ItalicFont={Roboto Italic},
]{Roboto Light}
\fi\fi
\renewcommand{\LettrineFontHook}{\bfseries\color{rubric}}
% \renewenvironment{labelblock}{\begin{center}\bfseries\color{rubric}}{\end{center}}

%%%%%%%%
% MISC %
%%%%%%%%

\setdefaultlanguage[frenchpart=false]{french} % bug on part


\newenvironment{quotebar}{%
    \def\FrameCommand{{\color{rubric!10!}\vrule width 0.5em} \hspace{0.9em}}%
    \def\OuterFrameSep{\itemsep} % séparateur vertical
    \MakeFramed {\advance\hsize-\width \FrameRestore}
  }%
  {%
    \endMakeFramed
  }
\renewenvironment{quoteblock}% may be used for ornaments
  {%
    \savenotes
    \setstretch{0.9}
    \normalfont
    \begin{quotebar}
  }
  {%
    \end{quotebar}
    \spewnotes
  }


\renewcommand{\headrulewidth}{\arrayrulewidth}
\renewcommand{\headrule}{{\color{rubric}\hrule}}

% delicate tuning, image has produce line-height problems in title on 2 lines
\titleformat{name=\chapter} % command
  [display] % shape
  {\vspace{1.5em}\centering} % format
  {} % label
  {0pt} % separator between n
  {}
[{\color{rubric}\huge\textbf{#1}}\bigskip] % after code
% \titlespacing{command}{left spacing}{before spacing}{after spacing}[right]
\titlespacing*{\chapter}{0pt}{-2em}{0pt}[0pt]

\titleformat{name=\section}
  [block]{}{}{}{}
  [\vbox{\color{rubric}\large\raggedleft\textbf{#1}}]
\titlespacing{\section}{0pt}{0pt plus 4pt minus 2pt}{\baselineskip}

\titleformat{name=\subsection}
  [block]
  {}
  {} % \thesection
  {} % separator \arrayrulewidth
  {}
[\vbox{\large\textbf{#1}}]
% \titlespacing{\subsection}{0pt}{0pt plus 4pt minus 2pt}{\baselineskip}

\ifaiv
  \fancypagestyle{main}{%
    \fancyhf{}
    \setlength{\headheight}{1.5em}
    \fancyhead{} % reset head
    \fancyfoot{} % reset foot
    \fancyhead[L]{\truncate{0.45\headwidth}{\fontrun\elbibl}} % book ref
    \fancyhead[R]{\truncate{0.45\headwidth}{ \fontrun\nouppercase\leftmark}} % Chapter title
    \fancyhead[C]{\thepage}
  }
  \fancypagestyle{plain}{% apply to chapter
    \fancyhf{}% clear all header and footer fields
    \setlength{\headheight}{1.5em}
    \fancyhead[L]{\truncate{0.9\headwidth}{\fontrun\elbibl}}
    \fancyhead[R]{\thepage}
  }
\else
  \fancypagestyle{main}{%
    \fancyhf{}
    \setlength{\headheight}{1.5em}
    \fancyhead{} % reset head
    \fancyfoot{} % reset foot
    \fancyhead[RE]{\truncate{0.9\headwidth}{\fontrun\elbibl}} % book ref
    \fancyhead[LO]{\truncate{0.9\headwidth}{\fontrun\nouppercase\leftmark}} % Chapter title, \nouppercase needed
    \fancyhead[RO,LE]{\thepage}
  }
  \fancypagestyle{plain}{% apply to chapter
    \fancyhf{}% clear all header and footer fields
    \setlength{\headheight}{1.5em}
    \fancyhead[L]{\truncate{0.9\headwidth}{\fontrun\elbibl}}
    \fancyhead[R]{\thepage}
  }
\fi

\ifav % a5 only
  \titleclass{\section}{top}
\fi

\newcommand\chapo{{%
  \vspace*{-3em}
  \centering % no vskip ()
  {\Large\addfontfeature{LetterSpace=25}\bfseries{\elauthor}}\par
  \smallskip
  {\large\eldate}\par
  \bigskip
  {\Large\selectfont{\eltitle}}\par
  \bigskip
  {\color{rubric}\hline\par}
  \bigskip
  {\Large TEXTE LIBRE À PARTICPATION LIBRE\par}
  \centerline{\small\color{rubric} {hurlus.fr, tiré le \today}}\par
  \bigskip
}}

\newcommand\cover{{%
  \thispagestyle{empty}
  \centering
  {\LARGE\bfseries{\elauthor}}\par
  \bigskip
  {\Large\eldate}\par
  \bigskip
  \bigskip
  {\LARGE\selectfont{\eltitle}}\par
  \vfill\null
  {\color{rubric}\setlength{\arrayrulewidth}{2pt}\hline\par}
  \vfill\null
  {\Large TEXTE LIBRE À PARTICPATION LIBRE\par}
  \centerline{{\href{https://hurlus.fr}{\dotuline{hurlus.fr}}, tiré le \today}}\par
}}

\begin{document}
\pagestyle{empty}
\ifbooklet{
  \cover\newpage
  \thispagestyle{empty}\hbox{}\newpage
  \cover\newpage\noindent Les voyages de la brochure\par
  \bigskip
  \begin{tabularx}{\textwidth}{l|X|X}
    \textbf{Date} & \textbf{Lieu}& \textbf{Nom/pseudo} \\ \hline
    \rule{0pt}{25cm} &  &   \\
  \end{tabularx}
  \newpage
  \addtocounter{page}{-4}
}\fi

\thispagestyle{empty}
\ifaiv
  \twocolumn[\chapo]
\else
  \chapo
\fi
{\it\elabstract}
\bigskip
\makeatletter\@starttoc{toc}\makeatother % toc without new page
\bigskip

\pagestyle{main} % after style

  \section[{Année 1661}]{Année 1661}\renewcommand{\leftmark}{Année 1661}

\subsection[{Livre premier}]{Livre premier}
\noindent Mon fils, beaucoup de raisons, et toutes fort importantes, m’ont fait résoudre à vous laisser, avec assez de travail pour moi, parmi mes occupations les plus grandes, ces Mémoires de mon règne et de mes principales actions. Je n’ai jamais cru que les rois, sentant, comme ils font, en eux toutes les tendresses paternelles, fussent dispensés de l’obligation commune des pères, qui est d’instruire leurs enfants par l’exemple et par le conseil. Au contraire, il m’a semblé qu’en ce haut rang où nous sommes, vous et moi, un devoir public se joignait au devoir de particulier, et qu’enfin tous les respects qu’on nous rend, toute l’abondance et tout l’éclat qui nous environnent, n’étant que des récompenses attachées par le Ciel même au soin qu’il nous confie des peuples et des états, ce soin n’était pas assez grand s’il ne passait au-delà de nous-mêmes, en nous faisant communiquer toutes nos lumières à celui qui doit régner après nous.\par
J’ai même espéré que dans ce dessein je pourrais vous être aussi utile, et par conséquent à mes sujets, que le saurait être personne du monde ; car ceux qui auront plus de talents et plus d’expérience que moi, n’auront pas régné, et régné en France ; et je ne crains pas de vous dire que plus la place est élevée, plus elle a d’objets qu’on ne peut ni voir ni connaître qu’en l’occupant.\par
J’ai considéré d’ailleurs ce que j’ai si souvent éprouvé moi-même : la foule de ceux qui s’empresseront autour de vous, chacun avec son propre dessein ; la peine que vous aurez à y trouver des avis sincères ; l’entière assurance que vous pourrez prendre en ceux d’un père qui n’aura eu d’intérêt que le vôtre, ni de passion que celle de votre grandeur.\par
Je me suis aussi quelquefois flatté de cette pensée, que, si les occupations, les plaisirs et le commerce du monde, comme il n’arrive que trop souvent, vous dérobaient quelque jour à celui des livres et des histoires, le seul toutefois où les jeunes princes trouvent mille vérités sans nul mélange de flatterie, la lecture de ces Mémoires pourrait suppléer en quelque sorte à toutes les autres lectures, conservant toujours son goût et sa distinction pour vous, par l’amitié et par le respect que vous conserveriez pour moi.\par
J’ai fait enfin quelque réflexion à la condition, en cela dure et rigoureuse, des rois, qui doivent, pour ainsi dire, un compte public de toutes leurs actions à tout l’univers et à tous les siècles, et ne peuvent toutefois le rendre à qui que ce soit dans le temps même, sans manquer à leurs plus grands intérêts et découvrir le secret de leur conduite. Et, ne doutant pas que les choses assez grandes et assez considérables où j’ai eu part, soit au-dedans, soit au-dehors de mon royaume, n’exercent un jour diversement le génie et la passion des écrivains, je ne serai pas fâché que vous ayez ici de quoi redresser l’histoire, si elle vient à s’écarter ou à se méprendre, faute de rapporter fidèlement ou d’avoir bien pénétré mes projets et leurs motifs. Je vous les expliquerai sans déguisement, aux endroits même où mes bonnes intentions n’auront pas été heureuses, persuadé qu’il est d’un petit esprit, et qui se trompe ordinairement, de vouloir ne s’être jamais trompé, et que ceux qui ont assez de mérite pour réussir le plus souvent, trouvent quelque magnanimité à reconnaître leurs fautes.\par
Je ne sais si je dois mettre au nombre des miennes de n’avoir pas pris d’abord à moi-même la conduite de mon État. J’ai tâché, si c’en est une, de la bien réparer par les suites ; et je puis hardiment vous assurer que ce ne fut jamais un effet ni de négligence ni de mollesse.\par
Dès l’enfance même, le seul nom des rois fainéants et de maires du palais me faisait peine quand on le prononçait en ma présence. Mais il faut se représenter l’état des choses : des agitations terribles par tout le royaume avant et après ma majorité ; une guerre étrangère, où ces troubles domestiques avaient fait perdre à la France mille et mille avantages ; un prince de mon sang et d’un très grand nom à la tête des ennemis ; beaucoup de cabales dans l’État ; les parlements encore en possession et en goût d’une autorité usurpée ; dans ma cour, très peu de fidélité sans intérêt, et par là mes sujets en apparence les plus soumis, autant à charge et autant à redouter pour moi que les plus rebelles ; un ministre rétabli malgré tant de factions, très habile, très adroit, qui m’aimait et que j’aimais, qui m’avait rendu de grands services, mais dont les pensées et les manières étaient naturellement très différentes des miennes, que je ne pouvais toutefois contredire ni lui ôter la moindre partie de son crédit sans exciter peut-être de nouveau contre lui, par cette image quoique fausse de disgrâce, les mêmes orages qu’on avait eu tant de peine à calmer ; moi-même, assez jeune encore, majeur à la vérité de la majorité des rois, que les lois de l’État ont avancée pour éviter de plus grands maux, mais non pas de celle où les simples particuliers commencent à gouverner librement leurs affaires ; qui ne connaissais entièrement que la grandeur du fardeau sans avoir pu jusques alors bien connaître mes propres forces ; préférant sans doute dans le cœur, à toutes choses et à la vie même, une haute réputation si je la pouvais acquérir, mais comprenant en même temps que mes premières démarches ou en jetteraient les fondements, ou m’en feraient perdre pour jamais jusques à l’espérance, et qui me trouvais de cette sorte pressé et retardé presque également dans mon dessein par un seul et même désir de gloire.\par
Je ne laissais pas cependant de m’éprouver en secret et sans confident, raisonnant seul et en moi-même sur tous les événements qui se présentaient, plein d’espérance et de joie quand je découvrais quelquefois que mes premières pensées étaient celles où s’arrêtaient à la fin les gens habiles et consommés, et persuadé au fond que je n’avais point été mis et conservé sur le trône avec une aussi grande passion de bien faire, sans en devoir trouver les moyens. Enfin quelques années s’étant écoulées de cette sorte, la paix générale, mon mariage, mon autorité plus affermie et la mort du cardinal Mazarin, m’obligèrent à ne pas différer davantage ce que je souhaitais et que je craignais tout ensemble depuis si longtemps.\par
Je commençai à jeter les yeux sur toutes les diverses parties de l’État, et non pas des yeux indifférents, mais des yeux de maître, sensiblement touché de n’en voir pas une qui ne m’invitât et ne me pressât d’y porter la main ; mais observant avec soin ce que le temps et la disposition des choses me pouvaient permettre. Le désordre régnait partout. Ma cour en général était encore assez éloignée des sentiments où j’espère que vous la trouverez. Les gens de qualité ou de service, accoutumés aux négociations continuelles avec un ministre qui n’y avait pas d’aversion, et à qui elles avaient été nécessaires, se faisaient toujours un droit imaginaire sur tout ce qui était à leur bienséance ; nul gouverneur de place que l’on n’eût peine à gouverner ; nulle demande qui ne fût mêlée d’un reproche du passé, ou d’un mécontentement à venir que l’on voulait laisser entrevoir et craindre. Les grâces exigées et arrachées plutôt qu’attendues, et toujours tirées à conséquence de l’un à l’autre, n’obligeaient plus personne, bonnes seulement désormais à maltraiter ceux à qui on les voudrait refuser.\par
Les finances, qui donnent le mouvement et l’action à tout ce grand corps de la monarchie, étaient entièrement épuisées, et à tel point qu’à peine y voyait-on de ressource. Plusieurs des dépenses les plus nécessaires et les plus privilégiées de ma maison et de ma propre personne étaient retardées contre toute bienséance ou soutenues par le seul crédit, dont les suites étaient à charge ; l’abondance paraissait en même temps chez les gens d’affaires, couvrant d’un côté leurs malversations par toute sorte d’artifice, et les découvrant de l’autre par un luxe insolent et audacieux, comme s’ils eussent appréhendé de me les laisser ignorer.\par
L’Église, sans compter ses maux ordinaires, après de longues disputes sur des matières de l’école, dont on avouait que la connaissance n’était nécessaire à personne pour le salut, les différends s’augmentant chaque jour avec la chaleur et l’opiniâtreté des esprits, et se mêlant même sans cesse de nouveaux intérêts humains, était enfin ouvertement menacée d’un schisme par des gens d’autant plus dangereux qu’ils pouvaient être très utiles, d’un grand mérite, s’ils en eussent été eux-mêmes moins persuadés. Il ne s’agissait plus seulement de quelques docteurs particuliers et cachés, mais d’évêques établis dans leur siège, capables d’entraîner la multitude après eux, de beaucoup de réputation, d’une piété digne en effet d’être révérée tant qu’elle serait suivie de soumission aux sentiments de l’Église, de douceur, de modération et de charité. Le cardinal de Retz, archevêque de Paris, que des raisons d’État très connues m’empêchaient alors de souffrir dans le royaume, ou par inclination ou par intérêt, favorisait toute cette secte naissante et en était favorisé.\par
Le moindre défaut dans l’ordre de la noblesse était de se trouver mêlée d’un nombre infini d’usurpateurs, sans aucun titre ou avec titre acquis à prix d’argent sans aucun service. La tyrannie qu’elle exerçait en quelques-unes de mes provinces sur ses vassaux et sur ses voisins, ne pouvait plus être soufferte ni réprimée que par des exemples de sévérité et de rigueur. La fureur des duels, un peu modérée depuis l’exacte observation des derniers règlements sur quoi je m’étais toujours rendu inflexible, montrait seulement par la guérison déjà avancée d’un mal si invétéré, qu’il n’y en avait point où il fallût désespérer du remède.\par
La Justice, à qui il appartenait de réformer tout le reste, me paraissait elle-même la plus difficile à réformer. Une infinité de choses y contribuaient : les charges remplies par le hasard et par l’argent, plutôt que par le choix et par le mérite ; peu d’expérience en une partie des juges, moins de savoir ; les ordonnances sur l’âge et le service, éludées presque partout ; la chicane établie par une possession de plusieurs siècles, fertile en inventions contre les meilleures lois ; et enfin ce qui la produit principalement, j’entends ce peuple excessif aimant les procès et les cultivant comme son propre héritage, sans autre application que d’en augmenter et la durée et le nombre. Mon conseil même, au lieu de régler les autres juridictions, ne les déréglait que trop souvent par une quantité étrange d’arrêts contraires, tous également donnés sous mon nom et comme par moi-même, ce qui rendait le désordre beaucoup plus honteux.\par
Tous ces maux ensemble, ou leurs suites et leurs effets, retombaient principalement sur le bas peuple, chargé d’ailleurs d’impositions, pressé de la misère en plusieurs endroits, incommodé en d’autres de sa propre oisiveté depuis la paix, et ayant surtout besoin d’être soulagé et occupé.\par
Parmi tant de difficultés dont quelques-unes se présentaient comme insurmontables, trois considérations me donnaient courage. La première, qu’en ces sortes de choses il n’est pas au pouvoir des rois, parce qu’ils sont hommes, et qu’ils ont affaire à des hommes, d’atteindre toute la perfection qu’ils se proposent, trop éloignée de notre faiblesse ; mais que cette impossibilité est une mauvaise raison de ne pas faire ce que l’on peut, et cet éloignement de ne se pas avancer toujours : ce qui ne peut être sans utilité et sans gloire. La seconde, qu’en toutes les entreprises justes et légitimes, le temps, l’action même, le secours du Ciel, ouvrent d’ordinaire mille voies et découvrent mille facilités qu’on n’attendait pas. La dernière enfin, qu’il semblait lui-même me promettre visiblement ce secours, disposant toute chose au même dessein qu’il m’inspirait.\par
En effet, tout était calme en tous lieux ; ni mouvement ni crainte ou apparence de mouvement dans le royaume qui pût m’interrompre ou s’opposer à mes projets ; la paix était établie avec mes voisins, vraisemblablement pour autant de temps que je le voudrais moi-même, par les dispositions où ils se trouvaient.\par
L’Espagne ne pouvait se remettre si promptement de ses grandes pertes. Elle était non seulement sans finances, mais sans crédit, incapable d’aucun grand effort en matière d’argent ni d’hommes, occupée par la guerre de Portugal qu’il m’était aisé de lui rendre plus difficile, et que la plupart des Grands du royaume étaient soupçonnés de ne vouloir pas finir. Le roi était vieux et d’une santé douteuse ; il n’avait qu’un fils en bas âge et assez infirme ; lui et son ministre don Louis de Haro appréhendaient également tout ce qui pouvait ramener la guerre, et elle n’était pas en effet de leur intérêt, ni par l’état de la nation, ni par celui de la maison royale.\par
Je ne voyais rien à craindre de l’Empereur, choisi seulement parce qu’il était de la maison d’Autriche, lié en mille sortes par une capitulation avec les États de l’Empire, peu porté de lui-même à rien entreprendre, et dont les résolutions suivraient apparemment le génie plutôt que l’âge et la dignité.\par
Les Électeurs qui lui avaient principalement imposé des conditions si dures, ne pouvant presque douter de son ressentiment, vivaient dans une défiance continuelle avec lui. Une partie des autres princes de l’Empire était dans mes intérêts.\par
La Suède n’en pouvait avoir de véritables ni de durables qu’avec moi : elle venait de perdre un grand prince, et c’était assez pour elle de se maintenir dans ses conquêtes durant l’enfance de son nouveau roi.\par
Le Danemark affaibli par une guerre précédente avec elle, où il avait été prêt à succomber, ne pensait plus qu’à la paix et au repos.\par
L’Angleterre respirait à peine de ses maux passés, et ne tâchait qu’à affermir le gouvernement sous un roi nouvellement rétabli, porté d’ailleurs d’inclination pour la France.\par
Toute la politique des Hollandais et de ceux qui les gouvernaient, n’avait alors pour but que deux choses : entretenir leur commerce, abaisser la maison d’Orange ; la moindre guerre leur nuisait à l’un et à l’autre, et leur principal support était en mon amitié.\par
Le Pape, seul en Italie, par un reste de son ancienne inimitié avec le cardinal Mazarin, conservait assez de mauvaise volonté pour les Français, mais elle n’allait qu’à me rendre difficile ce qui dépendrait de lui, et qui m’était au fond peu considérable. Ses voisins n’auraient pas suivi ses desseins, s’il en avait formé contre moi. La Savoie gouvernée par ma tante, m’était très favorable. Venise engagée dans la guerre contre le Turc, entretenait avec soin mon alliance, et espérait plus de mon secours que de celui des autres princes chrétiens. Le Grand-Duc s’alliait de nouveau avec moi, par le mariage de son fils avec une princesse de mon sang. Ces potentats enfin et tous les autres d’Italie, dont une partie m’était amis et alliés, comme Parme, Modène et Mantoue, étaient trop faibles séparément pour me faire peine, et ni crainte ni espérance ne les obligeait à se lier contre moi. Je pouvais même profiter de ce qui semblait un désavantage : on ne me connaissait point encore dans le monde ; mais aussi on me portait moins d’envie qu’on n’a fait depuis ; on observait moins ma conduite, et on pensait moins à traverser mes desseins.\par
C’eût été sans doute mal jouir d’une si parfaite tranquillité, qu’on rencontrerait quelquefois à peine en plusieurs siècles, que de ne la pas employer au seul usage qui me la pouvait faire estimer, pendant que mon âge et le plaisir d’être à la tête de mes armées, m’auraient fait souhaiter un peu plus d’affaires au-dehors. Mais comme la principale espérance de ces réformations était en ma volonté, leur premier fondement était de rendre ma volonté bien absolue, par une conduite qui imprimât la soumission et le respect : rendant exactement la justice à qui je la devais ; mais quant aux grâces, les faisant librement et sans contrainte à qui il me plairait, quand il me plairait, pourvu que la suite de mes actions fît connaître que, pour ne rendre raison à personne, je ne me gouvernais pas moins par la raison, et que, dans mon sentiment, le souvenir des services, favoriser et élever le mérite, faire du bien en un mot, ne devait pas seulement être la plus grande occupation, mais le plus grand plaisir d’un prince.\par
Deux choses sans doute m’étaient absolument nécessaires : un grand travail de ma part ; un grand choix de personnes qui pourraient le seconder.\par
Quant au travail, il se pourra faire, mon fils, que vous commenciez à lire ces Mémoires en un âge où l’on a bien plus accoutumé de le craindre que de l’aimer ; trop content d’être échappé à la sujétion des précepteurs et des maîtres, et de n’avoir plus d’heure réglée ni d’application longue et certaine.\par
Je ne vous avertirai pas seulement là-dessus que c’est toutefois par là qu’on règne, pour cela qu’on règne, et que ces conditions de la royauté qui pourront quelquefois vous sembler rudes et fâcheuses en une aussi grande place, vous paraîtraient douces et aisées s’il était question d’y parvenir.\par
Il y a quelque chose de plus, mon fils, et je souhaite que votre propre expérience ne vous l’apprenne jamais : rien ne vous saurait être plus laborieux qu’une grande oisiveté, si vous aviez le malheur d’y tomber, dégoûté premièrement des affaires, puis des plaisirs, puis d’elle-même, et cherchant partout inutilement ce qui ne se peut trouver, c’est-à-dire la douceur du repos et du loisir, sans quelque fatigue et quelque occupation qui précède.\par
Je m’imposai pour loi de travailler régulièrement deux fois par jour, et deux ou trois heures chaque fois avec diverses personnes, sans compter les heures que je passerais seul en particulier, ni le temps que je pourrais donner extraordinairement aux affaires extraordinaires s’il en survenait, n’y ayant pas un moment où il ne fût permis de m’en parler, pour peu qu’elles fussent pressées, à la réserve des ministres étrangers qui trouvent quelquefois dans la familiarité qu’on leur permet, de trop favorables conjonctures, soit pour obtenir, soit pour pénétrer, et que l’on ne doit guère écouter sans y être préparé.\par
Je ne puis vous dire quel fruit je recueillis aussitôt après de cette résolution. Je me sentis comme élever l’esprit et le courage, je me trouvai tout autre, je découvris en moi ce que je n’y connaissais pas, et je me reprochai avec joie de l’avoir trop longtemps ignoré. Cette première timidité qu’un peu de jugement donne toujours, et qui d’abord me faisait peine, surtout quand il fallait parler quelque temps et en public, se dissipa en moins de rien. Il me sembla seulement alors que j’étais roi, et né pour l’être. J’éprouvai enfin une douceur difficile à exprimer, et que vous ne connaîtrez point vous-même qu’en la goûtant comme moi. Car il ne faut pas vous imaginer, mon fils, que les affaires d’État soient comme quelques endroits obscurs et épineux des sciences qui vous auront peut-être fatigué, où l’esprit tâche à s’élever avec effort au-dessus de sa portée, le plus souvent pour ne rien faire, et dont l’inutilité, du moins apparente, nous rebute autant que la difficulté. La fonction de roi consiste principalement à laisser agir le bon sens, qui agit toujours naturellement et sans peine. Ce qui nous occupe est quelquefois moins difficile que ce qui nous amuserait seulement. L’utilité suit toujours. Un roi, quelque habiles et éclairés que soient ses ministres, ne porte point lui-même la main à l’ouvrage sans qu’il y paraisse. Le succès qui plaît en toutes les choses du monde, jusqu’aux moindres, charme en celle-ci comme en la plus grande de toutes, et nulle satisfaction n’égale celle de remarquer chaque jour quelque progrès à des entreprises glorieuses et hautes, et à la félicité des peuples dont on a soi-même formé le plan et le dessein. Tout ce qui est le plus nécessaire à ce travail est en même temps agréable ; car, c’est en un mot, mon fils, avoir les yeux ouverts sur toute la terre ; apprendre à toute heure les nouvelles de toutes les provinces et de toutes les nations, le secret de toutes les cours, l’humeur et le faible de tous les princes et de tous les ministres étrangers ; être informé d’un nombre infini de choses qu’on croit que nous ignorons ; pénétrer parmi nos sujets ce qu’ils nous cachent avec le plus de soin ; découvrir les vues les plus éloignées de nos propres courtisans, leurs intérêts les plus obscurs qui viennent à nous par des intérêts contraires. Et je ne sais enfin quel autre plaisir nous ne quitterions point pour celui-là, si la seule curiosité nous le donnait.\par
Je me suis arrêté sur cet endroit important au-delà de ce que j’avais résolu, et beaucoup plus pour vous que pour moi ; car en même temps que je vous découvre ces facilités et ces douceurs dans les soins les plus grands de la royauté, je n’ignore pas que je diminue d’autant presque l’unique mérite que je puis espérer au monde. Mais votre honneur, mon fils, m’est en cela plus cher que le mien ; et s’il arrive que Dieu vous appelle à gouverner avant que vous ayez pris encore cet esprit d’application et d’affaires dont je vous parle, la moindre déférence que vous puissiez rendre aux avis d’un père à qui j’ose dire que vous devez beaucoup en toutes sortes, est de faire d’abord et durant quelque temps, même avec contrainte, même avec dégoût, pour l’amour de moi qui vous en conjure, ce que vous ferez toute votre vie pour l’amour de vous-même, si vous avez une fois commencé.\par
Je commandai aux quatre secrétaires d’État de ne plus rien signer du tout sans m’en parler ; au surintendant de même, et qu’il ne se fît rien aux finances sans être enregistré dans un livre qui me devait demeurer, avec un extrait fort abrégé, où je pusse voir à tous moments et d’un coup d’œil, l’état des fonds et des dépenses faites ou à faire.\par
Le chancelier eut un pareil ordre, c’est-à-dire de ne rien sceller que par mon commandement, hors les seules lettres de justice, qu’on appelle ainsi parce que ce serait une injustice que de les refuser, étant nécessaires plus pour la forme que pour le fond des choses ; et je laissai alors en ce nombre les offices et les rémissions pour les cas manifestement graciables, quoique j’aie depuis changé d’avis sur ce sujet, comme je vous le dirai en son lieu. Je fis connaître qu’en quelque nature d’affaires que ce fût, il fallait me demander directement ce qui n’était que grâce, et je donnai à tous mes sujets sans distinction, la liberté de s’adresser à moi à toutes heures, de vive voix et par placets.\par
Les placets furent d’abord en un très grand nombre, qui ne me rebuta pas néanmoins. Le désordre où l’on avait mis mes affaires en produisait beaucoup ; la nouveauté et les espérances, ou vaines, ou injustes, n’en attiraient pas moins. On m’en donnait une grande quantité sur des procès, que je ne pouvais ni ne devais tirer à tous moments de la juridiction ordinaire, pour les faire juger devant moi. Mais dans ces choses mêmes qui paraissaient si inutiles, je découvrais de grandes utilités. Je m’instruisais par là en détail de l’état de mes peuples ; ils voyaient que je pensais à eux, et rien ne me gagnait tant leur cœur. L’oppression me pouvait être représentée de telle sorte dans les juridictions ordinaires, que je trouvais à propos de m’en faire informer davantage, pour y pourvoir extraordinairement au besoin. Un exemple ou deux de cette nature empêchaient mille maux semblables ; les plaintes, mêmes fausses et injustes, retenaient mes officiers de donner lieu à de plus véritables et de plus justes.\par
Quant aux personnes qui devaient seconder mon travail, je résolus sur toutes choses de ne point prendre de Premier ministre ; et si vous m’en croyez, mon fils, et tous vos successeurs après vous, le nom en sera pour jamais aboli en France, rien n’étant plus indigne que de voir d’un côté toutes les fonctions, et de l’autre le seul titre de Roi.\par
Pour cela, il était nécessaire de partager ma confiance et l’exécution de mes ordres, sans la donner tout entière à pas un, appliquant ces diverses personnes à diverses choses selon leurs divers talents, qui est peut-être le premier et le plus grand talent des princes.\par
Je résolus même quelque chose de plus, car afin de mieux réunir en moi seul toute l’autorité de maître, encore qu’il y ait en toutes sortes d’affaires un certain détail où nos occupations et notre dignité même ne nous permettent pas de descendre ordinairement, je fis dessein après que j’aurais choisi mes ministres, d’y entrer quelquefois avec chacun d’eux, et quand il s’y attendrait le moins, afin qu’il comprît que j’en pourrais faire autant sur d’autres sujets et à toutes les heures ; outre que la connaissance de ce petit détail prise seulement quelquefois, et par divertissement plutôt que par règle, instruit peu à peu, sans fatiguer, de mille choses qui ne sont pas inutiles aux résolutions générales, et que nous devrions savoir et faire nous-mêmes, s’il était possible qu’un seul homme sût tout et fît tout.\par
Il ne m’est pas aussi aisé de vous dire, mon fils, ce qu’il faut faire pour le choix de divers ministres. La fortune y a toujours, malgré nous, autant ou plus de part que la sagesse ; et dans cette part que la sagesse y peut prendre, le génie y peut beaucoup plus que le conseil.\par
Ni vous, ni moi, mon fils, n’irons pas chercher pour ces sortes d’emplois, ceux que l’éloignement ou leur obscurité dérobent à notre vue, quelque capacité qu’ils puissent avoir. Il faut par nécessité se déterminer sur un petit nombre que le hasard nous présente, c’est-à-dire qui se trouvent déjà dans les charges, ou que leur naissance ou leur inclination ont attachés de plus près à nous.\par
Et pour cet art de connaître les hommes, qui vous sera si important, non seulement en ceci, mais encore en toutes les occasions de votre vie, je vous dirai, mon fils, qu’il se peut apprendre, mais qu’il ne se peut enseigner.\par
En effet, il est juste sans doute de donner beaucoup à la réputation générale et établie, parce que le public n’a point d’intérêt, et qu’on lui impose difficilement pour longtemps. C’est sagement fait que d’écouter tout le monde, et de ne croire entièrement ceux qui nous approchent, ni sur leurs ennemis, hors le bien qu’ils sont contraints d’y reconnaître, ni sur leurs amis, hors le mal qu’ils tâchent d’y excuser, plus sagement encore d’éprouver aux petites choses ceux qu’on veut employer aux grandes. Mais l’abrégé des préceptes, pour bien distinguer les talents, les inclinations et la portée de chacun, c’est de s’y étudier et de s’y plaire, à quoi je vous exhorte, car en général, depuis les plus petites choses jusqu’aux plus grandes, vous ne vous connaîtrez jamais en pas une, si vous n’en faites votre plaisir et si vous ne l’aimez.\par
Dans ce partage que je fis des emplois, les personnes dont je me servais le plus souvent pour les matières de conscience, étaient mon confesseur, le père Annat, que j’estimais en particulier, pour avoir l’esprit droit et désintéressé, et ne se mêler d’aucune intrigue ; l’archevêque de Toulouse, Marca, que je fis depuis archevêque de Paris, homme d’un profond savoir et d’un esprit fort net ; l’évêque de Rennes, parce que la Reine ma mère l’avait souhaité, et celui de Rodez, depuis archevêque de Paris, qui avait été mon précepteur.\par
Pour les affaires de la justice, je les communiquai particulièrement au chancelier, fort ancien officier, reconnu généralement pour très habile en ces matières.\par
Je l’appelais aussi à tous les conseils publics que je tenais moi-même, et particulièrement deux jours la semaine, avec les quatre secrétaires d’État, pour les dépêches ordinaires du dedans du royaume, et pour les réponses aux placets.\par
Je voulus même assister quelquefois au Conseil des parties qu’il tient pour moi, et où il ne s’agit que de procès entre particuliers sur les juridictions. Et si des occupations plus importantes vous laissent le temps, vous ne ferez pas mal d’en user ainsi quelquefois, pour exciter et animer à leur devoir par votre présence ceux qui le composent et pour connaître par vous-même les maîtres des requêtes qui rapportent et qui opinent : d’où se prennent ordinairement les sujets pour les intendances des provinces, pour les ambassades et pour d’autres grands emplois.\par
Mais dans les intérêts les plus importants de l’État, et les affaires secrètes, où le petit nombre de têtes est à désirer autant qu’autre chose, et qui seules demandaient plus de temps et plus d’application que toutes les autres ensemble, ne voulant pas les confier à un seul ministre, les trois que je crus y pouvoir servir le plus utilement furent Le Tellier, Fouquet, et Lionne.\par
La charge de secrétaire d’État, exercée vingt ans par Le Tellier avec beaucoup d’attachement et d’assiduité, lui donnait une fort grande connaissance des affaires. On l’avait employé de tout temps en celles de la dernière confiance. Le cardinal Mazarin m’avait souvent dit qu’aux occasions les plus délicates, il avait reconnu sa suffisance et sa fidélité, que j’avais aussi remarquées moi-même. Il avait une conduite sage, précautionnée et modeste, dont je faisais état.\par
Lionne avait le même témoignage du cardinal Mazarin par qui il avait été formé. Je savais que pas un de mes sujets n’avait été plus souvent employé que lui aux négociations étrangères, ni avec plus de succès. Il connaissait les diverses cours de l’Europe, parlait et écrivait facilement plusieurs langues, avait des belles-lettres, l’esprit aisé, souple et adroit, propre à cette sorte de traités avec les étrangers.\par
Pour Fouquet, on pourra trouver étrange que j’aie voulu me servir de lui, quand on saura que dès ce temps-là ses voleries m’étaient connues ; mais je savais qu’il avait de l’esprit et une grande connaissance du dedans de l’État ; ce qui me faisait imaginer que pourvu qu’il avouât ses fautes passées, et qu’il me promît de se corriger, il pourrait me rendre de bons services.\par
Cependant, pour prendre avec lui mes sûretés, je lui donnai dans les finances Colbert pour contrôleur, sous le titre d’intendant, homme en qui je prenais toute la confiance possible, parce que je savais qu’il avait beaucoup d’application, d’intelligence et de probité, et je le commis dès lors à tenir ce registre des fonds dont je vous ai parlé.\par
J’ai su depuis que le choix de ces trois ministres avait été considéré diversement dans le monde, suivant les divers intérêts dont le monde est partagé. Mais pour connaître si je pouvais faire mieux, il n’y a qu’à considérer les autres sujets à qui j’aurais pu donner la même place.\par
Le chancelier était véritablement fort habile, mais plus dans les affaires de justice, comme j’ai dit, que dans celles d’État. Je le connaissais fort affectionné à mon service, mais il était en réputation de n’avoir pas toute la fermeté nécessaire aux grandes choses ; son âge et les continuelles occupations d’une charge si laborieuse, le pouvaient rendre moins assidu et moins propre à me suivre dans tous les lieux où les besoins du royaume et les guerres étrangères me pouvaient porter. Sa place était d’ailleurs si grande d’elle-même, par la qualité de premier officier du royaume et de chef de tous les conseils, qu’étant jointe à la même participation étroite des affaires secrètes, elle semblait faire, du moins en ce temps-là, un de mes ministres trop grand, et l’élever au-dessus des autres, ce que je ne voulais pas.\par
Le comte de Brienne, secrétaire d’État, qui avait le département des Étrangers, était vieux, présumant beaucoup de soi, et ne pensant d’ordinaire les choses, ni selon mon sens, ni selon la raison.\par
Son fils, qui avait la survivance de sa charge, semblait avoir intention de bien faire ; mais il était si jeune, que, bien loin de prendre ses avis sur mes autres intérêts, je ne pouvais même lui confier la fonction de son propre emploi, dont Lionne faisait la plus grande partie.\par
La Vrillière et du Plessis étaient de bonnes gens, mais dont les lumières paraissaient seulement proportionnées à l’exercice de leurs charges, dans lesquelles il ne tombait rien de bien important.\par
J’aurais pu sans doute jeter les yeux sur des gens de plus haute considération ; mais non pas qui eussent eu plus de capacité que ces trois ; et ce petit nombre, comme je vous l’ai déjà dit, me paraissait meilleur qu’un plus grand.\par
Pour vous découvrir même toute ma pensée, il n’était pas de mon intérêt de prendre des sujets d’une qualité plus éminente. Il fallait, avant toutes choses, établir ma propre réputation, et faire connaître au public, par le rang même d’où je les prenais, que mon intention n’était pas de partager mon autorité avec eux. Il m’importait qu’ils ne conçussent pas eux-mêmes de plus hautes espérances que celles qu’il me plairait de leur donner : ce qui est difficile aux gens d’une grande naissance ; et ces précautions m’étaient tellement nécessaires, qu’avec cela même le monde fut assez longtemps à me bien connaître.\par
Plusieurs se persuadaient que dans peu quelqu’un de ceux qui m’approchaient s’emparerait de mon esprit et de mes affaires. La plupart regardaient l’assiduité de mon travail comme une chaleur qui devait bientôt se ralentir ; et ceux qui voulaient en juger plus favorablement, attendaient à se déterminer par les suites.\par
Le temps a fait voir ce qu’il en fallait croire, et c’est ici la dixième année que je marche, comme il me semble, assez constamment dans la même route, ne relâchant rien de mon application ; informé de tout ; écoutant mes moindres sujets ; sachant à toute heure le nombre et la qualité de mes troupes, et l’état de mes places ; donnant incessamment mes ordres pour tous leurs besoins ; traitant immédiatement avec les ministres étrangers ; recevant et lisant les dépêches ; faisant moi-même une partie des réponses, et donnant à mes secrétaires la substance des autres ; réglant la recette et la dépense de mon État ; me faisant rendre compte directement par ceux que je mets dans les emplois importants ; tenant mes affaires aussi secrètes qu’aucun autre l’ait fait avant moi ; distribuant les grâces par mon propre choix, et retenant, si je ne me trompe, ceux qui me servent, quoique comblés de bienfaits pour eux-mêmes et pour les leurs, dans une modestie fort éloignée de l’élévation et du pouvoir des Premiers ministres.\par
L’observation que l’on fit à loisir de toutes ces choses, commença sans doute à donner quelque opinion de moi dans le monde ; et cette opinion n’a pas peu contribué au succès des affaires que j’ai entreprises depuis : rien ne faisant de si grands effets en si peu de temps que la réputation du prince.\par
Mais ne vous trompez pas, mon fils, comme tant d’autres, et ne pensez pas qu’il soit temps de l’établir quand il faut s’en servir. On ne la met point sur pied avec les armées : on aurait beau ouvrir ses trésors pour l’acquérir, il faut y avoir pensé auparavant, et ce n’est même qu’une possession assez longue qui nous en assure.\par
J’avais, dès les premières années, apparemment assez de sujet d’être content de ma conduite ; mais les applaudissements que cette nouveauté m’attirait, ne laissaient pas de me donner une continuelle inquiétude, par la crainte que j’avais, et dont je ne suis pas encore tout à fait exempt, de ne les pas assez bien mériter.\par
On vous dira dans quelle défiance j’ai vécu là-dessus avec mes courtisans, et combien de fois éprouvant leur génie, je les ai engagés à me louer des choses même que je croyais avoir mal faites, pour le leur reprocher aussitôt après, et les accoutumer à ne me point flatter.\par
Mais quelque obscures que puissent être leurs intentions, je vous enseignerai, mon fils, un moyen aisé de profiter de tout ce qu’ils diront à votre avantage : c’est de vous examiner secrètement vous-même, et d’en croire votre propre cœur plus que leurs louanges ; les prenant toujours, suivant l’humeur de ceux qui vous parleront, ou pour un reproche malin de quelque défaut opposé, ou pour une exhortation secrète à ce que vous ne sentiriez pas en vous ; persuadé de plus, quand même vous penserez les mériter, que vous n’en avez pas encore assez fait, que la réputation ne se peut conserver sans en acquérir tous les jours davantage ; que la gloire enfin n’est pas une maîtresse qu’on puisse jamais négliger, ni être digne de ses premières faveurs, si l’on n’en souhaite incessamment de nouvelles.
\subsection[{Livre second}]{Livre second}
\subsubsection[{Première section}]{Première section}
\noindent Les dispositions générales, dont je vous ai parlé, m’occupèrent tout le mois de mars ; car le cardinal Mazarin n’était mort que le neuf même. Et bien que durant sa maladie, qui fut longue, et quelque temps auparavant, j’eusse observé avec plus de soin que jamais l’état des choses, je ne crus pas devoir toucher au détail des affaires, qu’après m’en être fait rendre compte en particulier par chacun de ceux qui en avaient été chargés avec lui, leur demandant avec soin quelles vues ils avaient jusqu’alors ou croyaient qu’on pouvait avoir pour l’avenir, et persuadé que mes lumières, quand même elles auraient été plus grandes, pouvaient être fort aidées et fort augmentées par les leurs.\par
Il m’a semblé nécessaire de vous le marquer, mon fils, de peur que par un excès de bonne intention dans votre première jeunesse, et par l’ardeur même que ces Mémoires pourront exciter en vous, vous ne confondiez ensemble deux choses très différentes : je veux dire gouverner soi-même, et n’écouter aucun conseil, qui serait une autre extrémité aussi dangereuse que celle d’être gouverné. Les particuliers les plus habiles prennent avis d’autres personnes habiles dans leurs petits intérêts. Que sera-ce des rois qui ont en main l’intérêt public, et dont les résolutions font le mal ou le bien de toute la terre ? Il n’en faudrait jamais former d’aussi importantes, sans appeler, s’il était possible, tout ce qu’il y a de plus éclairé, de plus raisonnable et de plus sage parmi nos sujets.\par
La nécessité nous réduit à un petit nombre de personnes choisies entre les autres, et qu’il ne faut pas du moins négliger. Vous éprouverez de plus, mon fils, ce que je reconnus bientôt, qu’en parlant de nos affaires, quand nulle autre considération ne nous en doit empêcher, nous n’apprenons pas seulement beaucoup d’autrui, mais de nous-mêmes. L’esprit achève ses propres pensées en les mettant au-dehors, qu’il gardait auparavant confuses, imparfaites et seulement ébauchées. L’entretien qui l’excite et qui l’échauffe le porte insensiblement d’objet en objet, plus loin que n’avait fait la méditation solitaire et muette, et lui ouvre, par les difficultés même qu’on lui oppose, mille nouveaux expédients.\par
D’ailleurs, notre élévation nous éloigne en quelque sorte de nos peuples, dont nos ministres sont plus proches, capables de voir par conséquent mille particularités que nous ignorons, sur lesquelles il faut néanmoins se déterminer et prendre ses mesures. Ajoutez l’âge, l’expérience, l’étude, la liberté qu’ils ont bien plus grande que nous de prendre les connaissances et les lumières de quelques inférieurs, qui prennent eux-mêmes celles des autres, de degré en degré jusqu’aux moindres.\par
Mais quand dans les occasions importantes, ils nous ont rapporté tous les partis et toutes les raisons contraires, tout ce qu’on fait ailleurs en pareil cas, tout ce qu’on a fait autrefois et tout ce qu’on peut faire aujourd’hui, c’est à nous, mon fils, à choisir ce qu’il faut faire en effet ; et ce choix-là, j’oserai vous dire que si nous ne manquons ni de sens ni de courage, nul autre ne le fait mieux que nous ; car la décision a besoin d’un esprit de maître et il est sans comparaison plus facile de faire ce que l’on est, que d’imiter ce que l’on n’est pas. Que si l’on remarque presque toujours quelque différence entre les lettres particulières, que nous nous donnons la peine d’écrire nous-mêmes, et celles que nos secrétaires les plus habiles écrivent pour nous, découvrant en ces dernières je ne sais quoi de moins naturel, et l’inquiétude d’une plume qui craint éternellement d’en faire trop ou trop peu, ne doutez pas qu’aux affaires de plus grande conséquence, la différence ne soit encore plus grande entre nos propres résolutions, et celles que nous laisserons prendre à nos ministres sans nous, où plus ils seront habiles, plus ils hésiteront par la crainte des événements, et, d’en être chargés s’embarrassent quelquefois fort longtemps de difficultés qui ne nous arrêteraient pas un moment.\par
La sagesse veut qu’en certaines rencontres on donne beaucoup au hasard ; la raison elle-même conseille alors de suivre je ne sais quels mouvements ou instincts aveugles au-dessus de la raison, et qui semblent venir du Ciel, connus de tous les hommes, mais de plus grand poids sans doute et plus dignes de considération en ceux qu’il a placés lui-même aux premiers rangs. De dire quand c’est qu’il faut se défier de ces mouvements ou s’y abandonner, personne ne le peut ; ni livres, ni règles, ni expérience ne l’enseignent : une certaine justesse et une certaine hardiesse d’esprit le font trouver, toujours plus libres en celui qui ne doit compte de ses actions à personne.\par
Quoi qu’il en soit, et pour ne revenir plus sur ce sujet, aussitôt que j’eus commencé à tenir cette conduite avec mes ministres, je connus fort bien non pas tant à leurs discours qu’à un certain air de vérité qui se fait distinguer de la flatterie, comme une personne vivante de la plus belle statue, et il me revint depuis par plusieurs voies non suspectes, qu’ils n’étaient pas seulement satisfaits, mais en quelque sorte surpris de me voir dans les affaires les plus difficiles sans m’attacher précisément à leur avis, et sans affecter non plus de m’en éloigner, prendre aussi facilement mon parti, et le plus souvent celui que la suite des choses montrait clairement avoir été le meilleur. Et bien qu’ils vissent assez, dès lors, qu’ils seraient toujours auprès de moi ce que doivent être des ministres et rien de plus, ils n’en furent que plus contents d’un emploi où ils trouvaient, avec mille autres avantages, une sûreté entière en faisant leur devoir, rien n’étant plus dangereux à ceux qui occupent de pareils postes, qu’un roi qui dort ordinairement, pour s’éveiller de temps en temps comme en sursaut, après avoir perdu la suite des affaires, et qui, dans cette lumière trouble et confuse, s’en prend à tout le monde des mauvais succès, des cas fortuits ou des fautes dont il se devrait accuser lui-même.\par
Après m’être ainsi pleinement instruit par des entretiens particuliers avec eux, j’entrai plus hardiment en matière. Rien ne me sembla presser davantage que de soulager mes peuples : de quoi la misère des provinces, et la compassion que j’en avais, me sollicitaient puissamment. L’état de mes finances, tel que je vous l’ai représenté, semblait s’y opposer, et conseillait en tout cas de différer ; mais il faut toujours se hâter de faire le bien. Les réformations que j’entreprenais, quoiqu’utiles au public, devaient être fâcheuses à un grand nombre de particuliers. Il était à propos de commencer par quelque chose qui ne fût qu’agréable, et il n’y avait pas moyen enfin de soutenir plus longtemps le nom même de la paix, sans qu’il fût suivi d’aucune douceur de cette nature, qui donnât de meilleures espérances pour l’avenir. Je passai donc par-dessus toute autre considération, et en attendant plus de soulagement, je remis d’abord trois millions sur les tailles de l’année suivante, déjà réglées, et dont on allait faire l’imposition.\par
Je renouvelai en même temps, mais avec dessein de les faire mieux observer qu’auparavant, comme je l’ai fait aussi, les défenses de l’or et de l’argent sur les habits, et mille autres superfluités étrangères, qui étaient une espèce de charge et de contribution, volontaire en apparence, forcée en effet, que mes sujets, surtout les plus qualifiés et les personnes de ma cour, payaient tous les jours aux nations voisines, ou pour mieux dire au luxe et à la vanité.\par
Il fallait par mille raisons, même pour se préparer à la réformation de la justice qui en avait tant de besoin, diminuer l’autorité des principales compagnies qui, sous prétexte que leurs jugements sont sans appel, et comme on parle, souverains et en dernier ressort, ayant pris peu à peu le nom de cours souveraines, se regardaient comme autant de souverainetés séparées et indépendantes. Je fis connaître que je ne souffrirais plus leurs entreprises. Et pour en donner l’exemple, la Cour des Aides de Paris ayant commencé la première à s’écarter du devoir, en quelque nature de sa juridiction, j’en exilai quelques officiers les plus coupables, croyant que ce remède bien employé d’abord, m’empêcherait d’en avoir souvent besoin dans les suites ; ce qui m’a réussi.\par
Aussitôt après, je leur fis encore mieux entendre mes intentions par un arrêt solennel de mon Conseil d’en haut. Car il est bien vrai que ces compagnies n’ont rien à ordonner l’une à l’autre, dans leurs divers ressorts réglés par les lois et par les édits. Et cela suffisait autrefois pour les faire vivre en paix ; ou s’il survenait quelques différends entre elles, surtout dans les affaires des particuliers, ils étaient si rares et si peu embarrassés de procédure, que les rois eux-mêmes les terminaient d’un seul mot, le plus souvent en se promenant, sur le rapport des maîtres des requêtes, alors aussi en très petit nombre, jusqu’à ce que les affaires s’augmentant dans le royaume, et la chicane encore plus que les affaires, ce soin fut principalement confié au chancelier de France et au Conseil des parties dont je vous ai déjà parlé, qui doit être bien autorisé nécessairement pour régler ces autres compagnies sur leur juridiction, et même pour toutes les autres affaires dont nous jugeons quelquefois à propos, par des raisons de l’utilité publique et de notre service, de lui attribuer extraordinairement la connaissance, en l’ôtant à ces compagnies qui ne la tiennent elles-mêmes que de nous. Cependant par cet esprit de souveraineté, dans les désordres du temps, elles ne lui déféraient qu’autant que bon leur semblait, et passaient outre tous les jours et en toutes sortes d’affaires, nonobstant ses défenses, jusqu’à dire assez souvent qu’elles ne reconnaissaient point d’autre volonté du Roi que celle qui était dans les Ordonnances et dans les Édits vérifiés.\par
Je leur défendis à toutes en général, par cet arrêt, d’en donner jamais de contraires à ceux de mon conseil, sous quelque prétexte que ce pût être, soit de leur juridiction, soit du droit des particuliers ; et leur ordonnai, quand elles croiraient qu’on aurait blessé l’un ou l’autre, de s’en plaindre à moi, et de recourir à mon autorité, celle que je leur avais confiée n’étant que pour faire justice à mes sujets, et non pas pour se faire justice elles-mêmes, qui est une partie de la souveraineté tellement unie à la couronne et tellement propre au Roi seul, qu’elle ne peut être communiquée à nul autre.\par
Dans la même année, mais un peu plus tard, car je n’observerai pas si précisément l’ordre des dates, en une certaine affaire des finances sur tous les greffes en général, et qu’on n’avait jamais osé exécuter pour ceux du parlement de Paris, parce que la propriété en appartenait à des officiers du corps, et quelquefois à des chambres entières, j’affectai au contraire de faire voir que ces officiers devaient subir la loi commune, dont rien ne m’empêchait aussi de les dispenser, quand il me plairait de donner cette récompense à leurs services.\par
Presque en même temps, je fis une chose qui paraissait même trop hardie, tant la robe s’en était fait accroire jusqu’alors, et tant les esprits étaient pleins encore de cette considération qu’elle avait acquise dans les derniers troubles, en abusant de son pouvoir. Je réduisis à deux quartiers, au lieu de trois, toutes les nouvelles augmentations des gages qui étaient en aliénations de mon revenu, faites à très vil prix durant la guerre, consommant les baux de mes fermes, mais dont les officiers des compagnies avaient acquis la meilleure partie : ce qui faisait qu’on regardait comme une grande entreprise de les choquer d’abord aussi rudement dans leurs intérêts les plus sensibles. Mais le fond de cette affaire était juste ; car deux quartiers étaient encore beaucoup pour ce qu’ils en avaient payé. La réformation était nécessaire. Mes affaires n’étaient pas en état que je pusse rien craindre de leur chagrin. Il était plutôt à propos de leur témoigner qu’on n’en craignait rien, et que les temps étaient changés. Et ceux qui par divers intérêts eussent souhaité que ces compagnies s’emportassent, apprirent de leur soumission au contraire celle qu’ils me devaient.\par
En toutes ces choses, mon fils, et en plusieurs autres que vous verrez dans les suites, qui ont mortifié sans doute mes officiers de justice, je ne veux pas que vous me donniez, comme auront pu faire ceux qui me connaissent moins, des motifs d’aigreur, de haine et de vengeance pour tout ce qui s’était passé devant la Fronde, où l’on ne peut pas nier que ces compagnies ne se soient souvent fort oubliées, et jusqu’à d’étranges extrémités.\par
Mais en premier lieu, ce ressentiment qui paraît d’abord si juste, le serait peut-être un peu moins à l’examiner de près. Elles sont rentrées d’elles-mêmes et sans violence dans le devoir. Les bons serviteurs ont ramené les mauvais. Pourquoi imputer à tout le corps les fautes d’une partie, plutôt que les services qui ont prévalu, et par où l’on a fini ? Il faudrait plutôt oublier l’un en faveur de l’autre, et se souvenir seulement qu’à relire les histoires, à peine y a-t-il aucun ordre du royaume, Église, noblesse, tiers état, qui ne soit tombé quelquefois en des égarements terribles dont il est revenu.\par
Par-dessus cela, mon fils, encore que sur les offenses, autant ou plus que sur tout le reste des choses, les rois soient hommes, je ne crains pas de vous dire qu’ils le sont un peu moins quand ils sont véritablement rois, parce qu’une passion maîtresse et dominante, qui est celle de leur intérêt, de leur grandeur et de leur gloire, étouffe toutes les autres en eux.\par
Cette douceur qu’on se figure dans la vengeance, n’est presque pas faite pour nous. Elle ne flatte que ceux dont le pouvoir est en doute : ce qui est tellement vrai que les particuliers même, s’ils ont quelque chose d’honnête, ont peine à l’exercer sur un ennemi tout à fait abattu et qui ne s’en peut jamais relever. Pour nous, mon fils, nous ne sommes que très rarement dans cet état du milieu, où on prend plaisir de se venger ; car nous pouvons tout sans difficulté, ou bien nous nous trouvons au contraire en certaines conjonctures délicates et difficiles, qui ne veulent pas que nous éprouvions quel est notre pouvoir.\par
Enfin, comme nous sommes à nos peuples, nos peuples sont à nous, et je n’ai point vu encore qu’un homme sage se vengeât à son préjudice, en perdant ce qui lui appartient, sous prétexte qu’il en aura été mal servi, au lieu de donner ordre pour l’avenir qu’il le soit un peu mieux.\par
Ainsi, mon fils, le ressentiment et la colère des rois sages et habiles contre leurs sujets ne sont que justice et que prudence.\par
L’élévation trop grande des parlements avait été dangereuse à tout le royaume durant ma minorité. Il fallait les abaisser, moins pour le mal qu’ils avaient fait que pour celui qu’ils pouvaient faire à l’avenir. Leur autorité, tant qu’on la regardait comme opposée à la mienne, quelques bonnes que fussent leurs intentions, produisait de très méchants effets dans l’État, et traversait tout ce que je pouvais entreprendre de plus grand et de plus utile. Il était juste que cette utilité l’emportât sur tout le reste, et de réduire toutes choses dans leur ordre légitime et naturel, quand même, ce que j’ai évité néanmoins, il eût fallu ôter à ces corps ce qui leur avait été donné autrefois, comme le peintre ne fait aucune difficulté d’effacer lui-même ce qu’il aura fait de plus hardi et de plus beau, toutes les fois qu’il le trouve plus grand qu’il ne faut, et dans quelque disproportion visible avec le reste de l’ouvrage.\par
Mais je sais, mon fils, et je puis vous protester sincèrement, que je n’ai d’ailleurs ni aversion, ni aigreur dans l’esprit pour mes officiers de justice. Au contraire, si la vieillesse est vénérable dans les hommes, elle me le paraît davantage encore dans ces corps si anciens. Je suis persuadé qu’en nulle autre partie de l’État, le travail n’est peut-être plus grand, ni les récompenses moindres. J’ai pour eux l’affection et toute la considération que je dois ; et vous, mon fils, qui selon les apparences les trouverez encore plus éloignés de ces vaines prétentions d’autrefois, vous devez pratiquer avec d’autant plus de soin ce que je fais tous les jours moi-même : je veux dire de leur témoigner de l’estime dans les occasions, d’en connaître les principaux sujets et ceux qui ont le plus de mérite, de faire voir que vous les connaissez (car il est beau à un prince de montrer qu’il est informé de tous, et que les services que l’on rend loin de lui ne sont pas perdus), de les considérer, eux et leurs familles, dans la distribution des emplois et des bénéfices, de favoriser leurs desseins quand ils voudront s’attacher plus particulièrement à vous, de les accoutumer enfin par de bons traitements et des paroles honnêtes à vous voir quelquefois ; au lieu qu’au siècle passé une partie de leur intégrité était de ne pas approcher du Louvre, et cela, non pas par mauvais dessein, mais par la fausse imagination d’un prétendu intérêt du peuple opposé à celui du prince, et dont ils se faisaient les défenseurs sans considérer que ces deux intérêts ne sont qu’un, que la tranquillité des sujets ne se trouve qu’en l’obéissance, qu’il y a toujours moins de mal pour le public à supporter qu’à contrôler même le mauvais gouvernement des rois dont Dieu seul est le juge, et que ce qu’ils semblent faire contre la loi commune est fondé le plus souvent sur la raison d’État, qui est la première des lois, du consentement de tout le monde, mais la plus inconnue et la plus obscure à tous ceux qui ne gouvernent pas.\par
Les moindres démarches étaient importantes en ces commencements, qui faisaient voir à la France quel serait l’esprit de mon règne, et ma conduite pour tout l’avenir. J’étais blessé de la manière dont on s’était accoutumé à traiter avec le prince, ou plutôt avec le ministre, mettant presque toujours en conditions ce qu’il fallait attendre de ma justice ou de ma bonté.\par
L’assemblée du clergé, qui avait duré longtemps dans Paris, différait à l’ordinaire de se séparer, comme je l’avais témoigné souhaiter, jusques à l’expédition de certains édits qu’elle avait demandée avec instance. Je lui fis entendre qu’on n’obtenait plus rien par ces sortes de voies. Elle se sépara ; et ce fut alors seulement que les édits furent expédiés.\par
En ce même temps, la mort du duc d’Épernon fit vaquer la charge de colonel-général de l’infanterie française. Son père, le premier duc d’Épernon, élevé par la faveur d’Henri troisième, avait porté cette charge aussi haut que son ambition l’avait voulu. Le pouvoir en était infini : la nomination des officiers supérieurs qu’on y avait attachée, donnant moyen à celui qui la possédait, de mettre partout des créatures, le rendait plus maître que le Roi même des principales forces de l’État. Je trouvai à propos de la supprimer, quoique j’eusse déjà retranché auparavant de ce grand pouvoir, par diverses voies, tout ce que la bienséance et le temps m’avaient permis.\par
Quant aux gouverneurs des places, qui abusaient si souvent eux-mêmes de leur pouvoir, je leur ôtai premièrement le fonds des contributions qu’on leur avait abandonné durant la guerre, sous prétexte de pourvoir à la sûreté de leurs places sans attendre le secours des finances, et de les tenir en bon état, mais qui, allant à des sommes immenses pour des particuliers, les rendaient trop puissants et trop absolus. Je renouvelai en second lieu, insensiblement et peu à peu, presque toutes les garnisons, ne souffrant plus qu’elles fussent composées, comme auparavant, de troupes qui étaient dans leur dépendance, mais d’autres au contraire qui ne connaissaient que moi. Et ce que l’on n’eût osé proposer ni penser quelques mois auparavant, s’exécuta alors sans peine et sans bruit, chacun attendant de moi, et recevant en effet des récompenses plus légitimes et plus justes, en faisant son devoir.\par
Je fis cependant continuer à Bordeaux les fortifications du château Trompette, et à Marseille le bâtiment de la citadelle, non pas tant pour rien craindre alors de ces deux villes, que pour la sûreté de l’avenir, et pour servir d’exemple à toutes les autres. Il n’y avait aucun mouvement dans le royaume, mais tout ce qui approchait tant soit peu de la désobéissance, comme en quelques occasions à Montauban, à Dieppe, en Provence, à La Rochelle, était d’abord réprimé et châtié, sans le dissimuler ; de quoi la paix et les troupes que j’avais résolu d’entretenir en bon nombre, me donnaient assez de moyen.\par
Je crus enfin, mon fils, qu’en l’état des choses, un peu de sévérité était la plus grande douceur que je pouvais avoir pour mes peuples, une disposition contraire devant leur produire par elle-même et par ses suites une infinité de maux. Car aussitôt qu’un roi se relâche sur ce qu’il a commandé, l’autorité périt, et le repos avec elle. Ceux qui voient le prince de plus près, connaissant les premiers sa faiblesse, sont aussi les premiers à en abuser ; après eux, ceux du second rang, et ainsi dans les autres de suite pour ceux qui ont en main quelque sorte de pouvoir. Tout tombe sur la plus basse partie, opprimée par là de mille et mille petits tyrans, au lieu d’un roi légitime, dont la seule indulgence néanmoins a fait tout ce désordre.\par
Le mariage de ma cousine d’Orléans s’accomplit en ce temps-là. Je la dotai de mes deniers, et la fis conduire à mes dépens jusque dans les États de son beau-père. J’achevais encore celui que le cardinal Mazarin avait projeté d’une de ses nièces avec le connétable Colonna. Mais deux autres mariages plus importants méritent qu’on vous en parle.\par
Celui de mon frère avec la sœur du roi d’Angleterre avait été terminé au mois de mars, dont j’avais été fort aise, même par des raisons d’État, car mon alliance avec cette nation sous Cromwell avait comme frappé le dernier coup dans la guerre d’Espagne, réduisant les ennemis à ne pouvoir plus du tout défendre les Pays-Bas, et par conséquent à m’accorder, si je l’eusse voulu, même de plus grands avantages qu’ils ne firent par le traité des Pyrénées. Les affaires avaient depuis changé de face en Angleterre. Cromwell était mort, et le Roi rétabli. Les Espagnols se préparant des ressources pour la Flandre, en cas de rupture avec moi, et n’espérant rien alors de la Hollande, songeaient sur toutes choses à mettre ce prince dans leurs intérêts. Le mariage de mon frère servait à le retenir dans les miens ; mais celui que je résolus de proposer pour ce roi-là même, de la princesse de Portugal, semblait le devoir ôter entièrement à l’Espagne, et faire en ma faveur deux autres effets plus considérables. Le premier, de soutenir les Portugais que je voyais en danger de succomber bientôt sans cela ; le second, de me donner plus de moyen de les assister moi-même, si je le jugeais nécessaire, nonobstant le traité des Pyrénées qui me le défendait.\par
Je toucherai ici, mon fils, un endroit peut-être aussi délicat que pas un autre, dans la conduite des princes. Je suis bien éloigné de vouloir vous enseigner l’infidélité, et je crois avoir fait voir depuis peu à toute l’Europe en la paix d’Aix-la-Chapelle, quel état je faisais d’une parole donnée, en la préférant uniquement à mes plus grands intérêts. Mais il y a quelque distinction à faire en ces matières, que le jugement, l’équité, la conscience font beaucoup mieux qu’aucun discours. L’état des deux couronnes de France et d’Espagne est tel aujourd’hui, et depuis longtemps dans le monde, qu’on ne peut élever l’une sans abaisser l’autre. Cela fait entre elles une jalousie qui leur est, si je l’osais dire, essentielle, et une espèce d’inimitié permanente que les traités peuvent couvrir, mais qu’ils n’éteignent jamais, parce que le fondement en demeure toujours, et que l’une d’elles travaillant contre l’autre, ne croit pas tant nuire à autrui, que se maintenir et se conserver soi-même, devoir si naturel qu’il emporte facilement tous les autres.\par
Et à dire la vérité et sans déguisement, elles n’entrent jamais ensemble qu’avec cet esprit dans aucun traité. Quelques clauses spécieuses qu’on y mette d’union, d’amitié, de se procurer respectivement toutes sortes d’avantages, le véritable sens que chacun entend fort bien de son côté, par l’expérience de tant de siècles, est qu’on s’abstiendra de toute sorte d’hostilités et de toutes démonstrations publiques de mauvaise volonté, car pour les infractions secrètes et qui n’éclateront point, chacun les attend de l’autre, par le principe naturel que j’ai dit, et ne promet le contraire qu’au même sens qu’on le lui promet. Ainsi, l’on pourrait dire qu’en se dispensant également d’observer les traités, à la rigueur on n’y contrevient pas, parce qu’on n’en a point pris les paroles à la lettre, quoiqu’on n’ait pu employer que celles-là, comme il se fait, mais d’une autre sorte, dans le monde en celles des compliments, absolument nécessaires pour vivre ensemble, et qui n’ont qu’une signification bien au-dessous de ce qu’elles sonnent.\par
Les Espagnols nous en ont les premiers montré l’exemple, car en quelque profonde paix qu’on ait été avec eux, ont-ils jamais manqué à fomenter nos désordres domestiques et nos guerres civiles, et la qualité de catholiques par excellence les a-t-elle empêchés en nul temps de fournir de l’argent sous mains aux huguenots rebelles ? Ils accueillent sans cesse avec soin, avec dépense, tout ce qui se retire mécontent de ce pays-ci jusqu’à des personnes de néant et de nulle considération ; non pas pour ignorer ce qu’elles sont, mais pour montrer par là à celles qui valent mieux ce qu’on ferait en leur faveur. Je ne pouvais pas douter enfin qu’ils n’eussent violé les premiers et en mille sortes, le traité des Pyrénées, et j’aurais cru manquer à ce que je dois à mes États, si en l’observant plus scrupuleusement qu’eux, je leur laissais librement ruiner le Portugal, pour retomber ensuite sur moi avec toutes leurs forces, et me redemander, en troublant la paix de l’Europe, tout ce qu’ils m’avaient cédé par ce même traité. Les clauses par où ils me défendaient d’assister cette couronne encore mal affermie, plus elles étaient extraordinaires, réitérées et accompagnées de précautions, plus elles marquaient qu’on n’avait pas cru que je m’en dusse abstenir ; et tout ce que je croyais leur devoir déférer, était de ne le secourir que dans la nécessité, secrètement, avec modération et retenue ; ce qui se pouvait plus commodément par l’interposition et sous le nom du Roi d’Angleterre, s’il était une fois beau-frère de celui de Portugal.\par
Je n’oubliai donc rien pour le porter à cette alliance, et parce que c’est une cour où l’on fait d’ordinaire beaucoup par l’argent, que les ministres de cette nation ont été en général fort souvent suspects d’être pensionnaires d’Espagne, et que le chancelier Hyde, très habile homme pour le dedans du royaume, paraissait alors avoir un fort grand pouvoir sur l’esprit du Roi, je liai avec lui en particulier une négociation très secrète, dont mon ambassadeur même en Angleterre ne savait rien, et lui envoyai diverses fois un homme d’esprit qui en était connu, et qui, sous prétexte d’acheter du plomb pour mes bâtiments, avait des lettres de crédit jusqu’à cinq cent mille livres, qu’il offrit de ma part à ce ministre, sans lui demander que son amitié. Il refusa l’offre avec d’autant plus de mérite, qu’en même temps il avoua à cet envoyé qu’il était lui-même d’avis du mariage de Portugal pour l’intérêt du Roi son maître, à qui il le fit après cela parler en secret.\par
Les Espagnols lui faisaient proposer de leur côté la Princesse de Parme, qu’ils offraient de doter à leurs dépens comme une infante ; puis quand j’eus fait rejeter cette proposition, la fille du prince d’Orange, avec les mêmes avantages sans se souvenir alors de leur grand zèle pour la foi, et que donner à cet État une reine protestante, était ôter aux catholiques la seule consolation et le seul support qu’ils y peuvent espérer.\par
Mais je ménageai les choses de telle sorte que cette seconde proposition fut rejetée comme la première, et servit même à conclure plus promptement ce que je voulais pour le Portugal et pour son infante.
\subsubsection[{Deuxième section}]{Deuxième section}
\noindent De toutes les affaires étrangères de cette année, ce fut la plus importante. Je ne laisserai pas d’en toucher ici quelques autres de moindre conséquence, mais qui vous feront voir qu’en affermissant autant qu’il était possible mon autorité au-dedans, je n’oubliais pas de maintenir au-dehors en toutes rencontres les avantages et la dignité de la couronne.\par
Les ambassadeurs de Gênes, par un artifice souvent réitéré, dont ils se voulaient faire une espèce de possession et de titre, usurpaient depuis quelques années à ma cour le traitement royal. Ils s’étaient assujettis pour cela à ne prendre jamais leurs audiences qu’au même jour qu’on la donnait à quelque ambassadeur de roi, afin qu’entrant au Louvre immédiatement après lui et au même son du tambour, on ne pût distinguer si cet honneur les regardait ou non : vanité d’autant plus ridicule, que cet État longtemps possédé par nos ancêtres, n’a aucune souveraineté que celle qu’il s’est donnée à lui-même par sa rébellion depuis cent quarante ou tant d’années, nous appartenant légitimement à plusieurs bons titres, tels que sont les traités solennels et volontaires avec tout le peuple qui s’était donné à nous, souvent renouvelés avec un plein et entier consentement, et confirmés plus d’une fois par le droit des armes. Je fis connaître à ces ambassadeurs combien j’étais éloigné de souffrir leur folle prétention, dont ils avaient bien osé s’expliquer ; et ni eux ni leurs supérieurs n’ont eu garde d’en parler depuis, tremblant de peur au contraire aux moindres mouvements de mes troupes vers l’Italie, par la connaissance de ce que je pourrais justement leur demander.\par
L’Empereur avait cru de son intérêt de me donner part de son élection, comme ses prédécesseurs aux miens ; mais il s’était fait cette chimère qu’il n’était pas de sa dignité de m’écrire le premier, et avait adressé sa dépêche à l’ambassadeur d’Espagne, avec ordre de ne la point délivrer qu’il n’eût obtenu de moi quelque lettre de compliment, par où il parût que je l’avais prévenu. Je ne refusai pas seulement d’en écrire aucune ; mais pour apprendre à ce prince à me mieux connaître, je l’obligeai, aussitôt après, à rayer dans les pouvoirs de ses ministres, les qualités de comte de Ferrette et de landgrave d’Alsace, ces États m’ayant été cédés par le traité de Münster. Je lui fis aussi retrancher d’un projet de ligue contre les Turcs, le titre qu’il se donnait de chef du peuple chrétien, comme s’il eût véritablement possédé le même Empire et les mêmes droits qu’avait autrefois Charlemagne, après avoir défendu la religion contre les Saxons, les Huns et les Sarrazins.\par
Et sur ce sujet, mon fils, de peur qu’on ne veuille vous imposer quelquefois par les beaux noms d’Empire romain, de César ou de successeur de ces grands empereurs, dont nous tirons nous-mêmes notre origine, je me sens obligé de vous faire remarquer combien les empereurs d’aujourd’hui sont éloignés de cette grandeur dont ils affectent les titres. Quand ces titres furent mis dans notre maison, elle régnait tout à la fois sur la France, sur les Pays-Bas, sur l’Allemagne, sur l’Italie et sur la meilleure partie de l’Espagne, qu’elle avait distribuée à divers seigneurs particuliers, s’en réservant la souveraineté. Les sanglantes défaites de plusieurs peuples venus du Nord et du Midi, pour la ruine de la chrétienté, avaient porté la terreur du nom français par toute la terre. Charlemagne enfin ne voyant aucun roi en toute l’Europe, ni à dire la vérité, en tout le monde, qui pût se comparer à lui, ce nom semblait désormais impropre ou pour eux ou pour lui, par l’inégalité de leur fortune. Il était monté à ce haut point de gloire, non pas par l’élection de quelques princes, mais par le courage et par les victoires qui sont l’élection et les suffrages du Ciel même, quand il a résolu de soumettre les autres puissances à une seule. Et l’on n’avait point vu de domination aussi étendue que la sienne, hors les quatre fameuses monarchies, à qui l’on attribue l’empire du monde entier, quoiqu’elles n’en aient jamais conquis ni possédé qu’une assez petite partie, mais considérable et connue dans le monde le plus connu. Celle des Romains était la dernière, tout à fait éteinte en Occident, et dont on ne voyait plus en Orient que quelques restes faibles, misérables et languissants. Cependant, comme si l’Empire romain eût repris sa force et commencé à revivre en nos climats, ce qui n’était point en effet, ce nom, le plus grand qui fût alors dans la mémoire des hommes, sembla seul pouvoir distinguer et désigner l’élévation extraordinaire de Charlemagne, et bien que cette élévation même, qu’il ne tenait que de Dieu et de son épée, lui donnât assez de droit de prendre tel titre qu’il aurait voulu, le pape, qui, avec toute l’Église, lui avait d’extrêmes obligations, fût bien aise de contribuer ce qu’il pouvait à sa gloire, et de rendre en lui cette qualité d’empereur plus authentique par un couronnement solennel, comme le sacre, encore qu’il ne nous donne pas la royauté, ne laisse pas de la déclarer aux peuples, et de la rendre en nous plus auguste, plus inviolable et plus sainte. Mais cette grandeur de Charlemagne qui fondait si bien le titre d’empereur, ou de plus magnifiques encore si l’on en eût pu trouver, ne dura pas longtemps après lui, diminuée premièrement par les partages qui se faisaient alors entre les enfants de France, puis par la faiblesse et le peu d’application de ses descendants, en particulier de la branche qui s’était établie au deçà du Rhin ; car les empires, mon fils, ne se conservent que comme ils s’acquièrent, c’est-à-dire par la vigueur, par la vigilance et par le travail. Les Allemands, excluant les princes de notre sang, s’emparèrent aussitôt de cette dignité, ou plutôt en subrogèrent une autre en la place, qui n’avait rien de commun, ni avec l’ancien Empire romain, ni avec le nouvel Empire de nos aïeux, mais où l’on tâcha, comme dans tous les grands changements, à faire que chacun trouvât ses avantages, pour ne s’y pas opposer. Les peuples et les États particuliers s’y engagèrent, par les grands privilèges qu’on leur donna sous le nom de liberté ; les princes d’Allemagne, parce qu’on rendait cette dignité élective, d’héréditaire qu’elle était, et qu’ils acquéraient par là le droit d’y nommer ou d’y prétendre, ou tous les deux ensemble ; les papes enfin, parce qu’on faisait toujours profession de la tenir de leur autorité, et qu’au fond un grand et véritable empereur romain pouvait se donner plus de droit qu’ils n’eussent voulu sur Rome même ; d’où vient que ceux qui ont le plus curieusement recherché l’antiquité, tiennent que Léon troisième, en couronnant Charlemagne, ne lui attribua pas le titre d’Empereur romain, que la voix publique lui donna dans les suites, mais seulement celui d’Empereur et celui d’Avocat de l’Église et du Saint-Siège ; car ce mot d’avocat signifiait alors protecteur, et en ce sens, les rois d’Espagne se qualifiaient encore, il n’y a que quelques années, avocats d’une partie des villes que j’ai conquises en Flandre, ce pays étant presque tout divisé en différentes avocaties, ou protections de cette nature.\par
Mais pour en revenir aux empereurs d’aujourd’hui, il vous est aisé, mon fils, de comprendre par tout ce discours, qu’ils ne sont nullement ce qu’étaient les anciens empereurs romains, ni ce qu’étaient nos aïeux. Car à leur faire justice, on doit les regarder seulement comme les chefs et les capitaines-généraux d’une République d’Allemagne, assez nouvelle en comparaison de plusieurs autres États, et qui n’est ni si grande ni si puissante qu’elle doive prétendre aucune supériorité sur les nations voisines. Leurs résolutions les plus importantes sont soumises aux délibérations des États de l’Empire ; on leur impose, en les élisant, les conditions qu’on veut ; la plupart des membres de la République, c’est-à-dire des princes et des villes libres d’Allemagne, ne défèrent à leurs ordres qu’autant qu’il leur plaît. En cette qualité d’empereurs, ils n’ont que très peu de revenus et s’ils ne possédaient de leur chef d’autres États héréditaires, ils seraient réduits à n’avoir pour habitation dans tout l’Empire que l’unique ville de Bamberg, que l’évêque qui en est seigneur souverain, est obligé de leur céder en ce cas-là. Aussi plusieurs princes qui pouvaient par l’élection parvenir à cette dignité, n’en ont point voulu, la croyant plus onéreuse qu’honorable ; et de mon temps, l’électeur de Bavière était empereur, s’il n’eût refusé de se nommer lui-même, comme les lois le permettent, en joignant sa voix à celles dont je m’étais assuré pour lui dans le collège des Électeurs, et que je lui fis offrir.\par
Je ne vois donc pas, mon fils, par quelle raison des rois de France, rois héréditaires, et qui peuvent se vanter qu’il n’y a aujourd’hui dans le monde, sans exception, ni meilleure maison que la leur, ni monarchie aussi ancienne, ni puissance plus grande, ni autorité plus absolue, seraient inférieurs à ces princes électifs. Il ne faut pas dissimuler néanmoins que les papes, par une suite de ce qu’ils avaient fait, ont insensiblement donné dans la cour de Rome, la préséance aux ambassadeurs de l’Empereur sur tous les autres, et que la plupart des cours de la chrétienté ont imité cet exemple, sans que nos prédécesseurs aient fait effort pour l’empêcher ; mais en toute autre chose, ils ont défendu leurs droits. On trouve, dès le dixième siècle, des traités publics, où ils se nomment les premiers avant les empereurs avec qui ils traitent ; et à la Porte du Grand-Seigneur, nos ambassadeurs, et en dernier lieu, le marquis de Brèves, sous Henri le Grand, mon aïeul, n’ont pas seulement disputé, mais emporté la préséance sur ceux des empereurs. En un mot, mon fils, comme je n’ai pas cru devoir rien demander de nouveau dans la chrétienté sur cette matière, j’ai cru encore moins, en l’état où je me trouvais, devoir en façon du monde rien souffrir de nouveau, où ces princes affectassent de prendre le moindre avantage sur moi ; et je vous conseille d’en user de même, remarquant cependant combien la vertu est à estimer, puisqu’après tant de siècles celle des Romains, celle des deux premiers César et celle de Charlemagne font encore, malgré l’exacte raison, rendre plus d’honneur qu’on ne devrait au vain nom et à la vaine ombre de leur empire.\par
Ces légères contestations avec l’Empereur firent que je m’attachai encore davantage à diminuer en Allemagne son crédit ou celui que la maison d’Autriche s’y est acquis depuis deux siècles ; et m’étant encore plus exactement informé de la disposition des esprits, je détachai de cette cabale, par une négociation de quelques mois, l’électeur de Trèves. Il entra dans l’Alliance du Rhin, c’est-à-dire dans un parti puissant et considérable que j’avais formé au milieu de l’Empire, sous prétexte de maintenir le traité de Münster et la paix de l’Allemagne.\par
Dix villes impériales que ce même traité avait mises sous ma protection, me prêtèrent alors le serment qu’elles m’avaient toujours refusé.\par
Pour affermir mes conquêtes vers ce pays-là et vers la Flandre par une plus étroite union à mes anciens États, ne voyant pas lieu de pratiquer ce que faisaient les Grecs et les Romains, qui était d’envoyer des colonies de leurs sujets naturels dans les pays nouvellement subjugués, je tâchai du moins d’y établir les mœurs françaises. Je changeai les conseils souverains en présidiaux ; j’en fis ressortir les appellations à mes parlements. Je mis des Français et, autant qu’il me fut possible, des gens de mérite dans les premières charges. J’écrivis aux généraux d’ordre, afin qu’ils unissent les convents de ces pays-là aux anciennes provinces de France.\par
J’empêchai que les églises d’Artois et du Hainaut ne continuassent à recevoir les rescrits de Rome par la voie de l’internonce de Flandre, et ne permis plus que les abbés des trois évêchés de Metz, Toul et Verdun, fussent élus sans ma nomination ; mais je trouvai bon seulement qu’à chaque vacance, on me présentât trois sujets dont je promis d’agréer l’un.\par
J’avais accordé ma protection au prince d’Épinoi durant la guerre. Je le fis mettre en possession des biens du comte de Buquoi, jusqu’à ce que les siens lui eussent été rendus par les Espagnols, comme ils l’avaient promis.\par
Je délivrai le Pays de l’Alleu, alors en contestation entre eux et moi, de diverses oppressions dont ils le menaçaient ; car sous prétexte de quelques arrérages d’une somme de douze mille écus qu’ils avaient accoutumé d’y lever, ils avaient emprisonné douze des principaux habitants, et avaient déjà exigé d’eux, pour leur dépense, deux mille florins que je leur fis rendre avec la liberté, sans vouloir même jamais écouter l’expédient dur et ruineux pour ce pays, que l’Espagne me proposait, qui était de doubler cette imposition durant notre différend, afin que la France et elle y trouvassent chacune leur droit.\par
Je fis cesser dans l’Artois certaines levées que les magistrats des villes y faisaient, sous prétexte d’octrois accordés par le roi d’Espagne. Je voulus, pour soulager le peuple, que les officiers des garnisons eux-mêmes portassent, comme les habitants, tous les autres droits qui se levaient sur les denrées. Je fis donner trois ans aux pauvres familles de la frontière, que leurs créanciers pressaient cruellement depuis la paix. Je fis en sorte qu’une bonne partie des limites furent marquées, dès cette année, en exécution du traité des Pyrénées, les fortifications de Nancy démolies, toutes mes places réparées, mises en défense, et munies des choses nécessaires, comme si on eût été au milieu de la guerre, ne craignant rien tant que le reproche qu’on fait depuis si longtemps aux Français, mais que j’espère de bien effacer par ma conduite, qu’ils savent conquérir, et ne savent pas conserver.\par
J’avoue que dans ces commencements, voyant ma réputation s’augmenter chaque jour, toutes choses me devenir faciles et me réussir, je fus peut-être aussi sensiblement touché que je l’aie jamais été, du désir de le servir [Dieu] et de lui plaire.\par
Je donnai pouvoir au cardinal Antoine et à d’Auberville, chargés de mes affaires à Rome, de faire une ligue contre le Turc, où j’offrais de contribuer de mes deniers et de mes troupes, beaucoup plus que pas un des autres princes chrétiens. Je donnai cent mille écus aux Vénitiens pour leur guerre de Candie, m’engageant de nouveau à leur fournir des forces considérables toutes les fois qu’ils voudraient faire un effort pour chasser les infidèles de cette île. Je fis offrir à l’Empereur, contre cet ennemi commun, une armée de vingt mille hommes, toute composée de mes troupes ou de celles de mes alliés.\par
Je rétablis, par une nouvelle ordonnance, la rigueur des anciens édits contre les jurements et les blasphèmes, et voulus qu’on en fît aussitôt quelques exemples publics ; et je puis dire qu’à cet égard, mes soins et l’aversion que j’ai témoignée pour ce dérèglement scandaleux, n’ont pas été inutiles, ma cour en étant, grâce à Dieu, plus exempte qu’elle ne l’a été durant plusieurs siècles sous les rois mes prédécesseurs.\par
J’ajoutai de nouvelles précautions à celles que j’avais déjà prises contre les duels ; et pour montrer que ni rang, ni naissance ne dispenseraient personne, je bannis de ma cour le comte de Soissons, qui avait fait faire un appel au duc de Navailles, et je mis à la Bastille celui dont il s’était servi pour en porter la parole, quoique la chose n’eût eu aucun effet.\par
Je m’appliquai à détruire le jansénisme, et à dissiper les communautés où se fomentait cet esprit de nouveauté, bien intentionnées peut-être, mais qui ignoraient ou voulaient ignorer les dangereuses suites qu’il pourrait avoir.\par
Je fis diverses instances auprès des Hollandais pour les catholiques de Gueldre.\par
Je donnai ordre qu’on distribuât des aumônes considérables aux pauvres de Dunkerque, de peur que leur misère ne les tentât de suivre la religion des Anglais, à qui la guerre d’Espagne m’avait obligé de donner cette place durant le ministère du cardinal Mazarin. Et quant à ce grand nombre de mes sujets de la religion prétendue réformée, qui était un mal que j’avais toujours regardé et que je regarde encore avec beaucoup de douleur, je formai dès lors le plan de toute ma conduite envers eux, que je n’ai pas lieu de croire mauvaise, puisque Dieu a voulu qu’elle ait été suivie et le soit encore tous les jours, d’un très grand nombre de conversions.\par
Il me sembla, mon fils, que ceux qui voulaient employer des remèdes extrêmes et violents, ne connaissaient pas la nature de ce mal, causé en partie par la chaleur des esprits, qu’il faut laisser passer et s’éteindre insensiblement, plutôt que de la rallumer de nouveau par une forte contradiction, surtout quand la corruption n’est pas bornée à un certain nombre connu, mais répandue dans toutes les parties de l’État.\par
Autant que je l’ai pu comprendre jusqu’ici, l’ignorance des ecclésiastiques au siècle précédent, leur luxe, leur débauche, les mauvais exemples qu’ils donnaient, ceux qu’ils étaient obligés de souffrir par la même raison, les abus enfin qu’ils laissaient autoriser dans la conduite des particuliers, contre les règles et les sentiments publics de l’Église, donnèrent lieu, plus que toute autre chose, aux grandes blessures qu’elle reçut par le schisme et par l’hérésie. Les nouveaux réformateurs disaient vrai visiblement en plusieurs choses de cette nature, qu’ils reprenaient avec autant de justice que d’aigreur ; ils imposaient au contraire en toutes celles qui ne regardaient pas le fait, mais la croyance. Mais il n’est pas au pouvoir du peuple de distinguer une fausseté bien déguisée, quand elle se cache d’ailleurs parmi plusieurs vérités évidentes. On commença par de petits différends, dont j’ai appris que les protestants d’Allemagne ni les Huguenots de France ne tiennent presque plus de compte aujourd’hui. Ceux-là en produisirent de plus grands, principalement parce qu’on pressa trop un homme violent et hardi, qui ne voyant plus de retraite honnête pour lui, s’engagea plus avant dans le combat, et s’abandonnant à son propre sens, prit la liberté d’examiner tout ce qu’il recevait auparavant. Il promit au monde une voie facile et abrégée pour se sauver : moyen très propre à flatter le sens humain et à entraîner la multitude. L’amour de la nouveauté en séduisit plusieurs. Divers intérêts des princes se mêlèrent à cette querelle. Les guerres en Allemagne, puis en France, redoublèrent l’animosité du mauvais parti : le bas peuple douta encore moins que la religion ne fût bonne, pour laquelle on s’exposait à tant de périls. Les pères, pleins de cette préoccupation, la laissèrent à leurs enfants, la plus violente qu’il leur fût possible, mais, au fond, de la nature de toutes les autres passions que le temps modère toujours, et souvent avec d’autant plus de succès qu’on fait moins d’efforts pour les combattre.\par
Sur ces connaissances générales, je crus, mon fils, que le meilleur moyen pour réduire peu à peu les Huguenots de mon royaume était de ne les point presser du tout par aucune rigueur nouvelle contre eux, de faire observer ce qu’ils avaient obtenu sous les règnes précédents, mais aussi de ne leur accorder rien de plus, et d’en renfermer même l’exécution dans les plus étroites bornes que la justice et la bienséance le pouvaient permettre. Je nommai pour cela, dès cette année même, des commissaires exécuteurs de l’édit de Nantes. Je fis cesser avec soin partout les entreprises de ceux de cette religion, comme dans le faubourg Saint-Germain, où j’appris qu’ils commençaient d’établir des assemblées secrètes et des écoles de leur secte ; à Jamets en Lorraine, où n’ayant pas droit de s’assembler, ils s’étaient réfugiés en grand nombre durant les désordres de la guerre et y faisaient leurs exercices ; à La Rochelle, où l’habitation n’étant permise qu’aux anciens habitants et à leurs familles, elles en avaient attiré peu à peu quantité d’autres, que j’obligeai d’en sortir.\par
Mais quant aux grâces qui dépendaient de moi seul, je résolus et j’ai assez ponctuellement observé depuis, de n’en faire aucune à ceux de cette religion, et cela par bonté non par aigreur, pour les obliger par là à considérer de temps en temps, d’eux-mêmes et sans violence, si c’était par quelque bonne raison qu’ils se privaient volontairement des avantages qui pouvaient leur être communs avec mes autres sujets.\par
Pour profiter cependant de l’état où ils se trouvaient, d’écouter plus volontiers qu’autrefois ce qui pouvait les détromper, je résolus aussi d’attirer, même par les récompenses, ceux qui se rendraient dociles, d’animer autant que je pourrais les évêques, afin qu’ils travaillassent à leur instruction et leur ôtassent les scandales qui les éloignaient quelquefois de nous, de ne mettre enfin dans ces premières places, ni dans toutes celles dont j’ai la nomination, que des personnes de piété, d’application, de savoir, capables de réparer par une conduite toute contraire les désordres que celle de leurs anciens prédécesseurs avait principalement causés dans l’Église.\par
Mais il s’en faut encore beaucoup, mon fils, que je n’aie employé tous les moyens que j’ai dans l’esprit, pour ramener ceux que la naissance, l’éducation et le plus souvent un grand zèle sans connaissance, tiennent de bonne foi dans ces pernicieuses erreurs. Ainsi, j’aurai, comme je l’espère, d’autres occasions de vous en parler dans les suites de ces Mémoires, sans vous expliquer par avance des desseins où le temps et les circonstances des choses peuvent apporter mille changements.\par
Je prenais ces soins par une véritable reconnaissance des grâces que je recevais tous les jours. Mais je m’aperçus en même temps qu’ils me servaient beaucoup à me conserver l’affection des peuples, très contents de voir qu’étant sans comparaison beaucoup plus occupé qu’auparavant, je continuais à vivre, pour les exercices de la piété, dans la même régularité où la Reine ma mère m’avait fait élever, et édifiés particulièrement cette année de ce que je fis à pied avec toute ma maison les stations d’un jubilé, ce que je ne pensais pas même devoir être remarqué.\par
Et à vous dire la vérité, mon fils, nous ne manquons pas seulement de reconnaissance et de justice, mais de prudence et de bon sens, quand nous manquons de vénération pour celui dont nous ne sommes que les lieutenants. Notre soumission pour lui est la règle et l’exemple de celle qui nous est due. Les armées, les conseils, toute l’industrie humaine seraient de faibles moyens pour nous maintenir sur le trône, si chacun y croyait avoir même droit que nous, et ne révérait pas une puissance supérieure, dont la nôtre est une partie. Les respects publics que nous rendons à cette puissance invisible, pourraient enfin être nommés justement la première et la plus importante partie de notre politique, s’ils ne devaient avoir un motif plus noble et plus désintéressé.\par
Gardez-vous bien, mon fils, je vous en conjure, de n’avoir dans la religion que cette vue d’intérêt, très mauvaise quand elle est seule, mais qui d’ailleurs ne vous réussirait pas, parce que l’artifice se dément toujours, et ne produit pas longtemps les mêmes effets que la vérité. Tout ce que nous avons d’avantages sur les autres hommes dans la place que nous tenons, sont sans doute autant de nouveaux titres de sujétion pour celui qui nous les a donnés. Mais à son égard, l’extérieur sans l’intérieur n’est rien du tout, et sert plutôt à l’offenser qu’à lui plaire. Jugez-en par vous-même, mon fils, si jamais vous vous trouvez, comme il est difficile que cela n’arrive quelquefois dans le cours de votre vie, en l’état qui est si ordinaire aux rois, et où je me suis vu si souvent : mes sujets rebelles, lorsqu’ils ont eu l’audace de prendre les armes contre moi, m’ont donné peut-être moins d’indignation que ceux qui en même temps se tenant auprès de ma personne, me rendaient plus de devoirs et plus d’assiduité que tous les autres, pendant que je fusse bien informé qu’ils me trahissaient, et n’avaient pour moi ni véritable respect, ni véritable affection dans le cœur.\par
Pour conserver cette disposition intérieure que je désire avant toutes choses, et sur toutes choses en vous, il est utile, mon fils, de se remettre de temps en temps devant les yeux les vérités dont nous sommes persuadés, mais dont nos occupations, nos plaisirs, notre grandeur même effacent incessamment l’image dans nos esprits.\par
Ce n’est pas à moi à faire le théologien avec vous. J’ai pris un soin extrême de choisir pour votre éducation ceux que j’ai crus les plus propres à vous enseigner la piété par les discours et par l’exemple ; et je puis vous assurer que c’est la première qualité que j’ai cherchée et considérée en eux. Ils ne manqueront pas, et j’y prendrai garde, de vous confirmer dans les bonnes maximes, et tous les jours davantage, plus vous serez capable de raisonner avec eux.\par
Si toutefois, par une curiosité assez naturelle, vous vouliez savoir ce qui m’a le plus touché de ce que j’ai jamais vu ou entendu sur de semblables matières, je vous le dirai fort simplement, suivant que le bon sens me le pourra suggérer, sans affecter une profondeur de connaissances qui ne m’appartiennent pas.\par
J’ai donné beaucoup, en premier lieu, au consentement général de toutes les nations et de tous les siècles, et particulièrement de tous, ou presque tous les hommes les plus célèbres dont j’aie jamais entendu parler, soit pour les lettres, soit pour les armes, soit pour la conduite des États, qui en général ont estimé la piété, quoiqu’en différentes manières : au lieu qu’on ne compte depuis tant de temps pour impies et pour athées, qu’un très petit nombre d’esprits médiocres, qui ont voulu passer pour plus grands qu’ils n’étaient, ou du moins que le public ne les a trouvés, puisqu’ils n’ont pu jusqu’ici se faire, comme les autres, un parti considérable dans le monde, une longue suite d’approbateurs et d’admirateurs.\par
Ce consentement universel m’a toujours semblé d’un très grand poids. Car, après tout, il n’est pas étrange que la raison se trompe en un petit nombre de particuliers puisque les sens mêmes, dont la certitude est si grande, se trompent aussi en quelques particuliers, et qu’il y en a qui voient les choses tout à fait différentes de ce qu’elles sont en effet. Mais si en ce qu’il y avait de plus important au monde, et qu’on a étudié avec le plus de soin, la raison humaine généralement parlant s’était trompée en tous les temps et en toutes les nations, et toujours régulièrement de la même sorte, pour nous faire embrasser comme le plus grand et le plus important de tous nos devoirs un fantôme et une chimère, qui ne fût rien du tout, elle ne serait plus elle-même une raison, mais une folie à laquelle il faudrait renoncer, ce qui est la plus grande extravagance et la plus grande contradiction qu’un esprit raisonnable puisse soutenir, puisqu’il ne la soutiendrait qu’en raisonnant lui-même.\par
J’ai considéré ensuite que si l’imagination résiste d’abord à tout ce que nous n’avons pas vu, et par conséquent à tout ce qu’on nous enseigne de la divinité, le jugement s’y rend sans peine aussitôt que nous nous y attachons plus longtemps, et que nous l’examinons de plus près ; car nous ne pouvons juger des choses qui nous sont inconnues, qu’en les comparant à celles que nous connaissons et tirant des conséquences des unes aux autres.\par
Cependant nous ne voyons rien dans le monde, de tout ce qui a quelque rapport et quelque ressemblance avec le monde lui-même, comme sont les machines des bâtiments, et mille autres choses semblables, qui ne soit l’ouvrage de quelque raison ou de quelque esprit qui en a fait le dessin. Cela étant, pourquoi ne croirions-nous pas, quand même l’instinct naturel et la voix de tous les peuples ne nous l’auraient pas appris, que le monde lui-même, qui surpasse si fort toutes ces choses en ordre, en grandeur et en beauté, est aussi l’ouvrage de quelque esprit et de quelque raison, sans comparaison plus grande et plus élevée que la nôtre, dont si ensuite on nous dit mille merveilles, il nous faut seulement examiner qui nous le dit, et quelle assurance il en a, sans nous étonner de ne les pouvoir comprendre, puisque dans le monde même, qui n’en est que l’ouvrage, il y a tant d’autres miracles que nous ne pouvons entendre, encore que nous ne puissions les nier, et qu’ils soient incessamment devant nos yeux. Ainsi, ce qui serait incroyable en soi, s’il est appuyé d’ailleurs de quelque bonne autorité, ne devient pas seulement croyable, mais très vraisemblable, quand il s’agit de cette raison supérieure et si élevée, c’est-à-dire d’une chose très obscure pour nous, qui ne connaissons qu’imparfaitement ce que c’est que notre propre raison.\par
Ces premiers fondements posés, il m’a toujours semblé, mon fils, que tout le reste suivait facilement. La variété infinie des religions peut faire peine, mais elles ont toutes au fond tant de rapports l’une à l’autre, tant de principes et tant de fondements qui leur sont communs, que leur diversité même confirme visiblement une seule religion, dont toutes les autres sont des copies incomplètes ou falsifiées, qui ne laissent pas de conserver les traits les plus remarquables de l’originale.\par
Et quand il n’est plus question que de démêler cet original d’entre ces copies, quelle autre religion le peut emporter sur la nôtre, à laquelle tout ce qu’il y a eu de gens habiles et éclairés dans le monde se sont rendus quand elle a paru, qui est aujourd’hui embrassée et suivie, non pas comme les autres par des nations barbares, ignorantes et grossières, mais par toutes celles où l’esprit et le savoir sont le plus cultivés ; qui d’ailleurs, si on regarde l’ancienneté, est la même que la juive, la plus ancienne de toutes, et dont elle n’est que la perfection et, l’accomplissement, prédit, promis et annoncé plusieurs siècles auparavant par des hommes extraordinaires, en même temps qu’ils faisaient mille autres prédictions que l’événement confirmait chaque jour ; qui dès ce temps-là s’est vantée hardiment qu’aussitôt qu’elle serait à ce point de perfection qu’elle attendait, elle détruirait entièrement la païenne, dont elle était alors méprisée ou opprimée, et n’y a pas manqué ; tous ces dieux qu’elle adorait ayant disparu devant le sien, sans qu’il leur soit plus resté un seul adorateur dans le monde.\par
Les vérités qu’elle publie sont surprenantes, mais nous avons posé que rien ne doit nous surprendre, ni paraître trop grand en ce qui est si fort au-dessus de nous. Le monde les a apprises par ceux qui en étaient témoins oculaires, et que le bon sens ne nous permet pas encore aujourd’hui de soupçonner : ni de folie, puisque leur morale, du consentement des impies même, passe de bien loin celle des plus sages philosophes ; ni d’imposture, puisqu’on demeure d’accord qu’ils ont vécu sans intérêt, sans bien, sans ambition, sans plaisirs, fournissant le plus souvent, par le travail de leurs mains, au peu qui leur était nécessaire ; courant avec autant de fatigue que de péril par toute la terre pour la convertir ; méprisés, persécutés, et finissant presque tous leur vie par le martyre, mais ne se relâchant ni se démentant jamais par eux et par leurs successeurs ; cette religion qui prêchait des mystères si opposés au sens humain, et des maximes si dures et si fâcheuses aux gens du monde, sans les forcer par aucune violence, sans armer jamais le sujet contre le prince, ni le citoyen contre le citoyen, sans faire jamais que souffrir et que prier, a désarmé ses persécuteurs et toutes les puissances qui lui étaient contraires, s’est établie par tout le monde, s’est vue dominante en moins de trois siècles : ce qui ne peut arriver dans le bon sens que par les miracles dont l’histoire chrétienne est remplie, et que nous ne voyons plus aujourd’hui mais dont ce progrès si grand et si étonnant du christianisme nous prouve la vérité outre mille autres témoignages très authentiques.\par
Voilà, mon fils, les considérations dont j’ai été le plus touché ; je ne doute pas que celles-là même, ou d’autres, ne fassent un pareil effet sur vous, et que vous ne tâchiez de répondre sincèrement au nom de {\itshape très-chrétien} que nous portons ; si ce ne peut être en toutes vos actions, comme il serait à souhaiter, ne perdez jamais de vue pour le moins, ce qui fait tout le mérite des bonnes, et tout le remède des mauvaises et des faibles.\par
Plusieurs de mes ancêtres ont attendu l’extrémité de leur vie pour faire de pareilles exhortations à leurs enfants ; j’ai cru au contraire qu’elles auraient plus de force sur vous, lorsque la vigueur de mon âge, la liberté de mon esprit, l’état florissant de mes affaires ne vous permettraient point d’y soupçonner de déguisement, ou de les attribuer à la vue du péril. Ne me donnez pas ce déplaisir, mon fils, qu’elles n’aient un jour servi qu’à vous rendre plus coupable, comme elles le feraient sans doute si vous veniez à les oublier.
\subsubsection[{Troisième section}]{Troisième section}
\noindent Ce fut alors que je crus devoir mettre sérieusement la main au rétablissement des finances, et la première chose que je jugeai nécessaire, fut de déposer de leurs emplois les principaux officiers par qui le désordre avait été introduit. Car depuis le temps que je prenais soin de mes affaires, j’avais de jour en jour découvert de nouvelles marques de leurs dissipations, et principalement du surintendant. La vue des vastes établissements que cet homme avait projetés, et les insolentes acquisitions qu’il avait faites, ne pouvaient qu’elles ne convainquissent mon esprit du dérèglement de son ambition ; et la calamité générale de tous mes peuples sollicitait sans cesse ma justice contre lui.\par
Mais ce qui le rendait plus coupable envers moi, était que bien loin de profiter de la bonté que je lui avais témoignée en le retenant dans mes conseils, il en avait pris une nouvelle espérance de me tromper, et bien loin d’en devenir plus sage, tâchait seulement d’en être plus adroit. Mais quelque artifice qu’il pût pratiquer, je ne fus pas longtemps sans reconnaître sa mauvaise foi ; car il ne pouvait s’empêcher de continuer ses dépenses excessives, de fortifier des places, d’orner des palais, de former des cabales, et de mettre sous le nom de ses amis des charges importantes qu’il leur achetait à mes dépens, dans l’espoir de se rendre bientôt l’arbitre souverain de l’État.\par
Quoique ce procédé fût assurément fort criminel, je ne m’étais d’abord proposé que de l’éloigner des affaires ; mais ayant depuis considéré que de l’humeur inquiète dont il était, il ne supporterait point ce changement de fortune sans tenter quelque chose de nouveau, je pensai qu’il était plus sûr de l’arrêter. Je différai néanmoins l’exécution de ce dessein, et ce dessein me donna une peine incroyable. Car, non seulement je voyais que pendant ce temps-là, il pratiquait de nouvelles subtilités pour me voler ; mais ce qui m’incommodait davantage était que, pour augmenter la réputation de son crédit, il affectait de me demander des audiences particulières, et que pour ne lui pas donner de défiance, j’étais contraint de les lui accorder, et de souffrir qu’il m’entretînt de discours inutiles, pendant que je connaissais à fond toute son infidélité.\par
Vous pouvez juger qu’à l’âge où j’étais, il fallait que ma raison fît beaucoup d’effort sur mes ressentiments, pour agir avec tant de retenue. Mais, d’une part, je voyais que la déposition du surintendant avait une liaison nécessaire avec le changement des fermes ; et, d’autre côté, je savais que l’été où nous étions alors, était celle des saisons de l’année où ces innovations se faisaient avec le plus de désavantage, outre que je voulais avant toutes choses avoir un fonds en mes mains de quatre millions, pour les besoins qui pourraient survenir. Ainsi, je me résolus d’attendre l’automne pour exécuter ce projet.\par
Mais étant allé vers la fin du mois d’août à Nantes, où les États de Bretagne étaient assemblés, et, de là, voyant de plus près qu’auparavant les ambitieux projets de ce ministre, je ne pus m’empêcher de le faire arrêter en ce lieu même, le 5 septembre. Toute la France, persuadée aussi bien que moi de la mauvaise conduite du surintendant, applaudit à cette action, et loua particulièrement le secret dans lequel j’avais tenu, durant trois ou quatre mois, une résolution de cette nature, principalement à l’égard d’un homme qui avait des entrées si particulières auprès de moi, qui entretenait commerce avec tous ceux qui m’approchaient, qui recevait des avis du dedans et du dehors de l’État, et qui de soi-même devait tout appréhender par le seul témoignage de sa conscience.\par
Mais ce que je crus avoir fait en cette occasion de plus digne d’être observé et de plus avantageux pour mes peuples, c’est d’avoir supprimé la charge de surintendant, ou plutôt de m’en être chargé moi-même. Peut-être qu’en considérant la difficulté de cette entreprise, vous serez un jour étonné, comme l’a été toute la France, de ce que je me suis engagé à cette fatigue dans un âge où l’on n’aime ordinairement que le plaisir. Mais je vous dirai naïvement que j’eus à ce travail, quoique fâcheux, moins de répugnance qu’un autre, parce que j’ai toujours considéré comme le plus doux plaisir du monde la satisfaction qu’on trouve à faire son devoir. J’ai même souvent admiré comment il se pouvait faire que l’amour du travail, étant une qualité si nécessaire aux souverains, fût pourtant une de celles qu’on trouve plus rarement en eux.\par
La plupart des princes, parce qu’ils ont un grand nombre de serviteurs et de sujets, croient n’être obligés à se donner aucune peine, et ne considèrent pas que s’ils ont une infinité de gens qui travaillent sous leurs ordres, ils en ont infiniment davantage qui se reposent sur leur conduite, et qu’il faut beaucoup veiller et beaucoup travailler pour empêcher seulement que ceux qui agissent ne fassent rien que ce qu’ils doivent faire, et que ceux qui se reposent ne souffrent rien que ce qu’ils doivent souffrir. Toutes ces différentes conditions dont le monde est composé, ne sont unies les unes aux autres que par un commerce de devoirs réciproques. Ces obéissances et ces respects que nous recevons de nos sujets, ne sont pas un don qu’ils nous font, mais un échange avec la justice et la protection qu’ils prétendent recevoir de nous. Comme ils nous doivent honorer, nous les devons conserver et défendre ; et ces dettes dont nous sommes chargés envers eux, sont même d’une obligation plus indispensable que celles dont ils sont tenus envers nous : car enfin si l’un d’eux manque d’adresse ou de volonté pour exécuter ce que nous lui commandons, mille autres se présentent en foule pour remplir sa place, au lieu que l’emploi de souverain ne peut être bien rempli que par le souverain même.\par
Mais pour descendre plus particulièrement à la manière dont nous parlons, il faut ajouter à ceci que de toutes les fonctions souveraines, celle dont un prince doit être le plus jaloux, est le maniement des finances. C’est la plus délicate de toutes, parce que c’est celle de toutes qui est la plus capable de séduire celui qui l’exerce, et qui lui donne plus de facilité à corrompre que les autres. Il n’y a que le prince seul qui doive en avoir la souveraine direction, parce qu’il n’y a que lui seul qui n’ait point de fortune à établir que celle de l’État, point d’acquisition à faire que pour l’accroissement de la monarchie, point d’autorité à élever que celle des lois, point de dettes à payer que les charges publiques, point d’amis à enrichir que ses peuples.\par
Et, en effet, que peut-il y avoir de plus ruineux pour les provinces ou de plus honteux pour leur roi, que d’élever un homme qui a ses desseins et ses affaires particulières dans une place, qui prétend compter entre ses droits celui de disposer de tout sans rendre compte de rien, et de remplir incessamment ses coffres et ceux de ses créatures des plus clairs deniers du public ? Un prince peut-il faire de plus grande folie que d’établir des particuliers qui se servent de son autorité pour s’enrichir à ses propres dépens, et de qui la dissipation, quoiqu’elle ne lui produise aucun plaisir, ruine à la fois ses affaires et sa réputation ? Et pour parler encore plus chrétiennement, peut-il s’empêcher de considérer que ces grandes sommes, dont un petit nombre de financiers composent leurs richesses excessives et monstrueuses, proviennent toujours des sueurs, des larmes et du sang des misérables, dont la défense est commise à ses soins ?\par
Ces maximes que je vous apprends aujourd’hui, mon fils, ne m’ont été enseignées par personne, parce que mes devanciers ne s’en étaient pas avisés ; mais sachez que l’avantage que vous avez d’en être instruit de si bonne heure tournera quelque jour à votre confusion, si vous n’en savez profiter.\par
Outre les conseils de finances et les directions qui s’étaient tenus de tous temps, je voulus, pour m’acquitter avec plus de précaution de la surintendance, établir un conseil nouveau, que j’appelai Conseil royal. Je le composai du maréchal de Villeroi, de deux conseillers d’État, d’Aligre et de Sève, et d’un intendant des finances, qui fut Colbert ; et c’est dans ce conseil que j’ai travaillé continuellement depuis à démêler la terrible confusion qu’on avait mise dans mes affaires.\par
Ce n’était pas assurément une entreprise légère, et ceux qui ont vu les choses au point où elles étaient, et qui les regardent maintenant dans la netteté où je les ai réduites, s’étonnèrent avec raison que j’aie pu pénétrer en si peu de temps une obscurité que tant d’habiles surintendants n’avaient encore jamais éclaircie. Mais ce qui doit faire cesser cette surprise, est la différence qui se trouve naturellement entre l’intérêt du prince et celui de ses surintendants. Car ces particuliers n’ayant point de plus grand soin dans leur emploi que de se conserver la liberté de disposer de tout à leur fantaisie, mettent bien plus souvent leur adresse à rendre cette matière obscure qu’à l’éclaircir : au lieu qu’un roi qui en est le seigneur légitime, met autant qu’il peut l’ordre et la netteté en toutes choses, parce qu’il ne peut trouver que de la perte dans la confusion ; outre qu’en mon particulier je fus souvent soulagé dans ce travail par Colbert, que je chargeais de l’examen des choses qui demandaient trop de discussion, et dans lesquelles je n’eusse pas eu le loisir de descendre.\par
La manière en laquelle s’était faite la recette et la dépense, était une chose incroyable. Mes revenus n’étaient plus maniés par mes trésoriers, mais par les commis du surintendant qui lui en comptaient confusément avec ses dépenses particulières ; et l’argent se déboursait en tel temps, en telle forme, et pour telle cause qu’il leur plaisait ; et l’on cherchait après à loisir de fausses dépenses, des ordonnances de comptant, et des billets réformés pour consommer toutes ces sommes. Le continuel épuisement qui se faisait du trésor public, et l’avidité qu’on avait toujours de nouvel argent, faisait qu’on donnait sans peine des remises exorbitantes à ceux qui offraient d’en avancer. L’humeur déréglée de Fouquet lui avait toujours fait préférer les dépenses inutiles aux nécessaires, d’où il arrivait que les fonds les plus liquides étant consommés en gratifications distribuées à ses amis, en bâtiments faits pour son plaisir, ou en autres choses de pareille nature, on était contraint, au moindre besoin de l’État, d’avoir recours à des aliénations qu’on ne faisait jamais qu’à vil prix, à cause de l’extrême nécessité où on était. Par ces voies l’État s’était tellement appauvri, que, nonobstant les tailles immenses qui se levaient, il ne restait plus de net à l’Épargne que vingt et un millions par an, lesquels même étaient dépensés pour deux années par avance, sans compter soixante et dix millions dont on m’avait rendu redevable par billets faits au profit de divers particuliers.\par
La chose que j’eus le plus d’impatience de corriger dans cet abus général, fut l’usage des ordonnances de comptant, parce qu’elles avaient assurément plus servi qu’aucune autre à la dissipation de mes deniers, car en cette forme on donnait sans cesse et sans mesure à telle personne qu’on voulait, et on faisait sans honte et sans peur une dépense qui ne devait jamais être connue. Pour éviter à l’avenir cette confusion, je résolus de libeller et d’enregistrer moi-même toutes les ordonnances que je signerais, en sorte qu’il ne s’est fait ni pu faire depuis aucune dépense dont je n’aie su la raison.\par
Je voulus aussi rebailler mes fermes, qui jusqu’alors n’avaient pas été portées à leur juste prix ; et afin d’éviter les fraudes qui s’étaient souvent faites dans ces occasions, soit par la corruption des juges qui les adjugeaient, soit par les complots secrets que faisaient entre eux ceux qui les devaient enchérir, je me trouvai moi-même aux enchères, et ce premier essai de mon application me fit augmenter mon revenu de trois millions, outre que je rendis le prix des baux payable par mois, ce qui me donna dès lors de quoi fournir aux dépenses les plus pressées, et me fit épargner à l’État une perte de quinze millions par an, qui s’étaient jusque-là consumés dans les intérêts des sommes qu’on avait empruntées.\par
Pour les traites des recettes générales, au lieu de cinq sols de remise qui se donnaient auparavant, je ne laissai plus que quinze deniers pour livre : diminution qui, sur le total du royaume, montait à une somme si notable, qu’elle me donna lieu, dans le grand épuisement où j’étais, de rabaisser les tailles de quatre millions.\par
Je m’étonnais moi-même qu’en si peu de temps, et par des voies si pleines de justice, j’eusse pu trouver tant de profit pour le public. Mais ce qui pouvait causer un plus grand étonnement, c’est que ceux qui traitèrent avec moi à ces conditions firent un gain presque aussi grand et beaucoup plus solide que ceux qui avaient traité auparavant, parce que le respect que mes sujets avaient dès lors pour moi, et le soin que je prenais de protéger ceux qui me servaient dans tout ce qu’ils me demandaient avec justice, leur faisaient trouver alors autant de facilité dans leur recette qu’ils y avaient auparavant rencontré de chicane et d’endurcissement.\par
Je résolus, peu de temps après, de réduire à deux quartiers les augmentations de gages que les officiers avaient acquises à vil prix, et qui, ayant été payées jusque-là pour trois quartiers, avaient beaucoup diminué le prix de mes fermes. Mais je vous ai déjà expliqué, en parlant des compagnies souveraines, la justice de cette réduction et la facilité que j’y trouvai, et je ne vous la marquerai maintenant en passant que comme un des bons effets de cette économie qui était si nécessaires à mon État.\par
Mais la dernière résolution que je pris cette année-là, touchant les finances, fut l’établissement de la Chambre de justice, dans lequel j’eus deux principaux motifs. Le premier que dans l’état où les choses étaient réduites, il n’était pas possible de diminuer suffisamment les impositions ordinaires, et de soulager aussi promptement la pauvreté des peuples, qu’en faisant contribuer puissamment aux dépenses de l’État ceux qui s’étaient enrichis à ses dépens ; et le second, que cette chambre examinant les traités qui avaient été faits, c’était le seul moyen qui pouvait faciliter l’acquittement de mes dettes. Car on les faisait monter à des sommes si prodigieuses, que je n’aurais pu les payer toutes sans ruiner la plus grande partie de mes sujets, ni les abolir de ma pure autorité sans me mettre en danger de faire quelque injustice, outre que je ne voulais pas retomber dans l’abus qui s’était pratiqué dans le remboursement des billets de l’Épargne, par le moyen desquels les gens de crédit se faisaient payer tôt ou tard des sommes qui ne leur étaient point dues, pendant que les véritables créanciers n’auraient tiré qu’une très légère portion de leur dû.\par
C’est pourquoi je crus qu’il était bon de liquider exactement ce que je devais, et ce qu’on me devait, pour payer l’un et me faire payer de l’autre : mais parce que ces discussions étaient délicates, et que la plupart de ceux qui s’y trouvaient intéressés avaient beaucoup de crédit et beaucoup de parents dans les compagnies ordinaires de judicature, je fus obligé d’en former une exprès des hommes les plus désintéressés qui se trouvaient en toutes les autres.\par
Je ne doute point qu’en lisant tout ce détail, vous ne conceviez en vous-même que l’application qu’il fallait pour toutes ces sortes de choses n’avait pas en soi beaucoup d’agrément, et que ce grand nombre d’ordonnances, de baux, de déclarations, de registres et d’états, qu’il fallait non seulement voir et signer, mais concevoir et résoudre, n’était pas une matière qui satisfît beaucoup un esprit capable d’autres choses : et je veux bien en demeurer d’accord avec vous. Mais si vous considérez dans la suite les grands avantages que j’en ai tirés, les soulagements que j’ai accordés chaque année à mes sujets, de combien de dettes j’ai dégagé l’État, combien j’ai racheté de droits aliénés, avec quelle ponctualité j’ai payé toutes les charges légitimes, et quel nombre de pauvres ouvriers j’ai fait subsister en les occupant dans mes bâtiments ; combien de gratifications j’ai faites à des gens de mérite ; comment j’ai entretenu les ouvrages publics ; quels secours d’hommes et d’argent j’ai fournis à mes alliés ; de combien j’ai augmenté le nombre de mes vaisseaux ; quelles places j’ai achetées ; avec quelle vigueur je me suis mis en possession des droits qu’on m’a contestés, sans que pour cela j’aie jamais été réduit à la malheureuse nécessité de charger mes sujets d’aucune imposition extraordinaire : vous trouverez sans doute alors que les travaux par lesquels je me suis mis en cet état m’ont dû paraître fort agréables, puisqu’ils ont produit tant de fruits pour mes sujets.\par
Car enfin, mon fils, nous devons considérer le bien de nos sujets bien plus que le nôtre propre. Il semble qu’ils fassent une partie de nous-mêmes, puisque nous sommes la tête d’un corps dont ils sont les membres. Ce n’est que pour leurs propres avantages que nous devons leur donner des lois ; et ce pouvoir que nous avons sur eux ne nous doit servir qu’à travailler plus efficacement à leur bonheur. Il est beau de mériter d’eux le nom de père avec celui de maître, et si l’un nous appartient par le droit de notre naissance, l’autre doit être le plus doux objet de notre ambition. Je sais bien que ce titre si beau ne s’obtient pas sans beaucoup de peine ; mais dans les entreprises louables, il ne faut pas s’arrêter à la vue de la difficulté. Le travail n’épouvante que les âmes faibles ; et dès lors qu’un dessein est avantageux et juste, ne le pas exécuter est une faiblesse. La paresse chez ceux de notre rang est opposée à la grandeur de courage, aussi bien que la timidité ; et il est sans doute qu’un monarque obligé de veiller à l’intérêt public mérite plus de blâme en fuyant une peine utile, qu’en s’arrêtant à la vue d’un danger pressant : car enfin la crainte du danger peut être presque toujours colorée par un sentiment de prudence ; au lieu que l’appréhension du travail ne peut jamais être considérée que comme une mollesse inexcusable.\par
J’étais dans ces occupations quand il me vint nouvelles de Londres, que le 10 octobre, à l’entrée d’un ambassadeur de Suède, l’ambassadeur d’Espagne, le baron de Vatteville, avait prétendu former une concurrence de rang entre les ministres du roi son maître et les miens ; et que sur cette vision, ayant sous main et à force d’argent disposé les choses à une sédition populaire, il avait osé faire arrêter le carrosse du comte d’Estrades, mon ambassadeur, par une troupe de canaille armée, tué les chevaux à coups de mousquet, et l’avait empêché enfin de marcher en sa véritable place. Vous jugerez de mon indignation par la vôtre même, car je ne doute pas, mon fils, que vous n’en soyez encore ému en lisant ceci, et ne vous trouviez aussi sensible que je l’ai toujours été à l’honneur d’une couronne qui vous est destinée.\par
Ce qui me blessait davantage, c’est que je ne pouvais regarder cette offense comme l’effet d’une querelle prise sur-le-champ, où le hasard eût plus de part que le dessein. C’était au contraire une résolution faite de longue main, et dont ce ministre avait voulu flatter sa vanité et celle de sa nation. Il avait été très mortifié du mariage de Portugal, qu’il n’avait pu empêcher, quoiqu’il eût formé pour cela une grande cabale dans Londres, et des personnes les plus considérables de la cour, jusqu’à irriter le roi lui-même par ce procédé. L’argent qu’il avait demandé en Espagne pour rompre ce coup était arrivé, mais trop tard. Et ne se pouvant apparemment dégager de ses partisans, à qui il l’avait fait espérer, il cherchait du moins à employer cette dépense en quelque chose d’éclat qui pût faire honneur au roi son maître.\par
Avec ce dessein, quelque temps auparavant, dans une occasion toute semblable, qui était l’entrée d’un ambassadeur extraordinaire de Venise, il avait fait dire à d’Estrades, que pour conserver l’amitié entre les rois leurs maîtres, et pour imiter le cardinal Mazarin et don Louis de Haro qui, à l’île de la Conférence, avaient, disait-il, partagé toutes ces choses, la terre, l’eau et le soleil, il serait d’avis qu’ils n’envoyassent, ni l’un ni l’autre, leurs carrosses au-devant de cet ambassadeur : sur quoi n’ayant reçu qu’un refus bien formel, et d’Estrades lui ayant protesté au contraire qu’il entendait y envoyer et y conserver son rang, il témoigna de son côté la même chose, et qu’il enverrait aussi son carrosse, à moins, ajouta-t-il, que l’ambassadeur eût pris le même parti que d’autres ambassadeurs extraordinaires, qui était de ne notifier son arrivée et son entrée à personne, auquel cas personne n’était obligé de s’y trouver. Là-dessus ayant fait venir le résident de Venise, qui était son ami, et avec qui il était déjà d’accord, ce résident confirma que l’ambassadeur voulait imiter le prince de Ligne, qui, étant aussi ambassadeur extraordinaire quelque temps auparavant, avait cru se distinguer avantageusement des ambassadeurs ordinaires en ne notifiant son arrivée à qui que ce soit.\par
Le roi d’Angleterre, qui n’avait autre intérêt en cette dispute que d’empêcher toute sorte de bruit et d’émotion dans sa ville capitale, et qui était sollicité par Vatteville, n’eut pas de peine à intervenir ensuite, et à faire prier mon ambassadeur et tous les autres de ne point envoyer à l’entrée de celui de Venise, qui aussi ne le désirait pas, puisqu’il ne les en faisait point avertir : en un mot, on en usa ainsi pour cette fois. J’en fus très irrité aux premiers bruits qui m’en vinrent assez confusément : il me semblait que le roi d’Angleterre, qui alors me témoignait beaucoup d’amitié, avait eu tort de se mêler de ce différend ; que d’Estrades devait se défendre non seulement de ses prières, mais de ses ordres exprès, s’il en avait envoyé, et répondre qu’un ambassadeur ne recevait aucun ordre que de son maître, enfin, se retirer plutôt que de consentir à cet expédient qui me paraissait honteux.\par
Mais je n’eus rien à dire quand j’appris par ses lettres ce qui s’était passé, et que le roi n’avait ajouté que sa simple prière à la résolution déjà prise par l’ambassadeur de Venise, qui, dans l’ordre commun, devait empêcher tous les autres d’envoyer au-devant de lui ; et j’avais moins de sujet de m’en plaindre que personne, parce que dans ma propre cour j’avais pratiqué et comme inventé cet expédient, peu de temps auparavant, pour éviter la concurrence de quelques ambassadeurs, à la vérité mieux fondée que celle qu’on voulait établir entre l’Espagne et la France.\par
Mais je voyais à quoi allait la subtilité des Espagnols, et que par des négociations semblables avec les ambassadeurs qui entreraient à l’avenir, sur le prétexte toujours plausible d’éviter un désordre, ils tâcheraient de faire oublier une préséance qui m’appartient si légitimement. J’en étais en possession par toute l’Europe, et surtout à Rome, où les gardes mêmes du pape ont été quelquefois employés à la conserver à mes prédécesseurs ; et ni là, ni à Venise, les ambassadeurs d’Espagne ne se trouvaient plus depuis longtemps aux cérémonies publiques, où les miens assistaient. En nul temps, et même dans le plus florissant état de leur monarchie, elle n’est venue à bout d’établir l’égalité où elle aspirait. Et quand mes prédécesseurs, occupés par leurs troubles domestiques, se sont le plus relâchés sur ce sujet, tout ce que ses ministres ont pu faire, a été d’usurper, comme au concile de Trente, quelque rang bizarre, qui n’étant ni le premier, ni égal au premier, pût passer dans leur imagination pour n’être pas le second, quoiqu’il le fût en effet.\par
Ainsi, je ne pouvais digérer de voir mon droit éludé par l’artifice de Vatteville, et cet artifice souvent répété pouvait former à la fin non seulement la prétention, mais presque la possession d’un droit contraire. Au point où j’avais déjà porté la dignité du nom français, je ne pensais pas la devoir laisser à mes successeurs moindre que je ne l’avais reçue. Et me souvenant que dans les matières d’État il faut quelquefois couper ce qu’on ne peut dénouer, je mandai nettement à d’Estrades, qu’à la première entrée d’ambassadeur, soit ordinaire, soit extraordinaire, soit qu’elle lui eût été notifiée ou non, il ne manquât pas de lui envoyer son carrosse, et de lui faire prendre et conserver le premier rang.\par
Il se mit en état de m’obéir à cette entrée de l’ambassadeur de Suède, qui à la vérité lui avait notifié d’abord son arrivée, et le jour qu’il entrerait, mais qui, depuis, à la sollicitation des Espagnols, et peut-être du roi d’Angleterre même, l’avait fait prier de ne point envoyer au-devant de lui, comme ayant changé d’avis, et voulant en user de même que les derniers ambassadeurs extraordinaires. À cela, d’Estrades, instruit auparavant par mes lettres, répondit que l’alliance et l’amitié étroite qui étaient entre la France et la Suède ne lui permettaient pas de manquer à ce devoir, sans que je le trouvasse mauvais. Mais encore qu’il eût rassemblé tous les Français qui se trouvaient à Londres, qu’il eût fait venir de Gravelines, dont il était gouverneur, quelques officiers de son régiment et quelques cavaliers de la compagnie de son fils, que tout cela ensemble pût aller à quatre ou cinq cents hommes, que ceux qui accompagnaient son carrosse, ou ceux qui les devaient soutenir, et le marquis d’Estrades son fils qui était à leur tête, fissent tout ce que pouvaient de braves gens à un pareil tumulte, il ne leur fut pas possible de l’emporter sur une multitude infinie de peuple, déjà naturellement mal disposé contre les Français, mais encore alors excité par les émissaires de Vatteville qui, si on m’a dit la vérité, avait armé plus de deux mille hommes, et employé près de cinq cent mille livres à cette belle entreprise.\par
Le roi d’Angleterre, qui s’était secrètement engagé à d’Estrades de me conserver mon rang, avait fait publier, quelques jours devant, des défenses à tous sujets de prendre aucun parti, ni pour l’un, ni pour l’autre, et placé ses gardes en divers lieux de la ville pour empêcher ce qui arriva. Mais il n’en fut pas le maître ; et tout ce qu’il put faire, fut d’apaiser le tumulte après plusieurs personnes tuées et blessées de part et d’autre, et presque autant du côté des Espagnols que des Français.\par
Cependant ils croyaient déjà avoir défait mes armées par ce misérable avantage, qui leur coûta encore plus dans les suites qu’il n’avait fait jusqu’alors : mais ils changèrent d’avis quand ils virent de quelle sorte je ressentais cet outrage, et ce que j’étais capable de faire pour le réparer. Aussitôt après en avoir reçu la nouvelle, je fis commander au comte de Fuensaldagna, leur ambassadeur, de sortir incessamment du royaume, sans me voir, ni les reines, le chargeant de plus d’avertir le marquis de Fuentes, qui venait d’Allemagne pour prendre sa place, qu’il eût à ne point entrer dans mes États. Je révoquai le passeport que j’avais donné au marquis de Caracena, gouverneur de Flandre, pour passer par la France, en se retirant en Espagne ; j’ordonnai au gouverneur de Péronne de le lui faire savoir de ma part. Je mandai aux commissaires que j’avais nommés pour l’exécution de la paix de surseoir et de rompre tout commerce avec ceux du Roi Catholique. Je dépêchai en diligence à Madrid l’un des gentilshommes ordinaires de ma maison, avec ordre à l’archevêque d’Embrun, mon ambassadeur, de demander une punition personnelle et exemplaire de Vatteville, et une réparation non seulement proportionnée à l’offense, mais aussi qui m’assurât à l’avenir que les ministres d’Espagne ne feraient plus de pareilles entreprises sur les miens. Je lui commandai enfin de déclarer hautement que je saurais bien me rendre à moi-même la justice qui m’était due, si on me la refusait. Je fis aussi faire instance par d’Estrades auprès du roi d’Angleterre, pour le châtiment des coupables, et lui ordonnai ensuite de se retirer de cette cour, comme d’un lieu où il ne pouvait plus être ni avec sûreté, ni avec dignité et bienséance, jusqu’à la réparation de cet attentat.\par
Il ne fut pas difficile de persuader à tout le monde, par ces démonstrations, ce qui était en effet dans le fond de mon cœur. Car il est vrai que j’aurais porté jusqu’aux dernières extrémités un ressentiment aussi juste que celui-là, et que même dans ce mal j’aurais regardé comme un bien le sujet d’une guerre légitime, où je pusse acquérir de l’honneur, en me mettant à la tête de mes armées.\par
La cour d’Espagne n’était pas dans des sentiments pareils : mais elle se confiait en l’art de négocier, où cette nation croit être la maîtresse des autres. Don Louis de Haro, qui était sur la fin de sa vie, sentant la faiblesse de l’État et la sienne propre, ne craignait rien tant que cette rupture. Il cherchait seulement, par des conférences longues et réitérées avec mon ambassadeur, à gagner du temps en cette affaire, s’imaginant que tout y deviendrait plus facile, après qu’on aurait laissé passer la première chaleur. Il fut bien surpris quand il vit que les choses avaient changé de face entre la France et l’Espagne : car au traité des Pyrénées, c’était le cardinal Mazarin qui tâchait de le persuader par des raisonnements, auxquels il répondait toujours en deux mots par des ordres précis de son roi et du conseil d’Espagne, qu’il ne pouvait ni n’osait passer ; ici, au contraire, c’était lui qui raisonnait, et mon ambassadeur qui tenait ferme sur mes ordres précis, l’obligeant continuellement à descendre à des soumissions très fâcheuses.\par
Il mourut là-dessus. Je me servis de la conjoncture : je pris pour déjà décidées, avec des ministres nouveaux et encore incertains de leur conduite, toutes les conditions qui lui avaient seulement été proposées, pour avoir encore moyen de leur en demander d’autres. Chacun de mes courriers portait des ordres plus durs et plus pressants, et le conseil d’Espagne, voyant que tous les instants de délai rendaient sa condition plus mauvaise, se hâta lui-même de conclure aux conditions que je désirais.\par
Déjà, pour commencer à me satisfaire, on avait rappelé Vatteville et on l’avait relégué à Burgos, sans lui permettre d’aller à la cour, le punissant d’une faute qu’il n’avait peut-être pas faite sans aveu, mais où il avait plus de part que personne, par la facilité que trouvent toujours les ministres d’un prince en pays étranger à faire agréer de loin à leurs maîtres les entreprises qu’ils proposent comme glorieuses et aisées tout ensemble.\par
On régla outre cela par écrit une réparation publique qui fut ponctuellement exécutée ensuite, comme on me l’avait promis, et dont le procès-verbal a été publié, signé de mes quatre secrétaires d’État. Je crois nécessaire de vous en rapporter la substance : car encore que j’écrive ici les affaires de 1661, et que cette satisfaction ne m’ait été faite que le 4 mai 1662, je vous ai dit ailleurs que je ne prétends pas suivre si précisément l’ordre des dates, quand il s’agit de rassembler sur une même matière tout ce qui lui appartient. Le comte de Fuensaldagna, ambassadeur extraordinaire du Roi Catholique, se rendit au Louvre dans mon grand cabinet, où étaient déjà le nonce du pape, les ambassadeurs, résidents et envoyés de tous les princes qui en avaient alors auprès de moi, avec les personnes les plus considérables de mon État. Là, m’ayant premièrement présenté la lettre qui le déclarait ambassadeur, il m’en rendit aussitôt une seconde, en créance de ce qu’il me dirait sur cette affaire de la part du roi son maître. Ensuite il me déclara que Sa Majesté Catholique n’avait pas été moins fâchée et moins surprise que moi de ce qui s’était passé à Londres ; et qu’aussitôt qu’elle en avait eu avis, elle avait ordonné au baron de Vatteville, son ambassadeur, de sortir d’Angleterre et de se rendre en Espagne, le révoquant de l’emploi qu’il avait, pour me donner satisfaction, et témoigner contre lui le ressentiment que méritent ses excès ; qu’elle lui avait aussi commandé de m’assurer qu’elle avait déjà envoyé ses ordres à tous ses ambassadeurs et ministres, tant en Angleterre qu’en toutes les autres cours où se pourraient présenter à l’avenir de pareilles difficultés, afin qu’ils s’abstinssent et ne concourussent point avec mes ambassadeurs et ministres, en toutes les fonctions et cérémonies publiques où mes ambassadeurs et ministres assisteraient.\par
Je lui répondis que j’étais bien aise d’avoir entendu la déclaration qu’il m’avait faite de la part du roi son maître, parce qu’elle m’obligerait de continuer à bien vivre avec lui. Après quoi, cet ambassadeur s’étant retiré, j’adressai la parole au nonce du pape et à tous les ambassadeurs, résidents ou envoyés, qui étaient présents, et leur dis qu’ils avaient entendu la déclaration que l’ambassadeur d’Espagne m’avait faite, que je les priais de l’écrire à leurs maîtres, afin qu’ils sussent que le Roi Catholique avait donné ordre à ses ambassadeurs de céder la préséance aux miens en toutes sortes d’occasions.\par
Je ne serai pas fâché, mon fils, comme cette affaire est importante, que vous y fassiez quelques réflexions utiles. En premier lieu, cet exemple remarquable vous confirmera ce que j’ai déjà établi par la raison au commencement de ces Mémoires. Je veux dire qu’après avoir pris conseil, c’est à nous à former nos résolutions, personne n’osant ni ne pouvant quelquefois nous les inspirer aussi bonnes et aussi royales que nous les trouvons en nous-mêmes. Ce succès se peut sans doute appeler heureux, puisque j’ai obtenu ce que mes prédécesseurs n’avaient pas même espéré, obligeant les Espagnols non seulement à ne plus prétendre la concurrence, mais même à déclarer si solennellement et par un acte si authentique qu’ils ne la prétendraient plus. Et je ne sais si depuis le commencement de la monarchie il s’est rien passé de plus glorieux pour elle : car les rois et les souverains que nos ancêtres ont vus quelquefois à leurs pieds tous leur rendre hommage n’y étaient pas comme souverains et comme rois, mais comme seigneurs de quelque principauté moindre, qu’ils tenaient en fief et à laquelle ils pouvaient renoncer. Ici c’est une espèce d’hommage véritablement d’une autre sorte, mais de roi à roi, de couronne à couronne, qui ne laisse plus douter à nos ennemis mêmes que la nôtre ne soit la première de toute la chrétienté. Ce succès pourtant n’eût pas été tel, je le puis dire avec vérité, si depuis le commencement jusqu’à la fin je n’eusse suivi mes propres mouvements, beaucoup plus que ceux d’autrui : ce qui a été pour moi un long et durable sujet de joie.\par
Il ne faut pas croire que l’intérêt porte tout le monde à nous tromper. Ce serait une défiance injuste, aussi importune et aussi cruelle pour nous-mêmes que pour autrui. Mais il y a peu de gens au monde que l’intérêt ne trompe les premiers, en leur faisant considérer plus souvent et plus fortement les raisons qui les flattent que les raisons contraires. Le roi d’Angleterre n’était pas content de Vatteville, et préférait sans doute alors mon amitié à celle des Espagnols ; mais il ne la pouvait préférer à son unique intérêt, qui était d’éviter toute sorte de rumeur et de mouvement dans Londres, au commencement d’un règne encore mal établi, et par conséquent de favoriser et de me conseiller tous les expédients proposés par Vatteville et par l’Espagne pour ne rien décider.\par
D’Estrades sans doute n’était pas mal intentionné : je puis dire, au contraire, qu’il m’a rendu des services très utiles ; et j’avais enfin beaucoup de sujet d’estimer son zèle et sa conduite. Mais son intérêt, à le séparer du mien, n’était pas de se mettre sur les bras dans le cours de son ambassade une affaire aussi importante que celle-là, pleine de difficulté et d’incertitude, au lieu d’en sortir par un tempérament qui semblait devoir ne lui pas préjudicier : nul ambassadeur n’étant obligé de faire trouver son carrosse et ses domestiques à une entrée dont il n’est point averti. Aussi, quand je lui envoyai mes ordres précis pour celle-là, il me répondit, à la vérité, qu’il y serait le plus fort, les colonels écossais qui avaient servi en France lui ayant promis un bon nombre de leurs soldats, mais en même temps il ajoutait que la cabale d’Espagne étant grande et puissante dans Londres, tous les colonels irlandais dans les intérêts de cette nation, le peuple naturellement ennemi et envieux des Français, et Vatteville recevant et répandant pour ces sortes de choses un argent infini, il me laissait à considérer si on pourrait toujours conserver dans les sorties l’avantage qu’on aurait remporté une fois, et si par conséquent il ne serait pas meilleur de le supposer toujours comme entièrement acquis à la France, sans le hasarder jamais.\par
Il faisait bien sans doute, comme ambassadeur, de prévoir et de proposer ces difficultés ; mais je faisais bien, comme roi, de ne les pas craindre. Je fais assez connaître à toute la France si je crois mes ministres fidèles et éclairés ; mais il ne faudrait pas s’étonner quand leur état, leur condition, leur âge, leur inclination, leurs desseins leur auraient fait en ce temps-là un peu plus appréhender la guerre que je ne l’appréhendais, et craindre en particulier de demeurer responsable envers moi et envers le public de tout ce qui en pourrait arriver.\par
Quoi qu’il en soit, il est très certain, mon fils, que si j’eusse trop donné à leurs conseils, je me serais contenté d’une satisfaction beaucoup moindre, et ne vous laisserais que fort imparfait un avantage que vous devez infiniment estimer. Mais pour moi je raisonnais sur les circonstances du temps, sur l’état des choses en France et en Espagne, et au-dedans et au-dehors, qui me permettait de tout espérer. J’écoutais mon propre cœur, qui ne pouvait consentir à tout ce qui laissait mon droit et le vôtre en quelque sorte de doute. J’agissais enfin sur un principe général que je vous prie de bien remarquer : c’est, mon fils, qu’en ces sortes de rencontres fâcheuses, comme il n’est pas possible qu’il n’en arrive dans la vie des rois, ce n’est point assez de réparer le mal, si on n’ajoute quelque bien qu’on n’avait pas. Quand la blessure n’est que guérie et fermée, la marque ne laisse pas d’y demeurer. Peu de gens vous refuseront des paroles, quand ils vous auront offensé par des effets. Mais s’il ne leur en coûte rien de nouveau pour ce qu’ils ont entrepris, qui vous répond qu’ils ne l’entreprendront point encore ? On n’est pas trop rebuté de frapper un second coup, quand on a seulement manqué le premier. Il fallait, pour ne point reculer aux yeux de toute l’Europe, que je fisse un pas en avant comme je l’ai fait, tirant une nouvelle utilité de cette disgrâce. C’était un malheur que ce tumulte de Londres ; ce serait maintenant un malheur qu’il ne fût pas arrivé.\par
La seconde réflexion que vous devez faire ici, c’est qu’en ces accidents qui nous piquent vivement et jusqu’au fond du cœur, il faut garder un milieu entre la sagesse timide et le ressentiment emporté, tâchant, pour ainsi dire, d’imaginer pour nous-même ce que nous conseillerions à un autre en pareil cas. Car, quelque effort que nous fassions pour parvenir à ce point de tranquillité, notre propre passion, qui nous presse et nous sollicite au contraire, gagne toujours assez sur nous pour nous empêcher de raisonner avec trop de froideur et d’indifférence. J’ai remarqué en cette occasion, comme en mille autres, que les règles de la justice et de l’honneur conduisent presque toujours à l’utilité même. La guerre, quand elle est nécessaire, est une justice non seulement permise, mais commandée aux rois : c’est une injustice, au contraire, quand on s’en peut passer et obtenir la même chose par des voies plus douces. Je la regardai de cette sorte, et c’est ce qui me fit réussir. Si je n’eusse pas été intérieurement disposé à l’entreprendre au besoin pour l’honneur de ma couronne, la négociation ne m’aurait assurément point produit cet effet. Si j’eusse fermé la porte à toute négociation, portant d’abord les choses aux dernières extrémités, je ne sais quelles batailles et quelles victoires m’auraient acquis un pareil avantage, sans compter tant de sang à répandre, le sort des armes toujours douteux, et l’interruption de tous mes desseins pour le dedans du royaume.\par
Et de cette réflexion, mon fils, je passe à une plus générale, mais qui me paraît très nécessaire pour vous et pour moi ; je tâche et je tâcherai toujours dans ces Mémoires à élever, mais non pas à enfler votre courage. S’il y a une fierté légitime en notre rang, il y a une modestie et une humilité qui ne sont pas moins louables. Ne pensez pas, mon fils, que ces vertus ne soient pas faites pour nous. Au contraire, elles nous appartiennent plus proprement qu’au reste des hommes. Car, après tout, ceux qui n’ont rien d’éminent, ni par la fortune, ni par le mérite, quelque petite opinion qu’ils aient d’eux-mêmes, ne peuvent jamais être modestes ni humbles ; et ces qualités supposent nécessairement en celui qui les possède et quelque élévation et quelque grandeur dont il pourrait tirer de la vanité. Nous, mon fils, à qui toutes choses semblent inspirer ce défaut si naturel aux hommes, nous ne pouvons trop apporter de soin à nous en défendre. Mais si je puis vous expliquer ma pensée, il me semble que nous devons être en même temps humbles pour nous-mêmes, et fiers pour la place que nous occupons.\par
J’espère que je vous laisserai encore plus de puissance et plus de grandeur que je n’en ai, et je veux croire ce que je souhaite, c’est-à-dire que vous en ferez encore un meilleur usage que moi. Mais quand tout ce qui vous environnera fera effort pour ne vous remplir que de vous-même, ne vous comparez point, mon fils, à des princes moindres que vous, et à ceux qui ont porté ou qui porteront encore indignement le nom de roi : ce n’est un grand avantage de valoir un peu mieux ; pensez plutôt à tous ceux qu’on a le plus sujet d’estimer et d’admirer dans les siècles passés, qui, d’une fortune particulière ou d’une puissance très médiocre, par la seule force de leur mérite, sont venus à fonder de grands empires, ont passé comme des éclairs d’une partie du monde à l’autre, charmé toute la terre par leurs grandes qualités, et laissé depuis tant de siècles une longue et éternelle mémoire d’eux-mêmes, qui semble, au lieu de se détruire, s’augmenter et se fortifier tous les jours par le temps.\par
Si cela ne suffit pas, rendez-vous encore une justice plus exacte, et considérez de combien de choses on vous louera, que la fortune seule aura peut-être faites pour vous, et que vous devrez entièrement à ceux qu’elle aura mis elle-même dans votre service. Descendez avec quelque sévérité à la considération de vos propres faiblesses : car, bien que vous puissiez en imaginer de semblables en tous les hommes et même dans les plus grands, néanmoins, comme vous les imaginerez et les croirez seulement en eux avec quelque incertitude, au lieu que vous les sentirez véritablement et certainement en vous, elles diminueront sans doute la trop grande opinion que vous pourriez avoir de vous-même, qui est d’ordinaire l’écueil d’un mérite éclatant et connu.\par
Par là, mon fils, et en cela, vous serez humble. Mais quand il s’agira, comme dans l’occasion dont je viens de vous parler, du rang que vous tenez dans le monde, des droits de votre couronne, du Roi enfin et non pas du particulier, prenez hardiment l’élévation de cœur et d’esprit dont vous serez capable, ne trahissez point la gloire de vos prédécesseurs ni l’intérêt de vos successeurs à venir, dont vous n’êtes que le dépositaire. Car alors votre humilité deviendrait une bassesse, et c’est ce que j’avais eu à répondre moi-même aux partisans de l’Espagne, qui, étant préoccupés en sa faveur, murmuraient alors, quoique en secret, comme si j’avais usé avec un peu trop d’éclat de l’avantage que j’avais sur elle.\par
Je puis ajouter, comme une suite de cette affaire, un autre artifice des Espagnols que je découvris alors, et auquel je m’opposai. Avec le même dessein de venir à cette égalité prétendue, ils avaient gagné ceux qui dressaient les pouvoirs des ambassadeurs de Venise, qui, toutes les fois qu’on y parlait de la France et de l’Espagne, les joignaient ensemble par les mots {\itshape delle due corone.} Je m’en plaignis et fis cesser cette nouveauté.\par
J’obligeai encore le Roi Catholique à me faire justice sur un autre point, c’est-à-dire à ôter de ses titres la qualité de comte de Roussillon qu’il se donnait toujours, quoique ce pays me fût acquis par le droit des armes, et cédé par le traité des Pyrénées, sans compter le droit ancien que la France avait de le retirer des mains des Espagnols, qui n’ont jamais exécuté les conditions sous lesquelles il leur avait été donné.\par
Les Polonais et les Moscovites, environ ce temps-là, me prirent pour médiateur dans leurs différends. Les ducs de Savoie et de Modène remirent les leurs à mon jugement.\par
Vous naquîtes, mon fils, le premier du mois de novembre. Comme toutes ces choses glorieuses à mon État et à ma propre personne venaient d’être faites ou paraissaient fort avancées, j’en tirai un secret augure que le Ciel ne vous destinait pas à abaisser votre patrie. La joie de mes sujets, qui fut très grande pour votre naissance, me fit voir d’un côté combien ils sont naturellement affectionnés à leurs princes, et, de l’autre, tout ce qu’ils se promettaient un jour de vous, dont ils vous feraient, mon fils, un reproche éternel si vous ne remplissiez leur attente.\par
Je donnai ensuite divers ordres pour le dedans du royaume, sur lesquels je ne m’arrêterai pas, les ayant déjà touchés en partie quand je vous ai parlé des réformations que j’y avais entreprises. Je licenciai les mortes-payes, qui n’étaient qu’une dépense inutile ; je commençai à régler, par un édit, l’âge et la conduite des officiers de justice : à quoi néanmoins j’ai beaucoup ajouté depuis, comme vous le verrez en son lieu. Je continuai à faire entrer des troupes dans mes places, pour modérer l’excessive autorité des gouverneurs. Je mis la dernière main à cet utile règlement des duels, dont l’effet a été si grand et si prompt qu’il a presque exterminé un mal contre lequel mes prédécesseurs, avec d’aussi bonnes intentions que moi, avaient inutilement employé toutes sortes de remèdes.\par
J’achevai cette année, et commençai la suivante, par la promotion de huit prélats et soixante-trois chevaliers de l’ordre du Saint-Esprit : les places n’en avaient pas été remplies depuis l’année 1633, c’est ce qui en faisait le grand nombre ; mais j’aurais souhaité de pouvoir encore élever plus de gens à cet honneur, ne trouvant pas de joie plus pure pour un prince que celle d’obliger sensiblement plusieurs personnes de qualité dont il est satisfait, sans charger pas un de ses moindres sujets. Nulle récompense ne coûte moins à nos peuples, et nulle ne touche plus les cœurs bien faits que ces distinctions de rang, qui sont presque le premier motif de toutes les actions humaines, mais surtout des plus nobles et des plus grandes ; c’est d’ailleurs un des plus visibles effets de notre puissance, que de donner quand il nous plaît un prix infini à ce qui de soi-même n’est rien. Vous avez appris, mon fils, quel usage les Romains, et particulièrement Auguste, le plus sage de leurs empereurs, savaient faire de ces marques purement honorables, qui étaient bien plus fréquentes en leurs siècles que parmi nous. D’excellents hommes ont blâmé les derniers temps de n’en avoir pas assez : il est à propos, non seulement d’user de celles que nos pères ont introduites, quand nous le pouvons, mais même d’en inventer quelquefois de nouvelles, pourvu que ce soit avec jugement, avec choix, avec dignité, comme vous verrez ailleurs que j’ai tâché de vous en montrer l’exemple.
\section[{Année 1662}]{Année 1662}\renewcommand{\leftmark}{Année 1662}

\subsection[{Première section}]{Première section}
\noindent Je commençai l’année 1662 avec un ferme dessein, non seulement de continuer ce que j’avais entrepris pour le bien de mes peuples, mais encore d’y ajouter chaque jour ce que l’expérience me découvrait d’avantageux et d’utile. En travaillant au rétablissement des finances, je m’étais déjà assujetti, comme je vous l’ai dit, à signer moi-même toutes les ordonnances qui s’expédiaient pour les moindres dépenses de l’État. Je trouvai que ce n’était pas assez, et je voulus bien me donner la peine de marquer de ma propre main, sur un petit livre que je pusse voir à tous moments, d’un côté les fonds qui devaient me revenir chaque mois, de l’autre toutes les sommes payées par mes ordonnances dans ce mois-là, prenant pour ce travail toujours l’un des premiers jours du mois suivant, afin d’en avoir la mémoire plus présente.\par
Il se pourra faire, mon fils, que dans le grand nombre de courtisans dont vous serez environné, quelques-uns, attachés à leurs plaisirs et faisant gloire d’ignorer leurs propres affaires, vous représenteront quelque jour ce soin comme fort au-dessous de la royauté. Ils vous diront peut-être que les rois nos prédécesseurs n’en ont jamais usé de la sorte, non pas même leurs Premiers ministres qui auraient cru s’abaisser, s’ils ne se fussent pas reposés de ce détail sur le surintendant et celui-là encore sur le trésorier de l’Épargne, ou sur quelque commis inférieur et obscur. Mais ceux qui parlent ainsi n’ont jamais considéré que, dans le monde, les plus grandes affaires ne se font presque jamais que par les plus petites, et que ce qui serait bassesse en un prince, s’il agissait par un simple amour de l’argent, devient élévation et hauteur quand il a pour dernier objet l’utilité de ses sujets, l’exécution d’une infinité de grands desseins, sa propre splendeur et sa propre magnificence, dont ce soin et ce détail sont le plus assuré fondement. Que s’ils veulent vous avouer la vérité, et reconnaître combien de fois ils prennent de fausses mesures, ou sont contraints de rompre celles qu’ils avaient prises, parce qu’il plaît ainsi à leur intendant, qui seul est le maître de ce qu’ils peuvent ou ne peuvent pas faire, quelles contradictions et quels chagrins ils ont à essuyer là-dessus, vous jugerez aisément qu’ils auraient sans comparaison moins de peine à savoir leurs affaires qu’à ne les savoir pas.\par
Imaginez-vous, mon fils, que c’est encore tout autre chose pour un roi, dont les projets doivent être plus divers, plus étendus, et plus cachés que ceux de pas un particulier, de telle nature, enfin, qu’à peine se trouve-t-il quelquefois une seule personne au monde à qui il puisse les confier tous ensemble et tout entiers. Il n’y a cependant nul de ces projets où les finances n’entrent de quelque côté. Ce n’est pas assez dire : il n’y a pas un de ces projets qui n’en dépende absolument et essentiellement, car ce qui est grand et beau quand nous le pouvons par l’état où se trouvent nos finances, devient chimérique et ridicule quand nous ne le pouvons pas. Songez donc, je vous prie, comment un roi pourra gouverner et n’être pas gouverné, dont les meilleures pensées et les plus nobles, à cause qu’il ignore ce détail de ses finances, seront soumises au caprice du Premier ministre, ou du surintendant, ou du trésorier de l’Épargne, ou de ce commis obscur et inconnu, qu’il sera obligé de consulter comme autant d’oracles, en telle sorte qu’il ne puisse rien entreprendre sans s’expliquer à eux, qu’avec leur permission, et sous leur bon plaisir.\par
Mais on peut trouver, vous dira-t-on, des gens fidèles et sages qui, sans pénétrer dans vos desseins, ne vous tromperont point sur ce détail des finances toutes les fois que vous voudrez le savoir. Je veux, mon fils, que ces qualités soient aussi communes qu’elles sont rares. Ce n’est rien dire encore : s’ils ont seulement le cœur fait autrement que nous, et ils l’ont toujours ainsi, si leurs vues et leurs inclinations sont différentes des nôtres, ce qui ne manque jamais d’arriver, ils nous tromperont par affection. Ce sera alors, pour le bien de l’État entendu à leur fantaisie, qu’ils s’opposeront secrètement à nos volontés, et nous mettront dans l’impossibilité de rien faire, leurs bonnes intentions produisant le même effet que leur infidélité.\par
D’ailleurs, mon fils, ne vous y trompez jamais, nous n’avons pas affaire à des anges, mais à des hommes à qui le pouvoir excessif donne presque toujours à la fin quelque tentation d’en user. Dans les affaires du monde, la discussion du détail et le véritable crédit ont une liaison nécessaire et inévitable, et ne se séparent jamais. Nul ne partage votre travail, sans avoir un peu de part à votre puissance. N’en laissez à autrui que ce qu’il vous sera impossible de retenir : car quelque soin que vous puissiez prendre, il vous en échappera toujours beaucoup plus qu’il ne serait à souhaiter.\par
Il survint bientôt après une occasion en elle-même fâcheuse, mais utile par l’événement, qui fit assez remarquer à mes peuples combien j’étais capable de ce même soin du détail pour ce qui ne regardait que leurs intérêts et leurs avantages. La stérilité de 1661, quoique grande, ne se fit proprement sentir qu’au commencement de l’année 1662, lorsqu’on eut consumé, pour la plus grande partie, les blés des précédentes : mais alors elle affligea tout le royaume au milieu de ces premières prospérités, comme si Dieu qui prend soin de tempérer les biens et les maux eût voulu balancer les grandes et heureuses espérances de l’avenir par une infortune présente. Ceux qui en pareil cas ont accoutumé de profiter de la calamité publique ne manquèrent pas de fermer leurs magasins, se promettant dans les suites une plus grande cherté, et par conséquent un gain plus considérable.\par
On peut s’imaginer cependant, mon fils, quels effets produisaient dans le royaume les marchés vides de toutes sortes de grains, les laboureurs contraints de quitter le travail des terres pour aller chercher ailleurs la subsistance dont ils étaient pressés, ce qui faisait même appréhender que le malheur de cette année ne passât aux suivantes ; les artisans qui enchérissaient leurs ouvrages à proportion de ce qu’il leur fallait pour vivre ; les pauvres faisant entendre partout leurs plaintes et leurs murmures ; les familles médiocres qui retenaient leurs charités ordinaires par la crainte d’un besoin prochain ; les plus opulents chargés de leurs domestiques, et ne pouvant suffire à tout ; tous les ordres de l’État enfin menacés des grandes maladies que la mauvaise nourriture mène après elle, et qui, commençant par le peuple, s’étendent ensuite aux personnes de la plus haute qualité : tout cela ensemble causait par toute la France une désolation qu’il est difficile d’exprimer.\par
Elle eût été sans comparaison plus grande, mon fils, si je me fusse contenté de m’en affliger inutilement, ou si je me fusse reposé des remèdes qu’on y pouvait apporter, sur les magistrats ordinaires qui ne se rencontrent que trop souvent faibles et malhabiles, ou peu zélés, ou même corrompus. J’entrai moi-même en une connaissance très particulière et très exacte du besoin des peuples et de l’état des choses. J’obligeai les provinces les plus abondantes à secourir les autres, les particuliers à ouvrir leurs magasins, et à exposer leurs denrées à un prix équitable. J’envoyai en diligence mes ordres de tous côtés, pour faire venir par mer, de Dantzig et des autres pays étrangers, le plus de blés qu’il me fut possible ; je les fis acheter de mon épargne ; j’en distribuai gratuitement la plus grande partie au petit peuple des meilleures villes, comme Paris, Rouen, Tours et autres ; je fis vendre le reste à ceux qui en pouvaient acheter ; mais j’y mis un prix très modique, et dont le profit, s’il y en avait, était employé aussitôt au soulagement des pauvres, qui tiraient des plus riches, par ce moyen, un secours volontaire, naturel et insensible. À la campagne, où les distributions de blés n’auraient pu se faire si promptement, je les fis en argent, dont chacun tâchait ensuite de soulager sa nécessité. Je parus enfin à tous mes sujets comme un véritable père de famille qui fait la provision de sa maison, et partage avec équité les aliments à ses enfants et à ses domestiques.\par
Je n’ai jamais trouvé de dépense mieux employée que celle-là. Car nos sujets, mon fils, sont nos véritables richesses et les seules que nous conservons proprement pour les conserver, toutes les autres n’étant bonnes à rien, que quand nous savons l’art d’en user, c’est-à-dire de nous en défaire à propos. Que si Dieu me fait la grâce d’exécuter tout ce que j’ai dans l’esprit, je tâcherai de porter la félicité de mon règne jusqu’à faire en sorte, non pas à la vérité qu’il n’y ait plus ni pauvre ni riche, car la fortune, l’industrie et l’esprit laisseront éternellement cette distinction entre les hommes, mais au moins qu’on ne voie plus dans tout le royaume ni indigence, ni mendicité, je veux dire personne, quelque misérable qu’elle puisse être, qui ne soit assurée de sa subsistance, ou par son travail ou par un secours ordinaire et réglé.\par
Mais sans aller plus avant, je reçus à l’instant même une grande et ample récompense de mes soins, par le redoublement d’affection qu’ils produisirent pour moi dans l’esprit des peuples. Et c’est de cette sorte, mon fils, que nous pouvons quelquefois changer heureusement en biens les plus grands maux de l’État. Car si quelque chose peut resserrer le nœud sacré qui attache les sujets à leur souverain, et réveiller dans leur cœur les sentiments de respect, de reconnaissance et d’amour qu’ils ont naturellement pour lui, c’est sans doute le secours qu’ils en reçoivent dans quelque malheur public et non attendu. À peine remarquons-nous l’ordre admirable du monde, et le cours si réglé et si utile du soleil, jusqu’à ce que quelque dérèglement des saisons, ou quelque désordre apparent dans la machine, nous y fasse faire un peu plus de réflexions. Tant que tout prospère dans un État, on peut oublier les biens infinis que produit la royauté, et envier seulement ceux qu’elle possède : l’homme naturellement ambitieux et orgueilleux ne trouve jamais en lui-même pourquoi un autre lui doit commander, jusqu’à ce que son besoin propre le lui fasse sentir. Mais ce besoin même, aussitôt qu’il a un remède constant et réglé, la coutume le lui rend insensible. Ce sont les accidents extraordinaires qui lui font considérer ce qu’il en retire ordinairement d’utilité, et que, sans le commandement, il serait lui-même la proie du plus fort, il ne trouverait dans le monde ni justice, ni raison, ni assurance pour ce qu’il possède, ni ressource pour ce qu’il avait perdu : et c’est par là qu’il vient à aimer l’obéissance, autant qu’il aime sa propre vie et sa propre tranquillité.\par
J’eus encore presque en même temps diverses occasions de témoigner mon affection à mes peuples. La Chambre de justice, ayant reconnu qu’il s’était aliéné un million de rentes sur les tailles dont je n’avais point touché le prix, en ordonna la suppression à mon profit ; mais je commandai aussitôt que le fonds qui s’en devait lever fût diminué sur le brevet de la taille, sans en tirer nul avantage que celui de mes sujets.\par
La même raison m’empêcha de considérer, en une autre chose de cette nature, l’intérêt des rentiers contre celui de toute la France. Le droit commun permet à chaque particulier de racheter les rentes constituées ; en rendant le véritable prix qu’il en a reçu, et imputant sur ce prix principal ce qu’il a payé d’arrérages au-delà de l’intérêt légitime. La Chambre de justice jugea que je ne devais être de pire considération pour les rentes constituées en mon nom sur l’Hôtel de Ville de Paris. Les particuliers qui les avaient acquises à vil prix et en avaient joui longtemps ne trouvèrent pas leur compte à cette imputation, par où leur remboursement était réduit à peu de chose ; mais je ne crus pas devoir perdre une occasion si juste et si favorable d’acquitter facilement mes peuples plutôt que moi de quatre millions de rente annuelle qu’il eût fallu lever sur eux.\par
L’excès des impositions, durant la guerre et ma minorité, avait réduit presque toutes les communautés et toutes les villes de mon royaume à emprunter de grandes sommes, premièrement en engageant les droits d’octroi, leurs deniers et autres revenus publics, puis sur le crédit des principaux habitants qui s’obligeaient solidairement pour les autres. Les intérêts qui s’accumulaient incessamment les mettaient presque hors d’état d’y pouvoir jamais satisfaire de leur propre fonds. Les plus riches, poursuivis vivement pour ces dettes communes, devenaient plus misérables que les autres, forcés d’abandonner leurs héritages, la culture des terres et le commerce des choses les plus nécessaires à la vie, par les saisies continuelles que l’on faisait sur eux, et par la crainte de la prison. Le comble du mal était que les consuls et autres administrateurs se servaient du prétexte de ces dettes pour dissiper les deniers publics. Je délivrai les communautés de cette misère, en nommant des commissaires pour liquider leurs dettes, et pour en régler le payement suivant que l’état des choses pourrait le permettre, et ordonnant qu’il serait fait par mes propres receveurs.\par
Il me fut aisé de voir aussi que mes peuples répondaient à mon affection et dans les provinces les plus éloignées comme dans les plus proches. La taille qui, jusque-là, était à peine levée en deux ou trois ans, se leva dès lors en quatorze ou quinze mois, en partie à la vérité parce que les charges étant moindres devenaient plus aisées à porter, mais aussi principalement par la bonne volonté de ceux qui les portaient, qui, se voyant soulagés, faisaient, gaiement et sans chagrin, tout ce qu’ils pouvaient faire. Les pays d’États qui, en matière d’impositions, s’étaient autrefois estimés comme indépendants, commencèrent à ne plus se servir de leur liberté que pour me rendre leur soumission plus agréable. Déjà les États de Bretagne, l’année précédente, 1661, avaient accordé à mes commissaires, sans délibérer et sur le théâtre même, tout ce qui leur avait été demandé de ma part, prêts à aller plus loin pour peu que j’eusse témoigné le souhaiter. Mais j’étais à Nantes, et on pouvait croire que ma présence seule avait produit cet effet. Les États de Languedoc, qui se tenaient à deux cents lieues de moi au commencement de cette année 1662, suivirent un changement si avantageux pour moi, en m’accordant sans difficultés, comme ils faisaient auparavant, et sans en rien retrancher, la somme demandée.\par
L’usage avait été jusqu’alors non seulement de leur demander de grandes sommes pour en obtenir de médiocres, mais aussi de souffrir qu’ils missent tout en condition, de leur tout promettre, d’éluder bientôt après sous différents prétextes tout ce qu’on leur avait promis, de faire même un grand nombre d’édits sans autre dessein que de leur en accorder, ou plutôt de leur en vendre la révocation bientôt après. Je trouvai en cette méthode peu de dignité pour le souverain et peu d’agrément pour les sujets. J’en pris une toute contraire que j’ai toujours suivie depuis, qui fut de leur demander précisément ce que j’avais dessein d’obtenir, de promettre peu, de tenir exactement ce que j’avais promis, de ne recevoir presque jamais de condition, mais de passer leur attente quand, par la voie des supplications, ils se confiaient à ma justice et à ma bonté.\par
Je fis, cette année, par deux divers traités, deux acquisitions très considérables, celle de la Lorraine et celle de Dunkerque. Je les joins ici ensemble, mon fils, pour votre instruction, comme deux objets de même nature, quoique les traités aient été conclus et signés à quelques mois l’un de l’autre.\par
La situation de la Lorraine ne me permettait pas de douter qu’il ne me fût très avantageux d’en être le maître, et me le faisait souhaiter. C’était un passage à mes troupes pour l’Allemagne, pour l’Alsace, et pour quelque autre pays qui m’appartenait déjà, une porte jusqu’alors ouverte aux étrangers pour entrer dans nos États. C’était le siège d’une puissance voisine peu capable, à la vérité, d’inquiéter par elle-même un roi de France, mais prenant part de tout temps à toutes les brouilleries du royaume ; toujours prête à se lier avec les mécontents, et à les lier avec d’autres princes plus éloignés ; et s’il fallait ajouter l’honneur à l’utilité, c’était l’ancien patrimoine de nos pères, qu’il était beau de rejoindre au corps de la monarchie dont il avait été si longtemps séparé.\par
Il m’était aisé d’acquérir ce pays par les armes ; et la conduite du duc, toujours inquiet et inconstant, et ne tenant aucun compte de traités ni de promesses, ne m’en fournissait pas seulement des prétextes honnêtes, mais même d’assez légitimes sujets. Mais au fond c’était interrompre la paix de l’Europe : ce que je ne voulais pas faire alors sans une absolue nécessité. Le traité des Pyrénées donnait lieu aux autres potentats de s’intéresser dans cette querelle ; et la présomption qui est toujours contre le plus fort, pour peu que mon procédé eût été douteux, m’aurait fait accuser d’injustice et de violence.\par
D’un autre côté, on avait peine à comprendre qu’il fût possible d’en venir à bout par négociation et par traité. Comme il est vrai qu’on persuade difficilement à un prince libre et maître de ses actions une affaire telle que celle-là, son consentement même semble ne pas suffire sans celui des autres intéressés, c’est-à-dire de tous ceux qui ont droit à succession. À moins enfin que le traité ne soit bien solennel et bien authentique, à moins qu’il n’ait un grand fondement d’équité, on pourrait douter encore si les successeurs des successeurs, quoiqu’ils soient alors à naître, n’ont point droit de réclamer quelque jour contre le préjudice qu’on leur a fait. J’avais donc toutes ces difficultés à considérer, qui faisaient croire à une partie de mes ministres, qu’il n’y avait rien à espérer de ce dessein.\par
Mais il y a grande différence, mon fils, entre les lumières générales sur les choses et la connaissance particulière des temps, des circonstances, des personnes et intérêts. Je connaissais le duc de Lorraine pour un prince à qui son inquiétude naturelle rendait toutes les nouveautés agréables, fort attaché à l’argent, sans nuls enfants légitimes, avide d’amasser des trésors, et soigneux de les cacher en divers lieux de l’Europe, soit par la confiance qu’il y prenait lui-même dans ses diverses fortunes, soit pour enrichir quelque jour ses enfants naturels qu’il aimait. Il était maître de nom plutôt que d’effet d’un pays désolé par la guerre, où il ne tenait aucune place de considération, et par là même plus disposé à céder ce qu’il aurait en tout temps beaucoup de peine à défendre.\par
Quant à ceux de son sang et de sa maison, je savais la passion qu’ils avaient d’être tenus pour nos parents du côté de Charlemagne ; qu’en leur donnant quelque prérogative qui pût flatter cette prétention, on en obtiendrait toute chose ; qu’au fond leur maison était assez illustre pour être considérée, après la nôtre, au-dessus de toutes les autres dans l’État, surtout si l’État en pouvait recevoir dès lors même quelque grand et insigne avantage, comme de leur côté ils en recevaient un très grand et très glorieux par une pareille distinction.\par
Il manquait une occasion bien naturelle pour proposer ce que j’avais dans l’esprit, et elle se présenta d’elle-même plus favorable que je n’eusse osé l’espérer. Le prince Charles, neveu du duc, et le plus intéressé dans cette affaire comme son héritier présomptif, peu content de lui, et tenant pour suspecte l’affection qu’il témoignait à ses enfants naturels, voulait alors s’engager à un mariage avec Mademoiselle de Nemours, maintenant duchesse de Savoie, principalement par l’espérance de ma protection, et qu’après la mort de son oncle je le maintiendrais envers et contre tous dans les États qui lui devaient revenir.\par
Le duc, irrité et jaloux de la liaison que ce jeune prince tâchait de prendre avec moi, laissa échapper, dans son dépit, quelques paroles qui pouvaient être expliquées suivant mon dessein, et qu’on me rapporta. Je travaillai sur l’heure même à en profiter, de peur que, son chagrin passé, il ne changeât de pensée : ce qui lui était ordinaire même en des choses bien moins importantes. Lionne, que je chargeai de la négociation, me rendait compte, de temps à autre, de ce qui s’y passait ; je poussai l’affaire si vivement, qu’elle fut entièrement résolue bientôt après.\par
Le duc, par un traité que nous signâmes le 6 février, me fit une cession de tous ses États, à la réserve de l’usufruit durant sa vie, que je lui faisais valoir pour le revenu jusqu’à sept cent mille livres, sans rien augmenter aux impositions. Je lui donnais de plus cent mille écus de rente, qu’il pouvait faire passer au comte de Vaudémont, son fils naturel, ou à telle autre personne qu’il lui plairait : savoir, cent mille livres sur une de mes fermes et deux cent mille en terres, dont il y en avait une portant titre de duché et pairie. Je me chargeais de toutes les dettes du duc, ou de ses prédécesseurs, auxquelles ces trois cent mille livres de rente pourraient être hypothéquées, moyennant l’hôtel de Lorraine qu’il m’abandonnait en propriété. Je donnais enfin à ceux de la maison de Lorraine le privilège de princes après les derniers princes de mon sang, avec tous les droits que ce rang leur pourrait acquérir à l’avenir, plus éloignés sans doute, mais aussi, sans comparaison, plus grands que ceux dont ils se départaient pour eux et pour les leurs en consentant à ce traité. Mais j’étais d’accord avec le duc que pas un d’eux ne s’en pourrait prévaloir qu’ils ne l’eussent tous signé, et que cette condition serait ajoutée, comme elle le fut dans l’enregistrement au Parlement, où je le portai moi-même le 27 du même mois de février.\par
Mais, à dire la vérité, je ne hasardais rien pour une affaire dont je pouvais espérer de grandes suites. Le duc était du moins lié en son particulier par ce traité, obligé par là à vivre avec moi dans une plus grande dépendance : ce qui était toujours beaucoup. Ceux de sa maison, s’ils consentaient tous, établissaient tellement mon droit pour l’avenir, que la plus grande rigueur des lois ordinaires n’y pouvait rien trouver à redire : car ils quittaient seulement des droits incertains pour d’autres droits, mais sans comparaison plus grands et si illustres qu’ils s’en devaient tenir éternellement honorés. Quelqu’un de ceux qui, pour être plus proches de la succession, la regardaient comme présente, pouvait bien refuser de signer, mais en ce cas-là je n’étais engagé à rien pour les autres que je mettais par là même dans mes intérêts. Il me restait seulement un traité personnel avec le duc, qui subsistant à mon profit me donnait lieu de gagner peu à peu par d’autres avantages, généralement, tous les intéressés, en mille conjonctures que le temps pouvait produire.\par
Ce traité fut rendu public et enregistré au Parlement, avec la condition que j’ai dite, par le consentement de tous ceux de la maison de Lorraine, excepté du prince Charles, qui se retira de ma cour aussitôt qu’il vit la chose arrêtée, et me donna lieu de suspendre à tous les autres la jouissance des privilèges de prince du sang. Il est encore incertain, quand j’écris ces Mémoires, quels seront un jour à cet égard les avantages de ce traité pour moi, mais vous avez vu du moins qu’il ne me pouvait être nuisible.\par
L’acquisition de Dunkerque n’était pas de si grande étendue, mais elle était d’une importance non moindre et d’une utilité plus certaine. Peu de personnes ont su par quelle suite d’affaires cette place si considérable était passée entre les mains des Anglais, durant le ministère du cardinal Mazarin. Il faut pour cela remonter jusqu’à ma minorité et aux factions qui obligèrent deux fois ce ministre à sortir du royaume. Cromwell, à qui le génie, les occasions et le malheur de son pays avaient inspiré des pensées fort au-dessus de sa naissance, au commencement simple officier dans les troupes rebelles du Parlement, puis général, puis Protecteur de la République, et désirant en secret la qualité de roi qu’il refusait en public, enflé par le bon succès de la plupart de ses entreprises, ne voyait rien de si grand, ni au-dedans, ni au-dehors de son île, à quoi il ne pensât pouvoir prétendre ; et bien qu’il ne manquât pas d’affaires chez lui, il regarda les troubles de mon État comme un moyen de mettre le pied en France par quelque grand établissement : ce qui lui était également avantageux, soit que la puissance royale se confirmât en sa personne et en sa famille, soit que le caprice des peuples et la même fortune qui l’avaient élevé si haut entreprissent de le renverser.\par
Il savait de quelle sorte presque tous les gouverneurs des places et des provinces traitaient alors avec le cardinal Mazarin, et qu’à peine y avait-il de fidélité parmi ses sujets, qu’achetée à prix d’argent ou par des récompenses d’honneur, telles que chacun s’avisait de les souhaiter. Il dépêche le colonel de ses gardes au comte d’Estrades, gouverneur de Dunkerque, il l’exhorte à considérer l’état des choses pour en tirer ses avantages particuliers, lui offre jusqu’à deux millions payables à Amsterdam ou à Venise, s’il veut lui livrer la place, et de ne faire jamais de paix avec la France sans obtenir pour lui les dignités et les établissements où il peut aspirer. Il ajoute que les affaires du cardinal, son bienfaiteur, et qui l’avait mis dans ce poste, sont désespérées, n’y ayant pas d’apparence que ce ministre, dont on avait mis la tête à prix, puisse par ses propres forces revenir ni dans le ministère, ni dans l’État ; qu’il ne le soutiendra pas seul avec Dunkerque, mais périra avec lui. Si toutefois il veut porter son affection et sa reconnaissance pour lui jusqu’au bout, qu’il prenne cette occasion de le servir utilement par la seule voie peut-être que sa bonne fortune lui ait laissée de reste ; qu’il peut offrir au cardinal, avec la même condition de remettre Dunkerque aux Anglais, non seulement les deux millions, mais aussi tels secours de troupes qui lui seront nécessaires pour rentrer en France ; qu’il se fera par là, auprès de lui, un mérite après lequel, si ce ministre est rétabli, il n’y a rien qu’il n’en doive espérer.\par
D’Estrades, par une conduite très louable, après avoir obligé cet envoyé à lui faire ces propositions dans un conseil de guerre, et ensuite à les signer, le renvoie à Cromwell avec sa réponse : il se plaint qu’on l’ait cru capable d’une infidélité, ni de rendre cette place par d’autres ordres que les miens ; que tout ce qu’il peut est de me proposer à moi-même la condition des deux millions, et en même temps celle d’une étroite alliance avec moi par laquelle le Protecteur s’engagera à rompre sur mer et sur terre avec les Espagnols ; à me fournir dix mille hommes de pied et deux mille chevaux pour leur faire la guerre en Flandre ; à entretenir cinquante navires de guerre sur les côtes, durant les six mois de l’été, et une escadre de quinze durant l’hiver, pour croiser la mer, agissant de concert suivant les desseins qu’on pourrait former ensemble.\par
Cromwell accepta ces propositions qui me furent aussitôt envoyées par d’Estrades à Poitiers où j’étais, et n’arrivèrent que deux jours après le retour du cardinal Mazarin. Ce ministre les trouva très avantageuses, ayant pour maxime de pourvoir, à quelque prix que ce fût, aux affaires présentes, et persuadé que les maux à venir trouvaient leur remède dans l’avenir même. Mais le garde des sceaux Châteauneuf, qu’on avait été obligé de rappeler durant ces troubles, l’emporta contre lui dans le conseil et auprès de la Reine ma mère, et les fit absolument rejeter. Cromwell, ayant reçu cette réponse, signa le même jour un traité avec les Espagnols, leur fournit dix mille hommes et vingt-cinq vaisseaux pour le siège de Gravelines et de Dunkerque, qui par ce moyen furent prises sur moi en la même année, l’une à la fin de mai, l’autre au 22 septembre, mais au profit des Espagnols seulement.\par
Cependant mon autorité s’étant affermie dans le royaume et les factions qu’ils y fomentaient étant absolument dissipées, ils furent réduits quelque temps après à ne pouvoir soutenir que difficilement l’effort de mes armes en Flandre. Cromwell, qui ne s’était lié avec eux que pour cette entreprise particulière, et qui avait toujours augmenté depuis en pouvoir et en considération dans toute l’Europe, se voyait également recherché de leur côté et du mien ; ils le regardèrent comme l’unique ressource à leurs affaires de Flandre, et moi comme l’unique obstacle à leurs progrès, en un temps où je voyais la conquête entière de ces provinces presque certaine, si on ne m’accordait tout ce que je pouvais souhaiter pour la paix. Lui, qui n’avait pas oublié son premier dessein de s’acquérir un poste considérable au deçà de la mer, ne voulant se déterminer qu’à cette condition, proposait en même temps aux Espagnols de se joindre à eux dans cette guerre, d’assiéger Calais qui lui demeurerait, ce qu’ils étaient près d’accepter avec joie, et à moi d’assiéger Dunkerque et de le lui remettre.\par
Le cardinal Mazarin, à qui cette ouverture n’était pas nouvelle, et qui l’avait approuvée autrefois lors même que Dunkerque était au pouvoir des Français, s’en trouva sans doute moins éloigné. Et bien que j’y eusse beaucoup de répugnance, je m’y rendis enfin, non seulement par le cas que je faisais de ses conseils, mais aussi par les avantages essentiels que j’y trouvais pour la guerre de Flandre, et par la nécessité de choisir de deux maux le moindre, ne voyant pas de comparaison, puisqu’il fallait nécessairement voir les Anglais en France, entre les y voir mes ennemis ou mes amis, ni entre m’exposer à perdre Calais que j’avais, ou leur promettre Dunkerque que je n’avais pas encore.\par
Ce fut donc par cet accommodement, qu’après avoir repris Dunkerque, je le leur remis entre les mains, et il ne faut point douter que leur union avec moi ne fût comme le dernier coup qui mit l’Espagne hors d’état de se défendre, et qui produisit une paix si glorieuse et si avantageuse pour moi.\par
J’avoue pourtant que cette place au pouvoir des Anglais m’inquiétait beaucoup. Il me semblait que la religion catholique y était intéressée. Je me souvenais qu’ils étaient les anciens et irréconciliables ennemis de la France, dont elle ne s’était sauvée autrefois que par un miracle ; que leur premier établissement en Normandie nous avait coûté cent ans de guerre, et le second en Guyenne trois cents ans, durant lesquels la guerre se faisait toujours au milieu du royaume à nos dépens, de sorte qu’on s’estimait heureux quand on pouvait faire la paix et renvoyer les Anglais chez eux avec de grosses sommes d’argent pour les frais qu’ils avaient faits, ce qu’ils regardaient comme un revenu ou un tribut ordinaire.\par
Je n’ignorais pas que les temps étaient fort changés ; mais, parce qu’ils pouvaient encore changer d’une autre sorte, j’étais blessé de cette seule pensée que mes successeurs les plus éloignés me pussent reprocher quelque jour d’avoir donné lieu à de si grands maux, s’ils pouvaient jamais y retomber ; et sans passer même à ces extrémités, sans aller si loin dans le passé ou dans l’avenir, je savais combien la seule ville de Calais, qui leur était demeurée la dernière, avait coûté de sommes immenses aux Français, par les ravages ordinaires de la garnison, ou par les descentes qu’elle avait facilitées, ce poste, ni pas un autre dans mon royaume, ne pouvant d’ailleurs être à eux sans être en même temps un asile ouvert aux mutins, et sans fournir à cette nation des intelligences dans tout le royaume, surtout parmi ceux qu’un intérêt commun de religion liait naturellement avec elle. Peut-être qu’en donnant Dunkerque je n’avais point trop acheté la paix des Pyrénées et les avantages qu’elle m’apportait. Mais après cela il est certain que je ne pouvais trop donner pour racheter Dunkerque : ce que j’avais bien résolu dès lors, mais qui à la vérité était difficile à espérer.\par
Cependant, comme pour venir à bout des choses le premier pas est de les croire possibles, dès l’année 1661, renvoyant d’Estrades en Angleterre, je le chargeai très expressément d’étudier avec soin tout ce qui pourrait servir à ce dessein, et d’en faire son application principale.\par
Le roi d’Angleterre, nouvellement rétabli, avait un extrême besoin d’argent pour se maintenir. Je savais que par l’état de son revenu et de sa dépense il demeurait toujours en arrière de deux ou trois millions par an, et c’est le défaut essentiel de cette monarchie, que le prince n’y saurait faire de levées extraordinaires sans le Parlement, ni tenir le Parlement assemblé sans diminuer d’autant de son autorité qui en demeure quelquefois accablée, comme l’exemple du roi précédent l’avait assez fait voir.\par
Le chancelier Hyde avait toujours été assez favorable à la France ; il sentait alors diminuer son crédit dans l’esprit du roi, quoiqu’on ne s’en aperçût point encore, et voyait dans l’État une puissante cabale qui lui était opposée : ce qui l’obligeait d’autant plus à se faire des amis et protecteurs au-dehors. Toutes ces raisons ensemble le disposaient à me faire plaisir, quand mes intérêts pourraient s’accorder avec ceux du roi son maître.\par
D’Estrades, exécutant mes ordres, et se servant adroitement de l’accès libre et familier qu’il avait depuis longtemps auprès de ce prince, n’eut pas de peine dans les conversations ordinaires à le faire tomber sur Dunkerque. Le roi, qui disait alors qu’il en voulait faire sa place d’armes, l’entretenait volontiers de ce dessein, comme un homme qui pourrait lui donner des lumières utiles, en ayant été longtemps gouverneur. Pour lui, approuvant tout, il faisait seulement remarquer quelques incommodités dans la situation des lieux, et surtout la grande dépense dont cette place avait besoin nécessairement pour l’entretenir et la garder, jusque là que le cardinal Mazarin qui la connaissait par l’expérience du passé avait douté plusieurs fois s’il eût été avantageux à la France de la conserver quand elle l’aurait pu. Le roi répondait qu’il lui serait fort aisé quand il voudrait de se délivrer de cette dépense, les Espagnols lui offrant alors même de grandes sommes, s’il voulait leur vendre Dunkerque. D’Estrades lui conseillait toujours d’accepter leurs offres, jusqu’à ce que le roi, plus pressé que nous ne pensions, vînt de lui-même à dire que, s’il avait à en traiter, il aimerait mieux que ce fût avec moi qu’avec eux.\par
Ainsi commença cette négociation dont j’eus une extrême joie, et bien que sa demande fût de cinq millions, somme sans doute très considérable, qu’il fallait même payer fort promptement, je ne trouvai pas à propos de le laisser refroidir là-dessus, le bon état où commençaient d’être mes finances me permettant pour une chose aussi importante que celle-là, non seulement ces efforts, mais de plus grands. La conclusion du traité se fit toutefois à quatre millions payables en trois ans, tant pour la place que pour toutes les munitions de guerre, canons, pierres, briques et bois. Je gagnai même sur ce marché cinq cent mille livres, sans que les Anglais s’en aperçussent. Car ne pouvant s’imaginer qu’en l’état où on avait vu mes affaires peu de temps auparavant j’eusse moyen de leur fournir promptement cette grande somme comme ils le désiraient, ils acceptèrent avec joie l’offre que leur fit un banquier de la payer en argent comptant, moyennant cette remise de cinq cent mille livres ; mais le banquier était un homme interposé par moi, qui, faisant le payement de mes propres deniers, ne profitait point de la remise.\par
La conséquence de cette acquisition me donna une inquiétude continuelle, jusqu’à ce que tout fût achevé, et ce n’était pas sans raison ; car l’affaire, au commencement très secrète, ayant été éventée peu à peu, la ville de Londres qui en fut informée députa ses principaux magistrats, le maire et les aldermen, pour offrir au roi toutes les sommes qu’il voudrait, à condition de ne point aliéner Dunkerque. De deux courriers que d’Estrades m’avait dépêchés par deux divers chemins, avec deux copies du traité pour le ratifier, l’un fut arrêté sur le chemin de Calais par les ordres du roi d’Angleterre, l’autre étant déjà passé en France par Dieppe ; et ce roi, à qui d’Estrades représentait en même temps qu’il ne s’agissait plus de Dunkerque, mais de rompre pour jamais avec moi si l’on ne me tenait parole, quelque complaisance qu’il fût obligé d’avoir pour eux, leur fit approuver enfin comme une chose déjà faite et sans remède ce qu’ils avaient résolu d’empêcher.\par
Ces deux affaires, qui d’abord étaient hors de toute apparence, et qui me furent néanmoins si faciles, vous doivent apprendre, mon fils, à ne vous pas rebuter aisément dans les desseins que vous croirez d’ailleurs avantageux à l’État. Ne vous étonnez pas si je vous exhorte si souvent à travailler, à tout voir, à tout écouter, à tout connaître. Je vous l’ai déjà dit, il y a grande différence entre les lumières générales qui ne servent ordinairement qu’à discours, et les particulières qu’il faut presque toujours suivre dans l’action. Les maximes trompent la plupart du temps les esprits vulgaires ; les choses sont rarement comme elles devraient être. La paresse s’arrête aux notions communes, pour n’avoir rien à examiner et rien à faire. L’industrie est à relever les circonstances particulières, pour en profiter ; et on ne fait jamais rien d’extraordinaire, de grand et de beau, qu’en y pensant plus souvent et mieux que les autres.\par
Vous pouvez encore, mon fils, tirer une instruction de ces deux exemples. Ne doutez pas qu’en tout temps, et surtout en ces commencements et dans une plus grande jeunesse, je n’eusse mieux aimé conquérir des États que de les acquérir. Mais qui ne veut que pratiquer une vertu, il ne la connaît point du tout ; car il n’y en a point de véritable qui ne s’accorde avec toutes les autres, puisqu’elles consistent toutes à agir par raison, c’est-à-dire suivant que le temps et les occasions le demandent, même en faisant violence à nos propres inclinations. S’il n’est point beau de se faire un favori, quelque habile qu’il puisse être, pour ne plus écouter que lui, il ne l’est guère davantage de se faire une passion, quelque noble qu’elle soit, pour ne recevoir plus d’autre conseil que le sien, si ce n’est que vous entendiez par là celle du bien en général, qui se change en autant de formes qu’il y a de choses justes, honnêtes et utiles. Il faut de la variété dans la gloire comme partout ailleurs, et en celle des princes plus qu’en celle des particuliers ; car qui dit un grand roi dit presque tous les talents ensemble de ses plus excellents sujets.\par
La valeur est une de ces qualités principales, mais ce n’est pas l’unique, elle laisse beaucoup à faire à la justice, à la prudence et à la bonne conduite, et à l’habileté dans les négociations : plus la valeur même est parfaite, plus elle affecte de ne point paraître à contretemps, et de ne se montrer que la dernière, pour achever ce que toutes les autres ont trouvé impossible. Si les autres qualités ont moins d’éclat, elles ne laissent pas d’acquérir au prince un honneur d’autant plus solide, que leurs bons effets ne semblent être que son propre ouvrage, où la fortune n’a presque point eu de part. Soyez toujours, mon fils, en état de vous faire craindre par les armes, mais ne les employez qu’au besoin, et souvenez-vous que notre puissance, lors même qu’elle est à son comble, pour être plus redoutée, doit être plus rarement éprouvée : tel qui ne pensait pas se pouvoir défendre contre nous trouvant chez ses amis, chez ses voisins, chez nos envieux, et quelquefois même dans son propre désespoir les moyens de nous résister.
\subsection[{Deuxième section}]{Deuxième section}
\noindent Je passerai maintenant en peu de mots, mon fils, quantité de choses qui feraient des volumes, si je les voulais étendre, et qui allaient en général à me faire craindre, aimer ou considérer par toute l’Europe.\par
Dans ces vues générales, l’acquisition de Dunkerque ne m’empêcha pas de faire payer à l’archiduc d’Inspruck une bonne partie des trois millions qui lui étaient accordés par le traité de Münster pour le dédommagement de l’Alsace : dette qu’il était important d’acquitter, pour ne laisser à la maison d’Autriche aucune prétention sur ce pays.\par
L’évêque de Spire m’ayant envoyé son chancelier pour régler plusieurs différends que nous avions touchant Philippsbourg, je trouvai moyen de le satisfaire équitablement sans rien perdre de mes droits.\par
Le duc de Neubourg, prince très considérable en Allemagne, qui avait de grandes prétentions à la couronne de Pologne toutes les fois qu’il y avait lieu à une nouvelle élection, et qui m’était ami et allié, eut recours à moi pour retirer des Hollandais la comté souveraine de Ravestein, qui lui était échue par le partage de la maison de Clèves. Je fis promettre aux États de le récompenser d’autres terres, se confirmant par là eux-mêmes la possession de celle-là qui leur était importante.\par
Je terminai encore par mon entremise un autre différend de conséquence qu’il avait avec la maison de Brunswick. Je lui donnai enfin une preuve bien plus forte de mon amitié ; car sur la plainte qu’il me fit que l’électeur de Brandebourg l’avait fait exclure du traité d’Oliva, je refusai de signer un accord déjà résolu entre cet électeur et moi, quoiqu’il m’importât extrêmement de le détacher peu à peu des intérêts de la maison d’Autriche, dont il était un des plus considérables partisans dans l’Empire. Mais je crus, et il est très véritable, mon fils, que ce qu’on fait avec raison et avec vigueur tout ensemble, pour ceux qui sont dans nos intérêts contre ceux qui n’y sont pas, confirme puissamment les uns à demeurer toujours nos amis, et n’invite pas moins les autres à le devenir toutes les fois qu’ils en auront une occasion favorable.\par
L’Alliance du Rhin, qui m’était si utile en Allemagne, et dont je vous ai déjà parlé ailleurs, allait se partager malheureusement entre les protestants et les catholiques, par une querelle du landgrave de Hesse d’un côté, et du comte de Valdeck de l’autre, soutenu et protégé de l’électeur de Cologne. J’apaisai cette querelle en telle sorte que les uns et les autres m’en furent obligés, et demeurèrent plus amis entre eux et avec moi qu’ils ne l’étaient auparavant.\par
Il restait plusieurs difficultés entre mes commissaires et les députés des Provinces-Unies pour le renouvellement de notre alliance, et l’affaire traînait depuis dix-huit mois : je m’y appliquai moi-même et la terminai en peu de jours, avec une égale satisfaction de part et d’autre. Je rendis inutiles par là, et par mille autres moyens que je n’expliquerai point ici, les propositions que les Espagnols faisaient sans cesse aux Provinces-Unies d’une ligue pour la défense des Pays-Bas.\par
J’éludai de même leurs brigues envers les Suisses, pour leur faire solliciter la neutralité des deux Bourgognes.\par
Je traversai et fis échouer les propositions de l’Empereur aux électeurs de Bavière, de Saxe et de Brandebourg, d’une autre ligue pour s’opposer à l’Alliance du Rhin.\par
Ajoutez à cela, mon fils, le mariage du roi d’Angleterre avec l’infante de Portugal dont je vous ai parlé parce qu’il fut négocié en 1661, quoiqu’il n’ait été terminé qu’en cette année 1662 : mariage qui entraîna après lui l’accommodement de l’Angleterre avec la Hollande, l’accommodement de la Hollande avec le Portugal, et l’union plus étroite de tous ces potentats avec moi, qui était comme le lien de la leur.\par
Toutes ces choses ensemble, les unes déjà exécutées, les autres en une disposition manifeste de l’être bientôt, ne contribuèrent pas médiocrement à une autre, que je vous ai déjà expliquée par avance pour vous la faire voir tout entière en un seul lieu : je veux dire à la satisfaction que je reçus environ ce même temps sur l’affaire de Londres. Je ne répète pas ce que je vous en ai dit ; j’ai voulu seulement marquer ici en passant, et en leur véritable place, les circonstances du temps qui réduisaient d’elles-mêmes l’Espagne à me faire une raison entière, contre ses maximes et son inclination, et qui rendaient mes mesures certaines, encore qu’elles pussent ne le pas paraître à ceux qui n’en voyaient pas le détail.\par
Je ne puis même m’empêcher, mon fils, de faire là-dessus une réflexion avec vous : car en considérant combien il est vrai que tout l’art de la politique est de se servir des conjonctures, je viens à douter quelquefois si les discours qu’on en fait et ces propres Mémoires ne doivent pas être mis au rang des choses inutiles, puisque l’abrégé de tous les préceptes consiste au bon sens et en l’application que nous ne recevons pas d’autrui, et que nous trouvons plutôt chacun en nous-même. Mais ce dégoût qui nous prend de nos propres raisonnements n’est pas raisonnable ; car l’application nous vient principalement de la coutume, et le bon sens ne se forme que par une longue expérience, ou par une méditation réitérée et continuelle des choses de même nature, de sorte que nous devons aux règles mêmes et aux exemples l’avantage de nous pouvoir passer des exemples et des règles.\par
Une autre erreur également dangereuse se glisse parmi les hommes : car, comme cet art de profiter de toutes choses, de celles que le peuple ignore comme de celles qu’il sait, plus il est grand et parfait, plus il se cache et se dérobe à la vue, en cela contraire à sa propre gloire, il arrive souvent qu’on veut obscurcir le mérite des bonnes actions en s’imaginant que le monde se gouverne de lui-même, par certaines révolutions fortuites et naturelles qu’il était impossible de prévoir ni d’éviter : opinion que les esprits du commun reçoivent sans peine, parce qu’elle flatte leur peu de lumière et leur paresse, leur permettant d’appeler leurs fautes du nom de malheur, et l’industrie d’autrui du nom de bonne fortune.\par
Pour voir, mon fils, comme vous devez reconnaître avec soumission une puissance supérieure à la vôtre et capable de renverser, quand il lui plaira, vos desseins les mieux concertés, soyez toujours persuadé, d’un autre côté, qu’ayant établi elle-même l’ordre naturel des choses, elle ne les violera pas aisément ni à toutes les heures, ni à votre préjudice, ni en votre faveur. Elle peut nous assurer dans les périls, nous fortifier dans les travaux, nous éclairer dans les doutes, mais elle ne fait guère nos affaires sans nous, et quand elle veut rendre un roi heureux, puissant, autorisé, respecté, son chemin le plus ordinaire est de le rendre sage, clairvoyant, équitable, vigilant et laborieux.\par
Mais je reprends la suite des choses. Comme je n’avais fait que mon devoir en soutenant la dignité de ma couronne, ce différend avec une nation qui aura toujours des intérêts opposés aux nôtres, n’empêcha pas que le roi d’Espagne ne me donnât depuis en toute rencontre des marques de son estime et de son amitié.\par
Il me témoigna son estime d’une manière dont j’avoue que je fus agréablement flatté, quand, après la mort de don Louis de Haro, il dit publiquement, devant tous les ambassadeurs des princes étrangers, que c’était à mon exemple qu’il ne voulait plus avoir de Premier ministre. Car il me semblait tout ensemble bien généreux pour lui, et bien glorieux pour moi, qu’après une si longue expérience des affaires, il reconnût que je lui avais servi de guide dans le chemin de la royauté ; et sans me donner trop de vanité, j’ai lieu de croire qu’en cela même plusieurs autres princes ont regardé ma conduite pour régler la leur : ce qui nous doit bien exhorter, mon fils, et vous et moi, à peser toutes nos actions, quand nous considérons quel bien nous faisons en faisant bien, et quel mal par conséquent en faisant mal, puisque les mauvais exemples trouvent encore plus d’imitateurs que les bons.\par
Il me témoigna son amitié en une chose qu’il pouvait me refuser avec justice. Par le traité des Pyrénées, les Espagnols étaient en droit et en possession de visiter tous les bâtiments français qu’ils rencontraient à cinquante milles des côtes du Portugal, et cette possession leur était importante à conserver. Il voulut bien néanmoins s’en départir, sur les pressantes instances que je lui en fis, et favorisa en cela le commerce maritime de mes sujets, qui en recevait beaucoup de préjudice.\par
J’avais pris ombrage d’un moine français qui résidait secrètement à sa cour ; mais encore qu’il n’y fût pas obligé, pour me faire voir combien il souhaitait de bien vivre avec moi, il s’offrit à me déclarer, en parole de roi, que ce religieux n’avait parlé d’aucune affaire qui regardât la France ; et en effet, je sus qu’il n’avait fait que quelques propositions touchant le Portugal.\par
Je répondais à ces démonstrations d’estime et d’amitié par d’autres semblables, toutes les fois que l’occasion s’en présentait ; et c’est pour cela que je donnai alors au marquis de Fuentes, son ambassadeur, des entrées libres et familières auprès de moi, que les autres ambassadeurs n’ont jamais eues ni prétendues, le recevant comme mes propres domestiques dans ma maison et dans mes divertissements. Cela n’eût pas été sans danger en d’autres temps, quand tous ceux qui approchaient du roi ou du ministre avaient part aux secrets et presque aux résolutions, ou pouvaient du moins les pénétrer par cent marques extérieures. Je pense y avoir pourvu autrement ; et de quelque sorte qu’on ait les yeux ouverts sur mes desseins, si je ne me trompe, ceux qui ne bougent du Louvre n’en savent guère davantage que ceux qui n’en approchent jamais.\par
Je ne m’arrêterais pas avec vous, mon fils, à un carrousel qui fut fait au commencement de l’été, si ce n’était le premier divertissement de quelque éclat que je rencontre dans la suite de ces Mémoires, et si votre vie devant par nécessité être mêlée de ces sortes de choses aussi bien que de plus grandes il n’était bon de vous faire remarquer quel est l’usage légitime qu’on en peut faire. Je ne vous dirai pas seulement, comme on dirait à un simple particulier, que les plaisirs honnêtes ne nous ont pas été donnés sans raison par la nature ; qu’ils délassent du travail, fournissant de nouvelles forces pour s’y appliquer, servent à la santé, calment les troubles de l’âme et l’inquiétude des passions, inspirent l’humanité, polissent l’esprit, adoucissent les mœurs, et ôtent à la vertu je ne sais quelle trempe trop aigre, qui la rend quelquefois moins sociable et par conséquent moins utile.\par
Un prince, et un roi de France, peut encore considérer quelque chose de plus dans ces divertissements publics, qui ne sont pas tant les nôtres que ceux de notre cour et de tous nos peuples. Il y a des nations où la majesté des rois consiste, pour une grande partie, à ne se point laisser voir, et cela peut avoir ses raisons parmi des esprits accoutumés à la servitude, qu’on ne gouverne que par la crainte et la terreur ; mais ce n’est pas le génie de nos Français, et, d’aussi loin que nos histoires nous en peuvent instruire, s’il y a quelque caractère singulier dans cette monarchie, c’est l’accès libre et facile des sujets au prince. C’est une égalité de justice entre lui et eux, qui les tient pour ainsi dire dans une société douce et honnête, nonobstant la différence presque infinie de la naissance, du rang et du pouvoir. Que cette méthode soit pour nous bonne et utile, l’expérience l’a déjà montré, puisque dans tous les siècles passés il n’est mémoire d’aucun empire d’aussi longue durée que celui-ci l’a été, et qui toutefois ne semble pas prêt à finir.\par
Et c’est une chose remarquable, mon fils, que les politiques les plus intéressés, les moins touchés de l’équité, de la bonté et de l’honneur, semblent avoir prédit l’éternité à cet État, autant que les choses humaines se la peuvent promettre. Car ils prétendent que ces autres empires où la terreur domine et où le caprice du prince est la seule loi, sont peut-être plus difficiles à entamer, mais que la première blessure leur est mortelle, n’y ayant presque point de sujet qui ne souhaite le changement, et qui ne le favorise aussitôt qu’il le peut espérer : au lieu qu’en France, disent-ils, s’il est facile de broncher, il y est encore plus facile de revenir à l’état naturel des choses, n’y en ayant aucun autre sans exception où les particuliers, et surtout les principaux d’entre eux, aussitôt qu’ils l’ont un peu éprouvé, puissent trouver leur intérêt et leur compte, comme ils le trouvaient à celui-là.\par
Il vous semblera peut-être, mon fils, que je vais bien loin dans cette réflexion ; mais elle ne laisse pas de venir parfaitement au sujet. J’avoue, mon fils, et tout ce que je vous ai déjà dit vous le fait assez comprendre, que cette liberté, cette douceur, et, pour ainsi dire, cette facilité de la monarchie, avaient passé les justes bornes durant ma minorité et les troubles de mon État, et qu’elle était devenue licence, confusion, désordre. Mais plus j’étais obligé à retrancher de cet excès, et par des remèdes plus agréables, plus il fallait conserver et cultiver avec soin tout ce qui, sans diminuer mon autorité et le respect qui m’était dû, liait d’affection avec moi mes peuples et surtout les gens de qualité, afin de leur faire voir par là même que ce n’était point ni aversion pour eux, ni sévérité affectée, ni rudesse d’esprit, mais raison et devoir simplement, qui me rendaient en d’autres choses plus réservé et plus exact à leur égard. Cette société de plaisirs, qui donne aux personnes de la cour une honnête familiarité avec nous, les touche et les charme plus qu’on ne peut dire. Les peuples, d’un autre côté, se plaisent au spectacle, où au fond on a toujours pour but de leur plaire ; et tous nos sujets, en général, sont ravis de voir que nous aimons ce qu’ils aiment, ou à quoi ils réussissent le mieux. Par là nous tenons leur esprit et leur cœur, quelquefois plus fortement peut-être, que par les récompenses et les bienfaits ; et à l’égard des étrangers, dans un État qu’ils voient florissant d’ailleurs et bien réglé, ce qui se consume en ces dépenses qui peuvent passer pour superflues, fait sur eux une impression très avantageuse de magnificence, de puissance, de richesse et de grandeur, sans compter encore que l’adresse en tous les exercices du corps, qui ne peut être entretenue et confirmée que par là, est toujours de bonne grâce à un prince, et fait juger avantageusement, par ce qu’on voit, de ce qu’on ne voit pas.\par
Toutes ces considérations, mon fils, quand mon âge et mon inclination ne m’y auraient pas porté, m’obligeaient à favoriser des divertissements de cette nature, et vous y doivent obliger de même, sans aller pourtant à un excès d’attachement qui ne serait pas louable ; car alors, mon fils, quelque gravité que vous puissiez d’ailleurs affecter dans vos autres actions, ne vous y trompez pas, vous ne tromperiez point le public. Sous la couronne, quand vous l’auriez toujours en tête, et au travers du manteau royal, on aurait bientôt reconnu que vous faites de vos plaisirs vos affaires, et passez par-dessus les affaires comme il faut passer par-dessus les plaisirs. Par cette raison, il est quelquefois dangereux aux jeunes princes de réussir au-delà du commun à de certains exercices, et de ce genre surtout ; car ce fonds inépuisable d’amour-propre qui nous est si naturel nous porte toujours à cultiver, estimer et aimer sans mesure toutes les choses où nous pensons exceller au-dessus des autres. Si vous en croyez le maître à danser et le maître d’armes, et tous les autres, ils vous diront chacun, et il est vrai, que leur art demande l’homme tout entier, et qu’on y trouve toujours à apprendre ; mais c’est assez pour nous de connaître cette vérité sans en faire l’expérience, ni chercher les dernières bornes de leur savoir, qu’ils ne trouvent jamais eux-mêmes. Cette perfection, quand nous pourrions l’acquérir, marquerait une attention et un soin peu dignes de nous, qu’on ne peut avoir qu’en négligeant ce qui vaut beaucoup mieux. Vous savez le mot de ce roi d’autrefois à son fils : N’as-tu point de honte de jouer si bien de la lyre ? Souffrez qu’en toutes ces sortes de choses, il y ait parmi vos sujets des gens qui vous surpassent, mais que nul ne vous égale, s’il se peut, dans l’art de gouverner, que vous ne pouvez trop bien savoir, et qui doit être votre application principale.\par
Le carrousel, qui m’a fourni le sujet de ces réflexions, n’avait été projeté d’abord que comme un léger amusement ; mais on s’échauffa peu à peu, et il devint un spectacle assez grand et assez magnifique, soit par le nombre des exercices, soit par la nouveauté des habits ou par la variété des devises.\par
Ce fut là que je commençai à prendre celle que j’ai toujours gardée depuis, et que vous voyez en tant de lieux. Je crus que, sans s’arrêter à quelque chose de particulier et de moindre, elle devait représenter en quelque sorte les devoirs d’un prince, et m’exciter éternellement moi-même à les remplir. On choisit pour corps le soleil, qui, dans les règles de cet art, est le plus noble de tous, et qui, par la qualité d’unique, par l’éclat qui l’environne, par la lumière qu’il communique aux autres astres qui lui composent comme une espèce de cour, par le partage égal et juste qu’il fait de cette même lumière à tous les divers climats du monde, par le bien qu’il fait en tous lieux, produisant sans cesse de tous côtés la vie, la joie et l’action, par son mouvement sans relâche, où il paraît néanmoins toujours tranquille, par cette course constante et invariable, dont il ne s’écarte et ne se détourne jamais, est assurément la plus vive et la plus belle image d’un grand monarque.\par
Ceux qui me voyaient gouverner avec assez de facilité et sans être embarrassé de rien, dans ce nombre de soins que la royauté exige, me persuadèrent d’ajouter le globe de la terre, et pour âme {\itshape nec pluribus impar} : par où ils entendaient ce qui flattait agréablement l’ambition d’un jeune roi, que, suffisant seul à tant de choses, je suffirais sans doute encore à gouverner d’autres empires, comme le soleil à éclairer d’autres mondes, s’ils étaient également exposés à ses rayons. Je sais qu’on a trouvé quelque obscurité dans ces paroles, et je ne doute pas que ce même corps n’en pût fournir de plus heureuses. Il y en a même qui m’ont été présentées depuis ; mais celle-là étant déjà employée dans mes bâtiments et en une infinité d’autres choses, je n’ai pas jugé à propos de la changer.\par
Ce fut aussi cette année que, continuant dans le dessein de diminuer l’autorité des gouverneurs des places et des provinces, je résolus de ne plus donner nul gouvernement vacant que pour trois ans, me réservant seulement le pouvoir de prolonger ce terme par de nouvelles provisions toutes les fois que je le trouverais à propos.\par
Le gouvernement de Paris vaquait par la mort du maréchal de l’Hôpital : je le donnai, avec cette condition de trois ans, au maréchal d’Aumont, personne de considération, l’un des quatre capitaines de mes gardes du corps, attaché depuis longtemps à mon service personnel, afin qu’après cet exemple qui que ce soit ne se pût croire moins bien traité quand on pratiquerait à son égard le même règlement. Je l’ai toujours observé depuis, et j’ai trouvé qu’il produisait deux bons effets : l’un, que ceux qui servent sous les gouverneurs cessent de prendre avec eux les attachements et les mesures qu’ils y prenaient auparavant ; l’autre, que les gouverneurs eux-mêmes ne pouvant demeurer dans leur emploi que par une continuation de ma bonne volonté, ils vivent dans une soumission beaucoup plus grande.\par
Je donnai encore un archevêque à Paris, après lui avoir donné un gouverneur. On sait le peu de sujet que j’avais alors d’être content du cardinal de Retz, et de quelle conséquence il m’était que cette dignité fût remplie d’un autre. Tant qu’il avait espéré son rétablissement des intrigues ou des révolutions de la cour durant la vie du cardinal Mazarin, il avait opiniâtrement refusé sa démission, quelques propositions avantageuses qu’on lui eût pu faire. Il ne me vit pas plutôt agir par moi-même, et l’autorité affermie en mes mains qui rendait toutes les cabales inutiles, qu’il crut que le meilleur parti était de se remettre à ma volonté, comme il fit sans aucune condition.\par
J’avais nommé d’abord pour cette place importante l’archevêque de Toulouse, Marca, homme d’un savoir et d’un mérite extraordinaires ; mais il mourut aussitôt après, et je choisis pour lui succéder l’évêque de Rodez, qui avait été mon précepteur.\par
Je ne fus pas fâché sans doute, mon fils, de reconnaître, par cette marque de mon affection, le soin qu’il avait pris de mon enfance, et il n’y a personne à qui nous devions davantage qu’à ceux qui ont eu l’honneur et la peine tout ensemble de former notre esprit et nos mœurs. Mais je ne me serais jamais déterminé à ce choix, si je n’eusse connu en lui, avec plus de certitude qu’en aucun autre, les qualités qui me semblèrent les plus nécessaires en un poste aussi considérable que celui-là. J’ai très souvent résisté à mon inclination, je le puis dire avec vérité, pour ne faire de cette nature de bien à des personnes à qui j’aurais fait avec plaisir du bien de toute autre sorte, ne remarquant pas en elles ou la capacité ou l’application d’un véritable ecclésiastique.\par
Qui pourrait croire, mon fils, qu’il y eût quelque chose de plus important que notre service et que la tranquillité de nos sujets ? Cependant la distribution des bénéfices, par la suite nécessaire qu’elle entraîne après elle, l’est sans comparaison davantage, et autant que le ciel est élevé au-dessus de la terre. C’est en apparence une riche et abondante moisson qui nous revient en toutes les saisons de l’année pour combler de grâces ceux qui nous servent ou ceux que nous aimons. Mais peut-être n’y a-t-il rien de plus épineux en toute la royauté, s’il est vrai, comme on n’en peut douter, que notre conscience demeure engagée pour peu que nous donnions trop ou à notre propre penchant, ou au souvenir des services rendus, ou même à quelque utilité présente de l’État, en faveur de personnes d’ailleurs incapables, ou beaucoup moins capables que d’autres sur qui nous pourrions jeter les yeux.\par
Je ne veux pas toutefois, mon fils, vous porter à des opinions rigoureuses qui ne se réduisent presque jamais à la pratique, et s’éloignent aussi le plus souvent de la vérité. Un de nos aïeux, par la crainte de ne pouvoir bien répondre à une obligation si délicate, se dépouilla volontairement de la nomination aux bénéfices. Mais qui nous a dit si d’autres s’en acquitteront mieux que nous, et si ce ne serait point mal faire notre devoir pour le vouloir trop bien faire ? Dieu n’entend point très assurément, mon fils, que nous fassions le choix du plus digne comme il le pourrait faire lui-même, ce qui nous est impossible. C’est assez que nous le fassions en hommes, et en hommes bien intentionnés qui n’oublient rien pour ne se point tromper. Alors, j’ose le dire, nous pouvons nous assurer que c’est lui-même qui le fait par nous. Il n’est point vrai non plus que ceux qui nous servent ou qui nous approchent n’aient en cela nul avantage au-dessus des autres ; ils ont celui de nous faire mieux connaître ce qu’ils valent, grand sans doute auprès d’un prince éclairé, qui croit beaucoup plus à ce qu’il voit qu’à ce qui lui vient par le rapport d’autrui, toujours mêlé de bons ou de mauvais offices.\par
J’ai toujours cru que trois choses devaient entrer dans cet examen : le savoir, la piété et la conduite. À l’égard du savoir, il nous est peut-être plus difficile d’en juger que de tout le reste : car il arrive très rarement que les rois soient consommés dans ces sortes de choses, ou que, quand ils le seraient, ils trouvent le temps d’étudier en cela les talents et la portée de chacun. Contre cette difficulté, j’ai observé, autant que je l’ai pu, de ne donner les bénéfices importants qu’à des docteurs de Sorbonne : non pas qu’il n’y ait assez d’inégalité entre les connaissances et les lumières de ceux qui portent ce titre, mais au fond on ne peut jamais y être parvenu sans une capacité très raisonnable, fort éloignée de cette ancienne ignorance des prélats qui a fait tant de mal à l’Église. Ainsi, cette preuve, jointe à toutes les autres que nous en pouvons avoir, suffit sans doute pour nous mettre en repos sur ce sujet.\par
Quant à la piété et aux mœurs, ce qu’il y a de bien ou de mal ne se peut cacher longtemps aux yeux du monde. Écoutez sans préoccupation les divers rapports qu’on vous fera, même en faisant autre chose ; regardez vous-même, avec quelque sorte d’attention, ceux qui sont sous vos yeux : vous en saurez bientôt tout ce que les hommes en peuvent savoir, et vous n’êtes pas obligé de pénétrer le reste.\par
J’en dis de même de ce que je nomme conduite, qui est un troisième point bien important. Car si, dans la première simplicité, les apôtres mêmes ont voulu qu’on examinât pour faire un évêque quelle prudence il avait montrée dans son domestique et dans ses propres affaires, que sera-ce aujourd’hui où, par la constitution de l’État, ces sortes de dignités ont part en plusieurs choses au gouvernement civil ?\par
Ainsi, mon fils, je ne louerais pas volontiers qu’on pratiquât ordinairement ce qui se peut faire quelquefois avec dignité et avec éclat, pour rendre hommage à une piété éminente : je veux dire d’aller prendre dans les solitudes, sur une réputation assez souvent trompeuse, des sujets pour remplir ces places. Ils auront peut-être de très grandes perfections pour cet état où Dieu les a placés, et n’auront point celles qui leur sont nécessaires à cet autre état où nous les appelons.\par
Au contraire, j’ai souvent pensé que pour mieux connaître nos ecclésiastiques, et de quoi ils sont capables, il serait bon de faire observer dans cette milice sacrée ce que j’observe aujourd’hui avec soin dans la plupart de mes troupes où on monte par degré de charge en charge, ce que j’apprends aussi être tout à fait conforme au premier esprit de l’Église dans l’institution des cinq ordres sacrés. Mais comme le temps et les usages sont changés, il suffirait aujourd’hui, ce me semble, de n’admettre aux évêchés et autres dignités considérables que ceux qui auraient actuellement servi l’Église durant un certain temps, soit dans la prédication assidue et continuelle aux grandes paroisses de Paris, soit dans les missions des provinces, soit dans une application particulière à convertir les hérétiques, soit, ce qui serait le plus important, en faisant les fonctions de curé ou de vicaire, qui embrassent toutes ces choses et plusieurs autres : de quoi les jeunes gens de la plus haute naissance ne seraient pas plus à plaindre qu’ils le sont, quand ils portent le mousquet dans mes gardes, pour parvenir quelque jour à commander mes armées.\par
Mais il faut, mon fils, et pour vous et pour moi, gagner peu à peu ce que nous pouvons sur notre siècle, sans prétendre de le réformer en une seule fois, et cela même, je ne voudrais point le faire en ces matières par des édits publics, qui nous engagent ou à affaiblir l’autorité de nos propres lois, en ne les observant pas toujours, ou à pratiquer toujours les mêmes choses, encore qu’elles ne soient pas toujours à propos. Il suffit de montrer, par quelques paroles et par quelques exemples, le chemin des grâces, et vous verrez qu’on se pressera bientôt à le prendre.
\subsection[{Troisième section}]{Troisième section}
\noindent Ce fut environ ce même temps, mon fils, que je créai et mis sur pied votre compagnie de chevau-légers : ce ne fut pas seulement pour vous donner cette marque de mon affection, mais aussi par une occasion particulière qu’il est bon de vous expliquer.\par
La paix me permettait de licencier la plus grande partie de mes troupes : le dessein de soulager mes peuples m’y engageait. De dix-huit cents compagnies d’infanterie je n’en gardai que huit cents, et de mille cornettes de cavalerie que quatre cent neuf seulement ; mais la guerre pouvait facilement revenir, et l’on pouvait difficilement retrouver des troupes aussi aguerries que celles-là, surtout si on perdait ce grand nombre d’officiers qu’il avait fallu réformer, et qui en faisait la principale vigueur. Une partie n’avaient que leur emploi pour subsister et me touchaient de compassion. Plusieurs ne pouvant se résoudre à l’oisiveté entière, pensaient à prendre parti chez l’étranger. Je crus à propos d’en retenir autant que je le pouvais auprès de ma personne. J’en plaçai quantité dans mes gardes du corps et dans mes mousquetaires, et ce fut pour occuper les autres, que je formai votre compagnie de chevau-légers, leur donnant, outre la paye ordinaire des corps où ils entraient, des pensions proportionnées aux emplois où ils avaient été jusqu’alors ; et ainsi je faisais subsister un grand nombre de braves gens, et je me conservais à moi-même le moyen de remettre en moins de rien d’autres troupes sur pied, peu différentes des premières, puisque c’est d’ordinaire l’officier qui inspire, non seulement la discipline et l’adresse, mais aussi le courage au soldat ; et d’ailleurs j’avais souvent remarqué avec plaisir la différence presque infinie du reste des troupes d’avec celles de ma maison, que l’honneur d’être plus particulièrement à moi, la discipline plus exacte, l’espérance plus certaine des récompenses, des exemples du passé, l’esprit qui y régnait de tout temps, rendaient absolument incapables d’une mauvaise action. Ainsi, il me semblait utile d’en augmenter plutôt que d’en diminuer le nombre, à quoi je trouvais aussi de la dignité et de la grandeur.\par
Je m’appliquai aussi cette année à un règlement pour les forêts de mon royaume, où le désordre était extrême, et me déplaisait d’autant plus que j’avais formé de longue main de grands desseins pour la marine. Les causes principales de ce désordre peuvent servir à votre instruction, mon fils. C’est une simplicité sans doute que de confier nos intérêts, en matière d’argent, aux mêmes personnes à qui nous faisons, d’un autre côté, quelque tort considérable dans les levées, et qui peuvent le réparer en nous trompant. Il n’appartient à la vérité qu’aux rois de se faire justice eux-mêmes, depuis que les particuliers y ont renoncé pour l’utilité publique et pour la leur propre, en se soumettant à la loi civile. Mais, quand ils peuvent impunément et secrètement rentrer en possession de ce droit naturel, leur fidélité n’est guère à cette épreuve, et c’est alors une vertu presque héroïque dont le commun des hommes n’est pas capable.\par
La guerre et l’invention des partisans pour faire de l’argent avaient produit une infinité d’officiers des eaux et forêts comme de toutes les autres sortes ; la guerre et les mêmes inventions leur ôtaient ou leur retranchaient leurs gages, dont on ne leur avait fait qu’une vaine montre, en établissant leurs offices. Ils s’en vengeaient et s’en payaient, mais avec usure, aux dépens des forêts qui leur étaient commises, et cela d’autant plus facilement que peu de personnes étaient intelligentes en ces matières, hors celles qui avaient part au crime et au profit. Il n’y avait sortes d’artifices dont ces officiers ne se fussent avisés, jusqu’à brûler exprès une partie des bois sur pied, pour avoir lieu de prendre le reste comme brûlé par accident. J’avais su et déploré cette désolation de mes forêts dès l’année précédente ; mais mille autres choses plus pressées m’empêchant d’y pourvoir entièrement, j’avais seulement empêché le mal de s’augmenter, en défendant qu’il se fît aucune vente jusqu’à ce que j’en eusse autrement ordonné. Cette année j’y apportai, par le règlement dont je vous ai déjà parlé, deux remèdes principaux : l’un fut la réduction des officiers à un petit nombre qu’on pût payer de leurs gages sans peine, et sur lesquels il fût plus aisé d’avoir les yeux ; l’autre fut la recherche des malversations passées, qui ne servait pas seulement d’exemple pour l’avenir, mais qui, par les restitutions considérables auxquelles ils furent condamnés, fournissait en partie au remboursement des officiers supprimés, et rendait cette réduction également juste et facile.\par
J’augmentai d’ailleurs cette année mon revenu ordinaire de quatre millions en un seul article : d’un côté, en joignant les entrées de Paris à la ferme des aides, ce qui épargnait aux fermiers beaucoup de frais et leur donnait le moyen d’en porter les enchères plus haut ; de l’autre, en m’assujettissant moi-même à ne donner ni l’une ni l’autre qu’en la meilleure saison, qui est celle du quartier d’octobre, et sous certaines conditions où ils pouvaient trouver leur avantage et le mien ; mais principalement en retirant et réunissant à celle des aides quantité de droits qui en avaient été distraits durant la guerre, et aliénés aux personnes les plus puissantes, chacun en ayant acquis ce qui était à sa bienséance, aux lieux où il avait d’ailleurs le plus de revenu et de crédit, et cela ordinairement à très vil prix, ou même sans deniers comptants, pour de très mauvaises marchandises qu’on m’avait données en paiement. Je ne fis injustice à qui que ce soit, liquidant équitablement le remboursement qui leur était dû.\par
Mais cette justice même avait besoin d’une autorité aussi établie que la mienne l’était alors, pour se faire recevoir avec soumission et sans murmure. Après la réunion de ces droits aliénés, les deux fermes jointes ensemble furent portées à douze millions au lieu de huit, sans que j’eusse rien fait néanmoins que remettre toutes choses en leur situation naturelle où elles auraient dû être toujours.\par
Je ne me proposais pas seulement en cela pour but l’intérêt présent, quoique considérable, mais un bien sans comparaison plus grand et plus général pour l’avenir, qui était de faire en sorte, s’il était possible, qu’en nul temps, qu’en nulle occasion, on ne fût réduit désormais à ces aliénations misérables qui avaient désolé mes finances et mon État. Je savais jusqu’à quelles sommes par an on avait monté la plus forte dépense de la guerre. Je ne trouvai pas impossible de porter mon revenu jusque-là, et même par la seule économie dont je voyais tous les jours de si grands effets. Et je regardais comme une grande félicité pour moi d’établir à tel point celle de mes peuples, que la guerre même, si elle revenait, ne fût presque plus capable de la troubler, qu’ils ne fussent plus du moins exposés aux affaires extraordinaires, accompagnées de tant de vexations pour eux, ni obligés, comme autrefois, à gémir au-dedans des prospérités du dehors, où ils ne trouvaient qu’un vain honneur acquis par une véritable misère.\par
Mais je passais encore plus avant, mon fils. Car, en supposant, comme il est arrivé en effet depuis, que je porterais bientôt mon revenu jusqu’à cette somme que je m’étais fixée, suffisante pour soutenir la plus grande guerre sans crédit et sans secours extraordinaires, je résolus en moi-même de ne plus rien ajouter à ce revenu, mais de diminuer chaque année des impositions ordinaires, au profit de mes sujets, ce que j’aurais augmenté d’un autre côté à mes finances ou par la paix et par l’économie, ou par le rachat de mes anciens domaines, ou par d’autres voies justes et légitimes : en sorte qu’on n’eût jamais vu, s’il était possible, ni le prince plus riche, ni les peuples moins chargés.\par
Dans ces mêmes pensées, deux choses me paraissaient très nécessaires à leur soulagement. L’une était de diminuer dans les provinces le nombre de ceux qui étaient exempts des tailles et qui rejetaient par ce moyen tout le fardeau sur les plus misérables. De celle-là, j’en venais à bout en supprimant et remboursant tous les jours quantité de petits offices nouveaux et très inutiles, à qui cette exemption avait été attribuée durant la guerre pour les débiter.\par
L’autre était d’examiner de plus près les exemptions que certains pays particuliers prétendaient dans mon royaume, et dont ils étaient en possession, moins par aucun titre ou par aucun service considérable, que par la facilité des rois nos prédécesseurs, ou par la faiblesse de leurs ministres. Le Boulonnais était de ce nombre. Les peuples y sont aguerris depuis la guerre des Anglais, et ont même une espèce de milice dispersée dans les divers lieux du gouvernement, qui est assez exercée, et se rassemble facilement au besoin. Sous ce prétexte, ils se tenaient exempts depuis longtemps de contribuer en aucune sorte à la taille. Je voulus y faire imposer une très petite somme, seulement pour leur faire connaître que j’en avais le pouvoir et le droit. Cela produisit d’abord un mauvais effet ; mais l’usage que j’en fis, quoiqu’avec peine et avec douleur, l’a rendu bon pour les suites. Le bas peuple, ou effrayé d’une chose qui lui paraissait nouvelle, ou secrètement excité par la noblesse, s’émut séditieusement contre mes ordres. Les remontrances et la douceur de ceux à qui j’en avais confié l’exécution, étant prises pour timidité et pour faiblesse, augmentèrent le tumulte au lieu de l’apaiser. Les mutins se rassemblèrent en divers lieux jusqu’au nombre de six mille hommes : leur fureur ne pouvait plus être dissimulée. J’y envoyai des troupes pour la châtier ; ils se dispersèrent pour la plus grande partie. Je pardonnai sans peine à tous ceux dont la retraite témoignait le repentir. Quelques-uns, plus obstinés dans leurs fautes, furent pris les armes à la main et abandonnés à la justice. Leur crime méritait la mort. Je fis en sorte que la plupart fussent seulement condamnés aux galères, et je les aurais même exemptés de ce supplice, si je n’eusse cru devoir suivre en cette rencontre ma raison plutôt que mon inclination.\par
Nous serions trop heureux, mon fils, si nous n’avions jamais qu’à obliger et à faire des grâces. Mais Dieu même dont la bonté n’a point de bornes ne trouve pas toujours à récompenser, et est quelquefois contraint de punir. Quelque douleur que nous ayons de faire du mal, nous devons en être consolés, quand nous sentons en nous-même que nous le faisons comme lui, par la seule vue juste et légitime d’un bien mille fois plus considérable. Ce n’est pas répandre le sang de nos sujets, c’est plutôt le ménager et le conserver que d’exterminer les homicides et les malfaiteurs : c’est se laisser toucher de compassion plutôt pour un nombre infini d’innocents que pour un petit nombre de coupables. L’indulgence pour ces malheureux particuliers serait une cruauté universelle et publique.\par
Je ne vous parlerais pas ainsi, mon fils, si j’avais remarqué en vous le moindre penchant à une sévérité excessive, pour ne point dire à cette humeur sanguinaire et farouche, indigne d’un homme, bien loin d’être digne d’un roi. Au contraire, je tâcherais de vous faire connaître le charme de la clémence, la plus royale de toutes les vertus puisqu’elle ne peut jamais appartenir qu’à des rois, la seule par qui on peut nous devoir plus qu’on ne nous saurait jamais rendre, j’entends la vie et l’honneur, la plus grande enfin de toutes les choses qu’on peut révérer en nous, puisqu’elle est comme d’un degré au-dessus de notre puissance et de notre justice.\par
Mais, autant que j’en puis juger par les actions de votre enfance, les observant comme je le fais avec soin, vous serez, et j’en loue Dieu de tout mon cœur, compatissant, facile à être apaisé, et aurez beaucoup moins à vous défendre de la colère, de la haine et de la vengeance que des défauts opposés. Qu’on ne vous surprenne point seulement par le propre amour que vous aurez pour la gloire, en vous faisant passer ces défauts pour des vertus. L’applaudissement les suit d’abord, mais le mépris ne tarde guère à venir après lui, et l’on connaît que si ce ne sont en un prince les pères de tous les vices, ce sont du moins les plus dangereux. Ôter la rigueur aux lois, c’est ôter l’ordre, la paix et le repos au monde, c’est s’ôter à soi-même la royauté.\par
Quiconque pardonne trop souvent punit presque inutilement le reste du temps : car, dans cette terreur qui retient les hommes du mal, l’espérance de l’impunité ne fait guère moins d’effet que l’impunité même. Vous n’achèverez pas la lecture de ces Mémoires, mon fils, sans trouver des endroits où j’ai su me vaincre moi-même et pardonner des offenses que je pouvais justement ne jamais oublier. Mais en cette occasion où il s’agissait de l’État, des plus pernicieux exemples et du mal le plus contagieux du monde pour tout le reste de mes sujets, d’une révolte à main armée qui n’attaquait pas mon autorité en quelque partie moins importante mais dans son propre fondement, je crus me devoir vaincre d’une autre sorte, en laissant punir ces misérables à qui j’aurais voulu pouvoir pardonner. La douleur que j’en eus a été bien récompensée par la satisfaction de voir que leur châtiment m’a empêché depuis d’avoir jamais besoin d’un pareil remède.\par
Il était alors d’autant plus important de réprimer de pareils mouvements, que ma prospérité commençait à faire de l’envie, et que la coutume de nos voisins est d’attendre leurs ressources des révolutions de la France, se formant des espérances vaines et chimériques à la moindre apparence de nouveauté.\par
J’observais alors avec soin les démarches du prince Charles de Lorraine, mécontent du traité que j’avais fait avec son oncle, et je tâchais d’aller au-devant de tout ce qu’il pouvait émouvoir contre moi, n’étant nullement à craindre par ses propres forces.\par
À l’égard des Électeurs et des autres princes de l’Empire, sans attendre qu’il les engageât à me parler pour lui, et tâchât de me brouiller avec eux, par les réponses mêmes que je serais obligé de leur faire, je leur demandai moi-même le premier qu’ils ne me demandassent rien en sa faveur. Je l’obtins, et il me fut plus aisé de prévenir leurs instances, qu’il ne l’eût été de m’en défendre.\par
L’Empereur était très occupé par la guerre contre les Turcs, et il avait témoigné assez d’envie de bien vivre avec moi, s’étant résolu à m’écrire le premier, comme je vous l’ai déjà dit ailleurs, contre ses prétentions passées. Mais cette bonne disposition pouvait changer ; il pouvait faire la paix sans aucune participation des autres potentats de l’Europe, et se servir contre moi des mêmes secours qu’on lui avait donnés contre cet ennemi commun. Je fis en sorte, par diverses négociations, que ces secours ne lui fussent pas donnés en argent comme il le souhaitait, mais en troupes dont il ne pouvait abuser ; et je fus d’autant plus volontiers écouté partout, qu’en cela mon intérêt particulier s’accordait avec l’utilité publique.\par
Quant au roi d’Espagne, je souhaitais de lui faire approuver le traité de Lorraine et l’engager de telle sorte que le prince Charles ne pût non plus rien attendre de lui. Mais à connaître l’humeur des Espagnols, une négociation dans les formes m’aurait rendu le succès plus difficile, leur faisant connaître le désir et l’intérêt que j’avais de l’obtenir ; je pris un tour plus délicat et plus simple : j’écrivis sur ce sujet au Roi Catholique, mais une lettre conçue de telle sorte qu’il était impossible d’y faire réponse sans louer ou condamner mon procédé. Elle était de ma propre main, afin que l’honnêteté l’obligeât d’autant plus à y répondre. Il le fit ; et ne se trouvant pas en état ou en volonté de me contredire, il me donna aussi de sa main propre une approbation pour ce traité, plus formelle et plus précise que je n’aurais osé l’espérer.
\section[{Année 1666}]{Année 1666}\renewcommand{\leftmark}{Année 1666}

\noindent La mort du roi d’Espagne et la guerre des Anglais contre les Provinces-Unies, étant arrivées presque en même temps, m’offraient à la fois deux importantes occasions de faire la guerre, l’une contre l’Espagne pour la poursuite des droits qui m’étaient échus, et l’autre contre l’Angleterre pour la défense des Hollandais.\par
Ce n’est pas que le roi de la Grande-Bretagne ne me fournît un prétexte assez apparent pour me dégager de cette dernière querelle, disant que par le traité qui m’engageait aux Hollandais, je ne leur avais promis mon assistance qu’en cas qu’ils fussent attaqués, et qu’ainsi je ne leur devais aucun secours en cette rencontre dans laquelle ils étaient les agresseurs.\par
Mais quoiqu’il m’eût été fort commode de me laisser persuader à cette raison, comme je savais au vrai que l’agression venait de la part de l’Angleterre, je voulus agir de bonne foi, suivant les termes de mon traité.\par
Je différai pourtant, autant que je pus, à me déclarer pour tâcher à les mettre d’accord, mais mes efforts étant inutiles, craignant qu’enfin les deux partis ne s’accordassent d’eux-mêmes à mon préjudice, je résolus de prendre celui auquel ma parole était engagée.\par
Mais cette question étant terminée, il m’en restait une autre plus difficile à décider, qui était de savoir si, pour conserver ensemble mes intérêts et ceux de mes alliés, j’entrerais à la fois en guerre contre l’Angleterre et contre l’Espagne, ou si n’entreprenant d’abord que la querelle des Hollandais, j’attendrais un autre temps pour vider la mienne : délibération sans doute importante par le nombre et par le poids des raisons qui se pouvaient alléguer des deux côtés.\par
D’une part, j’envisageais avec plaisir le dessein de ces deux guerres comme un vaste champ où pouvaient naître à toute heure de grandes occasions de me signaler. Tant de braves gens que je voyais animés pour mon service semblaient à toute heure me solliciter de fournir quelque matière à leur valeur, plus avantageuse que la guerre maritime dans laquelle les plus vaillants n’ont presque jamais lieu de se distinguer des plus faibles. Mais dans mon intérêt propre, je considérais que le bien de l’État ne permettant pas qu’un roi s’expose aux caprices de la mer, je serais obligé de commettre à mes lieutenants tout le destin de mes armes, sans jamais pouvoir agir de mon chef ; que d’ailleurs, étant obligé à tout événement d’entretenir de grandes forces, il m’était plus expédient de les jeter dans les Pays-Bas que de les nourrir à mes dépens ; qu’aussi bien toute la maison d’Autriche, persuadée de mes intentions, ne manquerait pas de me nuire indirectement en toutes choses ; qu’ayant à faire la guerre, il valait mieux en avoir une où l’on vît quelque gain apparent que de porter mes efforts contre des insulaires sur qui je ne pouvais rien conquérir qui ne me fût plus onéreux que profitable ; que si je faisais ces deux affaires à la fois, les Hollandais m’en serviraient mieux contre les Espagnols, ayant besoin de moi contre l’Angleterre, au lieu qu’étant tout à fait hors de danger, ils craindraient peut-être plus l’augmentation de ma puissance qu’ils ne se souviendraient de mes bienfaits ; qu’enfin plusieurs de mes prédécesseurs s’étaient vus dans d’aussi grandes affaires que pouvait être celle-là et que, refusant de m’exposer aux difficultés qu’ils avaient surmontées, j’étais en danger de ne pas obtenir les éloges qu’ils avaient mérités.\par
Mais pour appuyer le sentiment contraire, je remarquais que, comme un prince acquiert de la gloire à vaincre les difficultés qu’il ne peut éviter, il se met en danger d’être accusé d’imprudence en se jetant trop aisément dans celles qu’un peu d’adresse lui pouvait épargner ; que la grandeur de notre courage ne nous doit pas faire négliger le secours de notre raison, et que plus on aime chèrement la gloire, plus on doit tâcher de l’acquérir avec sûreté ; que sous prétexte de la guerre d’Angleterre, je disposerais mes forces et mes intelligences pour commencer plus heureusement celle de Flandre ; que les Anglais seuls n’étaient pas à craindre, mais que leur secours serait d’un grand poids pour la défense des terres d’Espagne ; que d’attaquer ces deux puissants ennemis à la fois, c’était former entre eux une liaison qui ne se pourrait pas dissoudre quand on voudrait, et qui m’obligerait infailliblement ou à les combattre toujours ensemble ou à m’accorder avec tous deux à des conditions moins avantageuses ; que cette union de l’Espagne avec l’Angleterre avancerait la paix du Portugal, que considérant les Hollandais selon le temps présent et dans l’application qu’ils avaient à leur propre défense, leur secours ne pouvait me procurer tant d’avantage que les Anglais me feraient de préjudice et qu’en voulant raisonner sur l’avenir, il n’y avait pas de moyen plus honnête pour les engager dans mes intérêts que de leur faire paraître ma bonne foi en commençant la guerre purement pour eux ; mais que, du moins il me serait glorieux devant toutes les nations de la terre, qu’ayant d’un côté mes droits à poursuivre et de l’autre mes alliés à protéger, j’eusse été capable de négliger mon intérêt pour entreprendre leur défense.\par
Je fus quelque temps incertain entre ces deux opinions. Mais si la première flattait plus doucement mon humeur, la seconde touchait plus solidement ma raison ; et je crus que dans le poste où j’étais, je devais faire violence à mes sentiments pour m’attacher aux intérêts de ma couronne.\par
Ainsi je résolus de ne m’engager qu’à la seule guerre de mer, et pour la faire plus avantageusement, je désirai de mettre le roi de Danemark dans notre parti.\par
L’avantage que j’y remarquai était que fermant par son moyen aux Anglais l’entrée de la mer Baltique, ils étaient privés de toutes les commodités qu’ils en tiraient et principalement des choses nécessaires à la navigation, fort nécessaires pour entretenir la guerre. La difficulté qui s’y rencontrait était que les Hollandais, étant brouillés avec ce prince, s’arrêtaient à certaine somme pour faire leur accommodement. Mais j’accordai l’affaire en fournissant de mes deniers une partie de ce qu’il prétendait, moyennant quoi je fis comprendre dans un traité secret tous les articles que je désirais.\par
Cependant je faisais insensiblement des levées et garnissais mes frontières de troupes et de munitions. Mais l’état où était Mardick me donnait de l’inquiétude. Car cette place se trouvait alors à moitié démolie, et j’avais peine à juger si je devais ou la rétablir en diligence ou en achever la démolition. La réparant, je craignais que les ennemis ne la surprissent avant qu’elle fût en sa perfection ; et la démolissant, j’appréhendais qu’ils ne s’y portassent après pour la rétablir eux-mêmes. Mais en attendant que je me fusse déterminé, j’y fis demeurer le maréchal d’Aumont avec un petit corps d’armée.\par
J’eus aussi quelque appréhension pour les vaisseaux qui étaient ordinairement à la rade de Toulon. Car, sachant qu’elle n’était défendue que de deux tours fort éloignées et que les Anglais avaient des pilotes à qui l’état des lieux était connu, je craignais qu’ils ne s’avisassent d’y venir et pour cela j’envoyai Vivonne concerter avec le duc de Beaufort les moyens de prévenir cette perte.\par
La seule chose qui me restait à faire avant que de commencer la guerre était d’aviser comment je la déclarerais. Car dans le dessein que j’avais toujours de la terminer à la première rencontre, j’étais bien aise d’agir avec les Anglais le plus honnêtement qu’il se pourrait.\par
Dans cette pensée, je trouvai moyen de faire que la reine d’Angleterre qui était alors à Paris, se chargeât elle-même de cette déclaration, pensant ne se charger que d’un compliment. Car je la priai seulement de témoigner au roi son fils que, dans l’estime singulière que j’avais pour lui, je ne pouvais sans chagrin prendre la résolution à laquelle je me trouvais obligé par l’engagement de ma parole. Et cela parut si honnête à cette princesse, que non seulement elle se chargea de lui en donner avis, mais elle crut même qu’il s’en devait tenir obligé.\par
Cependant, ayant donné tous les ordres nécessaires en de pareilles occasions, je voulus faire connaître cette résolution à mes sujets aussi bien qu’à mes ennemis et la fis publier en la forme ordinaire.\par
Mais pendant que je préparais mes armes contre l’Angleterre, je n’oubliais pas de travailler contre la maison d’Autriche par tous les moyens que la négociation me pouvait fournir. Comme je savais que la guerre de Portugal était une espèce de mal intestin dont la durée affaiblissait infiniment l’Espagne, je me proposai pour l’entretenir de marier le roi de Portugal avec Mademoiselle de Nemours et envoyai Saint-Romain auprès de ce prince pour en faire la proposition et pour éluder toutes celles que faisaient continuellement les Espagnols.\par
Du côté d’Allemagne, le comte Guillaume de Furstemberg avait ordre de travailler, de concert avec l’électeur de Cologne et avec le duc de Neubourg, à persuader l’électeur de Mayence, le duc de Brunswick et les autres princes voisins du Rhin à s’unir avec moi pour empêcher le passage des troupes de l’Empereur en Flandre. Et la raison que je leur fournissais pour cela était qu’il n’y avait pas d’autre moyen de maintenir la paix dans leurs pays ni d’en éloigner mes armées.\par
J’envoyai à même fin l’abbé de Gravel pour résider particulièrement auprès de l’électeur de Mayence et pour observer de près ses déportements, parce que je savais bien qu’ils n’étaient pas toujours fort sincères.\par
J’envoyai aussi dans le même temps Pomponne en Suède avec ordre d’y négocier tant pour les affaires de Pologne que pour celles des Pays-Bas. Car en quelque manière que ce fût, j’eusse été bien aise de former quelque liaison avec cet État, non pas tant pour me servir de leurs forces que pour les ôter à mes ennemis.\par
J’entretenais encore une secrète intelligence avec le comte de Serin pour faire naître quelque trouble en Hongrie, si j’entrais en guerre avec l’Empereur.\par
J’avais à ma cour un religieux théatin envoyé par la duchesse de Bavière, avec participation de son mari, pour me faire diverses propositions, et je n’étais pas tout à fait hors d’espérance de détacher un jour cet électeur d’avec la maison d’Autriche.\par
J’écoutai aussi ce qui me fut proposé par les électeurs de Mayence et de Cologne, touchant le partage qui se pouvait faire entre l’Empereur et moi des États du roi d’Espagne, en cas que ce jeune prince vînt à mourir. Car, quoique la chose parût en soi peu faisable, je voulais y laisser former toutes les difficultés par l’Empereur afin de faire tomber sur lui tout le chagrin des auteurs de la proposition.\par
Les ducs Georges-Guillaume et Jean-Frédéric de Brunswick étant tombés en quelques différends, je crus que je devais les mettre d’accord et y fis travailler par Delombre, qui revenait alors de Pologne en France.\par
Je fus bien aise aussi d’être pris pour arbitre avec la couronne de Suède entre l’électeur de Mayence et le Palatin, sur la contestation qu’ils avaient pour le droit de {\itshape vilfranc}, mais je ne voulus pas souffrir que, sur la diversité de nos avis, l’Empereur pût être nommé surarbitre, considérant qu’encore que cela n’eût été fait que par le rencontre particulier de l’affaire, on aurait pu l’expliquer autrement.\par
Pour engager l’électeur de Brandebourg à la défense des États de Hollande, je lui envoyai d’abord Dumoulin avec des propositions générales, et depuis même voulant traiter la chose plus précisément, j’avais résolu d’y faire passer d’Estrades, mon ambassadeur en Hollande, que je fus obligé de contremander par le refus que cet électeur fit de lui donner la main. Mais quoique la fierté de cet électeur m’eût été fort désagréable, je ne voulus pas qu’elle rompît un traité qui m’était avantageux et qui d’ailleurs était rempli d’assez d’autres difficultés, car j’avais à combattre dans cette cour et les persuasions de la douairière et la considération du prince d’Orange. Mais à qui se peut vaincre soi-même, il est peu de chose qui puisse résister. Je dépêchai pour cette négociation Colbert, maître des requêtes, et je vins à bout de ce que je désirais. L’électeur de Brandebourg s’obligea d’entretenir dix mille hommes à ses frais pour la défense des Provinces-Unies.\par
Duquel exemple vous pouvez apprendre, mon fils, combien il est essentiel aux princes d’être maîtres de leurs ressentiments. En des occasions de cette nature, que nous pouvons, selon notre choix, ou dissimuler ou relever, il ne faut pas tant appliquer notre esprit à considérer les circonstances du tort que nous avons reçu qu’à peser les conjonctures du temps où nous sommes. Lorsque nous nous aigrissons mal à propos, il arrive d’ordinaire qu’en pensant seulement faire dépit à celui qui nous a fâchés, nous nous faisons tort à nous-mêmes. Pour la vaine satisfaction que nous trouvons à faire éclater notre vain courroux, nous perdons souvent l’occasion de ménager de solides avantages. Cette chaleur qui nous transporte s’évanouit en fort peu de temps, mais les pertes qu’elle nous a causées demeurent pour toujours présentes à notre esprit, avec la douleur de nous les être attirées par notre faute.\par
Je sais mieux que personne combien les moindres choses qui touchent à notre dignité intéressent sensiblement les cœurs jaloux de leur gloire. Mais cependant la raison ne veut pas que l’on relève tout avec scrupule, et peut-être même qu’il est convenable à l’élévation où nous nous trouvons, de négliger quelquefois par de nobles motifs ce qui se passe au-dessous de nous.\par
Exerçant ici-bas une fonction toute divine, nous devons paraître incapables des agitations qui pourraient la ravaler. Ou s’il est vrai que notre cœur, ne pouvant démentir la faiblesse de sa nature, sente encore naître malgré lui ces vulgaires émotions, notre raison doit du moins les cacher sitôt qu’elles nuisent au bien public, pour qui seul nous sommes nés.\par
L’on n’arrive jamais à la fin des vastes aventures sans essuyer diverses difficultés, et s’il s’en trouve quelqu’une qui nous oblige à relâcher en apparence quelque chose de notre fierté, la beauté du succès que nous en attendons nous en console doucement en nous-mêmes et les effets éclatants qui s’en découvrent enfin, nous en excusent glorieusement dans le public.\par
Dès la fin de l’année dernière, j’avais projeté de faire deux cents cornettes de cavalerie, dont je ne délivrai pourtant mes commissions que pour six-vingt, depuis que je fus borné à la seule guerre d’Angleterre. J’avais aussi incorporé deux cents compagnies nouvelles d’infanterie dans les anciens régiments, afin que, se conformant insensiblement aux autres, le nombre de mes gens s’augmentât sans que la discipline s’affaiblît, car déjà j’étais persuadé que toute l’infanterie française n’avait pas été fort bonne jusqu’ici. Et pour la rendre meilleure je fis tomber une partie des charges de colonels entre les mains des jeunes gens de ma cour, à qui le désir de me plaire et l’émulation qu’ils avaient l’un pour l’autre pouvaient, ce me semblait, donner plus d’application.\par
Et pour ôter aux divers corps toute sorte de différends et de jalousie, je réglai premièrement les contestations qui étaient entre eux pour le rang, ce que l’on n’avait encore osé faire, et ensuite je résolus que chacun des régiments d’infanterie aurait vingt-quatre compagnies dans le service, pendant que les autres demeureraient dans les garnisons pour en être tirées chacune à leur tour.\par
J’avais dessein de traiter avec le duc de Brunswick pour ses troupes en cas qu’elles me fussent nécessaires. Mais à l’égard du duc de Lorraine, quoique je voulusse avoir les siennes, je ne crus pas devoir en parler directement, sachant qu’il me les eût voulu vendre cher et à condition de les entretenir toujours ensemble. Mais pour l’obliger à me les offrir lui-même sans condition, je lui fis dire que, suivant les termes de notre traité, je désirais qu’il les licenciât et ce moyen ne manqua pas de produire l’effet que j’en avais attendu.\par
Je fis lever aussi seize compagnies suisses pour entrer dans les garnisons à la place des françaises que j’en tirais.\par
Et pour faire que généralement tous ceux qui faisaient pour moi des levées s’en acquittassent comme ils devaient, je leur fis entendre de bonne heure que je verrais de quelle manière ils m’auraient servi et publiai que de mois en mois je ferais moi-même des revues.\par
La première était assignée au 19 janvier à Breteuil. Mais je fus empêché de m’y trouver par le pressentiment que l’amour me donna du danger de la Reine ma mère contre l’opinion des médecins.\par
Cet accident, quoique préparé par un mal de longue durée, ne laissa pas de me toucher si sensiblement qu’il me rendit plusieurs jours incapable de m’entretenir d’aucune autre considération que de la perte que je faisais.\par
Car quoique je vous aie dit incontinent qu’un prince doit sacrifier au bien de son empire tous ses mouvements particuliers, il est des rencontres où cette maxime ne se peut pratiquer au premier abord.\par
La nature avait formé les premiers nœuds qui m’unissaient à la Reine ma mère. Mais les liaisons qui se font dans le cœur par le rapport des qualités de l’âme, se rompent bien plus malaisément que celles qui ne sont produites que par le seul commerce du sang. La vigueur avec laquelle cette princesse avait soutenu ma couronne dans les temps où je ne pouvais encore agir, m’était une marque de son affection et de sa vertu. Et les respects que je lui rendais de ma part n’étaient point de simples devoirs de bienséance. Cette habitude que j’avais formée à ne faire qu’un même logis et qu’une même table avec elle, cette assiduité avec laquelle je la voyais plusieurs fois chaque jour malgré l’empressement de mes affaires, n’était point une loi que je me fusse imposée par raison d’État, mais une marque du plaisir que je prenais en sa compagnie. Car enfin l’abandonnement qu’elle avait si pleinement fait de l’autorité souveraine, m’avait assez fait connaître que je n’avais rien à craindre de son ambition pour ne me pas obliger à la retenir par des tendresses affectées.\par
Ne pouvant après ce malheur souffrir la vue du lieu où il m’était arrivé, je quittai Paris à l’heure même et je me retirai premièrement à Versailles (comme à l’endroit où je pourrais être plus en particulier) et quelques jours après à Saint-Germain.\par
Les lettres qu’il fallut écrire sur cet accident à tous les princes de l’Europe me coûtèrent plus qu’on ne saurait penser et principalement celles à l’Empereur, aux rois d’Espagne et d’Angleterre, que la bienséance et la parenté m’obligeaient de faire de ma main.\par
Je fis soigneusement exécuter ses dernières volontés, excepté en ce qu’elle avait ordonné qu’on ne fît point de cérémonies à ses obsèques. Car ne trouvant pas d’autre soulagement à l’ennui que me causait sa mort que dans les honneurs qui se rendaient à sa mémoire, je commandai qu’on suivît en cette rencontre tout ce qu’elle-même avait fait pratiquer à la mort du feu roi mon père.\par
Dans cette cérémonie, il se forma comme à l’ordinaire différentes contestations. Mais celle qui fut la plus agitée fut entre le clergé et le Parlement. La chose ayant été par moi jugée en faveur du clergé, le Parlement en vit l’exécution avec tant de chagrin dans l’église de Saint-Denis que pour s’épargner de recevoir la même mortification dans Notre-Dame, il me députa les Gens du Roi. Je les entendis à Versailles, et je remarquai que Talon qui portait la parole, avait peine à venir à la conclusion, parce qu’il jugeait bien qu’elle ne me serait pas agréable, étant chargé de me prier que le Parlement ne se trouvât point à la cérémonie. Mais quoiqu’en effet cette proposition me déplût, je ne laissai pas, avant que d’y répondre, de reprendre tous les points de son discours plus exactement que je n’avais pensé. Et ma raison fut qu’ayant déjà décidé quelques autres affaires contre les intentions de ce corps, il était bon de lui faire voir que je ne jugeais rien dont je ne fusse pleinement instruit, afin qu’il ne se fît pas l’honneur de croire que je prenais intérêt à le ravaler. Mais, revenant enfin à ce qu’il m’avait demandé de ne point aller à Notre-Dame, je dis positivement que je voulais qu’on s’y trouvât, et même qu’il n’y manquât personne. En quoi je fus obéi ponctuellement.\par
Dès le commencement de l’année, il s’était présenté une autre affaire où le même procédé m’avait réussi. Les édits que je venais de faire publier, et principalement celui qui regardait la modération du prix des charges, ne plaisant à pas un des officiers, les Enquêtes demandèrent l’assemblée des Chambres, dans laquelle ils s’étaient promis de pouvoir indirectement, sous quelque prétexte, rentrer en délibération sur ce sujet ; et je sus que le Premier Président, pensant me faire un grand service, pratiquait avec soin divers délais, comme si les assemblées des Chambres eussent encore eu quelque chose de dangereux.\par
Mais pour faire voir qu’en mon esprit elles passaient pour fort peu de chose, je lui ordonnai moi-même d’assembler le Parlement, de lui dire que je ne voulais plus que l’on parlât des édits vérifiés en ma présence, et de voir si l’on oserait me désobéir. Car enfin, je voulais me servir de cette rencontre pour faire un exemple éclatant ou de l’entier assujettissement de cette compagnie ou de ma juste sévérité à la punir.\par
Et, en effet, l’obéissance qu’elle témoigna en se séparant sans rien entreprendre, fut imitée bientôt après par les parlements les plus éloignés, et fit voir que ces sortes de corps ne sont fâcheux que pour ceux qui les redoutent.\par
Entre les occupations que me produisit la mort de la Reine ma mère, je ne vous ai point parlé du partage de ses biens, parce que ni moi ni mon frère n’y donnâmes pas grande application. Mais j’aurais peut-être dû vous faire le récit d’une conversation que j’eus avec lui dans le plus violent accès de notre commune douleur : en laquelle après de pressants témoignages de tendresse que nous nous donnâmes l’un à l’autre, je lui promis de faire passer la mienne jusqu’à ses enfants et de faire élever son fils auprès de vous.\par
Car, quoique le temps où je lui disais ces choses et l’état où j’étais en les lui disant ne laissassent aucun lieu de douter qu’elles ne me fussent suggérées par un pur mouvement d’amitié, il est pourtant certain que, quand j’aurais médité ce discours dans une pleine liberté d’esprit, je n’eusse pu rien penser de plus délicat que de faire à la fois à mon frère un honneur dont il m’était obligé, et de prendre pour sûreté de sa conduite le plus précieux gage qu’il m’en pût donner.\par
Je ne sais si ce fut cette marque de tendresse qui lui donna lieu peu de jours après de me demander que sa femme eût chez la Reine une chaire à dos. De ma part, j’aurais bien désiré de ne lui refuser jamais aucune chose. Mais voyant la conséquence de celle-ci, ce que je pus fut de lui faire entendre que pour tout ce qui servirait à l’élever au-dessus de mes autres sujets, je le ferais toujours avec joie, mais que je ne crois pas lui pouvoir accorder ce qui semblerait l’approcher de moi, lui faisant voir par raison l’égard que je devais avoir à mon rang, la nouveauté de sa prétention, et combien il lui serait inutile d’y persister.\par
Mais tout ce que je lui pus dire ne satisfit aucunement son esprit ni celui de ma sœur. Ils prétendirent même qu’en mourant la Reine ma mère m’avait fait cette demande, quoiqu’en effet elle ne m’en eût point parlé et qu’elle ne fût pas capable de la faire, ayant assez fait voir par ses actions combien chèrement ceux de notre rang doivent en conserver la dignité. Mais enfin mon frère prit dès lors avec moi une certaine conduite qui m’aurait fait craindre quelque chose de fâcheux, si d’ailleurs je n’avais eu connaissance de la trempe de son cœur et du mien.\par
Et néanmoins, un mois après, la mort imprévue du prince de Conti lui fournit un nouveau sujet de prétention pour le gouvernement du Languedoc, fondée principalement sur ce que mon oncle l’avait autrefois possédé.\par
Mais je ne crus pas encore lui devoir accorder ce point, étant persuadé qu’après les désordres que nous avons vus si souvent dans le royaume, c’était manquer de prévoyance et de raison que de mettre les grands gouvernements entre les mains des Fils de France, lesquels, pour le bien de l’État, ne doivent jamais avoir d’autre retraite que la cour, ni d’autre place de sûreté que dans le cœur de leur frère. L’exemple de mon oncle, que mon frère alléguait, était une confirmation de ma pensée ; et ce qui s’était fait durant ma minorité m’obligeait à prévoir avec plus de soin ce qui eût pu arriver durant la vôtre, si je vous eusse manqué en cet état. Cependant mon frère et ma sœur qui n’entraient pas dans ces raisonnements et qui étaient peut-être encore aigris par les discours de quelques brouillons, témoignaient en diverses manières qu’ils étaient mécontents de mon refus. Mais de ma part, sans faire semblant de rien apercevoir, je leur laissai le loisir de se reconnaître. Et en effet, revenant à eux bientôt après, ils me demandèrent tous deux pardon de la chaleur qu’ils avaient témoignée.\par
Durant ce temps-là, Saint-Romain était arrivé en Portugal, où, trouvant une négociation fort avancée entre cette couronne et celle d’Espagne par l’entremise de l’ambassadeur d’Angleterre, il ne put la rompre sans faire espérer aux Portugais qu’outre le secours que je leur fournissais sous le nom de M. de Turenne, je me mettrais bientôt en état de les assister ouvertement. D’où il arriva que, comme ces propositions leur étaient infiniment agréables, ils ne manquèrent pas depuis d’en presser incessamment l’exécution, soit par la voie de Saint-Romain même ou par celle de l’ambassadeur anglais que j’avais alors à ma cour. Mais ne trouvant pas les choses disposées à leur donner sitôt cette satisfaction, je travaillais de jour en jour à les entretenir d’espérances sans vouloir rompre avec le temps les mesures que j’avais prises.\par
J’avais une autre affaire du côté du Nord qui n’était pas plus facile à démêler. Le roi de Danemark, alarmé par l’appareil que faisaient contre lui les Suédois, m’avait envoyé Annibal Chestet, son grand trésorier, pour me presser de me déclarer en sa faveur ; et les États de Hollande, qui étaient bien aise de prendre une si belle occasion pour me faire rompre avec la Suède, me faisaient de continuelles instances en faveur du roi de Danemark ; pendant que d’autre part les Suédois me faisaient aussi remontrer par Pomponne, que, voyant tous leurs voisins armés, ils ne pouvaient pas être seuls sans armes et que même ils avaient des raisons qui les obligeaient à faire la guerre au roi de Danemark, en cas qu’il attaquât celui d’Angleterre, me priant de ne pas croire qu’en cela ils eussent dessein de rien faire contre mes intérêts.\par
La conjoncture était assurément délicate. Car de laisser aux Suédois la liberté d’attaquer le roi de Danemark, c’était me priver de tout l’avantage que je m’étais promis en traitant avec lui. Mais de me déclarer pour cela contre la Suède, c’était aussi trop légèrement rompre avec une nation dont j’espérais me pouvoir bientôt servir dans une rencontre plus importante.\par
C’est pourquoi sans accorder alors précisément à l’une ni à l’autre des parties ce qu’elle désirait de moi, je m’appliquai à chercher des voies de milieu et y réussis de telle sorte qu’en peu de temps, je tirai assurance des Suédois qu’ils n’attaqueraient point le Danemark.\par
Sur lequel propos, on pourrait assez raisonnablement mettre en question s’il ne faut pas autant de force au prince pour se défendre des prétentions différentes de ses alliés, de ses sujets ou même de sa propre famille, que pour résister aux attaques de ses ennemis.\par
Et en effet, qui considérerait à combien de désirs, d’importunités et de murmures les rois sont continuellement exposés, s’étonnerait moins d’en voir quelques-uns se troubler dans un bruit si tumultueux, et trouverait plus digne d’estime ceux qui, dans ces agitations du dehors, gardent au-dedans le calme nécessaire pour la parfaite économie de la raison.\par
Il faut de la force assurément pour tenir toujours la balance droite, entre tant de gens qui font leurs efforts pour la faire pencher de leur côté. De tant de voisins qui nous environnent, de tant de sujets qui nous obéissent, de tant d’hommes qui nous font la cour, de tant de ministres et de serviteurs qui nous servent ou qui nous conseillent, il n’en est presque pas un qui n’ait dans l’esprit une prétention formée, et chacun d’eux s’appliquant tout entier à donner à ce qu’il veut l’apparence de justice, il n’est pas aisé que le prince seul, partagé par tant d’autres pensées, fasse toujours le parfait discernement du bon d’avec le mauvais.\par
Il serait difficile de vous fournir pour cela des règles certaines dans la diversité des sujets qui se présentent tous les jours. Mais il y a pourtant certaines maximes générales, dont il est bon que vous soyez informé.\par
La première est que quand vous auriez pour tous une complaisance universelle, vous ne pourriez pourtant satisfaire à tout, parce que la même chose qui contente l’un en fâche toujours plusieurs autres.\par
La deuxième, qu’il ne faut pas juger de l’équité d’une prétention par l’empressement avec lequel on l’appuie, parce que la passion et l’intérêt ont naturellement plus d’impétuosité que la raison.\par
La troisième, que ceux qui vous touchent de plus près, ou ceux même de qui vous prenez les avis sur les prétentions des autres, sont ceux sur les prétentions desquels vous devez le plus vous consulter vous-même ou prendre les conseils de gens qui ne soient pas en même degré qu’eux, de peur que, prenant le sentiment de l’un sur l’affaire de l’autre (quoique d’ailleurs ils ne fussent pas amis), ils ne se favorisassent réciproquement, dans la pensée que la grâce reçue par leur compagnon ferait exemple pour eux-mêmes.\par
Et enfin la quatrième est qu’il faut toujours considérer les suites de la chose prétendue, plutôt que le mérite de celui qui prétend, parce que le bien public se doit préférer à la satisfaction des particuliers, et qu’il n’y a point de roi si puissant au monde qui ne ruinât bientôt son État s’il s’était résolu de tout accorder seulement aux gens de mérite.\par
Je sais que l’on fâche toujours ceux à qui l’on refuse, et que plusieurs imputent à la mauvaise humeur ou au mauvais goût du souverain tout ce qui se trouve de difficulté dans leur demande. Il est même certain que l’on se fait toujours peine à soi-même en rejetant la prière des autres, et qu’il est naturellement plus doux de s’attirer des remerciements que des plaintes. Mais en cet endroit, mon fils, nous sommes obligés de nous sacrifier au bien général, et ce qu’il y a de plus fâcheux en ce sacrifice, c’est qu’encore qu’il nous coûte beaucoup, il est d’ordinaire fort peu prisé.\par
Car enfin la plupart de ceux qui distribuent les louanges aux princes ne prisent entre les vertus que celles qui leur sont utiles. Les beaux esprits de profession n’ont pas toujours de fort belles âmes et parmi les belles choses qu’ils débitent dans le public, ils se dépouillent rarement du soin de leurs intérêts particuliers.\par
Tandis que je m’engageais de jour en jour à la défense des États de Hollande, j’appris qu’ils avaient encore leur ambassadeur à Londres, et soupçonnant avec raison leur bonne foi, je voulus qu’ils le rappelassent et qu’ils me promissent formellement de ne rien négocier sans ma participation.\par
Pour m’assurer même plus fortement de leur foi, je travaillai à m’acquérir du crédit dans leurs délibérations et à éloigner des magistratures les partisans de la maison d’Orange, toujours liés avec les Anglais. Et dans ce dessein, je fis distribuer des pensions à plusieurs députés.\par
Cependant, les troupes que j’avais fait passer en Allemagne pour les secourir (quoique mal satisfaites du traitement qu’elles y recevaient) y vivaient pourtant avec toute la discipline possible et y firent diverses actions de conduite et de vigueur. L’Espagne même, quoique auparavant elle n’eût point de part dans cette querelle, éprouva leur valeur à son dommage. Car le dessein secret qu’elle a toujours eu de s’opposer aux intentions de la France, lui faisait embrasser les intérêts de l’évêque de Münster ; elle lui voulut donner lieu de surprendre deux places, Dalem et Villemstat, qui étaient voisines de leurs frontières et afin que ce prince n’alarmât point les Hollandais en envoyant des troupes de ce côté-là, elle trouva un expédient pour en fournir elle-même sur les lieux, qui fut de licencier deux régiments de leurs garnisons les plus proches, lesquels furent aussitôt pris à la solde de l’évêque. Mais toute cette trame fut éludée par la vigilance de mes chefs, qui en ayant eu nouvelle, surprirent et défirent les deux régiments espagnols.\par
Plusieurs actions de cette même nature qu’il serait ennuyeux de raconter ici, obligèrent les ennemis à demeurer chez eux et leur ôtant la liberté de la campagne, leur causèrent de si grandes incommodités que l’évêque de Münster m’envoya proposer la paix. Mais je ne voulus écouter aucune proposition et renvoyai absolument aux Hollandais la négociation de l’affaire, parce qu’ils y étaient les plus intéressés. J’en usai encore de la même manière avec eux touchant l’Angleterre. Car l’ambassadeur que le roi de Portugal entretenait auprès du roi de la Grande-Bretagne, étant venu pour d’autres affaires à ma cour, y fit diverses ouvertures d’accommodement auxquelles je ne voulus jamais entendre sans la participation des États.\par
Et en cela, je puis dire que j’eus pour eux une fidélité qu’ils n’auraient peut-être pas eue pour moi en pareille occasion. Car il est sans doute que, dans les desseins que j’avais formés, l’un et l’autre de ces accommodements m’était très avantageux, parce qu’en faisant la paix avec l’Anglais je pouvais épargner toutes les dépenses que me produisait l’armement de mer, pour entretenir de plus grandes forces par terre ; et en m’accommodant avec l’évêque de Münster, non seulement je retirais les forces que j’avais envoyées contre lui, mais je m’assurais des siennes propres que ce prince m’offrait avec empressement.\par
Cependant sur l’avis que j’eus que les Anglais étaient entrés dans la mer Méditerranée, je donnai ordre au duc de Beaufort, mon amiral, de se disposer au plus tôt à les en aller chasser. Mais dans la multiplicité des choses dont il est besoin pour l’équipement des vaisseaux, il se consomma quelques mois.\par
Ce retardement ne manqua pas de donner aux Anglais une grande présomption de leur force et de leur faire témoigner une grande impatience de voir mes gens en mer pour les combattre. Mais dès lors que le duc de Beaufort fut sorti, on ne les vit plus si vaillants et ils trouvèrent plus à propos de sortir de cette mer que de l’attendre.\par
J’eusse pu me contenter de les avoir chassés de là, après les bravades qu’ils y avaient faites. Mais je crus qu’il serait encore plus beau de les suivre dans la mer Océane. Ce qui fut exécuté à la vue de toute leur côte avec trente-deux vaisseaux seulement, sans que les Anglais, renfermés dans leurs ports, eussent l’audace d’en sortir, jusqu’à ce qu’ils eurent mis en état toutes leurs forces.\par
Elles se trouvèrent enfin assemblées au mois de juin et parurent en mer au nombre de 46 vaisseaux. Mais alors mes vaisseaux étaient fort éloignés de là, parce que j’avais été obligé de les envoyer en Portugal par des raisons que vous verrez dans la suite.\par
Les Hollandais levèrent l’ancre en même temps que leurs ennemis et se disposèrent même à donner bataille contre mon sentiment. Car mes vaisseaux étant éloignés, je leur fis remontrer qu’ils en devaient attendre le retour par deux raisons très importantes : la première qu’après notre jonction, nous aurions tant d’avantage sur les Anglais que la victoire nous serait infaillible ; et la seconde qu’il ne serait pas même nécessaire de combattre les ennemis pour les défaire, parce que leur laissant seulement consumer leurs provisions (lesquelles ne pouvaient pas beaucoup durer), ils seraient obligés de rentrer dans leurs ports, sans espérance de reparaître de longtemps à cause des longueurs et des difficultés qui se rencontraient à faire des levées sur des peuples mal obéissants ; et qu’ainsi ce n’était pas agir prudemment de mettre au hasard une victoire que le temps nous donnerait avec sûreté.\par
Cependant ces avis furent inutiles ; car les Hollandais, piqués du désir de vaincre sans mon secours, donnèrent le combat et le gagnèrent ; mais ils durent, suivant toutes les apparences, attribuer ce succès à l’excessive confiance de leurs ennemis plutôt qu’à leurs propres forces. Car en effet, leur flotte parut si méprisable aux Anglais qu’ils crurent n’avoir besoin pour s’en assurer la victoire que de l’engager au combat. Dans cette pensée, le prince Robert, avec vingt-trois des meilleurs vaisseaux, se détacha du gros de l’armée sous prétexte de venir au-devant des miens, mais en effet pour donner courage aux Hollandais de demeurer ferme devant ce qui leur restait d’ennemis en tête. Toute l’île attendait avidement la nouvelle de la bataille comme d’une victoire certaine. Les chefs l’avaient positivement promise, le Roi s’en tenait assuré, les peuples en triomphaient par avance. Mais le succès fut tout contraire à ces prétentions ; et après un combat de quatre jours, j’eus enfin la satisfaction de voir la fortune choisir le même parti que j’avais pris. Je ne vous dirai pas combien d’hommes et de vaisseaux furent perdus de part et d’autre : ce sont des circonstances que vous apprendrez de tous les autres écrivains, et qui sont de peu d’utilité. Mais, m’arrêtant à ce qui vous peut servir, je vous remarquerai que, de la part des Anglais, la vanité qu’ils avaient eue de se vanter trop tôt d’une victoire qu’ils n’obtinrent pas, les engagea dans un procédé tout à fait ridicule, et les obligea à faire par toute leur île des feux de joie pour leur propre défaite, comme s’ils eussent en effet vaincu : feux qui, dans leur allégresse impertinente, ne découvrirent que trop clairement à toute la terre quelle était la mauvaise disposition de cet État, dans lequel, pour conserver un peu d’autorité, le prince était contraint à se réjouir de ses propres pertes et à tenir ses sujets dans l’erreur, pour les empêcher de tomber dans la rébellion.\par
Et à l’égard des Hollandais, je vous ferai observer qu’encore que cette entreprise leur ait réussi, l’on ne doit pas conclure qu’ils aient eu raison de la faire, parce que, pour juger sainement des conseils, il ne faut pas toujours s’arrêter aux événements, qui, selon qu’il plaît au Dieu des armées, sont tantôt heureux et tantôt malheureux, mais qu’il faut se servir des lumières qu’il nous a données pour faire en toutes occasions ce qui est le plus conforme à la raison. Et sans chercher plus loin la confirmation de ce raisonnement, vous verrez dans cette même année les mêmes flottes combattant sur les mêmes principes avoir un succès tout différent.\par
Pour savoir ce qui retenait mes vaisseaux durant ce temps-là, il est besoin de vous reprendre le mariage de la reine de Portugal à l’endroit où je l’avais laissé l’année dernière.\par
Après divers empêchements apportés par les Espagnols, cette affaire fut enfin absolument résolue. Mais ceux mêmes qui avaient tâché d’en empêcher la résolution, employèrent toutes sortes d’artifices pour en éloigner encore l’exécution. Elle fut diverses fois différée.\par
Cependant je m’étais engagé à prêter huit de mes meilleurs vaisseaux pour le passage de cette princesse, lesquels je tenais toujours prêts pour cet effet. En sorte que, quand je fis passer le duc de Beaufort dans l’Océan pour y poursuivre les Anglais, quoique ses forces fussent peu considérables pour cette entreprise, je ne voulus pas qu’il se servît de ces huit vaisseaux, parce que l’ambassadeur de Portugal me faisait entendre qu’il attendait de jour en jour un ordre pour partir. Enfin l’affaire fut arrêtée pour le mois de juin et pour éviter certaines difficultés qui pouvaient naître touchant les cérémonies, l’on résolut que le mariage ne se ferait que sur mes vaisseaux.\par
Mais il me demeurait dans l’esprit une difficulté importante touchant ces vaisseaux mêmes, lesquels par ce voyage étaient exposés en prise et aux Anglais et aux Espagnols : parce que les Anglais, principalement dans le retour, pouvaient charger mes vaisseaux comme ennemis ; et les Espagnols, soit qu’ils les considérassent comme portant la reine de Portugal, les pouvaient attaquer par le droit de la guerre, et soit qu’ils les regardassent comme étant à moi, pouvaient s’en saisir en vertu d’un traité particulier par lequel nous étions convenus que tous les vaisseaux français, trouvés à cinquante milles du Portugal, seraient de bonne prise.\par
À l’égard des Anglais, l’expédient fut plus facile à trouver. Car je fis obtenir un passeport par la reine de Portugal, par lequel non seulement j’assurais mes vaisseaux, mais j’aurais pu même, si j’eusse été d’humeur à m’en prévaloir de mauvaise foi, en tirer un avantage considérable, parce que dans le retour de mes gens, ils pouvaient ne faire paraître leur passeport qu’en cas qu’ils se trouvassent les plus faibles, et charger les Anglais toutes les fois qu’ils se trouveraient plus forts qu’eux.\par
Mais à l’égard des Espagnols, l’affaire se trouvait plus difficile et ne doutant pas que, dans le puissant intérêt qu’ils avaient de traverser ce mariage, ils ne fissent leurs derniers efforts en cette occasion, je ne pus trouver d’autre expédient pour mettre les choses en sûreté que d’envoyer ma flotte à l’embouchure de la rivière de Lisbonne pour y attendre et l’arrivée et le retour de mes vaisseaux.\par
Durant que toutes ces choses se passaient, les Espagnols continuaient sans relâche leurs négociations pour l’accommodement du Portugal ; et j’employais aussi tous les moyens possibles pour l’empêcher tantôt vers le roi de Portugal, tantôt vers celui de la Grande-Bretagne et tantôt vers les Espagnols même.\par
Entre autres, une fois cette affaire toute prête à se conclure, je m’avisai, pour gagner quelques jours, de proposer à la reine d’Espagne de me recevoir pour médiateur. Mais contre mon espérance, la proposition fut acceptée, et l’archevêque d’Embrun, instruit par moi des couleurs qu’il lui pouvait donner pour la rendre plausible, obtint le consentement de cette princesse. Il est vrai que depuis, elle ne s’en voulut pas souvenir et que, reconnaissant après coup combien légèrement elle s’était engagée, elle se vit réduite à la nécessité de se désavouer elle-même.\par
Tout le monde convient, mon fils, qu’il n’y a rien de plus malhonnête que de se dédire de ce que l’on avait avancé. Mais vous devez savoir que le seul moyen de tenir inviolablement sa parole est de ne la jamais donner sans y avoir mûrement pensé.\par
L’imprudence attire presque toujours à sa suite le repentir et la mauvaise foi ; et tout homme qui peut s’engager sans raison, devient en peu de temps capable de se rétracter sans honte.\par
Délibérer à loisir sur toutes les choses importantes, et en prendre conseil de différentes gens, n’est pas, comme les sots se l’imaginent, un témoignage de faiblesse ou de dépendance, mais plutôt de prudence et de solidité. C’est une maxime surprenante, mais véritable pourtant, que ceux qui, pour se montrer plus maîtres de leur propre conduite, ne veulent prendre conseil en rien de ce qu’ils font, ne font presque jamais rien de ce qu’ils veulent. Et la raison en est que, dès lors qu’ils mettent au jour leurs résolutions mal digérées, ils y trouvent de si grands obstacles, et on leur y fait remarquer tant d’absurdités, qu’ils sont contraints de les rétracter eux-mêmes, s’acquérant ainsi justement la réputation de faiblesse et d’incapacité, par les mêmes voies par lesquelles ils s’étaient promis de s’en garantir.\par
Les conseils qui nous sont donnés, ne nous engagent à les suivre qu’en tant qu’ils nous paraissent raisonnables et loin de diminuer l’esprit de notre propre capacité, ils la relèvent plus assurément que toute autre chose, parce que tous les gens de bon sens sont d’accord que tout ce qui se fait ou se propose de bon dans l’administration de l’État, se doit rapporter principalement au prince, et qu’il n’y a rien qui fasse mieux voir son habileté que lorsqu’il sait se faire bien servir et bien conseiller par ses principaux ministres.\par
Il y a cette différence entre le sage monarque et le mal avisé, que ce dernier sera presque toujours mal servi par ceux même qui passent pour les plus honnêtes gens dans le monde, au lieu que l’autre saura très souvent tirer de bons services et de bons avis de ceux même de qui l’intégrité pourrait être la plus suspecte.\par
Car enfin, dans tout ce qui regarde la conduite des hommes, on peut établir pour un principe général que tous ont une pente secrète vers leur avantage particulier, et que la vertu des plus honnêtes gens est malaisément capable de les défendre de ce mouvement naturel, si elle n’est quelquefois soutenue par la crainte ou par l’espérance. Et quand même il se rencontrerait quelqu’un d’exempt de cette règle générale, c’est un bonheur tellement singulier que la prudence ne permet pas que l’on s’assure jamais pleinement de l’avoir trouvé.\par
Ainsi considérant les choses suivant leur cours ordinaire, dans lequel les hommes fuient le mal et recherchent le bien suivant qu’ils craignent ou qu’ils espèrent, il est certain que le prince malavisé qui ne sait pas faire jouer ces grands ressorts, et qui écoute et traite également tous ceux qui travaillent dans ses affaires, laisse quasi nécessairement corrompre auprès de lui ceux même qui s’y étaient mis avec les meilleures intentions du monde, parce que, comme rien ne les excite et ne les retient, ils se relâchent ou s’emportent insensiblement, suivant que leur humeur ou leur intérêt le désire, sans faire presque jamais aucune réflexion sur leur propre conduite.\par
Au lieu qu’auprès du prince intelligent, les plus avides et les plus intéressés n’osent s’éloigner tant soit peu du chemin qu’ils doivent tenir, parce qu’ils le voient toujours veillant sur leurs démarches, et qu’au moindre égarement, ils craignent de perdre son estime et sa créance, qui fait toujours leur premier intérêt. L’ambition de lui plaire les oblige à veiller sans cesse sur eux-mêmes. Ils ne se permettent rien, parce qu’ils savent qu’aucun mal ne lui sera caché ; ils ne se ménagent sur rien, parce qu’ils sont persuadés qu’aucun mérite ne manque de trouver auprès de lui l’agrément qui lui est dû. Et pour dire en un mot, ils font et conseillent toujours ce qu’ils estiment de mieux, parce qu’ils sont persuadés que la faveur, le crédit et l’élévation où ils aspirent, ne se donnent qu’à proportion du zèle et de la fidélité que chacun témoigne.\par
Pour conserver l’ancien usage pratiqué par mes prédécesseurs, de tenir un ambassadeur en Turquie, j’avais choisi pour aller en cette cour, La Haye, fils du précédent ambassadeur, lequel connaissait sans doute mieux que tout autre les manières de cette nation avec laquelle il avait traité longtemps sous son père.\par
Mais une haine particulière qui s’était contractée entre lui et le Grand-Vizir qui était alors, rendit la présence de ce ministre fort dommageable à mes affaires. Car il était arrivé que le Grand-Vizir, laissant sa place à son fils, lui avait laissé aussi toute sa haine : en sorte que La Haye fils, arrivant sur les lieux, y trouva pour capital ennemi celui-là même avec lequel il avait à négocier toutes choses.\par
Il en reconnut l’effet dès son arrivée, par le refus opiniâtre qu’on fit de lui accorder les mêmes traitements que venait tout nouvellement de recevoir un ambassadeur de l’Empereur. Car outre que le Grand-Vizir est presque maître absolu de toutes choses, il avait encore pris soin de prévenir le Grand-Seigneur contre La Haye, en lui faisant entendre que c’était par l’instigation de ce ministre que j’avais permis aux corsaires français d’interrompre le commerce de l’Archipel par les courses qu’ils y avaient faites.\par
La Haye, piqué de ce refus et s’en voulant éclaircir dans sa première audience, n’en reçut aucune satisfaction, et eut tant de chagrin de voir que sa querelle particulière faisait préjudice à mes intérêts, que non seulement il protesta de se retirer à ma cour, mais il jeta brusquement les capitulations qu’il tenait roulées, du côté du Grand-Vizir, lequel prétendit en avoir été frappé.\par
Ceux qui savent comme ce Premier ministre fait seul à la vue du public toutes les fonctions souveraines, et qu’il reçoit aussi dans le pays tous les honneurs souverains, ne douteront pas qu’il ne se tînt terriblement offensé par cette action, tant par la considération de son prince, dont on avait jeté le sceau par terre, qu’en sa propre personne sur laquelle ce coup avait porté.\par
Aussi fit-il retenir La Haye dans le palais même où l’action s’était passée, comme dans une espèce de prison. Mais ensuite, faisant réflexion sur le caractère d’ambassadeur dont cet homme était honoré, et sur le ressentiment que je pourrais avoir du traitement qu’il lui ferait, il témoigna quelque chagrin d’être engagé dans cette affaire. Et en effet, déjà le peuple de Constantinople disait ouvertement qu’il avait tort, et ceux de la Porte qui n’étaient pas dans ses intérêts, disaient que c’était une mauvaise politique d’avoir voulu, pour des démêlés particuliers, s’attirer un ennemi tel que moi.\par
Ces considérations le portèrent à chercher diverses voies d’accommodement, et la première qu’il essaya fut de tenter le marquis de Guitry, maître de ma garde-robe, qui était allé à Constantinople par pure curiosité, de se charger de l’ambassade, lui offrant tous les bons traitements qu’il pouvait désirer ; mais parce qu’il ne le trouva pas disposé à se charger sans ordre d’une pareille affaire, il pria le Premier Bacha, l’un de ses plus proches parents, d’adoucir l’esprit de La Haye par des paroles d’honnêteté, après lesquelles il lui rendit lui-même toutes les honnêtetés qu’il lui avait d’abord contestées, et d’ailleurs me fit écrire, pour excuser son procédé, qu’il n’avait arrêté La Haye dans son palais, que par la crainte qu’il avait eue que ce ministre, emporté par son chagrin, ne se retirât sur l’heure de la Porte et ne me vînt rapporter autre chose que la vérité. Ce qui fut sans doute une réparation fort remarquable eu égard à la manière d’agir de cette nation, qui se relâche très rarement dans les choses qu’elle a une fois entreprises.\par
Il s’était passé une autre affaire en cette cour, de laquelle je n’étais pas satisfait. Car, les Génois, corrompant l’ancien usage de toute la chrétienté qui n’avait eu commerce avec les Turcs que sous la bannière de France, avaient envoyé de leur chef des ambassadeurs à la Porte et prétendaient y trafiquer sous leur propre bannière. J’avais chargé La Haye de s’en plaindre en mon nom ; mais la mauvaise disposition où cette cour était pour lui m’empêcha d’en espérer la satisfaction que j’aurais désirée. Je me résolus à la tirer des Génois mêmes, lorsque j’aurais le loisir de traiter sérieusement cette affaire avec eux.\par
J’avais aussi eu affaire, cette même année, avec la barbare nation des Iroquois en Canada. Mais les choses y avaient réussi plus heureusement pour mon service. Car les troupes que j’y avais fait passer, firent en cette seule campagne trois marches différentes, de plus de trois cent lieues chacune, par des lieux absolument inhabités, où il fallait aller tout le jour sur la neige et coucher à découvert toutes les nuits. En sorte que les barbares, surpris dans leurs propres habitations, les virent désoler sans remède et se trouvant privés de leurs grains et même de plusieurs de leurs enfants, que mes gens enlevèrent à leur vue, ils s’humilièrent devant eux avec toute la soumission possible, et consentirent à des conditions de paix qui, dans la vraisemblance, assureront pour toujours le repos des colonies françaises.\par
Les habitants de Tunis, fatigués des continuelles alarmes que leur donnaient mes vaisseaux, désirèrent de faire la paix avec moi, et ceux d’Alger, passant plus avant, m’offraient encore de me servir contre l’Angleterre. Je ne voulus pas accepter cette dernière proposition. Mais au surplus, touché du désir de procurer la liberté à tant d’esclaves chrétiens que ces barbares tenaient en leurs fers, et de donner moyen à tous les fidèles de trafiquer sûrement sous la bannière de France, je fis partir Dumoulin, qui, en peu de temps, termina l’un et l’autre traité avec des conditions plus avantageuses qu’aucun autre prince de l’Europe en eût jamais obtenues de ces nations.\par
Il arriva néanmoins un incident au traité de Tunis, qui en rendit la conclusion difficile. Car le roi, avec qui les articles avaient été concertés, ayant été emprisonné par une sédition, ses ennemis, qui avaient toute l’autorité dans l’administration de la république, auraient sans doute rompu cet accord, comme ils firent tous les autres actes faits durant le règne de ce prince, si la terreur de mes armes ne les eût obligés à l’accomplir ; et le premier fruit que j’en recueillis fut de voir plus de trois mille esclaves français retirés des mains de ces infidèles.\par
J’étais dès lors si considéré sur la mer Méditerranée, que les Espagnols, ayant à faire passer l’Impératrice en Italie, n’osèrent l’entreprendre sans me demander passeport, lequel je leur accordai incontinent avec toute l’honnêteté dont je pus l’accompagner. Car je donnai ordre en même temps sur toutes mes côtes qu’en cas que l’Impératrice fût obligée d’y aborder, on la traitât avec les mêmes respects que l’on m’eût pu rendre à moi-même. Ces ordres furent pourtant sans effet à l’égard de l’Impératrice, qui, ayant eu le temps favorable, arriva sans danger en Italie. Mais ils furent fort utiles à sept galères espagnoles, qui, ayant un jour rencontré Vivonne, commandant les miennes, refusèrent de baisser l’étendard au commandement qu’il leur en fit. Car Vivonne qui était le plus fort et en nombre et en équipage, les pouvant prendre sans combat, les renvoya par ce seulement qu’elles étaient chargées des hardes de l’Impératrice.\par
Parmi les réjouissances qui se firent à Vienne à l’occasion de ses noces, l’on trouva mauvais que le chevalier de Gremonville, mon résident, n’eût pas quitté le deuil qu’il portait pour la mort de la Reine ma mère. Mais dès lors que j’en eus avis, quoique ce mariage ne fût pas une fête fort agréable pour moi, je donnai ordre à Gremonville de quitter son deuil, et lui envoyai même quelque argent pour augmenter, en cette occasion, sa dépense ordinaire.\par
Cependant je négociais en cette cour une affaire importante pour le duc d’Enghien. Car quoique la reine de Pologne lui eût cédé, par contrats de mariage, les duchés d’Oppeln et de Ratibor et qu’on ne lui pût pas contester ces seigneuries, parce qu’elles avaient été constamment engagées par l’empereur Ferdinand à Sigismond-Casimir, l’Empereur n’avait pourtant voulu jusqu’ici ni lui en donner l’investiture ni lui rembourser les sommes pour lesquelles elles avaient été engagées. Mais dès lors que l’affaire se traita sous mon nom, elle fut heureusement terminée ; et pour ces deux terres qui ne produisaient au duc d’Enghien que trente-cinq mille livres de revenu, l’Empereur lui fit payer en France seize cent mille livres, sans compter cent mille écus qui furent laissés à son surintendant pour nous avoir si favorablement traités et si mal servi son maître : en sorte qu’il en coûta près de deux millions de livres à ce prince.\par
Entre les ministres corrompus, il s’en trouve fort peu d’assez hardis pour mettre ouvertement la main dans la bourse de leur maître et pour s’approprier directement le bien dont il leur a donné la direction, parce que ce serait un crime dont ils seraient trop facilement convaincus. Mais la manière de voler qu’ils trouvent la plus commode et qu’ils croient la plus assurée contre la recherche des temps à venir, c’est de prendre sous le nom d’autrui ce dont ils ont dessein de profiter eux-mêmes. Les adresses qu’ils pratiquent en cela sont de tant d’espèces différentes que je n’entreprendrai pas de les expliquer par le menu. Mais je vous dirai seulement qu’elles ont toutes cela de commun qu’elles augmentent toujours le vol, qu’elles ont entrepris de cacher.\par
Car enfin il est sans doute que le particulier de qui le ministre se veut servir pour prendre ces sortes de profits, n’entrerait point dans ce commerce à moins d’y trouver quelque avantage de sa part, et il faut (sous quelque forme que ce puisse être) que le prince aux dépens de qui se fait le traité, porte en même temps sur ses coffres et le profit injuste que son ministre veut tirer et le gain qu’il fait faire encore à celui qui lui fournit le prétexte de ce vol.\par
Mais il est certain de plus que, de toutes ces conventions frauduleuses, il n’y en a point qui porte tant de préjudices au prince qui les souffre que celles qui se traitent avec des étrangers, non seulement parce que la perte qu’il y fait sort absolument de son État, mais parce qu’elle ruine encore sa réputation chez ses voisins, qui par de semblables épreuves ne connaissent que trop clairement le peu de soin ou le peu d’intelligence qu’il a de ses affaires.\par
Et cette seule considération devrait à mon avis donner plus de retenue aux infidèles serviteurs qui font de semblables marchés. Mais du moins devrait-elle apprendre aux maîtres à ne se pas contenter d’examiner les hommes avant que de les mettre dans l’emploi, parce que la plupart se déguisent aisément pour un temps, dans la passion de parvenir à l’autorité qu’ils se proposent, mais à les observer encore plus soigneusement lorsqu’ils sont actuellement dans le maniement des affaires, parce qu’alors étant en possession de ce qu’ils désirent, ils suivent souvent avec plus de liberté leurs mauvaises inclinaisons, dont l’effet retombe toujours ou sur les affaires ou sur la réputation de leurs princes.\par
Car enfin cette observation continuelle fera que le prince reconnaissant au vrai le faible de tous ceux qui le servent, pourra selon la diversité des sujets, ou les en corriger par ses bons avis, ou les éloigner quand ils seront incorrigibles, ou même, s’ils ont d’ailleurs des qualités qui méritent qu’on les supporte, se garantir du préjudice que leurs défauts pourraient apporter à ses affaires, en s’appliquant à distinguer, dans ce qu’ils font ou dans ce qu’ils proposent, ce qui peut être du bien de son service d’avec ce qui est de leur mauvaise inclination.\par
Je mêlais continuellement les soins du dedans à ceux du dehors. Comme j’apprenais d’Angleterre et d’Allemagne que la peste y continuait, je m’appliquai à rechercher toutes les précautions possibles pour empêcher qu’elle ne se communiquât dans mes États et, pour faire pratiquer plus exactement mes ordres, j’envoyai des commissaires exprès sur les frontières les plus suspectes.\par
Mais toute ma diligence ne put empêcher que par le continuel commerce qui se fait d’un pays à l’autre, Gravelines et Dunkerque ne fussent infectées. Dans ce malheur, je soulageai de mes soins et de ma bourse les pauvres familles affligées et me servis de tous les expédients dont je me pus aviser pour faire que le mal ne s’étendît pas aux places voisines. Celui de tous que je crus le plus efficace, ce fut d’augmenter considérablement la paye des garnisons infectées de la peste : parce que sans cela les soldats, qui d’ailleurs n’étaient pas retenus comme les habitants par la considération de leurs biens et de leurs familles, se seraient insensiblement écoulés et prenant parti dans mes autres troupes, y auraient porté le mauvais air.\par
Mais ce qui semblait demander plus de précaution fut le retour des troupes que Pradelle me ramenait alors d’Allemagne. Aussi lui prescrivis-je si précisément tout ce qu’il devait pratiquer dans sa marche pour ne nous point apporter de mal, qu’en effet le dedans du royaume n’en reçut aucun préjudice.\par
Je ne laissais naître aucune querelle entre les gens de considération que je n’accommodasse aussitôt ; et j’avançais autant que je pouvais le règlement général que j’avais résolu de faire pour l’abréviation et le retranchement des procès, résolu de donner bientôt aux vœux du public une bonne partie de ce travail.\par
Sur l’avis que j’eus qu’en plusieurs provinces, le peuple était tourmenté par certaines gens qui abusaient du nom des gouverneurs pour faire des exactions injustes, j’établis de toutes parts des hommes exprès pour être plus sûrement informé de ces concussions, afin de les punir ensuite comme elles méritaient.\par
Je fis même alors dans ma maison un changement, où toute la noblesse du royaume avait intérêt. Ce fut à l’égard de ma grande écurie, dans laquelle j’augmentai de plus de moitié le nombre des pages, et pris soin qu’ils fussent et mieux choisis et mieux instruits qu’ils n’avaient été jusque-là.\par
Car je savais que ce qui avait empêché les gens de qualité de prétendre à ces sortes de places, était ou la facilité qu’on avait eue d’y recevoir par recommandation toutes sortes de personnes, ou le peu d’occasions qu’avaient ordinairement ceux qui s’y trouvaient d’approcher de ma personne, ou la négligence qu’on avait insensiblement apportée à les perfectionner dans leurs exercices. Et pour remédier à tout, je résolus de prendre soin moi-même de nommer tous les pages, de leur faire partager avec ceux de la petite écurie tous les services domestiques qu’ils me rendaient, et de choisir pour leur instruction les meilleurs maîtres de mon État.\par
Le fruit que je prétendais tirer de là était, à l’égard du public, de donner une excellente éducation à un grand nombre de gentilshommes, et pour mon intérêt particulier d’avoir continuellement des gens qui, sortant de cette école, entreraient plus capables et plus affectionnés à mon service que le commun de mes sujets.\par
J’eus encore un autre soin qui regardait principalement les gens de condition, mais dont l’effet se répandait ensuite sur le général du royaume. Je savais les sommes immenses qui se déboursaient par les particuliers et se portaient sans cesse hors de l’État par le commerce des dentelles des manufactures étrangères. Je voyais que les Français ne manquaient ni d’industrie ni d’étoffes pour faire eux-mêmes ces ouvrages, et je ne doutais point qu’étant faites sur les lieux, elles ne se pussent donner à beaucoup meilleur prix que celles qu’on faisait venir de si loin. Sur ces considérations, je résolus d’en établir ici des fabriques, dont l’effet serait que les grands trouveraient de la modération dans leurs dépenses, que le menu peuple profiterait de tout ce que les riches dépenseraient, et que les grandes sommes qui sortaient de l’État y étant retenues, y produiraient insensiblement une abondance et une richesse extraordinaires ; outre que cela fournirait de l’occupation à plusieurs de mes sujets, qui jusqu’alors avaient été obligés ou de se corrompre ici dans l’oisiveté ou d’aller chercher de l’emploi chez nos voisins.\par
Cependant, comme les établissements les plus louables ne se font jamais sans contradiction, je prévis bien que les marchands de dentelles traverseraient celui-ci de tout leur pouvoir, parce que je ne doutais pas qu’ils ne trouvassent mieux leur compte à débiter des marchandises venues de loin et dont le juste prix ne pouvait être connu, que celles qui se feraient ici à la vue de tout le monde.\par
Mais je me résolus à retrancher par mon autorité toutes les chicanes qu’ils pourraient faire. Ainsi, je leur avais donné un temps suffisant pour vendre ce qu’ils avaient de dentelles étrangères alors que mon édit fut publié. Et ce temps étant expiré, je fis saisir tout ce qui se trouva chez eux comme venu depuis mes défenses ; pendant que d’autre part, je faisais ouvrir des magasins remplis de la nouvelle fabrique, auxquels j’obligeais les particuliers de se fournir.\par
Cet exemple fit établir en peu de temps dans mon État beaucoup d’autres manufactures, comme de draps, de verres, de glaces, de bas de soie et de semblables marchandises.\par
Je recherchais surtout avec soin le moyen d’augmenter et d’assurer le commerce de la mer à mes sujets, en rendant les ports que j’avais plus sûrs et trouvant lieu d’en faire de nouveaux. Mais je pris en même temps un autre dessein qui n’était pas de moindre utilité : ce fut de joindre par un canal l’Océan à la Méditerranée, en sorte qu’il ne fût plus besoin de faire le tour de l’Espagne pour passer d’une mer à l’autre. L’entreprise était grande et difficile. Mais elle était infiniment avantageuse à mon royaume qui devenait ainsi le centre et comme l’arbitre du commerce de toute l’Europe. Et il n’était pas moins glorieux pour moi qui, dans l’accomplissement de ce projet, m’élevais au-dessus des plus grands hommes des siècles passés qui l’avaient inutilement entrepris.\par
Outre ces sortes de dépenses et les autres dont je vous ai parlé touchant les armements de terre et de mer, j’étais encore obligé d’en faire plusieurs autres plus secrètes dans les négociations que j’entretenais avec les étrangers. Il y avait chez les Hollandais plusieurs députés auxquels je faisais payer des pensions. J’en donnais aussi de considérables à plusieurs seigneurs de Pologne pour disposer de leurs voix dans l’élection qui se méditait. J’entretenais des pensionnaires en Irlande pour y faire soulever les catholiques contre les Anglais. Et j’entrais en traité avec certains transfuges d’Angleterre, auxquels je promettais de fournir des sommes notables pour faire revivre les restes de la faction de Cromwell. J’avais fourni cent mille écus au roi de Danemark pour le faire entrer dans la ligue contre le roi de la Grande-Bretagne ; et depuis je fis donner un collier de prix à la reine sa femme. J’en fis porter un autre à l’électrice de Brandebourg et fis faire un présent considérable à la reine de Suède, ne doutant pas que ces princesses, outre les intérêts généraux de leurs États, ne se tinssent honorées en leur particulier du soin que je prenais de rechercher leur amitié. Sachant quel crédit avait en Suède le chancelier, et combien le prince d’Anhalt et le comte de Schwerin étaient puissants chez l’électeur de Brandebourg, je les voulus gagner par ma libéralité.\par
Car enfin comme d’une part, je travaillais continuellement à retrancher jusqu’aux moindres dépenses superflues, ainsi que je fis cette année en modérant l’ustensile des soldats, en supprimant la plupart des commissaires des guerres, en sursoyant mes bâtiments, en…, d’autre part aussi je n’épargnais aucune somme pour les choses importantes et principalement pour augmenter le nombre de mes amis ou pour diminuer celui de mes ennemis, dans la vue des importants desseins que je méditais continuellement.\par
Et en effet, mon fils, s’il est utile au prince de savoir ménager ses deniers quand l’état paisible de ses affaires lui en laisse la liberté, il n’est pas moins important qu’il sache les dépenser, même avec quelque sorte de profusion, quand le besoin de son État le désire ou que la fortune lui présente quelque occasion singulière de s’élever au-dessus de ses pareils.\par
Les souverains que le Ciel a faits dépositaires de la fortune publique font assurément contre leurs devoirs quand ils dissipent la substance de leurs sujets en des dépenses inutiles, mais ils font peut-être un plus grand mal encore, quand, par un ménage hors de propos, ils refusent de débourser ce qui peut servir à la gloire de leur nation ou à la défense de leurs provinces.\par
Il arrive souvent que des sommes médiocres dépensées dans leur temps et avec jugement épargnent aux États et des dépenses et des pertes incomparablement plus grandes. Faute d’un suffrage que l’on pouvait acquérir à bon marché, il faut quelquefois lever de grosses armées. Un voisin, qu’avec peu de dépense nous aurions pu faire notre ami, nous coûte quelquefois bien cher quand il devient notre ennemi. Les moindres troupes ennemies qui peuvent entrer dans nos États nous enlèvent en un mois plus qu’il n’eût été besoin pour entretenir dix ans d’intelligences. Et les imprudents ménagers qui ne comprennent pas ces maximes trouvent enfin tôt ou tard la punition de leur avare procédé, dans leurs provinces désolées, dans la cessation de leurs revenus, dans l’abandonnement de leurs alliés et dans le mépris de leurs peuples.\par
Pourquoi faire difficulté de débourser l’argent dans les nécessités publiques, puisque ce n’est que pour satisfaire à ces besoins que nous avons droit d’en lever ? Aimer l’argent pour l’amour de lui-même est une passion dont les belles âmes ne sont pas capables ; elles ne le considèrent jamais comme l’objet de leurs désirs, mais seulement comme un moyen nécessaire à l’exécution de leurs desseins. Le sage prince et le particulier avare sont absolument opposés dans leur conduite : le riche avare cherche toujours l’argent avec avidité, le reçoit avec un plaisir extrême, l’épargne sans discernement, le garde avec inquiétude, et n’en peut débourser la moindre partie sans un insupportable chagrin ; au lieu que le prince vertueux n’impose qu’avec retenue, n’exige qu’avec compassion, ne ménage que par devoir, ne réserve que par prudence et ne dépense jamais sans un contentement tout particulier, parce qu’il ne le fait que pour augmenter sa gloire, pour agrandir son État ou pour faire du bien à ses sujets.\par
Outre les cent mille écus que j’avais déjà fournis pour l’accommodement du roi de Danemark, les Hollandais me voulaient encore obliger à donner à ce prince une nouvelle somme. Et le sujet de cette demande était que l’on désirait qu’il fît passer ses vaisseaux dans la Manche pour se joindre à nos flottes, de quoi il se défendait en disant qu’il n’était obligé par notre traité de tenir ses vaisseaux que dans la mer Baltique, afin d’en défendre le commerce à nos ennemis, et que néanmoins si l’on voulait lui payer tous les frais qu’il serait obligé de faire pour ce passage, il contribuerait volontiers en cela au bien de la cause commune.\par
Mais sur cette proposition, je répondis qu’après les grandes sommes que j’avais déjà déboursées pour les États de Hollande, soit dans les armements de mer et de terre que j’avais faits pour leur défense, soit même à l’égard du roi de Danemark, je ne croyais pas devoir charger mes sujets d’une plus grande dépense.\par
Dès lors que je me résolus à déclarer la guerre aux Anglais, je ne doutai point que dans les îles Occidentales, où mes sujets étaient mêlés avec eux, l’on n’en vînt bientôt aux derniers actes d’hostilité ; et pour fortifier les miens dans cette occasion, je tirai des places les plus voisines de la mer huit cents hommes que j’envoyai diligemment à leur secours. Mais j’appris peu de temps après que mes vœux et ma fortune avaient de beaucoup devancé l’arrivée de mes vaisseaux. Car la nouvelle de la guerre ayant été portée dans l’île de Saint-Christophe plus diligemment qu’on n’eût pu le penser et ayant été sue en même temps de l’un et de l’autre parti, les ennemis, quoique plus forts incomparablement en nombre, furent pourtant battus et contre leur opinion et contre toute sorte de vraisemblance.\par
Et en effet les Français qui ne se voyaient que seize cents dans toute l’île, où les Anglais avaient au moins six mille hommes, proposèrent d’abord de vivre en paix comme auparavant. Mais ils apprirent que les ennemis avaient résolu de les mettre tous au fil de l’épée, suivant l’ordre exprès de leur vice-roi, qui fut depuis trouvé en original dans la poche de l’un des morts.\par
Cependant cet ordre si facile à donner ne fut pas si facile à exécuter. Car les Français, excités plutôt qu’abattus par la grandeur du péril, se comportèrent avec tant de valeur et de diligence, qu’ayant rendu dans un seul jour quatre combats contre diverses troupes d’ennemis, ils demeurèrent partout vainqueurs et forcèrent les ennemis qui en échappèrent, à quitter leurs forts et sortir de l’île, ou à me faire serment de fidélité.\par
Dans le même temps, la Tamise était comme assiégée par les Hollandais victorieux qui, durant plus d’un mois, demeurèrent à son embouchure, pendant que le roi de la Grande-Bretagne, pressé par les séditieux murmures de toute son île, travaillait à remettre sa flotte en état.\par
Il la mit enfin à la mer le 4 août et avec plus de bonheur que la première fois ; car il battit celle de Hollande, divisée par l’imprudence du vice-amiral Tromp, la disgrâce duquel causa depuis chez les Hollandais une contestation dangereuse. Car quoiqu’en effet ce capitaine eût failli de s’être détaché, sans l’ordre de son général, du corps de l’armée hollandaise, il avait d’ailleurs acquis tant de réputation par sa valeur et faisait si bien entendre que ce dessein qui avait mal réussi par hasard, avait pu se prendre par de bonnes raisons, qu’il formait dans les États un puissant parti pour sa défense. Mais l’intérêt que j’avais alors de maintenir cette république unie, me fit employer mon autorité à calmer la contestation qui s’y formait.\par
Pour ma flotte, elle était encore alors à l’embouchure du Tage, attendant toujours l’arrivée de la reine de Portugal. Car la navigation de cette princesse fut extraordinairement longue. Et cependant les Hollandais, abattus de leur défaite, me pressaient avec des instances continuelles de faire approcher mes vaisseaux. En quoi ils furent favorisés par la fortune. Car, comme il était malaisé de prévoir l’extrême longueur de ce voyage, les vivres de ma flotte se trouvèrent consommés, et les Portugais n’en ayant offert à mon amiral que de très mauvaise qualité, la nécessité, qui passe par-dessus toutes sortes d’ordres et de considérations, le força de reprendre la route de France avant que la princesse qu’il attendait fût arrivée.\par
Je fis néanmoins valoir son retour auprès des Hollandais comme résolu pour l’amour d’eux ; et tâchant en effet de le rendre utile à leurs affaires, je les en fis premièrement avertir par un courrier exprès. Puis j’eus soin de tenir dans mes ports des barques chargées de toutes sortes de vivres pour ravitailler mes vaisseaux sans les arrêter. Et changeant l’ordre que je leur avais donné de s’arrêter à Belle-Isle ou à Brest, je leur commandai de marcher incessamment à la rencontre des Hollandais, qui, déjà remis de leur déroute, étaient ressortis de leurs ports.\par
Et toutes ces choses furent exécutées par les miens avec tant de bonne foi et de ponctualité que les sept navires empruntés par la reine de Portugal ayant rejoint le reste de ma flotte durant cette navigation et se trouvant dépourvus de toutes choses, on aima mieux tirer des autres une partie de leurs provisions que de s’arrêter un moment pour ravitailler ces derniers. Enfin j’avais eu tant de prévoyance pour faciliter notre jonction que j’avais fait régler, par un traité particulier, toutes les difficultés qui l’auraient pu retarder, à la réserve du salut de l’amiral, pour lequel même j’avais donné mes ordres secrets dont mes gens se fussent servis, suivant le besoin, à terminer la contestation qui pouvait rester.\par
Mais autant j’agissais sincèrement de ma part, autant trouvais-je de mauvaise foi chez les Hollandais. Comme je croyais qu’on se battrait infailliblement devant ou après la jonction, parce que les Anglais et les Hollandais étaient fort proches les uns des autres, j’avais ordonné à mes vaisseaux de venir prendre à Dieppe six cents hommes d’armes choisis dans les troupes de ma maison, que j’y avais fait marcher, pour relever par leur exemple le courage des autres troupes. Mais incontinent après, je fus averti que les armées ennemies s’étant séparées sans combat, les Anglais étaient allés se poster à l’île de Wight, qui était absolument sur notre passage ; et que les Hollandais, au lieu de les suivre ou de venir au-devant de mes vaisseaux, comme ils s’y étaient engagés expressément par le traité de la jonction, s’étaient retirés vers leur pays comme pour donner commodité aux ennemis, trois fois plus forts que moi, de défaire ma flotte à leur aise.\par
Le comte de La Feuillade, que j’envoyai le premier pour les faire souvenir de leur parole et leur remontrer l’importance de ce manquement de foi, les trouva si peu disposés à faire ce qu’ils avaient promis qu’au lieu d’avancer vers les miens, ils levèrent l’ancre, de Boulogne où ils étaient, pour se retirer en sa présence beaucoup plus près de leurs ports. Et Vilquier, capitaine de mes gardes, que [je] renvoyai dans le même dessein, les trouva dans la même résolution. En quoi leur procédé sans doute est inexcusable. Car, soit que ce fût par un dessein formé de m’abandonner au besoin, ou que la maladie de Ruyter, leur général, leur eût absolument ôté la hardiesse de paraître devant leurs ennemis, c’était sans doute une infidélité ou une lâcheté très signalée.\par
Vous pouvez vous imaginer l’inquiétude que mon esprit souffrait durant ce temps-là, car je savais que mes vaisseaux étaient dans la Manche, et quelque diligence que je pusse faire en faisant sortir de toutes mes places maritimes des barques qui croisaient incessamment la mer pour donner avis à mon amiral de toutes choses, je ne pus jamais le faire avertir de la retraite des Hollandais et je fus surpris d’apprendre un jour qu’il était arrivé à Dieppe.\par
Ce m’était assurément un sujet de satisfaction de voir qu’ayant passé à la vue des ennemis, ils n’eussent pas eu la hardiesse de l’attaquer. Mais d’ailleurs, je considérais qu’apparemment lorsqu’ils auraient fait réflexion sur leurs avantages, ils ne manqueraient pas de s’en prévaloir ; que je n’avais dans la Manche aucun port capable de mettre ma flotte à couvert ; que si elle voulait aller joindre les Hollandais au lieu où ils s’étaient retirés, elle serait contrainte de passer le pas de Calais où les Anglais pouvaient se rendre plus tôt qu’elle ; et qu’enfin si elle prenait résolution de retourner vers Belle-Isle, qui était la retraite la plus proche que je leur pusse donner, il faudrait encore passer à la vue de l’île de Wight, où nous avions laissé les ennemis.\par
Dans cette perplexité, ne sachant que résoudre moi-même, je laissai à mon amiral la liberté de prendre son parti suivant les nouvelles qu’il pouvait avoir des Anglais. Mais, quoiqu’il n’en eût pu rien apprendre de certain, ma bonne fortune le fit pencher au meilleur avis ; et pendant que la flotte ennemie l’allait attendre vers Calais, il retourna sans danger vers la Bretagne : au lieu que certains vaisseaux hollandais qui, pour leur sûreté particulière, avaient jusque-là suivi les miens, ayant voulu s’en détacher alors pour retourner en leur pays, furent presque tous pris par la flotte d’Angleterre.\par
De mes vaisseaux il y en eut seulement sept qui, dans le premier trajet étant demeurés loin des autres, furent en danger de se perdre. Mais les uns ayant découvert les ennemis d’assez bonne heure les évitèrent. D’autres s’étant trouvés engagés parmi eux, s’en tirèrent à coups de canon ; et un seul, se voyant hors d’état de se sauver, se défendit avec tant d’opiniâtreté qu’il causa plus de dommages à ceux qui le prirent, qu’il ne leur apporta de profit.\par
Cependant le mariage de Portugal s’était accompli avec une satisfaction générale de tout le royaume ; et la nouvelle reine ayant acquis d’abord assez de crédit dans cette cour, je crus m’en devoir servir pour combattre l’autorité du comte de Castel Mayor, qui, tenant la première place dans les conseils du roi son maître, secondait de tout son pouvoir le dessein qu’avait le roi de la Grande-Bretagne d’accorder les Portugais avec les Espagnols ; et cette princesse, informée de mes intentions, les suivit avec tant de chaleur qu’elle se brouilla aussitôt ouvertement avec le comte.\par
La reine de Pologne n’avait pas moins d’affection pour la France et eût bien voulu faire tomber la couronne chancelante, qu’elle seule semblait soutenir par sa vertu, sur la tête d’un prince de ma maison. Mais ses affaires étaient alors en si mauvais état qu’il ne lui était pas facile de faire réussir ce dessein, si elle n’était puissamment assistée. Pour cela, j’avais résolu dans le commencement de cette année d’envoyer le prince de Condé en Pologne avec 500 chevaux et 6 000 hommes de pied, en cas que mes affaires le pussent permettre. Mais ayant été contraint, aussitôt après, de déclarer la guerre au roi d’Angleterre, je ne crus plus être en état d’exécuter ce projet : de quoi je donnai promptement avis à la reine de Pologne, lui envoyant en même temps par manière de consolation, une somme de 200 000 livres, que je ne lui avais point fait espérer. Cela n’empêcha pas que vers la fin du mois de mai, elle n’envoyât un gentilhomme à ma cour pour me demander de nouveaux secours. Mais prévoyant bien la difficulté qui se devait trouver à sa demande, elle avait fait expédier à celui qui venait deux différentes commissions, l’une de simple envoyé sous prétexte de me faire compliment sur la mort de la Reine ma mère, et l’autre d’ambassadeur extraordinaire pour me faire la demande dont je viens de vous parler, laissant au porteur la liberté de se servir de l’une ou de l’autre, selon l’espérance qu’il pourrait avoir du succès de sa négociation.\par
Dès lors que je fus informé de ces particularités, je voulus empêcher que le roi de Pologne ne fît éclater une célèbre ambassade pour ne rien obtenir. Et pour cela, je fis donner conseil à son ministre de ne paraître auprès de moi que comme envoyé. Mais soit qu’il voulût contenter sa vanité particulière par un titre plus éminent, ou qu’il s’imaginât en tirer pour son roi quelque autre avantage, il prit contre mon sentiment la qualité d’ambassadeur.\par
Je le reçus de ma part avec tous les honneurs accoutumés, quoique d’abord je fusse résolu à ne lui rien octroyer de ce qu’il demandait. Mais incontinent après, je ne pus m’empêcher de lui accorder une somme très importante. Car l’évêque de Béziers, mon ambassadeur, me fit savoir que l’armée de Lithuanie, dans laquelle consistait tout ce qui restait de force et d’autorité au roi de Pologne, étant sur le point de se mutiner, il avait cru devoir même sans mon ordre s’engager à lui payer un quintal, c’est-à-dire une certaine portion de sa solde, et j’estimai que je ne devais pas désavouer une parole donnée par une si pressante raison : comme je reconnus en effet par la suite, parce que l’attachement que cette armée continua de témoigner au service de son prince, fut principalement ce qui contraignit ses sujets rebelles à rentrer dans l’obéissance qu’ils lui devaient.\par
Mais dans l’une des conférences que j’eus sur les affaires de ce royaume avec l’ambassadeur, il s’avisa de me demander brusquement si je désirais encore insister à l’élection que j’avais jusque-là désirée ou si j’étais résolu de me désister. La proposition était délicate d’elle-même, mais elle semblait le devenir encore davantage par l’humeur de celui qui la faisait. Car j’étais averti de bonne part que c’était un esprit très difficile. Ainsi, j’avais lieu de craindre que, si je persistais dans le dessein de l’élection, cet homme chagrin ne se servît de ma réponse pour me brouiller avec les États de Pologne, qui avaient alors une entière répugnance à cette affaire, et si je déclarais que j’eusse intention de m’en désister, je voyais que c’était renoncer absolument à une prétention pour laquelle j’avais déjà fait des démarches et des dépenses très importantes.\par
C’est pourquoi rappelant à ce moment tout mon esprit pour former une réponse mitoyenne entre ces deux extrémités, je lui dis que, dans l’état présent des affaires, je ne pensais nullement à poursuivre mon premier dessein et qu’il fallait attendre qu’elles eussent repris une meilleure assiette pour examiner s’il serait à propos de reprendre nos premières pensées : par lequel discours je crus ne pouvoir blesser ni l’humeur présente des Polonais ni les espérances de la France.\par
Et sur cet événement, je prendrai sujet de vous faire observer combien il est important que les princes portent dans leurs propres têtes la meilleure partie de leur conseil et combien leurs paroles sont souvent importantes pour le succès ou la ruine de leurs affaires. Car enfin, quoique je vous parle ici continuellement des entretiens que j’ai avec les ministres étrangers, je ne prétendrais pas donner conseil indifféremment à tous ceux qui portent des couronnes de s’exposer à cette épreuve, sans avoir auparavant bien examiné s’ils sont capables d’en bien sortir. Et j’estime que ceux dont le génie est médiocre, font et plus honnêtement et plus sûrement de s’abstenir de cette fonction que d’y vouloir étaler leur faiblesse à la vue de leurs voisins et mettre en danger les intérêts de leurs provinces.\par
Beaucoup de monarques seraient capables de se gouverner sagement dans les choses où ils ont le temps de prendre conseil, qui ne seraient pas pour cela suffisants pour soutenir eux-mêmes leurs affaires contre des hommes habiles et consommés qui ne viennent jamais à eux sans préparation et qui cherchent toujours à prendre les avantages de leurs maîtres. Quelque notion que l’on puisse nous avoir donné du sujet qui se doit traiter, un ministre étranger peut à toute heure, ou par hasard ou avec dessein, nous proposer de certaines choses sur lesquelles nous ne sommes pas préparés. Et cependant ce qu’il y a de fâcheux est que les fausses démarches que fait alors un souverain, ne peuvent être désavouées par lui qu’en avouant son incapacité, et portent infailliblement leur coup ou contre l’intérêt de son État ou contre sa propre réputation.\par
Mais ce n’est pas seulement dans les négociations importantes qu’un prince doit prendre garde à ce qu’il dit. C’est même dans les discours les plus ordinaires qu’il est le plus souvent en danger de faillir. Car il faut bien se garder de penser qu’un souverain, parce qu’il a l’autorité de tout faire, ait aussi la liberté de tout dire. Au contraire, plus il est grand et considéré, plus il doit considérer lui-même ce qu’il dit. Les choses qui ne seraient rien dans la bouche d’un particulier, deviennent souvent importantes par la seule raison que c’est le prince qui les a dites. Surtout la moindre marque de mépris qu’il donne d’un particulier, ne peut qu’elle ne porte à cet homme un préjudice très grand, parce que dans la cour des princes chacun n’est estimé de ses pareils qu’à mesure qu’on le croit estimé du maître. Et de là vient que ceux qui sont offensés de la sorte, en portent ordinairement dans le cœur une plaie qui ne finit qu’avec la vie.\par
Deux choses peuvent consoler un homme d’une raillerie piquante ou d’une parole de mépris que son semblable a dite de lui : la première quand il se promet de trouver bientôt occasion de lui rendre la pareille ; et la seconde quand il peut se persuader que ce qu’on a dit à son désavantage, ne fera pas d’impression sur ceux qui l’ont entendu. Mais celui de qui le prince a parlé, sent d’autant plus vivement son mal qu’il n’y voit aucun de ces remèdes. Car enfin s’il ose parler mal de son maître, ce n’est au plus qu’en particulier et sans pouvoir lui faire savoir ce qu’il en dit, qui est la seule douceur de la vengeance. Et il ne peut non plus se persuader que ce qui a été dit de lui, n’a pas été écouté, parce qu’il sait avec quel agrément sont tous les jours reçues les paroles de ceux qui sont en autorité.\par
Ainsi je vous conseille, mon fils, très sérieusement de ne vous jamais rien permettre sur cette matière et de considérer que ces sortes d’injures non seulement blessent ceux qui les ont recues, mais offensent même bien souvent ceux qui feignent de les entendre avec le plus d’applaudissement, parce que, quand ils nous voient mépriser ceux qui nous servent comme eux, ils craignent avec sujet que nous ne les traitions de même en un autre rencontre.\par
Car enfin vous devez poser pour fondement de toute chose que l’on ne pardonne rien à ceux de notre rang. Au contraire, il se trouve souvent des paroles très indifférentes et dites par nous sans aucun dessein, qui sont appliquées par ceux qui les entendent ou à eux-mêmes ou à d’autres auxquels souvent nous ne pensons pas. Et quoiqu’à dire vrai, nous ne soyons pas obligés d’avoir égard en particulier à toutes les conjectures impertinentes, cela nous doit pourtant obliger en général à nous précautionner davantage dans nos paroles, pour ne pas donner du moins de raisonnable fondement aux pensées que l’on en pourrait former au désavantage de notre service.\par
Du côté d’Italie, je n’avais pas alors de grandes affaires. Il s’était formé une difficulté à la cour de Savoie sur le traitement que la nouvelle duchesse y devait faire à mon ambassadeur, parce qu’elle prétendait agir de même façon qu’avait fait la feue duchesse ma tante. Mais je fis considérer au duc son mari qu’encore que cette princesse, comme sa femme, portât dans ses États le même titre que sa mère, elle ne devait pas prétendre que dans le reste du monde on la considérât de la même sorte ; que la qualité de Fille de France avait de tout temps donné des prérogatives toutes singulières et que même autrefois les princesses qui en étaient honorées, à quelque prince qu’elles se mariassent, conservaient toujours de leur chef le titre et le rang de reines. Si bien que les respects qui leur étaient rendus ne devaient pas être tirés à conséquence. Lesquelles raisons furent trouvées si bonnes par le duc de Savoie qu’il ne crut pas devoir plus longtemps soutenir sa prétention.\par
Le Pape et la duchesse de Mantoue avaient un différend dans lequel ils m’avaient reconnu pour arbitre. Mais l’affaire s’accommoda d’elle-même.\par
Le duc de Chaulnes, que j’avais envoyé comme ambassadeur extraordinaire à Rome, y avait été reçu très honorablement. Car la mauvaise disposition où était alors le Pape, rendait ses neveux un peu plus honnêtes qu’ils n’avaient accoutumé. Mais sitôt que le Pape eut repris sa santé, ils reprirent aussi leur fierté ordinaire.\par
La seule affaire qui me restait à traiter en cette cour, était d’abolir les divisions qui s’étaient formées dans le clergé de ce royaume sur les propositions de Jansénius. Le Pape s’y était porté d’abord fort chaudement, comme dans une affaire qui regardait en effet ses intérêts plus que les miens, et se rendait solliciteur envers moi pour l’exécution des bulles qu’il avait données sur ce sujet, principalement en ce qui regardait les évêques qui avaient refusé d’y obéir. Et de ma part je lui prêtais volontiers le secours de mon autorité avec toute la précaution néanmoins qui se devait pour ne pas blesser les anciens privilèges de l’Église gallicane.\par
Mais depuis, comme je travaillais de bonne foi sur ce plan et que j’eus mis l’affaire au point que le Pape n’avait plus qu’à nommer des commissaires, je m’aperçus qu’il changeait de conduite. Et le sujet de ce changement était que ses neveux avaient pris le zèle chrétien avec lequel j’agissais en cette occasion, pour une puissante jalousie d’État, et s’imaginaient qu’ils pourraient tirer de moi tout ce qui leur plairait en échange de la satisfaction qu’ils me donneraient sur ce point.\par
Ainsi, lorsque mon ambassadeur leur parla de ma part de nommer les commissaires, ils firent premièrement diverses difficultés et ensuite s’expliquant plus nettement, osèrent bien proposer qu’en échange de cette expédition, je consentisse d’abattre la pyramide qu’ils avaient été contraints de me bâtir pour réparation du crime des Corses. Mais alors pour faire voir que je n’avais autre attachement à cette affaire que pour le bien de la religion et qu’en ce qui regardait l’intérêt de mon État, je ne craignais nullement le jansénisme, j’ordonnai à mon ambassadeur de dire simplement à ces messieurs qu’après avoir informé Sa Sainteté de l’état des choses et lui avoir proposé ce qui était à faire suivant les formes pour l’exécution de ses propres décrets, je croyais avoir satisfait à mon devoir envers Dieu et que ce serait désormais à Elle à faire le sien quand il lui plairait.\par
Cependant le cardinal Ursin, dont le procédé (comme vous avez vu dans les années passées) n’avait pas été tel qu’il devait être, se rendit à ma cour et m’ayant fait paraître un véritable repentir de sa faute, me fit résoudre à l’oublier et à lui rendre le titre de Comprotecteur de France que je lui avais ôté.\par
Peu de temps après, je rétablis la discipline et l’union dans l’ordre de Cîteaux. La division s’y était introduite par l’artifice ou le mauvais zèle de certains supérieurs particuliers, qui, sous prétexte d’une réforme plus austère, voulaient se soustraire de l’autorité du général, lequel de sa part soutenait que les nouvelles règles auxquelles les réformes s’étaient voulu soumettre, ne les pouvaient pas dispenser de l’obéissance qu’ils devaient à leur supérieur naturel.\par
Cette affaire me parut d’autant plus digne de mon application que cet ordre était infiniment célèbre et que le schisme que l’on y voyait, portait un scandale général à toute l’Église, outre qu’ayant été entreprise dès l’année 1633 par le cardinal de La Rochefoucauld, homme de suffisance et de piété singulière, et depuis poursuivie, non seulement dans toutes les juridictions du royaume, mais devant le Pape même, assisté pour ce sujet des plus habiles cardinaux, elle n’avait pu être achevée.\par
Ainsi je fis rapporter l’affaire en mon conseil. Mais comme s’il eût été du destin de cette affaire de n’être jamais terminée, mes conseillers se trouvèrent partagés en opinions et je me vis dans la nécessité de la décider par mon seul suffrage, lequel je donnai en faveur du général. Car outre les raisons du fond qu’il serait ennuyeux de déduire ici et le sentiment du Pape qui en avait jugé comme moi, je considérai qu’il était de l’avantage de l’État de conserver sous l’obéissance de ce chef d’ordre tous les couvents étrangers qui offraient de s’y ranger, et qu’il était de la prudence d’un souverain de maintenir en toutes les choses justes ceux qui ont le caractère de supériorité, contre la révolte des subalternes.\par
Je projetai encore alors un autre règlement qui regardait à la fois et l’État et l’Église. Ce fut à l’égard des fêtes dont le nombre, augmenté de temps en temps par des dévotions particulières, me semblait beaucoup trop grand. Car enfin il me parut qu’il nuisait à la fortune des particuliers en les détournant trop souvent de leur travail, qu’il diminuait la richesse du royaume en diminuant le nombre des ouvrages qui s’y fabriquaient, et qu’il était même préjudiciable à la religion par laquelle il était autorisé, parce que la plupart des artisans étant des hommes grossiers, donnaient ordinairement à la débauche et au désordre ces jours précieux qui n’étaient destinés que pour la prière et les bonnes œuvres.\par
Dans ces considérations, je pensai qu’il serait du bien des peuples et du service de Dieu d’apporter en cela quelque modération, et je fis entendre ma pensée à l’archevêque de Paris, lequel la jugeant pleine de raison, voulut bien, comme pasteur de la capitale de mon royaume, donner en cela l’exemple à tous ses confrères.\par
Les impiétés qui se commettaient dans le Vivarais, me donnèrent sujet d’y faire tenir des Grands Jours par les officiers du parlement de Toulouse, et quoique la chambre mi-partie de Castres me sollicitât avec instance afin d’obtenir place dans ce tribunal pour quelques-uns de leurs députés comme ayant droit d’y entrer, je crus qu’il serait plus avantageux à la religion de ne pas leur accorder cette demande, laquelle je sus éluder par divers délais, pendant que l’affaire se consommait.\par
Le même zèle me fit envoyer l’abbé de Bourséis jusqu’en Portugal pour essayer de convertir Schomberg qui s’y était acquis beaucoup de réputation, et me fit poursuivre chez les Hollandais la réparation du scandale qu’ils avaient fait peu de temps auparavant en la personne d’un aumônier de mon ambassadeur.\par
Je terminai bientôt après l’assemblée, qui durait depuis le commencement de juin, et lui fis arrêter le don extraordinaire qu’elle a coutume de me faire tous les cinq ans, à la somme de huit cent mille écus, quoiqu’elle s’en fût jusque-là défendue ou par l’envie que les députés avaient de continuer leur séjour à Paris ou par le désir de ménager leurs bourses.\par
Je n’ai jamais manqué de vous faire observer, lorsque l’occasion s’en est présentée, combien nous devons avoir de respect pour la religion et de déférence pour ses ministres, dans les choses principalement qui regardent leur mission, c’est-à-dire la célébration des mystères sacrés et la publication de la doctrine évangélique. Mais parce que les gens d’Église sont sujets à se flatter un peu trop des avantages de leur profession et s’en veulent quelquefois servir pour affaiblir leurs devoirs les plus légitimes, je crois être obligé de vous expliquer sur cette matière certains points qui peuvent être importants.\par
Le premier est que les rois sont seigneurs absolus et ont naturellement la disposition pleine et libre de tous les biens, tant des séculiers que des ecclésiastiques, pour en user comme sages économes, c’est-à-dire selon les besoins de leur État.\par
Le second, que ces noms mystérieux de franchises et de libertés de l’Église, dont on prétendra peut-être vous éblouir, regardent également tous les fidèles, soit laïques, soit tonsurés, qui sont tous également fils de cette commune mère, mais qu’ils n’exemptent ni les uns ni les autres de la sujétion des souverains, auxquels l’Évangile même leur enjoint précisément d’être soumis.\par
Le troisième, que tout ce qu’on dit de la destination particulière des biens de l’Église et de l’intention des fondateurs, n’est qu’un scrupule sans fondement : parce qu’il est constant que, comme ceux qui ont fondé les bénéfices n’ont pu, en donnant leurs héritages, les affranchir ni du cens ni des autres redevances qu’ils payaient aux seigneurs particuliers, à bien plus forte raison n’ont-ils pas pu les décharger de la première de toutes les redevances, qui est celle qui se reçoit par le prince comme seigneur universel pour le bien général de tout le royaume.\par
Le quatrième, que si l’on a permis jusqu’à présent aux ecclésiastiques de délibérer dans leurs assemblées sur la somme qu’ils doivent fournir, ils ne sauraient attribuer cet usage à aucun privilège particulier : parce que la même liberté est encore laissée aux peuples de plusieurs provinces comme une ancienne marque de la probité des premiers siècles, où la justice excitait suffisamment chaque particulier à faire ce qu’il devait selon ses forces, et cependant cela n’a jamais empêché que l’on ait contraint et les lais et les ecclésiastiques, lorsqu’ils ont refusé de s’acquitter volontairement de leur devoir.\par
Et le cinquième enfin, que, s’il y avait quelques-uns de ceux qui vivent sous notre empire plus tenus que les autres à nous servir de tous leurs biens, ce devrait être les bénéficiers, qui ne tiennent tout ce qu’ils ont que de notre choix. Les droits qui se perçoivent sur eux sont établis d’aussi longtemps que leurs bénéfices, et nous en avons des titres qui se sont conservés depuis le premier âge de la monarchie. Les papes mêmes qui se sont efforcés de nous dépouiller de ce droit, l’ont rendu plus clair et plus incontestable par la rétractation précise qu’ils ont été obligés de faire de leurs ambitieuses prétentions.\par
Mais on peut dire qu’il n’est pas ici besoin ni de titres ni d’exemples, parce que la seule équité naturelle suffit pour éclaircir absolument ce point. Serait-il juste que la noblesse donnât ses travaux et son sang pour la défense du royaume et consumât si souvent ses biens à soutenir les emplois dont elle est chargée, et que le peuple, qui, possédant si peu de fonds, a tant de têtes à nourrir, portât encore lui seul toutes les dépenses de l’État, pendant que les ecclésiastiques, exempts par leur profession des dangers de la guerre, des profusions du luxe et du poids des familles, jouiraient dans leur abondance de tous les avantages du public sans jamais rien contribuer à ses besoins ?\par
Mais au milieu de toutes mes autres applications, celle où mon esprit s’attachait davantage était de me mettre en état de retirer des mains du roi d’Espagne les provinces qui m’étaient échues par la mort du roi son père. Et comme rien n’était plus important à ce dessein que de terminer les autres affaires où j’étais engagé pour les Hollandais, afin d’avoir la liberté d’employer toutes mes forces dans ma propre querelle, je fis premièrement nouer une conférence en Allemagne entre les États et l’évêque de Münster, en laquelle Colbert, maître des requêtes, assistant de ma part, porta bientôt les choses à la conclusion. Et ensuite j’essayai de profiter des bonnes intentions qu’avait la reine d’Angleterre, pour faire la paix avec le roi son fils.\par
Le premier avantage que je tirai de l’entreprise de la reine d’Angleterre, fut de terminer une contestation incidente qui se formait par le roi d’Angleterre. Car il prétendait, non sans quelque raison, que le traité se devait faire chez lui, parce que les Hollandais, qui étaient ses véritables parties, lui étaient inférieurs en dignité. Et néanmoins, après quelques disputes, il se relâcha de cette prétention sur la remontrance que je lui fis que la maison de la Reine sa mère pouvait être prise entre nous comme un lieu neutre et qu’il était de l’honneur de cette princesse que la paix, ayant été proposée par elle, se négociât aussi en sa présence et avec sa participation. Mais ce qui me faisait le plus appuyer cette proposition était que je voyais que, par le seul respect dû à la présence de cette reine, l’on trouverait moyen de retrancher la plupart des questions préliminaires qui font ordinairement la plus grande longueur des traités ; outre que les choses se négociant si près de moi, j’aurais la commodité d’inspirer continuellement à mes ministres ce qui serait de mes intérêts. Nos députés s’assemblèrent donc au lieu accordé. Mais milord Ollis, qui traitait pour le roi d’Angleterre, ayant fait entendre dès la première conférence qu’il ne pouvait écouter aucune proposition si l’on ne convenait d’abord d’accorder au roi son maître tous les articles qu’il avait fait proposer quelque temps auparavant par l’entremise de l’ambassadeur de Portugal, l’on ne passa pas plus avant. Et ce milord fut bientôt rappelé par le roi son maître.\par
Cela ne fit pas néanmoins désespérer notre médiatrice de réussir dans ce qu’elle avait entrepris. Se promettant qu’Ollis ou quelque autre député reviendrait bientôt d’Angleterre, elle me pria que, s’il revenait durant un voyage qu’elle voulait faire à Bourbon pour sa santé, l’on continuât de s’assembler en son logis dans la même forme que si elle y eût été présente.\par
Cependant Ollis ne revint point, et, à dire le vrai, je n’en étais pas fort fâché, parce que dans le séjour qu’il avait fait auprès de moi comme ambassadeur, j’avais remarqué dans son esprit une rudesse mal propre à négocier un accord, et je ne sais pas même à quel dessein il avait affecté des prétextes de demeurer ici, depuis que la guerre eût fini son ambassade jusqu’à ce qu’il fût chargé de traiter de la paix. Mais nous ne fûmes pas longtemps sans renouer de nouvelles négociations par le commerce qu’entretenait Ruvigny avec le comte de Saint-Alban, lequel était alors en Angleterre et nous faisait espérer de jour en jour qu’il reviendrait avec des propositions raisonnables. Ce qu’il y avait de plus incommode pour moi dans ces pourparlers, est qu’ayant en effet toute l’impatience possible d’avancer l’affaire, je ne pouvais néanmoins la presser sans me faire un préjudice considérable. Car du côté des Anglais, cela faisait demander des conditions plus injustes ; et de la part des Hollandais, cela faisait naître des jalousies capables de reculer l’affaire au lieu de l’avancer, parce qu’ils ne craignaient rien au monde si fortement que de me voir établi dans leur voisinage.\par
Je tâchais néanmoins en toute rencontre de leur remettre l’esprit sur ce point ; et un jour entre autres m’entretenant avec Van Beuningen, je lui fis entendre, en termes généraux, que l’excès de la défiance entre des alliés faisait souvent de grands préjudices aux uns et aux autres. Sur quoi, lui m’ayant reparti qu’il ne voyait rien qui pût donner défiance aux États que l’entreprise que je semblais vouloir faire sur la Flandre, je lui témoignai qu’encore que les droits que j’avais dans les Pays-Bas s’étendissent sur certaines provinces qui leur étaient voisines, je serais toujours prêt de les transporter sur d’autres terres par manière d’équivalent. Mais je ne lui parlai néanmoins de cela que comme d’une affaire qui se pourrait traiter à loisir, afin que quelque résolution que les États prissent sur ce sujet, ma diligence les pût prévenir.\par
Dans ce dessein, je prenais toujours le prétexte de la guerre de mer pour couvrir tous les préparatifs que je faisais. J’avais besoin de grands magasins de vivres du côté de Flandre ; mais je les faisais dans mes places maritimes, comme pour fournir au besoin toutes choses nécessaires à mes vaisseaux, et j’avais même ordonné que l’on y donnât aux Hollandais toutes les choses qu’ils demanderaient. Il y avait à la vérité dans les autres places une grande quantité de farine ; mais ces provisions étaient couvertes par le nouveau soin que j’avais pris cette année de nourrir moi-même mes troupes ; et je tirais encore de cet établissement un prétexte spécieux pour en tenir la plus grande partie en Picardie et en Champagne, en sorte que dans ces deux seules provinces, j’avais pour le moins cinquante mille hommes prêts d’entrer dans la Flandre au premier ordre.\par
Je publiais cependant pour amuser le monde que j’allais faire un voyage à Brest, dont j’avais souvent compté les journées, parlant à mes domestiques, et même prescrit l’ordre qui devait être observé dans la marche des troupes de ma maison.\par
À l’égard des Espagnols, je les entretenais par la proposition d’un nouveau traité de commerce. Et sachant qu’il se traitait une ligue offensive et défensive entre eux et les Anglais, je m’avisai pour en retarder l’effet d’offrir au Roi Catholique d’en faire une pareille avec lui, et même d’y comprendre le roi de Portugal, que les Anglais n’y pouvaient comprendre. À laquelle proposition les Espagnols furent assez simples pour s’arrêter assez longtemps.\par
Cependant je passais par des diverses choses que j’aurais pu relever en d’autres temps. Car j’apprenais tous les jours que Castel Rodrigo, qui gouvernait les Pays-Bas, faisait faire ou souffrait que l’on fît outrage à tous les Français qui s’y trouvaient, sans leur rendre aucune justice : de quoi je ne témoignai aucun ressentiment. Mais je ne fus pas fâché que l’on volât aussi quelques courriers espagnols qui passaient ici pour aller en Flandre.\par
Je dissimulai tout de même la protection que l’on donnait à Madrid à Saint-Aunay, l’un de mes sujets qui tenait un procédé fort insolent à mon égard. Et ce fut sans mon aveu et sans ma participation que le comte de La Feuillade alla sur les lieux et contraignit cet homme à donner un billet de sa main par lequel il désavouait une devise qui lui était imputée.\par
Cependant je ne laissais pas de faire tirer exactement les plans de toutes les places de Flandre, et principalement de Bouchain par où j’avais alors dessein de commencer.\par
Mais de toutes les choses auxquelles je travaillais pour avancer cette entreprise, celle où je trouvais le plus de difficulté, était de m’assurer des Suédois qui, par la minorité de leur prince, étaient gouvernés par une espèce de Sénat, dont toutes les résolutions étaient réglées par le seul intérêt présent. Ils l’avaient déjà fait connaître assez ouvertement dans leur procédé entre moi et le roi d’Angleterre. Car ce prince les ayant fait solliciter de se liguer avec lui contre le roi de Danemark, leur ancien ennemi, ils lui en donnèrent d’abord leur parole et néanmoins se rétractèrent depuis en ma considération.\par
Cela ne se fit pas néanmoins sans difficulté. Car sur les premières remontrances que je leur fis faire à cet égard, ils s’efforcèrent assez longtemps de me payer de paroles générales qu’ils ne s’engageraient en aucun parti contre moi, et depuis encore, quand je leur déclarai que je savais l’engagement formel qu’ils avaient avec le roi de la Grande-Bretagne pour faire la guerre au roi de Danemark, et que, dans cette occasion, déclarer la guerre à ce prince c’était la faire à moi-même, ils me dirent que ce n’avait jamais été leur pensée de prendre parti contre la France, et que, s’ils s’étaient portés pour le roi d’Angleterre contre les Hollandais, c’était qu’ils avaient cru que je me rangerais moi-même de ce côté-là, vu l’étroite parenté que j’avais avec ce prince.\par
Mais pendant que je méditais divers moyens pour leur faire changer cette résolution, l’âpreté qu’ils avaient à suivre leur intérêt les fit venir d’eux-mêmes à tout ce que je pouvais souhaiter. Car je leur devais alors cent mille écus d’arrérages de vieilles pensions, lesquelles j’avais exprès différé de leur payer ; et de plus ils souhaitaient tirer de moi quelque nouveau secours d’argent pour leur entreprise de Brême, ce qui les fit résoudre à m’envoyer Koenigsmark comme ambassadeur extraordinaire, avec ordre de m’offrir leur médiation entre moi et le roi d’Angleterre : qui était sans doute s’engager à la neutralité. Mais encore y avait-il quelque chose de capiteux dans ces offres, parce que la médiation n’était offerte qu’à moi et aux États de Hollande, sans parler du roi de Danemark en faveur duquel je désirais que l’on s’expliquât. L’ambassadeur, tâchant de me faire passer cette omission comme une chose à laquelle on n’avait fait aucune réflexion et sur laquelle il n’avait nul ordre de me répondre, je l’obligeai de dépêcher en Suède pour en avoir la résolution. Et cependant je sursis de ma part à lui faire réponse sur ses demandes. La réponse vint bientôt après, conforme à mes intentions, en sorte que le roi de Danemark fut absolument exempt de la crainte des Suédois. Et alors je leur fis passer les cent mille écus d’ancienne dette.\par
Mais, à l’égard de la nouvelle demande qu’ils me faisaient, je répondis à l’ambassadeur que si l’on voulait conclure le traité qu’il négociait entre nous, je consentirais volontiers que l’on y comprît une somme payable comptant, pour le besoin qu’ils en avaient, que l’affaire n’avait reçu jusqu’ici et ne recevrait encore de ma part aucun retardement, et qu’il ne dépendait que d’eux-mêmes de la terminer.\par
Cependant, quelque temps après, craignant que ce refus ne les aigrît et ne voulant pas épargner une somme médiocre pour les gagner s’il se pouvait, je résolus de leur faire donner encore cent mille écus. Mais je ne voulus pas leur en donner la nouvelle par la voie de Kœnigsmark, quoiqu’il me sollicitât vivement sur ce sujet, parce que je crus qu’Arnauld, qui négociait en Suède pour moi, ayant à porter cette parole, en tirerait plus d’avantage pour mes affaires, qui, d’ailleurs, semblaient être en assez bonne disposition.\par
Car il n’y avait point de paroles honnêtes que je ne reçusse des Suédois. La reine Christine ayant demandé ma recommandation vers eux pour ce qui regardait ses intérêts, ils avaient promis de la traiter, en ma considération, le plus honorablement qu’il se pourrait. Et même en ce qui regardait notre traité, déjà ils m’avaient fait déclarer positivement qu’ils voulaient bien entrer en ligue avec moi contre la maison d’Autriche. Déjà, nous nous étions communiqué réciproquement les articles que nous prétendions insérer, et déjà même de ma part, j’avais fourni des réponses aux prétentions des Suédois.\par
Il arriva même en ce temps une chose fort capable d’avancer cette négociation : ce fut l’accommodement de Brême. Car, par ce moyen, leurs troupes leur demeurant inutiles, ils semblaient devoir être bien aises d’avoir occasion de les employer.\par
Et en effet, soit que touchés de cette considération, ils eussent alors dessein de faire ce que je désirais ou qu’ils voulussent seulement me mettre aux mains avec la maison d’Autriche pour en tirer après tout à loisir leurs avantages, ils me donnèrent plus d’espérance que jamais et travaillant sérieusement à m’accommoder avec le roi d’Angleterre, le pressèrent d’accepter leur médiation. Il s’en défendit d’abord sous divers prétextes. Mais la seule raison qui le touchait en effet était la répugnance qu’il avait d’accepter pour arbitres des gens qui s’étaient précisément engagés à suivre son parti.\par
Ce n’est pas que les Suédois n’alléguassent diverses excuses pour leur manquement de foi, disant entre autres choses qu’ils étaient menacés d’une prompte guerre de la part des Moscovites et qu’il n’était pas raisonnable qu’ils s’engageassent à soutenir la querelle de leurs alliés au préjudice de leur propre défense. Mais il était aisé de découvrir que la véritable raison de leur procédé était seulement qu’ils croyaient trouver plus d’avantage à quitter dans cette occasion le roi d’Angleterre qu’à le servir comme ils avaient promis.\par
D’où je crois que vous devez apprendre deux choses : l’une, que ni la religion des traités ni la foi des paroles données ne sont pas assez fortes pour retenir ceux qui naturellement sont de mauvaise foi ; et l’autre, que dans l’exécution de nos desseins, nous ne devons faire de fondement que sur la connaissance de nos propres forces. Encore qu’il soit de la probité d’un prince d’observer indispensablement ses paroles, il n’est pas de sa prudence de se fier absolument à celle d’autrui.\par
Il faut même savoir que, sur ce sujet, les plus fortes précautions sont inutiles. Il n’est point de clause si nette qui ne souffre quelque interprétation, et dès lors qu’on a pris la résolution de se dédire, on en trouve aisément le prétexte. Chacun parle dans les traités suivant ses intérêts présents ; mais la plupart tâchent après d’expliquer leurs paroles suivant les conjonctures qui se présentent, et quand la raison qui a fait promettre ne subsiste plus, on trouve peu de gens qui fassent subsister leurs promesses.\par
Mais je crois même vous devoir observer ici, pour votre instruction particulière, que cette façon d’agir est plus à craindre dans les États qui se conduisent par les suffrages de plusieurs, qu’en ceux qui se conduisent par l’ordre d’un seul. Les princes, en qui l’éclat de leur naissance et l’honnêteté de leur éducation ne produit d’ordinaire que des sentiments nobles et généreux, ne peuvent laisser tellement altérer ces bons principes qu’il n’en demeure toujours quelque impression dans leur esprit. Cette idée de vertu, quelque effacée qu’elle puisse être par la corruption du temps, donne pourtant toujours aux plus mauvais une espèce de répugnance pour le vice. Leurs cœurs, formés de bonne heure aux lois de l’honneur, s’en font une si forte habitude qu’ils ont peine de la corrompre entièrement, et le désir de la gloire qui les anime les fait passer en beaucoup de choses par-dessus le penchant de leur intérêt.\par
Mais il n’en est pas ainsi de ces gens de condition médiocre, par qui les États aristocratiques sont gouvernés. Les résolutions qui se prennent dans leurs conseils ne sont fondées sur un autre principe que sur celui de leur utilité. Ces corps formés de tant de têtes, n’ont point de cœur qui puisse être échauffé par le feu des belles passions. La joie qui naît des actions honnêtes, la honte qui suit les lâchetés, la reconnaissance des bienfaits, et le souvenir des services, lorsqu’ils sont partagés entre tant de personnes, s’affaiblissent enfin à tel point qu’elles ne produisent plus aucun effet, et il n’y a que l’intérêt seul qui, regardant les particuliers aussi bien que le général de l’État, puisse donner quelque règle à leur conduite.\par
De ces vérités, mon fils, l’instruction que vous pouvez tirer n’est pas qu’on doive s’abstenir absolument de toute sorte de société avec ces sortes d’États. Car, au contraire, je tiens qu’un prince habile doit savoir mettre toutes choses en usage pour parvenir à ses fins. Mais il faut seulement que dans le procédé que nous tenons avec eux, nous établissions pour principale maxime que, quoi que nous puissions faire pour eux, ou de fâcheux ou d’obligeant, ils ne manqueront jamais de nous rechercher toutes les fois qu’ils croiront y voir quelque profit et ne balanceront aussi jamais à nous quitter dès lors qu’ils trouveront quelque danger à nous suivre.\par
À l’égard du roi d’Angleterre, je ne pouvais prendre un meilleur chemin pour l’obliger à faire la paix avec moi, que de me mettre en état de lui faire fortement la guerre. Pour cela, je faisais dessein de lui enlever l’île de Jersey, qui était fort à ma bienséance, à cause du voisinage de la Bretagne, et qui me pouvait infiniment servir pour interrompre le commerce des Anglais. Car ce point était surtout de grande importance, et il n’y avait point de doute que, dès lors que ces insulaires seraient incommodés dans leur trafic, ils ne contraignissent incontinent leur prince à s’accommoder.\par
Ce fut aussi dans cette vue que je sommai les Hollandais de contribuer avec moi pour former durant l’automne une puissante escadre, avec laquelle j’espérais surprendre vingt frégates anglaises commandées pour escorter les marchands de Tanger. Mais les États, n’en étant pas demeurés d’accord, me firent perdre cette occasion.\par
Je faisais cependant bâtir de nouveaux vaisseaux et fondre du canon en divers lieux, au-dedans et au-dehors de mon royaume. Je faisais incessamment remplir mes magasins de toutes les choses nécessaires à la marine. Et afin que ma flotte fût plus tôt en état de se remettre en mer dans la campagne prochaine, je fis hiverner sur les lieux les troupes dont elle était armée, et contre l’usage ordinaire, je commandai que l’on entretînt tout l’hiver les officiers mariniers et les bons matelots.\par
D’ailleurs, j’entretenais commerce avec les catholiques d’Irlande, et sur quelque apparence de soulèvement en cette île, j’avais déjà destiné des troupes pour y passer. J’avais aussi quelque intelligence avec certains restes de la faction de Cromwell, d’où je reçus diverses propositions, et, entre autres, Sidney, gentilhomme anglais, me promettait d’exciter de grands soulèvements ; mais la proposition qu’il me fit d’avancer cent mille écus, me fit défier de ses paroles, et ne voulant pas exposer une si grande somme sur la foi d’un fugitif, je lui offris seulement vingt mille écus comptants, avec promesse de fournir le reste aux soulevés aussitôt qu’ils paraîtraient en état de me pouvoir servir.\par
Du côté d’Allemagne, je tâchais d’augmenter tous les jours mon crédit. Nonobstant la guerre que j’avais eue avec l’évêque de Münster, j’avais toujours agi fort honnêtement avec lui, comme avec un homme capable de divers desseins, et je m’étais si bien mis avec lui qu’aussitôt qu’il eut fait la paix avec les États de Hollande, il m’envoya offrir ses troupes. De ma part, j’eus quelque peine à me déterminer sur ce que je devais répondre à ses offres. Car d’un côté, je voyais qu’en les acceptant, non seulement elles grossissaient mes forces, mais encore elles diminuaient celles des ennemis, auxquels sans doute elles se donneraient après mon refus. Mais d’un autre côté, je considérais que j’étais encore alors chargé de mon armement de mer, qui allait à une très grande dépense et, ce qui me touchait le plus, il me semblait que c’était dès lors déclarer ouvertement mon dessein, et réveiller les soupçons de tous ceux qui s’y voulaient opposer.\par
J’avais fait proposer un traité à tous les princes voisins du Rhin, par lequel tous s’obligeaient de joindre leurs forces pour empêcher l’Empereur de jeter aucunes troupes dans les Pays-Bas, comme je leur promettais de ma part de ne faire point passer d’armée en Allemagne. Et déjà le comte Guillaume de Furstemberg les avait fait convenir du nombre d’hommes que chacun devait armer pour cet effet, moyennant deux millions, sur lesquels j’envoyai seulement cette année quatre cent mille livres pour commencer les levées, en attendant que les affaires s’avanceraient d’ailleurs.\par
Cependant, étant averti que la querelle émue entre l’électeur de Mayence et le Palatin pour ce droit de {\itshape vilfranc}, dont je crois vous avoir déjà parlé, s’échauffait de jour en jour, de telle sorte qu’elle pouvait consumer les forces de ces deux princes, j’envoyai vers eux Courtin, l’un des maîtres des requêtes de mon hôtel, pour les remettre en bonne intelligence.\par
Le marquis de Brandebourg semblait alors assez disposé à se joindre avec moi, ainsi qu’il s’en était lui-même expliqué à l’un de mes ministres, parce qu’il prétendait retirer le duché de Gueldre sur lequel il avait quelques droits.\par
Un théatin demeurait depuis quelque temps à ma cour, qui, paraissant seulement avoué par l’électrice de Bavière, m’avait néanmoins ouvert insensiblement une espèce de commerce avec ce duc, duquel j’avais depuis peu reçu diverses propositions.\par
Enfin j’avais encore diverses intelligences en Hongrie, pour y faire naître des affaires à l’Empereur en cas qu’il voulût se mêler des miennes. Et pour entretenir quelque correspondance entre tous ceux qui agissaient pour moi dans ces diverses cours, j’avais envoyé Milet au camp devant Brême, qui, en observant ce qui se traitait en ce lieu-là, était encore informé de ce qui se faisait dans les États voisins, et avait ordre de faire savoir à chacun de mes agents ce qui pouvait contribuer au bien de mon service.\par
Mais, quoique je travaillasse ainsi auprès des étrangers pour les mettre, autant qu’il se pouvait, dans mes intérêts, je mettais pourtant ma principale application à augmenter mes propres forces comme étant ce qui devait le plus sûrement contribuer au succès de mes desseins.\par
Il y avait longtemps que je prenais soin d’exercer les troupes de ma maison, et ce fut de là que je tirai presque tous les officiers des nouvelles compagnies que je levai, afin qu’ils y portassent la même discipline à laquelle ils étaient accoutumés ; et je remplis les places vacantes d’autres cavaliers choisis dans les vieux corps, ou de quelques jeunes gentilshommes qui ne pouvaient être instruits en meilleure école.\par
Cependant je faisais très souvent des revues, tant des nouvelles troupes, pour voir si elles étaient complètes, que des anciennes, pour reconnaître si elles n’étaient point affaiblies par les nouvelles levées. Car, je ne recommandais rien tant aux capitaines, à qui je délivrais mes commissions, que de prendre tous soldats nouveaux, parce qu’autrement c’eût été grossir le nombre des compagnies et la dépense de leur entretien sans augmenter le nombre des gens de guerre.\par
Mais pour prendre contre ce désordre une plus forte précaution et faire que mes troupes demeurassent toujours complètes, j’ordonnai que l’on m’envoyât de mois en mois les rôles des montres de tous les corps que je payais, quelque éloignés qu’ils pussent être, et pour connaître si j’étais en cela fidèlement servi, j’envoyais exprès des gentilshommes de toutes parts, pour surprendre et voir inopinément les troupes, ce qui tenait et les capitaines et les commissaires dans une continuelle obligation de faire leur devoir.\par
Mais afin que ni les uns ni les autres n’eussent plus lieu de s’excuser sur la désertion des soldats, qui était en effet un mal très ruineux pour les troupes, je me résolus d’y apporter des remèdes plus effectifs que ceux dont on s’était servi jusque-là ; et après avoir pris l’avis des gens les plus savants dans la guerre, je fis une ordonnance dont le fruit s’est fait connaître dans la suite du temps.\par
Je voulus aussi retrancher tous les sujets de contestation qui avaient si souvent causé du désordre dans nos troupes ; et sans m’étonner de toutes ces différentes prétentions que chacun des corps portait avec tant de chaleur que personne n’avait encore osé en décider, je réglai tous leurs rangs avec autant d’équité que je pus, et soutins si bien mon règlement par mon autorité que personne n’osa le contredire.\par
Pour égaler entre tous les régiments d’infanterie et la fatigue et l’honneur du service en toutes choses, je résolus que si la guerre de mer continuait, chacun servirait à son tour sur les vaisseaux ; et afin d’augmenter dans les matelots l’affection qu’ils témoignaient avoir pour mon service, je décidai en leur faveur une contestation qu’ils avaient avec les capitaines des vaisseaux pour leur solde, laquelle jusque-là ne leur avait pas toujours été bien fidèlement payée.\par
De ma part, je retranchais la plupart des dépenses que j’avais coutume de faire pour mon plaisir, mettant ma principale satisfaction à tenir mes gens en bon ordre.\par
Je ne pus m’empêcher d’augmenter à diverses fois les compagnies de mes gardes du corps, à cause du grand nombre des gens de qualité ou de service qui s’empressaient continuellement pour y avoir place. Mais enfin je les fixai à huit cents maîtres, tous anciens soldats ou officiers réformés, à la réserve de vingt jeunes gentilshommes que je laissai par compagnie pour y apprendre leur métier.\par
L’empressement de me servir était si grand que ma plus grande peine en toutes les occasions qui s’offraient de faire quelque chose, était de retenir ceux qui se présentaient, comme il parut lorsque je voulus jeter du monde sur mes vaisseaux de Dieppe. Car, outre les gens commandés, il se présenta un si grand nombre de volontaires, que je fus obligé de les refuser tous, et même d’en châtier quelques-uns de la première qualité, qui, sachant qu’ils seraient refusés, se mirent en chemin sans m’en demander congé.\par
J’entretenais une discipline si exacte dans mes troupes, qu’en ayant envoyé en divers temps chez mes alliés, en Italie, en Hongrie et en Hollande, elles ne donnèrent jamais le moindre sujet de plainte quoiqu’elles eussent eu quelquefois d’assez grands sujets de mécontentement. Aussi avais-je soin de les faire payer partout exactement par un trésorier que je tenais à leur suite, et même en Hollande j’augmentai leur paye ordinaire parce que je sus que les vivres y étaient plus chers qu’ailleurs.\par
Et toutes les fois que j’assemblais des troupes au-dedans de mon royaume, j’avais soin d’envoyer des commissaires qui faisaient sur le lieu des provisions de toutes les choses nécessaires à leur subsistance, au prix qu’elles avaient coutume de s’acheter, et les vendaient après aux soldats sur un pied proportionné à leur paye, afin qu’ils eussent toujours de quoi vivre sans être à charge aux paysans.\par
J’avais tant de considération sur ce point, qu’allant en Picardie au mois de mars pour y faire une grande revue, je ne voulus pas loger à Compiègne, qui était le lieu le plus proche et le plus commode pour moi et pour ma maison, parce que c’était une ville capable de loger une bonne partie de mon infanterie, qui sans cela se fût trouvée dans des villages où l’on eût eu plus de peine à la faire vivre régulièrement.\par
Comme l’ustensile était un des principaux sujets de querelle entre l’hôte et le soldat, ce fut une des choses que je pris plus de soin de régler, fixant celle des fantassins à un sol, celle des cavaliers à trois, dont un tiers seulement était payé par l’hôte, un autre par le corps de la ville, et le dernier par le général de l’élection. Et, en cas que, sur l’exécution de mes règlements, il arrivât quelque contravention ou quelque dispute, j’ordonnai aux commissaires et aux intendants de les régler sans aucune préférence entre l’habitant et le soldat. Et moi-même ayant appris qu’un capitaine d’Auvergne avait pris trois cents livres des habitants de Rethel pour les exempter d’un séjour, je le cassai sans vouloir entendre mille personnes de qualité qui m’importunèrent pour lui.\par
Ce n’est pas que je ne susse bien que dans les gens de basse condition dont se font les soldats et quelquefois même les petits officiers, l’esprit de libertinage est ordinairement l’un des principaux motifs qui font suivre la profession militaire, et qu’il s’est trouvé depuis peu des chefs qui ont fait longtemps subsister de grosses armées sans leur donner d’autre solde que la licence de piller partout. Mais cet exemple ne doit être imité que par ceux qui, n’ayant rien à perdre, n’ont plus rien à ménager. Car tout prince qui chérira sa réputation avec un peu de délicatesse, ne doutera pas qu’elle ne soit aussi bien engagée à défendre le bien de ses sujets du pillage de ses propres troupes que de celles de ses ennemis. Et celui qui entendra ses affaires ne manquera pas de s’apercevoir que tout ce qu’il laisse prendre sur ses peuples, en quelque manière que ce puisse être, ne se prend jamais qu’à ses dépens, parce qu’il est manifeste que plus les provinces sont épuisées, soit par les gens de guerre ou par quelqu’autre cause, moins elles sont capables de contribuer à tout le reste des charges publiques.\par
C’est une grande erreur parmi les souverains de s’approprier certaines choses et certaines personnes comme si elles étaient à eux d’une autre façon que le reste de ce qu’ils ont sous leur empire. Les deniers qui sont dans leur cassette, ceux qui demeurent entre les mains de leurs trésoriers et ceux qu’ils laissent dans le commerce de leurs peuples, doivent être par eux également ménagés. Les troupes qui sont sous leur nom ne sont pas pour cela plus à eux que celles auxquelles ils donnent des chefs particuliers. Et ceux qui suivent le métier des armes ne sont ni plus obligés ni plus utiles à leur service que le reste de leurs sujets.\par
Chaque profession contribue, en sa manière, au soutien de la monarchie. Le laboureur fournit par son travail la nourriture à tout ce grand corps ; l’artisan donne par son industrie toutes les choses qui servent à la commodité du public ; et le marchand assemble de mille endroits différents tout ce que le monde entier produit d’utile ou d’agréable pour le fournir à chaque particulier au moment qu’il en a besoin. Les financiers, en recueillant les deniers publics, servent à la subsistance de l’État ; les juges, en faisant l’application des lois, entretiennent la sûreté parmi les hommes ; et les ecclésiastiques, en instruisant les peuples à la religion, attirent les bénédictions du Ciel et conservent le repos sur la terre.\par
C’est pourquoi, bien loin de mépriser aucune de ces conditions, ou d’en favoriser l’une aux dépens de l’autre, nous devons être le père commun de toutes, prendre soin de les porter toutes, s’il se peut, à la perfection qui leur est convenable, et nous tenir persuadés que celle même que nous voudrions gratifier avec injustice, n’en aura ni plus d’affection ni plus d’estime pour nous, pendant que les autres tomberont avec raison dans la plainte et dans le murmure.\par
Si pourtant, malgré toutes ces raisons, vous ne pouvez vous défendre, mon fils, de cette secrète prédilection que les âmes généreuses ont presque toujours pour la profession des armes, prenez garde surtout que cette bienveillance particulière ne vous porte jamais à tolérer les emportements de ceux qui la suivent, et faites que l’affection que vous aurez pour eux, paraisse à prendre soin de leur subsistance et de leur fortune plutôt qu’à laisser corrompre leurs mœurs.
\section[{Année 1667}]{Année 1667}\renewcommand{\leftmark}{Année 1667}

\noindent Cette année commença par les couches de la Reine, lesquelles, paraissant un peu trop avancées, me donnèrent une juste appréhension pour elle : car je puis dire ici en passant qu’elle méritait le soin que j’en avais, et que le Ciel n’a peut-être jamais assemblé dans une seule femme plus de vertu, plus de beauté, plus de naissance, plus de tendresse pour ses enfants, plus d’amour et de respect pour son mari. Mais enfin ma crainte finit par la naissance d’une fille.\par
Ce fut dans le commencement de cette même année que je fis mettre la dernière main au traité qui se négociait en Allemagne pour empêcher le passage des troupes de l’Empereur.\par
J’en avais aussi projeté un avec le roi de Portugal, par lequel il me promettait de ne traiter de quatre ans avec l’Espagne ; mais avant que nous l’eussions signé, nous eûmes occasion d’en faire un autre dont je vous parlerai en son temps.\par
Cependant la reine de Pologne continuait à me demander du secours ; mais surtout après la mort de Lubomirski, comme elle croyait voir plus de jour que j’avais au rétablissement de ses affaires, elle me pressa plus fortement, et me dépêcha Morstain, son grand-référendaire, par lequel elle me fit entendre que si je voulais, sous prétexte de la secourir contre le Turc, lui envoyer un corps de troupes françaises commandé par le prince de Condé, elle pourrait calmer son royaume, et faire réussir l’élection du duc d’Enghien.\par
La proposition était glorieuse et bien pensée, mais dans la conjoncture où je me trouvais, l’exécution en était difficile. J’avais encore la guerre avec les Anglais ; j’étais près de la commencer avec les Espagnols ; je ne doutais point du parti que prendrait l’Empereur ; je savais la répugnance que les Hollandais avaient à mon accroissement, et j’étais toujours en doute de la Suède ; si bien que devant mettre dans mes seules forces tout l’espoir du succès de mes desseins, il était fâcheux de les diminuer.\par
Et néanmoins, sollicité vivement par le désir d’augmenter la gloire de ma couronne, je consentis à ce que l’on désirait, et les principales raisons qui m’y portèrent furent qu’en effet la guerre du Turc était un prétexte très favorable pour faire passer le prince de Condé ; que le roi de Pologne, déjà incommodé, venant une fois à mourir, la reine, sa femme, serait sans puissance ; que cette princesse même, ayant été menacée depuis peu d’apoplexie, pouvait nous manquer dans le besoin ; que les Suédois semblaient alors en disposition de l’assister de leur part. Mais, au vrai, la considération qui me touchait le plus, était qu’on trouvait rarement l’occasion de faire présent d’une couronne et de l’assurer à la France.\par
Suivant cette résolution, j’avais aussitôt fait demander passage à l’électeur de Brandebourg, et me disposais à faire partir mes troupes par terre ou par mer, selon que j’aurais la guerre ou la paix avec l’Angleterre. Mais bientôt après j’appris d’Allemagne que l’on n’accordait point les passages ; de Suède, que l’on ne voulait en rien contribuer à cette entreprise ; et de Pologne même, que la reine ne croyait pas pouvoir faire proposer l’élection. Sur quoi, je pensai qu’il n’était pas à propos que j’entreprisse tout de ma part, tandis que d’ailleurs on ne voulait rien faire.\par
Cependant je mêlais le soin des affaires du dedans à celles du dehors. Pour remédier aux désordres qui arrivaient ordinairement dans Paris, j’en voulus rétablir la police ; et après m’être fait représenter les anciennes ordonnances qui ont été faites sur ce sujet, je les trouvai si sagement digérées, que je me contentai d’en rétablir plusieurs articles abolis par la négligence des magistrats ; mais j’y ajoutai quelques précautions pour les faire mieux observer à l’avenir, principalement sur le port des armes, sur le nettoiement des rues, et sur quelques autres points particuliers, pour l’exacte observation desquels je formai même un conseil exprès.\par
Je crus aussi qu’il était de la police générale de mon royaume de diminuer ce grand nombre de religieux, dont la plupart, étant inutiles à l’Église, étaient onéreux à l’État. Dans cette pensée, je me persuadai que, comme rien ne contribuait tant à remplir les couvents que la facilité que l’on apportait à y recevoir les enfants de trop bonne heure, il serait bon de différer à l’avenir le temps des vœux ; qu’ainsi les esprits irrésolus, ne trouvant pas sitôt la porte des cloîtres ouverte, s’engageraient, en attendant, en quelque autre profession, où ils serviraient le public ; que même la plus grande partie, se trouvant dans un établissement, y demeurerait pour toujours, et formerait de nouvelles familles, dont l’État serait fortifié ; mais que l’Église même y trouverait son avantage, en ce que les particuliers, ne s’engageant plus dans les couvents sans avoir eu le loisir d’y bien penser, y vivraient après avec plus d’exemple.\par
Mon conseil, auquel j’avais communiqué ce dessein, m’y avait plusieurs fois confirmé par ses suffrages ; mais, sur le point de l’exécution, je fus arrêté par ces sentiments de respect que nous devons toujours avoir pour l’Église, en ce qui est de sa véritable juridiction, et je résolus de ne déterminer ce point que de concert avec le Pape. Et néanmoins, en attendant que je l’en eusse informé, je voulus empêcher le mal de croître par tous les moyens qui dépendaient purement de moi. Ainsi, je défendis tous les nouveaux établissements de monastères, je pourvus à la suppression de ceux qui s’étaient faits contre les formes, et je fis agir mon procureur général pour régler le nombre de religieux que chaque couvent pouvait porter.\par
À l’égard du règlement général de la justice, dont je vous ai déjà parlé, voyant un bon nombre d’articles rédigés en la forme que j’avais désirée, je ne voulus pas plus longtemps priver le public du soulagement qu’il en attendait, mais je ne crus ni les devoir simplement envoyer au Parlement, de peur que l’on y fît quelque chicane qui me fâchât, ni les porter aussi d’abord moi-même, de crainte que l’on ne pût alléguer un jour qu’ils auraient été vérifiés sans aucune connaissance de cause. C’est pourquoi, prenant une voie de milieu qui remédiait à la fois à ces deux inconvénients, je fis lire tous les articles chez mon chancelier, où se trouvaient les députés de toutes les Chambres, avec des commissaires de mon conseil ; et quand, dans la conférence qu’ils y faisaient, il se formait quelque difficulté raisonnable, elle m’était incontinent apportée pour y pourvoir ainsi que j’avisais. Après laquelle discussion, j’allai enfin en personne en faire publier l’édit.\par
Je réformai aussi dans le même temps la manière dont j’avais moi-même accoutumé de rendre la justice à ceux qui me la demandaient immédiatement : car je ne trouvais pas que la forme en laquelle j’avais jusque-là reçu leurs placets fût commode ni pour eux ni pour moi. Et, en effet, comme la plupart des gens qui ont des demandes ou des plaintes à me faire ne sont pas de condition à obtenir des entrées particulières auprès de moi, ils avaient peine à trouver une heure propre pour me parler et demeuraient souvent plusieurs jours à ma suite, éloignés de leurs familles et de leurs fonctions. C’est pourquoi je déterminai un jour de chaque semaine, auquel tous ceux qui avaient à me parler ou à me donner des mémoires avaient la liberté de venir dans mon cabinet, et m’y trouvaient précisément appliqué à écouter ce qu’ils désiraient me dire.\par
Mais, outre ces soins qui regardaient le public, je ne manquais aucune occasion de gratifier les particuliers avec justice. Ayant augmenté le nombre de mes gardes du corps, je pris occasion d’y créer de nouvelles charges en faveur de plusieurs hommes qui m’avaient bien servi.\par
Me souvenant de ce que La Feuillade avait fait en Hongrie, je consentis à faire passer en sa personne la qualité de duc de Roannez, dont la terre lui avait été cédée par mariage, et lui donnai même quelque argent pour faciliter l’exécution de ce contrat.\par
Je permis à mon procureur général de résigner à son fils cette charge, qui n’avait pas coutume de passer ainsi de père en fils.\par
Je soulageai en ce que je pus, et de mon autorité et de mes finances, plusieurs négociants, dont la guerre de mer avait mis les affaires en désordre.\par
Je secourus aussi, par divers moyens, ceux dont le receveur des consignations avait depuis peu emporté les deniers et j’accordai un long et fâcheux différend qui s’était formé entre les trois communautés des Carmélites de Paris.\par
Mais il est vrai que mes sujets faisaient chaque jour paraître de leur part plus d’ardeur et plus d’empressement pour mon service. La négligence qu’on avait eue de tout temps pour la marine m’avait quelquefois fait appréhender de ne pas trouver tous les matelots nécessaires pour armer le nombre de vaisseaux que j’équipais. Mais au moindre témoignage que je donnai de ma volonté, il s’en trouva plus que je n’en voulus, des provinces entières m’ayant offert d’abandonner leurs maisons pour mon service, et de n’y laisser que les femmes et les enfants.\par
Au premier bruit de la guerre de Flandre, ma cour se grossit en un instant d’une infinité de gentilshommes qui me demandaient de l’emploi. Les capitaines de tous les vieux corps me supplièrent de leur permettre de faire des recrues à leurs frais. D’autres ne demandaient que ma simple commission pour lever des compagnies nouvelles ; et tous, dans leurs divers emplois, cherchaient à l’envi des moyens de me faire connaître leur zèle.\par
Il est agréable assurément de recevoir de pareilles marques de l’estime et de l’affection de ses sujets : tous les princes demeurent d’accord que c’est le trésor le plus précieux qu’ils puissent jamais posséder ; tous l’estiment, tous le désirent, mais tous ne recherchent pas assez les moyens de l’acquérir.\par
Car pour y parvenir, mon fils, il faut diriger à cette fin toutes nos actions et toutes nos pensées ; il faut la préférer seule à tous les autres biens, et fuir, comme le plus grand mal du monde, tout ce qui peut nous en éloigner. C’est aux hommes du commun à borner leur application dans ce qui leur est utile et agréable ; mais les princes, dans tous leurs conseils, doivent avoir pour première vue d’examiner ce qui peut leur donner ou leur ôter l’applaudissement public. Les rois, qui sont nés pour posséder tout et commander à tout, ne doivent jamais être honteux de s’assujettir à la renommée : c’est un bien qu’il faut désirer sans cesse avec plus d’avidité, et qui seul, en effet, est plus capable que tous les autres de servir au succès de nos desseins. La réputation fait souvent elle seule plus que les armées les plus puissantes. Tous les conquérants ont plus avancé par leur nom que par leur épée ; et leur seule présence a mille fois abattu sans efforts des remparts capables de résister à toutes leurs forces assemblées.\par
Mais ce qu’il y a d’important à remarquer, c’est que ce bien si noble et si précieux est aussi le plus fragile du monde, que ce n’est pas assez de l’avoir acquis si l’on ne veille continuellement à sa conservation ; et que cette estime, qui ne se forme que par une longue suite de bonnes actions, peut être en un moment détruite par une seule faute que l’on commet.\par
Encore n’attend-on pas toujours que nous ayons failli pour nous condamner. C’est souvent assez que notre fortune s’affaiblisse pour diminuer l’opinion de notre vertu : et comme il arrive à l’homme heureux que tous les avantages qu’il a reçus du hasard tournent chez les peuples à sa gloire, il arrive de même aux infortunés qu’on leur impute à manque de prudence tout ce qui se fait contre leurs désirs.\par
Le caprice du sort, ou plutôt cette sage Providence qui dispose souverainement de nos intérêts par des motifs au-dessus de notre portée, se plaît quelquefois à rabattre ainsi le faste des hommes les plus élevés, pour les obliger, au milieu de nos plus grands avantages, à reconnaître la main dont ils tiennent tout, et à mériter, par un continuel aveu de leur dépendance, le concours nécessaire au succès de leurs desseins.\par
En ce même temps, les Vénitiens, menacés de perdre Candie, donnèrent ordre à leur ambassadeur de se faire assister par le nonce du Pape pour me demander quelque secours, mais je ne leur pus faire de réponse favorable, parce que les grands engagements où je me trouvais ne me permettaient pas de leur donner alors un corps de troupes considérable ; et je crus que de leur en donner un faible, c’était perdre inutilement les gens que j’y enverrais, étant certain que les petits corps ne reviennent jamais de ces longs voyages.\par
Ce n’est pas que, dans le vrai, je n’eusse bien désiré de les assister ; car, outre les intérêts communs du christianisme, j’avais, en mon particulier, été si mal satisfait de la Porte touchant l’entreprise des Génois, que je m’étais résolu de n’y plus parler de cette affaire, me réservant à en tirer raison de Gênes même lorsque j’aurais le loisir d’y penser.\par
La nouvelle qui arriva dans ce temps-là de l’extrémité du Pape me fit donner ordre aux cardinaux français d’être toujours prêts à se mettre à la voile, en cas qu’il survînt quelque chose de plus fâcheux, comme en effet il arriva bientôt après. Et les soins que je pris en cette occasion contribuèrent assurément à bien remplir cette grande place.\par
Cependant les Hollandais me faisaient continuellement demander qu’on réglât l’article qui était indécis sur le salut de nos amiraux, et coloraient leur empressement des plus belles raisons du monde, quoique la seule véritable fût qu’ils étaient persuadés que la guerre de mer durant encore me pouvait porter à quelque condescendance pour eux, au lieu qu’après la paix faite je conserverais avec plus de fermeté les avantages qui m’étaient dus. Mais comme leur pensée m’était connue, je les remettais de jour en jour, étant bien informé que chez eux-mêmes ils étaient fort pressés de faire la paix.\par
Déjà quatre de leurs provinces avaient déclaré qu’elles ne fourniraient plus aux frais de la guerre, et les autres étaient partagées sur ce sujet, parce que comme les politiques, par l’appréhension qu’ils avaient de mon accroissement, s’opposaient à la conclusion du traité, le peuple, au contraire, qui désirait surtout le rétablissement de son commerce, voulait que l’affaire se terminât ; et cela même passa si loin que je craignis de voir diviser cette république, et fus obligé d’employer mon entremise pour mettre la modération dans les esprits.\par
Au reste, il n’y avait plus à régler qu’un seul article touchant l’île de Polleron. Les Anglais prétendaient, de leur part, qu’elle devait leur être rendue, par les termes exprès du traité de 1662 ; et les Hollandais soutenaient, au contraire, qu’ils y avaient pleinement satisfait, en livrant alors l’île contestée, mais que les Anglais l’ayant abandonnée bientôt après, ils avaient pu de nouveau s’y établir, comme dans une terre qui était sans seigneur. Mais quoiqu’il en fût en effet, comme cette île était d’une valeur fort médiocre, il ne semblait pas que ni l’une ni l’autre des parties s’y dût fortement attacher, si bien que la paix déjà semblait infaillible.\par
Aussi la maison d’Autriche, ne pouvant plus s’imaginer d’autre expédient pour la rompre, me fit proposer la médiation de l’Empereur, prétendant que dans le détail des articles qui restaient encore à digérer, ses agents trouveraient peut-être moyen d’exciter quelque nouvelle contestation. Mais comme le motif de cette proposition n’était pas difficile à pénétrer, je ne manquai pas de prétexte pour m’en défendre disant que les Suédois avaient déjà été reçus pour médiateurs tant par moi que par les autres parties qui avaient intérêt dans ce traité, et qu’après que les choses avaient été portées par leur entremise au point où elles étaient alors, il n’était pas juste de leur donner un associé qui partageât avec eux la gloire du succès. À quoi le résident de l’Empereur ne manqua pas de repartir ; mais je sortis de cette conversation en rejetant toujours ses offres avec toutes les honnêtetés possibles.\par
Les Espagnols, pour me détourner par une autre voie de porter mes armes contre eux, me firent proposer un traité de commerce. Et depuis encore, dans le même dessein, le marquis de la Fuente, prenant congé de moi, me dit, de la part de la Reine régente, tout ce qu’il put de plus engageant, afin d’attirer de moi de pareilles civilités, desquelles aussitôt après il tâcha de prendre avantage, en faisant entendre dans le public que je lui avais positivement promis de ne point rompre avec les Espagnols, comme s’il eût espéré par là m’engager à ne le pas dédire. Mais parce que dans le vrai, je ne lui avais dit que des civilités fort générales, je fis fort peu de cas de tous ses discours, travaillant sans cesse à me tenir prêt et pour la guerre de mer et pour celle de terre, selon ce qui pouvait arriver.\par
Car enfin je craignais toujours que, comme j’avais beaucoup de passion pour faire réussir la paix d’Angleterre, je me pouvais tromper plus aisément qu’un autre dans les apparences que j’y croyais voir, et je tenais pour maxime qu’en tout ce qui est douteux, le seul moyen d’agir avec assurance est de faire son compte sur le pis.\par
Il n’est que trop naturel aux hommes de se promettre avec facilité ce qu’ils désirent avec ardeur, et nous ne saurions nous garantir d’un défaut si commun qu’en nous défiant de nos propres pensées dans toutes les choses où nous avons trop de penchant. Il n’est rien de si important ni de plus difficile au prince que de savoir combien et jusqu’où il doit estimer sa propre opinion. Je vous ai dit ailleurs, et il est vrai, qu’un souverain peut avoir cette persuasion en faveur de lui-même, que, comme il est d’un rang au-dessus des autres hommes, il voit aussi les choses qui se présentent d’une manière plus parfaite qu’eux, et qu’il se doit plus fier à ses propres lumières qu’aux rapports qui lui sont faits du dehors.\par
Mais je vous avertis ici que cette maxime ne se doit pas appliquer également à toutes nos différentes fonctions. Il en est sans doute de certaines, où tenant, pour ainsi dire, la place de Dieu, nous semblons être participants de sa connaissance, aussi bien que de son autorité, comme, par exemple, en ce qui regarde le discernement des esprits, le partage des emplois et la distribution des grâces, dans lesquelles choses nous décidons avec plus de succès par notre propre suffrage que par celui de nos conseillers, parce qu’étant postés dans une sphère supérieure, nous sommes plus éloignés qu’eux des petits intérêts qui nous pourraient porter à l’injustice. Mais il faut confesser de bonne foi qu’il se trouve aussi d’autres rencontres où quittant, ce semble, le personnage de souverains et d’indépendants, nous devenons aussi intéressés, et peut-être même davantage, que les moindres particuliers, parce que, plus les objets où nous aspirons sont grands et relevés, plus ils sont propres à troubler la tranquillité nécessaire pour former un juste raisonnement. Le feu des plus nobles passions, comme celui des plus obscures, produit toujours un peu de fumée, qui offusque notre raison. On admire souvent que, de plusieurs qui voient et entendent la même chose, à peine en est-il deux dont le rapport se trouve conforme l’un à l’autre, et cependant cette variété ne vient que de la différence des intérêts et des passions qui se trouvent toujours entre les hommes, lesquels, même sans s’en apercevoir, accommodent tout ce qu’ils voient au-dehors au mouvement qui domine dans leur âme.\par
C’est une des plus fortes raisons qui a obligé de tout temps les princes à tenir auprès d’eux des conseillers, et qui les doit même porter à entendre plus favorablement que les autres ceux qu’ils ne rencontrent pas de leur sentiment. Tandis que nous sommes dans la puissance, nous ne manquons jamais de gens qui s’étudient à suivre nos pensées et à paraître en tout de notre avis. Mais nous devons craindre de manquer, au besoin, de gens qui sachent nous contredire, parce que notre inclination paraît quelquefois si à découvert, que les plus hardis craignent de la choquer, et cependant il est bon qu’il y en ait qui puissent prendre cette liberté. Les fausses complaisances que l’on a pour nous en ces occasions nous peuvent nuire beaucoup plus que les contradictions les plus opiniâtres. Si nous nous trompons en notre avis, celui qui nous adhère achève de nous précipiter dans l’erreur, au lieu que, lors même que nous avons raison, celui qui nous contredit ne laisse pas que de nous être utile, quand ce ne serait qu’à nous faire chercher des remèdes aux inconvénients qu’il a proposés, et à nous laisser, en agissant, la satisfaction d’avoir auparavant examiné toutes les raisons de part et d’autre.\par
Dans l’accommodement que je désirais faire avec l’Angleterre le point qui m’arrêtait le plus était que les Anglais, encore affligés d’avoir perdu les îles Occidentales, prétendaient surtout y être rétablis : car, outre l’intérêt général que la France y pouvait avoir, j’étais particulièrement touché par la considération de la nouvelle Compagnie que j’avais formée pour ce commerce ; mais, d’autre part, considérant aussi la conjoncture où je me trouvais, la Flandre dépourvue d’argent et d’hommes, l’Espagne gouvernée par une princesse étrangère, l’Empereur incertain dans ses résolutions, la maison d’Autriche réduite à deux têtes, ses forces épuisées par diverses guerres, ses partisans presque tous refroidis, et mes sujets pleins de zèle pour mon service, je crus que je ne devais pas perdre une occasion si favorable d’avancer mes desseins, ni mettre en comparaison le gain de ces îles éloignées avec la conquête des Pays-Bas.\par
C’est pourquoi je pris en moi-même la résolution d’accorder la demande qui m’était faite. Et néanmoins, pour ne la pas déclarer sans en tirer quelque fruit important, je fis demander au roi de la Grande-Bretagne si, moyennant la parole secrète que je lui donnerais de passer cet article dans le traité, il voudrait aussi de sa part me promettre de ne prendre d’un an aucun engagement contre moi.\par
Mais tandis que cela se négociait entre nous, il me donna sujet de défiance par la proposition qu’il fit à mon insu aux États d’aller traiter de la paix à La Haye. Car, comme cette ville était pleine d’un fort grand peuple et fort facile à émouvoir, je ne doutai point que ce ne fût un choix fait de concert avec l’Espagne, dans le dessein d’y faire tramer des brigues par leurs ministres, soit pour rétablir l’autorité du prince d’Orange ou pour détacher cette république d’avec moi.\par
Mais j’éludai leur artifice en le faisant connaître aux États, qui, par mon avis, répondirent au roi d’Angleterre que, s’il voulait, on irait traiter en son royaume, ou que, s’il aimait mieux négocier chez eux, ils lui donnaient le choix de Bréda, de Bois-le-Duc et de Maëstricht, parce que, disaient-ils, La Haye n’étant pas fermée ne pouvait donner aux députés la sûreté convenable à leurs fonctions.\par
Mais le roi de la Grande-Bretagne, qui reconnut incontinent le véritable sujet de cette réponse, fut si fâché de voir son dessein découvert, qu’il ne voulut d’abord accepter aucune des places proposées. Et néanmoins, bientôt après, il choisit Bréda, témoignant même que c’était en ma considération qu’il apportait cette facilité aux affaires.\par
Ainsi, nos agents assemblés commencèrent à travailler ouvertement à la paix. Et je repris aussi, de ma part, la négociation commencée en secret, pour m’assurer au plus tôt de ce qui pouvait regarder mon dessein : car comme je ne doutais pas que dans les divers intérêts des différentes parties il ne se formât de jour en jour des contestations qui tireraient les choses en longueur, je crus que j’avais intérêt de me détacher de l’affaire pour profiter d’un temps qui m’était précieux.\par
La principale condition à laquelle je m’obligeais dans ce traité, était de rétablir les Anglais dans les îles Occidentales ; et, de leur part, ils me promettaient que l’article de l’île de Polleron n’empêcherait point la paix générale, et que, quand même elle ne serait pas conclue dans un an, ils ne traverseraient en rien mes projets. Pour dérober aux États de Hollande la connaissance de ces conventions, elles ne furent exprimées que dans des lettres missives écrites de ma main et de celle du roi d’Angleterre à la Reine ma tante et sa mère, qui en demeurait dépositaire entre nous. Et cela fait, je commençai à me préparer ouvertement à la guerre de Flandre.\par
Mais afin de ne rien oublier qui pût justifier mon procédé, je fis publier un écrit où mes droits étaient établis, et envoyai nouvel ordre en Espagne pour demander les États qui m’appartenaient, et pour déclarer que si on me les refusait, je m’en mettrais en possession moi-même, ou, du moins, de quelque chose d’équivalent. La Reine régente répondit que le testament du feu roi, son mari, défendant expressément l’aliénation de toutes les terres qu’il avait possédées, elle ne pouvait passer par-dessus cette loi. Mais Castel Rodrigo, qui me voyait de plus près qu’elle, ne témoigna pas tant de fermeté : car à peine étais-je parti de Saint-Germain, que je reçus de lui une lettre par laquelle, après quelques remontrances assez mal digérées, il me proposait de donner des députés, s’assurant, disait-il, que la Reine, sa maîtresse, entrerait dans un raisonnable accommodement. Mais comme il était aisé de voir que la seule crainte de mes armes lui faisait faire cette proposition, je ne fis autre réflexion sur cette lettre que pour y remarquer la frayeur dont celui qui l’écrivait était saisi.\par
Je me rendis le 19 mai dans Amiens, où j’avais résolu de voir faire l’assemblée de mes troupes. Et parce que je savais que les Espagnols manquaient principalement de gens de guerre, je leur voulus donner une égale terreur de tous côtés, afin qu’étant obligés de partager dans un grand nombre de garnisons le peu de forces qu’ils avaient, ils demeurassent partout également faibles.\par
Dans ce dessein, je faisais marcher un corps vers la mer sous le maréchal d’Aumont ; le marquis de Créqui en menait un autre du côté du Luxembourg ; il s’en formait un troisième sous Duras aux environs de La Fère, et j’en assemblais moi-même un quatrième vers Amiens.\par
Ma première pensée fut toujours de commencer par Charleroi : car, de l’importance dont était cette place, j’étais bien aise de m’en emparer, tandis que les fortifications encore nouvelles étaient plus faciles à ruiner. Et quoique dès Amiens je fusse averti que les Espagnols la ruinaient, je ne changeai pas pour cela de dessein, parce qu’en même temps j’appris que ceux à qui la démolition en avait été commandée avaient eu tant d’impatience d’en sortir qu’ils avaient laissé les dehors entiers.\par
Ainsi, j’envoyai devant moi le comte de Sault avec quinze cents hommes d’infanterie, et Podwits avec douze cents chevaux pour s’en saisir, lesquels je suivis aussitôt avec mon armée. En sorte que, dès l’ouverture de la campagne, je profitai sans coup férir de la dépense et des soins que Castel Rodrigo avait employés depuis deux ans à bâtir cette place.\par
Cependant le maréchal d’Aumont, ayant ordre d’aller à Bergues, la prit l’épée à la main ; d’où, passant aussitôt à Furnes, il ne trouva guère plus de résistance. Armentières et La Bassée ayant été abandonnées avant qu’on les attaquât, j’avais envoyé trois cents hommes pour se saisir de la première à cause de son pont sur la Lys : mais comme j’appris qu’elle était en tel état que j’y pouvais rentrer à toute heure, je ne voulus pas y laisser des hommes qui me pouvaient servir autre part.\par
De Charleroi, j’avais eu dessein d’aller d’abord à Bruxelles : mais voyant que mon infanterie, composée la plupart de nouveaux soldats, pourrait se rebuter ou se ruiner par un siège de longue durée, je résolus depuis d’attaquer Tournay, qui se pouvait prendre en bien moins de temps, et qui ne laissait pas d’être une grande ville et très avantageusement située.\par
Mais la difficulté était, qu’ayant fait premièrement état de m’avancer dans le Brabant, mon canon et mes vivres avaient marché de ce côté-là : en sorte que, sur le changement de ma résolution, il fallut donner de nouveaux ordres, afin que, ni dans la marche qu’il fallait faire au travers du pays ennemi, ni dans le siège que je formerais ensuite, mes gens ne pussent manquer de rien.\par
Car ce n’est pas assez, mon fils, de faire de vastes entreprises, sans penser comment les exécuter. Ces projets que forme notre valeur nous semblent d’abord les plus beaux du monde : mais ils ont peu de solidité, s’ils ne sont soutenus par une prévoyance qui sache disposer en même temps toutes choses qui doivent y concourir.\par
C’est en ce point sans doute que se peut voir une des principales différences qui sont entre les bons et les mauvais capitaines ; et jamais un habile général n’entreprend une affaire de durée sans avoir examiné par le menu d’où il tirera toutes les choses nécessaires pour la subsistance des gens qu’il conduit. Dans les autres désastres qui peuvent ruiner une armée, on peut presque toujours accuser ou la lâcheté des soldats, ou la malignité de la fortune. Mais dans le manquement de vivres, la prévoyance du général est la seule à qui l’on s’en prend : car, comme le soldat doit à celui qui commande l’obéissance et la soumission, le commandant doit à ses troupes la précaution et le soin de leur subsistance. C’est même une espèce d’inhumanité de mettre d’honnêtes gens dans un danger dont leur valeur ne peut les garantir, et où ils ne se peuvent consoler de leur mort par l’espérance d’aucune gloire.\par
Mais outre ces considérations, qui sont communes à tous les généraux, le prince qui commande en personne en doit avoir de toutes particulières. Comme la vie de ses sujets est son propre bien, il doit avoir bien plus de soin de la conserver ; et, comme il sait qu’ils ne s’exposent que pour son service, il doit pourvoir avec bien plus de tendresse à tous leurs besoins.\par
Pendant que l’on exécutait les ordres que j’avais donnés sur ce sujet, ne voyant rien à faire dans mon camp, je pris le temps de revenir vers ma frontière, où la Reine se rendit de son côté.\par
Durant ce temps-là, le duc de Lorraine témoignait une grande irrésolution touchant les troupes qu’il m’avait promises : car dans le fond il eût sans doute bien voulu se dispenser de me tenir parole, et n’osait néanmoins ouvertement y manquer. D’autre part, il se figurait qu’étant si près de l’Empereur, et n’ayant plus de places fortes à lui opposer, il demeurerait sans armes, exposé au ressentiment de ce prince ; mais d’autre part il voyait aussi qu’après s’être engagé vers moi, il était dangereux de se dédire, puisque j’étais alors plus que personne en état de m’en ressentir : tellement que, sans se déterminer ni d’un côté ni d’autre, il répondait toujours ambigument. Mais enfin, comme je connaissais la trempe de son esprit, je me persuadai que, si je le pouvais une fois mettre dans la nécessité de choisir sur-le-champ, il n’aurait pas assez de hardiesse pour se résoudre absolument à me fâcher : ainsi je lui fis dire un jour qu’il fallait précisément que ses troupes partissent d’auprès de lui le lendemain, parce que mes mesures étaient prises là-dessus. À quoi il obéit, comme je l’avais prévu.\par
Après avoir été quatre jours auprès de la Reine, je retournai au camp de Charleroi, et pris ma route par le milieu du pays ennemi pour donner une égale terreur à toutes leurs places.\par
Cependant mes ordres étaient donnés pour investir Tournay de trois endroits différents. Du côté de la mer, le maréchal d’Aumont y marchait avec sa cavalerie ; les Lorrains, que j’avais envoyés en Artois, s’y devaient rendre de ce côté-là ; et j’y venais en personne du côté de Bruxelles. En quoi nos marches se trouvèrent si bien concertées que nous nous y présentâmes tous à peu d’heures près les uns des autres.\par
En passant je me saisis d’Ath, petite place à la vérité, mais d’une situation avantageuse pour faciliter à mes gens le passage dans le pays, et pour incommoder les villes espagnoles au milieu desquelles elle est située.\par
Je ne crus pas devoir faire de circonvallation devant Tournay, tant parce que j’étais persuadé que le siège serait de peu de durée, que parce qu’il y avait des watregans que je fis joindre avec peu de travail. Mais je fis faire deux ponts sur l’Escaut, pour donner communication aux divers quartiers qui étaient séparés par la rivière.\par
Ainsi, étant arrivé le 21 juin devant la place, je fis dès le 22 ouvrir la tranchée. La nuit du 23 au 24, les habitants demandèrent à capituler ; la ville me fut livrée le 25 ; et la garnison retirée dans le château en sortit le 26.\par
Je marchai le même jour vers Courtrai, désirant que les ennemis vissent en un même jour la perte de la première place et le siège de la seconde. Mais en chemin je fis réflexion que cette place, dégarnie comme elle était, ne méritait pas que j’y fusse en personne, et que d’ailleurs Tournay se trouvant fort avancé dans le pays, il était besoin, pour le conserver, d’avoir quelque autre ville qui le joignît aux places de mon obéissance.\par
Douai me parut incontinent la plus commode pour ce dessein, et je crus qu’il était important de l’attaquer avant que les Espagnols s’en doutassent, parce que, s’ils eussent pu jeter dedans quelques troupes, et tenir le moindre corps du monde en campagne pour la rafraîchir, il eût été presque impossible de la prendre, vu l’étendue de la circonvallation qu’il eût fallu garder pour enfermer la ville et le château, qui sont fort éloignés l’un de l’autre.\par
Ainsi, jugeant la chose de conséquence, je cachai mon dessein aux ennemis, en faisant semblant d’aller à Lille, avec tant de succès qu’arrivant à Douai je n’y trouvai pour toute garnison que six-vingts chevaux et sept cents hommes d’infanterie. Il est vrai que le nombre des habitants était infiniment plus grand, et qu’ils témoignaient d’abord avoir intention de se bien défendre, tirant un si grand nombre de canons que jamais place n’en a tant tiré en si peu de temps. Mais après trois jours de tranchée ouverte, les Suisses s’étant logés dans le premier fossé, les habitants capitulèrent, quoiqu’il y eût encore un second fossé à gagner ; et, depuis les otages donnés, les régiments de Lyonnais et de Louvigny, qui se trouvaient dans une autre attaque, ayant passé le premier fossé sans avoir la capitulation, le peuple de la ville fit mille cris et m’envoya supplier humblement de faire cesser ce travail. Le fort, où la garnison s’était retirée, se rendit huit heures après : de sorte que le siège n’ayant en tout duré que quatre jours, j’entrai le sixième juillet dans la place.\par
Mon dessein était d’aller en ce moment, recommencer quelque nouveau siège : mais M. de Turenne me remontra qu’il fallait donner du repos à mon armée, pendant que je ferais prendre Courtrai par celle du maréchal d’Aumont. Et les considérations qu’il m’allégua pour cela étant effectivement très fortes, je m’y rendis, persuadé que, quelque envie qu’on ait de se signaler, le plus sûr chemin de la gloire est toujours celui que montre la raison.\par
Cependant, pour éviter l’oisiveté, je vins faire un tour à Compiègne, où je reçus la visite que me fit l’abbé Rospigliosi de la part du Pape, son oncle, au sujet de sa promotion. Mais je refusai les harangues que me voulut faire le Parlement, pour me congratuler de mes conquêtes, qui ne me semblaient pas encore assez grandes pour en recevoir des applaudissements publics.\par
Après quoi, ayant expédié les affaires qui regardaient le dedans de l’État, je voulus même que mon voyage pût servir au-dehors à faciliter le succès de mes armes ; et pour cela je menai, à mon retour, la Reine avec moi à dessein de la faire voir aux peuples des villes que je venais d’assujettir : de quoi ils se sentirent tellement obligés qu’après avoir tout mis en usage pour la bien recevoir, ils témoignèrent encore qu’ils étaient fâchés de n’avoir pas eu de temps de s’y préparer. Je la conduisis dans les meilleures villes : et ce fut une chose assez singulière de voir des dames faire ce trajet avec autant de tranquillité qu’elles eussent pu faire au centre de mon royaume.\par
Cependant, pour me poster toujours plus avant chez les ennemis, je résolus de tenter si je pourrais prendre Dendermonde, qui, par sa seule situation, les aurait fort incommodés et m’aurait donné de grands avantages. En chemin faisant, je pris Oudenarde, qui semblait utile au succès de ce dessein, et ensuite Alost se rendit à moi. De là je fis avancer Duras, lieutenant général, avec deux mille chevaux, sur les avenues de Bruxelles, d’où je prévoyais que le secours pouvait venir, et je marchai par un autre chemin, pour reconnaître la place en personne.\par
Mais comme je vis alors les choses de mes propres yeux, plus exactement qu’elles ne m’avaient été présentées, je trouvai, d’une part, que la rivière de l’Escaut était si large que, n’ayant pas de bâtiments propres à fermer entièrement son canal, il était absolument impossible, quelque garde que l’on fît sur les bords, d’empêcher que l’on ne passât dans le milieu avec le vent ou la marée ; et d’autre part j’appris que Duras, n’ayant pas fait assez de diligence, avait manqué de six heures les Espagnols, qui avaient jeté quinze cents hommes dans la place.\par
Ces deux considérations jointes ensemble me persuadèrent de quitter mon dessein, et je ne crois pas, ni en l’entreprenant, ni en l’abandonnant, avoir rien fait que je ne vous puisse donner pour exemple en de pareilles occasions. Car d’un côté, ayant des avis certains que cette place était dégarnie de monde, c’était peu sans doute de hasarder une marche de quelques journées contre l’un des meilleurs postes du pays. Comme au contraire, apprenant ensuite qu’il y était entré du secours, et voyant qu’il y en pouvait entrer encore à toute heure, je ne pouvais m’obstiner à l’assiéger qu’en hasardant d’y consumer sans fruit tout le reste de la campagne.\par
Ce n’est pas que je ne sache bien que l’on a parlé diversement de ma retraite ; et je vous dirai même pour votre instruction que, dès lors que je m’y résolus, je vis tout ce qui s’en est dit depuis, et le méprisai comme je devais.\par
Car enfin j’étais convaincu qu’aussitôt qu’on raisonnerait de bon sens sur cette affaire, l’on considérerait que la prudence des hommes n’est pas toujours maîtresse des événements, et qu’après avoir en si peu de jours exécuté si heureusement tant de choses, il n’était pas merveilleux que je me fusse déporté d’une seule pour m’occuper plus utilement ailleurs ; qu’il n’était pas même possible de m’attribuer un autre motif en cette action, puisque toute la terre savait que, ni dans ce temps-là, ni dans tout le reste de la campagne, les ennemis ne pouvaient être assez forts pour me faire retirer malgré moi ; et qu’enfin, comme le commun des hommes censure avec plaisir ce qui est au-dessus d’eux, les mêmes gens qui me blâmeraient d’avoir quitté Dendermonde sans l’attaquer, me condamneraient avec bien plus de sujet, si je l’attaquais sans la forcer, ou si, même en la prenant, je ruinais mon armée.\par
D’où vous pouvez conclure, mon fils, qu’il ne faut pas toujours s’alarmer des mauvais discours du vulgaire. Ces bruits qui s’élèvent avec tumulte, se détruisent bientôt par la raison, et font place aux sentiments des sages, qui, reconnus enfin pour vrais du peuple même, fondent par un consentement universel la solide et durable réputation. En attendant que le monde se détrompe de ses erreurs, ce doit être assez pour nous du témoignage que nous nous rendons à nous-mêmes : et c’est ce qui a fait que, repassant quelquefois mon esprit sur la retraite dont nous parlons, loin d’en être mal satisfait, je l’ai regardée comme la seule action de cette campagne où j’eusse véritablement fait quelque épreuve de ma vertu. Car, enfin, dans toutes les autres, quoique peut-être elles aient eu plus d’éclat, si j’ai fait quelque chose qu’on ait approuvé, ce n’a été que suivre les mouvements ordinaires à ceux de ma qualité, et, si j’ai eu quelques succès avantageux, la fortune y pourra prétendre autant ou plus de part que moi : au lieu que je ne dois tout le fruit de celle-ci qu’à la violence que je me fis à moi-même en méprisant tous les discours que je prévoyais.\par
Pour faire cesser la joie que les Espagnols faisaient éclater sur cette affaire, je résolus d’attaquer aussitôt après une de leurs meilleures places, et je me déterminai par mon propre sentiment à choisir Lille. Les ennemis, qui connaissaient de quelle importance elle était, et combien tombant entre mes mains, elle affermirait mes autres conquêtes, assemblèrent tout ce qu’ils avaient de troupes pour y jeter du secours. Il y avait même des gens en mon camp à qui la grandeur de la ville, le peuple dont elle était remplie, la force de la garnison et l’étendue des lignes qu’il fallait garder faisaient concevoir quelque doute du succès. Et néanmoins mes ordres furent exécutés avec tant de zèle que la ville fut réduite aux dernières extrémités avant que les Espagnols pussent apprendre seulement qu’elle fût en danger.\par
Mais cette ignorance où ils étaient me fit naître la pensée de leur donner un nouvel échec, en les allant attaquer où ils étaient, sitôt que la ville serait à moi. Pour cet effet, dès lors que l’on capitula, je fis partir, par divers chemins, deux de mes lieutenants généraux, Créqui et Bellefonds, lesquels je suivis de près moi-même, ne m’arrêtant dans la ville rendue qu’autant qu’il fallut pour remercier Dieu de l’avoir mise en mon pouvoir.\par
Les ennemis, qui avaient enfin su l’état des choses, marchaient déjà pour se retirer ; mais comme notre route allait à couper leur marche, ils furent trouvés, en même jour, par Créqui et par Bellefonds, devant lesquels, quoique trois fois plus forts en nombre, ils ne laissèrent pas de fuir, apprenant que je venais avec toute l’armée. Ils y perdirent environ deux mille hommes, en comptant les morts, les prisonniers et ceux que la fuite dissipa. Mais la joie que me devait donner leur défaite fut modérée par le chagrin que je sentis d’avoir eu si peu de part en l’exécution d’une entreprise dont j’avais seul formé tout le dessein.\par
Ce n’est pas que dans le vrai je ne susse bien que j’avais fait toute la diligence possible pour y arriver à temps, jusqu’à faire dire même à ceux qui me voulaient adroitement taxer d’imprudence qu’à la première nouvelle de l’ennemi j’y avais couru mal accompagné. Ce bruit était fondé sur ce qu’en effet j’avais été des premiers à cheval, et avais même marché fort vite : et la raison que j’en avais était parce qu’au sortir de mon quartier il y avait un grand défilé dans lequel, si mes troupes, qui sortaient du camp de toutes parts, fussent entrées avant moi, j’eusse perdu trop de temps à gagner la tête. Mais dès lors que le défilé fut passé, je mis tous mes gens en bataille, et les fis marcher avec tout l’ordre possible.\par
Après cela, je ne crus pas devoir entreprendre de nouveau siège. Et les raisons que j’en eus furent que les ennemis, qui n’avaient osé paraître devant moi toute la campagne, étant encore affaiblis par ce combat, jetteraient leurs troupes dans leurs villes ; que m’engageant à camper dans une mauvaise saison, je perdrais un si grand nombre d’hommes que la prise de la place même ne m’en pourrait pas dédommager ; qu’à mesure que l’hiver approcherait, mon armée serait plus abattue par les fatigues, et les Espagnols plus encouragés par l’espérance de nous rebuter ; et qu’enfin, ayant à ménager les esprits de tous mes voisins, et à pourvoir au recouvrement des deniers, des hommes et des munitions nécessaires pour l’achèvement de mon entreprise, je ne pouvais pas trouver, dans la chaleur d’une expédition, le loisir de penser à tant de choses.\par
Durant que je portais la guerre en Flandre, la paix qui se traitait à Bréda reçut un nouveau retardement. Car les Anglais, voyant que par mon entremise ils avaient presque obtenu tout ce qu’ils désiraient, s’avisèrent de redemander deux vaisseaux pris sur eux par les Hollandais, et qui avaient servi de prétexte à la déclaration de la guerre : sur laquelle demande les esprits, s’échauffant déjà, pouvaient en venir à une entière rupture. Mais comme je voyais l’importance de l’affaire et la modicité de la somme dont il s’agissait, qui n’allait qu’à cent mille francs, je résolus d’en fournir plutôt la moitié que de laisser la chose indécise ; et néanmoins, parce que je ne voulais pas faire connaître ouvertement l’intérêt que j’y prenais, je fis faire l’offre par Le Tellier, comme si de son chef il se fût porté à rendre ce service à ces deux États.\par
Mais il arriva dans le même temps que la flotte des Hollandais entra dans la Tamise, et qu’ayant pris ou brûlé plusieurs vaisseaux, elle jeta dans tout l’île une si furieuse consternation, que les Anglais se résolurent à conclure aussitôt le traité sans qu’il fût besoin de la somme que j’avais offerte.\par
Cet accord semblait, d’une part, me donner plus de jour à les tirer dans mon parti. Mais d’ailleurs, comme ils n’avaient été portés à se relâcher de leurs demandes que par l’insulte qu’ils avaient soufferte, et que ce malheur ne leur était arrivé que parce qu’ils n’avaient pas osé mettre leur flotte en mer de peur que je ne joignisse la mienne aux Hollandais, il y avait apparence qu’ils en garderaient du ressentiment contre moi. Et je savais de plus que le roi de la Grande-Bretagne était sollicité par les Espagnols et par les États même de Hollande, lesquels, quoique je les eusse récemment secourus, travaillaient pourtant à former contre moi une ligue de toute l’Europe.\par
Ainsi, je crus qu’il serait bon de lui envoyer Ruvigny pour faire ou qu’il se déclarât en ma faveur, ou que du moins il demeurât neutre, comme il semblait naturellement devoir faire, vu les fâcheuses nouveautés qui renaissaient à toute heure dans son État. Car il venait encore tout nouvellement d’être forcé à bannir son chancelier de ses conseils. Et bien qu’il fût vrai que ce ministre, pour avoir voulu prendre trop d’élévation, se fût de lui-même attiré beaucoup d’envie, il y a pourtant lieu de penser que la mauvaise volonté des Anglais ne se bornait pas tout à fait à sa personne, puisque ni son entière dépossession, ni son exil volontaire ne furent pas suffisants pour les contenter, mais qu’ils voulurent lui faire son procès sur des crimes qui semblaient lui être communs avec son maître.\par
D’un si notable événement, les ministres des rois peuvent apprendre à modérer leur ambition, parce que plus ils s’élèvent au-dessus de leur sphère, plus ils sont en péril de tomber. Mais les rois peuvent apprendre aussi à ne pas laisser trop agrandir leurs créatures, parce qu’il arrive presque toujours qu’après les avoir élevées avec emportement, ils sont obligés de les abandonner avec faiblesse, ou de les soutenir avec danger : car pour l’ordinaire ce ne sont pas des princes fort autorisés ou fort habiles qui souffrent ces monstrueuses élévations.\par
Je ne dis pas que nous ne puissions, par le propre intérêt de notre grandeur, désirer qu’il en paraisse quelque épanchement sur ceux qui ont part en nos bonnes grâces. Mais il faut prendre soigneusement garde que cela n’aille pas jusqu’à l’excès, et le conseil que je vous puis donner pour vous en garantir consiste en trois observations principales.\par
La première est que vous sachiez vos affaires à fond, parce qu’un roi qui ne les sait pas, dépendant toujours de ceux qui le servent, ne peut bien souvent se défendre de consentir à ce qui leur plaît.\par
La seconde, que vous partagiez votre confiance entre plusieurs, d’autant que chacun de ceux auxquels vous en faites part étant par une émulation naturelle opposé à l’élévation de ses rivaux, la jalousie de l’un sert souvent de frein à l’ambition de l’autre.\par
Et la troisième, qu’encore que, dans le secret de vos affaires ou dans vos entretiens de plaisir ou de familiarité, vous ne puissiez admettre qu’un petit nombre de personnes, vous ne souffriez pas pourtant que l’on se puisse imaginer que ceux qui auront cet avantage soient en pouvoir de vous donner à leur gré bonne ou mauvaise impression des autres ; mais qu’au contraire vous entreteniez exprès une espèce de commerce avec tous ceux qui tiendront quelque poste important dans l’État, que vous leur donniez à tous la même liberté de vous proposer ce qu’ils croiront être de votre service ; que pas un d’eux, en ses besoins, ne se croie obligé de s’adresser à d’autres qu’à vous ; qu’ils ne pensent avoir que vos bonnes grâces à ménager ; et qu’enfin les plus éloignés comme les plus familiers soient persuadés qu’ils ne dépendent en tout que de vous seul.\par
Car vous devez savoir que cette indépendance sur laquelle j’insiste si fort, étant bien établie entre les serviteurs, relève plus que toute autre chose l’autorité du maître, et que c’est elle seule qui fait voir qu’il les gouverne en effet, au lieu d’être gouverné par eux. Comme, au contraire, d’abord qu’elle cesse, on voit infailliblement les brigues, les liaisons et les engagements particuliers grossir la cour de ceux qui sont en crédit et affaiblir la réputation du prince.\par
Mais principalement s’il en est quelqu’un qui, par notre inclination ou par son industrie, vienne à se distinguer de ses pareils, on ne manque jamais de penser qu’il est maître absolu de notre esprit, on le regarde incontinent comme un favori déclaré, on lui attribue quelquefois des choses dont il n’a pas eu la moindre participation, et le bruit de sa faveur est infiniment plus grand dans le monde qu’elle ne l’est en effet dans notre cœur.\par
Et cependant ce n’est pas en cela, mon fils, qu’on peut mépriser les bruits populaires : au contraire, il faut y remédier sagement et promptement, parce que cette opinion, quoique de soi vaine, peut, en durant trop, nuire à notre réputation et augmenter effectivement le crédit de celui même qui l’a fait naître. Car comme chacun s’empresse à devenir de ses amis, il trouve souvent moyen de faire par les autres ce qu’il n’eût jamais entrepris de son chef, et parce qu’on s’imagine qu’il peut tout, on veut lui plaire par toutes voies. Ceux même à qui nous donnons le plus de familiarité auprès de nous cherchent à se fortifier par son appui. On prend avec lui des engagements secrets qu’on couvre en certaines occasions d’une indifférence affectée, pendant que dans les choses qu’il affectionne on l’informe de tout ce qu’on voit, on nous parle toujours dans ses sentiments, on approuve ou blâme ce qu’il veut, on éloigne ce qui lui déplaît, on facilite ce qu’il désire : en sorte que, sans qu’il paraisse y contribuer, nous nous trouvons, comme par merveille, mais merveille presque infaillible, portés dans tous ses sentiments.\par
Et cela, mon fils, est d’autant plus à remarquer, que c’est par où naît et s’établit d’ordinaire la puissance des favoris, et par où l’on parvient insensiblement à gouverner la plupart des princes. Car enfin, ce qu’on appelle être gouverné n’est pas toujours d’avoir un Premier ministre en titre, à qui l’on renvoie ouvertement la décision de toutes choses ; chez les esprits éclairés, c’est assez pour cela d’avoir une ou plusieurs personnes, de quelque qualité qu’elles soient, qui, séparées ou jointes ensemble, puissent nous mettre dans l’esprit ce qu’elles veulent, qui sachent, selon leurs intérêts, avancer ou reculer les affaires, et qui puissent, sans que nous y fassions réflexion, approcher de nous les gens qu’elles favorisent, ou nous dégoûter de ceux qu’elles n’aiment pas.\par
Après avoir entretenu mon armée de mer jusqu’au mois d’octobre, je l’avais licenciée, à la réserve d’une escadre, à qui je fis passer le détroit pour incommoder les côtes d’Espagne.\par
À l’égard de celle de terre, j’en avais laissé le commandement à M. de Turenne, lequel, bientôt après mon départ, marcha vers Alost, où les ennemis avaient remis des troupes, prit la place et la démantela. Après quoi, voyant qu’il n’y avait plus lieu de faire autre chose, il prit, suivant mes ordres, les quartiers d’hiver les plus étendus qu’il put pour resserrer d’autant plus les ennemis, et, revenant auprès de moi, laissa les troupes partagées entre quatre lieutenants généraux, qui, tous, ayant leur département séparé, avaient pourtant ordre de s’aider réciproquement en ce qui serait de mon service.\par
Du Passage commandait à tout ce qui était depuis la mer jusqu’à la Lys. Duras avait Tournai avec tous les postes avancés au-delà de l’Escaut. D’Humières tenait sous sa charge Lille et le plat pays qui était entre ces deux rivières. Bellefonds, détaché de tous les autres, veillait sur les places qui étaient entre la Sambre et la Meuse, où il fit, dès son arrivée, une action assez remarquable, ayant avec huit cents chevaux défait quinze cents hommes des ennemis, qui avaient infanterie et cavalerie, et étaient épaulés d’un bois.\par
Ce que j’avais le plus expressément ordonné à tous les commandants des places et aux officiers généraux, était de conserver les hommes que je leur laissais, et d’empêcher que l’on ne fît aucun tort aux habitants des villes. Mais pour y contribuer aussi de ma part en ce que je pouvais, j’eus soin que les troupes reçussent exactement leur solde, et fis même augmenter d’un tiers celle des officiers subalternes, afin qu’ils pussent commodément subsister sans être à charge aux gens du pays.\par
Mais je crus faire encore pour le bien des peuples, aussi bien que pour la sûreté de mes conquêtes, de bâtir des citadelles dans les plus grandes places, comme Lille et Tournai, parce que cela me dispensait de la nécessité d’y tenir de si fortes garnisons, et les délivrait de la crainte qu’ils avaient d’être pris ou repris toutes les campagnes.\par
Ce n’est pas que je ne donnasse bon ordre à les exempter de cette frayeur : car, bien loin de laisser aux ennemis la pensée de reprendre ce que je tenais, je me mettais en état de leur ôter une bonne partie de ce qui leur restait. Mon projet, en général, était de mettre le printemps suivant quatre armées en campagne, dont l’une, sous mon frère, devait passer en Catalogne pour attaquer les Espagnols dans l’Espagne même ; la seconde, sous le prince de Condé, se devait porter sur les bords du Rhin, afin d’arrêter ou de combattre ce qui viendrait d’Allemagne, et les deux autres devaient être dans la Flandre sans autre général que moi et M. de Turenne : car je ne voulais demeurer aucun moment sans occupation, et je désirais trouver toujours une armée fraîche, quand l’autre aurait besoin de se reposer.\par
Suivant ces projets, je faisais faire de nouvelles levées, non seulement dans mes États, mais en Allemagne, en Suisse et en Angleterre, d’où je tirai même plusieurs cavaliers licenciés de la compagnie des gendarmes du Roi, parce qu’ils étaient catholiques.\par
Monsieur de Lorraine voulait, après la campagne finie, reprendre les troupes qui étaient à lui : mais je lui en fis parler de telle sorte, qu’il fut obligé de s’en désister, et de me les laisser autant que je voulus.\par
Pour faire subsister tant de forces, je pris soin de faire emplir mes magasins que l’été passé avait dégarnis, et je fis un état exact de la recette et de la dépense que j’avais à faire l’année prochaine.\par
Mais, durant tous ces préparatifs de guerre, on ne laissait pas de parler de la paix. Les Hollandais, sollicités par leur propre appréhension, me pressaient sans cesse d’y consentir. Dès le temps que j’étais à Avesnes, Van Beuningen s’y était rendu pour cela, et demandait même à me suivre en mon camp : mais je n’estimai pas que je le dusse permettre, parce que, dans un si grand assemblage de gens, il est malaisé que toujours quelqu’un n’ait ou ne croie avoir sujet de se plaindre, et je ne voulais pas que cela fût observé de si près par un homme dont les pensées n’étaient pas conformes aux miennes.\par
Ainsi, je l’envoyai à Paris pour traiter avec Lionne ; et à mon retour, reprenant moi-même cette négociation, je résolus enfin de faire voir à toute l’Europe la modération de mon esprit, en offrant de me contenter pourvu qu’en échange des terres qui m’étaient échues, l’on me cédât seulement ce que j’avais pris, si l’on aimait mieux me donner la Franche-Comté ou le Luxembourg, avec Aire, Saint-Omer, Douai, Cambrai et Charleroi, consentant, de plus, que les Espagnols eussent trois mois de temps pour en délibérer, durant lesquels je n’attaquerais aucune de leurs places où il fût besoin de canon.\par
Le Pape travaillait aussi avec beaucoup de zèle à faire réussir cet accord. Et sa médiation ayant été acceptée par moi dès lors que son neveu me vit à Compiègne, les Espagnols n’avaient pas osé la refuser. Mais comme ils voyaient que l’affaire ne se terminerait pas sans qu’il leur en coûtât quelque chose, ils avaient peine de venir à la conclusion et affectaient diverses chicanes, tantôt sur le temps, tantôt sur le lieu de l’assemblée, pour voir si mes voisins, jaloux de mon accroissement, ne se ligueraient pas avec eux.\par
Cependant toute ma cour n’était pas aussi d’un même avis sur cette affaire, et plusieurs, réglant leurs pensées sur leurs intérêts, trouvaient des raisons pour la paix ou pour la guerre, selon que l’une ou l’autre pouvait augmenter ou leur fonction ou leur crédit. Mais comme leurs motifs m’étaient connus, leurs raisonnements ne faisaient d’impression sur mon esprit qu’autant qu’ils tendaient au bien de mes affaires, et ne me tiraient jamais de l’égalité que je m’étais proposé de garder en mon jugement : ou du moins, si l’on me voyait pencher quelquefois tant soit peu plus du côté des armes, ce n’était ni par la faveur, ni par l’adresse de ceux qui pouvaient y avoir intérêt, mais seulement par l’inclination que j’avais pour la gloire, qui, sans doute, par cette voie semble s’acquérir avec plus d’éclat.\par
À l’égard des princes d’Allemagne, je crois qu’il y en avait qui désiraient la continuation de la guerre, comme il y en avait qui demandaient la paix ; mais, à parler généralement, tous me traitaient sur ce sujet avec la plus grande honnêteté du monde. J’envoyai vers ceux qui s’étaient nouvellement engagés à défendre le passage du Rhin contre les troupes impériales, pour leur persuader de joindre leurs forces à celles du prince de Condé, que j’envoyais à même dessein.\par
Pour l’Empereur, je lui avais fait donner avis du voyage que je faisais en Flandre, et il l’avait reçu avec moins de chaleur que je ne m’étais figuré, me répondant seulement qu’il me priait de me contenter de choses raisonnables. Et même le comte de Furstemberg lui ayant fait quelque proposition touchant le traité éventuel, il témoigna ne s’en éloigner que par la peine qu’il aurait d’en faire l’ouverture aux ministres d’Espagne avant que le cas fût arrivé. Mais cela ne m’empêchait pas de prévoir que, ma querelle continuant avec les Espagnols, ce prince les assisterait sans doute, et, faisant mon compte là-dessus, je recherchais tous les expédients qui pouvaient divertir ses forces ailleurs.\par
Le dessein que j’avais eu de donner à la Pologne un prince de ma maison ayant été traversé par la mort de la reine, sur laquelle principalement il était fondé, le duc de Neubourg me fit prier de favoriser sa prétention ; et le prince de Condé auquel j’en fis parler, m’ayant répondu avec toute l’honnêteté et la soumission possibles, je promis au duc ce qu’il me demandait : même, pour le servir suivant ses intentions, je fis proposer le mariage de sa fille avec le roi nouvellement veuf, lequel je tâchai de dissuader de l’abdication qu’il avait projetée. Mais depuis, ayant entendu qu’il n’avait aucune inclination pour ce mariage, et voyant que le duc de Neubourg ne viendrait pas aisément à bout de son projet, je m’avisai de tirer un autre fruit de cette conjoncture. J’accommodai aussitôt ma conduite à ce nouveau dessein, et je résolus de favoriser moi-même l’abdication que j’avais auparavant retardée, afin que les contestations qu’elle produirait attirassent les armes des Allemands, pendant que je m’établirais en Flandre.\par
Quant aux Suédois, j’eusse bien désiré m’en assurer auparavant que la guerre fût déclarée. Mais voyant que plus on les pressait plus ils reculaient, je voulus tenter si, en leur témoignant plus de froideur, ils ne s’avanceraient pas davantage. Et, depuis encore, voyant que ce remède n’opérait pas et ne voulant pas épargner des démarches de cérémonie pour venir à de plus solides fins, je leur fis reparler de nouveau, mais ce fut toujours inutilement.\par
Je reçus du roi de Danemark des offres fort civiles ; mais, parce qu’il ne me semblait pas en état de rien faire d’important pour moi, je me contentai de lui répondre avec une honnêteté réciproque.\par
J’entretenais avec plus de soin les bonnes intentions du duc de Savoie, auquel j’avais d’abord donné part de mon dessein ; et parce qu’il me pouvait être utile en Italie, je tâchais de l’attacher à mes intérêts, en lui faisant toutes les propositions que je croyais capables de lui plaire.\par
Les Hollandais, qui ne pensaient peut-être pas que je connaissais les brigues qu’ils faisaient contre moi, me parlaient toujours avec même liberté de ce qui regardait leurs avantages, et s’efforcèrent de m’engager à ne rien conquérir près de leurs frontières ; mais je leur refusai précisément ce point. Et même les trois mois que par leur entremise j’avais donnés de surséance aux Espagnols étant expirés vers la fin de décembre, je déclarai que je ne prétendais plus la continuer.\par
Et, en effet, m’ennuyant déjà de demeurer en repos, je fis observer de divers côtés s’il n’y avait rien que l’on pût exécuter brusquement ; et entre autres, lorsque le prince de Condé alla tenir les États en Bourgogne, je le chargeai de reconnaître ce qui se pourrait faire dans la Franche-Comté.\par
J’avais depuis peu conclu un nouveau traité avec le roi de Portugal, par lequel il s’obligeait à ne faire ni paix ni trêve sans mon exprès consentement, et je lui promettais aussi de ne me point accommoder avec l’Espagne sans lui faire accorder le titre de roi qu’on lui avait jusque-là refusé. Mais vers la fin de cette année, il arriva une révolution dans cet État, qui rompit absolument mes mesures.\par
Car le Roi qui, de sa personne, était fort incommodé, s’étant encore rendu plus insupportable par ses mœurs, fut dépossédé et fait prisonnier dans son propre palais, sans que, de tout ce qu’il avait de sujets ou de domestiques, aucun se mît en devoir d’empêcher un si détestable attentat : aventure tellement singulière, que l’histoire des siècles passés ne nous peut fournir rien de pareil. Mais, tandis que le reste des hommes se contente d’admirer cet événement, il est bon que vous tâchiez d’en profiter, en observant quelles en ont été les causes.\par
Il faut assurément demeurer d’accord que, pour mauvais que puisse être un prince, la révolte de ses sujets est toujours infiniment criminelle. Celui qui a donné des rois aux hommes a voulu qu’on les respectât comme ses lieutenants, se réservant à lui seul le droit d’examiner leur conduite. Sa volonté est que, quiconque est né sujet, obéisse sans discernement ; et cette loi, si expresse et si universelle, n’est pas faite en faveur des princes seuls, mais est salutaire aux peuples mêmes auxquels elle est imposée, et qui ne la peuvent jamais violer sans s’exposer à des maux beaucoup plus terribles que ceux dont ils prétendent se garantir. Il n’est point de maxime plus établie par le christianisme que cette humble soumission des sujets envers ceux qui leur sont préposés. Et, en effet, ceux qui jetteront la vue sur les temps passés reconnaîtront aisément combien ont été rares, depuis la venue de Jésus-Christ, ces funestes révolutions d’États qui arrivaient si souvent durant le paganisme.\par
Mais il n’est pas juste que les souverains qui font profession de cette sainte doctrine se fondent sur l’innocence qu’elle inspire à leurs peuples pour vivre de leur part avec plus de dérèglement. Il faut qu’ils soutiennent par leurs propres exemples la religion dont ils veulent être appuyés, et qu’ils considèrent que leurs sujets, les voyant plongés dans le vice et le sang, ne peuvent presque rendre à leur personne le respect dû à leur dignité, ni les reconnaître pour les vivantes images de celui qui est tout saint aussi bien que tout puissant.\par
Je sais bien que ceux qui sont nés comme vous avec des inclinations vertueuses ne s’emportent jamais à ces scandaleuses extrémités qui blessent ouvertement la vue des peuples ; mais il est bon pourtant que vous sachiez que dans le haut rang que nous tenons, les moindres fautes ont toujours de fâcheuses suites. Celui qui les fait a ce malheur qu’il n’en connaît jamais la conséquence que quand il n’est plus temps d’y remédier. L’habitude qu’il prend au mal le lui fait croire de jour en jour plus excusable et moins connu, tandis qu’il paraît aux yeux du public plus honteux et plus manifeste ; car c’est une des plus grandes erreurs où puisse tomber un prince de penser que ses défauts demeurent cachés, ni qu’on se porte à les excuser.\par
Les rois, qui sont les arbitres souverains de la fortune et de la conduite des hommes, sont toujours eux-mêmes les plus sévèrement jugés et les plus curieusement observés. Dans le grand nombre de gens qui les environnent, ce qui échappe aux yeux de l’un est presque toujours découvert par un autre. Le moindre soupçon que l’on conçoit d’eux passe aussitôt d’oreille en oreille, comme une nouvelle agréable à débiter : celui qui parle, faisant toujours vanité de savoir plus que les autres, augmente les choses au lieu de les affaiblir ; et celui qui entend, prenant un plaisir malin à voir abaisser ce qu’il croit trop au-dessus de lui, apporte toute la facilité possible à se persuader de ce qu’on lui dit.\par
Plus le prince dont on s’entretient a d’ailleurs de mérite et de vertu, plus l’envie prend à tâche d’en obscurcir l’éclat : en sorte que, bien loin de dissimuler ses fautes, on lui en suppose même quelquefois dont il est absolument innocent. D’où vous devez conclure, mon fils, qu’un souverain ne saurait mener une vie trop sage et trop innocente ; que, pour régner heureusement et glorieusement, ce n’est pas assez de donner ordres aux affaires générales, si nous ne réglons aussi nos propres mœurs ; et que le seul moyen d’être vraiment indépendant et au-dessus du reste des hommes est de ne rien faire ni en public, ni en secret, qu’ils puissent légitimement censurer.\par
\subsubsection[{Appendice}]{Appendice}
\noindent Avant que de partir, j’envoyai un édit au Parlement par lequel j’érigeais en duché la terre de Vaujours en faveur de Mademoiselle de La Vallière, et reconnaissais une fille que j’avais eue d’elle. Car, n’étant pas résolu d’aller à l’armée pour y demeurer éloigné de tous les périls, je crus qu’il était juste d’assurer à cette enfant l’honneur de sa naissance, et de donner à la mère un établissement convenable à l’affection que j’avais pour elle depuis six ans.\par
J’aurais pu sans doute me passer de vous entretenir de cet attachement dont l’exemple n’est pas bon à suivre. Mais après avoir tiré plusieurs instructions des manquements que j’ai remarqués dans les autres, je n’ai pas voulu vous priver de celles que vous pouviez tirer des miens propres.\par
Je vous dirai premièrement que, comme le prince devrait toujours être un parfait modèle de vertu, il serait bon qu’on se garantît absolument des faiblesses communes au reste des hommes, d’autant plus qu’il est assuré qu’elles ne sauraient demeurer cachées. Et néanmoins, s’il arrive que nous tombions malgré nous dans quelqu’un de ces égarements, il faut du moins, pour en diminuer la conséquence, observer deux précautions que j’ai toujours pratiquées et dont je me suis fort bien trouvé.\par
La première, que le temps que nous donnons à notre amour ne soit jamais pris au préjudice de nos affaires, parce que notre premier objet doit toujours être la conservation de notre gloire et de notre autorité, lesquelles ne se peuvent absolument maintenir que par un travail assidu. Car, quelque transportés que nous puissions être, nous devons, par le propre intérêt de notre passion, considérer qu’en diminuant de crédit dans le public, nous diminuerions aussi l’estime auprès de la personne même pour qui nous nous serions relâchés.\par
Mais la seconde considération, qui est la plus délicate et la plus difficile à pratiquer, c’est qu’en abandonnant notre cœur, nous demeurions maîtres de notre esprit ; que nous séparions les tendresses d’amant d’avec les résolutions de souverain ; et que la beauté qui fait nos plaisirs n’ait jamais la liberté de nous parler de nos affaires, ni des gens qui nous y servent.\par
On attaque le cœur d’un prince comme une place. Le premier soin est de s’emparer de tous les postes par où on en peut approcher. Une femme adroite s’attache d’abord à éloigner tout ce qui n’est pas dans ses intérêts ; elle donne du soupçon des uns et du dégoût des autres, afin qu’elle seule et ses amis soient favorablement écoutés, et si nous ne sommes en garde contre cet usage, il faut, pour la contenter elle seule, mécontenter tout le reste du monde.\par
Dès lors que vous donnez la liberté à une femme de vous parler des choses importantes, il est impossible qu’elles ne nous fassent faillir. La tendresse que nous avons pour elles, nous faisant goûter leurs plus mauvaises raisons, nous fait tomber insensiblement du côté où elles penchent ; et la faiblesse qu’elles ont naturellement, leur faisant souvent préférer des intérêts de bagatelles aux plus solides considérations, leur fait presque toujours prendre le mauvais parti. Elles sont éloquentes dans leurs expressions, pressantes dans leurs prières, opiniâtres dans leurs sentiments, et tout cela n’est souvent fondé que sur une aversion qu’elles auront pour quelqu’un, sur le dessein d’en avancer un autre, ou sur une promesse qu’elles auront faite légèrement.\par
Le secret ne peut être chez elles dans aucune sûreté : car si elles manquent de lumières, elles peuvent par simplicité découvrir ce qu’il fallait le plus cacher ; et si elles ont de l’esprit, elles ne manquent jamais d’intrigues et de liaisons secrètes. Elles ont toujours quelque conseil particulier pour leur élévation ou pour leur conservation, et elles ne manquent point d’y étaler tout ce qu’elles savent, autant de fois qu’elles en croient tirer quelque raisonnement pour leurs intérêts.\par
C’est dans ces conseils qu’elles concertent en chaque affaire quel parti elles doivent prendre, de quels artifices elles se doivent servir pour faire réussir ce qu’elles ont entrepris, comment elles se déferont de ceux qui leur nuisent, comment elles établiront leurs amis, par quelles adresses elles nous pourront engager davantage et nous retenir plus longtemps. Enfin, tôt ou tard, elles font réussir toutes ces choses, et nous y donnons tôt ou tard sans nous apercevoir que nous perdons ou dégoûtons nos meilleurs serviteurs, que nous ruinons notre réputation, sans que nous nous en puissions garantir que par un seul moyen, qui est de ne leur donner la liberté de parler d’aucune chose que de celles qui sont purement de plaisir, et de nous préparer avec étude à ne les croire en rien de ce qui peut concerner nos affaires ou les personnes de ceux qui nous servent.\par
Je vous avouerai bien qu’un prince dont le cœur est fortement touché par l’amour, étant aussi toujours prévenu d’une forte estime pour ce qu’il aime, a peine de goûter toutes ces précautions. Mais c’est dans les choses difficiles que nous faisons paraître notre vertu. Et, d’ailleurs, il est certain qu’elles sont d’une nécessité absolue et c’est faute de les avoir observées que nous voyons dans l’histoire tant de funestes exemples des maisons éteintes, des trônes renversés, des provinces ruinées, des empires détruits.
\section[{Année 1668}]{Année 1668}\renewcommand{\leftmark}{Année 1668}

\noindent Les médiateurs voyant finir, au commencement de cette année, les trois mois que je leur avais accordés pour faire déterminer la reine d’Espagne à l’une des deux propositions de paix auxquelles je m’étais fixé, me demandèrent encore les trois mois suivants ; et malgré toute ma répugnance, je ne pus m’empêcher de les donner, principalement aux instances du Pape.\par
L’on me pressait fort d’accorder une suspension d’armes pour le même temps : mais dans la résistance que j’y faisais, je fus heureusement secondé par la fausse bravoure de Castel Rodrigo, qui, recevant avec froideur cette proposition, me donna prétexte de la refuser. Ainsi, continuant à faire observer de tous côtés ce qui se pouvait exécuter avec diligence, je faisais mon compte que ce que je prendrais servirait toujours, ou à rendre ma condition meilleure si la guerre durait, ou à faire éclater davantage ma bonne foi si je le rendais par la paix.\par
L’on me proposait quelques entreprises sur Ypres, sur Namur et sur quelques autres places. Mais je ne trouvai rien de si faisable, ni de si avantageux que ce que j’avais moi-même pensé touchant la comté de Bourgogne, principalement depuis que le prince de Condé, l’ayant observée de plus près, m’eût rendu compte de l’état où elle était. Car je considérais que c’était une province grande, fertile et importante, qui, par sa situation, par sa langue, et par des droits aussi justes qu’anciens, devait faire partie de ce royaume, et par qui, m’ouvrant un nouveau passage en Allemagne, je le fermais en même temps à mes ennemis.\par
Je voyais, de plus, que moi l’attaquant en cette saison, elle pouvait malaisément être secourue : que le gouverneur général des Pays-Bas avait peu de forces, et en était fort éloigné ; que le marquis d’Yenne, qui en avait le gouvernement particulier, était homme de médiocre intelligence et de plus médiocre crédit ; que toutes les forces du pays consistaient en certaines milices, dont il ne fallait pas craindre grand effet, et que toute l’autorité se trouvait alors entre les mains du seul parlement, qui, comme une assemblée de simples bourgeois, serait facile à tromper et à intimider. La plus grande difficulté de l’entreprise était d’en conserver le secret parmi tous les préparatifs qu’il fallait faire.\par
Mais après y avoir pensé, je trouvai moyen de faire assembler dix-huit mille hommes, sans qu’eux-mêmes se pussent apercevoir de mon dessein : car les uns étaient commandés pour aller en Catalogne avec mon frère ; les autres, pour se rendre à la Marche, où s’était faite une mutinerie de peu d’importance ; les autres pour m’attendre à Metz, où je feignais d’aller moi-même. Et leurs routes étaient tellement ajustées, qu’à considérer le lieu dont ils partaient et celui où ils avaient ordre d’aller, la Bourgogne se trouvait naturellement dans leur passage. Je faisais même qu’ils y étaient arrêtés par M. le Prince sous prétexte d’un défaut de formalité : car, comme il était gouverneur de la province, il leur refusait son attache pour passer outre, feignant qu’il n’avait point eu avis de leur route.\par
Il n’y eut que les troupes de ma maison avec lesquelles il fallut en user autrement : car je leur donnai d’abord une première route jusqu’à Troyes, où je leur fis porter un second ordre pour se joindre aux autres, mais cela ne se fit que dans le temps où il n’y avait plus rien à ménager.\par
Cependant le canon et les munitions, tant de bouche que d’artillerie, se portaient ou se préparaient dans la même province, sous des noms supposés et des raisons apparentes, tandis que j’amusais ceux qui pouvaient y avoir le plus d’intérêt par des propositions fort éloignées de mon dessein.\par
Il se rencontra par bonheur que les Francs-Comtois, alarmés de la campagne passée, avaient depuis peu fait demander qu’on renouvelât la neutralité qu’ils avaient souvent obtenue, et je crus que cette négociation serait bonne à occuper leurs esprits pendant que je ferais mes préparatifs. Mais pour en tirer à la fois tout le fruit qui s’en pouvait raisonnablement espérer, je la fis passer des mains de Mouillée, mon résident en Suisse, par qui elle avait été commencée, dans celles du prince de Condé qui, par ce moyen, pouvait sans soupçon envoyer et renvoyer dans le pays, autant de fois qu’il en serait besoin, pour prendre toutes nos mesures. Ce qui fut si bien ménagé par lui que les Francs-Comtois venant le trouver, et lui renvoyant à son tour chez eux, il apprit et régla tout ce qui était nécessaire, les entretenant toujours en tel état qu’il semblait ne tenir plus qu’à eux que la neutralité ne se conclût. D’où il arriva que non seulement ils ne prirent aucune sorte d’alarme, mais qu’entendant même la vérité, par ces bruits incertains qu’on ne peut empêcher de courir devant les choses les plus secrètes, ils les prirent pour un artifice étudié à dessein de leur faire augmenter leurs offres. Leur assurance fut si grande, que les Suisses, qui avaient déjà conçu quelque soupçon des démarches qui se faisaient, se rassurèrent, par la tranquillité de ceux qui avaient le premier intérêt dans l’affaire.\par
Castel Rodrigo même, auquel ils donnaient part de ce qu’ils négociaient, en fut longtemps abusé comme eux, et avec lui tous mes autres voisins, quoique, par les ministres qu’ils avaient à ma cour, ils observassent de plus près ma conduite. Car encore qu’il ne fût pas possible d’empêcher que quelqu’un ne s’imaginât ce qui était, je faisais voir tant d’apparences au contraire, que ceux même qui l’avaient cru les premiers cessaient quelquefois de le penser, et ceux auxquels on l’avait dit n’y pouvaient ajouter aucune foi.\par
Mais enfin, étant près de partir, je voulus donner moi-même avis à tous les États de l’Europe d’une chose que je ne pouvais plus leur cacher ; et, de peur que les plus mal intentionnés ne tirassent avantage de cette entreprise pour attirer les autres dans leurs sentiments, je déclarai que, quel qu’en pût être le succès, il ne m’empêcherait pas de garder les paroles que j’avais données.\par
Je partis accompagné de tout ce qu’il y avait de noblesse à ma cour. Et alors enfin les Francs-Comtois furent tirés de leur assoupissement, soit par le bruit de mon voyage, soit par les avis de Castel Rodrigo, ou même par la déclaration du prince de Condé qui, prenant occasion de quelque difficulté qu’ils faisaient, rompit brusquement avec eux.\par
Aussitôt ils demandèrent du secours en Flandre, envoyèrent offrir de grandes sommes aux Suisses pour en tirer des troupes, et convoquèrent leurs propres milices pour le huitième de février. Mais tout cela se faisait trop tard, car j’avais donné mes ordres au prince de Condé pour entrer dès le quatrième du même mois dans le pays, et pour se saisir de certains postes qui empêchaient à la fois et la jonction des milices et la communication des principales villes.\par
J’avais même résolu que l’on attaquerait en même temps Besançon et Salins, afin que l’une et l’autre fût prise avant qu’il leur pût venir aucun secours, et ne me souciai pas que l’on attendît pour cela mon arrivée, préférant le solide avantage que cette diligence me donnait à la vaine satisfaction qu’un autre eût peut-être trouvée à faire dire qu’il se fût lui-même rencontré à ces deux attaques. Ce n’est pas que ce ne fussent en effet deux places de réputation. Car Besançon, se prétendant ville impériale, ne reconnaissait le roi d’Espagne que pour protecteur, et passait pour la plus peuplée du pays, comme aussi Salins étant sans difficulté la plus riche par les fontaines qui lui fournissaient le sel. Mais, après tout, en l’état qu’étaient alors l’une et l’autre de ces places, il était malaisé qu’elles se défendissent longtemps.\par
Le prince de Condé marcha lui-même vers Besançon, n’ayant au plus que deux mille hommes, et la somma néanmoins si fièrement de se rendre, que les habitants, persuadés qu’il était suivi de toute mon armée, capitulèrent dès le même jour ; tandis qu’il envoyait le duc de Luxembourg à Salins, où, la consternation se trouvant pareille, on lui rendit sans combat la ville et les deux forts.\par
Ces deux nouvelles m’arrivèrent dans le même jour à Auxonne, d’où je partis dès le lendemain pour attaquer Dole, quoiqu’à dire le vrai la chose ne fût pas sans difficulté. Car le plan de la place m’apprenait qu’elle était garnie de sept grands bastions, la plupart bâtis sur le roc ; l’histoire me faisait voir qu’elle avait deux fois résisté à de puissantes armées ; et la saison où nous étions m’avertissait qu’il n’était pas possible de camper longtemps. Mais, d’autre part, je voyais aussi le peu de monde qui était dans la place, la consternation générale dont tout le pays était saisi, l’ardeur que mes gens témoignaient pour cette entreprise, et le bonheur qui m’avait suivi dans toutes les autres.\par
Ainsi j’envoyai mes ordres à M. le Prince, pour la venir investir du côté de Besançon. J’y fis marcher… du côté de… et je marchai moi-même du lieu où j’étais.\par
J’employai presque un jour et demi à reconnaître la place en personne, persuadé que ce temps n’était pas perdu, parce que du bon ou du mauvais choix des attaques dépend presque toujours le succès d’un siège. Enfin, je résolus que l’on en ferait trois et que, pour ménager le temps, on marcherait droit à la contrescarpe. Les deux attaques des Gardes et de Picardie firent ce qui leur était commandé ; mais celle de Lyonnais, passant au-delà de mes ordres, après avoir gagné le chemin couvert, entreprit de monter à la demi-lune, la força et s’y logea.\par
Ce fut une terreur inconcevable aux habitants de nous voir ainsi, dès le premier jour, postés au pied de leurs murailles. Et cela fit que le comte de Grammont, s’étant offert de leur aller proposer de se rendre, je crus qu’il pouvait réussir dans son dessein. Il eut quelque peine à parvenir jusqu’à la ville, mais il en eut peu à persuader les bourgeois, de la part desquels il m’amena des otages ; ensuite de quoi l’on capitula.\par
Cependant, pour ne rien laisser dans la province qui pût y redonner entrée aux Espagnols, je voulus m’assurer de plusieurs villes et châteaux qui tenaient encore pour eux. J’envoyai pour cela mes ordres à Noisy, gouverneur de Salins, lequel sut si bien se prévaloir de l’autorité de mon nom et de la frayeur des ennemis, qu’avec six-vingts hommes ou peu plus, il réduisit en deux jours six places, dont quelques-unes avaient souffert des sièges réguliers. Surtout les châteaux de Sainte-Anne et de Joux passaient pour imprenables dans le pays, et le marquis d’Yenne, retiré dans ce dernier, semblait le fortifier encore par sa présence. Mais, soit pour le mécontentement qu’il avait de l’Espagne, soit par le peu d’espoir d’en être secouru, ou même par la crainte d’y être quelque jour châtié d’avoir si mal gardé cette province, il se laissa persuader de se rendre à moi, et de me venir trouver devant Gray, où j’avais marché dès lors que j’avais pris Dole.\par
Je reçus avec joie ce présent de ma bonne fortune, et, pour m’en servir sur l’heure même aussi utilement qu’il se pouvait, je désirai que le marquis d’Yenne s’employât lui-même à moyenner la reddition de Gray. Les députés du parlement de Dole que j’avais déjà fait agir pour cela, y avaient été fort mal reçus, et la ville paraissant résolue à se défendre, j’avais aussi, de ma part, pris mes quartiers, reconnu la place en personne, et disposé toute chose à faire les attaques le lendemain. Mais pour n’épargner aucune chose qui pût ménager la vie de mes gens, je m’avisai de faire encore entrer le marquis d’Yenne dans la place, et crus que les habitants seraient peut-être bien aise d’être autorisés par les ordres du gouverneur de la province à faire une chose qui était de leur intérêt. En effet, dès le jour même, l’on me fit prier de surseoir les attaques, et la capitulation s’étant faite le lendemain, j’entrai dans Gray le 19 février, achevant ainsi en quinze jours d’hiver une conquête qui, étant entreprise avec moins de précaution, pouvait m’arrêter plus d’une campagne.\par
Sans m’amuser à visiter les villes qui s’étaient rendues en mon absence, je revins le plus vite que je pus à Saint-Germain, où j’avais des affaires importantes à régler ; mais je laissai la liberté à ceux qui étaient avec moi de me suivre ou de revenir à leur commodité.\par
Dans le temps de cette expédition, comme la saison était très fâcheuse, j’avais tâché d’en adoucir la rigueur aux gens de qualité, par la bonne chère que je leur faisais. Et, parce qu’étant à la campagne, on ne peut pas ménager tant de temps pour les affaires de cabinet, je m’entretenais plus librement avec tout le monde, tant en conversation générale qu’en particulier : mais je cherchais néanmoins, autant qu’il se pouvait, à tirer profit de ces entretiens ou pour avancer l’ouvrage auquel j’étais appliqué, ou pour connaître plus à fond les gens même à qui je parlais, ou pour tirer des éclaircissements sur diverses autres choses.\par
C’est une question fort agitée entre les politiques de savoir s’il est à propos que le prince se communique à peu de gens ou à plusieurs. Les uns disent qu’un roi, qui doit savoir tout, doit se communiquer à tout le monde ; d’autres prétendent qu’en partageant l’exécution de ses affaires entre un petit nombre de conseillers, il en recevrait plus de soulagement dans son travail et moins d’incertitude en ses conseils. Il s’en est même trouvé qui ont osé soutenir qu’un monarque, soit pour la tranquillité de son esprit, soit pour la solidité de ses résolutions, ne se devait ouvrir qu’à un seul ministre.\par
Mais, pour moi, mon fils, je crois qu’on peut accorder tous ces avis, en distinguant le temps et les personnes auxquelles ils seraient donnés. Car, pour commencer par celui qui paraît le plus dangereux, je croirais que, s’agissant d’un prince qui, par la faiblesse de son âge, ne serait point capable de gouverner, on pourrait, avec plus de raison, lui conseiller de se confier entièrement à un seul ministre qu’à plusieurs, parce qu’en ayant plusieurs et ne pouvant ni limiter leurs fonctions, ni régler leurs contestations, il les verrait bien plus appliqués à s’élever l’un au-dessus de l’autre, qu’à maintenir la grandeur de son État : au lieu que, remettant tout dans les mains d’un seul, il n’aurait de difficulté qu’à le choisir tel qu’il fût en effet par sa suffisance capable d’un si grand emploi, et par sa naissance hors d’état d’aspirer à rien davantage.\par
Il en serait autrement d’un roi qui, pourvu naturellement de lumières et de vigueur, manquerait seulement d’expérience : car, en ce cas, il ferait sans doute et plus honnêtement et plus sûrement de partager sa créance entre un certain nombre de gens habiles. Mais il faudrait que ce nombre fût petit. Car n’étant pas encore accoutumé aux malicieux artifices des hommes, il ne pourrait pas, entre un grand nombre de rapports différents, distinguer toujours le vrai du vraisemblable : d’où il naîtrait continuellement de la perplexité dans ses pensées, de l’inconstance dans ses résolutions et de l’inquiétude dans l’esprit de ceux même qui le serviraient avec plus de fidélité, lesquels craindraient toujours que la malignité de la cour ne ruinât le mérite de leurs services.\par
Mais enfin, quand il se pourra trouver un prince qui, par la beauté naturelle de son esprit, par la solide fermeté de son âme et par l’habitude prise aux grandes affaires, saura se défendre de la surprise aussi bien que ses plus habiles conseillers, qui entendra aussi bien ou mieux qu’eux ses plus délicats intérêts, et qui, prenant leurs avis parce qu’il lui plaît, pourra néanmoins, quand il sera besoin, se déterminer sagement par lui-même ; qui aura assez de retenue pour ne résoudre rien sur-le-champ de ce qui mériterait réflexion et qui serait assez maître de son visage et de ses paroles, pour apprendre les sentiments de tous sans découvrir les siens qu’à ceux qu’il voudrait, ou peut-être même à personne entièrement, je lui donnerais un conseil différent des autres. Car je désirerais qu’il n’évitât pas, hors du temps de son travail accoutumé, les occasions qui se pourraient naturellement offrir d’entendre parler diverses personnes sur toutes sortes de sujets, soit sous prétexte de jeu, de chasse, de conversation ou même d’audience particulière.\par
L’un des plus grands hommes de l’antiquité, prévenu de cette pensée, disait que celui qui gouverne un État doit se résoudre même fort souvent à écouter des sottises. Et sa raison, à mon avis, était que ce même homme qui nous dit une chose inutile aujourd’hui peut en dire demain une très importante, et que ceux encore qui ne disent rien de sérieux, ne laissent pas de faire que les autres qui traitent les plus grandes affaires auront plus de retenue à mentir, sachant par combien de voies différentes nous pouvons apprendre la vérité.\par
Mais un autre profit que le Prince tirera sans doute de ces différents entretiens, c’est qu’insensiblement il y connaîtra par lui-même les plus honnêtes gens de son État, avantage d’autant plus grand que la principale fonction du monarque est de mettre chacun des particuliers dans le poste où il peut être utile au public. On sait bien que nous ne pouvons pas faire tout ; mais nous devons donner ordre que tout soit bien fait, et cet ordre dépend principalement du choix de ceux que nous employons. Dans un grand État, il y a toujours des gens propres à toutes choses, et la seule question est de les connaître et de les mettre en leur place. Cette maxime, qui dit que pour être sage il suffit de se bien connaître soi-même, est bonne pour les particuliers ; mais le souverain, pour être habile et bien servi, est obligé de connaître tous ceux qui peuvent être à la portée de sa vue. Car enfin ceux de qui nous prenons conseil en toute autre chose, nous peuvent raisonnablement être suspects en celle-ci, parce que plus les places qu’il faut remplir sont importantes, plus l’envie qu’ils ont d’y poster des hommes dépendant d’eux peut, ou les abuser eux-mêmes, ou les tenter de nous abuser.\par
Je sais bien, mon fils, que ces observations sont un peu scrupuleuses, et qu’il y a peu de souverains qui se donnent la peine d’y prendre garde ; mais aussi s’en trouve-t-il bien peu qui s’acquittent pleinement de leur devoir. Si vous ne voulez vivre qu’en prince du commun, content de vous conduire ou plutôt de vous laisser conduire comme les autres, vous n’avez pas besoin de ces leçons. Mais si vous avez un jour, comme je l’espère, la noble ambition de vous signaler, si vous voulez éviter la honte non seulement d’être gouverné, mais seulement d’en être soupçonné, vous ne sauriez observer avec trop d’exactitude les principes que je vous donne ici, et que vous trouverez continuellement dans la suite de cet ouvrage.\par
Durant ce voyage, j’avais appris que les Hollandais, après de longues poursuites, avaient enfin fait résoudre les Anglais à s’unir avec eux par un traité fait à Bruxelles le 23 janvier, dont le principal article était qu’eux et les autres États qui entreraient dans cette ligue travailleraient jusqu’au mois de mai, par toutes sortes d’offices et de persuasions, à faire conclure la paix entre la France et l’Espagne, et que ce terme étant passé, ils y emploieraient des remèdes plus efficaces. Et je compris que cette convention, quoiqu’elle semblât regarder également les deux couronnes, était néanmoins faite contre moi seul, tant parce qu’elle s’était résolue chez mes ennemis, que parce qu’en l’état où étaient les affaires, la paix ne devait apparemment dépendre que de moi.\par
Les Suédois n’avaient pas encore signé ce traité, mais on les y croyait résolus, et moyennant sept cent mille livres de pension que leur payait la république de Hollande, ils s’engageaient à lui fournir dix mille hommes de pied.\par
De la part des princes d’Allemagne, je n’avais pas nouvelle qu’aucun fût encore entré dans ce complot. Mais ceux qui avaient traité avec moi pour défendre le passage du Rhin, sommés de se joindre à M. le Prince, ne m’avaient encore fait de réponse positive. Le duc de Lunebourg donnait ses troupes aux États de Hollande. L’évêque de Münster, sollicité par moi de faire quelque entreprise sur les États Généraux, m’avait témoigné qu’il manquait de forces ; et, en effet, je savais qu’alors même il craignait le ressentiment de l’électeur de Cologne, lequel, ayant prétendu être nommé coadjuteur à cet évêché, s’en trouvait exclu par l’évêque de Paderborn.\par
Le roi de Danemark me faisait parler fort honnêtement ; mais il avait une étroite liaison avec les Hollandais, et il armait alors un bon nombre de vaisseaux.\par
La Pologne, toujours agitée de ses troubles intestins, ne me pouvait assurément par elle-même donner sujet de rien appréhender ; mais j’apprenais aussi que je n’en devais pas attendre le sujet de diversion que j’avais prétendu faire naître de ce côté-là, parce que le roi témoignait encore de l’incertitude sur le sujet de son abdication, et la république n’était pas résolue d’y consentir.\par
L’Empereur paraissait assez tranquille ; mais il avait toujours sur pied d’anciennes troupes, et de la part de l’Espagne on le pressait de se déclarer, avec toute l’impatience possible.\par
Les Électeurs en corps avaient député vers moi comme simples médiateurs ; mais tous n’étaient pas, en effet, dans les mêmes sentiments, et l’électeur de Brandebourg avait un corps considérable de troupes qu’il pouvait donner à mes ennemis.\par
Du côté d’Italie je n’entendais que des exhortations à la paix, soit de la part du Pape comme père commun de tous les princes catholiques, soit de la part des Vénitiens, qui se promettaient d’en tirer quelque secours pour Candie ; et le duc de Savoie, que j’avais excité par diverses propositions à tenter de son chef quelque chose, n’avait pu se résoudre à rien.\par
Pour les Suisses, je sus qu’ils avaient appris avec tout le chagrin possible mon entreprise sur la Franche-Comté, laquelle se vantait d’être sous leur protection, jusque-là qu’ils avaient confisqué les biens des officiers de leur pays dont je m’étais servi dans cette conquête.\par
D’Espagne, l’on disait que don Juan devait passer avec six ou sept mille hommes pour prendre le gouvernement des Pays-Bas ; mais j’avais nouvelle qu’il n’était pas encore parti ; et le duc de Beaufort, par qui je le faisais observer, était en état de lui disputer le passage.\par
Touchant les propositions de paix, le Roi Catholique avait envoyé pouvoir à Castel Rodrigo pour choisir l’un des deux partis que j’avais proposés ; mais après les divers artifices que l’on avait déjà pratiqués pour m’amuser, j’étais en droit de douter de tout, et je me préparais en effet pour aller au plus tôt en Flandre : car enfin c’était là que je prétendais porter mon principal effort. J’avais même changé la résolution d’envoyer mon frère en Catalogne, me contentant d’y laisser seulement… avec un corps d’environ… hommes afin de mener trois grosses armées vers les Pays-Bas. L’une devait aller sur les bords du Rhin, commandée par le prince de Condé, l’autre vers la mer conduite par mon frère, et la troisième, où j’allai en personne, dans le milieu du pays, pour me trouver plus aisément partout où ma présence pouvait être utile.\par
Mais sitôt que la fin de mars approcha, les Hollandais vinrent, appuyés d’une députation célèbre du collège électoral, des suffrages du Pape et du roi d’Angleterre, pour me demander une nouvelle suspension jusqu’à la fin du mois de mai. Ils disaient pour raison que le roi d’Espagne avait fait dès lors, de sa part, tout ce que l’on devait attendre de lui ; que les choses que j’avais prescrites m’étaient accordées, et que, ne restant plus qu’à les revêtir des formes ordinaires, je ne devais pas refuser le temps qui serait absolument nécessaire pour cela. J’avais sans doute de ma part, de quoi répondre à ces remontrances : mais, dans le fond, il s’agissait de voir lequel m’était le plus avantageux et le plus honnête, ou de consentir à la paix aux conditions que j’avais moi-même réglées, ou de continuer la guerre contre les Espagnols et contre ceux qui prendraient leur parti.\par
La délibération était difficile assurément d’elle-même, par le nombre et par le poids des raisons qui se rencontraient des deux côtés ; mais l’embarras particulier que j’y trouvais encore était que je me voyais obligé de prendre ma résolution purement de moi, n’ayant personne que je pusse consulter avec une pleine confiance. Car, d’une part, je ne doutais pas que ceux qui avaient emploi dans la guerre ne s’attachassent insensiblement aux raisons qui me portaient à la continuer ; et d’ailleurs, il était aisé de connaître que les gens dont je me servais en mes autres conseils, ne pouvant ni me suivre à l’armée sans incommodité, ni s’éloigner de moi sans jalousie pour ceux qui me suivaient, se trouveraient naturellement d’accord entre eux à faire valoir dans leurs avis tout ce qui pourrait tendre à la paix.\par
Je ne laissai pas néanmoins d’entendre les uns et les autres pour pouvoir du moins comparer leurs raisons et en juger décisivement moi seul. D’un côté, l’on me représenta le nombre et la vigueur des troupes dont j’avais résolu de me servir, la faiblesse où étaient les Espagnols, et l’indifférence où toute l’Allemagne semblait demeurer. L’on me remontra que toutes mes mesures étaient déjà prises pour la campagne prochaine, mes recrues levées ou ordonnées, mes magasins remplis, et une bonne partie de la dépense faite ; que les Hollandais, qui faisaient tant de bruit, avaient plus de mauvaise volonté que de puissance ; que les Anglais qui s’unissaient à eux, n’avaient ni troupes, ni finances prêtes pour faire aucun effort important ; que les Suédois, n’étant pas encore absolument déclarés, balanceraient apparemment plus d’une fois avant que de quitter l’ancienne alliance de la France pour se joindre aux États naguère leurs ennemis, outre que leur pays étant fort éloigné de nous, leurs forces ne pourraient arriver que fort tard ; mais qu’enfin toutes ces puissances jointes ensemble n’iraient pas encore à la moitié de mes forces ; sans compter, disait-on, ma présence, ma vigueur et mon application, que l’on ne manquait pas de faire valoir pour beaucoup : en sorte qu’avant la fin de la campagne on me promettait infailliblement la conquête des Pays-Bas.\par
Mais quoique ces raisons fussent, en effet, spécieuses et capables de toucher un cœur ambitieux, j’en voyais à regret, de l’autre côté, de plus pressantes et de plus solides. Car ceux qui étaient de l’avis de la paix ne contestaient pas que je ne fusse plus fort que les Espagnols ; mais ils disaient qu’il fallait bien moins de force pour se défendre que pour attaquer ; que plus je ferais de progrès, plus mes armées seraient affaiblies par les grosses garnisons qu’il faudrait laisser chez des peuples nouvellement domptés ; qu’au contraire, mes ennemis s’augmenteraient tous les jours en nombre, par la jalousie qu’on aurait de moi ; que quand bien d’abord je ferais quelque conquête importante, il faudrait bien se résoudre enfin ou à rendre par la paix une bonne partie de ce que j’aurais pris, ou bien à soutenir moi seul une guerre éternelle contre tous mes voisins ; qu’ayant publiquement déclaré, dès le premier jour de cette querelle, que je ne demandais que la valeur de ce qui m’était justement échu, il n’était pas possible que je refusasse de me contenter de ce à quoi je l’avais moi-même estimée, sans attirer contre moi tous les États qui étaient dépositaires de mes paroles ; que l’Empereur, qui paraissait encore indifférent, ne laisserait pas perdre un si beau prétexte d’empêcher l’affaiblissement de sa maison, et d’engager, s’il pouvait, dans son parti tous les États et princes d’Allemagne ; que les Suisses mêmes, déjà fort irrités par la conquête de la Bourgogne, me voyant encore tenter de nouveaux projets, pourraient ou faire quelque chose de leur chef ou favoriser les desseins de mes ennemis ; que le Pape et toute la chrétienté me reprocheraient d’avoir, pour mon intérêt particulier, suspendu les forces de tous les princes chrétiens, pendant que Candie tomberait sans secours entre les mains des infidèles ; et qu’enfin mes peuples, frustrés par les dépenses d’une si grande guerre des soulagements qu’ils attendaient de moi, me pourraient soupçonner d’avoir préféré les intérêts de ma gloire particulière à ceux de leur avantage et de leur repos.\par
Mais, outre ces raisons qui pouvaient être alléguées par tout le monde, il y en avait d’autres qui dépendaient purement des vues secrètes que j’avais alors : car, à dire vrai, je ne regardais pas seulement à profiter de la conjoncture présente, mais encore à me mettre en état de me bien servir de celles qui vraisemblablement pouvaient arriver. Dans les grands accroissements que ma fortune pouvait recevoir, rien ne me semblait plus nécessaire que de m’établir, chez mes plus petits voisins, dans une estime de modération et de probité qui pût adoucir en eux ces mouvements de frayeur que chacun conçoit naturellement à l’aspect d’une trop grande puissance. Et je considérais que je ne pouvais faire paraître ces vertus avec plus d’éclat qu’en me faisant voir ici, les armes à la main, céder pourtant à l’intercession de mes alliés et me contenter d’un dédommagement médiocre.\par
Je remarquais, de plus, que ce dédommagement, pour médiocre qu’il parût à l’égard de ce que je pouvais acquérir par les armes, était néanmoins plus important qu’il me semblait, parce que, m’étant cédé par un traité volontaire, il portait un secret abandonnement des renonciations par lesquelles seules les Espagnols prétendaient exclure la Reine de toutes les successions de sa maison ; que si je m’opiniâtrais maintenant à la guerre, la ligue qui s’allait former pour la soutenir demeurerait ensuite pour toujours comme une barrière opposée à mes plus légitimes prétentions, au lieu qu’en m’accommodant promptement, je la dissipais dès sa naissance, et me donnais le temps de faire naître des affaires aux ligués, qui les empêcheraient de se mêler de celles que le temps me pouvait fournir ; que quand même il n’arriverait rien de nouveau, je ne manquerais pas d’occasions de rompre, quand je voudrais, avec l’Espagne ; que la Franche-Comté, que je rendais, se pouvait réduire en tel état que j’en serais le maître à toute heure, et que mes nouvelles conquêtes bien affermies m’ouvriraient une entrée plus sûre dans le reste des Pays-Bas ; que la paix me donnerait le loisir de me fortifier chaque jour de finances, de vaisseaux, d’intelligences et de tout ce que peuvent ménager les soins d’un prince appliqué dans un État puissant et riche ; et qu’enfin dans toute l’Europe je serais plus considéré, et plus en pouvoir d’obtenir de chaque État particulier ce qui pourrait aller à mes fins, tandis que l’on me verrait sans adversaire, que quand il y aurait un parti formé contre moi.\par
Et, en effet, peu de temps après que j’eus déclaré la résolution que j’avais prise de faire la paix, l’Empereur, convaincu de ma bonne foi, entra en négociation de ce traité éventuel qu’il avait jusque-là rejeté ; et l’affaire ayant été discutée par le comte de Furstemberg de ma part, et par… pour l’Empereur, fut terminée le…, à condition que le cas présupposé arrivant, l’Empereur aurait pour lui… et moi le… Ce qui fut encore une merveilleuse confirmation des droits de la Reine, et un aveu fort exprès de la nullité des renonciations : acte d’autant plus important qu’il était fait par la partie même qui seule alors avait intérêt de les soutenir.
 


% at least one empty page at end (for booklet couv)
\ifbooklet
  \pagestyle{empty}
  \clearpage
  % 2 empty pages maybe needed for 4e cover
  \ifnum\modulo{\value{page}}{4}=0 \hbox{}\newpage\hbox{}\newpage\fi
  \ifnum\modulo{\value{page}}{4}=1 \hbox{}\newpage\hbox{}\newpage\fi


  \hbox{}\newpage
  \ifodd\value{page}\hbox{}\newpage\fi
  {\centering\color{rubric}\bfseries\noindent\large
    Hurlus ? Qu’est-ce.\par
    \bigskip
  }
  \noindent Des bouquinistes électroniques, pour du texte libre à participation libre,
  téléchargeable gratuitement sur \href{https://hurlus.fr}{\dotuline{hurlus.fr}}.\par
  \bigskip
  \noindent Cette brochure a été produite par des éditeurs bénévoles.
  Elle n’est pas faîte pour être possédée, mais pour être lue, et puis donnée.
  Que circule le texte !
  En page de garde, on peut ajouter une date, un lieu, un nom ; pour suivre le voyage des idées.
  \par

  Ce texte a été choisi parce qu’une personne l’a aimé,
  ou haï, elle a en tous cas pensé qu’il partipait à la formation de notre présent ;
  sans le souci de plaire, vendre, ou militer pour une cause.
  \par

  L’édition électronique est soigneuse, tant sur la technique
  que sur l’établissement du texte ; mais sans aucune prétention scolaire, au contraire.
  Le but est de s’adresser à tous, sans distinction de science ou de diplôme.
  Au plus direct ! (possible)
  \par

  Cet exemplaire en papier a été tiré sur une imprimante personnelle
   ou une photocopieuse. Tout le monde peut le faire.
  Il suffit de
  télécharger un fichier sur \href{https://hurlus.fr}{\dotuline{hurlus.fr}},
  d’imprimer, et agrafer ; puis de lire et donner.\par

  \bigskip

  \noindent PS : Les hurlus furent aussi des rebelles protestants qui cassaient les statues dans les églises catholiques. En 1566 démarra la révolte des gueux dans le pays de Lille. L’insurrection enflamma la région jusqu’à Anvers où les gueux de mer bloquèrent les bateaux espagnols.
  Ce fut une rare guerre de libération dont naquit un pays toujours libre : les Pays-Bas.
  En plat pays francophone, par contre, restèrent des bandes de huguenots, les hurlus, progressivement réprimés par la très catholique Espagne.
  Cette mémoire d’une défaite est éteinte, rallumons-la. Sortons les livres du culte universitaire, cherchons les idoles de l’époque, pour les briser.
\fi

\ifdev % autotext in dev mode
\fontname\font — \textsc{Les règles du jeu}\par
(\hyperref[utopie]{\underline{Lien}})\par
\noindent \initialiv{A}{lors là}\blindtext\par
\noindent \initialiv{À}{ la bonheur des dames}\blindtext\par
\noindent \initialiv{É}{tonnez-le}\blindtext\par
\noindent \initialiv{Q}{ualitativement}\blindtext\par
\noindent \initialiv{V}{aloriser}\blindtext\par
\Blindtext
\phantomsection
\label{utopie}
\Blinddocument
\fi
\end{document}
