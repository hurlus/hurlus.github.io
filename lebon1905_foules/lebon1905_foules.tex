%%%%%%%%%%%%%%%%%%%%%%%%%%%%%%%%%
% LaTeX model https://hurlus.fr %
%%%%%%%%%%%%%%%%%%%%%%%%%%%%%%%%%

% Needed before document class
\RequirePackage{pdftexcmds} % needed for tests expressions
\RequirePackage{fix-cm} % correct units

% Define mode
\def\mode{a4}

\newif\ifaiv % a4
\newif\ifav % a5
\newif\ifbooklet % booklet
\newif\ifcover % cover for booklet

\ifnum \strcmp{\mode}{cover}=0
  \covertrue
\else\ifnum \strcmp{\mode}{booklet}=0
  \booklettrue
\else\ifnum \strcmp{\mode}{a5}=0
  \avtrue
\else
  \aivtrue
\fi\fi\fi

\ifbooklet % do not enclose with {}
  \documentclass[french,twoside]{book} % ,notitlepage
  \usepackage[%
    papersize={105mm, 297mm},
    inner=12mm,
    outer=12mm,
    top=20mm,
    bottom=15mm,
    marginparsep=0pt,
  ]{geometry}
  \usepackage[fontsize=9.5pt]{scrextend} % for Roboto
\else\ifav
  \documentclass[french,twoside]{book} % ,notitlepage
  \usepackage[%
    a5paper,
    inner=25mm,
    outer=15mm,
    top=15mm,
    bottom=15mm,
    marginparsep=0pt,
  ]{geometry}
  \usepackage[fontsize=12pt]{scrextend}
\else% A4 2 cols
  \documentclass[twocolumn]{report}
  \usepackage[%
    a4paper,
    inner=15mm,
    outer=10mm,
    top=25mm,
    bottom=18mm,
    marginparsep=0pt,
  ]{geometry}
  \setlength{\columnsep}{20mm}
  \usepackage[fontsize=9.5pt]{scrextend}
\fi\fi

%%%%%%%%%%%%%%
% Alignments %
%%%%%%%%%%%%%%
% before teinte macros

\setlength{\arrayrulewidth}{0.2pt}
\setlength{\columnseprule}{\arrayrulewidth} % twocol
\setlength{\parskip}{0pt} % classical para with no margin
\setlength{\parindent}{1.5em}

%%%%%%%%%%
% Colors %
%%%%%%%%%%
% before Teinte macros

\usepackage[dvipsnames]{xcolor}
\definecolor{rubric}{HTML}{800000} % the tonic 0c71c3
\def\columnseprulecolor{\color{rubric}}
\colorlet{borderline}{rubric!30!} % definecolor need exact code
\definecolor{shadecolor}{gray}{0.95}
\definecolor{bghi}{gray}{0.5}

%%%%%%%%%%%%%%%%%
% Teinte macros %
%%%%%%%%%%%%%%%%%
%%%%%%%%%%%%%%%%%%%%%%%%%%%%%%%%%%%%%%%%%%%%%%%%%%%
% <TEI> generic (LaTeX names generated by Teinte) %
%%%%%%%%%%%%%%%%%%%%%%%%%%%%%%%%%%%%%%%%%%%%%%%%%%%
% This template is inserted in a specific design
% It is XeLaTeX and otf fonts

\makeatletter % <@@@


\usepackage{blindtext} % generate text for testing
\usepackage[strict]{changepage} % for modulo 4
\usepackage{contour} % rounding words
\usepackage[nodayofweek]{datetime}
% \usepackage{DejaVuSans} % seems buggy for sffont font for symbols
\usepackage{enumitem} % <list>
\usepackage{etoolbox} % patch commands
\usepackage{fancyvrb}
\usepackage{fancyhdr}
\usepackage{float}
\usepackage{fontspec} % XeLaTeX mandatory for fonts
\usepackage{footnote} % used to capture notes in minipage (ex: quote)
\usepackage{framed} % bordering correct with footnote hack
\usepackage{graphicx}
\usepackage{lettrine} % drop caps
\usepackage{lipsum} % generate text for testing
\usepackage[framemethod=tikz,]{mdframed} % maybe used for frame with footnotes inside
\usepackage{pdftexcmds} % needed for tests expressions
\usepackage{polyglossia} % non-break space french punct, bug Warning: "Failed to patch part"
\usepackage[%
  indentfirst=false,
  vskip=1em,
  noorphanfirst=true,
  noorphanafter=true,
  leftmargin=\parindent,
  rightmargin=0pt,
]{quoting}
\usepackage{ragged2e}
\usepackage{setspace} % \setstretch for <quote>
\usepackage{tabularx} % <table>
\usepackage[explicit]{titlesec} % wear titles, !NO implicit
\usepackage{tikz} % ornaments
\usepackage{tocloft} % styling tocs
\usepackage[fit]{truncate} % used im runing titles
\usepackage{unicode-math}
\usepackage[normalem]{ulem} % breakable \uline, normalem is absolutely necessary to keep \emph
\usepackage{verse} % <l>
\usepackage{xcolor} % named colors
\usepackage{xparse} % @ifundefined
\XeTeXdefaultencoding "iso-8859-1" % bad encoding of xstring
\usepackage{xstring} % string tests
\XeTeXdefaultencoding "utf-8"
\PassOptionsToPackage{hyphens}{url} % before hyperref, which load url package

% TOTEST
% \usepackage{hypcap} % links in caption ?
% \usepackage{marginnote}
% TESTED
% \usepackage{background} % doesn’t work with xetek
% \usepackage{bookmark} % prefers the hyperref hack \phantomsection
% \usepackage[color, leftbars]{changebar} % 2 cols doc, impossible to keep bar left
% \usepackage[utf8x]{inputenc} % inputenc package ignored with utf8 based engines
% \usepackage[sfdefault,medium]{inter} % no small caps
% \usepackage{firamath} % choose firasans instead, firamath unavailable in Ubuntu 21-04
% \usepackage{flushend} % bad for last notes, supposed flush end of columns
% \usepackage[stable]{footmisc} % BAD for complex notes https://texfaq.org/FAQ-ftnsect
% \usepackage{helvet} % not for XeLaTeX
% \usepackage{multicol} % not compatible with too much packages (longtable, framed, memoir…)
% \usepackage[default,oldstyle,scale=0.95]{opensans} % no small caps
% \usepackage{sectsty} % \chapterfont OBSOLETE
% \usepackage{soul} % \ul for underline, OBSOLETE with XeTeX
% \usepackage[breakable]{tcolorbox} % text styling gone, footnote hack not kept with breakable


% Metadata inserted by a program, from the TEI source, for title page and runing heads
\title{\textbf{ Psychologie des foules }}
\date{1905}
\author{Lebon, Gustave}
\def\elbibl{Lebon, Gustave. 1905. \emph{Psychologie des foules}}
\def\elsource{\{source\}}

% Default metas
\newcommand{\colorprovide}[2]{\@ifundefinedcolor{#1}{\colorlet{#1}{#2}}{}}
\colorprovide{rubric}{red}
\colorprovide{silver}{lightgray}
\@ifundefined{syms}{\newfontfamily\syms{DejaVu Sans}}{}
\newif\ifdev
\@ifundefined{elbibl}{% No meta defined, maybe dev mode
  \newcommand{\elbibl}{Titre court ?}
  \newcommand{\elbook}{Titre du livre source ?}
  \newcommand{\elabstract}{Résumé\par}
  \newcommand{\elurl}{http://oeuvres.github.io/elbook/2}
  \author{Éric Lœchien}
  \title{Un titre de test assez long pour vérifier le comportement d’une maquette}
  \date{1566}
  \devtrue
}{}
\let\eltitle\@title
\let\elauthor\@author
\let\eldate\@date


\defaultfontfeatures{
  % Mapping=tex-text, % no effect seen
  Scale=MatchLowercase,
  Ligatures={TeX,Common},
}


% generic typo commands
\newcommand{\astermono}{\medskip\centerline{\color{rubric}\large\selectfont{\syms ✻}}\medskip\par}%
\newcommand{\astertri}{\medskip\par\centerline{\color{rubric}\large\selectfont{\syms ✻\,✻\,✻}}\medskip\par}%
\newcommand{\asterism}{\bigskip\par\noindent\parbox{\linewidth}{\centering\color{rubric}\large{\syms ✻}\\{\syms ✻}\hskip 0.75em{\syms ✻}}\bigskip\par}%

% lists
\newlength{\listmod}
\setlength{\listmod}{\parindent}
\setlist{
  itemindent=!,
  listparindent=\listmod,
  labelsep=0.2\listmod,
  parsep=0pt,
  % topsep=0.2em, % default topsep is best
}
\setlist[itemize]{
  label=—,
  leftmargin=0pt,
  labelindent=1.2em,
  labelwidth=0pt,
}
\setlist[enumerate]{
  label={\bf\color{rubric}\arabic*.},
  labelindent=0.8\listmod,
  leftmargin=\listmod,
  labelwidth=0pt,
}
\newlist{listalpha}{enumerate}{1}
\setlist[listalpha]{
  label={\bf\color{rubric}\alph*.},
  leftmargin=0pt,
  labelindent=0.8\listmod,
  labelwidth=0pt,
}
\newcommand{\listhead}[1]{\hspace{-1\listmod}\emph{#1}}

\renewcommand{\hrulefill}{%
  \leavevmode\leaders\hrule height 0.2pt\hfill\kern\z@}

% General typo
\DeclareTextFontCommand{\textlarge}{\large}
\DeclareTextFontCommand{\textsmall}{\small}

% commands, inlines
\newcommand{\anchor}[1]{\Hy@raisedlink{\hypertarget{#1}{}}} % link to top of an anchor (not baseline)
\newcommand\abbr[1]{#1}
\newcommand{\autour}[1]{\tikz[baseline=(X.base)]\node [draw=rubric,thin,rectangle,inner sep=1.5pt, rounded corners=3pt] (X) {\color{rubric}#1};}
\newcommand\corr[1]{#1}
\newcommand{\ed}[1]{ {\color{silver}\sffamily\footnotesize (#1)} } % <milestone ed="1688"/>
\newcommand\expan[1]{#1}
\newcommand\foreign[1]{\emph{#1}}
\newcommand\gap[1]{#1}
\renewcommand{\LettrineFontHook}{\color{rubric}}
\newcommand{\initial}[2]{\lettrine[lines=2, loversize=0.3, lhang=0.3]{#1}{#2}}
\newcommand{\initialiv}[2]{%
  \let\oldLFH\LettrineFontHook
  % \renewcommand{\LettrineFontHook}{\color{rubric}\ttfamily}
  \IfSubStr{QJ’}{#1}{
    \lettrine[lines=4, lhang=0.2, loversize=-0.1, lraise=0.2]{\smash{#1}}{#2}
  }{\IfSubStr{É}{#1}{
    \lettrine[lines=4, lhang=0.2, loversize=-0, lraise=0]{\smash{#1}}{#2}
  }{\IfSubStr{ÀÂ}{#1}{
    \lettrine[lines=4, lhang=0.2, loversize=-0, lraise=0, slope=0.6em]{\smash{#1}}{#2}
  }{\IfSubStr{A}{#1}{
    \lettrine[lines=4, lhang=0.2, loversize=0.2, slope=0.6em]{\smash{#1}}{#2}
  }{\IfSubStr{V}{#1}{
    \lettrine[lines=4, lhang=0.2, loversize=0.2, slope=-0.5em]{\smash{#1}}{#2}
  }{
    \lettrine[lines=4, lhang=0.2, loversize=0.2]{\smash{#1}}{#2}
  }}}}}
  \let\LettrineFontHook\oldLFH
}
\newcommand{\labelchar}[1]{\textbf{\color{rubric} #1}}
\newcommand{\milestone}[1]{\autour{\footnotesize\color{rubric} #1}} % <milestone n="4"/>
\newcommand\name[1]{#1}
\newcommand\orig[1]{#1}
\newcommand\orgName[1]{#1}
\newcommand\persName[1]{#1}
\newcommand\placeName[1]{#1}
\newcommand{\pn}[1]{\IfSubStr{-—–¶}{#1}% <p n="3"/>
  {\noindent{\bfseries\color{rubric}   ¶  }}
  {{\footnotesize\autour{ #1}  }}}
\newcommand\reg{}
% \newcommand\ref{} % already defined
\newcommand\sic[1]{#1}
\newcommand\surname[1]{\textsc{#1}}
\newcommand\term[1]{\textbf{#1}}

\def\mednobreak{\ifdim\lastskip<\medskipamount
  \removelastskip\nopagebreak\medskip\fi}
\def\bignobreak{\ifdim\lastskip<\bigskipamount
  \removelastskip\nopagebreak\bigskip\fi}

% commands, blocks
\newcommand{\byline}[1]{\bigskip{\RaggedLeft{#1}\par}\bigskip}
\newcommand{\bibl}[1]{{\RaggedLeft{#1}\par\bigskip}}
\newcommand{\biblitem}[1]{{\noindent\hangindent=\parindent   #1\par}}
\newcommand{\dateline}[1]{\medskip{\RaggedLeft{#1}\par}\bigskip}
\newcommand{\labelblock}[1]{\medbreak{\noindent\color{rubric}\bfseries #1}\par\mednobreak}
\newcommand{\salute}[1]{\bigbreak{#1}\par\medbreak}
\newcommand{\signed}[1]{\bigbreak\filbreak{\raggedleft #1\par}\medskip}

% environments for blocks (some may become commands)
\newenvironment{borderbox}{}{} % framing content
\newenvironment{citbibl}{\ifvmode\hfill\fi}{\ifvmode\par\fi }
\newenvironment{docAuthor}{\ifvmode\vskip4pt\fontsize{16pt}{18pt}\selectfont\fi\itshape}{\ifvmode\par\fi }
\newenvironment{docDate}{}{\ifvmode\par\fi }
\newenvironment{docImprint}{\vskip6pt}{\ifvmode\par\fi }
\newenvironment{docTitle}{\vskip6pt\bfseries\fontsize{18pt}{22pt}\selectfont}{\par }
\newenvironment{msHead}{\vskip6pt}{\par}
\newenvironment{msItem}{\vskip6pt}{\par}
\newenvironment{titlePart}{}{\par }


% environments for block containers
\newenvironment{argument}{\itshape\parindent0pt}{\vskip1.5em}
\newenvironment{biblfree}{}{\ifvmode\par\fi }
\newenvironment{bibitemlist}[1]{%
  \list{\@biblabel{\@arabic\c@enumiv}}%
  {%
    \settowidth\labelwidth{\@biblabel{#1}}%
    \leftmargin\labelwidth
    \advance\leftmargin\labelsep
    \@openbib@code
    \usecounter{enumiv}%
    \let\p@enumiv\@empty
    \renewcommand\theenumiv{\@arabic\c@enumiv}%
  }
  \sloppy
  \clubpenalty4000
  \@clubpenalty \clubpenalty
  \widowpenalty4000%
  \sfcode`\.\@m
}%
{\def\@noitemerr
  {\@latex@warning{Empty `bibitemlist' environment}}%
\endlist}
\newenvironment{quoteblock}% may be used for ornaments
  {\begin{quoting}}
  {\end{quoting}}

% table () is preceded and finished by custom command
\newcommand{\tableopen}[1]{%
  \ifnum\strcmp{#1}{wide}=0{%
    \begin{center}
  }
  \else\ifnum\strcmp{#1}{long}=0{%
    \begin{center}
  }
  \else{%
    \begin{center}
  }
  \fi\fi
}
\newcommand{\tableclose}[1]{%
  \ifnum\strcmp{#1}{wide}=0{%
    \end{center}
  }
  \else\ifnum\strcmp{#1}{long}=0{%
    \end{center}
  }
  \else{%
    \end{center}
  }
  \fi\fi
}


% text structure
\newcommand\chapteropen{} % before chapter title
\newcommand\chaptercont{} % after title, argument, epigraph…
\newcommand\chapterclose{} % maybe useful for multicol settings
\setcounter{secnumdepth}{-2} % no counters for hierarchy titles
\setcounter{tocdepth}{5} % deep toc
\markright{\@title} % ???
\markboth{\@title}{\@author} % ???
\renewcommand\tableofcontents{\@starttoc{toc}}
% toclof format
% \renewcommand{\@tocrmarg}{0.1em} % Useless command?
% \renewcommand{\@pnumwidth}{0.5em} % {1.75em}
\renewcommand{\@cftmaketoctitle}{}
\setlength{\cftbeforesecskip}{\z@ \@plus.2\p@}
\renewcommand{\cftchapfont}{}
\renewcommand{\cftchapdotsep}{\cftdotsep}
\renewcommand{\cftchapleader}{\normalfont\cftdotfill{\cftchapdotsep}}
\renewcommand{\cftchappagefont}{\bfseries}
\setlength{\cftbeforechapskip}{0em \@plus\p@}
% \renewcommand{\cftsecfont}{\small\relax}
\renewcommand{\cftsecpagefont}{\normalfont}
% \renewcommand{\cftsubsecfont}{\small\relax}
\renewcommand{\cftsecdotsep}{\cftdotsep}
\renewcommand{\cftsecpagefont}{\normalfont}
\renewcommand{\cftsecleader}{\normalfont\cftdotfill{\cftsecdotsep}}
\setlength{\cftsecindent}{1em}
\setlength{\cftsubsecindent}{2em}
\setlength{\cftsubsubsecindent}{3em}
\setlength{\cftchapnumwidth}{1em}
\setlength{\cftsecnumwidth}{1em}
\setlength{\cftsubsecnumwidth}{1em}
\setlength{\cftsubsubsecnumwidth}{1em}

% footnotes
\newif\ifheading
\newcommand*{\fnmarkscale}{\ifheading 0.70 \else 1 \fi}
\renewcommand\footnoterule{\vspace*{0.3cm}\hrule height \arrayrulewidth width 3cm \vspace*{0.3cm}}
\setlength\footnotesep{1.5\footnotesep} % footnote separator
\renewcommand\@makefntext[1]{\parindent 1.5em \noindent \hb@xt@1.8em{\hss{\normalfont\@thefnmark . }}#1} % no superscipt in foot
\patchcmd{\@footnotetext}{\footnotesize}{\footnotesize\sffamily}{}{} % before scrextend, hyperref


%   see https://tex.stackexchange.com/a/34449/5049
\def\truncdiv#1#2{((#1-(#2-1)/2)/#2)}
\def\moduloop#1#2{(#1-\truncdiv{#1}{#2}*#2)}
\def\modulo#1#2{\number\numexpr\moduloop{#1}{#2}\relax}

% orphans and widows
\clubpenalty=9996
\widowpenalty=9999
\brokenpenalty=4991
\predisplaypenalty=10000
\postdisplaypenalty=1549
\displaywidowpenalty=1602
\hyphenpenalty=400
% Copied from Rahtz but not understood
\def\@pnumwidth{1.55em}
\def\@tocrmarg {2.55em}
\def\@dotsep{4.5}
\emergencystretch 3em
\hbadness=4000
\pretolerance=750
\tolerance=2000
\vbadness=4000
\def\Gin@extensions{.pdf,.png,.jpg,.mps,.tif}
% \renewcommand{\@cite}[1]{#1} % biblio

\usepackage{hyperref} % supposed to be the last one, :o) except for the ones to follow
\urlstyle{same} % after hyperref
\hypersetup{
  % pdftex, % no effect
  pdftitle={\elbibl},
  % pdfauthor={Your name here},
  % pdfsubject={Your subject here},
  % pdfkeywords={keyword1, keyword2},
  bookmarksnumbered=true,
  bookmarksopen=true,
  bookmarksopenlevel=1,
  pdfstartview=Fit,
  breaklinks=true, % avoid long links
  pdfpagemode=UseOutlines,    % pdf toc
  hyperfootnotes=true,
  colorlinks=false,
  pdfborder=0 0 0,
  % pdfpagelayout=TwoPageRight,
  % linktocpage=true, % NO, toc, link only on page no
}

\makeatother % /@@@>
%%%%%%%%%%%%%%
% </TEI> end %
%%%%%%%%%%%%%%


%%%%%%%%%%%%%
% footnotes %
%%%%%%%%%%%%%
\renewcommand{\thefootnote}{\bfseries\textcolor{rubric}{\arabic{footnote}}} % color for footnote marks

%%%%%%%%%
% Fonts %
%%%%%%%%%
\usepackage[]{roboto} % SmallCaps, Regular is a bit bold
% \linespread{0.90} % too compact, keep font natural
\newfontfamily\fontrun[]{Roboto Condensed Light} % condensed runing heads
\ifav
  \setmainfont[
    ItalicFont={Roboto Light Italic},
  ]{Roboto}
\else\ifbooklet
  \setmainfont[
    ItalicFont={Roboto Light Italic},
  ]{Roboto}
\else
\setmainfont[
  ItalicFont={Roboto Italic},
]{Roboto Light}
\fi\fi
\renewcommand{\LettrineFontHook}{\bfseries\color{rubric}}
% \renewenvironment{labelblock}{\begin{center}\bfseries\color{rubric}}{\end{center}}

%%%%%%%%
% MISC %
%%%%%%%%

\setdefaultlanguage[frenchpart=false]{french} % bug on part


\newenvironment{quotebar}{%
    \def\FrameCommand{{\color{rubric!10!}\vrule width 0.5em} \hspace{0.9em}}%
    \def\OuterFrameSep{\itemsep} % séparateur vertical
    \MakeFramed {\advance\hsize-\width \FrameRestore}
  }%
  {%
    \endMakeFramed
  }
\renewenvironment{quoteblock}% may be used for ornaments
  {%
    \savenotes
    \setstretch{0.9}
    \normalfont
    \begin{quotebar}
  }
  {%
    \end{quotebar}
    \spewnotes
  }


\renewcommand{\headrulewidth}{\arrayrulewidth}
\renewcommand{\headrule}{{\color{rubric}\hrule}}

% delicate tuning, image has produce line-height problems in title on 2 lines
\titleformat{name=\chapter} % command
  [display] % shape
  {\vspace{1.5em}\centering} % format
  {} % label
  {0pt} % separator between n
  {}
[{\color{rubric}\huge\textbf{#1}}\bigskip] % after code
% \titlespacing{command}{left spacing}{before spacing}{after spacing}[right]
\titlespacing*{\chapter}{0pt}{-2em}{0pt}[0pt]

\titleformat{name=\section}
  [block]{}{}{}{}
  [\vbox{\color{rubric}\large\raggedleft\textbf{#1}}]
\titlespacing{\section}{0pt}{0pt plus 4pt minus 2pt}{\baselineskip}

\titleformat{name=\subsection}
  [block]
  {}
  {} % \thesection
  {} % separator \arrayrulewidth
  {}
[\vbox{\large\textbf{#1}}]
% \titlespacing{\subsection}{0pt}{0pt plus 4pt minus 2pt}{\baselineskip}

\ifaiv
  \fancypagestyle{main}{%
    \fancyhf{}
    \setlength{\headheight}{1.5em}
    \fancyhead{} % reset head
    \fancyfoot{} % reset foot
    \fancyhead[L]{\truncate{0.45\headwidth}{\fontrun\elbibl}} % book ref
    \fancyhead[R]{\truncate{0.45\headwidth}{ \fontrun\nouppercase\leftmark}} % Chapter title
    \fancyhead[C]{\thepage}
  }
  \fancypagestyle{plain}{% apply to chapter
    \fancyhf{}% clear all header and footer fields
    \setlength{\headheight}{1.5em}
    \fancyhead[L]{\truncate{0.9\headwidth}{\fontrun\elbibl}}
    \fancyhead[R]{\thepage}
  }
\else
  \fancypagestyle{main}{%
    \fancyhf{}
    \setlength{\headheight}{1.5em}
    \fancyhead{} % reset head
    \fancyfoot{} % reset foot
    \fancyhead[RE]{\truncate{0.9\headwidth}{\fontrun\elbibl}} % book ref
    \fancyhead[LO]{\truncate{0.9\headwidth}{\fontrun\nouppercase\leftmark}} % Chapter title, \nouppercase needed
    \fancyhead[RO,LE]{\thepage}
  }
  \fancypagestyle{plain}{% apply to chapter
    \fancyhf{}% clear all header and footer fields
    \setlength{\headheight}{1.5em}
    \fancyhead[L]{\truncate{0.9\headwidth}{\fontrun\elbibl}}
    \fancyhead[R]{\thepage}
  }
\fi

\ifav % a5 only
  \titleclass{\section}{top}
\fi

\newcommand\chapo{{%
  \vspace*{-3em}
  \centering % no vskip ()
  {\Large\addfontfeature{LetterSpace=25}\bfseries{\elauthor}}\par
  \smallskip
  {\large\eldate}\par
  \bigskip
  {\Large\selectfont{\eltitle}}\par
  \bigskip
  {\color{rubric}\hline\par}
  \bigskip
  {\Large TEXTE LIBRE À PARTICPATION LIBRE\par}
  \centerline{\small\color{rubric} {hurlus.fr, tiré le \today}}\par
  \bigskip
}}

\newcommand\cover{{%
  \thispagestyle{empty}
  \centering
  {\LARGE\bfseries{\elauthor}}\par
  \bigskip
  {\Large\eldate}\par
  \bigskip
  \bigskip
  {\LARGE\selectfont{\eltitle}}\par
  \vfill\null
  {\color{rubric}\setlength{\arrayrulewidth}{2pt}\hline\par}
  \vfill\null
  {\Large TEXTE LIBRE À PARTICPATION LIBRE\par}
  \centerline{{\href{https://hurlus.fr}{\dotuline{hurlus.fr}}, tiré le \today}}\par
}}

\begin{document}
\pagestyle{empty}
\ifbooklet{
  \cover\newpage
  \thispagestyle{empty}\hbox{}\newpage
  \cover\newpage\noindent Les voyages de la brochure\par
  \bigskip
  \begin{tabularx}{\textwidth}{l|X|X}
    \textbf{Date} & \textbf{Lieu}& \textbf{Nom/pseudo} \\ \hline
    \rule{0pt}{25cm} &  &   \\
  \end{tabularx}
  \newpage
  \addtocounter{page}{-4}
}\fi

\thispagestyle{empty}
\ifaiv
  \twocolumn[\chapo]
\else
  \chapo
\fi
{\it\elabstract}
\bigskip
\makeatletter\@starttoc{toc}\makeatother % toc without new page
\bigskip

\pagestyle{main} % after style

  \section[{Préface}]{Préface}\renewcommand{\leftmark}{Préface}

\noindent Notre précédent ouvrage a été consacré à décrire l’âme des races. Nous allons étudier maintenant l’âme des foules.\par
L’ensemble de caractères communs que l’hérédité impose à tous les individus d’une race constitue l’âme de cette race. Mais lorsqu’un certain nombre de ces indi­vidus se trouvent réunis en foule pour agir, l’observation démontre que, du fait même de leur rapprochement, résultent certains caractères psychologiques nouveaux qui se superposent aux caractères de race, et qui parfois en diffèrent profondément.\par
Les foules organisées ont toujours joué un rôle considérable dans la vie des peuples ; mais ce rôle n’a jamais été aussi important qu’aujourd’hui. L’action incon­sciente des foules se substituant à l’activité consciente des individus est une des principales caractéristiques de l’âge actuel.\par
J’ai essayé d’aborder le difficile problème des foules avec des procédés exclusi­vement scientifiques, c’est-à-dire en tâchant d’avoir une méthode et en laissant de côté les opinions, les théories et les doctrines. C’est là, je crois, le seul moyen d’arriver à découvrir quelques parcelles de vérité, surtout quand il s’agit, comme ici, d’une question passionnant vivement les esprits. Le savant, qui cherche à constater un phé­nomène, n’a pas à s’occuper des intérêts que ses constatations peuvent heurter. Dans une publication récente, un éminent penseur, M. Goblet d’Alviela, faisait observer que, n’appartenant à aucune des écoles contemporaines, je me trouvais par. fois en opposition avec certaines conclusions de toutes ces écoles. Ce nouveau travail méritera, je l’espère, la même observation. Appartenir à une école, c’est en épouser nécessairement les préjugés et les partis pris.\par
Je dois cependant expliquer au lecteur pourquoi il me verra tirer de mes études des conclusions différentes de celles qu’au premier abord on pourrait croire qu’elles comportent ; constater par exemple l’extrême infériorité mentale des foules, y compris les assemblées d’élite, et déclarer pourtant que, malgré cette infériorité, il serait dangereux de toucher à leur organisation.\par
C’est que l’observation la plus attentive des faits de l’histoire m’a toujours montré que les organismes sociaux étant aussi compliqués que ceux de tous les êtres, il n’est pas du tout en notre pouvoir de leur faire subir brusquement des transformations profondes. La nature est radicale parfois, mais jamais comme nous l’entendons, et c’est pourquoi la manie des grandes réformes est ce qu’il y a de plus funeste pour un peuple, quelque excellentes que ces réformes puissent théoriquement paraître. Elles ne seraient utiles que s’il était possible de changer instantanément l’âme des nations. Or le temps seul possède un tel pouvoir. Ce qui gouverne les hommes, ce sont les idées, les sentiments et les mœurs, choses qui sont en nous-mêmes. Les institutions et les lois sont la manifestation de notre âme, l’expression de ses besoins. Procédant de cette âme, institutions et lois ne sauraient la changer.\par
L’étude des phénomènes sociaux ne peut être séparée de celle des peuples chez lesquels ils se sont produits. Philosophiquement, ces phénomènes peuvent avoir une valeur absolue ; pratiquement ils n’ont qu’une valeur relative.\par
Il faut donc, en étudiant un phénomène social, le considérer successivement sous deux aspects très différents. On voit alors que les enseignements de la raison pure sont bien souvent contraires à ceux de la raison pratique. Il n’est guère de données, même physiques, auxquelles cette distinction ne soit applicable. Au point de vue de la vérité absolue, un cube, un cercle, sont des figures géométriques invariables, rigou­reusement définies par certaines formules. Au point de vue de notre œil, ces figures géométriques peuvent revêtir des formes très variées. La perspective peut transformer en effet le cube en pyramide ou en carré, le cercle en ellipse ou en ligne droite ; et ces formes fictives sont beaucoup plus importantes à considérer que les formes réelles, puisque ce sont les seules que nous voyons et que la photographie ou la peinture puissent reproduire. L’irréel est dans certains cas plus vrai que le réel. Figurer les objets avec leurs formes géométriques exactes serait déformer la nature et la rendre méconnaissable. Si nous supposons un monde dont les habitants ne puissent que copier ou photographier les objets sans avoir la possibilité de les toucher, ils n’arrive­raient que très difficilement à se faire une idée exacte de leur forme. La connaissance de cette forme, accessible seulement à un petit nombre de savants, ne présenterait d’ailleurs qu’un intérêt très faible.\par
Le philosophe qui étudie les phénomènes sociaux doit avoir présent à l’esprit, qu’à côté de leur valeur théorique ils ont une valeur pratique, et que, au point de vue de l’évolution des civilisations, cette dernière est la seule possédant quelque importance. Une telle constatation doit le rendre fort circonspect dans les conclusions que la loi que semble d’abord lui imposer.\par
D’autres motifs encore contribuent à lui dicter cette réserve. La complexité des faits sociaux est telle qu’il est impossible de les embrasser dans leur ensemble, et de prévoir les effets de leur influence réciproque. Il semble aussi que derrière les faits visibles se cachent parfois des milliers de causes invisibles. Les phénomènes sociaux visibles paraissent être la résultante d’un immense travail inconscient, inaccessible le plus souvent à notre analyse. On peut comparer les phénomènes perceptibles aux vagues qui viennent traduire à la surface de l’océan les bouleversements souterrains dont il est le siège, et que nous ne connaissons pas. Observées dans la plupart de leurs actes, les foules font preuve le plus souvent d’une mentalité singulièrement infé­rieure ; mais il est d’autres actes aussi où elles paraissent guidées par ces forces mystérieuses que les anciens appelaient destin, nature, providence, que nous appelons voix des morts, et dont nous ne saurions méconnaître la puissance, bien que nous ignorions leur essence. Il semblerait parfois que dans le sein des nations se trouvent des forces latentes qui les guident, Qu’y a-t-il, par exemple, de plus compliqué, de plus logique, de plus merveilleux qu’une langue ? Et d’où sort cependant cette chose si bien organisée et si subtile, sinon de l’âme inconsciente des foules ? Les académies les plus savantes, les grammairiens les plus estimés ne font qu’enregistrer pénible­ment les lois qui régissent ces langues, et seraient totalement incapables de les créer. Même pour les idées de génie des grands hommes, sommes-nous bien certains qu’elles soient exclusivement leur œuvre ? Sans doute elles sont toujours créées par des esprits solitaires ; mais les milliers de grains de poussière qui forment l’alluvion où ces idées ont germé, n’est-ce pas l’âme des foules qui les a formés ?\par
Les foules, sans doute, sont toujours inconscientes mais cette inconscience même est peut-être un des secrets de leur force. Dans la nature, les êtres soumis exclusi­vement à l’instinct exécutent des actes dont la complexité merveilleuse nous étonne. La raison est chose trop neuve dans l’humanité, et trop imparfaite encore pour pouvoir nous révéler les lois de l’inconscient et surtout le remplacer. Dans tous nos actes la part de l’inconscient est immense et celle de la raison très petite. L’inconscient agit comme une force encore inconnue.\par
Si donc nous voulons rester dans les limites étroites mais sûres des choses que la science peut connaître, et ne pas errer dans le domaine des conjectures vagues et des vaines hypothèses, il nous faut constater simplement les phénomènes qui nous sont accessibles, et nous borner à cette constatation. Toute conclusion tirée de nos obser­vations est le plus souvent prématurée, car, derrière les phénomènes que nous voyons bien, il en est d’autres que nous voyons mal, et peut-être même, derrière ces derniers, d’autres encore que nous ne voyons pas.
\section[{Introduction. L’ère des foules}]{Introduction. L’ère des foules}\renewcommand{\leftmark}{Introduction. L’ère des foules}


\begin{argument}\noindent Évolution de l’âge actuel. – Les grands chargements de civilisation sont la conséquence de change­ments dans la pensée des peuples. – La croyance moderne à la puissance des foules. – Elle transforme la politique traditionnelle des États. – Comment se produit l’avènement des classes populaires et comment s’exerce leur puissance. – Conséquences nécessaires de la puissance des foules. – Elles ne peuvent exercer qu’un rôle destructeur.- C’est par elles que s’achève la dissolution des civilisations devenues trop vieilles. – Ignorance générale de la psychologie des foules. – Importance de l’étude des foules pour les législateurs et les hommes d’État.
\end{argument}

\noindent Les grands bouleversements qui précèdent les changements de civilisations, tels que la chute de l’Empire romain et la fondation de l’Empire arabe par exemple sem­blent, au premier abord, déterminés surtout par des transformations politiques consi­dérables : invasions de peuples ou renversements de dynasties. Mais une étude plus attentive de ces événements montre que, derrière leurs causes apparentes, se trouve le plus souvent, comme cause réelle, une modification profonde dans les idées des peuples. Les véritables bouleversements historiques ne sont pas ceux qui nous étonnent par leur grandeur et leur violence. Les seuls changements importants, ceux d’où le renouvellement des civilisations découle, s’opèrent dans les idées, les concep­tions et les croyances. Les événements mémorables de l’histoire sont les effets visi­bles des invisibles changements de la pensée des hommes. Si ces grands événements se manifestent si rarement c’est qu’il n’est rien d’aussi stable dans une race que le fond héréditaire de ses pensées.\par
L’époque actuelle constitue un de ces moments critiques où la pensée des hommes est en voie de se transformer.\par
Deux facteurs fondamentaux sont à la base de cette transformation. Le premier est la destruction des croyances religieuses, politiques et sociales d’où dérivent tous les éléments de notre civilisation. Le second est la création de conditions d’existence et de pensée entièrement nouvelles, par suite des découvertes modernes des sciences et de l’industrie.\par
Les idées du passé, bien qu’à demi détruites, étant très puissantes encore, et les idées qui doivent les remplacer n’étant qu’en voie de formation, l’âge moderne repré­sente une période de transition et d’anarchie.\par
De cette période, forcément un peu chaotique, il n’est pas aisé de dire maintenant ce qui pourra sortir un jour. Quelles seront les idées fondamentales sur les­quelles s’édifieront les sociétés qui succéderont à la nôtre ? Nous ne le savons pas encore. Mais ce que, dès maintenant, nous voyons bien, c’est que, pour leur organisation, elles auront à compter avec une puissance, nouvelle, dernière souveraine de l’âge moder­ne : la puissance des foules. Sur les ruines de tant d’idées, tenues pour vraies jadis et qui sont mortes aujourd’hui, de tant de pouvoirs que les révolutions ont successive­ment brisés, cette puissance est la seule qui se soit élevée, et elle paraît devoir absorber bientôt les autres. Alors que toutes nos antiques croyances chancellent et disparaissent, que les vieilles colonnes des sociétés s’effondrent tour à tour, la puissance des foules est la seule force que rien ne menace et dont le prestige ne fasse que grandir. L’âge où nous entrons sera véritablement l’ÈRE DES FOULES.\par
Il y a un siècle à peine, la politique traditionnelle des États et les rivalités des princes étaient les principaux facteurs des événements. L’opinion des foules ne comp­tait guère, et même, le plus souvent, ne comptait pas. Aujourd’hui ce sont les tradi­tions politiques, les tendances individuelles des souverains, leurs rivalités qui ne comptent plus, et, au contraire, la voix des foules qui est devenue prépondérante. Elle dicte aux rois leur conduite, et c’est elle qu’ils tâchent d’entendre. Ce n’est plus dans les conseils des princes, mais dans l’âme des foules que se préparent les destinées des nations.\par
L’avènement des classes populaires à la vie politique, c’est-à-dire, en réalité, leur transformation progressive en classes dirigeantes, est une des caractéristiques les plus saillantes de notre époque de transition. Ce n’est pas, en réalité, par le suffrage univer­sel, si peu influent pendant longtemps et d’une direction d’abord si facile, que cet avènement a été marqué. La naissance progressive de la puissance des foules s’est faite d’abord par la propagation de certaines idées qui se sont lentement implantées dans les esprits, puis par l’association graduelle des individus pour amener la réalisation des conceptions théoriques. C’est par l’association que les foules ont fini par se former des idées, sinon très justes, au moins très arrêtées de leurs intérêts et par avoir conscience de leur force. Elles fondent des syndicats devant lesquels tous les pouvoirs capitulent tour à tour, des bourses du travail qui, en dépit de toutes les lois économiques tendent à régir les conditions du labeur et du salaire. Elles envoient dans les assemblées gouvernementales des représentants dépouillés de toute initiative, de toute indépendance, et réduits le plus souvent à n’être que les porte-parole des comités qui les ont choisis.\par
Aujourd’hui les revendications des foules deviennent de plus en plus nettes, et ne vont pas à moins qu’à détruire de fond en comble la société actuelle, pour la ramener à ce communisme primitif qui fut l’état normal de tous les groupes humains avant l’aurore de la civilisation. Limitation des heures de travail, expropriation des mines, des chemins de fer, des usines et du sol ; partage égal de tous les produits, élimina­tion de toutes les classes supérieures au profit des classes populaires, etc. Telles sont ces revendications.\par
Peu aptes au raisonnement, les foules sont au con­traire très aptes à l’action. Par leur organisation actuelle, leur force est devenue immense. Les dogmes que nous voyons naître auront bientôt la puissance des vieux dogmes c’est-à-dire, la force tyrannique et souveraine qui met à l’abri de la discussion. Le droit divin des foules va remplacer le droit divin des rois.\par
Les écrivains en faveur auprès de notre bourgeoisie actuelle, ceux qui représentent le mieux ses idées un peu étroites, ses vues un peu courtes, son scepticisme un peu sommaire, son égoïsme parfois un peu excessif, s’affolent tout à fait devant le pouvoir nouveau qu’ils voient grandir, et, pour combattre le désordre des esprits, ils adressent des appels désespérés aux forces morales de l’Église, tant dédaignées par eux jadis. Ils nous parlent de la banqueroute de la science, et revenus tout pénitents de Rome, nous rappellent aux enseignements des vérités révélées. Mais ces nouveaux convertis, oublient qu’il est trop tard. Si vraiment la grâce les a touchés, elle ne saurait avoir le même pouvoir sur des âmes peu soucieuses des préoccupations qui assiègent ces récents dévots. Les foules ne veulent plus aujourd’hui des dieux dont eux-mêmes ne voulaient pas hier et qu’ils ont contribué à briser. Il n’est pas de puissance divine ou humaine qui puisse obliger les fleuves à remonter vers leur source.\par
La science n’a fait aucune banqueroute et n’est pour rien dans l’anarchie actuelle des esprits ni dans la puissance nouvelle qui grandit au milieu de cette anarchie. Elle nous a promis la vérité, ou au moins la connaissance des relations que notre intelli­gence peut saisir ; elle ne nous a jamais promis ni la paix ni le bonheur. Souveraine­ment indifférente à nos sentiments, elle n’entend pas nos lamentations. C’est à nous de tâcher de vivre avec elle puisque rien ne pourrait ramener les illusions quelle a fait fuir.\par
D’universels symptômes, visibles chez toutes les nations, nous montrent l’accrois­sement rapide de la puissance des foules, et ne nous permettent pas de supposer que cette puissance doive cesser bientôt de grandir. Quoi qu’elle nous apporte, nous de­vrons le subir.\par
Toute dissertation contre elle ne représente que vaines paroles. Certes il est possible que l’avènement des foules marque une des dernières étapes des civilisations de l’Occident, un retour complet vers ces périodes d’anarchie confuse qui semblent devoir toujours précéder l’éclosion de chaque société nouvelle. Mais comment l’em­pêcherions-nous ?\par
Jusqu’ici ces grandes destructions de civilisations trop vieilles ont constitué le rôle le plus clair des foules. Ce n’est pas, en effet, d’aujourd’hui seulement que ce rôle apparaît dans le monde. L’histoire nous dit qu’au moment où les forces morales sur lesquelles reposait une civilisation ont perdu leur empire, la dissolution finale est effectuée par ces foules inconscientes et brutales assez juste­ment qualifiées de barbares. Les civilisations n’ont été créées et guidées jusqu’ici que par une petite aristocratie intellectuelle, jamais par les foules. Les foules n’ont de puissance que pour détruire. Leur domination représente toujours une phase de barbarie. Une civilisation implique des règles fixes, une discipline, le passage de l’instinctif au rationnel, la prévoyance de l’avenir, un degré élevé de culture, conditions que les foules, abandon­nées à elles-mêmes, se sont toujours montrées absolument incapables de réaliser. Par leur puissance uniquement destructive, elles agissent comme ces microbes qui acti­vent la dissolution des corps débilités ou des cadavres. Quand l’édifice d’une civili­sation est ver­moulu, ce sont toujours les foules qui en amènent l’écroulement. C’est alors qu’apparaît leur principal rôle, et que, pour un instant, la philosophie du nombre semble la seule philosophie de l’histoire.\par
En sera-t-il de même pour notre civilisation ? C’est ceque pouvons craindre, mais c’est ce que nous ne pouvons encore savoir.\par
Quoi qu’il en soit, il faut bien nous résigner à subir le règne des foules, puisque des mains imprévoyantes ont successivement renversé toutes les barrières qui pou­vaient les contenir.\par
Ces foules, dont on commence à tant parler, nous les connaissons bien peu. Les psychologues professionnels, ayant vécu loin d’elles, les ont toujours ignorées, et quand ils s’en sont occupés, ce n’a été qu’au point de vue des crimes qu’elles peuvent commettre. Sans doute, il existe des foules criminelles, mais il existe aussi des foules vertueuses, des foules héroïques, et encore bien d’autres. Les crimes des foules ne constituent qu’un cas particulier de leur psychologie, et on ne connaît pas plus la constitution mentale des foules en étudiant seulement leurs crimes, qu’on ne connaî­trait celle d’un individu en décrivant seulement ses vices.\par
A dire vrai pourtant, tous les maîtres du monde, tous les fondateurs de religions ou d’empires, les apôtres de toutes les croyances, les hommes d’État éminents, et, dans une sphère plus modeste, les simples chefs de petites collectivités humaines, ont toujours été des psychologues inconscients, ayant de l’âme des foules une connais­sance. instinctive, souvent très sûre ; et c’est parce qu’ils la connaissaient bien qu’ils sont si facilement devenus les maîtres. Napoléon pénétrait merveilleusement la psychologie des foules du pays où il a régné, mais il méconnut complètement parfois celle des foules appartenant à des races différentes \footnote{Ses plus subtils conseillers ne la comprirent pas d’ailleurs davantage. Talleyrand lui écrivait que “l’Espagne accueillerait en libérateurs ses soldats.. Elle les accueillit comme des bêtes fauves. Un psychologue, au courant des instincts héréditaires de la race, aurait pu aisément prévoir cet accueil.} ; et c’est parce qu’il la méconnut qu’il entreprit, en Espagne et en Russie notamment, des guerres où sa puissance reçut des chocs qui devaient bientôt l’abattre.\par
La connaissance de la psychologie des foules est aujourd’hui la dernière ressource de l’homme d’État qui veut, non pas les gouverner – la chose est devenue bien diffi­cile, – mais tout au moins ne pas être trop gouverné par elles.\par
Ce n’est qu’en approfondissant un peu la psychologie des foules qu’on comprend à quel point les lois et les institutions ont peu d’action sur elles ; combien elles sont incapables d’avoir des opinions quelconques en dehors de celles qui leur sont imposées ; que ce n’est pas avec des règles basées sur l’équité théorique pure qu’on les conduit, mais en recherchant ce qui peut les impressionner et les séduire. Si un législateur veut, par exemple, établir un nouvel impôt, devra-t-il choisir celui qui sera théoriquement le plus juste ? En aucune façon. Le plus injuste pourra être pratique­ment le meilleur pour les foules. S’il est en même temps le moins visible, et le moins lourd en apparence, il sera le plus facilement admis. C’est ainsi qu’un impôt indirect, si exorbitant qu’il soit, sera toujours accepté par la foule, parce que, étant journelle­ment payé sur des objets de consommation par fractions de centime, il ne gêne pas ses habitudes et ne l’impressionne pas. Remplacez-le par un impôt proportionnel sur les salaires ou autres revenus, à payer en une seule fois, fût-il, théoriquement dix fois moins lourd que l’autre, il soulèvera d’unanimes protestations. Aux centimes invisi­bles de chaque jour se substitue, en effet, une somme relative­ment élevée, qui paraî­tra immense, et par conséquent très impressionnante, le jour où il faudra la payer. Elle ne paraîtrait faible que si elle avait été mise de côté sou à sou ; mais ce procédé économique représente une dose de prévoyance dont les foules sont incapables.\par
L’exemple qui précède est des plus simples ; la justesse en est aisément perçue. Elle n’avait pas échappé à un psychologue comme Napoléon ; mais les législateurs, qui ignorent l’âme des foules, ne sauraient l’apercevoir. L’expérience ne leur a pas encore suffisamment enseigné que les hommes ne se conduisent jamais avec les pres­criptions de la raison pure.\par
Bien d’autres applications pourraient être faites de la psychologie des foules. Sa connaissance jette la plus vive lueur sur un grand nombre de phénomènes historiques et économiques totalement inintelligibles sans elle. J’aurai occasion de montrer que si le plus remarquable des historiens modernes, M. Taine, a si imparfaitement compris parfois les événements de notre grande Révolution, c’est qu’il n’avait jamais songé à étudier l’âme des foules. Il a pris pour guide, dans l’étude de cette période compliquée, la méthode descriptive des naturalistes ; mais, parmi les phénomènes que les natura­listes ont à étudier, les forces morales ne figurent guère. Or ce sont précisément ces forces-là qui constituent les vrais ressorts de l’histoire.\par
À n’envisager que son côté pratique, l’étude de la psychologie des foules méritait donc d’être tentée. N’eût-elle qu’un intérêt de curiosité pure, elle le mériterait encore. Il est aussi intéressant de déchiffrer les mobiles des actions des hommes que de déchiffrer un minéral ou une plante.\par
Notre étude de l’âme des foules ne pourra être qu’une brève synthèse, un simple résumé de nos recherches. Il ne faut lui demander que quelques vues suggestives. D’autres creuseront davantage le sillon. Nous ne faisons aujourd’hui que le tracer sur un terrain bien vierge encore\footnote{ \noindent Les rares auteurs qui se sont occupés de l’étude psychologique des foules ne les ont examinées, comme je le disais plus haut, qu’au point de vue criminel. N’ayant consacré à ce dernier sujet qu’un court chapitre de cet ouvrage, je renverrai le lecteur pour ce point spécial aux études de M. Tarde et à l’opuscule de M. Sighele : \emph{Les foules criminelles.} Ce dernier travail ne contient pas une seule idée personnelle à son auteur, mais il renferme une compilation de faits que les psychologues pourront utiliser. Mes conclusions sur la cri­minalité et la moralité des foules sont d’ailleurs tout à fait contraires à celles des deux écrivains que je viens de citer.\par
 On trouvera dans mon ouvrage, \emph{La Psychologie du Socia}­lisme quelques conséquences des lois qui régissent la psychologie des foules. Ces lois trouvent d’ailleurs des applica­tions dans les sujets les plus divers. M. A. Gevaert, directeur du Conservatoire royal de Bruxelles, a donné récemment une remarquable application des lois que nous avons expo­sées dans un travail sur la musique, qualifiées très juste­ment par lui d’“art des foules”. “Ce sont vos deux ouvrages, m’écrit cet éminent professeur, en m’envoyant son mémoire, qui m’ont donné la solution d’un problème considéré aupa­ravant par moi comme insoluble : l’aptitude étonnante de toute foule à sentir une œuvre musicale récente ou ancienne, indigène ou étrangère, simple ou compliquée, pourvu qu’elle soit produite dans une belle exécution et par des exécutants dirigés par un chef enthousiaste.” M. Gevaert montre admirablement pourquoi “une œuvre restée incomprise à des musi­ciens émérites lisant la partition dans la solitude de leur cabinet. sera parfois saisie d’emblée par un auditoire étran­ger à toute culture technique”. Il montre aussi fort bien pourquoi ces impressions esthétiques ne laissent aucune trace.
}
\section[{Livre I. L’âme des foules}]{Livre I. L’âme des foules}\renewcommand{\leftmark}{Livre I. L’âme des foules}

\subsection[{Chapitre 1. Caractéristiques générales des foules – Loi psychologique de leur unité mentale.}]{Chapitre 1. Caractéristiques générales des foules – Loi psychologique de leur unité mentale.}

\begin{argument}\noindent Ce qui constitue une foule au point de vue psychologique. – Une agglomération nombreuse d’indi­vidus ne suffit pas à former une foule. – Caractères spéciaux des foules psychologiques. – Orientation fixe des idées et sentiments chez les individus qui les composent et évanouissement de leur personna­lité. – La foule est toujours dominée par l’inconscient. – Disparition de la vie cérébrale et prédomi­nance de la vie médullaire. – Abaissement de l’intelligence et transformation complète des sentiments. – Les sentiments transformés peuvent être meilleurs ou pires que ceux des individus dont la foule est composée. – La foule est aussi aisément héroïque que criminelle.
\end{argument}

\noindent Au sens ordinaire le mot foule représente une réunion d’individus quelconques, quels que soient leur nationalité, leur profession ou leur sexe, et quels que soient aussi les hasards qui les rassemblent.\par
An point de vue psychologique, l’expression foule prend une signification tout autre. Dans certaines circonstances données, et seulement dans ces circonstances, une agglomération d’hommes possède des caractères nouveaux fort différents de ceux des individus composant cette agglomération. La personnalité consciente s’évanouit, les sentiments et les idées de toutes les unités sont orientés dans une même direction. Il se forme une âme collective, transitoire sans doute, mais présentant des caractères très nets. La collectivité est alors devenue ce que, faute d’une expression meilleure, j’ap­pellerai une foule organisée, ou, si l’on préfère, une foule psychologique. Elle forme un seul être et se trouve soumise à la \emph{loi de l’unité mentale des foules.}\par
Il est visible que ce n’est pas par le fait seul que beaucoup d’individus se trouvent accidentellement côte à côte, qu’ils acquièrent les caractères d’une foule organisée. Mille individus accidentellement réunis sur une place publique sans aucun but déter­miné, ne constituent nullement une foule au point de vue psychologique. Pour en acquérir les caractères spéciaux, il faut l’influence de certains excitants dont nous aurons à déterminer la nature.\par
L’évanouissement de la personnalité consciente et l’orientation des sentiments et des pensées dans un sens déterminé, qui sont les premiers traits de la foule en voie de s’organiser, n’impliquent pas toujours la présence simultanée de plusieurs individus sur un seul point. Des milliers d’individus séparés peuvent à certains moments, sous l’influence de certaines émotions violentes, un grand événement national par exemple, acquérir les caractères d’une foule psychologique. Il suffira alors qu’un hasard quel­conque les réunisse pour que leurs actes revêtent aussitôt les caractères spéciaux aux actes des foules. A certains moments, une demi-douzaine d’hommes peuvent constituer une foule psychologique, tandis que des centaines d’hommes réunis par hasard peuvent ne pas la constituer. D’autre part, un peuple entier, sans qu’il y ait agglomération visible, peut devenir foule sous l’action de certaines influences.\par
Lorsqu’une foule psychologique est constituée, elle acquiert des caractères géné­raux provisoires, mais déterminables. A ces caractères généraux s’ajoutent des carac­tères particuliers, variables, suivant les éléments dont la foule se compose et qui peuvent en modifier la constitution mentale.\par
Les foules psychologiques sont donc susceptibles d’une classification, et, lorsque nous arriverons à nous occuper de cette classification, nous verrons qu’une foule hété­rogène, c’est-à-dire composée d’éléments dissemblables, présente avec les foules homogènes, c’est-à-dire composées d’éléments plus ou moins semblables (sectes, castes et classes), des caractères communs, et, à côté de ces caractères communs, des particularités qui permettent de l’en différencier.\par
Mais avant de nous occuper des diverses catégories de foules, nous devons exa­miner d’abord les caractères communs à toutes. Nous opérerons comme le naturaliste, qui commence par décrire les caractères généraux communs à tous les individus d’une famille avant de s’occuper des caractères particuliers qui permettent de différencier les genres et les espèces que renferme cette famille.\par
Il n’est pas facile de décrire avec exactitude l’âme des foules, parce que son organisation varie non seulement suivant la race et la composition des collectivités, mais encore suivant la nature et le degré des excitants auxquels ces collectivités sont soumises. Mais la même difficulté se présente dans l’étude psychologique d’un individu quelconque. Ce n’est que dans les romans qu’on voit les individus traverser la vie avec un caractère constant. Seule l’uniformité des milieux crée l’uniformité apparente des caractères. J’ai montré ailleurs que toutes les constitutions mentales contiennent des possibilités de caractère qui peuvent se manifester dès que le milieu change brusquement. C’est ainsi que, parmi les Conventionnels les plus féroces se trouvaient d’inoffensifs bourgeois, qui, dans les circonstances ordinaires, eussent été de pacifiques notaires ou de vertueux magistrats. L’orage passé, ils reprirent leur caractère normal de bourgeois pacifiques. Napoléon trouva parmi eux ses plus dociles serviteurs.\par
Ne pouvant étudier ici tous les degrés de formation des foules, nous les envisa­gerons surtout ces dernières dans leur phase de complète organisation. Nous verrons ainsi ce qu’elles peuvent devenir mais non ce qu’elles sont toujours. C’est seulement à cette phase avancée d’organisation que, sur le fonds invariable et dominant de la race, se superposent certains caractères nouveaux et spéciaux, et que se produit l’orientation de tous les sentiments et pensées de la collectivité dans une direction identique. C’est alors seulement que se manifeste ce que j’ai nommé plus haut, la \emph{loi psychologique de l’unité mentale des foules.}\par
Parmi les caractères psychologiques des foules, il en est qu’elles peuvent présenter en commun avec des individus isolés ; d’autres, au contraire, leur sont absolument spéciaux et ne se rencontrent que chez les collectivités. Ce sont ces caractères spé­ciaux que nous allons étudier d’abord pour bien en montrer l’importance.\par
Le fait le plus frappant que présente une foule psychologique est le suivant : quels que soient les individus qui la composent, quelque semblables ou dissemblables que soient leur genre de vie, leurs occupations, leur caractère ou leur intelligence, par le fait seul qu’ils sont transformés en foule, ils possèdent une sorte d’âme collective qui les fait sentir, penser, et agir d’une façon tout à fait différente de celle dont senti­rait, penserait et agirait chacun d’eux isolément. il y a des idées, des sentiments qui ne surgissent ou ne se transforment en actes que chez les individus en foule. La foule psychologique est un être provisoire, formé d’éléments hétérogènes qui pour un instant se sont soudés, absolument comme les cellules qui constituent un corps vivant forment par leur réunion un être nouveau manifestant des caractères fort différents de ceux que chacune de ces cellules possède.\par
Contrairement à une opinion qu’on s’étonne de trouver sous la plume d’un philoso­phe aussi pénétrant qu’Herbert Spencer, dans l’agrégat qui constitue une foule, il n’y a nullement somme et moyenne des éléments, il y a combinaison et création de nou­veaux caractères, de même qu’en chimie certains éléments mis en présence, les bases et les acides par exemple, se combinent pour former un corps nouveau possédant des propriétés tout à fait différentes de celle des corps ayant servi à le constituer.\par
Il est facile de constater combien l’individu en foule diffère de l’individu isolé ; mais il est moins facile de découvrir les causes de cette différence.\par
Pour arriver à entrevoir au moins ces causes, il faut se rappeler d’abord cette constatation de la psychologie moderne à savoir que ce n’est pas seulement dans la vie organique, mais encore dans le fonctionnement de l’intelligence que les phénomè­nes inconscients jouent un rôle tout à fait prépondérant. La vie consciente de l’esprit ne représente qu’une bien faible part auprès de sa vie inconsciente. L’analyste le plus subtil, l’observateur le plus pénétrant n’arrive guère à découvrir qu’un bien petit nom­bre des mobiles inconscients qui le mènent. Nos actes conscients dérivent d’un substratum inconscient créé surtout par des influences d’hérédité. Ce substratum ren­ferme les innombrables résidus ancestraux qui constituent l’âme de la race. Derrière les causes avouées de nos actes, il y a sans doute les causes secrètes que nous n’avouons pas, mais derrière ces causes secrètes il y en a de beaucoup plus secrètes encore, puisque nous-mêmes les ignorons. La plupart de nos actions journalières ne sont que l’effet de mobiles cachés qui nous échappent.\par
C’est surtout par les éléments inconscients qui forment l’âme d’une race, que se ressemblent tous les individus de cette race, et c’est principalement par les éléments conscients, fruits de l’éducation mais surtout d’une hérédité exceptionnelle, qu’ils diffèrent. Les hommes les plus dissemblables par leur intelligence ont des instincts, des passions, des sentiments fort semblables. Dans tout ce qui est matière de senti­ment religion, politique, morale, affections et antipathies, etc., les hommes les plus éminents ne dépassent que bien rarement le niveau des individus les plus ordinaires. Entre un grand mathématicien et son bottier il peut exister un abîme, au point de vue intellectuel, mais au point de vue du caractère la différence est le plus sou­vent nulle ou très faible.\par
Or ce sont précisément ces qualités générales du caractère, régies par l’inconscient et que la plupart des individus normaux d’une race possèdent à peu près au même degré, qui, dans les foules, sont mises en commun. Dans l’âme collective, les aptitu­des intellectuelles des individus, et par conséquent leur individualité, s’effacent. L’hétérogène se noie dans l’homogène, et les qualités inconscientes dominent.\par
C’est justement cette mise en commun de qualités ordinaires qui nous explique pourquoi les foules ne sauraient jamais accomplir d’actes exigeant une intelligence élevée. Les décisions d’intérêt général prises par une assemblée d’hommes distingués, mais de spécialités différentes, ne sont pas sensiblement supérieures aux décisions que prendrait une réunion d’imbéciles. Ils ne peuvent mettre en commun en effet que ces qualités médiocres que tout le monde possède. Dans les foules, c’est la bêtise et non l’esprit, qui s’accumule. Ce n’est pas tout le monde, comme on le répète si souvent, qui a plus d’esprit que Voltaire, c’est certainement Voltaire qui a plus d’esprit que tout le monde, si par “tout le monde” il faut entendre les foules.\par
Mais si les individus en foule se bornaient à mettre en commun les qualités ordinaires dont chacun d’eux a sa part, il y aurait simplement moyenne, et non, com­me nous l’avons dit, création de caractères nouveaux.\par
Comment s’établissent ces caractères nouveaux ? C’est ce que nous devons rechercher maintenant.\par
Diverses causes déterminent l’apparition de ces caractères spéciaux aux foules, et que les individus isolés ne possèdent pas. La première est que l’individu en foule acquiert, par le fait seul du nombre, un sentiment de puissance invincible qui lui per­met de céder à des instincts que, seul, il eût forcément refrénés. Il sera d’autant moins porté à les refréner que, la foule étant anonyme, et par conséquent irresponsable, le sentiment de la responsabilité, qui retient toujours les individus, disparaît entièrement.\par
Une seconde cause, la contagion, intervient égale­ment pour déterminer chez les foules la manifestation de caractères spéciaux et en même temps leur orientation. La contagion est un phénomène aisé à constater, mais non expliqué, et qu’il faut rattacher aux phénomènes d’ordre hypnotique que nous étudierons dans un instant. Dans une foule, tout sentiment, tout acte est contagieux, et contagieux à ce point que l’individu sacrifie très facilement son intérêt personnel à l’intérêt collectif. C’est là une aptitude fort contraire à sa nature, et dont l’homme n’est guère capable que lorsqu’il fait partie d’une foule.\par
Une troisième cause, et celle-là est de beaucoup la plus importante, détermine dans les individus en foule des caractères spéciaux parfois tout à fait contraires à ceux de l’individu isolé. Je veux parler de la suggestibilité, dont la contagion mentionnée plus haut n’est d’ailleurs qu’un effet.\par
Pour comprendre ce phénomène, il faut avoir présentes à l’esprit certaines décou­vertes récentes de la physiologie. Nous savons aujourd’hui que, par des procédés variés, un individu peut être placé dans un état tel, qu’ayant perdu toute sa person­nalité consciente, il obéisse à toutes les suggestions de l’opérateur qui la lui a fait perdre, et commette les actes les plus contraires à son caractère et à ses habitudes. Or les observations les plus attentives paraissent prouver que l’individu plongé depuis quelque temps au sein d’une foule agissante, se trouve bientôt placé – par suite des effluves qui s’en dégagent, ou pour toute autre cause que nous ne connaissons pas – dans un état particulier, qui se rapproche beaucoup de l’état de fascination où se trouve l’hypnotisé dans les mains de son hypnotiseur. La vie du cerveau étant para­lysée chez le sujet hypnotisé, celui-ci devient l’esclave de toutes les activités inco­nscientes de sa moelle épinière, que l’hypnotiseur dirige à son gré. La personnalité consciente est entièrement évanouie, la volonté et le discernement sont perdus. Tous les sentiments et les pensées sont orientés dans le sens déterminé par l’hypnotiseur.\par
Tel est à peu près aussi l’état de l’individu faisant partie d’une foule psycholo­gique. Il n’est plus conscient de ses actes. Chez lui, comme chez l’hypnotisé, en même temps que certaines facultés sont détruites, d’autres peuvent être amenées à un degré d’exaltation extrême. Sous l’influence d’une suggestion, il se lancera avec une irrésis­tible impétuosité vers l’accomplissement de certains actes. Impétuosité plus irrésisti­ble encore dans les foules que chez le sujet hypnotisé, parce que la suggestion étant la même pour tous les individus s’exagère en devenant réciproque. Les individualités qui, dans la foule, posséderaient une personnalité assez forte pour résister à la suggestion, sont en nombre trop faible pour lutter contre le courant. Tout au plus elles pourront tenter une diversion par une suggestion différente. C’est ainsi, par exemple, qu’un mot heureux, une image évoquée à propos ont parfois détourné les foules des actes les plus sanguinaires.\par
Donc, évanouissement de la personnalité consciente, prédominance de la person­nalité inconsciente, orientation par voie de suggestion et de contagion des sentiments et des idées dans un même sens, tendance à transformer immédiatement en actes les idées suggérées, tels sont les principaux caractères de l’individu en foule. Il n’est plus lui-même, il est devenu un automate que sa volonté ne guide plus.\par
Aussi, par le fait seul qu’il fait partie d’une foule organisée, l’homme descend de plusieurs degrés sur l’échelle de la civilisation. Isolé, c’était peut-être un individu cultivé, en foule c’est un barbare, c’est-à-dire un instinctif. Il a la spontanéité, la violence, la férocité, et aussi les enthousiasmes et les héroïsmes des êtres primitifs. Il tend à s’en rapprocher encore par la facilité avec laquelle il se laisse impressionner par des mots, des images – qui sur chacun des individus isolés composant la foule seraient tout à fait sans action – et conduire à des actes contraires à ses intérêts les plus évidents et à ses habitudes les plus connues. L’individu en foule est un grain de sable au milieu d’autres grains de sable que le vent soulève à son gré.\par
Et c’est ainsi qu’on voit des jurys rendre des verdicts que désapprouverait chaque juré individuellement, des assemblées parlementaires adopter des lois et des mesures que réprouverait en particulier chacun des membres qui les composent. Pris séparé­ment, les hommes de la Convention étaient des bourgeois éclairés, aux habitudes pacifiques. Réunis en foule, ils n’hésitaient pas à approuver les propositions les plus féroces, à envoyer à la guillotine les individus les Plus manifestement innocents ; et, contrairement à tous leurs intérêts, à renoncer à leur inviolabilité et à se décimer eux-mêmes.\par
Et ce n’est pas seulement par ses actes que l’individu en foule, diffère essentielle­ment de lui-même. Avant même qu’il ait perdu toute indépendance, ses idées et ses sentiments se sont transformés, et la transformation est profonde, au point de changer l’avare en prodigue, le sceptique en croyant, l’honnête homme en criminel, le poltron en héros. La renonciation à tous ses privilèges que. dans un moment d’enthousiasme, la noblesse vota pendant la fameuse nuit du 4 août 1789, n’eût certes jamais été acceptée par aucun de ses membres pris isolément.\par
Concluons de ce qui précède, que la foule est toujours intellectuellement infé­rieure à l’homme isolé, mais que, au point de vue des sentiments et des actes que ces sentiments provoquent, elle peut, suivant les circonstances, être meilleure ou pire. Tout dépend de la façon dont la foule est suggestionnée. C’est là ce qu’ont parfaite­ment méconnu les écrivains qui n’ont étudié les foules qu’au point de vue criminel. La foule est souvent criminelle, sans doute, mais souvent aussi elle est héroïque. Ce sont surtout les foules qu’on amène à se faire tuer pour le triomphe d’une croyance ou d’une idée, qu’on enthousiasme pour la gloire et l’honneur, qu’on entraîne presque sans pain et sans armes comme à l’âge des croisades, pour délivrer de l’infidèle le tom­beau d’un Dieu, ou comme en 93, pour défendre le sol de la patrie. Héroïsmes un peu inconscients, sans doute, mais c’est avec ces héroïsmes-là que se fait l’histoire. S’il ne fallait mettre à l’actif des peuples que les grandes actions froidement raison­nées, les annales du monde en enregistreraient bien peu.
\subsection[{Chapitre 2. Sentiments et moralité des foules}]{Chapitre 2. Sentiments et moralité des foules}

\begin{argument}\noindent § 1.\emph{ Impulsivité, mobilité et irritabilité des foules.} – La foule est le jouet de toutes les excitations extérieures et en reflète les incessantes variations. – Les impulsions auxquelles elle obéit sont assez impérieuses pour que l’intérêt personnel s’efface. – Rien n’est prémédité chez les foules. – Action de la race. – § 2. – \emph{Suggestibilité et crédulité des foules. –} Leur obéissance aux suggestions. – Les images évoquées dans leur esprit sont prises par elles pour des réalités. – Pourquoi ces images sont semblables pour tous les individus qui composent une foule. – Égalisation du savant et de l’imbécile dans une foule. – Exemples divers des illusions auxquelles tous les individus d’une foule sont sujets. – Impossibilité d’accorder aucune créance au témoignage des foules. L’unanimité de nombreux témoins est une des plus mauvaises preuves qu’on puisse invoquer pour établir un fait. – Faible valeur des livres d’histoire. § 3.\emph{ Exagération et simplisme des sentiments des foules}. – Les foules ne connaissent ni le doute ni l’incertitude et vont toujours aux extrêmes. – Leurs sentiments sont toujours excessifs § 4. \emph{Intolérance, autoritarisme et conservatisme des foules}. – Raisons de ces sentiments. – Servilité des foules devant une autorité forte. – Les instincts révolutionnaires momentanés des foules ne les empê­chent pas d’être extrêmement conservatrices. – Elles sont d’instinct hostiles aux changements et au progrès. – § 5. – \emph{Moralité des foules.} – La moralité des foules peut, suivant les suggestions, être beaucoup plus basse ou beaucoup plus haute que celle des individus qui les composent. – Explication et exemples. Les foules ont rarement pour guide l’intérêt qui est, le plus souvent, le mobile exclusif de l’individu isolé. – Rôle moralisateur des foules.
\end{argument}

\noindent Après avoir indiqué d’une façon très générale les principaux caractères des foules, il nous reste à pénétrer dans le détail de ces caractères.\par
On remarquera que, parmi les caractères spéciaux des foules, il en est plusieurs, tels que l’impulsivité, l’irritabilité, l’incapacité de raisonner, l’absence de jugement et d’esprit critique, l’exagération des sentiments, et d’autres encore, que l’on observe également chez les êtres appartenant à des formes inférieures d’évolution, tels que la femme, le sauvage et l’enfant mais c’est là une analogie que je n’indique qu’en passant. Sa démonstration sortirait du cadre de cet ouvrage. Elle serait inutile, d’ailleurs, pour les personnes au courant de la psychologie des primitifs, et resterait toujours peu convaincante pour celles qui ne la connaissent pas.\par
J’aborde maintenant l’un après l’autre les divers caractères que l’on peut observer dans la plupart des foules.\par
\subsubsection[{§ 1. – Impulsivité, mobilité et irritabilité des foules}]{§ 1. – Impulsivité, mobilité et irritabilité des foules}
\noindent La foule, avons-nous dit en étudiant ses caractères fondamentaux, est conduite presque exclusivement par l’inconscient. Ses actes sont beaucoup plus sous l’influence de la moelle épinière que sous celle du cerveau. Elle se rapproche en cela des êtres tout à fait primitifs. Les actes exécutés peuvent être parfaits quant à leur exécution, mais, le cerveau ne les dirigeant pas, l’individu agit suivant les hasards des excita­tions. Une foule est le jouet de toutes les excitations extérieures et en reflète les incessantes variations. Elle est donc esclave des impulsions qu’elle reçoit. L’individu isolé peut être soumis aux mêmes excitants que l’homme en foule ; mais comme son cerveau lui montre les inconvénients d’y céder, il n’y cède pas. C’est ce qu’on peut physiologiquement exprimer en disant que l’individu isolé possède l’aptitude à domi­ner ses réflexes, alors que la foule ne la possède pas.\par
Ces impulsions diverses auxquelles obéissent les foules pourront être, suivant les excitations, généreuses ou cruelles, héroïques ou pusillanimes, mais elles seront toujours tellement impérieuses que l’intérêt personnel, l’intérêt de la conservation lui-même, ne les dominera pas. Les excitants qui peuvent agir sur les foules étant fort variés, et les foules y obéissant toujours, celles-ci sont par suite, extrêmement mobi­les ; et c’est pourquoi nous les voyons passer en un instant de la férocité la plus sanguinaire à la générosité ou à l’héroïsme le plus absolu. La foule devient très aisé­ment bourreau, mais non moins aisément elle devient martyre. C’est de son sein qu’ont coulé les torrents de sang exigés par le triomphe de chaque croyance. Il n’est pas besoin de remonter aux âges héroïques pour voir de quoi, a ce dernier point de vue, les foules sont capables. Elles ne marchandent jamais leur vie dans une émeute, et il y a bien peu d’années qu’un général, devenu subitement populaire, eût aisément trouvé cent mille hommes prêts à se faire tuer pour sa cause, s’il l’eût demandé. Rien donc ne saurait être prémédité chez les foules.\par
Elles peuvent parcourir successivement la gamme des sentiments les plus con­traires, mais elles seront toujours sous l’influence des excitations du moment. Elles sont semblables aux feuilles que l’ouragan soulève, disperse en tous sens, puis laisse retomber. En étudiant ailleurs certaines foules révolutionnaires, nous montrerons quelques exemples de la variabilité de leurs sentiments.\par
Cette mobilité des foules les rend très difficiles à gouverner, surtout lorsqu’une partie des pouvoirs publics est tombée entre leurs mains. Si les nécessités de la vie de chaque jour ne constituaient une sorte de régulateur invisible des choses, les démo­craties ne pourraient guère durer. Mais, si les foules veulent les choses avec frénésie, elles ne les veulent pas bien longtemps. Elles sont aussi incapables de volonté durable que de pensée.\par
La foule n’est pas seulement impulsive et mobile. Comme le sauvage, elle n’admet pas que quelque chose puisse s’interposer entre son désir et la réalisation de ce désir. Elle le comprend d’autant moins que le nombre lui donne le sentiment d’une puissance irrésistible. Pour l’individu en foule, la notion d’impossibilité disparaît. L’individu isolé sent bien qu’il ne pourrait à lui seul incendier un palais, piller un magasin, et, s’il en est tenté, il résistera aisément à sa tentation. Faisant partie d’une foule, il a con­science du pouvoir que lui donne le nombre, et il suffit de lui suggérer des idées de meurtre et de pillage pour qu’il cède immédiatement à la tentation. L’obstacle inatten­du sera brisé avec frénésie. Si l’organisme humain permettait la perpétuité de la fureur, on pourrait dire que l’état normal de la foule contrariée est la fureur.\par
Dans l’irritabilité des foules, dans leur impulsivité et leur mobilité, ainsi que dans tous les sentiments populaires que nous aurons à étudier, interviennent toujours les caractères fondamentaux de la race, qui constituent le sol invariable sur lequel ger­ment tous nos sentiments. Toutes les foules sont toujours irritables et impulsives, sans doute, mais avec de grandes variations de degré. La différence entre une foule latine et une foule anglo-saxonne est, par exemple, frappante. Les faits les plus récents de notre histoire jettent une vive lueur sur ce point. Il a suffi, en 1870, de la publication d’un simple télégramme relatant une insulte supposée faite à un ambassadeur pour déterminer une explosion de fureur dont une guerre terrible est immédiate­ment sortie. Quelques années plus tard, l’annonce télé­graphique d’un insignifiant échec à Langson provoqua une nouvelle explosion qui amena le renversement instantané du gouvernement. Au même moment, l’échec beaucoup plus grave d’une expédition anglaise devant Kartoum ne produisit en Angleterre qu’une émotion très faible, et aucun ministère ne fut renversé. Les foules sont partout féminines, mais les plus féminines de toutes sont les foules latines. Qui s’appuie sur elles peut monter très haut et très vite, mais en côtoyant sans cesse la roche Tarpéienne et avec la certitude d’en être précipité un jour.
\subsubsection[{§ 2. – Suggestibilité et crédulité des foules}]{§ 2. – Suggestibilité et crédulité des foules}
\noindent Nous avons dit, en définissant les foules, qu’un de leurs caractères généraux est une suggestibilité excessive, et nous avons montré combien, dans toute aggloméra­tion humaine, une suggestion est contagieuse ; ce qui explique l’orientation rapide des sentiments dans un sens déterminé.\par
Si neutre qu’on la suppose, la foule se trouve le plus souvent dans cet état d’atten­tion expectante qui rend la suggestion facile. La première suggestion formulée qui surgit s’impose immédiatement par contagion à tous les cerveaux, et aussitôt l’orienta­tion s’établit. Comme chez tous les êtres suggestionnés, l’idée qui a envahi le cerveau tend à se transformer en acte. Qu’il s’agisse d’un palais à incendier ou d’un acte de dévouement à accomplir, la foule s’y prête avec la même facilité. Tout dépendra de la nature de l’excitant, et non plus, comme chez l’être isolé, des rapports existant entre l’acte suggéré et la somme de raison qui peut être opposée à sa réalisation.\par
Aussi, errant toujours sur les limites de l’inconscience, subissant aisément toutes les suggestions, ayant toute la violence de sentiments propre aux êtres qui ne peuvent faire appel aux influences de la raison, dépourvue de tout esprit critique, la foule ne peut qu’être d’une crédulité excessive. L’invraisemblable n’existe pas pour elle, et il faut bien se le rappeler pour comprendre la facilité avec laquelle se créent et se pro­pagent les légendes et les récits les plus invraisemblables \footnote{Les personnes qui ont assisté au siège de Paris ont vu de nombreux exemples de cette crédulité des foules aux choses les plus invraisemblables. Une bougie allumée à un étage supérieur était considérée aussitôt comme un signal fait aux assiégeants, bien qu’il fût évident, après deux secondes de réflexion, qu’il leur était absolument impossible d’apercevoir de plusieurs lieues de distance la lueur de cette bougie.}.\par
La création des légendes qui circulent si aisément dans les foules n’est pas déter­minée seulement par une crédulité complète. Elle l’est encore par les déformations prodigieuses que subissent les événements dans l’imagination de gens assemblés. L’événement le plus simple vu par la foule est bientôt un événement transformé. Elle pense par images, et l’image évoquée en évoque elle-même une série d’autres n’ayant aucun lien logique avec la première. Nous concevons aisément cet état en songeant aux bizarres successions d’idées où nous sommes parfois conduits par l’évocation d’un fait quelconque. La raison nous montre ce que dans ces images il y a d’incohérence, mais la foule ne le voit guère ; et ce que son imagination déformante ajoute à l’événe­ment réel, elle le confondra avec lui. La foule ne sépare guère le subjectif de l’objec­tif. Elle admet comme réelles les images évoquées dans son esprit, et qui le plus souvent n’ont qu’une parenté, lointaine avec le fait observé.\par
Les déformations qu’une foule fait subir à un événement quelconque dont elle est témoin devraient, semble-t-il, être innombrables et de sens divers, puis­que les individus qui la composent sont de tempéraments fort différents. Mais il n’en est rien. Par suite de la contagion, les déformations sont de même nature et de même sens pour tous les individus. La première déformation perçue par un des individus de la collectivité est le noyau de la suggestion contagieuse. Avant d’apparaître sur les murs de Jérusalem à tous les croisés, saint Georges ne fut certainement aperçu que par un des assis­tants. Par voie de suggestion et de contagion le miracle signalé par un seul fut immédiatement accepté par tous.\par
Tel est toujours le mécanisme de ces hallucinations collectives si fréquentes dans l’histoire, et qui semblent avoir toutes les caractères classiques de l’authenticité, puisqu’il s’agit de phénomènes constatés par des milliers de personnes.\par
Il ne faudrait pas, pour combattre ce qui précède, faire intervenir la qualité mentale des individus dont se compose la foule. Cette qualité est sans importance. Du moment qu’ils sont en foule, l’ignorant et le savant sont également incapables d’obser­vation.\par
La thèse peut sembler paradoxale. Pour la démontrer à fond, il faudrait reprendre un grand nombre de faits historiques, et plusieurs volumes n’y suffiraient pas.\par
Ne voulant pas cependant laisser le lecteur sous l’impression d’assertions sans preuves, je vais lui donner quelques exemples pris au hasard parmi les monceaux de ceux que l’on pourrait citer.\par
Le fait suivant est un des plus typiques, parce qu’il est choisi parmi des halluci­nations collectives sévissant sur une foule où se trouvaient des individus de toutes sortes, les plus ignorants comme les plus instruits. Il est rapporté incidemment par le lieutenant de vaisseau Julien Félix dans son livre sur les courants de la mer, et a été autrefois reproduit dans la \emph{Revue Scientifique.}\par
La frégate la \emph{Belle-Poule} croisait en mer pour retrouver la corvette \emph{le Berceau} dont elle avait été séparée par un violent orage. On était en plein jour et en plein soleil. Tout à coup la vigie signale une embarcation désemparée. L’équipage dirige ses regards vers le point signalé, et tout le monde, officiers et matelots, aperçoit nettement un radeau chargé d’hommes remorqué par des embarcations sur lesquelles flottaient des signaux de détresse. Ce. n’était pourtant qu’une hallucination collective. L’amiral Desfossés fit armer une embarcation pour voler au secours des naufragés. En approchant, les matelots et les officiers qui la montaient voyaient “des masses d’hommes s’agiter, tendre les mains, et entendaient le bruit sourd et confus d’un grand nombre de voix”. Quand l’embarcation fut arrivée, on se trouva simplement devant quelques branches d’arbres couvertes de feuilles arrachées à la côte voisine. Devant une évidence aussi palpable, l’hallucination s’évanouit.\par
Dans cet exemple on voit se dérouler bien clairement le mécanisme de l’halluci­nation collective tel que nous l’avons expliqué. D’un côté, une foule en état d’atten­tion expectante ; de l’autre, une suggestion faite par la vigie signalant un bâtiment désemparé en mer, suggestion qui, par voie de contagion, fut acceptée par tous les assistants, officiers ou matelots.\par
Il n’est pas besoin qu’une, foule soit nombreuse pour que la faculté de voir correctement ce qui se passe devant elle soit détruite, et les faits réels remplacés par des hallucinations sans parenté avec eux. Dès que quelques individus sont réunis, ils constituent une foule, et, alors même qu’ils seraient des savants distingués, ils pren­nent tous les caractères des foules pour ce qui est en dehors de leur spécialité. La faculté d’observation et l’esprit critique possédés par chacun d’eux s’évanouissent aussitôt. Un psychologue ingénieux, M. Davey, nous en fournit un bien curieux exemple, récemment rapporté par les \emph{Annales des Sciences psychiques}, et qui mérite d’être relaté ici. M. Davey ayant convoqué une réunion d’observateurs distingués, parmi lesquels un des premiers savants de l’Angleterre, M. Wallace, exécuta devant eux, et après leur avoir laissé examiner les objets et poser des cachets où ils voulaient, tous les phénomènes classiques des spirites : matérialisation des esprits, écriture sur des ardoises, etc. Ayant ensuite obtenu de ces observateurs distingués des rapports écrits affirmant que les phénomènes observés n’avaient pu être obtenus que par des moyens surnaturels, il leur révéla qu’ils étaient le résultat de supercheries très simples. “Le plus étonnant de l’investigation de M. Davey, écrit l’auteur de la relation, n’est pas la merveille des tours en eux-mêmes, mais l’extrême faiblesse des rapports qu’en ont faits les témoins non initiés. Donc dit-il, les témoins peuvent faire de nombreux et positifs récits qui sont complètement erronés, mais dont le résultat est que, \emph{si l’on accepte leurs descriptions comme exactes}, les phénomènes qu’ils décrivent sont inexplicables par la supercherie. Les méthodes inventées par M. Davey étaient si simples qu’on est étonné qu’il ait eu la hardiesse de les employer ; mais il avait un tel pouvoir sur l’esprit de la foule qu’il pouvait lui persuader qu’elle voyait ce qu’elle ne voyait pas.” C’est toujours le pouvoir de l’hypnotiseur sur l’hypnotisé. Mais quant on voit ce pouvoir s’exercer sur des esprits supérieurs, préalablement mis en défiance pourtant, on conçoit à quel point il est facile d’illusionner les foules ordinaires.\par
Les exemples analogues sont innombrables. Au moment où j’écris ces lignes, les journaux sont remplis par l’histoire de deux petites filles noyées retirées de la Seine. Ces enfants furent d’abord reconnues de la façon la plus catégorique par une douzaine de témoins. Toutes les affirmations étaient si concordantes qu’il n’était resté aucun doute dans l’esprit du juge d’instruction. Il fit établir l’acte de décès. Mais au moment où on allait procéder à l’inhumation, le hasard fit découvrir que les victimes suppo­sées étaient parfaitement vivantes et n’avaient d’ailleurs qu’une très lointaine ressem­blance avec les petites noyées. Comme dans plusieurs des exemples précédemment cités l’affirmation du premier témoin, victime d’une illusion avait suffi à suggestion­ner tous les autres.\par
Dans les cas semblables, le point de départ de la suggestion est toujours l’illusion produite chez un individu par des réminiscences plus ou moins vagues, puis la contagion par voie d’affirmation de cette illusion primitive. Si le premier observateur est très impressionnable, il suffira souvent que le, cadavre qu’il croit reconnaître présente – en dehors de toute ressemblance réelle – quelque particularité, une cica­trice ou un détail de toilette, qui puisse évoquer l’idée d’une, autre personne.\par
L’idée évoquée peut alors devenir le noyau d’une sorte de cristallisation qui envahit le champ de l’entendement et paralyse toute faculté critique. Ce que l’observa­teur voit alors, ce n’est plus l’objet lui-même, mais l’image évoquée dans son esprit. Ainsi s’expliquent les reconnaissances erronées de cadavres d’enfants par leur propre mère, tel que le cas suivant, déjà ancien, mais qui a été rappelé récemment par les journaux, et où l’on voit se manifester précisément les deux ordres de suggestion dont je viens d’indiquer le mécanisme.\par
“L’enfant fut reconnu par un autre enfant – qui se trompait. La série des recon­naissances inexactes se déroula alors.\par
Et l’on vit une chose très extraordinaire. Le lendemain du jour où un écolier l’avait reconnu, une femme s’écria : “Ah ! mon Dieu, c’est mon enfant.”\par
On l’introduit près du cadavre, elle examine les effets, constate une cicatrice au front. “C’est bien, dit-elle, mon pauvre fils, perdu depuis juillet dernier. On me l’aura volé et on me l’a tué !”\par
La femme était concierge rue du Four et se nommait Chavandret. On fit venir son beau-frère qui, sans hésitation, dit : “Voilà le petit Philibert.” Plusieurs habitants de la rue reconnurent Philibert Chavandret dans l’enfant de la Villette, sans compter son propre maître d’école pour qui la médaille était un indice.\par
Eh bien ! les voisins, le beau-frère, le maître d’école et la mère se trompaient. Six semaines plus tard, l’identité de l’enfant fut établie. C’était un enfant de Bordeaux, tué à Bordeaux et, par les messageries, apporté à Paris \footnote{\emph{Éclair} du 21 avril 1895.}.\par
On remarque que ces reconnaissances se font généralement par des femmes et des enfants, c’est-à-dire précisément par les êtres les plus impressionnables. Elles nous montrent, du même coup, ce que peuvent valoir en justice de tels témoignages. En ce qui concerne les enfants, notamment, leurs affirmations ne devraient jamais être invoquées. Les magistrats répètent comme un lieu commun qu’à cet âge on ne ment pas. Avec une culture psychologique un peu moins sommaire ils sauraient qu’à cet âge, au contraire, on ment presque toujours. Le mensonge, sans doute, est innocent, mais n’en constitue pas moins un mensonge. Mieux vaudrait décider à pile ou face la condamnation d’un accusé que de la décider, comme on l’a fait tant de fois, d’après le témoignage d’un enfant.\par
Pour en revenir aux observations faites par les foules, nous conclurons que les observations collectives sont les plus erronées de toutes et que le plus souvent elles représentent la simple illusion d’un individu qui, par voie de contagion, a sugges­tionné les autres. On pourrait multiplier à l’infini les faits prouvant qu’il faut avoir la plus profonde défiance du témoignage des foules. Des milliers d’hommes ont assisté à la célèbre charge de cavalerie de la bataille de Sedan, et pourtant il est impossible, en présence des témoignages visuels les plus contradictoires, de savoir par qui elle fut commandée. Dans un livre récent, le général anglais Wolseley a prouvé que l’on avait commis jusqu’ici les plus graves erreurs sur les faits les plus considérables de la bataille de Waterloo, faits que des centaines de témoins avaient cependant attestés \footnote{Savons-nous, pour une seule bataille, comment elle s’est passée exactement ? J’en doute fort. Nous savons quels furent les vainqueurs et les vaincus, mais probablement rien de plus. Ce que M. d’Harcourt, acteur et témoin, rapporte de la bataille de Solférino peut s’appliquer à toutes les batailles : “Les généraux (renseignés naturellement par des centaines de témoignages) trans­mettent leurs rapports officiels ; les officiers chargés de porter les ordres modifient ces documents et rédigent le projet définitif ; le chef d’état-major le conteste et le refait sur nouveaux frais. On le porte au Maréchal, il s’écrie : “Vous vous trompez absolument !” et il substitue une nouvelle rédaction. il ne reste presque rien du rapport primitif.” M. d’Harcourt relate ce fait comme une preuve de l’impossibilité où l’on est d’établir la vérité sur l’événement le plus saisissant, le mieux observé.”}.\par
De tels faits nous montrent ce que valent les témoignages des foules. Les traités de logique font rentrer l’unanimité de nombreux témoins dans la catégorie des preu­ves les plus solides qu’on puisse invoquer pour prouver l’exactitude d’un fait. Mais ce que nous savons de la psychologie des foules montre que les traités de logique sont à refaire entièrement sur ce point. Les événements les plus douteux sont certainement ceux qui ont été observés par le plus grand nombre de personnes. Dire qu’an fait a été simultanément constaté par des milliers de témoins, c’est dire le plus souvent que le fait réel est fort différent du récit adopté.\par
Il découle clairement de ce qui précède qu’il faut considérer comme des ouvrages d’imagination pure les livres d’histoire. Ce sont des récits fantaisistes de faits mal observés, accompagnés d’explications faites après coup. Gâcher du plâtre est faire œuvre bien plus utile que de perdre son temps à écrire de tels livres. Si le passé ne nous avait pas légué ses œuvres littéraires, artistiques et monumentales, nous ne saurions absolument rien de réel sur ce passé. Connaissons-nous un seul mot de vrai concernant la vie des grands hommes qui ont joué les rôles prépondérants dans l’hu­manité, tels que Hercule, Bouddha, Jésus ou Mahomet ? Très probablement non. Au fond d’ailleurs, leur vie réelle nous importe fort peu. Ce que nous avons intérêt à con­naître ce sont les grands hommes tels que la légende populaire les a fabriqués. Ce sont les héros légendaires, et pas du tout les héros réels, qui ont impressionné l’âme des foules.\par
Malheureusement les légendes – alors même qu’elles sont fixées par les livres – n’ont elles-mêmes aucune consistance. L’imagination des foules les transforme sans cesse suivant les temps, et surtout suivant les races. il y a loin du Jéhovah sanguinaire de la Bible au Dieu d’amour de sainte Thérèse, et le Bouddha adoré en Chine n’a plus aucuns traits communs avec celui qui est vénéré dans l’Inde.\par
Il n’est même pas besoin que les siècles aient passé sur les héros pour que leur légende soit transformée par l’imagination des foules. La transformation se fait parfois en quelques années. Nous avons vu de nos jours la légende de l’un des plus grands héros de l’histoire se modifier plusieurs fois en moins de cinquante ans. Sous les Bourbons, Napoléon devint une sorte de personnage idyllique philanthrope et libéral, ami des humbles, qui, au dire des poètes, devaient conserver son souvenir sous le chaume pendant bien longtemps. Trente ans après, le héros débonnaire était devenu un despote sanguinaire qui, après avoir usurpé le pouvoir et la liberté, fit périr trois millions d’hommes uniquement pour satisfaire son ambition. De nos jours, nous assistons à une nouvelle transformation de la légende. Quand quelques dizaines de siècles auront passé sur elle, les savants de l’avenir, en présence de ces récits contra­dictoires, douteront peut-être, de l’existence du héros, comme ils doutent parfois de celle de Bouddha, et ne verront en lui que quelque mythe solaire ou un développe­ment de la, légende d’Hercule. Ils se consoleront aisément sans doute de cette incertitudes, car, mieux initiés qu’aujourd’hui à la connaissance de la psychologie des foules, ils sauront que l’histoire ne peut guère éterniser que des mythes.
\subsubsection[{§ 3. – Exagération et simplisme des sentiments}]{§ 3. – Exagération et simplisme des sentiments}
\noindent Quels que soient les sentiments, bons ou mauvais, manifestés par une foule, ils présentent ce double caractère d’être très simples et très exagérés. Sur ce point, comme sur tant d’autres, l’individu en foule se rapproche des êtres primitifs. Inacces­sible aux nuances, il voit les choses en bloc et ne connaît pas les transitions. Dans la foule, l’exagération des sentiments est fortifiée par ce fait, qu’un sentiment manifesté se propageant très vite par voie de suggestion et de contagion, l’approbation évidente dont il est l’objet accroît considérablement sa force.\par
La simplicité et l’exagération des sentiments des foules font que ces dernières ne connaissent ni le doute ni l’incertitude. Comme les femmes, elles vont tout de suite aux extrêmes. Le soupçon énoncé se transforme aussitôt en évidence indiscutable. Un commencement d’antipathie ou de désapprobation, qui, chez l’individu isolé, ne s’ac­centuerait pas, devient aussitôt haine féroce chez l’individu en foule.\par
La violence des sentiments des foules est encore exagérée, dans les foules hétéro­gènes surtout, par l’absence de responsabilité. La certitude de l’impunité, certitude d’autant plus forte que la foule est plus nombreuse et la notion d’une puissance momentanée considérable due au nombre, rendent possibles à la collectivité des senti­ments et des actes impossibles à l’individu isolé. Dans les foules, l’imbécile, l’ignorant et l’envieux sont libérés du sentiment de, leur nullité et de leur impuissance, que remplace la notion d’une force brutale, passagère, mais immense.\par
L’exagération, chez les foules, porte malheureusement souvent sur de mauvais sentiments, reliquat atavique des instincts de l’homme primitif, que la crainte du châ­timent oblige l’individu isolé et responsable à refréner. C’est ce qui fait que les foules sont si facilement conduites aux pires excès.\par
Ce n’est pas cependant que, suggestionnées habilement, les foules ne soient capables d’héroïsme, de dévouement et de vertus très hautes. Elles en sont même plus capables que l’individu isolé. Nous aurons bientôt occasion de revenir sur ce point en étudiant la moralité des foules.\par
Exagérée dans ses sentiments, la foule n’est impressionnée que par des sentiments excessifs. L’orateur qui veut la séduire doit abuser des affirmations violentes. Exagé­rer, affirmer, répéter, et ne jamais tenter de rien démontrer par un raisonnement, sont des procédés d’argumentation bien connus des orateurs des réunions populaires. La foule veut encore la même exagération dans les sentiments de ses héros. Leurs quali­tés et leurs vertus apparentes doivent toujours être amplifiées. On a très justement remarqué qu’au théâtre la foule exige du héros de la pièce des qualités de courage, de moralité, de vertu qui ne sont jamais pratiquées dans la vie.\par
On a parlé avec raison de l’optique spéciale du théâtre. Il en existe une, sans doute, mais ses règles n’ont le plus souvent rien à faire avec le bon sens et la logique. L’art de parler aux foules est d’ordre inférieur sans doute, mais exige des aptitudes toutes spéciales. Il est souvent impossible de s’expliquer à la lecture le succès de certaines pièces. Les directeurs des théâtres, quand ils les reçoivent, sont eux-mêmes le plus souvent très incertains de la réussite, parce que, pour juger, il faudrait qu’ils pussent se transformer en foule \footnote{C’est ce qui permet de comprendre pourquoi il arrive parfois que des pièces refusées par tous les directeurs de théâtre obtiennent de prodigieux succès lorsque, par hasard, elles sont jouées. On sait le succès de la pièce de M. Coppée, \emph{Pour la couronne}, refusée pendant dix ans par les directeurs des premiers théâtres, malgré le nom de son auteur. La \emph{marraine de Charley}, refusée par tous les théâtres et finalement montée aux frais d’un agent de change, a eu deux cents représentations en France et plus de mille en Angleterre. Sans l’explication donnée plus haut sur l’impossibilité où se trouvent les directeurs de théâtre de pouvoir se substituer mentalement à la foule, de telles aberrations de jugement de la part d’individus compétents et très intéressés à ne pas commettre d’aussi lourdes erreurs seraient inexplicables. C’est un sujet que je ne puis développer ici et qui mériterait d’être étudié longuement.}. Ici encore, si nous pouvions entrer dans les développements, nous montrerions l’influence prépondérante de la race. La pièce de théâtre qui enthousiasme la foule dans un pays n’a parfois aucun succès dans un autre ou n’a qu’un succès d’estime et de convention, parce qu’elle ne met pas en jeu les ressorts capables de soulever son nouveau public.\par
Je n’ai pas besoin d’ajouter que l’exagération des foules ne porte que sur les sentiments, et en aucune façon sur l’intelligence, J’ai déjà fait voir que, par le fait seul que l’individu est en foule, son niveau intellectuel baisse immédiatement et considéra­blement. C’est ce que M. Tarde a également constaté dans ses recherches sur les crimes des foules. Ce n’est donc que dans l’ordre du sentiment que les foules peuvent monter très haut ou descendre au contraire très bas.
\subsubsection[{§ 4. – Intolérance, autoritarisme et conservatisme des foules}]{§ 4. – Intolérance, autoritarisme et conservatisme des foules}
\noindent Les foules ne connaissant que les sentiments simples et extrêmes ; les opinions, idées et croyances qui leur sont suggérées sont acceptées ou rejetées par elles en bloc, et considérées comme des vérités absolues ou des erreurs non moins absolues. Il en est toujours ainsi des croyances déterminées par voie de suggestion, au lieu d’avoir été engendrées par voie de raisonnement. Chacun sait combien les croyances reli­gieuses sont intolérantes et quel empire despotique elles exercent sur les âmes.\par
N’ayant aucun doute sur ce qui est vérité ou erreur et ayant d’autre, part la notion claire de sa force, la foule est aussi autoritaire qu’intolérante. L’individu peut sup­porter la contradiction et la discussion, la foule ne les supportent jamais. Dans les réunions publiques, la plus légère contradiction de la part d’un orateur est immédiate­ment accueillie par des hurlements de fureur et de violentes invectives, bientôt suivis de voies de fait et d’expulsion pour peu que l’orateur insiste. Sans la présence inquié­tante des agents de l’autorité, le contradicteur serait même fréquemment massacré. L’autoritarisme et l’intolérance sont généraux chez toutes les catégories de foules, mais ils s’y présentent à des degrés forts divers ; et ici encore reparaît la notion fondamentale de la race, dominatrice de tous les sentiments et de toutes les pensées des hommes. C’est surtout chez les foules latines que l’autoritarisme et l’intolérance sont développés à un haut degré. Ils le sont au point d’avoir détruit entièrement ce sentiment de l’indépendance individuelle si puissant chez l’Anglo-Saxon. Les foules latines ne sont sensibles qu’à l’indépendance collective de la secte à laquelle elles appartiennent, et la caractéristique de cette indépendance est le besoin d’asservir immédiatement et violemment à leurs croyances tous les dissidents. Chez les peuples latins, les Jacobins de tous les âges, depuis ceux de l’inquisition, n’ont jamais pu s’élever à une autre conception de la liberté.\par
L’autoritarisme et l’intolérance sont pour les foules des sentiments très clairs, qu’elles conçoivent aisément et qu’elles acceptent aussi facilement qu’elles les prati­quent, dès qu’on les leur impose. Les foules respectent docilement la force et sont médiocrement impressionnées par la bonté, qui n’est guère pour elles qu’une forme de la faiblesse. Leurs sympathies n’ont jamais été aux maîtres débonnaires, mais aux tyrans qui les ont vigoureusement écrasées. C’est toujours à ces derniers qu’elles dres­sent les plus hautes statues. Si elles foulent volontiers aux pieds le despote renversé, c’est parce qu’ayant perdu sa force, il rentre dans cette catégorie des faibles qu’on méprise parce qu’on ne les craint pas. Le type du héros cher aux foules aura toujours la structure d’un César. Son panache les séduit, son autorité leur impose et son sabre leur fait peur.\par
Toujours prête à se soulever contre une autorité faible, la foule se courbe avec servilité devant une autorité forte. Si la force de l’autorité est intermittente, la foule, obéissant toujours à ses sentiments extrêmes, passe alternativement de l’anarchie à la servitude, et de la servitude à l’anarchie.\par
Ce serait d’ailleurs bien méconnaître la psychologie des foules que de croire, à la prédominance de leurs instincts révolutionnaires. Leurs violences seules nous illu­sionnent sur ce point. Leurs explosions de révolte et de destruction sont toujours très éphémères. Les foules sont trop régies par l’inconscient, et trop soumises par consé­quent à l’influence d’hérédités séculaires, pour n’être pas extrêmement conservatrices\par
Abandonnées à elles-mêmes, elles sont bientôt lasses de leurs désordres et se dirigent d’instinct vers la servitude. Ce furent les plus fiers et les plus intraitables des Jacobins qui acclamèrent le plus énergiquement Bonaparte, quand il supprima toutes les libertés et fit durement sentir sa main de fer.\par
Il est difficile de comprendre l’histoire, celle des révolutions populaires surtout, quand on ne se rend pas bien compte des instincts profondément conservateurs des foules. Elles veulent bien changer les noms de leurs institutions, et elles accomplis­sent parfois même de violentes révolutions pour obtenir ces changements ; mais le fond de ces institutions est trop l’expression des besoins héréditaires de la race pour qu’elles n’y reviennent pas toujours. Leur mobilité incessante ne porte que sur les choses tout à fait superficielles. En fait, elles ont des instincts conservateurs aussi irréductibles que ceux de tous les primitifs. Leur respect fétichiste pour les traditions est absolu, leur horreur inconsciente de toutes les nouveautés capables de changer leurs conditions réelles d’existence, est tout à fait profonde. Si les démocraties eussent possédé le pouvoir qu’elles ont aujourd’hui à l’époque où furent inventés les métiers mécaniques, la vapeur et les chemins de fer, la réalisation de ces inventions eût été impossible, ou ne l’eût été qu’au prix de révolutions et de massacres répétés. Il est heureux, pour les progrès de la civilisation, que la puissance des foules n’ait com­mencé à naître que lorsque les grandes découvertes de la science et de l’industrie étaient déjà accomplies.
\subsubsection[{§ 5. – Moralité des foules}]{§ 5. – Moralité des foules}
\noindent Si nous prenons le mot de moralité dans le sens de respect constant de certaines conventions sociales et de répression permanente des impulsions égoïstes, il est bien évident que les foules sont trop impulsives et trop mobiles pour être susceptibles de moralité. Mais si, dans le terme de moralité, nous faisons entrer l’apparition momenta­née de certaines qualités telles que l’abnégation, le dévouement, le désintéressement, le sacrifice de soi-même, le besoin d’équité, nous pouvons dire que les foules sont au contraire parfois susceptibles d’une moralité très haute.\par
Les rares psychologues qui ont étudié les foules ne les ont envisagées qu’au point de vue de leurs actes criminels ; et, voyant à quel point ces actes sont fréquents, ils les ont considérées comme ayant un niveau moral très bas.\par
Sans doute il en est souvent ainsi : mais pourquoi ? Simplement, parce que les instincts de férocité destructive sont des résidus des âges primitifs qui dorment au fond de chacun de nous. Dans la vie de l’individu isolé, il lui serait dangereux de les satisfaire, alors que son absorption dans une foule irresponsable, et où par conséquent l’impunité est assurée, lui donne toute liberté pour les suivre. Ne pouvant exercer habituellement ces instincts destructifs sur nos semblables, nous nous bornons à les exercer sur les animaux. C’est d’une même source que dérivent la passion si générale pour la chasse et les actes de férocité des foules. La foule qui écharpe lentement une victime sans défense fait preuve d’une férocité très lâche ; mais, pour le philosophe, cette férocité est bien proche parente de celle des chasseurs qui se réunissent par dou­zaines pour avoir le plaisir d’assister à la poursuite et à l’éventrement d’un malheureux cerf par leurs chiens.\par
Si la foule est capable de meurtre, d’incendie et de toutes sortes de crimes, elle est également capable d’actes de dévouement, de sacrifice et de désintéressement très élevés, beaucoup plus élevés même que ceux dont est capable l’individu isolé. C’est surtout sur l’individu en foule qu’on agit, et souvent jusqu’à obtenir le sacrifice de la vie, en invoquant des sentiments de gloire, d’honneur, de religion et de patrie. L’histoire fourmille d’exemples analogues à ceux des croisades et des volontaires de 93. Seules les collectivités sont capables de grands désintéressements et de grands dévouements.\par
Que de foules se sont fait héroïquement massacrer pour des croyances, des idées et des mots qu’elles comprenaient à peine. Les foules qui font des grèves les font bien plus pour obéir à un mot d’ordre que pour obtenir une augmentation du maigre salaire dont elles se contentent. L’intérêt personnel est bien rarement un mobile puissant chez les foules, alors qu’il est le mobile à peu près exclusif de l’individu isolé. Ce n’est certes pas l’intérêt qui a guidé les foules dans tant de guerres, incompréhensibles le plus souvent pour leur intelligence, et où elles se sont laissé aussi facilement massa­crer que les alouettes hypnotisées par le miroir que manœuvre le chasseur.\par
Même pour les parfaits gredins, il arrive fort souvent que le fait seul d’être réunis en foule leur donne momentanément des principes de moralité très stricts. Taine fait remarquer que les massacreurs de septembre venaient déposer sur la table des comités les portefeuilles et les bijoux qu’ils trouvaient sur leurs victimes, et qu’ils eussent pu aisément dérober. La foule hurlante, grouillante et misérable qui envahit les Tuileries pendant la Révolution de 1848, ne s’empara d’aucun des objets qui l’éblouirent et dont un seul eût représenté du pain pour bien des jours.\par
Cette moralisation de l’individu par la foule n’est certes pas une règle constante, mais c’est une règle qui s’observe fréquemment. Elle s’observe même dans des cir­constances beaucoup moins graves que celles que je viens de citer. J’ai déjà dit qu’au théâtre la foule veut chez le héros de la pièce des vertus exagérées, et il est d’une ob­servation banale qu’une assistance, même composée d’éléments inférieurs, se montre généralement très prude. Le viveur professionnel, le souteneur, le voyou gouailleur murmurent souvent devant une scène un peu risquée ou un propos léger, fort anodins pourtant auprès de leurs conversations habituelles.\par
Donc, si les foules se livrent souvent à de bas instincts, elles donnent aussi parfois l’exemple d’actes de moralité élevés. Si le désintéressement, la résignation, le dévoue­ment absolu à un idéal chimérique ou réel sont des vertus morales, on peut dire que les foules possèdent souvent ces vertus-là à un degré que les plus sages des philoso­phes ont rarement atteint. Elles les pratiquent sans doute avec inconscience, mais qu’importe. Ne nous plaignons pas trop que les foules soient guidées surtout par l’in­conscient., et ne raisonnent guère. Si elles avaient raisonné quelquefois et consulté leurs intérêts immédiats, aucune civilisation ne se fût développée peut-être à la sur­face de notre planète, et l’humanité n’aurait pas eu d’histoire.\par

\astermono

\subsection[{Chapitre 3. Idées, raisonnements et imagination des foules}]{Chapitre 3. Idées, raisonnements et imagination des foules}

\begin{argument}\noindent 1\emph{. Les idées des foules. –} Les idées fondamentales et les idées accessoires. – Comment peuvent subsister simultanément des idées contradictoires. – Transformations que doivent subir les idées supérieures pour être accessibles aux foules. – Le rôle social des idées est indépendant de la part de vérité qu’elles peuvent contenir. – § 2\emph{. Les raisonnements des foules. –} Les foules ne sont pas influençables par des raisonnements. – Les raisonnements des foules sont toujours d’ordre très inférieur. – Les idées qu’elles associent n’ont que des apparences d’analogie ou de succession. – § 3\emph{. L’imagination des foules. –} puissance de l’imagination des foules. – Elles pensent par images, et ces images se succèdent sans aucun lien. – Les foules sont frappées surtout par le côté merveilleux des choses. – Le merveil­leux et le légendaire sont les vrais supports des civilisa­tions. – L’imagination populaire a toujours été la base de la puissance des hommes d’État. – Comment se présentent les faits capables de frapper l’imagination des foules.
\end{argument}

\subsubsection[{§ 1. – Les idées des foules}]{§ 1. – Les idées des foules}
\noindent Étudiant dans notre précédent ouvrage le rôle des idées dans l’évolution des peu­ples, nous avons montré que chaque civilisation dérive d’un petit membre d’idées fondamentales fort rarement renouvelées. Nous avons exposé comment ces idées s’établissent dans l’âme des foules ; avec quelle difficulté elles y pénètrent, et la puissance qu’elles possèdent quand elles y ont pénétré. Nous avons vu enfin comment les grandes perturbations historiques dérivent le plus souvent des changements de ces idées fondamentales.\par
Ayant suffisamment traité ce sujet, je n’y reviendrai pas maintenant et me bornerai à dire quelques mots des idées qui sont accessibles aux foules et sous quelles formes celles-ci les conçoivent.\par
On peut les diviser en deux classes. Dans l’une nous placerons les idées acciden­telles et passagères créées sous des influences du moment : l’engouement pour un individu ou une doctrine par exemple. Dans l’autre, les idées fondamentales auxquel­les le milieu, l’hérédité, l’opinion donnent une stabilité très grande : telles les croyan­ces religieuses jadis, les idées démocratiques et sociales aujourd’hui.\par
Les idées fondamentales pourraient être figurées par la masse des eaux d’un fleuve déroulant lentement son cours ; les idées passagères par les petites vagues, toujours changeantes, qui agitent sa surface, et qui, bien que sans importance réelle, sont plus visibles que la marche du fleuve lui-même.\par
De nos jours, les grandes idées fondamentales dont ont vécu nos pères sont de plus en plus chancelantes. Elles ont perdu toute solidité, et, du même coup, les institu­tions qui reposaient sur elles se sont trouvées profondément ébranlées. Il se forme journellement beaucoup de ces petites idées transitoires dont je parlais à l’instant ; mais très peu d’entre elles paraissent visible­ment grandir et devoir acquérir une influence prépondérante.\par
Quelles que soient les idées suggérées aux foules, elles ne peuvent devenir domi­nantes qu’à la condition de revêtir une forme très absolue, et très simple. Elles se présentent alors sous l’aspect d’images, et ne sont accessibles aux masses que sous cette forme. Ces idées-images ne sont rattachées entre elles par aucun lien logique d’analogie ou de succession, et peuvent se substituer l’une à l’autre comme les verres de la lanterne magique que l’opérateur retire de la boîte où ils étaient superposés. Et c’est pourquoi on peut voir dans les foules se maintenir côte à côte les idées les plus contradictoires. Suivant les hasards du moment, la foule sera placée sous l’influence de l’une des idées diverses emmagasinées dans son entendement, et pourra par conséquent commettre les actes les plus dissemblables. Son absence complète d’esprit critique ne lui permet pas d’en percevoir les contradictions.\par
Ce n’est pas là un phénomène spécial aux foules ; on l’observe chez beaucoup d’individus isolés, non seulement parmi les êtres primitifs, mais chez tous ceux qui par un côté quelconque de leur esprit, – les sectateurs d’une foi religieuse intense par exemple, – se rapprochent des primitifs. Je l’ai observé à un degré curieux chez des Hindous lettrés, élevés dans nos universités européennes, et ayant obtenu tous les diplômes. Sur leur fonds immuable d’idées religieuses ou sociales héréditaires s’était superposé, sans nullement les altérer, un fonds d’idées occidentales sans parenté avec les premières. Suivant les hasards du moment, les unes ou les autres apparaissaient avec leur cortège spécial d’actes ou de discours, et le même individu présentait ainsi les contradictions les plus flagrantes. Contradictions, d’ailleurs, plus apparentes que réelles, car les idées héréditaires seules sont assez puissantes chez l’individu isolé pour devenir des mobiles de conduite. C’est seulement lorsque, par des croisements, l’homme se trouve entre les impulsions d’hérédités différentes, que les actes peuvent être réellement d’un moment à l’autre tout à fait contradictoires. Il serait inutile d’insister ici sur ces phénomènes, bien que leur importance psychologique soit capitale. Je considère qu’il faut au moins dix ans de voyages et d’observations pour arriver à les comprendre.\par
Les idées n’étant accessibles aux foules qu’après avoir revêtu une forme très simple, doivent, pour devenir populaires, subir souvent les plus complètes transfor­mations. C’est surtout quand il s’agit d’idées philosophiques ou scientifiques un peu élevées, qu’on peut constater la profondeur des modifications qui leur sont nécessaires pour descendre de couche en couche jusqu’au niveau des foules. Ces modifications dépendent des catégories des foules ou de la race à laquelle ces foules appartiennent ; mais elles sont toujours amoindrissantes et simplifiantes. Et c’est pour­quoi, au point de, vue social, il n’y a guère, en réalité, de hiérarchie des idées, c’est-à-dire d’idées plus ou moins élevées. Par le fait seul qu’une idée arrive aux foules et peut agir, si grande ou si vraie qu’elle ait été à son origine, elle est dépouillée de presque tout ce qui faisait son élévation et sa grandeur.\par
D’ailleurs, au point de vue social, la valeur hiérarchique d’une idée est sans impor­tance. Ce qu’il faut considérer, ce sont les effets qu’elle produit. Les idées chrétiennes du moyen âge, les idées démocratiques du siècle dernier, les idées sociales d’aujourd’hui, ne sont pas certes très élevées. On ne peut philosophiquement les consi­dérer que comme d’assez pauvres erreurs ; et cependant leur rôle a été et sera immense, et elles compteront longtemps parmi les plus essentiels facteurs de la conduite des États.\par
Alors même que l’idée a subi les transformations qui la rendent accessible aux foules, elle n’agit que lorsque, par des procédés divers qui seront étudiés ailleurs, elle a pénétré dans l’inconscient et est devenue un sentiment, ce qui est toujours fort long.\par
Il ne faut pas croire, en effet, que c’est simplement parce que la justesse d’une idée est démontrée qu’elle peut produire ses effets, même chez les esprits cultivés. On s’en rend vite compte en voyant combien la démonstration la plus claire a peu d’influence sur la majorité des hommes. L’évidence, si elle est éclatante, pourra être reconnue par un auditeur instruit ; mais ce nouveau converti sera vite ramené par son inconscient à ses conceptions primitives. Revoyez-le au bout de quel­ques jours, et il vous servira de nouveau ses anciens arguments, exactement dans les mêmes termes. Il est, en effet, sous l’influence d’idées antérieures devenues des sentiments ; et ce sont celles-là seules qui agissent sur les mobiles profonds de nos actes et de nos discours. Il ne saurait en être autrement pour les foules.\par
Mais lorsque, par des procédés divers, une idée a fini par pénétrer dans l’âme des foules, elle possède une puissance irrésistible et déroule toute une série d’effets qu’il faut subir. Les idées philosophiques qui aboutirent à la Révolution française mirent près d’un siècle à s’implanter dans l’âme des foules. On sait leur irrésistible force quand elles y furent établies. L’élan d’un peuple entier vers la conquête de l’égalité sociale, vers la réalisation de droits abstraits et de libertés idéales, fit chanceler tous les trônes et bouleversa profondément le monde occidental. Pendant vingt ans les peuples se précipitèrent les uns sur les autres, et l’Europe connut des hécatombes qui eussent effrayé Gengiskhan et Tamerlan. Jamais le monde ne vit à un tel degré ce que peut produire le déchaînement d’une idée.\par
Il leur faut bien longtemps, aux idées, pour s’établir dans l’âme des foules, mais il ne leur faut pas moins de temps pour en sortir. Aussi les foules sont-elles toujours, au point de vue des idées, en retard de plusieurs générations sur les savants et les philo­sophes. Tous les hommes d’État savent bien aujourd’hui ce que contiennent d’erroné les idées fondamentales que je citais à l’instant, mais comme leur influence est très puissante encore, ils sont obligés de gouverner suivant des principes à la vérité desquels ils ne croient plus.
\subsubsection[{§ 2. – Les raisonnements des foules}]{§ 2. – Les raisonnements des foules}
\noindent On ne peut dire d’une façon tout à fait absolue que les foules ne raisonnent pas et ne sont pas influençables par des raisonnements. Mais les arguments quelles em­ploient et ceux qui peuvent agir sur elles sont, au point de vue logique, d’un ordre tellement inférieur que c’est seulement par voie d’analogie qu’on peut les qualifier de raisonnements.\par
Les raisonnements inférieurs des foules sont, comme les raisonnements élevés, basés sur des associations ; mais les idées associées par les foules n’ont entre elles que des liens apparents d’analogie ou de succession. Elles s’enchaînent comme celles de l’Esquimau qui, sa­chant par expérience que la glace, corps transparent, fond dans la bouche, en conclut que le verre, corps également transparent, doit fondre aussi dans la bouche ; ou celles du sauvage qui se figure qu’en mangeant le cœur d’un ennemi courageux, il acquiert sa bravoure ; ou encore de l’ouvrier qui, ayant été exploité par un patron, en conclut immédiatement que tous les patrons sont des exploiteurs.\par
Association de choses dissemblables, n’ayant entre elles que des rapports appa­rents, et généralisation immédiate de cas particuliers, telles sont les caractéristiques des raisonnements des foules. Ce sont des raisonnements de cet ordre que leur présentent toujours ceux qui savent les manier ; ce sont les seuls qui peuvent les influencer. Une chaîne de raisonnements logiques est totalement incompréhensible, aux foules, et c’est pourquoi il est permis de dire qu’elles ne raisonnent pas ou raison­nent faux, et ne sont pas influençables par un raisonnement. On s’étonne parfois, à la lecture, de la faiblesse de certains discours qui ont eu pourtant une influence énorme, sur les foules qui les écoutaient ; mais on oublie qu’ils furent faits pour entraîner des collectivités, et non pour être lus par des philosophes. L’orateur, en communication intime avec la foule, sait évoquer les images qui la séduisent. S’il réussit, son but a été atteint ; et vingt volumes de harangues – toujours fabriquées après coup – ne valent pas les quelques phrases arrivées jusqu’aux cerveaux qu’il fallait convaincre.\par
Il serait superflu d’ajouter que l’impuissance des foules à raisonner juste les empê­che d’avoir aucune trace d’esprit critique, c’est-à-dire, d’être aptes à discerner la vérité de l’erreur, à porter un jugement précis sur quoi que ce soit. Les jugements que les foules acceptent ne sont que des jugements imposés et jamais des jugements discutés. A ce point de vue, nombreux sont les hommes qui ne s’élèvent pas au-dessus de la foule. La facilité avec laquelle certaines opinions deviennent générales tient surtout à l’impossibilité où sont la plupart des hommes de se former une opinion particulière basée sur leurs propres raisonnements.
\subsubsection[{§ 3. – L’imagination des foules}]{§ 3. – L’imagination des foules}
\noindent De même que pour les êtres chez qui le raisonnement n’intervient pas, l’imagina­tion représentative des foules est très puissante, très active, et susceptible d’être vive­ment impressionnée. Les images évoquées dans leur esprit par un personnage, un événement, un accident, ont presque la vivacité des choses réelles. Les foules sont un peu dans le cas du dormeur dont la raison, momentanément suspendue, laisse surgir dans l’esprit des images d’une intensité extrême, mais qui se dissiperaient vite si elles pouvaient être soumises à la réflexion. Les foules, n’étant capables ni de réflexion ni de raisonnement, ne connaissent pas l’invraisemblable : or, ce sont les choses les plus invraisemblables qui sont généralement les plus frappantes.\par
Et c’est pourquoi ce sont toujours les côtés merveilleux et légendaires des événe­ments qui frappent le plus les foules. Quand on analyse une civilisation, on voit que c’est, en réalité, le merveilleux et le légendaire qui en sont les vrais supports. Dans l’histoire, l’apparence a toujours joué un rôle beaucoup plus important que la réalité. L’irréel y prédomine toujours sur le réel.\par
Les foules, ne pouvant penser que par images, ne se laissent impressionner que par des images.\par
Seules les images les terrifient ou les séduisent, et deviennent des mobiles d’action.\par
Aussi, les représentations théâtrales, qui donnent l’image sous sa forme la plus nettement visible, ont-elles toujours une énorme influence sur les foules. Du pain et des spectacles constituaient jadis pour la plèbe romaine l’idéal du bonheur, et elle ne demandait rien de plus. Pendant la succession des âges cet idéal a peu varié. Rien ne frappe davantage l’imagination des foules de toutes catégories que les représentations théâtrales. Toute la salle éprouve en même temps les mêmes émotions, et si ces émotions ne se transforment pas aussitôt en actes, c’est que le spectateur le plus inconscient ne peut ignorer qu’il est victime d’illusions, et qu’il a ri ou pleuré à d’ima­ginaires aventures. Parfois cependant les sentiments suggérés par les images sont si forts qu’ils tendent, comme les suggestions habituelles, à se transformer en actes. On a raconté bien des fois l’histoire de ce théâtre populaire qui, ne jouant que des drames sombres, était obligé de faire protéger à la sortie l’acteur qui représentait le traître, pour le soustraire aux violences des spectateurs indignés des crimes, imaginaires pourtant, que ce traître avait commis. C’est là, je crois, un des indices les plus remar­quables de l’état mental des foules, et surtout de la facilité avec laquelle on les suggestionne. L’irréel a presque autant d’action sur elles que le réel. Elles ont une tendance évidente à ne pas les différencier.\par
C’est sur l’imagination populaire qu’est fondée la puissance des conquérants et la force des États. C’est surtout en agissant sur elle qu’on entraîne les foules. Tous les grands faits historiques, la création du Bouddhisme, du Christianisme, de l’Islamisme, la Réforme, la Révolution, et, de nos jours, l’invasion menaçante du Socialisme, sont les conséquences directes ou lointaines d’impressions fortes produites sur l’imagina­tion des foules.\par
Aussi, tous les grands hommes d’État de tous les âges et de tous les pays, y compris les plus absolus despotes, ont-ils considéré l’imagination populaire comme la base de leur puissance, et jamais ils n’ont essayé de gouverner contre elle. “C’est en me faisant catholique, disait Napoléon au Conseil d’État, que j’ai fini la guerre de Vendée ; en me faisant musulman que je me suis établi en Égypte, en me faisant ultramontain que j’ai gagné les prêtres en Italie. Si je gouvernais un peuple de Juifs, je rétablirais le temple de Salomon.” Jamais, peut-être, depuis Alexandre et César, aucun grand homme n’a mieux su comment l’imagination des foules doit être impres­sionnée. Sa préoccupation constante fut de la frapper. Il y songeait dans ses victoires, dans ses harangues, dans ses discours, dans tous ses actes. À son lit de mort il y songeait encore.\par
Comment impressionne-t-on l’imagination des foules ? Nous le verrons bientôt. Bornons-nous, pour le moment, à dire que ce n’est jamais en essayant d’agir sur l’intelligence et la raison, c’est-à-dire par voie de démonstration. Ce ne fut pas au moyen d’une rhétorique savante qu’Antoine réussit à ameuter le peuple contre les meurtriers de César. Ce fut en lui lisant son testament et en lui montrant son cadavre.\par
Tout ce qui frappe l’imagination des foules se pré­sente sous forme d’une image saisissante et bien nette, dégagée de toute interprétation accessoire, ou n’ayant d’autre accompagnement que quelques faits merveilleux ou mystérieux : une grande victoire, un grand miracle, un grand crime, un grand espoir. Il faut présenter les choses en bloc, et ne jamais en indiquer la genèse. Cent petits crimes ou cent petits accidents ne frapperont pas du tout l’imagination des foules ; tandis qu’un seul grand crime, un seul grand accident les frapperont profondément, même avec des résultats infiniment moins meur­triers que les cent petits accidents réunis. L’épidémie d’influenza qui, il y a peu d’années, fit périr, à Paris seulement, 5.000 personnes en quelques semaines, frappa très peu l’imagination populaire. Cette véritable hécatombe ne se traduisait pas, en effet, par quelque image visible, mais seulement par les indications hebdoma­daires de la statistique. Un accident qui, au lieu de ces 5.000 personnes, en eût seulement fait périr 500, mais le même jour, sur une place publique, par un accident bien visible, la chute de la tour Eiffel, par exemple, eût au contraire produit sur l’imagination une impression immense. La perte probable d’un transatlantique qu’on supposait, faute de nouvelles, coulé en pleine mer, frappa profondément pendant huit jours l’imagination des foules. Or les statistiques officielles montrent que dans la même année un millier de grands bâtiments se sont perdus. Mais, de ces pertes successives, bien autrement importantes comme destruction de vies et de marchan­dises qu’eût pu l’être celle du transatlantique eu question, les foules ne se sont pas préoccupées un seul instant.\par
Ce ne sont donc pas les faits en eux-mêmes qui frappent l’imagination populaire, mais bien la façon dont ils sont répartis et présentés. Il faut que par leur condensation, si je puis m’exprimer ainsi, ils produisent une image saisissante qui remplisse et obsède l’esprit. Qui connaît l’art d’impressionner l’imagination des foules connaît aussi l’art de les gouverner.
\subsection[{Chapitre 4. Formes religieuses que revêtent toutes les convictions des foules}]{Chapitre 4. Formes religieuses que revêtent toutes les convictions des foules}

\begin{argument}\noindent Ce qui constitue le sentiment religieux.. – Il est indépendant de l’adoration d’une divinité. – Ses caractéristiques. – Puissance des convictions revêtant la forme religieuse. – Exemples divers. – Les dieux populaires n’ont jamais disparu. – Formes nouvelles sous lesquelles ils renaissent. – Formes religieuses de l’athéisme. – Importance de ces notions au point de vue historique. – La Réforme, la Saint-Barthélemy, la Terreur et tous les événements ana­logues, sont la conséquence des sentiments religieux des foules, et non de la volonté d’individus isolés.
\end{argument}

\noindent Nous avons montré que les foules ne raisonnent pas, qu’elles admettent ou rejet­tent les idées en bloc ; ne supportent ni discussion, ni contradiction, et que les sugges­tions agissant sur elles envahissent entièrement le champ de leur entendement et tendent aussitôt à se transformer en actes. Nous avons montré que les foules conve­nablement suggestionnées sont prêtes à se sacrifier pour l’idéal qui leur a été suggéré. Nous avons vu aussi qu’elles ne connaissent que les sentiments violents et extrêmes, que, chez elles, la sympathie devient vite adoration, et qu’à peine née l’antipathie se transforme en haine. Ces indications générales permettent déjà de pressentir la nature de leurs convictions.\par
Quand on examine de près les convictions des foules, aussi bien aux époques de foi que dans les grands soulèvements politiques, tels que ceux du dernier siècle, on constate, que ces convictions revêtent toujours une forme spéciale, que je ne puis pas mieux déterminer qu’en lui donnant le nom de sentiment religieux.\par
Ce sentiment a des caractéristiques très simples : adoration d’un être supposé supérieur, crainte de la puissance magique qu’on lui suppose, soumission aveugle à ses commandements, impossibilité de discuter ses dogmes, désir de les répandre, ten­dance à considérer comme ennemis tous ceux qui ne les admettent pas. Qu’un tel sentiment s’applique à un Dieu invisible, à une idole de pierre ou de bois, à un héros ou à une idée politique, du moment qu’il présente les caractéristiques précédentes il reste toujours d’essence religieuse. Le surnaturel et le miraculeux s’y retrouvent au même degré. Inconsciemment les foules revêtent d’une puissance mystérieuse la formule politique ou le chef victorieux qui pour le moment les fanatise.\par
On n’est pas religieux seulement quand on adore une divinité, mais quand on met toutes les ressources de l’esprit, toutes les soumissions de la volonté, toutes les ardeurs du fanatisme au service d’une cause ou d’un être qui devient le but et le guide des pensées et des actions.\par
L’intolérance et le fanatisme constituent l’accompagnement nécessaire d’un senti­ment religieux. Ils sont inévitables chez ceux qui croient posséder le secret du bonheur terrestre ou éternel. Ces deux traits se retrouvent chez tous les hommes en groupe lorsqu’une conviction quelconque les soulève. Les Jacobins de la Terreur étaient aussi foncièrement religieux que les catholiques de l’Inquisition, et leur cruelle ardeur dérivait de la môme source.\par
Les convictions des foules revêtent ces caractères de soumission aveugle, d’into­lérance farouche, de besoin de propagande violente qui sont inhérents au sentiment religieux ; et c’est pourquoi on peut dire que toutes leurs croyances ont une forme religieuse. Le héros que la foule acclame est véritablement un dieu pour elle. Napo­léon le fut pendant quinze ans, et jamais divinité n’eut de plus parfaits adorateurs. Aucune n’envoya plus facilement les hommes à la mort. Les dieux du paganisme et du christianisme n’exercèrent jamais un empire plus absolu sur les âmes qu’ils avaient conquises.\par
Tous les fondateurs de croyances religieuses ou politiques ne les ont fondées que parce qu’ils ont su imposer aux foules ces sentiments de fanatisme qui font que l’homme trouve son bonheur dans l’adoration et l’obéissance et est prêt à donner sa vie pour son idole. Il en a été ainsi à toutes les époques. Dans son beau livre sur la Gaule romaine, Fustel de Coulanges fait juste­ment remarquer que ce ne fut nullement par la force que se maintint l’Empire romain, mais par l’admiration religieuse qu’il inspirait. “Il serait sans exemple dans l’histoire du monde, dit-il avec raison, qu’un régime détesté des populations ait duré cinq siècles… On ne s’expliquerait pas que trente légions de l’Empire eussent pu contraindre cent millions d’hommes à obéir.” S’ils obéissaient, c’est que l’empereur, qui personnifiait la grandeur romaine, était adoré comme une divinité, du consentement unanime. Dans la moindre bourgade de l’Empire, l’empereur avait ses autels. “On vit surgir en ce temps-là dans les âmes, d’un bout de l’Empire à l’autre, une religion nouvelle qui eut pour divinités les empe­reurs eux-mêmes. Quelques années avant l’ère chrétienne, la Gaule entière, repré­sentée par soixante cités, éleva en commun un temple, près de la ville de Lyon, à Auguste… Ses prêtres, élus par la réunion des cités gauloises, étaient les premiers personnages de leur pays… Il est impossible d’attribuer tout cela à la crainte et à la servilité. Des peuples entiers ne sont pas serviles, et ne le sont pas pendant trois siècles. Ce n’étaient pas les courtisans qui adoraient le prince, c’était Rome. Ce n’était pas Rome seulement, c’était la Gaule, c’était l’Es­pagne, c’était la Grèce et l’Asie.”\par
Aujourd’hui la plupart des grands conquérants d’âmes n’ont plus d’autels, mais ils ont des statues ou des images, et le culte qu’on leur rend n’est pas notablement dif­férent de celui qu’on leur rendait jadis. On n’arrive à comprendre un peu la philoso­phie de l’histoire que quand on est bien pénétré de ce point fondamental de la psychologie des foules. Il faut être dieu pour elles ou ne rien être.\par
Et il ne faudrait pas croire que ce sont là des superstitions d’un autre âge que la raison a définitivement chassées. Dans sa lutte éternelle contre la raison, le sentiment n’a jamais été vaincu. Les foules ne veulent plus entendre les mots de divinité et de religion, au nom desquelles elles ont été pendant si longtemps asservies mais elles n’ont jamais autant possédé de fétiches que depuis cent ans, et jamais les vieilles divinités ne firent s’élever autant de statues et d’autels. Ceux qui ont étudié dans ces dernières années le mouvement populaire connu sous le nom de boulangisme ont pu voir avec quelle facilité les instincts religieux des foules sont prêts à renaître. Il n’était pas d’auberge de village, qui ne possédât l’image du héros. On lui attribuait la puis­sance de remédier à toutes les injustices, à tous les maux ; et des milliers d’hommes auraient donné leur vie pour lui. Quelle place n’eût-il pas pris dans l’histoire si son caractère eût été de force à soutenir tant soit peu sa légende !\par
Aussi est-ce une bien inutile banalité de répéter qu’il faut une religion aux foules, puisque toutes les croyances politiques, divines et sociales ne s’établissent chez elles qu’à la condition de revêtir toujours la forme religieuse, qui les met à l’abri de la dis­cussion. L’athéisme, s’il était possible de le faire accepter aux foules, aurait toute l’ardeur intolérante d’un sentiment religieux, et, dans ses formes extérieures, devien­drait bientôt un culte. L’évolution de la petite secte positiviste nous en fournit une preuve curieuse. Il lui est arrivé bien vite ce qui arriva à ce nihiliste, dont le profond Dostoïewsky nous rapporte l’histoire. Éclairé un jour par les lumières de la raison, il brisa les images des divinités et des saints qui ornaient l’autel d’une chapelle, éteignit les cierges, et, sans perdre un instant, remplaça les images détruites par les ouvrages de quelques philosophes athées, tels que Büchner et Moleschott, puis ralluma pieusement les cierges. L’objet de ses croyances religieuses s’était transformé, mais ses sentiments religieux, peut-on dire vraiment qu’ils avaient changé ?\par
On ne comprend bien, je le répète encore, certains événements historiques – et ce sont précisément les plus importants – que lorsqu’on s’est rendu compte de cette forme religieuse que finissent toujours par prendre les convictions des foules. Il y a des phénomènes sociaux qu’il faut étudier en psychologue beaucoup plus qu’en natu­raliste. Notre grand historien Taine n’a étudié la Révolution qu’en naturaliste, et c’est pourquoi la genèse réelle des événements lui a bien souvent échappé. Il a parfai­tement observé les faits, mais, faute d’avoir étudié la psychologie des foules, il n’a pas toujours su remonter aux causes. Les faits l’ayant épouvanté par leur côté sanguinaire, anarchique et féroce, il n’a guère vu dans les héros de la grande épopée qu’une horde de sauvages épileptiques se livrant sans entraves à leurs instincts. Les violences de la Révolution, ses massacres, son besoin de propagande, ses déclarations de guerre à tous les rois, ne s’expliquent bien que si l’on réfléchit qu’elle fut simplement l’établis­sement d’une nouvelle croyance religieuse dans l’âme des foules. La Réforme, la Saint-Barthélemy, les guerres de Religion, l’Inquisition, la Terreur, sont des phéno­mènes d’ordre identique, accomplis par des foules animées de ces sentiments reli­gieux qui conduisent nécessairement à extirper sans pitié, par le fer et le feu, tout ce qui s’oppose à l’établissement de la nouvelle croyance. Les méthodes de l’inquisi­tion sont celles de tous les vrais convaincus. Ils ne seraient pas des convaincus s’ils en employaient d’autres.\par
Les bouleversements analogues à ceux que je viens de citer ne sont possibles que lorsque l’âme des foules les fait surgir. Les plus absolus despotes ne pourraient pas les déchaîner. Quand les historiens nous racontent que la Saint-Barthélemy fut l’œuvre d’un roi, ils montrent qu’ils ignorent la psychologie des foules tout autant que celle des rois. De semblables manifestations ne peuvent sortir que de l’âme des foules. Le pouvoir le plus absolu du monarque le plus despotique ne va guère plus loin que d’en hâter ou d’en retarder un peu, le moment. Ce ne sont pas les rois qui firent ni la Saint-Barthélemy, ni les guerres de religion, pas plus que ce ne fut Robespierre, Danton ou Saint-Just qui firent la Terreur. Derrière de tels événements on retrouve toujours l’âme des foules, et jamais la puissance des rois.
\section[{Livre II. Les opinions et les croyances des foules}]{Livre II. Les opinions et les croyances des foules}\renewcommand{\leftmark}{Livre II. Les opinions et les croyances des foules}

\subsection[{Chapitre 1. Facteur lointains des croyances et opinions des foules}]{Chapitre 1. Facteur lointains des croyances et opinions des foules}

\begin{argument}\noindent Facteurs préparatoires des croyances des foules. – L’éclosion des croyances des foules est la conséquence d’une élaboration antérieure. – Étude des divers facteurs de ces croyances. – § 1. La race. – Influence prédominante qu’elle exerce. – Elle représente les suggestions des ancêtres. – § \emph{2. Les Traditions. –} Elles sont la synthèse de l’âme de la race. – Importance sociale des traditions. – En quoi, après avoir été nécessaires, elles deviennent nuisibles. – Les foules sont les conservateurs les plus tenaces des idées traditionnelles. – § 3. \emph{Le temps. –} Il prépare successivement l’établissement des croyances, puis leur destruction. – C’est grâce à lui que l’ordre peut sortir du chaos. – § \emph{4. Les institutions politiques et sociales. –} Idée erronée de leur rôle. – Leur influence est extrêmement faible. Elles sont des effets, et non des causes. – Les peuples ne sauraient choisir les institutions qui leur semblent les meilleures. – Les institutions sont des étiquettes qui, sous un même titre, abritent les choses les plus dissemblables. – Comment les constitutions peuvent se créer. – Nécessité pour certains peuples de certaines institutions théoriquement mauvaises, telles que la centralisation. – § 5. – \emph{L’instruction et l’éducation.} – Erreur des idées actuelles sur l’influence de l’instruction chez les foules. – Indications statistiques. – Rôle démoralisateur de l’éducation latine. – Influence que l’instruction pourrait exercer. – Exemples fournis par divers peuples.
\end{argument}

\noindent Nous venons d’étudier la constitution mentale des fou­les. Nous connaissons leurs façons de sentir, de penser, de raisonner. Examinons maintenant comment naissent et s’établissent leurs opinions et leurs croyances.\par
Les facteurs qui déterminent ces opinions et ces croyances sont de deux ordres – les facteurs lointains et les facteurs immédiats.\par
Les facteurs lointains rendent les foules capables d’adopter certaines convictions et incapables de se laisser pénétrer par d’autres. Ils préparent le terrain où l’on voit germer tout à coup des idées nouvelles, dont la force et les résultats étonnent, mais qui n’ont de spontané que l’apparence. L’explosion et la mise en œuvre de certaines idées chez les foules présentent quelquefois une soudaineté foudroyante. Ce n’est là qu’un effet superficiel, derrière lequel on doit chercher tout un long travail antérieur.\par
Les facteurs immédiats sont ceux qui, se superposant à ce long travail, sans lequel ils n’auraient pas d’effet, provoquent la persuasion active chez les foules, c’est-à-dire font prendre forme à l’idée et la déchaînent avec toutes ses conséquences. Par ces facteurs immédiats surgissent les résolutions qui soulèvent brusquement les collecti­vités ; par eux éclate une émeute ou se décide une grève par eux des majorités énor­mes portent un homme au pouvoir ou renversent un gouvernement.\par
Dans tous les grands événements de l’histoire, nous constatons l’action successive de ces deux ordres de facteurs. La Révolution française, pour ne prendre qu’un des plus frappants exemples, eut parmi ses facteurs lointains les écrits des philosophes, les exactions de la noblesse, les progrès de la pensée scientifique. L’âme des foules, ainsi préparée, fut soulevée ensuite aisément par des facteurs immédiats, tels que les discours des orateurs, et les résistances de la cour à propos de réformes insignifiantes.\par
Parmi les facteurs lointains, il y en a de généraux, qu’on retrouve au fond de toutes les croyances et opinions des foules ; ce sont : la race, les traditions, le temps, les institutions, l’éducation.\par
Nous allons étudier le rôle de ces différents facteurs.\par
\subsubsection[{§ 1. – La race}]{§ 1. – La race}
\noindent Ce facteur, la race, doit être mis au premier rang, Car à lui seul il dépasse de beaucoup en importance tous les autres. Nous l’avons suffisamment étudié dans un autre ouvrage pour qu’il soit inutile d’y revenir encore. Nous avons fait voir, dans un précédent volume, ce qu’est une race historique, et comment, lorsque ses caractères sont formés, elle possède de par les lois de l’hérédité une puissance telle, que ses croyances, ses institutions, ses arts, en un mot tous les éléments de sa civilisation, ne sont que l’expression extérieure de son âme. Nous avons montré que la puissance de la race est telle qu’aucun élément ne peut passer d’un peuple à un autre sans subir les transformations les plus profondes \footnote{Cette proposition étant bien nouvelle encore, et l’histoire étant tout à fait inintelligible sans elle, j’ai consacré plusieurs chapitres de mon ouvrage \emph{(Les lois psychologiques de l’évolution des peuples)} à sa démonstration. Le lecteur y verra que, malgré de trompeuses apparences, ni la langue, ni la religion, ni les arts, ni, en un mot, aucun élément de civilisation, ne peut passer intact d’un peuple à un autre.}. Le milieu, les circonstances, les événements représentent les suggestions sociales du moment. Ils peuvent avoir une influence considérable, mais cette influence est toujours momentanée si elle est contraire aux suggestions de la race, c’est-à-dire de toute la série des ancêtres.\par
Dans plusieurs chapitres de cet ouvrage, nous aurons encore occasion de revenir sur l’influence de la race, et de montrer que cette influence est si grande qu’elle domine les caractères spéciaux à l’âme des foules de là ce, fait que les foules de divers pays présentent dans leurs croyances et leur conduite des différences très considé­rables, et ne peuvent être influencées de la même façon.
\subsubsection[{§ 2. – Les traditions}]{§ 2. – Les traditions}
\noindent Les traditions représentent les idées, les besoins, les sentiments du passé. Elles sont la synthèse de la race et pèsent de tout leur poids sur nous.\par
Les sciences biologiques ont été transformées depuis que l’embryologie a montré l’influence immense du passé dans l’évolution des êtres ; et les sciences historiques ne le seront pas moins quand cette notion sera plus répandue. Elle ne l’est pas suffi­samment encore, et bien des hommes d’État en sont restés aux idées des théoriciens du dernier siècle, qui croyaient qu’une société peut rompre avec son passé et être refaite de toutes pièces en ne prenant pour guide que les lumières de la raison.\par
Un peuple est un organisme créé par le passé, et qui, comme tout organisme, ne peut se modifier que par de lentes accumulations héréditaires.\par
Ce qui conduit les hommes, surtout lorsqu’ils sont en foule, ce sont les traditions ; et, comme je l’ai répété bien des fois, ils n’en changent facilement que les noms, les formes extérieures.\par
Il n’est pas à regretter qu’il en soit ainsi. Sans traditions, il n’y a ni âme nationale, ni civilisation possibles. Aussi les deux grandes occupations de l’homme depuis qu’il existe ont-elles été de se créer un réseau de traditions, puis de tâcher de les détruire lorsque leurs effets bienfaisants se sont usés. Sans les traditions, pas de civilisation ; sans la lente élimination de ces traditions, pas de progrès. La difficulté est de trouver un juste équilibre entre la stabilité et la variabilité ; et cette difficulté est immense. Quand un peuple a laissé des coutumes se fixer trop solidement chez lui pendant beaucoup de générations, il ne peut plus changer et devient, comme la Chine, incapa­ble de perfectionnement. Les révolutions violentes n’y peuvent rien, car il arrive alors, ou que les fragments brisés de la chaîne se ressoudent, et que le passé reprend sans changements son empire, ou que les fragments restent dispersés, et alors à l’anar­chie succède bientôt la décadence.\par
Aussi, l’idéal pour un peuple est-il de garder les institutions du passé, en ne les transformant qu’insensiblement et peu à peu. Cet idéal est difficilement accessible. Les Romains, dans les temps anciens, les Anglais, dans les temps modernes, sont à peu près les seuls qui l’aient réalisé.\par
Les conservateurs les plus tenaces des idées traditionnelles, et qui s’opposent le plus obstinément à leur changement, sont précisément les foules, et notamment les catégories de foules qui constituent les castes. J’ai déjà insisté sur l’esprit conservateur des foules, et montré que les plus violentes révoltes n’aboutissent qu’à un changement de mots. A la fin du dernier siècle, devant les églises détruites, devant les prêtres expulsés ou guillotinés, devant la persécution universelle du culte catholique, on pouvait croire que les vieilles idées religieuses avaient perdu tout pouvoir ; et cepen­dant quelques années s’étaient à peine écoulées que, devant les réclamations univer­selles, il fallut rétablir le culte aboli \footnote{ \noindent Le rapport de l’ancien conventionnel Fourcroy, cité par Taine, est à ce point de vue fort net :\par
 “Ce qu’on voit partout sur la célébration du dimanche et sur la fréquentation des églises prouve que la masse des Français veut revenir aux anciens usages, et il n’est plus temps de résister à cette pente nationale… La grande masse, des hommes a besoin de religion, de culte et de prêtres. \emph{C’est} une \emph{erreur de quelques philosophes modernes, à laquelle j’ai été moi-même entraîné}, que de croire à la possibilité d’une instruction assez répandue pour détruire les préjugés reli­gieux ; ils sont, pour le grand nombre des malheureux, une source de consolation… Il faut donc laisser à la masse du peuple, ses prêtres, ses autels et son culte.
}.\par
Effacées un instant, les vieilles traditions avaient repris leur empire.\par
Aucun exemple ne montre mieux la puissance des traditions sur l’âme des foules. Ce n’est pas dans les temples qu’habitent les idoles les plus redoutables, ni dans les palais les tyrans les plus despotiques ; ceux-ci peuvent être brisés en un instant ; mais les maîtres in­visibles qui règnent dans nos âmes échappent à tout effort de révolte, et ne cèdent qu’à la lente usure des siècles.
\subsubsection[{§ 3. – Le temps}]{§ 3. – Le temps}
\subsubsection[{§ 4. – Les institutions politiques et sociales}]{§ 4. – Les institutions politiques et sociales}
\noindent L’idée que les institutions peuvent remédier aux défauts des sociétés ; que le progrès des peuples est la conséquence du perfectionnement des constitutions et des gouvernements et que les changements sociaux peuvent se faire à coups de décrets ; cette idée, dis-je, est très généralement répandue encore. La Révolution française l’eut pour point de départ et les théories sociales actuelles y prennent leur point d’appui.\par
Les expériences les plus continues n’ont pas réussi encore à ébranler sérieusement cette redoutable chimère. C’est en vain que philosophes et historiens ont essayé d’en prouver l’absurdité. Il ne leur a pas été difficile pourtant de montrer que les institu­tions sont filles des idées, des sentiments et des mœurs ; et qu’on ne refait pas les idées, les sentiments et les mœurs en refaisant les codes. Un peuple ne choisit pas ses institutions à son gré, pas plus qu’il ne choisit la couleur de ses yeux ou de ses cheveux. Les institutions et les gouvernements sont le produit de la race. Loin d’être les créateurs d’une époque, ils sont ses créations. Les peuples ne sont pas gouvernés comme le voudraient leurs caprices d’un moment, mais comme l’exige leur caractère. Il faut des siècles pour former un régime politique, et des siècles pour le changer. Les institutions n’ont aucune vertu intrinsèque ; elles ne sont ni bonnes ni mauvaises en elles-mêmes. Celles qui sont bonnes à un moment donné pour un peuple donné, peuvent être détestables pour un autre.\par
Aussi n’est-il pas du tout dans le pouvoir d’un peuple de changer réellement ses institutions. Il peut assurément, au prix de révolutions violentes, changer le nom de ces institutions, mais le fond ne se modifie pas. Les noms ne sont que de vaines éti­quettes dont l’historien qui va un peu au fond des choses n’a pas à se préoccuper. C’est ainsi par exemple que le plus démocratique des pays du monde est l’Angleterre \footnote{ \noindent C’est ce que reconnaissent, même aux États-Unis, les républicains les plus avancés. Le journal américain \emph{Forum} exprimait récemment cette opinion catégorique dans les termes que je reproduis ici, d’après la \emph{Review of Reviews de} décembre 1894 :\par
 “On ne doit jamais oublier, même chez les plus fervents ennemis de l’aristocratie, que l’Angleterre est aujourd’hui le pays le plus démocratique de l’univers, celui où les droits de l’individu sont le plus respectés, et celui où les individus possèdent le plus de liberté.”
}, qui vit cependant sous un régime monarchique, alors que les pays où sévit le plus lourd despotisme sont les républiques hispano-américaines, malgré les constitutions répu­blicaines qui les régissent. Le caractère des peuples et non les gouvernements conduit leurs destinées. C’est un point de vue que j’ai essayé d’établir dans un précédent volu­me, en m’appuyant sur de catégoriques exemples.\par
C’est donc une tâche très puérile, un inutile exercice de rhéteur ignorant que de perdre son temps à fabriquer de toutes pièces des constitutions. La nécessité et le temps se chargent de les élaborer, quand nous avons la sagesse de laisser agir ces deux facteurs. C’est ainsi que les Anglo-Saxons s’y sont pris, et c’est ce que nous dit leur grand historien Macaulay dans un passage que devraient apprendre par cœur les politiciens de tous les pays latins. Après avoir montré tout le bien qu’ont pu faire des lois qui semblent, au point de vue de la raison pure, un chaos d’absurdités et de con­tradictions, il compare les douzaines de constitutions, mortes dans les convulsions, des peuples latins de l’Europe et de l’Amérique avec celle de l’Angleterre, et fait voir que cette dernière n’a été changée que très lentement, par parties, sous l’influence de nécessités immédiates et jamais de raisonnements spéculatifs. “Ne point s’inquiéter de la symétrie, et s’inquiéter beaucoup de l’utilité ; n’ôter jamais une anomalie unique­ment parce qu’elle est une anomalie ; ne jamais innover si ce n’est lorsque quelque malaise se fait sentir, et alors innover juste assez pour se débarrasser du malaise n’établir jamais une proposition plus large que le cas particulier auquel on remédie ; telles sont les règles qui, depuis l’âge de Jean jusqu’à l’âge de Victoria, ont généralement guidé les délibérations de nos 250 parlements.”\par
Il faudrait prendre une à une les lois, les institutions de chaque peuple, pour mon­trer à quel point elles sont l’expression des besoins de leur race, et ne sauraient pour cette raison être violemment transformées. On peut disserter philosophiquement, par exemple, sur les avantages et les inconvénients de la centralisation mais quand nous voyons un peuple,. composé de races très diverses, consacrer mille ans d’efforts pour arriver progressivement à cette centralisation ; quand nous constatons qu’une grande révolution ayant pour but de briser toutes les institutions du passé, a été forcée non seulement de respecter cette centralisation, mais d’en exagérer encore, nous pouvons dire qu’elle est fille de nécessités impérieuses, une condition même d’existence, et plaindre la faible portée mentale des hommes politiques qui parlent de la détruire. S’ils pouvaient par hasard y réussir, l’heure de la réussite serait aussitôt le signal d’une effroyable guerre civile \footnote{Si l’on rapproche les profondes dissensions religieuses et politiques qui séparent les diverses parties de la France, et sont surtout une question de races, des tendances séparatistes qui se sont manifestées à l’époque de la Révolution, et qui commençaient à se dessiner de nouveau vers la fin de la guerre franco-allemande, on voit que les races diverses qui subsistent sur notre sol sont bien loin d’être fusionnées encore. La centralisation énergique de la Révolution et la création de départements artificiels destinés à mêler les anciennes provinces fut certainement son œuvre la plus utile Si la décentralisation, dont parlent tant aujourd’hui des esprits imprévoyants, pouvait être créée, elle aboutirait promptement aux plus sanglantes discordes. Il faut pour le méconnaître oublier entièrement notre histoire.} qui ramènerait immédiatement d’ailleurs une nouvelle centralisation beaucoup plus lourde que l’ancienne.\par
Concluons de ce qui précède que ce n’est pas dans les institutions qu’il faut cher­cher le moyen d’agir profondément sur l’âme des foules ; et quand nous voyons cer­tains pays, comme les États-Unis, arriver à un haut degré de prospérité avec des institutions démocratiques, alors que nous en voyons d’autres, tels que les républi­ques hispano-américaines, vivre dans la plus triste anarchie malgré des institutions absolument semblables, disons-nous bien que ces institutions sont aussi étrangères à la grandeur des uns qu’à la décadence des autres. Les peuples sont gouvernés par leur caractère, et toutes les institutions qui ne sont pas intimement moulées sur ce carac­tère ne représentent qu’un vêtement d’emprunt, un déguisement transitoire. Certes, des guerres sanglantes, des révolutions violentes ont été faites, et se feront encore, pour imposer des institutions auxquelles est attribué, comme aux reliques des saints, le pouvoir surnaturel de créer le bonheur. On pourrait donc dire en un sens que les insti­tutions agissent sur l’âme des foules puisqu’elles engendrent de pareils soulèvements. Mais, en réalité, ce ne sont pas les institutions qui agissent alors, puisque nous savons que, triomphantes ou vaincues, elles ne possèdent par elles-mêmes aucune vertu. Ce qui a agi sur l’âme des foules, ce sont des illusions et des mots. Des mots surtout, ces mots chimériques et puissants dont nous montrerons bientôt l’étonnant empire.
\subsubsection[{§ 5. – L’instruction et l’éducation}]{§ 5. – L’instruction et l’éducation}
\noindent Au premier rang de ces idées dominantes d’une époque, dont nous avons marqué ailleurs le petit nombre et la force, bien qu’elles soient parfois des illusions pures, se trouve aujourd’hui celle-ci : que l’instruction est capable de changer considérablement les hommes, et a pour résultat certain de les améliorer, et même de les rendre égaux. Par le fait seul de la répétition, cette assertion a fini par devenir un des dogmes les plus inébranlables de la démocratie. Il serait aussi difficile d’y toucher maintenant qu’il l’eût été jadis de toucher à ceux de l’Église.\par
Mais sur ce point, comme sur bien d’autres, les idées démocratiques se sont trou­vées en profond désaccord avec les données de la psychologie et de l’expérience. Plusieurs philosophes éminents, Herbert Spencer entre autres, n’ont pas eu de peine à montrer que l’instruction ne rend l’homme ni plus moral ni plus heureux, qu’elle ne change pas ses instincts et ses passions héréditaires ; qu’elle est parfois – pour peu qu’elle soit mal dirigée – beaucoup plus pernicieuse qu’utile. Les statisticiens sont venus confirmer ces vues en nous disant que la criminalité augmente avec la géné­ralisation de l’instruction, ou tout au moins d’une certaine instruction ; que les pires ennemis de la société, les anarchistes, se recrutent souvent parmi les lauréats des écoles ; et, dans un travail récent, un magistrat distingué, M. Adolphe Guillot, faisait remarquer qu’on compte maintenant 3.000 criminels lettrés contre 1.000 illettrés, et que, en cinquante ans, la criminalité est passée de 227 pour 400.000 habitants, à 552, soit une augmentation de 133 p. 100. Il a noté également avec tous ses collègues que la criminalité augmente surtout chez les jeunes gens pour lesquels l’école gratuite et obligatoire a, comme on sait, remplacé le patronat.\par
Ce n’est pas certes, et personne ne l’a jamais soutenu, que l’instruction bien dirigée ne puisse donner des résultats pratiques fort utiles, sinon pour élever la moralité, au moins pour développer les capacités professionnelles. Malheureusement les peuples latins, sur­tout depuis vingt-cinq ans, ont basé leurs systèmes d’instruction sur des principes très erronés, et, malgré les observations des esprits les plus éminents, ils persistent dans leurs lamentables erreurs. J’ai moi-même, dans divers ouvrages \footnote{\emph{Voir Psychologie du socialisme}, 3° édit. \emph{Psychologie de l’éducation} (5° édition).}, montré que notre éducation actuelle transforme en ennemis de la société la plupart de ceux qui l’ont reçue, et recrute de nombreux disciples pour les pires formes du socialisme.\par
Ce qui constitue le premier danger de cette éducation – très justement qualifiée de latine – c’est quelle repose sur cette erreur psychologique fondamentale, que c’est en apprenant par cœur des manuels qu’on développe l’intelligence. Dès lors on a tâché d’en apprendre le plus possible ; et, de l’école primaire au doctorat ou à l’agrégation, le jeune homme ne fait qu’apprendre par cœur des livres, sans que son jugement et son initiative soient jamais exercés. L’instruction, pour lui, c’est réciter et obéir. “Apprendre des leçons, savoir par cœur une grammaire ou un abrégé, bien répéter, bien imiter, voilà, écrit un ancien ministre de l’instruction publique, M. Jules Simon, une plaisante éducation où tout effort est un acte de foi devant l’infaillibilité du maître, et n’aboutit qu’à nous diminuer et nous rendre impuissants.”\par
Si cette éducation n’était qu’inutile, on pourrait se borner à plaindre les malheu­reux enfants auxquels, au lieu de tant de choses nécessaires à apprendre à l’école primaire, on préfère enseigner la généalogie des fils de Clotaire, les luttes de la Neustrie et de l’Austrasie, ou des classifications zoologiques ; mais elle présente un danger beaucoup plus sérieux. Elle donne à celui qui l’a reçue un dégoût violent de la condition où il est né, et l’intense désir d’en sortir. L’ouvrier ne veut plus rester ouvrier, le paysan ne veut plus être paysan, et le dernier des bourgeois ne voit pour ses fils d’autre carrière possible que les fonctions salariées par l’État. Au lieu de préparer des hommes pour la vie, l’école ne les prépare qu’à des fonctions publiques où l’on peut réussir sans avoir à se diriger ni à manifester aucune lueur d’initiative. Au bas de l’échelle, elle crée ces armées de prolétaires mécontents de leur sort et toujours prêts à la révolte ; en haut, notre bourgeoisie frivole, à la fois sceptique et crédule, ayant une confiance superstitieuse dans l’État-providence, que cependant elle fronde sans cesse, s’en prenant toujours au gouvernement de ses propres fautes et incapable de rien entreprendre sans l’intervention de l’autorité.\par
L’État qui fabrique à coups de manuels tous ces diplômés, ne peut en utiliser qu’un petit nombre et laisse forcément sans emploi les autres. Il lui faut donc se résigner à nourrir les premiers et à avoir pour ennemis les seconds. Du haut en bas de la pyra­mide sociale, du simple commis au professeur et au préfet, la masse immense des diplômés assiège aujourd’hui les carrières. Alors qu’un négociant ne peut que très difficilement trouver un agent pour aller le représenter dans les colonies, c’est par des milliers de candidats que les plus modestes places officielles sont sollicitées. Le département de la Seine compte à lui seul 20.000 instituteurs et institutrices sans emploi, et qui, méprisant les champs et l’atelier, s’adressent à l’État pour vivre. Le nombre des élus étant restreint, celui des mécontents est forcément immense. Ces derniers sont prêts pour toutes les révolutions, quels qu’en soient les chefs et quelque but qu’elles poursuivent. L’acquisition de connaissances dont on ne peut trouver l’emploi est un moyen sûr de faire de l’homme un révolté \footnote{Ce n’est pas là d’ailleurs un phénomène spécial aux peuples latins ; on l’observe aussi en Chine, pays conduit également par une solide hiérarchie de mandarins, et où le mandarinat est, comme chez nous, obtenu par des concours dont la seule épreuve est la récitation imperturbable d’épais manuels. L’armée des lettrés sans emploi est considérée aujourd’hui en Chine comme une vérita­ble calamité nationale. il en est de même dans l’Inde, où, depuis que les Anglais ont ouvert des écoles, non pour éduquer, comme cela se fait en Angleterre, mais simplement pour instruire les indi­gènes, il s’est formé une classe spéciale de lettrés, les Babous, qui, lorsqu’ils ne peuvent recevoir un emploi, deviennent d’irréconciliables ennemis de la puissance anglaise. Chez tous les Babous, munis ou non d’emplois, le premier effet de l’instruction a été d’abaisser immensément le niveau de leur moralité. C’est un fait sur lequel j’ai longuement insisté dans mon livre \emph{Les Civilisalions de l’Inde}, et qu’ont également constaté tous les auteurs qui ont visité la grande péninsule.}.\par
Il est évidemment trop tard pour remonter un tel courant. Seule l’expérience, dernière éducatrice des peuples, se chargera de nous montrer notre erreur. Elle seule sera assez puissante pour prouver la nécessité de remplacer nos odieux manuels, nos pitoyables concours par une instruction professionnelle capable de ramener la jeu­nesse vers les champs, les ateliers, les entreprises coloniales, qu’aujourd’hui elle cherche à tout prix à fuir.\par
Cette instruction professionnelle que tous les esprits éclairés réclament mainte­nant fut celle qu’ont jadis reçue nos pères, et que les peuples qui dominent aujourd’hui le monde par leur volonté, leur initiative, leur esprit d’entreprise ont su conserver. Dans des pages remarquables, dont je reproduirai plus loin les parties les plus essen­tielles, un grand penseur, M. Taine, a montré nettement que notre éducation d’autre­fois était à peu près ce qu’est l’éducation anglaise ou américaine d’aujourd’hui, et, dans un remarquable parallèle entre le système latin et le système anglo-saxon, il a fait voir clairement les conséquences des deux méthodes.\par
On consentirait peut-être, à l’extrême rigueur, à accepter encore tous les inconv­énients de notre éducation classique, alors même qu’elle ne ferait que des déclassés et des mécontents, si l’acquisition superficielle de tant de connaissances, la récitation parfaite de tant de manuels élevait le niveau de l’intelligence. Mais l’élève-t-elle réel­lement ? Non, hélas ! C’est le jugement, l’expérience, l’initiative, le caractère qui sont les conditions de succès dans la vie, et ce n’est pas là ce que donnent les livres. Les livres sont des dictionnaires utiles à consulter, mais dont il est parfaitement inutile d’avoir de longs fragments dans la tête.\par
Comment l’instruction professionnelle peut-elle développer l’intelligence dans une mesure qui échappe tout à fait à l’instruction classique : c’est ce que M. Taine montre fort bien.\par
“Les idées ne se forment que dans leur milieu naturel et normal ; ce qui fait végéter leur germe, ce sont les innombrables impressions sensibles que le jeune hom­me reçoit tous les jours à l’atelier, dans la mine, au tribunal, à l’étude, sur le chantier, à l’hôpital, au spectacle des outils, des matériaux et des opérations, en présence des clients, des ouvriers, du travail, de l’ouvrage bien ou mal fait, dispendieux ou lucratif : voilà les petites perceptions particulières des yeux, de l’oreille, des mains et même de l’odorat, qui, involontairement recueillies et sourdement élaborées, s’organisent en lui pour lui suggérer tôt ou tard telle combinaison nouvelle, simplification, économie, perfectionnement ou invention. De tous ces contacts précieux, de tous ces éléments assimilables et indispensables, le jeune Français est privé, et justement pendant l’âge fécond ; sept ou huit années durant, il est séquestré dans une école, loin de l’expé­rience directe et personnelle qui lui aurait donné la notion exacte et vive des choses, des hommes et des diverses façons de les manier.\par
“… Au moins neuf sur dix ont perdu leur temps et leur peine, plusieurs années de leur vie, et des années efficaces, importantes ou même décisives : comptez d’abord la moitié ou les deux tiers de ceux qui se présentent à l’examen, je veux dire les refusés ­ensuite, parmi les admis, gradués, brevetés et diplômés, encore la moitié ou les deux tiers, je veux dire les surmenés. On leur a demandé trop en exigeant que tel jour, sur une chaise ou devant un tableau, ils fussent, deux heures durant et pour un groupe de sciences, des répertoires vivants de toute la connaissance humaine ; en effet, ils ont été cela, ou a peu près, ce jour-là, pendant deux heures ; mais, un mois plus tard, ils ne le sont plus ; ils ne pourraient pas subir de nouveau l’examen ; leurs acquisitions, trop nombreuses et trop lourdes, glissent incessamment hors de leur esprit, et ils n’en font pas de nouvelles. Leur vigueur mentale a fléchi ; la sève féconde est tarie ; l’hom­me fait apparaît, et, souvent c’est l’homme fini. Celui-ci, rangé, marié, résigné à tour­ner en cercle et indéfiniment dans le même cercle, se cantonne dans son office restreint ; il le remplit correctement, rien au delà. Tel est le rendement moyen ; certainement la recette n’équilibre pas la dépense. En Angleterre et en Amérique, où, comme jadis avant 1789, en France, on emploie le procédé inverse, le rendement obtenu est égal ou supérieur.”\par
L’illustre historien nous montre ensuite la différence de notre système avec celui des Anglo-Saxons. Ces derniers ne possèdent pas nos innombrables écoles spé­ciales ; chez eux l’enseignement n’est pas donné par le livre, mais par la chose elle-même. L’ingénieur, par exemple, se forme dans un atelier et jamais dans une école ; ce qui permet à chacun d’arriver exactement au degré que comporte son intelligence, ouvrier ou contremaître s’il ne peut aller plus loin, ingénieur si ses aptitudes l’y conduisent. C’est là un procédé autrement démocratique et autrement utile pour la société que de faire dépendre toute la carrière d’un individu d’un concours de quelques heures subi à dix-­huit ou vingt ans.\par
“A l’hôpital, dans la mine, dans la manufacture, chez l’architecte, chez l’homme de loi, l’élève, admis très jeune, fait son apprentissage et son stage, à peu près comme chez nous un clerc dans son étude ou un rapin dans son atelier. Au préalable et avant d’entrer, il a pu suivre quelque cours général et sommaire, afin d’avoir un cadre tout prêt pour y loger les observations que tout à l’heure il va faire. Cependant, à sa portée, il y a, le plus souvent, quelques cours techniques qu’il pourra suivre à ses heures libres, afin de coordonner au fur et à mesure les expériences quotidiennes qu’il fait. Sous un pareil régime, la capacité pratique croit et se développe d’elle-même, juste au degré que comportent les facultés de l’élève, et dans la direction requise par sa besogne future par l’œuvre spéciale à laquelle dès à présent il veut s’adapter. De cette façon, en Angleterre et aux États-Unis, le jeune homme parvient vite à tirer de lui-même tout ce qu’il contient. Dès vingt-cinq ans, et bien plus tôt, si la substance et le fonds ne lui manquent pas, il est, non seulement un exécutant utile, mais encore un entrepreneur spontané, non seulement un. rouage, mais de plus un moteur, – En France, où le procédé inverse a prévalu et, à chaque génération, devient plus chinois, le total des forces perdues est énorme.”\par
Et le grand philosophe arrive à la conclusion suivante sur la disconvenance croissante de notre éducation latine et de la vie.\par
“Aux trois étages de l’instruction, pour l’enfance, l’adolescence et la jeunesse, la préparation théorique et scolaire sur des bancs, par des livres, s’est prolongée et surchargée, en vue de l’examen, du grade, du diplôme et du brevet, en vue, de cela seulement, et par les pires moyens, par l’application d’un régime antinaturel et anti­social, par le retard excessif de l’apprentissage pratique, par l’internat, par l’entraîne­ment artificiel et le remplissage mécanique, par le surmenage, sans considération du temps qui suivra, de l’âge adulte et des offices virils que l’homme fait exercera, abstraction faite du monde réel où tout à l’heure le jeune homme va tomber, de la société ambiante à laquelle il faut l’adapter ou le résigner d’avance, du conflit humain où pour se défendre et se tenir debout, il doit être, au préalable, équipé, armé, exercé, endurci. Cet équipement indispensable, cette acquisition plus importante que toutes les autres, cette solidité du bon sens, de la volonté et des nerfs, nos écoles ne la lui procurent pas ; tout au rebours ; bien loin de le qualifier, elles le disqualifient pour sa condition prochaine et définitive. Partant, son entrée dans le monde et ses premiers pas dans le champ de l’action pratique ne sont, le plus souvent, qu’une suite de chutes douloureuses ; il en reste meurtri, et, pour longtemps, froissé, parfois estropié à demeure. C’est une rude et dangereuse épreuve ; l’équilibre moral et mental s’y altère, et court risque de ne passe rétablir ; la désillusion est venue, trop brusque et trop com­plète ; les déceptions ont été trop grandes et les déboires trop forts \footnote{TAINE. \emph{Le Régime moderne}, t. II, 1891. – Ces pages sont à peu près les dernières qu’écrivit Taine. Elles résument admirablement les résultats de la longue expérience du grand philosophe. Je les crois malheureusement totalement incom­préhensibles pour les professeurs de notre université n’ayant pas séjourné à l’étranger. L’éducation est le seul moyen que nous possédions pour agir un peu sur l’âme d’un peuple et il est profondément triste d’avoir à songer qu’il n’est à peu près personne, en France qui puisse arriver à comprendre que notre enseignement actuel est un redou­table élément de rapide décadence et qu’au lieu d’élever la jeunesse il l’abaisse et la pervertit.}.”\par
Nous sommes-nous éloignés, dans ce qui précède, de la psychologie des foules ? Non certes. Si nous voulons comprendre les idées, les croyances qui y germent aujourd’hui, et qui écloront demain, il faut savoir comment le terrain a été préparé. L’enseignement donné à la jeunesse d’un pays permet de savoir ce que sera ce pays un jour. L’éducation donnée à la génération actuelle justifie les prévisions les plus som­bres. C’est en partie avec l’instruction et l’éducation que s’améliore ou s’altère l’âme des foules. Il était donc nécessaire de montrer comment le système actuel l’a façon­née, et comment la masse des indifférents et des neutres est devenue progressivement une immense armée de mécontents, prête à obéir à toutes les suggestions des utopistes et des rhéteurs. C’est à l’école que se forment aujourd’hui les mécontents et les anar­chistes et que se préparent pour les peuples latins les heures prochaines de décadence.
\subsection[{Chapitre 2. Facteurs immédiats des opinions des foules}]{Chapitre 2. Facteurs immédiats des opinions des foules}

\begin{argument}\noindent § 1\emph{. Les images, les mots et les formules. –} Puissance magique des mots et des formules. – La puissance des mots est liée aux images qu’ils évoquent et est indépen­dante de leur sens réel. – Ces images varient d’âge en âge, de race en race. – L’usure des mots. – Exemples des variations considéra­bles du sens de quelques mots très usuels. – Utilité politique de baptiser de noms nouveaux les choses anciennes, lorsque les mots sous lesquels on les désignait produisent une fâcheuse impression sur les foules. – Variations du sens des mots suivant la race. – Sens différents du mot démocratie en Europe et en Amérique. – § 2\emph{. Les illusions.} – Leur importance. – On les retrouve à la base de toutes les civilisa­tions. – Nécessité sociale des illusions. – Les foules les préfèrent toujours aux vérités. – § 3.\emph{ L’expérience.} – L’expérience seule peut établir dans l’âme des foules des vérités devenues néces­saires et détruire des illusions devenues dangereuses. – L’expérience n’agit qu’à condition d’être fréquemment répétée. – Ce que coûtent les expériences nécessaires pour persuader les foules. – § 4. \emph{La raison. –} Nullité de son influence sur les foules. – On n’agit sur elles qu’en agissant sur leurs sentiments inconscients. – Le rôle de la ­logique dans l’histoire. – Les causes secrètes des événements invrai­semblables.
\end{argument}

\noindent Nous venons de rechercher les facteurs lointains et préparatoires qui donnent à l’âme des foules une réceptivité spéciale, rendant possible chez elle l’éclosion de certains sentiments et de certaines idées. Il nous reste à étudier maintenant les facteurs capables d’agir d’une façon immédiate. Nous verrons dans un prochain chapitre comment doivent être maniés ces facteurs pour qu’ils puissent produire tous leurs effets.\par
Dans la première partie de cet ouvrage nous avons étudié les sentiments, les idées, les raisonnements des collectivités ; et, de cette connaissance, on pourrait évidem­ment déduire d’une façon générale les moyens d’impressionner leur âme. Nous savons déjà ce qui frappe l’imagination des foules, la puissance et la contagion des sugges­tions, surtout de celles qui se présentent sous forme d’images. Mais les suggestions pouvant être d’origine fort diverses, les facteurs capables d’agir sur l’âme des foules peuvent être assez différents. Il est donc nécessaire de les examiner séparément. Ce n’est pas là une inutile étude. Les foules sont un peu comme le sphinx de la fable antique : il faut savoir résoudre les problèmes que leur psychologie nous pose, ou se résigner à être dévoré par elles.\par
\subsubsection[{§ 1. – Les images, les mots et les formules}]{§ 1. – Les images, les mots et les formules}
\noindent En étudiant l’imagination des foules, nous avons vu qu’elle est impressionnée surtout par des images. Ces images, on n’en dispose pas toujours, mais il est possible de les évoquer par l’emploi judicieux des mots et des formules. Maniés avec art, ils possèdent vraiment la puissance mystérieuse que leur attribuaient jadis les adeptes de la magie. Ils font naître dans l’âme des foules les plus formidables tempêtes, et savent aussi les calmer. On élèverait une pyramide beaucoup plus haute que celle du vieux Khéops avec les seuls ossements des hommes victimes de la puissance des mots et des formules.\par
La puissance des mots est liée aux images qu’ils évoquent et tout à fait indépen­dante de leur signification réelle. Ce sont parfois ceux dont le sens est le plus mal défini qui possèdent le plus d’action. Tels par exemple. les termes : démocratie, socia­lisme égalité, liberté, etc., dont le sens est si vague que de gros volumes ne suffisent pas à le préciser. Et pourtant il est certain qu’une puissance vraiment magique s’atta­che leurs brèves syllabes, comme si elles contenaient la solution de tous les problèmes. Ils synthétisent les aspirations inconscientes les plus diverses et l’espoir de leur réalisation.\par
La raison et les arguments ne sauraient lutter contre certains mots et certaines formules. On les prononce avec recueillement devant les foules ; et, dès qu’ils ont été prononcés, les visages deviennent respectueux et les fronts s’inclinent. Beaucoup les considèrent comme des forces de la nature, des puissances surnaturelles. Ils évoquent dans les âmes des images grandioses et vagues, mais le vague même qui les estompe augmente leur mystérieuse puissance. On peut les comparer à ces divinités redou­tables cachées derrière le tabernacle et dont le dévot ne s’approche qu’en tremblant.\par
Les images évoquées par les mots étant indépendantes de leur sens, varient d’âge en âge, de peuple à peuple, sous l’identité des formules. A certains mots s’attachent transitoirement certaines images : le mot n’est que le bouton d’appel qui les fait apparaître.\par
Tous les mots et toutes les formules ne possèdent pas la puissance d’évoquer des images ; et il en est qui, après en avoir évoqué, s’usent et ne réveillent plus rien dans l’esprit. Ils deviennent alors de vains sons, dont l’utilité principale est de dispenser celui qui les emploie de l’obligation de penser. Avec un petit stock de for­mules et de lieux communs appris dans la jeunesse, nous possédons tout ce qu’il faut pour traver­ser la vie sans la fatigante nécessité d’avoir à réfléchir sur quoi que ce soit.\par
Si l’on considère une, langue déterminée, on voit que les mots dont elle se com­pose changent assez lentement dans le cours des âges ; mais ce qui change sans cesse, ce sont les images qu’ils évoquent ou le sens qu’on y attache ; et c’est pourquoi je suis arrivé, dans un autre ouvrage, à cette conclusion que la traduction complète d’une langue, surtout quand il s’agit de peuples morts, est chose totalement impossible. Que faisons-nous, en réalité, quand nous substituons un terme français à un terme latin, grec ou sanscrit, ou même quand nous cherchons à comprendre un livre écrit dans notre propre langue il y a deux ou trois siècles ? Nous substituons simplement les images et les idées que la vie moderne a mises dans notre intelligence, aux notions et aux images absolument différentes que la vie ancienne avait fait naître dans l’âme de races soumises à des conditions d’existence sans analogie avec les nôtres. Quand les hommes de la Révolution croyaient copier les Grecs et les Romains, que faisaient-ils, sinon donner à des mots anciens un sens que ceux-ci n’eurent jamais. Quelle ressem­blance pouvait-il exister entre les institutions des Grecs et celles que désignent de nos jours les mots correspondants ? Qu’était alors une république, sinon une institution essentiellement aristocratique formée d’une réunion de petits despotes dominant une foule d’esclaves maintenus dans la plus absolue sujétion. Ces aristocraties communa­les, basées sur l’esclavage, n’auraient pu exister un instant sans lui.\par
Et le mot liberté, que pouvait-il signifier de semblable à ce que nous comprenons aujourd’hui, à une époque où la possibilité de la liberté de penser n’était même pas soupçonnée, et où il n’y avait pas de forfait plus grand et plus rare que de discuter les dieux, les lois et les coutumes de la cité ? Un mot comme celui de patrie, que signifiait-il dans l’âme d’un Athénien ou d’un Spartiate, sinon le culte d’Athènes ou de Sparte, et nullement celui de la Grèce, composée de cités rivales et toujours en guerre. Le même mot de patrie, quel sens avait-il chez les anciens Gaulois divisés en tribus rivales, de races, de langues et de religions différentes, que César vainquit facilement parce qu’il eut toujours parmi elles des alliées. Rome seule donna à la Gaule une patrie en lui donnant l’unité politique et religieuse. Sans même remonter si loin, et en reculant de deux siècles à peine, croit-on que le même mot de patrie était conçu comme aujourd’hui par des princes français, tels que le grand Condé, s’alliant à l’étranger contre leur souverain ? Et le même mot encore n’avait-il pas un sens bien différent du sens moderne pour les émigrés, qui croyaient obéir aux lois de l’honneur en combattant la France, et qui à leur point de vue y obéissaient en effet, puisque la loi féodale liait le vassal au seigneur et non à la terre, et que là où était le souverain, là était la vraie patrie.\par
Nombreux sont les mots dont le sens a ainsi profondément changé d’âge en âge, et que nous ne pouvons arriver à comprendre comme on les comprenait jadis qu’après un long effort. On a dit avec raison qu’il faut beaucoup de lecture pour arriver seulement à concevoir ce que signifiaient pour nos arrière-grands-pères des mots tels que le roi et la famille royale. Qu’est-ce alors pour des termes plus complexes encore ?\par
Les mots n’ont donc que des significations mobiles et transitoires, changeantes d’âge en âge et de peuple à peuple ; et, quand nous voulons agir par eux, sur la foule, ce qu’il faut savoir, c’est le sens qu’ils ont pour elle à un moment donné, et non celui qu’ils eurent jadis ou qu’ils peuvent avoir pour des individus de constitution mentale différente.\par
Aussi, quand les foules ont fini, à la suite de bouleversements politiques, de chan­gements de croyances, par acquérir une antipathie profonde pour les images évoquées par certains mots, le premier devoir de l’homme d’État véritable est de changer les mots sans, bien entendu, toucher aux choses en elles-mêmes, ces dernières étant trop liées à une constitution héréditaire pour pouvoir être transformées. Le judicieux Tocqueville a fait remarquer, il y a déjà longtemps, que le travail du Consulat et de l’Empire a surtout consisté à habiller de mots nouveaux la plupart des institutions du passé, c’est-à-dire à remplacer des mots évoquant de fâcheuses images dans l’imagi­nation des foules par d’autres mots dont la nouveauté empêchait de pareilles évoca­tions. La taille est devenue contribution foncière ; la gabelle, l’impôt du sel ; les aides, contributions indirectes et droit réunis la taxe des maîtrises et jurandes s’est appelée patente, etc.\par
Une des fonctions les plus essentielles des hommes d’État consiste donc à baptiser de mots populaires, ou au moins neutres, les choses que les foules ne peuvent supporter avec leurs anciens noms. La puissance des mots est si grande qu’il suffit de désigner par des termes bien choisis les choses les plus odieuses pour les faire accepter des foules. Taine remarque justement que c’est en invoquant la liberté et la fraternité, mots très populaires alors, que les Jacobins ont pu “installer un despotisme digne du Dahomey, un tribunal pareil à celui de l’inquisition, des hécatombes humai­nes semblables à celles de l’ancien Mexique”. L’art des gouvernants, comme celui des avocats, consiste surtout à savoir manier les mots. Une des grandes difficultés de cet art est que, dans une même société, les mêmes mots ont le plus souvent des sens fort différents pour les diverses couches sociales. Elles emploient en apparence les mêmes mots ; mais elles ne parlent jamais la même langue.\par
Dans les exemples qui précèdent nous avons fait surtout intervenir le temps comme principal facteur du changement de sens des mots. Mais si nous faisions intervenir aussi la race, nous verrions alors qu’à une même époque, chez des peuples également civilisés mais de races diverses, les mêmes mots correspondent fort souvent à des idées extrêmement dissemblables. Il est impossible de comprendre ces différences sans de nombreux voyages, et c’est pourquoi je ne saurais insister sur elles. Je me bornerai à faire remarquer que ce sont précisément les mots les plus employés par les foules qui d’un peuple à l’autre possèdent les sens les plus différents. Tels sont par exemple les mots de démocratie et de socialisme, d’un usage si fréquent aujourd’hui.\par
Ils correspondent en réalité à des idées et des images tout à fait opposées dans les âmes latines et dans les âmes anglo-saxonnes. Chez les Latins le mot démocratie, signifie surtout effacement de la volonté et de l’initiative de l’individu devant celles de la communauté représentées par l’État. C’est l’État qui est chargé de plus en plus de diriger tout, de centraliser, de monopoliser et de fabriquer tout. C’est à lui que tous les partis sans exception, radicaux, socialistes ou monarchistes, font constamment appel. Chez l’Anglosaxon, celui d’Amérique notamment, le même mot démocratie signifie au contraire développement intense de la volonté et de l’individu, effacement aussi complet. que possible de l’État, auquel en dehors de la police, de l’armée et des rela­tions diplomatiques, on ne laisse rien diriger, pas même l’instruction. Donc le même mot qui signifie, chez un peuple, effacement de la volonté et de l’initiative individu­elle et prépondérance de l’État, signifie chez un autre développement excessif de cette volonté, de cette initiative, et effacement complet de l’État \footnote{Dans \emph{Les Lois psychologiques de l’évolution des peuples}, j’ai longuement insisté sur la différence qui sépare l’idéal démocratique latin de l’idéal démocratique anglo-saxon.}, c’est-à-dire possède un sens absolument contraire.
\subsubsection[{§ 2. – Les illusions}]{§ 2. – Les illusions}
\noindent Depuis l’aurore des civilisations les foules ont toujours subi l’influence des illu­sions. C’est aux créateurs d’illusions qu’elles ont élevé le plus de temples, de statues et d’autels. Illusions religieuses jadis, illusions philosophiques et sociales aujourd’hui, on retrouve toujours ces formidables souveraines à la tête de toutes les civilisations qui ont successivement fleuri sur notre planète. C’est en leur nom que se sont édifiés les temples de la Chaldée et de l’Égypte, les édifices religieux du moyen âge, que l’Europe entière a été bouleversée il y a un siècle, et il n’est pas une seule de nos conceptions artistiques, politiques ou sociales qui ne porte leur puissante empreinte. L’homme les renverse parfois, au prix de bouleversements effroyables, mais il semble condamné à les relever toujours. Sans elles il n’aurait pu sortir de la barbarie primi­tive, et sans elles encore il y retomberait bientôt. Ce sont des ombres vaines, sans doute ; mais ces filles de nos rêves ont obligé les peuples à créer tout ce qui fait la splendeur des arts et la grandeur des civilisations.\par
“Si l’on détruisait, dans les musées et les bibliothèques, et que l’on fit écrouler, sur les dalles des parvis, toutes les œuvres et tous les monuments d’art qu’ont inspirés les religions, que resterait-il des grands rêves humains ? écrit un auteur qui résume nos doctrines. Donner aux hommes la part d’espoir et d’illusion sans laquelle ils ne peuvent exister, telle est la raison d’être des dieux, des héros et des poètes.\par
Pendant cinquante ans, la science parut assumer cette tâche. Mais ce qui l’a com­promise dans les cœurs affamés d’idéal, c’est qu’elle n’ose plus assez promettre et qu’elle ne sait pas assez mentir.”\par
Les philosophes du dernier siècle se sont consacrés avec ferveur à détruire les illusions religieuses, politiques et sociales dont, pendant de longs siècles, avaient vécu nos pères. En les détruisant ils ont tari les sources de l’espérance et de la résignation. Derrière les chimères immolées, ils ont trouvé les forces aveugles et sourdes de la nature. Inexorables pour la faiblesse elles ne connaissent pas la pitié.\par
Avec tous ses progrès la philosophie n’a pu encore offrir aux foules aucun idéal qui les puisse charmer ; mais, comme il leur faut des illusions à tout prix, elles se dirigent d’instinct, comme l’insecte allant à la lumière, vers les rhéteurs qui leur en présentent. Le grand facteur de l’évolution des peuples n’a jamais été la vérité, mais bien l’erreur. Et si le socialisme est si puissant aujourd’hui, c’est qu’il constitue la seule illusion qui soit vivante encore. Malgré toutes les démonstrations scientifiques, il continue à grandir. Sa principale force est d’être défendu par des esprits ignorant assez les réalités des choses pour oser promettre hardiment à l’homme le bonheur. L’illusion sociale règne aujourd’hui sur toutes les ruines amoncelées du passé, et l’avenir lui appartient. Les foules n’ont jamais eu soif de vérités. Devant les évidences qui leur déplaisent, elles se détournent, préférant déifier l’erreur, si l’erreur les séduit. Qui sait les illusionner est aisément leur maître ; qui tente de les désillusionner est toujours leur victime.
\subsubsection[{§ 3. – L’expérience}]{§ 3. – L’expérience}
\noindent L’expérience constitue à peu près le seul procédé efficace pour établir solidement une vérité dans l’âme des foules, et détruire des illusions devenues trop dangereuses. Encore est-il nécessaire que l’expérience soit réalisée sur une très large échelle et fort souvent répétée. Les expériences faites par une génération sont généralement inutiles pour la suivante ; et c’est pour­quoi les faits historiques invoqués comme éléments de démonstration ne sauraient servir. Leur seule utilité est de prouver à quel point les expériences doivent être répétées d’âge en âge pour exercer quelque influence, et réussir à ébranler seulement une erreur lorsqu’elle est solidement implantée dans l’âme des foules.\par
Notre siècle, et celui qui l’a précédé, seront cités sans doute par des historiens de l’avenir comme une ère de curieuses expériences. A aucun âge il n’en avait été autant tenté.\par
La plus gigantesque de ces expériences fut la Révolution française. Pour décou­vrir qu’on ne refait pas une société de toutes pièces sur les indications de la raison pure, il a fallu massacrer plusieurs millions d’hommes et bouleverser l’Europe entière pendant vingt ans. Pour nous prouver expérimentalement que les Césars coûtent cher aux peuples qui les acclament, il a fallu deux ruineuses expériences en cinquante ans, et, malgré leur clarté, elles ne semblent pas avoir été suffisamment convaincantes. La première a coûté pourtant trois millions d’hommes et une invasion, la seconde un démembrement et la nécessité des armées permanentes. Une troisième a failli être tentée il n’y a pas longtemps et le sera sûrement un jour. Pour faire admettre à tout un peuple que l’immense armée allemande n’était pas, comme on l’enseignait avant 1870, une sort de garde nationale inoffensive \footnote{L’opinion des foules était formée, dans ce cas, par ces associations grossières de choses dissem­blables dont j’ai pré­cédemment exposé le mécanisme. Notre garde nationale d’alors, étant composée de pacifiques boutiquiers sans trace de discipline, et ne pouvant être prise au sérieux, tout ce qui portait un nom analogue éveillait les mêmes images, et était considéré par conséquent comme aussi inoffensif. L’erreur des foules était partagée alors, ainsi que cela arrive si sou­vent pour les opinions générales, par leurs meneurs. Dans un discours prononcé le 31 décembre 1867 à la chambre des députés, et reproduit par M. E. Ollivier dans un livre récent, un homme d’État qui a bien souvent suivi l’opinion des foules, mais ne l’a jamais précédée, M. Thiers, répétait que la Prusse, en dehors d’une armée active à peu près égale en nombre à la nôtre, ne possédait qu’une garde nationale ana­logue à celle que nous possédions et par conséquent sans importance ; asser­tions aussi exactes que les prévisions du même homme d’État sur le peu d’avenir des chemins de fer.}, il a fallu l’effroyable guerre qui nous a coûté si cher. Pour reconnaître que le protectionnisme ruine les peuples qui l’acceptent, il faudra au moins vingt ans de désastreuses expériences. On pourrait multiplier indéfi­niment ces exemples.
\subsubsection[{§ 4. – La raison}]{§ 4. – La raison}
\noindent Dans l’énumération des facteurs capables d’impressionner l’âme des foules, on pourrait se dispenser entièrement de mentionner la raison, s’il n’était nécessaire d’indiquer la valeur négative de son influence.\par
Nous avons déjà montré que les foules ne sont pas influençables par des raison­nements, et ne comprennent que de grossières associations d’idées. Aussi est-ce à leurs sentiments et jamais à leur raison que font appel les orateurs qui savent les impressionner. Les lois de la logique n’ont aucune action sur elles \footnote{Mes premières observations sur l’art d’impressionner les foules et sur les faibles ressources qu’offrent sur ce point les règles de la logique remontent à l’époque du siège de Paris, le jour où je vis conduire au Louvre, où siégeait alors le gouvernement, le maréchal V…, qu’une foule furieuse prétendait avoir surpris levant le plan des fortifications pour le vendre aux Prussiens. Un membre du gouvernement, G.P…, orateur fort célèbre, sortit pour haranguer la foule qui récla­mait l’exécution immédiate du prisonnier. Je m’attendais à ce que l’orateur démontrât l’absurdité de l’accusation, en disant que le maréchal accusé était précisément un des constructeurs de ces fortifi­cations dont le plan se vendait d’ailleurs chez tous les libraires. A ma grande stupéfaction – j’étais fort jeune alors – le discours fut tout autre… “Justice sera faite, cria l’orateur en s’avançant vers le prisonnier, et une justice impitoyable. Laissez le gouvernement de la défense nationale terminer votre enquête. Nous allons, en attendant, enfermer l’accusé.” Calmée aussitôt par cette satisfaction apparente, la foule s’écoula, et au bout d’un quart d’heure le maréchal put regagner son domicile. Il eût été infaillible­ment écharpé si l’orateur eût tenu à la foule en fureur les raisonnements logiques que ma grande jeunesse me faisaient trouver très convaincants.}. Pour convaincre les foules, il faut d’abord se rendre bien compte des sentiments dont elles sont animées, feindre de les partager, puis tenter de les modifier, en provoquant, au moyen d’associations rudimentaires, certaines images bien suggestives ; savoir revenir au besoin sur ses pas, deviner surtout à chaque instant les sentiments qu’on fait naître. Cette nécessité de varier sans cesse son langage suivant l’effet produit à l’instant où l’on parle frappe d’avance d’impuissance tout discours étudié et préparé : l’orateur y suit sa pensée et non celle de ses auditeurs, et, par ce seul fait, son influence devient parfaitement nulle.\par
Les esprits logiques, habitués à être convaincus par des chaînes de raisonnements un peu serrées, ne peuvent s’empêcher d’avoir recours à ce mode de persuasion quand ils s’adressent aux foules, et le manque d’effet de leurs arguments les surprend toujours. “Les conséquences mathématiques usuelles fondées sur le syllogisme, c’est-à-dire sur des associations d’identités, écrit un logicien, sont nécessaires… La néces­sité forcerait l’assentiment même d’une masse inorganique, si celle-ci était capable de suivre des associations d’identités. Sans doute ; mais la foule n’est pas plus capable que la masse inorganique de les suivre, ni même de les entendre. Qu’on essaie de convaincre par un raisonnement des esprits primitifs, des sauvages ou des enfants, par exemple, et l’on se rendra compte de la faible valeur que possède ce mode d’argu­mentation.\par
Il n’est même pas besoin de descendre jusqu’aux êtres primitifs pour voir la complète impuissance des raisonnements quand ils ont à lutter contre des sentiments. Rappelons-nous simplement combien ont été tenaces pendant de longs siècles des superstitions religieuses, contraires à la plus simple logique. Pendant près de deux mille ans les plus lumineux génies ont été courbés sous leurs lois, et il a fallu arriver aux temps modernes pour que leur véracité ait pu seulement être contestée. Le moyen-âge et la Renaissance ont possédé bien des hommes éclairés ; ils n’en ont pas possédé un seul auquel le raisonnement ait montré les côtés enfantins de ses supers­titions, et fait naître un faible doute sur les méfaits du diable ou sur la nécessité de brûler les sorciers.\par
Faut-il regretter que ce ne soit jamais la raison qui guide les foules ? Nous n’oserions le dire. La raison humaine n’eût pas réussi sans doute à entraîner l’huma­nité dans les voies de la civilisation avec l’ardeur et la hardiesse dont l’ont soulevée ses chimères. Filles de l’inconscient qui nous mène, ces chimères étaient sans doute nécessaires. Chaque race porte dans sa constitution mentale les lois de ses destinées, et c’est peut-être à ces lois qu’elle obéit par un inéluctable instinct, même dans ses impulsions en apparence les plus irraisonnées. Il semble parfois que les peuples soient sou­mis à des forces secrètes analogues à celles qui obligent le gland à se transformer en chêne ou la comète à suivre son orbite.\par
Le peu que nous pouvons pressentir de ces forces doit être cherché dans la marche générale de l’évolution d’un peuple et non dans les faits isolés d’où cette évolution semble parfois surgir. Si l’on ne considérait que ces faits isolés l’histoire semblerait régie par d’invraisemblables hasards. Il était invraisemblable qu’un ignorant char­pentier de Galilée pût devenir pendant deux mille ans un dieu tout-puissant, au nom duquel fussent fondées les plus importantes civilisations ; invraisemblable aussi que quelques bandes d’Arabes sortis de leurs déserts pussent conquérir la plus grande partie du vieux monde gréco-romain, et fonder un empire plus grand que celui d’Alexandre ; invraisemblable encore que, dans une Europe très vieille et très hiérar­chisée, un obscur lieutenant d’artillerie pût réussir à régner sur une foule de peuples et de rois.\par
Laissons donc la raison aux philosophes, mais ne lui demandons pas trop d’inter­venir dans le gouvernement des hommes. Ce n’est pas avec la raison et c’est le plus souvent malgré elle, que se sont créés des sentiments tels que l’honneur, l’abnégation. la foi religieuse, l’amour de la gloire et de la patrie, qui ont été jusqu’ici les grands ressorts de toutes les civilisations.
\subsection[{Chapitre 3. Les meneurs des fouleset leurs moyens de persuasion}]{Chapitre 3. Les meneurs des fouleset leurs moyens de persuasion}

\begin{argument}\noindent § 1.\emph{ Les meneurs des foules. –} Besoin instinctif de tous les êtres en foule d’obéir à un meneur. – Psychologie des meneurs. – Eux seuls peuvent créer la foi et donner une organisation aux foules. – Despotisme forcé des meneurs. – Classification des meneurs. – Rôle de la volonté. – § 2. \emph{Les moyens d’action des meneurs. –} L’affirmation, la répé­tition, la contagion. – Rôle respectif de ces divers facteurs. – Comment la contagion peut remonter des couches inférieures aux couches supérieures d’une société. – Une opinion populaire devient bientôt une opinion générale. – § 3. \emph{Le prestige. –} Définition et classification du pres­tige. – Le prestige acquis et le prestige personnel. – Exemples divers. – Comment meurt le prestige.
\end{argument}

\noindent La constitution mentale des foules nous est mainte­nant connue, et nous savons aussi quels sont les mobiles capables d’impressionner leur âme. Il nous reste à recher­cher comment doivent être appliqués ces mobiles, et par qui ils peuvent être utilement mis en œuvre.\par
\subsubsection[{§ 1. – Les meneurs des foules}]{§ 1. – Les meneurs des foules}
\noindent Dès qu’un certain nombre d’êtres vivants sont réunis, qu’il s’agisse d’un troupeau d’animaux ou d’une foule d’hommes, ils se placent d’instinct sous l’autorité d’un chef.\par
Dans les foules humaines, le chef réel n’est souvent qu’un meneur, mais, comme tel, il joue un rôle considérable. Sa volonté est le noyau autour duquel se forment et s’identifient les opinions. Il constitue le premier élément d’organisation des foules hétérogènes et prépare leur organisation en sectes. En attendant, il les dirige. La foule est un troupeau servile qui ne saurait jamais se passer de maître.\par
Le meneur a d’abord été le plus souvent un mené. Il a lui-même été hypnotisé par l’idée dont il est ensuite devenu l’apôtre. Elle l’a envahi au point que tout disparaît en dehors d’elle, et que toute opinion contraire lui parait erreur et superstition. Tel, par exemple, Robespierre, hypnotisé par les idées philosophiques de Rousseau, et employant les procédés de l’inquisition pour les propager.\par
Les meneurs ne sont pas le plus souvent des hommes de pensée, mais des hom­mes d’action. Ils sont peu clairvoyants, et ne pourraient l’être, la clairvoyance condui­sant généralement au doute et à l’inaction. Ils se recrutent surtout parmi ces névrosés, ces excités, ces demi-aliénés qui côtoient les bords de la folie. Quelque absurde que puisse être l’idée qu’ils défendent ou le but qu’ils poursuivent, tout raisonnement s’émousse contre leur conviction. Le mépris et les persécutions ne les touchent pas, ou ne font que les exciter davantage. Intérêt personnel, famille, tout est sacrifié. L’ins­tinct de la conservation lui-même est annulé chez eux, au point que la seule récom­pense qu’ils sollicitent souvent est de devenir des martyrs. L’intensité de leur foi donne à leurs paroles une grande puissance suggestive. La multitude est toujours prête à écouter l’homme doué de volonté forte qui sait s’imposer à elle. Les hommes réunis en foule perdent toute volonté et se tournent d’instinct vers qui en possède une.\par
De meneurs, les peuples n’ont jamais manqué : mais il s’en faut que tous soient animés des convictions fortes qui font les apôtres. Ce sont souvent des rhéteurs subtils, ne poursuivant que des intérêts personnels et cherchant à persuader en flattant de bas instincts. L’influence qu’ils exercent ainsi peut être très grande, mais elle reste toujours très éphémère. Les grands convaincus qui ont soulevé l’âme des foules, les Pierre l’Ermite, les Luther, les Savonarole, les hommes de la Révolution, n’ont exercé de fascination qu’après avoir été eux mêmes d’abord fascinés par une croyance. Ils purent alors créer dans les âmes cette puissance formidable nommée la foi, qui rend l’homme esclave absolu de son rêve.\par
Créer la foi, qu’il s’agisse de foi religieuse, de foi politique ou sociale, de foi en une œuvre, en un personnage, en une idée, tel est surtout le rôle des grands meneurs, et c’est pourquoi leur influence est toujours considérable. De toutes les forces dont l’humanité dispose, la foi a toujours été une des plus grandes, et c’est avec raison que l’Évangile lui attribue le pouvoir de transporter les montagnes. Donner à l’homme une foi, c’est décupler sa force. Les grands événements de l’histoire ont été réalisés par d’obscurs croyants n’ayant guère que leur foi pour eux. Ce n’est pas avec des lettrés et des philosophes, ni surtout avec des sceptiques qu’ont été édifiées les grandes religions qui ont gouverné le monde, ni les vastes empires qui se sont étendus d’un hémisphère à l’autre.\par
Mais, dans de tels exemples, il s’agit des grands meneurs, et ils sont assez rares pour que l’histoire en puisse aisément marquer le nombre. Ils forment le sommet d’une série continue descendant de ces puissants manieurs d’hommes à l’ouvrier qui, dans une auberge fumeuse, fascine lentement ses camarades en remâchant sans cesse quelques formules qu’il ne comprend guère, mais dont, selon lui, l’application doit amener sûrement la réalisation de tous les rêves et de toutes les espé­ances.\par
Dans toutes les sphères sociales, des plus hautes aux plus basses, dès que l’homme n’est plus isolé, il tombe bientôt sous la loi d’un meneur. La plupart des hommes, dans les masses populaires surtout, ne possèdent, en dehors de leur spécialité, d’idée nette et raisonnée sur quoi que ce soit. Ils sont incapables de se conduire. Le meneur leur sert de guide. il peut être remplacé à la rigueur, mais très insuffisamment par ces pu­blications périodiques qui fabriquent des opinions pour leurs lecteurs et leur procurent ces phrases toutes faites qui dispensent de raisonner.\par
L’autorité des meneurs est très despotique, et n’arrive même à s’imposer qu’à cause de ce despotisme. On a remarqué souvent combien facilement ils se faisaient obéir, bien que n’ayant aucun moyen d’appuyer leur autorité, dans les couches ouvrières les plus turbulentes. Ils fixent les heures de travail, le taux des salaires, décident les grèves, les font commencer et cesser à heure fixe.\par
Les meneurs tendent aujourd’hui à remplacer de plus en plus les pouvoirs publics à mesure que. ces derniers se laissent discuter et affaiblir. La tyrannie de ces nouveaux maîtres fait que les foules leur obéissent beaucoup plus docilement qu’elles n’ont obéi à aucun gouvernement. Si, par suite d’un accident quelconque, le meneur disparaît et n’est pas immédiatement remplacé, la foule redevient une collectivité sans cohésion ni résistance. Pendant une des grèves des employés des omnibus à Paris, il a suffi d’arrêter les deux meneurs qui la dirigeaient pour la faire aussitôt cesser. Ce n’est pas le besoin de la liberté, mais celui de la servitude qui domine toujours dans l’âme des foules. Elles ont une telle soif d’obéir qu’elles se soumettent d’instinct à qui se déclare leur maître.\par
On peut établir une division assez tranchée dans la classe des meneurs. Les uns sont des hommes énergiques, à volonté forte, mais momentanée ; les autres, beau­coup plus rares que les précédents, sont des hommes possédant une volonté à la fois forte et durable. Les premiers sont violents, braves, hardis. Ils sont utiles surtout pour diriger un coup de main, entraîner les masses malgré le danger, et transformer en hé­ros les recrues de la veille. Tels, par exemple, Ney et Murat, sous le premier Empire. Tel encore, de nos jours, Garibaldi, aventurier sans talent, mais énergique, réussissant avec une poignée d’hommes à s’emparer de l’ancien royaume de Naples défendu pourtant par une armée disciplinée.\par
Mais si l’énergie de ces meneurs est puissante, elle est momentanée et ne survit guère à l’excitant qui l’a fait naître. Rentrés dans le courant de la vie ordinaire, les héros qui en étaient animés font souvent preuve, comme ceux que je citais à l’instant, de la plus étonnante faiblesse. Ils semblent incapables de réfléchir et de se conduire dans les circonstances les plu simples, alors qu’ils avaient si bien su conduire les autres. Ce sont des meneurs qui ne peuvent exercer leur fonction qu’à la condition d’être menés eux-mêmes et excités sans cesse, d’avoir toujours au-dessus d’eux un homme ou une idée, de suivre une ligne de conduite bien tracée.\par
La seconde catégorie des meneurs, celle des hommes à volonté durable, a, malgré des formes moins brillantes, une influence beaucoup plus considérable. En elle on trouve les vrais fondateurs de religions ou de grandes œuvres : saint Paul, Mahomet, Christophe Colomb, Lesseps. Qu’ils soient intelligents ou bornés, il n’importe, le monde sera toujours à eux. La volonté persistante qu’ils possèdent est une faculté infiniment rare et infiniment puissante qui fait tout plier. On ne se rend pas toujours suffisamment compte de ce que peut une volonté forte et continue : rien ne lui résiste, ni la nature, ni les dieux, ni les hommes.\par
Le plus récent exemple de ce que peut une volonté forte et continue, nous est don­né par l’homme illustre qui sépara deux mondes et réalisa la tâche inutilement tentée depuis trois mille ans par les plus grands souverains. Il échoua plus tard dans une entreprise identique ; mais la vieillesse était venue, et tout s’éteint devant elle, même la volonté.\par
Lorsqu’on voudra montrer ce que peut la seule volonté, il n’y aura qu’à présenter dans ses détails l’histoire des difficultés qu’il fallut surmonter pour creuser le canal de Suez. Un témoin oculaire, le docteur Cazalis, a résumé en quelques lignes saisissantes la synthèse de cette grande œuvre racontée par son immortel auteur. “Et il contait, de jour en jour, par épisodes, l’épopée du canal. Il contait tout ce qu’il avait dû vaincre, tout l’impossible qu’il avait fait possible, toutes les résistances, les coalitions contre lui, et les déboires, les revers, les défaites, mais qui n’avaient pu jamais le décourager, ni l’abattre ; il rappelait l’Angleterre le combattant, l’attaquant sans relâche, et l’Égypte et la France hésitantes, et le consul de France s’opposant plus que tout autre aux premiers travaux, et comme on lui résistait, prenant les ouvriers par la soif, leur faisant refuser l’eau douce ; et le ministère de la marine et les ingénieurs, tous les hommes sérieux, d’expérience et de science, tous naturellement hostiles, et tous scien­tifiquement assurés du désastre, le calculant et le promettant, comme pour tel jour ou telle heure on promet l’éclipse.”\par
Le livre qui raconterait la vie de tous ces grands meneurs ne contiendrait pas beaucoup de noms ; mais ces noms ont été à la tête des événements les plus impor­tants de la civilisation et de l’histoire.
\subsubsection[{§ 2. – Les moyens d’action des meneurs ;l’affirmation, la répétition, la contagion.}]{§ 2. – Les moyens d’action des meneurs ;l’affirmation, la répétition, la contagion.}
\noindent Lorsqu’il s’agit d’entraîner une foule pour un instant, et de la déterminer à com­mettre un acte quelconque piller un palais, se faire massacrer pour défendre une place forte ou une barricade, il faut agir sur elle par des suggestions rapides, dont la plus énergique est encore l’exemple ; mais il faut alors que la foule soit déjà préparée par certaines circonstances, et surtout que celui qui veut l’entraîner possède la qualité que j’étudierai plus loin sous le nom de prestige.\par
Mais quand il s’agit de faire pénétrer des idées et des croyances dans l’esprit des foules – les théories sociales modernes, par exemple – les procédés des meneurs sont différents. Ils ont principalement recours à trois procédés très nets : l’affirmation, la répétition, la contagion. L’action en est assez lente, mais les effets de cette action une fois produits sont fort durables.\par
L’affirmation pure et simple, dégagée de tout raisonnement et de toute preuve, est un des plus sûrs moyens de faire pénétrer une idée dans l’esprit des foules. Plus l’affirmation est concise, plus elle est dépourvue de toute apparence de preuves et de démonstration, plus elle a d’autorité. Les livres religieux et les codes de tous les âges ont toujours procédé par simple affirmation. Les hommes d’État appelés à défendre une cause politique quelconque, les industriels propageant leurs produits par l’annon­ce, savent la valeur de l’affirmation.\par
L’affirmation n’a cependant d’influence réelle qu’à la condition d’être constamment répétée, et, le plus possible, dans les mêmes termes. C’est Napoléon, je crois, qui a dit qu’il n’y a qu’une seule figure sérieuse de rhétorique, la répétition. La chose affirmée arrive, par la répétition, à s’établir dans les esprits au point qu’ils finissent par l’accep­ter comme une vérité démontrée.\par
On comprend bien l’influence de la répétition sur les foules, en voyant à quel point elle est puissante sur les esprits les plus éclairés. Cette puissance vient de ce que la chose répétée finit par s’incruster dans ces régions profondes de l’inconscient où s’élaborent les motifs de nos actions. Au bout de quelque temps, nous ne savons plus quel est l’auteur de l’assertion répétée, et nous finissons par y croire. De là la force étonnante de l’annonce. Quand nous avons lu cent fois, mille fois que le meilleur chocolat est le chocolat X, nous nous imaginons l’avoir entendu dire de bien des côtés, et nous finissons par en avoir la certitude. Quand nous avons lu mille fois que la farine Y a guéri les plus grands personnages des maladies les plus tenaces, nous finissons être tentés de l’essayer le jour où nous sommes par atteints d’une maladie du même genre. Si nous lisons toujours dans le même journal que A est un parfait gre­din et B un très honnête homme, nous finissons par en être convaincus, à moins, bien entendu, que nous ne lisions souvent un autre journal d’opinion contraire, ou les deux qualificatifs soient inversés. L’affirmation et la répétition sont seules assez puissantes pour pouvoir se combattre.\par
Lorsqu’une affirmation a été suffisamment répétée, et qu’il y a unanimité dans la répétition, comme cela est arrivé pour certaines entreprises financières célèbres assez riches pour acheter tous les concours, il se forme ce qu’on appelle un courant d’opi­nion et le puissant mécanisme de la contagion intervient. Dans les foules, les idées, les sentiments, les émotions, les croyances possèdent un pouvoir contagieux aussi intense que celui des microbes. Ce phénomène est très naturel puisqu’on l’observe chez les animaux eux-mêmes dès qu’il sont en foule. Le tic d’un cheval dans une écurie est bientôt imité par les autres chevaux de la même écurie. Une panique, un mouvement désordonné de quelques mou­tons s’étend bientôt à tout le troupeau. Chez l’homme en foule toutes les émotions sont très rapidement contagieuses, et c’est ce qui explique la soudaineté des paniques. Les désordres cérébraux, comme la folie, sont eux-mêmes contagieux. On sait combien est fréquente l’aliénation chez les médecins aliénistes. On a même cité récemment des formes de folie, l’agoraphobie par exemple, communiquées de l’homme aux animaux.\par
La contagion n’exige pas la présence simultanée d’individus sur un seul point ; elle peut se faire à distance sous l’influence de certains événements qui orientent tous les esprits dans le même sens et leur donnent les caractères spéciaux aux foules, surtout quand les esprits sont préparés par les facteurs lointains que j’ai étudiés plus haut. C’est ainsi par exemple que l’explosion révolutionnaire de 1848, partie de Paris, s’étendit brusquement à une grande partie de l’Europe et ébranla plusieurs monar­chies.\par
L’imitation, à laquelle on a attribué tant d’influence dans les phénomènes sociaux, n’est en réalité qu’un simple effet de la contagion. Ayant montré ailleurs son influence je me bornerai à reproduire ce que j’en disais il y a plus de vingt ans et qui depuis a été développé par d’autres écrivains dans des publications récentes :\par
“Semblable aux animaux, l’homme est naturellement imitatif. L’imitation est un besoin pour lui, à condition bien entendu, que cette imitation soit tout à fait facile. C’est ce besoin qui rend si puissante l’influence de ce que nous appelons la mode. Qu’il s’agisse d’opinions, d’idées, de manifestations littéraires, ou simplement de costumes, combien osent se soustraire à son empire ? Ce n’est pas avec des argu­ments, mais avec des modèles, qu’on guide les foules. A chaque époque il y a un petit nombre d’individualités qui impriment leur action et que la masse inconsciente imite. Il ne faudrait pas cependant que ces individualités s’écartassent par trop des idées reçues. Les imiter serait alors trop difficile et leur influence serait nulle. C’est préci­sément pour cette raison que les hommes trop supérieurs à leur époque n’ont généralement aucune influence sur elle. L’écart est trop grand. C’est pour la même raison que les Européens, avec tous les avantages de leur civilisation, ont une influence si insignifiante sur les peuples de l’Orient ils en diffèrent trop.\par
“La double action du passé et de l’imitation réciproque finit par rendre tous les hommes d’un même pays et d’une même époque à ce point semblables que, même chez ceux qui sembleraient devoir le plus s’y soustraire, philosophes, savants et littérateurs, la pensée et le style ont un air de famille qui fait immédiatement recon­naître le temps auquel ils appartiennent. Il ne faut pas causer longtemps avec un individu pour connaître à fond ses lectures, ses occupations habituelles et le milieu où il vit \footnote{GUSTAVE LE BON. \emph{L’homme et} les \emph{Sociétés}, t. II, p. 116, 1881.}.”\par
La contagion est si puissante qu’elle impose aux individus non seulement certai­nes opinions mais encore certaines façons de sentir. C’est la contagion qui fait mépriser à une époque certaines œuvres, telles que le \emph{Tanhauser}, par exemple, et qui, quelques années plus tard, les fait admirer par ceux-là mêmes qui les avaient déni­grées le plus.\par
C’est surtout par le mécanisme de la contagion, jamais par celui du raisonnement, que se propagent les opinions et les croyances des foules. C’est au cabaret, par affirmation, répétition et contagion que s’établissent les conceptions actuelles des ouvriers ; et les croyances des foules de tous les âges ne se sont guère créées autre­ment. Renan compare avec justesse les premiers fondateurs du christianisme “aux ouvriers socialistes répandant leurs idées de cabaret en cabaret” ; et Voltaire avait déjà fait observer à propos de la religion chrétienne que “la plus vile canaille l’avait seule embrassée pendant plus de cent ans”.\par
On remarquera que, dans les exemples analogues à ceux que je viens de citer, la contagion, après s’être exercée dans les couches populaires, passe ensuite aux couches supérieures de la société. C’est ce que nous voyons de nos jours pour les doctrines socialistes, qui commencent à gagner ceux qui pourtant sont marqués pour en devenir les premières victimes. Le mécanisme de la contagion est si puissant que, devant son action, l’intérêt personnel lui-même s’évanouit.\par
Et c’est pourquoi toute opinion devenue populaire finit toujours par s’imposer avec une grande force aux couches sociales les plus élevées, quelque visible que puisse être l’absurdité de l’opinion triomphante. Il y a là une réaction des couches sociales inférieures sur les couches supérieures d’autant plus curieuse que les croyances de la foule dérivent toujours plus ou moins de quelque idée supérieure restée souvent sans influence dans le milieu où elle avait pris naissance. Cette idée supérieure, les me­neurs subjugués par elle s’en emparent, la déforment et créent une secte qui la déforme de nouveau, puis la répand dans le sein des foules qui continuent à la défor­mer de plus en plus.\par
Devenue vérité populaire, elle remonte en quelque façon à sa source et agit alors sur les couches supérieures d’une nation. C’est en définitive l’intelligence qui guide le monde, mais elle le guide vraiment de fort loin. Les philosophes qui créent les idées sont depuis bien longtemps retournés à la poussière, lorsque, par l’effet du mécanisme que je viens de décrire, leur pensée finit par triompher.
\subsubsection[{§ 3. – Le prestige}]{§ 3. – Le prestige}
\noindent Ce, qui contribue surtout à donner aux idées propagées par l’affirmation, la répéti­tion et la contagion, une puissance très grande, c’est qu’elles finissent par acquérir le pouvoir mystérieux nommé prestige.\par
Tout ce qui a dominé dans le monde, les idées ou les hommes, s’est imposé principalement par cette force irrésistible qu’exprime le mot prestige. C’est un terme dont nous saisissons tous le sens, mais qu’on applique de façons trop diverses pour qu’il soit facile de le définir. Le prestige peut comporter certains sentiments tels que l’admiration ou la crainte ; il lui arrive parfois même de les avoir pour base, mais il peut parfaitement exister sans eux. Ce sont des morts, et par conséquent des êtres que nous ne craignons pas, Alexandre, César, Mahomet, Bouddha, par exemple, qui possèdent le plus de prestige. D’un autre côté, il y a des êtres ou des fictions que nous n’admirons pas, les divinités monstrueuses des temples souterrains de l’Inde, par exemple, et qui nous paraissent pourtant revêtues d’un grand prestige.\par
Le prestige est en réalité une sorte de domination qu’exerce sur notre esprit un individu, une œuvre, ou une idée. Cette domination paralyse toutes nos facultés criti­ques et remplit notre âme d’étonnement et de respect. Le sentiment provoqué est inexplicable, comme tous les sentiments, mais il doit être du même ordre que la fasci­nation subie par un sujet magnétisé. Le prestige est le plus puissant ressort de toute domination. Les dieux, les rois et les femmes n’auraient jamais régné sans lui.\par
On peut ramener à deux formes principales les diverses variétés de prestige : le prestige acquis et le prestige personnel. Le prestige acquis est celui que, donnent le nom, la fortune, la réputation. Il peut être indépendant du prestige personnel. Le pres­tige personnel est au contraire quelque chose d’individuel qui peut coexister avec la réputation, la gloire, la fortune, ou être renforcé par elles, mais qui peut parfaitement exister sans elles.\par
Le prestige acquis, ou artificiel, est de beaucoup le plus répandu. Par le fait seul qu’un individu occupe une certaine position, possède une certaine fortune, est affublé de certains titres, il a du prestige, quelque nulle que puisse être sa valeur personnelle. Un militaire en uniforme, un magistrat en robe rouge ont toujours du prestige. Pascal avait très justement noté la nécessité pour les juges des robes et des perruques. Sans elles ils perdraient les trois quarts de leur autorité. Le socialiste le plus farouche est toujours un peu émotionné par la vue d’un prince ou d’un marquis ; et il suffit de prendre de tels titres pour escroquer à un commerçant tout ce qu’on veut \footnote{ \noindent Cette influence des titres, des rubans, des uniformes sur les foules se rencontre dans tous les pays, même dans ceux où le sentiment de l’indépendance personnelle est le plus développé. Je reproduis à ce propos un passage curieux du livre récent d’un voyageur sur le prestige de certains personnages en Angleterre.\par
 “En diverses rencontres, je ne m’étais aperçu de l’ivresse particulière à laquelle le contact ou la vue d’un pair d’Angleterre exposent les Anglais les plus raisonnables.\par
 “Pourvu que son état soutienne son rang, ils l’aiment d’avance, et mis en présence supportent tout de lui avec enchantement. On les voit rougir de plaisir à son approche et, s’il leur parle, la joie qu’ils contiennent augmente cette rougeur et fait briller leurs yeux d’un éclat inaccoutumé. Ils ont le lord dans le sang, si l’on peut dire, comme l’Espa­gnol la danse, l’Allemand la musique et le Français la Révo­lution. Leur passion pour les chevaux et Shakspeare est moins violente, la satisfaction et l’orgueil qu’ils en tirent moins fondamentaux. Le Livre de la Pairie a un débit considérable, et si loin qu’on aille, on le trouve, comme la Bible, entre toutes les mains.
}.\par
Le prestige dont je viens de parler est celui qu’exercent les personnes ; on peut placer à côté le prestige qu’exercent les opinions, les œuvres littéraires ou artistiques, etc. Ce n’est le plus souvent que de la répétition accumulée. L’histoire, l’histoire littéraire et artistique surtout, n’étant que la répétition des mêmes jugements que personne n’essaie de contrôler, chacun finit par répéter ce qu’il a appris à l’école, et il y a des noms et des choses auxquels nul n’oserait toucher. Pour un lecteur moderne, l’œuvre d’Homère dégage un incontestable et immense ennui mais qui oserait le dire ? Le Parthénon, dans son état actuel, est une ruine dépourvue d’intérêt ; mais il possède un tel prestige qu’on ne le voit plus qu’avec tout son cortège de souvenirs historiques. Le propre du prestige est d’empêcher de voir les choses telles qu’elles sont et de paralyser tous nos jugements. Les foules toujours, les individus le plus souvent, ont besoin, sur tous les sujets, d’opinions toutes faites. Le succès de ces opinions est indépendant de la part de vérité ou d’erreur qu’elles contiennent ; il dépend unique­ment de leur prestige,\par
J’arrive maintenant au prestige personnel. Il est d’une nature fort différente du prestige artificiel ou acquis dont je viens de m’occuper. C’est une faculté indépendante de tout titre, de toute autorité, que possèdent un petit nombre de personnes, et qui leur permet d’exercer une fascination véritablement magnétique sur ceux qui les entourent, alors même qu’ils sont socialement leurs égaux et ne possèdent aucun moyen ordi­naire de domination. Ils imposent leurs idées, leurs sentiments à ceux qui les entourent, et on leur obéit comme la bête féroce obéit au dompteur qu’elle pourrait si facilement dévorer.\par
Les grands meneurs de foules, tels que Bouddha, Jésus, Mahomet, Jeanne d’Arc, Napoléon, ont possédé à un haut degré cette forme de prestige ; et c’est surtout par elle qu’ils se sont imposés. Les dieux, les héros et les dogmes s’imposent et ne se discutent pas ; ils s’évanouissent même dès qu’on les discute.\par
Les grands personnages que je viens de citer possédaient leur puissance fasci­natrice bien avant de devenir illustres, et ils ne le fussent pas devenus sans elle. Il est évident, par exemple, que Napoléon, au zénith de la gloire, exerçait, par le seul fait de sa puissance, un prestige immense ; mais ce prestige, il en était doué déjà en partie alors qu’il n’avait aucun pouvoir et était complète­ment inconnu. Lorsque, général ignoré, il fut envoyé par protection commander l’armée d’Italie, il tomba au milieu de rudes généraux qui s’apprêtaient à faire, un dur accueil au jeune intrus que le Directoire leur expédiait. Dès la première minute, dès la première entrevue, sans phrases, sans gestes, sans menaces, au premier regard du futur grand homme, ils étaient domptés. Taine donne, d’après les mémoires des contemporains, un curieux récit de cette entrevue.\par
“Les généraux de division, entre autres Augereau, sorte de soudard héroïque et grossier, fier de sa haute taille et de sa bravoure, arrivent au quartier général très mal disposés pour le petit parvenu qu’on leur expédie de Paris. Sur la description qu’on leur en a faite, Augereau est injurieux, insubordonné d’avance : un favori de Barras, un général de vendémiaire, un général de rue, regardé comme un ours, parce qu’il est toujours seul à penser, une petite mine, une réputation de mathématicien et de rêveur. On les introduit, et Bonaparte se fait attendre. Il paraît enfin, ceint de son épée, se couvre, explique ses dispositions, leur donne ses ordres et les congédie. Augereau est resté muet ; c’est dehors seulement qu’il se ressaisit et retrouve ses jurons ordinaires ; il convient, avec Masséna, que ce petit b… de général lui a fait peur ; il ne peut pas comprendre l’ascendant dont il s’est senti écrasé au premier coup d’œil.”\par
Devenu grand homme, son prestige s’accrut de toute sa gloire et devint au moins égal à celui d’une divinité pour les dévots. Le général Vandamme, soudard révo­lutionnaire, plus brutal et plus énergique encore qu’Augereau, disait de lui au maré­chal d’Ornano, en 1815, un jour qu’ils montaient ensemble l’escalier des Tuileries :\par
“Mon cher, ce diable d’homme exerce sur moi une fascination dont je ne puis me rendre compte. C’est au point que moi, qui ne crains ni dieu ni diable, quand je l’approche, je suis prêt à trembler comme un enfant, et il me ferait passer par le trou d’une aiguille pour me jeter dans le feu.”\par
Napoléon exerça la même fascination sur tous ceux qui l’approchèrent \footnote{Très conscient de son prestige, Napoléon savait qu’il l’accroissait encore en traitant un peu moins bien que des palefreniers les grands personnages qui l’entouraient, et parmi lesquels figuraient plu­sieurs de ces célèbres conventionnels qu’avait tant redoutés l’Europe. Les récits du temps sont pleins de faits significatifs sur ce point. Un jour, en plein conseil d’État, Napoléon rudoie grossiè­rement Beugnot qu’il traite comme un valet mal appris. L’effet produit, il s’approche et lui dit : “Eh bien, grand imbécile, avez-vous retrouvé votre tête ?” Là-dessus, Beugnot, haut comme un tambour-major se courbe très bas, et le petit homme, levant la main, prend le grand par l’oreille, “signe de faveur enivrante, écrit Beugnot, geste familier du maître qui s’humanise”. De tels exemples donnent une notion nette du degré de basse platitude que peut provoquer le prestige. Ils font comprendre l’immense mépris du grand despote pour les hommes qui l’entouraient et qu’il traitait simplement de chair à canon”.}.\par
Davoust disait, parlant du dévouement de Maret et du sien : “Si l’Empereur nous disait à tous deux : Il importe aux intérêts de ma politique de détruire Paris sans que personne en sorte et s’en échappe, Maret garderait le secret, j’en suis sûr, mais il ne pourrait s’empêcher de le compromettre cependant en faisant sortir sa famille. Eh bien, moi, de peur de le laisser deviner, j’y laisserais ma femme et mes enfants.”\par
Il faut se souvenir de cette étonnante puissance de fascination pour comprendre ce merveilleux retour de l’île d’Elbe ; cette conquête immédiate de la France par un homme isolé, ayant devant lui toutes les forces organisées d’un grand pays, qu’on pouvait croire lassé de sa tyrannie. Il n’eut qu’à regarder les généraux envoyés pour s’emparer de lui, et qui avaient juré de s’en emparer. Tous se soumirent sans discussion.\par
“Napoléon, écrit le général anglais Wolseley, débarque en France presque seul, et comme un fugitif, de la petite île d’Elbe qui était son royaume, et réussit en quelques semaines à bouleverser, sans effusion de sang, toute l’organisation du pouvoir de la France sous son roi légitime : l’ascendant personnel d’un homme s’affirma-t-il jamais plus étonnamment ? Mais d’un bout à l’autre de cette campagne, qui fut sa dernière, combien est remarquable l’ascendant qu’il exerçait également sur les alliés, les obligeant à suivre son initiative, et combien peu s’en fallut qu’il ne les écrasât ?”\par
Son prestige lui survécut et continua à grandir. C’est lui qui fit sacrer empereur un neveu obscur. En voyant renaître aujourd’hui sa légende, on voit combien cette grande ombre est puissante encore. Malmenez les hommes tant qu’il vous plaira, massacrez-les par mil­lions, amenez invasions sur invasions, tout vous est permis si vous possédez un degré suffisant de prestige et le talent nécessaire pour le maintenir.\par
J’ai invoqué ici un exemple de prestige tout à fait exceptionnel, sans doute, mais qu’il était utile de citer pour faire comprendre la genèse des grandes religions, des grandes doctrines et des grands empires. Sans la puissance exercée sur la foule par le prestige, cette genèse ne serait pas compréhensible.\par
Mais le prestige ne se fonde pas uniquement sur l’ascendant personnel, la gloire militaire et la terreur religieuse ; il peut avoir des origines plus modestes, et cepen­dant être considérable encore. Notre siècle en peut fournir plusieurs exemples – Un des plus frappants, celui que la postérité rappellera d’âge en âge, sera donné par l’histoire de l’homme célèbre qui modifia la face du globe et les relations commer­ciales des peuples en séparant deux continents. Il réussit dans son entreprise par son immense volonté, mais aussi parla fascination qu’il exerçait sur tous ceux qui l’entouraient. Pour vaincre l’opposition unanime qu’il rencontrait, il n’avait qu’à se montrer. Il parlait un instant, et, devant le charme qu’il exerçait, les opposants devenaient des amis. Les Anglais surtout combattaient son projet avec acharnement ; il n’eut qu’à paraître en Angleterre pour rallier tous les suffrages. Quand, plus tard, il passa par Southampton, les cloches sonnèrent sur son passage, et aujourd’hui l’Angleterre s occupe de lui élever une statue. Ayant tout vaincu, les hommes et les choses, il ne croyait plus aux obstacles et voulut recommencer Suez à Panama. Il recommença avec les mêmes moyens ; mais l’âge était venu, et, d’ailleurs, la foi qui soulève les montagnes ne les soulève qu’à la condition qu’elles ne soient pas trop hautes. Les montagnes résistèrent, et la catastrophe qui s’en suivit détruisit l’éblouis­sante auréole de gloire qui enveloppait le héros. Sa vie enseigne comment peut grandir le prestige, et comment il peut disparaître. Après avoir égalé en grandeur les plus célèbres héros de l’histoire, il fut abaissé par les magistrats de son pays au rang des plus vils criminels. Quand il mourut, son cercueil passa isolé au milieu des foules indifférentes. Seuls, les souverains étrangers rendirent hommage à sa mémoire comme à celle de l’un des plus grands hommes qu’ait connus l’histoire \footnote{ \noindent Un journal étranger, la \emph{Neu Freie Presse}, de Vienne, s’est livré au sujet de la destinée de Lesseps à des réflexions d’une très judicieuse psychologie, et que, pour cette raison, je reproduis ici :\par
 “Après la condamnation de Ferdinand de Lesseps, on n’a plus le droit de s’étonner de la triste fin de Christophe Colomb. Si Ferdinand de Lesseps est un escroc, toute noble illusion est un crime. L’antiquité aurait couronné la mémoire de Lesseps d’une auréole de gloire, et lui aurait fait boire à la coupe du nectar au milieu de l’Olympe, car il a changé la face de la terre, et il a accompli des œuvres qui perfection­nent la création. En condamnant Ferdinand de Lesseps, le président de la Cour d’appel s’est fait immortel, car toujours les peuples demanderont le nom de l’homme qui ne craignit pas d’abaisser son siècle pour habiller de la casaque du forçat un vieillard dont la vie a été la gloire de ses contem­porains.\par
 “Qu’on ne nous parle plus désormais de justice inflexible, là où règne la haine bureaucratique contre les grandes œuvres hardies. Les nations ont besoin de ces hommes audacieux qui croient en eux-mêmes et franchissent tous les obstacles, sans égard pour leur propre personne. Le génie ne peut pas être prudent ; avec la prudence il ne pourrait jamais élargir le cercle de l’activité humaine.\par
 “… Ferdinand de Lesseps a connu l’ivresse du triomphe et l’amertume des déceptions : Suez et Panama. Ici le cœur se révolte contre la morale du succès. Lorsque de Lesseps eut réussi à relier deux mers, princes et nations lui rendirent leurs hommages ; aujourd’hui qu’il échoue contre les rochers des Cordillères, il n’est plus qu’un vulgaire escroc… Il y a là une guerre des classes de la société, un mécontentement de bureaucrates et d’employés qui se vengent par le code cri­minel contre ceux qui voudraient s’élever au-dessus des autres… Les législateurs modernes se trouvent embarrassés devant ces grandes idées du génie humain ; le publie y comprend moins encore, et il est facile à un avocat général de prouver que Stanley est un assassin et Lesseps un trompeur.”
}.\par
Mais les divers exemples qui viennent d’être cités représentent des formes extrêmes. Pour établir dans ses détails la psychologie du prestige, il faudrait les placer à l’extrémité d’une série qui descendrait des fondateurs de religions et d’empires jusqu’au particulier essayant d’éblouir ses voisins par un habit neuf ou une décoration.\par
Entre les termes les plus éloignés de cette série, on placerait toutes les formes du prestige dans les divers éléments d’une civilisation : sciences, arts, littérature, etc., et l’on verrait qu’il constitue l’élément fondamental de la persuasion. Consciemment ou non, l’être, l’idée ou la chose possédant du prestige sont par voie de contagion imités immédiatement et imposent à toute une génération certaines façons de sentir et de traduire leur pensée. L’imitation est d’ailleurs le plus souvent inconsciente, et c’est précisément ce qui la rend parfaite. Les peintres modernes, qui reproduisent les couleurs effacées et les attitudes rigides de certains primitifs, ne se doutent guère d’où vient leur inspiration ; ils croient à leur propre sincérité, alors que si un maître éminent n’avait pas ressuscité cette forme d’art, on aurait continué à n’en voir que les côtés naïfs et inférieurs. Ceux qui, à l’instar d’un autre maître illustre, inondent leurs toiles d’ombres violettes, ne voient pas dans la nature plus de violet qu’on n’en voyait il y a cinquante ans, mais ils sont suggestionnés par l’impression personnelle et spé­ciale d’un peintre qui, malgré cette bizarrerie, sut acquérir un grand prestige. Dans tous les éléments de la civilisation, de tels exemples pourraient être aisément invoqués.\par
On voit, par ce qui précède, que bien des facteurs peuvent entrer dans la genèse du prestige : un des plus importants fut toujours le succès. Tout homme qui réussit, toute idée qui s’impose, cessent par ce fait même d’être contestée. La preuve que le succès est une des bases principales du prestige, c’est que ce dernier disparaît presque toujours avec lui. Le héros, que la foule acclamait la veille, est conspué par elle le lende­main si l’insuccès l’a frappé. La réaction sera même d’autant plus vive que le prestige aura été plus grand. La foule considère, alors le héros tombé comme un égal, et se venge de s’être inclinée devant la supériorité qu’elle ne lui reconnaît plus. Lorsque Robespierre faisait couper le cou à ses collègues et à un grand nombre de ses contemporains, il possédait un immense prestige Lorsqu’un déplacement de quelques voix lui ôta son pouvoir, il perdit immédiatement ce prestige, et la foule le suivit à la guillotine avec autant d’imprécations qu’elle suivait la veille ses victimes. C’est toujours avec fureur que les croyants brisent les statues de leurs anciens dieux.\par
Le prestige enlevé par l’insuccès est perdu brusquement. Il. peut s’user aussi par la discussion, mais d’une façon plus lente. Ce procédé est cependant d’un effet très sûr. Le prestige discuté n’est déjà plus du prestige. Les dieux et les hommes qui ont su garder longtemps leur prestige n’ont jamais toléré la discussion. Pour se faire admirer des foules, il faut toujours les tenir à distance.
\subsection[{Chapitre 4. Limites de variabilité des croyances et opinions des foules}]{Chapitre 4. Limites de variabilité des croyances et opinions des foules}

\begin{argument}\noindent § 1. – \emph{Les croyances fixes. –} Invariabilité de certaines croyances générales. – Elles sont les guides d’une civilisation. – Difficulté de les déraciner. – En quoi l’intolérance constitue pour les peuples une vertu. – L’absurdité philosophique d’une croyance générale ne peut nuire à sa propagation. § 2. \emph{Les opinions mobiles des foules. –} Extrême mobilité des opinions qui ne dérivent pas des croyances générales. Variations apparentes des idées et des croyances en moins d’un siècle. – Limites réelles de ces variations. – Éléments sur lesquels la variation a porté. – La disparition actuelle des croyances générales et la diffusion extrême de la presse rendent de nos jours les opinions de plus en plus mobiles. – Comment les opinions des foules tendent sur la plupart des sujets vers l’indifférence. – Impuissance des gouvernements à diriger comme jadis l’opinion. – L’émiettement actuel des opinions empêche leur tyrannie.
\end{argument}

\subsubsection[{§ 1. – Les croyances fixes}]{§ 1. – Les croyances fixes}
\noindent Il y a un parallélisme étroit entre les caractères anatomiques des êtres et leurs caractères psychologiques. Dans les caractères anatomiques nous trouvons certains éléments invariables, Ou si peu variables, qu’il faut la durée des âges géologiques pour les changer, et, à côté de ces caractères fixes, irréductibles, se voient des carac­tères très mobiles que le milieu, l’art de l’éleveur et de l’horticulteur modifient aisément, et parfois au point de dissimuler, pour l’observateur peu attentif, les carac­tères fondamentaux.\par
Nous observons le même phénomène dans les caractères moraux. A côté des éléments psychologiques irréductibles d’une race se rencontrent des éléments mobiles et changeants. Et c’est pourquoi, en étudiant les croyances et les opinions d’un peuple, on constate toujours un fonds très fixe sur lequel se greffent des opinions aussi mobiles que le sable qui recouvre le rocher.\par
Les croyances et les opinions des foules forment donc deux classes bien distinc­tes. D’une part, les grandes croyances permanentes, qui durent plusieurs siècles, et sur lesquelles une civilisation entière repose, telles, par exemple, autrefois, la conception féodale, les idées chrétiennes, celles de la Réforme ; tels de nos jours, le principe des nationalités, les idées démocratiques et sociales. D’autre part, les opinions momenta­nées et changeantes, dérivées le plus souvent des conceptions générales, que chaque âge voit naître et mourir : telles sont les théories qui guident les arts et la littérature à certains moments, celles, par exemple, qui ont produit le romantisme, le naturalisme, le mysticisme, etc. Elles sont aussi superficielles, le plus souvent, que la mode, et changent comme elle. Ce sont les petites vagues qui naissent et s’évanouissent sans cesse à la surface d’un lac aux eaux profondes.\par
Les grandes croyances générales sont en nombre fort restreint. Leur naissance et leur mort forment pour chaque race historique les points culminants de son histoire. Elles constituent la vraie charpente des civilisations.\par
Il est très facile d’établir une opinion passagère dans l’âme des foules, Mais il est très difficile d’y établir une croyance durable. Il est également fort difficile de détruire cette dernière lorsqu’elle a été établie. Ce n’est, le plus souvent, qu’au prix de révo­lutions violentes qu’on peut la changer. Les révolutions n’ont même ce pouvoir que lorsque la croyance a perdu presque entièrement son empire sur les âmes. Les révolutions servent alors à balayer finalement ce qui était à peu près abandonné déjà, mais ce que le joug de la coutume empêchait d’abandonner entièrement. Les révo­lutions qui commencent sont en réalité des croyances qui finissent.\par
Le jour précis où une grande croyance est marquée pour mourir est facile à reconnaître ; c’est celui où sa valeur commence à être discutée. Toute croyance géné­rale n’étant guère qu’une fiction ne saurait subsister qu’à la condition de n’être pas soumise à l’examen.\par
Mais alors même qu’une croyance est fortement ébranlée, les institutions qui en dérivent conservent leur puissance et ne s’effacent que, lentement. Lorsqu’elle a enfin perdu complètement son pouvoir, tout ce qu’elle soutenait s’écroule bientôt. Il n’a pas encore été donné à un peuple de pouvoir changer ses croyances sans être aussitôt condamné à transformer tous les éléments de sa civilisation.\par
Il les transforme, jusqu’à ce qu’il ait trouvé une nouvelle croyance générale qui soit acceptée ; et jusque-là il vit forcément dans l’anarchie. Les croyances générales sont les supports nécessaires des civilisations ; elles impriment une orientation aux idées. Elles seules peuvent inspirer la foi et créer le devoir.\par
Les peuples ont toujours senti l’utilité d’acquérir des croyances générales, et compris d’instinct que la disparition de celles-ci devait marquer pour eux l’heure de la décadence. Le culte fanatique de Rome fut pour les Romains la croyance qui les rendit maîtres du monde, et quand cette croyance fut morte, Rome dut mourir. Ce fut seulement lorsqu’ils eurent acquis quelques croyances communes que les barbares, qui détruisirent la civilisation romaine, atteignirent à une certaine cohésion et purent sortir de l’anarchie.\par
Ce n’est donc pas sans cause que les peuples ont toujours défendu leurs convic­tions avec intolérance. Cette intolérance, si critiquable au point de vue philoso­phique, représente dans la vie des peuples la plus nécessaire des vertus. C’est pour fonder ou maintenir des croyances générales que le moyen âge a élevé tant de bûchers, que tant d’inventeurs et de novateurs sont morts dans le désespoir quand ils évitaient les supplices. C’est pour les défendre que le monde a été tant de fois bouleversé, que tant de millions d’hommes sont morts sur les champs de bataille, et y mourront encore.\par
Il y a de grandes difficultés à établir une croyance générale, mais, quand elle est définitivement établie, sa puissance est pour longtemps invincible ; et quelle que soit sa fausseté philosophique, elle s’impose aux plus lumineux esprits. Les peuples de l’Europe n’ont-ils pas, depuis plus de quinze siècles, considéré comme des vérités indiscutables des légendes religieuses aussi barbares \footnote{Barbares philosophiquement, j’entends. Pratiquement, elles ont créé une civilisation entièrement nouvelle et pendant quinze siècles laissé entrevoir à l’homme ces paradis enchantés du rêve et de l’espoir qu’il ne connaîtra plus.}, quand on les examine de près, que celles de Moloch. L’effrayante absurdité de la légende d’un Dieu se vengeant sur son fils par d’horribles supplices de la désobéissance d’une de ses créatures, n’a pas été aperçue pendant bien des siècles. Les plus puissants génies, un Galilée, un Newton, un Leibniz, n’ont pas même supposé un instant que la vérité de tels dogmes pût être discutée. Rien ne démontre mieux l’hynotisation produite par les croyances générales, mais rien ne marque mieux aussi les humiliantes limites de notre esprit.\par
Dès qu’un dogme nouveau est implanté dans l’âme des foules, il devient l’inspira­teur de ses institutions, de ses arts et de sa conduite. L’empire qu’il exerce alors sur les âmes est absolu. Les hommes d’action ne songent qu’à le réaliser, les législateurs ne font que l’appliquer, les philosophes, les artistes, les littérateurs ne sont préoccupés que de le traduire sous des formes diverses.\par
De la croyance fondamentale, des idées momentanées accessoires peuvent surgir, mais elles portent toujours l’empreinte de la croyance dont elles sont issues. La civili­sation égyptienne, la civilisation européenne du moyen âge, la civilisation musulmane des Arabes dérivent d’un tout petit nombre de croyances religieuses qui ont imprimé leur marque sur les moindres éléments de ces civilisations, et permettent de les reconnaître aussitôt.\par
Et c’est ainsi que grâce aux croyances générales, les hommes de chaque âge sont entourés d’un réseau de traditions, d’opinions et de coutumes, au joug desquelles ils ne sauraient se soustraire et qui les rendent toujours très semblables les uns aux autres. Ce qui mène sur­tout les hommes, ce sont les croyances et les coutumes dérivées de ces croyances. Elles règlent les moindres actes de notre existence, et l’esprit le plus indépendant ne songe pas à s’y soustraire. Il n’y a de véritable tyran­nie que celle qui s’exerce inconsciemment sur les âmes, parce que c’est la seule qui ne se puisse com­battre. Tibère, Gengiskhan, Napoléon ont été des tyrans redoutables, sans doute, mais, du fond de leur tombeau, Moïse, Bouddha, Jésus, Mahomet, Luther ont exercé sur les âmes un despotisme bien autrement profond. Une conspiration peut abattre un tyran, mais que peut-elle sur une croyance bien établie ? Dans sa lutte violente contre le catholicisme, et malgré l’assentiment apparent des foules, malgré des procédés de destruction aussi impitoyables que ceux de l’Inquisition, c’est notre grande Révolution qui a été vaincue. Les seuls tyrans réels que l’humanité ait connus ont toujours été les ombres des morts ou les illusions qu’elle s’est créées.\par
L’absurdité philosophique que présentent souvent les croyances générales n’a jamais été un obstacle à leur triomphe. Ce triomphe ne semble même possible qu’à la condition qu’elles renferment quelque mystérieuse absurdité. Ce n’est donc pas l’évidente faiblesse des croyances socialistes actuelles qui les empêchera de triompher dans l’âme des foules. Leur véritable infériorité par rapport à toutes les croyances religieuses tient uniquement à ceci : l’idéal de bonheur que promettaient ces dernières ne devant être réalisé que dans une vie future, personne ne pouvait contester cette réalisation. L’idéal de bonheur socialiste devant être réalisé sur terre, dès les premiè­res tentatives de réalisation, la vanité des promesses apparaîtra aussitôt, et la croyance nouvelle perdra du même coup tout prestige. Sa puissance ne grandira donc que jusqu’au jour où, ayant triomphé, la réalisation pratique commencera. Et c’est pour­quoi, si la religion nouvelle exerce d’abord, comme toutes celles qui l’ont précédée, un rôle destructeur, elle ne saurait exercer ensuite, comme elles, un rôle créateur
\subsubsection[{§ 2. – Les opinions mobiles des foules}]{§ 2. – Les opinions mobiles des foules}
\noindent Au-dessus des croyances fixes, dont nous venons de montrer la puissance se trouve une couche d’opinions, d’idées, de pensées qui naissent et meurent constam­ment. Quelques-unes ont la durée d’un jour, et les plus importantes ne dépassent guère la vie d’une génération. Nous avons marqué déjà que les changements qui surviennent dans ces opinions sont parfois beaucoup plus superficiels que réels, et que toujours ils portent l’empreinte des qualités de la race. Considérant par exemple les institutions politiques du pays où nous vivons, nous avons fait voir que les partis en apparence les plus contraires : monarchistes, radicaux, impérialistes, socialistes, etc., ont un idéal absolument identique, et que cet idéal tient uniquement à la structure mentale de notre race, puisque, sous des noms analogues, on retrouve dans d’autres races un idéal tout à fait contraire. Ce n’est pas le nom donné aux opinions, ni des adaptations trompeu­ses qui changent le fond des choses. Les bourgeois de la Révolution, tout imprégnés de littérature latine, et qui, les yeux fixés sur la république romaine, adoptèrent ses lois, ses faisceaux et ses toges, et tachèrent d’imiter ses institutions et ses exemples, n’étaient pas devenus des Romains parce qu’ils étaient sous l’empire d’une puissante suggestion historique. Le rôle du philosophe est de rechercher ce qui subsiste des croyances anciennes sous les changements apparents, et de distinguer ce qui, dans le flot mouvant des opinions, est déterminé par les croyances générales et l’âme de la race.\par
Sans ce critérium philosophique on pourrait croire que les foules changent de croyances politiques ou religieuses fréquemment et à volonté. L’histoire tout entière, politique, religieuse, artistique, littéraire, semble le prouver en effet.\par
Prenons, par exemple, une bien courte période de notre histoire, de 1790 à 1820 seulement, c’est-à-dire trente ans, la durée d’une génération. Nous y voyons les foules, d’abord monarchiques, devenir révolutionnaires, puis impérialistes, puis redevenir monarchiques. En religion, elles vont pendant le même temps du catholicisme à l’athéisme, puis au déisme, puis retournent aux formes les plus exagérées du catho­licisme. Et ce ne sont pas seulement les foules, mais également ceux qui les dirigent. Nous contemplons avec étonnement ces grands conventionnels, ennemis jurés des rois et ne voulant ni dieux ni maîtres, qui deviennent les humbles serviteurs de Napoléon, puis portent pieusement des cierges dans les processions sous Louis XVIII.\par
Et dans les soixante-dix années qui suivent, quels changements encore dans les opinions des foules. La “Perfide Albion” du début de ce siècle devenant l’alliée de la France sous l’héritier de Napoléon ; la Russie, deux fois envahie par nous, et qui avait tant applaudi à nos derniers revers, considérée subitement comme une amie.\par
En littérature, en art, en philosophie, les successions d’opinions sont plus rapides encore. Romantisme, naturalisme, mysticisme, etc., naissent et meurent tour à tour. L’artiste et l’écrivain acclamés hier sont profondément dédaignés demain.\par
Mais, quand nous analysons tous ces changements, en apparence si profonds, que voyons-nous ? Tous ceux contraires aux croyances générales et aux sentiments de la race n’ont qu’une durée éphémère, et le fleuve détourné reprend bientôt son cours. Les opinions qui ne se rattachent à aucune croyance générale, à aucun sentiment de la race, et qui, par conséquent, ne sauraient avoir de fixité, sont à la merci de tous les hasards ou, si l’on préfère, des moindres changements de milieu. Formées par sugges­tion et contagion, elles sont toujours momentanées ; elles naissent et disparaissent parfois aussi rapidement que les dunes de sable formées par le vent au bord de la mer.\par
De nos jours, la somme des opinions mobiles des foules est plus grande qu’elle ne le fut jamais ; et cela, pour trois raisons différentes :\par
La première est que les anciennes croyances perdant de plus en plus leur empire, n’agissent plus comme jadis sur les opinions passagères pour leur donner une certaine orientation. L’effacement des croyances générales laisse place à une foule d’opinions particulières sans passé ni avenir.\par
La seconde raison est que la puissance des foules devenant de plus en plus grande et ayant de moins en moins de contrepoids. la mobilité extrême d’idées que nous avons constatée chez elles, peut se manifester librement.\par
La troisième raison enfin est la diffusion récente de la presse qui met sans cesse sous les yeux des foules les opinions les plus contraires. Les suggestions que cha­cune d’elles pourrait engendrer sont bientôt détruites par des suggestions opposées. Il en résulte que chaque opinion n’arrive pas à s’étendre et est vouée à une existence très éphémère. Elle est morte avant d’avoir pu se répandre assez pour devenir générale.\par
De ces causes diverses est résulté un phénomène très nouveau dans l’histoire du monde, et tout à fait caractéristique de l’âge actuel, je veux parler de l’impuissance des gouvernements à diriger l’opinion.\par
Jadis, et ce jadis n’est pas fort loin, l’action des gouvernements, l’influence de quelques écrivains et d’un tout petit nombre de journaux constituaient les vrais régu­lateurs de l’opinion. Aujourd’hui, les écrivains ont perdu toute influence, et les journaux ne font plus que refléter l’opinion. Quant aux hommes d’État, loin de la diriger, ils ne cherchent qu’à la suivre. Ils ont une crainte de l’opinion qui va parfois jusqu’à la terreur et ôte toute fixité à leur ligne de conduite.\par
L’opinion des foules tend donc à devenir de plus en plus le révélateur suprême de la politique. Elle arrive aujourd’hui à imposer des alliances, comme nous l’avons vu récemment pour l’alliance russe, exclusivement sortie d’un mouvement populaire. C’est un symptôme bien curieux de voir de nos jours papes, rois et empereurs, se soumettre au mécanisme de l’interview, pour exposer leur pensée, sur un sujet donné, au jugement des foules. On a pu dire jadis que la politique n’était pas chose de sentiment. Pourrait-on le dire encore aujourd’hui où elle a de plus en plus pour guide les impulsions de foules mobiles qui ne connaissent pas la raison, et que le sentiment seul peut guider ?\par
Quant à la presse, autrefois directrice de l’opinion, elle a dû, comme les gouverne­ments, s’effacer devant le pouvoir des foules. Elle possède certes une puissance considérable, mais seulement parce qu’elle est exclusivement le reflet des opinions des foules et de leurs incessantes variations. Devenue simple agence d’information, elle a renoncé à chercher à imposer aucune idée, aucune doctrine. Elle suit tous les changements de la pensée publique, et les nécessités de la concurrence l’obligent à bien les suivre sous peine de perdre ses lecteurs. Les vieux organes solennels et influents d’autrefois, comme le \emph{Constitutionnel}, les \emph{Débats}, le\emph{ Siècle}, dont la précé­dente génération écoutait pieusement les oracles, ont disparu ou sont devenus feuilles d’informations encadrées de chroniques amusantes, de cancans mondains et de réclames financières. Où serait aujourd’hui le journal assez riche pour permettre à ses rédacteurs des opinions personnelles, et de quel poids seraient ces opinions auprès de lecteurs qui ne demandent qu’à être renseignés ou amusés, et qui, derrière chaque recommandation, redoutent toujours le spéculateur. La critique n’a même plus le pou­voir de lancer un livre ou une pièce de théâtre. Elle peut leur nuire, mais non les servir. Les journaux ont tellement conscience de l’inutilité de tout ce qui est critique ou opinion personnelle, qu’ils ont progressivement supprimé les critiques littéraires, se bornant à donner le titre du livre avec deux ou trois lignes de réclame, et, dans vingt ans, il en sera probablement de même pour la critique théâtrale.\par
Épier l’opinion est devenu aujourd’hui la préoccupation essentielle de la presse et des gouvernements. Quel est l’effet produit par un événement, un projet législatif, un discours, voilà ce qu’il leur faut savoir sans cesse ; et la chose n’est pas facile, car rien n’est plus mobile et plus changeant que la pensée des foules, et rien n’est plus fréquent que de les voir accueillir avec des anathèmes ce qu’elles avaient acclamé la veille.\par
Cette absence totale de direction de l’opinion, et en même temps la dissolution des croyances générales, ont en pour résultat final un émiettement complet de toutes les convictions, et l’indifférence croissante des foules pour ce qui ne touche pas nette­ment leurs intérêts immédiats. Les questions de doctrines, telles que le socialisme, ne recrutent de défenseurs réellement convaincus que dans les couches tout à fait illettrées : ouvriers des mines et des usines, par exemple. Le petit bourgeois, l’ouvrier ayant quelque teinte d’instruction soit devenus d’un scepticisme ou tout au moins d’une mobilité complète.\par
L’évolution qui s’est ainsi opérée depuis trente ans est frappante. A l’époque précé­dente, peu éloignée pourtant, les opinions possédaient encore une orientation géné­rale ; elles dérivaient de l’adoption de quelque croyance fondamentale. Par le fait seul qu’on était monarchiste, on avait fatalement, aussi bien en histoire que dans les sciences, certaines idées très arrêtées et, par le fait seul qu’on était républicain, on avait des idées tout à fait contraires. Un monarchiste savait pertinemment que l’hom­me ne descend pas du singe, et un républicain savait non moins pertinemment qu’il en descend. Le monarchiste devait parler de la Révolution avec horreur, et le républicain avec vénération. Il y avait des noms, tels que ceux de Robespierre et de Marat, qu’il fallait prononcer avec des mines de dévot, et d’autres noms, tels que ceux de César, d’Auguste et de Napoléon qu’on ne devait pas articuler sans les couvrir d’invectives. Jusque dans notre Sorbonne, cette naïve façon de concevoir l’histoire était générale \footnote{ \noindent Certaines pages des livres de nos professeurs officiels sont, à ce point de vue, bien curieuses, et montrent à quel point l’esprit critique est peu développé par notre éducation universitaire. Je citerai comme exemple les lignes suivantes extraites de la Révolution française d’un ancien professeur d’histoire à la Sorbonne, qui fut ministre de l’instruction publique.\par
 “La prise de la Bastille est un fait culminant dans l’histoire non seulement de la France, mais de l’Europe entière ; elle inaugurait une époque nouvelle de l’histoire du monde” !\par
 Quant à, Robespierre, nous y apprenons avec stupeur, que sa dictature fut surtout d’opinion, de persuasion, d’autorité morale ; elle fut une sorte de pontificat entre les mains d’un homme vertueux ! (pp.91 et 220.)
}.\par
Aujourd’hui, devant la discussion et l’analyse, toutes les opinions perdent leur prestige ; leurs angles s’usent vite, et il en survit bien peu qui nous puissent passion­ner. L’homme moderne est de plus en plus envahi par l’indifférence.\par
Ne déplorons pas trop cet effritement général des opinions. Que ce soit un symp­tôme de décadence dans la vie d’un peuple, on ne saurait le contester. Il est certain que les voyants, les apôtres, les meneurs, les convaincus en un mot, ont une bien autre force que les négateurs, les critiques et les indifférents ; mais n’oublions pas non plus qu’avec la puissance actuelle des foules, si une seule, opinion pouvait acquérir assez de prestige pour s’imposer, elle serait bientôt revêtue d’un pouvoir tellement tyranni­que que tout devrait aussitôt plier devant elle, et que l’âge de la libre discussion serait clos pour longtemps. Les foules représentent des maîtres pacifiques parfois, comme l’étaient à leurs heures Héliogabale et Tibère ; mais elles ont aussi de furieux caprices. Quand une civilisation est prête à tomber entre leurs mains, elle est à la merci de trop de hasards pour durer bien longtemps. Si quelque chose pouvait retarder un peu l’heure de l’effondrement, ce serait précisément l’extrême mobilité des opinions et l’indifférence croissante des foules pour toute croyance générale.\par

\labelblock{Classification et description des diverses catégories de foules}

\section[{Livre III. Classification et descriptions des diverses catégories de foules}]{Livre III. Classification et descriptions des diverses catégories de foules}\renewcommand{\leftmark}{Livre III. Classification et descriptions des diverses catégories de foules}

\subsection[{Chapitre 1. Classification des foules}]{Chapitre 1. Classification des foules}

\begin{argument}\noindent Divisions générales des foules. – Leur classification. § 1\emph{. Les foules hétérogènes. –} Comment elles se différen­cient. – Influence de la race. – L’âme de la foule est d’au­tant plus faible que l’âme de la race est plus forte. – L’âme de la race représente l’état de civilisation et l’âme de la foule l’état de barbarie. – § 2\emph{. Les foules homo}­\emph{gènes. –} Division des foules homogènes. – Les sectes, les castes et les classes.
\end{argument}

\noindent Nous avons indiqué dans cet ouvrage les caractères généraux communs aux foules psychologiques. Il nous reste à montrer les caractères particuliers qui s’ajoutent à ces caractères généraux suivant les diverses catégories de collectivités lorsque, sous l’influence d’excitants convenables, elles se transforment en foule.\par
Exposons d’abord en quelques mots une classification des foules.\par
Notre point de départ sera la simple multitude. Sa forme la plus inférieure se présente, lorsqu’elle est composée d’individus appartenant à des races différentes. Elle n’a d’autre lien commun que la volonté, lus ou moins respectée d’un chef. On peut donner comme type de telles multitudes, les barbares d’origines fort diverses, qui pen­dant plusieurs siècles envahirent l’empire Romain.\par
Au-dessus de ces multitudes de races diverses, se trouvent celles qui, sous l’influ­ence de certains facteurs, ont acquis des caractères communs et ont fini par former une race. Elles présenteront à l’occasion les caractéristiques spéciales des foules, mais ces caractéristiques seront plus ou moins dominées par celles de la race.\par
Ces deux catégories de multitudes peuvent, sous l’influence des facteurs étudiés dans cet ouvrage, se transformer en foules organisées ou psychologiques. Dans ces foules organisées, nous établirons les divisions suivantes :\par

\begin{itemize}[itemsep=\baselineskip,]
\item A. Foules hétérogènes 
\begin{itemize}[itemsep=0pt,]
\item 1° Anonymes. (Foules de rues, par exemple)
\item 2° Non anonymes (Jurys, assemblées parlementaires, etc.)
\end{itemize}

 
\item B. Foules homogènes 
\begin{itemize}[itemsep=0pt,]
\item 1° Sectes. (Sectes politiques, Sectes religieuses, etc.)
\item 2° Castes. (Caste militaire, caste sacerdotale, castes ouvrières, etc.)
\item 3° Classes. (Classe bourgeoise, classe des paysans, etc.)
\end{itemize}

 
\end{itemize}

\noindent Indiquons en quelques mots les caractères différentiels de ces diverses catégories de foules.\par
\subsubsection[{§ 1. – Foules hétérogènes}]{§ 1. – Foules hétérogènes}
\noindent Ces collectivités sont celles dont nous avons étudié les caractères dans ce volume. Elles se composent d’individus quelconques, quelle que soit leur profession ou leur intelligence.\par
Nous savons maintenant que, par le fait seul que des hommes forment une foule agissante, leur psychologie collective diffère essentiellement de leur psychologie individuelle, et que l’intelligence ne les soustrait pas à cette différenciation. Nous avons vu que, dans les collectivités, l’intelligence ne joue aucun rôle. Seuls des senti­ments inconscients agissent.\par
Un facteur fondamental, la race, permet de différencier assez profondément les diverses foules hétérogènes.\par
Nous sommes plusieurs fois déjà revenus sur le rôle de la race, et nous avons montré qu’elle est le plus puissant des facteurs capables de déterminer les actions des hommes. Elle manifeste également son action dans les caractères des foules. Une foule composée d’individus quelconques, mais tous Anglais ou Chinois, différera pro­fondément d’une autre foule composée d’individus également quelconques, mais de races différentes Russes, Français, Espagnols, par exemple.\par
Les profondes divergences que la constitution mentale héréditaire crée dans la façon de sentir et de penser des hommes, éclatent immédiatement dès que des cir­constances, assez rares d’ailleurs, réunissent dans une même foule, en proportions à peu près égales, des individus de nationalités différentes, quelque identiques que soient en apparence les intérêts qui les rassemblent. Les tentatives faites par les socialistes pour réunir dans de grands congrès des représentants de la population ouvrière de chaque pays, ont toujours abouti aux plus furieuses discordes. Une foule latine, si révolutionnaire ou si conservatrice qu’on la suppose, fera invariablement appel, pour réaliser ses exigences, à l’intervention de l’État. Elle est toujours centra­lisatrice et plus ou moins césarienne. Une foule anglaise ou américaine, au con­traire, ne connaît pas l’État et ne fait appel qu’à l’initiative privée. Une foule française tient avant tout à l’égalité, et une foule anglaise à la liberté. Ce sont précisément ces différences de races qui font qu’il y a presque autant de formes de socialisme et de démocratie que de nations.\par
L’âme de la race domine donc entièrement l’âme de la foule. Elle est le substratum puissant qui limite ses oscillations. Considérons comme une loi essentielle que \emph{les caractères inférieurs des foules sont d’autant moins accentués que l’âme de la race est plus forte.} L’état de foule et la domination des foules, c’est la barbarie ou le retour à la barbarie. C’est en acquérant une âme solidement constituée que la race se sous­trait de plus en plus à la puissance irréfléchie des foules et sort de la barbarie.\par
En dehors de la race, la seule classification importante à faire pour les foules hétérogènes est de les séparer en foules anonymes, comme celles des rues, et en foules non anonymes, – les assemblées délibérantes et les jurés par exemple. Le sentiment de la responsabilité, nul chez les premières et développé chez les secondes, donne à leurs actes des orientations souvent fort différentes.
\subsubsection[{§ 2. – Foules homogènes}]{§ 2. – Foules homogènes}
\noindent Les foules homogènes comprennent : 1°\emph{ les sectes} ; 2°\emph{ les castes ;} 3°\emph{ les classes.}\par
La \emph{secte} marque le premier degré dans l’organisation des foules homogènes. Elle comprend des individus d’éducation, de professions, de milieux parfois fort différents, n’ayant entre eux que le lien unique des croyances. Telles sont les sectes religieuses et politiques, par exemple.\par
La \emph{caste} représente le plus haut degré d’organisation dont la foule soit susceptible. Alors que la secte comprend des individus de professions, d’éducation, de mi­lieux fort différents et rattachés seulement par la communauté des croyances, la caste ne comprend que des individus de même profession et par conséquent d’éducation et de milieux à peu près semblables. Telles sont la caste militaire et la caste sacerdotale, par exemple.\par
La \emph{classe} est formée par des individus d’origines diverses réunis, non par la com­munauté des croyances, comme le sont les membres d’une secte, ni par la commu­nauté des occupations professionnelles, comme le sont les membres d’une caste, mais par certains intérêts, certaines habitudes de vie et d’éducation fort semblables. Telles sont, par exemple, la classe bourgeoise, la classe agricole, etc.\par
Ne m’occupant dans cet ouvrage que des foules hétérogènes, et réservant l’étude des foules homogènes (sectes, castes et classes) pour un autre volume, je n’insisterai pas ici sur les caractères de ces dernières, et ne m’occuperai maintenant que de quel­ques catégories de foules hétérogènes choisies comme types.
\subsection[{Chapitre 2. Les foules dites criminelles}]{Chapitre 2. Les foules dites criminelles}

\begin{argument}\noindent Les foules dites criminelles. – Une foule peut être légale­ment mais non psychologiquement criminelle. – Complète inconscience des actes des foules. – Exemples divers. – Psychologie des septembriseurs. – Leurs raisonnements, leur sensibilité, leur férocité et leur moralité.
\end{argument}

\noindent Les foules tombant, après une certaine période d’excitation, à l’état de simples automates inconscients menés par des suggestions, il semble difficile de les qualifier dans aucun cas de criminelles. Je ne conserve ce qualificatif erroné que parce qu’il a été consacré par des recherches psychologiques récentes. Certains actes des foules sont assurément criminels si on ne les considère qu’en eux-mêmes, niais alors au même titre que l’acte d’un tigre dévorant un Hindou, après l’avoir d’abord laissé un peu déchiqueter par ses petits pour les distraire.\par
Les crimes des foules ont généralement pour mobile une suggestion puissante, et les individus qui y ont pris part sont persuadés ensuite qu’ils ont obéi à un devoir, ce qui n’est pas du tout le cas du criminel ordinaire.\par
L’histoire des crimes commis par les foules met en évidence ce qui précède.\par
On peut citer comme exemple typique le meurtre du gouverneur de la Bastille, M. de Launay. Après la prise de cette forteresse, le gouverneur, entouré d’une foule très excitée, recevait des coups de tous côtés. On proposait de le pendre, de lui couper la tête, ou de l’attacher à la queue d’un cheval. En se débattant, il donna par mégarde un coup de pied à l’un des assistants. qu’un proposa, et sa suggestion fut acclamée aussitôt par la foule, que l’individu atteint par le coup de pied coupât le cou au gouverneur.\par
“Celui-ci, cuisinier sans place, demi-badaud qui est allé à la Bastille pour voir ce qui s’y passait, juge que, puisque tel est l’avis général, l’action est patriotique, et croit même mériter une médaille en détruisant un monstre. Avec un sabre qu’on lui prête, il frappe sur le col nu ; mais le sabre mal affilé ne coupant pas, il tire de sa poche un petit couteau à manche noir et (comme, en sa qualité de cuisinier, il sait travailler les viandes) il achève heureusement l’opération.”\par
On voit clairement ici le mécanisme indiqué précédemment. Obéissance à une suggestion d’autant plus puissante qu’elle est collective, conviction chez le meurtrier qu’il a commis un acte fort méritoire, et conviction d’autant plus naturelle qu’il a pour lui l’approbation unanime de ses concitoyens. Un acte semblable peut être légalement, mais non psychologiquement, qualifié de criminel.\par
Les caractères généraux des foules dites criminelles sont exactement ceux que nous avons constatés chez toutes les foules : suggestibilité, crédulité, mobilité, exagé­ration des sentiments bons ou mauvais, manifestation de certaines formes de moralité., etc.\par
Nous allons retrouver tous ces caractères chez une des foules qui ont laissé un des plus sinistres souvenirs dans notre histoire : celle des septembriseurs. Elle pré­sente d’ailleurs beaucoup d’analogie avec celles qui firent la Saint-Barthélemy. J’emprunte les détails du récit à M. Taine, qui les a puisés dans les mémoires du temps.\par
On ne sait pas exactement qui donna l’ordre ou suggéra de vider les prisons en massacrant les prisonniers. Que ce soit Danton, comme cela est probable, ou tout autre, il n’importe ; le seul fait intéressant pour nous est celui de la suggestion puis­sante que reçut la foule chargée du massacre.\par
La foule des massacreurs comprenait environ trois cents personnes, et constituait le type parfait d’une foule hétérogène. A part un très petit nombre de gredins profes­sionnels, elle se composait surtout de boutiquiers et d’artisans de tous les corps d’états : cordonniers, serruriers, perruquiers, maçons, employés, commissionnaires, etc. Sous l’influence de la suggestion reçue, ils sont, comme le cuisinier cité plus haut, parfaitement convaincus qu’ils accomplissent un devoir patriotique. Ils remplissent une double fonction, juges et bourreaux, mais ne se considèrent en aucune façon com­me des criminels.\par
Pénétrés de l’importance de leur devoir, ils commencent par former une sorte de tribunal, et immédiatement apparaissent l’esprit simpliste, et l’équité non moins sim­pliste des foules. Vu le nombre considérable des accusés, on décide tout d’abord que les nobles, les prêtres, les officiers, les serviteurs du roi, c’est-à-dire tous les individus dont la profession seule est une preuve de culpabilité aux yeux d’un bon patriote, seront massacrés en tas sans qu’il soit besoin de décision spéciale. Pour les autres, ils seront jugés sur la mine et la réputation. La conscience rudimentaire de la foule étant ainsi satisfaite, elle va pouvoir procéder légalement au massacre et donner libre cours à ces instincts de férocité dont j’ai montré ailleurs la genèse, et que les collectivités ont toujours le pouvoir de développer à un haut degré. Ils n’empêcheront pas d’ailleurs – ainsi que cela est la règle dans les foules – la manifestation concomitante d’autres sentiments contraires, tels qu’une sensibilité souvent aussi extrême que la férocité.\par
“Ils ont la sympathie expansive et la sensibilité prompte de l’ouvrier parisien. A l’Abbaye, un fédéré, apprenant que depuis vingt-six heures on avait laissé les détenus sans eau, voulait absolument exterminer le guichetier négligent, et l’eût fait sans les supplications des détenus eux-mêmes. Lorsqu’un prisonnier est acquitté : (par leur tribunal improvisé), gardes et tueurs, tout le monde l’embrasse avec transport, on applaudit à outrance,” puis on retourne tuer les autres en tas. Pendant le massacre, une aimable gaieté ne cesse de régner. Ils dansent et chantent autour des cadavres, disposent des bancs “pour les dames” heureuses de voir tuer des aristocrates. Ils continuent aussi à faire preuve d’une équité spéciale. Un tueur s’étant plaint, à l’Abbaye, que les dames placées un peu loin voient mal, et que quel­ques assistants seuls ont le plaisir de frapper les aristocrates, ils se rendent à la justesse de cette observation, et décident que l’on fera passer lentement les victimes entre deux haies d’égorgeurs qui ne pourront frapper qu’avec le dos du sabre, afin de prolonger le supplice. A la Force on met les victimes entièrement nues, on les déchiquette pendant une demi-heure ; puis, quand tout le monde a bien vu on les finit en leur ouvrant le ventre.\par
Les massacreurs sont d’ailleurs fort scrupuleux, et manifestent la moralité dont nous avons déjà signalé l’existence au sein des foules. Ils refusent de s’emparer de l’argent et des bijoux des victimes, et les rapportent sur la table des comités.\par
Dans tous leurs actes, on retrouve toujours ces formes rudimentaires de raisonne­ment, caractéristiques de l’âme des foules. C’est ainsi qu’après l’égorgement des 12 ou 1500 ennemis de la nation, quelqu’un fait observer, et immédiatement sa suggestion est acceptée, que les autres prisons, celles qui contiennent de vieux mendiants, des vagabonds, des jeunes détenus, renferment en réalité des bouches inutiles, et dont il serait bon, pour cette raison, de se débarrasser. D’ail­leurs il doit y avoir certainement parmi eux des ennemis du peuple, tels, par exemple, qu’une certaine dame Delarue, veuve d’un empoisonneur : “Elle doit être furieuse d’être en prison ; si elle pouvait, elle mettrait le feu à Paris ; elle doit l’avoir dit, elle l’a dit. Encore un coup de balai.” La démonstration parait évidente, et tout est massacré en bloc, y compris une cin­quantaine d’enfants de douze, à dix-sept ans, qui, d’ailleurs, eux-mêmes auraient pu devenir des ennemis de la nation, et dont par conséquent il y avait un intérêt évident à se débarrasser.\par
Au bout d’une semaine de travail, toutes ces opérations étant terminées, les massa­creurs purent songer au repos. Étant intimement persuadés qu’ils avaient bien mérité de la patrie, ils vinrent réclamer aux autorités une récompense ; les plus zélés allèrent même jusqu’à exiger une médaille.\par
L’histoire de la Commune de 1871 nous offre plusieurs faits analogues à ceux qui précèdent. Avec l’influence grandissante des foules et les capitulations successives des pouvoirs devant elles, nous sommes certainement, appelés à en voir bien d’autres.
\subsection[{Chapitre 3. Les Jurés de cour d’assises}]{Chapitre 3. Les Jurés de cour d’assises}

\begin{argument}\noindent Les jurés de cour d’assises. – Caractères généraux des jurys. – La statistique montre que leurs décisions sont indépendantes de leur composition. – Comment sont impressionnés les jurés. – Faible action du raisonnement. – Méthodes de persuasion des avocats célèbres. – Nature des crimes pour lesquels les jurés sont indulgents ou sévères. – Utilité de l’institution du jury et danger extrême que présenterait son remplacement par des magistrats.
\end{argument}

\noindent Ne pouvant étudier ici toutes les catégories de jurés, j’examinerai seulement la plus importante, celle des jurés de cours d’assises. Ces jurés constituent un excellent exemple de foule hétérogène non anonyme. Nous y retrouvons la suggestibilité, la prédominance des sentiments inconscients, la faible aptitude au raisonnement, influ­ence des meneurs, etc. En les étudiant nous aurons l’occasion d’observer d’intéres­sants spécimens des erreurs que peuvent commettre les personnes non initiées à la psychologie des collectivités.\par
Les jurés nous fournissent tout d’abord une preuve de la faible importance que présente au point de vue des décisions, le niveau mental des divers éléments compo­sant une foule. Nous avons vu que lorsqu’une assemblée délibérante est appelée à donner son opinion sur une question n’ayant pas un caractère tout a fait technique, l’intelligence ne joue aucun rôle ; et qu’une réunion de savants ou d’artistes, par ce fait seul qu’ils sont réunis, n’a pas, sur des sujets généraux, des jugements sensiblement différents de ceux d’une assemblée de maçons ou d’épiciers. A diverses époques, l’administration faisait un choix soigneux parmi les personnes appelées à composer le jury, et on les recrutait parmi les classes éclairées : professeurs, fonctionnaires, lettrés, etc. Aujourd’hui le jury se recrute surtout parmi les petits marchands, les petits patrons, les employés. Or, au grand étonnement des écrivains spéciaux, quelle qu’ait été la composition des jurys, la statistique prouve que leurs décisions ont été iden­tiques. Les magistrats eux-mêmes, si hostiles pourtant à l’institution du jury, ont dû reconnaître l’exactitude de cette assertion. Voici comment s’exprime à ce sujet un ancien président de cour d’assises, M. Bérard des Glajeux, dans ses \emph{Souvenirs.}\par
“Aujourd’hui les choix du jury sont, en réalité, dans les mains des conseillers municipaux, qui admettent ou éliminent, à leur gré, suivant les préoccupations poli­tiques et électorales inhérentes à leur situation… La majorité des élus se compose de commerçants moins importants qu’on ne les choisissait autrefois, et des employés de certaines administrations… Toutes les opinions se fondant avec toutes les professions dans le rôle de juge, beaucoup ayant l’ardeur des néophytes, et les hommes de meil­leure volonté se rencontrant dans les situations les plus humbles, l’esprit du jury n’a pas changé : \emph{ses verdicts sont restés les mêmes.”}\par
Retenons du passage que je viens de citer les conclusions qui sont très justes, et non les explications qui sont très faibles. Il ne faut pas trop s’étonner de cette faiblesse, car la psychologie des foules, et par conséquent des jurés, semble avoir été le plus souvent aussi inconnue des avocats que des magistrats. J’en trouve la preuve dans ce fait rapporté par l’auteur cité à l’instant, qu’un des plus illustres avocats de cour d’assises, Lachaud, usait systématiquement de son droit de récusation à l’égard de tous les individus intelligents faisant partie du jury. Or, l’expérience – l’expérience seule – a fini par apprendre l’entière inutilité de ces récusations. La preuve en est qu’aujourd’hui le ministère public et les avocats, à Paris du moins, y ont entièrement renoncé ; et, comme le fait remarquer M. des Glajeux, les verdicts n’ont pas changé, ils ne sont ni meilleurs ni pires”.\par
Comme toutes les foules, les jurés sont très fortement impressionnes par des sentiments et très faiblement par des raisonnements Ils ne résistent pas, écrit un avocat, “à la vue d’une femme donnant à téter, ou à un défilé d’orphelins.” “Il suffit qu’une femme soit agréable, dit M. des Glajeux, pour obtenir la bienveillance du jury.”\par
Impitoyables aux crimes qui semblent pouvoir les atteindre – et qui sont précisé­ment d’ailleurs les plus redoutables pour la société – les jurés sont au contraire très indulgents pour les crimes dits passionnels. Ils sont rarement sévères pour l’infanti­cide des filles-mères et moins encore pour la fille abandonnée qui vitriolise un peu son séducteur, sentant fort bien d’instinct que ces crimes-là sont peu dangereux pour la société, et que dans un pays où la loi ne protège, pas les filles abandonnées, le crime de celle qui se venge est plus utile que nuisible, en intimidant d’avance les futurs séducteurs \footnote{Remarquons en passant que cette division, très bien faite d’instinct par les jurés, entre les crimes dangereux pour la société et les crimes non dangereux pour elle n’est pas du tout dénuée de justesse. Le but des lois criminelles doit être évidemment de protéger la société contre les criminels dangereux et non pas de la venger. Or nos codes, et surtout l’esprit de nos magistrats, sont tout imprégnés encore de l’esprit de vengeance du vieux droit primitif, et le terme de vindicte (vindicta, vengeance) est encore d’un usage journalier. Nous avons la preuve de cette tendance des magis­trats dans le refus de beaucoup d’entre eux d’appliquer l’excellente loi Bérenger, qui permet au condamné de ne subir sa peine que s’il récidive. Or, il n’est pas un magistrat qui puisse ignorer, car la statistique le prouve, que l’application d’une première peine crée presque infailliblement la récidive. Quand les juges relâchent un coupable, il leur semble toujours que la société n’a pas été vengée. Plutôt que de ne la pas venger, ils préfèrent créer un récidiviste dangereux.}.\par
Les jurys, comme toutes les foules, sont fort éblouis par le prestige, et le président des Glajeux fait justement remarquer que, très démocratiques dans leur composition, ils sont très aristocratiques dans leurs affections : “Le nom, la naissance, la grande fortune, la renommée, l’assistance d’un avocat illustre, les choses qui distinguent et les choses qui reluisent forment un appoint très considérable dans la main des accusés.” Agir sur les sentiments des jurés, et, comme avec toutes les foules, raisonner fort peu, ou n’employer que des formes rudimentaires de raisonnement, doit être la préoccu­pation de tout bon avocat. Un avocat anglais célèbre par ses succès en cour d’assises a bien montré la façon d’agir.\par
“Il observait attentivement le jury tout en plaidant. C’est le moment favorable. Avec du flair et de l’habitude, l’avocat lit sur les physionomies l’effet de chaque phrase, de chaque mot, et il en tire ses conclusions. Il s’agit tout d’abord de distinguer les membres acquis d’avance à la cause. Le défenseur achève en un tour de main de se les assurer, après quoi il passe aux membres qui semblent au contraire mal disposés, et il s’efforce de deviner pourquoi ils sont contraires à l’accusé. C’est la partie délicate du travail, car il peut y avoir une infinité de raisons d’avoir envie de condamner un homme, en dehors du sentiment de la justice.”\par
Ces quelques lignes résument très bien le but de l’art oratoire, et nous montrent aussi pourquoi le discours fait d’avance est inutile puisqu’il faut pouvoir à chaque instant modifier les termes employés suivant l’impression produite.\par
L’orateur n’a pas besoin de convertir tous les membres d’un jury, mais seulement les meneurs qui détermineront l’opinion générale. Comme dans toutes les foules, il y a toujours un petit nombre, d’individus qui conduisent les autres. “J’ai fait l’expérience, dit l’avocat que je citais plus haut, qu’au moment de rendre le verdict, il suffisait d’un ou deux hommes énergiques pour entraîner le reste du jury.” Ce sont ces deux ou trois-là qu’il faut convaincre par d’habiles suggestions. Il faut d’abord et avant tout leur plaire. L’homme en foule à qui on a plu est près d’être convaincu, et tout disposé à trouver excellentes les raisons quelconques qu’on lui présente. Je trouve, dans un travail intéressant sur Me Lachaud, l’anecdote suivante :\par
“On sait que pendant toute la durée des plaidoiries qu’il prononçait aux assises, Lachaud ne perdait pas de vue deux ou trois jurés qu’il savait, ou sentait, influents, mais revêches. Généralement, il parvenait à réduire ces récalcitrants. Pourtant, une fois, en province, il en trouva un qu’il dardait vainement de son argumentation la plus tenace depuis trois quarts d’heure : le premier du deuxième banc, le septième juré. C’était désespérant ! Tout à coup, au milieu d’une démonstration passion­nante, Lachaud s’arrête, et s’adressant au président de la cour d’assises : “Monsieur le président, dit-il, ne pourriez-vous pas faire tirer le rideau, là, en face. Monsieur le septième juré est aveuglé par le soleil.” Le septième juré rougit, sourit, remercia. Il était acquis à la défense.”\par
Plusieurs écrivains, et parmi eux de très distingués, ont fortement combattu dans ces derniers temps l’institution du jury, seule protection que nous ayons pour­tant contre les erreurs vraiment bien fréquentes d’une caste sans contrôle \footnote{La magistrature représente, en effet, la seule administration dont les actes ne soient soumis à aucun contrôle. Malgré toutes ses révolutions, la France démocratique ne possède pas ce droit d’\emph{habeas corpus} dont l’Angleterre est si fière. Nous avons banni tous les tyrans ; mais dans chaque cité nous avons établi un magistrat qui dispose à son gré de l’honneur et de la liberté des citoyens. Un petit juge d’instruction, à peine sorti de l’école de droit, possède le pouvoir révoltant d’envoyer à son gré en prison, sur une simple supposition de culpabilité de sa part, et dont il ne doit la justifi­cation à personne, les citoyens les plus considérables. Il peut les y garder six mois ou même un an sous prétexte d’instruction, et les relâcher ensuite sans leur devoir ni indemnité, ni excuses. Le mandat d’amener est absolument l’équivalent de la lettre de cachet, avec cette différence que cette dernière, si justement reprochée à l’ancienne monarchie, n’était à la portée que de très grands personnages, alors qu’elle est aujourd’hui entre les mains de toute une classe de citoyens, qui est loin de passer pour la plus éclairée et la plus indépendante.}. Les uns vou­draient un jury recruté seulement parmi les classes éclairées ; mais nous avons déjà prouvé que, même dans ce cas, les décisions seront identiques à celles qui sont main­tenant rendues, D’autres, se basant sur les erreurs commises par les jurés, voudraient supprimer ces derniers et les remplacer par des juges. Mais comment peuvent-ils oublier que ces erreurs tant reprochées au jury, ce sont toujours des juges qui les ont d’abord commises, puisque, quand l’accusé arrive devant le jury, il a été considéré comme, coupable par plusieurs magistrats : le juge d’instruction, le procureur de la République et la chambre des mises en accusation. Et ne voit-on pas alors que si l’accusé était définitivement jugé par des magistrats au lieu de l’être par des jurés, il perdrait sa seule chance d’être reconnu innocent. Les erreurs des jurés ont toujours été d’abord des erreurs de magistrats. C’est donc uniquement à ces derniers qu’il faut s’en prendre quand on voit des erreurs judiciaires particulièrement monstrueuses, comme la condamnation de ce docteur X. qui, poursuivi par un juge d’instruction véritable­ment par trop borné, sur la dénonciation d’une fille demi-idiote qui accusait ce méde­cin de l’avoir fait avorter pour 30 francs aurait été envoyé au bagne sans l’explosion d’indignation publique qui le fit gracier immédiatement par le chef de l’État. L’hono­rabilité du condamné proclamée par tous ses concitoyens rendait évidente la grossiè­reté de l’erreur. Les magistrats la reconnaissaient eux-mêmes ; et cependant, par esprit de caste, ils firent tout ce qu’ils purent pour empêcher la grâce d’être signée. Dans toutes les affaires analogues, entourées de détails techniques où il ne peut rien com­prendre, le jury écoute naturellement le ministère public, se disant qu’après tout l’affaire a été instruite par des magistrats rompus à toutes les subtilités. Quels sont alors les auteurs véritables de l’erreur : les jurés ou les magistrats ? Gardons précieusement le jury. Il constitue peut-être la seule catégorie de foule qu’aucune individualité ne saurait remplacer. Lui seul peut tempérer les duretés de la loi qui, égale pour tous, doit être aveugle en principe, et ne pas connaître les cas particuliers. Inaccessible à la pitié, et ne connaissant que le texte de la loi, le juge, avec sa dureté professionnelle, frapperait de la même peine le cambrioleur assassin et la fille pauvre que l’abandon de son séducteur et la misère ont con­duite à l’infanticide alors que le jury sent très bien d’instinct que la fille séduite est beaucoup moins coupable que le séducteur, qui, lui, cependant, échappe à la loi et qu’elle mérite toute son indulgence.\par
Sachant très bien ce qu’est la psychologie des castes, et ce qu’est aussi la psycho­logie des autres catégories de foules, je ne vois aucun cas où, accusé à tort d’un crime, je ne préférerais pas avoir affaire à des jurés plutôt qu’à des magistrats. J’aurais beaucoup de chances d’être reconnu innocent avec les premiers, et très peu avec les seconds. Redoutons la puissance des foules, mais redoutons beaucoup plus encore la puissance de certaines castes. Les premières peuvent se laisser convaincre, les secon­des ne fléchissent jamais.
\subsection[{Chapitre 4. Les foules électorales.}]{Chapitre 4. Les foules électorales.}

\begin{argument}\noindent Caractères généraux des foules électorales. – Comment on les persuade. – Qualités que doit possé­der le candidat. – Nécessité du prestige. – Pourquoi ouvriers et paysans choisissent si rarement les candidats dans leur sein. – Puissance des mots et des formules sur l’électeur. – Aspect général des discussions électorales. – Comment se forment les opinions de l’électeur. – Puissance des co­mités. – Ils représentent la forme la plus redoutable de la tyrannie. – Les comités de la Révolution. – Malgré si faible valeur psychologique, le suffrage universel ne peut être remplacé. – Pourquoi les votes seraient identiques, alors même qu’on restreindrait le droit de suffrage à une classe limitée de citoyens. – Ce que traduit le suffrage universel dans tous les pays.
\end{argument}

\noindent Les foules électorales, c’est-à-dire les collectivités appelées à élire les titulaires de certaines fonctions, constituent des foules hétérogènes ; mais, comme elles n’agissent que sur un point bien déterminé : choisir entre divers candidats, on ne peut observer chez elles que quelques-uns des caractères précédemment décrits.\par
Les caractères des foules qu’elles manifestent surtout, sont la faible aptitude au raisonnement, l’absence d’esprit critique, l’irritabilité, la crédulité et le simplisme. on découvre aussi dans leurs décisions l’influence des meneurs et le rôle des facteurs précédemment énumérés : l’affirmation, la répétition, le prestige et la contagion.\par
Recherchons comment on les séduit. Des procédés qui réussissent le mieux, leur psychologie se déduira clairement.\par
La première des conditions à posséder pour le candidat est le prestige. Le prestige personnel ne peut être remplacé que par celui de la fortune. Le talent, le génie même ne sont pas des éléments de succès.\par
Cette nécessité pour le candidat de posséder du pres­tige, c’est-à-dire de pouvoir s’imposer sans discussion, est capitale. Si les électeurs, dont la majorité est composée d’ouvriers et de paysans, choisissent si rarement un des leurs pour les représenter, c’est que les personnalités sorties de leurs rangs n’ont pour eux aucun prestige. Quand, par hasard, ils nomment un de leurs égaux, c’est le plus souvent pour des raisons accessoires, par exemple pour contrecarrer un homme éminent, un patron puissant dans la dépendance duquel se trouve chaque jour l’électeur, et dont il a ainsi l’illusion de devenir pour un instant le maître.\par
Mais la possession du prestige ne suffit pas pour assurer au candidat le succès. L’électeur tient à ce qu’on flatte ses convoitises et ses vanités ; il faut l’accabler des plus extravagantes flagorneries, ne pas hésiter à lui faire les plus fantastiques promes­ses. S’il est ouvrier, on ne saurait trop injurier et flétrir ses patrons. Quant au candidat adverse, on doit tâcher de l’écraser en établissant par affirmation, répétition et conta­gion qu’il est le dernier des gredins, et que personne n’ignore qu’il a commis plusieurs crimes. Inutile, bien entendu, de chercher aucun semblant de preuve. Si l’adversaire connaît mal la psychologie des foules, il essaiera de se justifier par des arguments, au lieu de se borner à répondre aux affirmations par d’autres affirmations ; et il n’aura dès lors aucune chance de triompher.\par
Le programme écrit du candidat ne doit pas être trop catégorique, parce que ses adversaires pourraient le lui opposer plus tard ; mais son programme verbal ne sau­rait être trop excessif. Les réformes les plus considérables peuvent être promises sans crainte. Sur le moment, ces exagérations produisent beaucoup d’effet, et pour l’avenir elles n’engagent en rien. Il est d’observation constante, en effet, que l’électeur ne s’est jamais préoccupé de savoir jusqu’à quel point l’élu a suivi la profession de foi acclamée, et sur laquelle l’élection est supposée avoir eu lieu.\par
Nous reconnaissons ici tous les facteurs de persuasion que nous avons décrits. Nous allons les retrouver encore dans l’action des \emph{mots} et des \emph{formules} dont nous avons déjà montré le puissant empire. L’orateur qui sait les manier conduit à volonté les foules où il veut. Des expressions telles que : l’infâme capital, les vils exploi­teurs, l’admirable ouvrier, la socialisation des richesses, etc., produisent toujours le même effet, bien qu’un peu usées déjà. Mais le candidat qui trouve une formule neuve, bien dépourvue de sens précis, et par conséquent pouvant répondre aux aspira­tions les plus diverses, obtient un succès infaillible. La sanglante révolution espagnole de 1873 a été faite avec un de ces mots magiques, au sens complexe, que chacun peut interpréter à sa façon. Un écrivain contemporain en a raconté la genèse en termes qui méritent d’être rapportés.\par
“Les radicaux avaient découvert qu’une république unitaire est une monarchie déguisée, et, pour leur faire plaisir, les Cortès avaient proclamé d’une seule voix la république fédérale sans qu’aucun des votants eût pu dire ce qui venait d’être voté. Mais cette formule enchantait tout le monde, c’était une ivresse, un délire. On venait d’inaugurer sur la terre le règne de la vertu et du bonheur. Un républicain, à qui son ennemi refusait le titre de fédéral, s’en offensait comme d’une mortelle injure. On s’abordait dans les rues en se disant : \emph{Salud y republica federal} ! Après quoi on entonnait des hymnes à la sainte indiscipline et à l’autonomie du soldat. Qu’était-ce que la “république fédérale ?” Les uns entendaient par là l’émancipation des provin­ces, des institutions pareilles à celles des États-Unis ou la décentralisation adminis­trative ; autres visaient à l’anéantissement de toute autorité, à l’ouverture prochaine de la grande liquidation sociale. Les socialistes de Barcelone et de l’Andalousie prêchaient la souveraineté absolue des communes, ils entendaient donner à l’Espagne dix mille municipes indépendants, ne recevant de lois que d’eux-mêmes, en suppri­mant du même coup et l’armée et la gendarmerie. On vit bientôt dans les provinces du Midi l’insurrection se propager de ville en ville, de village en village. Dès qu’une commune avait fait son pronunciamiento, son premier soin était de détruire le télégraphe et les chemins de fer pour couper toutes ses communications avec ses voisins et avec Madrid. Il n’était pas de méchant bourg qui n’entendit faire sa cuisine à part. Le fédéralisme avait fait place à un cantonalisme brutal, incendiaire et massa­creur, et partout se célébraient de sanglantes saturnales.”\par
Quant à l’influence que pourraient avoir des raisonnements sur l’esprit des élec­teurs, il faudrait n’avoir jamais lu le compte rendu d’une réunion électorale pour n’être pas fixé à ce sujet. On y échange des affirmations, des invectives, parfois des horions, jamais des raisons. Si le silence s’établit pour un instant, c’est qu’un assistant au caractère difficile annonce qu’il va poser au candidat une de ces questions embar­rassantes qui réjouissent toujours l’auditoire. Mais la satisfaction des opposants ne dure pas bien longtemps, car la voix du préopinant est bientôt couverte par les hurlements des adversaires. On peut considérer comme type des réunions publiques les comptes rendus suivants, pris entre des centaines d’autres semblables, et que j’emprunte aux journaux quotidiens\par
“Un organisateur ayant prié les assistants de nommer un président, l’orage se déchaîne. Les anarchistes bon­dissent sur la scène pour enlever le bureau d’assaut. Les socialistes le défendent avec énergie ; on se cogne, on se traite mutuellement de mouchards, vendus, etc. un citoyen se retire avec un œil poché.\par
“Enfin, le bureau est installé tant bien que mal au milieu du tumulte, et la tribune reste au compagnon X.\par
“L’orateur exécute une charge à fond de train contre les socialistes, qui l’inter­rompent en criant : “Crétin ! bandit ! canaille !”, etc., épithètes auxquelles le compa­gnon X… répond par l’exposé d’une théorie selon laquelle les socialistes sont des “idiots” ou des “farceurs”.\par
“… Le parti allemaniste avait organisé, hier soir, à la salle du Commerce, rue du Faubourg-du-Temple, une grande réunion préparatoire à la fête des Travailleurs du premier mai. Le mot d’ordre était : “Calme et tranquillité.”\par
“Le compagnon G… traite les socialistes de “crétins et de “fumistes”.\par
“Sur ces mots, orateurs et auditeurs s’invectivent et en viennent aux mains ; les chaises, les bancs, les tables entrent en scène, etc., etc.”\par
N’imaginons pas un instant que ce genre de discussion soit spécial à une classe déterminée d’électeurs, et dépende de leur situation sociale. Dans toute assemblée anonyme, quelle qu’elle soit, fût-elle exclusivement composée de lettrés, la discus­sion revêt facilement les mêmes formes. J’ai montré que les hommes en foule tendent vers l’égalisation mentale, et à chaque instant nous en retrouvons la preuve. Voici, comme exemple, un extrait du compte rendu d’une réunion exclusivement composée d’étudiants, que j’emprunte à un journal :\par
“Le tumulte n’a fait que croître à mesure que la soirée s’avançait ; je ne crois pas qu’un seul orateur ait pu dire deux phrases sans être interrompu. A chaque instant les cris partaient d’un point ou de l’autre, ou de tous les points à la fois ; on applaudissait, on sifflait ; des discussions violentes s’engageaient entre divers auditeurs ; les cannes étaient brandies, menaçantes ; on frappait le plan­cher en cadence ; des clameurs poursuivaient les interrupteurs “A la porte ! À la tribune !”\par
“M-C… prodigue à l’association les épithètes d’odieuse et lâche, monstrueuse, vile, vénale et vindicative, et déclare qu’il veut la détruire, etc., etc.”.\par
On pourrait se demander comment, dans des conditions pareilles, peut se former l’opinion d’un électeur Mais poser une pareille question serait se faire une étrange illusion sur le degré de liberté dont peut jouir une collectivité. Les foules ont des opinions imposées, jamais des opinions raisonnées. Dans le cas qui nous occupe, les opinions et les votes des électeurs sont entre les mains de comités électoraux, dont les meneurs sont le plus souvent quelques marchands de vins, fort influents sur les ouvriers, auxquels ils font crédit.\par
Savez-vous ce qu’est un comité électoral, écrit un des plus vaillants défenseurs de la démocratie actuelle, M. Schérer ? Tout simplement la clef de nos institutions, la maîtresse pièce de la machine politique. La France est aujourd’hui gouvernée par les comités \footnote{Les comités, quels que soient leurs noms : clubs, syndicats, etc., constituent peut-être le plus redoutable danger de la puissance des foules. Ils représentent, en effet, la forme la plus imper­sonnelle, et, par conséquent, la plus oppressive de la tyrannie. Les meneurs qui dirigent les comités étant censés parler et agir au nom d’une collectivité sont dégagés de toute responsabilité et peuvent tout se permettre. Le tyran le plus farouche n’eût jamais osé rêver les proscriptions ordonnées par les comités révolutionnaires. Ils avaient, dit Barras, décimés et mis en coupe réglée la Convention. Robes­pierre fut maître absolu tant qu’il put parler en leur nom. Le jour où l’effroyable dictateur se sépara d’eux pour des questions d’amour-propre, il fut perdu. Le règne des foules c’est le règne des comités, c’est-à-dire des meneurs. On ne saurait rêver de despotisme plus dur.}.” Aussi n’est-il pas trop difficile d’agir sur eux, pour peu que le candidat soit acceptable et possède des ressources suffisantes. D’après les aveux des donateurs, 3 millions suffirent pour obtenir les élections multiples du général Boulanger.\par
Telle est la psychologie des foules électorales. Elle est identique à celle des autres foules. Ni meilleure ni pire.\par
Aussi ne tirerai-je de ce qui précède aucune conclusion contre le suffrage univer­sel. Si j’avais à décider de son sort, je le conserverais tel qu’il est, pour des motifs pratiques qui découlent précisément de notre étude de la psychologie des foules, et que pour cette raison je vais exposer.\par
Sans doute, les inconvénients du suffrage universel sont trop visibles pour être méconnus. On ne saurait contester que les civilisations ont été l’œuvre d’une petite minorité d’esprits supérieurs constituant la pointe d’une pyramide, dont les étages, s’élargissant à mesure que décroît la valeur mentale, représentent les couches profon­des d’une nation. Ce n’est pas assurément du suffrage d’éléments inférieurs, représen­tant uniquement le nombre, que la grandeur d’une civilisation peut dépendre. Sans doute encore les suffrages des foules sont souvent bien dangereux. Ils nous ont déjà coûté plusieurs invasions ; et avec le triomphe du socialisme, les fantaisies de la souveraineté populaire nous coûteront sûrement beaucoup plus cher encore.\par
Mais ces objections théoriquement excellentes perdent pratiquement toute leur force, si l’on veut se souvenir de la puissance invincible des idées transformées en dogmes. Le dogme de la souveraineté des foules est, au point de vue philosophique, aussi peu défendable que les dogmes religieux du moyen âge, mais il en a aujourd’hui l’absolue puissance. Il est donc aussi inattaquable que le furent jadis nos idées reli­gieuses. Supposez un libre-penseur moderne, transporté par un pouvoir magique en plein moyen âge. Croyez-vous qu’après avoir constaté la puissance souveraine des idées religieuses qui régnaient alors il eût tenté de les combattre ? Tombé dans les mains d’un juge voulant le faire brûler sous l’imputation d’avoir conclu un pacte avec le diable, ou d’avoir été au sabbat, eût-il songé à contester l’existence du diable et du sabbat ? On ne discute pas plus avec les croyances des foules qu’avec les cyclones. Le dogme du suffrage universel possède aujourd’hui le pouvoir qu’eurent jadis les dogmes chrétiens. Orateurs et écrivains en parlent avec un respect et des adulations que n’a pas connus Louis XIV. Il faut donc se conduire à son égard comme à l’égard de tous les dogmes religieux. Le temps seul agit sur eux.\par
Il serait d’ailleurs d’autant plus inutile d’essayer d’ébranler ce dogme qu’il a des raisons apparentes pour lui : “Dans les temps d’égalité, dit justement Tocqueville, les hommes n’ont aucune foi les uns dans les autres, à cause de leur similitude ; mais cette même similitude leur donne une confiance presque illimitée dans le jugement du public ; car il ne leur parait pas vraisemblable, qu’ayant tous des lumières pareilles, la vérité ne se rencontre pas du côté du plus grand nombre.”\par
Faut-il supposer maintenant qu’avec un suffrage restreint – restreint aux capacités, si l’on veut – on améliorerait les votes des foules ? Je ne puis l’admettre un seul instant, et cela pour les raisons que j’ai déjà dites de l’infériorité mentale de toutes les collectivités, quelle que puisse être leur composition. En foule les hommes s’égali­sent toujours, et, sur des questions générales, le suffrage de quarante académiciens n’est pas meilleur que celui de quarante porteurs d’eau. Je ne crois pas du tout qu’aucun des votes tant reprochés au suffrage universel, tel que le rétablissement de l’Empire, par exemple, eût été différent si les votants avaient été recrutés exclusive­ment parmi des savants et des lettrés. Ce n’est pas parce qu’un individu sait le grec ou les mathématiques, est architecte, vétérinaire, médecin ou avocat, qu’il acquiert sur les questions sociales des clartés particulières. Tous nos économistes sont des gens instruits, professeurs et académiciens pour la plupart. Est-il une seule question générale : protectionnisme, bimétallisme, etc., sur laquelle ils aient réussi à se mettre d’accord ? C’est que leur science n’est qu’une forme très atténuée de l’universelle ignorance. Devant des problèmes sociaux, où entrent de si multiples inconnues, toutes les ignorances s’égalisent.\par
Si donc des gens bourrés de science formaient à eux seuls le corps électoral, leurs votes ne seraient pas meilleurs que ceux d’aujourd’hui. Ils se guideraient surtout d’après leurs sentiments et l’esprit de leur parti. Nous n’aurions aucune des difficultés actuelles en moins, et en plus nous aurions sûrement la lourde tyrannie des castes.\par
Restreint ou général, sévissant dans un pays républicain ou dans un pays monar­chique, pratiqué en France, en Belgique, en Grèce, en Portugal ou en Espagne le suffrage des foules est partout semblable, et ce qu’il traduit souvent, ce sont les aspira­tions et les besoins inconscients de la race. La moyenne des élus représente pour chaque pays l’âme moyenne de la race. D’une génération à l’autre on la retrouve à peu près identique.\par
Et c’est ainsi qu’une fois encore nous retombons sur cette notion fondamentale de race, déjà rencontrée si souvent, et sur cette autre notion, qui découle de la première que les institutions et les gouvernements ne jouent qu’un rôle insignifiant dans la vie des peuples. Ces derniers sont surtout conduits par l’âme de leur race, c’est-à-dire par les résidus ancestraux dont cette âme est la somme. La race et l’engrenage des néces­sités de chaque jour, tels sont les maîtres mystérieux qui régissent nos destinées.
\subsection[{Chapitre 5. Les assemblées parlementaires}]{Chapitre 5. Les assemblées parlementaires}

\begin{argument}\noindent Les foules parlementaires présentent la plupart des caractères communs aux foules hétérogènes non anonymes. – Simplisme des opinions. – Suggestibilité et limites de cette suggestibilité. – Opinions fixes irréductibles et opi­nions mobiles. – Pourquoi l’indécision prédomine. – Rôle des meneurs. – Raison de leur prestige. – Ils sont les vrais maîtres d’une assemblée dont les votes ne sont ainsi que ceux d’une petite minorité. – Puissance absolue qu’ils exercent. – Les éléments de leur art oratoire. – Les mots et les images. – Nécessité psychologique pour les meneurs d’être généralement convaincus et bornés. – Impossi­bilité pour l’orateur sans prestige de faire admettre ses raisons. – L’exagération des sentiments, bons ou mauvais, dans les assemblées. – Automatisme auquel elles arrivent à certains moments. – Les séances de la Convention. – Cas dans lesquels une assemblée perd les caractères des foules. – Influence des spécialistes dans les questions techniques. – Avantages et dangers du régime parlementaire dans tous les pays. – Il est adapté aux nécessités modernes ; mais il entraîne le gaspillage des finances et la restriction progressive de toutes les libertés. – \emph{Conclusion de l’ouvrage.}
\end{argument}

\noindent Les assemblées parlementaires représentent des foules hétérogènes non anony­mes. Malgré leur recrutement, variable suivant les époques et les peuples, elles se ressemblent beaucoup par leurs caractères. L’influence de la race s’y fait sentir, pour atténuer ou exagérer, mais non pour empêcher la manifestation des caractères. Les assemblées parlementaires des contrées les plus différentes, celles de Grèce, d’Italie, de Portugal, d’Espagne, de France et d’Amérique, présentent dans leurs discussions et leurs votes de grandes analogies et laissent les gouvernements aux prises avec des difficultés identiques.\par
Le régime parlementaire représente d’ailleurs l’idéal de tous les peuples civilisés modernes. Il traduit cette idée, psychologiquement erronée mais généralement admi­se, que beaucoup d’hommes réunis sont bien plus capables qu’un petit nombre de prendre une décision sage et indépendante sur un sujet donné.\par
Nous retrouverons dans les assemblées parlementaires les caractéristiques géné­rales des foules : le simplisme des idées, l’irritabilité, la suggestibilité, l’exagération des sentiments, l’influence prépondérante des meneurs. Mais, en raison de leur com­position spéciale, les foules parlementaires présentent quelques différences que nous indiquerons bientôt.\par
Le simplisme des opinions est une des caractéristiques les plus importantes de ces assemblées. On y rencontre dans tous les partis, chez les peuples latins surtout, une tendance invariable à résoudre les problèmes sociaux les plus compliqués par les principes abstraits les plus simples, et par des lois générales applicables à tous les cas. Les principes varient naturellement avec chaque parti ; mais, par le fait seul que les individus sont en foule, ils tendent toujours à exagérer la valeur de ces principes et à les pousser jusqu’à leurs dernières conséquences. Aussi ce que les parlements représentent sur­tout, ce sont des opinions extrêmes.\par
Le type le plus parfait du simplisme des assemblées fut réalisé par les jacobins de notre grande Révolution Tous dogmatiques et logiques, la cervelle pleine de géné­ralités vagues, ils s’occupaient d’appliquer des principes fixes sans se soucier des événements ; et on a pu dire avec raison qu’ils avaient traversé la Révolution sans la voir. Avec les dogmes très simples qui leur servaient de guide, ils s’imaginaient refaire une société de toutes pièces, et ramener une civilisation raffinée à une phase très antérieure de l’évolution sociale. Les moyens qu’ils employèrent pour réaliser leur rêve étaient également empreints d’un absolu simplisme. Ils se bornaient en effet, à détruire violemment ce qui les gênait. Tous, d’ailleurs : girondins, montagnards, ther­midoriens, etc., étaient animés du même esprit.\par
Les foules parlementaires sont très suggestibles ; et, comme pour toutes les foules, la suggestion émane de meneurs possédant du prestige ; mais, dans les assem­blées parlementaires, la suggestibilité a des limites très nettes qu’il importe de marquer.\par
Sur toutes les questions d’intérêt local ou régional, chaque membre d’une assem­blée a des opinions fixes, irréductibles, et qu’aucune argumentation ne pourrait ébranler. Le talent d’un Démosthène n’arriverait pas à changer le vote d’un député sur des questions telles que le protectionnisme ou le privilège des bouilleurs de cru, qui représentent des exigences d’électeurs influents. La suggestion antérieure de ces élec­teurs est assez prépondérante pour annuler toutes les autres suggestions, et maintenir une fixité absolue d’opinion \footnote{C’est à ces opinions antérieurement fixées et rendues irréductibles par des nécessités électorales, que s’applique sans doute cette réflexion d’un vieux parlementaire anglais : “Depuis cinquante ans que je siège à Westminster, j’ai entendu des milliers de discours ; il en est peu qui aient changé mon opinion ; mais pas un seul n’a changé mon vote.”}.\par
Sur des questions générales : renversement d’un ministère, établissement d’un impôt, etc., il n’y a plus du tout de fixité d’opinion, et les suggestions des meneurs peuvent agir, mais pas tout à fait comme dans une foule ordinaire. Chaque parti a ses meneurs, qui ont parfois une égale influence. Il en résulte que le député se trouve entre des suggestions contraires et devient fatalement très hésitant. C’est pourquoi on le voit souvent, à un quart d’heure de distance, voter de façon contraire, ajouter à une loi un article qui la détruit : ôter par exemple aux industriels le droit de choisir et de congédier leurs ouvriers, puis annuler à peu près cette mesure par un amendement.\par
Et c’est pourquoi, à chaque législature, une Chambre a des opinions très fixes et d’autres opinions très indécises. Au fond, les questions générales étant les plus nom­breuses, c’est l’indécision qui domine, indécision entretenue par la crainte constante de l’électeur, dont la suggestion latente tend toujours à contrebalancer l’influence des meneurs.\par
Ce sont cependant les meneurs qui sont en définitive les vrais maîtres dans les discussions nombreuses où les membres d’une assemblée n’ont pas d’opinions antérieures bien arrêtées.\par
La nécessité de ces meneurs est évidente puisque, sous le nom de chefs de grou­pes, on les retrouve dans les assemblées de tous les pays. Ils sont les vrais souverains d’une assemblée. Les hommes en foule ne sauraient se passer d’un maître. Et c’est pourquoi les votes d’une assemblée ne représentent généralement que les opinions d’une petite minorité.\par
Les meneurs agissent très peu par leurs raisonnements, beaucoup par leur presti­ge. Et la meilleure preuve, c’est que si une circonstance quelconque les en dépouille, ils n’ont plus d’influence.\par
Ce prestige des meneurs est individuel et ne tient ni au nom ni à la célébrité. M. Jules Simon parlant des grands hommes de l’assemblée de 1848, où il a siégé, nous en donne de bien curieux exemples.\par
“Deux mois avant d’être tout-puissant, Louis-Napoléon n’était rien.\par
“Victor Hugo monta à la tribune. Il n’y eut pas de succès. On l’écouta, comme on écoutait Félix Pyat ; on ne l’applaudit pas autant. “Je n’aime pas ses idées, me dit Vaulabelle en parlant de Félix Pyat ; mais c’est un des plus grands écrivains et le plus grand orateur de la France.” Edgar Quinet, ce rare et puissant esprit, n’était compté pour rien. Il avait eu son moment de popularité avant l’ouverture de l’Assemblée ; dans l’Assemblée, il n’en eut aucune.\par
“Les assemblées politiques sont le lieu de la terre où l’éclat du génie se fait le moins sentir. On n’y tient compte que d’une éloquence appropriée au temps et au lieu, et des services rendus non à la patrie, mais aux partis. Pour qu’on rendit hommage à Lamartine en 1848 et à Thiers en 1871, il fallut le stimulant de l’intérêt urgent, inexo­rable. Le danger passé, on fut guéri à la fois de la recon­naissance et de la peur.\par
J’ai reproduit le passage qui précède pour les faits qu’il contient, mais non pour les explications, qu’il propose.\par
Elles sont d’une psychologie médiocre. Une foule perdrait aussitôt son caractère de foule si elle tenait compte aux meneurs des services rendus, que ce soit à la patrie ou aux partis. La foule qui obéit au meneur subit son prestige, et n’y fait intervenir aucun sentiment d’intérêt ou de reconnaissance.\par
Aussi le meneur doué d’un prestige suffisant possède-t-il un pouvoir presque absolu. On sait l’influence immense qu’eut pendant de longues années, grâce à son prestige, un député célèbre, battu dans les dernières élections à la suite de certains événements financiers. Sur un simple signe de lui, les ministres étaient renversés. Un écrivain a marqué nettement dans les lignes suivantes la portée de son action.\par
“C’est à M. X… principalement que nous devons d’avoir acheté le Tonkin trois fois plus cher qu’il n’aurait dû coûter, de n’avoir pris dans Madagascar qu’un pied incertain, de nous être laissé frustrer de tout un empire sur le bas Niger, d’avoir perdu la situation prépondérante que nous occupions en Égypte. – Les théories de M. X… nous ont coûté plus de territoires que les désastres de Napoléon I\textsuperscript{er}.\par
Il ne faudrait pas trop en vouloir au meneur en question. Il nous a coûté fort cher évidemment ; mais une grande partie de son influence tenait à ce qu’il suivait l’opi­nion publique, qui, en matière coloniale, n’était pas du tout alors ce qu’elle est devenue aujourd’hui. Il est rare qu’un meneur précède l’opinion ; presque toujours il se borne à la suivre et à en épouser toutes les erreurs.\par
Les moyens de persuasion des meneurs, en dehors du prestige, sont les facteurs que nous avons déjà énumérés plusieurs fois. Pour les manier habilement, le meneur doit avoir pénétré, au moins d’une façon inconsciente, la psychologie des foules, et savoir comment leur parler. Il doit surtout connaître la fascinante influence des mots, des formules et des images. Il doit posséder une éloquence spéciale, composée : d’affirmations énergiques, dégagées de preuves, et d’images impressionnantes enca­drées de raisonnements fort sommaires. C’est un genre d’éloquence qu’on rencontre dans toutes les assemblées, y compris le parlement anglais, le plus pondéré pourtant de tous.\par
“Nous pouvons lire constamment, dit le philosophe anglais Maine, des débats à la Chambre des communes, où toute la discussion consiste à échanger des généralités assez faibles et des personnalités assez violentes. Sur l’imagination d’une démocratie pure, ce genre de for­mules générales exerce un effet prodigieux. Il sera toujours aisé de faire accepter à une foule des assertions générales présentées en termes saisissants, quoiqu’elles n’aient jamais été vérifiées et ne soient peut-être susceptibles d’aucune vérification.”\par
L’importance des “termes saisissants”, indiquée dans la citation qui précède, ne saurait être exagérée. Nous avons plusieurs fois déjà insisté sur la puissance spéciale des mots et des formules. Il faut les choisir de façon à ce qu’ils évoquent des images très vives. La phrase suivante, empruntée au discours d’un des meneurs de nos assemblées, en constitue un excellent spécimen :\par
“Le jour où le même navire emportera vers les terres fiévreuses de la relégation le politicien véreux et l’anarchiste meurtrier, ils pourront lier conversation et ils s’apparaîtront l’un à l’autre comme les deux aspects complémentaires d’un même ordre social.”\par
L’image ainsi évoquée est bien visible, et tous les adversaires de l’orateur se sentent menacés par elle. Ils voient du même coup les pays fiévreux, le bâtiment qui pourra les emporter, car ne font-ils pas peut-être partie de la catégorie assez mal limitée des politiciens menacés ? Ils éprouvent alors la sourde crainte que devaient ressentir les conventionnels, que les vagues discours de Robespierre menaçaient plus ou moins du couperet de la guillotine, et qui, sous l’influence de cette crainte, lui cédaient toujours.\par
Les meneurs ont tout intérêt à verser dans les plus invraisemblables exagérations. L’orateur dont je viens de citer une phrase, a pu affirmer, sans soulever de grandes protestations, que les banquiers et les prêtres soudoyaient les lanceurs de bombes, et que les administrateurs des grandes compagnies financières méritent les mêmes peines que les anarchistes. Sur les foules, de pareilles affirmations agissent toujours. L’affirmation n’est jamais trop furieuse, ni la déclamation trop menaçante. Rien n’inti­mide plus les auditeurs que cette éloquence. En protestant, ils craignent de passer pour traîtres ou complices.\par
Cette éloquence spéciale a toujours régné, comme je le disais à l’instant, sur toutes les assemblées ; et, dans les périodes critiques, elle ne fait que s’accentuer. La lecture des discours des grands orateurs qui composaient les assemblées de la Révolution est très intéressant à ce point de vue. A chaque instant ils se croyaient obligés de s’inter­rompre pour flétrir le crime et exalter la vertu ; puis, ils éclataient en imprécations contre les tyrans, et juraient de vivre libres ou de mourir. L’assistance se levait, applaudissait avec fureur, puis calmée, se rasseyait.\par
Le meneur peut être quelquefois intelligent et instruit ; mais cela lui est géné­ralement plus nuisible qu’utile. En montrant la complexité des choses, en permettant d’expliquer et de comprendre, l’intelligence rend toujours indulgent, et émousse fortement l’intensité et la violence des convictions nécessaires aux apôtres. Les grands meneurs de tous les âges, ceux de la Révolution surtout ont été lamentablement bornés ; et ce sont justement les plus bornés qui ont exercé la plus grande influence.\par
Les discours du plus célèbre d’entre eux, Robes­pierre, stupéfient souvent par leur incohérence ; en se bornant à les lire, on n’y trouverait aucune explication plausible du rôle immense du puissant dictateur :\par
“Lieux communs et redondances de l’éloquence pédagogique et de la culture latine au service d’une âme plu­tôt puérile que plate, et qui semble se borner, dans l’attaque ou la défense, au “Viens-y donc !”, des écoliers. Pas une idée, pas un tour, pas un trait, c’est l’ennui dans la tempête. Quand on sort de cette lecture morne, on a envie de pousser le ouf ! de l’aimable Camille Desmoulins.”\par
Il est quelquefois effrayant de songer au pouvoir que donne à un homme possé­dant du prestige une conviction forte unie à une extrême étroitesse d’esprit. Il faut pourtant réaliser ces conditions pour ignorer les obstacles et savoir vouloir. D’instinct les foules reconnaissent dans ces convaincus énergiques le maître qu’il leur faut toujours.\par
Dans une assemblée parlementaire, le succès d’un discours dépend presque uni­quement du prestige que l’orateur possède, et pas du tout des raisons qu’il propose. Et, la meilleure preuve, c’est que lorsqu’une cause quelconque fait perdre à un orateur son prestige, il perd du même coup toute son influence, c’est-à-dire le pouvoir de diriger à son gré les votes.\par
Quant à l’orateur inconnu qui arrive avec un discours contenant de bonnes raisons, mais seulement des raisons, il n’a aucune chance d’être seulement écouté. Un ancien député M. Descubes a récemment tracé dans les lignes suivantes l’image du député sans prestige :\par
“Quand il a pris place à la tribune, il tire de sa serviette un dossier qu’il étale méthodiquement devant lui et débute avec assurance.\par
Il se flatte de faire passer dans l’âme des auditeurs la conviction qui l’anime. Il a pesé et repesé ses arguments ; il est tout bourré de chiffres et de preuves ; il est sûr d’avoir raison. Toute résistance, devant l’évidence qu’il apporte, sera vaine. Il com­mence, confiant dans son bon droit et aussi dans l’attention de ses collègues, qui certainement ne demandent qu’à s’incliner devant la vérité..\par
Il parle, et, tout de suite il est surpris du mouvement de la salle, un peu agacé par le brouhaha qui s’en élève.\par
Comment le silence ne se fait-il pas ? Pourquoi cette inattention générale ? A quoi pensent donc ceux-là qui causent entre eux ? Quel motif si urgent fait quitter sa place à cet autre ?\par
Une inquiétude passe sur son front. Il fronce les sourcils, s’arrête. Encouragé par le président, il repart, haussant la voix. On ne l’en écoute que moins. Il force le ton, il s’agite : le bruit redouble autour de lui. Il ne s’entend plus lui-même, s’arrête enco­re ; puis, craignant que son silence ne provoque le fâcheux cri de : \emph{Clôture} ! il reprend de plus belle. Le vacarme devient insupportable.”\par
Lorsque les assemblées parlementaires se trouvent montées à un certain degré d’excitation, elles deviennent identiques aux foules hétérogènes ordinaire, et leurs sentiments présentent par conséquent la particularité d’être toujours extrêmes. On les verra se porter aux plus grands actes d’héroïsme ou aux pires excès. L’individu n’est plus lui-même, et il l’est si peu qu’il votera les mesures les plus contraires à ses intérêts personnels.\par
L’histoire de la Révolution montre à quel point les assemblées peuvent devenir inconscientes et obéir aux suggestions les plus contraires à leurs intérêts. C’était un sacrifice énorme pour la noblesse de renoncer à ses privilèges, et pourtant, dans une nuit célèbre de la Constituante, elle le fit sans hésiter. C’était une menace permanente de mort pour les conventionnels de renoncer à leur inviolabilité, et pourtant ils le firent et ne craignirent pas de se décimer réciproquement, sachant bien cependant que l’échafaud où ils envoyaient aujourd’hui des collègues leur était réservé demain.\par
Mais ils étaient arrivés à ce degré d’automatisme complet que j’ai décrit, et aucune considération ne pouvait les empêcher de céder aux suggestions qui les hypnotisaient. Le passage suivant des mémoires de l’un d’eux, Billaud-Varennes, est absolument typique sur ce point : “Les décisions que l’on nous reproche tant, dit-il, \emph{nous ne les voulions pas le plus souvent deux jours, un jour auparavant : la crise seule les suscitait.”} Rien n’est plus juste.\par
Les mêmes phénomènes d’inconscience se manifestèrent pendant toutes les séances orageuses de la Convention.\par
“Ils approuvent et décrètent, dit Taine, ce dont ils ont horreur, non seulement les sottises et les folies, mais les crimes, le meurtre des innocents, le meurtre de leur amis. A l’unanimité et avec les plus vifs applaudissements, la gauche, réunie à la droite, envoie à l’échafaud Danton, son chef naturel, le grand promoteur et conducteur de la Révolution. À l’unanimité et avec les plus grands applaudissements, la droite, réunie à la gauche, vote les pires décrets du gouvernement révolutionnaire. A l’unani­mité, et avec des cris d’admiration et d’enthousiasme, avec des témoignages de sympathie passionnée pour Collot d’Herbois, pour Couthon et pour Robespierre, la Convention, par des réélections spontanées et multiples, maintient en place le gouvernement homicide que la Plaine déteste parce qu’il est homicide, et que la Mon­tagne déteste parce qu’il la décime. Plaine et Montagne, la majorité et la minorité finissent par consentir à aider à leur propre suicide. Le 22 prairial, la Convention tout entière a tendu la gorge ; le 8 thermidor, pendant le premier quart d’heure qui a suivi le discours de Robes­pierre, elle l’a tendue encore.”\par
Le tableau peut paraître sombre. Il est exact pourtant. Les assemblées parlemen­taires suffisamment excitées et hypnotisées présentent les mêmes caractères. Elles deviennent un troupeau mobile obéissant à toutes les impulsions. La description suivante de l’assemblée de 1848, due à un parlementaire dont on ne suspectera pas la foi démocratique, M. Spuller, et que je reproduis d’après la \emph{Revue littéraire}, est bien typique. On y retrouve tous les sentiments exagérés que j’ai décrits dans les foules, et cette mobilité excessive qui permet de passer d’un instant à l’autre par la gamme des sentiments les plus contraires.\par
“Les divisions, les jalousies, les soupçons, et tour à tour la confiance aveugle et les espoirs illimités ont conduit le parti républicain à sa perte. Sa naïveté et sa candeur n’avaient d’égale que sa défiance universelle. Aucun sens de la légalité, nulle intelli­gence de la discipline : des terreurs et des illusions sans bornes : le paysan et l’enfant se rencontrent en ce point. Leur calme rivalise avec leur impatience. Leur sauvagerie est pareille à leur docilité. C’est le propre d’un tempérament qui n’est point fait et d’une éducation absente. Bien ne les étonne et tout les déconcerte. Tremblants, peu­reux, intrépides, héroïques, ils se jetteront à travers les flammes et ils reculeront devant une ombre.\par
“Ils ne connaissent point les effets et les rapports des choses. Aussi prompts aux découragements qu’aux exaltations, sujets à toutes les paniques, toujours trop haut ou trop bas, jamais au degré qu’il faut et dans la mesure qui convient. Plus fluides que l’eau, ils reflètent toutes les couleurs et prennent toutes les formes. Quelle base de gouvernement pouvait-on espérer d’asseoir en eux ?”\par
Il s’en faut de beaucoup heureusement que tous les caractères que nous venons de décrire dans les assemblées parlementaires se manifestent constamment. Elles ne sont foules qu’à certains moments. Les individus qui les composent arrivent à garder leur individualité dans un grand nombre de cas ; et c’est pourquoi une assemblée peut élaborer des lois techniques excellentes. Ces lois ont, il est vrai, pour auteur un homme spécial qui les a préparées dans le silence du cabinet ; et la loi votée est en réalité l’œuvre d’un individu, et non plus celle d’une assemblée. Ce sont naturelle­ment ces lois qui sont les meilleures. Elles ne deviennent désastreuses que lorsqu’une série d’amendements malheureux les rendent collectives. L’œuvre d’une foule est partout et toujours inférieure à celle d’un individu isolé. Ce sont les spécialistes qui sauvent les assemblées des mesures trop désordonnées et trop inexpérimentées. Le spécialiste est alors un meneur momentané. L’assemblée n’agit pas sur lui et il agit sur elle.\par
Malgré toutes les difficultés de leur fonctionnement, les assemblées parlemen­taires représentent ce que les peuples ont encore trouvé de meilleur pour se gouver­ner et surtout pour se soustraire le plus possible au joug des tyrannies personnelles. Elles sont certainement l’idéal d’un gouvernement, au moins pour les philosophes, les penseurs, les écrivains, les artistes et les savants, en un mot pour tout ce qui constitue le sommet d’une civilisation.\par
En fait, d’ailleurs, elles ne présentent que deux dangers sérieux, l’un est un gas­pillage forcé des finances, l’autre une restriction progressive des libertés individuelles.\par
Le premier de ces dangers est la conséquence forcée des exigences et de l’impré­voyance des foules électorales. Qu’un membre d’une assemblée propose une mesure donnant une satisfaction apparente à des idées démocratiques, telle qu’assurer, par exemple, des retraites à tous les ouvriers, augmenter le traitement des cantonniers, des instituteurs, etc., les autres députés, suggestionnés par la crainte des électeurs, n’oseront pas avoir l’air de dédaigner les intérêts de ces derniers en repoussant la mesure proposée, bien que sachant qu’elle grèvera lourdement le budget et nécessitera la création de nouveaux impôts. Hésiter dans le vote leur est impossible. Les consé­quences de l’accroissement des dépenses sont lointaines et sans résultats bien fâcheux pour eux, alors que les conséquences d’un vote négatif pourraient apparaître claire­ment le jour prochain où il faudra se représenter devant l’électeur.\par
À côté de cette première cause d’exagération des dépenses il en est une autre, non moins impérative obligation d’accorder toutes les dépenses d’intérêt pure­ment local. Un député ne saurait s’y opposer, parce qu’elles représentent encore des exigences d’électeurs, et que chaque député ne peut obtenir ce dont il a besoin pour sa circons­cription qu’à la condition de céder aux demandes analogues de ses collègues \footnote{Dans son numéro du 6 avril 1895, \emph{l’Economiste} faisait une revue curieuse de ce que peuvent coûter en une année ces dépenses d’intérêt purement électoral, notamment celles des chemins de fer. Pour relier Langayes (ville de 3.000 habi­tants), juchée sur une montagne, au Puy, vote d’un chemin de fer qui coûtera 15 millions. Pour relier Beaumont (3.500 ha­bitants) à Castel-Sarrazin, 7 millions. Pour relier le village de Oust (523 habitants) à celui de Seix (1.200 habitants) 7 mil­lions. Pour relier Prades à la bourgade d’Olette (717 habi­tants), 6 millions, etc. Rien que pour 1895, 90 millions de voies ferrées dépourvues de tout intérêt général ont été votés. D’autres dépenses de nécessités également électorales ne sont pas moins importantes. La loi sur les retraites ouvrières coûtera bientôt un minimum annuel de 165 millions d’après le ministre des finances, et de 800 millions suivant l’académi­cien Leroy-Beaulieu. Évidemment la progression continue de telles dépenses a forcément cour issue la faillite. Beaucoup de pays en Europe : le Portugal, la Grèce, l’Espagne, la Turquie, y sont arrivés ; d’autres vont y être acculés bientôt ; mais il ne faut pas trop s’en préoccuper, puisque le public a successivement accepté sans grandes protes­tations des réductions des quatre cinquièmes dans le paie­ment des coupons par divers pays. Ces ingénieuses faillites permettent alors de remettre instantanément les budgets avariés en équilibre. Les guerres, le socialisme, les luttes économiques nous préparent d’ailleurs de bien autres catastrophes, et à l’époque de désagrégation universelle où nous sommes entrés, il faut se résigner à vivre au jour le jour sans trop se soucier de lendemains qui nous échappent.}.\par
Le second des dangers mentionnés plus haut, la restriction forcée des libertés par les assemblées parlementaires, moins visible en apparence est cependant fort réel. Il est la conséquence des innombrables lois, toujours restrictives, dont les parlements, avec leur esprit simpliste, voient mal les conséquences, et qu’ils se croient obligés de voter.\par
Il faut que ce danger soit bien inévitable, puisque l’Angleterre elle-même, qui offre assurément le type le plus parfait du régime parlementaire, celui où le repré­sentant est le plus indépendant de son électeur, n’a pas réussi à s’y soustraire. Herbert Spencer, dans un travail déjà ancien, avait montré que l’accroissement de la liberté apparente devait être suivi d’une diminution de la liberté réelle. Reprenant la même thèse dans son livre récent, \emph{l’Individu contre l’État}, il s’exprime ainsi au sujet du parlement anglais :\par
“Depuis cette époque la législation a suivi le cours que j’indiquais. Des mesures dictatoriales, se multipliant rapidement, ont continuellement tendu à restreindre les libertés individuelles, et cela de deux manières : des réglementations ont été établies, chaque année en plus grand nombre, qui imposent une contrainte au citoyen là où ses actes étaient auparavant complètement libres, et le forcent à accomplir des actes qu’il pouvait auparavant accomplir ou ne pas accomplir, à volonté. En même temps des charges publiques, de plus en plus lourdes, surtout locales, ont restreint davantage sa liberté en diminuant cette portion de ses profits qu’il peut dépenser à sa guise, et en augmentant la portion qui lui est enlevée pour être dépensée selon le bon plaisir des agents publics.”\par
Cette restriction progressive des libertés se manifeste pour tous les pays sous une forme spéciale, que Herbert Spencer n’a pas indiquée, et qui est celle-ci : La création de ces séries innombrables de mesures législatives, toutes généralement d’ordre restrictif, conduit nécessairement à augmenter le nombre, le pouvoir et l’influence des fonctionnaires chargés de les appliquer. Ils tendent ainsi progressivement à devenir les véritables maîtres des pays civilisés. Leur puissance est d’autant plus grande, que, dans les incessants changements de pou­voir, la caste administrative est la seule qui échappe à ces changements, la seule qui possède l’irresponsabilité, l’impersonnalité et la perpétuité. Or, de tous les despotismes, il n’en est pas de plus lourds que ceux qui se présentent sous cette triple forme.\par
Cette création incessante de lois et de règlements restrictifs entourant des forma­lités les plus byzantines les moindres actes de la vie, a pour résultat fatal de rétrécir de plus en plus la sphère dans laquelle les citoyens peuvent se mouvoir librement. Victimes de cette illusion qu’en multipliant les lois l’égalité et la liberté se trouvent mieux assurées, les peuples acceptent chaque jour de plus pesantes entraves.\par
Ce n’est pas impunément qu’ils les acceptent. Habitués à supporter tous les jougs, ils finissent bientôt par les rechercher, et arrivent à perdre toute spontanéité et toute énergie. Ils ne sont plus alors que des ombres vaines, des automates passifs,. sans volonté, sans résistance et sans force.\par
Mais alors les ressorts que l’homme ne trouve plus en lui-même, il est bien forcé de les chercher hors de lui-même. Avec l’indifférence et l’impuissance croissantes des citoyens, le rôle des gouvernements est obligé de grandir encore. Ce sont eux qui doivent avoir forcément l’esprit d’initiative, d’entreprise et de conduite que les particuliers n’ont plus. Il leur faut tout entreprendre, tout diriger, tout protéger. L’État devient un dieu tout-puissant. Mais l’expérience enseigne que le pouvoir de tels dieux ne fut jamais ni bien durable, ni bien fort.\par
Cette restriction progressive de toutes les libertés chez certains peuples, malgré une licence extérieure qui leur donne l’illusion de les posséder, semble être une conséquence de leur vieillesse tout autant que celle d’un régime quelconque. Elle constitue un des symptômes précurseurs de cette phase de décadence à laquelle aucune civilisation n’a pu échapper jusqu’ici.\par
Si l’on en juge par les enseignements du passé et par des symptômes qui éclatent de toutes parts, plusieurs de nos civilisations modernes sont arrivées à cette phase d’extrême vieillesse qui précède la décadence. Il semble que des phases identiques soient fatales pour tous les peuples, puisque l’on voit si souvent l’histoire en répéter le cours.\par
Ces phases d’évolution générale des civilisations, il est facile de les marquer sommairement, et c’est avec leur résumé que se terminera notre ouvrage.\par
Si nous envisageons dans leurs grandes lignes la genèse de la grandeur et de la décadence des civilisations qui ont précédé la nôtre, que voyons-nous ?\par
A l’aurore de ces civilisations une poussière d’hommes, d’origines variées, réunie par les hasards des migrations, des invasions et des conquêtes. De sangs divers, de langues et de croyances également diverses, ces hommes n’ont de lien commun que la loi à demi reconnue d’un chef. Dans ces agglomérations confuses se retrouvent au plus haut degré les caractères psychologiques des foules. Elles en ont la cohésion momentanée, les héroïsmes, les faiblesses, les impulsions et les violences. Rien n’est stable en elles. Ce sont des barbares.\par
Puis le temps accomplit son œuvre. L’identité des milieux, la répétition des croisements, les nécessités d’une vie commune, agissent lentement. L’agglomération d’unités dissemblables commence à se fusionner et à former une race, c’est-à-dire un agrégat possédant des caractères et des sentiments communs, que l’hérédité va fixer de plus en plus. La foule est devenue un peuple, et ce peuple va pouvoir sortir de la barbarie.\par
Il n’en sortira tout à fait pourtant que quand, après de longs efforts, des luttes sans cesse répétées et d’innombrables recommencements, il aura acquis un idéal. Peu importe la nature de cet idéal, que ce soit le culte de Rome, la puissance d’Athènes ou le triomphe d’Allah, il suffira pour donner à tous les individus de la race en voie de formation une parfaite unité de sentiments et de pensées.\par
C’est alors que peut naître une civilisation nouvelle avec ses institutions, ses croyances et ses arts. Entraînée par son rêve, la race acquerra successivement tout ce qui donne l’éclat, la force et la grandeur. Elle sera foule encore sans doute à certaines heures, mais alors, derrière les caractères mobiles et changeants des foules, se trou­vera ce substratum solide, l’âme de la race, qui limite étroitement l’étendue des oscillations d’un peuple et règle le hasard.\par
Mais, après avoir exercé son action créatrice, le temps commence cette œuvre de destruction à laquelle n’échappent ni les dieux ni les hommes. Arrivée à un certain niveau de puissance et de complexité, la civilisation cesse de grandir, et, dès qu’elle ne grandit plus, elle est condamnée à décliner bientôt. L’heure de la vieillesse va sonner pour elle.\par
Cette heure inévitable est toujours marquée par l’affaiblissement de l’idéal qui soutenait l’âme de la race. A mesure que cet idéal pâlit, tous les édifices religieux, politiques ou sociaux dont il était l’inspirateur commencent à s’ébranler.\par
Avec l’évanouissement progressif de son idéal, la race perd de plus en plus ce qui faisait sa cohésion, son unité et sa force. L’individu peut croître en personnalité et en intelligence, mais en même temps aussi l’égoïsme collectif de la race est remplacé par un développement excessif de l’égoïsme individuel accompagné par l’affaissement du caractère et par l’amoindrissement de l’aptitude à l’action. Ce qui formait un peuple, une unité, un bloc, finit par devenir une agglomération d’individus sans cohésion et que maintiennent artificiellement pour quelque temps encore les traditions et les institutions.\par
C’est alors que, divisé par leurs intérêts et leurs aspirations, ne sachant plus se gouverner, les hommes demandent à être dirigés dans leurs moindres actes, et que l’État exerce son influence absorbante.\par
Avec la perte définitive de l’idéal ancien, la race finit par perdre entièrement son âme ; elle n’est plus qu’une poussière d’individus isolés et redevient ce qu’elle était à son point de départ : une foule. Elle en a tous les caractères transitoires sans consis­tance et sans lendemain. La civilisation n’a plus aucune fixité et est à la merci de tous les hasards. La plèbe est reine et les barbares avancent. La civilisation peut sembler brillante encore parce qu’elle possède la façade extérieure qu’un long passé a créée, mais c’est en réalité un édifice vermoulu que rien ne soutient plus et qui s’effondrera au premier orage.\par
Passer de la barbarie à la civilisation en poursuivant un rêve, puis décliner et mourir dès que ce rêve a perdu sa force, tel est le cycle de la vie d’un peuple.\par
Fin
 


% at least one empty page at end (for booklet couv)
\ifbooklet
  \pagestyle{empty}
  \clearpage
  % 2 empty pages maybe needed for 4e cover
  \ifnum\modulo{\value{page}}{4}=0 \hbox{}\newpage\hbox{}\newpage\fi
  \ifnum\modulo{\value{page}}{4}=1 \hbox{}\newpage\hbox{}\newpage\fi


  \hbox{}\newpage
  \ifodd\value{page}\hbox{}\newpage\fi
  {\centering\color{rubric}\bfseries\noindent\large
    Hurlus ? Qu’est-ce.\par
    \bigskip
  }
  \noindent Des bouquinistes électroniques, pour du texte libre à participation libre,
  téléchargeable gratuitement sur \href{https://hurlus.fr}{\dotuline{hurlus.fr}}.\par
  \bigskip
  \noindent Cette brochure a été produite par des éditeurs bénévoles.
  Elle n’est pas faîte pour être possédée, mais pour être lue, et puis donnée.
  Que circule le texte !
  En page de garde, on peut ajouter une date, un lieu, un nom ; pour suivre le voyage des idées.
  \par

  Ce texte a été choisi parce qu’une personne l’a aimé,
  ou haï, elle a en tous cas pensé qu’il partipait à la formation de notre présent ;
  sans le souci de plaire, vendre, ou militer pour une cause.
  \par

  L’édition électronique est soigneuse, tant sur la technique
  que sur l’établissement du texte ; mais sans aucune prétention scolaire, au contraire.
  Le but est de s’adresser à tous, sans distinction de science ou de diplôme.
  Au plus direct ! (possible)
  \par

  Cet exemplaire en papier a été tiré sur une imprimante personnelle
   ou une photocopieuse. Tout le monde peut le faire.
  Il suffit de
  télécharger un fichier sur \href{https://hurlus.fr}{\dotuline{hurlus.fr}},
  d’imprimer, et agrafer ; puis de lire et donner.\par

  \bigskip

  \noindent PS : Les hurlus furent aussi des rebelles protestants qui cassaient les statues dans les églises catholiques. En 1566 démarra la révolte des gueux dans le pays de Lille. L’insurrection enflamma la région jusqu’à Anvers où les gueux de mer bloquèrent les bateaux espagnols.
  Ce fut une rare guerre de libération dont naquit un pays toujours libre : les Pays-Bas.
  En plat pays francophone, par contre, restèrent des bandes de huguenots, les hurlus, progressivement réprimés par la très catholique Espagne.
  Cette mémoire d’une défaite est éteinte, rallumons-la. Sortons les livres du culte universitaire, cherchons les idoles de l’époque, pour les briser.
\fi

\ifdev % autotext in dev mode
\fontname\font — \textsc{Les règles du jeu}\par
(\hyperref[utopie]{\underline{Lien}})\par
\noindent \initialiv{A}{lors là}\blindtext\par
\noindent \initialiv{À}{ la bonheur des dames}\blindtext\par
\noindent \initialiv{É}{tonnez-le}\blindtext\par
\noindent \initialiv{Q}{ualitativement}\blindtext\par
\noindent \initialiv{V}{aloriser}\blindtext\par
\Blindtext
\phantomsection
\label{utopie}
\Blinddocument
\fi
\end{document}
