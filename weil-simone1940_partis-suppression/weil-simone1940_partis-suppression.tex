%%%%%%%%%%%%%%%%%%%%%%%%%%%%%%%%%
% LaTeX model https://hurlus.fr %
%%%%%%%%%%%%%%%%%%%%%%%%%%%%%%%%%

% Needed before document class
\RequirePackage{pdftexcmds} % needed for tests expressions
\RequirePackage{fix-cm} % correct units

% Define mode
\def\mode{a4}

\newif\ifaiv % a4
\newif\ifav % a5
\newif\ifbooklet % booklet
\newif\ifcover % cover for booklet

\ifnum \strcmp{\mode}{cover}=0
  \covertrue
\else\ifnum \strcmp{\mode}{booklet}=0
  \booklettrue
\else\ifnum \strcmp{\mode}{a5}=0
  \avtrue
\else
  \aivtrue
\fi\fi\fi

\ifbooklet % do not enclose with {}
  \documentclass[french,twoside]{book} % ,notitlepage
  \usepackage[%
    papersize={105mm, 297mm},
    inner=12mm,
    outer=12mm,
    top=20mm,
    bottom=15mm,
    marginparsep=3pt,
    marginpar=7mm,
  ]{geometry}
  \usepackage[fontsize=9.5pt]{scrextend} % for Roboto
\else\ifav % A5
  \documentclass[french,twoside]{book} % ,notitlepage
  \usepackage[%
    a5paper
  ]{geometry}
  \usepackage[fontsize=12pt]{scrextend}
\else% A4 2 cols
  \documentclass[twocolumn]{report}
  \usepackage[%
    a4paper,
    inner=15mm,
    outer=10mm,
    top=25mm,
    bottom=18mm,
    marginparsep=0pt,
  ]{geometry}
  \setlength{\columnsep}{20mm}
  \usepackage[fontsize=9.5pt]{scrextend}
\fi\fi

%%%%%%%%%%%%%%
% Alignments %
%%%%%%%%%%%%%%
% before teinte macros

\setlength{\arrayrulewidth}{0.2pt}
\setlength{\columnseprule}{\arrayrulewidth} % twocol
\setlength{\parskip}{0pt} % 1pt allow better vertical justification
\setlength{\parindent}{1.5em}

%%%%%%%%%%
% Colors %
%%%%%%%%%%
% before Teinte macros

\usepackage[dvipsnames]{xcolor}
\definecolor{rubric}{HTML}{0c71c3} % the tonic
\def\columnseprulecolor{\color{rubric}}
\colorlet{borderline}{rubric!30!} % definecolor need exact code
\definecolor{shadecolor}{gray}{0.95}
\definecolor{bghi}{gray}{0.5}

%%%%%%%%%%%%%%%%%
% Teinte macros %
%%%%%%%%%%%%%%%%%
%%%%%%%%%%%%%%%%%%%%%%%%%%%%%%%%%%%%%%%%%%%%%%%%%%%
% <TEI> generic (LaTeX names generated by Teinte) %
%%%%%%%%%%%%%%%%%%%%%%%%%%%%%%%%%%%%%%%%%%%%%%%%%%%
% This template is inserted in a specific design
% It is XeLaTeX and otf fonts

\makeatletter % <@@@

\usepackage{alphalph} % for alph couter z, aa, ab…
\usepackage{blindtext} % generate text for testing
\usepackage{booktabs} % for tables: \toprule, \midrule…
\usepackage[strict]{changepage} % for modulo 4
\usepackage{contour} % rounding words
\usepackage[nodayofweek]{datetime}
\usepackage{enumitem} % <list>
\usepackage{etoolbox} % patch commands
\usepackage{fancyvrb}
\usepackage{fancyhdr}
\usepackage{float}
\usepackage{fontspec} % XeLaTeX mandatory for fonts
\usepackage{footnote} % used to capture notes in minipage (ex: quote)
\usepackage{framed} % bordering correct with footnote hack
\usepackage{graphicx}
\usepackage{lettrine} % drop caps
\usepackage{lipsum} % generate text for testing
\usepackage{manyfoot} % for parallel footnote numerotation
\usepackage[framemethod=tikz,]{mdframed} % maybe used for frame with footnotes inside
\usepackage[defaultlines=2,all]{nowidow} % at least 2 lines by par (works well!)
\usepackage{pdftexcmds} % needed for tests expressions
\usepackage{poetry} % <l>, bad for theater
\usepackage{polyglossia} % bug Warning: "Failed to patch part"
\usepackage[%
  indentfirst=false,
  vskip=1em,
  noorphanfirst=true,
  noorphanafter=true,
  leftmargin=\parindent,
  rightmargin=0pt,
]{quoting}
\usepackage{ragged2e}
\usepackage{setspace} % \setstretch for <quote>
\usepackage{scrextend} % KOMA-common, used for addmargin
\usepackage{tabularx} % <table>
\usepackage[explicit]{titlesec} % wear titles, !NO implicit
\usepackage{tikz} % ornaments
\usepackage{tocloft} % styling tocs
\usepackage[fit]{truncate} % used im runing titles
\usepackage{unicode-math}
\usepackage[normalem]{ulem} % breakable \uline, normalem is absolutely necessary to keep \emph
\usepackage{xcolor} % named colors
\usepackage{xparse} % @ifundefined
\XeTeXdefaultencoding "iso-8859-1" % bad encoding of xstring
\usepackage{xstring} % string tests
\XeTeXdefaultencoding "utf-8"

\defaultfontfeatures{
  % Mapping=tex-text, % no effect seen
  Scale=MatchLowercase,
  Ligatures={TeX,Common},
}
\newfontfamily\zhfont{Noto Sans CJK SC}

% Metadata inserted by a program, from the TEI source, for title page and runing heads
\title{\textbf{ Note sur la suppression générale des partis politiques }\par
}
\date{1940}
\author{Simone Weil}
\def\elbibl{Simone Weil. 1940. \emph{Note sur la suppression générale des partis politiques}}
\def\elsource{\emph{Écrits de Londres}, p. 126 et s.}
\def\eltitlepage{%
{\centering\parindent0pt
  {\LARGE\addfontfeature{LetterSpace=25}\bfseries Simone Weil\par}\bigskip
  {\Large 1940\par}\bigskip
  {\LARGE
\bigskip\textbf{Note sur la suppression générale des partis politiques}\par

  }
}

}

% Default metas
\newcommand{\colorprovide}[2]{\@ifundefinedcolor{#1}{\colorlet{#1}{#2}}{}}
\colorprovide{rubric}{red}
\colorprovide{silver}{lightgray}
\@ifundefined{syms}{\newfontfamily\syms{DejaVu Sans}}{}
\newif\ifdev
\@ifundefined{elbibl}{% No meta defined, maybe dev mode
  \newcommand{\elbibl}{Titre court ?}
  \newcommand{\elbook}{Titre du livre source ?}
  \newcommand{\elabstract}{Résumé\par}
  \newcommand{\elurl}{http://oeuvres.github.io/elbook/2}
  \author{Éric Lœchien}
  \title{Un titre de test assez long pour vérifier le comportement d’une maquette}
  \date{1566}
  \devtrue
}{}
\let\eltitle\@title
\let\elauthor\@author
\let\eldate\@date




% generic typo commands
\newcommand{\astermono}{\medskip\centerline{\color{rubric}\large\selectfont{\syms ✻}}\medskip\par}%
\newcommand{\astertri}{\medskip\par\centerline{\color{rubric}\large\selectfont{\syms ✻\,✻\,✻}}\medskip\par}%
\newcommand{\asterism}{\bigskip\par\noindent\parbox{\linewidth}{\centering\color{rubric}\large{\syms ✻}\\{\syms ✻}\hskip 0.75em{\syms ✻}}\bigskip\par}%

% lists
\newlength{\listmod}
\setlength{\listmod}{\parindent}
\setlist{
  itemindent=!,
  listparindent=\listmod,
  labelsep=0.2\listmod,
  parsep=0pt,
  % topsep=0.2em, % default topsep is best
}
\setlist[itemize]{
  label=—,
  leftmargin=0pt,
  labelindent=1.2em,
  labelwidth=0pt,
}
\setlist[enumerate]{
  label={\arabic*°},
  labelindent=0.8\listmod,
  leftmargin=\listmod,
  labelwidth=0pt,
}
% list for big items
\newlist{decbig}{enumerate}{1}
\setlist[decbig]{
  label={\bf\color{rubric}\arabic*.},
  labelindent=0.8\listmod,
  leftmargin=\listmod,
  labelwidth=0pt,
}
\newlist{listalpha}{enumerate}{1}
\setlist[listalpha]{
  label={\bf\color{rubric}\alph*.},
  leftmargin=0pt,
  labelindent=0.8\listmod,
  labelwidth=0pt,
}
\newcommand{\listhead}[1]{\hspace{-1\listmod}\emph{#1}}

\renewcommand{\hrulefill}{%
  \leavevmode\leaders\hrule height 0.2pt\hfill\kern\z@}

% General typo
\DeclareTextFontCommand{\textlarge}{\large}
\DeclareTextFontCommand{\textsmall}{\small}

% commands, inlines
\newcommand{\anchor}[1]{\Hy@raisedlink{\hypertarget{#1}{}}} % link to top of an anchor (not baseline)
\newcommand\abbr[1]{#1}
\newcommand{\autour}[1]{\tikz[baseline=(X.base)]\node [draw=rubric,thin,rectangle,inner sep=1.5pt, rounded corners=3pt] (X) {\color{rubric}#1};}
\newcommand\corr[1]{#1}
\newcommand{\ed}[1]{ {\color{silver}\sffamily\footnotesize (#1)} } % <milestone ed="1688"/>
\newcommand\expan[1]{#1}
\newcommand\foreign[1]{\emph{#1}}
\newcommand\gap[1]{#1}
\renewcommand{\LettrineFontHook}{\color{rubric}}
\newcommand{\initial}[2]{\lettrine[lines=2, loversize=0.3, lhang=0.3]{#1}{#2}}
\newcommand{\initialiv}[2]{%
  \let\oldLFH\LettrineFontHook
  % \renewcommand{\LettrineFontHook}{\color{rubric}\ttfamily}
  \IfSubStr{QJ’}{#1}{
    \lettrine[lines=4, lhang=0.2, loversize=-0.1, lraise=0.2]{\smash{#1}}{#2}
  }{\IfSubStr{É}{#1}{
    \lettrine[lines=4, lhang=0.2, loversize=-0, lraise=0]{\smash{#1}}{#2}
  }{\IfSubStr{ÀÂ}{#1}{
    \lettrine[lines=4, lhang=0.2, loversize=-0, lraise=0, slope=0.6em]{\smash{#1}}{#2}
  }{\IfSubStr{A}{#1}{
    \lettrine[lines=4, lhang=0.2, loversize=0.2, slope=0.6em]{\smash{#1}}{#2}
  }{\IfSubStr{V}{#1}{
    \lettrine[lines=4, lhang=0.2, loversize=0.2, slope=-0.5em]{\smash{#1}}{#2}
  }{
    \lettrine[lines=4, lhang=0.2, loversize=0.2]{\smash{#1}}{#2}
  }}}}}
  \let\LettrineFontHook\oldLFH
}
\newcommand{\labelchar}[1]{\textbf{\color{rubric} #1}}
\newcommand{\lnatt}[1]{\reversemarginpar\marginpar[\sffamily\scriptsize #1]{}}
\newcommand{\milestone}[1]{\autour{\footnotesize\color{rubric} #1}} % <milestone n="4"/>
\newcommand\name[1]{#1}
\newcommand\orig[1]{#1}
\newcommand\orgName[1]{#1}
\newcommand\persName[1]{#1}
\newcommand\placeName[1]{#1}
\newcommand{\pn}[1]{\IfSubStr{-—–¶}{#1}% <p n="3"/>
  {\noindent{\bfseries\color{rubric}   ¶  }}
  {{\footnotesize\autour{#1}}}}
\newcommand\reg{}
% \newcommand\ref{} % already defined
\newcommand\sic[1]{#1}
\newcommand\surname[1]{\textsc{#1}}
\newcommand\term[1]{\textbf{#1}}
\newcommand\zh[1]{{\zhfont #1}}


\def\mednobreak{\ifdim\lastskip<\medskipamount
  \removelastskip\nopagebreak\medskip\fi}
\def\bignobreak{\ifdim\lastskip<\bigskipamount
  \removelastskip\nopagebreak\bigskip\fi}

% commands, blocks

\newcommand{\byline}[1]{\bigskip{\RaggedLeft{#1}\par}\bigskip}
% \setlength{\RaggedLeftLeftskip}{2em plus \leftskip}
\newcommand{\bibl}[1]{{\smallskip\RaggedLeft\normalsize\normalfont #1\par\medskip}}
\newcommand{\biblitem}[1]{{\noindent\hangindent=\parindent   #1\par}}
\newcommand{\castItem}[1]{{\noindent\hangindent=\parindent #1\par}}
\newcommand{\dateline}[1]{\medskip{\RaggedLeft{#1}\par}\bigskip}
\newcommand{\docAuthor}[1]{{\large\bigskip #1 \par\bigskip}}
\newcommand{\docDate}[1]{#1 \ifvmode\par\fi }
\newcommand{\docImprint}[1]{\ifvmode\medskip\fi #1 \ifvmode\par\fi }
\newcommand{\labelblock}[1]{\medbreak{\noindent\color{rubric}\bfseries #1}\par\mednobreak}
\newcommand{\salute}[1]{\bigbreak{#1}\par\medbreak}
\newcommand{\signed}[1]{\medskip{\RaggedLeft #1\par}\bigbreak} % supposed bottom
\newcommand{\speaker}[1]{\medskip{\Centering\sffamily #1 \par\nopagebreak}} % supposed bottom
\newcommand{\stagescene}[1]{{\Centering\sffamily\textsf{#1}\par}\bigskip}
\newcommand{\stageblock}[1]{\begingroup\leftskip\parindent\noindent\it\sffamily\footnotesize #1\par\endgroup} % left margin, better than list envs
\newcommand{\spl}[1]{\noindent\hangindent=2\parindent  #1\par} % sp/l
\newcommand{\trailer}[1]{{\Centering\bigskip #1\par}} % sp/l

% environments for blocks (some may become commands)
\newenvironment{borderbox}{}{} % framing content
\newenvironment{citbibl}{\ifvmode\hfill\fi}{\ifvmode\par\fi }
\newenvironment{msHead}{\vskip6pt}{\par}
\newenvironment{msItem}{\vskip6pt}{\par}


% environments for block containers
\newenvironment{argument}{\itshape\parindent0pt}{\bigskip}
\newenvironment{biblfree}{}{\ifvmode\par\fi }
\newenvironment{bibitemlist}[1]{%
  \list{\@biblabel{\@arabic\c@enumiv}}%
  {%
    \settowidth\labelwidth{\@biblabel{#1}}%
    \leftmargin\labelwidth
    \advance\leftmargin\labelsep
    \@openbib@code
    \usecounter{enumiv}%
    \let\p@enumiv\@empty
    \renewcommand\theenumiv{\@arabic\c@enumiv}%
  }
  \sloppy
  \clubpenalty4000
  \@clubpenalty \clubpenalty
  \widowpenalty4000%
  \sfcode`\.\@m
}%
{\def\@noitemerr
  {\@latex@warning{Empty `bibitemlist' environment}}%
\endlist}
\newenvironment{docTitle}{\LARGE\bigskip\bfseries\onehalfspacing}{\bigskip}
% leftskip makes big bugs in Lexmark printing \sffamily
\newenvironment{epigraph}{\begin{addmargin}[2\parindent]{0em}\sffamily\large\setstretch{0.95}}{\end{addmargin}\bigskip}
\newenvironment{quoteblock}% may be used for ornaments
  {\begin{quoting}}
  {\end{quoting}}
\newenvironment{titlePage}
  {\Centering}
  {}






% table () is preceded and finished by custom command
\renewcommand\tabularxcolumn[1]{m{#1}}% for vertical centering text in X column
\newcommand{\tableopen}[1]{%
  \ifnum\strcmp{#1}{wide}=0{%
    \begin{center}
  }
  \else\ifnum\strcmp{#1}{long}=0{%
    \begin{center}
  }
  \else{%
    \begin{center}
  }
  \fi\fi
}
\newcommand{\tableclose}[1]{%
  \ifnum\strcmp{#1}{wide}=0{%
    \end{center}
  }
  \else\ifnum\strcmp{#1}{long}=0{%
    \end{center}
  }
  \else{%
    \end{center}
  }
  \fi\fi
}


% text structure
\newcommand\chapteropen{} % before chapter title
\newcommand\chaptercont{} % after title, argument, epigraph…
\newcommand\chapterclose{} % maybe useful for multicol settings
\setcounter{secnumdepth}{-2} % no counters for hierarchy titles
\setcounter{tocdepth}{5} % deep toc
\renewcommand\tableofcontents{\@starttoc{toc}}
% toclof format
% \renewcommand{\@tocrmarg}{0.1em} % Useless command?
% \renewcommand{\@pnumwidth}{0.5em} % {1.75em}
\renewcommand{\@cftmaketoctitle}{}
\setlength{\cftbeforesecskip}{\z@ \@plus.2\p@}
\renewcommand{\cftchapfont}{}
\renewcommand{\cftchapdotsep}{\cftdotsep}
\renewcommand{\cftchapleader}{\normalfont\cftdotfill{\cftchapdotsep}}
\renewcommand{\cftchappagefont}{\bfseries}
\setlength{\cftbeforechapskip}{0em \@plus\p@}
% \renewcommand{\cftsecfont}{\small\relax}
\renewcommand{\cftsecpagefont}{\normalfont}
% \renewcommand{\cftsubsecfont}{\small\relax}
\renewcommand{\cftsecdotsep}{\cftdotsep}
\renewcommand{\cftsecpagefont}{\normalfont}
\renewcommand{\cftsecleader}{\normalfont\cftdotfill{\cftsecdotsep}}
\setlength{\cftsecindent}{1em}
\setlength{\cftsubsecindent}{2em}
\setlength{\cftsubsubsecindent}{3em}
\setlength{\cftchapnumwidth}{1em}
\setlength{\cftsecnumwidth}{1em}
\setlength{\cftsubsecnumwidth}{1em}
\setlength{\cftsubsubsecnumwidth}{1em}

% footnotes
\newif\ifheading
\newcommand*{\fnmarkscale}{\ifheading 0.70 \else 1 \fi}
\renewcommand\footnoterule{\vspace*{0.3cm}\hrule height \arrayrulewidth width 3cm \vspace*{0.3cm}}
\setlength\footnotesep{1.5\footnotesep} % footnote separator
\renewcommand\@makefntext[1]{\parindent 1.5em \noindent \hb@xt@1.8em{\hss{\normalfont\@thefnmark . }}#1} % no superscipt in foot
\patchcmd{\@footnotetext}{\footnotesize}{\footnotesize\sffamily}{}{} % before scrextend, hyperref
\DeclareNewFootnote{A}[alph] % for editor notes
\renewcommand*{\thefootnoteA}{\alphalph{\value{footnoteA}}} % z, aa, ab…

% poem
\setlength{\poembotskip}{0pt}
\setlength{\poemtopskip}{0pt}
\setlength{\poemindent}{0pt}
\poemlinenumsfalse

%   see https://tex.stackexchange.com/a/34449/5049
\def\truncdiv#1#2{((#1-(#2-1)/2)/#2)}
\def\moduloop#1#2{(#1-\truncdiv{#1}{#2}*#2)}
\def\modulo#1#2{\number\numexpr\moduloop{#1}{#2}\relax}

% orphans and widows, nowidow package in test
% from memoir package
\clubpenalty=9996
\widowpenalty=9999
\brokenpenalty=4991
\predisplaypenalty=10000
\postdisplaypenalty=1549
\displaywidowpenalty=1602
\hyphenpenalty=400
% report h or v overfull ?
\hbadness=4000
\vbadness=4000
% good to avoid lines too wide
\emergencystretch 3em
\pretolerance=750
\tolerance=2000
\def\Gin@extensions{.pdf,.png,.jpg,.mps,.tif}

\PassOptionsToPackage{hyphens}{url} % before hyperref and biblatex, which load url package
\usepackage{hyperref} % supposed to be the last one, :o) except for the ones to follow
\hypersetup{
  % pdftex, % no effect
  pdftitle={\elbibl},
  % pdfauthor={Your name here},
  % pdfsubject={Your subject here},
  % pdfkeywords={keyword1, keyword2},
  bookmarksnumbered=true,
  bookmarksopen=true,
  bookmarksopenlevel=1,
  pdfstartview=Fit,
  breaklinks=true, % avoid long links, overrided by url package
  pdfpagemode=UseOutlines,    % pdf toc
  hyperfootnotes=true,
  colorlinks=false,
  pdfborder=0 0 0,
  % pdfpagelayout=TwoPageRight,
  % linktocpage=true, % NO, toc, link only on page no
}
\urlstyle{same} % after hyperref



\makeatother % /@@@>
%%%%%%%%%%%%%%
% </TEI> end %
%%%%%%%%%%%%%%


%%%%%%%%%%%%%
% footnotes %
%%%%%%%%%%%%%
\renewcommand{\thefootnote}{\bfseries\textcolor{rubric}{\arabic{footnote}}} % color for footnote marks

%%%%%%%%%
% Fonts %
%%%%%%%%%
% \linespread{0.90} % too compact, keep font natural
\ifav % A5
  \usepackage{DejaVuSans} % correct
  \setsansfont{DejaVuSans} % seen, if not set, problem with printer
\else\ifbooklet
  \usepackage[]{roboto} % SmallCaps, Regular is a bit bold
  \setmainfont[
    ItalicFont={Roboto Light Italic},
  ]{Roboto}
  \setsansfont{Roboto Light} % seen, if not set, problem with printer
  \newfontfamily\fontrun[]{Roboto Condensed Light} % condensed runing heads
\else
  \usepackage[]{roboto} % SmallCaps, Regular is a bit bold
  \setmainfont[
    ItalicFont={Roboto Italic},
  ]{Roboto Light}
  \setsansfont{Roboto Light} % seen, if not set, problem with printer
  \newfontfamily\fontrun[]{Roboto Condensed Light} % condensed runing heads
\fi\fi
\renewcommand{\LettrineFontHook}{\bfseries\color{rubric}}
% \renewenvironment{labelblock}{\begin{center}\bfseries\color{rubric}}{\end{center}}

%%%%%%%%
% MISC %
%%%%%%%%

\setdefaultlanguage[frenchpart=false]{french} % bug on part


\newenvironment{quotebar}{%
    \def\FrameCommand{{\color{rubric!10!}\vrule width 0.5em} \hspace{0.9em}}%
    \def\OuterFrameSep{0pt} % séparateur vertical
    \MakeFramed {\advance\hsize-\width \FrameRestore}
  }%
  {%
    \endMakeFramed
  }
\renewenvironment{quoteblock}% may be used for ornaments
  {%
    \savenotes
    \setstretch{0.9}
    \begin{quotebar}
    \smallskip
  }
  {%
    \smallskip
    \end{quotebar}
    \spewnotes
  }


\renewcommand{\headrulewidth}{\arrayrulewidth}
\renewcommand{\headrule}{{\color{rubric}\hrule}}
\renewcommand{\lnatt}[1]{\marginpar{\sffamily\scriptsize #1}}

% delicate tuning, image has produce line-height problems in title on 2 lines
\titleformat{name=\chapter} % command
  [display] % shape
  {\vspace{1.5em}\centering} % format
  {} % label
  {0pt} % separator between n
  {}
[{\color{rubric}\huge\textbf{#1}}\bigskip] % after code
% \titlespacing{command}{left spacing}{before spacing}{after spacing}[right]
\titlespacing*{\chapter}{0pt}{-2em}{0pt}[0pt]

\titleformat{name=\section}
  [display]{}{}{}{}
  [\vbox{\color{rubric}\large\centering\textbf{#1}}]
\titlespacing{\section}{0pt}{0pt plus 4pt minus 2pt}{\baselineskip}

\titleformat{name=\subsection}
  [block]
  {}
  {} % \thesection
  {} % separator \arrayrulewidth
  {}
[\vbox{\large\textbf{#1}}]
% \titlespacing{\subsection}{0pt}{0pt plus 4pt minus 2pt}{\baselineskip}

\ifaiv
  \fancypagestyle{main}{%
    \fancyhf{}
    \setlength{\headheight}{1.5em}
    \fancyhead{} % reset head
    \fancyfoot{} % reset foot
    \fancyhead[L]{\truncate{0.45\headwidth}{\fontrun\elbibl}} % book ref
    \fancyhead[R]{\truncate{0.45\headwidth}{ \fontrun\nouppercase\leftmark}} % Chapter title
    \fancyhead[C]{\thepage}
  }
  \fancypagestyle{plain}{% apply to chapter
    \fancyhf{}% clear all header and footer fields
    \setlength{\headheight}{1.5em}
    \fancyhead[L]{\truncate{0.9\headwidth}{\fontrun\elbibl}}
    \fancyhead[R]{\thepage}
  }
\else
  \fancypagestyle{main}{%
    \fancyhf{}
    \setlength{\headheight}{1.5em}
    \fancyhead{} % reset head
    \fancyfoot{} % reset foot
    \fancyhead[RE]{\truncate{0.9\headwidth}{\fontrun\elbibl}} % book ref
    \fancyhead[LO]{\truncate{0.9\headwidth}{\fontrun\nouppercase\leftmark}} % Chapter title, \nouppercase needed
    \fancyhead[RO,LE]{\thepage}
  }
  \fancypagestyle{plain}{% apply to chapter
    \fancyhf{}% clear all header and footer fields
    \setlength{\headheight}{1.5em}
    \fancyhead[L]{\truncate{0.9\headwidth}{\fontrun\elbibl}}
    \fancyhead[R]{\thepage}
  }
\fi

\ifav % a5 only
  \titleclass{\section}{top}
\fi

\newcommand\chapo{{%
  \vspace*{-3em}
  \centering\parindent0pt % no vskip ()
  \eltitlepage
  \bigskip
  {\color{rubric}\hline}
  \bigskip
  {\Large TEXTE LIBRE À PARTICIPATIONS LIBRES\par}
  \centerline{\small\color{rubric} {\href{https://hurlus.fr}{\dotuline{hurlus.fr}}}, tiré le \today}\par
  \bigskip
}}

\newcommand\cover{{%
  \thispagestyle{empty}
  \centering\parindent0pt
  \eltitlepage
  \vfill\null
  {\color{rubric}\setlength{\arrayrulewidth}{2pt}\hline}
  \vfill\null
  {\Large TEXTE LIBRE À PARTICIPATIONS LIBRES\par}
  \centerline{\href{https://hurlus.fr}{\dotuline{hurlus.fr}}, tiré le \today}\par
}}

\begin{document}
\pagestyle{empty}
\ifbooklet{
  \cover\newpage
  \thispagestyle{empty}\hbox{}\newpage
  \cover\newpage\noindent Les voyages de la brochure\par
  \bigskip
  \begin{tabularx}{\textwidth}{l|X|X}
    \textbf{Date} & \textbf{Lieu}& \textbf{Nom/pseudo} \\ \hline
    \rule{0pt}{25cm} &  &   \\
  \end{tabularx}
  \newpage
  \addtocounter{page}{-4}
}\fi

\thispagestyle{empty}
\ifaiv
  \twocolumn[\chapo]
\else
  \chapo
\fi
{\it\elabstract}
\bigskip
\makeatletter\@starttoc{toc}\makeatother % toc without new page
\bigskip

\pagestyle{main} % after style
\setcounter{footnote}{0}
\setcounter{footnoteA}{0}
  \noindent Le mot parti est pris ici dans la signification qu’il a sur le continent européen. Le même mot dans les pays anglo-saxons désigne une réalité tout autre. Elle a sa racine dans la tradition anglaise et n’est pas transplantable. Un siècle et demi d’expérience le montre assez. Il y a dans les partis anglo-saxons un élément de jeu, de sport, qui ne peut exister que dans une institution d’origine aristocratique ; tout est sérieux dans une institution qui, au départ, est plébéienne.\par
L’idée de parti n’entrait pas dans la conception politique française de 1789, sinon comme mal à éviter. Mais il y eut le club des Jacobins. C’était d’abord seulement un lieu de libre discussion. Ce ne fut aucune espèce de mécanisme fatal qui le transforma. C’est uniquement la pression de la guerre et de la guillotine qui en fit un parti totalitaire.\par
Les luttes des factions sous la Terreur furent gouvernées par la pensée si bien formulée par Tomski : « Un parti au pouvoir et tous les autres en prison. » Ainsi sur le continent d’Europe le totalitarisme est le péché originel des partis.\par
C’est d’une part l’héritage de la Terreur, d’autre part l’influence de l’exemple anglais, qui installa les partis dans la vie publique européenne. Le fait qu’ils existent n’est nullement un motif de les conserver. Seul le bien est un motif légitime de conservation. Le mal des partis politiques saute aux yeux. Le problème à examiner, c’est s’il y a en eux un bien qui l’emporte sur le mal et rende ainsi leur existence désirable.\par
Mais il est beaucoup plus à propos de demander : Y a-t-il en eux même une parcelle infinitésimale de bien ? Ne sont-ils pas du mal à l’état pur ou presque ?\par
S’ils sont du mal, il est certain qu’en fait et dans la pratique ils ne peuvent produire que du mal. C’est un article de foi. « Un bon arbre ne peut jamais porter de mauvais fruits, ni un arbre pourri de beaux fruits. »\par
Mais il faut d’abord reconnaître quel est le critère du bien.\par
Ce ne peut être que la vérité, la justice, et, en second lieu, l’utilité publique.\par
La démocratie, le pouvoir du plus grand nombre, ne sont pas des biens. Ce sont des moyens en vue du bien, estimés efficaces à tort ou à raison. Si la République de Weimar, au lieu de Hitler, avait décidé par les voies les plus rigoureusement parlementaires et légales de mettre les Juifs dans des camps de concentration et de les torturer avec raffinement jusqu’à la mort, les tortures n’auraient pas eu un atome de légitimité de plus qu’elles n’ont maintenant. Or pareille chose n’est nullement inconcevable.\par
Seul ce qui est juste est légitime. Le crime et le mensonge ne le sont en aucun cas.\par
Notre idéal républicain procède entièrement de la notion de volonté générale due à Rousseau, Mais le sens de la notion a été perdu presque tout de suite, parce qu’elle est complexe et demande un degré d’attention élevé.\par
Quelques chapitres mis à part, peu de livres sont beaux, forts, lucides et clairs comme \emph{Le Contrat Social.} On dit que peu de livres ont eu autant d’influence. Mais en fait tout s’est passé et se passe encore comme s’il n’avait jamais été lu.\par
Rousseau partait de deux évidences. L’une, que la raison discerne et choisit la justice et l’utilité innocente, et que tout crime a pour mobile la passion. L’autre, que la raison est identique chez tous les hommes, au lieu que les passions, le plus souvent, diffèrent. Par suite si, sur un problème général, chacun réfléchit tout seul et exprime une opinion, et si ensuite les opinions sont comparées entre elles, probablement elles coïncideront par la partie juste et raisonnable de chacune et différeront par les injustices et les erreurs.\par
C’est uniquement en vertu d’un raisonnement de ce genre qu’on admet que le \emph{consensus} universel indique la vérité.\par
La vérité est une. La justice est une. Les erreurs, les injustices sont indéfiniment variables. Ainsi les hommes convergent dans le juste et le vrai, au lieu que le mensonge et le crime les font indéfiniment diverger. L’union étant une force matérielle, on peut espérer trouver là une ressource pour rendre ici-bas la vérité et la justice matériellement plus fortes que le crime et l’erreur.\par
Il y faut un mécanisme convenable. Si la démocratie constitue un tel mécanisme, elle est bonne. Autrement non.\par
Un vouloir injuste commun à toute la nation n’était aucunement supérieur aux yeux de Rousseau – et il était dans le vrai – au vouloir injuste d’un homme.\par
Rousseau pensait seulement que le plus souvent un vouloir commun à tout un peuple est en fait conforme à la justice, par la neutralisation mutuelle et la compensation des passions particulières. C’était là pour lui l’unique motif de préférer le vouloir du peuple à un vouloir particulier.\par
C’est ainsi qu’une certaine masse d’eau, quoique composée de particules qui se meuvent et se heurtent sans cesse, est dans un équilibre et un repos parfaits. Elle renvoie aux objets leurs images avec une vérité irréprochable. Elle indique parfaitement le plan horizontal. Elle dit sans erreur la densité des objets qu’on y plonge.\par
Si des individus passionnés, enclins par la passion au crime et au mensonge, se composent de la même manière en un peuple véridique et juste, alors il est bon que le peuple soit souverain. Une constitution démocratique est bonne si d’abord elle accomplit dans le peuple cet état d’équilibre, et si ensuite seulement elle fait en sorte que les vouloirs du peuple soient exécutés.\par
Le véritable esprit de 1789 consiste à penser, non pas qu’une chose est juste parce que le peuple la veut, mais qu’à certaines conditions le vouloir du peuple a plus de chances qu’aucun autre vouloir d’être conforme à la justice.\par
Il y a plusieurs conditions indispensables pour pouvoir appliquer la notion de volonté générale. Deux doivent particulièrement retenir l’attention.\par
L’une est qu’au moment où le peuple prend conscience d’un de ses vouloirs et l’exprime, il n’y ait aucune espèce de passion collective.\par
Il est tout à fait évident que le raisonnement de Rousseau tombe dès qu’il y a passion collective. Rousseau le savait bien. La passion collective est une impulsion de crime et de mensonge infiniment plus puissante qu’aucune passion individuelle. Les impulsions mauvaises, en ce cas, loin de se neutraliser, se portent mutuellement à la millième puissance. La pression est presque irrésistible, sinon pour les saints authentiques.\par
Une eau mise en mouvement par un courant violent, impétueux, ne reflète plus les objets, n’a plus une surface horizontale, n’indique plus les densités.\par
Et il importe très peu qu’elle soit mue par un seul courant ou par cinq ou six courants qui se heurtent et font des remous. Elle est également troublée dans les deux cas.\par
Si une seule passion collective saisit tout un pays, le pays entier est unanime dans le crime. Si deux ou quatre ou cinq ou dix passions collectives le partagent, il est divisé en plusieurs bandes de criminels. Les passions divergentes ne se neutralisent pas, comme c’est le cas pour une poussière de passions individuelles fondues dans une masse ; le nombre est bien trop petit, la force de chacune est bien trop grande, pour qu’il puisse y avoir neutralisation. La lutte les exaspère. Elles se heurtent avec un bruit vraiment infernal, et qui rend impossible d’entendre même une seconde la voix de la justice et de la vérité, toujours presque imperceptible.\par
Quand il y a passion collective dans un pays, il y a probabilité pour que n’importe quelle volonté particulière soit plus proche de la justice et de la raison que la volonté générale, ou plutôt que ce qui en constitue la caricature.\par
La seconde condition est que le peuple ait à exprimer son vouloir à l’égard des problèmes de la vie publique, et non pas à faire seulement un choix de personnes. Encore moins un choix de collectivités irresponsables. Car la volonté générale est sans aucune relation avec un tel choix.\par
S’il y a eu en 1789 une certaine expression de la volonté générale, bien qu’on eût adopté le système représentatif faute de savoir en imaginer un autre, c’est qu’il y avait eu bien autre chose que des élections. Tout ce qu’il y avait de vivant à travers tout le pays – et le pays débordait alors de vie – avait cherché à exprimer une pensée par l’organe des cahiers de revendications. Les représentants s’étaient en grande partie fait connaître au cours de cette coopération dans la pensée ; ils en gardaient l’a chaleur ; ils sentaient le pays attentif à leurs paroles, jaloux de surveiller si elles traduisaient exactement ses aspirations. Pendant quelque temps – peu de temps – ils furent vraiment de simples organes d’expression pour la pensée publique.\par
Pareille chose ne se produisit jamais plus.\par
Le seul énoncé de ces deux conditions montre que nous n’avons jamais rien connu qui ressemble même de loin à une démocratie. Dans ce que nous nommons de ce nom, jamais le peuple n’a l’occasion ni le moyen d’exprimer un avis sur aucun problème de la vie publique ; et tout ce qui échappe aux intérêts particuliers est livré aux passions collectives, lesquelles sont systématiquement, officiellement encouragées.\par
L’usage même des mots de démocratie et de république oblige à examiner avec une attention extrême les deux problèmes que voici :\par
Comment donner en fait aux hommes qui composent le peuple de France la possibilité d’exprimer parfois un jugement sur les grands problèmes de la vie publique ?\par
Comment empêcher, au moment où le peuple est interrogé, qu’il circule à travers lui aucune espèce de passion collective ?\par
Si on ne pense pas à ces deux points, il est inutile de parler de légitimité républicaine.\par
Des solutions ne sont pas faciles à concevoir. Mais il est évident, après examen attentif, que toute solution impliquerait d’abord la suppression des partis politiques.\par
Pour apprécier les partis politiques selon le critère de la vérité, de la justice, du..bien public, il convient de commencer par en discerner les caractères essentiels.\par
On peut en énumérer trois :\par
— Un parti politique est une machine à fabriquer de la passion collective.\par
— Un parti politique est une organisation construite de manière à exercer une pression collective sur la pensée de chacun des êtres humains qui en sont membres.\par
— La première fin, et, en dernière analyse, l’unique fin de tout parti politique est sa propre croissance, et cela sans aucune limite.\par
Par ce triple caractère, tout parti est totalitaire en germe et en aspiration. S’il ne l’est pas en fait, c’est seulement parce que ceux qui l’entourent ne le sont pas moins que lui.\par
Ces trois caractères sont des vérités de fait évidentes à quiconque s’est approché de la vie des partis.\par
Le troisième est un cas particulier d’un phénomène qui se produit partout où le collectif domine les êtres pensants. C’est le retournement de la relation entre fin et moyen. Partout, sans exception, toutes les choses généralement considérées comme des fins sont par nature, par définition, par essence et de la manière la plus évidente uniquement des moyens. On pourrait en citer autant d’exemples qu’on voudrait dans tous les domaines. Argent, pouvoir, État, grandeur nationale, production économique, diplômes universitaires ; et beaucoup d’autres.\par
Le bien seul est une fin. Tout ce qui appartient au domaine des faits est de l’ordre des moyens. Mais la pensée collective est incapable de s’élever au-dessus du domaine des faits. C’est une pensée animale. Elle n’a la notion du bien que juste assez pour commettre l’erreur de prendre tel ou tel moyen pour un bien absolu.\par
Il en est ainsi des partis. Un parti est en principe un instrument pour servir une certaine conception du bien public.\par
Cela est vrai même de ceux qui sont liés aux intérêts d’une catégorie sociale, car il est toujours une certaine conception du bien public en vertu de laquelle il y aurait coïncidence entre le bien public et ces intérêts. Mais cette conception est extrêmement vague. Cela est vrai sans exception et presque sans différence de degrés. Les partis les plus inconsistants et les plus strictement organisés sont égaux par le vague de la doctrine. Aucun homme, si profondément qu’il ait étudié la politique, ne serait capable d’un exposé précis et clair relativement à la doctrine d’aucun parti, y compris, le cas échéant, le sien propre.\par
Les gens ne s’avouent guère cela à eux-mêmes. S’ils se l’avouaient, ils seraient naïvement tentés d’y voir une marque d’incapacité personnelle, faute d’avoir reconnu que l’expression : « Doctrine d’un parti politique » ne peut jamais, par la nature des choses, avoir aucune signification.\par
Un homme, passât-il sa vie à écrire et à examiner des problèmes d’idées, n’a que très rarement une doctrine. Une collectivité n’en a jamais. Ce n’est pas une marchandise collective.\par
On peut parler, il est vrai, de doctrine chrétienne, doctrine hindoue, doctrine pythagoricienne, et ainsi de suite. Ce qui est alors désigné par ce mot n’est ni individuel ni collectif ; c’est une chose située infiniment au-dessus de l’un et l’autre domaine. C’est, purement et simplement, la vérité.\par
La fin d’un parti politique est chose vague et irréelle. Si elle était réelle, elle exigerait un très grand effort d’attention, car une conception du bien public n’est pas chose facile à penser. L’existence du parti est palpable, évidente, et n’exige aucun effort pour être reconnue. Il est ainsi inévitable qu’en fait le parti soit à lui-même sa propre fin.\par
Il y a dès lors idolâtrie, car Dieu seul est légitimement une fin pour soi-même. La transition est facile. On pose en axiome que la condition nécessaire et suffisante pour que le parti serve efficacement la conception du bien public en vue duquel il existe est qu’il possède une large quantité de pouvoir.\par
Mais aucune quantité finie de pouvoir ne peut jamais être en fait regardée comme suffisante, surtout une fois obtenue. Le parti se trouve en fait, par l’effet de l’absence de pensée, dans un état continuel d’impuissance qu’il attribue toujours à l’insuffisance du pouvoir dont il dispose. Serait-il maître absolu du pays, les nécessités internationales imposent des limites étroites.\par
Ainsi la tendance essentielle des partis est totalitaire, non seulement relativement à une nation, mais relativement au globe terrestre. C’est précisément parce que la conception du bien public propre à tel ou tel parti est une fiction, une chose vide, sans réalité, qu’elle impose la recherche de la puissance totale. Toute réalité implique par elle-même une limite. Ce qui n’existe pas du tout n’est jamais limitable.\par
C’est pour cela qu’il y a affinité, alliance entre le totalitarisme et le mensonge.\par
Beaucoup de gens, il est vrai, ne songent jamais à une puissance totale ; cette pensée leur ferait peur. Elle est vertigineuse, et il faut une espèce de grandeur pour la soutenir. Ces gens-là, quand ils s’intéressent à un parti, se contentent d’en désirer la croissance ; mais comme une chose qui ne comporte aucune limite. S’il y a trois membres de plus cette année que l’an dernier, ou si la collecte a rapporté cent francs de plus, ils sont contents. Mais ils désirent que cela continue indéfiniment dans la même direction. Jamais ils ne concevraient que leur parti puisse avoir en aucun cas trop de membres, trop d’électeurs, trop d’argent.\par
Le tempérament révolutionnaire mène à concevoir la totalité. Le tempérament petit-bourgeois mène à s’installer dans l’image d’un progrès lent, continu et sans limite. Mais dans les deux cas la croissance matérielle du parti devient l’unique critère par, rapport auquel se définissent en toutes choses le bien et le mal. Exactement comme si le parti était un animal à l’engrais, et que l’univers eût été créé pour le faire engraisser.\par
On ne peut servir Dieu et Mammon. Si on a un critère du bien autre que le bien, on perd la notion du bien.\par
Dès lors que la croissance du parti constitue un critère du bien, il s’ensuit inévitablement une pression collective du parti sur les pensées des hommes. Cette pression s’exerce en fait. Elle s’étale publiquement. Elle est avouée, proclamée. Cela nous ferait horreur si l’accoutumance ne nous avait pas tellement endurcis.\par
Les partis sont des organismes publiquement, officiellement constitués de manière à tuer dans les âmes le sens de la vérité et de la justice.\par
La pression collective est exercée sur le grand public par la propagande. Le but avoué de la propagande est de persuader et non pas de communiquer de la lumière. Hitler a très bien vu que la propagande est toujours une tentative d’asservissement des esprits. Tous les partis font de la propagande. Celui qui n’en ferait pas disparaîtrait du fait que les autres en font. Tous avouent qu’ils font de la propagande. Aucun n’est audacieux dans le mensonge au point d’affirmer qu’il entreprend l’éducation du public, qu’il forme le jugement du peuple.\par
Les partis parlent, il est vrai, d’éducation à l’égard de ceux qui sont venus à eux, sympathisants, jeunes, nouveaux adhérents. Ce mot est un mensonge. Il s’agit d’un dressage pour préparer l’emprise bien plus rigoureuse exercée par le parti sur la pensée de ses membres.\par
Supposons un membre d’un parti – député, candidat à la députation, ou simplement militant – qui prenne en public l’engagement que voici : « Toutes les fois que j’examinerai n’importe quel problème politique ou social, je m’engage à oublier absolument le fait que je suis membre de tel groupe, et à me préoccuper exclusivement de discerner le bien public et la justice. »\par
Ce langage serait très mal accueilli. Les siens et même beaucoup d’autres l’accuseraient de trahison. Les moins hostiles diraient : « Pourquoi alors a-t-il adhéré à un parti ? » – avouant ainsi naïvement qu’en entrant dans un parti on renonce à chercher uniquement le bien public et la justice. Cet homme serait exclu de son parti, ou au moins en perdrait l’investiture ; il ne serait certainement pas élu.\par
Mais bien plus, il ne semble même pas possible qu’un tel langage soit tenu. En fait, sauf erreur, il ne l’a jamais été. Si des mots en apparence voisins de ceux-là ont été prononcés, c’était seulement par des hommes désireux de gouverner avec l’appui de partis autres que le leur. De telles paroles sonnaient alors comme une sorte de manquement à l’honneur.\par
En revanche on trouve tout à fait naturel, raisonnable et honorable que quelqu’un dise : « Comme conservateur — » ou : « Comme socialiste – je pense que… »\par
Cela, il est vrai, n’est pas propre aux partis. On ne rougit pas non plus de dire : « Comme Français, je pense que… » « Comme catholique, je pense que… »\par
Des petites filles, qui se disaient attachées au gaullisme comme à l’équivalent français de l’hitlérisme, ajoutaient : « La vérité est relative, même en géométrie. » Elles touchaient le point central.\par
S’il n’y a pas de vérité, il est légitime de penser de telle ou telle manière en tant qu’on se trouve être en fait telle ou telle chose. Comme on a des cheveux noirs, bruns, roux ou blonds, parce qu’on est comme cela, on émet aussi telles et telles pensées. La pensée, comme les cheveux, est alors le produit d’un processus physique d’élimination.\par
Si on reconnaît qu’il y a une vérité, il n’est permis de penser que ce qui est vrai. On pense alors telle chose, non parce qu’on se trouve être en fait Français, ou catholique, ou socialiste, mais parce que la lumière irrésistible de l’évidence oblige à penser ainsi et \emph{non} autrement.\par
S’il n’y a pas évidence, s’il y a doute, il est alors évident que dans l’état de connaissances dont \emph{on} dispose la question est douteuse. S’il y a une faible probabilité d’un côté, il est évident qu’il y a une faible probabilité ; et ainsi de suite. Dans tous les cas, la lumière intérieure accorde toujours à quiconque la consulte une \emph{réponse} manifeste. Le contenu de la réponse est plus ou moins affirmatif ; peu importe. Il est toujours susceptible de révision ; mais aucune correction ne peut être apportée, sinon par davantage de lumière intérieure.\par
Si un homme, membre d’un parti, est absolument résolu à n’être fidèle en toutes ses pensées qu’à la lumière intérieure exclusivement et à rien d’autre, il ne peut pas faire connaître cette résolution à son parti, Il est alors vis-à-vis de lui en état de mensonge.\par
C’est une situation qui ne peut être acceptée qu’à cause de la nécessité qui contraint à se trouver dans un parti pour prendre part efficacement aux affaires publiques. Mais alors \emph{cette} nécessité est un mal, et \emph{il} faut y mettre fin en supprimant les partis.\par
Un homme qui n’a pas pris la résolution de fidélité exclusive à la lumière intérieure installe le mensonge au centre même de l’âme. Les ténèbres intérieures en sont la punition.\par
On tenterait vainement de s’en tirer par la distinction entre la liberté intérieure et la discipline extérieure. Car il faut alors mentir au public, envers qui tout candidat, tout élu, a une obligation particulière de vérité.\par
Si je m’apprête à dire, au nom de mon parti, des choses que j’estime contraires à la vérité et à la justice, vais-je l’indiquer dans un avertissement préalable ? Si je ne le fais pas, je mens.\par
De ces trois formes de mensonge – au parti, au public, à soi-même – la première est de loin la moins mauvaise. Mais si l’appartenance à un parti contraint toujours, en tout cas, au mensonge, l’existence des partis est absolument, inconditionnellement un mal.\par
Il était fréquent de voir dans des annonces de réunion : M. X. exposera le point de vue communiste (sur le problème qui est l’objet de la réunion). M. Y. exposera le point de vue socialiste. M. Z. exposera le point de vue radical.-\par
Comment ces malheureux s’y prenaient-ils pour connaître le point de vue qu’ils devaient exposer ? Qui pouvaient-ils consulter ? Quel oracle ? Une collectivité n’a pas de langue ni de plume. Les organes d’expression sont tous individuels. La collectivité socialiste ne réside en aucun individu. La collectivité radicale non plus. La collectivité communiste réside en Staline, mais il est loin ; on ne peut pas lui téléphoner avant de parler dans une réunion.\par
Non, MM. X., Y. et Z. se consultaient eux-mêmes. Mais comme ils étaient honnêtes, ils se mettaient d’abord dans un état mental spécial, un état semblable à celui où les avait mis si souvent l’atmosphère des milieux communiste, socialiste, radical.\par
Si, s’étant mis dans cet état, on se laisse aller à ses réactions, on produit naturellement un langage conforme aux « points de vue » communiste, socialiste, radical.\par
A condition, bien entendu, de s’interdire rigoureusement tout effort d’attention en vue de discerner la justice et la vérité. Si on accomplissait un tel effort, on risquerait – comble d’horreur – d’exprimer un « point de vue personnel ».\par
Car de nos jours la tension vers la justice et la vérité est regardée comme répondant à un point de vue personnel.\par
Quand Ponce Pilate a demandé au Christ : « Qu’est-ce que la vérité ? » le Christ n’a pas répondu. Il avait répondu d’avance en disant : « Je suis venu porter témoignage pour la vérité. »\par
Il n’y a qu’une réponse. La vérité, ce sont les pensées qui surgissent dans l’esprit d’une créature pensante uniquement, totalement, exclusivement désireuse de la vérité.\par
Le mensonge, l’erreur – mots synonymes – ce sont les pensées de ceux qui ne désirent pas la vérité, et de ceux qui désirent la vérité et autre chose en plus. Par exemple qui désirent la vérité et en plus la conformité avec telle ou telle pensée établie.\par
Mais comment « désirer la vérité sans rien savoir d’elle ? C’est là le mystère des mystères. Les mots qui expriment une perfection inconcevable à l’homme – Dieu, vérité, justice – prononcés intérieurement avec désir, sans être joints à aucune conception, ont le pouvoir d’élever l’âme et de l’inonder de lumière.\par
C’est en désirant la vérité à vide et sans tenter d’en deviner d’avance le contenu qu’on reçoit la lumière. C’est là tout le mécanisme de l’attention.\par
Il est impossible d’examiner les problèmes effroyablement complexes de la vie publique en étant attentif à la fois, d’une part à discerner la vérité, la justice, le bien public, d’autre part à conserver l’attitude qui convient à un membre de tel groupement. La faculté humaine d’attention n’est pas capable simultanément \emph{des} deux soucis. En fait quiconque s’attache à l’un abandonne l’autre.\par
Mais aucune souffrance, n’attend celui qui abandonne la justice et la vérité. Au lieu que le système des partis comporte les pénalités les plus douloureuses pour l’indocilité. Des pénalités qui atteignent presque tout – la carrière, les sentiments, l’amitié, la réputation, la partie extérieure de l’honneur, parfois même la vie de famille. Le parti communiste a porté le système à sa perfection.\par
Même chez celui qui intérieurement ne cède pas, l’existence de pénalités fausse inévitablement le discernement. Car s’il veut réagir contre l’emprise du parti, cette volonté de réaction est elle-même un mobile étranger à la vérité et dont il faut se méfier. Mais cette méfiance aussi ; et ainsi de suite. L’attention véritable est un état tellement difficile à l’homme, tellement violent, que tout trouble personnel de la sensibilité suffit à y faire obstacle. Il en résulte l’obligation impérieuse de protéger autant qu’on peut la faculté de discernement qu’on porte en soi-même contre le tumulte des espérances et \emph{des} craintes personnelles.\par
Si un homme fait des calculs numériques très complexes en sachant qu’il sera fouetté toutes les fois qu’il obtiendra comme résultat un nombre pair, sa situation est très difficile. Quelque chose dans la partie charnelle de l’âme le poussera à donner un petit coup de pouce aux calculs pour obtenir toujours un nombre impair. En voulant réagir il risquera de trouver un nombre pair même là où il n’en faut pas. Prise dans cette oscillation, son attention n’est plus intacte. Si les calculs sont complexes au point d’exiger de sa part la plénitude de l’attention, il est inévitable qu’il se trompe très souvent. Il ne servira à rien qu’il soit très intelligent, très courageux, très soucieux de vérité.\par
Que doit-il faire ? C’est très simple. S’il peut échapper \emph{des} mains de ces gens qui le menacent du fouet, il doit fuir. S’il a pu éviter de tomber entre leurs mains, il devait l’éviter.\par
Il en est exactement ainsi des partis politiques.\par
Quand il y a des partis dans un pays, il en résulte tôt ou tard un état de fait tel qu’il est impossible d’intervenir efficacement dans les affaires publiques sans entrer dans un parti et jouer le jeu. Quiconque s’intéresse à la chose publique désire s’y intéresser efficacement. Ainsi ceux qui inclinent au souci du bien public, ou renoncent à y penser et se tournent vers autre chose, ou passent par le laminoir des partis. En ce cas aussi il leur vient des soucis qui excluent celui du bien public.\par
Les partis sont un merveilleux mécanisme, par la vertu duquel, dans toute l’étendue d’un pays, pas un esprit ne donne son attention à l’effort de discerner, dans les affaires publiques, le bien, la justice, la vérité.\par
Il en résulte que – sauf un très petit nombre de coïncidences fortuites – il n’est décidé et exécuté que des mesures contraires au bien public, à la justice et à la vérité.\par
Si on confiait au diable l’organisation de la vie publique, il ne pourrait rien imaginer de plus ingénieux.\par
Si la réalité a été un peu moins sombre, c’est que les partis n’avaient pas encore tout dévoré. Mais en fait, a-t-elle été un peu moins sombre ? N’était-elle pas exactement aussi sombre que le tableau esquissé ici ? L’événement ne l’a-t-il pas montré ?\par
Il faut avouer que le mécanisme d’oppression spirituelle et mentale propre aux partis a été introduit dans l’histoire par l’Église catholique dans sa lutte contre l’hérésie.\par
Un converti qui entre dans l’Église – ou un fidèle qui délibère avec lui-même et résout d’y demeurer – a aperçu dans le dogme du vrai et du bien. Mais en franchissant le seuil il professe du même coup n’être pas frappé par les \emph{anathema sit}, c’est-à-dire accepter en bloc tous les articles dits « de foi stricte ». Ces articles, il ne les a pas étudiés. Même avec un haut degré d’intelligence et de culture, une vie entière ne suffirait pas à cette étude, vu qu’elle implique celle des circonstances historiques de chaque condamnation.\par
Comment adhérer à des affirmations qu’on ne connaît pas ? Il suffît de se soumettre inconditionnellement à l’autorité d’où elles émanent.\par
C’est pourquoi saint Thomas ne veut soutenir ses affirmations que par l’autorité de l’Église, à l’exclusion de tout autre argument. Car, dit-il, il n’en faut pas davantage pour ceux qui l’acceptent ; et aucun argument ne persuaderait ceux qui la refusent.\par
Ainsi la lumière intérieure de l’évidence, cette faculté de discernement accordée d’en haut à l’âme humaine comme réponse au désir de vérité, est mise au rebut, condamnée aux tâches serviles, comme de faire des additions, exclue de toutes les recherches relatives à la destinée spirituelle de l’homme. Le mobile de la pensée n’est plus le désir inconditionné, non défini, de la vérité, mais le désir de la conformité avec un enseignement établi d’avance.\par
Que l’Église fondée par le Christ ait ainsi dans une si large mesure étouffé l’esprit de vérité – et si, malgré l’Inquisition, elle ne l’a pas fait totalement, c’est que la mystique offrait un refuge sûr – c’est une ironie tragique. On l’a souvent remarqué. Mais on a moins remarqué une autre ironie tragique. C’est que le mouvement de révolte contre l’étouffement des esprits sous le régime inquisitorial a pris une orientation telle qu’il a poursuivi l’œuvre d’étouffement des esprits.\par
La Réforme et l’humanisme de la Renaissance, double produit de cette révolte, ont largement contribué à susciter, après trois siècles de maturation, l’esprit de 1789. Il en est résulté après un certain délai notre démocratie fondée sur le jeu des partis, dont chacun est une petite Église profane armée de la menace d’excommunication. L’influence des partis a contaminé toute la vie mentale de notre époque.\par
Un homme qui adhère à un parti a vraisemblablement aperçu dans l’action et la propagande de ce parti \emph{des} choses qui lui ont paru justes et bonnes. Mais il n’a jamais étudié la position du parti relativement à tous les problèmes de la vie publique. En entrant dans le parti, il accepte des positions qu’il ignore. Ainsi il soumet sa pensée à l’autorité du parti. Quand, peu à peu, il connaîtra ces positions, il les admettra sans examen.\par
C’est exactement la situation de celui qui adhère à l’orthodoxie catholique conçue comme fait saint Thomas.\par
Si un homme disait, en demandant sa carte de membre : « Je suis d’accord avec le parti sur tel, tel, tel point ; je n’ai pas étudié ses autres positions et je réserve entièrement mon opinion tant que je n’en aurai pas fait l’étude », on le prierait sans doute de repasser plus tard.\par
Mais en fait, sauf exceptions très rares, un homme qui entre dans un parti adopte docilement l’attitude d’esprit qu’il exprimera plus tard par les mots : « Comme monarchiste, comme socialiste, je pense que… » C’est tellement confortable ! Car c’est ne pas penser. Il n’y a rien de plus confortable que de ne pas penser.\par
Quant au troisième caractère des partis, à savoir qu’ils sont \emph{des} machines à fabriquer de la passion collective, il \emph{est} si visible qu’il n’a pas à être établi. La passion collective est l’unique énergie dont disposent les partis pour la propagande extérieure et pour la pression exercée sur l’âme de chaque membre.\par
On avoue que l’esprit de parti aveugle, rend sourd à la justice, pousse même d’honnêtes gens à l’acharnement le plus cruel contre des innocents. \emph{On} l’avoue, mais on ne pense pas à supprimer les organismes qui fabriquent un tel esprit.\par
Cependant on interdit les stupéfiants.\par
Il y a quand même des gens adonnés aux stupéfiants.\par
Mais il y en aurait davantage si l’Etat organisait la vente de l’opium et de la cocaïne dans tous les bureaux de tabac, avec affiches de publicité pour encourager les consommateurs.\par
La conclusion, c’est que l’institution des partis semble bien constituer du mal à peu près sans mélange. Ils sont mauvais dans leur principe, et pratiquement leurs effets sont mauvais.\par
La suppression des partis serait du bien presque pur. Elle est éminemment légitime en principe et ne paraît susceptible pratiquement que de bons effets.\par
Les candidats diront aux électeurs, non pas : « J’ai telle étiquette » – ce qui pratiquement n’apprend rigoureusement rien au public sur leur attitude concrète concernant les problèmes concrets – mais : « Je pense telle, telle et telle chose à l’égard de tel, tel, tel grand problème. »\par
Les élus s’associeront et se dissocieront selon le jeu naturel et mouvant des affinités. Je peux très bien être en accord avec M. A. sur la colonisation et en désaccord avec lui sur la propriété paysanne ; et inversement pour M. B. Si on parle de colonisation, j’irai, avant la séance, causer un peu avec M. A.; si on parle de propriété paysanne, avec M. B.\par
La cristallisation artificielle en partis coïncidait si peu avec les affinités réelles qu’un député pouvait être en désaccord, pour toutes les attitudes concrètes, avec un collègue de son parti, et en accord avec un homme d’un autre parti.\par
Combien de fois, en Allemagne, en 1932, un communiste et un nazi, discutant dans la rue, ont été frappés de vertige mental en constatant qu’ils étaient d’accord sur tous les points !\par
Hors du Parlement, comme il existerait des revues d’idées, il y aurait tout naturellement autour d’elles des milieux. Mais ces milieux devraient être maintenus à l’état de fluidité. C’est la fluidité qui distingue du parti un milieu d’affinité et l’empêche d’avoir une influence mauvaise. Quand on fréquente amicalement celui qui dirige telle revue, ceux qui y écrivent souvent, quand on y écrit soi-même, on sait qu’on est en contact avec le milieu de cette revue. Mais on ne sait pas soi-même si on en fait partie ; il n’y a pas de distinction nette entre le dedans et le dehors. Plus loin, il y a ceux qui lisent la revue et connaissent un ou deux de ceux qui y écrivent. Plus loin, les lecteurs réguliers qui y puisent une inspiration. Plus loin, les lecteurs occasionnels. Mais personne ne songerait à penser ou à dire : « En tant que lié à telle revue, je pense que… »\par
Quand des collaborateurs à une revue se présentent aux élections, il doit leur être interdit de se réclamer de la revue. Il doit être interdit à la revue de leur donner une investiture, ou d’aider directement ou indirectement leur candidature, ou même d’en faire mention. Tout groupe d’ « amis » de telle revue devrait être interdit.\par
Si une revue empêche ses collaborateurs, sous peine de rupture, de collaborer à d’autres publications quelles qu’elles soient, elle doit être supprimée dès que le fait est prouvé.\par
Ceci implique un régime de la presse rendant impossibles les publications auxquelles il est déshonorant de collaborer (genre \emph{Gringoire, Marie-Claire}, etc.).\par
Toutes les fois qu’un milieu tentera de se cristalliser en donnant un caractère défini à la qualité de membre, il y aura répression pénale quand le fait semblera établi.\par
Bien entendu il y aura des partis clandestins. Mais leurs membres auront mauvaise conscience. Ils ne pourront plus faire profession publique de servilité d’esprit. Ils ne pourront faire aucune propagande au nom du parti. Le parti ne pourra plus les tenir dans un réseau sans issue d’intérêts, de sentiments et d’obligations.\par
Toutes les fois qu’une loi est impartiale, équitable, et fondée sur une vue du bien public facilement assimilable pour le peuple, elle affaiblit tout ce qu’elle interdit. Elle l’affaiblit du fait seul qu’elle existe, et indépendamment \emph{des} mesures répressives qui cherchent à en assurer l’application.\par
Cette majesté intrinsèque de la loi est un facteur de la vie publique qui est oublié depuis longtemps et dont il faut faire usage.\par
Il semble n’y avoir dans l’existence de partis clandestins aucun inconvénient « qui ne se trouve à un degré bien plus élevé du fait des partis légaux.\par
D’une manière générale, un examen attentif ne semble laisser voir à aucun égard aucun inconvénient d’aucune espèce attaché à la suppression des partis.\par
Par un singulier paradoxe les mesures de ce genre, qui sont sans inconvénients, sont en fait celles qui ont le moins de chances d’être décidées. On se dit : si c’était si simple, pourquoi est-ce que cela n’aurait pas été fait depuis longtemps ?\par
Pourtant, généralement, les grandes choses sont faciles et simples.\par
Celle-ci étendrait sa vertu d’assainissement bien au-delà des affaires publiques. Car l’esprit de parti en était arrivé à tout contaminer.\par
Les institutions qui déterminent le jeu de la vie publique influencent toujours dans un pays la totalité de la pensée, à cause du prestige du pouvoir.\par
On en est arrivé à ne presque plus penser, dans aucun domaine, qu’en prenant position « pour » ou « contre » une opinion. Ensuite on cherche des arguments, selon le cas, soit pour, soit contre. C’est exactement la transposition de l’adhésion à un parti.\par
Comme, dans les partis politiques, il y a des démocrates qui admettent plusieurs partis, de même dans le domaine des opinions les gens larges reconnaissent une valeur aux opinions avec lesquelles ils se disent en désaccord.\par
C’est avoir complètement perdu le \emph{sens} même du vrai et du faux.\par
D’autres, ayant pris position pour une opinion, ne consentent à examiner rien qui lui soit contraire. C’est la transposition de l’esprit totalitaire.\par
Quand Einstein vint en France, tous les gens des milieux plus ou moins intellectuels, y compris les savants eux-mêmes, se divisèrent en deux camps, pour et contre. Toute pensée scientifique nouvelle a dans les milieux scientifiques ses partisans et ses adversaires animés les uns et les autres, à un degré regrettable, de l’esprit de parti. Il y a d’ailleurs dans ces milieux des tendances, des coteries, à l’état plus ou moins cristallisé.\par
Dans l’art et la littérature, c’est bien plus visible encore. Cubisme et surréalisme ont été des espèces de partis. On était « gidien » comme on était « maurrassien ». Pour avoir un nom, il est utile d’être entouré d’une bande d’admirateurs animés de l’esprit de parti.\par
De même il n’y avait pas grande différence entre l’attachement à un parti et l’attachement à une Église ou bien à l’attitude antireligieuse. On était pour ou contre la croyance en Dieu, pour ou contre le christianisme, et ainsi de suite. On en est arrivé, en matière de religion, à parler de militants.\par
Même dans les écoles on ne sait plus stimuler autrement la pensée des enfants qu’en les invitant à prendre parti pour ou contre. On leur cite une phrase de grand auteur et \emph{on} leur dit : « Êtes-vous d’accord ou non ? Développez vos arguments. » A l’examen les malheureux, devant avoir fini leur dissertation au bout de trois heures, ne peuvent passer plus de cinq minutes à se demander s’ils sont d’accord. Et il serait si facile de leur dire : « Méditez ce texte et exprimez les réflexions qui vous viennent à l’esprit ».\par
Presque partout – et même souvent pour des problèmes purement techniques – l’opération de prendre parti, de prendre position pour ou contre, s’est substituée à l’obligation de la pensée.\par
C’est là une lèpre qui a pris origine dans les milieux politiques, et s’est étendue, à travers tout le pays, presque à la totalité de la pensée.\par
Il est douteux qu’on puisse remédier à cette lèpre, qui nous tue, sans commencer par la suppression des partis politiques.
 


% at least one empty page at end (for booklet couv)
\ifbooklet
  \pagestyle{empty}
  \clearpage
  % 2 empty pages maybe needed for 4e cover
  \ifnum\modulo{\value{page}}{4}=0 \hbox{}\newpage\hbox{}\newpage\fi
  \ifnum\modulo{\value{page}}{4}=1 \hbox{}\newpage\hbox{}\newpage\fi


  \hbox{}\newpage
  \ifodd\value{page}\hbox{}\newpage\fi
  {\centering\color{rubric}\bfseries\noindent\large
    Hurlus ? Qu’est-ce.\par
    \bigskip
  }
  \noindent Des bouquinistes électroniques, pour du texte libre à participations libres,
  téléchargeable gratuitement sur \href{https://hurlus.fr}{\dotuline{hurlus.fr}}.\par
  \bigskip
  \noindent Cette brochure a été produite par des éditeurs bénévoles.
  Elle n’est pas faite pour être possédée, mais pour être lue, et puis donnée, ou déposée dans une boîte à livres.
  En page de garde, on peut ajouter une date, un lieu, un nom ;
  comme une fiche de bibliothèque en papier qui enregistre \emph{les voyages de la brochure}.
  \par

  Ce texte a été choisi parce qu’une personne l’a aimé,
  ou haï, elle a pensé qu’il partipait à la formation de notre présent ;
  sans le souci de plaire, vendre, ou militer pour une cause.
  \par

  L’édition électronique est soigneuse, tant sur la technique
  que sur l’établissement du texte ; mais sans aucune prétention scolaire, au contraire.
  Le but est de s’adresser à tous, sans distinction de science ou de diplôme.
  \par

  Cet exemplaire en papier a été tiré sur une imprimante personnelle
   ou une photocopieuse. Tout le monde peut le faire.
  Il suffit de
  télécharger un fichier sur \href{https://hurlus.fr}{\dotuline{hurlus.fr}},
  d’imprimer, et agrafer (puis lire et donner).\par

  \bigskip

  \noindent PS : Les hurlus furent aussi des rebelles protestants qui cassaient les statues dans les églises catholiques. En 1566 démarra la révolte des gueux dans le pays de Lille. L’insurrection enflamma la région jusqu’à Anvers où les gueux de mer bloquèrent les bateaux espagnols.
  Ce fut une rare guerre de libération dont naquit un pays toujours libre : les Pays-Bas.
  En plat pays francophone, par contre, restèrent des bandes de huguenots, les hurlus, progressivement réprimés par la très catholique Espagne.
  Cette mémoire d’une défaite est éteinte, rallumons-la. Sortons les livres du culte universitaire, débusquons les idoles de l’époque, pour les démonter.
\fi

\end{document}
