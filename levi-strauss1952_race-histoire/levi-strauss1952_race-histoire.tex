%%%%%%%%%%%%%%%%%%%%%%%%%%%%%%%%%
% LaTeX model https://hurlus.fr %
%%%%%%%%%%%%%%%%%%%%%%%%%%%%%%%%%

% Needed before document class
\RequirePackage{pdftexcmds} % needed for tests expressions
\RequirePackage{fix-cm} % correct units

% Define mode
\def\mode{a4}

\newif\ifaiv % a4
\newif\ifav % a5
\newif\ifbooklet % booklet
\newif\ifcover % cover for booklet

\ifnum \strcmp{\mode}{cover}=0
  \covertrue
\else\ifnum \strcmp{\mode}{booklet}=0
  \booklettrue
\else\ifnum \strcmp{\mode}{a5}=0
  \avtrue
\else
  \aivtrue
\fi\fi\fi

\ifbooklet % do not enclose with {}
  \documentclass[french,twoside]{book} % ,notitlepage
  \usepackage[%
    papersize={105mm, 297mm},
    inner=12mm,
    outer=12mm,
    top=20mm,
    bottom=15mm,
    marginparsep=0pt,
  ]{geometry}
  \usepackage[fontsize=9.5pt]{scrextend} % for Roboto
\else\ifav
  \documentclass[french,twoside]{book} % ,notitlepage
  \usepackage[%
    a5paper,
    inner=25mm,
    outer=15mm,
    top=15mm,
    bottom=15mm,
    marginparsep=0pt,
  ]{geometry}
  \usepackage[fontsize=12pt]{scrextend}
\else% A4 2 cols
  \documentclass[twocolumn]{report}
  \usepackage[%
    a4paper,
    inner=15mm,
    outer=10mm,
    top=25mm,
    bottom=18mm,
    marginparsep=0pt,
  ]{geometry}
  \setlength{\columnsep}{20mm}
  \usepackage[fontsize=9.5pt]{scrextend}
\fi\fi

%%%%%%%%%%%%%%
% Alignments %
%%%%%%%%%%%%%%
% before teinte macros

\setlength{\arrayrulewidth}{0.2pt}
\setlength{\columnseprule}{\arrayrulewidth} % twocol
\setlength{\parskip}{0pt} % 1pt allow better vertical justification
\setlength{\parindent}{1.5em}

%%%%%%%%%%
% Colors %
%%%%%%%%%%
% before Teinte macros

\usepackage[dvipsnames]{xcolor}
\definecolor{rubric}{HTML}{800000} % the tonic 0c71c3
\def\columnseprulecolor{\color{rubric}}
\colorlet{borderline}{rubric!30!} % definecolor need exact code
\definecolor{shadecolor}{gray}{0.95}
\definecolor{bghi}{gray}{0.5}

%%%%%%%%%%%%%%%%%
% Teinte macros %
%%%%%%%%%%%%%%%%%
%%%%%%%%%%%%%%%%%%%%%%%%%%%%%%%%%%%%%%%%%%%%%%%%%%%
% <TEI> generic (LaTeX names generated by Teinte) %
%%%%%%%%%%%%%%%%%%%%%%%%%%%%%%%%%%%%%%%%%%%%%%%%%%%
% This template is inserted in a specific design
% It is XeLaTeX and otf fonts

\makeatletter % <@@@

\usepackage{alphalph} % for alph couter z, aa, ab…
\usepackage{blindtext} % generate text for testing
\usepackage[strict]{changepage} % for modulo 4
\usepackage{contour} % rounding words
\usepackage[nodayofweek]{datetime}
\usepackage{enumitem} % <list>
\usepackage{epigraph} % <epigraph>
\usepackage{etoolbox} % patch commands
\usepackage{fancyvrb}
\usepackage{fancyhdr}
\usepackage{float}
\usepackage{fontspec} % XeLaTeX mandatory for fonts
\usepackage{footnote} % used to capture notes in minipage (ex: quote)
\usepackage{framed} % bordering correct with footnote hack
\usepackage{graphicx}
\usepackage{lettrine} % drop caps
\usepackage{lipsum} % generate text for testing
\usepackage{manyfoot} % for parallel footnote numerotation
\usepackage[framemethod=tikz,]{mdframed} % maybe used for frame with footnotes inside
\usepackage[defaultlines=2,all]{nowidow} % at least 2 lines by par (works well!)
\usepackage{pdftexcmds} % needed for tests expressions
\usepackage{poetry} % <l>, bad for theater
\usepackage{polyglossia} % bug Warning: "Failed to patch part"
\usepackage[%
  indentfirst=false,
  vskip=1em,
  noorphanfirst=true,
  noorphanafter=true,
  leftmargin=\parindent,
  rightmargin=0pt,
]{quoting}
\usepackage{ragged2e}
\usepackage{setspace} % \setstretch for <quote>
\usepackage{tabularx} % <table>
\usepackage[explicit]{titlesec} % wear titles, !NO implicit
\usepackage{tikz} % ornaments
\usepackage{tocloft} % styling tocs
\usepackage[fit]{truncate} % used im runing titles
\usepackage{unicode-math}
\usepackage[normalem]{ulem} % breakable \uline, normalem is absolutely necessary to keep \emph
\usepackage{xcolor} % named colors
\usepackage{xparse} % @ifundefined
\XeTeXdefaultencoding "iso-8859-1" % bad encoding of xstring
\usepackage{xstring} % string tests
\XeTeXdefaultencoding "utf-8"


% TOTEST
% \usepackage{hypcap} % links in caption ?
% \usepackage{marginnote}
% TESTED
% \usepackage{background} % doesn’t work with xetek
% \usepackage{bookmark} % prefers the hyperref hack \phantomsection
% \usepackage[color, leftbars]{changebar} % 2 cols doc, impossible to keep bar left
% \usepackage{DejaVuSans} % override too much, was for for symbols
% \usepackage[utf8x]{inputenc} % inputenc package ignored with utf8 based engines
% \usepackage[sfdefault,medium]{inter} % no small caps
% \usepackage{firamath} % choose firasans instead, firamath unavailable in Ubuntu 21-04
% \usepackage{flushend} % bad for last notes, supposed flush end of columns
% \usepackage[stable]{footmisc} % BAD for complex notes https://texfaq.org/FAQ-ftnsect
% \usepackage{helvet} % not for XeLaTeX
% \usepackage{multicol} % not compatible with too much packages (longtable, framed, memoir…)
% \usepackage[default,oldstyle,scale=0.95]{opensans} % no small caps
% \usepackage{sectsty} % \chapterfont OBSOLETE
% \usepackage{soul} % \ul for underline, OBSOLETE with XeTeX
% \usepackage[breakable]{tcolorbox} % text styling gone, footnote hack not kept with breakable
% \usepackage{verse} % not enough control on indent, poetry is better

\defaultfontfeatures{
  % Mapping=tex-text, % no effect seen
  Scale=MatchLowercase,
  Ligatures={TeX,Common},
}
\newfontfamily\zhfont{Noto Sans CJK SC}

% Metadata inserted by a program, from the TEI source, for title page and runing heads
\title{\textbf{ Race et Histoire }\par
\medskip
\textit{ l’édition Unesco }\par
}
\date{1952}
\author{Lévi-Strauss}
\def\elbibl{Lévi-Strauss. 1952. \emph{Race et Histoire}}
\def\elsource{ \href{https://unesdoc.unesco.org/ark:/48223/pf0000005546}{\dotuline{Unesco}}\footnote{\href{https://unesdoc.unesco.org/ark:/48223/pf0000005546}{\url{https://unesdoc.unesco.org/ark:/48223/pf0000005546}}} }
\def\eltitlepage{%
{\centering\parindent0pt
  {\LARGE\addfontfeature{LetterSpace=25}\bfseries Lévi-Strauss\par}\bigskip
  {\Large 1952\par}\bigskip
  {\LARGE
\bigskip\textbf{Race et Histoire}\par
\bigskip\emph{l’édition Unesco}\par

  }
}

}

% Default metas
\newcommand{\colorprovide}[2]{\@ifundefinedcolor{#1}{\colorlet{#1}{#2}}{}}
\colorprovide{rubric}{red}
\colorprovide{silver}{lightgray}
\@ifundefined{syms}{\newfontfamily\syms{DejaVu Sans}}{}
\newif\ifdev
\@ifundefined{elbibl}{% No meta defined, maybe dev mode
  \newcommand{\elbibl}{Titre court ?}
  \newcommand{\elbook}{Titre du livre source ?}
  \newcommand{\elabstract}{Résumé\par}
  \newcommand{\elurl}{http://oeuvres.github.io/elbook/2}
  \author{Éric Lœchien}
  \title{Un titre de test assez long pour vérifier le comportement d’une maquette}
  \date{1566}
  \devtrue
}{}
\let\eltitle\@title
\let\elauthor\@author
\let\eldate\@date




% generic typo commands
\newcommand{\astermono}{\medskip\centerline{\color{rubric}\large\selectfont{\syms ✻}}\medskip\par}%
\newcommand{\astertri}{\medskip\par\centerline{\color{rubric}\large\selectfont{\syms ✻\,✻\,✻}}\medskip\par}%
\newcommand{\asterism}{\bigskip\par\noindent\parbox{\linewidth}{\centering\color{rubric}\large{\syms ✻}\\{\syms ✻}\hskip 0.75em{\syms ✻}}\bigskip\par}%

% lists
\newlength{\listmod}
\setlength{\listmod}{\parindent}
\setlist{
  itemindent=!,
  listparindent=\listmod,
  labelsep=0.2\listmod,
  parsep=0pt,
  % topsep=0.2em, % default topsep is best
}
\setlist[itemize]{
  label=—,
  leftmargin=0pt,
  labelindent=1.2em,
  labelwidth=0pt,
}
\setlist[enumerate]{
  label={\arabic*°},
  labelindent=0.8\listmod,
  leftmargin=\listmod,
  labelwidth=0pt,
}
% list for big items
\newlist{decbig}{enumerate}{1}
\setlist[decbig]{
  label={\bf\color{rubric}\arabic*.},
  labelindent=0.8\listmod,
  leftmargin=\listmod,
  labelwidth=0pt,
}
\newlist{listalpha}{enumerate}{1}
\setlist[listalpha]{
  label={\bf\color{rubric}\alph*.},
  leftmargin=0pt,
  labelindent=0.8\listmod,
  labelwidth=0pt,
}
\newcommand{\listhead}[1]{\hspace{-1\listmod}\emph{#1}}

\renewcommand{\hrulefill}{%
  \leavevmode\leaders\hrule height 0.2pt\hfill\kern\z@}

% General typo
\DeclareTextFontCommand{\textlarge}{\large}
\DeclareTextFontCommand{\textsmall}{\small}

% commands, inlines
\newcommand{\anchor}[1]{\Hy@raisedlink{\hypertarget{#1}{}}} % link to top of an anchor (not baseline)
\newcommand\abbr[1]{#1}
\newcommand{\autour}[1]{\tikz[baseline=(X.base)]\node [draw=rubric,thin,rectangle,inner sep=1.5pt, rounded corners=3pt] (X) {\color{rubric}#1};}
\newcommand\corr[1]{#1}
\newcommand{\ed}[1]{ {\color{silver}\sffamily\footnotesize (#1)} } % <milestone ed="1688"/>
\newcommand\expan[1]{#1}
\newcommand\foreign[1]{\emph{#1}}
\newcommand\gap[1]{#1}
\renewcommand{\LettrineFontHook}{\color{rubric}}
\newcommand{\initial}[2]{\lettrine[lines=2, loversize=0.3, lhang=0.3]{#1}{#2}}
\newcommand{\initialiv}[2]{%
  \let\oldLFH\LettrineFontHook
  % \renewcommand{\LettrineFontHook}{\color{rubric}\ttfamily}
  \IfSubStr{QJ’}{#1}{
    \lettrine[lines=4, lhang=0.2, loversize=-0.1, lraise=0.2]{\smash{#1}}{#2}
  }{\IfSubStr{É}{#1}{
    \lettrine[lines=4, lhang=0.2, loversize=-0, lraise=0]{\smash{#1}}{#2}
  }{\IfSubStr{ÀÂ}{#1}{
    \lettrine[lines=4, lhang=0.2, loversize=-0, lraise=0, slope=0.6em]{\smash{#1}}{#2}
  }{\IfSubStr{A}{#1}{
    \lettrine[lines=4, lhang=0.2, loversize=0.2, slope=0.6em]{\smash{#1}}{#2}
  }{\IfSubStr{V}{#1}{
    \lettrine[lines=4, lhang=0.2, loversize=0.2, slope=-0.5em]{\smash{#1}}{#2}
  }{
    \lettrine[lines=4, lhang=0.2, loversize=0.2]{\smash{#1}}{#2}
  }}}}}
  \let\LettrineFontHook\oldLFH
}
\newcommand{\labelchar}[1]{\textbf{\color{rubric} #1}}
\newcommand{\milestone}[1]{\autour{\footnotesize\color{rubric} #1}} % <milestone n="4"/>
\newcommand\name[1]{#1}
\newcommand\orig[1]{#1}
\newcommand\orgName[1]{#1}
\newcommand\persName[1]{#1}
\newcommand\placeName[1]{#1}
\newcommand{\pn}[1]{\IfSubStr{-—–¶}{#1}% <p n="3"/>
  {\noindent{\bfseries\color{rubric}   ¶  }}
  {{\footnotesize\autour{#1}}}}
\newcommand\reg{}
% \newcommand\ref{} % already defined
\newcommand\sic[1]{#1}
\newcommand\surname[1]{\textsc{#1}}
\newcommand\term[1]{\textbf{#1}}
\newcommand\zh[1]{{\zhfont #1}}


\def\mednobreak{\ifdim\lastskip<\medskipamount
  \removelastskip\nopagebreak\medskip\fi}
\def\bignobreak{\ifdim\lastskip<\bigskipamount
  \removelastskip\nopagebreak\bigskip\fi}

% commands, blocks
\newcommand{\byline}[1]{\bigskip{\RaggedLeft{#1}\par}\bigskip}
\newcommand{\bibl}[1]{{\RaggedLeft{#1}\par\bigskip}}
\newcommand{\biblitem}[1]{{\noindent\hangindent=\parindent   #1\par}}
\newcommand{\castItem}[1]{{\noindent\hangindent=\parindent #1\par}}
\newcommand{\dateline}[1]{\medskip{\RaggedLeft{#1}\par}\bigskip}
\newcommand{\labelblock}[1]{\medbreak{\noindent\color{rubric}\bfseries #1}\par\mednobreak}
\newcommand{\salute}[1]{\bigbreak{#1}\par\medbreak}
\newcommand{\signed}[1]{\medskip{\raggedleft #1\par}\bigbreak} % supposed bottom
\newcommand{\speaker}[1]{\medskip{\centering #1\par\nopagebreak}} % supposed bottom
\newcommand{\spl}[1]{\noindent\hangindent=2\parindent  #1\par} % sp/l


% environments for blocks (some may become commands)
\newenvironment{borderbox}{}{} % framing content
\newenvironment{citbibl}{\ifvmode\hfill\fi}{\ifvmode\par\fi }
\newenvironment{docAuthor}{\ifvmode\vskip4pt\fontsize{16pt}{18pt}\selectfont\fi\itshape}{\ifvmode\par\fi }
\newenvironment{docDate}{}{\ifvmode\par\fi }
\newenvironment{docImprint}{\vskip6pt}{\ifvmode\par\fi }
\newenvironment{docTitle}{\vskip6pt\bfseries\fontsize{18pt}{22pt}\selectfont}{\par }
\newenvironment{msHead}{\vskip6pt}{\par}
\newenvironment{msItem}{\vskip6pt}{\par}
\newenvironment{titlePart}{}{\par }


% environments for block containers
\newenvironment{argument}{\itshape\parindent0pt}{\bigskip}
\newenvironment{biblfree}{}{\ifvmode\par\fi }
\newenvironment{bibitemlist}[1]{%
  \list{\@biblabel{\@arabic\c@enumiv}}%
  {%
    \settowidth\labelwidth{\@biblabel{#1}}%
    \leftmargin\labelwidth
    \advance\leftmargin\labelsep
    \@openbib@code
    \usecounter{enumiv}%
    \let\p@enumiv\@empty
    \renewcommand\theenumiv{\@arabic\c@enumiv}%
  }
  \sloppy
  \clubpenalty4000
  \@clubpenalty \clubpenalty
  \widowpenalty4000%
  \sfcode`\.\@m
}%
{\def\@noitemerr
  {\@latex@warning{Empty `bibitemlist' environment}}%
\endlist}
\newenvironment{quoteblock}% may be used for ornaments
  {\begin{quoting}}
  {\end{quoting}}
% \newenvironment{epigraph}{\parindent0pt\raggedleft\it}{\bigskip}
 % epigraph pack
\setlength{\epigraphrule}{0pt}
\setlength{\epigraphwidth}{0.8\textwidth}
% \renewcommand{\epigraphflush}{center} ? dont work

% table () is preceded and finished by custom command
\newcommand{\tableopen}[1]{%
  \ifnum\strcmp{#1}{wide}=0{%
    \begin{center}
  }
  \else\ifnum\strcmp{#1}{long}=0{%
    \begin{center}
  }
  \else{%
    \begin{center}
  }
  \fi\fi
}
\newcommand{\tableclose}[1]{%
  \ifnum\strcmp{#1}{wide}=0{%
    \end{center}
  }
  \else\ifnum\strcmp{#1}{long}=0{%
    \end{center}
  }
  \else{%
    \end{center}
  }
  \fi\fi
}


% text structure
\newcommand\chapteropen{} % before chapter title
\newcommand\chaptercont{} % after title, argument, epigraph…
\newcommand\chapterclose{} % maybe useful for multicol settings
\setcounter{secnumdepth}{-2} % no counters for hierarchy titles
\setcounter{tocdepth}{5} % deep toc
\renewcommand\tableofcontents{\@starttoc{toc}}
% toclof format
% \renewcommand{\@tocrmarg}{0.1em} % Useless command?
% \renewcommand{\@pnumwidth}{0.5em} % {1.75em}
\renewcommand{\@cftmaketoctitle}{}
\setlength{\cftbeforesecskip}{\z@ \@plus.2\p@}
\renewcommand{\cftchapfont}{}
\renewcommand{\cftchapdotsep}{\cftdotsep}
\renewcommand{\cftchapleader}{\normalfont\cftdotfill{\cftchapdotsep}}
\renewcommand{\cftchappagefont}{\bfseries}
\setlength{\cftbeforechapskip}{0em \@plus\p@}
% \renewcommand{\cftsecfont}{\small\relax}
\renewcommand{\cftsecpagefont}{\normalfont}
% \renewcommand{\cftsubsecfont}{\small\relax}
\renewcommand{\cftsecdotsep}{\cftdotsep}
\renewcommand{\cftsecpagefont}{\normalfont}
\renewcommand{\cftsecleader}{\normalfont\cftdotfill{\cftsecdotsep}}
\setlength{\cftsecindent}{1em}
\setlength{\cftsubsecindent}{2em}
\setlength{\cftsubsubsecindent}{3em}
\setlength{\cftchapnumwidth}{1em}
\setlength{\cftsecnumwidth}{1em}
\setlength{\cftsubsecnumwidth}{1em}
\setlength{\cftsubsubsecnumwidth}{1em}

% footnotes
\newif\ifheading
\newcommand*{\fnmarkscale}{\ifheading 0.70 \else 1 \fi}
\renewcommand\footnoterule{\vspace*{0.3cm}\hrule height \arrayrulewidth width 3cm \vspace*{0.3cm}}
\setlength\footnotesep{1.5\footnotesep} % footnote separator
\renewcommand\@makefntext[1]{\parindent 1.5em \noindent \hb@xt@1.8em{\hss{\normalfont\@thefnmark . }}#1} % no superscipt in foot
\patchcmd{\@footnotetext}{\footnotesize}{\footnotesize\sffamily}{}{} % before scrextend, hyperref
\DeclareNewFootnote{A}[alph] % for editor notes
\renewcommand*{\thefootnoteA}{\alphalph{\value{footnoteA}}} % z, aa, ab…

% poem
\setlength{\poembotskip}{0pt}
\setlength{\poemtopskip}{0pt}
\setlength{\poemindent}{0pt}
\poemlinenumsfalse

%   see https://tex.stackexchange.com/a/34449/5049
\def\truncdiv#1#2{((#1-(#2-1)/2)/#2)}
\def\moduloop#1#2{(#1-\truncdiv{#1}{#2}*#2)}
\def\modulo#1#2{\number\numexpr\moduloop{#1}{#2}\relax}

% orphans and widows, nowidow package in test
% from memoir package
\clubpenalty=9996
\widowpenalty=9999
\brokenpenalty=4991
\predisplaypenalty=10000
\postdisplaypenalty=1549
\displaywidowpenalty=1602
\hyphenpenalty=400
% report h or v overfull ?
\hbadness=4000
\vbadness=4000
% good to avoid lines too wide
\emergencystretch 3em
\pretolerance=750
\tolerance=2000
\def\Gin@extensions{.pdf,.png,.jpg,.mps,.tif}

\PassOptionsToPackage{hyphens}{url} % before hyperref and biblatex, which load url package
\usepackage{hyperref} % supposed to be the last one, :o) except for the ones to follow
\hypersetup{
  % pdftex, % no effect
  pdftitle={\elbibl},
  % pdfauthor={Your name here},
  % pdfsubject={Your subject here},
  % pdfkeywords={keyword1, keyword2},
  bookmarksnumbered=true,
  bookmarksopen=true,
  bookmarksopenlevel=1,
  pdfstartview=Fit,
  breaklinks=true, % avoid long links, overrided by url package
  pdfpagemode=UseOutlines,    % pdf toc
  hyperfootnotes=true,
  colorlinks=false,
  pdfborder=0 0 0,
  % pdfpagelayout=TwoPageRight,
  % linktocpage=true, % NO, toc, link only on page no
}
\urlstyle{same} % after hyperref



\makeatother % /@@@>
%%%%%%%%%%%%%%
% </TEI> end %
%%%%%%%%%%%%%%


%%%%%%%%%%%%%
% footnotes %
%%%%%%%%%%%%%
\renewcommand{\thefootnote}{\bfseries\textcolor{rubric}{\arabic{footnote}}} % color for footnote marks

%%%%%%%%%
% Fonts %
%%%%%%%%%
\usepackage[]{roboto} % SmallCaps, Regular is a bit bold
% \linespread{0.90} % too compact, keep font natural
\newfontfamily\fontrun[]{Roboto Condensed Light} % condensed runing heads
\ifav
  \setmainfont[
    ItalicFont={Roboto Light Italic},
  ]{Roboto}
\else\ifbooklet
  \setmainfont[
    ItalicFont={Roboto Light Italic},
  ]{Roboto}
\else
\setmainfont[
  ItalicFont={Roboto Italic},
]{Roboto Light}
\fi\fi
\renewcommand{\LettrineFontHook}{\bfseries\color{rubric}}
% \renewenvironment{labelblock}{\begin{center}\bfseries\color{rubric}}{\end{center}}

%%%%%%%%
% MISC %
%%%%%%%%

\setdefaultlanguage[frenchpart=false]{french} % bug on part


\newenvironment{quotebar}{%
    \def\FrameCommand{{\color{rubric!10!}\vrule width 0.5em} \hspace{0.9em}}%
    \def\OuterFrameSep{0pt} % séparateur vertical
    \MakeFramed {\advance\hsize-\width \FrameRestore}
  }%
  {%
    \endMakeFramed
  }
\renewenvironment{quoteblock}% may be used for ornaments
  {%
    \savenotes
    \setstretch{0.9}
    \normalfont
    \begin{quotebar}
  }
  {%
    \end{quotebar}
    \spewnotes
  }


\renewcommand{\headrulewidth}{\arrayrulewidth}
\renewcommand{\headrule}{{\color{rubric}\hrule}}

% delicate tuning, image has produce line-height problems in title on 2 lines
\titleformat{name=\chapter} % command
  [display] % shape
  {\vspace{1.5em}\centering} % format
  {} % label
  {0pt} % separator between n
  {}
[{\color{rubric}\huge\textbf{#1}}\bigskip] % after code
% \titlespacing{command}{left spacing}{before spacing}{after spacing}[right]
\titlespacing*{\chapter}{0pt}{-2em}{0pt}[0pt]

\titleformat{name=\section}
  [display]{}{}{}{}
  [\vbox{\color{rubric}\large\raggedleft\textbf{#1}}]
\titlespacing{\section}{0pt}{0pt plus 4pt minus 2pt}{\baselineskip}

\titleformat{name=\subsection}
  [block]
  {}
  {} % \thesection
  {} % separator \arrayrulewidth
  {}
[\vbox{\large\textbf{#1}}]
% \titlespacing{\subsection}{0pt}{0pt plus 4pt minus 2pt}{\baselineskip}

\ifaiv
  \fancypagestyle{main}{%
    \fancyhf{}
    \setlength{\headheight}{1.5em}
    \fancyhead{} % reset head
    \fancyfoot{} % reset foot
    \fancyhead[L]{\truncate{0.45\headwidth}{\fontrun\elbibl}} % book ref
    \fancyhead[R]{\truncate{0.45\headwidth}{ \fontrun\nouppercase\leftmark}} % Chapter title
    \fancyhead[C]{\thepage}
  }
  \fancypagestyle{plain}{% apply to chapter
    \fancyhf{}% clear all header and footer fields
    \setlength{\headheight}{1.5em}
    \fancyhead[L]{\truncate{0.9\headwidth}{\fontrun\elbibl}}
    \fancyhead[R]{\thepage}
  }
\else
  \fancypagestyle{main}{%
    \fancyhf{}
    \setlength{\headheight}{1.5em}
    \fancyhead{} % reset head
    \fancyfoot{} % reset foot
    \fancyhead[RE]{\truncate{0.9\headwidth}{\fontrun\elbibl}} % book ref
    \fancyhead[LO]{\truncate{0.9\headwidth}{\fontrun\nouppercase\leftmark}} % Chapter title, \nouppercase needed
    \fancyhead[RO,LE]{\thepage}
  }
  \fancypagestyle{plain}{% apply to chapter
    \fancyhf{}% clear all header and footer fields
    \setlength{\headheight}{1.5em}
    \fancyhead[L]{\truncate{0.9\headwidth}{\fontrun\elbibl}}
    \fancyhead[R]{\thepage}
  }
\fi

\ifav % a5 only
  \titleclass{\section}{top}
\fi

\newcommand\chapo{{%
  \vspace*{-3em}
  \centering\parindent0pt % no vskip ()
  \eltitlepage
  \bigskip
  {\color{rubric}\hline\par}
  \bigskip
  {\Large TEXTE LIBRE À PARTICIPATIONS LIBRES\par}
  \centerline{\small\color{rubric} {\href{https://hurlus.fr}{\dotuline{hurlus.fr}}}, tiré le \today}\par
  \bigskip
}}

\newcommand\cover{{%
  \thispagestyle{empty}
  \centering\parindent0pt
  \eltitlepage
  \vfill\null
  {\color{rubric}\setlength{\arrayrulewidth}{2pt}\hline\par}
  \vfill\null
  {\Large TEXTE LIBRE À PARTICIPATIONS LIBRES\par}
  \centerline{\href{https://hurlus.fr}{\dotuline{hurlus.fr}}, tiré le \today}\par
}}

\begin{document}
\pagestyle{empty}
\ifbooklet{
  \cover\newpage
  \thispagestyle{empty}\hbox{}\newpage
  \cover\newpage\noindent Les voyages de la brochure\par
  \bigskip
  \begin{tabularx}{\textwidth}{l|X|X}
    \textbf{Date} & \textbf{Lieu}& \textbf{Nom/pseudo} \\ \hline
    \rule{0pt}{25cm} &  &   \\
  \end{tabularx}
  \newpage
  \addtocounter{page}{-4}
}\fi

\thispagestyle{empty}
\ifaiv
  \twocolumn[\chapo]
\else
  \chapo
\fi
{\it\elabstract}
\bigskip
\makeatletter\@starttoc{toc}\makeatother % toc without new page
\bigskip

\pagestyle{main} % after style
\setcounter{footnote}{0}
\setcounter{footnoteA}{0}
  
\section[{Race et culture}]{Race et culture}
\renewcommand{\leftmark}{Race et culture}

\noindent Parler de contribution des races humaines à la civilisation mondiale pourrait avoir de quoi surprendre, dans une collection de brochures destinées à lutter contre le préjugé raciste. Il serait vain d’avoir consacré tant de talent et tant d’efforts à montrer que rien, dans l’état actuel de la science, ne permet d’affirmer la supériorité ou l’infériorité intellectuelle d’une race par rapport à une autre, si c’était seulement pour restituer subrepticement sa consistance à la notion de race, en paraissant démontrer que les grands groupes ethniques qui composent l’humanité ont apporté, \emph{en tant que tels}, des contributions spécifiques au patrimoine commun.\par
Mais rien n’est plus éloigné de notre dessein qu’une telle entreprise qui aboutirait seulement à formuler la doctrine raciste à l’envers. Quand on cherche à caractériser les races biologiques par des propriétés psychologiques particulières, on s’écarte autant de la vérité scientifique en les définissant de façon positive que négative. Il ne faut pas oublier que Gobineau, dont l’histoire a fait le père des théories racistes, ne concevait pourtant pas l’« inégalité des races humaines » de manière quantitative, mais qualitative : pour lui, les grandes races primitives qui formaient l’humanité à ses débuts – blanche, jaune, noire – n’étaient pas tellement inégales en valeur absolue que diverses dans leurs aptitudes particulières. La tare de la dégénérescence s’attachait pour lui au phénomène du métissage plutôt qu’à la position de chaque race dans une échelle de valeurs commune à toutes ; elle était donc destinée à frapper l’humanité tout entière, condamnée, sans distinction de race, à un métissage de plus en plus poussé. Mais le péché originel de l’anthropologie consiste dans la confusion entre la notion purement biologique de race (à supposer, d’ailleurs, que, même sur ce terrain limité, cette notion puisse prétendre à l’objectivité, ce que la génétique moderne conteste) et les productions sociologiques et psychologiques des cultures humaines. Il a suffi à Gobineau de l’avoir commis pour se trouver enfermé dans le cercle infernal qui conduit d’une erreur intellectuelle n’excluant pas la bonne foi à la légitimation involontaire de toutes les tentatives de descrimi – nation et d’exploitation.\par
Aussi, quand nous parlons, en cette étude, de contribution des races humaines à la civilisation, ne voulons-nous pas dire que les apports culturels de l’Asie ou de l’Europe, de l’Afrique ou de l’Amérique tirent une quelconque originalité du fait que ces continents sont, en gros, peuplés par des habitants de souches raciales différentes. Si cette originalité existe – et la chose n’est pas douteuse – elle tient à des circonstances géographiques, historiques et sociologiques, non à des aptitudes distinctes liées à la constitution anatomique ou physiologique des noirs, des jaunes ou des blancs. Mais il nous est apparu que, dans la mesure même où cette série de brochures s’est efforcée de faire droit à ce point de vue négatif, elle risquait, en même temps, de reléguer au second plan un aspect également très important de la vie de l’humanité : à savoir que celle-ci ne se développe pas sous le régime d’une uniforme monotonie, mais à travers des modes extraordinairement diversifiés de sociétés et de civilisations ; cette diversité intellectuelle, esthétique, sociologique n’est unie par aucune relation de cause à effet à celle qui existe, sur le plan biologique, entre certains aspects observables des groupements humains : elle lui est seulement parallèle sur un autre terrain. Mais, en même temps, elle s’en distingue par deux caractères importants. D’abord elle se situe dans un autre ordre de grandeur. Il y a beaucoup plus de cultures humaines que de races humaines, puisque les unes se comptent par milliers et les autres par unités : deux cultures élaborées par des hommes appartenant à la même race peuvent différer autant, ou davantage, que deux cultures relevant de groupes racialement éloignés. En second lieu, à l’inverse de la diversité entre les races, qui présente pour principal intérêt celui de leur origine historique et de leur distribution dans l’espace, la diversité entre les cultures pose de nombreux problèmes, car on peut se demander si elle constitue pour l’humanité un avantage ou un inconvénient, question d’ensemble qui se subdivise, bien entendu, en beaucoup d’autres.\par
Enfin et surtout on doit se demander en quoi consiste cette diversité, au risque de voir les préjugés racistes, à peine déracinés de leur base biologique, se reformer sur un nouveau terrain. Car il serait vain d’avoir obtenu de l’homme de la rue qu’il renonce à attribuer une signification intellectuelle ou morale au fait d’avoir la peau noire ou blanche, le cheveu lisse ou crépu, pour rester silencieux devant une autre question à laquelle l’expérience prouve qu’il se raccroche immédiatement : s’il n’existe pas d’aptitudes raciales innées, comment expliquer que la civilisation développée par l’homme blanc ait fait les immenses progrès que l’on sait, tandis que celles des peuples de couleur sont restées en arrière, les unes à mi-chemin, les autres frappées d’un retard qui se chiffre par milliers ou dizaines de milliers d’années ? On ne saurait donc prétendre avoir résolu par la négative le problème de l’inégalité des \emph{races} humaines, si l’on ne se penche pas aussi sur celui de l’inégalité – ou de la diversité – des \emph{cultures} humaines qui, en fait sinon en droit, lui est, dans l’esprit public, étroitement lié.

\section[{Diversité des cultures}]{Diversité des cultures}
\renewcommand{\leftmark}{Diversité des cultures}

\noindent Pour comprendre comment, et dans quelle mesure, les cultures humaines diffèrent entre elles, si ces différences s’annulent ou se contredisent, ou si elles concourent à former un ensemble harmonieux, il faut d’abord essayer d’en dresser l’inventaire. Mais c’est ici que les difficultés commencent, car nous devons nous rendre compte que les cultures humaines ne diffèrent pas entre elles de la même façon, ni sur le même plan. Nous sommes d’abord en présence de sociétés juxtaposées dans l’espace, les unes proches, les autres lointaines, mais, à tout prendre, contemporaines. Ensuite nous devons compter avec des formes de la vie sociale qui se sont succédé dans le temps et que nous sommes empêchés de connaître par expérience directe. Tout homme peut se transformer en ethnographe et aller partager sur place l’existence d’une société qui l’intéresse ; par contre, même s’il devient historien ou archéologue, il n’entrera jamais directement en contact avec une civilisation disparue, mais seulement à travers les documents écrits ou les monuments figurés que cette société – ou d’autres – auront laissés à son sujet. Enfin, il ne faut pas oublier que les sociétés contemporaines restées ignorantes de l’écriture, comme celles que nous appelons « sauvages » ou « primitives », furent, elles aussi, précédées par d’autres formes, dont la connaissance est pratiquement impossible, fût-ce de manière indirecte ; un inventaire consciencieux se doit de leur réserver des cases blanches, sans doute en nombre infiniment plus élevé que celui des cases où nous nous sentons capables d’inscrire quelque chose. Une première constatation s’impose : la diversité des cultures humaines est, en fait dans le présent, en fait et aussi en droit dans le passé, beaucoup plus grande et plus riche que tout ce que nous sommes destinés à en connaître jamais.\par
Mais, même pénétrés d’un sentiment d’humilité et convaincus de cette limitation, nous rencontrons d’autres problèmes. Que faut-il entendre par cultures différentes ? Certaines semblent l’être, mais si elles émergent d’un tronc commun elles ne diffèrent pas au même titre que deux sociétés qui à aucun moment de leur développement n’ont entretenu de rapports. Ainsi l’ancien empire des Incas au Pérou et celui du Dahomey en Afrique diffèrent entre eux de façon plus absolue que, disons, l’Angleterre et les États-Unis d’aujourd’hui, bien que ces deux sociétés doivent aussi être traitées comme des sociétés distinctes. Inversement, des sociétés entrées récemment en contact très intime paraissent offrir l’image de la même civilisation alors qu’elles y ont accédé par des chemins différents, que l’on n’a pas le droit de négliger. Il y a simultanément à l’œuvre, dans les sociétés humaines, des forces travaillant dans des directions opposées : les unes tendant au maintien et même à l’accentuation des particularismes ; les autres agissant dans le sens de la convergence et de l’affinité. L’étude du langage offre des exemples frappants de tels phénomènes : ainsi, en même temps que des langues de même origine ont tendance à se différencier les unes par rapport aux autres (tels : le russe, le français et l’anglais), des langues d’origines variées, mais parlées dans des territoires contigus, développent des caractères communs : par exemple, le russe s’est, à certains égards, différencié d’autres langues slaves pour se rapprocher, au moins par certains traits phonétiques, des langues finno-ougriennes et turques parlées dans son voisinage géographique immédiat.\par
Quand on étudie de tels faits – et d’autres domaines de la civilisation, comme les institutions sociales, l’art, la religion, en fourniraient aisément de semblables – on en vient à se demander si les sociétés humaines ne se définissent pas, eu égard à leurs relations mutuelles, par un certain \emph{optimum} de diversité au-delà duquel elles ne sauraient aller, mais en dessous duquel elles ne peuvent, non plus, descendre sans danger. Cet optimum varierait en fonction du nombre des sociétés, de leur importance numérique, de leur éloignement géographique et des moyens de communication (matériels et intellectuels) dont elles disposent. En effet, le problème de la diversité ne se pose pas seulement à propos des cultures envisagées dans leurs rapports réciproques ; il existe aussi au sein de chaque société, dans tous les groupes qui la constituent : castes, classes, milieux professionnels ou confessionnels, etc., développent certaines différences auxquelles chacun d’eux attache une extrême importance. On peut se demander si cette \emph{diversification interne} ne tend pas à s’accroître lorsque la société devient, sous d’autres rapports, plus volumineuse et plus homogène ; tel fut, peut-être, le cas de l’Inde ancienne, avec son système de castes s’épanouissant à la suite de l’établissement de l’hégémonie aryenne.\par
On voit donc que la notion de la diversité des cultures humaines ne doit pas être conçue d’une manière statique. Cette diversité n’est pas celle d’un échantillonnage inerte ou d’un catalogue desséché. Sans doute les hommes ont-ils élaboré des cultures différentes en raison de l’éloignement géographique, des propriétés particulières du milieu et de l’ignorance où ils étaient du reste de l’humanité ; mais cela ne serait rigoureusement vrai que si chaque culture ou chaque société était née et s’était développée dans l’isolement de toutes les autres. Or cela n’est jamais le cas, sauf peut-être dans des exemples exceptionnels comme celui des Tasmaniens (et là encore, pour une période limitée). Les sociétés humaines ne sont jamais seules ; quand elles semblent le plus séparées, c’est encore sous forme de groupes ou de paquets. Ainsi, il n’est pas exagéré de supposer que les cultures nord-américaines et sud-américaines ont été coupées de presque tout contact avec le reste du monde pendant une période dont la durée se situe entre dix mille et vingt-cinq mille années. Mais ce gros fragment d’humanité détachée consistait en une multitude de sociétés, grandes et petites, qui avaient entre elles des contacts fort étroits. Et, à côté des différences dues à l’isolement, il y a celles, tout aussi importantes, dues à la proximité : désir de s’opposer, de se distinguer, d’être soi. Beaucoup de coutumes sont nées, non de quelque nécessité interne ou accident favorable, mais de la seule volonté de ne pas demeurer en reste par rapport à un groupe voisin qui soumettait à un usage précis un domaine où l’on n’avait pas songé soi-même à édicter des règles. Par conséquent, la diversité des cultures humaines ne doit pas nous inviter à une observation morcelante ou morcelée. Elle est moins fonction de l’isolement des groupes que des relations qui les unissent.

\section[{L’ethnocentrisme}]{L’ethnocentrisme}
\renewcommand{\leftmark}{L’ethnocentrisme}

\noindent Et pourtant, il semble que la diversité des cultures soit rarement apparue aux hommes pour ce qu’elle est : un phénomène naturel, résultant des rapports directs ou indirects entre les sociétés ; ils y ont plutôt vu une sorte de monstruosité ou de scandale ; dans ces matières, le progrès de la connaissance n’a pas tellement consisté à dissiper cette illusion au profit d’une vue plus exacte qu’à l’accepter ou à trouver le moyen de s’y résigner.\par
L’attitude la plus ancienne, et qui repose sans doute sur des fondements psychologiques solides puisqu’elle tend à réapparaître chez chacun de nous quand nous sommes placés dans une situation inattendue, consiste à répudier purement et simplement les formes culturelles : morales, religieuses, sociales, esthétiques, qui sont les plus éloignées de celles auxquelles nous nous identifions. « Habitudes de sauvages », « cela n’est pas de chez nous », « on ne devrait pas permettre cela », etc., autant de réactions grossières qui traduisent ce même frisson, cette même répulsion, en présence de manières de vivre, de croire ou de penser qui nous sont étrangères. Ainsi l’Antiquité confondait-elle tout ce qui ne participait pas de la culture grecque (puis gréco-romaine) sous le même nom de barbare ; la civilisation occidentale a ensuite utilisé le terme de sauvage dans le même sens. Or derrière ces épithètes se dissimule un même jugement : il est probable que le mot barbare se réfère étymologiquement à la confusion et à l’inarticulation du chant des oiseaux, opposées à la valeur signifiante du langage humain ; et sauvage, qui veut dire « de la forêt », évoque aussi un genre de vie animal, par opposition à la culture humaine. Dans les deux cas, on refuse d’admettre le fait même de la diversité culturelle ; on préfère rejeter hors de la culture, dans la nature, tout ce qui ne se conforme pas à la norme sous laquelle on vit.\par
Ce point de vue naïf, mais profondément ancré chez la plupart des hommes, n’a pas besoin d’être discuté puisque cette brochure – avec toutes celles de la même collection – en constitue précisément la réfutation. Il suffira de remarquer ici qu’il recèle un paradoxe assez significatif. Cette attitude de pensée, au nom de laquelle on rejette les « sauvages » (ou tous ceux qu’on choisit de considérer comme tels) hors de l’humanité, est justement l’attitude la plus marquante et la plus distinctive de ces sauvages mêmes. On sait, en effet, que la notion d’humanité, englobant, sans distinction de race ou de civilisation, toutes les formes de l’espèce humaine, est d’apparition fort tardive et d’expansion limitée. Là même où elle semble avoir atteint son plus haut développement, il n’est nullement certain — l’histoire récente le prouve — qu’elle soit établie à l’abri des équivoques ou des régressions. Mais, pour de vastes fractions de l’espèce humaine et pendant des dizaines de millénaires, cette notion paraît être totalement absente. L’humanité cesse aux frontières de la tribu, du groupe linguistique, parfois même du village ; à tel point qu’un grand nombre de populations dites primitives se désignent d’un nom qui signifie les « hommes » (ou parfois – dirons-nous avec plus de discrétion – les « bons », les « excellents », les « complets »), impliquant ainsi que les autres tribus, groupes ou villages ne participent pas des vertus – ou même de la nature – humaines, mais sont tout au plus composés de « mauvais », de « méchants », de « singes de terre » ou d’« œufs de pou ». On va souvent jusqu’à priver l’étranger de ce dernier degré de réalité en en faisant un « fantôme » ou une « apparition ». Ainsi se réalisent de curieuses situations où deux interlocuteurs se donnent cruellement la réplique. Dans les Grandes Antilles, quelques années après la découverte de l’Amérique, pendant que les Espagnols envoyaient des commissions d’enquête pour rechercher si les indigènes possédaient ou non une âme, ces derniers s’employaient à immerger des Blancs prisonniers afin de vérifier par une surveillance prolongée si leur cadavre était, ou non, sujet à la putréfaction.\par
Cette anecdote à la fois baroque et tragique illustre bien le paradoxe du relativisme culturel (que nous retrouverons ailleurs sous d’autres formes) : c’est dans la mesure même où l’on prétend établir une discrimination entre les cultures et les coutumes que l’on s’identifie le plus complètement avec celles qu’on essaye de nier. En refusant l’humanité à ceux qui apparaissent comme les plus « sauvages » ou « barbares » de ses représentants, on ne fait que leur emprunter une de leurs attitudes typiques. Le barbare, c’est d’abord l’homme qui croit à la barbarie.\par
Sans doute les grands systèmes philosophiques et religieux de l’humanité – qu’il s’agisse du bouddhisme, du christianisme ou de l’islam, des doctrines stoïcienne, kantienne ou marxiste — se sont-ils constamment élevés contre cette aberration. Mais la simple proclamation de l’égalité naturelle entre tous les hommes et de la fraternité qui doit les unir, sans distinction de races ou de cultures, a quelque chose de décevant pour l’esprit, parce qu’elle néglige une diversité de fait, qui s’impose à l’observation, et dont il ne suffit pas de dire qu’elle n’affecte pas le fond du problème pour que l’on soit théoriquement et pratiquement autorisé à faire comme si elle n’existait pas. Ainsi le préambule à la seconde déclaration de l’Unesco sur le problème des races remarque judicieusement que, ce qui convainc l’homme de la rue que les races existent, c’est l’« évidence immédiate de ses sens quand il aperçoit ensemble un Africain, un Européen, un Asiatique et un Indien américain ».\par
Les grandes déclarations des droits de l’homme ont, elles aussi, cette force et cette faiblesse d’énoncer un idéal trop souvent oublieux du fait que l’homme ne réalise pas sa nature dans une humanité abstraite, mais dans des cultures traditionnelles où les changements les plus révolutionnaires laissent subsister des pans entiers, et s’expliquent eux-mêmes en fonction d’une situation strictement définie dans le temps et dans l’espace. Pris entre la double tentation de condamner des expériences qui le heurtent affectivement, et de nier des différences qu’il ne comprend pas intellectuellement, l’homme moderne s’est livré à cent spéculations philosophiques et sociologiques pour établir de vains compromis entre ces pôles contradictoires, et rendre compte de la diversité des cultures tout en cherchant à supprimer ce qu’elle conserve pour lui de scandaleux et de choquant.\par
Mais, si différentes et parfois si bizarres qu’elles puissent être, toutes ces spéculations se ramènent en fait à une seule recette, que le terme de \emph{faux évolutionnisme} est sans doute le mieux apte à caractériser. En quoi consiste-t-elle ? Très exactement, il s’agit d’une tentative pour supprimer la diversité des cultures tout en feignant de la reconnaître pleinement. Car, si l’on traite les différents états où se trouvent les sociétés humaines, tant anciennes que lointaines, comme des \emph{stades} ou des \emph{étapes} d’un développement unique qui, partant du même point, doit les faire converger vers le même but, on voit bien que la diversité n’est plus qu’apparente. L’humanité devient une et identique à elle-même ; seulement, cette unité et cette identité ne peuvent se réaliser que progressivement et la variété des cultures illustre les moments d’un processus qui dissimule une réalité plus profonde ou en retarde la manifestation.\par
Cette définition peut paraître sommaire quand on a présent à l’esprit les immenses conquêtes du darwinisme. Mais celui-ci n’est pas en cause, car l’évolutionnisme biologique et le pseudoévolutionnisme que nous avons ici en vue sont deux doctrines très différentes. La première est née comme une vaste hypothèse de travail, fondée sur des observations où la part laissée à l’interprétation est fort petite. Ainsi, les différents types constituant la généalogie du cheval peuvent être rangés dans une série évolutive pour deux raisons : la première est qu’il faut un cheval pour engendrer un cheval ; la seconde, que des couches de terrain superposées, donc historiquement de plus en plus anciennes, contiennent des squelettes qui varient de façon graduelle depuis la forme la plus récente jusqu’à la plus archaïque. Il devient ainsi hautement probable que \emph{Hipparion} soit l’ancêtre réel de \emph{Equus caballus.} Le même raisonnement s’applique sans doute à l’espèce humaine et à ses races. Mais quand on passe des faits biologiques aux faits de culture, les choses se compliquent singulièrement. On peut recueillir dans le sol des objets matériels, et constater que, selon la profondeur des couches géologiques, la forme ou la technique de fabrication d’un certain type d’objet varie progressivement. Et pourtant une hache ne donne pas physiquement naissance à une hache, à la façon d’un animal. Dire, dans ce dernier cas, qu’une hache a évolué à partir d’une autre constitue donc une formule métaphorique et approximative, dépourvue de la rigueur scientifique qui s’attache à l’expression similaire appliquée aux phénomènes biologiques. Ce qui est vrai d’objets matériels dont la présence physique est attestée dans le sol, pour des époques déterminables, l’est plus encore pour les institutions, les croyances, les goûts, dont le passé nous est généralement inconnu. La notion d’évolution biologique correspond à une hypothèse dotée d’un des plus hauts coefficients de probabilité qui puissent se rencontrer dans le domaine des sciences naturelles ; tandis que la notion d’évolution sociale ou culturelle n’apporte, tout au plus, qu’un procédé séduisant, mais dangereusement commode, de présentation des faits.\par
D’ailleurs, cette différence, trop souvent négligée, entre le vrai et le faux évolutionnisme s’explique par leurs dates d’apparition respectives. Sans doute, l’évolutionnisme sociologique devait recevoir une impulsion vigoureuse de la part de l’évolutionnisme biologique ; mais il lui est antérieur dans les faits. Sans remonter jusqu’aux conceptions antiques, reprises par Pascal, assimilant l’humanité à un être vivant qui passe par les stades successifs de l’enfance, de l’adolescence et de la maturité, c’est au XVIII\textsuperscript{e} siècle qu’on voit fleurir les schémas fondamentaux qui seront, par la suite, l’objet de tant de manipulations : les « spirales » de Vico, ces « trois âges » annonçant les « trois états » de Comte, l’« escalier » de Condorcet. Les deux fondateurs de l’évolutionnisme social, Spencer et Tylor, élaborent et publient leur doctrine avant \emph{L’origine des espèces} ou sans avoir lu cet ouvrage. Antérieur à l’évolutionnisme biologique, théorie scientifique, l’évolutionnisme social n’est, trop souvent, que le maquillage faussement scientifique d’un vieux problème philosophique dont il n’est nullement certain que l’observation et l’induction puissent un jour fournir la clef.

\section[{Cultures archaïques et cultures primitives}]{Cultures archaïques et cultures primitives}
\renewcommand{\leftmark}{Cultures archaïques et cultures primitives}

\noindent Nous avons suggéré que chaque société peut, de son propre point de vue, répartir les cultures en trois catégories : celles qui sont ses contemporaines, mais se trouvent situées en un autre lieu du globe ; celles qui se sont manifestées approximativement dans le même espace, mais l’ont précédée dans le temps ; celles, enfin, qui ont existé à la fois dans un temps antérieur au sien et dans un espace différent de celui où elle se place.\par
On a vu que ces trois groupes sont très inégalement connaissables. Dans le cas du dernier, et quand il s’agit de cultures sans écriture, sans architecture et à techniques rudimentaires (comme c’est le cas pour la moitié de la terre habitée et pour 90 à 99 \%, selon les régions, du laps de temps écoulé depuis le début de la civilisation), on peut dire que nous ne pouvons rien en savoir et que tout ce qu’on essaie de se représenter à leur sujet se réduit à des hypothèses gratuites.\par
Par contre, il est extrêmement tentant de chercher à établir, entre les cultures du premier groupe, des relations équivalant à un ordre de succession dans le temps. Comment des sociétés contemporaines, restées ignorantes de l’électricité et de la machine à vapeur, n’évoqueraient-elles pas la phase correspondante du développement de la civilisation occidentale ? Comment ne pas comparer les tribus indigènes, sans écriture et sans métallurgie, mais traçant des figures sur les parois rocheuses et fabriquant des outils de pierre, avec les formes archaïques de cette même civilisation, dont les vestiges trouvés dans les grottes de France et d’Espagne attestent la similarité ? C’est là surtout que le faux évolutionnisme s’est donné libre cours. Et pourtant ce jeu séduisant, auquel nous nous abandonnons presque irrésistiblement chaque fois que nous en avons l’occasion (le voyageur occidental ne se complaît-il pas à retrouver le « moyen âge » en Orient, le « siècle de Louis XIV » dans le Pékin d’avant la première guerre mondiale, l’« âge de la pierre » parmi les indigènes d’Australie ou de Nouvelle – Guinée ?), est extraordinairement pernicieux. Des civilisations disparues, nous ne connaissons que certains aspects, et ceux-ci sont d’autant moins nombreux que la civilisation considérée est plus ancienne, puisque les aspects connus sont ceux-là seuls qui ont pu survivre aux destructions du temps. Le procédé consiste donc à prendre la partie pour le tout, à conclure, du fait que \emph{certains} aspects de deux civilisations (l’une actuelle, l’autre disparue) offrent des ressemblances, à l’analogie de \emph{tous} les aspects. Or non seulement cette façon de raisonner est logiquement insoutenable, mais dans bon nombre de cas elle est démentie par les faits.\par
Jusqu’à une époque relativement récente, les Tasmaniens, les Patagons possédaient des instruments de pierre taillée, et certaines tribus australiennes et américaines en fabriquent encore. Mais l’étude de ces instruments nous aide fort peu à comprendre l’usage des outils de l’époque paléolithique. Comment se servait-on des fameux « coups-de-poing » dont l’utilisation devait pourtant être si précise que leur forme et leur technique de fabrication sont restées standardisées de façon rigide pendant cent ou deux cent mille années, et sur un territoire s’étendant de l’Angleterre à l’Afrique du Sud, de la France à la Chine ? À quoi servaient les extraordinaires pièces levalloisiennes, triangulaires et aplaties, qu’on trouve par centaines dans les gisements et dont aucune hypothèse ne parvient à rendre compte ? Qu’étaient les prétendus « bâtons de commandement » en os de renne ? Quelle pouvait être la technologie des cultures tardenoisiennes qui ont abandonné derrière elles un nombre incroyable de minuscules morceaux de pierre taillée, aux formes géométriques infiniment diversifiées, mais fort peu d’outils à l’échelle de la main humaine ? Toutes ces incertitudes montrent qu’entre les sociétés paléolithiques et certaines sociétés indigènes contemporaines existe sans doute une ressemblance : elles se sont servies d’un outillage de pierre taillée. Mais, même sur le plan de la technologie, il est difficile d’aller plus loin : la mise en œuvre du matériau, les types d’instruments, donc leur destination, étaient différentes et les uns nous apprennent peu sur les autres à ce sujet. Comment donc pourraient-ils nous instruire sur le langage, les institutions sociales ou les croyances religieuses ?\par
Une des interprétations les plus populaires, parmi celles qu’inspire l’évolutionnisme culturel, traite les peintures rupestres que nous ont laissées les sociétés du paléolithique moyen comme des figurations magiques liées à des rites de chasse. La marche du raisonnement est la suivante : les populations primitives actuelles ont des rites de chasse, qui nous apparaissent souvent dépourvus de valeur utilitaire ; les peintures rupestres préhistoriques, tant par leur nombre que par leur situation au plus profond des grottes, nous semblent sans valeur utilitaire ; leurs auteurs étaient des chasseurs : donc elles servaient à des rites de chasse. Il suffit d’énoncer cette argumentation implicite pour en apprécier l’inconséquence. De reste, c’est surtout parmi les non-spécialistes qu’elle a cours, car les ethnographes, qui ont, eux, l’expérience de ces populations primitives si volontiers mises a à toutes les sauces » par un cannibalisme pseudo – scientifique peu respectueux de l’intégrité des cultures humaines, sont d’accord pour dire que rien, dans les faits observés, ne permet de formuler une hypothèse quelconque sur les documents en question. Et puisque nous parlons ici des peintures rupestres, nous soulignerons qu’à l’exception des peintures rupestres sud-africaines (que certains considèrent comme l’œuvre d’indigènes récents), les arts « primitifs » sont aussi éloignés de l’art magdalénien et aurignacien que de l’art européen contemporain. Car ces arts se caractérisent par un très haut degré de stylisation allant jusqu’aux plus extrêmes déformations, tandis que l’art préhistorique offre un saisissant réalisme. On pourrait être tenté de voir dans ce dernier trait l’origine de l’art européen ; mais cela même serait inexact, puisque, sur le même territoire, l’art paléolithique a été suivi par d’autres formes qui n’avaient pas le même caractère ; la continuité de l’emplacement géographique ne change rien au fait que, sur le même sol, se sont succédé des populations différentes, ignorantes ou insouciantes de l’œuvre de leurs devanciers et apportant chacune avec elle des croyances, des techniques et des styles opposés.\par
Par l’état de ses civilisations, l’Amérique précolombienne, à la veille de la découverte, évoque la période néolithique européenne. Mais cette assimilation ne résiste pas davantage à l’examen : en Europe, l’agriculture et la domestication des animaux vont de pair, tandis qu’en Amérique un développement exceptionnellement poussé de la première s’accompagne d’une presque complète ignorance (ou, en tout cas, d’une extrême limitation) de la seconde. En Amérique, l’outillage lithique se perpétue dans une économie agricole qui, en Europe, est associée au début de la métallurgie.\par
Il est inutile de multiplier les exemples. Car les tentatives faites pour connaître la richesse et l’originalité des cultures humaines, et pour les réduire à l’état de répliques inégalement arriérées de la civilisation occidentale, se heurtent à une autre difficulté, qui est beaucoup plus profonde : en gros (et exception faite de l’Amérique, sur laquelle nous allons revenir), toutes les sociétés humaines ont derrière elles un passé qui est approximativement du même ordre de grandeur. Pour traiter certaines sociétés comme des « étapes » du développement de certaines autres, il faudrait admettre qu’alors que, pour ces dernières, il se passait quelque chose, pour celles-là il ne se passait rien – ou fort peu de choses. Et en effet, on parle volontiers des « peuples sans histoire » (pour dire parfois que ce sont les plus heureux). Cette formule elliptique signifie seulement que leur histoire est et restera inconnue, mais non qu’elle n’existe pas. Pendant des dizaines et même des centaines de millénaires, là-bas aussi, il y a eu des hommes qui ont aimé, haï, souffert, inventé, combattu. En vérité, il n’existe pas de peuples enfants ; tous sont adultes, même ceux qui n’ont pas tenu le journal de leur enfance et de leur adolescence.\par
On pourrait sans doute dire que les sociétés humaines ont inégalement utilisé un temps passé qui, pour certaines, aurait même été du temps perdu ; que les unes mettaient les bouchées doubles tandis que les autres musaient le long du chemin. On en viendrait ainsi à distinguer entre deux sortes d’histoires : une histoire progressive, acquisitive, qui accumule les trouvailles et les inventions pour construire de grandes civilisations, et une autre histoire, peut-être également active et mettant en œuvre autant de talents, mais où manquerait le don synthétique qui est le privilège de la première. Chaque innovation, au lieu de venir s’ajouter à des innovations antérieures et orientées dans le même sens, s’y dissoudrait dans une sorte de flux ondulant qui ne parviendrait jamais à s’écarter durablement de la direction primitive.\par
Cette conception nous paraît beaucoup plus souple et nuancée que les vues simplistes dont on a fait justice aux paragraphes précédents. Nous pourrons lui conserver une place dans notre essai d’interprétation de la diversité des cultures et cela sans faire injustice à aucune. Mais avant d’en venir là, il faut examiner plusieurs questions.

\section[{L’idée de progrès}]{L’idée de progrès}
\renewcommand{\leftmark}{L’idée de progrès}

\noindent Nous devons d’abord considérer les cultures appartenant au second des groupes que nous avons distingués : celles qui ont précédé historiquement la culture – quelle qu’elle soit – au point de vue de laquelle on se place. Leur situation est beaucoup plus compliquée que dans les cas précédemment envisagés. Car l’hypothèse d’une évolution, qui semble si incertaine et si fragile quand on l’utilise pour hiérarchiser des sociétés contemporaines éloignées dans l’espace, paraît ici difficilement contestable, et même directement attestée par les faits. Nous savons, par le témoignage concordant de l’archéologie, de la préhistoire et de la paléontologie, que l’Europe actuelle fut d’abord habitée par des espèces variées du genre \emph{Homo} se servant d’outils de silex grossièrement taillés ; qu’à ces premières cultures en ont succédé d’autres, où la taille de la pierre s’affine, puis s’accompagne du polissage et du travail de l’os et de l’ivoire ; que la poterie, le tissage, l’agriculture, l’élevage font ensuite leur apparition, associés progressivement à la métallurgie, dont nous pouvons aussi distinguer les étapes. Ces formes successives s’ordonnent donc dans le sens d’une évolution et d’un progrès ; les unes sont supérieures et les autres inférieures. Mais, si tout cela est vrai, comment ces distinctions ne réagiraient-elles pas inévitablement sur la façon dont nous traitons des forces contemporaines, mais présentant entre elles des écarts analogues ? Nos conclusions antérieures risquent donc d’être remises en cause par ce nouveau biais.\par
Les progrès accomplis par l’humanité depuis ses origines sont si manifestes et si éclatants que toute tentative pour les discuter se réduirait à un exercice de rhétorique. Et pourtant, il n’est pas si facile qu’on le croit de les ordonner en une série régulière et continue. Il y a quelque cinquante ans, les savants utilisaient, pour se les représenter, des schémas d’une admirable simplicité : âge de la pierre taillée, âge de la pierre polie, âges du cuivre, du bronze, du fer. Tout cela est trop commode. Nous soupçonnons aujourd’hui que le polissage et la taille de la pierre ont parfois existé côte à côte ; quand la seconde technique éclipse complètement la première, ce n’est pas comme le résultat d’un progrès technique spontanément jailli de l’étape antérieure, mais comme une tentative pour copier, en pierre, les armes et les outils de métal que possédaient des civilisations, plus « avancées » sans doute, mais en fait contemporaines de leurs imitateurs. Inversement, la poterie, qu’on croyait solidaire de I’ « âge de la pierre polie », est associée à la taille de la pierre dans certaines régions du nord de l’Europe.\par
Pour ne considérer que la période de la pierre taillée, dite paléolithique, on pensait, il y a quelques années encore, que les différentes formes de cette technique – caractérisant respectivement les industries « à nucléi », les industries « à éclats » et les industries « à lames » – correspondaient à un progrès historique en trois étapes qu’on appelait paléolithique inférieur, paléolithique moyen et paléolithique supérieur. On admet aujourd’hui que ces trois formes ont coexisté, constituant, non des étapes d’un progrès à sens unique, mais des aspects ou, comme on dit, des « faciès » d’une réalité non pas sans doute statique, mais soumise à des variations et transformations fort complexes. En fait, le levalloisien, que nous avons déjà cité et dont la floraison se situe entre le 250\textsuperscript{e} et le 70\textsuperscript{e} millénaire avant l’ère chrétienne, atteint une perfection dans la technique de la taille qui ne devait guère se retrouver qu’a la fin du néolithique, deux cent quarante-cinq à soixante-cinq mille ans plus tard, et que nous serions fort en peine de reproduire aujourd’hui.\par
Tout ce qui est vrai des cultures l’est aussi sur le plan des races, sans qu’on puisse établir (en raison des ordres de grandeur différents) aucune corrélation entre les deux processus : en Europe, l’homme de Néanderthal n’a pas précédé les plus anciennes formes \emph{d’Homo sapiens} ; celles-ci ont été ses contemporaines, peut-être même ses devancières. Et il n’est pas exclu que les types les plus variables d’hominiens aient coexisté dans le temps, sinon dans l’espace : « pygmées » d’Afrique du Sud, « géants » de Chine et d’Indonésie, etc.\par
Encore une fois, tout cela ne vise pas à nier la réalité d’un progrès de l’humanité, mais nous invite à le concevoir avec plus de prudence. Le développement des connaissances préhistoriques et archéologiques tend à \emph{étaler dans l’espace} des formes de civilisation que nous étions portés à imaginer comme \emph{échelonnées dans le temps.} Cela signifie deux choses : d’abord que le « progrès » (si ce terme convient encore pour désigner une réalité très différente de celle à laquelle on l’avait d’abord appliqué) n’est ni nécessaire, ni continu ; il procède par sauts, par bonds, ou, comme diraient les biologistes, par mutations. Ces sauts et ces bonds ne consistent pas à aller toujours plus loin dans la même direction ; ils s’accompagnent de changements d’orientation, un peu à la manière du cavalier des échecs qui a toujours à sa disposition plusieurs progressions mais jamais dans le même sens. L’humanité en progrès ne ressemble guère à un personnage gravissant un escalier, ajoutant par chacun de ses mouvements une marche nouvelle à toutes celles dont la conquête lui est acquise ; elle évoque plutôt le joueur dont la chance est répartie sur plusieurs dés et qui, chaque fois qu’il les jette, les voit s’éparpiller sur le tapis, amenant autant de comptes différents. Ce que l’on gagne sur un, on est toujours exposé à le perdre sur l’autre, et c’est seulement de temps à autre que l’histoire est cumulative, c’est-à-dire que les comptes s’additionnent pour former une combinaison favorable.\par
Que cette histoire cumulative ne soit pas le privilège d’une civilisation ou d’une période de l’histoire, l’exemple de l’Amérique le montre de manière convaincante. Cet immense continent voit arriver l’homme, sans doute par petits groupes de nomades passant le détroit de Behring à la faveur des dernières glaciations, à une date qui ne saurait être fort antérieure au 20\textsuperscript{e} millénaire. En vingt ou vingt-cinq mille ans, ces hommes réussissent une des plus étonnantes démonstrations d’histoire cumulative qui soient au monde : explorant de fond en comble les ressources d’un milieu naturel nouveau, y domestiquant (à côtés de certaines espèces animales) les espèces végétales les plus variées pour leur nourriture, leurs remèdes et leurs poisons, et – fait inégalé ailleurs – promouvant des substances vénéneuses comme le manioc au rôle d’aliment de base, ou d’autres à celui de stimulant ou d’anesthésique ; collectionnant certains poisons ou stupéfiants en fonction des espèces animales sur lesquelles chacun d’eux exerce une action élective ; poussant enfin certaines industries comme le tissage, la céramique et le travail des métaux précieux au plus haut point de perfection. Pour apprécier cette œuvre immense, il suffit de mesurer la contribution de l’Amérique aux civilisations de l’Ancien Monde. En premier lieu, la pomme de terre, le caoutchouc, le tabac et la coca (base de l’anesthésie moderne) qui, à des titres sans doute divers, constituent quatre piliers de la culture occidentale ; le maïs et l’arachide qui devaient révolutionner l’économie africaine avant peut-être de se généraliser dans le régime alimentaire de l’Europe ; ensuite le cacao, la vanille, la tomate, l’ananas, le piment, plusieurs espèces de haricots, de cotons et de cucurbitacées. Enfin le zéro, base de l’arithmétique et, indirectement, des mathématiques modernes, était connu et utilisé par les Mayas au moins un demi-millénaire avant sa découverte par les savants indiens de qui l’Europe l’a reçu par l’intermédiaire des Arabes. Pour cette raison peut-être leur calendrier était, à époque égale, plus exact que celui de l’Ancien Monde. La question de savoir si le régime politique des Incas était socialiste ou totalitaire a déjà fait couler beaucoup d’encre. Il relevait de toute façon des formules les plus modernes et était en avance de plusieurs siècles sur les phénomènes européens du même type. L’attention renouvelée dont le curare a fait récemment l’objet rappellerait, s’il en était besoin, que les connaissances scientifiques des indigènes américains, qui s’appliquent à tant de substances végétales inemployées dans le reste du monde, peuvent encore fournir à celui-ci d’importantes contributions.

\section[{Histoire stationnaire et histoire cumulative}]{Histoire stationnaire et histoire cumulative}
\renewcommand{\leftmark}{Histoire stationnaire et histoire cumulative}

\noindent La discussion de l’exemple américain qui précède doit nous inviter à pousser plus avant notre réflexion sur la différence entre « histoire stationnaire » et « histoire cumulative ». Si nous avons accordé à l’Amérique le privilège de l’histoire cumulative, n’est-ce pas, en effet, seulement parce que nous lui reconnaissons la paternité d’un certain nombre de contributions que nous lui avons empruntées ou qui ressemblent aux nôtres ? Mais quelle serait notre position, en présence d’une civilisation qui se serait attachée à développer des valeurs propres, dont aucune ne serait susceptible d’intéresser la civilisation de l’observateur ? Celui-ci ne serait-il pas porté à qualifier cette civilisation de stationnaire ? En d’autres termes la distinction entre les deux formes d’histoire dépend-elle de la nature intrinsèque des cultures auxquelles on l’applique, ou ne résulte-t-elle pas de la perspective ethnocentrique dans laquelle nous nous plaçons toujours pour évaluer une culture différente ? Nous considérerions ainsi comme cumulative toute culture qui se développerait dans un sens analogue au nôtre, c’est-à-dire dont le développement serait doté pour nous de \emph{signification.} Tandis que les autres cultures nous apparaîtraient comme stationnaires, non pas nécessairement parce qu’elles le sont, mais parce que leur ligne de développement ne signifie rien pour nous, n’est pas mesurable dans les termes du système de référence que nous utilisons.\par
Que tel est bien le cas, cela résulte d’un examen, même sommaire, des conditions dans lesquelles nous appliquons la distinction entre les deux histoires, non pas pour caractériser des sociétés différentes de la nôtre, mais à l’intérieur même de celle – ci. Cette application est plus fréquente qu’on ne croit. Les personnes âgées considèrent généralement comme stationnaire l’histoire qui s’écoule pendant leur vieillesse en opposition avec l’histoire cumulative dont leurs jeunes ans ont été témoins. Une époque dans laquelle elles ne sont plus activement engagées, où elles ne jouent plus de rôle, n’a plus de sens : il ne s’y passe rien, ou ce qui s’y passe n’offre à leurs yeux que des caractères négatifs : tandis que leurs petits-enfants vivent cette période avec toute la ferveur qu’ont oubliée leurs aînés. Les adversaires d’un régime politique ne reconnaissent pas volontiers que celui-ci évolue ; ils le condamnent en bloc, le rejettent hors de l’histoire, comme une sorte de monstrueux entracte à la fin duquel seulement la vie reprendra. Tout autre est la conception des partisans, et d’autant plus, remarquons-le, qu’ils participent étroitement, et à un rang élevé, au fonctionnement de l’appareil. L’historicité, ou, pour parler exactement, l’\emph{événementialité} d’une culture ou d’un processus culturel sont ainsi fonction, non de leurs propriétés intrinsèques, mais de la situation où nous nous trouvons par rapport à eux, du nombre et de la diversité de nos intérêts qui sont gagés sur eux.\par
L’opposition entre cultures progressives et cultures inertes semble ainsi résulter, d’abord, d’une différence de focalisation. Pour l’observateur au microscope, qui s’est « mis au point » sur une certaine distance mesurée à partir de l’objectif, les corps placés en deçà ou au-delà, l’écart serait-il de quelques centièmes de millimètre seulement, apparaissent confus et brouillés, ou même n’apparaissent pas du tout : on voit au travers. Une autre comparaison permettra de déceler la même illusion. C’est celle qu’on emploie pour expliquer les premiers rudiments de la théorie de la relativité. Afin de montrer que la dimension et la vitesse de déplacement des corps ne sont pas des valeurs absolues, mais des fonctions de la position de l’observateur, on rappelle que, pour un voyageur assis à la fenêtre d’un train, la vitesse et la longueur des autres trains varient selon que ceux-ci se déplacent dans le même sens ou dans un sens opposé. Or tout membre d’une culture en est aussi étroitement solidaire que ce voyageur idéal l’est de son train. Car, dès notre naissance, l’entourage fait pénétrer en nous, par mille démarches conscientes et inconscientes, un système complexe de référence consistant en jugements de valeur, motivations, centres d’intérêt, y compris la vue réflexive que l’éducation nous impose du devenir historique de notre civilisation, sans laquelle celle-ci deviendrait impensable, ou apparaîtrait en contradiction avec les conduites réelles. Nous nous déplaçons littéralement avec ce système de références, et les réalités culturelles du dehors ne sont observables qu’à travers les déformations qu’il leur impose, quand il ne va pas jusqu’à nous mettre dans l’impossibilité d’en apercevoir quoi que ce soit.\par
Dans une très large mesure, la distinction entre les « cultures qui bougent » et les « cultures qui ne bougent pas » s’explique par la même différence de position qui fait que, pour notre voyageur, un train en mouvement bouge ou ne bouge pas. Avec, il est vrai, une différence dont l’importance apparaîtra pleinement le jour – dont nous pouvons déjà entrevoir la lointaine venue – où l’on cherchera à formuler une théorie de la relativité généralisée dans un autre sens que celui d’Einstein, nous voulons dire s’appliquant à la fois aux sciences physiques et aux sciences sociales : dans les unes et les autres, tout semble se passer de façon symétrique mais inverse. À l’observateur du monde physique (comme le montre l’exemple du voyageur), ce sont les systèmes évoluant dans le même sens que le sien qui paraissent immobiles, tandis que les plus rapides sont ceux qui évoluent dans des sens différents. C’est le contraire pour les cultures, puisqu’elles nous paraissent d’autant plus actives qu’elles se déplacent dans le sens de la nôtre, et stationnaires quand leur orientation diverge. Mais, dans le cas des sciences de l’homme, le facteur \emph{vitesse} n’a qu’une valeur métaphorique. Pour rendre la comparaison valable, on doit le remplacer par celui d’\emph{information} et de \emph{signification.} Or nous savons qu’il est possible d’accumuler beaucoup plus d’informations sur un train qui se meut parallèlement au nôtre et à une vitesse voisine (ainsi, examiner la tête des voyageurs, les compter, etc.) que sur un train qui nous dépasse ou que nous dépassons à très grande vitesse, ou qui nous paraît d’autant plus court qu’il circule dans une autre direction. À la limite, il passe si vite que nous n’en gardons qu’une impression confuse d’où les signes mêmes de vitesse sont absents ; il se réduit à un brouillage momentané du champ visuel : ce n’est plus un train, il ne \emph{signifie} plus rien. Il y a donc, semble-t-il, une relation entre la notion physique de \emph{mouvement apparent} et une autre notion qui, elle, relève également de la physique, de la psychologie et de la sociologie : celle de \emph{quantité d’information} susceptible de « passer » entre deux individus ou groupes, en fonction de la plus ou moins grande diversité de leurs cultures respectives.\par
Chaque fois que nous sommes portés à qualifier une culture humaine d’inerte ou de stationnaire, nous devons donc nous demander si cet immobilisme apparent ne résulte pas de l’ignorance où nous sommes de ses intérêts véritables, conscients ou inconscients, et si, ayant des critères différents des nôtres, cette culture n’est pas, à notre égard, victime de la même illusion. Autrement dit, nous nous apparaîtrions l’un à l’autre comme dépourvus d’intérêt, tout simplement parce que nous ne nous ressemblons pas.\par
La civilisation occidentale s’est entièrement tournée, depuis deux ou trois siècles, vers la mise à la disposition de l’homme de moyens mécaniques de plus en plus puissants. Si l’on adopte ce critère, on fera de la quantité d’énergie disponible par tête d’habitant l’expression du plus ou moins haut degré de développement des sociétés humaines. La civilisation occidentale, sous sa forme nord-américaine, occupera la place de tête, les sociétés européennes venant ensuite, avec, à la traîne, une masse de sociétés asiatiques et africaines qui deviendront vite indistinctes. Or ces centaines ou même ces milliers de sociétés qu’on appelle « insuffisamment développées » et « primitives », qui se fondent dans un ensemble confus quand on les envisage sous le rapport que nous venons de citer (et qui n’est guère propre à les qualifier, puisque cette ligne de développement leur manque ou occupe chez elles une place très secondaire), elles ne sont tout de même pas identiques. Sous d’autres rapports, elle se placent aux antipodes les unes des autres ; selon le point de vue choisi, on aboutirait donc à des classements différents.\par
Si le critère retenu avait été le degré d’aptitude à triompher des milieux géographiques les plus hostiles, il n’y a guère de doute que les Eskimos d’une part, les Bédouins de l’autre, emporteraient la palme. L’Inde a su, mieux qu’aucune autre civilisation, élaborer un système philosophico-religieux, et la Chine, un genre de vie capable de réduire les conséquences psychologiques d’un déséquilibre démographique. Il y a déjà treize siècles, l’islam a formulé une théorie de la solidarité de toutes les formes de la vie humaine : technique, économique, sociale, spirituelle, que l’Occident ne devait retrouver que tout récemment, avec certains aspects de la pensée marxiste et la naissance de l’ethnologie moderne. On sait quelle place prééminente cette vision prophétique a permis aux Arabes d’occuper dans la vie intellectuelle du moyen âge. L’Occident, maître des machines, témoigne de connaissances très élémentaires sur l’utilisation et les ressources de cette suprême machine qu’est le corps humain. Dans ce domaine au contraire, comme dans celui, connexe, des rapports entre le physique et le moral, l’Orient et l’Extrême-Orient possèdent sur lui une avance de plusieurs millénaires ; ils ont produit ces vastes sommes théoriques et pratiques que sont le yoga de l’Inde, les techniques du souffle chinoises ou la gymnastique viscérale des anciens Maoris. L’agriculture sans terre, depuis peu à l’ordre du jour, a été pratiquée pendant plusieurs siècles par certains peuples polynésiens qui eussent pu aussi enseigner au monde l’art de la navigation, et qui l’ont profondément bouleversé, au XVIII\textsuperscript{e} siècle, en lui révélant un type de vie sociale et morale plus libre et plus généreuse que tout ce que l’on soupçonnait.\par
Pour tout ce qui touche à l’organisation de la famille et à l’harmonisation des rapports entre groupe familial et groupe social, les Australiens, arriérés sur le plan économique, occupent une place si avancée par rapport au reste de l’humanité qu’il est nécessaire, pour comprendre les systèmes de règles élaborés par eux de façon consciente et réfléchie, de faire appel aux formes les plus raffinées des mathématiques modernes. Ce sont eux qui ont vraiment découvert que les liens du mariage forment le canevas sur lequel les autres institutions sociales ne sont que des broderies ; car, même dans les sociétés modernes où le rôle de la famille tend à se restreindre, l’intensité des liens de famille n’est pas moins grande : elle s’amortit seulement dans un cercle plus étroit aux limites duquel d’autres liens, intéressant d’autres familles, viennent aussitôt la relayer. L’articulation des familles au moyen des intermariages peut conduire à la formation de larges charnières entre quelques ensembles, ou de petites charnières entre des ensembles très nombreux ; mais, petites ou grandes, ce sont ces charnières qui maintiennent tout l’édifice social et qui lui dorment sa souplesse. Avec une admirable lucidité, les Australiens ont fait la théorie de ce mécanisme et inventorié les principales méthodes permettant de le réaliser, avec les avantages et les inconvénients qui s’attachent à chacune. Ils ont ainsi dépassé le plan de l’observation empirique pour s’élever à la connaissance des lois mathématiques qui régissent le système. Si bien qu’il n’est nullement exagéré de saluer en eux, non seulement les fondateurs de toute sociologie générale, mais encore les véritables introducteurs de la mesure dans les sciences sociales.\par
La richesse et l’audace de l’invention esthétique des Mélanésiens, leur talent pour intégrer dans la vie sociale les produits les plus obscurs de l’activité inconsciente de l’esprit, constituent un des plus hauts sommets que les hommes aient atteints dans ces directions. La contribution de l’Afrique est plus complexe, mais aussi plus obscure, car c’est seulement à une date récente qu’on a commencé à soupçonner l’importance de son rôle comme \emph{melting pot} culturel de l’Ancien Monde : lieu où toutes les influences sont venues se fondre pour repartir ou se tenir en réserve, mais toujours transformées dans des sens nouveaux. La civilisation égyptienne, dont on connaît l’importance pour l’humanité, n’est intelligible que comme un ouvrage commun de l’Asie et de l’Afrique ; et les grands systèmes politiques de l’Afrique ancienne, ses constructions juridiques, ses doctrines philosophiques longtemps cachées aux Occidentaux, ses arts plastiques et sa musique, qui explorent méthodiquement toutes les possibilités offertes par chaque moyen d’expression, sont autant d’indices d’un passé extraordinairement fertile. Celui-ci est, d’ailleurs, directement attesté par la perfection des anciennes techniques du bronze et de l’ivoire, qui dépassent de loin tout ce que l’Occident pratiquait dans ces domaines à la même époque. Nous avons déjà évoqué la contribution américaine, et il est inutile d’y revenir ici.\par
D’ailleurs, ce ne sont pas tellement ces apports morcelés qui doivent retenir l’attention, car ils risqueraient de nous donner l’idée, doublement fausse, d’une civilisation mondiale composée comme un habit d’Arlequin. On a trop fait état de toutes les priorités : phénicienne pour l’écriture ; chinoise pour le papier, la poudre à canon, la boussole ; indienne pour le verre et l’acier… Ces éléments sont moins importants que la façon dont chaque culture les groupe, les retient ou les exclut. Et ce qui fait l’originalité de chacune d’elles réside plutôt dans sa façon particulière de résoudre des problèmes, de mettre en perspective des valeurs, qui sont approximativement les mêmes pour tous les hommes : car tous les hommes sans exception possèdent un langage, des techniques, un art, des connaissances de type scientifique, des croyances religieuses, une organisation sociale, économique et politique. Or ce dosage n’est jamais exactement le même pour chaque culture, et de plus en plus l’ethnologie moderne s’attache à déceler les origines secrètes de ces options plutôt qu’à dresser un inventaire de traits séparés.

\section[{Place de la civilisation occidentale}]{Place de la civilisation occidentale}
\renewcommand{\leftmark}{Place de la civilisation occidentale}

\noindent Peut-être formulera-t-on des objections contre une telle argumentation à cause de son caractère théorique. Il est possible, dira-t-on, sur le plan d’une logique abstraite, que chaque culture soit incapable de porter un jugement vrai sur une autre, puisqu’une culture ne peut s’évader d’elle-même et que son appréciation reste, par conséquent, prisonnière d’un relativisme sans appel. Mais regardez autour de vous ; soyez attentifs à ce qui se passe dans le monde depuis un siècle, et toutes vos spéculations s’effondreront. Loin de rester enfermées en elles-mêmes, toutes les civilisations reconnaissent, l’une après l’autre, la supériorité de l’une d’entre elles, qui est la civilisation occidentale. Ne voyons-nous pas le monde entier lui emprunter progressivement ses techniques, son genre de vie, ses distractions et jusqu’à ses vêtements ? Comme Diogène prouvait le mouvement en marchant, c’est la marche même des cultures humaines qui, depuis les vastes masses de l’Asie jusqu’aux tribus perdues dans la jungle brésilienne ou africaine, prouve, par une adhésion unanime sans précédent dans l’histoire, qu’une des formes de la civilisation humaine est supérieure à toutes les autres : ce que les pays « insuffisamment développés » reprochent aux autres dans les assemblées internationales n’est pas de les occidentaliser, mais de ne pas leur donner assez vite les moyens de s’occidentaliser.\par
Nous touchons là au point le plus sensible de notre débat ; il ne servirait à rien de vouloir défendre l’originalité des cultures humaines contre elles-mêmes. De plus, il est extrêmement difficile à l’ethnologue d’apporter une juste estimation d’un phénomène comme l’universalisation de la civilisation occidentale, et cela pour plusieurs raisons. D’abord, l’existence d’une civilisation mondiale est un fait probablement unique dans l’histoire ou dont les précédents seraient à chercher dans une préhistoire lointaine, sur laquelle nous ne savons à peu près rien. Ensuite, une grande incertitude règne sur la consistance du phénomène en question. Il est de fait que, depuis un siècle et demi, la civilisation occidentale tend, soit en totalité, soit par certains de ses éléments clefs comme l’industrialisation, à se répandre dans le monde ; et que, dans la mesure où les autres cultures cherchent à préserver quelque chose de leur héritage traditionnel, cette tentative se réduit généralement aux superstructures, c’est-à-dire aux aspects les plus fragiles et dont on peut supposer qu’ils seront balayés par les transformations profondes qui s’accomplissent. Mais le phénomène est en cours, nous n’en connaissons pas encore le résultat. S’achèvera-t-il par une occidentalisation intégrale de la planète avec des variantes, russe ou américaine ? Des formes syncrétiques apparaîtront-elles, comme on en aperçoit la possibilité pour le monde islamique, l’Inde et la Chine ? Ou bien le mouvement de flux touche-t-il déjà à son terme et va-t-il se résorber, le monde occidental étant près de succomber, comme ces monstres préhistoriques, à une expansion physique incompatible avec les mécanismes internes qui assurent son existence ? C’est en tenant compte de toutes ces réserves que nous tâcherons d’évaluer le processus qui se déroule sous nos yeux et dont nous sommes, consciemment ou inconsciemment, les agents, les auxiliaires ou les victimes.\par
On commencera par remarquer que cette adhésion au genre de vie occidental, ou à certains de ses aspects, est loin d’être aussi spontanée que les Occidentaux aimeraient le croire. Elle résulte moins d’une décision libre que d’une absence de choix. La civilisation occidentale a établi ses soldats, ses comptoirs, ses plantations et ses missionnaires dans le monde entier ; elle est, directement ou indirectement, intervenue dans la vie des populations de couleur ; elle a bouleversé de fond en comble leur mode traditionnel d’existence, soit en imposant le sien, soit en instaurant des conditions qui engendraient l’effondrement des cadres existants sans les remplacer par autre chose. Les peuples subjugués ou désorganisés ne pouvaient donc qu’accepter les solutions de remplacement qu’on leur offrait, ou, s’ils n’y étaient pas disposés, espérer s’en rapprocher suffisamment pour être en mesure de les combattre sur le même terrain. En l’absence de cette inégalité dans le rapport des forces, les sociétés ne se livrent pas avec une telle facilité ; leur \emph{Weltanschauung} se rapproche plutôt de celle de ces pauvres tribus du Brésil oriental, où l’ethnographe Curt Nimuendaju avait su se faire adopter, et dont les indigènes, chaque fois qu’il revenait parmi eux après un séjour dans les centres civilisés, sanglotaient de pitié à la pensée des souffrances qu’il devait avoir subies, loin du seul endroit – leur village – où ils jugeaient que la vie valût la peine d’être vécue.\par
Toutefois, en formulant cette réserve, nous n’avons fait que déplacer la question. Si ce n’est pas le consentement qui fonde la supériorité occidentale, n’est-ce pas alors cette plus grande énergie dont elle dispose et qui lui a précisément permis de forcer le consentement ? Nous atteignons ici le roc. Car cette inégalité de force ne relève plus de la subjectivité collective, comme les faits d’adhésion que nous évoquions tout à l’heure. C’est un phénomène objectif que seul l’appel à des causes objectives peut expliquer.\par
Il ne s’agit pas d’entreprendre ici une étude de philosophie des civilisations ; on peut discuter pendant des volumes sur la nature des valeurs professées par la civilisation occidentale. Nous ne relèverons que les plus manifestes, celles qui sont les moins sujettes à la controverse. Elles se ramènent, semble-t-il, à deux : la civilisation occidentale cherche d’une part, selon l’expression de M. Leslie White, à accroître continuellement la quantité d’énergie disponible par tête d’habitant ; d’autre part à protéger et à prolonger la vie humaine, et si l’on veut être bref on considérera que le second aspect est une modalité du premier puisque la quantité d’énergie disponible s’accroît en valeur absolue, avec la durée et l’intérêt de l’existence individuelle. Pour écarter toute discussion, on admettra aussi d’emblée que ces caractères peuvent s’accompagner de phénomènes compensateurs servant, en quelque sorte, de frein : ainsi, les grands massacres que constituent les guerres mondiales, et l’inégalité qui préside à la répartition de l’énergie disponible entre les individus et entre les classes.\par
Cela posé, on constate aussitôt que si la civilisation occidentale s’est, en effet, adonnée à ces tâches avec un exclusivisme où réside peut-être sa faiblesse, elle n’est certainement pas la seule. Toutes les sociétés humaines, depuis les temps les plus reculés, ont agi dans le même sens ; et ce sont des sociétés très lointaines et très archaïques, que nous égalerions volontiers aux peuples « sauvages » d’aujourd’hui, qui ont accompli, dans ce domaine, les progrès les plus décisifs. À l’heure actuelle, ceux-ci constituent toujours la majeure partie de ce que nous nommons civilisation. Nous dépendons encore des immenses découvertes qui ont marqué ce qu’on appelle, sans exagération aucune, la révolution néolithique : l’agriculture, l’élevage, la poterie, le tissage… À tous ces « arts de la civilisation », nous n’avons, depuis huit mille ou dix mille ans, apporté que des perfectionnements.\par
Il est vrai que certains esprits ont une fâcheuse tendance à réserver le privilège de l’effort, de l’intelligence et de l’imagination aux découvertes récentes, tandis que celles qui ont été accomplies par l’humanité dans sa période « barbare » seraient le fait du hasard, et qu’elle n’y aurait, somme toute, que peu de mérite. Cette aberration nous paraît si grave et si répandue, et elle est si profondément de nature à empêcher de prendre une vue exacte du rapport entre les cultures que nous croyons indispensable de la dissiper complètement.

\section[{Hasard et civilisation}]{Hasard et civilisation}
\renewcommand{\leftmark}{Hasard et civilisation}

\noindent On lit dans des traités d’ethnologie – et non des moindres – que l’homme doit la connaissance du feu au hasard de la foudre ou d’un incendie de brousse ; que la trouvaille d’un gibier accidentellement rôti dans ces conditions lui a révélé la cuisson des aliments ; que l’invention de la poterie résulte de l’oubli d’une boulette d’argile au voisinage d’un foyer. On dirait que l’homme aurait d’abord vécu dans une sorte d’âge d’or technologique, où les inventions se cueillaient avec la même facilité que les fruits et les fleurs. À l’homme moderne seraient réservées les fatigues du labeur et les illuminations du génie.\par
Cette vue naïve résulte d’une totale ignorance de la complexité et de la diversité des opérations impliquées dans les techniques les plus élémentaires. Pour fabriquer un outil de pierre taillée efficace, il ne suffit pas de frapper sur un caillou jusqu’à ce qu’il éclate : on s’en est bien aperçu le jour où l’on a essayé de reproduire les principaux types d’outils préhistoriques. Alors – et aussi en observant la même technique chez les indigènes qui la possèdent encore – on a découvert la complication des procédés indispensables et qui vont, quelquefois, jusqu’à la fabrication préliminaire de véritables « appareils à tailler » : marteaux à contrepoids pour contrôler l’impact et sa direction ; dispositifs amortisseurs pour éviter que la vibration ne rompe l’éclat. Il faut aussi un vaste ensemble de notions sur l’origine locale, les procédés d’extraction, la résistance et la structure des matériaux utilisés, un entraînement musculaire approprié, la connaissance des « tours de main », etc. ; en un mot, une véritable « liturgie » correspondant, \emph{mutatis mutandis}, aux divers chapitres de la métallurgie.\par
De même, des incendies naturels peuvent parfois griller ou rôtir ; mais il est très difficilement concevable (hors le cas des phénomènes volcaniques dont la distribution géographique est restreinte) qu’ils fassent bouillir ou cuire à la vapeur. Or ces méthodes de cuisson ne sont pas moins universelles que les autres. Donc on n’a pas de raison d’exclure l’acte inventif, qui a certainement été requis pour les dernières méthodes, quand on veut expliquer les premières.\par
La poterie offre un excellent exemple parce qu’une croyance très répandue veut qu’il n’y ait rien de plus simple que de creuser une motte d’argile et la durcir au feu. Qu’on essaye. Il faut d’abord découvrir des argiles propres à la cuisson ; or, si un grand nombre de conditions naturelles sont nécessaires à cet effet, aucune n’est suffisante, car aucune argile non mêlée à un corps inerte, choisi en fonction de ses caractéristiques particulières, ne donnerait après cuisson un récipient utilisable. Il faut élaborer les techniques du modelage qui permettent de réaliser ce tour de force de maintenir en équilibre pendant un temps appréciable, et de modifier en même temps, un corps plastique qui ne « tient » pas ; il faut enfin découvrir le combustible particulier, la forme du foyer, le type de chaleur et la durée de la cuisson, qui permettront de le rendre solide et imperméable, à travers tous les écueils des craquements, effritements et déformations. On pourrait multiplier les exemples.\par
Toutes ces opérations sont beaucoup trop nombreuses et trop complexes pour que le hasard puisse en rendre compte. Chacune d’elles, prise isolément, ne signifie rien, et c’est leur combinaison imaginée, voulue, cherchée et expérimentée qui seule permet la réussite. Le hasard existe sans doute, mais ne donne par lui-même aucun résultat. Pendant deux mille cinq cents ans environ, le monde occidental a connu l’existence de l’électricité – découverte sans doute par hasard – mais ce hasard devait rester stérile jusqu’aux efforts intentionnels et dirigés par des hypothèses des Ampère et des Faraday. Le hasard n’a pas joué un plus grand rôle dans l’invention de l’arc, du boomerang ou de la sarbacane, dans la naissance de l’agriculture et de l’élevage, que dans la découverte de la pénicilline – dont on sait, au reste, qu’il n’a pas été absent. On doit donc distinguer avec soin la transmission d’une technique d’une génération à une autre, qui se fait toujours avec une aisance relative grâce à l’observation et à l’entraînement quotidien, et la création ou l’amélioration des techniques au sein de chaque génération. Celles-ci supposent toujours la même puissance imaginative et les mêmes efforts acharnés de la part de certains individus, quelle que soit la technique particulière qu’on ait en vue. Les sociétés que nous appelons primitives ne sont pas moins riches en Pasteur et en Palissy que les autres.\par
Nous retrouverons tout à l’heure le hasard et la probabilité, mais à une autre place et avec un autre rôle. Nous ne les utiliserons pas pour expliquer paresseusement la naissance d’inventions toutes faites, mais pour interpréter un phénomène qui se situe à un autre niveau de réalité : à savoir que, malgré une dose d’imagination, d’invention, d’effort créateur dont nous avons tout lieu de supposer qu’elle reste à peu près constante à travers l’histoire de l’humanité, cette combinaison ne détermine des mutations culturelles importantes qu’à certaines périodes et en certains lieux. Car, pour aboutir à ce résultat, les facteurs purement psychologiques ne suffisent pas : ils doivent d’abord se trouver présents, avec une orientation similaire, chez un nombre suffisant d’individus pour que le créateur soit aussitôt assuré d’un public ; et cette condition dépend elle-même de la réunion d’un nombre considérable d’autres facteurs, de nature historique, économique et sociologique. On en arriverait donc, pour expliquer les différences dans le cours des civilisations, à invoquer des ensembles de causes si complexes et si discontinus qu’ils seraient inconnaissables, soit pour des raisons pratiques, soit même pour des raisons théoriques telles que l’apparition, impossible à éviter, de perturbations liées aux techniques d’observation. En effet, pour débrouiller un écheveau formé de fils aussi nombreux et ténus, il ne faudrait pas faire moins que soumettre la société considérée (et aussi le monde qui l’entoure) à une étude ethnographique globale et de tous les instants. Même sans évoquer l’énormité de l’entreprise, on sait que les ethnographes, qui travaillent pourtant à une échelle infiniment plus réduite, sont souvent limités dans leurs observations par les changements subtils que leur seule présence suffit à introduire dans le groupe humain objet de leur étude. Au niveau des sociétés modernes, on sait aussi que les \emph{poils} d’opinion publique, un des moyens les plus efficaces de sondage, modifient l’orientation de cette opinion du fait même de leur emploi, qui met en jeu dans la population un facteur de réflexion sur soi jusqu’alors absent.\par
Cette situation justifie l’introduction dans les sciences sociales de la notion de probabilité, présente depuis longtemps déjà dans certaines branches de la physique, dans la thermodynamique par exemple. Nous y reviendrons ; pour le moment, il suffira de se rappeler que la complexité des découvertes modernes ne résulte pas d’une plus grande fréquence ou d’une meilleure disponibilité du génie chez nos contemporains. Bien au contraire, puisque nous avons reconnu qu’à travers les siècles chaque génération, pour progresser, n’aurait besoin que d’ajouter une épargne constante au capital légué par les générations antérieures. Les neuf dixièmes de notre richesse leur sont dus ; et même davantage, si, comme on s’est amusé à le faire, on évalue la date d’apparition des principales découvertes par rapport à celle, approximative, du début de la civilisation. On constate alors que l’agriculture naît au cours d’une phase récente correspondant à 2 \% de cette durée ; la métallurgie à 0,7 \%, l’alphabet à 0,35 \%, la physique galiléenne à 0,035 \% et le darwinisme à 0,009 \%\footnote{Leslie A. White, \emph{The science of culture}, p. 356, New York, 1949.} (1) La révolution scientifique et industrielle de l’Occident s’inscrit tout entière dans une période égale à un demi-millième environ de la vie écoulée de l’humanité. On peut donc se montrer prudent avant d’affirmer qu’elle est destinée à en changer totalement la signification.\par
Il n’en est pas moins vrai – et c’est l’expression définitive que nous croyons pouvoir donner à notre problème – que, sous le rapport des inventions techniques (et de la réflexion scientifique qui les rend possibles), la civilisation occidentale s’est montrée plus cumulative que les autres ; qu’après avoir disposé du même capital néolithique initial, elle a su apporter des améliorations (écriture alphabétique, arithmétique et géométrie), dont elle a d’ailleurs rapidement oublié certaines ; mais qu’après une stagnation qui, en gros, s’étale sur deux mille ou deux mille cinq cents ans (du 1\textsuperscript{er} millénaire avant l’ère chrétienne jusqu’au XVIII\textsuperscript{e} siècle environ), elle s’est soudainement révélée comme le foyer d’une révolution industrielle dont, par son ampleur, son universalité et l’importance de ses conséquences, la révolution néolithique seule avait offert jadis un équivalent.\par
Deux fois dans son histoire, par conséquent, et à environ dix mille ans d’intervalle, l’humanité a su accumuler une multiplicité d’inventions orientées dans le même sens ; et ce nombre, d’une part, cette continuité, de l’autre, se sont concentrés dans un laps de temps suffisamment court pour que de hautes synthèses techniques s’opèrent ; synthèses qui ont entraîné des changements significatifs dans les rapports que l’homme entretient avec la nature et qui ont, à leur tour, rendu possibles d’autres changements. L’image d’une réaction en chaîne, déclenchée par des corps catalyseurs, permet d’illustrer ce processus qui s’est, jusqu’à présent, répété deux fois, et deux fois seulement, dans l’histoire de l’humanité. Comment cela s’est-il produit ?\par
D’abord il ne faut pas oublier que d’autres révolutions, présentant les mêmes caractères cumulatifs, ont pu se dérouler ailleurs et à d’autres moments, mais dans des domaines différents de l’activité humaine. Nous avons expliqué plus haut pourquoi notre propre révolution industrielle avec la révolution néolithique (qui l’a précédée dans le temps, mais relève des mêmes préoccupations) sont les seules qui peuvent nous apparaître telles, parce que notre système de références permet de les mesurer. Tous les autres changements, qui se sont certainement produits, ne se révèlent que sous forme de fragments, ou profondément déformés. Ils ne peuvent pas \emph{prendre un sens} pour l’homme occidental moderne (en tout cas, pas tout leur sens) ; ils peuvent même être pour lui comme s’ils n’existaient pas.\par
En second lieu, l’exemple de la révolution néolithique (la seule que l’homme occidental moderne parvienne à se représenter assez clairement) doit lui inspirer quelque modestie quant à la prééminence qu’il pourrait être tenté de revendiquer au profit d’une race, d’une région ou d’un pays. La révolution industrielle est née en Europe occidentale ; puis elle est apparue aux États-Unis, ensuite au Japon ; depuis 1917 elle s’accélère en Union soviétique, demain sans doute elle surgira ailleurs ; d’un demi-siècle à l’autre, elle brille d’un feu plus ou moins vif dans tel ou tel de ses centres. Que deviennent, à l’échelle des millénaires, les questions de priorité, dont nous tirons tant de vanité ?\par
À mille ou deux mille ans près, la révolution néolithique s’est déclenchée simultanément dans le bassin égéen, l’Égypte, le Proche-Orient, la vallée de l’Indus et la Chine ; et depuis l’emploi du carbone radio-actif pour la détermination des périodes archéologiques, nous soupçonnons que le néolithique américain, plus ancien qu’on ne le croyait jadis, n’a pas dû débuter beaucoup plus tard que dans l’Ancien Monde. Il est probable que trois ou quatre petites vallées pourraient, dans ce concours, réclamer une priorité de quelques siècles. Qu’en savons-nous aujourd’hui ? Par contre, nous sommes certains que la question de priorité n’a pas d’importance, précisément parce que la simultanéité d’apparition des mêmes bouleversements technologiques (suivis de près par des bouleversements sociaux), sur des territoires aussi vastes et dans des régions aussi écartées, montre bien qu’elle n’a pas dépendu du génie d’une race ou d’une culture, mais de conditions si générales qu’elles se situent en dehors de la conscience des hommes. Soyons donc assurés que, si la révolution industrielle n’était pas apparue d’abord en Europe occidentale et septentrionale, elle se serait manifestée un jour sur un autre point du globe. Et si, comme il est vraisemblable, elle doit s’étendre à l’ensemble de la terre habitée, chaque culture y introduira tant de contributions particulières que l’historien des futurs millénaires considérera légitimement comme futile la question de savoir qui peut, d’un ou de deux siècles, réclamer la priorité pour l’ensemble.\par
Cela posé, il nous faut introduire une nouvelle limitation, sinon à la validité, tout au moins à la rigueur de la distinction entre histoire stationnaire et histoire cumulative. Non seulement cette distinction est relative à nos intérêts, comme nous l’avons déjà montré, mais elle ne réussit jamais à être nette. Dans le cas des inventions techniques, il est bien certain qu’aucune période, aucune culture, n’est absolument stationnaire. Tous les peuples possèdent et transforment, améliorent ou oublient des techniques suffisamment complexes pour leur permettre de dominer leur milieu. Sans quoi ils auraient disparu depuis longtemps. La différence n’est donc jamais entre histoire cumulative et histoire non cumulative ; toute histoire est cumulative, avec des différences de degrés. On sait, par exemple, que les anciens Chinois, les Eskimos avaient poussé très loin les arts mécaniques ; et il s’en est fallu de fort peu qu’ils n’arrivent au point où la « réaction en chaîne » se déclenche, déterminant le passage d’un type de civilisation à un autre. On connaît l’exemple de la poudre à canon : les Chinois avaient résolu, techniquement parlant, tous les problèmes qu’elle posait, sauf celui de son utilisation en vue de résultats massifs. Les anciens Mexicains n’ignoraient pas la roue, comme on le dit souvent ; ils la connaissaient fort bien, pour fabriquer des animaux à roulettes destinés aux enfants ; il leur eût suffi d’une démarche supplémentaire pour posséder le chariot.\par
Dans ces conditions, le problème de la rareté relative (pour chaque système de référence) de cultures « plus cumulatives » par rapport aux cultures « moins cumulatives » se réduit à un problème connu qui relève du calcul des probabilités. C’est le même problème qui consiste à déterminer la probabilité relative d’une combinaison complexe par rapport à d’autres combinaisons du même type, mais de complexité moindre. À la roulette, par exemple, une suite de deux numéros consécutifs (7 et 8, 12 et 13, 30 et 31, par exemple) est assez fréquente ; une de trois numéros est déjà rare, une de quatre l’est beaucoup plus. Et c’est une fois seulement sur un nombre extrêmement élevé de lancers que se réalisera peut-être une série de six, sept ou huit numéros conforme à l’ordre naturel des nombres. Si notre attention est exclusivement fixée sur des séries longues (par exemple, si nous parions sur les séries de cinq numéros consécutifs), les séries plus courtes deviendront pour nous équivalentes à des séries non ordonnées. C’est oublier qu’elles ne se distinguent des nôtres que par la valeur d’une fraction, et qu’envisagées sous un autre angle elles présentent peut-être d’aussi grandes régularités. Poussons encore plus loin notre comparaison. Un joueur, qui transférerait tous ses gains sur des séries de plus en plus longues, pourrait se décourager, après des milliers ou des millions de coups, de ne voir jamais apparaître la série de neuf numéros consécutifs, et penser qu’il eût mieux fait de s’arrêter plus tôt. Pourtant, il n’est pas dit qu’un autre joueur, suivant la même formule de pari, mais sur des séries d’un autre type (par exemple, un certain rythme d’alternance entre rouge et noir, ou entre pair et impair) ne saluerait pas des combinaisons significatives là où le premier joueur n’apercevrait que le désordre. L’humanité n’évolue pas dans un sens unique. Et si, sur un certain plan, elle semble stationnaire ou même régressive, cela ne signifie pas que, d’un autre point de vue, elle n’est pas le siège d’importantes transformations.\par
Le grand philosophe anglais du XVIII\textsuperscript{e} siècle Hume s’est un jour attaché à dissiper le faux problème que se posent beaucoup de gens quand ils se demandent pourquoi toutes les femmes ne sont pas jolies, mais seulement une petite minorité. Il n’a eu nulle peine à montrer que la question n’a aucun sens. Si toutes les femmes étaient au moins aussi jolies que la plus belle, nous les trouverions banales et réserverions notre qualificatif à la petite minorité qui surpasserait le modèle commun. De même, quand nous sommes intéressés à un certain type de progrès, nous en réservons le mérite aux cultures qui le réalisent au plus haut point, et nous restons indifférents devant les autres. Ainsi le progrès n’est jamais que le maximum de progrès dans un sens prédéterminé par le goût de chacun.

\section[{La collaboration des cultures}]{La collaboration des cultures}
\renewcommand{\leftmark}{La collaboration des cultures}

\noindent Il nous faut enfin envisager notre problème sous un dernier aspect. Un joueur comme celui dont il a été question aux paragraphes précédents qui ne parierait jamais que sur les séries les plus longues (de quelque façon qu’il conçoive ces séries) aurait toute chance de se ruiner. Il n’en serait pas de même d’une coalition de parieurs jouant les mêmes séries en valeur absolue, mais sur plusieurs roulettes et en s’accordant le privilège de mettre en commun les résultats favorables aux combinaisons de chacun. Car si, ayant tiré tout seul le 21 et le 22, j’ai besoin du 23 pour continuer ma série, il y a évidemment plus de chances pour qu’il sorte entre dix tables que sur une seule.\par
Or cette situation ressemble beaucoup à celle des cultures qui sont parvenues à réaliser les formes d’histoire les plus cumulatives. Ces formes extrêmes n’ont jamais été le fait de cultures isolées, mais bien de cultures combinant, volontairement ou involontairement, leurs jeux respectifs, et réalisant par des moyens variés (migrations, emprunts, échanges commerciaux, guerres) ces \emph{coalitions} dont nous venons d’imaginer le modèle. Et c’est ici que nous touchons du doigt l’absurdité qu’il y a à déclarer une culture supérieure à une autre. Car, dans la mesure où elle serait seule, une culture ne pourrait jamais être « supérieure » ; comme le joueur isolé, elle ne réussirait jamais que des petites séries de quelques éléments, et la probabilité pour qu’une série longue « sorte » dans son histoire (sans être théoriquement exclue) serait si faible qu’il faudrait disposer d’un temps infiniment plus long que celui dans lequel s’inscrit le développement total de l’humanité pour espérer la voir se réaliser. Mais – nous l’avons dit plus haut – aucune culture n’est seule ; elle est toujours donnée en coalition avec d’autres cultures, et c’est cela qui lui permet d’édifier des séries cumulatives. La probabilité pour que, parmi ces séries, en apparaisse une longue dépend naturellement de l’étendue, de la durée et de la variabilité du régime de coalition.\par
De ces remarques découlent deux conséquences.\par
Au cours de cette étude, nous nous somme demandé à plusieurs reprises comment il se faisait que l’humanité soit restée stationnaire pendant les neuf dixièmes de son histoire, et même davantage : les premières civilisations sont vieilles de deux cent mille à cinq cent mille années, les conditions de vie se transforment seulement au cours des derniers dix mille ans. Si notre analyse est exacte, ce n’est pas parce que l’homme paléolithique était moins intelligent, moins doué que son successeur néolithique, c’est tout simplement parce que, dans l’histoire humaine, une combinaison de degré \emph{n} a mis un temps de durée \emph{t} à sortir ; elle aurait pu se produire beaucoup plus tôt, ou beaucoup plus tard. Le fait n’a pas plus de signification que n’en a ce nombre de coups qu’un joueur doit attendre pour voir une combinaison donnée se produire : cette combinaison pourra se produire au premier coup, au millième, au millionième, ou jamais. Mais pendant tout ce temps l’humanité, comme le joueur, n’arrête pas de spéculer. Sans toujours le vouloir, et sans jamais exactement s’en rendre compte, elle « monte des affaires » culturelles, se lance dans des « opérations civilisation », dont chacune est couronnée d’un inégal succès. Tantôt elle frôle la réussite, tantôt elle compromet les acquisitions antérieures. Les grandes simplifications qu’autorise notre ignorance de la plupart des aspects des sociétés préhistoriques permettent d’illustrer cette marche incertaine et ramifiée, car rien n’est plus frappant que ces repentirs qui conduisent de l’apogée levalloisien à la médiocrité moustérienne, des splendeurs aurignacienne et solutréenne à la rudesse du magdalénien, puis aux contrastes extrêmes offerts par les divers aspects du mésolithique.\par
Ce qui est vrai dans le temps ne l’est pas moins dans l’espace, mais doit s’exprimer d’une autre façon. La chance qu’a une culture de totaliser cet ensemble complexe d’inventions de tous ordres que nous appelons une civilisation est fonction du nombre et de la diversité des cultures avec lesquelles elle participe à l’élaboration – le plus souvent involontaire – d’une commune stratégie. Nombre et diversité, disons-nous. La comparaison entre l’Ancien Monde et le Nouveau à la veille de la découverte illustre bien cette double nécessité.\par
L’Europe du début de la Renaissance était le lieu de rencontre et de fusion des influences les plus diverses : les traditions grecque, romaine, germanique et anglo-saxonne ; les influences arabe et chinoise. L’Amérique précolombienne ne jouissait pas, quantitativement parlant, de moins de contacts culturels puisque les cultures américaines entretenaient des rapports, et que les deux Amériques forment ensemble un vaste hémisphère. Mais, tandis que les cultures qui se fécondent mutuellement sur le sol européen sont le produit d’une différenciation vieille de plusieurs dizaines de millénaires, celles de l’Amérique, dont le peuplement est plus récent, ont eu moins de temps pour diverger ; elles offrent un tableau relativement plus homogène. Aussi, bien qu’on ne puisse pas dire que le niveau culturel du Mexique ou du Pérou fût, au moment de la découverte, inférieur à celui de l’Europe (nous avons même vu qu’à certains égards il lui était supérieur), les divers aspects de la culture y étaient peut-être moins bien articulés. À côté d’étonnantes réussites, les civilisations précolombiennes sont pleines de lacunes, elles ont, si l’on peut dire, des « trous ». Elles offrent aussi le spectacle, moins contradictoire qu’il ne semble, de la coexistence de formes précoces et de formes abortives. Leur organisation peu souple et faiblement diversifiée explique vraisemblablement leur effondrement devant une poignée de conquérants. Et la cause profonde peut en être cherchée dans le fait que la « coalition » culturelle américaine était établie entre des partenaires moins différents entre eux que ne l’étaient ceux de l’Ancien Monde.\par
Il n’y a donc pas de société cumulative en soi et par soi. L’histoire cumulative n’est pas la propriété de certaines races ou de certaines cultures qui se distingueraient ainsi des autres. Elle résulte de leur \emph{conduite} plutôt que de leur \emph{nature.} Elle exprime une certaine modalité d’existence des cultures qui n’est autre que leur \emph{manière d’être ensemble.} En ce sens, on peut dire que l’histoire cumulative est la forme d’histoire caractéristique de ces superorganismes sociaux que constituent les groupes de sociétés, tandis que l’histoire stationnaire – si elle existait vraiment – serait la marque de ce genre de vie inférieur qui est celui des sociétés solitaires.\par
L’exclusive fatalité, l’unique tare qui puissent affliger un groupe humain et l’empêcher de réaliser pleinement sa nature, c’est d’être seul.\par
On voit ainsi ce qu’il y a souvent de maladroit et de peu satisfaisant pour l’esprit, dans les tentatives dont on se contente généralement pour justifier la contribution des races et des cultures humaines à la civilisation. On énumère des traits, on épluche des questions d’origine, on décerne des priorités. Pour bien intentionnés qu’ils soient, ces efforts sont futiles, parce qu’ils manquent triplement leur but. D’abord, le mérite d’une invention accordé à telle ou telle culture n’est jamais sûr. Pendant un siècle, on a cru fermement que le maïs avait été créé à partir du croisement d’espèces sauvages par les Indiens d’Amérique, et l’on continue à l’admettre provisoirement, mais non sans un doute croissant, car il se pourrait qu’après tout, le maïs fût venu en Amérique (on ne sait trop quand ni comment) à partir de l’Asie du Sud-Est.\par
En second lieu, les contributions culturelles peuvent toujours se répartir en deux groupes. D’un côté, nous avons des traits, des acquisitions isolées dont l’importance est facile à évaluer, et qui offrent aussi un caractère limité. Que le tabac soit venu d’Amérique est un fait, mais après tout, et malgré toute la bonne volonté déployée à cette fin par les institutions internationales, nous ne pouvons nous sentir fondre de gratitude envers les Indiens américains chaque fois que nous fumons une cigarette. Le tabac est une adjonction exquise à l’art de vivre, comme d’autres sont utiles (ainsi le caoutchouc) ; nous leur devons des plaisirs et des commodités supplémentaires, mais, si elles n’étaient pas là, les racines de notre civilisation n’en seraient pas ébranlées ; et, en cas de pressant besoin, nous aurions su les retrouver ou mettre autre chose à la place.\par
Au pôle opposé (avec, bien entendu, toute une série de formes intermédiaires), il y a les contributions offrant un caractère de système, c’est-à-dire correspondant à la façon propre dont chaque société a choisi d’exprimer et de satisfaire l’ensemble des aspirations humaines. L’originalité et la nature irremplaçables de ces styles de vie ou, comme disent les Anglo-Saxons, de ces \emph{patterns} ne sont pas niables, mais comme ils représentent autant de choix exclusifs on aperçoit mal comment une civilisation pourrait espérer profiter du style de vie d’une autre, à moins de renoncer à être elle-même. En effet, les tentatives de compromis ne sont susceptibles d’aboutir qu’à deux résultats : soit une désorganisation et un effondrement du \emph{pattern} d’un des groupes ; soit une synthèse originale, mais qui, alors, consiste en l’émergence d’un troisième \emph{pattern} lequel devient irréductible par rapport aux deux autres. Le problème n’est d’ailleurs pas même de savoir si une société peut ou non tirer profit du style de vie de ses voisines, mais si, et dans quelle mesure, elle peut arriver à les comprendre, et même à les connaître. Nous avons vu que cette question ne comporte aucune réponse catégorique.\par
Enfin, il n’y a pas de contribution sans bénéficiaire. Mais s’il existe des cultures concrètes, que l’on peut situer dans le temps et dans l’espace, et dont on peut dire qu’elles ont « contribué » et continuent de le faire, qu’est ce que cette « civilisation mondiale » supposée bénéficiaire de toutes ces contributions ? Ce n’est pas une civilisation distincte de toutes les autres, jouissant d’un même coefficient de réalité. Quand nous parlons de civilisation mondiale, nous ne désignons pas une époque ou un groupe d’hommes : nous utilisons une notion abstraite, à laquelle nous prêtons une valeur, soit morale, soit logique : morale, s’il s’agit d’un but que nous proposons aux sociétés existantes ; logique, si nous entendons grouper sous un même vocable les éléments communs que l’analyse permet de dégager entre les différentes cultures. Dans les deux cas, il ne faut pas se dissimuler que la notion de civilisation mondiale est fort pauvre, schématique, et que son contenu intellectuel et affectif n’offre pas une grande densité. Vouloir évaluer des contributions culturelles lourdes d’une histoire millénaire, et de tout le poids des pensées, des souffrances, des désirs et du labeur des hommes qui les ont amenées à l’existence, en les rapportant exclusivement à l’étalon d’une civilisation mondiale qui est encore une forme creuse, serait les appauvrir singulièrement, les vider de leur substance et n’en conserver qu’un corps décharné.\par
Nous avons, au contraire, cherché à montrer que la véritable contribution des cultures ne consiste pas dans la liste de leurs inventions particulières, mais dans \emph{Yécart différentiel} qu’elles offrent entre elles. Le sentiment de gratitude et d’humilité que chaque membre d’une culture donnée peut et doit éprouver envers toutes les autres ne saurait se fonder que sur une seule conviction : c’est que les autres cultures sont différentes de la sienne, de la façon la plus variée ; et cela, même si la nature dernière de ces différences lui échappe ou si, malgré tous ses efforts, il n’arrive que très imparfaitement à la pénétrer.\par
D’autre part, nous avons considéré la notion de civilisation mondiale comme une sorte de concept limite, ou comme une manière abrégée de désigner un processus complexe. Car si notre démonstration est valable, il n’y a pas, il ne peut y avoir, une civilisation mondiale au sens absolu que l’on donne souvent à ce terme, puisque la civilisation implique la coexistence de cultures offrant entre elles le maximum de diversité, et consiste même en cette coexistance. La civilisation mondiale ne saurait être autre chose que la coalition, à l’échelle mondiale, de cultures préservant chacune son originalité.

\section[{Le double sens du progrès}]{Le double sens du progrès}
\renewcommand{\leftmark}{Le double sens du progrès}

\noindent Ne nous trouvons-nous pas alors devant un étrange paradoxe ? En prenant les termes dans le sens que nous leur avons donné, on a vu que tout \emph{progrès} culturel est fonction d’une \emph{coalition} entre les cultures. Cette coalition consiste dans la mise en commun (consciente ou inconsciente, volontaire ou involontaire, intentionnelle ou accidentelle, cherchée ou contrainte) des \emph{chances} que chaque culture rencontre dans son développement historique ; enfin, nous avons admis que cette coalition était d’autant plus féconde qu’elle s’établissait entre des cultures plus diversifiées. Cela posé, il semble bien que nous nous trouvions en face de conditions contradictoires. Car ce \emph{jeu en commun} dont résulte tout progrès doit entraîner comme conséquence, à échéance plus ou moins brève, une \emph{homogénéisation} des ressources de chaque joueur. Et si la diversité est une condition initiale, il faut reconnaître que les chances de gain deviennent d’autant plus faibles que la partie doit se prolonger.\par
À cette conséquence inéluctable il n’existe, semble-t-il, que deux remèdes. L’un consiste, pour chaque joueur, à provoquer dans son jeu des \emph{écarts différentiels} ; la chose est possible puisque chaque société (le « joueur » de notre modèle théorique) se compose d’une coalition de groupes : confessionnels, professionnels et économiques, et que la mise sociale est faite des mises de tous ces constituants. Les inégalités sociales sont l’exemple le plus frappant de cette solution. Les grandes révolutions que nous avons choisies comme illustration : néolithique et industrielle, se sont accompagnées, non seulement d’une diversification du corps social comme l’avait bien vu Spencer, mais aussi de l’instauration de statuts différentiels entre les groupes, surtout au point de vue économique. On a remarqué depuis longtemps que les découvertes néolithiques avaient rapidement entraîné une différenciation sociale, avec la naissance dans l’Orient ancien des grandes concentrations urbaines, l’apparition des États, des castes et des classes. La même observation s’applique à la révolution industrielle, conditionnée par l’apparition d’un prolétariat et aboutissant à des formes nouvelles, et plus poussées, d’exploitation du travail humain. Jusqu’à présent, on avait tendance à traiter ces transformations sociales comme la conséquence des transformations techniques, à établir entre celles-ci et celles-là un rapport de cause à effet. Si notre interprétation est exacte, la relation de causalité (avec la succession temporelle qu’elle implique) doit être abandonnée – comme la science moderne tend d’ailleurs généralement à le faire – au profit d’une corrélation fonctionnelle entre les deux phénomènes. Remarquons au passage que la reconnaissance du fait que le progrès technique ait eu, pour corrélatif historique, le développement de l’exploitation de l’homme par l’homme peut nous inciter à une certaine discrétion dans les manifestations d’orgueil que nous inspire si volontiers le premier nommé de ces deux phénomènes.\par
Le deuxième remède est, dans une large mesure, conditionné par le premier : c’est d’introduire de gré ou de force dans la coalition de nouveaux partenaires, externes cette fois, dont les « mises » soient très différentes de celles qui caractérisent l’association initiale. Cette solution a également été essayée, et si le terme de capitalisme permet, en gros, d’identifier la première, ceux d’impérialisme ou de colonialisme aideront à illustrer la seconde. L’expansion coloniale du XIX\textsuperscript{e} siècle a largement permis à l’Europe industrielle de renouveler (et non certes à son profit exclusif) un élan qui, sans l’introduction des peuples colonialisés dans le circuit, aurait risqué de s’épuiser beaucoup plus rapidement.\par
On voit que, dans les deux cas, le remède consiste à élargir la coalition, soit par diversification interne, soit par admission de nouveaux partenaires ; en fin de compte, il s’agit toujours d’augmenter le nombre des joueurs, c’est-à-dire de revenir à la complexité et à la diversité de la situation initiale. Mais on voit aussi que ces solutions ne peuvent que ralentir provisoirement le processus. Il ne peut y avoir exploitation qu’au sein d’une coalition : entre les deux groupes, dominant et dominé, existent des contacts et se produisent des échanges. À leur tour, et malgré la relation unilatérale qui les unit en apparence, ils doivent, consciemment ou inconsciemment, mettre en commun leurs mises, et progressivement les différences qui les opposent tendent à diminuer. Les améliorations sociales d’une part, l’accession graduelle des peuples colonisés à l’indépendance de l’autre, nous font assister au déroulement de ce phénomène ; et bien qu’il y ait encore beaucoup de chemin à parcourir dans ces deux directions, nous savons que les choses iront inévitablement dans ce sens. Peut-être, en vérité, faut-il interpréter comme une troisième solution l’apparition dans le monde de régimes politiques et sociaux antagonistes ; on peut concevoir qu’une diversification, se renouvelant chaque fois sur un autre plan, permette de maintenir indéfiniment, à travers des formes variables et qui ne cesseront jamais de surprendre les hommes, cet état de déséquilibre dont dépend la survie biologique et culturelle de l’humanité.\par
Quoi qu’il en soit, il est difficile de se représenter autrement que comme contradictoire un processus que l’on peut résumer de la façon suivante : pour progresser, il faut que les hommes collaborent ; et au cours de cette collaboration, ils voient graduellement s’identifier les apports dont la diversité initiale était précisément ce qui rendait leur collaboration féconde et nécessaire.\par
Mais même si cette contradiction est insoluble, le devoir sacré de l’humanité est d’en conserver les deux termes également présents à l’esprit, de ne jamais perdre de vue l’un au profit exclusif de l’autre ; de se garder, sans doute, d’un particularisme aveugle qui tendrait à réserver le privilège de l’humanité à une race, une culture ou une société ; mais aussi de ne jamais oublier qu’aucune fraction de l’humanité ne dispose de formules applicables à l’ensemble, et qu’une humanité confondue dans un genre de vie unique est inconcevable, parce que ce serait une humanité ossifiée.\par
À cet égard, les institutions internationales ont devant elles une tâche immense, et elles portent de lourdes responsabilités. Les unes et les autres sont plus complexes qu’on ne pense. Car la mission des institutions internationales est double ; elle consiste pour une part dans une liquidation, et pour une autre part dans un éveil. Elles doivent d’abord assister l’humanité, et rendre aussi peu douloureuse et dangereuse que possible la résorption de ces diversités mortes, résidus sans valeur de modes de collaboration dont la présence à l’état de vestiges putréfiés constitue un risque permanent d’infection pour le corps international. Elles doivent élaguer, amputer s’il est besoin, et faciliter la naissance d’autres formes d’adaptation.\par
Mais, en même temps, elles doivent être passionnément attentives au fait que, pour posséder la même valeur fonctionnelle que les précédents, ces nouveaux modes ne peuvent les reproduire, ou être conçus sur le même modèle, sans se réduire à des solutions de plus en plus insipides et finalement impuissantes. Il faut qu’elles sachent, au contraire, que l’humanité est riche de possibilités imprévues dont chacune, quand elle apparaîtra, frappera toujours les hommes de stupeur ; que le progrès n’est pas fait à l’image confortable de cette « similitude améliorée » où nous nous cherchons un paresseux repos, mais qu’il est tout plein d’aventures, de ruptures et de scandales. L’humanité est constamment aux prises avec deux processus contradictoires dont l’un tend à instaurer l’unification, tandis que l’autre vise à maintenir ou à rétablir la diversification. La position de chaque époque ou de chaque culture dans le système, l’orientation selon laquelle elle s’y trouve engagée sont telles qu’un seul des deux processus lui paraît avoir un sens, l’autre semblant être la négation du premier. Mais dire, comme on pourrait y être enclin, que l’humanité se défait en même temps qu’elle se fait, procéderait encore d’une vision incomplète. Car, sur deux plans et à deux niveaux opposés, il s’agit bien de deux manières différentes de se \emph{faire.}\par
La nécessité de préserver la diversité des cultures dans un monde menacé par la monotonie et l’uniformité n’a certes pas échappé aux institutions internationales. Elles comprennent aussi qu’il ne suffira pas, pour atteindre ce but, de choyer des traditions locales et d’accorder un répit aux temps révolus. C’est le fait de la diversité qui doit être sauvé, non le contenu historique que chaque époque lui a donné et qu’aucune ne saurait perpétuer au-delà d’elle-même. Il faut donc écouter le blé qui lève, encourager les potentialités secrètes, éveiller toutes les vocations à vivre ensemble que l’histoire tient en réserve ; il faut aussi être prêt à envisager sans surprise, sans répugnance et sans révolte ce que toutes ces nouvelles formes sociales d’expression ne pourront manquer d’offrir d’inusité. La tolérance n’est pas une position contemplative, dispensant les indulgences à ce qui fut ou à ce qui est. C’est une attitude dynamique, qui consiste à prévoir, à comprendre et à promouvoir ce qui veut être. La diversité des cultures humaines est derrière nous, autour de nous et devant nous. La seule exigence que nous puissions faire valoir à son endroit (créatrice pour chaque individu des devoirs correspondants) est qu’elle se réalise sous des formes dont chacune soit une contribution à la plus grande générosité des autres.

\section[{Bibliographie}]{Bibliographie}
\renewcommand{\leftmark}{Bibliographie}

\biblitem{\surname{Auger}, P. \emph{L’homme microscopique.} Paris, 1952.}
\biblitem{\surname{Boas}, F. \emph{The mind of primitive man.} New York, 1931.}
\biblitem{\surname{Dilthey}, W. \emph{Gesammelte Schriften.} Leipzig, 1914-1931.}
\biblitem{\surname{Dixon}, R. B. \emph{The building of culture.} New York et Londres, 1928.}
\biblitem{\surname{Gobineau, A. de.} \emph{Essai sur l’inégalité des races humaines}, 2\textsuperscript{e} éd. Paris, 1884.}
\biblitem{\surname{Hawkes}, C. F. C. \emph{Prehistoric foundations of Europe.} Londres, 1939.}
\biblitem{\surname{Herskovits}, M. J. \emph{Man and his works.} New York, 1948.}
\biblitem{\surname{Kroeber}, A. L. \emph{Anthropology}, nouv. éd. New York, 1948.}
\biblitem{\surname{Leroi-Gourhan}, A. \emph{L’homme et la matière.} Paris, 1943.}
\biblitem{\surname{Linton}, R. \emph{The study of mon.} New York, 1936.}
\biblitem{\surname{Morazé}, Ch. \emph{Essai sur la civilisation d’Occident}, 1.1. Paris, 1949.}
\biblitem{\surname{Pirenne}, J. \emph{Les grands courants de l’histoire universelle}, t. I. Paris, 1947.}
\biblitem{\surname{Pittard}, E. \emph{Les races et l’histoire.} Paris, 1922.}
\biblitem{\surname{Spengler}, O. \emph{Le déclin de l’Occident.} Paris, 1948.}
\biblitem{\surname{Toynbee}, A. J. \emph{A study of history.} Londres, 1948.}
\biblitem{\surname{White}, L. A. \emph{The science of culture.} New York, 1949.}
 


% at least one empty page at end (for booklet couv)
\ifbooklet
  \pagestyle{empty}
  \clearpage
  % 2 empty pages maybe needed for 4e cover
  \ifnum\modulo{\value{page}}{4}=0 \hbox{}\newpage\hbox{}\newpage\fi
  \ifnum\modulo{\value{page}}{4}=1 \hbox{}\newpage\hbox{}\newpage\fi


  \hbox{}\newpage
  \ifodd\value{page}\hbox{}\newpage\fi
  {\centering\color{rubric}\bfseries\noindent\large
    Hurlus ? Qu’est-ce.\par
    \bigskip
  }
  \noindent Des bouquinistes électroniques, pour du texte libre à participations libres,
  téléchargeable gratuitement sur \href{https://hurlus.fr}{\dotuline{hurlus.fr}}.\par
  \bigskip
  \noindent Cette brochure a été produite par des éditeurs bénévoles.
  Elle n’est pas faite pour être possédée, mais pour être lue, et puis donnée.
  Que circule le texte !
  En page de garde, on peut ajouter une date, un lieu, un nom ;
  comme une fiche de bibliothèque en papier,
  pour suivre le voyage du texte. Qui sait, un jour, vous la retrouverez ?
  \par

  Ce texte a été choisi parce qu’une personne l’a aimé,
  ou haï, elle a pensé qu’il partipait à la formation de notre présent ;
  sans le souci de plaire, vendre, ou militer pour une cause.
  \par

  L’édition électronique est soigneuse, tant sur la technique
  que sur l’établissement du texte ; mais sans aucune prétention scolaire, au contraire.
  Le but est de s’adresser à tous, sans distinction de science ou de diplôme.
  Au plus direct ! (possible)
  \par

  Cet exemplaire en papier a été tiré sur une imprimante personnelle
   ou une photocopieuse. Tout le monde peut le faire.
  Il suffit de
  télécharger un fichier sur \href{https://hurlus.fr}{\dotuline{hurlus.fr}},
  d’imprimer, et agrafer ; puis de lire et donner.\par

  \bigskip

  \noindent PS : Les hurlus furent aussi des rebelles protestants qui cassaient les statues dans les églises catholiques. En 1566 démarra la révolte des gueux dans le pays de Lille. L’insurrection enflamma la région jusqu’à Anvers où les gueux de mer bloquèrent les bateaux espagnols.
  Ce fut une rare guerre de libération dont naquit un pays toujours libre : les Pays-Bas.
  En plat pays francophone, par contre, restèrent des bandes de huguenots, les hurlus, progressivement réprimés par la très catholique Espagne.
  Cette mémoire d’une défaite est éteinte, rallumons-la. Sortons les livres du culte universitaire, débusquons les idoles de l’époque, pour les démonter.
\fi

\end{document}
